% Created 2024-11-07 jeu. 12:19
% Intended LaTeX compiler: pdflatex
\documentclass[a4paper,12pt]{article}
\usepackage[utf8]{inputenc}
\usepackage[T1]{fontenc}
\usepackage{graphicx}
\usepackage{longtable}
\usepackage{wrapfig}
\usepackage{rotating}
\usepackage[normalem]{ulem}
\usepackage{amsmath}
\usepackage{amssymb}
\usepackage{capt-of}
\usepackage{hyperref}
\usepackage[french]{babel}
\usepackage[frenchb]{babel}
\usepackage{lmodern}
\DeclareUnicodeCharacter{00A0}{~}
\DeclareUnicodeCharacter{200B}{}
\author{Baptiste Mélès}
\date{7 novembre 2024}
\title{Myriade des sujets de dissertation philosophique}
\hypersetup{
 pdfauthor={Baptiste Mélès},
 pdftitle={Myriade des sujets de dissertation philosophique},
 pdfkeywords={},
 pdfsubject={},
 pdfcreator={Emacs 24.5.1 (Org mode 8.2.10)}, 
 pdflang={French}}
\begin{document}

\maketitle
La « Myriade des sujets de dissertation philosophique » est une
compilation de quelque 15 000 sujets de dissertation de philosophie,
classés par ordre alphabétique, par concours, par thème et par type.

Elle peut être utile :
\begin{itemize}
\item aux étudiantes et étudiants souhaitant se préparer aux examens ou aux
concours ;
\item aux enseignantes et enseignants de méthodologie philosophique
souhaitant montrer quel type de traitement appelle chaque type
de sujet (question totale, couple de notion, etc.) ;
\item aux examinatrices et examinateurs souhaitant reprendre un sujet déjà
donné ou au contraire s'assurer qu'un sujet ne l'a jamais été.
\end{itemize}

Cette méthode peut être utilisée en complément de la « Méthode de la
dissertation philosophique » du même auteur
(\url{http://baptiste.meles.free.fr/site/B.Meles-Methode\_dissertation.pdf}).

Tous les fichiers et programmes ayant servi à produire le présent
document, ainsi que sa version la plus récente, sont archivés et
partagés de manière pérenne sur Software Heritage à l'adresse
\url{https://archive.softwareheritage.org/browse/origin/directory/?origin\_url=https://github.com/BapMel/SujetsDeDissertation}.

L'auteur vous sera reconnaissant de lui signaler toute coquille
ou erreur.



\section*{Concours compilés}

Sont compilés ici tous les sujets donnés aux épreuves écrites et orales
de plus d'une centaine de concours :
\begin{enumerate}
\item agrégation externe (22 années : 2003-2024) ;
\item agrégation interne (20 années : 2005-2024) ;
\item CAPES externe (21 années : 2004-2024) ;
\item CAPES interne (14 années : 2011-2024) ;
\item École normale supérieure de Paris, concours A​/​L (23 années :
2002-2024) ;
\item École normale supérieure de Paris, concours B​/​L (22 années :
2002-2023).
\end{enumerate}


\section*{L'auteur}

\subsection*{Présentation}

Baptiste Mélès est chargé de recherche au CNRS, attaché aux Archives
Henri-Poincaré—PReST (CNRS, Université de Lorraine). Il étudie les
relations entre la philosophie, les sciences formelles (logique,
mathématiques, informatique) et la linguistique.

Il travaille sur l'histoire structurale de la philosophie de Jules
Vuillemin, la philosophie mathématique de Jean Cavaillès et d'Albert
Lautman, et étudie les textes de programmes informatiques comme des
textes à part entière. Il est l'auteur des \emph{Classifications des systèmes
philosophiques} (Paris, Vrin, 2016).

\subsection*{Autres documents pédagogiques du même auteur}

\begin{enumerate}
\item « Méthode de la dissertation philosophique »
(\url{http://baptiste.meles.free.fr/site/B.Meles-Methode\_dissertation.pdf},
2010-2024) ;
\item « Méthode du commentaire de texte philosophique »
(\url{http://baptiste.meles.free.fr/site/B.Meles-Methode\_commentaire\_texte.pdf},
2007-2024)
\item « Méthodologie du mémoire de Master »
(\url{http://baptiste.meles.free.fr/site/B.Meles-Memoire\_Master.pdf},
2014-2024) ;
\item « Le travail personnel en philosophie, de la licence à l'agrégation »
(\url{http://baptiste.meles.free.fr/site/B.Meles-Travail\_perso.pdf},
2008-2024) ;
\item « Bibliographie de philosophie » (avec Paul Clavier,
\url{http://baptiste.meles.free.fr/site/P\_Clavier-B\_Meles-Bibliographie\_philosophie.pdf}, 2022)
\item « Les tables de vérité en braille »
(\url{http://baptiste.meles.free.fr/site/B.Meles-Table\_verite\_braille.pdf},
2011).
\end{enumerate}


\newpage

\tableofcontents

\newpage

\section{Tri par ordre alphabétique}
\label{sec:orgeade547}

\noindent
1, 2, 3 \\[0pt]
2+2 = 4 \\[0pt]
2+2 pourrait-il ne pas être égal à 4 ? \\[0pt]
Abjection et sainteté \\[0pt]
Abolir la propriété \\[0pt]
Absolu / relatif \\[0pt]
Abstraire \\[0pt]
Abstraire, est-ce se couper du réel ? \\[0pt]
Abuser du pouvoir \\[0pt]
Accepter l'injustice \\[0pt]
Accuser et défendre \\[0pt]
À chacun des goûts \\[0pt]
« À chacun sa morale » \\[0pt]
À chacun sa morale \\[0pt]
« À chacun sa vérité » \\[0pt]
À chacun sa vérité \\[0pt]
À chacun sa vérité ? \\[0pt]
À chacun selon son mérite \\[0pt]
À chacun ses goûts \\[0pt]
À chacun son dû \\[0pt]
Acteurs sociaux et usages sociaux \\[0pt]
Action et contemplation \\[0pt]
Action et événement \\[0pt]
Action et fabrication \\[0pt]
Action et production \\[0pt]
Actions et pratiques \\[0pt]
Activité et passivité \\[0pt]
Additionner et soustraire \\[0pt]
Admettre le hasard, est-ce nier l'ordre de la nature ? \\[0pt]
Admettre une cause première, est-ce faire une pétition de principe ? \\[0pt]
Affirmation et négation \\[0pt]
Affirmer et nier \\[0pt]
Agir \\[0pt]
Agir contre ses intérêts \\[0pt]
Agir en conscience \\[0pt]
Agir en politique, est-ce agir dans l'incertain ? \\[0pt]
Agir et faire \\[0pt]
Agir et réagir \\[0pt]
Agir justement fait-il de moi un homme juste ? \\[0pt]
Agir librement \\[0pt]
Agir moralement, est-ce lutter contre ses idées ? \\[0pt]
Agir moralement, est-ce lutter contre soi-même ? \\[0pt]
Agir par devoir, est-ce agir contre son intérêt ? \\[0pt]
Agir par devoir, est-ce évaluer les conséquences de ses actes \\[0pt]
Agir par devoir, est-ce renoncer à ses intérêts ? \\[0pt]
Agir sans raison \\[0pt]
Agir volontairement, est-ce nécessairement savoir ce que l'on fait ? \\[0pt]
Aide-toi, le ciel t'aidera \\[0pt]
Ai-je des devoirs envers moi-même ? \\[0pt]
Ai-je intérêt à la liberté d'autrui ? \\[0pt]
Ai-je le pouvoir de me changer ? \\[0pt]
Ai-je un corps ? \\[0pt]
Ai-je un corps ou suis-je mon corps ? \\[0pt]
Ai-je une âme ? \\[0pt]
« Aime, et fais ce que tu veux » \\[0pt]
Aimer ce qui est beau \\[0pt]
Aimer, est-ce vraiment connaître ? \\[0pt]
Aimer et être aimé \\[0pt]
Aimer la nature \\[0pt]
Aimer la vérité \\[0pt]
Aimer la vie \\[0pt]
Aimer le monde tel qu'il est. \\[0pt]
Aimer l'éphémère \\[0pt]
Aimer les lois \\[0pt]
Aimer peut-il être un devoir ? \\[0pt]
« Aimer » se dit-il en un seul sens ? \\[0pt]
Aimer ses proches \\[0pt]
Aimer son prochain \\[0pt]
Aimer son prochain comme soi-même \\[0pt]
Aimer une œuvre d'art \\[0pt]
Aime ton prochain comme toi-même \\[0pt]
« Aimez vos ennemis » \\[0pt]
Aliénation et servitude \\[0pt]
« À l'impossible nul n'est tenu » \\[0pt]
« À l'impossible, nul n'est tenu » \\[0pt]
À l'impossible nul n'est tenu \\[0pt]
Aller au-delà des apparences \\[0pt]
Aller au fond des choses \\[0pt]
Aller au vrai \\[0pt]
Altérité, diversité, pluralité \\[0pt]
Âme, conscience, esprit \\[0pt]
Ami et ennemi \\[0pt]
Amitié et société \\[0pt]
Amour et amitié \\[0pt]
Amour et inconscient \\[0pt]
Amour et respect \\[0pt]
Analogie et ressemblance \\[0pt]
Analyse et intuition \\[0pt]
Analyse et synthèse \\[0pt]
Analyser \\[0pt]
Analyser les mœurs \\[0pt]
Animal politique ou social ? \\[0pt]
Anomalie et anomie \\[0pt]
Antérieur / postérieur \\[0pt]
Anthropologie et humanisme \\[0pt]
Anthropologie et ontologie \\[0pt]
Anthropologie et politique \\[0pt]
Apparaître \\[0pt]
Apparence et réalité \\[0pt]
Apparition et manifestation \\[0pt]
Appartenons-nous à une culture ? \\[0pt]
Appartient-il aux lois d'éduquer les hommes ? \\[0pt]
Appliquer la loi \\[0pt]
Apprend-on à aimer ? \\[0pt]
Apprend-on à être artiste ? \\[0pt]
Apprend-on à être libre ? \\[0pt]
Apprend-on à penser ? \\[0pt]
Apprend-on à percevoir ? \\[0pt]
Apprend-on à percevoir la beauté ? \\[0pt]
Apprend-on à voir ? \\[0pt]
Apprendre \\[0pt]
Apprendre à aimer \\[0pt]
Apprendre à gouverner \\[0pt]
Apprendre à parler \\[0pt]
Apprendre à penser \\[0pt]
Apprendre à philosopher \\[0pt]
Apprendre à vivre \\[0pt]
Apprendre à voir \\[0pt]
Apprendre à voir ? \\[0pt]
Apprendre des autres \\[0pt]
Apprendre et devenir \\[0pt]
Apprendre et enseigner \\[0pt]
Apprendre s'apprend-il ? \\[0pt]
Apprentissage et conditionnement \\[0pt]
Approcher du vrai \\[0pt]
Après-coup \\[0pt]
« Après moi, le déluge » \\[0pt]
Après moi le déluge \\[0pt]
\emph{A priori} et \emph{a posteriori} \\[0pt]
À quelle condition une démarche est-elle scientifique ? \\[0pt]
À quelle condition un travail est-il humain ? \\[0pt]
À quelle expérience l'art nous convie-t-il ? \\[0pt]
À quelle généralité peuvent prétendre les sciences humaines ? \\[0pt]
À quelle objectivité peuvent prétendre les sciences sociales ? \\[0pt]
À quelles conditions est-il acceptable de travailler ? \\[0pt]
À quelles conditions le vivant peut-il être objet de science ? \\[0pt]
À quelles conditions penser une seconde nature ? \\[0pt]
À quelles conditions peut-on dire qu'une action est historique ? \\[0pt]
À quelles conditions un choix peut-il être rationnel ? \\[0pt]
À quelles conditions une activité est-elle un travail ? \\[0pt]
À quelles conditions une croyance devient-elle religieuse ? \\[0pt]
À quelles conditions une démarche est-elle scientifique ? \\[0pt]
À quelles conditions une expérience est-elle possible ? \\[0pt]
À quelles conditions une explication est-elle scientifique ? \\[0pt]
À quelles conditions une hypothèse est-elle scientifique ? \\[0pt]
À quelles conditions un énoncé est-il doué de sens ? \\[0pt]
À quelles conditions une pensée est-elle libre ? \\[0pt]
À quelles conditions un État peut-il être juste ? \\[0pt]
À quelles conditions une théorie est-elle scientifique ? \\[0pt]
À quelles conditions une théorie peut-elle être scientifique ? \\[0pt]
À quelles conditions une vérité s'impose-t-elle à nous ? \\[0pt]
À quelles conditions un jugement est-il moral ? \\[0pt]
À quelles conditions y a-t-il progrès technique ? \\[0pt]
À quelque chose le malheur peut-il être bon ? \\[0pt]
À quelque chose, le malheur peut-il être bon ? \\[0pt]
« À quelque chose malheur est bon » \\[0pt]
À quelque chose malheur est bon \\[0pt]
À quels signes reconnaît-on la vérité ? \\[0pt]
À qui appartient-il de décider du juste et de l'injuste ? \\[0pt]
À qui appartient la Terre ? \\[0pt]
À qui devons-nous obéir ? \\[0pt]
À qui dois-je la vérité ? \\[0pt]
À qui doit-on la vérité ? \\[0pt]
À qui doit-on le respect ? \\[0pt]
À qui doit-on obéir ? \\[0pt]
À qui est mon corps ? \\[0pt]
À qui faut-il obéir ? \\[0pt]
À qui la faute ? \\[0pt]
À qui le pouvoir appartient-il ? \\[0pt]
À qui peut-on faire confiance ? \\[0pt]
À qui profite le crime ? \\[0pt]
À qui profite le travail ? \\[0pt]
À qui se fier ? \\[0pt]
À quoi bon ? \\[0pt]
À quoi bon avoir mauvaise conscience ? \\[0pt]
À quoi bon chercher à prouver l'existence de Dieu ? \\[0pt]
À quoi bon critiquer les autres ? \\[0pt]
À quoi bon culpabiliser ? \\[0pt]
À quoi bon démontrer ? \\[0pt]
À quoi bon des héros ? \\[0pt]
À quoi bon des philosophes ? \\[0pt]
À quoi bon discuter ? \\[0pt]
À quoi bon imaginer la cité idéale ? \\[0pt]
À quoi bon imiter la nature ? \\[0pt]
À quoi bon la morale ? \\[0pt]
À quoi bon la science ? \\[0pt]
À quoi bon les regrets ? \\[0pt]
À quoi bon les romans ? \\[0pt]
À quoi bon les sciences humaines et sociales ? \\[0pt]
À quoi bon les systèmes ? \\[0pt]
À quoi bon parler ? \\[0pt]
À quoi bon penser la fin du monde ? \\[0pt]
À quoi bon raconter des histoires ? \\[0pt]
À quoi bon rêver ? \\[0pt]
À quoi bon se parler ? \\[0pt]
À quoi bon voyager ? \\[0pt]
À quoi est-il impossible de s'habituer ? \\[0pt]
À quoi est-il nécessaire d'obéir ? \\[0pt]
À quoi faut-il être fidèle ? \\[0pt]
À quoi faut-il renoncer ? \\[0pt]
À quoi faut-il renoncer pour dialoguer ? \\[0pt]
À quoi juger l'action d'un gouvernement ? \\[0pt]
À quoi la conscience nous donne-t-elle accès ? \\[0pt]
À quoi la logique peut-elle servir dans les sciences ? \\[0pt]
À quoi la perception donne-t-elle accès ? \\[0pt]
À quoi la religion sert-elle ? \\[0pt]
À quoi l'art est-il bon ? \\[0pt]
À quoi l'art nous rend-il sensibles ? \\[0pt]
À quoi la valeur d'un homme se mesure-t-elle ? \\[0pt]
À quoi les sciences nous forment-elles ? \\[0pt]
À quoi nos illusions tiennent-elles ? \\[0pt]
À quoi peut-on reconnaître la vérité ? \\[0pt]
À quoi peut-on reconnaître une œuvre d'art ? \\[0pt]
À quoi reconnaît-on la rationalité ? \\[0pt]
À quoi reconnaît-on la vérité ? \\[0pt]
À quoi reconnaît-on le faux ? \\[0pt]
À quoi reconnaît-on le réel ? \\[0pt]
À quoi reconnaît-on l'humanité en chaque homme ? \\[0pt]
À quoi reconnaît-on l'injustice ? \\[0pt]
À quoi reconnaît-on qu'une activité est un travail ? \\[0pt]
À quoi reconnaît-on qu'une expérience est scientifique ? \\[0pt]
À quoi reconnaît-on qu'une pensée est vraie ? \\[0pt]
À quoi reconnaît-on qu'une politique est juste ? \\[0pt]
À quoi reconnaît-on qu'une science est une science ? \\[0pt]
À quoi reconnaît-on qu'une théorie est scientifique ? \\[0pt]
À quoi reconnaît-on qu'un événement est historique ? \\[0pt]
À quoi reconnaît-on un acte libre ? \\[0pt]
À quoi reconnaît-on un acte vraiment libre ? \\[0pt]
À quoi reconnaît-on un bon artisan ? \\[0pt]
À quoi reconnaît-on un bon gouvernement ? \\[0pt]
À quoi reconnaît-on un comportement humain ? \\[0pt]
À quoi reconnaît-on un devoir ? \\[0pt]
À quoi reconnaît-on une action morale ? \\[0pt]
À quoi reconnaît-on une attitude religieuse ? \\[0pt]
À quoi reconnaît-on une bonne interprétation ? \\[0pt]
À quoi reconnaît-on une idéologie ? \\[0pt]
À quoi reconnaît-on une œuvre d'art ? \\[0pt]
À quoi reconnaît-on une religion ? \\[0pt]
À quoi reconnaît-on une science ? \\[0pt]
À quoi reconnaît-on une théorie scientifique ? \\[0pt]
À quoi reconnaît-on un être vivant ? \\[0pt]
À quoi renonçons-nous quand nous obéissons ? \\[0pt]
À quoi rêve-t-on ? \\[0pt]
À quoi sert la bioéthique ? \\[0pt]
À quoi sert la connaissance du passé ? \\[0pt]
À quoi sert la critique ? \\[0pt]
À quoi sert la dialectique ? \\[0pt]
À quoi sert la logique ? \\[0pt]
À quoi sert la mémoire ? \\[0pt]
À quoi sert la métaphysique ? \\[0pt]
À quoi sert la négation ? \\[0pt]
À quoi sert la notion de contrat social ? \\[0pt]
À quoi sert la notion d'état de nature ? \\[0pt]
À quoi sert la technique ? \\[0pt]
À quoi sert la théodicée ? \\[0pt]
À quoi sert la vérité ? \\[0pt]
À quoi sert le contrat social ? \\[0pt]
À quoi sert l'écriture ? \\[0pt]
À quoi sert l'État ? \\[0pt]
À quoi sert l'histoire ? \\[0pt]
À quoi sert l'intuition ? \\[0pt]
À quoi sert l'ontologie ? \\[0pt]
À quoi sert un critique d'art ? \\[0pt]
À quoi sert une métaphore ? \\[0pt]
À quoi sert un exemple ? \\[0pt]
À quoi servent les cérémonies ? \\[0pt]
À quoi servent les constitutions ? \\[0pt]
À quoi servent les croyances ? \\[0pt]
À quoi servent les doctrines morales ? \\[0pt]
À quoi servent les élections ? \\[0pt]
À quoi servent les émotions ? \\[0pt]
À quoi servent les encyclopédies ? \\[0pt]
À quoi servent les expériences ? \\[0pt]
À quoi servent les fictions ? \\[0pt]
À quoi servent les hypothèses ? \\[0pt]
À quoi servent les images ? \\[0pt]
À quoi servent les lois ? \\[0pt]
À quoi servent les machines ? \\[0pt]
À quoi servent les mythes ? \\[0pt]
À quoi servent les œuvres d'art ? \\[0pt]
À quoi servent les preuves ? \\[0pt]
À quoi servent les preuves de l'existence de Dieu ? \\[0pt]
À quoi servent les règles ? \\[0pt]
À quoi servent les religions ? \\[0pt]
À quoi servent les sciences ? \\[0pt]
À quoi servent les sciences humaines ? \\[0pt]
À quoi servent les statistiques ? \\[0pt]
À quoi servent les symboles ? \\[0pt]
À quoi servent les théories ? \\[0pt]
À quoi servent les utopies ? \\[0pt]
À quoi servent les voyages ? \\[0pt]
À quoi suis-je obligé ? \\[0pt]
À quoi tenons-nous ? \\[0pt]
À quoi tient la fermeté du vouloir ? \\[0pt]
À quoi tient la force de l'État ? \\[0pt]
À quoi tient la force des religions ? \\[0pt]
À quoi tient l'autorité ? \\[0pt]
À quoi tient la valeur d'une pensée ? \\[0pt]
À quoi tient la vérité d'une interprétation ? \\[0pt]
À quoi tient l'efficacité d'une technique ? \\[0pt]
À quoi tient le pouvoir des mots ? \\[0pt]
À quoi tient le pouvoir des sciences ? \\[0pt]
À quoi tient notre humanité ? \\[0pt]
Arbitrer \\[0pt]
Architecture et religion \\[0pt]
Architecture et utopie \\[0pt]
Argent et liberté \\[0pt]
Argumenter \\[0pt]
Argumenter et démontrer \\[0pt]
Arriver à ses fins \\[0pt]
Arrive-t-il que l'impossible se produise ? \\[0pt]
Art et abstraction \\[0pt]
Art et apparence \\[0pt]
Art et apparences \\[0pt]
Art et artifice \\[0pt]
Art et authenticité \\[0pt]
Art et beauté \\[0pt]
Art et connaissance \\[0pt]
Art et création \\[0pt]
Art et critique \\[0pt]
Art et décadence \\[0pt]
Art et divertissement \\[0pt]
Art et émotion \\[0pt]
Art et expression \\[0pt]
Art et finitude \\[0pt]
Art et folie \\[0pt]
Art et forme \\[0pt]
Art et illusion \\[0pt]
Art et image \\[0pt]
Art et imagination \\[0pt]
Art et industrie \\[0pt]
Art et interdit \\[0pt]
Art et jeu \\[0pt]
Art et langage \\[0pt]
Art et marchandise \\[0pt]
Art et matière \\[0pt]
Art et mélancolie \\[0pt]
Art et mémoire \\[0pt]
Art et métaphysique \\[0pt]
Art et morale \\[0pt]
Art et politique \\[0pt]
Art et pouvoir \\[0pt]
Art et présence \\[0pt]
Art et propagande \\[0pt]
Art et religieux \\[0pt]
Art et religion \\[0pt]
Art et représentation \\[0pt]
Art et société \\[0pt]
Art et Société \\[0pt]
Art et symbole \\[0pt]
Art et technique \\[0pt]
Art et tradition \\[0pt]
Art et transgression \\[0pt]
Art et utilité \\[0pt]
Art et vérité \\[0pt]
Artiste et artisan \\[0pt]
Artistes et ingénieurs \\[0pt]
Art populaire et art savant \\[0pt]
Arts de l'espace et arts du temps \\[0pt]
Arts du temps et arts de l'espace \\[0pt]
À science nouvelle, nouvelle philosophie ? \\[0pt]
Aspect et apparence \\[0pt]
A-t-on besoin de certitudes ? \\[0pt]
A-t-on besoin de fonder la connaissance ? \\[0pt]
A-t-on besoin de maîtres ? \\[0pt]
A-t-on besoin de spécialistes ? \\[0pt]
A-t-on besoin de spécialistes en politique ? \\[0pt]
A-t-on besoin des poètes ? \\[0pt]
A-t-on besoin d'experts ? \\[0pt]
A-t-on besoin d'un chef ? \\[0pt]
A-t-on des devoirs envers les animaux ? \\[0pt]
A-t-on des devoirs envers l'humanité ? \\[0pt]
A-t-on des devoirs envers qui n'a aucun droit ? \\[0pt]
A-t-on des devoirs envers soi-même ? \\[0pt]
A-t-on des droits contre l'État ? \\[0pt]
A-t-on des raisons de croire ? \\[0pt]
A-t-on des raisons de croire ce qu'on croit ? \\[0pt]
A-t-on expliqué un phénomène quand on l'a rapporté à des lois ? \\[0pt]
A-t-on intérêt à tout savoir ? \\[0pt]
A-t-on le droit de faire tout ce qui est permis par la loi ? \\[0pt]
A-t-on le droit de mentir ? \\[0pt]
A-t-on le droit de résister ? \\[0pt]
A-t-on le droit de se désintéresser de la politique ? \\[0pt]
A-t-on le droit de se révolter ? \\[0pt]
A-t-on le droit de s'évader ? \\[0pt]
A-t-on l'obligation de pardonner ? \\[0pt]
A-t-on raison de mettre la technique en accusation ? \\[0pt]
A-t-on raison de tenir à la démocratie ? \\[0pt]
Attendre \\[0pt]
Attente et espérance \\[0pt]
Au-delà \\[0pt]
Au-delà de la nature ? \\[0pt]
Au nom de qui rend-on justice ? \\[0pt]
Au nom de quoi le plaisir serait-il condamnable ? \\[0pt]
Au nom de quoi rend-on justice ? \\[0pt]
Au nom du peuple \\[0pt]
Aussitôt dit, aussitôt fait \\[0pt]
Autorité et autoritarisme \\[0pt]
Autorité et domination \\[0pt]
Autorité et pouvoir \\[0pt]
Autorité et souveraineté \\[0pt]
Autrui \\[0pt]
Autrui, est-ce n'importe quel autre ? \\[0pt]
Autrui est-il aimable ? \\[0pt]
Autrui est-il inconnaissable ? \\[0pt]
Autrui est-il mon prochain ? \\[0pt]
Autrui est-il mon semblable ? \\[0pt]
Autrui est-il pour moi un mystère ? \\[0pt]
Autrui est-il un autre moi ? \\[0pt]
Autrui est-il un autre moi-même ? \\[0pt]
Autrui me connaît-il mieux que moi-même ? \\[0pt]
Autrui m'est-il étranger ? \\[0pt]
« Aux armes citoyens ! » \\[0pt]
Aux armes, citoyens ! \\[0pt]
Avez-vous une âme ? \\[0pt]
Avoir \\[0pt]
Avoir bonne conscience \\[0pt]
Avoir confiance \\[0pt]
Avoir de la chance \\[0pt]
Avoir de la suite dans les idées \\[0pt]
Avoir de l'autorité \\[0pt]
Avoir de la volonté \\[0pt]
Avoir de l'esprit \\[0pt]
Avoir de l'expérience \\[0pt]
Avoir des ennemis \\[0pt]
Avoir des ennemis, est-ce le principe de la politique ? \\[0pt]
Avoir des principes \\[0pt]
Avoir des remords \\[0pt]
Avoir des scrupules \\[0pt]
Avoir des valeurs \\[0pt]
Avoir du caractère \\[0pt]
Avoir du goût \\[0pt]
Avoir du jugement \\[0pt]
Avoir du mérite \\[0pt]
Avoir du métier \\[0pt]
Avoir du pouvoir \\[0pt]
Avoir du style \\[0pt]
Avoir et être \\[0pt]
Avoir la conscience tranquille \\[0pt]
Avoir la foi \\[0pt]
Avoir la parole, est-ce avoir le pouvoir ? \\[0pt]
Avoir la santé \\[0pt]
Avoir le choix \\[0pt]
Avoir le droit \\[0pt]
Avoir le sens de la situation \\[0pt]
Avoir le sens du devoir \\[0pt]
Avoir le temps \\[0pt]
Avoir mauvaise conscience \\[0pt]
Avoir peur \\[0pt]
Avoir peur des mots \\[0pt]
Avoir raison \\[0pt]
Avoir raison de ses émotions \\[0pt]
Avoir raison, est-ce nécessairement être raisonnable ? \\[0pt]
Avoir ses raisons \\[0pt]
Avoir tout pour être heureux \\[0pt]
Avoir un corps \\[0pt]
Avoir un destin \\[0pt]
Avoir une bonne mémoire \\[0pt]
Avoir une idée \\[0pt]
Avoir un passé, avoir un avenir \\[0pt]
Avoir un sens \\[0pt]
Avons-nous à apprendre des images ? \\[0pt]
Avons-nous accès aux choses-mêmes ? \\[0pt]
Avons-nous besoin d'amis ? \\[0pt]
Avons-nous besoin de cérémonies ? \\[0pt]
Avons-nous besoin de Dieu ? \\[0pt]
Avons-nous besoin de héros ? \\[0pt]
Avons-nous besoin de l'État ? \\[0pt]
Avons-nous besoin de lois ? \\[0pt]
Avons-nous besoin de maîtres ? \\[0pt]
Avons-nous besoin de métaphysique ? \\[0pt]
Avons-nous besoin de méthodes ? \\[0pt]
Avons-nous besoin de nous tromper nous-même ? \\[0pt]
Avons-nous besoin de partis politiques ? \\[0pt]
Avons-nous besoin de rêver ? \\[0pt]
Avons-nous besoin de spectacles ? \\[0pt]
Avons-nous besoin de traditions ? \\[0pt]
Avons-nous besoin d'être gouvernés ? \\[0pt]
Avons-nous besoin d'experts en matière d'art ? \\[0pt]
Avons-nous besoin d'idéal \\[0pt]
Avons-nous besoin d'un chef ? \\[0pt]
Avons-nous besoin d'une conception métaphysique du monde ? \\[0pt]
Avons-nous besoin d'une définition de l'art ? \\[0pt]
Avons-nous besoin d'une philosophie politique ? \\[0pt]
Avons-nous besoin d'un libre arbitre ? \\[0pt]
Avons-nous besoin d'utopies ? \\[0pt]
Avons-nous des devoirs à l'égard de la vérité ? \\[0pt]
Avons-nous des devoirs envers la nature ? \\[0pt]
Avons-nous des devoirs envers le passé ? \\[0pt]
Avons-nous des devoirs envers les animaux ? \\[0pt]
Avons-nous des devoirs envers les autres êtres vivants ? \\[0pt]
Avons-nous des devoirs envers les générations futures ? \\[0pt]
Avons-nous des devoirs envers les morts ? \\[0pt]
Avons-nous des devoirs envers le vivant ? \\[0pt]
Avons-nous des devoirs envers nous-mêmes ? \\[0pt]
Avons-nous des devoirs envers tous les vivants ? \\[0pt]
Avons-nous des droits sur la nature ? \\[0pt]
Avons-nous des raisons d'espérer ? \\[0pt]
Avons-nous encore besoin de la nature ? \\[0pt]
Avons-nous intérêt à la liberté d'autrui ? \\[0pt]
Avons-nous le devoir de dire la vérité ? \\[0pt]
Avons-nous le devoir d'être heureux ? \\[0pt]
Avons-nous le devoir de vivre ? \\[0pt]
Avons-nous le droit de juger autrui ? \\[0pt]
Avons-nous le droit d'être heureux ? \\[0pt]
Avons-nous le temps d'apprendre à vivre ? \\[0pt]
Avons-nous peur de la liberté ? \\[0pt]
Avons-nous raison de croire ? \\[0pt]
Avons-nous raison d'exiger toujours des raisons ? \\[0pt]
Avons-nous un accès direct aux phénomènes ? \\[0pt]
Avons-nous un corps ? \\[0pt]
Avons-nous un devoir de vérité ? \\[0pt]
Avons-nous un droit au droit ? \\[0pt]
Avons-nous une âme ? \\[0pt]
Avons-nous une identité ? \\[0pt]
Avons-nous une intuition du temps ? \\[0pt]
Avons-nous une obligation envers les générations à venir ? \\[0pt]
Avons-nous une responsabilité envers le passé ? \\[0pt]
Avons-nous un libre arbitre ? \\[0pt]
Avons-nous un monde commun ? \\[0pt]
Axiomatique et vérité \\[0pt]
Axiomatiser, est-ce fonder ? \\[0pt]
Bâtir un monde \\[0pt]
Beauté et moralité \\[0pt]
Beauté et vérité \\[0pt]
Beauté naturelle et beauté artistique \\[0pt]
Beauté réelle, beauté idéale \\[0pt]
Besoin et désir \\[0pt]
Besoins et désirs \\[0pt]
Bêtise et méchanceté \\[0pt]
Bien agir \\[0pt]
Bien agir, est-ce nécessairement faire son devoir ? \\[0pt]
Bien agir, est-ce toujours être moral ? \\[0pt]
Bien commun et bien public \\[0pt]
Bien commun et droits individuels \\[0pt]
Bien commun et intérêt particulier \\[0pt]
Bien faire \\[0pt]
« Bienheureuse faute » \\[0pt]
Bien jouer son rôle \\[0pt]
Bien juger \\[0pt]
Bien parler \\[0pt]
Bien parler, est-ce bien penser ? \\[0pt]
Bien parler exige-t-il des qualités morales ? \\[0pt]
Bien vivre \\[0pt]
Bonheur de chacun bonheur de tous \\[0pt]
Bonheur et autarcie \\[0pt]
Bonheur et bien-être \\[0pt]
Bonheur et plaisir \\[0pt]
Bonheur et satisfaction \\[0pt]
Bonheur et société \\[0pt]
Bonheur et technique \\[0pt]
Bonheur et vertu \\[0pt]
Bon sens et philosophie \\[0pt]
Bon sens et sens commun \\[0pt]
Calculer \\[0pt]
Calculer et penser \\[0pt]
Calculer et raisonner \\[0pt]
Cartographier \\[0pt]
Castes et classes \\[0pt]
Catégories de langue, catégories de pensée \\[0pt]
Catégories de l'être, catégories de langue \\[0pt]
Catégories de pensée, catégories de langue \\[0pt]
Catégories logiques et catégories linguistiques \\[0pt]
Causalité et finalité \\[0pt]
Cause et condition \\[0pt]
Cause et effet \\[0pt]
Cause et loi \\[0pt]
Cause et motivation \\[0pt]
Cause et raison \\[0pt]
Causes et lois \\[0pt]
Causes et motivations \\[0pt]
Causes et raisons \\[0pt]
Causes premières et causes secondes \\[0pt]
« Ceci » \\[0pt]
Ceci \\[0pt]
Cela a-t-il un sens de penser par soi-même ? \\[0pt]
« Ce ne sont que des mots » \\[0pt]
Ce que je pense est-il nécessairement vrai ? \\[0pt]
Ce que la morale autorise, l'État peut-il légitimement l'interdire ? \\[0pt]
Ce que la technique rend possible, peut-on jamais en empêcher la réalisation ? \\[0pt]
Ce que nous avons le devoir de faire peut-il toujours s'exprimer sous forme de loi ? \\[0pt]
Ce que sait le poète \\[0pt]
Ce qui dépasse la raison est-il nécessairement irréel ? \\[0pt]
Ce qui dépend de moi \\[0pt]
Ce qui doit-être, est-ce autre chose que ce qui est ? \\[0pt]
Ce qui est à moi \\[0pt]
Ce qui est contingent peut-il être objet de science ? \\[0pt]
Ce qui est contradictoire peut-il exister ? \\[0pt]
Ce qui est démontré est-il nécessairement vrai ? \\[0pt]
Ce qui est faux est-il dénué de sens ? \\[0pt]
Ce qui est ordinaire est-il normal ? \\[0pt]
Ce qui est passé est-il dépassé ? \\[0pt]
Ce qui est subjectif est-il arbitraire ? \\[0pt]
Ce qui est vrai est-il toujours vérifiable ? \\[0pt]
Ce qui fut et ce qui sera \\[0pt]
Ce qui importe \\[0pt]
Ce qu'il y a \\[0pt]
Ce qui n'a pas de prix \\[0pt]
Ce qui n'a pas lieu d'être \\[0pt]
Ce qui ne dépend pas de nous \\[0pt]
Ce qui ne peut s'acheter est-il dépourvu de valeur ? \\[0pt]
Ce qui n'est pas \\[0pt]
Ce qui n'est pas démontré peut-il être vrai ? \\[0pt]
Ce qui n'est pas matériel peut-il être réel ? \\[0pt]
Ce qui n'est pas réel est-il impossible ? \\[0pt]
Ce qui n'existe pas \\[0pt]
Ce qui passe et ce qui demeure \\[0pt]
Ce qui subsiste et ce qui change \\[0pt]
Ce qui vaut en théorie vaut-il toujours en pratique ? \\[0pt]
Ce qu'on ne peut pas vendre \\[0pt]
Certaines œuvres d'art ont-elles plus de valeur que d'autres ? \\[0pt]
Certitude et conviction \\[0pt]
Certitude et pari \\[0pt]
Certitude et probabilité \\[0pt]
Certitude et vérité \\[0pt]
Cesser d'espérer \\[0pt]
« C'est en forgeant que l'on devient forgeron » \\[0pt]
« C'est humain » \\[0pt]
« C'est la vie » \\[0pt]
« C'est plus fort que moi » \\[0pt]
« C'est pour ton bien » \\[0pt]
C'est pour ton bien \\[0pt]
« C'est scientifique » \\[0pt]
« C'est tout un art » \\[0pt]
C'est trop beau pour être vrai ! \\[0pt]
Ceux qui oppriment sont-ils libres ? \\[0pt]
Ceux qui savent doivent-ils gouverner ? \\[0pt]
Chacun a-t-il le droit d'invoquer sa vérité ? \\[0pt]
Chacun a-t-il sa propre vérité ? \\[0pt]
« Chacun ses goûts » \\[0pt]
Chance et bonheur \\[0pt]
Changement et mouvement \\[0pt]
Changer \\[0pt]
Changer d'opinion \\[0pt]
Changer, est-ce devenir un autre ? \\[0pt]
Changer, est-ce se trahir ? \\[0pt]
Changer la vie \\[0pt]
« Changer le monde » \\[0pt]
Changer le monde \\[0pt]
Changer ses désirs plutôt que l'ordre du monde \\[0pt]
Change-t-on avec le temps ? \\[0pt]
Chanter \\[0pt]
Chanter et parler \\[0pt]
Chaos et cosmos \\[0pt]
Chaque science porte-t-elle une métaphysique qui lui est propre ? \\[0pt]
Châtier, est ce faire honneur au criminel ? \\[0pt]
Chercher ses mots \\[0pt]
Chercher son intérêt, est-ce être immoral ? \\[0pt]
Choisir \\[0pt]
Choisir, est-ce renoncer ? \\[0pt]
Choisir ses souvenirs ? \\[0pt]
Choisissons-nous qui nous sommes ? \\[0pt]
Choisit-on ses souvenirs ? \\[0pt]
Choisit-on son corps ? \\[0pt]
Choix et raison \\[0pt]
Chose et objet \\[0pt]
Chose et personne \\[0pt]
Choses et personnes \\[0pt]
Choses naturelles, choses fabriquées \\[0pt]
Cinéma et réalité \\[0pt]
Cinéma et vérité \\[0pt]
Cité juste ou citoyen juste ? \\[0pt]
Citoyen du monde ? \\[0pt]
Citoyen et soldat \\[0pt]
Citoyenneté et solidarité \\[0pt]
Citoyens du monde \\[0pt]
Civilisation et barbarie \\[0pt]
Civilisé, barbare, sauvage \\[0pt]
Classer \\[0pt]
Classer, est-ce penser ? \\[0pt]
Classer et ordonner \\[0pt]
Classes et essences \\[0pt]
Classes et histoire \\[0pt]
Classicisme et romantisme \\[0pt]
Cohérence et vérité \\[0pt]
Colère et indignation \\[0pt]
Collectionner \\[0pt]
Commander \\[0pt]
Comme d'habitude \\[0pt]
Commémorer \\[0pt]
Commencer \\[0pt]
Commencer en philosophie \\[0pt]
Comment assumer les conséquences de ses actes ? \\[0pt]
Comment autrui peut-il m'aider à rechercher le bonheur ? \\[0pt]
Comment bien appliquer une règle ? \\[0pt]
Comment bien vivre ? \\[0pt]
Comment caractériser une idée confuse ? \\[0pt]
Comment chercher ce qu'on ignore ? \\[0pt]
Comment comprendre les faits sociaux ? \\[0pt]
Comment comprendre une croyance qu'on ne partage pas ? \\[0pt]
Comment conduire ses pensées ? \\[0pt]
Comment connaissons-nous nos devoirs ? \\[0pt]
Comment connaître l'autre ? \\[0pt]
Comment connaître le passé ? \\[0pt]
Comment connaître l'inconscient ? \\[0pt]
Comment connaître nos devoirs ? \\[0pt]
Comment considérer le devenir ? \\[0pt]
Comment croire au progrès ? \\[0pt]
Comment décider, sinon à la majorité ? \\[0pt]
Comment définir la démocratie ? \\[0pt]
Comment définir la raison ? \\[0pt]
Comment définir la responsabilité politique \\[0pt]
Comment définir la signification \\[0pt]
Comment définir le bien commun ? \\[0pt]
Comment définir le laid ? \\[0pt]
Comment définir le peuple ? \\[0pt]
Comment définir le style ? \\[0pt]
Comment définir l'État de droit ? \\[0pt]
Comment définir l'excès ? \\[0pt]
Comment définir l'infini ? \\[0pt]
Comment déterminer le sens d'une conduite ? \\[0pt]
Comment deux personnes peuvent-elles partager la même pensée ? \\[0pt]
Comment devient-on artiste ? \\[0pt]
Comment devient-on citoyen ? \\[0pt]
Comment devient-on raisonnable ? \\[0pt]
Comment dire la vérité ? \\[0pt]
Comment dire l'individuel ? \\[0pt]
Comment distinguer culture et civilisation ? \\[0pt]
Comment distinguer désirs et besoins ? \\[0pt]
Comment distinguer entre l'amour et l'amitié ? \\[0pt]
Comment distinguer l'amour de l'amitié ? \\[0pt]
Comment distinguer le rêvé du perçu ? \\[0pt]
Comment distinguer le vrai du faux ? \\[0pt]
Comment distingue-t-on le vrai du réel ? \\[0pt]
Comment entrer dans l'histoire ? \\[0pt]
Comment établir des critères d'équité ? \\[0pt]
Comment être naturel ? \\[0pt]
Comment évaluer la qualité de la vie ? \\[0pt]
Comment évaluer l'art ? \\[0pt]
Comment expliquer l'erreur ? \\[0pt]
Comment expliquer les phénomènes mentaux ? \\[0pt]
Comment exprimer l'identité ? \\[0pt]
Comment faire de l'homme un objet d'étude ? \\[0pt]
Comment fonder la propriété ? \\[0pt]
Comment fonder nos devoirs ? \\[0pt]
Comment fonder une psychologie sociale ? \\[0pt]
Comment juger de la justesse d'une interprétation ? \\[0pt]
Comment juger de la politique ? \\[0pt]
Comment juger d'une œuvre d'art ? \\[0pt]
Comment juger son éducation ? \\[0pt]
Comment juger une œuvre d'art ? \\[0pt]
Comment justifier l'autonomie des sciences de la vie ? \\[0pt]
Comment justifier une méthode ? \\[0pt]
Comment la psychanalyse use-t-elle de l'interprétation ? \\[0pt]
Comment la réalité peut-elle dépasser la fiction ? \\[0pt]
Comment la science progresse-t-elle ? \\[0pt]
Comment la volonté peut-elle être faible ? \\[0pt]
Comment la volonté peut-elle être indéterminée ? \\[0pt]
Comment le devoir peut-il déterminer l'action ? \\[0pt]
Comment le passé nous est-il présent ? \\[0pt]
Comment le passé peut-il demeurer présent ? \\[0pt]
Comment l'erreur est-elle possible ? \\[0pt]
Comment les mathématiques peuvent-elles s'appliquer au réel ? \\[0pt]
Comment les sciences sociales expliquent-elles l'irrationnel ? \\[0pt]
Comment les sociétés changent-elles ? \\[0pt]
Comment l'homme peut-il se représenter le temps ? \\[0pt]
Comment l'inconscient peut-il se manifester ? \\[0pt]
Comment mesurer ? \\[0pt]
Comment mesurer une sensation ? \\[0pt]
Comment ne pas être humaniste ? \\[0pt]
Comment ne pas être libéral ? \\[0pt]
Comment penser la diversité des langues ? \\[0pt]
Comment penser l'écoulement du temps ? \\[0pt]
Comment penser le futur ? \\[0pt]
Comment penser le hasard ? \\[0pt]
Comment penser le mouvement ? \\[0pt]
Comment penser les universaux ? \\[0pt]
Comment penser l'éternel ? \\[0pt]
Comment penser un pouvoir qui ne corrompe pas ? \\[0pt]
Comment percevons-nous l'espace ? \\[0pt]
Comment peut-on choisir entre différentes hypothèses ? \\[0pt]
Comment peut-on définir la politique ? \\[0pt]
Comment peut-on définir un être vivant ? \\[0pt]
Comment peut-on être heureux ? \\[0pt]
« Comment peut-on être persan ? » \\[0pt]
Comment peut-on être sceptique ? \\[0pt]
Comment peut-on savoir que l'on a raison ? \\[0pt]
Comment peut-on se trahir soi-même ? \\[0pt]
Comment prend-on connaissance de ses devoirs ? \\[0pt]
Comment prouver la liberté ? \\[0pt]
Comment puis-je devenir ce que je suis ? \\[0pt]
Comment reconnaît-on une œuvre d'art ? \\[0pt]
Comment reconnaît-on un vivant ? \\[0pt]
Comment réfuter une thèse métaphysique ? \\[0pt]
Comment rendre la justice ? \\[0pt]
Comment représenter la douleur ? \\[0pt]
Comment retrouver la nature ? \\[0pt]
Comment sait-on qu'on se comprend ? \\[0pt]
Comment sait-on qu'une chose existe ? \\[0pt]
Comment s'assurer de ce qui est réel ? \\[0pt]
Comment s'assurer qu'on est libre ? \\[0pt]
Comment savoir ce qui échappe à la conscience \\[0pt]
Comment savoir ce qui existe ? \\[0pt]
Comment savoir quand nous sommes libres ? \\[0pt]
Comment savoir que l'on est dans l'erreur ? \\[0pt]
Comment savoir quels sont nos devoirs ? \\[0pt]
Comment se libérer du temps ? \\[0pt]
Comment se mettre à la place d'autrui ? \\[0pt]
Comment s'entendre ? \\[0pt]
Comment s'orienter dans la pensée ? \\[0pt]
Comment traiter les animaux ? \\[0pt]
Comment trancher une controverse ? \\[0pt]
Comment vérifier les hypothèses ? \\[0pt]
Comment vivre ensemble ? \\[0pt]
Comment voyager dans le temps ? \\[0pt]
Comme on dit \\[0pt]
Commettre une faute \\[0pt]
Communauté, collectivité, société \\[0pt]
Communauté et conflit \\[0pt]
Communauté et société \\[0pt]
Communiquer \\[0pt]
Communiquer et enseigner \\[0pt]
Comparaison n'est pas raison \\[0pt]
Comparer les cultures \\[0pt]
Comparer les cultures ? \\[0pt]
Comparer les langues \\[0pt]
Compatir \\[0pt]
Compétence et autorité \\[0pt]
Complet / incomplet \\[0pt]
Comportement et conduite \\[0pt]
Composer avec les circonstances \\[0pt]
Composition et construction \\[0pt]
Comprendre \\[0pt]
Comprendre autrui \\[0pt]
Comprendre, est-ce excuser ? \\[0pt]
Comprendre, est-ce expliquer par les causes ? \\[0pt]
Comprendre, est-ce interpréter ? \\[0pt]
Comprendre le réel, est-ce le dominer ? \\[0pt]
Comprendre le sens d'un texte \\[0pt]
Comprendre l'inconscient \\[0pt]
Comprendre une démonstration \\[0pt]
Comprendre une œuvre d'art \\[0pt]
Compter \\[0pt]
Compter et mesurer \\[0pt]
Compter sur soi \\[0pt]
Concept et existence \\[0pt]
Concept et image \\[0pt]
Concept et intuition \\[0pt]
Concept et métaphore \\[0pt]
Conception et perception \\[0pt]
Concevoir et expérimenter \\[0pt]
Concevoir et juger \\[0pt]
Concevoir et percevoir \\[0pt]
Concevoir le possible \\[0pt]
Conclure \\[0pt]
Concurrence et égalité \\[0pt]
Conduire sa vie \\[0pt]
Conduire ses pensées \\[0pt]
Conduite et comportement \\[0pt]
Confiance et crédulité \\[0pt]
Conflit et démocratie \\[0pt]
Conflit et liberté \\[0pt]
Confondre \\[0pt]
Connaissance commune et connaissance scientifique \\[0pt]
Connaissance, croyance, conjecture \\[0pt]
Connaissance de soi et conscience de soi \\[0pt]
Connaissance du futur et connaissance du passé \\[0pt]
Connaissance et croyance \\[0pt]
Connaissance et expérience \\[0pt]
Connaissance et liberté \\[0pt]
Connaissance et perception \\[0pt]
Connaissance et progrès \\[0pt]
Connaissance historique et action politique \\[0pt]
Connaissons-nous la réalité des choses ? \\[0pt]
Connaissons-nous la réalité telle qu'elle est ? \\[0pt]
Connaissons-nous mieux le présent que le passé ? \\[0pt]
« Connais-toi toi-même » \\[0pt]
Connais-toi toi-même \\[0pt]
Connaît-on jamais pour le plaisir ? \\[0pt]
Connaît-on la vie ou bien connaît-on le vivant ? \\[0pt]
Connaît-on la vie ou connaît-on le vivant ? \\[0pt]
Connaît-on la vie ou le vivant ? \\[0pt]
Connaît-on les choses telles qu'elles sont ? \\[0pt]
Connaît-on pour le plaisir ? \\[0pt]
Connaître autrui \\[0pt]
Connaître, est-ce connaître par les causes ? \\[0pt]
Connaître, est-ce découvrir le réel ? \\[0pt]
Connaître, est-ce dépasser les apparences ? \\[0pt]
Connaître et agir \\[0pt]
Connaître et comprendre \\[0pt]
Connaître et penser \\[0pt]
Connaître, expliquer, comprendre \\[0pt]
Connaître la vie ou le vivant ? \\[0pt]
Connaître les causes \\[0pt]
Connaître les choses, en quoi est-ce déterminer leurs différences ? \\[0pt]
Connaître l'infini \\[0pt]
Connaître par les causes \\[0pt]
Connaître ses limites \\[0pt]
Connaître ses origines \\[0pt]
Connaître une chose, est-ce en connaître la cause ? \\[0pt]
Conquérir \\[0pt]
Conscience de soi et amour de soi \\[0pt]
Conscience de soi et connaissance de soi \\[0pt]
Conscience et attention \\[0pt]
Conscience et connaissance \\[0pt]
Conscience et conscience de soi \\[0pt]
Conscience et existence \\[0pt]
Conscience et liberté \\[0pt]
Conscience et mauvaise conscience \\[0pt]
Conscience et mémoire \\[0pt]
Conscience et responsabilité \\[0pt]
Conscience et subjectivité \\[0pt]
Conscience et volonté \\[0pt]
Conseiller le prince \\[0pt]
Consensus et conflit \\[0pt]
Consentir \\[0pt]
Conservatisme et tradition \\[0pt]
Considère-t-on jamais le temps en lui-même ? \\[0pt]
Consistance et précarité \\[0pt]
Constitution et lois \\[0pt]
Construire la vérité \\[0pt]
Construire l'espace \\[0pt]
Consumérisme et démocratie \\[0pt]
Contemplation et action \\[0pt]
Contemplation et création \\[0pt]
Contemplation et distraction \\[0pt]
Contempler \\[0pt]
Contempler une œuvre d'art \\[0pt]
Contingence et nécessité \\[0pt]
Contingence et rationalité \\[0pt]
Continuité et discontinuité \\[0pt]
Contradiction et opposition \\[0pt]
Contrainte et désobéissance \\[0pt]
Contrainte et obligation \\[0pt]
Contrôle et vigilance \\[0pt]
Convaincre \\[0pt]
Convaincre et persuader \\[0pt]
Convention et observation \\[0pt]
Conventions sociales et moralité \\[0pt]
Conviction et certitude \\[0pt]
Conviction et responsabilité \\[0pt]
Convient-il d'opposer explication et interprétation ? \\[0pt]
Convient-il d'opposer liberté et nécessité ? \\[0pt]
Corps et espace \\[0pt]
Corps et esprit \\[0pt]
Corps et identité \\[0pt]
Corps et matière \\[0pt]
Corps et nature \\[0pt]
Correspondre \\[0pt]
Corriger ses erreurs \\[0pt]
Couleur et forme \\[0pt]
Crainte et espoir \\[0pt]
Création et critique \\[0pt]
Création et fabrication \\[0pt]
Création et production \\[0pt]
Création et réception \\[0pt]
Créativité et contrainte \\[0pt]
Créer \\[0pt]
Créer et produire \\[0pt]
Crime et châtiment \\[0pt]
Crimes et châtiments \\[0pt]
Crise et création \\[0pt]
Crise et critique \\[0pt]
Crise et progrès \\[0pt]
Crise et révolution \\[0pt]
Critiquer \\[0pt]
Critiquer la démocratie \\[0pt]
Croire \\[0pt]
Croire au bonheur \\[0pt]
Croire aux fictions \\[0pt]
Croire en Dieu \\[0pt]
Croire en soi \\[0pt]
Croire, est-ce être faible ? \\[0pt]
Croire, est-ce obéir ? \\[0pt]
Croire, est-ce renoncer à l'usage de la raison ? \\[0pt]
Croire, est-ce renoncer à savoir ? \\[0pt]
Croire, est-ce renoncer au savoir ? \\[0pt]
Croire et savoir \\[0pt]
Croire pour savoir \\[0pt]
Croire que Dieu existe, est-ce croire en lui ? \\[0pt]
Croire savoir \\[0pt]
Croit-on ce que l'on veut ? \\[0pt]
Croit-on comme on veut ? \\[0pt]
Croyance et certitude \\[0pt]
Croyance et choix \\[0pt]
Croyance et confiance \\[0pt]
Croyance et connaissance \\[0pt]
Croyance et passivité \\[0pt]
Croyance et probabilité \\[0pt]
Croyance et société \\[0pt]
Croyance et vérité \\[0pt]
Culpabilité et responsabilité \\[0pt]
Cultes et rituels \\[0pt]
Cultivons notre jardin \\[0pt]
Culture et artifice \\[0pt]
Culture et civilisation \\[0pt]
Culture et communauté \\[0pt]
Culture et conscience \\[0pt]
Culture et différence \\[0pt]
Culture et éducation \\[0pt]
Culture et identité \\[0pt]
Culture et langage \\[0pt]
Culture et moralité \\[0pt]
Culture et savoir \\[0pt]
Culture et sociétés \\[0pt]
Culture et technique \\[0pt]
Culture et violence \\[0pt]
Danser \\[0pt]
Dans l'action, est-ce l'intention qui compte ? \\[0pt]
Dans les sciences humaines et sociales, peut-on préconiser l'imitation des sciences de la nature ? \\[0pt]
Dans quel but les hommes se donnent-ils des lois ? \\[0pt]
Dans quelle mesure est-on l'auteur de sa propre vie ? \\[0pt]
Dans quelle mesure l'art est-il un fait social ? \\[0pt]
Dans quelle mesure la technique nous libère-t-elle de la nature ? \\[0pt]
Dans quelle mesure la traduction est-elle une activité cognitive ? \\[0pt]
Dans quelle mesure les sciences ne sont-elles pas à l'abri de l'erreur ? \\[0pt]
Dans quelle mesure le temps nous appartient-il ? \\[0pt]
Dans quelle mesure pouvons-nous échapper à notre passé ? \\[0pt]
Dans quelle mesure suis-je responsable de mon inconscient ? \\[0pt]
Dans quelle mesure toute philosophie est-elle critique du langage ? \\[0pt]
« Dans un bois aussi courbe que celui dont l'homme est fait on ne peut rien tailler de tout à fait droit » \\[0pt]
D'après nature \\[0pt]
Débat et démocratie \\[0pt]
Débattre \\[0pt]
Débattre et dialoguer \\[0pt]
Déchiffrer \\[0pt]
Décider \\[0pt]
Décomposer les choses \\[0pt]
Décorer \\[0pt]
Découverte et invention \\[0pt]
Découverte et invention dans les sciences \\[0pt]
Découverte et justification \\[0pt]
Découvrir \\[0pt]
Découvrir et inventer \\[0pt]
Décrire \\[0pt]
Décrire, est-ce déjà expliquer ? \\[0pt]
Décrire et expliquer \\[0pt]
Décrire une langue \\[0pt]
Déduction et démonstration \\[0pt]
Déduction et expérience \\[0pt]
Défendre ses droits \\[0pt]
Défendre son honneur \\[0pt]
Définir \\[0pt]
Définir, est-ce déterminer l'essence ? \\[0pt]
Définir et démontrer : à quoi bon ? \\[0pt]
Définir l'art : à quoi bon ? \\[0pt]
Définir la vérité, est-ce la connaître ? \\[0pt]
Définition et démonstration \\[0pt]
Définition nominale et définition réelle \\[0pt]
Définitions, axiomes, postulats \\[0pt]
Dégager la structure, est-ce rendre compte de ce qu'est une chose ? \\[0pt]
Déjouer \\[0pt]
« De la musique avant toute chose » \\[0pt]
Délibérer \\[0pt]
Délibérer, est-ce être dans l'incertitude ? \\[0pt]
De l'utilité des voyages \\[0pt]
Dématérialiser \\[0pt]
Démêler le vrai du faux \\[0pt]
Démériter \\[0pt]
Démocrates et démagogues \\[0pt]
Démocratie ancienne et démocratie moderne \\[0pt]
Démocratie et anarchie \\[0pt]
Démocratie et consensus \\[0pt]
Démocratie et démagogie \\[0pt]
Démocratie et impérialisme \\[0pt]
Démocratie et opinion \\[0pt]
Démocratie et religion \\[0pt]
Démocratie et représentation \\[0pt]
Démocratie et république \\[0pt]
Démocratie et technocratie \\[0pt]
Démocratie et transparence \\[0pt]
Démocratie et vérité \\[0pt]
Démocratie représentative \\[0pt]
Démonstration et argumentation \\[0pt]
Démonstration et déduction \\[0pt]
Démonstration et intuition \\[0pt]
Démontrer \\[0pt]
Démontrer, argumenter, expérimenter \\[0pt]
Démontrer est-il le privilège du mathématicien ? \\[0pt]
Démontrer et argumenter \\[0pt]
Démontrer et expérimenter \\[0pt]
Démontrer par l'absurde \\[0pt]
Démontre-t-on ailleurs qu'en mathématiques ? \\[0pt]
Dénaturer \\[0pt]
Dépasser les apparences ? \\[0pt]
Dépasser les bornes \\[0pt]
Dépasser l'humain \\[0pt]
Dépend-il de nous d'être heureux ? \\[0pt]
Dépend-il de soi d'être heureux ? \\[0pt]
Déplaire \\[0pt]
De quel bonheur sommes-nous capables ? \\[0pt]
De quel droit ? \\[0pt]
De quel droit l'État exerce-t-il un pouvoir ? \\[0pt]
De quel droit punit-on ? \\[0pt]
De quel homme parlent les sciences humaines ? \\[0pt]
De quelle certitude la science est-elle capable ? \\[0pt]
De quelle expertise les sciences humaines peuvent-elles se prévaloir ? \\[0pt]
De quelle liberté témoigne l'œuvre d'art ? \\[0pt]
De quelle réalité nos perceptions témoignent-elles ? \\[0pt]
De quelle réalité parlons-nous lorsque nous nous exprimons ? \\[0pt]
De quelle réalité témoignent nos perceptions ? \\[0pt]
De quelle science humaine la folie peut-elle être l'objet ? \\[0pt]
De quelle transgression l'art est-il susceptible ? \\[0pt]
De quelle vérité l'art est-il capable ? \\[0pt]
De quelle vérité l'inconscient est-il porteur ? \\[0pt]
De quelle vérité l'opinion est-elle capable ? \\[0pt]
De qui sommes-nous les obligés ? \\[0pt]
De quoi a-t-on besoin ? \\[0pt]
De quoi a-t-on conscience lorsqu'on a conscience de soi ? \\[0pt]
De quoi avons-nous besoin ? \\[0pt]
De quoi avons-nous vraiment besoin ? \\[0pt]
De quoi dépend le bonheur ? \\[0pt]
De quoi dépend notre bonheur ? \\[0pt]
De quoi doute un sceptique ? \\[0pt]
De quoi est fait mon présent ? \\[0pt]
De quoi est fait notre présent ? \\[0pt]
De quoi est-on conscient ? \\[0pt]
De quoi est-on malheureux ? \\[0pt]
De quoi faisons-nous l'expérience ? \\[0pt]
De quoi fait-on l'expérience face à une œuvre ? \\[0pt]
De quoi faut-il attendre son bonheur ? \\[0pt]
De quoi faut-il avoir peur ? \\[0pt]
De quoi « humain » est-il le nom ? \\[0pt]
De quoi la forme est-elle la forme ? \\[0pt]
De quoi la logique est-elle la science ? \\[0pt]
De quoi la musique est-elle l'art ? \\[0pt]
De quoi la philosophie est-elle le désir ? \\[0pt]
De quoi la politique s'occupe-t-elle ? \\[0pt]
De quoi la religion sauve-t-elle ? \\[0pt]
De quoi l'art nous console-t-il ? \\[0pt]
De quoi l'art nous délivre-t-il ? \\[0pt]
De quoi l'art peut-il être contemporain ? \\[0pt]
De quoi l'art peut-il nous libérer ? \\[0pt]
De quoi la technique ne peut-elle pas nous libérer ? \\[0pt]
De quoi la vérité libère-t-elle ? \\[0pt]
De quoi le devoir libère-t-il ? \\[0pt]
De quoi le phénomène est-il l'apparaître ? \\[0pt]
De quoi le réel est-il constitué ? \\[0pt]
De quoi les logiciens parlent-ils ? \\[0pt]
De quoi les métaphysiciens parlent-ils ? \\[0pt]
De quoi les sciences humaines nous instruisent-elles ? \\[0pt]
De quoi l'État doit-il être propriétaire ? \\[0pt]
De quoi l'État ne doit-il pas se mêler ? \\[0pt]
De quoi le tissu social est-il fait ? \\[0pt]
De quoi le tyran est-il libre ? \\[0pt]
De quoi l'expérience esthétique est-elle expérience ? \\[0pt]
De quoi l'expérience esthétique est-elle l'expérience ? \\[0pt]
De quoi l'historien fait-il l'histoire ? \\[0pt]
De quoi l'inconscient peut-il être la cause ? \\[0pt]
De quoi l'outil nous libère-t-il ? \\[0pt]
De quoi n'avons-nous pas conscience ? \\[0pt]
De quoi ne peut-on pas répondre ? \\[0pt]
De quoi parlent les mathématiques ? \\[0pt]
De quoi parlent les théories physiques ? \\[0pt]
De quoi pâtit-on ? \\[0pt]
De quoi peut-il y avoir science ? \\[0pt]
De quoi peut-on être certain ? \\[0pt]
De quoi peut-on être inconscient ? \\[0pt]
De quoi peut-on faire l'expérience ? \\[0pt]
De quoi peut-on réclamer la propriété ? \\[0pt]
De quoi pouvons-nous être certains ? \\[0pt]
De quoi pouvons-nous être sûrs ? \\[0pt]
De quoi puis-je répondre ? \\[0pt]
De quoi rit-on ? \\[0pt]
De quoi sommes-nous coupables ? \\[0pt]
De quoi sommes-nous prisonniers ? \\[0pt]
De quoi sommes-nous responsables ? \\[0pt]
De quoi suis-je inconscient ? \\[0pt]
De quoi suis-je responsable ? \\[0pt]
De quoi une œuvre d'art nous instruit-elle ? \\[0pt]
De quoi y a-t-il expérience ? \\[0pt]
De quoi y a-t-il histoire ? \\[0pt]
Déraisonner \\[0pt]
Déraisonner, est-ce perdre de vue le réel ? \\[0pt]
Désacraliser \\[0pt]
Des comportements économiques peuvent-ils être rationnels ? \\[0pt]
Description et explication \\[0pt]
Désespérer de l'homme \\[0pt]
Des événements aléatoires peuvent-ils obéir à des lois ? \\[0pt]
Des goûts et des couleurs \\[0pt]
« Des goûts et des couleurs, on ne dispute pas » \\[0pt]
Des hommes et des dieux \\[0pt]
Des inégalités peuvent-elles être justes ? \\[0pt]
Désintérêt et désintéressement \\[0pt]
Désirer \\[0pt]
Désirer, est-ce être aliéné ? \\[0pt]
Désirer, est-ce refuser de se satisfaire de la réalité \\[0pt]
Désirer et vouloir \\[0pt]
Désir et besoin \\[0pt]
Désir et bonheur \\[0pt]
Désir et interdit \\[0pt]
Désir et langage \\[0pt]
Désir et manque \\[0pt]
Désire-t-on la reconnaissance ? \\[0pt]
Désire-t-on un autre que soi ? \\[0pt]
Désir et ordre \\[0pt]
Désir et plaisir \\[0pt]
Désir et politique \\[0pt]
Désir et pouvoir \\[0pt]
Désir et raison \\[0pt]
Désir et réalité \\[0pt]
Désir et société \\[0pt]
Désir et temps \\[0pt]
Désir et volonté \\[0pt]
Désirons-nous les choses parce qu'elles sont bonnes ? \\[0pt]
Des lois justes suffisent-elles à assurer la justice ? \\[0pt]
Des motivations peuvent-elles être sociales ? \\[0pt]
Des mythes peuvent-ils encore apparaître ? \\[0pt]
Des nations peuvent-elles former une société ? \\[0pt]
Désobéir \\[0pt]
Désobéir aux lois \\[0pt]
Désobéissance et résistance \\[0pt]
Désordre et chaos \\[0pt]
Des peuples sans histoire \\[0pt]
Des sciences molles ? \\[0pt]
Dessiner \\[0pt]
Des sociétés sans État sont-elles des sociétés politiques ? \\[0pt]
Des sociétés sans histoire \\[0pt]
Des statistiques et des hommes \\[0pt]
Des structures et des hommes \\[0pt]
Détermination, déterminisme, liberté \\[0pt]
Déterminer \\[0pt]
Déterminisme et responsabilité \\[0pt]
Déterminisme naturel et déterminisme social \\[0pt]
Déterminisme psychique et déterminisme physique \\[0pt]
Détruire et construire \\[0pt]
Détruire pour reconstruire \\[0pt]
Deux et deux font quatre \\[0pt]
Deux personnes peuvent-elles partager la même pensée ? \\[0pt]
Devant qui est-on responsable ? \\[0pt]
Devant qui sommes-nous responsables ? \\[0pt]
Devenir \\[0pt]
Devenir autre \\[0pt]
Devenir ce que l'on est \\[0pt]
Devenir citoyen \\[0pt]
Devenir et évolution \\[0pt]
Devenir humain \\[0pt]
« Deviens ce que tu es » \\[0pt]
« Deviens qui tu es » \\[0pt]
Devient-on raisonnable ? \\[0pt]
Devoir, est-ce avoir une dette envers quelqu'un ? \\[0pt]
Devoir, est-ce pouvoir ? \\[0pt]
Devoir, est-ce vouloir ? \\[0pt]
Devoir et bonheur \\[0pt]
Devoir et conformisme \\[0pt]
Devoir et contrainte \\[0pt]
Devoir et intérêt \\[0pt]
Devoir et liberté \\[0pt]
Devoir et plaisir \\[0pt]
Devoir et prudence \\[0pt]
Devoir et utilité \\[0pt]
Devoir et vertu \\[0pt]
Devoir mourir \\[0pt]
Devoirs envers les autres et devoirs envers soi-même \\[0pt]
Devoirs et passions \\[0pt]
Devons-nous aimer notre destin ? \\[0pt]
Devons-nous apprendre à vivre ? \\[0pt]
Devons-nous dire la vérité ? \\[0pt]
Devons-nous douter de l'existence des choses ? \\[0pt]
Devons-nous espérer vivre sans travailler ? \\[0pt]
Devons-nous être heureux ? \\[0pt]
Devons-nous être obéissants ? \\[0pt]
Devons-nous nous faire confiance ? \\[0pt]
Devons-nous nous libérer de nos désirs ? \\[0pt]
Devons-nous quelque chose à la nature ? \\[0pt]
Devons-nous tenir certaines connaissances pour acquises ? \\[0pt]
Devons-nous toujours dire la vérité ? \\[0pt]
Devons-nous vivre comme si nous ne devions jamais mourir ? \\[0pt]
Diachronie et synchronie \\[0pt]
Dialectique et Philosophie \\[0pt]
Dialectique et vérité \\[0pt]
Dialogue et délibération en démocratie \\[0pt]
Dialoguer \\[0pt]
Dialoguer et s'entendre \\[0pt]
Dieu aurait-il pu mieux faire ? \\[0pt]
Dieu des philosophes et Dieu des croyants \\[0pt]
Dieu est-il l'être ? \\[0pt]
Dieu est-il mort ? \\[0pt]
Dieu est-il mortel ? \\[0pt]
Dieu est-il un concept ? \\[0pt]
Dieu est-il une hypothèse dont le savant et le philosophe peuvent se passer ? \\[0pt]
Dieu est-il une invention humaine ? \\[0pt]
Dieu est-il une limite de la pensée ? \\[0pt]
« Dieu est mort » \\[0pt]
Dieu est mort \\[0pt]
Dieu et César \\[0pt]
Dieu et l'âme \\[0pt]
Dieu et le monde \\[0pt]
Dieu, l'âme, le monde \\[0pt]
Dieu pense-t-il ? \\[0pt]
Dieu peut-il tout faire ? \\[0pt]
Dieu, prouvé ou éprouvé ? \\[0pt]
Dieu tout-puissant \\[0pt]
Différer \\[0pt]
Dire ce qui est \\[0pt]
Dire, est-ce autre chose que vouloir dire ? \\[0pt]
Dire, est-ce faire ? \\[0pt]
Dire, est-ce trahir ? \\[0pt]
Dire et exprimer \\[0pt]
Dire et faire \\[0pt]
Dire et montrer \\[0pt]
Dire « je » \\[0pt]
Dire je \\[0pt]
Dire la vérité \\[0pt]
Dire la vérité, est-ce un devoir ? \\[0pt]
Dire le monde \\[0pt]
Dire le singulier \\[0pt]
Dire l'individuel \\[0pt]
Dire oui \\[0pt]
Dire que l'être humain est un animal, est-ce méconnaître sa dignité ? \\[0pt]
Diriger son esprit \\[0pt]
Discerner et juger \\[0pt]
Discret et continu \\[0pt]
Discrimination et revendication \\[0pt]
Discussion et conversation \\[0pt]
Discussion et dialogue \\[0pt]
Discuter de la beauté d'une chose, est-ce discuter sur une réalité ? \\[0pt]
Disposer de son corps \\[0pt]
Distinguer \\[0pt]
Diversité des sciences humaines et unité de l'homme \\[0pt]
Diversité ou universalité \\[0pt]
Diviser pour mieux régner \\[0pt]
Division du travail et cohésion sociale \\[0pt]
Document, archive, témoignage \\[0pt]
Documents et monuments \\[0pt]
Dogme et opinion \\[0pt]
Dois-je admettre tout ce que je ne peux réfuter ? \\[0pt]
Dois-je mériter mon bonheur ? \\[0pt]
Dois-je obéir à la loi ? \\[0pt]
Dois-je tenir compte de ce que font les autres pour orienter ma conduite ? \\[0pt]
Doit-on apprendre à devenir soi-même ? \\[0pt]
Doit-on apprendre à percevoir ? \\[0pt]
Doit-on apprendre à vivre ? \\[0pt]
Doit-on attendre de la technique qu'elle mette fin au travail ? \\[0pt]
Doit-on bien juger pour bien faire ? \\[0pt]
Doit-on cesser de chercher à définir l'œuvre d'art ? \\[0pt]
Doit-on changer ses désirs, plutôt que l'ordre du monde ? \\[0pt]
Doit-on chasser les artistes de la cité ? \\[0pt]
Doit-on chercher à être heureux ? \\[0pt]
Doit-on corriger les inégalités sociales ? \\[0pt]
Doit-on croire au progrès ? \\[0pt]
Doit-on croire en l'humanité ? \\[0pt]
Doit-on cultiver l'ironie ? \\[0pt]
Doit-on déplorer le désaccord ? \\[0pt]
Doit-on distinguer devoir moral et obligation sociale ? \\[0pt]
Doit-on identifier l'âme à la conscience ? \\[0pt]
Doit-on interpréter les rêves ? \\[0pt]
Doit-on justifier les inégalités ? \\[0pt]
Doit-on le respect au vivant ? \\[0pt]
Doit-on mûrir pour la liberté ? \\[0pt]
Doit-on préférer l'injustice au désordre ? \\[0pt]
Doit-on rechercher le bonheur ? \\[0pt]
Doit-on rechercher l'harmonie ? \\[0pt]
Doit-on refuser d'interpréter ? \\[0pt]
Doit-on répondre de ce qu'on est devenu ? \\[0pt]
Doit-on respecter la nature ? \\[0pt]
Doit-on respecter les êtres vivants ? \\[0pt]
Doit-on se faire l'avocat du diable ? \\[0pt]
Doit-on se justifier d'exister ? \\[0pt]
Doit-on se mettre à la place d'autrui ? \\[0pt]
Doit-on se passer des utopies ? \\[0pt]
Doit-on souffrir de n'être pas compris ? \\[0pt]
Doit-on tenir le plaisir pour une fin ? \\[0pt]
Doit-on toujours dire la vérité ? \\[0pt]
Doit-on toujours rechercher la vérité ? \\[0pt]
Doit-on tout accepter de l'État ? \\[0pt]
Doit-on tout attendre de l'État ? \\[0pt]
Doit-on tout calculer ? \\[0pt]
Doit-on tout contrôler ? \\[0pt]
Doit-on tout pardonner ? \\[0pt]
Doit-on travailler ? \\[0pt]
Doit-on vraiment tout pardonner ? \\[0pt]
Dominer la nature \\[0pt]
Don, échange, potlatch \\[0pt]
Don et échange \\[0pt]
Don Juan \\[0pt]
Donner \\[0pt]
Donner à chacun son dû \\[0pt]
Donner, à quoi bon ? \\[0pt]
Donner des exemples \\[0pt]
Donner des preuves \\[0pt]
Donner des raisons \\[0pt]
Donner du sens \\[0pt]
Donner et recevoir \\[0pt]
Donner l'exemple ? \\[0pt]
Donner raison \\[0pt]
Donner raison, rendre raison \\[0pt]
Donner sa parole \\[0pt]
Donner sens \\[0pt]
Donner son assentiment \\[0pt]
Donner une représentation \\[0pt]
Donner un exemple \\[0pt]
D'où la politique tire-t-elle sa légitimité ? \\[0pt]
Doute et raison \\[0pt]
Douter \\[0pt]
Douter de tout \\[0pt]
D'où viennent les concepts ? \\[0pt]
D'où viennent les idées générales ? \\[0pt]
D'où viennent les préjugés ? \\[0pt]
D'où viennent nos idées ? \\[0pt]
D'où vient aux objets techniques leur beauté ? \\[0pt]
D'où vient la certitude ? \\[0pt]
D'où vient la certitude dans les sciences ? \\[0pt]
D'où vient la force d'une religion ? \\[0pt]
D'où vient l'amour de Dieu ? \\[0pt]
D'où vient la servitude ? \\[0pt]
D'où vient la signification des mots ? \\[0pt]
D'où vient le mal ? \\[0pt]
D'où vient le plaisir de lire ? \\[0pt]
D'où vient le sens du devoir ? \\[0pt]
D'où vient que l'histoire soit autre chose qu'un chaos ? \\[0pt]
Dressage et éducation \\[0pt]
Droit et coutume \\[0pt]
Droit et démocratie \\[0pt]
Droit et devoir \\[0pt]
Droit et devoir sont-ils liés ? \\[0pt]
Droit et interprétation \\[0pt]
Droit et morale \\[0pt]
Droit et moralité \\[0pt]
Droit et nature \\[0pt]
Droit et protection \\[0pt]
Droit et violence \\[0pt]
Droit naturel et loi naturelle \\[0pt]
« Droits de\ldots{} » et « droits à\ldots{} » \\[0pt]
Droits de l'homme, droits du citoyen \\[0pt]
Droits de l'homme et droits du citoyen \\[0pt]
Droits de l'homme ou droits du citoyen ? \\[0pt]
Droits et devoirs \\[0pt]
Droits et devoirs sont-ils réciproques ? \\[0pt]
Droits et obligations \\[0pt]
Droits, garanties, protection \\[0pt]
« Du passé, faisons table rase » \\[0pt]
Du passé pouvons-nous faire table rase ? \\[0pt]
Du penser à l'être, la conséquence est-elle bonne ? \\[0pt]
Durée et instant \\[0pt]
Durer \\[0pt]
Échange et don \\[0pt]
Échange et partage \\[0pt]
Échange et valeur \\[0pt]
Échanger \\[0pt]
Échanger des idées \\[0pt]
Échanger, est-ce créer de la valeur ? \\[0pt]
Échanger, est-ce partager ? \\[0pt]
Échanger, est-ce risquer ? \\[0pt]
Échappe-t-on à la nécessité ? \\[0pt]
Éclairer \\[0pt]
Économie et politique \\[0pt]
Économie et prédiction \\[0pt]
Économie et société \\[0pt]
Économie politique et politique économique \\[0pt]
Écouter \\[0pt]
Écouter et entendre \\[0pt]
Écouter son corps \\[0pt]
Écrire \\[0pt]
Écrire et parler \\[0pt]
Écrire l'histoire \\[0pt]
Écrire l'histoire, est-ce donner un sens à la violence ? \\[0pt]
Écrire l'histoire, est-ce la faire ? \\[0pt]
Éducation de l'homme, éducation du citoyen \\[0pt]
Éducation et citoyenneté \\[0pt]
Éducation et dressage \\[0pt]
Éducation et instruction \\[0pt]
Éduquer \\[0pt]
Éduquer à la liberté \\[0pt]
Éduquer, est-ce dénaturer ? \\[0pt]
Éduquer et instruire \\[0pt]
Éduquer le citoyen \\[0pt]
Éduquer le désir \\[0pt]
Éduquer le goût \\[0pt]
Efficacité et justice \\[0pt]
Égalité des droits, égalité des conditions \\[0pt]
Égalité et différence \\[0pt]
Égalité et identité \\[0pt]
Égalité et liberté \\[0pt]
Égalité et solidarité \\[0pt]
Égoïsme et altruisme \\[0pt]
Égoïsme et individualisme \\[0pt]
Égoïsme et méchanceté \\[0pt]
Élire \\[0pt]
Élite et démocratie \\[0pt]
Empathie et sympathie \\[0pt]
Empirique et expérimental \\[0pt]
Empirique et transcendantal \\[0pt]
En apparence \\[0pt]
Enfance et moralité \\[0pt]
En finir avec les préjugés \\[0pt]
En histoire, tout est-il affaire d'interprétation ? \\[0pt]
En mathématiques, la notion d'existence a-t-elle un sens ? \\[0pt]
En morale, est-ce seulement l'intention qui compte ? \\[0pt]
En morale, peut-on dire : « C'est l'intention qui compte » ? \\[0pt]
En morale, y a-t-il quelque chose à savoir ? \\[0pt]
En politique, faut-il refuser l'utopie ? \\[0pt]
En politique, nécessité fait loi \\[0pt]
En politique, ne fait-on que défendre ses intérêts ? \\[0pt]
En politique, ne faut-il croire qu'aux rapports de force ? \\[0pt]
En politique n'y a-t-il que des rapports de force ? \\[0pt]
En politique, peut-on faire table rase du passé ? \\[0pt]
En politique, y a-t-il des modèles ? \\[0pt]
En quel sens la maladie peut-elle transformer notre vie ? \\[0pt]
En quel sens la métaphysique a-t-elle une histoire ? \\[0pt]
En quel sens la métaphysique est-elle une science ? \\[0pt]
En quel sens l'anthropologie peut-elle être historique ? \\[0pt]
En quel sens la perception engage-t-elle le corps ? \\[0pt]
En quel sens les sciences ont-elles une histoire ? \\[0pt]
En quel sens l'État est-il rationnel ? \\[0pt]
En quel sens le vécu est-il l'objet des sciences humaines ? \\[0pt]
En quel sens le vivant a-t-il une histoire ? \\[0pt]
En quel sens parler de lois de la pensée ? \\[0pt]
En quel sens parler de structure métaphysique ? \\[0pt]
En quel sens parler d'identité culturelle ? \\[0pt]
En quel sens parler d'illusion religieuse ? \\[0pt]
En quel sens peut-on appeler les sciences humaines « sciences de l'esprit » ? \\[0pt]
En quel sens peut-on dire de l'écologie qu'elle est politique ? \\[0pt]
En quel sens peut-on dire que la vérité s'impose ? \\[0pt]
En quel sens peut-on dire que le langage est la condition de la pensée ? \\[0pt]
En quel sens peut-on dire que le mal n'existe pas ? \\[0pt]
En quel sens peut-on dire que l'homme est un animal politique ? \\[0pt]
En quel sens peut-on dire que nos paroles nous trahissent ? \\[0pt]
En quel sens peut-on dire qu'« on expérimente avec sa raison » ? \\[0pt]
En quel sens peut-on parler de la mort de l'art ? \\[0pt]
En quel sens peut-on parler de la vie sociale comme d'un jeu ? \\[0pt]
En quel sens peut-on parler de responsabilité collective ? \\[0pt]
En quel sens peut-on parler de transcendance ? \\[0pt]
En quel sens peut-on parler d'expérience possible ? \\[0pt]
En quel sens peut-on parler d'une « culture technique » ? \\[0pt]
En quel sens peut-on parler d'une culture technique ? \\[0pt]
En quel sens peut-on parler d'une interprétation de la nature ? \\[0pt]
En quel sens une œuvre d'art est-elle un document ? \\[0pt]
En quels sens les corps peuvent-ils être objets de la métaphysique ? \\[0pt]
Enquêter \\[0pt]
En quoi la connaissance de la matière peut-elle relever de la métaphysique ? \\[0pt]
En quoi la connaissance du vivant contribue-t-elle à la connaissance de l'homme ? \\[0pt]
En quoi la croyance religieuse se distingue-t-elle des autres croyances ? \\[0pt]
En quoi la justice met-elle fin à la violence ? \\[0pt]
En quoi la liberté n'est-elle pas une illusion ? \\[0pt]
En quoi la linguistique est-elle une science ? \\[0pt]
En quoi la matière s'oppose-t-elle à l'esprit ? \\[0pt]
En quoi la métaphysique est-elle une science ? \\[0pt]
En quoi la méthode est-elle un art de penser ? \\[0pt]
En quoi la nature constitue-t-elle un modèle ? \\[0pt]
En quoi la nature peut-elle être source de création ? \\[0pt]
En quoi l'animal intéresse-t-il les sciences humaines ? \\[0pt]
En quoi la patience est-elle une vertu ? \\[0pt]
En quoi la physique a-t-elle besoin des mathématiques ? \\[0pt]
En quoi la question de Dieu est-elle philosophique ? \\[0pt]
En quoi l'art peut-il intéresser le philosophe ? \\[0pt]
En quoi la sexualité peut-elle être objet de science ? \\[0pt]
En quoi la sociologie est-elle fondamentale ? \\[0pt]
En quoi la technique fait-elle question ? \\[0pt]
En quoi le bien d'autrui m'importe-t-il ? \\[0pt]
En quoi le bonheur est-il l'affaire de l'État ? \\[0pt]
En quoi le langage est-il constitutif de l'homme ? \\[0pt]
En quoi le langage est-il objet de science ? \\[0pt]
En quoi les hommes restent-ils des enfants ? \\[0pt]
En quoi les sciences humaines nous éclairent-elles sur la barbarie ? \\[0pt]
En quoi les sciences humaines sont-elles normatives ? \\[0pt]
En quoi les vivants témoignent-ils d'une histoire ? \\[0pt]
En quoi l'histoire est-elle une science du réel ? \\[0pt]
En quoi l'œuvre d'art donne-t-elle à penser ? \\[0pt]
En quoi une culture peut-elle être la mienne ? \\[0pt]
En quoi une discussion est-elle politique ? \\[0pt]
En quoi une insulte est-elle blessante ? \\[0pt]
En quoi une œuvre d'art est-elle moderne ? \\[0pt]
Enseigner \\[0pt]
Enseigner, est-ce transmettre un savoir ? \\[0pt]
Enseigner et éduquer \\[0pt]
Enseigner, instruire, éduquer \\[0pt]
Enseigner l'art \\[0pt]
Entendement et raison \\[0pt]
Entendre \\[0pt]
Entendre raison \\[0pt]
Entité et identité \\[0pt]
Entre croire et savoir, y a-t-il une différence de nature ? \\[0pt]
Entre l'art et la nature, qui imite l'autre ? \\[0pt]
Entre la sécurité de tous et la liberté de chacun, le politique doit-il choisir ? \\[0pt]
Entre le vrai et le faux y a-t-il une place pour le probable ? \\[0pt]
Entre l'opinion et la science, n'y a-t-il qu'une différence de degré ? \\[0pt]
Entre l'utile et l'honnête, peut-on choisir ? \\[0pt]
Entrer en scène \\[0pt]
Énumérer \\[0pt]
Envers qui avons-nous des devoirs ? \\[0pt]
Épistémologie et psychologie \\[0pt]
Épistémologie générale et épistémologie des sciences particulières \\[0pt]
Éprouver \\[0pt]
Éprouver sa valeur \\[0pt]
Errer \\[0pt]
Erreur et faute \\[0pt]
Erreur et illusion \\[0pt]
Erreur, illusion, hallucination \\[0pt]
Espace et réalité \\[0pt]
Espace et représentation \\[0pt]
Espace et structure sociale \\[0pt]
Espace mathématique et espace physique \\[0pt]
Espace privé, espace public \\[0pt]
Espace public et vie privée \\[0pt]
Espace vécu, espace géométrique \\[0pt]
Espèce et genre \\[0pt]
Espérer \\[0pt]
Esprit et intériorité \\[0pt]
Essayer \\[0pt]
Essence et existence \\[0pt]
Essence et nature \\[0pt]
Est beau ce qui ne sert à rien \\[0pt]
Est-ce à la fin que le sens apparaît ? \\[0pt]
Est-ce à la raison de déterminer ce qui est réel ? \\[0pt]
Est-ce à l'État de faire le bonheur du peuple ? \\[0pt]
Est-ce de la force que l'État tient son autorité ? \\[0pt]
Est-ce la certitude qui fait la science ? \\[0pt]
Est-ce la démonstration qui fait la science ? \\[0pt]
Est-ce la majorité qui doit décider ? \\[0pt]
Est-ce la mémoire qui constitue mon identité ? \\[0pt]
Est-ce l'autorité qui fait la loi ? \\[0pt]
Est-ce le cerveau qui pense ? \\[0pt]
Est-ce l'échange utilitaire qui fait le lien social ? \\[0pt]
Est-ce le corps qui perçoit ? \\[0pt]
Est-ce l'ignorance qui nous fait croire ? \\[0pt]
Est-ce l'ignorance qui rend les hommes croyants ? \\[0pt]
Est-ce l'intérêt qui fonde le lien social ? \\[0pt]
Est-ce l'utilité qui définit un objet technique ? \\[0pt]
Est-ce par désir de la vérité que l'homme cherche à savoir ? \\[0pt]
Est-ce par son objet ou par ses méthodes qu'une science peut se définir ? \\[0pt]
Est-ce pour des raisons morales qu'il faut protéger l'environnement ? \\[0pt]
Est-ce seulement l'intention qui compte ? \\[0pt]
Est-ce un devoir d'aimer son prochain ? \\[0pt]
Est-ce un devoir de rechercher la vérité ? \\[0pt]
Est-ce un devoir d'être sincère ? \\[0pt]
Est-ce une chance de naître humain ? \\[0pt]
Esthétique et éthique \\[0pt]
Esthétique et poétique \\[0pt]
Esthétique et politique \\[0pt]
Esthétisme et moralité \\[0pt]
Est-il bien vrai qu'« on n'arrête pas le progrès » ? \\[0pt]
Est-il bon qu'un seul commande ? \\[0pt]
Est-il dans mon intérêt d'accomplir mes devoirs ? \\[0pt]
Est-il difficile de savoir ce que l'on veut ? \\[0pt]
Est-il difficile de savoir ce qu'on veut ? \\[0pt]
Est-il difficile d'être heureux ? \\[0pt]
Est-il difficile de vivre en société ? \\[0pt]
Est-il facile d'être libre ? \\[0pt]
Est-il immoral de se rendre heureux ? \\[0pt]
Est-il impossible de moraliser la politique ? \\[0pt]
Est-il judicieux de revenir sur ses décisions ? \\[0pt]
Est-il juste de payer l'impôt ? \\[0pt]
Est-il juste d'interpréter la loi ? \\[0pt]
Est-il justifié de parler de « corps social » ? \\[0pt]
Est-il légitime d'affirmer que seul le présent existe ? \\[0pt]
Est-il légitime d'opposer liberté et nécessité ? \\[0pt]
Est-il mauvais de suivre son désir ? \\[0pt]
Est-il méritoire de ne faire que son devoir ? \\[0pt]
Est-il naturel à l'homme de parler ? \\[0pt]
Est-il naturel de s'aimer soi-même ? \\[0pt]
Est-il nécessaire d'espérer pour entreprendre ? \\[0pt]
Est-il parfois bon de mentir ? \\[0pt]
Est-il pertinent d'opposer les actes et les paroles ? \\[0pt]
Est-il pertinent d'opposer l'individuel et le collectif ? \\[0pt]
Est-il possible d'améliorer l'homme ? \\[0pt]
Est-il possible de croire en la vie éternelle ? \\[0pt]
Est-il possible de douter de tout ? \\[0pt]
Est-il possible de ne croire à rien ? \\[0pt]
Est-il possible de ne pas savoir ce que l'on pense ? \\[0pt]
Est-il possible de préparer l'avenir ? \\[0pt]
Est-il possible de tout avoir pour être heureux ? \\[0pt]
Est-il possible d'être immoral sans le savoir ? \\[0pt]
Est-il possible d'être neutre politiquement ? \\[0pt]
Est-il possible d'ignorer toute vérité ? \\[0pt]
Est-il raisonnable d'aimer ? \\[0pt]
Est-il raisonnable d'avoir des certitudes ? \\[0pt]
Est-il raisonnable de chercher à interpréter les rêves ? \\[0pt]
Est-il raisonnable de lutter contre le temps ? \\[0pt]
Est-il raisonnable de se quereller pour des mots ? \\[0pt]
Est-il raisonnable d'être rationnel ? \\[0pt]
Est-il raisonnable de vouloir maîtriser la nature ? \\[0pt]
Est-il si difficile d'accéder à la vérité ? \\[0pt]
Est-il toujours avantageux de promouvoir son propre intérêt ? \\[0pt]
Est-il toujours illicite d'inférer ce qui doit être à partir de ce qui est ? \\[0pt]
Est-il toujours meilleur d'avoir le choix ? \\[0pt]
Est-il toujours moral de faire son devoir ? \\[0pt]
Est-il toujours possible de faire ce que l'on dit ? \\[0pt]
Est-il toujours possible de savoir ce que l'on doit faire ? \\[0pt]
Est-il utile d'avoir mal ? \\[0pt]
Est-il vrai que les animaux ne pensent pas ? \\[0pt]
Est-il vrai que l'ignorant n'est pas libre ? \\[0pt]
Est-il vrai que ma liberté s'arrête là où commence celle des autres ? \\[0pt]
Est-il vrai que nous ne nous tenons jamais au temps présent ? \\[0pt]
Est-il vrai qu'en science, « rien n'est donné, tout est construit » ? \\[0pt]
Est-il vrai que plus on échange, moins on se bat ? \\[0pt]
Est-il vrai qu'on apprenne de ses erreurs ? \\[0pt]
Estime et respect \\[0pt]
Estimer \\[0pt]
Est-on fondé à distinguer la justice et le droit ? \\[0pt]
Est-on fondé à parler d'une imperfection du langage ? \\[0pt]
Est-on l'auteur de sa propre vie ? \\[0pt]
Est-on le produit d'une culture ? \\[0pt]
Est-on libre de ne pas vouloir ce que l'on veut ? \\[0pt]
Est-on libre de travailler ? \\[0pt]
Est-on libre face à la vérité ? \\[0pt]
Est-on libre par hasard ? \\[0pt]
Est-on propriétaire de son corps ? \\[0pt]
Est-on responsable de ce qu'on n'a pas voulu ? \\[0pt]
Est-on responsable de l'avenir de l'humanité \\[0pt]
Est-on responsable de ses croyances ? \\[0pt]
Est-on responsable de ses rêves ? \\[0pt]
Est-on responsable de son passé ? \\[0pt]
Est-on sociable par nature ? \\[0pt]
Est-on vraiment libre d'exprimer ce que l'on veut ? \\[0pt]
Établir la vérité, est-ce nécessairement démontrer ? \\[0pt]
Établir les faits \\[0pt]
État et gouvernement \\[0pt]
État et institutions \\[0pt]
État et nation \\[0pt]
État et société \\[0pt]
État et Société \\[0pt]
État et société civile \\[0pt]
État et vie privée \\[0pt]
État et violence \\[0pt]
État, peuple, nation \\[0pt]
Éternité et immortalité \\[0pt]
Éternité et perpétuité \\[0pt]
Éthique et authenticité \\[0pt]
Éthique et esthétique \\[0pt]
Éthique et Morale \\[0pt]
Ethnologie et cinéma \\[0pt]
Ethnologie et ethnocentrisme \\[0pt]
Ethnologie et sociologie \\[0pt]
Étonnement et sidération \\[0pt]
« Et pourtant elle tourne » \\[0pt]
Être absolument libre \\[0pt]
Être acteur \\[0pt]
Être adulte \\[0pt]
Être affairé \\[0pt]
Être à l'écoute de son désir, est-ce nier le désir de l'autre ? \\[0pt]
Être aliéné \\[0pt]
Être apolitique \\[0pt]
Être à sa place \\[0pt]
Être asocial \\[0pt]
Être attentif \\[0pt]
Être au monde \\[0pt]
Être bien élevé \\[0pt]
Être bon juge \\[0pt]
Être cause de soi \\[0pt]
Être certain \\[0pt]
Être, c'est agir \\[0pt]
Être chez soi \\[0pt]
Être citoyen \\[0pt]
Être citoyen du monde \\[0pt]
Être civilisé \\[0pt]
Être compris \\[0pt]
Être conformiste \\[0pt]
Être conscient de soi, est-ce être maître de soi ? \\[0pt]
Être conscient, est-ce être maître de soi ? \\[0pt]
Être conséquent \\[0pt]
Être conséquent avec soi-même \\[0pt]
Être content de soi \\[0pt]
« Être contre » \\[0pt]
Être cultivé, est-ce tout connaître ? \\[0pt]
Être cultivé rend-il meilleur ? \\[0pt]
Être cynique \\[0pt]
Être dans l'esprit \\[0pt]
Être dans le temps \\[0pt]
Être dans le vrai \\[0pt]
Être dans l'opposition \\[0pt]
Être dans son bon droit \\[0pt]
Être dans son droit \\[0pt]
Être de mauvaise humeur \\[0pt]
Être démocrate \\[0pt]
Être de son temps \\[0pt]
Être déterminé \\[0pt]
Être dogmatique \\[0pt]
Être éduqué, est-ce devenir soi-même ? \\[0pt]
Être égal à soi-même \\[0pt]
Être égaux \\[0pt]
Être en bonne santé \\[0pt]
Être en désaccord \\[0pt]
Être en dette \\[0pt]
Être en paix \\[0pt]
Être en puissance \\[0pt]
Être enraciné \\[0pt]
Être en règle avec soi-même \\[0pt]
Être ensemble \\[0pt]
Être équitable \\[0pt]
Être, est-ce agir ? \\[0pt]
Être et apparaître \\[0pt]
Être et avoir \\[0pt]
Être et avoir été \\[0pt]
Être et devenir \\[0pt]
Être et devoir être \\[0pt]
Être et devoir-être \\[0pt]
Être et être pensé \\[0pt]
Être et être perçu \\[0pt]
Être et exister \\[0pt]
Être et ne plus être \\[0pt]
Être et paraître \\[0pt]
Être et penser, est-ce la même chose ? \\[0pt]
Être et représentation \\[0pt]
Être et sens \\[0pt]
Être exemplaire \\[0pt]
Être fidèle \\[0pt]
Être heureux \\[0pt]
Être heureux, est-ce devoir ? \\[0pt]
Être hors de soi \\[0pt]
Être hors la loi \\[0pt]
Être hors-la-loi \\[0pt]
Être identique \\[0pt]
Être impossible \\[0pt]
Être inspiré \\[0pt]
Être juge et partie \\[0pt]
Être là \\[0pt]
Être l'entrepreneur de soi-même \\[0pt]
Être libre, cela s'apprend-il ? \\[0pt]
Être libre, est-ce avoir le choix ? \\[0pt]
Être libre, est-ce dire non ? \\[0pt]
Être libre, est-ce échapper aux prévisions ? \\[0pt]
Être libre, est-ce faire ce que l'on veut ? \\[0pt]
Être libre, est-ce n'avoir aucun maître ? \\[0pt]
Être libre, est-ce n'obéir qu'à soi-même ? \\[0pt]
Être libre, est-ce pouvoir choisir ? \\[0pt]
Être libre, est-ce se suffire à soi-même ? \\[0pt]
Être libre, est-ce une question de volonté ? \\[0pt]
Être libre, est-ce vivre comme on l'entend ? \\[0pt]
Être libre, même dans les fers \\[0pt]
Être logique \\[0pt]
Être logique avec soi-même \\[0pt]
Être maître de soi \\[0pt]
Être majeur \\[0pt]
Être malade \\[0pt]
Être matérialiste \\[0pt]
Être méchant \\[0pt]
Être méchant volontairement \\[0pt]
Être membre de L'État \\[0pt]
Être mère \\[0pt]
Être méthodique \\[0pt]
Être moderne \\[0pt]
Être mortel \\[0pt]
Être naturel \\[0pt]
Être né \\[0pt]
« Être négatif » \\[0pt]
Être normal \\[0pt]
Être objectif \\[0pt]
Être ou avoir \\[0pt]
Être ou ne pas être \\[0pt]
Être ou ne pas être ? \\[0pt]
Être ou ne pas être, est-ce la question ? \\[0pt]
Être par soi \\[0pt]
Être par soi / être par autre chose \\[0pt]
Être passionné \\[0pt]
Être patient \\[0pt]
Être pauvre \\[0pt]
Être père \\[0pt]
Être poli \\[0pt]
Être politique \\[0pt]
Être possible \\[0pt]
Être pragmatique \\[0pt]
Être précurseur \\[0pt]
Être présent \\[0pt]
Être quelqu'un \\[0pt]
Être raisonnable, est-ce accepter la réalité telle qu'elle est ? \\[0pt]
Être raisonnable, est-ce renoncer à ses désirs ? \\[0pt]
Être réaliste \\[0pt]
Être relativiste \\[0pt]
Être religieux, est-ce nécessairement être dogmatique ? \\[0pt]
Être riche \\[0pt]
Être sans cause \\[0pt]
Être sans cœur \\[0pt]
Être sans foi ni loi \\[0pt]
Être sans scrupule \\[0pt]
Être sceptique \\[0pt]
« Être » se dit-il en plusieurs sens ? \\[0pt]
Être sensible \\[0pt]
Être seul avec sa conscience \\[0pt]
Être seul avec soi-même \\[0pt]
Être soi \\[0pt]
Être soi-même \\[0pt]
« Être soi-même », cela a-t-il un sens ? \\[0pt]
Être solidaire \\[0pt]
Être spectateur \\[0pt]
Être spirituel \\[0pt]
Être spontané, est-ce être libre ? \\[0pt]
Être sujet \\[0pt]
Être systématique \\[0pt]
Être un artiste \\[0pt]
Être un corps \\[0pt]
Être une chose qui pense \\[0pt]
Être une personne \\[0pt]
Être un sujet, est-ce être maître de soi ? \\[0pt]
Être vertueux \\[0pt]
Être, vie et pensée \\[0pt]
Être vulgaire \\[0pt]
Étudier \\[0pt]
Évaluer \\[0pt]
Évaluer l'art \\[0pt]
Événement et structure \\[0pt]
Évidence et certitude \\[0pt]
Évidence et raison \\[0pt]
Évidence et vérité \\[0pt]
Évidences et préjugés \\[0pt]
Évolution biologique et culture \\[0pt]
Évolution et progrès \\[0pt]
Évolution et révolution \\[0pt]
Exactitude et objectivité \\[0pt]
Excuser et pardonner \\[0pt]
Existe-il des mentalités primitives ? \\[0pt]
Existence et contingence \\[0pt]
Existence et essence \\[0pt]
Existence et religion \\[0pt]
Existence et transcendance \\[0pt]
Exister \\[0pt]
Exister, est-ce simplement vivre ? \\[0pt]
Exister hors du temps \\[0pt]
Existe-t-il au moins un devoir universel ? \\[0pt]
Existe-t-il dans le monde des réalités identiques ? \\[0pt]
Existe-t-il de faux besoins ? \\[0pt]
Existe-t-il des autorités en matière d'art ? \\[0pt]
Existe-t-il des choses en soi ? \\[0pt]
Existe-t-il des choses réellement sublimes ? \\[0pt]
Existe-t-il des choses sans prix ? \\[0pt]
Existe-t-il des comportements contraires à la nature ? \\[0pt]
Existe-t-il des connaissances a priori ? \\[0pt]
Existe-t-il des croyances collectives ? \\[0pt]
Existe-t-il des degrés de vérité ? \\[0pt]
Existe-t-il des démonstrations métaphysiques ? \\[0pt]
Existe-t-il des désirs coupables ? \\[0pt]
Existe-t-il des devoirs envers soi-même ? \\[0pt]
Existe-t-il des dilemmes moraux ? \\[0pt]
Existe-t-il des émotions proprement esthétiques ? \\[0pt]
Existe-t-il des états mentaux collectifs ? \\[0pt]
Existe-t-il des expériences métaphysiques ? \\[0pt]
Existe-t-il des identités collectives ? \\[0pt]
Existe-t-il des intuitions métaphysiques ? \\[0pt]
Existe-t-il des langages naturels ? \\[0pt]
Existe-t-il des mondes sociaux ? \\[0pt]
Existe-t-il des paroles vraies ? \\[0pt]
Existe-t-il des plaisirs purs ? \\[0pt]
Existe-t-il des preuves irréfutables ? \\[0pt]
Existe-t-il des principes premiers ? \\[0pt]
Existe-t-il des problèmes insolubles ? \\[0pt]
Existe-t-il des questions sans réponse ? \\[0pt]
Existe-t-il des sciences de différentes natures ? \\[0pt]
Existe-t-il des signes naturels ? \\[0pt]
Existe-t-il des sujets collectifs ? \\[0pt]
Existe-t-il des violences légitimes ? \\[0pt]
Existe-t-il différentes sortes de sciences ? \\[0pt]
Existe-t-il plusieurs déterminismes ? \\[0pt]
Existe-t-il plusieurs mondes ? \\[0pt]
Existe-t-il un art de la parole ? \\[0pt]
Existe-t-il un art de penser ? \\[0pt]
Existe-t-il un bien commun qui soit la norme de la vie politique ? \\[0pt]
Existe-t-il un bien souverain ? \\[0pt]
Existe-t-il un déterminisme social ? \\[0pt]
Existe-t-il un droit de mentir ? \\[0pt]
Existe-t-il une autorité naturelle ? \\[0pt]
Existe-t-il une hiérarchie des êtres ? \\[0pt]
Existe-t-il une histoire du présent ? \\[0pt]
Existe-t-il une méthode pour rechercher la vérité ? \\[0pt]
Existe-t-il une méthode pour trouver la vérité ? \\[0pt]
Existe-t-il une opinion publique ? \\[0pt]
Existe-t-il une réalité subjective ? \\[0pt]
Existe-t-il une réalité symbolique ? \\[0pt]
Existe-t-il une science de la morale ? \\[0pt]
Existe-t-il une science de l'être ? \\[0pt]
Existe-t-il une sensibilité philosophique ? \\[0pt]
Existe-t-il une unité des arts ? \\[0pt]
Existe-t-il une violence symbolique ? \\[0pt]
Existe-t-il une vision scientifique du monde ? \\[0pt]
Existe-t-il un vocabulaire neutre des droits fondamentaux ? \\[0pt]
Expérience esthétique et sens commun \\[0pt]
Expérience et approximation \\[0pt]
Expérience et expérimentation \\[0pt]
Expérience et habitude \\[0pt]
Expérience et interprétation \\[0pt]
Expérience et jugement \\[0pt]
Expérience et phénomène \\[0pt]
Expérience et vérité \\[0pt]
Expérience, expérimentation \\[0pt]
Expérience immédiate et expérimentation scientifique \\[0pt]
Expérimentation et vérification \\[0pt]
Expérimenter \\[0pt]
Explication causale, explication génétique dans les sciences humaines \\[0pt]
Explication et compréhension \\[0pt]
Explication et prévision \\[0pt]
Expliquer \\[0pt]
Expliquer, est-ce excuser ? \\[0pt]
Expliquer, est-ce interpréter ? \\[0pt]
Expliquer, est-ce justifier ? \\[0pt]
Expliquer et comprendre \\[0pt]
Expliquer et interpréter \\[0pt]
Expliquer et justifier \\[0pt]
Expliquer, justifier comprendre \\[0pt]
« Expliquer les faits sociaux par des faits sociaux » \\[0pt]
Exploiter la nature, est-ce lui vouloir du mal ? \\[0pt]
Expression et création \\[0pt]
Expression et signification \\[0pt]
Extension et compréhension \\[0pt]
Fabriquer et créer \\[0pt]
Faire apprendre \\[0pt]
Faire autorité \\[0pt]
Faire ce que l'on dit \\[0pt]
Faire ce qu'on dit \\[0pt]
Faire comme les autres \\[0pt]
Faire comme si \\[0pt]
Faire confiance \\[0pt]
Faire corps \\[0pt]
Faire croire \\[0pt]
Faire de la métaphysique, est-ce se détourner du monde ? \\[0pt]
Faire de la politique \\[0pt]
Faire de nécessité vertu \\[0pt]
Faire de sa vie une œuvre d'art \\[0pt]
Faire des choix \\[0pt]
Faire douter \\[0pt]
Faire école \\[0pt]
Faire est-il nécessairement savoir faire ? \\[0pt]
Faire et agir \\[0pt]
Faire et laisser faire \\[0pt]
Faire justice \\[0pt]
Faire la loi \\[0pt]
Faire la morale \\[0pt]
« Faire la paix » \\[0pt]
Faire la paix \\[0pt]
Faire la part des choses \\[0pt]
Faire la révolution \\[0pt]
Faire la vérité \\[0pt]
Faire le bien \\[0pt]
Faire le mal \\[0pt]
Faire le nécessaire \\[0pt]
Faire l'histoire \\[0pt]
Faire preuve d'humanité \\[0pt]
Faire sa vie \\[0pt]
Faire ses preuves \\[0pt]
Faire société \\[0pt]
Faire son devoir \\[0pt]
Faire son devoir, est-ce faire son propre malheur ? \\[0pt]
Faire son devoir, est-ce là toute la morale ? \\[0pt]
Faire son devoir, est-ce obéir ? \\[0pt]
Faire son devoir, est-ce payer une dette ? \\[0pt]
Faire son devoir, est-ce perdre toute sensibilité ? \\[0pt]
Faire son devoir, est-ce se soumettre ? \\[0pt]
Faire son possible \\[0pt]
Faire table rase \\[0pt]
Faire une expérience \\[0pt]
Faire voir \\[0pt]
Faisons-nous l'histoire ? \\[0pt]
Faisons-nous l'Histoire ? \\[0pt]
Faisons-nous notre propre malheur ? \\[0pt]
Fait et essence \\[0pt]
Fait et fiction \\[0pt]
Fait et preuve \\[0pt]
Fait et théorie \\[0pt]
Fait et valeur \\[0pt]
Fait-on de la politique pour changer les choses ? \\[0pt]
Faits et preuves \\[0pt]
Faits et valeurs \\[0pt]
Falsifier \\[0pt]
Famille et société \\[0pt]
Famille et tribu \\[0pt]
Familles, je vous hais \\[0pt]
Faudrait-il bannir la polysémie du langage ? \\[0pt]
Faudrait-il ne rien oublier ? \\[0pt]
Faudrait-il vivre sans passion ? \\[0pt]
Faut-avoir peur de la technique ? \\[0pt]
Faut-il accepter sa condition ? \\[0pt]
Faut-il accorder de l'importance aux mots ? \\[0pt]
Faut-il accorder l'esprit aux bêtes ? \\[0pt]
Faut-il affirmer son identité ? \\[0pt]
Faut-il aimer autrui pour le respecter ? \\[0pt]
Faut-il aimer la nature plus que soi-même ? \\[0pt]
Faut-il aimer la vie ? \\[0pt]
Faut-il aimer le destin ? \\[0pt]
Faut-il aimer son prochain ? \\[0pt]
Faut-il aimer son prochain comme soi-même ? \\[0pt]
Faut-il aller au-delà des apparences ? \\[0pt]
Faut-il aller toujours plus vite ? \\[0pt]
Faut-il apprendre à être libre ? \\[0pt]
Faut-il apprendre à obéir ? \\[0pt]
Faut-il apprendre à vivre en renonçant au bonheur ? \\[0pt]
Faut-il apprendre à voir ? \\[0pt]
Faut-il assimiler la pensée au langage ? \\[0pt]
Faut-il avoir des ennemis ? \\[0pt]
Faut-il avoir des principes ? \\[0pt]
Faut-il avoir foi en la raison ? \\[0pt]
Faut-il avoir foi en la technique ? \\[0pt]
Faut-il avoir peur de la liberté ? \\[0pt]
Faut-il avoir peur de la nature ? \\[0pt]
Faut-il avoir peur de la technique ? \\[0pt]
Faut-il avoir peur de l'avenir ? \\[0pt]
Faut-il avoir peur de la vérité ? \\[0pt]
Faut-il avoir peur des contradictions ? \\[0pt]
Faut-il avoir peur des habitudes ? \\[0pt]
Faut-il avoir peur des machines ? \\[0pt]
Faut-il avoir peur d'être libre ? \\[0pt]
Faut-il avoir peur du désordre ? \\[0pt]
Faut-il changer le monde ? \\[0pt]
Faut-il changer ses désirs plutôt que l'ordre du monde ? \\[0pt]
Faut-il chasser le naturel ? \\[0pt]
Faut-il chasser les poètes ? \\[0pt]
Faut-il chercher à être heureux ? \\[0pt]
Faut-il chercher à satisfaire tous nos désirs ? \\[0pt]
Faut-il chercher à se connaître ? \\[0pt]
Faut-il chercher à tout démontrer ? \\[0pt]
Faut-il chercher la paix à tout prix ? \\[0pt]
Faut-il chercher le bonheur à tout prix ? \\[0pt]
Faut-il chercher un sens à l'histoire ? \\[0pt]
Faut-il choisir entre être heureux et être libre ? \\[0pt]
Faut-il combattre ses illusions ? \\[0pt]
Faut-il comprendre pour croire ? \\[0pt]
Faut-il concilier les contraires ? \\[0pt]
Faut-il condamner la fiction ? \\[0pt]
Faut-il condamner la rhétorique ? \\[0pt]
Faut-il condamner le luxe ? \\[0pt]
Faut-il condamner les illusions ? \\[0pt]
Faut-il connaître la vérité pour pouvoir mentir ? \\[0pt]
Faut-il connaître l'histoire des sciences ? \\[0pt]
Faut-il connaître l'Histoire pour gouverner ? \\[0pt]
Faut-il considérer le droit pénal comme instituant une violence légitime ? \\[0pt]
Faut-il considérer les faits sociaux comme des choses ? \\[0pt]
Faut-il contrôler les mœurs ? \\[0pt]
Faut-il craindre de perdre son temps ? \\[0pt]
Faut-il craindre la démocratie ? \\[0pt]
Faut-il craindre la mort ? \\[0pt]
Faut-il craindre la révolution ? \\[0pt]
Faut-il craindre la tyrannie du bonheur ? \\[0pt]
Faut-il craindre le développement des techniques ? \\[0pt]
Faut-il craindre le pire ? \\[0pt]
Faut-il craindre le pouvoir des machines ? \\[0pt]
Faut-il craindre le pouvoir souverain ? \\[0pt]
Faut-il craindre le regard d'autrui ? \\[0pt]
Faut-il craindre les foules ? \\[0pt]
Faut-il craindre les machines ? \\[0pt]
Faut-il craindre les masses ? \\[0pt]
Faut-il craindre l'État ? \\[0pt]
Faut-il craindre l'ordre ? \\[0pt]
Faut-il croire au progrès ? \\[0pt]
Faut-il croire en la science ? \\[0pt]
Faut-il croire en quelque chose ? \\[0pt]
Faut-il croire les historiens ? \\[0pt]
Faut-il croire que l'histoire a un sens ? \\[0pt]
Faut-il cultiver le naturel ? \\[0pt]
Faut-il défendre la démocratie ? \\[0pt]
Faut-il défendre les faibles ? \\[0pt]
Faut-il défendre l'ordre à tout prix ? \\[0pt]
Faut-il défendre ses convictions \\[0pt]
Faut-il dépasser les apparences ? \\[0pt]
Faut-il désespérer de l'humanité ? \\[0pt]
Faut-il des frontières ? \\[0pt]
Faut-il des héros ? \\[0pt]
Faut-il désirer la vérité ? \\[0pt]
Faut-il des outils pour penser ? \\[0pt]
Faut-il des techniques du raisonnement ? \\[0pt]
Faut-il détruire l'État ? \\[0pt]
Faut-il détruire pour créer ? \\[0pt]
Faut-il dire de la justice qu'elle n'existe pas ? \\[0pt]
Faut-il dire tout haut ce que les autres pensent tout bas ? \\[0pt]
Faut-il diriger l'économie ? \\[0pt]
Faut-il distinguer ce qui est de ce qui doit être ? \\[0pt]
Faut-il distinguer culture et civilisation ? \\[0pt]
Faut-il distinguer désir et besoin ? \\[0pt]
Faut-il distinguer devoir moral et obligation sociale ? \\[0pt]
Faut-il distinguer esthétique et philosophie de l'art ? \\[0pt]
Faut-il donner un sens à la souffrance ? \\[0pt]
Faut-il douter de ce qu'on ne peut pas démontrer ? \\[0pt]
Faut-il douter de l'évidence \\[0pt]
Faut-il douter de tout ? \\[0pt]
Faut-il du passé faire table rase ? \\[0pt]
Faut-il écouter sa conscience ? \\[0pt]
Faut-il enfermer ? \\[0pt]
Faut-il enfermer les œuvres dans les musées ? \\[0pt]
Faut-il en finir avec l'esthétique ? \\[0pt]
Faut-il espérer pour agir ? \\[0pt]
Faut-il être à l'écoute du corps ? \\[0pt]
Faut-il être bon ? \\[0pt]
Faut-il être cohérent ? \\[0pt]
Faut-il être connaisseur pour apprécier une œuvre d'art ? \\[0pt]
Faut-il être cosmopolite ? \\[0pt]
Faut-il être courageux pour être libre ? \\[0pt]
Faut-il être discipliné ? \\[0pt]
Faut-il être fidèle à soi-même ? \\[0pt]
Faut-il être heureux ? \\[0pt]
Faut-il être idéaliste ? \\[0pt]
Faut-il être libre pour être heureux ? \\[0pt]
Faut-il être logique avec soi-même ? \\[0pt]
Faut-il être mesuré ? \\[0pt]
Faut-il être mesuré en toutes choses ? \\[0pt]
Faut-il être modéré ? \\[0pt]
Faut-il être naturel ? \\[0pt]
Faut-il être objectif ? \\[0pt]
Faut-il être original ? \\[0pt]
Faut-il être positif ? \\[0pt]
Faut-il être pragmatique ? \\[0pt]
Faut-il être réaliste ? \\[0pt]
Faut-il être réaliste en politique ? \\[0pt]
Faut-il être relativiste ? \\[0pt]
Faut-il expliquer la morale par son utilité ? \\[0pt]
Faut-il faire confiance au progrès technique ? \\[0pt]
Faut-il faire de la philosophie ? \\[0pt]
Faut-il faire de nécessité vertu ? \\[0pt]
Faut-il faire de sa vie une œuvre d'art ? \\[0pt]
Faut-il faire des efforts pour être soi-même ? \\[0pt]
Faut-il faire table rase du passé ? \\[0pt]
Faut-il faire une place à la croyance ? \\[0pt]
Faut-il forcer les gens à participer à la vie politique ? \\[0pt]
Faut-il fuir la politique ? \\[0pt]
Faut-il garder ses illusions ? \\[0pt]
Faut-il hiérarchiser les désirs ? \\[0pt]
Faut-il hiérarchiser les formes de vie ? \\[0pt]
Faut-il hiérarchiser ses désirs ? \\[0pt]
Faut-il imaginer que nous sommes heureux ? \\[0pt]
Faut-il imposer la vérité ? \\[0pt]
Faut-il interpréter la loi ? \\[0pt]
Faut-il joindre l'utile à l'agréable ? \\[0pt]
Faut-il laisser faire la nature ? \\[0pt]
Faut-il laisser parler la nature ? \\[0pt]
Faut-il libérer la parole ? \\[0pt]
Faut-il libérer l'humanité du travail ? \\[0pt]
Faut-il limiter la souveraineté ? \\[0pt]
Faut-il limiter la souveraineté de l'État ? \\[0pt]
Faut-il limiter le pouvoir de l'État ? \\[0pt]
Faut-il limiter les prétentions de la science ? \\[0pt]
Faut-il limiter l'exercice de la puissance publique ? \\[0pt]
Faut-il lire des romans ? \\[0pt]
Faut-il maîtriser ses émotions ? \\[0pt]
Faut-il mathématiser le réel pour le connaître ? \\[0pt]
Faut-il ménager les apparences ? \\[0pt]
Faut-il mépriser le luxe ? \\[0pt]
Faut-il mériter son bonheur ? \\[0pt]
Faut-il mieux vivre comme si nous ne devions jamais mourir ? \\[0pt]
Faut-il ne manquer de rien pour être heureux ? \\[0pt]
Faut-il n'être jamais méchant ? \\[0pt]
Faut-il obéir à la voix de sa conscience ? \\[0pt]
Faut-il opposer à la politique la souveraineté du droit ? \\[0pt]
Faut-il opposer culture et nature ? \\[0pt]
Faut-il opposer droits et devoirs ? \\[0pt]
Faut-il opposer éduquer et instruire ? \\[0pt]
Faut-il opposer histoire et mémoire ? \\[0pt]
Faut-il opposer la foi et la raison ? \\[0pt]
Faut-il opposer la matière et l'esprit ? \\[0pt]
Faut-il opposer l'art à la connaissance ? \\[0pt]
Faut-il opposer la société et l'État ? \\[0pt]
Faut-il opposer la théorie et la pratique ? \\[0pt]
Faut-il opposer la vie à la mort ? \\[0pt]
Faut-il opposer le devoir-être à l'être ? \\[0pt]
Faut-il opposer le don et l'échange ? \\[0pt]
Faut-il opposer le réel et l'imaginaire ? \\[0pt]
Faut-il opposer l'esprit et la matière ? \\[0pt]
Faut-il opposer l'État et la société ? \\[0pt]
Faut-il opposer le temps vécu et le temps des choses ? \\[0pt]
Faut-il opposer l'histoire et la fiction ? \\[0pt]
Faut-il opposer nature et culture ? \\[0pt]
Faut-il opposer produire et créer ? \\[0pt]
Faut-il opposer raison et sensation ? \\[0pt]
Faut-il opposer rhétorique et philosophie ? \\[0pt]
Faut-il opposer science et croyance ? \\[0pt]
Faut-il opposer science et métaphysique ? \\[0pt]
Faut-il opposer sciences humaines et sciences de la nature ? \\[0pt]
Faut-il opposer sens et vérité ? \\[0pt]
Faut-il opposer subjectivité et objectivité ? \\[0pt]
Faut-il opposer technique et nature ? \\[0pt]
Faut-il opposer théorie et expérience ? \\[0pt]
Faut-il oublier le passé ? \\[0pt]
Faut-il oublier le passé pour se donner un avenir ? \\[0pt]
Faut-il pardonner ? \\[0pt]
Faut-il parfois désobéir aux lois ? \\[0pt]
Faut-il parfois sacrifier la vérité ? \\[0pt]
Faut-il parler pour avoir des idées générales ? \\[0pt]
Faut-il partager la souveraineté ? \\[0pt]
Faut-il penser l'État comme un corps ? \\[0pt]
Faut-il penser l'État comme un individu ? \\[0pt]
Faut-il perdre ses illusions ? \\[0pt]
Faut-il perdre son temps ? \\[0pt]
Faut-il poser des limites à l'activité rationnelle ? \\[0pt]
Faut-il pour le connaître faire du vivant un objet ? \\[0pt]
Faut-il préférer la liberté à l'égalité ? \\[0pt]
Faut-il préférer l'art à la nature ? \\[0pt]
Faut-il préférer le bonheur à la vérité ? \\[0pt]
Faut-il préférer l'injustice au désordre ? \\[0pt]
Faut-il préférer une injustice au désordre ? \\[0pt]
Faut-il prendre soin de soi ? \\[0pt]
Faut-il privilégier le naturel ? \\[0pt]
Faut-il protéger la dignité humaine ? \\[0pt]
Faut-il protéger la nature ? \\[0pt]
Faut-il protéger les faibles contre les forts ? \\[0pt]
Faut-il que le réel ait un sens ? \\[0pt]
Faut-il que les meilleurs gouvernent ? \\[0pt]
Faut-il rechercher la certitude ? \\[0pt]
Faut-il rechercher la perfection dans le langage ? \\[0pt]
Faut-il rechercher la simplicité ? \\[0pt]
Faut-il rechercher le bonheur ? \\[0pt]
Faut-il rechercher l'harmonie ? \\[0pt]
Faut-il reconnaître pour connaître ? \\[0pt]
Faut-il regretter l'équivocité du langage ? \\[0pt]
Faut-il réguler la technique ? \\[0pt]
Faut-il rejeter tous les préjugés ? \\[0pt]
Faut-il rejeter toute norme ? \\[0pt]
Faut-il renoncer à connaître la nature des choses ? \\[0pt]
Faut-il renoncer à croire ? \\[0pt]
Faut-il renoncer à faire du travail une valeur ? \\[0pt]
Faut-il renoncer à la certitude ? \\[0pt]
Faut-il renoncer à la distinction entre nature et culture ? \\[0pt]
Faut-il renoncer à la vérité ? \\[0pt]
Faut-il renoncer à l'idée d'âme ? \\[0pt]
Faut-il renoncer à l'idée de fondement ? \\[0pt]
Faut-il renoncer à l'impossible ? \\[0pt]
Faut-il renoncer à rechercher la vérité ? \\[0pt]
Faut-il renoncer à son désir ? \\[0pt]
Faut-il résister à la peur de mourir ? \\[0pt]
Faut-il respecter la nature ? \\[0pt]
Faut-il respecter la tradition ? \\[0pt]
Faut-il respecter la vie ? \\[0pt]
Faut-il respecter les convenances ? \\[0pt]
Faut-il respecter le vivant ? \\[0pt]
Faut-il restaurer les œuvres d'art ? \\[0pt]
Faut-il rester impartial ? \\[0pt]
Faut-il rester naturel ? \\[0pt]
Faut-il rester soi-même ? \\[0pt]
Faut-il rire ou pleurer ? \\[0pt]
Faut-il rompre avec le passé ? \\[0pt]
Faut-il s'adapter ? \\[0pt]
Faut-il s'adapter aux événements ? \\[0pt]
Faut-il s'affranchir des désirs ? \\[0pt]
Faut-il s'aimer pour vivre ensemble ? \\[0pt]
Faut-il s'aimer soi-même ? \\[0pt]
Faut-il s'attendre à l'inattendu ? \\[0pt]
Faut-il sauver des vies à tout prix ? \\[0pt]
Faut-il sauver les apparences ? \\[0pt]
Faut-il savoir ce que l'on veut ? \\[0pt]
Faut-il savoir désobéir ? \\[0pt]
Faut-il savoir mentir ? \\[0pt]
Faut-il savoir obéir pour gouverner ? \\[0pt]
Faut-il savoir pour agir ? \\[0pt]
Faut-il savoir prendre des risques ? \\[0pt]
Faut-il se chercher pour se trouver ? \\[0pt]
Faut-il se connaître pour être maître de soi ? \\[0pt]
Faut-il se contenter de peu ? \\[0pt]
Faut-il se cultiver ? \\[0pt]
Faut-il se délivrer de la peur ? \\[0pt]
Faut-il se délivrer des passions ? \\[0pt]
Faut-il se demander si l'homme est bon ou méchant par nature \\[0pt]
Faut-il se détacher du monde ? \\[0pt]
Faut-il s'efforcer d'être moins personnel ? \\[0pt]
Faut-il se fier à ce que l'on ressent ? \\[0pt]
Faut-il se fier à la majorité ? \\[0pt]
Faut-il se fier à sa propre raison ? \\[0pt]
Faut-il se fier au témoignage des sens ? \\[0pt]
Faut-il se fier aux apparences ? \\[0pt]
Faut-il se libérer de soi-même ? \\[0pt]
Faut-il se libérer du travail ? \\[0pt]
Faut-il se libérer pour être libre ? \\[0pt]
Faut-il se méfier de la technique ? \\[0pt]
Faut-il se méfier de l'écriture ? \\[0pt]
Faut-il se méfier de l'imagination ? \\[0pt]
Faut-il se méfier de l'inspiration ? \\[0pt]
Faut-il se méfier de l'intuition ? \\[0pt]
Faut-il se méfier des apparences ? \\[0pt]
Faut-il se méfier de ses désirs ? \\[0pt]
Faut-il se méfier des images ? \\[0pt]
Faut-il se méfier de soi-même ? \\[0pt]
Faut-il se méfier du progrès technique ? \\[0pt]
Faut-il se méfier du volontarisme politique ? \\[0pt]
Faut-il s'en remettre à l'État pour limiter le pouvoir de l'État ? \\[0pt]
Faut-il s'en tenir aux faits ? \\[0pt]
Faut-il séparer la science et la technique ? \\[0pt]
Faut-il séparer morale et politique ? \\[0pt]
Faut-il se poser des questions métaphysiques ? \\[0pt]
Faut-il se réjouir d'exister ? \\[0pt]
Faut-il se rendre à l'évidence ? \\[0pt]
Faut-il se résigner aux inégalités ? \\[0pt]
Faut-il se ressembler pour former une société ? \\[0pt]
Faut-il s'intéresser aux œuvres mineures ? \\[0pt]
Faut-il souhaiter la fin du travail ? \\[0pt]
Faut-il souhaiter une langue universelle ? \\[0pt]
Faut-il suivre la nature ? \\[0pt]
Faut-il suivre ses intuitions ? \\[0pt]
Faut-il supposer un énoncé vrai pour en comprendre le sens ? \\[0pt]
Faut-il surmonter son enfance ? \\[0pt]
Faut-il tolérer les intolérants ? \\[0pt]
Faut-il toujours avoir raison ? \\[0pt]
Faut-il toujours chercher à savoir ? \\[0pt]
Faut-il toujours dire la vérité ? \\[0pt]
Faut-il toujours être en accord avec soi-même ? \\[0pt]
Faut-il toujours éviter de se contredire ? \\[0pt]
Faut-il toujours faire son devoir ? \\[0pt]
Faut-il toujours garder espoir ? \\[0pt]
Faut-il toujours respecter ses engagements ? \\[0pt]
Faut-il tout critiquer ? \\[0pt]
Faut-il tout démontrer ? \\[0pt]
Faut-il tout interpréter ? \\[0pt]
Faut-il un commencement à tout ? \\[0pt]
Faut-il un corps pour penser ? \\[0pt]
Faut-il une guerre pour mettre fin à toutes les guerres ? \\[0pt]
Faut-il une méthode pour découvrir la vérité ? \\[0pt]
Faut-il une théorie de la connaissance ? \\[0pt]
Faut-il vaincre ses désirs plutôt que l'ordre du monde ? \\[0pt]
Faut-il vivre avec son temps ? \\[0pt]
Faut-il vivre comme si l'on ne devait jamais mourir ? \\[0pt]
Faut-il vivre comme si nous étions immortels ? \\[0pt]
Faut-il vivre comme si nous ne devions jamais mourir ? \\[0pt]
Faut-il vivre comme si on ne devait jamais mourir ? \\[0pt]
Faut-il vivre conformément à la nature ? \\[0pt]
Faut-il vivre dangereusement ? \\[0pt]
Faut-il vivre hors de la société pour être heureux ? \\[0pt]
Faut-il voir pour croire ? \\[0pt]
Faut-il vouloir changer le monde ? \\[0pt]
Faut-il vouloir être heureux ? \\[0pt]
Faut-il vouloir la paix ? \\[0pt]
Faut-il vouloir la paix de l'âme ? \\[0pt]
Faut-il vouloir la transparence ? \\[0pt]
Faut-il vouloir savoir ? \\[0pt]
Féminité et citoyenneté \\[0pt]
Fiction et mensonge \\[0pt]
Fiction et réalité \\[0pt]
Fiction et vérité \\[0pt]
Fiction et virtualité \\[0pt]
Figurer le pouvoir \\[0pt]
Finalité et causalité \\[0pt]
Fins et moyens \\[0pt]
Foi et bonne foi \\[0pt]
Foi et raison \\[0pt]
Foi et savoir \\[0pt]
Foi et superstition \\[0pt]
Folie et raison \\[0pt]
Folie et société \\[0pt]
Fonction et limites des modèles en sciences sociales \\[0pt]
Fonction et prédicat \\[0pt]
Fonder \\[0pt]
Fonder la justice \\[0pt]
Fonder la morale \\[0pt]
Fonder une cité \\[0pt]
Fonder une famille \\[0pt]
Force et droit \\[0pt]
Force et violence \\[0pt]
Forcer à être libre \\[0pt]
Forger des hypothèses \\[0pt]
Formaliser et axiomatiser \\[0pt]
Forme et contenu \\[0pt]
Forme et fonction \\[0pt]
Forme et matière \\[0pt]
Forme et rythme \\[0pt]
Former et éduquer \\[0pt]
Former les esprits \\[0pt]
Forme-t-on son esprit en transformant la matière ? \\[0pt]
Fuir la civilisation \\[0pt]
Fuir le monde \\[0pt]
Gagner \\[0pt]
Gagner sa vie \\[0pt]
Garder en mémoire \\[0pt]
Garder la mesure \\[0pt]
Généralité de la règle, contingence des faits \\[0pt]
Généralité et universalité \\[0pt]
Génération, fabrication, création \\[0pt]
Genèse et structure \\[0pt]
Génie et technique \\[0pt]
Genre et espèce \\[0pt]
Géographie et géométrie \\[0pt]
Gérer et gouverner \\[0pt]
Gestion et politique \\[0pt]
Gouvernement des hommes et administration des choses \\[0pt]
Gouvernement et société \\[0pt]
Gouverner \\[0pt]
Gouverner, administrer, gérer \\[0pt]
Gouverner, est-ce dominer ? \\[0pt]
Gouverner, est-ce prévoir ? \\[0pt]
Gouverner, est-ce régner ? \\[0pt]
Gouverner et se gouverner \\[0pt]
Gouverner la nature \\[0pt]
Gouverner les passions \\[0pt]
Grammaire et logique \\[0pt]
Grammaire et métaphysique \\[0pt]
Grammaire et philosophie \\[0pt]
Grandeur et décadence \\[0pt]
Grandir \\[0pt]
Grève et politique \\[0pt]
Groupe, classe, société \\[0pt]
Guérir \\[0pt]
Guerre et paix \\[0pt]
Guerre et politique \\[0pt]
Guerres justes et injustes \\[0pt]
Habiter \\[0pt]
Habiter le monde \\[0pt]
Habiter l'espace \\[0pt]
Habiter sur la terre \\[0pt]
Haïr \\[0pt]
Haïr la raison \\[0pt]
Hasard et destin \\[0pt]
Hériter \\[0pt]
Herméneutique et sciences humaines \\[0pt]
Hésiter \\[0pt]
Heureux les simples d'esprit \\[0pt]
Hiérarchiser les arts \\[0pt]
Hier a-t-il plus de réalité que demain ? \\[0pt]
Histoire des techniques et histoire de l'art \\[0pt]
Histoire et anthropologie \\[0pt]
Histoire et causalité \\[0pt]
Histoire et devenir \\[0pt]
Histoire et écriture \\[0pt]
Histoire et ethnologie \\[0pt]
Histoire et fiction \\[0pt]
Histoire et géographie \\[0pt]
Histoire et géographie font-elles couple ? \\[0pt]
Histoire et mémoire \\[0pt]
Histoire et morale \\[0pt]
Histoire et narration \\[0pt]
Histoire et politique \\[0pt]
Histoire et préhistoire \\[0pt]
Histoire et progrès \\[0pt]
Histoire et structure \\[0pt]
Histoire et violence \\[0pt]
Histoire individuelle et histoire collective \\[0pt]
Historicité et vérité \\[0pt]
Homo religiosus \\[0pt]
Honte, pudeur, embarras \\[0pt]
Humanisation et déshumanisation \\[0pt]
Humour et ironie \\[0pt]
Hypothèse et vérité \\[0pt]
Ici et maintenant \\[0pt]
Idéal et utopie \\[0pt]
Idée et concept \\[0pt]
Idée et réalité \\[0pt]
Identité et appartenance \\[0pt]
Identité et changement \\[0pt]
Identité et communauté \\[0pt]
Identité et différence \\[0pt]
Identité et égalité \\[0pt]
Identité et indiscernabilité \\[0pt]
Identité et représentation politique \\[0pt]
Identité et responsabilité \\[0pt]
Identité et ressemblance \\[0pt]
Idéologie et utopie \\[0pt]
Ignorer \\[0pt]
« Il faudrait rester des années entières pour contempler une telle œuvre » \\[0pt]
« Il faut de tout pour faire un monde » \\[0pt]
Il faut de tout pour faire un monde \\[0pt]
Illégalité et injustice \\[0pt]
Illusion et apparence \\[0pt]
Illusion et hallucination \\[0pt]
« Il ne lui manque que la parole » \\[0pt]
« Il n'y a pas de fait, il n'y a que des interprétations » \\[0pt]
Il y a \\[0pt]
« Il y a un temps pour tout » \\[0pt]
Image, concept et signe \\[0pt]
Image et concept \\[0pt]
Image et idée \\[0pt]
Image, fiction, illusion \\[0pt]
Image, signe, symbole \\[0pt]
Imaginaire et politique \\[0pt]
Imagination et conception \\[0pt]
Imagination et connaissance \\[0pt]
Imagination et culture \\[0pt]
Imagination et fantasme \\[0pt]
Imagination et perception \\[0pt]
Imagination et pouvoir \\[0pt]
Imagination et raison \\[0pt]
Imaginer \\[0pt]
Imaginer, est-ce créer ? \\[0pt]
Imitation et création \\[0pt]
Imitation et identification \\[0pt]
Imitation et représentation \\[0pt]
Imiter \\[0pt]
Imiter, est-ce copier ? \\[0pt]
Imiter, est-ce manquer d'imagination ? \\[0pt]
Imiter et créer \\[0pt]
Immoralité et amoralité \\[0pt]
Immortalité et éternité \\[0pt]
Improviser \\[0pt]
Incarner le pouvoir \\[0pt]
Incertitude et action \\[0pt]
Inconscient et déterminisme \\[0pt]
Inconscient et identité \\[0pt]
Inconscient et inconscience \\[0pt]
Inconscient et instinct \\[0pt]
Inconscient et langage \\[0pt]
Inconscient et liberté \\[0pt]
Inconscient et mythes \\[0pt]
Incréé / créé \\[0pt]
Indépendance et autonomie \\[0pt]
Indépendance et liberté \\[0pt]
Indépendant / dépendant \\[0pt]
Individualisme et égoïsme \\[0pt]
Individuation et identité \\[0pt]
Individu et citoyen \\[0pt]
Individu et communauté \\[0pt]
Individu et société \\[0pt]
Inégalités et différences \\[0pt]
Inégalités et discriminations \\[0pt]
Infini et indéfini \\[0pt]
Information et communication \\[0pt]
Information et opinion \\[0pt]
Innocence et ignorance \\[0pt]
Innocenter le devenir \\[0pt]
Instinct et morale \\[0pt]
Instruction et éducation \\[0pt]
Instruire et éduquer \\[0pt]
Intégration et assimilation \\[0pt]
Intégration et exclusion \\[0pt]
Intentions, plans et stratégies \\[0pt]
Interdire et prohiber \\[0pt]
Intérêt général et bien commun \\[0pt]
Intériorité et extériorité \\[0pt]
Interprétation et création \\[0pt]
Interprétation et perception \\[0pt]
Interprétation et scientificité \\[0pt]
Interpréter \\[0pt]
Interpréter, est-ce connaître ? \\[0pt]
Interpréter, est-ce renoncer à prouver ? \\[0pt]
Interpréter, est-ce renoncer à toute objectivité ? \\[0pt]
Interpréter, est-ce savoir ? \\[0pt]
Interpréter, est-ce un art ? \\[0pt]
Interpréter est-il subjectif ? \\[0pt]
Interpréter et comprendre \\[0pt]
Interpréter et expliquer \\[0pt]
Interpréter et formaliser dans les sciences humaines \\[0pt]
Interpréter et traduire \\[0pt]
Interpréter ou expliquer \\[0pt]
Interpréter une œuvre d'art \\[0pt]
Interprète-t-on à défaut de connaître ? \\[0pt]
Interroger \\[0pt]
Interroger et répondre \\[0pt]
Intuition et concept \\[0pt]
Intuition et déduction \\[0pt]
Intuition et intellection \\[0pt]
Invariance et pluralité des langues \\[0pt]
Inventer et découvrir \\[0pt]
Invention, création, découverte \\[0pt]
Invention et création \\[0pt]
Invention et découverte \\[0pt]
Invention et imitation \\[0pt]
« J'ai le droit » \\[0pt]
J'ai un corps \\[0pt]
Je \\[0pt]
Je est un autre \\[0pt]
« Je mens » \\[0pt]
Je mens \\[0pt]
« Je m'en veux » \\[0pt]
« Je n'ai pas voulu cela » \\[0pt]
« Je ne crois que ce que je vois » \\[0pt]
« Je ne l'ai pas fait exprès » \\[0pt]
Je ne l'ai pas fait exprès \\[0pt]
« Je ne voulais pas cela » : en quoi les sciences humaines permettent-elles de comprendre cette excuse ? \\[0pt]
« Je préfère une injustice à un désordre » \\[0pt]
Je sens donc je suis \\[0pt]
Je sens, donc je suis \\[0pt]
Je, tu, il \\[0pt]
Jouer \\[0pt]
Jouer son rôle \\[0pt]
Jouer un rôle \\[0pt]
Joue-t-on toujours un rôle ? \\[0pt]
Jouir sans entraves \\[0pt]
Jugement analytique et jugement synthétique \\[0pt]
Jugement de goût et jugement esthétique \\[0pt]
Jugement esthétique et jugement de valeur \\[0pt]
Jugement esthétique et objectivité \\[0pt]
Jugement et proposition \\[0pt]
Jugement et réflexion \\[0pt]
Jugement et vérité \\[0pt]
Jugement moral et jugement empirique \\[0pt]
Juger \\[0pt]
Juger en conscience \\[0pt]
Juger et connaître \\[0pt]
Juger et décider \\[0pt]
Juger et raisonner \\[0pt]
Juger et sentir \\[0pt]
Juge-t-on un acte ou son auteur ? \\[0pt]
Jusqu'à quel point la nature est-elle objet de science ? \\[0pt]
Jusqu'à quel point pouvons-nous juger autrui ? \\[0pt]
Jusqu'à quel point sommes-nous responsables de nos passions ? \\[0pt]
Jusqu'à quel point suis-je mon propre maître ? \\[0pt]
Jusqu'où interpréter ? \\[0pt]
Jusqu'où peut-on dialoguer ? \\[0pt]
Jusqu'où peut-on soigner ? \\[0pt]
Jusqu'où s'étend le domaine de la science ? \\[0pt]
Justice et bonheur \\[0pt]
Justice et charité \\[0pt]
Justice et égalité \\[0pt]
Justice et équité \\[0pt]
Justice et fiscalité \\[0pt]
Justice et force \\[0pt]
Justice et impartialité \\[0pt]
Justice et pardon \\[0pt]
Justice et ressentiment \\[0pt]
Justice et utilité \\[0pt]
Justice et vengeance \\[0pt]
Justice et violence \\[0pt]
Justice sociale et charité publique \\[0pt]
Justification et politique \\[0pt]
Justifier \\[0pt]
Justifier et prouver \\[0pt]
Justifier le mensonge \\[0pt]
La banalité \\[0pt]
L'abandon \\[0pt]
La barbarie \\[0pt]
La barbarie de la technique \\[0pt]
La barbarie peut-elle avoir un visage humain ? \\[0pt]
La barrière de la langue \\[0pt]
La bassesse \\[0pt]
La béatitude \\[0pt]
La beauté \\[0pt]
La beauté a-t-elle une histoire ? \\[0pt]
La beauté comporte-t-elle des degrés ? \\[0pt]
La beauté de la nature \\[0pt]
La beauté demande-t-elle des preuves ? \\[0pt]
La beauté des corps \\[0pt]
La beauté des ruines \\[0pt]
La beauté des villes \\[0pt]
La beauté du diable \\[0pt]
La beauté du geste \\[0pt]
La beauté du monde \\[0pt]
La beauté du vrai \\[0pt]
« La beauté est dans l'œil de celui qui regarde » \\[0pt]
La beauté est-elle affaire de goût ? \\[0pt]
La beauté est-elle dans le regard ou dans la chose vue ? \\[0pt]
La beauté est-elle dans les choses ? \\[0pt]
La beauté est-elle intemporelle ? \\[0pt]
La beauté est-elle l'objet d'une connaissance ? \\[0pt]
La beauté est-elle partout ? \\[0pt]
La beauté est-elle sensible ? \\[0pt]
La beauté est-elle une promesse de bonheur ? \\[0pt]
La beauté et la grâce \\[0pt]
La beauté et le mal \\[0pt]
La beauté et l'ennui \\[0pt]
La beauté idéale \\[0pt]
La beauté morale \\[0pt]
La beauté naturelle \\[0pt]
La beauté naturelle est-elle une catégorie esthétique périmée ? \\[0pt]
La beauté n'est-elle qu'un effet artistique ? \\[0pt]
La beauté n'est-elle qu'une idée ? \\[0pt]
La beauté nous rend-elle meilleurs ? \\[0pt]
La beauté peut-elle délivrer une vérité ? \\[0pt]
La beauté peut-elle être cachée ? \\[0pt]
La beauté peut-elle laisser indifférent ? \\[0pt]
La beauté s'explique-t-elle ? \\[0pt]
La belle âme \\[0pt]
La belle nature \\[0pt]
La bestialité \\[0pt]
La bête \\[0pt]
La bête et l'animal \\[0pt]
La bêtise \\[0pt]
La bêtise et la méchanceté sont-elles liées intrinsèquement ? \\[0pt]
La bêtise et la méchanceté sont-elles liées nécessairement ? \\[0pt]
La bêtise n'est-elle pas proprement humaine ? \\[0pt]
La bibliothèque \\[0pt]
La bienfaisance \\[0pt]
La bienséance \\[0pt]
La bienveillance \\[0pt]
La bioéthique \\[0pt]
La biographie \\[0pt]
La biologie peut-elle se passer de causes finales ? \\[0pt]
L'abnégation \\[0pt]
L'abondance \\[0pt]
La bonne conscience \\[0pt]
La bonne éducation \\[0pt]
La bonne intention \\[0pt]
La bonne volonté \\[0pt]
La bonté \\[0pt]
L'absence \\[0pt]
L'absence de fondement \\[0pt]
L'absence de générosité \\[0pt]
L'absence de preuves \\[0pt]
L'absence d'œuvre \\[0pt]
L'absolu \\[0pt]
L'absolu est-il connaissable ? \\[0pt]
L'Absolu est-il un objet de savoir ? \\[0pt]
L'absolu et le relatif \\[0pt]
L'abstention \\[0pt]
L'abstraction \\[0pt]
L'abstraction en art \\[0pt]
L'abstraction est-elle toujours utile à la science empirique ? \\[0pt]
L'abstrait est-il en dehors de l'espace et du temps ? \\[0pt]
L'abstrait et le concret \\[0pt]
L'abstrait et l'immatériel \\[0pt]
L'absurde \\[0pt]
L'absurde peut-il avoir un sens pour une pensée 155 \\[0pt]
La bureaucratie \\[0pt]
L'abus de langage \\[0pt]
L'abus de pouvoir \\[0pt]
L'abus du droit \\[0pt]
L'académisme \\[0pt]
L'académisme dans l'art \\[0pt]
L'académisme et les fins de l'art \\[0pt]
La calomnie \\[0pt]
La caricature \\[0pt]
La carte et le territoire \\[0pt]
La casuistique \\[0pt]
La catastrophe \\[0pt]
La catharsis \\[0pt]
La causalité \\[0pt]
La causalité dans les sciences humaines \\[0pt]
La causalité en histoire \\[0pt]
La causalité en sciences humaines \\[0pt]
La causalité historique \\[0pt]
La causalité suppose-t-elle des lois ? \\[0pt]
La cause \\[0pt]
La cause de soi \\[0pt]
La cause efficiente \\[0pt]
La cause et la raison \\[0pt]
La cause et l'effet \\[0pt]
La cause finale \\[0pt]
La cause formelle \\[0pt]
La cause motrice \\[0pt]
La cause première \\[0pt]
L'accélération du temps \\[0pt]
L'accès à la vérité \\[0pt]
L'accident \\[0pt]
L'accidentel \\[0pt]
L'accomplissement \\[0pt]
L'accomplissement de soi \\[0pt]
L'accord \\[0pt]
L'accord des esprits \\[0pt]
L'accord des esprits est-il un critère de vérité ? \\[0pt]
L'accord est-il le terme ou le principe du dialogue ? \\[0pt]
L'accusation \\[0pt]
La censure \\[0pt]
La certitude \\[0pt]
La certitude de mourir \\[0pt]
La certitude est-elle une marque de vérité ? \\[0pt]
La certitude ne peut-elle naître que de l'évidence ? \\[0pt]
La chair \\[0pt]
La chair et l'esprit \\[0pt]
La chance \\[0pt]
La charité \\[0pt]
La charité est-elle la négation de la justice ? \\[0pt]
La charité est-elle une vertu ? \\[0pt]
La chasse et la guerre \\[0pt]
L'achèvement de l'œuvre \\[0pt]
La chose \\[0pt]
La chose en soi \\[0pt]
La chose et l'objet \\[0pt]
La chose publique \\[0pt]
La chronologie \\[0pt]
La chute \\[0pt]
La circonspection \\[0pt]
La circonstance \\[0pt]
La citation \\[0pt]
La cité \\[0pt]
La cité idéale \\[0pt]
La cité sans dieux \\[0pt]
La citoyenneté \\[0pt]
La civilisation \\[0pt]
La civilité \\[0pt]
La clarté \\[0pt]
La clarté suffit-elle au savoir ? \\[0pt]
La classe moyenne \\[0pt]
La classification \\[0pt]
La classification des arts \\[0pt]
La classification des sciences \\[0pt]
La clause de conscience \\[0pt]
La clémence \\[0pt]
La coexistence des libertés \\[0pt]
La cohérence \\[0pt]
La cohérence des attributs divins \\[0pt]
La cohérence est-elle la norme du vrai ? \\[0pt]
La cohérence est-elle un critère de la vérité ? \\[0pt]
La cohérence est-elle un critère de vérité ? \\[0pt]
La cohérence est-elle une vertu ? \\[0pt]
La cohérence logique est-elle une condition suffisante de la démonstration ? \\[0pt]
La cohérence suffit-elle à la vérité ? \\[0pt]
La cohésion nationale \\[0pt]
La cohésion sociale \\[0pt]
La colère \\[0pt]
La colère peut-elle être justifiée ? \\[0pt]
La collection \\[0pt]
La comédie \\[0pt]
La comédie du pouvoir \\[0pt]
La comédie humaine \\[0pt]
La comédie sociale \\[0pt]
La communauté \\[0pt]
La communauté des individus \\[0pt]
La communauté des savants \\[0pt]
La communauté internationale \\[0pt]
La communauté morale \\[0pt]
La communauté scientifique \\[0pt]
La communication \\[0pt]
La communication est-elle nécessaire à la démocratie ? \\[0pt]
La comparaison \\[0pt]
La comparaison en sociologie \\[0pt]
La compassion \\[0pt]
La compassion risque-t-elle d'abolir l'exigence politique ? \\[0pt]
La compétence \\[0pt]
La compétence en politique \\[0pt]
La compétence fonde-t-elle la compétence juridique ? \\[0pt]
La compétence fonde-t-elle l'autorité politique ? \\[0pt]
La compétence politique \\[0pt]
La compétence technique peut-elle fonder l'autorité publique ? \\[0pt]
La composition \\[0pt]
La compréhension \\[0pt]
La concorde \\[0pt]
La concurrence \\[0pt]
La condition \\[0pt]
La condition de mortel \\[0pt]
La condition humaine \\[0pt]
La condition sociale \\[0pt]
La confiance \\[0pt]
La confiance en la raison \\[0pt]
La confiance est-elle une vertu ? \\[0pt]
La conflictualité politique est-elle indépassable ? \\[0pt]
La confusion \\[0pt]
La connaissance adéquate \\[0pt]
La connaissance animale \\[0pt]
La connaissance a-t-elle des limites ? \\[0pt]
La connaissance commune est-elle le point de départ de la science ? \\[0pt]
La connaissance commune fait-elle obstacle à la vérité ? \\[0pt]
La connaissance de Dieu \\[0pt]
La connaissance de l'amour \\[0pt]
La connaissance de la nécessité a priori peut-elle évoluer ? \\[0pt]
La connaissance de l'animal \\[0pt]
La connaissance de la vie \\[0pt]
La connaissance de la vie se confond-elle avec celle du vivant ? \\[0pt]
La connaissance de l'histoire est-elle utile à l'action ? \\[0pt]
La connaissance de l'histoire nous rend-elle plus libres ? \\[0pt]
La connaissance de l'individuel \\[0pt]
La connaissance de l'infini \\[0pt]
La connaissance des causes \\[0pt]
La connaissance des faits \\[0pt]
La connaissance de soi \\[0pt]
La connaissance de soi est-elle évidente ? \\[0pt]
La connaissance des passions \\[0pt]
La connaissance des principes \\[0pt]
La connaissance du bien \\[0pt]
La connaissance du futur \\[0pt]
La connaissance du monde \\[0pt]
La connaissance du passé \\[0pt]
La connaissance du passé est-elle nécessaire à la compréhension du présent ? \\[0pt]
La connaissance du singulier \\[0pt]
La connaissance du singulier dans les sciences humaines \\[0pt]
La connaissance du vivant \\[0pt]
La connaissance du vivant est-elle désintéressée ? \\[0pt]
La connaissance du vivant peut-elle être désintéressée ? \\[0pt]
La connaissance est-elle une contemplation ? \\[0pt]
La connaissance est-elle une croyance justifiée ? \\[0pt]
La connaissance et la croyance \\[0pt]
La connaissance et la morale \\[0pt]
La connaissance et le vivant \\[0pt]
La connaissance historique \\[0pt]
La connaissance historique est-elle une interprétation des faits ? \\[0pt]
La connaissance historique est-elle utile à l'homme ? \\[0pt]
La connaissance intuitive \\[0pt]
La connaissance mathématique \\[0pt]
La connaissance objective \\[0pt]
La connaissance objective doit-elle s'interdire toute interprétation ? \\[0pt]
La connaissance objective exclut-elle toute forme de subjectivité ? \\[0pt]
La connaissance peut-elle être pratique ? \\[0pt]
La connaissance peut-elle se passer de l'imagination ? \\[0pt]
La connaissance scientifique \\[0pt]
La connaissance scientifique abolit-elle toute croyance ? \\[0pt]
La connaissance scientifique est-elle désintéressée ? \\[0pt]
La connaissance scientifique est-elle fondée sur l'expérience ? \\[0pt]
La connaissance scientifique n'est-elle qu'une croyance argumentée ? \\[0pt]
La connaissance sensible \\[0pt]
La connaissance s'interdit-elle tout recours à l'imagination ? \\[0pt]
La connaissance suppose-t-elle une éthique ? \\[0pt]
La connexion des choses et la connexion des idées \\[0pt]
La conquête \\[0pt]
La conquête de l'espace \\[0pt]
La conquête du pouvoir \\[0pt]
La conscience \\[0pt]
La conscience animale \\[0pt]
La conscience a-t-elle des degrés ? \\[0pt]
La conscience a-t-elle des moments ? \\[0pt]
La conscience collective \\[0pt]
La conscience d'agir suffit-elle à garantir notre liberté ? \\[0pt]
La conscience d'autrui est-elle impénétrable ? \\[0pt]
La conscience de classe \\[0pt]
La conscience définit-elle l'homme en propre ? \\[0pt]
La conscience de la mort est-elle une condition de la sagesse ? \\[0pt]
La conscience des autres \\[0pt]
La conscience de soi \\[0pt]
La conscience de soi de l'art \\[0pt]
La conscience de soi est-elle une donnée immédiate ? \\[0pt]
La conscience de soi est-elle une méconnaissance de soi ? \\[0pt]
La conscience de soi et l'identité personnelle \\[0pt]
La conscience de soi suppose-t-elle autrui ? \\[0pt]
La conscience du temps rend-elle l'existence tragique ? \\[0pt]
La conscience entrave-t-elle l'action ? \\[0pt]
La conscience est-elle ce qui fait le sujet ? \\[0pt]
La conscience est-elle intrinsèquement morale ? \\[0pt]
La conscience est-elle le produit de la vie réelle ? \\[0pt]
La conscience est-elle nécessairement malheureuse ? \\[0pt]
La conscience est-elle ou n'est-elle pas ? \\[0pt]
La conscience est-elle source d'illusions ? \\[0pt]
La conscience est-elle temporelle ? \\[0pt]
La conscience est-elle toujours morale ? \\[0pt]
La conscience est-elle une activité ? \\[0pt]
La conscience est-elle une connaissance ? \\[0pt]
La conscience est-elle une illusion ? \\[0pt]
La conscience et l'inconscient \\[0pt]
La conscience historique \\[0pt]
La conscience malheureuse \\[0pt]
La conscience morale \\[0pt]
La conscience morale est-elle innée ? \\[0pt]
La conscience morale est-elle naturelle ? \\[0pt]
La conscience morale est-elle une illusion ? \\[0pt]
La conscience morale n'est-elle que le fruit de l'éducation ? \\[0pt]
La conscience morale n'est-elle que le produit de l'éducation ? \\[0pt]
La conscience nous leurre-t-elle ? \\[0pt]
La conscience peut-elle être collective ? \\[0pt]
La conscience peut-elle être objet de science ? \\[0pt]
La conscience peut-elle nous tromper ? \\[0pt]
La conscience politique \\[0pt]
La conscience universelle \\[0pt]
La conséquence \\[0pt]
La conservation \\[0pt]
La conservation de soi \\[0pt]
La considération \\[0pt]
La considération de l'utilité doit-elle déterminer toutes nos actions ? \\[0pt]
La consolation \\[0pt]
La consommation \\[0pt]
La constance \\[0pt]
La constitution \\[0pt]
La contemplation \\[0pt]
La contemplation de la nature \\[0pt]
La contestation \\[0pt]
La contingence \\[0pt]
La contingence de l'existence \\[0pt]
La contingence des lois de la nature \\[0pt]
La contingence du futur \\[0pt]
La contingence du mal \\[0pt]
La contingence du monde \\[0pt]
La contingence est-elle la condition de la liberté ? \\[0pt]
La contingence est-elle une condition de la liberté ? \\[0pt]
La continuité \\[0pt]
La contradiction \\[0pt]
La contradiction réside-t-elle dans les choses ? \\[0pt]
La contrainte \\[0pt]
La contrainte déontologique \\[0pt]
La contrainte des lois est-elle une violence ? \\[0pt]
La contrainte en art \\[0pt]
La contrainte morale \\[0pt]
La contrainte peut-elle être légitime ? \\[0pt]
La contrainte s'oppose-t-elle à la liberté ? \\[0pt]
La contrainte supprime-t-elle la responsabilité ? \\[0pt]
La contrôle social \\[0pt]
La controverse \\[0pt]
La controverse scientifique \\[0pt]
La convalescence \\[0pt]
La convention \\[0pt]
La convention et l'arbitraire \\[0pt]
La conversation \\[0pt]
La conversion \\[0pt]
La conviction \\[0pt]
La coopération \\[0pt]
La copie \\[0pt]
La correspondance des arts \\[0pt]
La corruption \\[0pt]
La corruption des mœurs \\[0pt]
La corruption politique \\[0pt]
La cosmogonie \\[0pt]
La couleur \\[0pt]
La couleur en peinture \\[0pt]
La couleur et le dessin \\[0pt]
La courtoisie \\[0pt]
La coutume \\[0pt]
L'acquis \\[0pt]
La crainte \\[0pt]
La crainte des Dieux \\[0pt]
« La crainte est le commencement de la sagesse » \\[0pt]
La crainte et l'ignorance \\[0pt]
La création \\[0pt]
La création artistique \\[0pt]
La création dans l'art \\[0pt]
La création de l'humanité \\[0pt]
La création de valeur \\[0pt]
La créativité \\[0pt]
La crédibilité \\[0pt]
La crédulité \\[0pt]
La criminalité \\[0pt]
La crise \\[0pt]
La crise du sens \\[0pt]
La crise sociale \\[0pt]
La critique \\[0pt]
La critique d'art \\[0pt]
La critique de l'art \\[0pt]
La critique de l'État \\[0pt]
La critique des bons sentiments \\[0pt]
La critique des théories \\[0pt]
La critique du pouvoir peut-elle conduire à la désobéissance ? \\[0pt]
« La critique est aisée » \\[0pt]
La critique et la crise \\[0pt]
La critique fait-elle partie de l'art ? \\[0pt]
La croissance \\[0pt]
La croissance du savoir \\[0pt]
La croyance \\[0pt]
La croyance en Dieu est-elle irrationnelle ? \\[0pt]
La croyance est-elle l'asile de l'ignorance ? \\[0pt]
La croyance est-elle signe de faiblesse ? \\[0pt]
La croyance est-elle une opinion ? \\[0pt]
La croyance est-elle une opinion comme les autres ? \\[0pt]
La croyance et la foi \\[0pt]
La croyance et la raison \\[0pt]
La croyance peut-elle être rationnelle ? \\[0pt]
La croyance peut-elle s'accompagner de certitude ? \\[0pt]
La croyance peut-elle tenir lieu de savoir ? \\[0pt]
La croyance religieuse échappe-t-elle à toute logique ? \\[0pt]
La croyance religieuse se distingue-t-elle des autres formes de croyance ? \\[0pt]
La cruauté \\[0pt]
La cruauté politique \\[0pt]
L'acte \\[0pt]
L'acte et la parole \\[0pt]
L'acte et la puissance \\[0pt]
L'acte et l'œuvre \\[0pt]
L'acte gratuit \\[0pt]
L'acte manqué \\[0pt]
L'acteur \\[0pt]
L'acteur et l'observateur dans les sciences humaines \\[0pt]
L'acteur et son rôle \\[0pt]
L'action \\[0pt]
L'action collective \\[0pt]
L'action du temps \\[0pt]
L'action et la passion \\[0pt]
L'action et le risque \\[0pt]
L'action et son contexte \\[0pt]
L'action humaine nécessite-t-elle la contingence du monde ? \\[0pt]
L'action intentionnelle \\[0pt]
L'action morale dépend-elle des circonstances ? \\[0pt]
L'action politique \\[0pt]
L'action politique a-t-elle un fondement rationnel ? \\[0pt]
L'action politique peut-elle se passer de mots ? \\[0pt]
L'action requiert-elle décision d'un sujet ? \\[0pt]
L'activité \\[0pt]
L'activité philosophique conduit-elle à la métaphysique ? \\[0pt]
L'activité se laisse-t-elle programmer ? \\[0pt]
L'activité technique dévalorise-t-elle l'homme ? \\[0pt]
L'actualité \\[0pt]
L'actuel \\[0pt]
La cuisine \\[0pt]
La culpabilité \\[0pt]
La culture \\[0pt]
La culture artistique \\[0pt]
La culture commence-t-elle avec le refus de la nature ? \\[0pt]
La culture de masse \\[0pt]
La culture démocratique \\[0pt]
La culture d'entreprise \\[0pt]
La culture engendre-t-elle le progrès ? \\[0pt]
La culture est-elle affaire de politique ? \\[0pt]
La culture est-elle ce qui nous relie au passé ? \\[0pt]
La culture est-elle ce qui unit ou ce qui sépare les hommes ? \\[0pt]
La culture est-elle la négation de la nature ? \\[0pt]
La culture est-elle l'objet naturel des sciences humaines ? \\[0pt]
La culture est-elle nécessaire à l'appréciation d'une œuvre d'art ? \\[0pt]
La culture est-elle une question politique ? \\[0pt]
La culture est-elle une seconde nature ? \\[0pt]
La culture est-elle un luxe ? \\[0pt]
La culture est-elle un objet ? \\[0pt]
La culture et le langage \\[0pt]
La culture et les cultures \\[0pt]
La culture garantit-elle l'excellence humaine ? \\[0pt]
La culture générale \\[0pt]
La culture historique forge-t-elle le jugement moral ? \\[0pt]
La culture libère-t-elle des préjugés ? \\[0pt]
La culture morale \\[0pt]
La culture nous rend-elle meilleurs ? \\[0pt]
La culture nous rend-elle plus humains ? \\[0pt]
La culture nous unit-elle ? \\[0pt]
La culture peut-elle être instituée ? \\[0pt]
La culture peut-elle être objet de science ? \\[0pt]
La culture populaire \\[0pt]
La culture : pour quoi faire ? \\[0pt]
La culture rend-elle plus humain ? \\[0pt]
La culture savante et la culture populaire \\[0pt]
La culture scientifique \\[0pt]
La culture technique \\[0pt]
La curiosité \\[0pt]
La curiosité est-elle à l'origine du savoir ? \\[0pt]
La danse \\[0pt]
La danse est-elle l'œuvre du corps ? \\[0pt]
La danse et l'espace \\[0pt]
L'adaptation \\[0pt]
La décadence \\[0pt]
La décence \\[0pt]
La déception \\[0pt]
La décision \\[0pt]
La décision a-t-elle besoin de raisons ? \\[0pt]
La décision morale \\[0pt]
La décision politique \\[0pt]
La découverte de la vérité peut-elle être le fait du hasard ? \\[0pt]
La découverte scientifique a-t-elle une logique ? \\[0pt]
La déduction \\[0pt]
La défense de la liberté \\[0pt]
La défense de l'intérêt général est-il la fin dernière de la politique ? \\[0pt]
La défense nationale \\[0pt]
La déficience \\[0pt]
La définition \\[0pt]
La délibération \\[0pt]
La délibération en morale \\[0pt]
La délibération politique \\[0pt]
La démagogie \\[0pt]
La démarche scientifique exclut-elle tout recours à l'imagination ? \\[0pt]
La démence \\[0pt]
La démesure \\[0pt]
La démocratie \\[0pt]
La démocratie a-t-elle des limites ? \\[0pt]
La démocratie a-t-elle une histoire ? \\[0pt]
La démocratie conduit-elle au règne de l'opinion ? \\[0pt]
La démocratie, est-ce la fin du despotisme ? \\[0pt]
La démocratie, est-ce le pouvoir du plus grand nombre ? \\[0pt]
La démocratie est-elle la loi du plus fort ? \\[0pt]
La démocratie est-elle le pire des régimes politiques ? \\[0pt]
La démocratie est-elle le règne de l'opinion ? \\[0pt]
La démocratie est-elle moyen ou fin ? \\[0pt]
La démocratie est-elle nécessairement libérale ? \\[0pt]
La démocratie est-elle possible ? \\[0pt]
La démocratie est-elle un mythe ? \\[0pt]
La démocratie et les experts \\[0pt]
La démocratie et les institutions de la justice \\[0pt]
La démocratie et le statut de la loi \\[0pt]
La démocratie n'est-elle que la force des faibles ? \\[0pt]
La démocratie participative \\[0pt]
La démocratie peut-elle échapper à la démagogie ? \\[0pt]
La démocratie peut-elle être représentative ? \\[0pt]
La démocratie peut-elle se passer de représentation ? \\[0pt]
La démonstration \\[0pt]
La démonstration mathématique peut-elle servir de modèle à toute démonstration ? \\[0pt]
La démonstration n'est-elle qu'une méthode d'exposition des découvertes ? \\[0pt]
La démonstration nous garantit-elle l'accès à la vérité ? \\[0pt]
La démonstration obéit-elle à des lois ? \\[0pt]
La démonstration par l'absurde \\[0pt]
La démonstration suffit-elle à établir la vérité ? \\[0pt]
La démonstration supprime-t-elle le doute ? \\[0pt]
La déontologie \\[0pt]
La dépendance \\[0pt]
La dépense \\[0pt]
L'adéquation aux choses suffit-elle à définir la vérité ? \\[0pt]
La déraison \\[0pt]
La dérision \\[0pt]
La descendance \\[0pt]
La description \\[0pt]
La désillusion \\[0pt]
La désinvolture \\[0pt]
La désobéissance \\[0pt]
La désobéissance civile \\[0pt]
La désobéissance peut-elle être légitime ? \\[0pt]
La destruction \\[0pt]
La détermination \\[0pt]
La dette \\[0pt]
La deuxième chance \\[0pt]
La déviance \\[0pt]
La dialectique \\[0pt]
La dialectique est-elle une science ? \\[0pt]
La dialectique remet-elle en question le principe de contradiction ? \\[0pt]
La dictature \\[0pt]
La différence \\[0pt]
La différence culturelle \\[0pt]
La différence de l'être et de l'étant \\[0pt]
La différence des arts \\[0pt]
La différence des sexes \\[0pt]
La différence des sexes est-elle une question philosophique ? \\[0pt]
La différence des sexes est-elle un problème philosophique ? \\[0pt]
La différence homme-femme \\[0pt]
La différence sexuelle \\[0pt]
La difformité \\[0pt]
La dignité \\[0pt]
La dignité de l'homme \\[0pt]
La dignité humaine \\[0pt]
La digression \\[0pt]
La direction de l'esprit \\[0pt]
La discipline \\[0pt]
La discipline de vie \\[0pt]
La discorde \\[0pt]
La discrétion \\[0pt]
La discrimination \\[0pt]
La discursivité \\[0pt]
La discussion \\[0pt]
La disgrâce \\[0pt]
La disharmonie \\[0pt]
La disponibilité \\[0pt]
La disposition \\[0pt]
La disposition. Le possible et le réel \\[0pt]
La disposition morale \\[0pt]
La dispute \\[0pt]
La dissidence \\[0pt]
La dissimulation \\[0pt]
La distance \\[0pt]
La distinction \\[0pt]
La distinction de genre \\[0pt]
La distinction de la nature et de la culture est-elle un fait de culture ? \\[0pt]
La distinction du sacré et du profane \\[0pt]
La distinction entre vraie et fausse sciences a-t-elle un sens ? \\[0pt]
La distinction sociale \\[0pt]
La distraction \\[0pt]
La diversion \\[0pt]
La diversité \\[0pt]
La diversité des cultures \\[0pt]
La diversité des cultures fait-elle obstacle à l'unité du genre humain ? \\[0pt]
La diversité des langues \\[0pt]
La diversité des langues est-elle une diversité des pensées ? \\[0pt]
La diversité des langues est-elle un obstacle à l'entente entre les hommes ? \\[0pt]
La diversité des opinions conduit-elle à douter de tout ? \\[0pt]
La diversité des perceptions \\[0pt]
La diversité des religions \\[0pt]
La diversité des sciences \\[0pt]
La diversité humaine \\[0pt]
La division \\[0pt]
La division de la volonté \\[0pt]
La division des pouvoirs \\[0pt]
La division des tâches \\[0pt]
La division du travail \\[0pt]
L'admiration \\[0pt]
La docilité est-elle un vice ou une vertu ? \\[0pt]
La domestication \\[0pt]
La domination \\[0pt]
La domination du corps \\[0pt]
La domination sociale \\[0pt]
La domination technique de la nature doit-elle susciter la crainte ou l'espoir ? \\[0pt]
La douceur \\[0pt]
L'adoucissement des mœurs \\[0pt]
La douleur \\[0pt]
La douleur est-elle utile ? \\[0pt]
La douleur et le plaisir \\[0pt]
La douleur nous apprend-elle quelque chose ? \\[0pt]
La droit de conquête \\[0pt]
La droiture \\[0pt]
La dualité \\[0pt]
La duplicité \\[0pt]
La durée \\[0pt]
La durée et l'instant \\[0pt]
L'adversité \\[0pt]
La faiblesse \\[0pt]
La faiblesse de croire \\[0pt]
La faiblesse de la démocratie \\[0pt]
La faiblesse de la volonté \\[0pt]
La faiblesse d'esprit \\[0pt]
La faiblesse de volonté \\[0pt]
La familiarité \\[0pt]
La famille \\[0pt]
La famille est-elle le lieu de la formation morale ? \\[0pt]
La famille est-elle naturelle ? \\[0pt]
La famille est-elle une communauté naturelle ? \\[0pt]
La famille est-elle une institution politique ? \\[0pt]
La famille est-elle une invention culturelle ? \\[0pt]
La famille est-elle un modèle de société ? \\[0pt]
La famille et la cité \\[0pt]
La famille et le droit \\[0pt]
La fatalité \\[0pt]
La fatigue \\[0pt]
La fausseté \\[0pt]
La faute \\[0pt]
La faute de goût \\[0pt]
La faute et le péché \\[0pt]
La faute et l'erreur \\[0pt]
La fécondité du raisonnement mathématique \\[0pt]
La femme est-elle l'avenir de l'homme ? \\[0pt]
La fermeté \\[0pt]
La fête \\[0pt]
La fête nationale \\[0pt]
L'affection \\[0pt]
L'affirmation \\[0pt]
La fiction \\[0pt]
La fiction est-elle fausse ? \\[0pt]
La fidélité \\[0pt]
La fidélité à soi \\[0pt]
La fierté \\[0pt]
La fièvre \\[0pt]
La figuration \\[0pt]
La figuration sensible de l'intelligible \\[0pt]
La figure de l'ennemi \\[0pt]
La figure humaine \\[0pt]
La fin \\[0pt]
La finalité \\[0pt]
La finalité des sciences humaines \\[0pt]
La finalité est-elle nécessaire pour penser le vivant ? \\[0pt]
La fin de la discussion \\[0pt]
La fin de la guerre \\[0pt]
La fin de la métaphysique \\[0pt]
La fin de la nature \\[0pt]
La fin de la politique \\[0pt]
La fin de la politique est-elle d'éliminer la violence ? \\[0pt]
La fin de la politique est-elle l'établissement de la justice ? \\[0pt]
La fin de l'art \\[0pt]
La fin de la technique se résume-t-elle à son utilité ? \\[0pt]
La fin de la vie \\[0pt]
La fin de l'État \\[0pt]
La fin de l'histoire \\[0pt]
La fin de l'homme \\[0pt]
La fin des désirs \\[0pt]
La fin des guerres \\[0pt]
La fin des temps \\[0pt]
La fin du monde \\[0pt]
La fin du mythe \\[0pt]
La fin du travail \\[0pt]
La fin et les moyens \\[0pt]
La finitude \\[0pt]
La fin justifie-t-elle les moyens ? \\[0pt]
La foi \\[0pt]
La foi est-elle aveugle ? \\[0pt]
La foi est-elle irrationnelle ? \\[0pt]
La foi est-elle rationnelle ? \\[0pt]
La folie \\[0pt]
La folie des grandeurs \\[0pt]
La folie et l'étude du psychisme \\[0pt]
La fonction \\[0pt]
La fonction de l'art \\[0pt]
La fonction de penser peut-elle se déléguer ? \\[0pt]
La fonction des exemples \\[0pt]
La fonction des rêves \\[0pt]
La fonction du philosophe est-elle de diriger l'État ? \\[0pt]
La fonction et l'organe \\[0pt]
La fonction première de l'État est-elle de durer ? \\[0pt]
La fonction sociale de l'écrivain \\[0pt]
La fonction symbolique \\[0pt]
La force \\[0pt]
La force d'âme \\[0pt]
La force de conviction \\[0pt]
La force de la croyance \\[0pt]
La force de la loi \\[0pt]
La force de la morale \\[0pt]
La force de la nature \\[0pt]
La force de l'art \\[0pt]
La force de la vérité \\[0pt]
La force de l'esprit \\[0pt]
La force de l'État est-elle nécessaire à la liberté des citoyens ? \\[0pt]
La force de l'expérience \\[0pt]
La force de l'habitude \\[0pt]
La force de l'idée \\[0pt]
La force de l'inconscient \\[0pt]
La force de l'obligation \\[0pt]
La force de l'oubli \\[0pt]
La force des choses \\[0pt]
La force des faibles \\[0pt]
La force des idées \\[0pt]
La force des institutions \\[0pt]
La force des lois \\[0pt]
La force des objections \\[0pt]
La force des récits \\[0pt]
La force donne-t-elle des droits ? \\[0pt]
La force du droit \\[0pt]
La force du génie \\[0pt]
La force du pouvoir \\[0pt]
La force du social \\[0pt]
La force du vrai \\[0pt]
La force est-elle une vertu ? \\[0pt]
La force et la violence \\[0pt]
La force et le droit \\[0pt]
La force et le droit s'opposent-ils nécessairement ? \\[0pt]
La force fait-elle le droit ? \\[0pt]
La force publique \\[0pt]
La formalisation \\[0pt]
La formation de l'esprit \\[0pt]
La formation des citoyens \\[0pt]
La formation du goût \\[0pt]
La formation d'une conscience \\[0pt]
La forme \\[0pt]
La forme du droit \\[0pt]
La forme et la couleur \\[0pt]
La forme et le fond \\[0pt]
La fortune \\[0pt]
La foule \\[0pt]
La fragilité \\[0pt]
La fragilité du beau \\[0pt]
La franchise \\[0pt]
La franchise est-elle une vertu ? \\[0pt]
La fraternité \\[0pt]
La fraternité a-t-elle un sens politique ? \\[0pt]
La fraternité est-elle un idéal moral ? \\[0pt]
La fraternité peut-elle se passer d'un fondement religieux ? \\[0pt]
La fraude \\[0pt]
La frivolité \\[0pt]
La frontière \\[0pt]
La fuite du temps \\[0pt]
La fuite du temps est-elle nécessairement un malheur ? \\[0pt]
La futilité \\[0pt]
La gauche et la droite \\[0pt]
L'âge atomique \\[0pt]
L'âge de la réflexion \\[0pt]
L'âge de raison \\[0pt]
L'âge d'or \\[0pt]
La généalogie \\[0pt]
La généalogie peut-elle être une méthode historique ? \\[0pt]
La généralisation \\[0pt]
La généralisation empirique \\[0pt]
La généralité \\[0pt]
La générosité \\[0pt]
La genèse \\[0pt]
La genèse de l'œuvre \\[0pt]
L'agent \\[0pt]
La gentillesse \\[0pt]
La géographie \\[0pt]
La géographie est-elle une science humaine ? \\[0pt]
La géométrie \\[0pt]
La géopolitique \\[0pt]
La gloire \\[0pt]
La gloire est-elle un bien ? \\[0pt]
La grâce \\[0pt]
La grammaire \\[0pt]
La grammaire contraint-elle la pensée ? \\[0pt]
La grammaire contraint-elle notre pensée ? \\[0pt]
La grammaire et la logique \\[0pt]
La grammaire générative \\[0pt]
La grammaire véhicule-t-elle une métaphysique ? \\[0pt]
La grandeur \\[0pt]
La grandeur d'âme \\[0pt]
La grandeur d'une culture \\[0pt]
La gratitude \\[0pt]
La gratuité \\[0pt]
La gravité \\[0pt]
L'agression \\[0pt]
L'agressivité \\[0pt]
La grève \\[0pt]
L'agriculture \\[0pt]
La grossièreté \\[0pt]
La guérison \\[0pt]
La guérison en psychanalyse \\[0pt]
La guerre \\[0pt]
La guerre civile \\[0pt]
La guerre est-elle la continuation de la politique ? \\[0pt]
La guerre est-elle la continuation de la politique par d'autres moyens ? \\[0pt]
La guerre est-elle la plus grande violence ? \\[0pt]
La guerre est-elle la politique continuée par d'autres moyens ? \\[0pt]
La guerre est-elle l'essentiel de toute politique ? \\[0pt]
La guerre et la paix \\[0pt]
La guerre juste \\[0pt]
La guerre met-elle fin au droit ? \\[0pt]
La guerre mondiale \\[0pt]
La guerre peut-elle être juste ? \\[0pt]
La guerre peut-elle être justifiée ? \\[0pt]
La guerre peut-elle être un droit ? \\[0pt]
La guerre totale \\[0pt]
La haine \\[0pt]
La haine de la pensée \\[0pt]
La haine de la raison \\[0pt]
La haine de la vérité \\[0pt]
La haine des images \\[0pt]
La haine des machines \\[0pt]
La haine de soi \\[0pt]
La haine et le mépris \\[0pt]
La hiérarchie \\[0pt]
La hiérarchie des arts \\[0pt]
La hiérarchie des cultures \\[0pt]
La hiérarchie des énoncés scientifiques \\[0pt]
La hiérarchie des êtres \\[0pt]
La honte \\[0pt]
Laisser faire \\[0pt]
Laisser faire la nature \\[0pt]
Laisser mourir, est-ce tuer ? \\[0pt]
La jalousie \\[0pt]
La jeunesse \\[0pt]
La jeunesse est mécontente \\[0pt]
La joie \\[0pt]
La joie de vivre \\[0pt]
La jouissance \\[0pt]
La jurisprudence \\[0pt]
La juste colère \\[0pt]
La juste mesure \\[0pt]
La juste peine \\[0pt]
La justice \\[0pt]
La justice a-t-elle besoin des institutions ? \\[0pt]
La justice a-t-elle pour fonction de compenser les inégalités naturelles ? \\[0pt]
La justice a-t-elle un fondement rationnel ? \\[0pt]
La justice comme avantage mutuel \\[0pt]
La justice consiste-t-elle à traiter tout le monde de la même manière ? \\[0pt]
La justice consiste-t-elle dans l'application de la loi ? \\[0pt]
La justice de l'État \\[0pt]
La justice distributive \\[0pt]
La justice divine \\[0pt]
La justice entre les générations \\[0pt]
La justice est-elle affaire de raison ? \\[0pt]
La justice est-elle de ce monde ? \\[0pt]
La justice est-elle de l'ordre du sentiment ? \\[0pt]
La justice est-elle l'affaire de l'État ? \\[0pt]
La justice est-elle le ciment de la paix sociale ? \\[0pt]
La justice est-elle une idée ? \\[0pt]
La justice est-elle une notion morale ? \\[0pt]
La justice est-elle une vertu ? \\[0pt]
La justice est-elle un idéal rationnel ? \\[0pt]
La justice et la force \\[0pt]
La justice et la loi \\[0pt]
La justice et la paix \\[0pt]
La justice et le droit \\[0pt]
La justice et l'égalité \\[0pt]
La justice exclut-elle l'arbitraire ? \\[0pt]
La justice internationale \\[0pt]
La justice : moyen ou fin de la politique ? \\[0pt]
La justice n'est-elle qu'une institution ? \\[0pt]
La justice n'est-elle qu'un idéal ? \\[0pt]
La justice peut-elle être fondée en nature ? \\[0pt]
La justice peut-elle se fonder sur le compromis ? \\[0pt]
La justice peut-elle se passer de la force ? \\[0pt]
La justice peut-elle se passer d'institutions ? \\[0pt]
La justice requiert-elle l'infaillibilité des juges ? \\[0pt]
La justice requiert-elle toujours l'impartialité ? \\[0pt]
La justice sociale \\[0pt]
La justice suppose-t-elle l'égalité ? \\[0pt]
La justification \\[0pt]
La lâcheté \\[0pt]
La laïcité \\[0pt]
La laideur \\[0pt]
La laideur est-elle une valeur esthétique ? \\[0pt]
La langue de la raison \\[0pt]
La langue et la parole \\[0pt]
La langue et la pensée \\[0pt]
La langue maternelle \\[0pt]
La langue nous parle-t-elle ? \\[0pt]
La langue officielle \\[0pt]
La lassitude \\[0pt]
L'aléatoire \\[0pt]
La leçon des choses \\[0pt]
La lecture \\[0pt]
La légende \\[0pt]
La légèreté \\[0pt]
La légitimation \\[0pt]
La légitime défense \\[0pt]
La légitimité \\[0pt]
La légitimité de l'État \\[0pt]
La légitimité démocratique \\[0pt]
La légitimité d'un clonage humain ? \\[0pt]
La lettre et l'esprit \\[0pt]
La liaison des idées \\[0pt]
La libération des mœurs \\[0pt]
La liberté \\[0pt]
La liberté admet-elle des degrés ? \\[0pt]
La liberté artistique \\[0pt]
La liberté a-t-elle une histoire ? \\[0pt]
La liberté a-t-elle un prix ? \\[0pt]
La liberté civile \\[0pt]
La liberté comporte-t-elle des degrés ? \\[0pt]
La liberté, concept moral ? \\[0pt]
La liberté connaît-elle des excès ? \\[0pt]
La liberté créatrice \\[0pt]
La liberté d'autrui \\[0pt]
La liberté de croire \\[0pt]
La liberté de culte \\[0pt]
La liberté de l'artiste \\[0pt]
La liberté de la science \\[0pt]
La liberté de la volonté \\[0pt]
La liberté de l'interprète \\[0pt]
La liberté de parole \\[0pt]
La liberté de penser \\[0pt]
La liberté des autres \\[0pt]
La liberté des citoyens \\[0pt]
La liberté des uns s'arrête-elle où commence celle des autres ? \\[0pt]
La liberté d'expression \\[0pt]
La liberté d'expression a-t-elle des limites ? \\[0pt]
La liberté d'expression est-elle nécessaire à la liberté de pensée ? \\[0pt]
La liberté d'expression est-elle nécessaire à la liberté de penser ? \\[0pt]
La liberté d'imaginer \\[0pt]
La liberté d'indifférence \\[0pt]
La liberté d'obéir \\[0pt]
La liberté doit-elle être limitée ? \\[0pt]
La liberté doit-elle se conquérir ? \\[0pt]
La liberté d'opinion \\[0pt]
La liberté du choix \\[0pt]
La liberté du citoyen \\[0pt]
La liberté du savant \\[0pt]
La liberté, est-ce l'indépendance à l'égard des passions ? \\[0pt]
La liberté est-elle ce qui définit l'homme ? \\[0pt]
La liberté est-elle contraire au principe de causalité ? \\[0pt]
La liberté est-elle innée ? \\[0pt]
La liberté est-elle la condition du bonheur ? \\[0pt]
La liberté est-elle le fondement de la responsabilité ? \\[0pt]
La liberté est-elle le pouvoir de refuser ? \\[0pt]
La liberté est-elle menacée par l'égalité ? \\[0pt]
La liberté est-elle une illusion ? \\[0pt]
La liberté est-elle une illusion nécessaire ? \\[0pt]
La liberté est-elle un fait ? \\[0pt]
La liberté et la grâce \\[0pt]
La liberté et la loi \\[0pt]
La liberté et la nécessité \\[0pt]
La liberté et l'égalité sont-elles compatibles ? \\[0pt]
La liberté et le hasard \\[0pt]
La liberté et les libertés \\[0pt]
La liberté et le temps \\[0pt]
La liberté fait-elle de nous des êtres meilleurs ? \\[0pt]
La liberté implique-t-elle l'indifférence ? \\[0pt]
La liberté impose-t-elle des devoirs ? \\[0pt]
La liberté individuelle \\[0pt]
La liberté intéresse-t-elle les sciences humaines ? \\[0pt]
La liberté morale \\[0pt]
La liberté ne s'éprouve-t-elle que dans la solitude ? \\[0pt]
La liberté n'est-elle qu'un droit ? \\[0pt]
La liberté n'est-elle qu'une illusion ? \\[0pt]
La liberté nous rend-elle inexcusables ? \\[0pt]
La liberté peut-elle être prouvée ? \\[0pt]
La liberté peut-elle être une illusion ? \\[0pt]
La liberté peut-elle faire peur ? \\[0pt]
La liberté peut-elle s'affirmer sans violence ? \\[0pt]
La liberté peut-elle s'aliéner ? \\[0pt]
La liberté peut-elle se constater ? \\[0pt]
La liberté peut-elle se prouver ? \\[0pt]
La liberté peut-elle se refuser ? \\[0pt]
La liberté politique \\[0pt]
La liberté requiert-elle le libre échange ? \\[0pt]
La liberté s'achète-t-elle ? \\[0pt]
La liberté s'apprend-elle ? \\[0pt]
La liberté se mérite-t-elle ? \\[0pt]
La liberté se prouve-t-elle ? \\[0pt]
La liberté s'éprouve-t-elle ? \\[0pt]
La liberté se prouve-t-elle ou s'éprouve-t-elle ? \\[0pt]
La liberté se réduit-elle au libre arbitre ? \\[0pt]
La liberté suppose-t-elle l'absence de déterminisme ? \\[0pt]
La libre interprétation \\[0pt]
L'aliénation \\[0pt]
L'aliéné \\[0pt]
La limitation des créatures \\[0pt]
La limite \\[0pt]
La limite du politique \\[0pt]
La linguistique est-elle une science de l'information ? \\[0pt]
La linguistique fait-elle partie des sciences humaines ? \\[0pt]
La linguistique peut-elle se passer de la psychologie ? \\[0pt]
La littérature engagée \\[0pt]
La littérature est-elle la mémoire de l'humanité ? \\[0pt]
La littérature et le mythe \\[0pt]
La littérature peut-elle suppléer les sciences de l'homme ? \\[0pt]
L'allégorie \\[0pt]
La logique a-t-elle une histoire ? \\[0pt]
La logique a-t-elle un intérêt philosophique ? \\[0pt]
La logique : calcul ou langage ? \\[0pt]
La logique : découverte ou invention ? \\[0pt]
La logique décrit-elle le monde ? \\[0pt]
La logique des sens \\[0pt]
La logique du pire \\[0pt]
La logique du sens \\[0pt]
La logique est-elle indépendante de la psychologie ? \\[0pt]
La logique est-elle la norme du vrai ? \\[0pt]
La logique est-elle l'art de penser ? \\[0pt]
La logique est-elle un art de penser ? \\[0pt]
La logique est-elle un art de raisonner ? \\[0pt]
La logique est-elle une discipline normative ? \\[0pt]
La logique est-elle une forme de calcul ? \\[0pt]
La logique est-elle une science ? \\[0pt]
La logique est-elle une science de la vérité ? \\[0pt]
La logique est-elle utile à la métaphysique ? \\[0pt]
La logique et le réel \\[0pt]
La logique nous apprend-elle quelque chose sur le langage ordinaire ? \\[0pt]
« La logique » ou bien « les logiques » ? \\[0pt]
La logique peut-elle se passer de la métaphysique ? \\[0pt]
La logique pourrait-elle nous surprendre ? \\[0pt]
La loi \\[0pt]
La loi dit-elle ce qui est juste ? \\[0pt]
La loi doit-elle être la même pour tous ? \\[0pt]
La loi du désir \\[0pt]
La loi du genre \\[0pt]
La loi du marché \\[0pt]
La loi du plus fort \\[0pt]
La loi éduque-t-elle ? \\[0pt]
La loi est-elle une garantie contre l'injustice ? \\[0pt]
La loi et la coutume \\[0pt]
La loi et la règle \\[0pt]
La loi et le contrat \\[0pt]
La loi et le règlement \\[0pt]
La loi et les mœurs \\[0pt]
La loi et l'ordre \\[0pt]
La loi peut-elle changer les mœurs ? \\[0pt]
La loi peut-elle être injuste ? \\[0pt]
La louange et le blâme \\[0pt]
La loyauté \\[0pt]
L'alter ego \\[0pt]
L'altérité \\[0pt]
L'altérité culturelle \\[0pt]
L'alternative \\[0pt]
L'altruisme \\[0pt]
L'altruisme est-il une attitude morale ? \\[0pt]
L'altruisme n'est-il qu'un égoïsme bien compris ? \\[0pt]
La lumière \\[0pt]
La lumière de la vérité \\[0pt]
La lumière et les ténèbres \\[0pt]
La lumière naturelle \\[0pt]
La lutte des classes \\[0pt]
La machine \\[0pt]
La machine fournit-elle un modèle pour penser le vivant ? \\[0pt]
La magie \\[0pt]
La magie des mots \\[0pt]
La magie peut-elle être efficace ? \\[0pt]
La magnanimité \\[0pt]
La main \\[0pt]
La main et l'esprit \\[0pt]
La main et l'outil \\[0pt]
La maîtrise \\[0pt]
La maîtrise de la langue \\[0pt]
La maîtrise de la nature \\[0pt]
La maîtrise de soi \\[0pt]
La maîtrise du feu \\[0pt]
La maîtrise du geste \\[0pt]
La maîtrise du temps \\[0pt]
La majesté \\[0pt]
La majorité \\[0pt]
La majorité a-t-elle toujours raison ? \\[0pt]
La majorité doit-elle toujours l'emporter ? \\[0pt]
La majorité, force ou droit ? \\[0pt]
La majorité peut-elle être tyrannique ? \\[0pt]
La maladie \\[0pt]
La maladie est-elle à l'organisme vivant ce que la panne est à la machine ? \\[0pt]
La maladie est-elle indispensable à la connaissance du vivant ? \\[0pt]
La maladie mentale a-t-elle une histoire ? \\[0pt]
La malchance \\[0pt]
La malveillance \\[0pt]
La manière \\[0pt]
La manifestation \\[0pt]
La manifestation de la vérité \\[0pt]
La marchandisation \\[0pt]
La marchandise \\[0pt]
La marge \\[0pt]
La marginalité \\[0pt]
La masse \\[0pt]
L'amateur \\[0pt]
L'amateur d'art \\[0pt]
L'amateurisme \\[0pt]
La mathématique est-elle une ontologie ? \\[0pt]
La mathématisation du réel \\[0pt]
La matière \\[0pt]
La matière de la pensée \\[0pt]
La matière de l'œuvre \\[0pt]
La matière, est-ce le mal ? \\[0pt]
La matière, est-ce l'informe ? \\[0pt]
La matière est-elle amorphe ? \\[0pt]
La matière est-elle connaissable ? \\[0pt]
La matière est-elle plus aisée à connaître que l'esprit ? \\[0pt]
La matière est-elle plus facile à connaître que l'esprit ? \\[0pt]
La matière est-elle une substance ? \\[0pt]
La matière est-elle une vue de l'esprit ? \\[0pt]
La matière et la forme \\[0pt]
La matière et la vie \\[0pt]
La matière et l'esprit \\[0pt]
La matière et l'esprit peuvent-ils constituer une seule chose ? \\[0pt]
La matière n'est-elle que ce que l'on perçoit ? \\[0pt]
La matière n'est-elle qu'une idée ? \\[0pt]
La matière n'est-elle qu'un obstacle ? \\[0pt]
La matière pense-t-elle ? \\[0pt]
La matière peut-elle agir ? \\[0pt]
La matière peut-elle être objet de connaissance ? \\[0pt]
La matière peut-elle penser ? \\[0pt]
La matière première \\[0pt]
La matière sensible \\[0pt]
La matière vivante \\[0pt]
La maturité \\[0pt]
La mauvaise conscience \\[0pt]
La mauvaise éducation \\[0pt]
La mauvaise foi \\[0pt]
La mauvaise volonté \\[0pt]
L'ambiguïté \\[0pt]
L'ambiguïté des mots peut-elle être heureuse ? \\[0pt]
L'ambition \\[0pt]
L'ambition politique \\[0pt]
L'âme \\[0pt]
La méchanceté \\[0pt]
L'âme concerne-t-elle les sciences humaines ? \\[0pt]
La méconnaissance \\[0pt]
La méconnaissance de soi \\[0pt]
La médecine est-elle une science ? \\[0pt]
L'âme des bêtes \\[0pt]
La médiation \\[0pt]
La médiocrité \\[0pt]
La médiocrité artistique \\[0pt]
La médisance \\[0pt]
La méditation \\[0pt]
L'âme est-elle immortelle ? \\[0pt]
L'âme et le corps \\[0pt]
L'âme et le corps sont-ils une seule et même chose ? \\[0pt]
L'âme et l'esprit \\[0pt]
La méfiance \\[0pt]
La meilleure constitution \\[0pt]
L'âme jouit-elle d'une vie propre ? \\[0pt]
La mélancolie \\[0pt]
L'âme, le monde et Dieu \\[0pt]
L'amélioration des hommes peut-elle être considérée comme un objectif politique ? \\[0pt]
La mémoire \\[0pt]
La mémoire collective \\[0pt]
La mémoire est-elle une fonction vitale ? \\[0pt]
La mémoire et l'histoire \\[0pt]
La mémoire et l'individu \\[0pt]
La mémoire et l'oubli \\[0pt]
La mémoire sélective \\[0pt]
La menace \\[0pt]
La mesure \\[0pt]
La mesure dans les sciences humaines \\[0pt]
La mesure de l'homme \\[0pt]
La mesure de l'intelligence \\[0pt]
La mesure des choses \\[0pt]
La mesure du temps \\[0pt]
La métamorphose \\[0pt]
La métaphore \\[0pt]
La métaphore n'est-elle qu'un ornement du discours ? \\[0pt]
La métaphysique a-t-elle ses fictions ? \\[0pt]
La métaphysique est-elle affaire de raisonnement ? \\[0pt]
La métaphysique est-elle le fondement de la morale ? \\[0pt]
La métaphysique est-elle nécessairement une réflexion sur Dieu ? \\[0pt]
La métaphysique est-elle un discours sans objet ? \\[0pt]
La métaphysique est-elle une discipline théorique ? \\[0pt]
La métaphysique est-elle une science ? \\[0pt]
La métaphysique peut-elle être autre chose qu'une science recherchée ? \\[0pt]
La métaphysique peut-elle faire appel à l'expérience ? \\[0pt]
La métaphysique procure-t-elle un savoir ? \\[0pt]
La métaphysique relève-t-elle de la philosophie ou de la poésie ? \\[0pt]
La métaphysique répond-elle à un besoin ? \\[0pt]
La métaphysique repose-t-elle sur des croyances ? \\[0pt]
La métaphysique se définit-elle par son objet ou sa démarche ? \\[0pt]
La méthode \\[0pt]
La méthode de la science \\[0pt]
La méthode en sciences sociales \\[0pt]
La méthode est-elle nécessaire pour la recherche de la vérité ? \\[0pt]
La méthode expérimentale est-elle appropriée à l'étude du vivant ? \\[0pt]
La méthode hypothético-déductive \\[0pt]
La méthode statistique \\[0pt]
L'ami \\[0pt]
L'ami du prince \\[0pt]
L'ami et l'ennemi \\[0pt]
La minorité \\[0pt]
La misanthropie \\[0pt]
La mise à l'épreuve des hypothèses en sciences humaines et sociales \\[0pt]
La mise en scène \\[0pt]
La misère \\[0pt]
La miséricorde \\[0pt]
La misologie \\[0pt]
L'amitié \\[0pt]
L'amitié est-elle une vertu ? \\[0pt]
L'amitié est-elle un principe politique ? \\[0pt]
L'amitié peut-elle obliger ? \\[0pt]
L'amitié relève-t-elle d'une décision ? \\[0pt]
La modalité \\[0pt]
La mode \\[0pt]
La modélisation en sciences sociales \\[0pt]
La modération \\[0pt]
La modération est-elle l'essence de la vertu ? \\[0pt]
La modération est-elle une vertu politique ? \\[0pt]
La modernité \\[0pt]
La modernité dans les arts \\[0pt]
La modestie \\[0pt]
La mondialisation \\[0pt]
La monnaie \\[0pt]
La monnaie n'est-elle qu'un moyen d'échange ? \\[0pt]
La monumentalité \\[0pt]
La morale \\[0pt]
La morale a-t-elle à décider de la sexualité ? \\[0pt]
La morale a-t-elle besoin de la notion de sainteté ? \\[0pt]
La morale a-t-elle besoin d'être fondée ? \\[0pt]
La morale a-t-elle besoin d'un au-delà ? \\[0pt]
La morale a-t-elle besoin d'un fondement ? \\[0pt]
La morale a-t-elle sa place dans l'économie ? \\[0pt]
La morale commune \\[0pt]
La morale consiste-t-elle à respecter le droit ? \\[0pt]
La morale consiste-t-elle à suivre la nature ? \\[0pt]
La morale de l'athée \\[0pt]
La morale de l'intérêt \\[0pt]
La morale dépend-elle de la culture ? \\[0pt]
La morale des fables \\[0pt]
La morale doit-elle en appeler à la nature ? \\[0pt]
La morale doit-elle être rationnelle ? \\[0pt]
La morale doit-elle fournir des préceptes ? \\[0pt]
La morale du citoyen \\[0pt]
La morale du plus fort \\[0pt]
La morale est-elle affaire de convention ? \\[0pt]
La morale est-elle affaire de jugement ? \\[0pt]
La morale est-elle affaire de sentiment ? \\[0pt]
La morale est-elle affaire de sentiments ? \\[0pt]
La morale est-elle affaire individuelle ? \\[0pt]
La morale est-elle condamnée à n'être qu'un champ de bataille ? \\[0pt]
La morale est-elle désintéressée ? \\[0pt]
La morale est-elle émotion ? \\[0pt]
La morale est-elle en conflit avec le désir ? \\[0pt]
La morale est-elle ennemie du bonheur ? \\[0pt]
La morale est-elle fondée sur la liberté ? \\[0pt]
La morale est-elle incompatible avec le déterminisme ? \\[0pt]
La morale est-elle l'ennemie de la vie ? \\[0pt]
La morale est-elle nécessairement répressive ? \\[0pt]
La morale est-elle objet de science ? \\[0pt]
La morale est-elle un art de vivre ? \\[0pt]
La morale est-elle un besoin ? \\[0pt]
La morale est-elle une affaire de raison ? \\[0pt]
La morale est-elle une affaire d'habitude ? \\[0pt]
La morale est-elle une affaire solitaire ? \\[0pt]
La morale est-elle une médecine de l'âme ? \\[0pt]
La morale est-elle une preuve de la liberté ? \\[0pt]
La morale est-elle une science ? \\[0pt]
La morale est-elle un fait de culture ? \\[0pt]
La morale est-elle un fait social ? \\[0pt]
La morale et la politique \\[0pt]
La morale et la religion visent-elles les mêmes fins ? \\[0pt]
La morale et le droit \\[0pt]
La morale et les mœurs \\[0pt]
La morale n'est-elle qu'un art de vivre ? \\[0pt]
La morale n'est-elle qu'un dressage ? \\[0pt]
La morale n'est-elle qu'un ensemble de conventions ? \\[0pt]
La morale peut-elle être fondée sur la science ? \\[0pt]
La morale peut-elle être naturelle ? \\[0pt]
La morale peut-elle être personnelle ? \\[0pt]
La morale peut-elle être un calcul ? \\[0pt]
La morale peut-elle être une science ? \\[0pt]
La morale peut-elle se définir comme l'art d'être heureux ? \\[0pt]
La morale peut-elle se fonder sur les sentiments ? \\[0pt]
La morale peut-elle s'enseigner ? \\[0pt]
La morale peut-elle se passer d'un fondement religieux ? \\[0pt]
La morale politique \\[0pt]
La morale requiert-elle une casuistique ? \\[0pt]
La morale requiert-elle un fondement ? \\[0pt]
La morale s'apprend-elle ? \\[0pt]
La morale se déduit-elle de la métaphysique ? \\[0pt]
La morale s'enracine-t-elle dans le sens commun ? \\[0pt]
La morale s'enseigne-t-elle ? \\[0pt]
La morale s'oppose-t-elle à la politique ? \\[0pt]
La morale suppose-t-elle le libre arbitre ? \\[0pt]
La moralité consiste-t-elle à se contraindre soi-même ? \\[0pt]
La moralité des lois \\[0pt]
La moralité est-elle affaire de principes ou de conséquences ? \\[0pt]
La moralité est-elle utile à la vie sociale ? \\[0pt]
La moralité et le traitement des animaux \\[0pt]
La moralité n'est-elle que dressage ? \\[0pt]
La moralité réside-t-elle dans l'intention ? \\[0pt]
La moralité se juge-t-elle aux actes ? \\[0pt]
La moralité se réduit-elle aux sentiments ? \\[0pt]
La mort \\[0pt]
La mort a-t-elle un sens ? \\[0pt]
La mort dans l'âme \\[0pt]
La mort d'autrui \\[0pt]
La mort de Dieu \\[0pt]
La mort de l'art \\[0pt]
La mort de l'homme \\[0pt]
La mort fait-elle partie de la vie ? \\[0pt]
La motivation et le jugement moral \\[0pt]
L'amour \\[0pt]
L'amour a-t-il des raisons ? \\[0pt]
L'amour de la liberté \\[0pt]
L'amour de la nature \\[0pt]
L'amour de la patrie \\[0pt]
L'amour de l'argent \\[0pt]
L'amour de l'art \\[0pt]
L'amour de la vérité \\[0pt]
L'amour de la vie \\[0pt]
L'amour de l'humanité \\[0pt]
L'amour des lois \\[0pt]
L'amour de soi \\[0pt]
L'amour de soi est-il immoral ? \\[0pt]
L'amour du destin \\[0pt]
L'amour du pouvoir est-il compatible avec le dévouement à la chose publique ? \\[0pt]
L'amour du prochain \\[0pt]
L'amour du travail \\[0pt]
L'amour est-il aveugle ? \\[0pt]
L'amour est-il désir ? \\[0pt]
L'amour est-il sans raison ? \\[0pt]
L'amour est-il toujours une passion ? \\[0pt]
L'amour est-il une vertu ? \\[0pt]
L'amour et la haine \\[0pt]
L'amour et la justice \\[0pt]
L'amour et l'amitié \\[0pt]
L'amour et la mort \\[0pt]
L'amour et la philosophie \\[0pt]
L'amour et le devoir \\[0pt]
L'amour et le respect \\[0pt]
L'amour et l'humanité \\[0pt]
L'amour fou \\[0pt]
L'amour implique-t-il le respect ? \\[0pt]
L'amour maternel \\[0pt]
L'amour peut-il être absolu ? \\[0pt]
L'amour peut-il être raisonnable ? \\[0pt]
L'amour peut-il être un devoir ? \\[0pt]
L'amour-propre \\[0pt]
L'amour vrai \\[0pt]
La multiplicité \\[0pt]
La multiplicité des acceptions de l'étant \\[0pt]
La multitude \\[0pt]
La musique a-t-elle une essence ? \\[0pt]
La musique a-t-elle un sens ? \\[0pt]
La musique de film \\[0pt]
La musique donne-t-elle à penser ? \\[0pt]
La musique est-elle un discours ? \\[0pt]
La musique est-elle un langage ? \\[0pt]
La musique et la danse \\[0pt]
La musique et le bruit \\[0pt]
La musique et le temps \\[0pt]
La musique exprime-t-elle quelque chose ? \\[0pt]
La musique peut-elle être narrative ? \\[0pt]
L'anachronisme \\[0pt]
La naissance \\[0pt]
La naissance de l'art \\[0pt]
La naissance de la science \\[0pt]
La naissance de l'homme \\[0pt]
La naïveté \\[0pt]
La naïveté est-elle une vertu ? \\[0pt]
L'analogie \\[0pt]
L'analyse \\[0pt]
L'analyse des comportements humains permet-elle de réduire le hasard ? \\[0pt]
L'analyse des mythes doit-elle être uniquement linguistique \\[0pt]
L'analyse du langage ordinaire peut-elle avoir un intérêt philosophique ? \\[0pt]
L'analyse du vécu \\[0pt]
L'analyse économique rend-elle compte des comportements humains ? \\[0pt]
L'analyse et la synthèse \\[0pt]
L'anarchie \\[0pt]
La nation \\[0pt]
La nation est-elle dépassée ? \\[0pt]
La nation et l'État \\[0pt]
La nature \\[0pt]
La nature artiste \\[0pt]
La nature a-t-elle des droits ? \\[0pt]
La nature a-t-elle une histoire ? \\[0pt]
La nature a-t-elle une valeur ? \\[0pt]
La nature a-t-elle un langage ? \\[0pt]
La nature a-t-elle un ordre ? \\[0pt]
La nature des choses \\[0pt]
La nature des objets mathématiques \\[0pt]
La nature donne-t-elle un sens à notre existence ? \\[0pt]
La nature du bien \\[0pt]
La nature du droit \\[0pt]
La nature du fait moral \\[0pt]
La nature, est-ce le donné ? \\[0pt]
La nature est-elle artiste ? \\[0pt]
La nature est-elle belle ? \\[0pt]
La nature est-elle bien faite ? \\[0pt]
La nature est-elle digne de respect ? \\[0pt]
La nature est-elle écrite en langage mathématique ? \\[0pt]
La nature est-elle muette ? \\[0pt]
La nature est-elle politique ? \\[0pt]
La nature est-elle prévisible ? \\[0pt]
La nature est-elle sacrée ? \\[0pt]
La nature est-elle sans histoire ? \\[0pt]
La nature est-elle sauvage ? \\[0pt]
La nature est-elle une grande artiste ? \\[0pt]
La nature est-elle une idée ? \\[0pt]
La nature est-elle une invention de l'homme ? \\[0pt]
La nature est-elle une norme ? \\[0pt]
La nature est-elle une ressource ? \\[0pt]
La nature est-elle un guide ? \\[0pt]
La nature est-elle un modèle ? \\[0pt]
La nature est-elle un système ? \\[0pt]
La nature et la grâce \\[0pt]
La nature et l'artifice \\[0pt]
La nature et le beau \\[0pt]
La nature et le monde \\[0pt]
La nature existe-t-elle ? \\[0pt]
« La nature fait bien les choses » \\[0pt]
La nature fait-elle bien les choses ? \\[0pt]
La nature humaine \\[0pt]
La nature imite-t-elle l'art ? \\[0pt]
La nature morte \\[0pt]
La nature ne fait-elle rien en vain ? \\[0pt]
La nature ne fait pas de saut \\[0pt]
La nature nous dicte-t-elle des fins ? \\[0pt]
La nature nous domine-t-elle ? \\[0pt]
La nature nous donne-t-elle des commandements ? \\[0pt]
La nature nous indique-t-elle ce qui est bon ? \\[0pt]
La nature obéit-elle à des fins ? \\[0pt]
La nature parle-t-elle le langage des mathématiques ? \\[0pt]
La nature peut-elle avoir des droits ? \\[0pt]
La nature peut-elle constituer une norme ? \\[0pt]
La nature peut-elle être belle ? \\[0pt]
La nature peut-elle être détruite ? \\[0pt]
La nature peut-elle être un modèle ? \\[0pt]
La nature peut-elle nous indiquer ce que nous devons faire ? \\[0pt]
La nature reprend-elle toujours ses droits ? \\[0pt]
La nature se donne-t-elle à penser ? \\[0pt]
La nature s'oppose-t-elle à l'esprit ? \\[0pt]
L'anéantissement \\[0pt]
L'anecdotique \\[0pt]
La nécessité \\[0pt]
La nécessité de l'oubli \\[0pt]
La nécessité des contradictions \\[0pt]
La nécessité des signes \\[0pt]
La nécessité fait-elle loi ? \\[0pt]
La nécessité historique \\[0pt]
La nécessité sociale des cérémonies \\[0pt]
La négation \\[0pt]
La négligence \\[0pt]
La négligence est-elle une faute ? \\[0pt]
La négociation \\[0pt]
La neige est-elle blanche ? \\[0pt]
La neutralité \\[0pt]
La neutralité axiologique \\[0pt]
La neutralité de l'État \\[0pt]
La névrose \\[0pt]
Langage et communication \\[0pt]
Langage et logique \\[0pt]
Langage et passions \\[0pt]
Langage et pensée \\[0pt]
Langage et pouvoir \\[0pt]
Langage et réalité \\[0pt]
Langage et réification \\[0pt]
Langage et société \\[0pt]
Langage, langue et parole \\[0pt]
Langage, langue, parole \\[0pt]
Langage ordinaire et langage de la science \\[0pt]
L'angélisme \\[0pt]
L'angoisse \\[0pt]
L'angoisse et la peur \\[0pt]
Langue et langage \\[0pt]
Langue et parole \\[0pt]
L'animal \\[0pt]
L'animal apprend-il à l'homme quelque chose sur l'homme ? \\[0pt]
L'animal a-t-il des droits ? \\[0pt]
L'animal domestique \\[0pt]
L'animal est-il une personne ? \\[0pt]
L'animal et la bête \\[0pt]
L'animal et l'homme \\[0pt]
L'animalité \\[0pt]
L'animalité de l'animal, l'animalité de l'homme \\[0pt]
L'animal nous apprend-il quelque chose sur l'homme ? \\[0pt]
L'animal peut-il être un sujet moral ? \\[0pt]
L'animal politique \\[0pt]
L'animisme \\[0pt]
La noblesse \\[0pt]
L'anomalie \\[0pt]
L'anomie \\[0pt]
La non-contradiction : principe logique ou ontologique ? \\[0pt]
La non-violence \\[0pt]
L'anonymat \\[0pt]
L'anonyme \\[0pt]
L'anormal \\[0pt]
La normalité \\[0pt]
La normativité \\[0pt]
La norme \\[0pt]
La norme du beau \\[0pt]
La norme du goût \\[0pt]
La norme et l'écart \\[0pt]
La norme et le fait \\[0pt]
La norme et son application \\[0pt]
La nostalgie \\[0pt]
La notion d'administration \\[0pt]
La notion d'art contemporain \\[0pt]
La notion d'art poétique \\[0pt]
La notion de barbarie a-t-elle un sens ? \\[0pt]
La notion de caractère relève-t- elle de la morale ou de la psychologie ? \\[0pt]
La notion de civilisation \\[0pt]
La notion de classe dominante \\[0pt]
La notion de classe sociale \\[0pt]
La notion de communauté morale \\[0pt]
La notion de comportement \\[0pt]
La notion de corps social \\[0pt]
La notion de démocratie représentative est-elle contradictoire ? \\[0pt]
La notion de droit international est-elle contradictoire ? \\[0pt]
La notion de fait social \\[0pt]
La notion de finalité a-t-elle de l'intérêt pour le savant ? \\[0pt]
La notion de genre littéraire \\[0pt]
La notion de loi a-t-elle une unité ? \\[0pt]
La notion de loi dans les sciences de la nature et dans les sciences de l'homme \\[0pt]
La notion de milieu \\[0pt]
La notion de monde \\[0pt]
La notion de nature humaine \\[0pt]
La notion de nature humaine a-t-elle un sens ? \\[0pt]
La notion de paradis a-t-elle un sens exclusivement religieux ? \\[0pt]
La notion de peuple \\[0pt]
La notion de point de vue \\[0pt]
La notion d'époque \\[0pt]
La notion de possible \\[0pt]
La notion de progrès a-t-elle un sens en politique ? \\[0pt]
La notion de progrès moral a-t-elle encore un sens ? \\[0pt]
La notion de race a-t-elle un statut scientifique ? \\[0pt]
La notion de responsabilité est-elle suspecte ? \\[0pt]
La notion de signification est-elle indispensable aux sciences humaines ? \\[0pt]
La notion de souveraineté partagée \\[0pt]
La notion de sujet en politique \\[0pt]
La notion de système \\[0pt]
La notion d'évolution \\[0pt]
La notion d'histoire commune a-t-elle un sens ? \\[0pt]
La notion d'intérêt \\[0pt]
La notion d'ordre \\[0pt]
La notion économique d'abondance \\[0pt]
La notion physique de force \\[0pt]
La nouveauté \\[0pt]
La nouveauté en art \\[0pt]
L'antériorité \\[0pt]
L'anthropocentrisme \\[0pt]
L'anthropologie est-elle une ontologie ? \\[0pt]
L'anticipation \\[0pt]
L'antinomie \\[0pt]
La nuance \\[0pt]
La nudité \\[0pt]
La nuit \\[0pt]
La nuit et le jour \\[0pt]
La ou les vertus ? \\[0pt]
La paix \\[0pt]
La paix civile \\[0pt]
La paix de la conscience \\[0pt]
La paix est-elle l'absence de guerre ? \\[0pt]
La paix est-elle l'absence de guerres ? \\[0pt]
La paix est-elle la visée de la politique ? \\[0pt]
La paix est-elle le plus grand des biens ? \\[0pt]
La paix est-elle moins naturelle que la guerre ? \\[0pt]
La paix est-elle possible ? \\[0pt]
La paix est-elle toujours préférable ? \\[0pt]
La paix met-elle fin à l'histoire ? \\[0pt]
La paix n'est-elle que l'absence de conflit ? \\[0pt]
La paix n'est-elle que l'absence de guerre ? \\[0pt]
La paix n'est-elle qu'un idéal ? \\[0pt]
La paix perpétuelle \\[0pt]
La paix sociale \\[0pt]
La paix sociale est-elle la finalité de la politique ? \\[0pt]
La paix sociale est-elle le but de la politique ? \\[0pt]
La paix sociale est-elle une fin en soi ? \\[0pt]
La panne et la maladie \\[0pt]
La parenté \\[0pt]
La parenté et la famille \\[0pt]
La paresse \\[0pt]
La parole \\[0pt]
La parole comme acte social \\[0pt]
La parole donnée \\[0pt]
La parole et l'action \\[0pt]
La parole et l'écriture \\[0pt]
La parole et le geste \\[0pt]
La parole intérieure \\[0pt]
La parole nous libère-t-elle ? \\[0pt]
La parole peut-elle agir ? \\[0pt]
La parole peut-elle être étudiée comme un acte social ? \\[0pt]
La parole peut-elle être une arme ? \\[0pt]
La parole politique \\[0pt]
La parole publique \\[0pt]
La part de l'ombre \\[0pt]
La participation \\[0pt]
La participation des citoyens \\[0pt]
La participation politique \\[0pt]
La partie et le tout \\[0pt]
La parure \\[0pt]
La passion \\[0pt]
La passion amoureuse \\[0pt]
La passion de la connaissance \\[0pt]
La passion de la justice \\[0pt]
La passion de la liberté \\[0pt]
La passion de la vérité \\[0pt]
La passion de la vérité peut-elle être source d'erreur ? \\[0pt]
La passion de l'égalité \\[0pt]
La passion du juste \\[0pt]
La passion est-elle immorale ? \\[0pt]
La passion est-elle l'ennemi de la raison ? \\[0pt]
La passion exclut-elle la lucidité ? \\[0pt]
La passion n'est-elle que souffrance ? \\[0pt]
La passivité \\[0pt]
La paternité \\[0pt]
L'apathie \\[0pt]
La patience \\[0pt]
La patience de la pensée \\[0pt]
La patience est-elle une vertu ? \\[0pt]
L'apatride \\[0pt]
La patrie \\[0pt]
La pauvreté \\[0pt]
La pauvreté est-elle une injustice ? \\[0pt]
La pédagogie, philosophie pratique ou psychologie appliquée ? \\[0pt]
La peine \\[0pt]
La peine capitale \\[0pt]
La peine de mort \\[0pt]
La peine de mort est-elle juste, injuste, et pourquoi ? \\[0pt]
La peinture apprend-elle à voir ? \\[0pt]
La peinture des mœurs \\[0pt]
La peinture est-elle une poésie muette ? \\[0pt]
La peinture peut-elle être un art du temps ? \\[0pt]
La pénibilité du travail \\[0pt]
La pensée \\[0pt]
La pensée a-t-elle besoin d'images ? \\[0pt]
La pensée a-t-elle des lois ? \\[0pt]
La pensée a-t-elle une histoire ? \\[0pt]
La pensée collective \\[0pt]
La pensée de la mort a-t-elle un objet ? \\[0pt]
La pensée de l'espace \\[0pt]
La pensée des machines \\[0pt]
La pensée doit-elle se soumettre aux règles de la logique ? \\[0pt]
La pensée échappe-t-elle à la grammaire ? \\[0pt]
La pensée est-elle en lutte avec le langage ? \\[0pt]
La pensée est-elle une activité assimilable à un travail ? \\[0pt]
La pensée et la conscience sont-elles une seule et même chose ? \\[0pt]
La pensée formelle \\[0pt]
La pensée formelle est-elle privée d'objet ? \\[0pt]
La pensée formelle est-elle une pensée vide ? \\[0pt]
La pensée formelle peut-elle avoir un contenu ? \\[0pt]
La pensée magique \\[0pt]
La pensée mythique \\[0pt]
La pensée obéit-elle à des lois ? \\[0pt]
La pensée peut-elle devenir une technique ? \\[0pt]
La pensée peut-elle s'écrire ? \\[0pt]
La pensée peut-elle se passer de mots ? \\[0pt]
La pensée philosophique doit-elle être une pensée systématique ? \\[0pt]
La pensée scientifique \\[0pt]
La pensée technique \\[0pt]
L'aperception \\[0pt]
La perception \\[0pt]
La perception construit-elle son objet ? \\[0pt]
La perception de la folie change-t-elle de nature avec les sciences humaines ? \\[0pt]
La perception de l'espace est-elle innée ou acquise ? \\[0pt]
La perception est-elle le premier degré de la connaissance ? \\[0pt]
La perception est-elle l'interprétation du réel ? \\[0pt]
La perception est-elle source de connaissance ? \\[0pt]
La perception est-elle une interprétation ? \\[0pt]
La perception est-elle un mode de connaissance ? \\[0pt]
La perception me donne-t-elle le réel ? \\[0pt]
La perception musicale \\[0pt]
La perception peut-elle être désintéressée ? \\[0pt]
La perception peut-elle s'éduquer ? \\[0pt]
La perfectibilité \\[0pt]
La perfection \\[0pt]
La perfection artistique \\[0pt]
La perfection en art \\[0pt]
La perfection est-elle désirable ? \\[0pt]
La perfection morale \\[0pt]
La performance \\[0pt]
La permanence \\[0pt]
La persécution \\[0pt]
La persévérance \\[0pt]
La personnalité \\[0pt]
La personne \\[0pt]
La personne et la mémoire \\[0pt]
La personne et l'individu \\[0pt]
La personne politique \\[0pt]
La perspective \\[0pt]
La persuasion \\[0pt]
La pertinence \\[0pt]
La perversion \\[0pt]
La perversion morale \\[0pt]
La perversité \\[0pt]
La pesanteur \\[0pt]
La pétition de principe \\[0pt]
La peur \\[0pt]
La peur de la mort \\[0pt]
La peur de la nature \\[0pt]
La peur de la science \\[0pt]
La peur de la technique \\[0pt]
La peur de l'autre \\[0pt]
La peur de l'avenir \\[0pt]
La peur de la vérité \\[0pt]
La peur de l'inconnu \\[0pt]
La peur de manquer \\[0pt]
La peur des machines \\[0pt]
La peur des mots \\[0pt]
La peur du châtiment \\[0pt]
La peur du désordre \\[0pt]
La peur est-elle mauvaise conseillère ? \\[0pt]
La philanthropie \\[0pt]
La philosophie a-t-elle une histoire ? \\[0pt]
La philosophie a-t-elle un objet qui lui soit propre ? \\[0pt]
La philosophie doit-elle être systématique ? \\[0pt]
La philosophie doit-elle être une science ? \\[0pt]
La philosophie doit-elle se préoccuper du salut ? \\[0pt]
La philosophie est-elle abstraite ? \\[0pt]
La philosophie est-elle la servante de la science ? \\[0pt]
La philosophie est-elle une méditation de la mort ? \\[0pt]
La philosophie est-elle une méditation de la vie ou de la mort ? \\[0pt]
La philosophie est-elle une science ? \\[0pt]
La philosophie et le sens commun \\[0pt]
La philosophie et les sciences \\[0pt]
La philosophie et son histoire \\[0pt]
La philosophie peut-elle disparaître ? \\[0pt]
La philosophie peut-elle être expérimentale ? \\[0pt]
La philosophie peut-elle être populaire ? \\[0pt]
La philosophie peut-elle être une science ? \\[0pt]
La philosophie peut-elle ne pas parler de Dieu ? \\[0pt]
La philosophie peut-elle se passer de l'idée de vérité ? \\[0pt]
La philosophie peut-elle se passer de théologie ? \\[0pt]
La philosophie première \\[0pt]
La philosophie rend-elle inefficace la propagande ? \\[0pt]
La photographie est-elle un art ? \\[0pt]
La physique est-elle le modèle de toutes les sciences ? \\[0pt]
La physique et la chimie \\[0pt]
La pitié \\[0pt]
La pitié a-t-elle une valeur ? \\[0pt]
La pitié est-elle dangereuse ? \\[0pt]
La pitié est-elle morale ? \\[0pt]
La pitié est-elle un sentiment moral ? \\[0pt]
La pitié fonde-t-elle la morale ? \\[0pt]
La pitié peut-elle fonder la morale ? \\[0pt]
La place d'autrui \\[0pt]
La place de l'animal \\[0pt]
La place de la philosophie dans la culture \\[0pt]
La place de l'art est-elle sur le marché de l'art ? \\[0pt]
La place du hasard dans la science \\[0pt]
La place du sujet dans la science \\[0pt]
La place du sujet dans les sciences humaines \\[0pt]
La place publique \\[0pt]
La plaisanterie \\[0pt]
La plénitude \\[0pt]
La pluralité \\[0pt]
La pluralité des arts \\[0pt]
La pluralité des biens \\[0pt]
La pluralité des cultures \\[0pt]
La pluralité des interprétations \\[0pt]
La pluralité des langues \\[0pt]
La pluralité des méthodes en sociologie \\[0pt]
La pluralité des mondes \\[0pt]
La pluralité des opinions \\[0pt]
La pluralité des pouvoirs \\[0pt]
La pluralité des religions \\[0pt]
La pluralité des représentations \\[0pt]
La pluralité des sciences \\[0pt]
La pluralité des sciences de la nature \\[0pt]
La pluralité des sens de l'être \\[0pt]
La pluralité des valeurs \\[0pt]
La pluralité des vérités condamne-t-elle l'idée de vérité ? \\[0pt]
La plus grande perfection \\[0pt]
La poésie \\[0pt]
La poésie dit-elle le réel ? \\[0pt]
La poésie est-elle comme une peinture ? \\[0pt]
La poésie et l'idée \\[0pt]
La poésie pense-t-elle ? \\[0pt]
La polémique \\[0pt]
La police \\[0pt]
La police et l'armée \\[0pt]
La politesse \\[0pt]
La politesse est-elle une vertu ? \\[0pt]
La politique \\[0pt]
La politique : affaire de compétence ? \\[0pt]
La politique a-t-elle besoin de héros ? \\[0pt]
La politique a-t-elle besoin de la fiction ? \\[0pt]
La politique a-t-elle besoin de la religion ? \\[0pt]
La politique a-t-elle besoin de modèles ? \\[0pt]
La politique a-t-elle besoin d'experts ? \\[0pt]
La politique a-t-elle pour but de nous faire vivre dans un monde meilleur ? \\[0pt]
La politique a-t-elle pour fin d'éliminer la violence ? \\[0pt]
La politique consiste-t-elle à faire cause commune ? \\[0pt]
La politique consiste-t-elle à faire des compromis ? \\[0pt]
La politique consiste-t-elle à gérer l'urgence ? \\[0pt]
La politique de la santé \\[0pt]
La politique doit-elle avoir pour visée le bonheur ? \\[0pt]
La politique doit-elle être morale ? \\[0pt]
La politique doit-elle être rationnelle ? \\[0pt]
La politique doit-elle protéger la liberté des citoyens ? \\[0pt]
La politique doit-elle refuser l'utopie ? \\[0pt]
La politique doit-elle se mêler de l'art ? \\[0pt]
La politique doit-elle se mêler du bonheur ? \\[0pt]
La politique doit-elle viser la concorde ? \\[0pt]
La politique doit-elle viser le consensus ? \\[0pt]
La politique du pire \\[0pt]
La politique échappe-t-elle à l'exigence de vérité ? \\[0pt]
La politique est-elle affaire de compétence ? \\[0pt]
La politique est-elle affaire de décision ? \\[0pt]
La politique est-elle affaire de jugement ? \\[0pt]
La politique est-elle affaire de science ? \\[0pt]
La politique est-elle affaire d'expérience ou de théorie ? \\[0pt]
La politique est-elle architectonique ? \\[0pt]
La politique est-elle continuation de la guerre, par d'autres moyens ? \\[0pt]
La politique est-elle extérieure au droit ? \\[0pt]
La politique est-elle la continuation de la guerre ? \\[0pt]
La politique est-elle la continuation de la guerre par d'autres moyens ? \\[0pt]
La politique est-elle l'affaire des spécialistes ? \\[0pt]
La politique est-elle l'affaire de tous ? \\[0pt]
La politique est-elle l'art de convaincre le peuple ? \\[0pt]
La politique est-elle l'art de la délibération ? \\[0pt]
La politique est-elle l'art des possibles ? \\[0pt]
La politique est-elle l'art du possible ? \\[0pt]
La politique est-elle le refus de la violence ? \\[0pt]
La politique est-elle le règne des passions ? \\[0pt]
La politique est-elle naturelle ? \\[0pt]
La politique est-elle par nature sujette à dispute ? \\[0pt]
La politique est-elle plus importante que tout ? \\[0pt]
La politique est-elle un art ? \\[0pt]
La politique est-elle une affaire d'experts ? \\[0pt]
La politique est-elle une science ? \\[0pt]
La politique est-elle une technique ? \\[0pt]
La politique est-elle un métier ? \\[0pt]
La politique est-elle un moyen ou une fin ? \\[0pt]
La politique et la gloire \\[0pt]
La politique et la guerre \\[0pt]
La politique et la ville \\[0pt]
La politique et le bonheur \\[0pt]
La politique et le mal \\[0pt]
La politique et le politique \\[0pt]
La politique et le pouvoir \\[0pt]
La politique et les passions \\[0pt]
La politique et l'opinion \\[0pt]
La politique et l'utopie \\[0pt]
La politique exclut-elle le désordre ? \\[0pt]
La politique implique-t-elle la hiérarchie ? \\[0pt]
La politique n'est-elle que l'art de conquérir et de conserver le pouvoir ? \\[0pt]
La politique peut-elle changer la société ? \\[0pt]
La politique peut-elle changer le monde ? \\[0pt]
La politique peut-elle disparaître ? \\[0pt]
La politique peut-elle échapper au mythe ? \\[0pt]
La politique peut-elle être indépendante de la morale ? \\[0pt]
La politique peut-elle être morale ? \\[0pt]
La politique peut-elle être objet de science ? \\[0pt]
La politique peut-elle être un objet de science ? \\[0pt]
La politique peut-elle n'être qu'une pratique ? \\[0pt]
La politique peut-elle se passer de croyance ? \\[0pt]
La politique peut-elle se passer de croyances ? \\[0pt]
La politique peut-elle se passer des idéologies ? \\[0pt]
La politique peut-elle unir les hommes ? \\[0pt]
La politique peut-elle viser à autre chose qu'à l'efficacité ? \\[0pt]
La politique repose-t-elle sur un contrat ? \\[0pt]
La politique requière-t-elle le compromis \\[0pt]
La politique scientifique \\[0pt]
La politique se réduit-elle à des rapports de force ? \\[0pt]
La politique suppose-t-elle la morale ? \\[0pt]
La politique suppose-t-elle une idée de l'homme ? \\[0pt]
La politique vise-t-elle à dépasser les injustices ? \\[0pt]
L'apolitisme \\[0pt]
La populace \\[0pt]
La population \\[0pt]
L'aporie \\[0pt]
La pornographie \\[0pt]
La position \\[0pt]
La possession \\[0pt]
La possibilité \\[0pt]
La possibilité logique \\[0pt]
La possibilité métaphysique \\[0pt]
La possibilité réelle \\[0pt]
La postérité \\[0pt]
La potentialité \\[0pt]
La poursuite de mon intérêt m'oppose-t-elle aux autres ? \\[0pt]
L'apparence \\[0pt]
L'apparence du pouvoir \\[0pt]
L'apparence est-elle toujours trompeuse ? \\[0pt]
L'apparence fait-elle obstacle à la vérité ? \\[0pt]
L'apparence fait-elle partie de la réalité ? \\[0pt]
L'appartenance sociale \\[0pt]
L'appel \\[0pt]
L'appréciation de la nature \\[0pt]
L'apprentissage \\[0pt]
L'apprentissage de la langue \\[0pt]
L'apprentissage de la liberté \\[0pt]
L'approche psychologique de l'homme met-elle en question la notion d'âme ? \\[0pt]
L'appropriation \\[0pt]
L'approximation \\[0pt]
La pratique \\[0pt]
La pratique de l'espace \\[0pt]
La pratique des sciences met-elle à l'abri des préjugés ? \\[0pt]
La précarité \\[0pt]
La précaution \\[0pt]
La précaution peut-elle être un principe ? \\[0pt]
La précision \\[0pt]
La préhistoire \\[0pt]
La première fois \\[0pt]
La première vérité \\[0pt]
La présence \\[0pt]
La présence à soi \\[0pt]
La présence d'autrui nous évite-t-elle la solitude ? \\[0pt]
La présence de l'œuvre d'art \\[0pt]
La présence d'esprit \\[0pt]
La présence du passé \\[0pt]
La présomption \\[0pt]
La présomption d'innocence \\[0pt]
La pression du groupe \\[0pt]
La preuve \\[0pt]
La preuve de l'existence de Dieu \\[0pt]
La preuve expérimentale \\[0pt]
La prévision \\[0pt]
La prévision des conduites rationnelles \\[0pt]
La prévoyance \\[0pt]
La prière \\[0pt]
L'\emph{a priori} \\[0pt]
L'a priori peut-il être historique ? \\[0pt]
La prise de parti est-elle essentielle en politique ? \\[0pt]
La prise du pouvoir \\[0pt]
La prise en charge de l'individuel par la psychanalyse \\[0pt]
La prison \\[0pt]
La prison est-elle utile ? \\[0pt]
La privation \\[0pt]
La privation de liberté \\[0pt]
La probabilité \\[0pt]
La probité \\[0pt]
La production \\[0pt]
La productivité de l'art \\[0pt]
La profondeur \\[0pt]
La prohibition de l'inceste \\[0pt]
La promenade \\[0pt]
La promesse \\[0pt]
La promesse et le contrat \\[0pt]
La promesse n'est-elle qu'un acte de langage ? \\[0pt]
La propagande \\[0pt]
La propagation des idées \\[0pt]
La proportion \\[0pt]
L'à propos \\[0pt]
La proposition \\[0pt]
La propreté \\[0pt]
La propriété \\[0pt]
La propriété, est-ce un vol ? \\[0pt]
La propriété est-elle un droit ? \\[0pt]
La propriété est-elle une garantie de liberté ? \\[0pt]
La propriété et le travail \\[0pt]
La prose du monde \\[0pt]
La protection \\[0pt]
La protection sociale \\[0pt]
La providence \\[0pt]
La prudence \\[0pt]
La prudence est-elle une vertu politique ? \\[0pt]
La psychanalyse donne-t-elle un sens au rêve ? \\[0pt]
La psychanalyse est-elle une science ? \\[0pt]
La psychanalyse est-elle une science humaine ? \\[0pt]
La psychanalyse permet-elle la construction du sujet ? \\[0pt]
La psychanalyse s'applique-t-elle aux phénomènes sociaux ? \\[0pt]
La psychologie dit-elle ce qu'est le moi ? \\[0pt]
La psychologie est-elle l'étude de ce qui n'est pas social ? \\[0pt]
La psychologie est-elle l'étude des comportements individuels ? \\[0pt]
La psychologie est-elle une science ? \\[0pt]
La psychologie est-elle une science de la nature ? \\[0pt]
La psychologie est-elle une science humaine ? \\[0pt]
La publicité \\[0pt]
La pudeur \\[0pt]
La puissance \\[0pt]
La puissance de la raison \\[0pt]
La puissance de la technique \\[0pt]
La puissance de l'État \\[0pt]
La puissance de l'image \\[0pt]
La puissance de l'imagination \\[0pt]
La puissance des contraires \\[0pt]
La puissance des images \\[0pt]
La puissance du désir \\[0pt]
La puissance du faux \\[0pt]
La puissance du langage \\[0pt]
La puissance du peuple \\[0pt]
La puissance et l'acte \\[0pt]
La puissance et le possible \\[0pt]
La puissance technique est- elle sans limites ? \\[0pt]
La pulsion \\[0pt]
La punition \\[0pt]
La pureté \\[0pt]
La qualité \\[0pt]
La quantité \\[0pt]
La quantité et la qualité \\[0pt]
La question de la nature humaine relève-t-elle des sciences humaines ? \\[0pt]
La question de l'essence \\[0pt]
La question de l'œuvre d'art \\[0pt]
La question de l'origine \\[0pt]
La question des origines \\[0pt]
La question du sens de la vie peut-elle être sans réponses ? \\[0pt]
La question « qu'est-ce que l'homme ? » est-elle scientifique ? \\[0pt]
La question : « qui ? » \\[0pt]
La question « qui suis-je » admet-elle une réponse exacte ? \\[0pt]
La question sociale \\[0pt]
La quête des origines \\[0pt]
La quête du sens ultime \\[0pt]
La racine du mal \\[0pt]
La radicalité \\[0pt]
La radicalité est-elle une exigence philosophique ? \\[0pt]
La raison \\[0pt]
La raison a-t-elle des limites ? \\[0pt]
La raison a-t-elle le droit d'expliquer ce que morale condamne ? \\[0pt]
La raison a-t-elle pour fin la connaissance ? \\[0pt]
La raison a-t-elle toujours raison ? \\[0pt]
La raison a-t-elle une histoire ? \\[0pt]
La raison des mythes \\[0pt]
La raison d'État \\[0pt]
La raison d'État peut-elle être justifiée ? \\[0pt]
La raison d'être \\[0pt]
La raison doit-elle critiquer la croyance ? \\[0pt]
La raison doit-elle être cultivée ? \\[0pt]
La raison doit-elle être notre guide ? \\[0pt]
La raison doit-elle faire une place à la croyance ? \\[0pt]
La raison doit-elle se soumettre au réel ? \\[0pt]
La raison du plus fort \\[0pt]
La raison engendre-t-elle des illusions ? \\[0pt]
La raison épuise-t-elle le réel ? \\[0pt]
La raison, est-ce la modération ? \\[0pt]
La raison est-elle impersonnelle ? \\[0pt]
La raison est-elle le pouvoir de distinguer le vrai du faux ? \\[0pt]
La raison est-elle l'esclave des passions ? \\[0pt]
La raison est-elle l'esclave du désir ? \\[0pt]
La raison est-elle morale par elle-même ? \\[0pt]
La raison est-elle plus fiable que l'expérience ? \\[0pt]
La raison est-elle seulement affaire de logique ? \\[0pt]
La raison est-elle suffisante ? \\[0pt]
La raison est-elle toujours raisonnable ? \\[0pt]
La raison est-elle une valeur ? \\[0pt]
La raison est-elle un instrument ? \\[0pt]
La raison est-elle un obstacle au bonheur ? \\[0pt]
La raison et la démonstration \\[0pt]
La raison et la liberté \\[0pt]
La raison et le bonheur \\[0pt]
La raison et le calcul \\[0pt]
La raison et le réel \\[0pt]
La raison et l'existence \\[0pt]
La raison et l'expérience \\[0pt]
La raison et l'irrationnel \\[0pt]
La raison gouverne-t-elle le monde ? \\[0pt]
La raison ne connaît-elle du réel que ce qu'elle y met d'elle-même ? \\[0pt]
La raison ne veut-elle que connaître ? \\[0pt]
La raison peut-elle entrer en conflit avec elle-même ? \\[0pt]
La raison peut-elle entrer en contradiction avec elle-même ? \\[0pt]
La raison peut-elle errer ? \\[0pt]
La raison peut-elle être immédiatement pratique ? \\[0pt]
La raison peut-elle être pratique ? \\[0pt]
La raison peut-elle nous commander de croire ? \\[0pt]
La raison peut-elle nous égarer ? \\[0pt]
La raison peut-elle nous induire en erreur ? \\[0pt]
La raison peut-elle rendre compte du réel entièrement ? \\[0pt]
La raison peut-elle rendre raison de tout ? \\[0pt]
La raison peut-elle s'aveugler elle-même ? \\[0pt]
La raison peut-elle se contredire ? \\[0pt]
La raison peut-elle servir le mal ? \\[0pt]
La raison peut-elle s'opposer à elle-même ? \\[0pt]
La raison pratique \\[0pt]
La raison s'identifie-t-elle au calcul ? \\[0pt]
La raison s'oppose-t-elle aux passions ? \\[0pt]
La raison suffisante \\[0pt]
La raison transforme-t-elle le réel ? \\[0pt]
La rareté \\[0pt]
La rationalité \\[0pt]
La rationalité de la psychanalyse \\[0pt]
La rationalité de l'homo œconomicus \\[0pt]
La rationalité des choix politiques \\[0pt]
La rationalité des comportements économiques \\[0pt]
La rationalité des émotions \\[0pt]
La rationalité du langage \\[0pt]
La rationalité du marché \\[0pt]
La rationalité économique \\[0pt]
La rationalité en sciences sociales \\[0pt]
La rationalité politique \\[0pt]
La rationalité se confond-elle avec la systématicité ? \\[0pt]
La rationalité technique \\[0pt]
L'arbitraire \\[0pt]
L'arbitraire de la règle \\[0pt]
L'arbitraire du signe \\[0pt]
L'arbitraire en politique \\[0pt]
L'archéologie \\[0pt]
L'architecte : artiste ou ingénieur ? \\[0pt]
L'architecte : artiste ou l'ingénieur \\[0pt]
L'architecte dans la cité \\[0pt]
L'architecte et la cité \\[0pt]
L'architecte et le législateur \\[0pt]
L'architecte et l'ingénieur \\[0pt]
L'architecture du monde \\[0pt]
L'architecture est-elle un art ? \\[0pt]
L'architecture et l'espace \\[0pt]
L'archive \\[0pt]
La réaction \\[0pt]
La réaction en politique \\[0pt]
La réalisation du devoir exclut-elle toute forme de plaisir ? \\[0pt]
La réalité \\[0pt]
La réalité a-t-elle une forme logique ? \\[0pt]
La réalité décrite par la science s'oppose-t-elle à la démonstration ? \\[0pt]
La réalité de la contradiction \\[0pt]
La réalité de la vie s'épuise-t-elle dans celle des vivants ? \\[0pt]
La réalité de l'espace \\[0pt]
La réalité de l'idéal \\[0pt]
La réalité de l'idée \\[0pt]
La réalité de l'inconscient \\[0pt]
La réalité des idées \\[0pt]
La réalité des phénomènes \\[0pt]
La réalité du beau \\[0pt]
La réalité du bien \\[0pt]
La réalité du corps \\[0pt]
La réalité du désordre \\[0pt]
La réalité du futur \\[0pt]
La réalité du mal \\[0pt]
La réalité du monde extérieur \\[0pt]
La réalité du mouvement \\[0pt]
La réalité du passé \\[0pt]
La réalité du possible \\[0pt]
La réalité du progrès \\[0pt]
La réalité du rêve \\[0pt]
La réalité du sensible \\[0pt]
La réalité du temps \\[0pt]
La réalité du temps se réduit-elle à la conscience que nous en avons ? \\[0pt]
La réalité est-elle une idée ? \\[0pt]
La réalité mentale \\[0pt]
La réalité n'est-elle qu'une construction ? \\[0pt]
La réalité nourrit-elle la fiction ? \\[0pt]
La réalité peut-elle être virtuelle ? \\[0pt]
La réalité physique \\[0pt]
La réalité sensible \\[0pt]
La réalité sociale \\[0pt]
La réalité symbolique \\[0pt]
La réalité virtuelle \\[0pt]
La réception de l'œuvre d'art \\[0pt]
La recherche \\[0pt]
La recherche-action en sciences sociales \\[0pt]
La recherche de l'absolu \\[0pt]
La recherche de la perfection \\[0pt]
La recherche de l'authenticité \\[0pt]
La recherche de la vérité \\[0pt]
La recherche de la vérité a-t- elle une fin ? \\[0pt]
La recherche de la vérité dans les sciences humaines \\[0pt]
La recherche de la vérité peut-elle être désintéressée ? \\[0pt]
La recherche de la vérité peut-elle être une passion ? \\[0pt]
La recherche des causes \\[0pt]
La recherche des invariants \\[0pt]
La recherche des origines \\[0pt]
La recherche d'identité \\[0pt]
La recherche du bonheur \\[0pt]
La recherche du bonheur est-elle égoïste ? \\[0pt]
La recherche du bonheur est-elle un idéal égoïste ? \\[0pt]
La recherche du bonheur est-il un guide suffisant dans la vie ? \\[0pt]
La recherche du bonheur peut-elle être un devoir ? \\[0pt]
La recherche du bonheur suffit-elle à déterminer une morale ? \\[0pt]
La recherche scientifique est-elle désintéressée ? \\[0pt]
La réciprocité \\[0pt]
La réciprocité est-elle indispensable à la communauté politique ? \\[0pt]
La réconciliation \\[0pt]
La reconnaissance \\[0pt]
La reconnaissance de la diversité des cultures condamne-t-elle au relativisme ? \\[0pt]
La rectitude \\[0pt]
La rectitude du droit \\[0pt]
La référence \\[0pt]
La référence à la mémoire en histoire \\[0pt]
La référence aux faits suffit-elle à garantir l'objectivité de la connaissance ? \\[0pt]
La réflexion \\[0pt]
La réflexion sur l'expérience participe-t-elle de l'expérience ? \\[0pt]
La réforme \\[0pt]
La réforme des institutions \\[0pt]
La réfutation \\[0pt]
La règle \\[0pt]
La règle du jeu \\[0pt]
La règle et l'exception \\[0pt]
La régression \\[0pt]
La régression à l'infini \\[0pt]
La régularité \\[0pt]
La régulation \\[0pt]
La relation \\[0pt]
La relation à autrui est-elle symétrique ? \\[0pt]
La relation de causalité est-elle temporelle ? \\[0pt]
La relation de cause à effet \\[0pt]
La relation de nécessité \\[0pt]
La relation d'identité \\[0pt]
La relativité \\[0pt]
La religion \\[0pt]
La religion a-t-elle besoin d'un dieu ? \\[0pt]
La religion a-t-elle des vertus ? \\[0pt]
La religion a-t-elle les mêmes fins que la morale ? \\[0pt]
La religion a-t-elle une fonction de cohésion sociale ? \\[0pt]
La religion a-t-elle une fonction sociale ? \\[0pt]
La religion civile \\[0pt]
La religion conduit-elle l'homme au-delà de lui-même ? \\[0pt]
La religion de l'art \\[0pt]
La religion divise-t-elle les hommes ? \\[0pt]
La religion est-elle à craindre ? \\[0pt]
La religion est-elle au fondement de la morale ? \\[0pt]
La religion est-elle contraire à la raison ? \\[0pt]
La religion est-elle fondée sur la peur de la mort ? \\[0pt]
La religion est-elle la sagesse des pauvres ? \\[0pt]
La religion est-elle l'asile de l'ignorance ? \\[0pt]
La religion est-elle l'opium du peuple ? \\[0pt]
La religion est-elle relation à l'absolu ? \\[0pt]
La religion est-elle simple affaire de croyance ? \\[0pt]
La religion est-elle source de conflit ? \\[0pt]
La religion est-elle une affaire privée ? \\[0pt]
La religion est-elle une consolation pour les hommes ? \\[0pt]
La religion est-elle une production culturelle comme les autres ? \\[0pt]
La religion est-elle un facteur de lien social ? \\[0pt]
La religion est-elle un instrument de pouvoir ? \\[0pt]
La religion est-elle un obstacle à la liberté ? \\[0pt]
La religion et la croyance \\[0pt]
La religion implique-t-elle la croyance en un être divin ? \\[0pt]
La religion impose t-elle un joug salutaire à l'intelligence ? \\[0pt]
La religion naturelle \\[0pt]
La religion n'est-elle que l'affaire des croyants ? \\[0pt]
La religion n'est-elle qu'une affaire privée ? \\[0pt]
La religion n'est-elle qu'un fait de culture ? \\[0pt]
La religion outrepasse-t-elle les limites de la raison ? \\[0pt]
La religion peut-elle être civile ? \\[0pt]
La religion peut-elle être naturelle ? \\[0pt]
La religion peut-elle faire lien social ? \\[0pt]
La religion peut-elle n'être qu'une affaire privée ? \\[0pt]
La religion peut-elle se passer de transcendant ? \\[0pt]
La religion peut-elle se passer du recours au sacré ? \\[0pt]
La religion peut-elle suppléer la raison ? \\[0pt]
La religion relève-t-elle de l'irrationnel ? \\[0pt]
La religion relève-t-elle de l'opinion ? \\[0pt]
La religion relie-t-elle les hommes ? \\[0pt]
La religion rend-elle l'homme heureux ? \\[0pt]
La religion rend-elle meilleur ? \\[0pt]
La religion repose-t-elle sur une illusion ? \\[0pt]
La religion se distingue-t-elle de la superstition ? \\[0pt]
La religion se réduit-elle à la foi ? \\[0pt]
La réminiscence \\[0pt]
La renaissance \\[0pt]
La Renaissance \\[0pt]
La rencontre \\[0pt]
La rencontre d'autrui \\[0pt]
La réparation \\[0pt]
La répétition \\[0pt]
La représentation \\[0pt]
La représentation artistique \\[0pt]
La représentation en politique \\[0pt]
La représentation politique \\[0pt]
La reproductibilité de l'œuvre d'art \\[0pt]
La reproduction \\[0pt]
La reproduction des œuvres d'art \\[0pt]
La reproduction des œuvres d'art nuit-elle à l'art ? \\[0pt]
La reproduction sociale \\[0pt]
La république \\[0pt]
La réputation \\[0pt]
La résignation \\[0pt]
La résilience \\[0pt]
La résistance \\[0pt]
La résistance à l'oppression \\[0pt]
La résistance de la matière \\[0pt]
La résistance du réel \\[0pt]
La résolution \\[0pt]
La responsabilité \\[0pt]
La responsabilité collective \\[0pt]
La responsabilité de l'artiste \\[0pt]
La responsabilité envers les générations futures \\[0pt]
La responsabilité est-elle une fiction juridique ? \\[0pt]
La responsabilité implique-t-elle autrui ? \\[0pt]
La responsabilité peut-elle être collective ? \\[0pt]
La responsabilité politique \\[0pt]
La responsabilité politique n'est-elle le fait que de ceux qui gouvernent ? \\[0pt]
La ressemblance \\[0pt]
La restauration des œuvres d'art \\[0pt]
La réussite \\[0pt]
La révélation \\[0pt]
La rêverie \\[0pt]
La révolte \\[0pt]
La révolte peut-elle être un droit ? \\[0pt]
La révolution \\[0pt]
L'argent \\[0pt]
L'argent est-il la mesure de tout échange ? \\[0pt]
L'argent est-il un mal nécessaire ? \\[0pt]
L'argent et la valeur \\[0pt]
L'argumentation \\[0pt]
L'argumentation morale \\[0pt]
L'argument d'autorité \\[0pt]
La rhétorique \\[0pt]
La rhétorique a-t-elle une valeur ? \\[0pt]
La rhétorique est-elle un art ? \\[0pt]
La richesse \\[0pt]
La richesse du sensible \\[0pt]
La richesse intérieure \\[0pt]
La rigueur \\[0pt]
La rigueur de la loi \\[0pt]
La rigueur des lois \\[0pt]
La rigueur des lois ? \\[0pt]
La rigueur morale \\[0pt]
La rime et la raison \\[0pt]
L'aristocratie \\[0pt]
La rivalité \\[0pt]
L'arme rhétorique \\[0pt]
L'art \\[0pt]
L'art abstrait \\[0pt]
L'art ajoute-t-il quelque chose à la vie ? \\[0pt]
L'art à l'épreuve du goût \\[0pt]
L'art apprend-il à percevoir ? \\[0pt]
L'art a-t-il à être populaire ? \\[0pt]
L'art a-t-il besoin de théorie ? \\[0pt]
L'art a-t-il besoin d'un discours sur l'art ? \\[0pt]
L'art a-t-il des règles ? \\[0pt]
L'art a-t-il des vertus thérapeutiques ? \\[0pt]
L'art a-t-il plus de valeur que la vérité ? \\[0pt]
L'art a-t-il pour fin la vérité ? \\[0pt]
L'art a-t-il pour fin le plaisir ? \\[0pt]
L'art a-t-il pour fonction de sublimer le réel ? \\[0pt]
L'art a-t-il une fin morale ? \\[0pt]
L'art a-t-il une histoire ? \\[0pt]
L'art a-t-il une responsabilité morale ? \\[0pt]
L'art a-t-il une valeur sociale ? \\[0pt]
L'art a-t-il un rôle à jouer dans l'éducation ? \\[0pt]
L'art change-t-il la vie ? \\[0pt]
L'art cinématographique \\[0pt]
L'art contre la beauté ? \\[0pt]
L'art décoratif \\[0pt]
L'art d'écrire \\[0pt]
L'art décrit-il ? \\[0pt]
L'art de faire croire \\[0pt]
L'art de gouverner \\[0pt]
L'art de juger \\[0pt]
L'art de la discussion \\[0pt]
L'art de la guerre \\[0pt]
L'art de masse \\[0pt]
L'art de persuader \\[0pt]
L'art des images \\[0pt]
L'art de vivre \\[0pt]
L'art de vivre est-il un art ? \\[0pt]
L'art d'interpréter \\[0pt]
L'art d'inventer \\[0pt]
L'art doit-il divertir ? \\[0pt]
L'art doit-il être critique ? \\[0pt]
L'art doit-il nécessairement représenter la réalité ? \\[0pt]
L'art doit-il nous étonner ? \\[0pt]
L'art doit-il refaire le monde ? \\[0pt]
L'art donne-t-il à penser ? \\[0pt]
L'art donne-t-il à rêver ou à penser ? \\[0pt]
L'art donne-t-il à voir l'invisible ? \\[0pt]
L'art donne-t-il nécessairement lieu à la production d'une œuvre ? \\[0pt]
L'art dramatique \\[0pt]
L'art du comédien \\[0pt]
L'art du corps \\[0pt]
L'art du mensonge \\[0pt]
L'art du portrait \\[0pt]
L'art échappe-t-il à la raison ? \\[0pt]
L'art éduque-t-il la perception ? \\[0pt]
L'art éduque-t-il l'homme ? \\[0pt]
L'art engagé \\[0pt]
L'art, est-ce ce qui résiste à la certitude ? \\[0pt]
L'art est-il affaire d'apparence ? \\[0pt]
L'art est-il affaire de goût ? \\[0pt]
L'art est-il affaire d'imagination ? \\[0pt]
L'art est-il à lui-même son propre but ? \\[0pt]
L'art est-il au service du beau ? \\[0pt]
L'art est-il ce qui permet de partager ses émotions ? \\[0pt]
L'art est-il désintéressé ? \\[0pt]
L'art est-il destiné à embellir ? \\[0pt]
L'art est-il hors du temps ? \\[0pt]
L'art est-il imitatif ? \\[0pt]
L'art est-il interprétation ? \\[0pt]
L'art est-il inutile ? \\[0pt]
L'art est-il le miroir du monde ? \\[0pt]
L'art est-il le produit de l'inconscient ? \\[0pt]
L'art est-il le propre de l'homme ? \\[0pt]
L'art est-il le règne des apparences ? \\[0pt]
L'art est-il mensonger ? \\[0pt]
L'art est-il méthodique ? \\[0pt]
L'art est-il moins nécessaire que la science ? \\[0pt]
L'art est-il objet de compréhension ? \\[0pt]
L'art est-il politique ? \\[0pt]
L'art est-il révolutionnaire ? \\[0pt]
L'art est-il subversif ? \\[0pt]
L'art est-il un divertissement ? \\[0pt]
L'art est-il une affaire sérieuse ? \\[0pt]
L'art est-il une connaissance ? \\[0pt]
L'art est-il une critique de la culture ? \\[0pt]
L'art est-il une expérience de la liberté ? \\[0pt]
L'art est-il une forme de langage ? \\[0pt]
L'art est-il une histoire ? \\[0pt]
L'art est-il une promesse de bonheur ? \\[0pt]
L'art est-il une valeur ? \\[0pt]
L'art est-il universel ? \\[0pt]
L'art est-il un jeu ? \\[0pt]
L'art est-il un langage ? \\[0pt]
L'art est-il un langage universel ? \\[0pt]
L'art est-il un luxe ? \\[0pt]
L'art est-il un mode de connaissance ? \\[0pt]
L'art est-il un modèle pour la philosophie ? \\[0pt]
L'art est-il un monde ? \\[0pt]
L'art est-il un moyen de connaître ? \\[0pt]
L'art est-il un refuge ? \\[0pt]
L'art est par-delà beauté et laideur ? \\[0pt]
L'art et la manière \\[0pt]
L'art et la morale \\[0pt]
L'art et la mort \\[0pt]
L'art et la nature \\[0pt]
L'art et la nouveauté \\[0pt]
L'art et la raison \\[0pt]
L'art et l'artifice \\[0pt]
L'art et la société \\[0pt]
L'art et la technique \\[0pt]
L'art et la tradition \\[0pt]
L'art et la vérité \\[0pt]
L'art et la vie \\[0pt]
L'art et le beau \\[0pt]
L'art et le divin \\[0pt]
L'art et le jeu \\[0pt]
L'art et le mouvement \\[0pt]
L'art et le mythe \\[0pt]
L'art et l'éphémère \\[0pt]
L'art et le réel \\[0pt]
L'art et le rêve \\[0pt]
L'art et le sacré \\[0pt]
L'art et les arts \\[0pt]
L'art et l'espace \\[0pt]
L'art et le temps \\[0pt]
L'art et le travail \\[0pt]
L'art et le vivant \\[0pt]
L'art et l'illusion \\[0pt]
L'art et l'immoralité \\[0pt]
L'art et l'interdit \\[0pt]
L'art et l'interprétation \\[0pt]
L'art et l'invisible \\[0pt]
L'art et morale \\[0pt]
L'art et ses institutions \\[0pt]
L'art éveille-t-il des émotions spécifiques ? \\[0pt]
L'art : expérience, exercice ou habitude ? \\[0pt]
L'art exprime-t-il ce que nous ne saurions dire ? \\[0pt]
L'art fait-il penser ? \\[0pt]
L'artifice \\[0pt]
L'artificiel \\[0pt]
L'art imite-t-il la nature ? \\[0pt]
L'art industriel \\[0pt]
L'artisan et l'ingénieur \\[0pt]
L'artiste \\[0pt]
L'artiste a-t-il besoin de modèle ? \\[0pt]
L'artiste a-t-il besoin d'une idée de l'art ? \\[0pt]
L'artiste a-t-il besoin d'un public ? \\[0pt]
L'artiste a-t-il toujours raison ? \\[0pt]
L'artiste a-t-il une méthode ? \\[0pt]
L'artiste dans la cité \\[0pt]
L'artiste de soi-même \\[0pt]
L'artiste doit-il être de son temps ? \\[0pt]
L'artiste doit-il être original ? \\[0pt]
L'artiste doit-il se donner des modèles ? \\[0pt]
L'artiste doit-il se soucier du goût du public ? \\[0pt]
L'artiste est-il le mieux placé pour comprendre son œuvre ? \\[0pt]
L'artiste est-il le seul travailleur libre ? \\[0pt]
L'artiste est-il le technicien du beau ? \\[0pt]
L'artiste est-il maître de son œuvre ? \\[0pt]
L'artiste est-il souverain ? \\[0pt]
L'artiste est-il un créateur ? \\[0pt]
L'artiste est-il un métaphysicien ? \\[0pt]
L'artiste est-il un travailleur ? \\[0pt]
L'artiste et l'artisan \\[0pt]
L'artiste et la sensation \\[0pt]
L'artiste et la société \\[0pt]
L'artiste et le savant \\[0pt]
L'artiste et son public \\[0pt]
L'artiste exprime-t-il quelque chose ? \\[0pt]
L'artiste peut-il se passer d'un maître ? \\[0pt]
L'artiste recherche-t-il le beau ? \\[0pt]
L'artiste sait-il ce qu'il fait ? \\[0pt]
L'artiste travaille-t-il ? \\[0pt]
L'art modifie-t-il notre rapport à la réalité ? \\[0pt]
L'art modifie-t-il notre rapport au réel ? \\[0pt]
L'art n'est-il pas toujours politique ? \\[0pt]
L'art n'est-il pas toujours religieux ? \\[0pt]
L'art n'est-il qu'apparence ? \\[0pt]
L'art n'est-il qu'un artifice ? \\[0pt]
L'art n'est-il qu'une affaire d'esthétique ? \\[0pt]
L'art n'est-il qu'une question de sentiment ? \\[0pt]
L'art n'est-il qu'un mode d'expression subjectif ? \\[0pt]
L'art n'est qu'une affaire de goût ? \\[0pt]
L'art nous détourne-t-il de la réalité ? \\[0pt]
L'art nous donne-t-il des raisons d'espérer ? \\[0pt]
L'art nous enseigne-t-il quelque chose ? \\[0pt]
L'art nous fait-il mieux percevoir le réel ? \\[0pt]
L'art nous libère-t-il de l'insignifiance ? \\[0pt]
L'art nous mène-t-il au vrai ? \\[0pt]
L'art nous permet-il de lutter contre l'irréversibilité ? \\[0pt]
L'art nous ramène-t-il à la réalité ? \\[0pt]
L'art nous réconcilie-t-il avec le monde ? \\[0pt]
L'art officiel \\[0pt]
L'art ou les arts \\[0pt]
L'art parachève-t-il la nature ? \\[0pt]
L'art participe-t-il à la vie politique ? \\[0pt]
L'art pense-t-il ? \\[0pt]
L'art permet-il un accès au divin ? \\[0pt]
L'art peut-il changer le monde ? \\[0pt]
L'art peut-il contribuer à éduquer les hommes ? \\[0pt]
L'art peut-il encore imiter la nature ? \\[0pt]
L'art peut-il être abstrait ? \\[0pt]
L'art peut-il être brut ? \\[0pt]
L'art peut-il être conceptuel ? \\[0pt]
L'art peut-il être enseigné ? \\[0pt]
L'art peut-il être populaire ? \\[0pt]
L'art peut-il être réaliste \\[0pt]
L'art peut-il être révolutionnaire ? \\[0pt]
L'art peut-il être sans œuvre ? \\[0pt]
L'art peut-il être utile ? \\[0pt]
L'art peut-il finir ? \\[0pt]
L'art peut-il mourir ? \\[0pt]
L'art peut-il ne pas être sacré ? \\[0pt]
L'art peut-il n'être aucunement mimétique ? \\[0pt]
L'art peut-il n'être pas conceptuel ? \\[0pt]
L'art peut-il nous rendre meilleurs ? \\[0pt]
L'art peut-il prétendre à la vérité ? \\[0pt]
L'art peut-il quelque chose contre la morale ? \\[0pt]
L'art peut-il quelque chose pour la morale ? \\[0pt]
L'art peut-il rendre le mouvement ? \\[0pt]
L'art peut-il s'affranchir des lois ? \\[0pt]
L'art peut-il sauver le monde ? \\[0pt]
L'art peut-il s'enseigner ? \\[0pt]
L'art peut-il se passer de la beauté ? \\[0pt]
L'art peut-il se passer de règles ? \\[0pt]
L'art peut-il se passer des critères du beau et du laid ? \\[0pt]
L'art peut-il se passer d'idéal ? \\[0pt]
L'art peut-il se passer d'œuvres ? \\[0pt]
L'art peut-il tenir lieu de métaphysique ? \\[0pt]
L'art politique \\[0pt]
L'art populaire \\[0pt]
L'art pour l'art \\[0pt]
L'art produit-il nécessairement des œuvres ? \\[0pt]
L'art progresse-t-il ? \\[0pt]
L'art prolonge-t-il la nature ? \\[0pt]
L'art rend-il heureux ? \\[0pt]
L'art rend-il les hommes meilleurs ? \\[0pt]
L'art s'adresse-t-il à la sensibilité ? \\[0pt]
L'art s'adresse-t-il à tous ? \\[0pt]
L'art sait-il montrer ce que le langage ne peut pas dire ? \\[0pt]
L'art s'apparente-t-il à la philosophie ? \\[0pt]
L'art s'apprend-il ? \\[0pt]
L'art : une arithmétique sensible ? \\[0pt]
L'art vise-t-il le beau ? \\[0pt]
L'art vise-t-il nécessairement le beau ? \\[0pt]
La ruine \\[0pt]
La rumeur \\[0pt]
La rupture \\[0pt]
La ruse \\[0pt]
La ruse technique \\[0pt]
La sacralisation de l'œuvre \\[0pt]
La sagesse \\[0pt]
La sagesse du corps \\[0pt]
La sagesse et la passion \\[0pt]
La sagesse et l'expérience \\[0pt]
La sagesse rend-elle heureux ? \\[0pt]
La sainteté \\[0pt]
La sanction \\[0pt]
La santé \\[0pt]
La santé est-elle un devoir ? \\[0pt]
La santé est-elle un droit ou un devoir ? \\[0pt]
La santé mentale \\[0pt]
La santé n'est-elle que l'absence de maladie ? \\[0pt]
La satisfaction \\[0pt]
La satisfaction des penchants \\[0pt]
La scène \\[0pt]
La scène du monde \\[0pt]
La scène théâtrale \\[0pt]
L'ascèse \\[0pt]
L'ascétisme \\[0pt]
L'ascétisme est-il une vertu ? \\[0pt]
La science admet-elle des degrés de croyance ? \\[0pt]
La science a-t-elle besoin d'imagination ? \\[0pt]
La science a-t-elle besoin d'un critère de démarcation entre science et non science ? \\[0pt]
La science a-t-elle besoin d'une méthode ? \\[0pt]
La science a-t-elle besoin du principe de causalité ? \\[0pt]
La science a-t-elle des limites ? \\[0pt]
La science a-t-elle le monopole de la raison ? \\[0pt]
La science a-t-elle le monopole de la vérité ? \\[0pt]
La science a-t-elle pour fin de prévoir ? \\[0pt]
La science a-t-elle réponse à tout ? \\[0pt]
La science a-t-elle toujours raison ? \\[0pt]
La science a-t-elle une histoire ? \\[0pt]
La science commence-t-elle avec la perception ? \\[0pt]
La science connaît-elle ses limites ? \\[0pt]
La science découvre-t-elle ou construit-elle son objet ? \\[0pt]
La science de l'être \\[0pt]
La science de l'individuel \\[0pt]
La science dépend-elle nécessairement de l'expérience ? \\[0pt]
La science désenchante-t-elle le monde ? \\[0pt]
La science des mœurs \\[0pt]
La science dévoile-t-elle le réel ? \\[0pt]
La science doit-elle nous fournir des représentations du monde ? \\[0pt]
La science doit-elle se fonder sur une idée de la nature ? \\[0pt]
La science doit-elle se passer de l'idée de finalité ? \\[0pt]
La science du vivant peut-elle se passer de l'idée de finalité ? \\[0pt]
La science est-elle austère ? \\[0pt]
La science est-elle indépendante de toute métaphysique ? \\[0pt]
La science est-elle inhumaine ? \\[0pt]
La science est-elle le lieu de la vérité ? \\[0pt]
La science est-elle une affaire politique ? \\[0pt]
La science est-elle une connaissance du réel ? \\[0pt]
La science est-elle une langue bien faite ? \\[0pt]
La science est-elle un jeu ? \\[0pt]
La science et la foi \\[0pt]
La science et le faux \\[0pt]
La science et le mythe \\[0pt]
La science et les sciences \\[0pt]
La science et l'irrationnel \\[0pt]
La science exclut-elle l'imagination ? \\[0pt]
La science explique-t-elle le réel ? \\[0pt]
La science historique peut-elle être prédictive ? \\[0pt]
« La science ne pense pas » \\[0pt]
La science n'est-elle qu'une activité théorique ? \\[0pt]
La science n'est-elle qu'une fiction ? \\[0pt]
La science nous éloigne-t-elle de la religion ? \\[0pt]
La science nous éloigne-t-elle des choses ? \\[0pt]
La science nous indique-t-elle ce que nous devons faire ? \\[0pt]
La science pense-t-elle ? \\[0pt]
La science permet-elle de comprendre le monde ? \\[0pt]
La science permet-elle de mieux comprendre la religion ? \\[0pt]
La science permet-elle d'expliquer toute la réalité ? \\[0pt]
La science peut-elle errer ? \\[0pt]
La science peut-elle être une métaphysique ? \\[0pt]
La science peut-elle guider notre conduite ? \\[0pt]
La science peut-elle lutter contre les préjugés ? \\[0pt]
La science peut-elle produire des croyances ? \\[0pt]
La science peut-elle se passer de fondement ? \\[0pt]
La science peut-elle se passer de l'idée de finalité ? \\[0pt]
La science peut-elle se passer de métaphysique ? \\[0pt]
La science peut-elle se passer de méthode ? \\[0pt]
La science peut-elle se passer d'hypothèses ? \\[0pt]
La science peut-elle se passer d'institutions ? \\[0pt]
La science peut-elle tout expliquer ? \\[0pt]
La science politique \\[0pt]
La science porte-elle au scepticisme ? \\[0pt]
La science procède-t-elle par rectification ? \\[0pt]
La science rend-elle la religion caduque ? \\[0pt]
La science se limite-t-elle à constater les faits ? \\[0pt]
La science s'oppose-t-elle à la religion ? \\[0pt]
La scientificité des sciences humaines \\[0pt]
La sculpture \\[0pt]
La sculpture et la peinture \\[0pt]
La sculpture et le mouvement \\[0pt]
La sculpture et l'espace \\[0pt]
La seconde nature \\[0pt]
La sécularisation \\[0pt]
La sécurité \\[0pt]
La sécurité nationale \\[0pt]
La sécurité publique \\[0pt]
La séduction \\[0pt]
La séduction en politique \\[0pt]
La ségrégation \\[0pt]
La sensation \\[0pt]
La sensation est-elle une connaissance ? \\[0pt]
La sensibilité \\[0pt]
La sensibilité animale \\[0pt]
La sensibilité peut-elle délivrer une vérité ? \\[0pt]
La sensibilité se cultive-t-elle ? \\[0pt]
La séparation \\[0pt]
La séparation des pouvoirs \\[0pt]
La sérénité \\[0pt]
La servitude \\[0pt]
La servitude humaine \\[0pt]
La servitude peut-elle être volontaire ? \\[0pt]
La servitude volontaire \\[0pt]
La sévérité \\[0pt]
La sexualité \\[0pt]
La signification \\[0pt]
La signification dans l'œuvre \\[0pt]
La signification des mots \\[0pt]
La signification des rites \\[0pt]
La signification en musique \\[0pt]
La signification et l'usage \\[0pt]
L'asile de l'ignorance \\[0pt]
La simplicité \\[0pt]
La simplicité du bien \\[0pt]
La simulation \\[0pt]
La sincérité \\[0pt]
La singularité \\[0pt]
La singularité du réel \\[0pt]
La singularité en morale \\[0pt]
La situation \\[0pt]
La sobriété \\[0pt]
La sociabilité \\[0pt]
La socialisation \\[0pt]
La socialisation des comportements \\[0pt]
La société \\[0pt]
La société civile \\[0pt]
La société civile et l'État \\[0pt]
La société contre l'État \\[0pt]
La société de consommation \\[0pt]
La société des individus \\[0pt]
La société des loisirs \\[0pt]
La société des nations \\[0pt]
La société des savants \\[0pt]
La société doit-elle reconnaître les désirs individuels ? \\[0pt]
La société du genre humain \\[0pt]
La société est-elle concevable sans le travail ? \\[0pt]
La société est-elle un grand marché ? \\[0pt]
La société est-elle un organisme ? \\[0pt]
La société et le droit \\[0pt]
La société et les échanges \\[0pt]
La société et l'État \\[0pt]
La société et l'individu \\[0pt]
La société existe-t-elle ? \\[0pt]
La société fait-elle l'homme ? \\[0pt]
La société peut-elle être l'objet d'une science ? \\[0pt]
La société peut-elle être objet de connaissance ? \\[0pt]
La société peut-elle se passer de l'État ? \\[0pt]
La société précède-t-elle l'individu ? \\[0pt]
La société repose-t-elle sur l'altruisme ? \\[0pt]
La société sans l'État \\[0pt]
La sociologie de la religion \\[0pt]
La sociologie de l'art \\[0pt]
La sociologie de l'art nous permet-elle de comprendre l'art ? \\[0pt]
La sociologie est-elle déterministe ? \\[0pt]
La sociologie est-elle une physique sociale ? \\[0pt]
La sociologie relativise-t-elle la valeur des œuvres d'art ? \\[0pt]
La solidarité \\[0pt]
La solidarité est-elle naturelle ? \\[0pt]
La solitude \\[0pt]
La solitude constitue-t-elle un obstacle à la citoyenneté ? \\[0pt]
La solitude de l'artiste \\[0pt]
La sollicitude \\[0pt]
La somme et le tout \\[0pt]
La souffrance \\[0pt]
La souffrance a-t-elle une valeur morale ? \\[0pt]
La souffrance a-t-elle un sens ? \\[0pt]
La souffrance a-t-elle un sens moral ? \\[0pt]
La souffrance au travail \\[0pt]
La souffrance d'autrui \\[0pt]
La souffrance d'autrui m'importe-t-elle ? \\[0pt]
La souffrance des animaux \\[0pt]
La souffrance morale \\[0pt]
La souffrance peut-elle avoir un sens ? \\[0pt]
La souffrance peut-elle être partagée ? \\[0pt]
La souffrance peut-elle être un mode de connaissance ? \\[0pt]
La soumission \\[0pt]
La soumission à l'autorité \\[0pt]
La soumission volontaire \\[0pt]
La souveraineté \\[0pt]
La souveraineté de l'État \\[0pt]
La souveraineté du peuple \\[0pt]
La souveraineté est-elle indivisible ? \\[0pt]
La souveraineté nationale \\[0pt]
La souveraineté peut-elle être déléguée \\[0pt]
La souveraineté peut-elle être limitée ? \\[0pt]
La souveraineté peut-elle se partager ? \\[0pt]
La souveraineté populaire \\[0pt]
La spécificité des sciences humaines \\[0pt]
La spéculation \\[0pt]
La sphère privée échappe-t-elle au politique ? \\[0pt]
L'aspiration esthétique \\[0pt]
La spontanéité \\[0pt]
L'assassinat politique \\[0pt]
L'assentiment \\[0pt]
L'asservissement à l'utile \\[0pt]
L'asservissement de la pensée \\[0pt]
L'association \\[0pt]
L'association des idées \\[0pt]
La standardisation \\[0pt]
La structure \\[0pt]
La structure et les éléments \\[0pt]
La structure et le sujet \\[0pt]
La subjectivité \\[0pt]
La subsistance \\[0pt]
La substance \\[0pt]
La substance et l'accident \\[0pt]
La substance et le substrat \\[0pt]
La subtilité \\[0pt]
La succession des théories scientifiques \\[0pt]
La superstition \\[0pt]
La sûreté \\[0pt]
La surface et la profondeur \\[0pt]
La surprise \\[0pt]
La surveillance de la société \\[0pt]
La survie \\[0pt]
La sympathie \\[0pt]
La sympathie peut-elle tenir lieu de morale ? \\[0pt]
La sympathie peut-elle tenir lieu de moralité ? \\[0pt]
La systématicité \\[0pt]
La table rase \\[0pt]
La tâche d'exister \\[0pt]
La tautologie \\[0pt]
La technique \\[0pt]
La technique accroît-elle notre liberté ? \\[0pt]
La technique a-t-elle sa place en politique ? \\[0pt]
La technique a-t-elle une finalité ? \\[0pt]
La technique a-t-elle une histoire ? \\[0pt]
La technique augmente-t-elle notre puissance d'agir ? \\[0pt]
La technique change-t-elle l'homme ? \\[0pt]
La technique change-t-elle notre rapport au monde ? \\[0pt]
La technique crée-t-elle son propre monde ? \\[0pt]
La technique déshumanise-t-elle le monde ? \\[0pt]
La technique détermine-t-elle les rapports sociaux ? \\[0pt]
La technique doit-elle nous libérer du travail ? \\[0pt]
La technique doit-elle permettre de dépasser les limites de l'humain ? \\[0pt]
La technique donne-t-elle une illusion de pouvoir ? \\[0pt]
La technique est-elle civilisatrice ? \\[0pt]
La technique est-elle contre nature ? \\[0pt]
La technique est-elle dangereuse ? \\[0pt]
La technique est-elle l'application de la science ? \\[0pt]
La technique est-elle le propre de l'homme ? \\[0pt]
La technique est-elle libératrice ? \\[0pt]
La technique est-elle moralement neutre ? \\[0pt]
La technique est-elle neutre ? \\[0pt]
La technique est-elle une forme de savoir ? \\[0pt]
La technique est-elle un savoir ? \\[0pt]
La technique et la liberté \\[0pt]
La technique et la morale \\[0pt]
La technique et le corps \\[0pt]
La technique et le travail \\[0pt]
La technique facilite-t-elle la vie ? \\[0pt]
La technique fait-elle des miracles ? \\[0pt]
La technique fait-elle violence à la nature ? \\[0pt]
La technique imite-t-elle la nature ? \\[0pt]
La technique invente-t-elle des formes ? \\[0pt]
La technique libère-t-elle les hommes ? \\[0pt]
La technique met-elle le monde à notre disposition ? \\[0pt]
La technique ne fait-elle qu'appliquer la science ? \\[0pt]
La technique ne pose-t-elle que des problèmes techniques ? \\[0pt]
La technique n'est-elle pour l'homme qu'un moyen ? \\[0pt]
La technique n'est-elle qu'une application de la science ? \\[0pt]
La technique n'est-elle qu'une ruse ? \\[0pt]
La technique n'est-elle qu'un moyen ? \\[0pt]
La technique n'est-elle qu'un moyen de domination ? \\[0pt]
La technique n'est-elle qu'un outil au service de l'homme ? \\[0pt]
La technique n'est-elle qu'un prolongement de nos organes ? \\[0pt]
La technique n'est-elle qu'un savoir-faire ? \\[0pt]
La technique n'existe-elle que pour satisfaire des besoins ? \\[0pt]
La technique nous délivre-t-elle d'un rapport irrationnel au monde ? \\[0pt]
La technique nous éloigne-t-elle de la nature ? \\[0pt]
La technique nous éloigne-t-elle de la réalité ? \\[0pt]
La technique nous libère-t-elle ? \\[0pt]
La technique nous libère-t-elle du travail ? \\[0pt]
La technique nous oppose-t-elle à la nature ? \\[0pt]
La technique nous permet-elle de comprendre la nature ? \\[0pt]
La technique pense-t-elle ? \\[0pt]
La technique permet-elle de réaliser tous les désirs ? \\[0pt]
La technique peut-elle améliorer l'homme ? \\[0pt]
La technique peut-elle être tenue pour la forme moderne de la culture ? \\[0pt]
La technique peut-elle respecter la nature ? \\[0pt]
La technique peut-elle se déduire de la science ? \\[0pt]
La technique peut-elle se passer de la science ? \\[0pt]
La technique pose-t-elle plus de problèmes qu'elle n'en résout ? \\[0pt]
La technique produit-elle autre chose que condition de la pensée ? \\[0pt]
La technique produit-elle son propre savoir ? \\[0pt]
La technique provoque-t-elle inévitablement des catastrophes ? \\[0pt]
La technique repose-t-elle sur le génie du technicien ? \\[0pt]
La technique sert-elle nos désirs ? \\[0pt]
La technique s'oppose-t-elle à la nature ? \\[0pt]
La technocratie \\[0pt]
La technologie modifie-t-elle les rapports sociaux ? \\[0pt]
La téléologie \\[0pt]
La télévision \\[0pt]
La témérité \\[0pt]
La tempérance \\[0pt]
La temporalité de l'œuvre d'art \\[0pt]
La tendance \\[0pt]
La tentation \\[0pt]
La tentation réductionniste \\[0pt]
La terre \\[0pt]
La terre est-elle un bien commun ? \\[0pt]
La terre et le ciel \\[0pt]
La Terre et le Ciel \\[0pt]
La terreur \\[0pt]
La terreur morale \\[0pt]
L'athée peut-il être vertueux ? \\[0pt]
L'athéisme \\[0pt]
L'athéisme condamne-t-il l'existence à l'absurdité ? \\[0pt]
L'athéisme est-il une croyance ? \\[0pt]
La théodicée \\[0pt]
La théogonie \\[0pt]
La théologie est-elle une science ? \\[0pt]
La théologie peut-elle être rationnelle ? \\[0pt]
La théologie rationnelle \\[0pt]
La théorie \\[0pt]
La théorie et la pratique \\[0pt]
La théorie et l'expérience \\[0pt]
La théorie nous éloigne-t-elle de la réalité ? \\[0pt]
La théorie peut-elle nous égarer ? \\[0pt]
La théorie scientifique \\[0pt]
La tolérance \\[0pt]
La tolérance a-t-elle des limites ? \\[0pt]
La tolérance envers les intolérants \\[0pt]
La tolérance est-elle un concept politique ? \\[0pt]
La tolérance est-elle une vertu ? \\[0pt]
La tolérance peut-elle constituer un problème pour la démocratie ? \\[0pt]
L'atome \\[0pt]
La totalitarisme \\[0pt]
La totalité \\[0pt]
La toute-puissance \\[0pt]
La toute-puissance de la pensée \\[0pt]
La trace \\[0pt]
La trace et l'indice \\[0pt]
La tradition \\[0pt]
La traductibilité des langues \\[0pt]
La traduction \\[0pt]
La tragédie \\[0pt]
La trahison \\[0pt]
La tranquillité \\[0pt]
La transcendance \\[0pt]
La transe \\[0pt]
La transgression \\[0pt]
La transgression des règles \\[0pt]
La transmission \\[0pt]
La transmission de pensée \\[0pt]
La transparence \\[0pt]
La transparence des consciences \\[0pt]
La transparence est-elle un idéal démocratique ? \\[0pt]
La tristesse \\[0pt]
La tristesse du fini \\[0pt]
La tristesse est-elle mauvaise ? \\[0pt]
L'attachement \\[0pt]
L'attente \\[0pt]
L'attention \\[0pt]
L'attention caractérise-t-elle la conscience ? \\[0pt]
L'attitude religieuse \\[0pt]
L'attraction \\[0pt]
L'attrait du beau \\[0pt]
La tyrannie \\[0pt]
La tyrannie de la majorité \\[0pt]
La tyrannie des désirs \\[0pt]
La tyrannie du bonheur \\[0pt]
L'audace \\[0pt]
L'audace politique \\[0pt]
L'au-delà \\[0pt]
L'au-delà de l'être \\[0pt]
L'autarcie \\[0pt]
L'auteur et le créateur \\[0pt]
L'authenticité \\[0pt]
L'authenticité artistique \\[0pt]
L'authenticité de l'œuvre d'art \\[0pt]
L'autobiographie \\[0pt]
L'autocritique \\[0pt]
L'automate \\[0pt]
L'automatisation \\[0pt]
L'automatisation du raisonnement \\[0pt]
L'autonomie \\[0pt]
L'autonomie de l'art \\[0pt]
L'autonomie de l'œuvre d'art \\[0pt]
L'autonomie du théorique \\[0pt]
L'autonomie du vivant \\[0pt]
L'autoportrait \\[0pt]
L'autorité \\[0pt]
L'autorité de la chose jugée \\[0pt]
L'autorité de la loi \\[0pt]
L'autorité de la parole \\[0pt]
L'autorité de la science \\[0pt]
L'autorité de l'écrit \\[0pt]
L'autorité de l'État \\[0pt]
L'autorité des lois \\[0pt]
L'autorité des savants \\[0pt]
L'autorité du droit \\[0pt]
L'autorité morale \\[0pt]
L'autorité politique \\[0pt]
L'autre est-il le fondement de la conscience morale ? \\[0pt]
L'autre et les autres \\[0pt]
L'autre monde \\[0pt]
La valeur \\[0pt]
La valeur artistique \\[0pt]
La valeur d'échange \\[0pt]
La valeur de la culture \\[0pt]
La valeur de la loi \\[0pt]
La valeur de la nature \\[0pt]
La valeur de la pitié \\[0pt]
La valeur de la raison \\[0pt]
La valeur de l'argent \\[0pt]
La valeur de l'art \\[0pt]
La valeur de la science \\[0pt]
La valeur de la vérité \\[0pt]
La valeur de la vie \\[0pt]
La valeur de l'échange \\[0pt]
La valeur de l'exemple \\[0pt]
La valeur de l'hypothèse \\[0pt]
La valeur de l'opinion \\[0pt]
La valeur des arts \\[0pt]
La valeur des choses \\[0pt]
La valeur des conventions \\[0pt]
La valeur des exceptions \\[0pt]
La valeur des hypothèses \\[0pt]
La valeur des images \\[0pt]
La valeur des normes \\[0pt]
La valeur du beau \\[0pt]
La valeur du consensus \\[0pt]
La valeur du consentement \\[0pt]
La valeur du don \\[0pt]
La valeur d'une action se mesure-t-elle à sa réussite ? \\[0pt]
La valeur d'une action se mesure-t-elle à ses conséquences ? \\[0pt]
La valeur d'une œuvre \\[0pt]
La valeur d'une théorie scientifique se mesure-t-elle à son efficacité ? \\[0pt]
La valeur du plaisir \\[0pt]
La valeur d'usage \\[0pt]
La valeur du témoignage \\[0pt]
La valeur du temps \\[0pt]
La valeur du travail \\[0pt]
La valeur et le prix \\[0pt]
La valeur morale \\[0pt]
La valeur morale de l'amour \\[0pt]
La valeur morale d'une action se juge-t-elle à ses conséquences ? \\[0pt]
« La valeur travail » \\[0pt]
La validité \\[0pt]
La vanité \\[0pt]
La vanité est-elle toujours sans objet ? \\[0pt]
L'avant-garde \\[0pt]
L'avarice \\[0pt]
La variété \\[0pt]
La veille et le sommeil \\[0pt]
La vénalité \\[0pt]
La vénération \\[0pt]
La vengeance \\[0pt]
L'avenir \\[0pt]
L'avenir a-t-il une réalité ? \\[0pt]
L'avenir de l'humanité \\[0pt]
L'avenir est-il imaginable ? \\[0pt]
L'avenir est-il incertain ? \\[0pt]
L'avenir est-il l'affaire de la pensée ? \\[0pt]
L'avenir est-il prévisible ? \\[0pt]
L'avenir est-il sans image ? \\[0pt]
L'avenir est-il une page blanche ? \\[0pt]
L'avenir existe-t-il ? \\[0pt]
L'avenir peut-il être objet de connaissance ? \\[0pt]
L'aventure \\[0pt]
La véracité \\[0pt]
La vérification \\[0pt]
La vérification expérimentale \\[0pt]
La vérification fait-elle la vérité ? \\[0pt]
La vérité \\[0pt]
La vérité admet-elle des degrés ? \\[0pt]
La vérité a-t-elle une histoire ? \\[0pt]
La vérité de la fiction \\[0pt]
La vérité de la perception \\[0pt]
La vérité de l'apparence \\[0pt]
La vérité de la religion \\[0pt]
La vérité demande-t-elle du courage ? \\[0pt]
La vérité des arts \\[0pt]
La vérité des images \\[0pt]
La vérité des sciences \\[0pt]
La vérité des sentiments \\[0pt]
La vérité doit-elle toujours être démontrée ? \\[0pt]
La vérité donne-t-elle le droit d'être injuste ? \\[0pt]
La vérité du déterminisme \\[0pt]
La vérité d'une théorie dépend-elle de sa correspondance avec les faits ? \\[0pt]
La vérité du roman \\[0pt]
La vérité échappe-t-elle au temps ? \\[0pt]
La vérité en art \\[0pt]
La vérité en politique \\[0pt]
La vérité est-elle affaire de cohérence ? \\[0pt]
La vérité est-elle affaire de croyance ou de savoir ? \\[0pt]
La vérité est-elle affaire d'unanimité ? \\[0pt]
La vérité est-elle contraignante ? \\[0pt]
La vérité est-elle éternelle ? \\[0pt]
La vérité est-elle fille de son temps ? \\[0pt]
La vérité est-elle hors de notre portée ? \\[0pt]
La vérité est-elle intemporelle ? \\[0pt]
La vérité est-elle la valeur suprême ? \\[0pt]
La vérité est-elle libératrice ? \\[0pt]
La vérité est-elle morale ? \\[0pt]
La vérité est-elle objective ? \\[0pt]
La vérité est-elle préférable à l'illusion ? \\[0pt]
La vérité est-elle relative ? \\[0pt]
La vérité est-elle triste ? \\[0pt]
La vérité est-elle une ? \\[0pt]
La vérité est-elle une construction ? \\[0pt]
La vérité est-elle une idole ? \\[0pt]
La vérité est-elle une valeur ? \\[0pt]
La vérité et les vérités \\[0pt]
La vérité historique \\[0pt]
La vérité judiciaire \\[0pt]
La vérité mathématique \\[0pt]
La vérité n'est-elle qu'une erreur rectifiée ? \\[0pt]
La vérité n'est-elle qu'un idéal inaccessible ? \\[0pt]
La vérité nous appartient-elle ? \\[0pt]
La vérité nous contraint-elle ? \\[0pt]
La vérité nous rend-elle libres ? \\[0pt]
La vérité peut-elle changer avec le temps ? \\[0pt]
La vérité peut-elle être équivoque ? \\[0pt]
La vérité peut-elle être indicible ? \\[0pt]
La vérité peut-elle être relative ? \\[0pt]
La vérité peut-elle être tolérante ? \\[0pt]
La vérité peut-elle faire peur ? \\[0pt]
La vérité peut-elle laisser indifférent ? \\[0pt]
La vérité peut-elle se définir par le consensus ? \\[0pt]
La vérité peut-elle se discuter ? \\[0pt]
La vérité philosophique \\[0pt]
La vérité rend-elle heureux ? \\[0pt]
La vérité rend-elle libre ? \\[0pt]
La vérité requiert-elle du courage ? \\[0pt]
La vérité résiste-elle à sa révélation ? \\[0pt]
La vérité scientifique est-elle relative ? \\[0pt]
La vérité se communique-t-elle ? \\[0pt]
La vérité se discute-t-elle ? \\[0pt]
La vertu \\[0pt]
La vertu de l'abstraction \\[0pt]
La vertu de l'exemple \\[0pt]
La vertu de l'homme politique \\[0pt]
La vertu de l'oubli \\[0pt]
La vertu du citoyen \\[0pt]
La vertu du plaisir \\[0pt]
La vertu est-elle une forme de santé ? \\[0pt]
La vertu est-elle utile ? \\[0pt]
La vertu, les vertus \\[0pt]
La vertu peut-elle être excessive ? \\[0pt]
La vertu peut-elle être purement morale ? \\[0pt]
La vertu peut-elle s'enseigner ? \\[0pt]
La vertu politique \\[0pt]
La vertu requiert-elle l'idée de bien ? \\[0pt]
La vertu s'enseigne-t-elle ? \\[0pt]
L'aveu \\[0pt]
L'aveu diminue-t-il la faute ? \\[0pt]
L'aveuglement \\[0pt]
La vie \\[0pt]
La vie active \\[0pt]
La vie après la mort \\[0pt]
La vie a-t-elle un sens ? \\[0pt]
La vie brève \\[0pt]
La vie collective est-elle nécessairement frustrante ? \\[0pt]
La vie de la cité \\[0pt]
La vie de la langue \\[0pt]
La vie de l'esprit \\[0pt]
La vie de plaisirs \\[0pt]
« La vie des formes » \\[0pt]
La vie des machines \\[0pt]
La vie des rêves \\[0pt]
La vie des signes \\[0pt]
La vie du droit \\[0pt]
La vie en société est-elle naturelle à l'homme ? \\[0pt]
La vie en société impose-t-elle de n'être pas soi-même ? \\[0pt]
La vie en société menace-t-elle la liberté ? \\[0pt]
La vie est-elle la valeur suprême ? \\[0pt]
La vie est-elle le bien le plus précieux ? \\[0pt]
La vie est-elle le bien suprême ? \\[0pt]
La vie est-elle l'objet des sciences de la vie ? \\[0pt]
La vie est-elle objet de science ? \\[0pt]
La vie est-elle sacrée ? \\[0pt]
La vie est-elle trop courte ? \\[0pt]
La vie est-elle une notion métaphysique ? \\[0pt]
La vie est-elle une valeur ? \\[0pt]
La vie est-elle un roman ? \\[0pt]
La vie est-elle un songe ? \\[0pt]
« La vie est une scène » \\[0pt]
« La vie est un songe » \\[0pt]
La vie éternelle \\[0pt]
La vie et le vivant \\[0pt]
La vie heureuse \\[0pt]
La vieillesse \\[0pt]
« La vieillesse est un naufrage » \\[0pt]
La vie intérieure \\[0pt]
La vie intérieure est-elle une illusion ? \\[0pt]
La vie moderne \\[0pt]
La vie morale \\[0pt]
La vie ordinaire \\[0pt]
La vie peut-elle être éternelle ? \\[0pt]
La vie peut-elle être objet de science ? \\[0pt]
La vie peut-elle être sans histoire ? \\[0pt]
La vie politique \\[0pt]
La vie politique est-elle aliénante ? \\[0pt]
La vie privée \\[0pt]
La vie psychique \\[0pt]
La vie quotidienne \\[0pt]
La vie sauvage \\[0pt]
La vie sexuelle est-elle volontaire ? \\[0pt]
La vie sociale \\[0pt]
La vie sociale est-elle toujours conflictuelle ? \\[0pt]
La vie sociale est-elle une comédie ? \\[0pt]
La vie sociale relève-t-elle des habitudes ? \\[0pt]
La vie suffit-elle à créer des valeurs ? \\[0pt]
La vigilance \\[0pt]
La ville \\[0pt]
La ville et la campagne \\[0pt]
La violence \\[0pt]
La violence a-t-elle des degrés ? \\[0pt]
La violence a-t-elle un sens dans l'histoire ? \\[0pt]
La violence de l'art \\[0pt]
La violence de l'État \\[0pt]
La violence de l'interprétation \\[0pt]
La violence d'État \\[0pt]
La violence du désir \\[0pt]
La violence engendre-t-elle toujours la violence ? \\[0pt]
La violence est-elle la condition du progrès ? \\[0pt]
La violence est-elle le fondement du droit ? \\[0pt]
La violence est-elle le prix du progrès ? \\[0pt]
La violence est-elle toujours destructrice ? \\[0pt]
La violence est-elle toujours injustifiable ? \\[0pt]
La violence peut-elle avoir raison ? \\[0pt]
La violence peut-elle être gratuite ? \\[0pt]
La violence peut-elle être morale ? \\[0pt]
La violence politique \\[0pt]
La violence révolutionnaire \\[0pt]
La violence sociale \\[0pt]
La violence verbale \\[0pt]
La virtualité \\[0pt]
La virtuosité \\[0pt]
La vision et le toucher \\[0pt]
La vision peut-elle être le modèle de toute connaissance ? \\[0pt]
La vision scientifique du monde \\[0pt]
La vitesse \\[0pt]
L'avocat du diable \\[0pt]
La vocation \\[0pt]
La vocation utopique de l'art \\[0pt]
La voix \\[0pt]
La voix de la conscience \\[0pt]
La voix de la raison \\[0pt]
La voix du peuple \\[0pt]
La volonté constitue-t-elle le principe de la politique ? \\[0pt]
La volonté de croire \\[0pt]
La volonté de savoir \\[0pt]
La volonté divine \\[0pt]
La volonté du peuple \\[0pt]
La volonté est-elle la somme des inclinations ? \\[0pt]
La volonté est-elle libre ? \\[0pt]
La volonté et le désir \\[0pt]
La volonté générale \\[0pt]
La volonté générale est-elle la volonté de tous ? \\[0pt]
La volonté peut-elle être collective ? \\[0pt]
La volonté peut-elle être générale ? \\[0pt]
La volonté peut-elle être indéterminée ? \\[0pt]
La volonté peut-elle être libre ? \\[0pt]
La volonté peut-elle nous manquer ? \\[0pt]
La volonté politique peut-elle être commune ? \\[0pt]
La volupté \\[0pt]
« La vraie morale se moque de la morale » \\[0pt]
La vraie morale se moque-t-elle de la morale ? \\[0pt]
La vraie vie \\[0pt]
La vraisemblance \\[0pt]
La vue et le toucher \\[0pt]
La vue et l'ouïe \\[0pt]
La vulgarisation \\[0pt]
La vulgarité \\[0pt]
La vulnérabilité \\[0pt]
L'axiome \\[0pt]
Le barbare \\[0pt]
Le baroque \\[0pt]
Le bavardage \\[0pt]
Le beau a-t-il une histoire ? \\[0pt]
Le beau est-il aimable ? \\[0pt]
Le beau est-il dicible ? \\[0pt]
Le beau est-il l'objet de l'esthétique ? \\[0pt]
Le beau est-il toujours moral ? \\[0pt]
Le beau est-il une valeur commune ? \\[0pt]
Le beau est-il universel ? \\[0pt]
Le beau et l'agréable \\[0pt]
Le beau et le bien \\[0pt]
Le beau et le bien sont-ils, au fond, identiques ? \\[0pt]
Le beau et le bon \\[0pt]
Le beau et le joli \\[0pt]
Le beau et le sublime \\[0pt]
Le beau et l'utile \\[0pt]
Le beau existe-t-il indépendamment du bien ? \\[0pt]
Le beau geste \\[0pt]
Le beau naturel \\[0pt]
Le beau n'est-il qu'une idée ? \\[0pt]
Le beau peut-il être bizarre ? \\[0pt]
Le beau peut-il être effrayant ? \\[0pt]
Le bénéfice du doute \\[0pt]
Le besoin \\[0pt]
Le besoin d'absolu \\[0pt]
Le besoin de beauté \\[0pt]
Le besoin de métaphysique est-il un besoin de connaissance ? \\[0pt]
Le besoin de philosophie \\[0pt]
Le besoin de reconnaissance \\[0pt]
Le besoin de rêve \\[0pt]
Le besoin de sens \\[0pt]
Le besoin de signes \\[0pt]
Le besoin de théorie \\[0pt]
Le besoin de vérité \\[0pt]
Le besoin et le désir \\[0pt]
Le besoin métaphysique \\[0pt]
Le bien commun \\[0pt]
Le bien commun est-il une illusion ? \\[0pt]
Le bien commun et l'intérêt de tous \\[0pt]
Le bien d'autrui \\[0pt]
Le bien détermine-t-il la volonté ? \\[0pt]
Le bien, est-ce l'utile ? \\[0pt]
Le bien est-il conformité à la nature ? \\[0pt]
Le bien est-il l'utile ? \\[0pt]
Le bien est-il relatif ? \\[0pt]
Le bien et le beau \\[0pt]
Le bien et le mal \\[0pt]
Le bien et les biens \\[0pt]
Le bien et l'utile \\[0pt]
Le bien-être \\[0pt]
Le bien n'est-il réalisable que comme moindre mal ? \\[0pt]
Le bien public \\[0pt]
Le Bien public \\[0pt]
Le bien suppose-t-il la transcendance ? \\[0pt]
Le biologisme \\[0pt]
Le biologisme en sciences sociales \\[0pt]
L'éblouissement \\[0pt]
Le bon Dieu \\[0pt]
Le bon et l'utile \\[0pt]
Le bon goût \\[0pt]
Le bon gouvernement \\[0pt]
Le bonheur \\[0pt]
Le bonheur a-t-il nécessairement un objet ? \\[0pt]
Le bonheur collectif \\[0pt]
Le bonheur comporte-t-il des degrés ? \\[0pt]
Le bonheur dans le mal \\[0pt]
Le bonheur de la passion est-il sans lendemain ? \\[0pt]
Le bonheur des autres \\[0pt]
Le bonheur des citoyens est-il un idéal politique ? \\[0pt]
Le bonheur des méchants \\[0pt]
Le bonheur des sens \\[0pt]
Le bonheur des uns, le malheur des autres \\[0pt]
Le bonheur du citoyen \\[0pt]
Le bonheur du juste \\[0pt]
Le bonheur du plus grand nombre \\[0pt]
Le bonheur est-il affaire de calcul ? \\[0pt]
Le bonheur est-il affaire de hasard ou de nécessité ? \\[0pt]
Le bonheur est-il affaire de vertu ? \\[0pt]
Le bonheur est-il affaire de volonté ? \\[0pt]
Le bonheur est-il affaire privée ? \\[0pt]
Le bonheur est-il au nombre de nos devoirs ? \\[0pt]
Le bonheur est-il dans l'inconscience ? \\[0pt]
Le bonheur est-il l'absence de maux ? \\[0pt]
Le bonheur est-il l'affaire du politique ? \\[0pt]
Le bonheur est-il la fin de la vie ? \\[0pt]
Le bonheur est-il le bien suprême ? \\[0pt]
Le bonheur est-il le but de la politique ? \\[0pt]
Le bonheur est-il le prix de la vertu ? \\[0pt]
Le bonheur est-il nécessairement lié au plaisir ? \\[0pt]
Le bonheur est-il un accident ? \\[0pt]
Le bonheur est-il un but politique ? \\[0pt]
Le bonheur est-il un droit ? \\[0pt]
Le bonheur est-il une affaire de circonstances ? \\[0pt]
Le bonheur est-il une affaire privée ? \\[0pt]
Le bonheur est-il une fin morale ? \\[0pt]
Le bonheur est-il une fin politique ? \\[0pt]
Le bonheur est-il une récompense ? \\[0pt]
Le bonheur est-il une valeur morale ? \\[0pt]
Le bonheur est-il un idéal ? \\[0pt]
Le bonheur est-il un principe politique ? \\[0pt]
Le bonheur et la raison \\[0pt]
Le bonheur et la technique \\[0pt]
Le bonheur et la vertu \\[0pt]
Le bonheur et le plaisir \\[0pt]
Le bonheur n'est-il qu'une idée ? \\[0pt]
Le bonheur n'est-il qu'un idéal ? \\[0pt]
Le bonheur peut-il être collectif ? \\[0pt]
Le bonheur peut-il être le but de la politique ? \\[0pt]
Le bonheur peut-il être un droit ? \\[0pt]
Le bonheur peut-il être un objectif politique ? \\[0pt]
Le bonheur s'apprend-il ? \\[0pt]
Le bonheur se calcule-t-il ? \\[0pt]
Le bonheur se fonde-t-il sur un savoir ? \\[0pt]
Le bonheur se mérite-t-il ? \\[0pt]
Le bon plaisir \\[0pt]
Le bon régime \\[0pt]
Le bon sens \\[0pt]
Le bon usage des passions \\[0pt]
Le bouc émissaire \\[0pt]
Le bourgeois et le citoyen \\[0pt]
Le bricolage \\[0pt]
Le bruit \\[0pt]
Le bruit et la musique \\[0pt]
Le but de l'association politique \\[0pt]
Le cadavre \\[0pt]
Le cadre \\[0pt]
Le calcul \\[0pt]
Le calcul des plaisirs \\[0pt]
Le calendrier \\[0pt]
Le cannibalisme \\[0pt]
Le canon \\[0pt]
Le capital et le travail \\[0pt]
Le capitalisme \\[0pt]
Le capital social \\[0pt]
Le caractère \\[0pt]
Le caractère sacré de la vie \\[0pt]
L'écart \\[0pt]
Le cas de conscience \\[0pt]
Le cas particulier \\[0pt]
Le catéchisme moral \\[0pt]
Le certain et le probable \\[0pt]
Le cerveau et la pensée \\[0pt]
Le cerveau pense-t-il ? \\[0pt]
L'échange \\[0pt]
L'échange constitue-t-il un lien social ? \\[0pt]
L'échange des marchandises et les rapports humains \\[0pt]
L'échange des signes \\[0pt]
L'échange économique fonde-t-il la société humaine \\[0pt]
L'échange est-il un facteur de paix ? \\[0pt]
L'échange et l'usage \\[0pt]
L'échange inégal \\[0pt]
Le changement \\[0pt]
L'échange n'a-t-il de fondement qu'économique ? \\[0pt]
L'échange ne porte-t-il que sur les choses ? \\[0pt]
L'échange peut-il être désintéressé ? \\[0pt]
L'échange symbolique \\[0pt]
Le chant \\[0pt]
Le chant et le cri \\[0pt]
Le chaos \\[0pt]
Le chaos du monde \\[0pt]
Le charisme en politique \\[0pt]
Le charisme politique \\[0pt]
Le charme \\[0pt]
Le charme et la grâce \\[0pt]
Le châtiment \\[0pt]
Le châtiment peut-il ne rien devoir au désir de vengeance ? \\[0pt]
Le châtiment rend-il meilleur ? \\[0pt]
L'échec \\[0pt]
Le chef \\[0pt]
Le chef-d'œuvre \\[0pt]
Le chemin \\[0pt]
Le choc des idées \\[0pt]
Le choc esthétique \\[0pt]
Le choix \\[0pt]
Le choix de philosopher \\[0pt]
Le choix des moyens \\[0pt]
Le choix d'un destin \\[0pt]
Le choix d'un métier \\[0pt]
Le choix en économie \\[0pt]
Le choix et la liberté \\[0pt]
Le choix peut-il être éclairé ? \\[0pt]
Le ciel et la terre \\[0pt]
Le cinéma, art de la représentation ? \\[0pt]
Le cinéma est-il un art ? \\[0pt]
Le cinéma est-il un art comme les autres ? \\[0pt]
Le cinéma est-il un art narratif ? \\[0pt]
Le cinéma est-il un art ou une industrie ? \\[0pt]
Le cinéma est-il un art populaire ? \\[0pt]
Le citoyen \\[0pt]
Le citoyen a-t-il perdu toute naturalité ? \\[0pt]
Le citoyen peut-il être à la fois libre et soumis à l'État ? \\[0pt]
Le civisme \\[0pt]
Le clair et le distinct \\[0pt]
Le clair et l'obscur \\[0pt]
Le clair-obscur \\[0pt]
Le classicisme \\[0pt]
L'éclat \\[0pt]
Le cliché \\[0pt]
Le code \\[0pt]
Le cœur \\[0pt]
Le cœur et la raison \\[0pt]
L'école de la vie \\[0pt]
L'école des vertus \\[0pt]
L'écologie est-elle aussi une science humaine ? \\[0pt]
L'écologie est-elle une science de la nature ou une science sociale ? \\[0pt]
L'écologie est-elle un problème politique ? \\[0pt]
L'écologie politique \\[0pt]
L'écologie, une science humaine ? \\[0pt]
Le combat \\[0pt]
Le combat contre l'injustice a-t-il une source morale ? \\[0pt]
Le comédien \\[0pt]
Le comique \\[0pt]
Le comique et le tragique \\[0pt]
Le commencement \\[0pt]
Le commencement du monde \\[0pt]
Le comment et le pourquoi \\[0pt]
Le commerce \\[0pt]
Le commerce adoucit-il les mœurs ? \\[0pt]
Le commerce des corps \\[0pt]
Le commerce des hommes \\[0pt]
Le commerce des idées \\[0pt]
Le commerce équitable \\[0pt]
Le commerce est-il pacificateur ? \\[0pt]
Le commerce est-il un facteur de paix ? \\[0pt]
Le commerce peut-il être équitable ? \\[0pt]
Le commerce unit-il les hommes ? \\[0pt]
Le commun \\[0pt]
Le commun et le propre \\[0pt]
Le comparatisme dans les sciences humaines \\[0pt]
Le complexe \\[0pt]
Le comportement \\[0pt]
Le compromis \\[0pt]
Le concept \\[0pt]
Le concept de civilisation \\[0pt]
Le concept de comportement rationnel dans les sciences humaines \\[0pt]
Le concept de matière \\[0pt]
Le concept de nature est-il un concept scientifique ? \\[0pt]
Le concept de pulsion \\[0pt]
Le concept de race est-il anthropologique ? \\[0pt]
Le concept de société sans écriture est-il pertinent ? \\[0pt]
Le concept de structure \\[0pt]
Le concept de structure sociale \\[0pt]
Le concept d'évolution \\[0pt]
Le concept d'histoire universelle \\[0pt]
Le concept d'inconscient est-il nécessaire en sciences humaines ? \\[0pt]
Le concept et la vie \\[0pt]
Le concept et l'exemple \\[0pt]
Le concept et l'image \\[0pt]
Le concret \\[0pt]
Le concret et l'abstrait \\[0pt]
Le conditionnel \\[0pt]
Le conflit \\[0pt]
Le conflit de devoirs \\[0pt]
Le conflit des devoirs \\[0pt]
Le conflit des interprétations \\[0pt]
Le conflit entre la science et la religion est-il inévitable ? \\[0pt]
Le conflit entre l'individu et la société est-il irréductible ? \\[0pt]
Le conflit esthétique \\[0pt]
Le conflit est-il au fondement de la vie sociale ? \\[0pt]
Le conflit est-il constitutif de la politique ? \\[0pt]
Le conflit est-il la raison d'être de la politique ? \\[0pt]
Le conflit est-il une maladie sociale ? \\[0pt]
Le conflit social \\[0pt]
Le conformisme \\[0pt]
Le conformisme moral \\[0pt]
Le conformisme social \\[0pt]
Le confort intellectuel \\[0pt]
L'économie \\[0pt]
L'économie a-t-elle des lois ? \\[0pt]
L'économie a-t-elle ses propres lois ? \\[0pt]
L'économie de la langue \\[0pt]
L'économie des moyens \\[0pt]
L'économie du don \\[0pt]
L'économie est-elle politique ? \\[0pt]
L'économie est-elle une science ? \\[0pt]
L'économie est-elle une science humaine ? \\[0pt]
L'économie et la politique \\[0pt]
L'économie et la valeur humaine \\[0pt]
L'économie et le politique obéissent-ils à une même rationalité ? \\[0pt]
L'économie fait-elle partie des sciences humaines ? \\[0pt]
L'économie politique \\[0pt]
L'économie primitive \\[0pt]
L'économie psychique \\[0pt]
L'économique et le politique \\[0pt]
Le conscient et l'inconscient \\[0pt]
Le conseil \\[0pt]
Le conseiller du prince \\[0pt]
Le consensus \\[0pt]
Le consensus peut-il être critère de vérité ? \\[0pt]
Le consensus peut-il faire le vrai ? \\[0pt]
Le consensus social \\[0pt]
Le consentement \\[0pt]
Le consentement des gouvernés \\[0pt]
Le consentement du peuple \\[0pt]
Le conservatisme \\[0pt]
Le contemporain \\[0pt]
Le contentement \\[0pt]
Le contenu empirique \\[0pt]
Le contingent \\[0pt]
Le continu \\[0pt]
Le contradictoire peut-il exister ? \\[0pt]
Le contrat \\[0pt]
Le contrat de travail \\[0pt]
Le contrat est-il au fondement de la politique ? \\[0pt]
Le contrat peut-il remplacer la loi ? \\[0pt]
Le contrat social \\[0pt]
Le contrôle social \\[0pt]
Le convenable \\[0pt]
Le corps dansant \\[0pt]
Le corps de l'autre \\[0pt]
Le corps dit-il quelque chose ? \\[0pt]
Le corps du prince \\[0pt]
Le corps du travailleur \\[0pt]
Le corps est-il le reflet de l'âme ? \\[0pt]
Le corps est-il négociable ? \\[0pt]
Le corps est-il porteur de valeurs ? \\[0pt]
Le corps est-il respectable ? \\[0pt]
Le corps est-il une entrave pour l'esprit ? \\[0pt]
Le corps est-il un enjeu des sciences humaines ? \\[0pt]
Le corps est-il un tout ? \\[0pt]
Le corps et la danse \\[0pt]
Le corps et la machine \\[0pt]
Le corps et l'âme \\[0pt]
Le corps et l'esprit \\[0pt]
Le corps et le temps \\[0pt]
Le corps et l'instrument \\[0pt]
Le corps humain \\[0pt]
Le corps humain est-il naturel ? \\[0pt]
Le corps impose-t-il des perspectives ? \\[0pt]
Le corps n'est-il que matière ? \\[0pt]
Le corps n'est-il qu'un mécanisme ? \\[0pt]
Le corps obéit-il à l'esprit ? \\[0pt]
Le corps pense-t-il ? \\[0pt]
Le corps peut-il être objet d'art ? \\[0pt]
Le corps politique \\[0pt]
Le corps propre \\[0pt]
Le cosmopolitisme \\[0pt]
Le cosmopolitisme peut-il devenir réalité ? \\[0pt]
Le cosmopolitisme peut-il être réaliste ? \\[0pt]
Le coupable \\[0pt]
Le coup d'État \\[0pt]
Le couple \\[0pt]
Le courage \\[0pt]
Le courage de la vérité \\[0pt]
Le courage de penser \\[0pt]
Le courage est-il une vertu ? \\[0pt]
Le courage politique \\[0pt]
Le cours de l'histoire \\[0pt]
Le cours des choses \\[0pt]
Le cours du temps \\[0pt]
Le créé et l'incréé \\[0pt]
Le cri \\[0pt]
Le crime \\[0pt]
Le crime contre l'humanité \\[0pt]
Le crime inexpiable \\[0pt]
Le crime suppose-t-il la loi ? \\[0pt]
Le critère \\[0pt]
L'écrit et l'oral \\[0pt]
Le critique d'art \\[0pt]
L'écriture \\[0pt]
L'écriture de l'histoire \\[0pt]
L'écriture des lois \\[0pt]
L'écriture est-elle une technique parmi d'autres ? \\[0pt]
L'écriture et la parole \\[0pt]
L'écriture et la pensée \\[0pt]
L'écriture ne sert-elle qu'à consigner la pensée ? \\[0pt]
L'écriture peut-elle porter secours à la pensée ? \\[0pt]
Le cru et le cuit \\[0pt]
Le culte des ancêtres \\[0pt]
Le culte des images \\[0pt]
Le cynique \\[0pt]
Le cynisme \\[0pt]
Le dandysme \\[0pt]
Le danger \\[0pt]
Le débat \\[0pt]
Le débat politique \\[0pt]
Le décentrement ethnologique \\[0pt]
Le déchet \\[0pt]
Le déclassement \\[0pt]
Le dedans et le dehors \\[0pt]
Le défaut \\[0pt]
Le dégoût \\[0pt]
Le déguisement \\[0pt]
Le délire \\[0pt]
Le démoniaque \\[0pt]
Le déni et la loi \\[0pt]
Le dépaysement \\[0pt]
Le dérèglement \\[0pt]
Le dernier mot \\[0pt]
Le désaccord \\[0pt]
Le désenchantement \\[0pt]
Le désespoir \\[0pt]
Le désespoir est-il une faute morale ? \\[0pt]
Le désespoir peut-il passer pour une sagesse ? \\[0pt]
Le déshonneur \\[0pt]
Le design \\[0pt]
Le désintéressement \\[0pt]
Le désintéressement esthétique \\[0pt]
Le désir \\[0pt]
Le désir a-t-il un objet ? \\[0pt]
Le désir d'absolu \\[0pt]
Le désir de connaissance \\[0pt]
Le désir de connaître \\[0pt]
Le désir de domination \\[0pt]
Le désir de dominer \\[0pt]
Le désir d'égalité \\[0pt]
Le désir de gloire \\[0pt]
Le désir de l'autre \\[0pt]
Le désir de pouvoir \\[0pt]
Le désir de reconnaissance \\[0pt]
Le désir de savoir \\[0pt]
Le désir de savoir est-il comblé par la science ? \\[0pt]
Le désir de savoir est-il naturel ? \\[0pt]
Le désir d'éternité \\[0pt]
Le désir d'être autre \\[0pt]
Le désir de vérité \\[0pt]
Le désir de vérité peut-il être interprété comme un désir de pouvoir ? \\[0pt]
Le désir de vivre \\[0pt]
Le désir d'immortalité \\[0pt]
Le désir d'infini \\[0pt]
Le désir d'originalité \\[0pt]
Le désir du bonheur est-il universel ? \\[0pt]
Le désir est-il aveugle ? \\[0pt]
Le désir est-il ce qui nous fait vivre ? \\[0pt]
Le désir est-il désir de l'autre ? \\[0pt]
Le désir est-il le signe d'un manque ? \\[0pt]
Le désir est-il l'essence de l'homme ? \\[0pt]
Le désir est-il nécessairement l'expression d'un manque ? \\[0pt]
Le désir est-il par nature illimité ? \\[0pt]
Le désir est-il sans limite ? \\[0pt]
Le désir et la culpabilité \\[0pt]
Le désir et la liberté \\[0pt]
Le désir et la loi \\[0pt]
Le désir et le besoin \\[0pt]
Le désir et le mal \\[0pt]
Le désir et le manque \\[0pt]
Le désir et le réel \\[0pt]
Le désir et le rêve \\[0pt]
Le désir et le temps \\[0pt]
Le désir et le travail \\[0pt]
Le désir et l'instinct \\[0pt]
Le désir et l'interdit \\[0pt]
Le désir métaphysique \\[0pt]
Le désir n'est-il pas qu'inquiétude ? \\[0pt]
Le désir n'est-il que l'épreuve d'un manque ? \\[0pt]
Le désir n'est-il que manque ? \\[0pt]
Le désir n'est-il qu'inquiétude ? \\[0pt]
Le désir peut-il atteindre son objet ? \\[0pt]
Le désir peut-il être désintéressé ? \\[0pt]
Le désir peut-il ne pas avoir d'objet ? \\[0pt]
Le désir peut-il nous rendre libre ? \\[0pt]
Le désir peut-il se satisfaire de la réalité ? \\[0pt]
Le désœuvrement \\[0pt]
Le désordre \\[0pt]
Le désordre des choses \\[0pt]
Le désordre est-il étonnant ? \\[0pt]
Le désordre est-il un mal ? \\[0pt]
Le despote peut-il être éclairé ? \\[0pt]
Le despotisme \\[0pt]
Le despotisme peut-il être éclairé ? \\[0pt]
Le dessin \\[0pt]
Le dessin et la couleur \\[0pt]
Le destin \\[0pt]
Le désuet \\[0pt]
Le détachement \\[0pt]
Le détail \\[0pt]
Le déterminisme \\[0pt]
Le déterminisme psychique \\[0pt]
Le déterminisme social \\[0pt]
Le deuil \\[0pt]
Le développement de la technique est-il toujours facteur de progrès ? \\[0pt]
Le développement des techniques fait-il reculer la croyance ? \\[0pt]
Le développement moral \\[0pt]
Le devenir \\[0pt]
Le devenir est-il pensable ? \\[0pt]
Le devoir \\[0pt]
Le devoir d'aimer \\[0pt]
Le devoir d'assistance \\[0pt]
Le devoir de loyauté \\[0pt]
Le devoir de mémoire \\[0pt]
Le devoir de vérité \\[0pt]
Le devoir d'obéissance \\[0pt]
Le devoir est-il l'expression de la contrainte sociale ? \\[0pt]
Le devoir et la dette \\[0pt]
Le devoir et le bonheur \\[0pt]
Le devoir-être \\[0pt]
Le devoir nous apparaît-il d'autant mieux qu'on ne l'a pas accompli ? \\[0pt]
Le devoir rend-il libre ? \\[0pt]
Le devoir s'apprend-il ? \\[0pt]
Le devoir se présente-t-il avec la force de l'évidence ? \\[0pt]
Le devoir supprime-t-il la liberté ? \\[0pt]
Le dévouement \\[0pt]
Le diable \\[0pt]
Le dialogue \\[0pt]
Le dialogue conduit-il à la vérité ? \\[0pt]
Le dialogue des philosophes \\[0pt]
Le dialogue entre les cultures \\[0pt]
Le dialogue entre nations \\[0pt]
Le dialogue suffit-il à rompre la solitude ? \\[0pt]
Le dictionnaire \\[0pt]
Le dieu artiste \\[0pt]
Le dieu des philosophes \\[0pt]
Le Dieu des philosophes \\[0pt]
L'édification morale \\[0pt]
Le dilemme \\[0pt]
Le dire et le faire \\[0pt]
Le discernement \\[0pt]
Le discontinu \\[0pt]
Le discours politique \\[0pt]
Le divers \\[0pt]
Le divertissement \\[0pt]
Le divertissement est-il vain ? \\[0pt]
Le divin \\[0pt]
Le dogmatisme \\[0pt]
Le domestique \\[0pt]
Le don \\[0pt]
Le don de soi \\[0pt]
Le don est-il toujours généreux ? \\[0pt]
Le don est-il une modalité de l'échange ? \\[0pt]
Le don et la dette \\[0pt]
Le don et l'échange \\[0pt]
Le don exclut-il le calcul ? \\[0pt]
Le donné \\[0pt]
Le donné et le construit \\[0pt]
Le double \\[0pt]
Le doute \\[0pt]
Le doute dans les sciences \\[0pt]
Le doute est-il le principe de la méthode scientifique ? \\[0pt]
Le doute est-il une faiblesse de la pensée ? \\[0pt]
Le doute métaphysique \\[0pt]
Le doute peut-il être méthodique ? \\[0pt]
Le drame \\[0pt]
Le droit \\[0pt]
Le droit à la citoyenneté \\[0pt]
Le droit à la différence met-il en péril l'égalité des droits ? \\[0pt]
Le droit à la paresse \\[0pt]
Le droit à la révolte \\[0pt]
Le droit à la vérité \\[0pt]
Le droit à l'erreur \\[0pt]
Le droit au bonheur \\[0pt]
Le droit au Bonheur \\[0pt]
Le droit au respect de la vie privée \\[0pt]
Le droit au travail \\[0pt]
Le droit d'asile \\[0pt]
Le droit d'auteur \\[0pt]
Le droit de disposer de son corps \\[0pt]
Le droit de guerre \\[0pt]
Le droit de la guerre \\[0pt]
Le droit de la majorité \\[0pt]
Le droit de mentir \\[0pt]
Le droit de propriété \\[0pt]
Le droit de punir \\[0pt]
Le droit de résistance \\[0pt]
Le droit de résistance à l'oppression \\[0pt]
Le droit de révolte \\[0pt]
Le droit des animaux \\[0pt]
Le droit des gens \\[0pt]
Le droit des minorités \\[0pt]
Le droit des peuples \\[0pt]
Le droit des peuples à disposer d'eux-mêmes \\[0pt]
Le droit de veto \\[0pt]
Le droit de vie et de mort \\[0pt]
Le droit de vivre \\[0pt]
Le droit d'ingérence \\[0pt]
Le droit d'intervention \\[0pt]
Le droit divin \\[0pt]
Le droit doit-il être indépendant de la morale ? \\[0pt]
Le droit doit-il être le seul régulateur de la vie sociale ? \\[0pt]
Le droit doit-il être moral ? \\[0pt]
Le droit du plus faible \\[0pt]
Le droit du plus fort \\[0pt]
Le droit du premier occupant \\[0pt]
Le droit est-il facteur de paix ? \\[0pt]
Le droit est-il le fondement de l'État ? \\[0pt]
Le droit est-il raison du plus fort ? \\[0pt]
Le droit est-il une science ? \\[0pt]
Le droit est-il une science humaine ? \\[0pt]
Le droit et la convention \\[0pt]
Le droit et la force \\[0pt]
Le droit et la liberté \\[0pt]
Le droit et la loi \\[0pt]
Le droit et la morale \\[0pt]
Le droit et la violence \\[0pt]
Le droit et le devoir \\[0pt]
Le Droit et l'État \\[0pt]
Le droit exclut-il la force ? \\[0pt]
Le droit humanitaire \\[0pt]
Le droit international \\[0pt]
Le droit naturel \\[0pt]
Le droit ne peut-il se fonder sur des faits ? \\[0pt]
Le droit n'est-il qu'une justice par défaut ? \\[0pt]
Le droit n'est-il qu'un ensemble de conventions ? \\[0pt]
Le droit peut-il échapper à l'histoire ? \\[0pt]
Le droit peut-il être flexible ? \\[0pt]
Le droit peut-il être naturel ? \\[0pt]
Le droit peut-il se fonder sur la force ? \\[0pt]
Le droit peut-il se passer de la morale ? \\[0pt]
Le droit positif \\[0pt]
Le droit sert-il à établir l'ordre ou la justice ? \\[0pt]
Le dualisme \\[0pt]
L'éducation \\[0pt]
L'éducation artistique \\[0pt]
L'éducation civique \\[0pt]
L'éducation des esprits \\[0pt]
L'éducation du goût \\[0pt]
L'éducation du goût est-elle la condition de l'expérience esthétique ? \\[0pt]
L'éducation est-elle affaire de politique ? \\[0pt]
L'éducation est-elle par nature conservatrice ? \\[0pt]
L'éducation esthétique \\[0pt]
L'éducation peut-elle être sentimentale ? \\[0pt]
L'éducation physique \\[0pt]
L'éducation politique \\[0pt]
L'éducation relève-t-elle du politique ? \\[0pt]
Le factice \\[0pt]
Le fait \\[0pt]
Le fait de vivre constitue-t-il un bien en soi ? \\[0pt]
Le fait de vivre est-il un bien en soi ? \\[0pt]
Le fait d'exister \\[0pt]
Le fait divers \\[0pt]
Le fait du prince \\[0pt]
Le fait et le droit \\[0pt]
Le fait et l'événement \\[0pt]
Le fait historique \\[0pt]
Le fait religieux \\[0pt]
Le fait scientifique \\[0pt]
Le fait social est-il une chose ? \\[0pt]
Le familier \\[0pt]
Le fanatisme \\[0pt]
Le fantasme \\[0pt]
Le fantastique \\[0pt]
Le fatalisme \\[0pt]
Le fatalisme l'incarnation \\[0pt]
Le faux \\[0pt]
Le faux en art \\[0pt]
Le faux et l'absurde \\[0pt]
Le faux et le fictif \\[0pt]
Le fédéralisme \\[0pt]
Le féminin \\[0pt]
Le féminin et le masculin \\[0pt]
Le féminisme \\[0pt]
Le fétichisme \\[0pt]
Le fétichisme de la marchandise \\[0pt]
L'effectivité \\[0pt]
L'effet de la musique \\[0pt]
L'effet et la cause \\[0pt]
L'efficacité \\[0pt]
L'efficacité des discours \\[0pt]
L'efficacité est-elle une vertu ? \\[0pt]
L'efficacité technique est-elle la norme de toute réussite ? \\[0pt]
L'efficacité thérapeutique de la psychanalyse \\[0pt]
L'efficience \\[0pt]
L'effort \\[0pt]
L'effort moral \\[0pt]
Le fil conducteur \\[0pt]
Le finalisme \\[0pt]
Le fini \\[0pt]
Le fini et l'infini \\[0pt]
Le fin mot de l'histoire \\[0pt]
Le flegme \\[0pt]
Le folklore \\[0pt]
Le fonctionnaire \\[0pt]
Le fonctionnel et le beau \\[0pt]
Le fond \\[0pt]
Le fondement \\[0pt]
Le fondement de l'autorité \\[0pt]
Le fondement de l'induction \\[0pt]
Le fond et la forme \\[0pt]
Le for intérieur \\[0pt]
Le formalisme \\[0pt]
Le formalisme dans les sciences humaines \\[0pt]
Le formalisme moral \\[0pt]
Le fou \\[0pt]
Le fragment \\[0pt]
Le frivole \\[0pt]
Le futur est-il contingent ? \\[0pt]
Le futur est-il nécessairement contingent ? \\[0pt]
Le futur existe-t-il ? \\[0pt]
Le futur nous appartient-il ? \\[0pt]
L'égalité \\[0pt]
L'égalité civile \\[0pt]
L'égalité des chances \\[0pt]
L'égalité des citoyens \\[0pt]
L'égalité des conditions \\[0pt]
L'égalité des hommes est-elle un fait ou une idée ? \\[0pt]
L'égalité des hommes et des femmes est-elle une question politique ? \\[0pt]
L'égalité des sexes \\[0pt]
L'égalité devant la loi \\[0pt]
L'égalité est-elle contre nature ? \\[0pt]
L'égalité est-elle l'ennemie de la liberté ? \\[0pt]
L'égalité est-elle souhaitable ? \\[0pt]
L'égalité est-elle toujours juste ? \\[0pt]
L'égalité est-elle une condition de la liberté ? \\[0pt]
Légalité et causalité \\[0pt]
Légalité et légitimité \\[0pt]
Légalité et moralité \\[0pt]
L'égalité peut-elle être une menace pour la liberté ? \\[0pt]
L'égarement \\[0pt]
Le général et le particulier \\[0pt]
Le génie \\[0pt]
Le génie du lieu \\[0pt]
Le génie du mal \\[0pt]
Le génie est-il la marque de l'excellence artistique ? \\[0pt]
Le génie et la règle \\[0pt]
Le génie et le savant \\[0pt]
Le génie n'a-t-il sa place que dans les arts ? \\[0pt]
Le genre \\[0pt]
Le genre et l'espèce \\[0pt]
Le genre humain \\[0pt]
Le genre humain : unité ou pluralité ? \\[0pt]
Le geste \\[0pt]
Le geste créateur \\[0pt]
Le geste et la parole \\[0pt]
Le geste technique exprime t-il une liberté sans fin ? \\[0pt]
Légitimité et légalité \\[0pt]
L'égoïsme \\[0pt]
Le goût \\[0pt]
Le goût : certitude ou conviction ? \\[0pt]
Le goût de la beauté \\[0pt]
Le goût de la fête \\[0pt]
Le goût de la liberté \\[0pt]
Le goût de la polémique \\[0pt]
Le goût de l'artiste \\[0pt]
Le goût des autres \\[0pt]
Le goût du beau \\[0pt]
Le goût du pouvoir \\[0pt]
Le goût du risque \\[0pt]
Le goût est-il affaire d'éducation ? \\[0pt]
Le goût est-il nécessairement subjectif ? \\[0pt]
Le goût est-il une faculté ? \\[0pt]
Le goût est-il une question de classe ? \\[0pt]
Le goût est-il une vertu sociale ? \\[0pt]
Le goût et la distinction \\[0pt]
Le goût s'éduque-t-il ? \\[0pt]
Le goût se forme-t-il ? \\[0pt]
Le gouvernement des experts \\[0pt]
Le gouvernement des hommes et l'administration des choses \\[0pt]
Le gouvernement des hommes libres \\[0pt]
Le gouvernement des juges \\[0pt]
Le gouvernement des meilleurs \\[0pt]
Le gouvernement de soi et des autres \\[0pt]
Le gouvernement par le peuple est-il nécessairement pour le peuple ? \\[0pt]
Le grand art est-il de plaire ? \\[0pt]
Le grandiose \\[0pt]
Le grand livre de la nature \\[0pt]
Le grotesque \\[0pt]
Le groupe \\[0pt]
Le handicap \\[0pt]
Le hasard \\[0pt]
Le hasard est-il injuste ? \\[0pt]
Le hasard et la nécessité \\[0pt]
Le hasard existe-t-il ? \\[0pt]
Le hasard fait-il bien les choses ? \\[0pt]
Le hasard gouverne-t-il le monde ? \\[0pt]
Le hasard n'est-il que la mesure de notre ignorance ? \\[0pt]
Le hasard n'est-il que le nom de notre ignorance ? \\[0pt]
Le hasard peut-il être un concept explicatif ?La morale doit-elle s'adapter à la réalité ? \\[0pt]
Le haut \\[0pt]
Le haut et le bas \\[0pt]
Le héros \\[0pt]
Le héros moral \\[0pt]
Le hors-la-loi \\[0pt]
Le je et le tu \\[0pt]
Le « je ne sais quoi » \\[0pt]
Le je-ne-sais-quoi \\[0pt]
Le jeu \\[0pt]
Le jeu de mots \\[0pt]
Le jeu des apparences \\[0pt]
Le jeu des possibles \\[0pt]
Le jeu du monde \\[0pt]
Le jeu et la règle \\[0pt]
Le jeu et le divertissement \\[0pt]
Le jeu et le hasard \\[0pt]
Le jeu et le sérieux \\[0pt]
Le jeu social \\[0pt]
Le joli, le beau \\[0pt]
Le juge \\[0pt]
Le jugement \\[0pt]
Le jugement artistique se fait-il sans concept ? \\[0pt]
Le jugement critique peut-il s'exercer sans culture ? \\[0pt]
Le jugement de goût \\[0pt]
Le jugement de goût est-il désintéressé ? \\[0pt]
Le jugement de goût est-il universel ? \\[0pt]
Le jugement de l'histoire \\[0pt]
Le jugement dernier \\[0pt]
Le jugement de valeur \\[0pt]
Le jugement de valeur est-il indifférent à la vérité ? \\[0pt]
Le jugement moral \\[0pt]
Le jugement politique \\[0pt]
Le jugement pratique \\[0pt]
Le juste et le bien \\[0pt]
Le juste et le légal \\[0pt]
Le juste et le vrai \\[0pt]
Le juste milieu \\[0pt]
Le laboratoire \\[0pt]
Le laid \\[0pt]
Le langage \\[0pt]
Le langage animal \\[0pt]
Le langage a-t-il une prétention à la vérité ? \\[0pt]
Le langage de la morale \\[0pt]
Le langage de la peinture \\[0pt]
Le langage de la pensée \\[0pt]
Le langage de l'art \\[0pt]
Le langage de la science \\[0pt]
Le langage des sciences \\[0pt]
Le langage du corps \\[0pt]
Le langage est-il assimilable à un outil ? \\[0pt]
Le langage est-il d'essence poétique ? \\[0pt]
Le langage est-il l'auxiliaire de la pensée ? \\[0pt]
Le langage est-il le lieu de la vérité ? \\[0pt]
Le langage est-il le miroir du monde ? \\[0pt]
Le langage est-il le moyen d'expression le plus efficace ? \\[0pt]
Le langage est-il le propre de l'homme ? \\[0pt]
Le langage est-il l'instrument de la pensée ? \\[0pt]
Le langage est-il logique ? \\[0pt]
Le langage est-il naturel ? \\[0pt]
Le langage est-il une prise de possession des choses ? \\[0pt]
Le langage est-il un instrument ? \\[0pt]
Le langage est-il un instrument de connaissance ? \\[0pt]
Le langage est-il un obstacle pour la pensée ? \\[0pt]
Le langage et la pensée \\[0pt]
Le langage et la raison \\[0pt]
Le langage et la vérité \\[0pt]
Le langage et le réel \\[0pt]
Le langage et les langues \\[0pt]
Le langage et l'image \\[0pt]
Le langage fait-il obstacle à la connaissance ? \\[0pt]
Le langage masque-t-il la pensée ? \\[0pt]
Le langage mathématique \\[0pt]
Le langage ne sert-il qu'à communiquer ? \\[0pt]
Le langage n'est-il qu'un instrument de communication ? \\[0pt]
Le langage n'est-il qu'un outil ? \\[0pt]
Le langage nous donne-t-il une connaissance des choses ? \\[0pt]
Le langage nous éloigne-t-il de la réalité ? \\[0pt]
Le langage permet-il seulement de communiquer ? \\[0pt]
Le langage peut-il être un obstacle à la recherche de la vérité ? \\[0pt]
Le langage poétique \\[0pt]
Le langage rapproche-t-il ou sépare-t-il les hommes ? \\[0pt]
Le langage rend-il l'homme plus puissant ? \\[0pt]
Le langage reste-t-il toujours à la surface des choses ? \\[0pt]
Le langage scientifique \\[0pt]
Le langage traduit-il la pensée ? \\[0pt]
Le langage trahit-il la pensée ? \\[0pt]
Le langage usuel est-il un obstacle à la connaissance ? \\[0pt]
Le langage voile-t-il les choses ? \\[0pt]
Le latent et le manifeste \\[0pt]
L'élection \\[0pt]
Le légal et le légitime \\[0pt]
L'élégance \\[0pt]
Le législateur \\[0pt]
Le législateur et le juge \\[0pt]
Le légitime et le légal \\[0pt]
L'élément \\[0pt]
L'élémentaire \\[0pt]
Le libertin \\[0pt]
Le libertin, le libertaire \\[0pt]
Le libre arbitre \\[0pt]
Le libre cours de l'imagination est-il libérateur ? \\[0pt]
Le libre échange \\[0pt]
Le libre jeu des formes \\[0pt]
Le lien causal \\[0pt]
Le lien politique \\[0pt]
Le lien social \\[0pt]
Le lien social peut-il être compassionnel ? \\[0pt]
Le lien social relève-t-il de la politique ? \\[0pt]
Le lieu \\[0pt]
Le lieu commun \\[0pt]
Le lieu de la pensée \\[0pt]
Le lieu de l'esprit \\[0pt]
Le lieu et l'espace \\[0pt]
Le littéral et le figuré \\[0pt]
Le livre \\[0pt]
Le livre de la nature \\[0pt]
L'éloge de la démesure \\[0pt]
Le logique \\[0pt]
Le logique est-elle un art de penser ? \\[0pt]
Le loisir \\[0pt]
Le loisir caractérise-t-il l'homme libre ? \\[0pt]
Le luxe \\[0pt]
Le lyrisme \\[0pt]
Le magistrat \\[0pt]
Le maintien de l'ordre \\[0pt]
Le maître \\[0pt]
Le maître et le disciple \\[0pt]
Le maître et l'esclave \\[0pt]
Le mal \\[0pt]
Le mal apparaît-il toujours ? \\[0pt]
Le mal a-t-il des raisons ? \\[0pt]
Le mal constitue-t-il une objection à l'existence de Dieu ? \\[0pt]
Le malentendu \\[0pt]
Le mal est-il une erreur ? \\[0pt]
Le mal est-il une objection à l'existence de Dieu ? \\[0pt]
Le mal est-il une réalité objective ? \\[0pt]
Le mal être \\[0pt]
Le mal existe-t-il ? \\[0pt]
Le malheur \\[0pt]
Le malheur donne-t-il le droit d'être injuste ? \\[0pt]
Le malheur est-il injuste ? \\[0pt]
Le malin plaisir \\[0pt]
Le mal métaphysique \\[0pt]
Le mal peut-il être absolu ? \\[0pt]
Le mal peut-il être involontaire ? \\[0pt]
L'émancipation \\[0pt]
L'émancipation des femmes \\[0pt]
Le maniérisme \\[0pt]
Le manifeste politique \\[0pt]
Le manque de culture \\[0pt]
Le manque de rigueur \\[0pt]
Le marché \\[0pt]
Le marché de l'art \\[0pt]
Le marché du travail \\[0pt]
Le mariage \\[0pt]
Le mariage est-il un contrat ? \\[0pt]
Le masculin \\[0pt]
Le masculin et le féminin \\[0pt]
Le masque \\[0pt]
Le matérialisme \\[0pt]
Le matériau musical \\[0pt]
Le matériel \\[0pt]
Le matériel et le virtuel \\[0pt]
Le mauvais goût \\[0pt]
L'embarras \\[0pt]
L'embarras du choix \\[0pt]
L'embryon humain est-il une personne ? \\[0pt]
L'embryon humain peut-il être l'objet d'expérimentation ? \\[0pt]
Le mécanisme \\[0pt]
Le mécanisme et la mécanique \\[0pt]
Le mécénat \\[0pt]
Le méchant \\[0pt]
Le méchant est-il malheureux ? \\[0pt]
Le méchant peut-il être heureux ? \\[0pt]
Le médiat et l'immédiat \\[0pt]
Le meilleur \\[0pt]
Le meilleur des mondes \\[0pt]
Le meilleur des mondes possible \\[0pt]
Le meilleur est-il l'ennemi du bien ? \\[0pt]
Le meilleur gouvernement est-il le gouvernement des meilleurs ? \\[0pt]
Le meilleur régime \\[0pt]
Le meilleur régime politique \\[0pt]
Le mélange \\[0pt]
Le même et l'autre \\[0pt]
Le ménage \\[0pt]
Le mensonge \\[0pt]
Le mensonge de l'art ? \\[0pt]
Le mensonge en politique \\[0pt]
Le mensonge est-il la plus grande transgression ? \\[0pt]
Le mensonge est-il une forme d'indifférence à la vérité ? \\[0pt]
Le mensonge peut-il être au service de la vérité ? \\[0pt]
Le mensonge politique \\[0pt]
Le mépris \\[0pt]
Le mépris de classe \\[0pt]
Le mépris des idées \\[0pt]
Le mépris peut-il être justifié ? \\[0pt]
Le mérite \\[0pt]
Le mérite est-il le critère de la vertu ? \\[0pt]
Le mérite et les talents \\[0pt]
Le merveilleux \\[0pt]
Le mesurable \\[0pt]
Le métaphysicien est-il un physicien à sa façon ? \\[0pt]
Le métier \\[0pt]
Le métier de philosophe \\[0pt]
Le métier de politique \\[0pt]
Le métier de savant \\[0pt]
Le métier d'historien \\[0pt]
Le métier d'homme \\[0pt]
Le métier du politique \\[0pt]
Le mien et le tien \\[0pt]
Le mieux est-il l'ennemi du bien ? \\[0pt]
Le milieu \\[0pt]
Le milieu social \\[0pt]
Le miracle \\[0pt]
Le miroir \\[0pt]
Le misanthrope \\[0pt]
Le mode \\[0pt]
Le mode d'existence de l'œuvre d'art \\[0pt]
Le modèle \\[0pt]
Le modèle du jeu dans les sciences humaines ? \\[0pt]
Le modèle en morale \\[0pt]
Le modèle et la copie \\[0pt]
Le modèle évolutionniste en histoire \\[0pt]
Le modèle organiciste \\[0pt]
Le modèle organiciste en sciences humaines \\[0pt]
Le moi \\[0pt]
Le moi est-il haïssable ? \\[0pt]
Le moi est-il insaisissable ? \\[0pt]
Le moi est-il objet de connaissance ? \\[0pt]
Le moi est-il une collection d'idées ? \\[0pt]
Le moi est-il une fiction ? \\[0pt]
Le moi est-il une illusion ? \\[0pt]
Le moi et la conscience \\[0pt]
Le moi et ses images \\[0pt]
Le moindre mal \\[0pt]
Le moi n'est-il qu'une fiction ? \\[0pt]
Le moi n'est-il qu'une idée ? \\[0pt]
Le moi peut-il être l'objet d'un savoir ? \\[0pt]
Le moi reste-t-il identique à lui-même au cours du temps ? \\[0pt]
Le moment propice \\[0pt]
Le monde \\[0pt]
Le monde à l'envers \\[0pt]
Le monde a-t-il besoin de moi ? \\[0pt]
Le monde a-t-il une histoire ? \\[0pt]
Le monde aurait-il un sens si l'homme n'existait pas ? \\[0pt]
Le monde de la culture \\[0pt]
Le monde de l'animal \\[0pt]
Le monde de l'animal nous est-il étranger ? \\[0pt]
Le monde de l'art \\[0pt]
Le monde de l'art et l'ordinateur \\[0pt]
Le monde de la technique \\[0pt]
Le monde de la vie \\[0pt]
Le monde de l'entreprise \\[0pt]
Le monde de l'être humain \\[0pt]
Le monde des idées \\[0pt]
Le monde des images \\[0pt]
Le monde des machines \\[0pt]
Le monde des œuvres \\[0pt]
Le monde des physiciens \\[0pt]
Le monde des rêves \\[0pt]
Le monde des sens \\[0pt]
Le monde du rêve \\[0pt]
Le monde du travail \\[0pt]
Le monde est-il autre chose que ce que j'en dis ? \\[0pt]
Le monde est-il écrit en langage mathématique ? \\[0pt]
Le monde est-il en progrès ? \\[0pt]
Le monde est-il éternel ? \\[0pt]
Le monde est-il ma représentation ? \\[0pt]
Le monde est-il notre représentation ? \\[0pt]
Le monde est-il une marchandise ? \\[0pt]
Le monde est-il un théâtre ? \\[0pt]
Le monde et la nature \\[0pt]
Le monde et la technique \\[0pt]
Le monde extérieur \\[0pt]
Le monde extérieur existe-t-il ? \\[0pt]
Le monde intelligible \\[0pt]
Le monde intérieur \\[0pt]
Le monde n'est-il que l'ensemble de ce qui existe ? \\[0pt]
Le monde n'est-il que ma représentation ? \\[0pt]
Le monde n'est-il qu'une représentation ? \\[0pt]
Le monde peut-il être nouveau ? \\[0pt]
Le monde politique \\[0pt]
Le monde sensible \\[0pt]
Le monde se réduit-il à ce que nous en voyons ? \\[0pt]
Le monde se suffit-il à lui-même ? \\[0pt]
Le monde vrai \\[0pt]
Le monopole de la violence légitime \\[0pt]
Le monstre \\[0pt]
Le monstrueux \\[0pt]
Le moralisme \\[0pt]
Le moraliste \\[0pt]
Le mot d'esprit \\[0pt]
Le mot et la chose \\[0pt]
Le mot et le geste \\[0pt]
L'émotion \\[0pt]
L'émotion esthétique \\[0pt]
L'émotion esthétique peut-elle se communiquer ? \\[0pt]
L'émotion esthétique peut-elle se partager ? \\[0pt]
L'émotion morale \\[0pt]
Le mot juste \\[0pt]
Le mot vie a-t-il plusieurs sens ? \\[0pt]
Le mouvement \\[0pt]
Le mouvement de la pensée \\[0pt]
L'empathie \\[0pt]
L'empathie est-elle nécessaire aux sciences sociales ? \\[0pt]
L'empathie est-elle possible ? \\[0pt]
L'empire \\[0pt]
L'empire sur soi \\[0pt]
L'empirisme \\[0pt]
L'empirisme des sciences humaines \\[0pt]
L'empirisme exclut-il l'abstraction ? \\[0pt]
L'emploi du temps \\[0pt]
Le multiculturalisme \\[0pt]
Le multiple \\[0pt]
Le multiple et l'un \\[0pt]
Le musée \\[0pt]
Le Musée \\[0pt]
Le mystère \\[0pt]
Le mysticisme \\[0pt]
Le mythe \\[0pt]
Le mythe est-il objet de science ? \\[0pt]
Le naïf \\[0pt]
Le narcissisme \\[0pt]
Le naturalisme \\[0pt]
Le naturalisme des sciences humaines et sociales \\[0pt]
Le naturel \\[0pt]
Le naturel a-t-il toutes les vertus ? \\[0pt]
Le naturel et l'artificiel \\[0pt]
Le naturel et le fabriqué \\[0pt]
Le naturel et le normal \\[0pt]
L'enchevêtrement des causes \\[0pt]
L'encyclopédie \\[0pt]
L'Encyclopédie \\[0pt]
Le néant \\[0pt]
Le néant est-il ? \\[0pt]
Le nécessaire et le contingent \\[0pt]
Le nécessaire et le superflu \\[0pt]
Le négatif \\[0pt]
Le néologisme \\[0pt]
L'énergie \\[0pt]
L'énergie du désespoir \\[0pt]
L'enfance \\[0pt]
L'enfance de l'art \\[0pt]
L'enfance est-elle ce qui doit être surmonté ? \\[0pt]
L'enfance est-elle en nous ce qui doit être abandonné ? \\[0pt]
L'enfance peut-elle servir de modèle ? \\[0pt]
L'enfant \\[0pt]
L'enfant et l'adulte \\[0pt]
L'enfant sauvage \\[0pt]
L'enfer est-il véritablement pavé de bonnes intentions ? \\[0pt]
« L'enfer est pavé de bonnes intentions » \\[0pt]
L'enfer est pavé de bonnes intentions \\[0pt]
L'engagement \\[0pt]
L'engagement dans l'art \\[0pt]
L'engagement politique \\[0pt]
L'engendrement \\[0pt]
L'énigme \\[0pt]
L'énigme, le mystère et le secret \\[0pt]
Le nihilisme \\[0pt]
L'ennemi \\[0pt]
L'ennemi intérieur \\[0pt]
L'ennui \\[0pt]
Le noble et le vil \\[0pt]
Le noir et blanc \\[0pt]
Le nomade \\[0pt]
Le nomadisme \\[0pt]
Le nombre \\[0pt]
Le nombre et la mesure \\[0pt]
Le nom et le verbe \\[0pt]
Le nominalisme \\[0pt]
Le nom propre \\[0pt]
Le nom propre et le nom commun \\[0pt]
Le non-être \\[0pt]
Le non-sens \\[0pt]
Le normal et le pathologique \\[0pt]
L'enquête \\[0pt]
L'enquête de terrain \\[0pt]
L'enquête empirique rend-elle la métaphysique inutile ? \\[0pt]
L'enquête sociale \\[0pt]
L'enseignement de la vérité \\[0pt]
L'enseignement peut-il se passer d'exemples ? \\[0pt]
L'entendement et la volonté \\[0pt]
L'enthousiasme \\[0pt]
L'enthousiasme est-il moral ? \\[0pt]
L'entraide \\[0pt]
Le nu \\[0pt]
Le nu et la nudité \\[0pt]
L'envie \\[0pt]
L'environnement \\[0pt]
L'environnement est-il un nouvel objet pour les sciences humaines ? \\[0pt]
L'environnement est-il un problème politique ? \\[0pt]
Le oui-dire \\[0pt]
Le pacifisme \\[0pt]
Le paradigme \\[0pt]
Le paradoxal \\[0pt]
Le paradoxe \\[0pt]
Le pardon \\[0pt]
Le pardon et l'oubli \\[0pt]
Le pardon peut-il être une obligation ? \\[0pt]
Le parfait \\[0pt]
Le pari \\[0pt]
Le partage \\[0pt]
Le partage des biens \\[0pt]
Le partage des connaissances \\[0pt]
Le partage des savoirs \\[0pt]
Le partage est-il une obligation morale ? \\[0pt]
Le particulier \\[0pt]
Le passage à l'acte \\[0pt]
Le passé \\[0pt]
Le passé a-t-il plus de réalité que l'avenir ? \\[0pt]
Le passé a-t-il une réalité ? \\[0pt]
Le passé a-t-il un intérêt ? \\[0pt]
Le passé détermine-t-il notre présent ? \\[0pt]
Le passé, est-ce du passé ? \\[0pt]
Le passé est-il ce qui a disparu ? \\[0pt]
Le passé est-il indépassable ? \\[0pt]
Le passé est-il objet de science ? \\[0pt]
Le passé est-il perdu ? \\[0pt]
Le passé est-il réel ? \\[0pt]
Le passé et le présent \\[0pt]
Le passé existe-t-il ? \\[0pt]
Le passé peut-il être un objet de connaissance ? \\[0pt]
Le passionné mérite-t-il d'être plaint ? \\[0pt]
Le paternalisme \\[0pt]
Le pathologique \\[0pt]
Le patriarcat \\[0pt]
Le patrimoine \\[0pt]
Le patrimoine artistique \\[0pt]
Le patrimoine de l'humanité \\[0pt]
Le patriotisme \\[0pt]
Le patriotisme est-il une vertu ? \\[0pt]
Le paysage \\[0pt]
Le paysage urbain \\[0pt]
Le pays natal \\[0pt]
Le péché \\[0pt]
Le pédagogue \\[0pt]
Le peintre et le sculpteur \\[0pt]
Le peintre et son modèle \\[0pt]
Le père, la mère, l'enfant \\[0pt]
Le personnage et la personne \\[0pt]
Le pessimisme \\[0pt]
Le peuple \\[0pt]
Le peuple a-t-il toujours raison ? \\[0pt]
Le peuple est-il bête ? \\[0pt]
Le peuple est-il souverain ? \\[0pt]
Le peuple est-il un acteur politique ? \\[0pt]
Le peuple et la nation \\[0pt]
Le peuple et les élites \\[0pt]
Le peuple peut-il se tromper ? \\[0pt]
Le peuple possède-t-il une volonté ? \\[0pt]
Le phantasme \\[0pt]
L'éphémère \\[0pt]
L'éphémère a-t-il une valeur ? \\[0pt]
Le phénomène \\[0pt]
Le phénomène religieux \\[0pt]
Le philanthrope \\[0pt]
Le philosophe a-t-il besoin de l'histoire ? \\[0pt]
Le philosophe a-t-il des leçons à donner au politique ? \\[0pt]
Le philosophe est-il le vrai politique ? \\[0pt]
Le philosophe et l'écrivain \\[0pt]
Le philosophe et le médecin \\[0pt]
Le philosophe et l'enfant \\[0pt]
Le philosophe et le sophiste \\[0pt]
Le philosophe-roi \\[0pt]
Le philosophe s'écarte-t-il du réel ? \\[0pt]
Le philosophe se détourne-t-il du réel ? \\[0pt]
Le physique et le mental \\[0pt]
L'épistémologie des sciences humaines \\[0pt]
L'épistémologie est-elle une logique de la science ? \\[0pt]
Le plagiat \\[0pt]
Le plaisir \\[0pt]
Le plaisir artistique est-il affaire de jugement ? \\[0pt]
Le plaisir a-t-il un rôle à jouer dans la morale ? \\[0pt]
Le plaisir d'avoir mal \\[0pt]
Le plaisir de l'art \\[0pt]
Le plaisir de parler \\[0pt]
Le plaisir des sens \\[0pt]
Le plaisir d'être libre \\[0pt]
Le plaisir d'imiter \\[0pt]
Le plaisir esthétique \\[0pt]
Le plaisir esthétique est-il affaire de jugement ? \\[0pt]
Le plaisir esthétique est-il un plaisir ? \\[0pt]
Le plaisir esthétique peut-il se communiquer ? \\[0pt]
Le plaisir esthétique peut-il se partager ? \\[0pt]
Le plaisir esthétique suppose-t-il une culture ? \\[0pt]
Le plaisir esthétique suppose-t-il une culture esthétique ? \\[0pt]
Le plaisir est-il immoral ? \\[0pt]
Le plaisir est-il la fin du désir ? \\[0pt]
Le plaisir est-il tout le bonheur ? \\[0pt]
Le plaisir est-il un bien ? \\[0pt]
Le plaisir et la douleur \\[0pt]
Le plaisir et la joie \\[0pt]
Le plaisir et la jouissance \\[0pt]
Le plaisir et la peine \\[0pt]
Le plaisir et le bien \\[0pt]
Le plaisir et le contentement \\[0pt]
Le plaisir peut-il être immoral ? \\[0pt]
Le plaisir peut-il être partagé ? \\[0pt]
Le plaisir se mérite-t-il ? \\[0pt]
Le plaisir suffit-il au bonheur ? \\[0pt]
Le pluralisme \\[0pt]
Le pluralisme politique \\[0pt]
Le plus et le moins \\[0pt]
Le plus grand bonheur pour le plus grand nombre \\[0pt]
Le plus grand des maux \\[0pt]
Le poète réinvente-t-il la langue ? \\[0pt]
Le poétique \\[0pt]
Le poids de la culture \\[0pt]
Le poids de la société \\[0pt]
Le poids des circonstances \\[0pt]
Le poids du passé \\[0pt]
Le poids du préjugé en politique \\[0pt]
Le poids du souvenir \\[0pt]
Le point de vue \\[0pt]
Le point de vue d'autrui \\[0pt]
Le point de vue de l'auteur \\[0pt]
Le point de vue du spectateur \\[0pt]
Le politique a-t-il à régler les passions ? \\[0pt]
Le politique a-t-il à régler les passions humaines ? \\[0pt]
Le politique doit-il être un technicien ? \\[0pt]
Le politique doit-il s'appuyer sur la science ? \\[0pt]
Le politique doit-il se soucier des émotions ? \\[0pt]
Le politique et l'économique \\[0pt]
Le politique et le religieux \\[0pt]
Le politique et le social \\[0pt]
Le politique peut-il faire abstraction de la morale ? \\[0pt]
Le populaire \\[0pt]
Le populisme \\[0pt]
Le portrait \\[0pt]
Le possible \\[0pt]
Le possible et le contingent \\[0pt]
Le possible et le probable \\[0pt]
Le possible et le réel \\[0pt]
Le possible et le virtuel \\[0pt]
Le possible et l'existence \\[0pt]
Le possible et l'impossible \\[0pt]
Le possible existe-t-il ? \\[0pt]
Le postulat et l'axiome \\[0pt]
Le pour et le contre \\[0pt]
Le pourquoi et le comment \\[0pt]
Le pouvoir \\[0pt]
Le pouvoir absolu \\[0pt]
Le pouvoir causal de l'inconscient \\[0pt]
Le pouvoir corrompt-il ? \\[0pt]
Le pouvoir corrompt-il nécessairement ? \\[0pt]
Le pouvoir corrompt-il toujours ? \\[0pt]
Le pouvoir de la beauté \\[0pt]
Le pouvoir de la minorité \\[0pt]
Le pouvoir de l'argent \\[0pt]
Le pouvoir de la science \\[0pt]
Le pouvoir de l'État est-il arbitraire ? \\[0pt]
Le pouvoir de l'habitude \\[0pt]
Le pouvoir de l'image \\[0pt]
Le pouvoir de l'imagination \\[0pt]
Le pouvoir de l'opinion \\[0pt]
Le pouvoir des idées \\[0pt]
Le pouvoir des images \\[0pt]
Le pouvoir des meilleurs \\[0pt]
Le pouvoir des mots \\[0pt]
Le pouvoir des paroles \\[0pt]
Le pouvoir des savants \\[0pt]
Le pouvoir des sciences humaines et sociales \\[0pt]
Le pouvoir du concept \\[0pt]
Le pouvoir du peuple \\[0pt]
Le pouvoir est-il une forme de domination ? \\[0pt]
Le pouvoir et l'autorité \\[0pt]
Le pouvoir et la violence \\[0pt]
Le pouvoir législatif \\[0pt]
Le pouvoir magique \\[0pt]
Le pouvoir peut-il arrêter le pouvoir ? \\[0pt]
Le pouvoir peut-il être limité ? \\[0pt]
Le pouvoir peut-il ignorer la morale ? \\[0pt]
Le pouvoir peut-il limiter le pouvoir ? \\[0pt]
Le pouvoir peut-il se déléguer ? \\[0pt]
Le pouvoir peut-il se passer de sa mise en scène ? \\[0pt]
Le pouvoir peut-il se transmettre ? \\[0pt]
Le pouvoir politique est-il nécessairement coercitif ? \\[0pt]
Le pouvoir politique peut-il échapper à l'arbitraire ? \\[0pt]
Le pouvoir politique repose-t-il sur la contrainte ? \\[0pt]
Le pouvoir politique repose-t-il sur un savoir ? \\[0pt]
Le pouvoir se partage-t-il ? \\[0pt]
Le pouvoir souverain \\[0pt]
Le pouvoir spirituel \\[0pt]
Le pouvoir tend-il toujours à l'excès ? \\[0pt]
Le pouvoir traditionnel \\[0pt]
Le pragmatisme \\[0pt]
Le préférable \\[0pt]
Le préjugé \\[0pt]
Le premier \\[0pt]
Le premier devoir de l'État est-il de se défendre ? \\[0pt]
Le premier et le primitif \\[0pt]
Le premier principe \\[0pt]
Le préscientifique \\[0pt]
Le présent \\[0pt]
Le présent historique \\[0pt]
L'épreuve \\[0pt]
L'épreuve de la liberté \\[0pt]
L'épreuve de la réalité \\[0pt]
L'épreuve des faits \\[0pt]
L'épreuve du pouvoir \\[0pt]
L'épreuve du réel \\[0pt]
Le primitif \\[0pt]
Le primitivisme en art \\[0pt]
Le prince \\[0pt]
Le principe \\[0pt]
Le principe de causalité \\[0pt]
Le principe de contradiction \\[0pt]
Le principe d'égalité \\[0pt]
Le principe de non-contradiction \\[0pt]
Le principe de précaution \\[0pt]
Le principe de raison \\[0pt]
Le principe de raison suffisante \\[0pt]
Le principe de réalité \\[0pt]
Le principe de réciprocité \\[0pt]
Le principe de simplicité \\[0pt]
Le principe d'identité \\[0pt]
Le privé et le public \\[0pt]
Le privilège de l'original \\[0pt]
Le prix \\[0pt]
Le prix de la liberté \\[0pt]
Le prix de la vérité \\[0pt]
Le prix des choses \\[0pt]
Le prix du travail \\[0pt]
Le probabilisme \\[0pt]
Le probable \\[0pt]
Le problème \\[0pt]
Le problème de l'être \\[0pt]
Le procès d'intention \\[0pt]
Le processus \\[0pt]
Le processus de civilisation \\[0pt]
Le prochain \\[0pt]
Le prochain et le semblable \\[0pt]
Le proche et le lointain \\[0pt]
Le profane \\[0pt]
Le profit \\[0pt]
Le profit est-il la fin de l'échange ? \\[0pt]
Le profit est-il le moteur de la production économique ? \\[0pt]
Le progrès \\[0pt]
Le progrès de l'humanité se réduit-il au progrès technique ? \\[0pt]
Le progrès des sciences \\[0pt]
Le progrès des sciences infirme-t-il les résultats anciens ? \\[0pt]
Le progrès en logique \\[0pt]
Le progrès est-il réversible ? \\[0pt]
Le progrès est-il un mythe ? \\[0pt]
Le progrès moral \\[0pt]
Le progrès scientifique fait-il disparaître la superstition ? \\[0pt]
Le progrès technique \\[0pt]
Le progrès technique a-t-il une fin ? \\[0pt]
Le progrès technique est-il source de bonheur ? \\[0pt]
Le progrès technique peut-il être aliénant ? \\[0pt]
Le projet \\[0pt]
Le projet d'une paix perpétuelle est-il insensé ? \\[0pt]
Le projet politique \\[0pt]
Le propre \\[0pt]
Le propre de la musique \\[0pt]
Le propre de l'homme \\[0pt]
Le propre du vivant est-il de tomber malade ? \\[0pt]
Le propre et l'impropre \\[0pt]
Le propriétaire \\[0pt]
Le protectionnisme \\[0pt]
Le provisoire \\[0pt]
Le psychisme est-il objet de connaissance ? \\[0pt]
Le psychologisme \\[0pt]
Le public \\[0pt]
Le public et le privé \\[0pt]
Le pur et l'impur \\[0pt]
Le quelconque \\[0pt]
Lequel, de l'art ou du réel, est-il une imitation de l'autre ? \\[0pt]
Le quelque chose \\[0pt]
L'équilibre \\[0pt]
L'équilibre des pouvoirs \\[0pt]
L'équilibre économique \\[0pt]
L'équité \\[0pt]
L'équivalence \\[0pt]
L'équivalence des hypothèses \\[0pt]
L'équivocité \\[0pt]
L'équivocité du langage \\[0pt]
L'équivoque \\[0pt]
Le quotidien \\[0pt]
Le racisme \\[0pt]
Le raffinement \\[0pt]
Le raisonnable et le rationnel \\[0pt]
Le raisonnement par l'absurde \\[0pt]
Le raisonnement scientifique \\[0pt]
Le raisonnement suit-il des règles ? \\[0pt]
Le rapport de force \\[0pt]
Le rapport de l'homme à son milieu a-t-il une dimension morale ? \\[0pt]
Le rationalisme \\[0pt]
Le rationalisme peut-il être une passion ? \\[0pt]
Le rationnel \\[0pt]
Le rationnel et le raisonnable \\[0pt]
Le rationnel et l'irrationnel \\[0pt]
Le réalisme \\[0pt]
Le réalisme de la science \\[0pt]
Le réalisme politique \\[0pt]
Le récit \\[0pt]
Le récit en histoire \\[0pt]
Le récit historique \\[0pt]
Le récit national \\[0pt]
Le reconnaissance \\[0pt]
Le recours à la force signifie-t-il l'échec de la justice ? \\[0pt]
Le recours à l'Histoire \\[0pt]
Le recours au symbole est-il la marque de la finitude de la pensée ? \\[0pt]
Le réel \\[0pt]
Le réel est-il ce que l'on croit ? \\[0pt]
Le réel est-il ce que nous expérimentons ? \\[0pt]
Le réel est-il ce que nous percevons ? \\[0pt]
Le réel est-il ce qui apparaît ? \\[0pt]
Le réel est-il ce qui est perçu ? \\[0pt]
Le réel est-il ce qui résiste ? \\[0pt]
Le réel est-il inaccessible ? \\[0pt]
Le réel est-il l'objet de la science ? \\[0pt]
Le réel est-il objet d'interprétation ? \\[0pt]
Le réel est-il rationnel ? \\[0pt]
Le réel et la fiction \\[0pt]
Le réel et le matériel \\[0pt]
Le réel et le nécessaire \\[0pt]
Le réel et le possible \\[0pt]
Le réel et le virtuel \\[0pt]
Le réel et le vrai \\[0pt]
Le réel et l'idéal \\[0pt]
Le réel et l'imaginaire \\[0pt]
Le réel et l'impossible \\[0pt]
Le réel et l'irréel \\[0pt]
Le réel excède-t-il l'intelligible ? \\[0pt]
Le réel n'est-il qu'un ensemble de contraintes ? \\[0pt]
Le réel obéit-il à la raison ? \\[0pt]
Le réel peut-il échapper à la logique ? \\[0pt]
Le réel peut-il être beau ? \\[0pt]
Le réel peut-il être contradictoire ? \\[0pt]
Le réel résiste-t-il à la connaissance ? \\[0pt]
Le réel se donne-t-il à voir ? \\[0pt]
Le réel se limite-t-il à ce que font connaître les théories scientifiques ? \\[0pt]
Le réel se limite-t-il à ce que nous percevons ? \\[0pt]
Le réel se réduit-il à ce que l'on perçoit ? \\[0pt]
Le réel se réduit-il à l'objectivité ? \\[0pt]
Le refoulement \\[0pt]
Le refus \\[0pt]
Le refus de l'autorité \\[0pt]
Le refus de la vérité \\[0pt]
Le regard \\[0pt]
Le regard de l'autre \\[0pt]
Le regard du photographe \\[0pt]
Le regard éloigné \\[0pt]
Le règlement politique des conflits ? \\[0pt]
Le règne de la technique \\[0pt]
Le règne de l'homme \\[0pt]
Le règne des experts \\[0pt]
Le règne des fins \\[0pt]
Le règne des passions \\[0pt]
Le regret \\[0pt]
Le relativisme \\[0pt]
Le relativisme culturel \\[0pt]
Le relativisme culturel est-il indépassable ? \\[0pt]
Le relativisme dans les sciences humaines \\[0pt]
Le relativisme moral \\[0pt]
Le relativisme peut-il être un principe de méthode dans les sciences humaines ? \\[0pt]
Le religieux est-il inutile ? \\[0pt]
Le remords \\[0pt]
Le remords est-il stérile ? \\[0pt]
Le renoncement \\[0pt]
Le repas \\[0pt]
Le repentir \\[0pt]
Le repos \\[0pt]
Le respect \\[0pt]
Le respect de la nature \\[0pt]
Le respect de la tradition \\[0pt]
Le respect des convenances \\[0pt]
Le respect des institutions \\[0pt]
Le respect de soi \\[0pt]
Le respect de soi-même \\[0pt]
Le respect n'est-il dû qu'aux personnes ? \\[0pt]
Le ressentiment \\[0pt]
Le retour à la nature \\[0pt]
Le retour à la nature est-il souhaitable ? \\[0pt]
Le retour à l'expérience \\[0pt]
Le rêve \\[0pt]
Le rêve et la réalité \\[0pt]
Le rêve et la veille \\[0pt]
Le révisionnisme \\[0pt]
Le riche et le pauvre \\[0pt]
Le ridicule \\[0pt]
Le rien \\[0pt]
Le rigorisme \\[0pt]
Le rire \\[0pt]
Le risque \\[0pt]
Le risque calculé \\[0pt]
Le risque de la liberté \\[0pt]
Le risque technique \\[0pt]
Le risque technologique \\[0pt]
Le rite \\[0pt]
Le rituel \\[0pt]
Le rôle de la théorie dans l'expérience scientifique \\[0pt]
Le rôle de l'État est-il de faire régner la justice ? \\[0pt]
Le rôle de l'État est-il de préserver la liberté de l'individu ? \\[0pt]
Le rôle de l'historien est-il de juger ? \\[0pt]
Le rôle des exemples \\[0pt]
Le rôle des institutions \\[0pt]
Le rôle des théories est-il d'expliquer ou de décrire ? \\[0pt]
Le roman \\[0pt]
Le roman national \\[0pt]
Le roman peut-il être philosophique ? \\[0pt]
Le romantisme \\[0pt]
L'érotisme \\[0pt]
Le royaume du possible \\[0pt]
L'erreur \\[0pt]
L'erreur est-elle humaine ? \\[0pt]
L'erreur est humaine \\[0pt]
L'erreur et la faute \\[0pt]
L'erreur et l'ignorance \\[0pt]
L'erreur et l'illusion \\[0pt]
L'erreur peut-elle donner un accès à la vérité ? \\[0pt]
L'erreur peut-elle jouer un rôle dans la connaissance scientifique ? \\[0pt]
L'erreur politique, la faute politique \\[0pt]
L'erreur scientifique \\[0pt]
L'érudition \\[0pt]
Le rythme \\[0pt]
Le rythme en peinture \\[0pt]
Le sacré \\[0pt]
Le sacré et le profane \\[0pt]
Le sacrifice \\[0pt]
Le sacrifice de soi \\[0pt]
Les acteurs de l'histoire en sont-ils les auteurs ? \\[0pt]
Les affaires publiques \\[0pt]
Les affections \\[0pt]
Les affects sont-ils déraisonnables ? \\[0pt]
Les affects sont-ils des objets sociologiques ? \\[0pt]
Le sage a-t-il besoin d'autrui ? \\[0pt]
Le sage est-il insensible ? \\[0pt]
Les agents économiques ont-ils un comportement rationnel ? \\[0pt]
Les agents sociaux poursuivent-ils l'utilité ? \\[0pt]
Les agents sociaux sont-ils rationnels ? \\[0pt]
Les âges de la vie \\[0pt]
Les âges de l'humanité \\[0pt]
Le salaire \\[0pt]
Le salut \\[0pt]
Le salut du peuple \\[0pt]
Le salut vient-il de la raison ? \\[0pt]
Les amis \\[0pt]
Les analogies dans les sciences humaines \\[0pt]
Les anciens et les modernes \\[0pt]
Les Anciens et les Modernes \\[0pt]
Les animaux échappent-ils à la moralité ? \\[0pt]
Les animaux ont-ils des droits ? \\[0pt]
Les animaux pensent-ils ? \\[0pt]
Les animaux peuvent-ils avoir des droits ? \\[0pt]
Les animaux révèlent-ils ce que nous sommes ? \\[0pt]
Les animaux sont-ils nos amis ? \\[0pt]
Les antagonismes sociaux \\[0pt]
Les apparences font-elles partie du monde ? \\[0pt]
Les apparences sont-elles toujours trompeuses ? \\[0pt]
Les archives \\[0pt]
Les artistes sont-ils sérieux ? \\[0pt]
Les arts admettent-ils une hiérarchie ? \\[0pt]
Les arts appliqués \\[0pt]
Les arts communiquent-ils entre eux ? \\[0pt]
Les arts décoratifs \\[0pt]
Les arts de la mémoire \\[0pt]
Les arts industriels \\[0pt]
Les arts mineurs \\[0pt]
Les arts nobles \\[0pt]
Les arts ont-ils besoin de théorie ? \\[0pt]
Les arts ont-ils pour fonction de divertir ? \\[0pt]
Les arts populaires \\[0pt]
Les arts sont-ils des jeux ? \\[0pt]
Les arts vivants \\[0pt]
Les atomes \\[0pt]
Les autorités \\[0pt]
Les autres \\[0pt]
Le sauvage \\[0pt]
Le sauvage et le barbare \\[0pt]
Le sauvage et le civilisé \\[0pt]
Le sauvage et le cultivé \\[0pt]
Le savant et le politique \\[0pt]
Le savant et l'ignorant \\[0pt]
Le savant et l'ingénieur \\[0pt]
Les avant-gardes \\[0pt]
Le savant, l'artiste et le politique \\[0pt]
Le savoir absolu \\[0pt]
Le savoir a-t-il besoin d'être fondé ? \\[0pt]
Le savoir a-t-il des degrés ? \\[0pt]
Le savoir du corps \\[0pt]
Le savoir du peintre \\[0pt]
Le savoir émancipe-t-il ? \\[0pt]
Le savoir est-il libérateur ? \\[0pt]
Le savoir est-il une condition du bonheur ? \\[0pt]
Le savoir est-il une forme de pouvoir ? \\[0pt]
Le savoir exclut-il toute forme de croyance ? \\[0pt]
Le savoir-faire \\[0pt]
Le savoir peut-il être gratuit ? \\[0pt]
Le savoir rend-il libre ? \\[0pt]
Le savoir se vulgarise-t-il ? \\[0pt]
Le savoir total \\[0pt]
Le savoir utile au citoyen \\[0pt]
Les beautés de la nature \\[0pt]
Les beaux-arts \\[0pt]
Les beaux-arts sont-ils compatibles entre eux ? \\[0pt]
Les belles actions \\[0pt]
Les belles choses \\[0pt]
Les bénéfices du doute \\[0pt]
Les bénéfices moraux \\[0pt]
Les besoins et les désirs \\[0pt]
Les bêtes \\[0pt]
Les bêtes travaillent-elles ? \\[0pt]
Les bienfaits de la coopération \\[0pt]
Les biens communs \\[0pt]
Les biens culturels \\[0pt]
Les biotechnologies \\[0pt]
Les blessures de l'esprit \\[0pt]
Les bonnes intentions \\[0pt]
Les bonnes lois font-elles les bonnes mœurs ? \\[0pt]
Les bonnes manières \\[0pt]
Les bonnes mœurs \\[0pt]
Les bonnes résolutions \\[0pt]
Les bons comptes font-ils les bons amis ? \\[0pt]
« Les bons comptes font les bons amis » \\[0pt]
Les bons sentiments \\[0pt]
Les bornes de la conscience \\[0pt]
Le scandale \\[0pt]
Les canons de la beauté \\[0pt]
Les caractères \\[0pt]
Les caractères moraux \\[0pt]
Les cas de conscience \\[0pt]
Les catastrophes \\[0pt]
Les catégories \\[0pt]
Les catégories sont-elles définitives ? \\[0pt]
Les catégories sont-elles des effets de langue ? \\[0pt]
Les causes \\[0pt]
Les causes et les conditions \\[0pt]
Les causes et les effets \\[0pt]
Les causes et les lois \\[0pt]
Les causes et les raisons \\[0pt]
Les causes et les signes \\[0pt]
Les causes finales \\[0pt]
Les causes naturelles expliquent-elles la nature ? \\[0pt]
Le scepticisme \\[0pt]
Le scepticisme a-t-il des limites ? \\[0pt]
Les cérémonies \\[0pt]
Les changements de théorie s'expliquent-ils seulement par le résultat des expériences ? \\[0pt]
Les changements scientifiques et la réalité \\[0pt]
Les chemins de traverse \\[0pt]
Les choses \\[0pt]
Les choses en soi \\[0pt]
Les choses et les événements \\[0pt]
Les choses humaines \\[0pt]
Les choses ont-elles l'air de ce qu'elles sont ? \\[0pt]
Les choses ont-elles quelque chose en commun ? \\[0pt]
Les choses ont-elles une essence ? \\[0pt]
Les choses ont-elles un sens ? \\[0pt]
Les choses peuvent-elles avoir un sens ? \\[0pt]
Les choses sont-elles dans l'espace ? \\[0pt]
Les cinq sens \\[0pt]
Les circonstances \\[0pt]
Les classes sociales \\[0pt]
L'esclavage \\[0pt]
L'esclavage des passions \\[0pt]
L'esclave \\[0pt]
L'esclave et son maître \\[0pt]
Les clichés \\[0pt]
Les coïncidences ont-elles des causes ? \\[0pt]
Les commandements divins \\[0pt]
Les commencements \\[0pt]
Les comportements humains s'expliquent-il par l'instinct naturel ? \\[0pt]
Les concepts sont-ils des fictions ? \\[0pt]
Les concitoyens doivent-ils être des amis ? \\[0pt]
Les conditions de la démocratie \\[0pt]
Les conditions d'existence \\[0pt]
Les conditions du dialogue \\[0pt]
Les conditions d'une science historique \\[0pt]
Les conflits de devoirs \\[0pt]
Les conflits menacent-ils la société ? \\[0pt]
Les conflits politiques \\[0pt]
Les conflits politiques ne sont-ils que des conflits sociaux ? \\[0pt]
Les conflits sociaux \\[0pt]
Les conflits sociaux sont-ils des conflits de classe ? \\[0pt]
Les conflits sociaux sont-ils des conflits politiques ? \\[0pt]
Les connaissances scientifiques peuvent-elles être à la fois vraies et provisoires ? \\[0pt]
Les connaissances scientifiques peuvent-elles être vulgarisées ? \\[0pt]
Les connaissances scientifiques sont-elles vraies ? \\[0pt]
Les conquêtes de la science \\[0pt]
Les conséquences \\[0pt]
Les conséquences de l'action \\[0pt]
Les considérations morales ont-elles leur place en politique ? \\[0pt]
Les contradictions de la raison \\[0pt]
Les contradictions sont-elles un échec de la raison ? \\[0pt]
Les contre-pouvoirs \\[0pt]
Les convenances \\[0pt]
Les conventions \\[0pt]
Les convictions d'autrui sont-elles un argument ? \\[0pt]
Les couleurs \\[0pt]
Les couleurs existent-elles en dehors de notre esprit ? \\[0pt]
Les coutumes \\[0pt]
Les critères de vérité dans les sciences humaines \\[0pt]
Les croyances politiques \\[0pt]
Les croyances religieuses sont-elles indiscutables ? \\[0pt]
Les croyances sont-elles utiles ? \\[0pt]
Le scrupule \\[0pt]
Les cultures sont-elles incommensurables ? \\[0pt]
Les décisions politiques peuvent-elles être collectives ? \\[0pt]
Les degrés de conscience \\[0pt]
Les degrés de la beauté \\[0pt]
Les degrés du beau \\[0pt]
Les démonstrations ont-elles des propriétés esthétiques ? \\[0pt]
Les désaccords politiques peuvent-ils être surmontés ? \\[0pt]
Les désirs et les valeurs \\[0pt]
Les désirs ont-ils nécessairement un objet ? \\[0pt]
Les devoirs à l'égard de la nature \\[0pt]
Les devoirs de l'État \\[0pt]
Les devoirs de l'homme varient-ils selon la culture ? \\[0pt]
Les devoirs de l'homme varient-ils selon les cultures ? \\[0pt]
Les devoirs du citoyen \\[0pt]
Les devoirs envers nos proches \\[0pt]
Les devoirs envers soi-même \\[0pt]
Les dictionnaires \\[0pt]
Les différences entre les sciences humaines sont-elles des différences d'objet ? \\[0pt]
Les différentes preuves de l'existence de Dieu prouvent-elles le même Dieu ? \\[0pt]
Les difficultés de l'observation \\[0pt]
Les dilemmes moraux \\[0pt]
Les disciplines scientifiques et leurs interfaces \\[0pt]
Les dispositions sociales \\[0pt]
Les distinctions sociales \\[0pt]
Les divers sens du mot culture peuvent-ils être ramenés à l'unité ? \\[0pt]
Les divisions entre les hommes \\[0pt]
Les divisions sociales \\[0pt]
Les dogmes \\[0pt]
Les droits de la nature \\[0pt]
Les droits de l'enfant \\[0pt]
Les droits de l'homme \\[0pt]
Les droits de l'homme et ceux du citoyen \\[0pt]
Les droits de l'homme ont-ils besoin d'être fondés ? \\[0pt]
Les droits de l'homme ont-ils un fondement moral ? \\[0pt]
Les droits de l'homme sont-ils ceux du citoyen ? \\[0pt]
Les droits de l'homme sont-ils les droits de la femme ? \\[0pt]
Les droits de l'homme sont-ils une abstraction ? \\[0pt]
Les droits de l'individu \\[0pt]
Les droits des animaux \\[0pt]
Les droits et les devoirs \\[0pt]
Les droits naturels imposent-ils une limite à la politique ? \\[0pt]
Les droits sociaux \\[0pt]
Les échanges \\[0pt]
Les échanges économiques sont-ils facteurs de paix ? \\[0pt]
Les échanges, facteurs de paix ? \\[0pt]
Les échanges favorisent-ils la paix ? \\[0pt]
Les échanges sont-ils facteurs de paix ? \\[0pt]
Les écrans \\[0pt]
Le secret \\[0pt]
Le secret d'État \\[0pt]
Les effets de l'esclavage \\[0pt]
Les éléments \\[0pt]
Les élites \\[0pt]
Le semblable \\[0pt]
Les émotions politiques \\[0pt]
Les enfants \\[0pt]
Le sensationnel \\[0pt]
Le sens caché \\[0pt]
Le sens commun \\[0pt]
Le sens commun existe-t-il ? \\[0pt]
Le sens de la famille \\[0pt]
Le sens de la fête \\[0pt]
Le sens de la justice \\[0pt]
Le sens de la mesure \\[0pt]
Le sens de la réalité \\[0pt]
Le sens de la situation \\[0pt]
Le sens de la vie \\[0pt]
Le sens de l'État \\[0pt]
Le sens de l'être \\[0pt]
Le sens de l'existence \\[0pt]
Le sens de l'histoire \\[0pt]
Le sens de l'Histoire \\[0pt]
Le sens de l'humour \\[0pt]
Le sens de l'opportunité est-il moral ? \\[0pt]
Le sens des mots \\[0pt]
Le sens des mots dépend-il de notre connaissance des choses ? \\[0pt]
Le sens de tout énoncé dépend-il de son interprétation ? \\[0pt]
Le sens du destin \\[0pt]
Le sens du devoir \\[0pt]
Le sens du devoir nous fait-il connaître le bien de façon indiscutable ? \\[0pt]
Le sens d'un texte \\[0pt]
Le sens du problème \\[0pt]
Le sens du sacrifice \\[0pt]
Le sens du silence \\[0pt]
Le sens du travail \\[0pt]
Les ensembles \\[0pt]
Le sens et le non-sens \\[0pt]
Le sensible \\[0pt]
Le sensible est-il communicable ? \\[0pt]
Le sensible est-il irréductible à l'intelligible ? \\[0pt]
Le sensible et la science \\[0pt]
Le sensible et l'intelligible \\[0pt]
Le sensible peut-il être connu ? \\[0pt]
Le sensible peut-il fonder des certitudes ? \\[0pt]
Le sens interne \\[0pt]
Le sens moral \\[0pt]
Le sens moral est-il naturel ? \\[0pt]
Le sens musical \\[0pt]
Le sens pratique \\[0pt]
Le sens propre \\[0pt]
Le sentiment \\[0pt]
Le sentiment de culpabilité \\[0pt]
Le sentiment de la faute \\[0pt]
Le sentiment de l'existence \\[0pt]
Le sentiment de liberté \\[0pt]
Le sentiment de l'injustice \\[0pt]
Le sentiment d'injustice \\[0pt]
Le sentiment d'injustice est-il naturel ? \\[0pt]
Le sentiment du devoir accompli \\[0pt]
Le sentiment du juste et de l'injuste \\[0pt]
Le sentiment du libre arbitre \\[0pt]
Le sentiment esthétique \\[0pt]
Le sentiment moral \\[0pt]
Les entités mathématiques sont-elles des fictions ? \\[0pt]
Les envieux \\[0pt]
Les épreuves ont-elles un sens ? \\[0pt]
Le sérieux \\[0pt]
Le serment \\[0pt]
Le service de l'État \\[0pt]
Le service public \\[0pt]
Les États ont-ils pour fin d'assurer la paix ? \\[0pt]
Les éthiques professionnelles \\[0pt]
Les êtres vivants sont-ils des machines ? \\[0pt]
Les études comparatives \\[0pt]
Les études de genre \\[0pt]
« Le seul problème philosophique vraiment sérieux, c'est le suicide » \\[0pt]
Les événements \\[0pt]
Les événements historiques sont-ils de nature imprévisible ? \\[0pt]
Les exceptions ont-elles une place en morale ? \\[0pt]
Les excuses \\[0pt]
Le sexe \\[0pt]
Les experts \\[0pt]
Les factions politiques \\[0pt]
Les faits \\[0pt]
Les faits et les valeurs \\[0pt]
Les faits existent-ils indépendamment de leur établissement par l'esprit humain ? \\[0pt]
Les faits parlent-ils d'eux-mêmes ? \\[0pt]
Les faits peuvent-ils faire autorité ? \\[0pt]
« Les faits, rien que les faits » \\[0pt]
Les faits sociaux se répètent-ils ? \\[0pt]
Les faits sont-ils des preuves ? \\[0pt]
Les faits sont-ils têtus ? \\[0pt]
« Les faits sont là » \\[0pt]
Les fausses sciences \\[0pt]
Les fictions \\[0pt]
Les fins de la culture \\[0pt]
Les fins de l'art \\[0pt]
Les fins de la science \\[0pt]
Les fins de la technique sont-elles techniques ? \\[0pt]
Les fins de l'éducation \\[0pt]
Les fins dernières \\[0pt]
Les fins et les moyens \\[0pt]
Les fins naturelles et les fins morales \\[0pt]
Les fins sont-elles toujours intentionnelles ? \\[0pt]
Les fonctions de l'image \\[0pt]
Les fondements de l'État \\[0pt]
Les « forces de l'ordre » \\[0pt]
Les forces de l'ordre \\[0pt]
Les formes de vie \\[0pt]
Les formes du vivant \\[0pt]
Les forts et les faibles \\[0pt]
Les foules \\[0pt]
Les fous \\[0pt]
Les frontières \\[0pt]
Les frontières de la connaissance \\[0pt]
Les frontières de l'art \\[0pt]
Les frontières du moi \\[0pt]
Les frontières du vivant \\[0pt]
Les fruits du travail \\[0pt]
Les générations \\[0pt]
Les générations futures \\[0pt]
Les générations futures ont-elles des droits ? \\[0pt]
Les genres de Dieu \\[0pt]
Les genres de l'être \\[0pt]
Les genres de vie \\[0pt]
Les genres esthétiques \\[0pt]
Les genres naturels \\[0pt]
Les grands hommes \\[0pt]
Les habitudes de pensée \\[0pt]
Les habitudes nous forment-elles ? \\[0pt]
Les hasards de la vie \\[0pt]
Les héros \\[0pt]
Les hommes croient-ils dans leurs mythes ? \\[0pt]
Les hommes de pouvoir \\[0pt]
Les hommes et les dieux \\[0pt]
Les hommes et les femmes \\[0pt]
Les hommes n'agissent-ils que par intérêt ? \\[0pt]
Les hommes naissent-ils libres ? \\[0pt]
Les hommes ont-ils besoin de maîtres ? \\[0pt]
Les hommes peuvent-ils s'associer sans renoncer à leur liberté ? \\[0pt]
Les hommes savent-ils ce qu'ils désirent ? \\[0pt]
Les hommes sont-ils des animaux ? \\[0pt]
Les hommes sont-ils faits pour s'entendre ? \\[0pt]
Les hommes sont-ils frères ? \\[0pt]
Les hommes sont-ils naturellement sociables ? \\[0pt]
Les hommes sont-ils seulement le produit de leur culture ? \\[0pt]
Les hors-la-loi \\[0pt]
Les hypothèses scientifiques ont-elles pour nature d'être confirmées ou infirmées ? \\[0pt]
Les idées et les choses \\[0pt]
Les idées existent-elles ? \\[0pt]
Les idées fixes \\[0pt]
Les idées mènent-elles le monde ? \\[0pt]
Les idées ont-elles une existence éternelle ? \\[0pt]
Les idées ont-elles une histoire ? \\[0pt]
Les idées ont-elles une réalité ? \\[0pt]
Les idées politiques \\[0pt]
Les idées reçues \\[0pt]
Les idées sont-elles vivantes ? \\[0pt]
Les idoles \\[0pt]
Le signe \\[0pt]
Le signe et le symbole \\[0pt]
Le signe et l'indice \\[0pt]
Le silence \\[0pt]
Le silence a-t-il un sens ? \\[0pt]
Le silence des lois \\[0pt]
Le silence est-il signifiant ? \\[0pt]
Le silence signifie-t-il toujours l'échec du langage ? \\[0pt]
Les images empêchent-elles de penser ? \\[0pt]
Les images nous égarent-elles ? \\[0pt]
Les images ont-elles un sens ? \\[0pt]
Les images peuvent-elles instruire ? \\[0pt]
Les images peuvent-elles mentir ? \\[0pt]
Le simple \\[0pt]
Le simple et le complexe \\[0pt]
Le simulacre \\[0pt]
Les individus \\[0pt]
Les individus se réduisent-ils à leurs qualités ? \\[0pt]
Les industries culturelles \\[0pt]
Les inégalités de la nature doivent-elles être compensées ? \\[0pt]
Les inégalités menacent-elles la société ? \\[0pt]
Les inégalités sociales \\[0pt]
Les inégalités sociales sont-elles inévitables ? \\[0pt]
Les inégalités sociales sont-elles naturelles ? \\[0pt]
Le singulier \\[0pt]
Le singulier est-il objet de connaissance ? \\[0pt]
Le singulier et le pluriel \\[0pt]
Les institutions artistiques \\[0pt]
Les institutions de l'art \\[0pt]
Les institutions politiques : qu'ont-elles en propre ? \\[0pt]
Les instruments de la pensée \\[0pt]
Les intentions de l'artiste \\[0pt]
Les intentions et les actes \\[0pt]
Les intentions et les conséquences \\[0pt]
Les interdits \\[0pt]
Les intérêts particuliers peuvent-ils tempérer l'autorité politique ? \\[0pt]
Les invariants culturels \\[0pt]
Les jeux de hasard \\[0pt]
Les jeux du pouvoir \\[0pt]
Les jugements analytiques \\[0pt]
Les jugements de goût sont-ils susceptibles de vérité ? \\[0pt]
Les langages de l'art \\[0pt]
Les langages formels \\[0pt]
Les langues meurent-elles ? \\[0pt]
Les langues que nous parlons sont-elles imparfaites ? \\[0pt]
Les langues sont-elles condamnées à l'approximation ? \\[0pt]
Les leçons de l'expérience \\[0pt]
Les leçons de l'histoire \\[0pt]
Les leçons de morale \\[0pt]
Les lettres et les sciences \\[0pt]
Les libertés civiles \\[0pt]
Les libertés fondamentales \\[0pt]
Les libertés politiques \\[0pt]
Les libertés publiques \\[0pt]
Les liens \\[0pt]
Les liens sociaux \\[0pt]
Les lieux de mémoire \\[0pt]
Les lieux du pouvoir \\[0pt]
Les limites de la connaissance \\[0pt]
Les limites de la connaissance scientifique \\[0pt]
Les limites de la démocratie \\[0pt]
Les limites de la description \\[0pt]
Les limites de la discussion \\[0pt]
Les limites de la raison \\[0pt]
Les limites de la science \\[0pt]
Les limites de la tolérance \\[0pt]
Les limites de la vérité \\[0pt]
Les limites de la vertu \\[0pt]
Les limites de l'État \\[0pt]
Les limites de l'expérience \\[0pt]
Les limites de l'humain \\[0pt]
Les limites de l'imaginaire \\[0pt]
Les limites de l'imagination \\[0pt]
Les limites de l'interprétation \\[0pt]
Les limites de l'obéissance \\[0pt]
Les limites de ma langue sont-elles les limites de mon monde ? \\[0pt]
Les limites du corps \\[0pt]
Les limites du déterminisme \\[0pt]
Les limites du langage \\[0pt]
Les limites du plaisir \\[0pt]
Les limites du possible \\[0pt]
Les limites du pouvoir \\[0pt]
Les limites du pouvoir politique \\[0pt]
Les limites du réel \\[0pt]
Les limites du vivant \\[0pt]
Les livres \\[0pt]
Les lois \\[0pt]
Les lois causales \\[0pt]
Les lois de la guerre \\[0pt]
Les lois de la nature \\[0pt]
Les lois de la nature sont-elles contingentes ? \\[0pt]
Les lois de la nature sont-elles de simples régularités ? \\[0pt]
Les lois de la nature sont elles nécessaires ? \\[0pt]
Les lois de la pensée \\[0pt]
Les lois de l'art \\[0pt]
Les lois de l'histoire \\[0pt]
Les lois de l'hospitalité \\[0pt]
Les lois du sang \\[0pt]
Les lois et les armes \\[0pt]
Les lois et les mœurs \\[0pt]
Les lois naturelles \\[0pt]
Les lois nous rendent-elles meilleurs ? \\[0pt]
Les lois scientifiques sont-elles des lois de la nature ? \\[0pt]
Les lois sont-elles sacrées ? \\[0pt]
Les lois sont-elles seulement utiles ? \\[0pt]
Les lois suffisent-elles à garantir nos libertés ? \\[0pt]
Les luttes politiques \\[0pt]
Les machines \\[0pt]
Les machines nous rendent-elles libres ? \\[0pt]
Les machines pensent-elles ? \\[0pt]
Les machines permettent-elles de mieux connaître le corps humain ? \\[0pt]
Les maîtres de vérité \\[0pt]
Les maladies de l'âme \\[0pt]
Les maladies de l'esprit \\[0pt]
Les maladies mentales sont-elles des maladies sociales ? \\[0pt]
Les marginaux \\[0pt]
Les masses \\[0pt]
Les matériaux \\[0pt]
Les mathématiques consistent-elles seulement en des opérations de l'esprit ? \\[0pt]
Les mathématiques du mouvement \\[0pt]
Les mathématiques et la pensée de l'infini \\[0pt]
Les mathématiques et la quantité \\[0pt]
Les mathématiques et le monde sensible \\[0pt]
Les mathématiques et l'expérience \\[0pt]
Les mathématiques ont-elles affaire au réel ? \\[0pt]
Les mathématiques ont-elles besoin d'un fondement ? \\[0pt]
Les mathématiques parlent-elles du réel ? \\[0pt]
Les mathématiques peuvent-elles se passer de l'intuition ? \\[0pt]
Les mathématiques se réduisent-elles à une pensée cohérente ? \\[0pt]
Les mathématiques sont-elles le modèle de toute science ? \\[0pt]
Les mathématiques sont-elles réductibles à la logique ? \\[0pt]
Les mathématiques sont-elles une découverte ou une invention ? \\[0pt]
Les mathématiques sont-elles une science ? \\[0pt]
Les mathématiques sont-elles une science de l'ordre ? \\[0pt]
Les mathématiques sont-elles un instrument ? \\[0pt]
Les mathématiques sont-elles un jeu de l'esprit ? \\[0pt]
Les mathématiques sont-elles un langage ? \\[0pt]
Les mathématiques sont-elles utiles au philosophe ? \\[0pt]
Les mauvaises habitudes \\[0pt]
Les mauvaises intentions \\[0pt]
Les mécanismes cérébraux \\[0pt]
Les mécanismes de la vie \\[0pt]
Les mécanismes d'identification \\[0pt]
Les méchants peuvent-ils être amis ? \\[0pt]
Les méchants peuvent-ils faire société ? \\[0pt]
Les métamorphoses du goût \\[0pt]
Les miracles \\[0pt]
« Les miracles de la technique » \\[0pt]
Les miracles de la technique \\[0pt]
Les modalités \\[0pt]
Les modèles \\[0pt]
Les mœurs \\[0pt]
Les mœurs et la morale \\[0pt]
Les mœurs sont-elles objet de science ? \\[0pt]
Les mœurs sont-ils l'objet d'une science ? \\[0pt]
Les mondes de l'art \\[0pt]
Les mondes possibles \\[0pt]
Les monstres \\[0pt]
Les morts \\[0pt]
Les mots disent-ils les choses ? \\[0pt]
Les mots et la signification \\[0pt]
Les mots et les choses \\[0pt]
Les mots et les concepts \\[0pt]
Les mots expriment-ils les choses ? \\[0pt]
Les mots justes \\[0pt]
Les mots n'ont-ils que le pouvoir qu'on leur donne ? \\[0pt]
Les mots nous éloignent-ils des autres ? \\[0pt]
Les mots nous éloignent-ils des choses ? \\[0pt]
Les mots ont-ils un pouvoir ? \\[0pt]
Les mots parviennent-ils à tout exprimer ? \\[0pt]
Les mots peuvent-ils nous manquer ? \\[0pt]
Les mots rendent-ils compte de la nature des choses ? \\[0pt]
Les mots sont-ils trompeurs ? \\[0pt]
Les moyens de l'art \\[0pt]
Les moyens de l'autorité \\[0pt]
Les moyens et la fin \\[0pt]
Les moyens et les fins \\[0pt]
Les moyens et les fins en art \\[0pt]
Les muses \\[0pt]
Les mythes collectifs \\[0pt]
Les mythes relèvent-ils de l'anthropologie ? \\[0pt]
Les nombres gouvernent-ils le monde ? \\[0pt]
Les noms \\[0pt]
Les noms divins \\[0pt]
Les noms propres \\[0pt]
Les noms propres ont-ils une signification ? \\[0pt]
Les normes \\[0pt]
Les normes du savoir sont-elles universelles ? \\[0pt]
Les normes du vivant \\[0pt]
Les normes esthétiques \\[0pt]
Les normes et les valeurs \\[0pt]
Les nouvelles technologies transforment-elles l'idée de l'art ? \\[0pt]
Les objets de la pensée \\[0pt]
Les objets de pensée \\[0pt]
Les objets du désir \\[0pt]
Les objets existent-ils ? \\[0pt]
Les objets impossibles \\[0pt]
Les objets scientifiques \\[0pt]
Les objets sont-ils colorés ? \\[0pt]
Les objets techniques nous imposent-ils une manière de vivre ? \\[0pt]
Le social et le politique \\[0pt]
Le sociologisme \\[0pt]
Le sociologue et l'historien \\[0pt]
Les œuvres d'art doivent-elles faire l'objet d'une évaluation morale ? \\[0pt]
Les œuvres d'art ont-elles besoin d'un commentaire ? \\[0pt]
Les œuvres d'art sont-elles des choses ? \\[0pt]
Les œuvres d'art sont-elles des réalités comme les autres ? \\[0pt]
Les œuvres d'art sont-elles éternelles ? \\[0pt]
Le soi est-il inaccessible ? \\[0pt]
Le soi et le je \\[0pt]
Le soin \\[0pt]
Le soldat \\[0pt]
Le soleil se lèvera demain \\[0pt]
Le soleil se lèvera-t-il demain ? \\[0pt]
Le solipsisme \\[0pt]
Le sommeil \\[0pt]
Le sommeil de la raison \\[0pt]
Le sommeil et la veille \\[0pt]
Les opérations de la pensée \\[0pt]
Les opérations de l'esprit \\[0pt]
Le sophiste et le philosophe \\[0pt]
Les opinions politiques \\[0pt]
Les origines de l'humanité \\[0pt]
L'ésotérisme \\[0pt]
Le souci \\[0pt]
Le souci d'autrui résume-t-il la morale ? \\[0pt]
Le souci de l'avenir \\[0pt]
Le souci de l'ordre \\[0pt]
Le souci de soi \\[0pt]
Le souci de soi est-il une attitude morale ? \\[0pt]
Le souci du bien-être est-il politique ? \\[0pt]
Le soupçon \\[0pt]
Les outils \\[0pt]
Le souvenir \\[0pt]
Le souverain bien \\[0pt]
L'espace \\[0pt]
L'espace dans la peinture \\[0pt]
L'espace de la perception \\[0pt]
L'espace du tableau \\[0pt]
L'espace et le lieu \\[0pt]
L'espace et le territoire \\[0pt]
L'espace intérieur \\[0pt]
L'espace nous sépare-t-il ? \\[0pt]
L'espace public \\[0pt]
L'espace social \\[0pt]
Les paroles et les actes \\[0pt]
« Les paroles s'envolent, les écrits restent » \\[0pt]
Les parties de l'âme \\[0pt]
Les partis politiques \\[0pt]
Les passions ont-elles une place en politique ? \\[0pt]
Les passions peuvent-elles être raisonnables ? \\[0pt]
Les passions politiques \\[0pt]
Les passions sont-elles mauvaises ? \\[0pt]
Les passions sont-elles toujours mauvaises ? \\[0pt]
Les passions sont-elles toutes bonnes ? \\[0pt]
Les passions sont-elles un obstacle à la vie sociale ? \\[0pt]
Les passions s'opposent-elles à la raison ? \\[0pt]
Les pauvres \\[0pt]
L'espèce et l'individu \\[0pt]
L'espèce humaine \\[0pt]
Le spectacle \\[0pt]
Le spectacle de la nature \\[0pt]
Le spectacle de la pensée \\[0pt]
Le spectacle du monde \\[0pt]
Le spectateur \\[0pt]
L'espérance \\[0pt]
L'espérance est-elle une mauvaise passion ? \\[0pt]
L'espérance est-elle une vertu ? \\[0pt]
Les perceptions sont-elles des jugements ? \\[0pt]
Les personnages de fiction peuvent-ils avoir une réalité ? \\[0pt]
Les personnes et les choses \\[0pt]
Les peuples font-ils l'histoire ? \\[0pt]
Les peuples ont-ils les gouvernements qu'ils méritent ? \\[0pt]
Les phénomènes \\[0pt]
Les phénomènes de mode \\[0pt]
Les phénomènes inconscients sont-ils réductibles à une mécanique cérébrale ? \\[0pt]
Les philosophes doivent-ils être rois ? \\[0pt]
Les philosophies se classent-elles ? \\[0pt]
Le spirituel et le temporel \\[0pt]
Les plaisirs \\[0pt]
Les plaisirs de l'amitié \\[0pt]
Les poètes et la cité \\[0pt]
L'espoir \\[0pt]
L'espoir et la crainte \\[0pt]
L'espoir peut-il être raisonnable ? \\[0pt]
Les politiques publiques \\[0pt]
Le sport \\[0pt]
Le sport : s'accomplir ou se dépasser ? \\[0pt]
Les possibles \\[0pt]
Les pouvoirs de la parole \\[0pt]
Les pouvoirs de la religion \\[0pt]
Les préjugés moraux \\[0pt]
Les prêtres \\[0pt]
Les preuves de la liberté \\[0pt]
Les preuves de l'existence de Dieu \\[0pt]
Les principes \\[0pt]
Les principes de la démonstration \\[0pt]
Les principes de la morale dépendent-ils de la culture ? \\[0pt]
Les principes d'une science sont-ils des conventions ? \\[0pt]
Les principes et les éléments \\[0pt]
Les principes moraux \\[0pt]
Les principes sont-ils indémontrables ? \\[0pt]
L'esprit \\[0pt]
L'esprit appartient-il à la nature ? \\[0pt]
L'esprit critique \\[0pt]
L'esprit de corps \\[0pt]
L'esprit de finesse \\[0pt]
L'esprit dépend-il du corps ? \\[0pt]
L'esprit de sérieux \\[0pt]
L'esprit des lois \\[0pt]
L'esprit de système \\[0pt]
L'esprit d'invention \\[0pt]
L'esprit domine-t-il la matière ? \\[0pt]
L'esprit du christianisme \\[0pt]
L'esprit est-il chose ? \\[0pt]
L'esprit est-il irréductible à la matière ? \\[0pt]
L'esprit est-il libre ? \\[0pt]
L'esprit est-il matériel ? \\[0pt]
L'esprit est-il mieux connu que le corps ? \\[0pt]
L'esprit est-il objet de science ? \\[0pt]
L'esprit est-il plus aisé à connaître que le corps ? \\[0pt]
L'esprit est-il plus difficile à connaître que la matière ? \\[0pt]
L'esprit est-il plus facile à connaître que le corps ? \\[0pt]
L'esprit est-il réductible à la matière ? \\[0pt]
L'esprit est-il une chose ? \\[0pt]
L'esprit est-il une machine ? \\[0pt]
L'esprit est-il un ensemble de facultés ? \\[0pt]
L'esprit est-il une partie du corps ? \\[0pt]
L'esprit et la lettre \\[0pt]
L'esprit et la machine \\[0pt]
L'esprit et la matière peuvent-ils agir l'un sur l'autre ? \\[0pt]
L'esprit et le cerveau \\[0pt]
L'esprit humain progresse-t-il ? \\[0pt]
L'esprit n'a-t-il jamais affaire qu'à lui-même ? \\[0pt]
L'esprit peut-il être divisé ? \\[0pt]
L'esprit peut-il être malade ? \\[0pt]
L'esprit peut-il être mesuré ? \\[0pt]
L'esprit peut-il être objet de science ? \\[0pt]
L'esprit scientifique \\[0pt]
L'esprit s'explique-t-il par une activité cérébrale ? \\[0pt]
L'esprit tranquille \\[0pt]
Les problèmes politiques peuvent-ils se ramener à des problèmes techniques ? \\[0pt]
Les problèmes politiques sont-ils des problèmes techniques ? \\[0pt]
Les procédures de validation dans les sciences humaines ? \\[0pt]
Les progrès de la technique sont-ils nécessairement des progrès de la raison ? \\[0pt]
Les progrès techniques constituent-ils des progrès de la civilisation ? \\[0pt]
Les propositions métaphysiques sont-elles des illusions ? \\[0pt]
Les proverbes \\[0pt]
Les proverbes enseignent-ils quelque chose ? \\[0pt]
Les proverbes nous instruisent-ils moralement ? \\[0pt]
Les qualités esthétiques \\[0pt]
Les qualités sensibles sont-elles dans les choses ou dans l'esprit ? \\[0pt]
Les querelles de mots sont-elles toujours stériles ? \\[0pt]
Les questions métaphysiques ont-elles un sens ? \\[0pt]
L'esquisse \\[0pt]
Les raisons d'aimer \\[0pt]
Les raisons de croire \\[0pt]
Les raisons d'espérer \\[0pt]
Les raisons de vivre \\[0pt]
Les raisons du choix \\[0pt]
Les rapports entre les hommes sont-ils des rapports de force ? \\[0pt]
Les rapports sociaux sont-ils des rapports entre individus ? \\[0pt]
Les règles de l'art \\[0pt]
Les règles de l'interprétation \\[0pt]
Les règles du jeu \\[0pt]
Les règles du langage limitent-elles notre liberté d'expression ? \\[0pt]
Les règles du langage sont- elles un obstacle à la communication ? \\[0pt]
Les règles d'un bon gouvernement \\[0pt]
Les règles sociales \\[0pt]
Les règles sociales sont-elles répressives ? \\[0pt]
Les relations \\[0pt]
Les relations ont-elles une existence objective ? \\[0pt]
Les relations sociales \\[0pt]
Les religions naissent-elles du besoin de justice ? \\[0pt]
Les religions peuvent-elles être objets de science ? \\[0pt]
Les religions peuvent-elles prétendre libérer les hommes ? \\[0pt]
Les religions sont-elles affaire de foi ? \\[0pt]
Les religions sont-elles des illusions ? \\[0pt]
Les remords \\[0pt]
Les représentants du peuple \\[0pt]
Les représentations collectives \\[0pt]
Les reproductions \\[0pt]
Les réseaux sociaux \\[0pt]
Les ressources de l'analogie \\[0pt]
Les ressources humaines \\[0pt]
Les ressources naturelles \\[0pt]
Les rêves ont-ils un sens ? \\[0pt]
Les rêves sont-ils des pensées ? \\[0pt]
Les révolutions scientifiques \\[0pt]
Les révolutions techniques suscitent-elles des révolutions dans l'art ? \\[0pt]
Les riches et les pauvres \\[0pt]
Les rites \\[0pt]
Les rites funéraires \\[0pt]
Les rituels \\[0pt]
Les robots \\[0pt]
Les rôles sociaux \\[0pt]
Les ruines \\[0pt]
Les sacrifices \\[0pt]
Les sans-papiers \\[0pt]
Les sauvages \\[0pt]
Les scélérats peuvent-ils être heureux ? \\[0pt]
Les sciences appliquées \\[0pt]
Les sciences décrivent-elles le réel ? \\[0pt]
Les sciences décrivent-elles ou construisent-elles leurs objets ? \\[0pt]
Les sciences de la nature \\[0pt]
Les sciences de la vie et de la Terre \\[0pt]
Les sciences de la vie visent-elles un objet irréductible à la matière ? \\[0pt]
Les sciences de l'éducation \\[0pt]
Les sciences de l'esprit \\[0pt]
Les sciences de l'homme et l'évolution \\[0pt]
Les sciences de l'homme ont-elles inventé leur objet ? \\[0pt]
Les sciences de l'homme permettent-elles d'affiner la notion de responsabilité ? \\[0pt]
Les sciences de l'homme peuvent-elles expliquer l'impuissance de la liberté ? \\[0pt]
Les sciences de l'homme rendent-elles l'homme prévisible ? \\[0pt]
Les sciences de l'homme sont-elles des sciences ? \\[0pt]
Les sciences de l'homme sont-elles utiles à la société ? \\[0pt]
Les sciences doivent-elle prétendre à l'unification ? \\[0pt]
Les sciences du comportement \\[0pt]
Les sciences et le vivant \\[0pt]
Les sciences exactes \\[0pt]
Les sciences forment-elle un système ? \\[0pt]
Les sciences historiques \\[0pt]
Les sciences humaines doivent-elles être transdisciplinaires ? \\[0pt]
Les sciences humaines échappent-elles à l'abstraction ? \\[0pt]
Les sciences humaines éliminent-elles la contingence du futur ? \\[0pt]
Les sciences humaines, est-ce la mort de l'homme ? \\[0pt]
Les sciences humaines et le droit \\[0pt]
Les sciences humaines et l'expérience \\[0pt]
Les sciences humaines et sociales n'ont-elles de sciences que le nom ? \\[0pt]
Les sciences humaines finissent-elles de ruiner l'idée de liberté ? \\[0pt]
Les sciences humaines jouent elles un rôle dans l'exploitation de l'homme par l'homme ? \\[0pt]
Les sciences humaines nient-elles la liberté de l'homme ? \\[0pt]
Les sciences humaines nient-elles la liberté du sujet humain ? \\[0pt]
Les sciences humaines nous font-elles connaître l'homme ? \\[0pt]
Les sciences humaines nous obligent-elles à renoncer à l'idée de libre arbitre ? \\[0pt]
Les sciences humaines nous protègent-elles de l'essentialisme ? \\[0pt]
Les sciences humaines ont-elles une utilité ? \\[0pt]
Les sciences humaines ont-elles un objet commun ? \\[0pt]
Les sciences humaines opposent-elles histoire et système ? \\[0pt]
Les sciences humaines permettent-elles de comprendre la vie d'un homme ? \\[0pt]
Les sciences humaines peuvent-elles adopter les méthodes des sciences de la nature ? \\[0pt]
Les sciences humaines peuvent-elles éclairer la réflexion éthique ? \\[0pt]
Les sciences humaines peuvent-elles être objectives ? \\[0pt]
Les sciences humaines peuvent-elles faire abstraction de la nature ? \\[0pt]
Les sciences humaines peuvent-elles s'affranchir de l'ethnocentrisme ? \\[0pt]
Les sciences humaines peuvent-elles se passer de la notion d'inconscient ? \\[0pt]
Les sciences humaines présupposent-elles une définition de l'homme ? \\[0pt]
Les sciences humaines recherchent-elles des invariants ? \\[0pt]
Les sciences humaines rendent-elle l'homme prévisible ? \\[0pt]
Les sciences humaines reposent-elles sur des faits ? \\[0pt]
Les sciences humaines sont-elles des sciences ? \\[0pt]
Les sciences humaines sont-elles des sciences de la nature humaine ? \\[0pt]
Les sciences humaines sont-elles des sciences de la vie humaine ? \\[0pt]
Les sciences humaines sont-elles des sciences de l'esprit ? \\[0pt]
Les sciences humaines sont-elles des sciences de l'interprétation ? \\[0pt]
Les sciences humaines sont-elles des sciences d'interprétation ? \\[0pt]
Les sciences humaines sont-elles explicatives ou compréhensives ? \\[0pt]
Les sciences humaines sont-elles les sciences de l'esprit ? \\[0pt]
Les sciences humaines sont-elles nécessairement empiriques ? \\[0pt]
Les sciences humaines sont-elles normatives ? \\[0pt]
Les sciences humaines sont-elles relativistes ? \\[0pt]
Les sciences humaines sont-elles subversives ? \\[0pt]
Les sciences humaines suffisent-elles à connaître l'homme ? \\[0pt]
Les sciences humaines supposent-elles une définition de l'homme ? \\[0pt]
Les sciences humaines supposent-elles une rupture avec l'ordre naturel ? \\[0pt]
Les sciences humaines traitent-elles de l'homme ? \\[0pt]
Les sciences humaines traitent-elles de l'individu ? \\[0pt]
Les sciences humaines transforment-elles la notion de causalité ? \\[0pt]
Les sciences naturelles \\[0pt]
Les sciences ne sont-elles qu'une description du monde ? \\[0pt]
Les sciences nous donnent-elles des normes ? \\[0pt]
Les sciences ont-elles à penser leurs fondements ? \\[0pt]
Les sciences ont-elles besoin de principes fondamentaux ? \\[0pt]
Les sciences ont-elles besoin d'une fondation ? \\[0pt]
Les sciences ont-elles besoin d'une fondation métaphysique ? \\[0pt]
Les sciences ont-elles le monopole de la vérité ? \\[0pt]
Les sciences pensent-elles leurs fondements ? \\[0pt]
Les sciences permettent-elles de connaître la réalité-même ? \\[0pt]
Les sciences peuvent-elles exclure toute notion de finalité ? \\[0pt]
Les sciences peuvent-elles penser leurs fondements ? \\[0pt]
Les sciences peuvent-elles penser l'individu ? \\[0pt]
Les sciences peuvent-elles se passer de fondements métaphysiques ? \\[0pt]
Les sciences peuvent-elles se passer d'images ? \\[0pt]
Les sciences produisent-elles des normes ? \\[0pt]
Les sciences sociales \\[0pt]
Les sciences sociales ont-elles un objet ? \\[0pt]
Les sciences sociales peuvent-elles être expérimentales ? \\[0pt]
Les sciences sociales peuvent-elles servir à gouverner ? \\[0pt]
Les sciences sociales sont-elles nécessairement inexactes ? \\[0pt]
Les sciences sont-elles une description du monde ? \\[0pt]
Les secrets \\[0pt]
L'essence \\[0pt]
L'essence de la technique \\[0pt]
L'essence et l'existence \\[0pt]
Les sensations peuvent-elles se partager ? \\[0pt]
Les sens humains \\[0pt]
Les sens jugent-ils ? \\[0pt]
Les sens nous trompent-ils ? \\[0pt]
Les sens peuvent-ils nous tromper ? \\[0pt]
Les sens sont-ils source d'illusion ? \\[0pt]
Les sens sont-ils trompeurs ? \\[0pt]
L'essentiel \\[0pt]
Les sentiments \\[0pt]
Les sentiments ont-ils une histoire ? \\[0pt]
Les sentiments peuvent-ils s'apprendre ? \\[0pt]
Les services publics \\[0pt]
Les signes \\[0pt]
Les signes de l'humanité \\[0pt]
Les signes de l'intelligence \\[0pt]
Les sociétés animales \\[0pt]
Les sociétés évoluent-elles ? \\[0pt]
Les sociétés humaines se construisent-elles contre la nature ? \\[0pt]
Les sociétés ont-elles besoin de théâtre ? \\[0pt]
Les sociétés ont-elles un inconscient ? \\[0pt]
Les sociétés sans histoire \\[0pt]
Les sociétés savantes \\[0pt]
Les sociétés sont-elles hiérarchisables ? \\[0pt]
Les sociétés sont-elles imprévisibles ? \\[0pt]
Les statistiques comme outil dans les sciences humaines \\[0pt]
Les structures expliquent-elles tout ? \\[0pt]
Les structures que dégagent les sciences humaines agissent-elles ? \\[0pt]
Les structures sociales sont-elles économiques ? \\[0pt]
Les styles \\[0pt]
Les substances \\[0pt]
Les symboles du pouvoir \\[0pt]
Les systèmes \\[0pt]
Les tabous \\[0pt]
Le statut de l'axiome \\[0pt]
Le statut des hypothèses dans la démarche scientifique \\[0pt]
Les techniques artistiques \\[0pt]
Les techniques de manipulation des hommes peuvent-elles s'appuyer sur les sciences humaines ? \\[0pt]
Les techniques du corps \\[0pt]
Les témoignages et la preuve \\[0pt]
Les théories scientifiques décrivent-elles la réalité ? \\[0pt]
Les théories scientifiques sont-elles vraies ? \\[0pt]
L'esthète \\[0pt]
L'esthète et l'artiste \\[0pt]
L'esthétique \\[0pt]
L'esthétique des ruines \\[0pt]
L'esthétique est-elle une métaphysique de l'art ? \\[0pt]
L'esthétisme \\[0pt]
L'estime \\[0pt]
L'estime de soi \\[0pt]
Les traditions \\[0pt]
Les transcendantaux \\[0pt]
Le style \\[0pt]
Le style et le beau \\[0pt]
Le sublime \\[0pt]
Le sublime est-il dans les choses ? \\[0pt]
Le substitut \\[0pt]
Le substrat \\[0pt]
Le succès \\[0pt]
Le succès peut-il être un critère de vérité ? \\[0pt]
Le suffrage universel \\[0pt]
Le suicide \\[0pt]
Le sujet \\[0pt]
Le sujet de droit \\[0pt]
Le sujet de l'action \\[0pt]
Le sujet de la pensée \\[0pt]
Le sujet de l'histoire \\[0pt]
Le sujet du droit \\[0pt]
Le sujet et la personne \\[0pt]
Le sujet et l'individu \\[0pt]
Le sujet et l'objet \\[0pt]
Le sujet moral \\[0pt]
Le sujet n'est-il qu'une fiction ? \\[0pt]
Le sujet peut-il s'aliéner par un libre choix ? \\[0pt]
Les universaux \\[0pt]
Les universaux existent-ils ? \\[0pt]
Le superflu \\[0pt]
Le surnaturel \\[0pt]
Les usages \\[0pt]
Les usages de l'art \\[0pt]
Les valeurs \\[0pt]
Les valeurs de la République \\[0pt]
Les valeurs morales ont-elles leur origine dans la raison ? \\[0pt]
Les valeurs ont-elles une histoire ? \\[0pt]
Les valeurs ont-elles une origine ? \\[0pt]
Les valeurs universelles \\[0pt]
Les vérités empiriques \\[0pt]
Les vérités éternelles \\[0pt]
Les vérités scientifiques sont-elles relatives ? \\[0pt]
Les vérités sont-elles intemporelles ? \\[0pt]
Les vérités sont-elles toujours démontrables ? \\[0pt]
Les vertus \\[0pt]
Les vertus cardinales \\[0pt]
Les vertus de la formalisation \\[0pt]
Les vertus de l'amour \\[0pt]
Les vertus du commerce \\[0pt]
Les vertus ne sont-elles que des vices déguisés ? \\[0pt]
Les vertus politiques \\[0pt]
Les vertus sont-elles au principe de la morale ? \\[0pt]
Les vices de l'esprit \\[0pt]
Les vices privés peuvent-ils faire le bien public ? \\[0pt]
Les visages du mal \\[0pt]
Les vivants \\[0pt]
Les vivants et les morts \\[0pt]
Les vivants peuvent-ils se passer des morts ? \\[0pt]
Le syllogisme \\[0pt]
Le symbole \\[0pt]
Le symbole, comme concept, dans les sciences humaines \\[0pt]
Le symbolique \\[0pt]
Le symbolisme \\[0pt]
Le symbolisme mathématique \\[0pt]
Le système \\[0pt]
Le système des arts \\[0pt]
Le système des beaux-arts \\[0pt]
Le système des besoins \\[0pt]
Le tableau \\[0pt]
Le tableau ? \\[0pt]
Le tableau vivant \\[0pt]
Le tacite \\[0pt]
Le tact \\[0pt]
Le talent \\[0pt]
Le talent et le génie \\[0pt]
Le talent s'enseigne-t-il ? \\[0pt]
L'étant est-il le premier pensable ? \\[0pt]
Le tas et le tout \\[0pt]
L'État \\[0pt]
L'État a-t-il des intérêts propres ? \\[0pt]
L'État a-t-il le droit de contrôler notre habillement ? \\[0pt]
L'État a-t-il pour but de maintenir l'ordre ? \\[0pt]
L'État a-t-il pour finalité de maintenir l'ordre ? \\[0pt]
L'État a-t-il tous les droits ? \\[0pt]
L'État a-t-il un fondement rationnel ? \\[0pt]
« L'État, c'est moi » \\[0pt]
L'État contribue-t-il à pacifier les relations entre les hommes ? \\[0pt]
L'État crée-t-il la liberté ? \\[0pt]
L'État de droit \\[0pt]
L'État de droit ? \\[0pt]
L'état de guerre \\[0pt]
L'état de la nature \\[0pt]
L'état de nature \\[0pt]
L'état d'exception \\[0pt]
L'État doit-il disparaître ? \\[0pt]
L'État doit-il éduquer le citoyen ? \\[0pt]
L'État doit-il éduquer le peuple ? \\[0pt]
L'État doit-il éduquer les citoyens ? \\[0pt]
L'État doit-il être fort ? \\[0pt]
L'État doit-il être le plus fort ? \\[0pt]
L'État doit-il être neutre ? \\[0pt]
L'État doit-il être sans pitié ? \\[0pt]
L'État doit-il faire le bonheur des citoyens ? \\[0pt]
L'État doit-il nous rendre meilleurs ? \\[0pt]
L'État doit-il préférer l'injustice au désordre ? \\[0pt]
L'État doit-il reconnaître des limites à sa puissance ? \\[0pt]
L'État doit-il se mêler de religion ? \\[0pt]
L'État doit-il se préoccuper des arts ? \\[0pt]
L'État doit-il se préoccuper du bonheur des citoyens ? \\[0pt]
L'État doit-il se soucier de la morale ? \\[0pt]
L'État doit-il veiller au bonheur des individus ? \\[0pt]
L'État est-il appelé à disparaître ? \\[0pt]
L'État est-il au service de la société ? \\[0pt]
L'État est-il ennemi de la liberté ? \\[0pt]
L'État est-il fin ou moyen ? \\[0pt]
L'État est-il la seule source du droit ? \\[0pt]
L'État est-il le garant de la propriété privée ? \\[0pt]
L'État est-il le garant du bien commun ? \\[0pt]
L'État est-il l'ennemi de la liberté ? \\[0pt]
L'État est-il l'ennemi de l'individu ? \\[0pt]
L'État est-il libérateur ? \\[0pt]
L'État est-il nécessaire ? \\[0pt]
L'État est-il notre ennemi ? \\[0pt]
L'État est-il oppresseur ? \\[0pt]
L'État est-il souverain ? \\[0pt]
L'État est-il toujours juste ? \\[0pt]
L'État est-il un arbitre ? \\[0pt]
L'État est-il un mal nécessaire ? \\[0pt]
L'État est-il un moindre mal ? \\[0pt]
L'État est-il un « monstre froid » ? \\[0pt]
L'État est-il un tiers impartial ? \\[0pt]
L'État et la culture \\[0pt]
L'État et la guerre \\[0pt]
L'État et la justice \\[0pt]
L'État et la nation \\[0pt]
L'État et la Nation \\[0pt]
L'État et la protection \\[0pt]
L'État et la société \\[0pt]
L'État et la violence \\[0pt]
L'État et le droit \\[0pt]
L'État et le marché \\[0pt]
L'État et le peuple \\[0pt]
L'État et les communautés \\[0pt]
L'État et les Églises \\[0pt]
L'État et l'individu \\[0pt]
L'État libéral \\[0pt]
L'État mondial \\[0pt]
L'État n'a-t-il d'existence que contractuelle ? \\[0pt]
L'État n'est-il qu'un instrument de domination ? \\[0pt]
L'État n'est-il qu'un moyen ? \\[0pt]
L'État nous rend-il meilleurs ? \\[0pt]
L'État peut-il avoir tous les droits ? \\[0pt]
L'État peut-il créer la liberté ? \\[0pt]
L'État peut-il demeurer indifférent à la religion ? \\[0pt]
L'État peut-il disparaître ? \\[0pt]
L'État peut-il être fondé sur un contrat ? \\[0pt]
L'État peut-il être impartial ? \\[0pt]
L'État peut-il être indifférent à la religion ? \\[0pt]
L'État peut-il être libéral ? \\[0pt]
L'État peut-il fonder sa propre légitimité ? \\[0pt]
L'État peut-il limiter son pouvoir ? \\[0pt]
L'État peut-il poursuivre une autre fin que sa propre puissance ? \\[0pt]
L'État peut-il renoncer à la violence ? \\[0pt]
L'État peut-il se désintéresser de la religion ? \\[0pt]
L'État-Providence \\[0pt]
L'État sert-il l'intérêt général ? \\[0pt]
L'État universel \\[0pt]
Le technicien n'est-il qu'un exécutant ? \\[0pt]
Le témoignage \\[0pt]
Le témoignage des sens \\[0pt]
Le témoin \\[0pt]
Le temporel et le spirituel \\[0pt]
Le temps \\[0pt]
Le temps de la liberté \\[0pt]
Le temps de la mémoire \\[0pt]
Le temps de la réflexion \\[0pt]
Le temps de l'art \\[0pt]
Le temps de la science \\[0pt]
Le temps de la vie et le temps de l'existence \\[0pt]
Le temps de l'histoire \\[0pt]
Le temps dépend-il de la mémoire ? \\[0pt]
Le temps des origines \\[0pt]
Le temps détruit-il tout ? \\[0pt]
Le temps de vivre \\[0pt]
Le temps du bonheur \\[0pt]
Le temps du désir \\[0pt]
Le temps du monde \\[0pt]
Le temps du monde et le temps de la conscience \\[0pt]
Le temps est-il destructeur ? \\[0pt]
Le temps est-il en nous ou hors de nous ? \\[0pt]
Le temps est-il essentiellement destructeur ? \\[0pt]
Le temps est-il la marque de notre impuissance ? \\[0pt]
Le temps est-il notre allié ? \\[0pt]
Le temps est-il notre ennemi ? \\[0pt]
Le temps est-il une contrainte ? \\[0pt]
Le temps est-il une dimension de la nature ? \\[0pt]
Le temps est-il une expérience de la continuité ou de la discontinuité ? \\[0pt]
Le temps est-il une prison ? \\[0pt]
Le temps est-il une réalité ? \\[0pt]
Le temps et la liberté \\[0pt]
Le temps et le bonheur \\[0pt]
Le temps et l'espace \\[0pt]
Le temps et l'éternité \\[0pt]
Le temps existe-t-il ? \\[0pt]
Le temps libre \\[0pt]
Le temps ne fait-il que passer ? \\[0pt]
Le temps n'est-il pour l'homme que ce qui le limite ? \\[0pt]
Le temps n'existe-t-il que subjectivement ? \\[0pt]
Le temps nous appartient-il ? \\[0pt]
Le temps nous est-il compté ? \\[0pt]
Le temps passe-t-il ? \\[0pt]
Le temps perdu \\[0pt]
Le temps peut-il s'arrêter ? \\[0pt]
Le temps physique est-il comparable au temps psychique ? \\[0pt]
Le temps politique \\[0pt]
Le temps rend-il tout vain ? \\[0pt]
Le temps s'écoule-t-il ? \\[0pt]
Le temps se laisse-t-il décrire logiquement ? \\[0pt]
Le temps se mesure-t-il ? \\[0pt]
L'éternel présent \\[0pt]
L'éternel retour \\[0pt]
L'éternité \\[0pt]
L'éternité des œuvres d'art \\[0pt]
L'éternité est-elle désirable ? \\[0pt]
L'éternité n'est-elle qu'une illusion ? \\[0pt]
Le « terrain » \\[0pt]
Le terrain \\[0pt]
Le territoire \\[0pt]
Le terrorisme est-il un acte de guerre ? \\[0pt]
Le théâtral \\[0pt]
Le théâtre de l'histoire \\[0pt]
Le théâtre du monde \\[0pt]
Le théâtre et les mœurs \\[0pt]
Le théâtre et l'existence \\[0pt]
L'éthique à l'épreuve du tragique \\[0pt]
L'éthique des plaisirs \\[0pt]
L'éthique du spectateur \\[0pt]
L'éthique est-elle affaire de choix ? \\[0pt]
L'éthique suppose-t-elle la liberté ? \\[0pt]
L'ethnocentrisme \\[0pt]
L'ethnocentrisme comme problème pour l'ethnologie \\[0pt]
L'ethnologie est-elle une science mélancolique ? \\[0pt]
Le tiers exclu \\[0pt]
L'étonnement \\[0pt]
Le totalitarisme \\[0pt]
Le totémisme \\[0pt]
Le toucher \\[0pt]
Le tourment moral \\[0pt]
Le tout est-il la somme de ses parties ? \\[0pt]
Le tout et la partie \\[0pt]
Le tout et les parties \\[0pt]
Le tragique \\[0pt]
Le tragique et le comique \\[0pt]
Le trait d'esprit \\[0pt]
L'étranger \\[0pt]
L'étrangeté \\[0pt]
Le transcendantal \\[0pt]
Le travail \\[0pt]
Le travail artistique \\[0pt]
Le travail artistique doit-il demeurer caché ? \\[0pt]
Le travail a-t-il une valeur morale ? \\[0pt]
Le travail bien fait \\[0pt]
Le travail de la pensée \\[0pt]
Le travail de la raison \\[0pt]
Le travail du droit \\[0pt]
Le travail du négatif \\[0pt]
Le travail est-il le propre de l'homme ? \\[0pt]
Le travail est-il libérateur ? \\[0pt]
Le travail est-il nécessaire au bonheur ? \\[0pt]
Le travail est-il toujours une activité productrice ? \\[0pt]
Le travail est-il un besoin ? \\[0pt]
Le travail est-il une fin ? \\[0pt]
Le travail est-il une marchandise ? \\[0pt]
Le travail est-il une valeur ? \\[0pt]
Le travail est-il une valeur morale ? \\[0pt]
Le travail est-il un rapport naturel de l'homme à la nature ? \\[0pt]
Le travail et la propriété \\[0pt]
Le travail et la technique \\[0pt]
Le travail et le bonheur \\[0pt]
Le travail et le jeu \\[0pt]
Le travail et le labeur \\[0pt]
Le travail et le loisir \\[0pt]
Le travail et le temps \\[0pt]
Le travail et l'œuvre \\[0pt]
Le travail fait-il de l'homme un être moral ? \\[0pt]
Le travail fonde-t-il la propriété ? \\[0pt]
Le travail intellectuel \\[0pt]
Le travaille libère-t-il ? \\[0pt]
Le travail manuel \\[0pt]
Le travail manuel est-il sans pensée ? \\[0pt]
Le travail nous libère-t-il ? \\[0pt]
Le travail nous rend-il heureux ? \\[0pt]
Le travail nous rend-il solidaires ? \\[0pt]
Le travail peut-il être solitaire ? \\[0pt]
Le travail rapproche-t-il les hommes ? \\[0pt]
« Le travail rend libre » \\[0pt]
Le travail sur le terrain \\[0pt]
Le travail sur soi \\[0pt]
Le travail unit-il ou sépare-t-il les hommes ? \\[0pt]
L'être de la conscience \\[0pt]
L'être de la vérité \\[0pt]
L'être de l'image \\[0pt]
L'être du possible \\[0pt]
L'être en tant qu'être \\[0pt]
L'être en tant qu'être est-il connaissable ? \\[0pt]
L'être est-il un genre ? \\[0pt]
L'être et l'apparence \\[0pt]
L'être et la relation \\[0pt]
L'être et la volonté \\[0pt]
L'être et le bien \\[0pt]
L'être et le devenir \\[0pt]
L'être et le devoir-être \\[0pt]
L'être et le néant \\[0pt]
L'être et les êtres \\[0pt]
L'être et l'esprit \\[0pt]
L'être et l'essence \\[0pt]
L'être et l'étant \\[0pt]
L'être et le temps \\[0pt]
L'être humain désire-t-il naturellement connaître ? \\[0pt]
L'être humain est-il la mesure de toute chose ? \\[0pt]
L'être humain est-il par nature un être religieux \\[0pt]
L'être imaginaire et l'être de raison \\[0pt]
L'être intelligent \\[0pt]
L'être se confond-il avec l'être perçu ? \\[0pt]
L'Être suprême \\[0pt]
Le tribunal de la raison \\[0pt]
Le tribunal de l'histoire \\[0pt]
Le troc \\[0pt]
Le trompe-l'œil \\[0pt]
L'étude \\[0pt]
L'étude de l'histoire conduit-elle à désespérer l'homme ? \\[0pt]
Le tyran \\[0pt]
L'eugénisme \\[0pt]
L'Europe \\[0pt]
L'euthanasie \\[0pt]
Le vainqueur a-t-il tous les droits ? \\[0pt]
L'évaluation \\[0pt]
L'évaluation de l'action politique \\[0pt]
L'évaluation des œuvres d'art \\[0pt]
L'évasion \\[0pt]
Le vécu \\[0pt]
Le vécu et la vérité \\[0pt]
L'événement \\[0pt]
L'événement et le fait divers \\[0pt]
L'événement historique a-t-il un sens par lui-même ? \\[0pt]
L'événement manque-t-il d'être ? \\[0pt]
Le verbalisme \\[0pt]
Le verbe \\[0pt]
Le vertige \\[0pt]
Le vertige de la liberté \\[0pt]
Le vestige \\[0pt]
Le vêtement \\[0pt]
Le vice \\[0pt]
Le vice et la vertu \\[0pt]
Le vide \\[0pt]
Le vide et le plein \\[0pt]
L'évidence \\[0pt]
L'évidence a-t-elle une valeur absolue ? \\[0pt]
L'évidence est-elle critère de vérité ? \\[0pt]
L'évidence est-elle le signe de la vérité ? \\[0pt]
L'évidence est-elle toujours un critère de vérité ? \\[0pt]
L'évidence est-elle un critère de vérité ? \\[0pt]
L'évidence est-elle un obstacle ou un instrument de la recherche de la vérité ? \\[0pt]
L'évidence et la démonstration \\[0pt]
L'évidence se passe-t-elle de démonstration ? \\[0pt]
Le village global \\[0pt]
Le virtuel \\[0pt]
Le virtuel existe-t-il ? \\[0pt]
Le visage \\[0pt]
Le visage n'est-il qu'un masque ? \\[0pt]
Le visible et l'invisible \\[0pt]
Le vivant \\[0pt]
Le vivant a-t-il des droits ? \\[0pt]
Le vivant a-t-il une temporalité propre ? \\[0pt]
Le vivant comme problème pour la philosophie des sciences \\[0pt]
Le vivant définit-il ses propres normes ? \\[0pt]
Le vivant échappe-t-il à la connaissance ? \\[0pt]
Le vivant échappe-t-il au déterminisme ? \\[0pt]
Le vivant est-il entièrement connaissable ? \\[0pt]
Le vivant est-il entièrement explicable ? \\[0pt]
Le vivant est-il l'affaire du politique ? \\[0pt]
Le vivant est-il réductible au physico-chimique ? \\[0pt]
Le vivant est-il un objet de science comme un autre ? \\[0pt]
Le vivant et la machine \\[0pt]
Le vivant et la mort \\[0pt]
Le vivant et la sensibilité \\[0pt]
Le vivant et la technique \\[0pt]
Le vivant et le vécu \\[0pt]
Le vivant et l'expérimentation \\[0pt]
Le vivant et l'inerte \\[0pt]
Le vivant et ses lois \\[0pt]
Le vivant et son milieu \\[0pt]
Le vivant n'est-il que matière ? \\[0pt]
Le vivant n'est-il qu'une machine ingénieuse ? \\[0pt]
Le vivant obéit-il à des lois ? \\[0pt]
Le vivant obéit-il à une nécessité ? \\[0pt]
Le vivant se définit-il par sa résistance à la mort ? \\[0pt]
Le voile des mots \\[0pt]
Le vol \\[0pt]
Le volontaire et l'involontaire \\[0pt]
Le volontarisme \\[0pt]
L'évolution \\[0pt]
L'évolution des langues \\[0pt]
L'évolution des sociétés dépend-elle du progrès technique \\[0pt]
Le voyage \\[0pt]
Le voyage dans le temps \\[0pt]
Le vrai admet-il des degrés ? \\[0pt]
Le vrai a-t-il une histoire ? \\[0pt]
Le vrai doit-il être démontré ? \\[0pt]
Le vrai est-il à lui-même sa propre marque ? \\[0pt]
Le vrai et le bien \\[0pt]
Le vrai et le bien sont-ils analogues ? \\[0pt]
Le vrai et le faux \\[0pt]
Le vrai et le réel \\[0pt]
Le vrai et le vraisemblable \\[0pt]
Le vrai et l'imaginaire \\[0pt]
Le vrai n'est-il que ce sur quoi l'on s'accorde ? \\[0pt]
Le vrai peut-il rester invérifiable ? \\[0pt]
Le vraisemblable \\[0pt]
Le vraisemblable et le probable \\[0pt]
Le vraisemblable et le romanesque \\[0pt]
Le vraisemblable présente-t-il un intérêt pour la pensée ? \\[0pt]
Le vrai se perçoit-il ? \\[0pt]
Le vrai se réduit-il à ce qui est vérifiable ? \\[0pt]
Le vrai se réduit-il à l'utile ? \\[0pt]
Le vulgaire \\[0pt]
L'exactitude \\[0pt]
L'excellence \\[0pt]
L'excellence des sens \\[0pt]
L'exception \\[0pt]
L'exception est-elle instructive ? \\[0pt]
L'exception peut-elle confirmer la règle ? \\[0pt]
L'excès \\[0pt]
L'excès et le défaut \\[0pt]
L'exclusion \\[0pt]
L'excuse \\[0pt]
L'exécution d'une œuvre d'art est-elle toujours une œuvre d'art ? \\[0pt]
L'exemplaire \\[0pt]
L'exemplarité \\[0pt]
L'exemple \\[0pt]
L'exemple en morale \\[0pt]
L'exercice \\[0pt]
L'exercice de la raison \\[0pt]
L'exercice de la vertu \\[0pt]
L'exercice de la volonté \\[0pt]
L'exercice du pouvoir \\[0pt]
L'exercice du pouvoir est-il nécessairement solitaire ? \\[0pt]
L'exercice solitaire du pouvoir \\[0pt]
L'exigence de vérité a-t-elle un sens moral ? \\[0pt]
L'exigence morale \\[0pt]
L'exil \\[0pt]
L'existence \\[0pt]
L'existence a-t-elle un sens ? \\[0pt]
L'existence d'autrui \\[0pt]
L'existence de Dieu \\[0pt]
L'existence de l'État dépend-elle d'un contrat ? \\[0pt]
L'existence des idées \\[0pt]
L'existence du mal \\[0pt]
L'existence du mal met-elle en échec la raison ? \\[0pt]
L'existence du passé \\[0pt]
L'existence du possible \\[0pt]
L'existence est-elle pensable ? \\[0pt]
L'existence est-elle une propriété ? \\[0pt]
L'existence est-elle un jeu ? \\[0pt]
L'existence est-elle vaine ? \\[0pt]
L'existence et le temps \\[0pt]
L'existence se démontre-t-elle ? \\[0pt]
L'existence se laisse-t-elle penser ? \\[0pt]
L'existence se prouve-t-elle ? \\[0pt]
L'existence suppose-t-elle la conscience du temps ? \\[0pt]
L'expérience \\[0pt]
L'expérience artistique \\[0pt]
L'expérience a-t-elle le même sens dans toutes les sciences ? \\[0pt]
L'expérience cruciale \\[0pt]
L'expérience dans les sciences humaines \\[0pt]
L'expérience d'autrui nous est-elle utile ? \\[0pt]
L'expérience de la beauté \\[0pt]
L'expérience de la beauté est-elle naturelle ? \\[0pt]
L'expérience de la liberté \\[0pt]
L'expérience de la maladie \\[0pt]
L'expérience de la vie \\[0pt]
L'expérience de la vie produit-elle un savoir ? \\[0pt]
L'expérience de l'injustice \\[0pt]
L'expérience de l'irrationnel \\[0pt]
L'expérience démontre-t-elle quelque chose ? \\[0pt]
L'expérience de pensée \\[0pt]
L'expérience de soi-même \\[0pt]
L'expérience directe est-elle une connaissance ? \\[0pt]
L'expérience du beau peut-elle être une expérience métaphysique ? \\[0pt]
L'expérience du danger \\[0pt]
L'expérience du désir \\[0pt]
L'expérience du mal \\[0pt]
L'expérience du présent \\[0pt]
L'expérience du sociologue \\[0pt]
L'expérience du temps \\[0pt]
L'expérience en psychologie \\[0pt]
L'expérience en sciences humaines \\[0pt]
L'expérience enseigne-elle quelque chose ? \\[0pt]
L'expérience, est-ce l'observation ? \\[0pt]
L'expérience est-elle la seule source de nos connaissances ? \\[0pt]
L'expérience est-elle un guide suffisant ? \\[0pt]
L'expérience esthétique \\[0pt]
L'expérience esthétique peut-elle se communiquer ? \\[0pt]
L'expérience esthétique relève-t-elle de la contemplation ? \\[0pt]
L'expérience et la sensation \\[0pt]
L'expérience et l'expérimentation \\[0pt]
L'expérience imaginaire \\[0pt]
L'expérience instruit-elle ? \\[0pt]
L'expérience métaphysique \\[0pt]
L'expérience morale \\[0pt]
L'expérience nous apprend-elle quelque chose ? \\[0pt]
L'expérience permet-elle de départager des théories concurrentes ? \\[0pt]
L'expérience peut-elle avoir raison des principes ? \\[0pt]
L'expérience peut-elle contredire la théorie ? \\[0pt]
L'expérience religieuse \\[0pt]
L'expérience rend-elle raisonnable ? \\[0pt]
L'expérience rend-elle responsable ? \\[0pt]
L'expérience scientifique \\[0pt]
L'expérience sensible est-elle la seule source légitime de connaissance ? \\[0pt]
L'expérience suffit-elle pour établir une vérité ? \\[0pt]
L'expérimentation \\[0pt]
L'expérimentation en art \\[0pt]
L'expérimentation en psychologie \\[0pt]
L'expérimentation en sciences sociales \\[0pt]
L'expérimentation est-elle facteur de vérité ? \\[0pt]
L'expérimentation médicale \\[0pt]
L'expérimentation sur l'être humain \\[0pt]
L'expérimentation sur le vivant \\[0pt]
L'expert et l'amateur \\[0pt]
L'expertise \\[0pt]
L'expertise politique \\[0pt]
L'explication \\[0pt]
L'explication causale dans les sciences sociales \\[0pt]
L'explication de l'œuvre d'art \\[0pt]
L'explication scientifique \\[0pt]
L'exploitation de l'homme par l'homme \\[0pt]
L'exposition \\[0pt]
L'exposition de l'œuvre d'art \\[0pt]
L'expression \\[0pt]
L'expression artistique \\[0pt]
L'expression de l'inconscient \\[0pt]
L'expression des citoyens dans le vote suffit-elle à faire la démocratie ? \\[0pt]
L'expression des citoyens dans le vote suffit-elle à faire vivre la démocratie ? \\[0pt]
L'expression des sentiments \\[0pt]
L'expression du désir \\[0pt]
L'expression du mouvement \\[0pt]
L'expression « perdre son temps » a-t-elle un sens ? \\[0pt]
L'expression peut-elle être libre ? \\[0pt]
L'expressivité musicale \\[0pt]
L'extériorité \\[0pt]
L'extinction du désir \\[0pt]
L'extinction du désir est-elle le commencement du bonheur ? \\[0pt]
L'extraordinaire \\[0pt]
L'extrémisme \\[0pt]
L'extrémité \\[0pt]
L'habileté \\[0pt]
L'habileté et la prudence \\[0pt]
L'habitation \\[0pt]
L'habitude \\[0pt]
L'habitude a-t-elle des vertus ? \\[0pt]
L'habitude est-elle notre guide dans la vie ? \\[0pt]
L'habitude est-elle une force ? \\[0pt]
L'habitude est-elle une seconde nature ? \\[0pt]
L'habitude morale \\[0pt]
L'habitus \\[0pt]
L'harmonie \\[0pt]
L'harmonie du monde \\[0pt]
L'hégémonie politique \\[0pt]
L'hérédité \\[0pt]
L'hérésie \\[0pt]
L'héritage \\[0pt]
L'héroïsme \\[0pt]
L'hésitation \\[0pt]
L'hétérogénéité sociale \\[0pt]
L'hétéronomie \\[0pt]
L'hétéronomie de l'art \\[0pt]
L'histoire \\[0pt]
L'Histoire \\[0pt]
L'histoire a-t-elle des lois ? \\[0pt]
L'Histoire a-t-elle un commencement ? \\[0pt]
L'histoire a-t-elle un commencement et une fin ? \\[0pt]
L'histoire a-t-elle une fin ? \\[0pt]
L'histoire a-t-elle un sens ? \\[0pt]
L'histoire comme science \\[0pt]
L'histoire de l'art \\[0pt]
L'histoire de l'art est-elle celle des styles ? \\[0pt]
L'histoire de l'art est-elle finie ? \\[0pt]
L'histoire de l'art peut-elle arriver à son terme ? \\[0pt]
L'histoire des arts est-elle liée à l'histoire des techniques ? \\[0pt]
L'histoire des civilisations \\[0pt]
L'histoire des sciences \\[0pt]
L'Histoire des sciences \\[0pt]
L'histoire des sciences est-elle une histoire ? \\[0pt]
L'histoire du droit est-elle celle du progrès de la justice ? \\[0pt]
L'histoire : enquête ou science ? \\[0pt]
L'histoire est-elle avant tout mémoire ? \\[0pt]
L'histoire est-elle cyclique ? \\[0pt]
L'histoire est-elle déterministe ? \\[0pt]
L'histoire est-elle écrite par les vainqueurs ? \\[0pt]
L'histoire est-elle la connaissance du passé humain ? \\[0pt]
L'histoire est-elle la mémoire de l'humanité ? \\[0pt]
L'histoire est-elle la science de ce qui ne se répète jamais ? \\[0pt]
L'histoire est-elle la science de l'événementiel ? \\[0pt]
L'histoire est-elle la science du passé ? \\[0pt]
L'histoire est-elle la vérité de la mémoire ? \\[0pt]
L'histoire est-elle le récit objectif des faits passés ? \\[0pt]
L'histoire est-elle le règne du hasard ? \\[0pt]
L'histoire est-elle le théâtre des passions ? \\[0pt]
L'histoire est-elle rationnelle ? \\[0pt]
L'histoire est-elle sans fin ? \\[0pt]
L'histoire est-elle tragique ? \\[0pt]
L'histoire est-elle une explication ou une justification du passé ? \\[0pt]
L'histoire est-elle une science ? \\[0pt]
L'histoire est-elle une science comme les autres ? \\[0pt]
L'histoire est-elle une science de l'individuel ? \\[0pt]
L'histoire est-elle un genre littéraire ? \\[0pt]
L'histoire est-elle un récit ? \\[0pt]
L'histoire est-elle un roman ? \\[0pt]
L'histoire est-elle un roman vrai ? \\[0pt]
L'histoire est-elle utile ? \\[0pt]
L'histoire est-elle utile à la politique ? \\[0pt]
L'histoire et la géographie \\[0pt]
L'histoire et la raison \\[0pt]
« L'histoire jugera » \\[0pt]
L'histoire jugera \\[0pt]
« L'histoire jugera » : quel sens faut-il accorder à cette expression ? \\[0pt]
L'histoire jugera-t-elle ? \\[0pt]
L'histoire n'a-t-elle pour objet que le passé ? \\[0pt]
L'histoire naturelle \\[0pt]
L'histoire n'est-elle que bruit et fureur ? \\[0pt]
L'histoire n'est-elle que la connaissance du passé ? \\[0pt]
L'histoire n'est-elle que le produit de rapports de force ? \\[0pt]
L'histoire n'est-elle qu'un déchaînement de violence ? \\[0pt]
L'histoire n'est-elle qu'une suite d'événements ? \\[0pt]
L'histoire n'est-elle qu'un récit ? \\[0pt]
L'histoire nous appartient-elle ? \\[0pt]
L'histoire obéit-elle à des lois ? \\[0pt]
L'histoire pèse-t-elle comme un destin sur les individus ? \\[0pt]
L'histoire peut-elle être contemporaine ? \\[0pt]
L'histoire peut-elle être objective ? \\[0pt]
L'histoire peut-elle être universelle ? \\[0pt]
L'histoire peut-elle ne pas avoir de sens ? \\[0pt]
L'histoire peut-elle s'écrire au présent ? \\[0pt]
L'histoire peut-elle se répéter ? \\[0pt]
L'histoire : science ou récit ? \\[0pt]
L'histoire se réduit-elle à l'étude du passé ? \\[0pt]
L'histoire se répète-t-elle ? \\[0pt]
L'histoire sert-elle le présent ? \\[0pt]
L'histoire universelle est-elle l'histoire des guerres ? \\[0pt]
L'historicisme \\[0pt]
L'historicité \\[0pt]
L'historicité des sciences \\[0pt]
L'historien \\[0pt]
L'historien est-il un homme qui se souvient du passé ? \\[0pt]
L'historien et le politique \\[0pt]
L'historien peut-il être impartial ? \\[0pt]
L'historien peut-il se passer du concept de causalité ? \\[0pt]
L'homicide \\[0pt]
L'hominisation \\[0pt]
L'homme aime-t-il la justice pour elle-même ? \\[0pt]
L'homme a-t-il besoin de l'art ? \\[0pt]
L'homme a-t-il besoin de religion ? \\[0pt]
L'homme a-t-il une nature ? \\[0pt]
L'homme a-t-il une place dans la nature ? \\[0pt]
L'homme cultivé \\[0pt]
L'homme de l'art \\[0pt]
L'homme de la rue \\[0pt]
L'homme des droits de l'homme \\[0pt]
L'homme des droits de l'homme n'est-il qu'une fiction ? \\[0pt]
L'homme des foules \\[0pt]
L'homme des sciences de l'homme \\[0pt]
L'homme des sciences humaines \\[0pt]
L'homme d'État \\[0pt]
L'homme est-il chez lui dans l'univers ? \\[0pt]
L'homme est-il face à la nature ? \\[0pt]
L'homme est-il fait pour le travail ? \\[0pt]
L'homme est-il la mesure de toute chose ? \\[0pt]
L'homme est-il la mesure de toutes choses ? \\[0pt]
L'homme est-il l'artisan de sa dignité ? \\[0pt]
L'homme est-il le seul être à avoir une histoire ? \\[0pt]
L'homme est-il le seul sujet de droit ? \\[0pt]
L'homme est-il le sujet de son histoire ? \\[0pt]
L'homme est-il libre par nature ? \\[0pt]
L'homme est-il meilleur seul ou en société ? \\[0pt]
L'homme est-il naturellement artiste ? \\[0pt]
L'homme est-il objet de science ? \\[0pt]
L'homme est-il par nature un être religieux ? \\[0pt]
L'homme est-il prisonnier du temps ? \\[0pt]
L'homme est-il raisonnable par nature ? \\[0pt]
L'homme est-il religieux par nature ? \\[0pt]
L'homme est-il un animal ? \\[0pt]
L'homme est-il un animal comme les autres ? \\[0pt]
L'homme est-il un animal comme un autre ? \\[0pt]
L'homme est-il un animal dénaturé ? \\[0pt]
L'homme est-il un animal métaphysique ? \\[0pt]
L'homme est-il un animal politique ? \\[0pt]
L'homme est-il un animal rationnel ? \\[0pt]
L'homme est-il un animal religieux ? \\[0pt]
L'homme est-il un animal social ? \\[0pt]
L'homme est-il un animal symbolique ? \\[0pt]
L'homme est-il un animal technicien ? \\[0pt]
L'homme est-il un corps pensant ? \\[0pt]
L'homme est-il un être de devoir ? \\[0pt]
L'homme est-il un être inachevé ? \\[0pt]
L'homme est-il un être social par nature ? \\[0pt]
L'homme est-il un loup pour l'homme ? \\[0pt]
L'homme est-il vraiment un animal politique ? \\[0pt]
« L'homme est la mesure de toute chose » \\[0pt]
« L'homme est la mesure de toutes choses » \\[0pt]
L'homme et la bête \\[0pt]
L'homme et la machine \\[0pt]
L'homme et la nature sont-ils commensurables ? \\[0pt]
L'homme et l'animal \\[0pt]
L'homme et le citoyen \\[0pt]
L'homme et le sacré \\[0pt]
L'homme fait-il partie de la nature ? \\[0pt]
L'homme injuste est-il heureux ? \\[0pt]
L'homme injuste peut-il être heureux ? \\[0pt]
L'homme injuste peut-il être malheureux ? \\[0pt]
L'homme intérieur \\[0pt]
L'homme, le citoyen, le soldat \\[0pt]
L'homme libre est-il un homme seul ? \\[0pt]
L'homme-machine \\[0pt]
L'homme moyen \\[0pt]
L'homme n'est-il qu'un animal comme les autres ? \\[0pt]
L'homme pense-t-il toujours ? \\[0pt]
L'homme peut-il changer ? \\[0pt]
L'homme peut-il être inhumain ? \\[0pt]
L'homme peut-il être libéré du besoin ? \\[0pt]
L'homme peut-il être objet d'expérience ? \\[0pt]
L'homme peut-il faire l'objet d'une science ? \\[0pt]
L'homme peut-il se représenter un monde sans l'homme ? \\[0pt]
L'homme se réalise-t-il dans le travail ? \\[0pt]
L'homme se reconnaît-il mieux dans le travail ou dans le loisir ? \\[0pt]
L'homme trouble-t-il l'ordre de la nature ? \\[0pt]
L'homme vertueux doit-il s'attendre à être malheureux ? \\[0pt]
L'homo œconomicus est-il rationnel ? \\[0pt]
L'honnêteté \\[0pt]
L'honneur \\[0pt]
L'horizon \\[0pt]
L'horizon de la mort \\[0pt]
L'horreur \\[0pt]
L'horreur du vide \\[0pt]
L'horrible \\[0pt]
L'hospitalité \\[0pt]
L'hospitalité a-t-elle un sens politique ? \\[0pt]
L'hospitalité est-elle un devoir ? \\[0pt]
L'hospitalité est-elle une vertu politique ? \\[0pt]
L'humain \\[0pt]
L'humain est-il un animal irrationnel ? \\[0pt]
L'humanité \\[0pt]
L'humanité a-t-elle eu une enfance ? \\[0pt]
L'humanité a-t-elle une histoire ? \\[0pt]
L'humanité est-elle aimable ? \\[0pt]
L'humanité : un concept normatif ? \\[0pt]
L'humiliation \\[0pt]
L'humilité \\[0pt]
L'humilité est-elle une vertu ? \\[0pt]
L'humour \\[0pt]
L'humour et l'ironie \\[0pt]
L'hybridation des arts \\[0pt]
L'hypocrisie \\[0pt]
L'hypothèse \\[0pt]
L'hypothèse de la liberté est-elle compatible avec les exigences de la raison ? \\[0pt]
L'hypothèse de l'inconscient \\[0pt]
Libéral et libertaire \\[0pt]
Libéralisme et démocratie \\[0pt]
Libéralité et libéralisme \\[0pt]
Liberté d'agir, liberté de penser \\[0pt]
« Liberté, égalité, fraternité » \\[0pt]
Liberté, égalité, fraternité \\[0pt]
Liberté, égalité, solidarité \\[0pt]
Liberté et courage \\[0pt]
Liberté et démocratie \\[0pt]
Liberté et déterminisme \\[0pt]
Liberté et éducation \\[0pt]
Liberté et égalité \\[0pt]
Liberté et engagement \\[0pt]
Liberté et existence \\[0pt]
Liberté et habitude \\[0pt]
Liberté et indépendance \\[0pt]
Liberté et libération \\[0pt]
Liberté et licence \\[0pt]
Liberté et nécessité \\[0pt]
Liberté et négativité \\[0pt]
Liberté et pouvoir \\[0pt]
Liberté et propriété \\[0pt]
Liberté et responsabilité \\[0pt]
Liberté et savoir \\[0pt]
Liberté et sécurité \\[0pt]
Liberté et société \\[0pt]
Liberté et solitude \\[0pt]
Liberté et transgression \\[0pt]
Liberté et vérité \\[0pt]
Liberté humaine et liberté divine \\[0pt]
Liberté réelle, liberté formelle \\[0pt]
Libertés publiques et culture politique \\[0pt]
Libre arbitre et déterminisme sont-ils compatibles ? \\[0pt]
Libre arbitre et liberté \\[0pt]
Libre-arbitre, impulsion, contrainte \\[0pt]
Libre et heureux \\[0pt]
L'idéal \\[0pt]
L'idéal dans l'art \\[0pt]
L'idéal de l'art \\[0pt]
L'idéal démonstratif \\[0pt]
L'idéal de vérité \\[0pt]
L'idéal et le réel \\[0pt]
L'idéalisme \\[0pt]
L'idéaliste \\[0pt]
L'idéalité \\[0pt]
L'idéal moral est-il vain ? \\[0pt]
L'idéal systématique \\[0pt]
L'idéal-type \\[0pt]
L'idée d'anthropologie \\[0pt]
L'idée de beaux arts \\[0pt]
L'idée de bioéthique \\[0pt]
L'idée de bonheur \\[0pt]
L'idée de bonheur collectif a-t-elle un sens ? \\[0pt]
L'idée de civilisation \\[0pt]
L'idée de communauté \\[0pt]
L'idée de communisme \\[0pt]
L'idée de connaissance approchée \\[0pt]
L'idée de conscience collective \\[0pt]
L'idée de continuité \\[0pt]
L'idée de contrat \\[0pt]
L'idée de contrat social \\[0pt]
L'idée de création \\[0pt]
L'idée de crise \\[0pt]
L'idée de critique \\[0pt]
L'idée de culpabilité collective \\[0pt]
L'idée de décadence \\[0pt]
L'idée de destin \\[0pt]
L'idée de destin a-t-elle un sens ? \\[0pt]
L'idée de destin est-elle une idée périmée ? \\[0pt]
L'idée de déterminisme \\[0pt]
L'idée de devoir requiert-elle l'idée de liberté ? \\[0pt]
L'idée de Dieu \\[0pt]
L'idée de domination \\[0pt]
L'idée de forme sociale \\[0pt]
L'idée d'égalité \\[0pt]
L'idée de guerre juste \\[0pt]
L'idée de justice \\[0pt]
L'idée de langue parfaite \\[0pt]
L'idée de langue universelle \\[0pt]
L'idée de logique \\[0pt]
L'idée de logique transcendantale \\[0pt]
L'idée de logique universelle \\[0pt]
L'idée de loi de la nature \\[0pt]
L'idée de loi logique \\[0pt]
L'idée de loi naturelle \\[0pt]
L'idée de mal nécessaire \\[0pt]
L'idée de mathesis universalis \\[0pt]
L'idée de métier \\[0pt]
L'idée de milieu \\[0pt]
L'idée de modernité \\[0pt]
L'idée de monde \\[0pt]
L'idée de morale appliquée \\[0pt]
L'idée de nation \\[0pt]
L'idée de nature humaine \\[0pt]
L'idée de « nature » n'est-elle qu'un mythe ? \\[0pt]
L'idée d'encyclopédie \\[0pt]
L'idée de norme \\[0pt]
L'idée de paix \\[0pt]
L'idée de perfection \\[0pt]
L'idée de philosophie première \\[0pt]
L'idée de préhistoire \\[0pt]
L'idée de progrès \\[0pt]
L'idée de progrès historique est-elle devenue désuète ? \\[0pt]
L'idée de réduction \\[0pt]
L'idée de république \\[0pt]
L'idée de République \\[0pt]
L'idée de rétribution est-elle nécessaire à la morale ? \\[0pt]
L'idée de révolution \\[0pt]
L'idée de science \\[0pt]
L'idée de science expérimentale \\[0pt]
L'idée de « sciences exactes » \\[0pt]
L'idée de société primitive \\[0pt]
L'idée de substance \\[0pt]
L'idée de système \\[0pt]
L'idée d'éternité \\[0pt]
L'idée d'Europe \\[0pt]
L'idée d'évolution \\[0pt]
L'idée d'exactitude \\[0pt]
L'idée d'histoire universelle \\[0pt]
L'idée d'humanité \\[0pt]
L'idée d'inconscient \\[0pt]
L'idée d'ordre \\[0pt]
L'idée d'organisme \\[0pt]
L'idée d'origine \\[0pt]
L'idée d'un commencement absolu \\[0pt]
L'idée d'une langue universelle \\[0pt]
L'idée d'une liberté totale a-t- elle un sens ? \\[0pt]
L'idée d'une pensée inconsciente est-elle contradictoire ? \\[0pt]
L'idée d'une religion personnelle a-t-elle un sens ? \\[0pt]
L'idée d'une science bien faite \\[0pt]
L'idée d'une science générale des signes \\[0pt]
L'idée d'univers \\[0pt]
L'idée d'université \\[0pt]
L'idée esthétique \\[0pt]
L'idée et l'idéal \\[0pt]
L'identification \\[0pt]
L'identité \\[0pt]
L'identité collective \\[0pt]
L'identité des hommes \\[0pt]
L'identité et la différence \\[0pt]
L'identité nationale \\[0pt]
L'identité personnelle \\[0pt]
L'identité personnelle est-elle donnée ou construite ? \\[0pt]
L'identité relève-t-elle du champ politique ? \\[0pt]
L'idéologie \\[0pt]
L'idiot \\[0pt]
L'idolâtrie \\[0pt]
L'idole \\[0pt]
Lieu et milieu \\[0pt]
L'ignoble \\[0pt]
L'ignorance \\[0pt]
L'ignorance des savants \\[0pt]
L'ignorance est-elle la raison du mal ? \\[0pt]
L'ignorance est-elle préférable à l'erreur ? \\[0pt]
L'ignorance est-elle une excuse ? \\[0pt]
L'ignorance nous excuse-t-elle ? \\[0pt]
L'ignorance peut-elle être une excuse ? \\[0pt]
L'illimité \\[0pt]
L'illusion \\[0pt]
L'illusion de la liberté \\[0pt]
L'illusion est-elle nécessaire au bonheur des hommes ? \\[0pt]
L'illusoire \\[0pt]
L'illustration \\[0pt]
L'image \\[0pt]
L'image du monde \\[0pt]
L'image et le modèle \\[0pt]
L'image et le réel \\[0pt]
L'imaginaire \\[0pt]
L'imaginaire et le réel \\[0pt]
L'imaginaire politique \\[0pt]
L'imaginaire social \\[0pt]
L'imagination \\[0pt]
L'imagination a-t-elle des limites ? \\[0pt]
L'imagination a-t-elle sa place dans la connaissance scientifique ? \\[0pt]
L'imagination dans l'art \\[0pt]
L'imagination dans les sciences \\[0pt]
L'imagination enrichit-elle la connaissance ? \\[0pt]
L'imagination en science \\[0pt]
L'imagination est-elle créatrice ? \\[0pt]
L'imagination est-elle dangereuse ? \\[0pt]
L'imagination est-elle le refuge de la liberté ? \\[0pt]
L'imagination est-elle libre ? \\[0pt]
L'imagination est-elle maîtresse d'erreur et de fausseté ? \\[0pt]
L'imagination esthétique \\[0pt]
L'imagination et la raison \\[0pt]
L'imagination nous éloigne-t-elle du réel ? \\[0pt]
L'imagination nous instruit-elle ? \\[0pt]
L'imagination politique \\[0pt]
L'imagination scientifique \\[0pt]
L'imbécile heureux \\[0pt]
L'imitation \\[0pt]
L'imitation a-t-elle une fonction morale ? \\[0pt]
Limite, borne, frontière \\[0pt]
L'immanence \\[0pt]
L'immatériel \\[0pt]
L'immédiat \\[0pt]
L'immémorial \\[0pt]
L'immensité \\[0pt]
L'immoralisme \\[0pt]
L'immoraliste \\[0pt]
L'immoralité \\[0pt]
L'immortalité \\[0pt]
L'immortalité de l'âme \\[0pt]
L'immortalité des œuvres d'art \\[0pt]
L'immuable \\[0pt]
L'immutabilité \\[0pt]
L'impardonnable \\[0pt]
L'imparfait \\[0pt]
L'impartialité \\[0pt]
L'impartialité des historiens \\[0pt]
L'impartialité est-elle toujours désirable ? \\[0pt]
L'impassibilité \\[0pt]
L'impatience \\[0pt]
L'impensable \\[0pt]
L'impératif \\[0pt]
L'impératif d'impartialité \\[0pt]
L'impératif hypothétique \\[0pt]
L'imperceptible \\[0pt]
L'impersonnel \\[0pt]
L'impiété \\[0pt]
L'implicite \\[0pt]
L'implicite dans la communication \\[0pt]
L'importance des détails \\[0pt]
L'impossible \\[0pt]
L'impossible est-il concevable ? \\[0pt]
L'imposteur \\[0pt]
L'imprescriptible \\[0pt]
L'impression \\[0pt]
L'imprévisible \\[0pt]
L'imprévu \\[0pt]
L'improbable \\[0pt]
L'improvisation \\[0pt]
L'improvisation dans l'art \\[0pt]
L'imprudence \\[0pt]
L'impuissance \\[0pt]
L'impuissance de la raison \\[0pt]
L'impuissance de l'art \\[0pt]
L'impuissance de l'État \\[0pt]
L'impunité \\[0pt]
L'inacceptable \\[0pt]
L'inaccessible \\[0pt]
L'inachevé \\[0pt]
L'inachèvement \\[0pt]
L'inaction \\[0pt]
L'inaliénable \\[0pt]
L'inaperçu \\[0pt]
L'inapparent \\[0pt]
L'inattendu \\[0pt]
L'incarnation \\[0pt]
L'incarnation du pouvoir \\[0pt]
L'incertain \\[0pt]
L'incertitude \\[0pt]
L'incertitude du passé \\[0pt]
L'incertitude est-elle dans les choses ou dans les idées ? \\[0pt]
L'incertitude interdit-elle de raisonner ? \\[0pt]
L'incommensurabilité \\[0pt]
L'incommensurable \\[0pt]
L'incompréhensible \\[0pt]
L'inconcevable \\[0pt]
L'inconcevable et l'impossible \\[0pt]
L'inconnaissable \\[0pt]
L'inconnu \\[0pt]
L'inconscience \\[0pt]
L'inconscience peut-elle être coupable ? \\[0pt]
L'inconscient \\[0pt]
L'inconscient a-t-il son propre langage ? \\[0pt]
L'inconscient a-t-il une histoire ? \\[0pt]
L'inconscient collectif \\[0pt]
L'inconscient de l'art \\[0pt]
L'inconscient est-il dans l'âme ou dans le corps ? \\[0pt]
L'inconscient est-il l'animal en nous ? \\[0pt]
L'inconscient est-il pure négation de la conscience ? \\[0pt]
L'inconscient est-il un concept scientifique ? \\[0pt]
L'inconscient est-il un destin ? \\[0pt]
L'inconscient est-il une découverte ou une invention ? \\[0pt]
L'inconscient est-il une dimension de la conscience ? \\[0pt]
L'inconscient est-il une excuse ? \\[0pt]
L'inconscient est-il un obstacle à la liberté ? \\[0pt]
L'inconscient et l'involontaire \\[0pt]
L'inconscient et l'oubli \\[0pt]
L'inconscient n'est-il qu'un défaut de conscience ? \\[0pt]
L'inconscient n'est-il qu'une hypothèse ? \\[0pt]
L'inconscient nous révèle-t-il à nous-même ? \\[0pt]
L'inconscient peut-il se manifester ? \\[0pt]
L'inconscient psychanalytique réfute-t-il la notion philosophique de conscience ? \\[0pt]
L'inconséquence \\[0pt]
L'inconstance \\[0pt]
L'incorporel \\[0pt]
L'incrédulité \\[0pt]
L'incroyable \\[0pt]
L'inculture \\[0pt]
L'indécence \\[0pt]
L'indécidable \\[0pt]
L'indécision \\[0pt]
L'indéfini \\[0pt]
L'indémontrable \\[0pt]
L'indépassable \\[0pt]
L'indépendance \\[0pt]
L'indescriptible \\[0pt]
L'indésirable \\[0pt]
L'indétermination \\[0pt]
L'indéterminé \\[0pt]
L'indice \\[0pt]
L'indice et la preuve \\[0pt]
L'indicible \\[0pt]
L'indicible et l'impensable \\[0pt]
L'indicible et l'ineffable \\[0pt]
L'indifférence \\[0pt]
L'indifférence à la politique \\[0pt]
L'indifférence peut-elle être une vertu ? \\[0pt]
L'indigène et l'étranger \\[0pt]
L'indignation \\[0pt]
L'indignité \\[0pt]
L'indiscernable \\[0pt]
L'indiscutable \\[0pt]
L'indistinct \\[0pt]
L'individu \\[0pt]
L'individualisme \\[0pt]
L'individualisme a-t-il sa place en politique ? \\[0pt]
L'individualisme est-il un égoïsme ? \\[0pt]
L'individualisme méthodologique \\[0pt]
L'individualisme rend-il la politique impossible ? \\[0pt]
L'individualité \\[0pt]
L'individu a-t-il des droits ? \\[0pt]
L'individuation \\[0pt]
L'individu échappe-t-il aux sciences humaines ? \\[0pt]
L'individuel \\[0pt]
L'individuel et le collectif \\[0pt]
L'individu est-il définissable ? \\[0pt]
L'individu est-il ineffable ? \\[0pt]
L'individu et la multitude \\[0pt]
L'individu et la société \\[0pt]
L'individu et le groupe \\[0pt]
L'individu et les masses \\[0pt]
L'individu et l'espèce \\[0pt]
L'individu face à L'État \\[0pt]
L'individu particulier peut-il être l'objet d'une connaissance sociologique ? \\[0pt]
L'indivisible \\[0pt]
L'indubitable \\[0pt]
L'induction \\[0pt]
L'induction et la déduction \\[0pt]
L'indulgence \\[0pt]
L'industrie culturelle \\[0pt]
L'industrie du beau \\[0pt]
L'ineffable \\[0pt]
L'ineffable et l'innommable \\[0pt]
L'inégalité a-t-elle des vertus ? \\[0pt]
L'inégalité des chances \\[0pt]
L'inégalité entre les hommes \\[0pt]
L'inégalité est-elle nécessairement injuste ? \\[0pt]
L'inégalité naturelle \\[0pt]
L'inégalité sociale \\[0pt]
L'inéluctable \\[0pt]
L'inerte \\[0pt]
L'inertie \\[0pt]
L'inespéré \\[0pt]
L'inesthétique \\[0pt]
L'inestimable \\[0pt]
L'inexactitude et le savoir scientifique \\[0pt]
L'inexact peut-il avoir une valeur ? \\[0pt]
L'inexistant \\[0pt]
L'inexpérience \\[0pt]
L'infâme \\[0pt]
L'infamie \\[0pt]
L'inférence \\[0pt]
L'infériorité \\[0pt]
L'infidélité \\[0pt]
L'infini \\[0pt]
L'infini est-il la négation du fini ? \\[0pt]
L'infini et l'indéfini \\[0pt]
L'infini se réduit-il à l'indéfini ? \\[0pt]
L'infinité de l'espace \\[0pt]
L'infinité de l'univers a-t-elle de quoi nous effrayer ? \\[0pt]
L'infinité du monde \\[0pt]
L'influence \\[0pt]
L'information \\[0pt]
L'informe \\[0pt]
L'informe et le difforme \\[0pt]
L'ingénieur \\[0pt]
L'ingénieur et l'artiste \\[0pt]
L'ingénieur et le politique \\[0pt]
L'ingéniosité \\[0pt]
L'ingénuité \\[0pt]
L'ingérence \\[0pt]
L'ingratitude \\[0pt]
Linguistique et sens \\[0pt]
L'inhibition \\[0pt]
L'inhumain \\[0pt]
L'inhumanité \\[0pt]
L'inimaginable \\[0pt]
L'inimitié \\[0pt]
L'inintelligible \\[0pt]
L'iniquité \\[0pt]
L'initiation \\[0pt]
L'injonction \\[0pt]
L'injure \\[0pt]
L'injustice \\[0pt]
L'injustice de la loi \\[0pt]
L'injustice est-elle préférable au désordre ? \\[0pt]
L'injustifiable \\[0pt]
L'inné et l'acquis \\[0pt]
L'innocence \\[0pt]
L'innommable \\[0pt]
L'innovation \\[0pt]
L'innovation technique nous impose-t-elle de nouveaux devoirs ? \\[0pt]
L'inobservable \\[0pt]
L'inoubliable \\[0pt]
L'inquiétant \\[0pt]
L'inquiétude \\[0pt]
L'inquiétude de l'avenir \\[0pt]
L'inquiétude peut-elle définir l'existence humaine ? \\[0pt]
L'inquiétude peut-elle devenir l'existence humaine ? \\[0pt]
L'insatisfaction \\[0pt]
L'insensé \\[0pt]
L'insensibilité \\[0pt]
L'insignifiant \\[0pt]
« L'insociable sociabilité » \\[0pt]
L'insociable sociabilité \\[0pt]
L'insolite \\[0pt]
L'insoluble \\[0pt]
L'insouciance \\[0pt]
L'insoumission \\[0pt]
L'insoutenable \\[0pt]
L'inspiration \\[0pt]
L'instant \\[0pt]
L'instant a-t-il une réalité ? \\[0pt]
L'instant de la décision est-il une folie ? \\[0pt]
L'instant et la durée \\[0pt]
L'instinct \\[0pt]
L'institution \\[0pt]
L'institution de la société \\[0pt]
L'institution du mariage \\[0pt]
L'institutionnalisation des conduites \\[0pt]
L'institution scientifique \\[0pt]
L'institution scolaire \\[0pt]
L'instruction \\[0pt]
L'instruction est-elle facteur de moralité ? \\[0pt]
L'instrument \\[0pt]
L'instrument et la machine \\[0pt]
L'instrument, l'outil, le jouet \\[0pt]
L'instrument mathématique en sciences humaines \\[0pt]
L'instrument scientifique \\[0pt]
L'insulte \\[0pt]
L'insurrection \\[0pt]
L'insurrection est-elle un droit ? \\[0pt]
L'intangible \\[0pt]
L'intégration sociale \\[0pt]
L'intégrité \\[0pt]
L'intellect \\[0pt]
L'intellectuel \\[0pt]
L'intelligence \\[0pt]
L'intelligence artificielle \\[0pt]
L'intelligence de la machine \\[0pt]
L'intelligence de la main \\[0pt]
L'intelligence de la matière \\[0pt]
L'intelligence de la technique \\[0pt]
L'intelligence des bêtes \\[0pt]
L'intelligence des foules \\[0pt]
L'intelligence du lointain \\[0pt]
L'intelligence du sensible \\[0pt]
L'intelligence du vivant \\[0pt]
L'intelligence peut-elle être artificielle ? \\[0pt]
L'intelligence peut-elle être inhumaine ? \\[0pt]
L'intelligence politique \\[0pt]
L'intelligible \\[0pt]
L'intempérance \\[0pt]
L'intemporel \\[0pt]
L'intention \\[0pt]
L'intention morale \\[0pt]
L'intention morale suffit-elle à constituer la valeur morale de l'action ? \\[0pt]
L'intentionnalité \\[0pt]
L'interaction \\[0pt]
L'interdépendance \\[0pt]
L'interdépendance des faits humains \\[0pt]
L'interdiction \\[0pt]
L'interdisciplinarité \\[0pt]
L'interdit \\[0pt]
L'interdit a-t-il une origine sociale ? \\[0pt]
L'interdit est-il au fondement de la culture ? \\[0pt]
L'interdit et le sacré \\[0pt]
L'intéressant \\[0pt]
L'intérêt \\[0pt]
L'intérêt bien compris \\[0pt]
L'intérêt commun \\[0pt]
L'intérêt constitue-t-il l'unique lien social ? \\[0pt]
L'intérêt de la justice \\[0pt]
L'intérêt de l'art \\[0pt]
L'intérêt de la société l'emporte-t-il sur celui des individus ? \\[0pt]
L'intérêt de l'État \\[0pt]
L'intérêt des machines \\[0pt]
L'intérêt est-il le principe de tout échange ? \\[0pt]
L'intérêt général \\[0pt]
L'intérêt général est-il la somme des intérêts particuliers ? \\[0pt]
L'intérêt général est-il le bien commun ? \\[0pt]
L'intérêt général est-il un mythe ? \\[0pt]
L'intérêt général n'est-il qu'un mythe ? \\[0pt]
L'intérêt gouverne-t-il le monde ? \\[0pt]
L'intérêt peut-il avoir une valeur morale ? \\[0pt]
L'intérêt peut-il être général ? \\[0pt]
L'intérêt peut-il être une valeur morale ? \\[0pt]
L'intérêt public est-il une illusion ? \\[0pt]
L'intérieur et l'extérieur \\[0pt]
L'intériorisation des normes \\[0pt]
L'intériorité \\[0pt]
L'intériorité de l'œuvre \\[0pt]
L'intériorité est-elle un mythe ? \\[0pt]
L'interprétation \\[0pt]
L'interprétation de la loi \\[0pt]
L'interprétation de la nature \\[0pt]
L'interprétation des œuvres \\[0pt]
L'interprétation est-elle sans fin ? \\[0pt]
L'interprétation est-elle un art ? \\[0pt]
L'interprétation est-elle un art ou une science ? \\[0pt]
L'interprétation est-elle une activité sans fin ? \\[0pt]
L'interprétation est-elle une forme de connaissance ? \\[0pt]
L'interprétation est-elle une science ? \\[0pt]
L'interprétation : mode d'être ou mode de connaissance ? \\[0pt]
L'interprète est-il un créateur ? \\[0pt]
L'interprète et le créateur \\[0pt]
L'interprète sait-il ce qu'il cherche ? \\[0pt]
L'interrogation humaine \\[0pt]
L'intersectionnalité \\[0pt]
L'intersubjectivité \\[0pt]
L'intime \\[0pt]
L'intime conviction \\[0pt]
L'intime est-il politique ? \\[0pt]
L'intimité \\[0pt]
L'intolérable \\[0pt]
L'intolérance \\[0pt]
L'intraduisible \\[0pt]
L'intransigeance \\[0pt]
L'intransmissible \\[0pt]
L'introspection \\[0pt]
L'introspection est-elle une connaissance ? \\[0pt]
L'intuition \\[0pt]
L'intuition a-t-elle une place en logique ? \\[0pt]
L'intuition en mathématiques \\[0pt]
L'intuition intellectuelle \\[0pt]
L'intuition morale \\[0pt]
L'inutile \\[0pt]
L'inutile a-t-il de la valeur ? \\[0pt]
L'inutile est-il sans valeur ? \\[0pt]
L'invention \\[0pt]
L'invention de soi \\[0pt]
L'invention des valeurs \\[0pt]
L'invention en politique \\[0pt]
L'invention et la découverte \\[0pt]
L'invention technique \\[0pt]
L'invérifiable \\[0pt]
L'invisibilité \\[0pt]
L'invisible \\[0pt]
L'invivable \\[0pt]
L'involontaire \\[0pt]
L'invraisemblable \\[0pt]
Lire \\[0pt]
Lire et écrire \\[0pt]
L'ironie \\[0pt]
L'irrationalité \\[0pt]
L'irrationnel \\[0pt]
L'irrationnel est-il nécessairement absurde ? \\[0pt]
L'irrationnel est-il pensable ? \\[0pt]
L'irrationnel est-il toujours absurde ? \\[0pt]
L'irrationnel et l'absurde \\[0pt]
L'irrationnel et le politique \\[0pt]
L'irrationnel existe-t-il ? \\[0pt]
L'irréductible \\[0pt]
L'irréel \\[0pt]
L'irréfléchi \\[0pt]
L'irréfutable \\[0pt]
L'irrégularité \\[0pt]
L'irréparable \\[0pt]
L'irreprésentable \\[0pt]
L'irrésolution \\[0pt]
L'irrespect \\[0pt]
L'irresponsabilité \\[0pt]
L'irréversibilité \\[0pt]
L'irréversible \\[0pt]
L'irrévocable \\[0pt]
Littérature et philosophie \\[0pt]
Littérature et réalité \\[0pt]
L'ivresse \\[0pt]
L'obéissance \\[0pt]
L'obéissance à l'autorité \\[0pt]
L'obéissance est-elle compatible avec la liberté ? \\[0pt]
L'obéissance peut-elle être un acte de liberté ? \\[0pt]
L'objection de conscience \\[0pt]
L'objectivation \\[0pt]
L'objectivité \\[0pt]
L'objectivité de l'art \\[0pt]
L'objectivité de l'historien \\[0pt]
L'objectivité de l'œuvre d'art \\[0pt]
L'objectivité des sciences humaines \\[0pt]
L'objectivité des sciences sociales \\[0pt]
L'objectivité des valeurs \\[0pt]
L'objectivité en histoire \\[0pt]
L'objectivité existe-t-elle ? \\[0pt]
L'objectivité historique \\[0pt]
L'objectivité historique est-elle synonyme de neutralité ? \\[0pt]
L'objectivité scientifique \\[0pt]
L'objet \\[0pt]
L'objet d'amour \\[0pt]
L'objet de culte \\[0pt]
L'objet de la littérature \\[0pt]
L'objet de l'amour \\[0pt]
L'objet de la physique \\[0pt]
L'objet de la politique \\[0pt]
L'objet de la psychologie \\[0pt]
L'objet de la réflexion \\[0pt]
L'objet de l'art \\[0pt]
L'objet de l'intention \\[0pt]
L'objet des mathématiques \\[0pt]
L'objet du désir \\[0pt]
L'objet du désir en est-il la cause ? \\[0pt]
L'objet du jugement esthétique \\[0pt]
L'objet et la chose \\[0pt]
L'objet technique \\[0pt]
L'obligation \\[0pt]
L'obligation d'échanger \\[0pt]
L'obligation morale \\[0pt]
L'obligation morale peut-elle se réduire à une obligation sociale ? \\[0pt]
L'obscène \\[0pt]
L'obscénité \\[0pt]
L'obscur \\[0pt]
L'obscurantisme \\[0pt]
L'obscurité \\[0pt]
L'observable et l'inobservable \\[0pt]
L'observation \\[0pt]
L'observation de l'homme \\[0pt]
L'observation participante \\[0pt]
L'obsession \\[0pt]
L'obstacle \\[0pt]
L'obstacle épistémologique \\[0pt]
L'occasion \\[0pt]
Locke \\[0pt]
L'œil du photographe \\[0pt]
L'œil et l'oreille \\[0pt]
L'œuvre \\[0pt]
L'œuvre anonyme \\[0pt]
L'œuvre d'art a-t-elle un sens ? \\[0pt]
L'œuvre d'art doit-elle être belle ? \\[0pt]
L'œuvre d'art doit-elle nous émouvoir ? \\[0pt]
L'œuvre d'art donne-t-elle à penser ? \\[0pt]
L'œuvre d'art échappe-t-elle au monde ? \\[0pt]
L'œuvre d'art échappe-t-elle au temps ? \\[0pt]
L'œuvre d'art échappe-t-elle nécessairement au temps ? \\[0pt]
L'œuvre d'art est-elle anhistorique ? \\[0pt]
L'œuvre d'art est-elle intemporelle ? \\[0pt]
L'œuvre d'art est-elle l'expression d'une idée ? \\[0pt]
L'œuvre d'art est-elle toujours destinée à un public ? \\[0pt]
L'œuvre d'art est-elle une belle apparence ? \\[0pt]
L'œuvre d'art est-elle une marchandise ? \\[0pt]
L'œuvre d'art est-elle un monde ? \\[0pt]
L'œuvre d'art est-elle un objet d'échange ? \\[0pt]
L'œuvre d'art est-elle un symbole ? \\[0pt]
L'œuvre d'art et le plaisir \\[0pt]
L'œuvre d'art et sa reproduction \\[0pt]
L'œuvre d'art et son auteur \\[0pt]
L'œuvre d'art instruit-elle ? \\[0pt]
L'œuvre d'art nous apprend-elle quelque chose ? \\[0pt]
L'œuvre d'art représente-t-elle quelque chose ? \\[0pt]
L'œuvre d'art totale \\[0pt]
L'œuvre d'art traduit-elle une vision du monde ? \\[0pt]
L'œuvre de fiction \\[0pt]
L'œuvre de l'historien \\[0pt]
L'œuvre du temps \\[0pt]
L'œuvre et le produit \\[0pt]
L'œuvre inachevée \\[0pt]
L'œuvre : l'universel et le particulier \\[0pt]
L'œuvre totale \\[0pt]
L'offense \\[0pt]
Logique et dialectique \\[0pt]
Logique et écriture \\[0pt]
Logique et existence \\[0pt]
Logique et grammaire \\[0pt]
Logique et logiques \\[0pt]
Logique et mathématique \\[0pt]
Logique et mathématiques \\[0pt]
Logique et métaphysique \\[0pt]
Logique et méthode \\[0pt]
Logique et ontologie \\[0pt]
Logique et psychologie \\[0pt]
Logique et réalité \\[0pt]
Logique et vérité \\[0pt]
Logique générale et logique transcendantale \\[0pt]
Loi et commandement \\[0pt]
Loi morale et loi politique \\[0pt]
Loi naturelle et loi morale \\[0pt]
Loi naturelle et loi politique \\[0pt]
Loi scientifique et causalité \\[0pt]
Lois et coutumes \\[0pt]
Lois et normes \\[0pt]
Lois et règles en logique \\[0pt]
Lois et valeurs \\[0pt]
Loisir et oisiveté \\[0pt]
L'oisiveté \\[0pt]
Lois naturelles et lois civiles \\[0pt]
L'oligarchie \\[0pt]
L'ombre \\[0pt]
L'ombre et la lumière \\[0pt]
L'omniscience \\[0pt]
L'ontologie peut-elle être relative ? \\[0pt]
L'opéra est-il un art complet ? \\[0pt]
L'opinion \\[0pt]
L'opinion a-t-elle nécessairement tort ? \\[0pt]
L'opinion droite \\[0pt]
L'opinion du citoyen \\[0pt]
L'opinion est-elle un savoir ? \\[0pt]
L'opinion publique \\[0pt]
L'opinion vraie \\[0pt]
L'opportunisme \\[0pt]
L'opportunité \\[0pt]
L'opposant \\[0pt]
L'opposition \\[0pt]
L'optimisme \\[0pt]
L'oral et l'écrit \\[0pt]
L'ordinaire \\[0pt]
L'ordinaire est-il ennuyeux ? \\[0pt]
L'ordinateur \\[0pt]
L'ordre \\[0pt]
L'ordre des choses \\[0pt]
L'ordre des échanges est-il un ordre rationnel ? \\[0pt]
L'ordre des raisons \\[0pt]
L'ordre du monde \\[0pt]
L'ordre du temps \\[0pt]
L'ordre du vivant est-il façonné par le hasard ? \\[0pt]
L'ordre est-il dans les choses ? \\[0pt]
L'ordre établi \\[0pt]
L'ordre et la beauté \\[0pt]
L'ordre et la mesure \\[0pt]
L'ordre et le désordre \\[0pt]
L'ordre international \\[0pt]
L'ordre, le nombre, la mesure \\[0pt]
L'ordre moral \\[0pt]
L'ordre naturel \\[0pt]
L'ordre politique \\[0pt]
L'ordre politique exclut-il la violence ? \\[0pt]
L'ordre politique peut-il exclure la violence ? \\[0pt]
L'ordre public \\[0pt]
L'ordre social \\[0pt]
L'ordre social peut-il être juste ? \\[0pt]
L'ordre symbolique \\[0pt]
L'organique \\[0pt]
L'organique et le mécanique \\[0pt]
L'organique et l'inorganique \\[0pt]
L'organisation \\[0pt]
L'organisation du travail \\[0pt]
L'organisation du vivant \\[0pt]
L'organisme \\[0pt]
L'orgueil \\[0pt]
L'orientation \\[0pt]
L'Orient et l'Occident \\[0pt]
L'original et la copie \\[0pt]
L'originalité \\[0pt]
L'originalité en art \\[0pt]
L'origine \\[0pt]
L'origine de la culpabilité \\[0pt]
L'origine de la famille \\[0pt]
L'origine de la négation \\[0pt]
L'origine de l'art \\[0pt]
L'origine de la violence \\[0pt]
L'origine des croyances \\[0pt]
L'origine des idées \\[0pt]
L'origine des langues \\[0pt]
L'origine des langues est-elle un faux problème ? \\[0pt]
L'origine des valeurs \\[0pt]
L'origine des vertus \\[0pt]
L'origine du droit \\[0pt]
L'origine du langage \\[0pt]
L'origine du mal \\[0pt]
L'origine du symbolisme \\[0pt]
L'origine et le fondement \\[0pt]
L'ornement \\[0pt]
L'orthodoxie \\[0pt]
L'oubli \\[0pt]
L'oubli des fautes \\[0pt]
L'oubli est-il nécessaire à la vie ? \\[0pt]
L'oubli est-il un échec de la mémoire ? \\[0pt]
L'oubli et le pardon \\[0pt]
L'oubli n'est-il qu'une perte de mémoire ? \\[0pt]
L'outil \\[0pt]
L'outil et la machine \\[0pt]
L'outil, l'instrument, la machine \\[0pt]
L'ouverture d'esprit \\[0pt]
L'ouvrier et l'ingénieur \\[0pt]
Lumière et couleur \\[0pt]
L'un \\[0pt]
L'Un \\[0pt]
L'unanimité \\[0pt]
L'unanimité est-elle un critère de légitimité ? \\[0pt]
L'unanimité est-elle un critère de vérité ? \\[0pt]
L'un et le multiple \\[0pt]
L'un et le tout \\[0pt]
L'un et l'être \\[0pt]
L'uniformité \\[0pt]
L'unité \\[0pt]
L'unité dans le beau \\[0pt]
L'unité de la nature \\[0pt]
L'unité de l'art \\[0pt]
L'unité de la science \\[0pt]
L'unité de l'État \\[0pt]
L'unité de l'œuvre d'art \\[0pt]
L'unité des arts \\[0pt]
L'unité des contraires \\[0pt]
L'unité des langues \\[0pt]
L'unité des sciences \\[0pt]
L'unité des sciences humaines \\[0pt]
L'unité des sciences humaines ? \\[0pt]
L'unité des vices \\[0pt]
L'unité du corps politique \\[0pt]
L'unité du genre humain \\[0pt]
L'unité du moi \\[0pt]
L'unité du vivant \\[0pt]
L'univers \\[0pt]
L'universalisme \\[0pt]
L'universalité du vrai \\[0pt]
L'universel \\[0pt]
L'universel en histoire \\[0pt]
L'universel est-il une illusion ? \\[0pt]
L'universel et le particulier \\[0pt]
L'universel et le singulier \\[0pt]
L'universel implique-t-il l'indifférence à la singularité ? \\[0pt]
L'univers infini \\[0pt]
L'univocité de l'étant \\[0pt]
L'un, le vrai, le bien \\[0pt]
L'urbanité \\[0pt]
L'urgence \\[0pt]
L'urgence de vivre \\[0pt]
L'usage \\[0pt]
L'usage de la langue \\[0pt]
L'usage de l'analogie \\[0pt]
L'usage de l'analogie dans les sciences humaines \\[0pt]
L'usage des fictions \\[0pt]
L'usage des généalogies \\[0pt]
L'usage des modèles dans les sciences humaines \\[0pt]
L'usage des mots \\[0pt]
L'usage des passions \\[0pt]
L'usage des principes \\[0pt]
L'usage du concept de race \\[0pt]
L'usage du doute \\[0pt]
L'usage du monde \\[0pt]
L'usure \\[0pt]
L'usure des mots \\[0pt]
L'usure du pouvoir \\[0pt]
L'utile \\[0pt]
L'utile et l'agréable \\[0pt]
L'utile et le beau \\[0pt]
L'utile et le bien \\[0pt]
L'utile et le bon \\[0pt]
L'utile et l'honnête \\[0pt]
L'utile et l'inutile \\[0pt]
L'utilité \\[0pt]
L'utilité de croire \\[0pt]
L'utilité de la peine \\[0pt]
L'utilité de la poésie \\[0pt]
L'utilité de l'art \\[0pt]
L'utilité des préjugés \\[0pt]
L'utilité des sciences humaines \\[0pt]
L'utilité est-elle étrangère à la morale ? \\[0pt]
L'utilité est-elle une valeur morale ? \\[0pt]
L'utilité peut-elle être le principe de la moralité ? \\[0pt]
L'utilité peut-elle être un critère pour juger de la valeur de nos actions ? \\[0pt]
L'utilité publique \\[0pt]
L'utopie \\[0pt]
L'utopie a-t-elle une signification politique ? \\[0pt]
L'utopie a-t-elle un lien avec la pratique ? \\[0pt]
L'utopie en politique \\[0pt]
L'utopie et l'idéologie \\[0pt]
« Machinalement » \\[0pt]
Machine et organisme \\[0pt]
Machines et liberté \\[0pt]
Machines et mémoire \\[0pt]
Ma conscience est-elle digne de confiance ? \\[0pt]
Ma douleur \\[0pt]
Magie et religion \\[0pt]
Maître et disciple \\[0pt]
Maître et serviteur \\[0pt]
Maîtrise et puissance \\[0pt]
Maîtriser l'absence \\[0pt]
Maîtriser la nature \\[0pt]
Maîtriser la technique \\[0pt]
Maîtriser le vivant \\[0pt]
Majorité et minorité \\[0pt]
Maladie de l'âme et maladie du corps \\[0pt]
Maladie et convalescence \\[0pt]
Maladies du corps, maladies de l'âme \\[0pt]
Malaise dans la civilisation \\[0pt]
Mal et liberté \\[0pt]
Mal faire \\[0pt]
« Malheur aux vaincus » \\[0pt]
Malheur aux vaincus \\[0pt]
Ma liberté s'arrête-t-elle où commence celle des autres ? \\[0pt]
Manger \\[0pt]
Manifester \\[0pt]
Manquer de jugement \\[0pt]
Ma parole m'engage-t-elle ? \\[0pt]
Masculin et féminin \\[0pt]
Masculin et Féminin \\[0pt]
Masculin, féminin \\[0pt]
Mass-media et création artistique \\[0pt]
Matérialisme et mécanisme \\[0pt]
Matériel / immatériel \\[0pt]
Mathématique et imagination \\[0pt]
Mathématiques et réalité \\[0pt]
Mathématiques et sciences humaines \\[0pt]
Mathématiques et universel \\[0pt]
Mathématiques pures et mathématiques appliquées \\[0pt]
Matière et corps \\[0pt]
Matière et forme \\[0pt]
Matière et matériaux \\[0pt]
Ma vraie nature \\[0pt]
Mécanisme et finalité \\[0pt]
Médecine et morale \\[0pt]
Médecine et philosophie \\[0pt]
Méditer \\[0pt]
Mémoire et anticipation \\[0pt]
Mémoire et fiction \\[0pt]
Mémoire et identité \\[0pt]
Mémoire et imagination \\[0pt]
Mémoire et responsabilité \\[0pt]
Mémoire et souvenir \\[0pt]
Ménager les apparences \\[0pt]
Mensonge et politique \\[0pt]
Mensonge, vérité, véracité \\[0pt]
Mentir \\[0pt]
Mérite et démocratie \\[0pt]
Mérite-t-on d'être heureux ? \\[0pt]
Mesure et démesure \\[0pt]
Mesurer \\[0pt]
Mesurer le risque \\[0pt]
Mesurer le temps \\[0pt]
Métaphysique et histoire \\[0pt]
Métaphysique et mystique \\[0pt]
Métaphysique et ontologie \\[0pt]
Métaphysique et poésie \\[0pt]
Métaphysique et politique \\[0pt]
Métaphysique et psychologie \\[0pt]
Métaphysique et religion \\[0pt]
Métaphysique et théologie \\[0pt]
Métaphysique spéciale, métaphysique générale \\[0pt]
Méthode et métaphysique \\[0pt]
Métier et vocation \\[0pt]
Mettre en commun \\[0pt]
Mettre en ordre \\[0pt]
Microscope et télescope \\[0pt]
Mieux vaut subir que commettre l'injustice \\[0pt]
Milieux naturels et milieux sociaux \\[0pt]
Misère et pauvreté \\[0pt]
Modèle et copie \\[0pt]
Modèle, type et paradigme \\[0pt]
Mœurs, coutumes, lois \\[0pt]
Mœurs et moralité \\[0pt]
Moi d'abord \\[0pt]
Mon corps \\[0pt]
Mon corps est-il ma propriété ? \\[0pt]
Mon corps est-il naturel ? \\[0pt]
Mon corps fait-il obstacle à ma liberté ? \\[0pt]
Mon corps m'appartient-il ? \\[0pt]
Monde et nature \\[0pt]
Monde naturel et monde humain \\[0pt]
Monde perçu et monde conçu \\[0pt]
Mon devoir dépend-il de moi ? \\[0pt]
Mon existence est-elle contingente ? \\[0pt]
Monologue et dialogue \\[0pt]
Mon passé m'appartient-il ? \\[0pt]
Mon prochain est-il mon semblable ? \\[0pt]
Mon semblable \\[0pt]
Montrer, est-ce démontrer ? \\[0pt]
Montrer et démontrer \\[0pt]
Montrer et dire \\[0pt]
Montrer l'exemple \\[0pt]
Morale et calcul \\[0pt]
Morale et convention \\[0pt]
Morale et dispositions \\[0pt]
Morale et économie \\[0pt]
Morale et éducation \\[0pt]
Morale et histoire \\[0pt]
Morale et identité \\[0pt]
Morale et intérêt \\[0pt]
Morale et liberté \\[0pt]
Morale et politique sont-elles indépendantes ? \\[0pt]
Morale et pratique \\[0pt]
Morale et prudence \\[0pt]
Morale et religion \\[0pt]
Morale et science des mœurs \\[0pt]
Morale et sentiments \\[0pt]
Morale et sexualité \\[0pt]
Morale et société \\[0pt]
Morale et technique \\[0pt]
Morale et violence \\[0pt]
Moralité et connaissance \\[0pt]
Moralité et utilité \\[0pt]
Mourir \\[0pt]
Mourir dans la dignité \\[0pt]
Mourir pour des principes \\[0pt]
Mourir pour la patrie \\[0pt]
Mouvement et changement \\[0pt]
Murs et frontières \\[0pt]
Musique et architecture \\[0pt]
Musique et bruit \\[0pt]
Musique et émotion \\[0pt]
Musique et poésie \\[0pt]
Musique et rhétorique \\[0pt]
Mythe et connaissance \\[0pt]
Mythe et histoire \\[0pt]
Mythe et pensée \\[0pt]
Mythe et philosophie \\[0pt]
Mythe et symbole \\[0pt]
Mythe et vérité \\[0pt]
Mythes et idéologies \\[0pt]
N'aimer que soi \\[0pt]
Naît-on sujet ou le devient-on ? \\[0pt]
Naître \\[0pt]
Naître et mourir \\[0pt]
N'apprend-on que par l'expérience ? \\[0pt]
Narration et identité \\[0pt]
Nation et richesse \\[0pt]
N'a-t-on des devoirs qu'envers autrui ? \\[0pt]
Naturaliser l'esprit \\[0pt]
Nature et artifice \\[0pt]
Nature et convention \\[0pt]
Nature et culture \\[0pt]
Nature et fonction du sacrifice \\[0pt]
Nature et histoire \\[0pt]
Nature et institution \\[0pt]
Nature et institutions \\[0pt]
Nature et liberté \\[0pt]
Nature et loi \\[0pt]
Nature et mesure du temps \\[0pt]
Nature et monde \\[0pt]
Nature et morale \\[0pt]
Nature et nature humaine \\[0pt]
Naturel et artificiel \\[0pt]
Nature, monde, univers \\[0pt]
Naviguer \\[0pt]
N'avons-nous affaire qu'au réel ? \\[0pt]
N'avons-nous de devoir qu'envers autrui ? \\[0pt]
N'avons-nous de devoirs qu'envers autrui ? \\[0pt]
Nécessaire / contingent \\[0pt]
Nécessité et contingence \\[0pt]
Nécessité et fatalité \\[0pt]
Nécessité et liberté \\[0pt]
Nécessité fait loi \\[0pt]
N'échange-t-on que ce qui a de la valeur ? \\[0pt]
N'échange-t-on que des biens ? \\[0pt]
N'échange-t-on que des symboles ? \\[0pt]
N'échange-t-on que par intérêt ? \\[0pt]
Ne désire-t-on que ce dont on manque ? \\[0pt]
Ne désirons-nous que ce qui est bon pour nous ? \\[0pt]
Ne désirons-nous que les choses que nous estimons bonnes ? \\[0pt]
Ne doit-on tenir pour vrai que ce qui est démontré ? \\[0pt]
Ne faire que son devoir \\[0pt]
« Ne fais pas à autrui ce que tu ne voudrais pas qu'on te fasse » \\[0pt]
Ne faut-il pas craindre la liberté ? \\[0pt]
Ne faut-il vivre que dans le présent ? \\[0pt]
Ne faut-il vivre que pour faire son devoir ? \\[0pt]
Négation et privation \\[0pt]
Ne lèse personne \\[0pt]
N'en faire qu'à sa tête \\[0pt]
Ne pas choisir \\[0pt]
Ne pas comprendre \\[0pt]
Ne pas multiplier en vain les entités \\[0pt]
Ne pas raconter d'histoires \\[0pt]
Ne pas rire, ne pas pleurer, mais comprendre \\[0pt]
Ne pas savoir ce que l'on fait \\[0pt]
Ne pas tuer \\[0pt]
Ne penser à rien \\[0pt]
Ne penser qu'à soi \\[0pt]
Ne prêche-t-on que les convertis ? \\[0pt]
Ne rien devoir à personne \\[0pt]
Ne sait-on rien que par expérience ? \\[0pt]
Ne sommes-nous justes que par intérêt ? \\[0pt]
Ne sommes-nous véritablement maîtres que de nos pensées ? \\[0pt]
N'est-il d'inférence que démonstrative ? \\[0pt]
N'est-on juste que par crainte du châtiment ? \\[0pt]
Ne travaille-t-on que par nécessité ? \\[0pt]
Ne travaille-t- on que pour un salaire ? \\[0pt]
Ne veut-on que ce qui est désirable ? \\[0pt]
Ne vit-on bien qu'avec ses amis ? \\[0pt]
Névroses et psychoses \\[0pt]
N'existe-t-il que des individus ? \\[0pt]
N'existe-t-il que le présent ? \\[0pt]
N'existe-t-il qu'un seul temps ? \\[0pt]
N'exprime-t-on que ce dont on a conscience ? \\[0pt]
« Ni Dieu ni maître » \\[0pt]
« Ni Dieu, ni maître ! » \\[0pt]
Ni Dieu ni maître \\[0pt]
Ni Dieu, ni maître \\[0pt]
Nier la liberté de l'homme, est-ce ôter tout sens à la morale ? \\[0pt]
Nier la vérité \\[0pt]
Nier le libre arbitre, est-ce nier la liberté ? \\[0pt]
Nier le monde \\[0pt]
Nier l'évidence \\[0pt]
N'interprète-t-on que ce qui est équivoque ? \\[0pt]
Ni regrets, ni remords \\[0pt]
Nomade et sédentaire \\[0pt]
Nommer \\[0pt]
Nom propre et nom commun \\[0pt]
Non-sens et absurdité \\[0pt]
Normalité et normativité \\[0pt]
Norme et moyenne \\[0pt]
Normes et valeurs \\[0pt]
Normes morales et normes vitales \\[0pt]
« Nos amis les animaux » \\[0pt]
Nos convictions morales sont-elles le simple reflet de notre temps ? \\[0pt]
Nos désirs nous appartiennent-ils ? \\[0pt]
Nos désirs nous opposent-ils ? \\[0pt]
Nos devoirs nous motivent-ils ? \\[0pt]
Nos devoirs se ramènent-ils aux conventions sociales ? \\[0pt]
Nos goûts sont-ils justifiables ? \\[0pt]
Nos habitudes compromettent-elles notre moralité ? \\[0pt]
Nos paroles peuvent-elles dépasser notre pensée ? \\[0pt]
Nos pensées dépendent-elles de nous ? \\[0pt]
Nos pensées sont-elles entièrement en notre pouvoir ? \\[0pt]
Nos sens nous font-ils connaître la matière ? \\[0pt]
Nos sens nous trompent-ils ? \\[0pt]
Nos sens peuvent-ils s'éduquer ? \\[0pt]
Notre besoin de fictions \\[0pt]
Notre connaissance du réel se limite-t-elle au savoir scientifique ? \\[0pt]
Notre corps pense-t-il ? \\[0pt]
Notre existence a-t-elle un sens si l'histoire n'en a pas ? \\[0pt]
Notre ignorance nous excuse-t-elle ? \\[0pt]
Notre liberté de pensée a-t-elle des limites ? \\[0pt]
Notre liberté est-elle toujours relative ? \\[0pt]
Notre pensée est-elle prisonnière de la langue que nous parlons ? \\[0pt]
Notre rapport au monde est-il essentiellement technique ? \\[0pt]
Notre rapport au monde peut-il être exclusivement technique ? \\[0pt]
Notre rapport au monde peut-il n'être que technique ? \\[0pt]
Nous et les autres \\[0pt]
Nous trouvons-nous nous-mêmes dans l'animal ? \\[0pt]
Nouveauté et tradition \\[0pt]
« Nul n'est censé ignorer la loi » \\[0pt]
Nul n'est censé ignorer la loi \\[0pt]
Nul n'est méchant volontairement \\[0pt]
N'y a-t-il d'amitié qu'entre égaux ? \\[0pt]
N'y a-t-il de beauté qu'artistique ? \\[0pt]
N'y a-t-il de bonheur que dans l'instant ? \\[0pt]
N'y a-t-il de bonheur qu'éphémère ? \\[0pt]
N'y a-t-il de certitude que mathématique ? \\[0pt]
N'y a-t-il de connaissance que de l'universel ? \\[0pt]
N'y a-t-il de connaissance que par les causes ? \\[0pt]
N'y a-t-il de connaissance que scientifique ? \\[0pt]
N'y a-t-il de démocratie que représentative ? \\[0pt]
N'y a-t-il de devoirs qu'envers autrui ? \\[0pt]
N'y a-t-il de droit qu'écrit ? \\[0pt]
N'y a-t-il de droit que de l'homme ? \\[0pt]
N'y a-t-il de droits que de l'homme ? \\[0pt]
N'y a-t-il de foi que religieuse ? \\[0pt]
N'y a-t-il de liberté qu'individuelle ? \\[0pt]
N'y a-t-il de propriété que privée ? \\[0pt]
N'y a-t-il de rationalité que scientifique ? \\[0pt]
N'y a-t-il de réalité que de l'individuel ? \\[0pt]
N'y a-t-il de réel que le présent ? \\[0pt]
N'y a-t-il de savoir que livresque ? \\[0pt]
N'y a-t-il de science qu'autant qu'il s'y trouve de mathématique ? \\[0pt]
N'y a-t-il de science que de ce qui est mathématisable ? \\[0pt]
N'y a-t-il de science que du général ? \\[0pt]
N'y a-t-il de science que du mesurable ? \\[0pt]
N'y a-t-il de science que du nécessaire ? \\[0pt]
N'y a-t-il de science que provisoire ? \\[0pt]
N'y a-t-il de science que systématique ? \\[0pt]
N'y a-t-il de science qu'exacte ? \\[0pt]
N'y a-t-il des droits que de l'homme ? \\[0pt]
N'y a-t-il de sens que par le langage ? \\[0pt]
N'y a-t-il d'être que sensible ? \\[0pt]
N'y a-t-il de vérité que scientifique ? \\[0pt]
N'y a-t-il de vérité que vérifiable ? \\[0pt]
N'y a-t-il de vérités que scientifiques ? \\[0pt]
N'y a-t-il de vrai que le vérifiable ? \\[0pt]
N'y a-t-il d'origine que mythique ? \\[0pt]
N'y a-t-il que des individus ? \\[0pt]
N'y a-t-il que du mesurable ? \\[0pt]
N'y a-t-il que l'intention qui compte ? \\[0pt]
N'y a-t-il qu'une substance ? \\[0pt]
N'y a-t-il qu'un seul monde ? \\[0pt]
N'y a-t-il rien au monde de certain ? \\[0pt]
Obéir \\[0pt]
Obéir aux lois, est-ce renoncer à la liberté ? \\[0pt]
Obéir, est-ce se soumettre ? \\[0pt]
Obéir et désobéir \\[0pt]
Obéissance et esclavage \\[0pt]
Obéissance et liberté \\[0pt]
Obéissance et servitude \\[0pt]
Obéissance et soumission \\[0pt]
Objecter et répondre \\[0pt]
Objectivé et subjectivité \\[0pt]
Objectiver \\[0pt]
Objectivité et impartialité \\[0pt]
Objectivité et subjectivité \\[0pt]
Objet d'art, objet d'échange \\[0pt]
Objet et œuvre \\[0pt]
Observation et expérience \\[0pt]
Observation et expérimentation \\[0pt]
Observer \\[0pt]
Observer, décrire et expliquer \\[0pt]
Observer et comprendre \\[0pt]
Observer et expérimenter \\[0pt]
Observer et interpréter \\[0pt]
« Œil pour œil, dent pour dent » \\[0pt]
Œil pour œil, dent pour dent \\[0pt]
Œuvre et événement \\[0pt]
Œuvrer \\[0pt]
Offre et demande \\[0pt]
On dit \\[0pt]
« On n'arrête pas le progrès » \\[0pt]
Ontologie et métaphysique \\[0pt]
Ontologie et théologie \\[0pt]
Opinion, croyance, jugement \\[0pt]
Opinion et ignorance \\[0pt]
Opinion publique et conscience universelle \\[0pt]
Optimisme et pessimisme \\[0pt]
Ordre et chaos \\[0pt]
Ordre et désordre \\[0pt]
Ordre et justice \\[0pt]
Ordre et liberté \\[0pt]
Ordre et progrès \\[0pt]
Organe et fonction \\[0pt]
Organiser \\[0pt]
Organisme et milieu \\[0pt]
Origine des hommes et principe de l'humanité \\[0pt]
Origine et commencement \\[0pt]
Origine et fondement \\[0pt]
Oublier \\[0pt]
Où commence la liberté ? \\[0pt]
Où commence la servitude ? \\[0pt]
Où commence la vie privée ? \\[0pt]
Où commence la violence ? \\[0pt]
Où commence le territoire de la psychologie ? \\[0pt]
Où commence l'interprétation ? \\[0pt]
Où commence ma liberté ? \\[0pt]
Où est le danger ? \\[0pt]
Où est le passé ? \\[0pt]
Où est le pouvoir ? \\[0pt]
Où est l'esprit ? \\[0pt]
Où est mon esprit ? \\[0pt]
Où est-on quand on pense ? \\[0pt]
Où s'arrête la responsabilité ? \\[0pt]
Où s'arrête l'espace public ? \\[0pt]
Où sont les relations ? \\[0pt]
Où suis-je ? \\[0pt]
Où suis-je quand je pense ? \\[0pt]
Outil et machine \\[0pt]
Outil et organe \\[0pt]
Paraître \\[0pt]
Par-delà beauté et laideur \\[0pt]
Pardonner \\[0pt]
Pardonner et oublier \\[0pt]
Parfaire \\[0pt]
« Par hasard » \\[0pt]
Parier \\[0pt]
Par le langage, peut-on agir sur la réalité ? \\[0pt]
Parler de Dieu, est-ce nécessairement tenir un discours religieux ? \\[0pt]
Parler de soi \\[0pt]
Parler de soi est-il intéressant ? \\[0pt]
Parler, écrire, penser \\[0pt]
Parler, est-ce agir ? \\[0pt]
Parler, est-ce communiquer ? \\[0pt]
Parler, est-ce donner sa parole ? \\[0pt]
Parler, est-ce ne pas agir ? \\[0pt]
Parler et agir \\[0pt]
Parler et penser \\[0pt]
Parler, n'est-ce que désigner ? \\[0pt]
Parler pour ne rien dire \\[0pt]
Parler pour quelqu'un \\[0pt]
Parler tout seul \\[0pt]
Parler vrai \\[0pt]
Parole et pouvoir \\[0pt]
Paroles et actes \\[0pt]
Par où commencer ? \\[0pt]
Par quoi un individu diffère-t-il réellement d'un autre ? \\[0pt]
Par quoi un individu se distingue-t-il d'un autre ? \\[0pt]
Partager \\[0pt]
Partager les richesses \\[0pt]
Partager sa vie \\[0pt]
Partager ses sentiments \\[0pt]
Participation et imitation \\[0pt]
« Pas de liberté pour les ennemis de la liberté » \\[0pt]
« Pas de liberté pour les ennemis de la liberté » ? \\[0pt]
Passer du fait au droit \\[0pt]
Passer le temps \\[0pt]
Passions et intérêts \\[0pt]
Passions, intérêt, raison \\[0pt]
Pâtir \\[0pt]
« Pauvre bête » \\[0pt]
Peindre \\[0pt]
Peindre d'après nature \\[0pt]
Peindre, est-ce nécessairement feindre ? \\[0pt]
Peindre, est-ce traduire ? \\[0pt]
Peindre la présence \\[0pt]
Peindre un paysage, est-ce représenter la nature ? \\[0pt]
Peinture et abstraction \\[0pt]
Peinture et histoire \\[0pt]
Peinture et photographie \\[0pt]
Peinture et réalité \\[0pt]
Peinture et vérité \\[0pt]
Pensée et calcul \\[0pt]
Pensée et réalité \\[0pt]
Penser à rien \\[0pt]
« Penser, c'est dire non » \\[0pt]
Penser, est-ce aller contre l'évidence ? \\[0pt]
Penser, est-ce calculer ? \\[0pt]
Penser, est-ce comparer ? \\[0pt]
Penser, est-ce désobéir ? \\[0pt]
Penser, est-ce dire non ? \\[0pt]
Penser, est-ce se parler à soi-même ? \\[0pt]
Penser est-il assimilable à un travail ? \\[0pt]
Penser et calculer \\[0pt]
Penser et connaître \\[0pt]
Penser et être \\[0pt]
Penser et imaginer \\[0pt]
Penser et parler \\[0pt]
Penser et raisonner \\[0pt]
Penser et savoir \\[0pt]
Penser et sentir \\[0pt]
Penser la matière \\[0pt]
Penser la technique \\[0pt]
Penser l'avenir \\[0pt]
Penser le changement \\[0pt]
Penser le devenir \\[0pt]
Penser le mouvement \\[0pt]
Penser le réel \\[0pt]
Penser le rien, est-ce ne rien penser ? \\[0pt]
Penser les sociétés comme des organismes \\[0pt]
Penser l'impossible \\[0pt]
Penser l'infini \\[0pt]
Penser par soi-même \\[0pt]
Penser par soi-même, est-ce être l'auteur de ses pensées ? \\[0pt]
Penser peut-il nous rendre heureux ? \\[0pt]
Penser requiert-il d'avoir un corps ? \\[0pt]
Penser requiert-il un corps ? \\[0pt]
Penser sans corps \\[0pt]
Penser sans sujet \\[0pt]
Pense-t-on jamais seul ? \\[0pt]
Pensez-vous que vous avez une âme ? \\[0pt]
Perception et aperception \\[0pt]
Perception et connaissance \\[0pt]
Perception et création artistique \\[0pt]
Perception et imagination \\[0pt]
Perception et interprétation \\[0pt]
Perception et jugement \\[0pt]
Perception et mémoire \\[0pt]
Perception et mouvement \\[0pt]
Perception et observation \\[0pt]
Perception et passivité \\[0pt]
Perception et sensation \\[0pt]
Perception et souvenir \\[0pt]
Perception et vérité \\[0pt]
Percer les apparences \\[0pt]
Percevoir \\[0pt]
Percevoir, est-ce connaître ? \\[0pt]
Percevoir, est-ce interpréter ? \\[0pt]
Percevoir, est-ce juger ? \\[0pt]
Percevoir, est-ce nécessaire pour penser ? \\[0pt]
Percevoir, est-ce reconnaître ? \\[0pt]
Percevoir, est-ce savoir ? \\[0pt]
Percevoir, est-ce s'ouvrir au monde ? \\[0pt]
Percevoir et concevoir \\[0pt]
Percevoir et imaginer \\[0pt]
Percevoir et juger \\[0pt]
Percevoir et sentir \\[0pt]
Percevoir et témoigner \\[0pt]
Percevoir s'apprend-il ? \\[0pt]
Percevons-nous le monde extérieur ? \\[0pt]
Percevons-nous les choses telles qu'elles sont ? \\[0pt]
Perçoit-on le réel ? \\[0pt]
Perçoit-on le réel tel qu'il est ? \\[0pt]
Perçoit-on les choses comme elles sont ? \\[0pt]
Perdre connaissance \\[0pt]
Perdre la face \\[0pt]
Perdre la mémoire \\[0pt]
Perdre la raison \\[0pt]
Perdre le contrôle \\[0pt]
Perdre ses habitudes \\[0pt]
Perdre ses illusions \\[0pt]
Perdre son âme \\[0pt]
Perdre son identité \\[0pt]
Perdre son temps \\[0pt]
Permanence et identité \\[0pt]
Permanent / successif \\[0pt]
Permettre \\[0pt]
Persévérer dans son être \\[0pt]
Persister \\[0pt]
Personne et individu \\[0pt]
Personne n'est innocent \\[0pt]
Persuader \\[0pt]
Persuader et convaincre \\[0pt]
« Petites causes, grands effets » \\[0pt]
Peuple et culture \\[0pt]
Peuple et masse \\[0pt]
Peuple et multitude \\[0pt]
Peuple et société \\[0pt]
Peuples et masses \\[0pt]
Peut-il être moral de tuer ? \\[0pt]
Peut-il être préférable de ne pas savoir ? \\[0pt]
Peut-il exister une action désintéressée ? \\[0pt]
Peut-il exister une morale sceptique ? \\[0pt]
Peut-il y avoir conflit entre nos devoirs ? \\[0pt]
Peut-il y avoir de bonnes raisons de croire ? \\[0pt]
Peut-il y avoir de bons tyrans ? \\[0pt]
Peut-il y avoir de la politique sans conflit ? \\[0pt]
Peut-il y avoir des choix collectifs ? \\[0pt]
Peut-il y avoir des conflits de devoirs ? \\[0pt]
Peut-il y avoir des degrés dans la vérité ? \\[0pt]
Peut-il y avoir des échanges équitables ? \\[0pt]
Peut-il y avoir des expériences métaphysiques ? \\[0pt]
Peut-il y avoir des lois de l'histoire ? \\[0pt]
Peut-il y avoir des lois injustes ? \\[0pt]
Peut-il y avoir des modèles en morale ? \\[0pt]
Peut-il y avoir des nations sans État ? \\[0pt]
Peut-il y avoir des vérités partielles ? \\[0pt]
Peut-il y avoir esprit sans corps ? \\[0pt]
Peut-il y avoir plusieurs vérités religieuses ? \\[0pt]
Peut-il y avoir savoir-faire sans savoir ? \\[0pt]
Peut-il y avoir science sans intuition du vrai ? \\[0pt]
Peut-il y avoir un art conceptuel ? \\[0pt]
Peut-il y avoir un bien commun ? \\[0pt]
Peut-il y avoir un droit à désobéir ? \\[0pt]
Peut-il y avoir un droit de la guerre ? \\[0pt]
Peut-il y avoir une anthropologie non anthropocentriste ? \\[0pt]
Peut-il y avoir une ethnologie non ethnocentriste ? \\[0pt]
Peut-il y avoir une histoire universelle ? \\[0pt]
Peut-il y avoir une méthode commune à toutes les sciences ? \\[0pt]
Peut-il y avoir une morale dans les échanges ? \\[0pt]
Peut-il y avoir une pensée sans langage ? \\[0pt]
Peut-il y avoir une philosophie applicable ? \\[0pt]
Peut-il y avoir une philosophie appliquée ? \\[0pt]
Peut-il y avoir une philosophie politique sans Dieu ? \\[0pt]
Peut-il y avoir une politique de la langue ? \\[0pt]
Peut-il y avoir une pratique sans théorie ? \\[0pt]
Peut-il y avoir une science de la morale ? \\[0pt]
Peut-il y avoir une science de l'éducation ? \\[0pt]
Peut-il y avoir une science de l'homme ? \\[0pt]
Peut-il y avoir une science de l'inconscient ? \\[0pt]
Peut-il y avoir une science des signes ? \\[0pt]
Peut-il y avoir une science du désordre ? \\[0pt]
Peut-il y avoir une science du moi ? \\[0pt]
Peut-il y avoir une science politique ? \\[0pt]
Peut-il y avoir une servitude volontaire ? \\[0pt]
Peut-il y avoir une société des nations ? \\[0pt]
Peut-il y avoir une société sans État ? \\[0pt]
Peut-il y avoir un État mondial ? \\[0pt]
Peut-il y avoir une vérité en art ? \\[0pt]
Peut-il y avoir une vérité en politique ? \\[0pt]
Peut-il y avoir une vérité religieuse ? \\[0pt]
Peut-il y avoir un intérêt collectif ? \\[0pt]
Peut-il y avoir un langage universel ? \\[0pt]
Peut-on abolir la religion ? \\[0pt]
Peut-on admettre ce qui n'est pas vérifié ? \\[0pt]
Peut-on admettre un droit à la révolte ? \\[0pt]
Peut-on affirmer que le monde a un ordre ? \\[0pt]
Peut-on affirmer qu'être, c'est être perçu ou percevoir ? \\[0pt]
Peut-on agir de manière désintéressée ? \\[0pt]
Peut-on agir machinalement ? \\[0pt]
Peut-on agir sans prendre de risques ? \\[0pt]
Peut-on agir sans raison ? \\[0pt]
Peut-on aimer ce qu'on ne connaît pas ? \\[0pt]
Peut-on aimer l'autre tel qu'il est ? \\[0pt]
Peut-on aimer la vie plus que tout ? \\[0pt]
Peut-on aimer les animaux ? \\[0pt]
Peut-on aimer l'humanité ? \\[0pt]
Peut-on aimer sans perdre sa liberté ? \\[0pt]
Peut-on aimer son prochain comme soi-même ? \\[0pt]
Peut-on aimer son travail ? \\[0pt]
Peut-on aimer une œuvre d'art sans la comprendre ? \\[0pt]
Peut-on à la fois préserver et dominer la nature ? \\[0pt]
Peut-on aller à l'encontre de la nature ? \\[0pt]
Peut-on appréhender les choses telles qu'elles sont ? \\[0pt]
Peut-on apprendre à être heureux ? \\[0pt]
Peut-on apprendre à être juste ? \\[0pt]
Peut-on apprendre à être libre ? \\[0pt]
Peut-on apprendre à mourir ? \\[0pt]
Peut-on apprendre à penser ? \\[0pt]
Peut-on apprendre à vivre ? \\[0pt]
Peut-on argumenter en morale ? \\[0pt]
Peut-on arrêter le progrès ? \\[0pt]
Peut-on assimiler le vivant à une machine ? \\[0pt]
Peut-on atteindre une certitude ? \\[0pt]
Peut-on attendre de la politique qu'elle soit conforme aux exigences de la raison ? \\[0pt]
Peut-on attribuer à chacun son dû ? \\[0pt]
Peut-on avoir conscience de soi sans avoir conscience d'autrui ? \\[0pt]
Peut-on avoir de bonnes raisons de ne pas dire la vérité ? \\[0pt]
Peut-on avoir des droits sans avoir de devoirs ? \\[0pt]
Peut-on avoir le droit de se révolter ? \\[0pt]
Peut-on avoir peur de la liberté ? \\[0pt]
Peut-on avoir peur de soi-même ? \\[0pt]
Peut-on avoir peur d'être libre ? \\[0pt]
Peut-on avoir raison contre la science ? \\[0pt]
Peut-on avoir raison contre les faits ? \\[0pt]
Peut-on avoir raison contre tous ? \\[0pt]
Peut-on avoir raison contre tout le monde ? \\[0pt]
Peut-on avoir raisons contre les faits ? \\[0pt]
Peut-on avoir raison tout.e seul.e ? \\[0pt]
Peut-on avoir raison tout seul ? \\[0pt]
Peut-on avoir trop d'imagination ? \\[0pt]
Peut-on bien agir sans le savoir ? \\[0pt]
Peut-on cesser de croire ? \\[0pt]
Peut-on cesser de désirer ? \\[0pt]
Peut-on changer au point de ne plus être le même ? \\[0pt]
Peut-on changer de culture ? \\[0pt]
Peut-on changer de logique ? \\[0pt]
Peut-on changer le cours de l'histoire ? \\[0pt]
Peut-on changer le monde ? \\[0pt]
Peut-on changer le passé ? \\[0pt]
Peut-on changer ses désirs ? \\[0pt]
Peut-on changer ses désirs à volonté ? \\[0pt]
Peut-on choisir de renoncer à sa liberté ? \\[0pt]
Peut-on choisir le mal ? \\[0pt]
Peut-on choisir sa vie ? \\[0pt]
Peut-on choisir ses désirs ? \\[0pt]
Peut-on classer les arts ? \\[0pt]
Peut-on combattre une croyance par le raisonnement ? \\[0pt]
Peut-on commander à la nature ? \\[0pt]
Peut-on communiquer ses perceptions à autrui ? \\[0pt]
Peut-on communiquer son expérience ? \\[0pt]
Peut-on comparer deux philosophies ? \\[0pt]
Peut-on comparer les cultures ? \\[0pt]
Peut-on comparer l'organisme à une machine ? \\[0pt]
Peut-on comprendre ce qui est illogique ? \\[0pt]
Peut-on comprendre le présent ? \\[0pt]
Peut-on comprendre les actions humaines ? \\[0pt]
Peut-on comprendre un acte que l'on désapprouve ? \\[0pt]
Peut-on comprendre un auteur mieux qu'il ne s'est compris lui-même ? \\[0pt]
Peut-on concevoir la science achevée ? \\[0pt]
Peut-on concevoir une humanité sans art ? \\[0pt]
Peut-on concevoir une morale sans sanction ni obligation ? \\[0pt]
Peut-on concevoir une religion dans les limites de la simple raison ? \\[0pt]
Peut-on concevoir une science qui ne soit pas démonstrative ? \\[0pt]
Peut-on concevoir une science sans expérience ? \\[0pt]
Peut-on concevoir une société juste sans que les hommes ne le soient ? \\[0pt]
Peut-on concevoir une société qui n'aurait plus besoin du droit ? \\[0pt]
Peut-on concevoir une société sans État ? \\[0pt]
Peut-on concevoir un État mondial ? \\[0pt]
Peut-on concilier bonheur et liberté ? \\[0pt]
Peut-on concilier le droit à la liberté et l'égalité des droits ? \\[0pt]
Peut-on concilier nécessité et liberté ? \\[0pt]
Peut-on conclure de l'être au devoir-être ? \\[0pt]
Peut-on connaître autrui ? \\[0pt]
Peut-on connaître les causes ? \\[0pt]
Peut-on connaître les choses telles qu'elles sont ? \\[0pt]
Peut-on connaître le singulier ? \\[0pt]
Peut-on connaître l'esprit ? \\[0pt]
Peut-on connaître le vivant sans le dénaturer ? \\[0pt]
Peut-on connaître le vivant sans recourir à la notion de finalité ? \\[0pt]
Peut-on connaître l'individuel ? \\[0pt]
Peut-on connaître par intuition ? \\[0pt]
Peut-on connaître sans concept ? \\[0pt]
Peut-on connaître une vie humaine ? \\[0pt]
Peut-on considérer la conscience comme une chose ? \\[0pt]
Peut-on considérer la main comme un outil ? \\[0pt]
Peut-on considérer l'art comme un langage ? \\[0pt]
Peut-on considérer les hommes de science comme des autorités morales ? \\[0pt]
Peut-on contester les droits de l'homme ? \\[0pt]
Peut-on contredire l'expérience ? \\[0pt]
Peut-on convaincre quelqu'un de la beauté d'une œuvre d'art ? \\[0pt]
Peut-on craindre la liberté ? \\[0pt]
Peut-on créer un homme nouveau ? \\[0pt]
Peut-on critiquer la démocratie ? \\[0pt]
Peut-on critiquer la religion ? \\[0pt]
Peut-on croire ce qu'on veut ? \\[0pt]
Peut-on croire en rien ? \\[0pt]
Peut-on croire librement ? \\[0pt]
Peut-on croire sans être crédule ? \\[0pt]
Peut-on croire sans savoir pourquoi ? \\[0pt]
Peut-on décider de croire ? \\[0pt]
Peut-on décider de tout ? \\[0pt]
Peut-on décider d'être heureux ? \\[0pt]
Peut-on définir ce qu'est un progrès technique ? \\[0pt]
Peut-on définir la matière ? \\[0pt]
Peut-on définir la morale comme l'art d'être heureux ? \\[0pt]
Peut-on définir la santé ? \\[0pt]
Peut-on définir la vérité ? \\[0pt]
Peut-on définir la vérité par la correspondance ? \\[0pt]
Peut-on définir la vie ? \\[0pt]
Peut-on définir le bien ? \\[0pt]
Peut-on définir le bonheur ? \\[0pt]
Peut-on délimiter le réel ? \\[0pt]
Peut-on délimiter l'humain ? \\[0pt]
Peut-on démontrer l'existence de la matière ? \\[0pt]
Peut-on démontrer que l'on est libre ? \\[0pt]
Peut-on démontrer qu'on ne rêve pas ? \\[0pt]
Peut-on démontrer une théorie ? \\[0pt]
Peut-on dépasser la subjectivité ? \\[0pt]
Peut-on dériver des concepts de l'expérience ? \\[0pt]
Peut-on désirer autre chose qu'être heureux ? \\[0pt]
Peut-on désirer ce qui est ? \\[0pt]
Peut-on désirer ce qu'on ne veut pas ? \\[0pt]
Peut-on désirer ce qu'on possède ? \\[0pt]
Peut-on désirer ce qu'on sait être impossible ? \\[0pt]
Peut-on désirer l'absolu ? \\[0pt]
Peut-on désirer l'impossible ? \\[0pt]
Peut-on désirer sans souffrir ? \\[0pt]
Peut-on désobéir à l'État ? \\[0pt]
Peut-on désobéir aux lois ? \\[0pt]
Peut-on désobéir par devoir ? \\[0pt]
Peut-on dialoguer avec soi-même ? \\[0pt]
Peut-on dialoguer avec un ordinateur ? \\[0pt]
Peut-on dire ce que l'on pense ? \\[0pt]
Peut-on dire ce qui n'est pas ? \\[0pt]
Peut-on dire de la connaissance scientifique qu'elle procède par approximation ? \\[0pt]
Peut-on dire de l'art qu'il donne un monde en partage ? \\[0pt]
Peut-on dire d'une image qu'elle parle ? \\[0pt]
Peut-on dire d'une œuvre d'art qu'elle est ratée ? \\[0pt]
Peut-on dire d'une théorie scientifique qu'elle n'est jamais plus vraie qu'une autre mais seulement plus commode ? \\[0pt]
Peut-on dire d'un homme qu'il est supérieur à un autre homme ? \\[0pt]
Peut-on dire la vérité ? \\[0pt]
Peut-on dire le singulier ? \\[0pt]
Peut-on dire l'être ? \\[0pt]
Peut-on dire que la science désenchante le monde ? \\[0pt]
Peut-on dire que la science ne nous fait pas connaître les choses mais les rapports entre les choses ? \\[0pt]
Peut-on dire que les hommes font l'histoire ? \\[0pt]
Peut-on dire que les machines travaillent pour nous ? \\[0pt]
Peut-on dire que les mots pensent pour nous ? \\[0pt]
Peut-on dire que l'humanité progresse ? \\[0pt]
Peut-on dire que rien n'échappe à la technique ? \\[0pt]
Peut-on dire qu'est vrai ce qui correspond aux faits ? \\[0pt]
Peut-on dire que toutes les croyances se valent ? \\[0pt]
Peut-on dire que tout est politique ? \\[0pt]
Peut-on dire que tout est relatif ? \\[0pt]
Peut-on dire qu'une théorie physique en contredit une autre ? \\[0pt]
Peut-on dire toute la vérité ? \\[0pt]
Peut-on discuter des goûts et des couleurs ? \\[0pt]
Peut-on disposer de son corps ? \\[0pt]
Peut-on distinguer différents types de causes ? \\[0pt]
Peut-on distinguer entre de bons et de mauvais désirs ? \\[0pt]
Peut-on distinguer entre les bons et les mauvais désirs ? \\[0pt]
Peut-on distinguer la vie et l'existence ? \\[0pt]
Peut-on distinguer le réel de l'imaginaire ? \\[0pt]
Peut-on distinguer les faits de leurs interprétations ? \\[0pt]
Peut-on distinguer le vrai du faux ? \\[0pt]
Peut-on donner un sens à la souffrance ? \\[0pt]
Peut-on donner un sens à l'existence ? \\[0pt]
Peut-on donner un sens à son existence ? \\[0pt]
Peut-on douter de la raison ? \\[0pt]
Peut-on douter de l'humanité d'un homme ? \\[0pt]
Peut-on douter de sa propre existence ? \\[0pt]
Peut-on douter de soi ? \\[0pt]
Peut-on douter de tout ? \\[0pt]
Peut-on douter de toute vérité ? \\[0pt]
Peut-on douter du sens des mots ? \\[0pt]
Peut-on échanger des idées ? \\[0pt]
Peut-on échapper à ses désirs ? \\[0pt]
Peut-on échapper à son temps ? \\[0pt]
Peut-on échapper au cours de l'histoire ? \\[0pt]
Peut-on échapper au temps ? \\[0pt]
Peut-on échapper aux relations de pouvoir ? \\[0pt]
Peut-on éclairer la liberté ? \\[0pt]
Peut-on écrire comme on parle ? \\[0pt]
Peut-on éduquer la conscience ? \\[0pt]
Peut-on éduquer la sensibilité ? \\[0pt]
Peut-on éduquer le goût ? \\[0pt]
Peut-on éduquer quelqu'un à être libre ? \\[0pt]
Peut-on en appeler à la conscience contre la loi ? \\[0pt]
Peut-on en appeler à la conscience contre l'État ? \\[0pt]
Peut-on encore être cosmopolite ? \\[0pt]
Peut-on encore parler d'une nature humaine ? \\[0pt]
Peut-on encore soutenir que l'homme est un animal rationnel ? \\[0pt]
Peut-on en finir avec la violence ? \\[0pt]
Peut-on en finir avec les préjugés ? \\[0pt]
Peut-on en savoir trop ? \\[0pt]
Peut-on entreprendre d'éliminer la métaphysique ? \\[0pt]
Peut-on espérer être libéré du travail ? \\[0pt]
Peut-on établir une hiérarchie des arts ? \\[0pt]
Peut-on être à la fois lucide et heureux ? \\[0pt]
Peut-on être aliéné sans le savoir ? \\[0pt]
Peut-on être amoral ? \\[0pt]
Peut-on être apolitique ? \\[0pt]
Peut-on être assuré d'avoir raison ? \\[0pt]
Peut-on être athée ? \\[0pt]
Peut-on être citoyen du monde ? \\[0pt]
Peut-on être citoyen sans être soldat ? \\[0pt]
Peut-on être complètement athée ? \\[0pt]
Peut-on être content de soi ? \\[0pt]
Peut-on être dans le présent ? \\[0pt]
Peut-on être désintéressé ? \\[0pt]
Peut-on être en avance sur son temps ? \\[0pt]
Peut-on être en conflit avec soi-même ? \\[0pt]
Peut-on être esclave de soi-même ? \\[0pt]
Peut-on être étranger à soi-même ? \\[0pt]
Peut-on être étranger au monde ? \\[0pt]
Peut-on être fidèle à soi-même ? \\[0pt]
Peut-on être heureux dans la solitude ? \\[0pt]
Peut-on être heureux sans être sage ? \\[0pt]
Peut-on être heureux sans le savoir ? \\[0pt]
Peut-on être heureux sans philosophie ? \\[0pt]
Peut-on être heureux sans s'en rendre compte ? \\[0pt]
Peut-on être heureux tout seul ? \\[0pt]
Peut-on être homme sans être citoyen ? \\[0pt]
Peut-on être hors de soi ? \\[0pt]
Peut-on être ignorant ? \\[0pt]
Peut-on être impartial ? \\[0pt]
Peut-on être indifférent à l'histoire ? \\[0pt]
Peut-on être indifférent à son bonheur ? \\[0pt]
Peut-on être injuste avec soi-même ? \\[0pt]
Peut-on être injuste envers soi-même ? \\[0pt]
Peut-on être injuste et heureux ? \\[0pt]
Peut-on être insensible à l'art ? \\[0pt]
Peut-on être insensible au vrai ? \\[0pt]
Peut-on être juste dans une situation injuste ? \\[0pt]
Peut-on être juste dans une société injuste ? \\[0pt]
Peut-on être juste sans être impartial ? \\[0pt]
Peut-on être libre sans le savoir ? \\[0pt]
Peut-on être maître de soi ? \\[0pt]
Peut-on être méchant volontairement ? \\[0pt]
Peut-on être moral sans religion ? \\[0pt]
Peut-on être objectif en matière d'art ? \\[0pt]
Peut-on être obligé d'aimer ? \\[0pt]
Peut-on être plus ou moins libre ? \\[0pt]
Peut-on être prisonnier de soi-même ? \\[0pt]
Peut-on être responsable de ce que l'on n'a pas fait ? \\[0pt]
Peut-on être sage inconsciemment ? \\[0pt]
Peut-on être sage tout seul ? \\[0pt]
Peut-on être sans opinion ? \\[0pt]
Peut-on être sceptique ? \\[0pt]
Peut-on être sceptique de bonne foi ? \\[0pt]
Peut-on être seul ? \\[0pt]
Peut-on être seul avec soi-même ? \\[0pt]
Peut-on être soi-même en société ? \\[0pt]
Peut-on être sûr d'avoir raison ? \\[0pt]
Peut-on être sûr de bien agir ? \\[0pt]
Peut-on être sûr de ne pas se tromper ? \\[0pt]
Peut-on être trop religieux ? \\[0pt]
Peut-on être trop sage ? \\[0pt]
Peut-on être trop sensible ? \\[0pt]
Peut-on étudier le passé de façon objective ? \\[0pt]
Peut-on étudier l'homme de l'extérieur ? \\[0pt]
Peut-on exercer son esprit ? \\[0pt]
Peut-on expérimenter sur le vivant ? \\[0pt]
Peut-on expliquer le mal ? \\[0pt]
Peut-on expliquer le monde par la matière ? \\[0pt]
Peut-on expliquer le vivant ? \\[0pt]
Peut-on expliquer un acte libre ? \\[0pt]
Peut-on expliquer une œuvre d'art ? \\[0pt]
Peut-on faire ce qu'on ne veut pas ? \\[0pt]
Peut-on faire comme si le passé n'existait pas ? \\[0pt]
Peut-on faire de la politique sans supposer les hommes méchants ? \\[0pt]
Peut-on faire de l'art avec tout ? \\[0pt]
Peut-on faire de l'esprit un objet de science ? \\[0pt]
Peut-on faire de sa vie une œuvre d'art ? \\[0pt]
Peut-on faire du dialogue un modèle de relation morale ? \\[0pt]
Peut-on faire la paix ? \\[0pt]
Peut-on faire la philosophie de l'histoire ? \\[0pt]
Peut-on faire le bien d'autrui malgré lui ? \\[0pt]
Peut-on faire le bien de quelqu'un malgré lui ? \\[0pt]
Peut-on faire le bonheur d'autrui ? \\[0pt]
Peut-on faire l'économie de la notion de forme ? \\[0pt]
Peut-on faire l'éloge de l'illusion ? \\[0pt]
Peut-on faire le mal en vue du bien ? \\[0pt]
Peut-on faire le mal innocemment ? \\[0pt]
Peut-on faire l'expérience de la nécessité ? \\[0pt]
Peut-on faire l'inventaire du monde ? \\[0pt]
Peut-on faire son devoir par habitude ? \\[0pt]
Peut-on faire table rase du passé ? \\[0pt]
Peut-on faire un mauvais usage de la raison ? \\[0pt]
Peut-on feindre la vertu ? \\[0pt]
Peut-on fixer des limites à la science ? \\[0pt]
Peut-on fonder la liberté ? \\[0pt]
Peut-on fonder la morale ? \\[0pt]
Peut-on fonder la morale sur la pitié ? \\[0pt]
Peut-on fonder le droit de propriété ? \\[0pt]
Peut-on fonder le droit sur la morale ? \\[0pt]
Peut-on fonder les droits de l'homme ? \\[0pt]
Peut-on fonder les mathématiques ? \\[0pt]
Peut-on fonder un droit de désobéir ? \\[0pt]
Peut-on fonder une éthique sur la biologie ? \\[0pt]
Peut-on fonder une morale sur la nature ? \\[0pt]
Peut-on fonder une morale sur le plaisir ? \\[0pt]
Peut-on forcer quelqu'un à être libre ? \\[0pt]
Peut-on forcer un homme à être libre ? \\[0pt]
Peut-on fuir hors du monde ? \\[0pt]
Peut-on fuir la société ? \\[0pt]
Peut-on gâcher son talent ? \\[0pt]
Peut-on gouverner le développement des innovations techniques ? \\[0pt]
Peut-on gouverner sans lois ? \\[0pt]
Peut-on guérir par la parole ? \\[0pt]
Peut-on haïr la raison ? \\[0pt]
Peut-on haïr la vie ? \\[0pt]
Peut-on haïr les images ? \\[0pt]
Peut-on hiérarchiser les arts ? \\[0pt]
Peut-on hiérarchiser les devoirs ? \\[0pt]
Peut-on hiérarchiser les œuvres d'art ? \\[0pt]
Peut-on hiérarchiser les sociétés ? \\[0pt]
Peut-on identifier le désir au besoin ? \\[0pt]
Peut-on ignorer sa propre liberté ? \\[0pt]
Peut-on ignorer volontairement la vérité ? \\[0pt]
Peut-on imaginer l'avenir ? \\[0pt]
Peut-on imaginer un langage universel ? \\[0pt]
Peut-on imposer la liberté ? \\[0pt]
Peut-on innover en politique ? \\[0pt]
Peut-on interpréter la nature ? \\[0pt]
Peut-on inventer en morale ? \\[0pt]
Peut-on invoquer la nature comme une excuse ? \\[0pt]
Peut-on jamais aimer son prochain ? \\[0pt]
Peut-on jamais avoir la conscience tranquille ? \\[0pt]
Peut-on juger autrui ? \\[0pt]
Peut-on juger de la valeur d'une vie humaine ? \\[0pt]
Peut-on juger des œuvres d'art sans recourir à l'idée de beauté ? \\[0pt]
Peut-on justifier la discrimination ? \\[0pt]
Peut-on justifier la guerre ? \\[0pt]
Peut-on justifier la raison d'État ? \\[0pt]
Peut-on justifier la violence ? \\[0pt]
Peut-on justifier le mal ? \\[0pt]
Peut-on justifier le mensonge ? \\[0pt]
Peut-on justifier le recours à l'idée de finalité ? \\[0pt]
Peut-on justifier ses choix ? \\[0pt]
Peut-on légitimer la violence ? \\[0pt]
Peut-on limiter l'expression de la volonté du peuple ? \\[0pt]
Peut-on lutter contre le destin ? \\[0pt]
Peut-on lutter contre soi-même ? \\[0pt]
Peut-on maîtriser la nature ? \\[0pt]
Peut-on maîtriser la technique ? \\[0pt]
Peut-on maîtriser le risque ? \\[0pt]
Peut-on maîtriser le temps ? \\[0pt]
Peut-on maîtriser l'évolution de la technique ? \\[0pt]
Peut-on maîtriser l'inconscient ? \\[0pt]
Peut-on maîtriser ses désirs ? \\[0pt]
Peut-on manipuler les esprits ? \\[0pt]
Peut-on manquer de culture ? \\[0pt]
Peut-on manquer de volonté ? \\[0pt]
Peut-on me contraindre à faire mon devoir ? \\[0pt]
Peut-on mentir par humanité ? \\[0pt]
Peut-on mesurer les phénomènes sociaux ? \\[0pt]
Peut-on mesurer le temps ? \\[0pt]
Peut-on montrer en cachant ? \\[0pt]
Peut-on moraliser la guerre ? \\[0pt]
Peut-on ne croire en rien \\[0pt]
Peut-on ne croire en rien ? \\[0pt]
Peut-on ne pas connaître son bonheur ? \\[0pt]
Peut-on ne pas croire ? \\[0pt]
Peut-on ne pas croire à la science ? \\[0pt]
Peut-on ne pas croire au progrès ? \\[0pt]
Peut-on ne pas être de son temps ? \\[0pt]
Peut-on ne pas être égoïste ? \\[0pt]
Peut-on ne pas être matérialiste ? \\[0pt]
Peut-on ne pas être soi-même ? \\[0pt]
Peut-on ne pas interpréter ? \\[0pt]
Peut-on ne pas manquer de temps ? \\[0pt]
Peut-on ne pas perdre son temps ? \\[0pt]
Peut-on ne pas rechercher le bonheur ? \\[0pt]
Peut-on ne pas savoir ce que l'on dit ? \\[0pt]
Peut-on ne pas savoir ce que l'on fait ? \\[0pt]
Peut-on ne pas savoir ce que l'on veut ? \\[0pt]
Peut-on ne pas savoir ce qu'on veut ? \\[0pt]
Peut-on ne pas vouloir être heureux ? \\[0pt]
Peut-on ne penser à rien ? \\[0pt]
Peut-on ne rien aimer ? \\[0pt]
Peut-on ne rien devoir à personne ? \\[0pt]
Peut-on ne rien vouloir ? \\[0pt]
Peut-on ne vivre qu'au présent ? \\[0pt]
Peut-on nier la réalité ? \\[0pt]
Peut-on nier le réel ? \\[0pt]
Peut-on nier l'évidence ? \\[0pt]
Peut-on nier l'existence de la matière ? \\[0pt]
Peut-on nier l'existence du temps ? \\[0pt]
Peut-on objectiver le psychisme ? \\[0pt]
Peut-on opposer connaissance scientifique et création artistique ? \\[0pt]
Peut-on opposer justice et liberté ? \\[0pt]
Peut-on opposer le loisir au travail ? \\[0pt]
Peut-on opposer morale et technique ? \\[0pt]
Peut-on opposer nature et culture ? \\[0pt]
Peut-on opposer religion et raison ? \\[0pt]
Peut-on ôter à l'homme sa liberté ? \\[0pt]
Peut-on oublier ? \\[0pt]
Peut-on oublier de vivre ? \\[0pt]
Peut-on pactiser avec un brigand ? \\[0pt]
Peut-on parler d'art primitif ? \\[0pt]
Peut-on parler de capital culturel ? \\[0pt]
Peut-on parler de ce qui n'existe pas ? \\[0pt]
Peut-on parler de conscience collective ? \\[0pt]
Peut-on parler de corruption des mœurs ? \\[0pt]
Peut-on parler de despotisme démocratique ? \\[0pt]
Peut-on parler de dialogue des cultures ? \\[0pt]
Peut-on parler de droits des animaux ? \\[0pt]
Peut-on parler de mondes imaginaires ? \\[0pt]
Peut-on parler de « nature humaine » ? \\[0pt]
Peut-on parler de nourriture spirituelle ? \\[0pt]
Peut-on parler de problèmes techniques ? \\[0pt]
Peut-on parler de progrès en art ? \\[0pt]
Peut-on parler des miracles de la technique ? \\[0pt]
Peut-on parler des œuvres d'art ? \\[0pt]
Peut-on parler de « travail intellectuel » ? \\[0pt]
Peut-on parler de travail intellectuel ? \\[0pt]
Peut-on parler de vérité en art ? \\[0pt]
Peut-on parler de vérités métaphysiques ? \\[0pt]
Peut-on parler de vérité subjective ? \\[0pt]
Peut-on parler de vérité théâtrale ? \\[0pt]
Peut-on parler de vertu politique ? \\[0pt]
Peut-on parler de violence d'État ? \\[0pt]
Peut-on parler d'un droit de la guerre ? \\[0pt]
Peut-on parler d'un droit de résistance ? \\[0pt]
Peut-on parler d'une expérience religieuse ? \\[0pt]
Peut-on parler d'une morale collective ? \\[0pt]
Peut-on parler d'une religion de l'humanité ? \\[0pt]
Peut-on parler d'une santé de l'âme ? \\[0pt]
Peut-on parler d'une science de l'art ? \\[0pt]
Peut-on parler d'univers symbolique ? \\[0pt]
Peut-on parler d'un progrès dans l'histoire ? \\[0pt]
Peut-on parler d'un progrès de la liberté ? \\[0pt]
Peut-on parler d'un progrès moral ? \\[0pt]
Peut-on parler d'un règne de la technique ? \\[0pt]
Peut-on parler d'un savoir poétique ? \\[0pt]
Peut-on parler d'un sens moral ? \\[0pt]
Peut-on parler d'un style moral ? \\[0pt]
Peut-on parler d'un travail intellectuel ? \\[0pt]
Peut-on parler pour ne rien dire ? \\[0pt]
Peut-on parler sans incohérence d'une libre obéissance ? \\[0pt]
Peut-on partager ses goûts ? \\[0pt]
Peut-on partager une expérience ? \\[0pt]
Peut-on penser ce qu'on ne peut dire ? \\[0pt]
Peut-on penser contre l'expérience ? \\[0pt]
Peut-on penser illogiquement ? \\[0pt]
Peut-on penser la création ? \\[0pt]
Peut-on penser la douleur ? \\[0pt]
Peut-on penser la fin de toute chose ? \\[0pt]
Peut-on penser la justice comme une compétence ? \\[0pt]
Peut-on penser la matière ? \\[0pt]
Peut-on penser la mort ? \\[0pt]
Peut-on penser la nouveauté ? \\[0pt]
Peut-on penser l'art comme un langage ? \\[0pt]
Peut-on penser l'art sans référence au beau ? \\[0pt]
Peut-on penser l'avenir ? \\[0pt]
Peut-on penser la vie ? \\[0pt]
Peut-on penser la vie sans penser la mort ? \\[0pt]
Peut-on penser le changement ? \\[0pt]
Peut-on penser le divin ? \\[0pt]
Peut-on penser le monde sans la technique ? \\[0pt]
Peut-on penser le réel comme un tout ? \\[0pt]
Peut-on penser le social sans la politique ? \\[0pt]
Peut-on penser le temps ? \\[0pt]
Peut-on penser le temps sans l'espace ? \\[0pt]
Peut-on penser l'extériorité ? \\[0pt]
Peut-on penser l'histoire comme progrès ? \\[0pt]
Peut-on penser l'homme à partir de la nature ? \\[0pt]
Peut-on penser l'impossible ? \\[0pt]
Peut-on penser l'infini ? \\[0pt]
Peut-on penser l'irrationnel ? \\[0pt]
Peut-on penser l'œuvre d'art sans référence à l'idée de beauté ? \\[0pt]
Peut-on penser sans concept ? \\[0pt]
Peut-on penser sans concepts ? \\[0pt]
Peut-on penser sans image ? \\[0pt]
Peut-on penser sans images ? \\[0pt]
Peut-on penser sans les mots ? \\[0pt]
Peut-on penser sans les signes ? \\[0pt]
Peut-on penser sans méthode ? \\[0pt]
Peut-on penser sans ordre ? \\[0pt]
Peut-on penser sans préjugé ? \\[0pt]
Peut-on penser sans préjugés ? \\[0pt]
Peut-on penser sans règles ? \\[0pt]
Peut-on penser sans savoir ce que l'on pense ? \\[0pt]
Peut-on penser sans savoir que l'on pense ? \\[0pt]
Peut-on penser sans signes ? \\[0pt]
Peut-on penser sans son corps ? \\[0pt]
Peut-on penser un art sans œuvres ? \\[0pt]
Peut-on penser un droit international ? \\[0pt]
Peut-on penser une conscience sans mémoire ? \\[0pt]
Peut-on penser une conscience sans objet ? \\[0pt]
Peut-on penser une fin de l'histoire ? \\[0pt]
Peut-on penser une métaphysique sans Dieu ? \\[0pt]
Peut-on penser une religion sans le recours au divin ? \\[0pt]
Peut-on penser une société sans État ? \\[0pt]
Peut-on penser un État sans violence ? \\[0pt]
Peut-on penser une volonté diabolique ? \\[0pt]
Peut-on penser un pouvoir absolu ? \\[0pt]
Peut-on percevoir le temps ? \\[0pt]
Peut-on percevoir sans juger ? \\[0pt]
Peut-on percevoir sans s'en apercevoir ? \\[0pt]
Peut-on perdre la raison ? \\[0pt]
Peut-on perdre sa dignité ? \\[0pt]
Peut-on perdre sa liberté ? \\[0pt]
Peut-on perdre son identité ? \\[0pt]
Peut-on perdre son temps ? \\[0pt]
Peut-on permettre le mensonge ? \\[0pt]
Peut-on préconiser, dans les sciences humaines et sociales, l'imitation des sciences de la nature ? \\[0pt]
Peut-on prédire les événements ? \\[0pt]
Peut-on prédire l'histoire ? \\[0pt]
Peut-on préférer le bonheur à la vérité ? \\[0pt]
Peut-on préférer l'injustice au désordre ? \\[0pt]
Peut-on préférer l'ordre à la justice ? \\[0pt]
Peut-on prendre le risque de donner la mort en voulant soulager la souffrance ? \\[0pt]
Peut-on prendre les moyens pour la fin ? \\[0pt]
Peut-on prévoir l'avenir ? \\[0pt]
Peut-on prévoir le futur ? \\[0pt]
Peut-on prévoir les hommes ? \\[0pt]
Peut-on promettre le bonheur ? \\[0pt]
Peut-on protéger les libertés sans les réduire ? \\[0pt]
Peut-on prouver la réalité de l'esprit ? \\[0pt]
Peut-on prouver l'existence ? \\[0pt]
Peut-on prouver l'existence de Dieu ? \\[0pt]
Peut-on prouver l'existence de l'inconscient ? \\[0pt]
Peut-on prouver l'existence du monde ? \\[0pt]
Peut-on prouver une existence ? \\[0pt]
Peut-on raconter sa vie ? \\[0pt]
Peut-on raisonner sans règles ? \\[0pt]
Peut-on ralentir la course du temps ? \\[0pt]
Peut-on recommencer sa vie ? \\[0pt]
Peut-on reconnaître un sens à l'histoire sans lui assigner une fin ? \\[0pt]
Peut-on réduire la pensée à une espèce de comportement ? \\[0pt]
Peut-on réduire la vie à la matière ? \\[0pt]
Peut-on réduire le raisonnement au calcul ? \\[0pt]
Peut-on réduire l'esprit à la matière ? \\[0pt]
Peut-on réduire une métaphysique à une conception du monde ? \\[0pt]
Peut-on réduire un homme à la somme de ses actes ? \\[0pt]
Peut-on refaire le monde ? \\[0pt]
Peut-on refuser de voir la vérité ? \\[0pt]
Peut-on refuser la loi ? \\[0pt]
Peut-on refuser la vérité ? \\[0pt]
Peut-on refuser la violence ? \\[0pt]
Peut-on refuser le bonheur ? \\[0pt]
Peut-on refuser l'évidence ? \\[0pt]
Peut-on refuser le vrai ? \\[0pt]
Peut-on réfuter le scepticisme ? \\[0pt]
Peut-on régner innocemment ? \\[0pt]
Peut-on rejeter le principe de non-contradiction ? \\[0pt]
Peut-on rendre raison des émotions ? \\[0pt]
Peut-on rendre raison de tout ? \\[0pt]
Peut-on rendre raison du réel ? \\[0pt]
Peut-on renoncer à comprendre ? \\[0pt]
Peut-on renoncer à être libre ? \\[0pt]
Peut-on renoncer à la liberté ? \\[0pt]
Peut-on renoncer à la vérité ? \\[0pt]
Peut-on renoncer à sa liberté ? \\[0pt]
Peut-on renoncer à ses droits ? \\[0pt]
Peut-on renoncer à soi ? \\[0pt]
Peut-on renoncer au bonheur ? \\[0pt]
Peut-on réparer le vivant ? \\[0pt]
Peut-on réparer une injustice ? \\[0pt]
Peut-on répondre au sceptique ? \\[0pt]
Peut-on répondre d'autrui ? \\[0pt]
Peut-on représenter le peuple ? \\[0pt]
Peut-on représenter l'espace ? \\[0pt]
Peut-on représenter l'invisible ? \\[0pt]
Peut-on reprocher à la morale d'être abstraite ? \\[0pt]
Peut-on reprocher au langage d'être équivoque ? \\[0pt]
Peut-on reprocher au langage d'être parfait ? \\[0pt]
Peut-on résister à la vérité ? \\[0pt]
Peut-on résister au vrai ? \\[0pt]
Peut-on rester dans le doute ? \\[0pt]
Peut-on rester insensible à la beauté ? \\[0pt]
Peut-on rester sceptique ? \\[0pt]
Peut-on restreindre la logique à la pensée formelle ? \\[0pt]
Peut-on résumer une philosophie ? \\[0pt]
Peut-on retenir le temps ? \\[0pt]
Peut-on réunir des arts différents dans une même œuvre ? \\[0pt]
Peut-on revendiquer la paix comme un droit ? \\[0pt]
Peut-on revendiquer le droit de désobéir ? \\[0pt]
Peut-on revenir sur ses erreurs ? \\[0pt]
Peut-on rire de tout ? \\[0pt]
Peut-on rompre avec la société ? \\[0pt]
Peut-on rompre avec le passé ? \\[0pt]
Peut-on s'abstenir de penser politiquement ? \\[0pt]
Peut-on s'accorder sur des vérités morales ? \\[0pt]
Peut-on s'affranchir de la subjectivité ? \\[0pt]
Peut-on s'affranchir de sa propre nature ? \\[0pt]
Peut-on s'affranchir des lois ? \\[0pt]
Peut-on saisir le temps ? \\[0pt]
Peut-on s'attendre à tout ? \\[0pt]
Peut-on savoir ce que l'on fait ? \\[0pt]
Peut-on savoir ce qui est bien ? \\[0pt]
Peut-on savoir quelque chose de l'avenir ? \\[0pt]
Peut-on savoir sans croire ? \\[0pt]
Peut-on se choisir un destin ? \\[0pt]
Peut-on se connaître soi-même ? \\[0pt]
Peut-on se désintéresser de la politique ? \\[0pt]
Peut-on se désintéresser de son bonheur ? \\[0pt]
Peut-on se duper soi-même ? \\[0pt]
Peut-on se faire une idée de tout ? \\[0pt]
Peut-on se fier à la technique ? \\[0pt]
Peut-on se fier à l'expérience vécue ? \\[0pt]
Peut-on se fier à l'intuition ? \\[0pt]
Peut-on se fier à sa propre raison ? \\[0pt]
Peut-on se fier à son intuition ? \\[0pt]
Peut-on se fier aux apparences ? \\[0pt]
Peut-on se gouverner soi-même ? \\[0pt]
Peut-on se méfier de soi-même ? \\[0pt]
Peut-on se mentir à soi-même ? \\[0pt]
Peut-on se mettre à la place d'autrui ? \\[0pt]
Peut-on se mettre à la place de l'autre ? \\[0pt]
Peut-on se mettre à la place des autres ? \\[0pt]
Peut-on s'en tenir au présent ? \\[0pt]
Peut-on sentir le temps qui passe ? \\[0pt]
Peut-on séparer la politique de l'exigence de vérité ? \\[0pt]
Peut-on séparer l'homme et l'œuvre ? \\[0pt]
Peut-on séparer politique et économie ? \\[0pt]
Peut-on se passer de chef ? \\[0pt]
Peut-on se passer de croire ? \\[0pt]
Peut-on se passer de croyance ? \\[0pt]
Peut-on se passer de croyances ? \\[0pt]
Peut-on se passer de Dieu ? \\[0pt]
Peut-on se passer de frontières ? \\[0pt]
Peut-on se passer de la religion ? \\[0pt]
Peut-on se passer de la technique ? \\[0pt]
Peut-on se passer de l'État ? \\[0pt]
Peut-on se passer de l'idée de cause finale ? \\[0pt]
Peut-on se passer de l'idée de Dieu ? \\[0pt]
Peut-on se passer de l'idée de droit naturel ? \\[0pt]
Peut-on se passer de l'idée de finalité ? \\[0pt]
Peut-on se passer de l'idée de nature humaine ? \\[0pt]
Peut-on se passer de maître ? \\[0pt]
Peut-on se passer de métaphysique ? \\[0pt]
Peut-on se passer de méthode ? \\[0pt]
Peut-on se passer de mythes ? \\[0pt]
Peut-on se passer de principes ? \\[0pt]
Peut-on se passer de religion ? \\[0pt]
Peut-on se passer de représentants ? \\[0pt]
Peut-on se passer des causes finales ? \\[0pt]
Peut-on se passer de spiritualité ? \\[0pt]
Peut-on se passer des relations ? \\[0pt]
Peut-on se passer d'État ? \\[0pt]
Peut-on se passer de technique ? \\[0pt]
Peut-on se passer de techniques de raisonnement ? \\[0pt]
Peut-on se passer de toute religion ? \\[0pt]
Peut-on se passer d'idéal ? \\[0pt]
Peut-on se passer d'ontologie ? \\[0pt]
Peut-on se passer d'un maître ? \\[0pt]
Peut-on se peindre soi-même ? \\[0pt]
Peut-on se prescrire une loi ? \\[0pt]
Peut-on se promettre quelque chose à soi-même ? \\[0pt]
Peut-on se punir soi-même ? \\[0pt]
Peut-on se régler sur des exemples en politique ? \\[0pt]
Peut-on se rendre maître de la technique ? \\[0pt]
Peut-on se retirer du monde ? \\[0pt]
Peut-on servir deux maîtres à la fois ? \\[0pt]
Peut-on se soustraire à son devoir ? \\[0pt]
Peut-on se tromper en se croyant heureux ? \\[0pt]
Peut-on se tromper sur son devoir ? \\[0pt]
Peut-on se vouloir parfait ? \\[0pt]
Peut-on s'exprimer par le silence ? \\[0pt]
Peut-on s'obliger soi-même ? \\[0pt]
Peut-on sortir de la subjectivité ? \\[0pt]
Peut-on sortir de sa conscience ? \\[0pt]
Peut-on sortir de sa culture ? \\[0pt]
Peut-on s'oublier ? \\[0pt]
Peut-on souhaiter le gouvernement des meilleurs ? \\[0pt]
Peut-on souhaiter ne pas être libre ? \\[0pt]
Peut-on suivre une règle ? \\[0pt]
Peut-on suspendre le temps ? \\[0pt]
Peut-on suspendre son jugement ? \\[0pt]
Peut-on sympathiser avec l'ennemi ? \\[0pt]
Peut-on tirer des leçons de l'histoire ? \\[0pt]
Peut-on tolérer l'injustice ? \\[0pt]
Peut-on toujours faire ce qu'on doit ? \\[0pt]
Peut-on toujours raison garder ? \\[0pt]
Peut-on toujours savoir entièrement ce que l'on dit ? \\[0pt]
Peut-on tout analyser ? \\[0pt]
Peut-on tout attendre de l'État ? \\[0pt]
Peut-on tout définir ? \\[0pt]
Peut-on tout démontrer ? \\[0pt]
Peut-on tout désirer ? \\[0pt]
Peut-on tout dire ? \\[0pt]
Peut-on tout donner ? \\[0pt]
Peut-on tout échanger ? \\[0pt]
Peut-on tout enseigner ? \\[0pt]
Peut-on tout expliquer ? \\[0pt]
Peut-on tout exprimer ? \\[0pt]
Peut-on tout imaginer ? \\[0pt]
Peut-on tout imiter ? \\[0pt]
Peut-on tout interpréter ? \\[0pt]
Peut-on tout justifier ? \\[0pt]
Peut-on tout mathématiser ? \\[0pt]
Peut-on tout mesurer ? \\[0pt]
Peut-on tout ordonner ? \\[0pt]
Peut-on tout pardonner ? \\[0pt]
Peut-on tout partager ? \\[0pt]
Peut-on tout prévoir ? \\[0pt]
Peut-on tout prouver ? \\[0pt]
Peut-on tout soumettre à la discussion ? \\[0pt]
Peut-on tout tolérer ? \\[0pt]
Peut-on traiter autrui comme un moyen ? \\[0pt]
Peut-on traiter un être vivant comme une machine ? \\[0pt]
Peut-on trancher les conflits entre interprétations ? \\[0pt]
Peut-on transformer le réel ? \\[0pt]
Peut-on transiger avec les principes ? \\[0pt]
Peut-on trop penser ? \\[0pt]
Peut-on trouver du plaisir à l'ennui ? \\[0pt]
Peut-on vivre avec les autres ? \\[0pt]
Peut-on vivre dans le doute ? \\[0pt]
Peut-on vivre en dehors de l'histoire ? \\[0pt]
Peut-on vivre en marge de la société ? \\[0pt]
Peut-on vivre en paix avec son inconscient ? \\[0pt]
Peut-on vivre en sceptique ? \\[0pt]
Peut-on vivre hors du temps ? \\[0pt]
Peut-on vivre pour la vérité ? \\[0pt]
Peut-on vivre sans aimer ? \\[0pt]
Peut-on vivre sans art ? \\[0pt]
Peut-on vivre sans aucune certitude ? \\[0pt]
Peut-on vivre sans croyance ? \\[0pt]
Peut-on vivre sans croyances ? \\[0pt]
Peut-on vivre sans désir ? \\[0pt]
Peut-on vivre sans échange ? \\[0pt]
Peut-on vivre sans foi ni loi ? \\[0pt]
Peut-on vivre sans illusions ? \\[0pt]
Peut-on vivre sans l'art ? \\[0pt]
Peut-on vivre sans le plaisir de vivre ? \\[0pt]
Peut-on vivre sans lois ? \\[0pt]
Peut-on vivre sans opinions ? \\[0pt]
Peut-on vivre sans passion ? \\[0pt]
Peut-on vivre sans peur ? \\[0pt]
Peut-on vivre sans principes ? \\[0pt]
Peut-on vivre sans réfléchir ? \\[0pt]
Peut-on vivre sans ressentiment ? \\[0pt]
Peut-on vivre sans rien espérer ? \\[0pt]
Peut-on vivre sans sacré ? \\[0pt]
Peut-on vivre sans travailler ? \\[0pt]
Peut-on vivre sans valeurs ? \\[0pt]
Peut-on voir sans croire ? \\[0pt]
Peut-on vouloir ce qu'on ne désire pas ? \\[0pt]
Peut-on vouloir le bien sans le faire ? \\[0pt]
Peut-on vouloir le bonheur d'autrui ? \\[0pt]
Peut-on vouloir le mal ? \\[0pt]
Peut-on vouloir le mal pour le mal ? \\[0pt]
Peut-on vouloir le mal sachant que c'est le mal ? \\[0pt]
Peut-on vouloir l'impossible ? \\[0pt]
Peut-on vouloir ne pas être libre ? \\[0pt]
Peut-on vouloir sans désirer ? \\[0pt]
Peut-on vraiment créer ? \\[0pt]
Peut-on vraiment dire ce qu'on pense ? \\[0pt]
Peut-on vraiment parler de soi ? \\[0pt]
Peut-on vraiment tirer des leçons du passé ? \\[0pt]
Phénomène ou fait social ? \\[0pt]
Philosopher, est-ce apprendre à vivre ? \\[0pt]
Philosophe-t-on pour être heureux ? \\[0pt]
Philosophie et mathématiques \\[0pt]
Philosophie et métaphysique \\[0pt]
Philosophie et poésie \\[0pt]
Philosophie et religion \\[0pt]
Philosophie et système \\[0pt]
Philosophie et théologie \\[0pt]
Philosophie, psychologie, sociologie \\[0pt]
Photographier le réel \\[0pt]
Physique et cosmologie \\[0pt]
Physique et mathématique \\[0pt]
Physique et mathématiques \\[0pt]
Physique et métaphysique \\[0pt]
Pitié et compassion \\[0pt]
Pitié et cruauté \\[0pt]
Pitié et mépris \\[0pt]
Plaider \\[0pt]
Plaisir et bonheur \\[0pt]
Plaisir et douleur \\[0pt]
Plaisir et préférences \\[0pt]
Plaisirs, honneurs, richesses \\[0pt]
Pluralisme et politique \\[0pt]
Pluralité et unité \\[0pt]
Plusieurs religions valent-elles mieux qu'une seule ? \\[0pt]
« Plus vite, plus haut, plus fort » \\[0pt]
Poésie et philosophie \\[0pt]
Poésie et vérité \\[0pt]
Poétique et prosaïque \\[0pt]
Point de vue du créateur et point de vue du spectateur \\[0pt]
Police et politique \\[0pt]
Politique et coopération \\[0pt]
Politique et esthétique \\[0pt]
Politique et médias \\[0pt]
Politique et mémoire \\[0pt]
Politique et parole \\[0pt]
Politique et participation \\[0pt]
Politique et passions \\[0pt]
Politique et propagande \\[0pt]
Politique et religion \\[0pt]
Politique et secret \\[0pt]
Politique et technique \\[0pt]
Politique et technologie \\[0pt]
Politique et territoire \\[0pt]
Politique et trahison \\[0pt]
Politique et unité \\[0pt]
Politique et vérité \\[0pt]
Politique et vertu \\[0pt]
Position, rôle, fonction \\[0pt]
Possession et propriété \\[0pt]
Possibilité et intelligibilité \\[0pt]
Pour agir moralement, faut-il ne pas se soucier de soi ? \\[0pt]
Pour apprécier une œuvre, faut-il être cultivé ? \\[0pt]
Pour bien agir, faut-il vouloir le bien d'autrui ? \\[0pt]
Pour bien penser, faut-il renoncer au langage courant ? \\[0pt]
Pour connaître, suffit-il de démontrer ? \\[0pt]
Pour dialoguer faut-il parler le même langage ? \\[0pt]
Pour être heureux, faut-il ne rien désirer ? \\[0pt]
Pour être heureux, faut-il renoncer à la perfection ? \\[0pt]
Pour être homme, faut-il être citoyen ? \\[0pt]
Pour être libre, faut-il renoncer à être heureux ? \\[0pt]
Pour être un bon observateur faut-il être un bon théoricien ? \\[0pt]
Pour juger, faut-il seulement apprendre à raisonner ? \\[0pt]
Pour penser rigoureusement, faut-il renoncer au langage ordinaire ? \\[0pt]
Pour qui se prend-on ? \\[0pt]
« Pourquoi » \\[0pt]
Pourquoi ? \\[0pt]
Pourquoi accomplir son devoir ? \\[0pt]
Pourquoi aimer la liberté ? \\[0pt]
Pourquoi aimons-nous la musique ? \\[0pt]
Pourquoi aller contre son désir ? \\[0pt]
Pourquoi a-t-on peur de la folie ? \\[0pt]
Pourquoi avoir recours à la notion d'inconscient ? \\[0pt]
Pourquoi avons-nous besoin des autres pour être heureux ? \\[0pt]
Pourquoi avons-nous du mal à reconnaître la vérité ? \\[0pt]
Pourquoi avons-nous tant de mal à renoncer à nos illusions ? \\[0pt]
Pourquoi châtier ? \\[0pt]
Pourquoi chercher à connaître le passé ? \\[0pt]
Pourquoi chercher à se connaître soi-même ? \\[0pt]
Pourquoi chercher à se distinguer ? \\[0pt]
Pourquoi chercher à vivre libre ? \\[0pt]
Pourquoi chercher la vérité ? \\[0pt]
Pourquoi chercher un sens à l'histoire ? \\[0pt]
Pourquoi cherche-t-on à connaître ? \\[0pt]
Pourquoi commémorer ? \\[0pt]
Pourquoi communiquer ? \\[0pt]
Pourquoi conserver des œuvres d'art ? \\[0pt]
Pourquoi conserver les œuvres d'art ? \\[0pt]
Pourquoi construire des monuments ? \\[0pt]
Pourquoi critiquer la raison ? \\[0pt]
Pourquoi critiquer le conformisme ? \\[0pt]
Pourquoi croyons-nous ? \\[0pt]
Pourquoi défendre le faible ? \\[0pt]
Pourquoi définir ? \\[0pt]
Pourquoi délibérer ? \\[0pt]
Pourquoi démontrer ? \\[0pt]
Pourquoi démontrer ce que l'on sait être vrai ? \\[0pt]
Pourquoi des artifices ? \\[0pt]
Pourquoi des artistes ? \\[0pt]
Pourquoi des cérémonies ? \\[0pt]
Pourquoi des châtiments ? \\[0pt]
Pourquoi des classifications ? \\[0pt]
Pourquoi des comités d'éthique ? \\[0pt]
Pourquoi des conflits ? \\[0pt]
Pourquoi des constitutions ? \\[0pt]
Pourquoi des corps intermédiaires ? \\[0pt]
Pourquoi des définitions ? \\[0pt]
Pourquoi des devoirs ? \\[0pt]
Pourquoi des élections ? \\[0pt]
Pourquoi des exemples ? \\[0pt]
Pourquoi des fêtes ? \\[0pt]
Pourquoi des fictions ? \\[0pt]
Pourquoi des géométries ? \\[0pt]
Pourquoi des guerres ? \\[0pt]
Pourquoi des historiens ? \\[0pt]
Pourquoi des hommes politiques ? \\[0pt]
Pourquoi des hypothèses ? \\[0pt]
Pourquoi des idoles ? \\[0pt]
Pourquoi des institutions ? \\[0pt]
Pourquoi des interdits ? \\[0pt]
Pourquoi désirer ? \\[0pt]
Pourquoi désirer la sagesse ? \\[0pt]
Pourquoi désirer l'immortalité ? \\[0pt]
Pourquoi désire-t-on ce dont on n'a nul besoin ? \\[0pt]
Pourquoi désirons-nous ? \\[0pt]
Pourquoi des logiciens ? \\[0pt]
Pourquoi des lois ? \\[0pt]
Pourquoi des maîtres ? \\[0pt]
Pourquoi des métaphores ? \\[0pt]
Pourquoi des modèles ? \\[0pt]
Pourquoi des musées ? \\[0pt]
Pourquoi des œuvres d'art ? \\[0pt]
Pourquoi des philosophes ? \\[0pt]
Pourquoi des poètes ? \\[0pt]
Pourquoi des psychologues ? \\[0pt]
Pourquoi des religions ? \\[0pt]
Pourquoi des rites ? \\[0pt]
Pourquoi des sociologues ? \\[0pt]
Pourquoi des syllogismes ? \\[0pt]
Pourquoi des symboles ? \\[0pt]
Pourquoi des traditions ? \\[0pt]
Pourquoi des utopies ? \\[0pt]
Pourquoi dialogue-t-on ? \\[0pt]
Pourquoi Dieu se soucierait-il des affaires humaines ? \\[0pt]
Pourquoi dire la vérité ? \\[0pt]
Pourquoi distinguer nature et culture ? \\[0pt]
Pourquoi domestiquer ? \\[0pt]
Pourquoi donner ? \\[0pt]
Pourquoi donner des leçons de morale ? \\[0pt]
Pourquoi douter ? \\[0pt]
Pourquoi échanger des idées ? \\[0pt]
Pourquoi écrire ? \\[0pt]
Pourquoi écrit-on ? \\[0pt]
Pourquoi écrit-on des lois ? \\[0pt]
Pourquoi écrit-on les lois ? \\[0pt]
Pourquoi écrit-on l'Histoire ? \\[0pt]
Pourquoi est-il difficile de rectifier une erreur ? \\[0pt]
Pourquoi être exigeant ? \\[0pt]
Pourquoi être moral ? \\[0pt]
Pourquoi être raisonnable ? \\[0pt]
Pourquoi étudier le vivant ? \\[0pt]
Pourquoi étudier l'Histoire ? \\[0pt]
Pourquoi exiger la cohérence \\[0pt]
Pourquoi exposer les œuvres d'art ? \\[0pt]
Pourquoi faire confiance ? \\[0pt]
Pourquoi faire confiance au dialogue ? \\[0pt]
Pourquoi faire de la métaphysique ? \\[0pt]
Pourquoi faire de la politique ? \\[0pt]
Pourquoi faire de l'histoire ? \\[0pt]
Pourquoi faire la guerre ? \\[0pt]
Pourquoi faire l'hypothèse de l'inconscient ? \\[0pt]
Pourquoi faire son devoir ? \\[0pt]
Pourquoi fait-on le mal ? \\[0pt]
Pourquoi faudrait-il avoir peur de la technique ? \\[0pt]
Pourquoi faudrait-il être cohérent ? \\[0pt]
Pourquoi faudrait-il sauver les apparences ? \\[0pt]
Pourquoi faut-il agir ? \\[0pt]
Pourquoi faut-il diviser le travail ? \\[0pt]
Pourquoi faut-il être cohérent ? \\[0pt]
Pourquoi faut-il être juste ? \\[0pt]
Pourquoi faut-il être poli ? \\[0pt]
Pourquoi faut-il travailler ? \\[0pt]
Pourquoi formaliser des arguments ? \\[0pt]
Pourquoi il n'y a pas de société sans art ? \\[0pt]
Pourquoi imiter ? \\[0pt]
Pourquoi interprète-t-on ? \\[0pt]
Pourquoi joue-t-on ? \\[0pt]
Pourquoi la critique ? \\[0pt]
Pourquoi la curiosité est-elle un vilain défaut ? \\[0pt]
Pourquoi la guerre ? \\[0pt]
Pourquoi la justice a-t-elle besoin d'institutions ? \\[0pt]
Pourquoi la musique intéresse-t-elle le philosophe ? \\[0pt]
Pourquoi la prison ? \\[0pt]
Pourquoi la prohibition de l'inceste ? \\[0pt]
Pourquoi la raison recourt-elle à l'hypothèse ? \\[0pt]
Pourquoi la réalité peut-elle dépasser la fiction ? \\[0pt]
Pourquoi l'art intéresse-t-il les philosophes ? \\[0pt]
Pourquoi l'économie est-elle politique ? \\[0pt]
Pourquoi le droit international est-il imparfait ? \\[0pt]
Pourquoi les droits de l'homme sont-ils universels ? \\[0pt]
Pourquoi les États se font-ils la guerre ? \\[0pt]
Pourquoi les hommes jouent-ils ? \\[0pt]
Pourquoi les hommes mentent-ils ? \\[0pt]
Pourquoi les hommes se soumettent-ils à l'autorité ? \\[0pt]
Pourquoi les mathématiques s'appliquent-elles à la réalité ? \\[0pt]
Pourquoi les œuvres d'art résistent-elles au temps ? \\[0pt]
Pourquoi le sport ? \\[0pt]
Pourquoi les sciences humaines s'intéressent-elles à la parenté ? \\[0pt]
Pourquoi les sciences ont-elles une histoire ? \\[0pt]
Pourquoi les sociétés ont-elles besoin de lois ? \\[0pt]
Pourquoi l'État ? \\[0pt]
Pourquoi le théâtre ? \\[0pt]
Pourquoi l'ethnologue s'intéresse-t-il à la vie urbaine ? \\[0pt]
Pourquoi l'homme a-t-il des droits ? \\[0pt]
Pourquoi l'homme est-il l'objet de plusieurs sciences ? \\[0pt]
Pourquoi l'Homme se pose-t-il des questions métaphysiques ? \\[0pt]
Pourquoi l'homme travaille-t-il ? \\[0pt]
Pourquoi lire des romans ? \\[0pt]
Pourquoi lire les poètes ? \\[0pt]
Pourquoi lit-on des romans ? \\[0pt]
Pourquoi mentir ? \\[0pt]
Pourquoi ne peut-on concevoir la science comme achevée ? \\[0pt]
Pourquoi ne s'entend-on pas sur la nature de ce qui est réel ? \\[0pt]
Pourquoi nous racontons-nous des histoires ? \\[0pt]
Pourquoi nous soucier du sort des générations futures ? \\[0pt]
Pourquoi nous souvenons-nous ? \\[0pt]
Pourquoi nous trompons-nous ? \\[0pt]
Pourquoi n'y aurait-il pas de sots métiers ? \\[0pt]
Pourquoi obéir ? \\[0pt]
Pourquoi obéir aux lois ? \\[0pt]
Pourquoi obéit-on ? \\[0pt]
Pourquoi obéit-on aux lois ? \\[0pt]
Pourquoi oublie-t-on ? \\[0pt]
Pourquoi pardonner ? \\[0pt]
Pourquoi parler de fautes de goût ? \\[0pt]
Pourquoi parler de jugement de goût ? \\[0pt]
Pourquoi parler de « sciences exactes » ? \\[0pt]
Pourquoi parler du travail comme d'un droit ? \\[0pt]
Pourquoi parle-t-on ? \\[0pt]
Pourquoi parle-t-on d'économie politique ? \\[0pt]
Pourquoi parle-t-on d'une « société civile » ? \\[0pt]
Pourquoi parlons-nous ? \\[0pt]
Pourquoi pas ? \\[0pt]
Pourquoi pas plusieurs dieux ? \\[0pt]
Pourquoi peindre la nature ? \\[0pt]
Pourquoi penser à la mort ? \\[0pt]
Pourquoi penser l'impossible ? \\[0pt]
Pourquoi pensons-nous ? \\[0pt]
Pourquoi philosopher ? \\[0pt]
Pourquoi pleure-t-on ? \\[0pt]
Pourquoi pleure-t-on au cinéma ? \\[0pt]
Pourquoi plusieurs sciences ? \\[0pt]
Pourquoi préférer l'original ? \\[0pt]
Pourquoi préférer l'original à la copie ? \\[0pt]
Pourquoi préférer l'original à la reproduction ? \\[0pt]
Pourquoi préférer l'original à sa reproduction ? \\[0pt]
Pourquoi préserver la nature ? \\[0pt]
Pourquoi préserver l'environnement ? \\[0pt]
Pourquoi prier ? \\[0pt]
Pourquoi promettre ? \\[0pt]
Pourquoi prouver l'existence de Dieu ? \\[0pt]
Pourquoi punir ? \\[0pt]
Pourquoi punit-on ? \\[0pt]
Pourquoi raconter des histoires ? \\[0pt]
Pourquoi rechercher la vérité ? \\[0pt]
Pourquoi rechercher le bonheur ? \\[0pt]
Pourquoi refuser de faire son devoir ? \\[0pt]
Pourquoi refuse-t-on la conscience à l'animal ? \\[0pt]
Pourquoi respecter autrui ? \\[0pt]
Pourquoi respecter la nature ? \\[0pt]
Pourquoi respecter le droit ? \\[0pt]
Pourquoi respecter les anciens ? \\[0pt]
Pourquoi rit-on ? \\[0pt]
Pourquoi sauver les apparences ? \\[0pt]
Pourquoi sauver les phénomènes ? \\[0pt]
Pourquoi se confesser ? \\[0pt]
Pourquoi se cultiver ? \\[0pt]
Pourquoi se divertir ? \\[0pt]
Pourquoi se fier à autrui ? \\[0pt]
Pourquoi se mettre à la place d'autrui ? \\[0pt]
Pourquoi s'ennuie-t-on ? \\[0pt]
Pourquoi séparer les pouvoirs ? \\[0pt]
Pourquoi se référer au destin ? \\[0pt]
Pourquoi se révolter ? \\[0pt]
Pourquoi se soucier du futur ? \\[0pt]
Pourquoi se taire ? \\[0pt]
Pourquoi s'étonner ? \\[0pt]
Pourquoi s'exprimer ? \\[0pt]
Pourquoi s'inspirer de l'art antique ? \\[0pt]
Pourquoi s'intéresser à l'histoire ? \\[0pt]
Pourquoi s'intéresser à l'origine ? \\[0pt]
Pourquoi s'interroger sur l'origine du langage ? \\[0pt]
Pourquoi soigner son apparence ? \\[0pt]
Pourquoi sommes-nous déçus par les œuvres d'un faussaire ? \\[0pt]
Pourquoi sommes-nous des êtres moraux ? \\[0pt]
Pourquoi sommes-nous moraux ? \\[0pt]
Pourquoi suivre l'actualité ? \\[0pt]
Pourquoi suspendre son jugement ? \\[0pt]
Pourquoi tenir ses promesses ? \\[0pt]
Pourquoi théoriser ? \\[0pt]
Pourquoi transformer le monde ? \\[0pt]
Pourquoi transmettre ? \\[0pt]
Pourquoi travailler ? \\[0pt]
Pourquoi travaille-t-on ? \\[0pt]
Pourquoi un droit du travail ? \\[0pt]
Pourquoi une instruction publique ? \\[0pt]
Pourquoi un fait devrait-il être établi ? \\[0pt]
Pourquoi veut-on changer le monde ? \\[0pt]
Pourquoi veut-on la vérité ? \\[0pt]
Pourquoi vivons-nous ? \\[0pt]
Pourquoi vivre ensemble ? \\[0pt]
Pourquoi vouloir avoir raison ? \\[0pt]
Pourquoi vouloir devenir « comme maîtres et possesseurs de la nature » ? \\[0pt]
Pourquoi vouloir être libre ? \\[0pt]
Pourquoi vouloir gagner ? \\[0pt]
Pourquoi vouloir la vérité ? \\[0pt]
Pourquoi vouloir s'abstraire du quotidien ? \\[0pt]
Pourquoi vouloir se connaître ? \\[0pt]
Pourquoi voulons-nous savoir ? \\[0pt]
Pourquoi voyager ? \\[0pt]
Pourquoi y a-t-il des conflits insolubles ? \\[0pt]
Pourquoi y a-t-il des institutions ? \\[0pt]
Pourquoi y a-t-il des lois ? \\[0pt]
Pourquoi y a-t-il des religions ? \\[0pt]
Pourquoi y a-t-il du mal dans le monde ? \\[0pt]
Pourquoi y a-t-il plusieurs façons de démontrer ? \\[0pt]
Pourquoi y a-t-il plusieurs langues ? \\[0pt]
Pourquoi y a-t-il plusieurs philosophies ? \\[0pt]
Pourquoi y a-t-il plusieurs sciences ? \\[0pt]
Pourquoi y a-t-il quelque chose plutôt que rien ? \\[0pt]
Pourquoi y a-t-il une philosophie de la vie ? \\[0pt]
Pourrait-on se passer de l'argent ? \\[0pt]
Pourrait-on vivre sans art ? \\[0pt]
Pourrait-on vivre sans idéal ? \\[0pt]
Pourrions-nous comprendre une pensée non humaine ? \\[0pt]
Pourrions-nous nous passer des musées ? \\[0pt]
Pourrions-nous vivre sans religion ? \\[0pt]
Pour s'aimer, faut-il se connaître ? \\[0pt]
Pour se trouver, faut-il savoir se perdre ? \\[0pt]
Pour vivre heureux, vivons cachés \\[0pt]
Pouvoir et autorité \\[0pt]
Pouvoir et contre-pouvoir \\[0pt]
Pouvoir et devoir \\[0pt]
Pouvoir et domination \\[0pt]
Pouvoir et politique \\[0pt]
Pouvoir et puissance \\[0pt]
Pouvoir et savoir \\[0pt]
Pouvoir, magie, secret \\[0pt]
Pouvoirs et libertés \\[0pt]
Pouvoir temporel et pouvoir spirituel \\[0pt]
Pouvons-nous communiquer ce que nous sentons ? \\[0pt]
Pouvons-nous connaître sans interpréter ? \\[0pt]
Pouvons-nous désirer ce qui nous nuit ? \\[0pt]
Pouvons-nous devenir meilleurs ? \\[0pt]
Pouvons-nous dissocier le réel de nos interprétations ? \\[0pt]
Pouvons-nous être certains que nous ne rêvons pas ? \\[0pt]
Pouvons-nous être objectifs ? \\[0pt]
Pouvons-nous faire l'expérience de la liberté ? \\[0pt]
Pouvons-nous justifier nos croyances ? \\[0pt]
Pouvons-nous réellement tirer des leçons du passé ? \\[0pt]
Pouvons-nous savoir ce que nous ignorons ? \\[0pt]
Pouvons-nous vivre sans préjugés ? \\[0pt]
Prédicats et relations \\[0pt]
Prédiction et prévision \\[0pt]
Prédiction et probabilité \\[0pt]
Prédire et expliquer \\[0pt]
Prémisses et conclusions \\[0pt]
Prendre connaissance \\[0pt]
Prendre conscience \\[0pt]
Prendre des risques \\[0pt]
Prendre la parole \\[0pt]
Prendre la parole, est-ce prendre le pouvoir ? \\[0pt]
Prendre le pouvoir \\[0pt]
Prendre les armes \\[0pt]
« Prendre ses désirs pour des réalités » \\[0pt]
Prendre ses désirs pour des réalités \\[0pt]
Prendre ses responsabilités \\[0pt]
Prendre soin \\[0pt]
Prendre son temps \\[0pt]
Prendre son temps, est-ce le perdre ? \\[0pt]
Prendre sur soi \\[0pt]
Prendre une décision \\[0pt]
Prendre une décision politique \\[0pt]
Prescrire et décrire \\[0pt]
Prescrire et interdire \\[0pt]
Présence et absence \\[0pt]
Présence et représentation \\[0pt]
Prêter attention \\[0pt]
Preuve et démonstration \\[0pt]
Preuve et résultat dans les sciences humaines \\[0pt]
Prévoir \\[0pt]
Prévoir les comportements humains \\[0pt]
Primitif ou premier ? \\[0pt]
Prince : nature ou fonction ? \\[0pt]
Principe et axiome \\[0pt]
Principe et cause \\[0pt]
Principe et commencement \\[0pt]
Principe et conséquence \\[0pt]
Principe et fondement \\[0pt]
Principes et stratégie \\[0pt]
Privation et négation \\[0pt]
Probabilité et explication scientifique \\[0pt]
Problème, énigme, mystère \\[0pt]
Production et action \\[0pt]
Production et création \\[0pt]
Production et réception de l'œuvre d'art \\[0pt]
Produire et consommer \\[0pt]
Produire et créer \\[0pt]
Promettre \\[0pt]
Promettre, est-ce renoncer à sa liberté ? \\[0pt]
Prophétie, utopie, pronostic \\[0pt]
Proposition et jugement \\[0pt]
Propriété privée, propriété collective \\[0pt]
Propriétés artistiques, propriétés esthétiques \\[0pt]
Propriétés et dispositions \\[0pt]
Prose et poésie \\[0pt]
Prospérité et sécurité \\[0pt]
Protester \\[0pt]
Prouver \\[0pt]
Prouver Dieu \\[0pt]
Prouver en métaphysique \\[0pt]
Prouver et démontrer \\[0pt]
Prouver et éprouver \\[0pt]
Prouver et justifier \\[0pt]
Prouver et réfuter \\[0pt]
Prouver la force d'âme \\[0pt]
Prouver l'existence de Dieu \\[0pt]
Prouver l'existence du monde extérieur \\[0pt]
Prouvez-le ! \\[0pt]
Providence et destin \\[0pt]
Prudence et liberté \\[0pt]
Psychanalyse et critique littéraire \\[0pt]
Psychanalyse et esthétique \\[0pt]
Psychanalyse et interprétation \\[0pt]
Psychologie et contrôle des comportements \\[0pt]
Psychologie et ethnologie \\[0pt]
Psychologie et métaphysique \\[0pt]
Psychologie et neurosciences \\[0pt]
Psychologie et philosophie \\[0pt]
Publier \\[0pt]
Puis-je aimer tous les hommes ? \\[0pt]
Puis-je avoir la certitude que mes choix sont libres ? \\[0pt]
Puis-je comprendre autrui ? \\[0pt]
Puis-je décider de croire ? \\[0pt]
Puis-je devenir moi-même ? \\[0pt]
Puis-je devenir un autre ? \\[0pt]
Puis-je dire « ceci est mon corps » ? \\[0pt]
Puis-je douter de ma propre existence ? \\[0pt]
Puis-je être dans le vrai sans le savoir ? \\[0pt]
Puis-je être heureux dans un monde chaotique ? \\[0pt]
Puis-je être libre sans être responsable ? \\[0pt]
Puis-je être libre tout seul ? \\[0pt]
Puis-je être sûr de bien agir ? \\[0pt]
Puis-je être sûr de ne pas me tromper ? \\[0pt]
Puis-je être sûr que je ne rêve pas ? \\[0pt]
Puis-je être universel ? \\[0pt]
Puis-je faire ce que je veux de mon corps ? \\[0pt]
Puis-je faire confiance à mes sens ? \\[0pt]
Puis-je invoquer l'inconscient sans ruiner la morale ? \\[0pt]
Puis-je juger une culture à laquelle je n'appartiens pas ? \\[0pt]
Puis-je me connaître sans l'aide des autres ? \\[0pt]
Puis-je me connaître si je change à travers le temps ? \\[0pt]
Puis-je me mettre à la place d'un autre ? \\[0pt]
Puis-je me passer d'imiter autrui ? \\[0pt]
Puis-je me trouver moi-même sans l'aide d'autrui ? \\[0pt]
Puis-je ne croire que ce que je vois ? \\[0pt]
Puis-je ne pas vouloir ce que je désire ? \\[0pt]
Puis-je ne rien croire ? \\[0pt]
Puis-je ne rien devoir à personne ? \\[0pt]
Puis-je répondre des autres ? \\[0pt]
Puis-je savoir avec assurance en quoi consiste mon devoir ? \\[0pt]
Puis-je savoir ce qui m'est propre ? \\[0pt]
Puissance et pouvoir \\[0pt]
Pulsion et instinct \\[0pt]
Pulsions et passions \\[0pt]
Punir \\[0pt]
Punir ou soigner ? \\[0pt]
Punition et vengeance \\[0pt]
Qu'admire-t-on dans une œuvre d'art ? \\[0pt]
Qu'ai-je le droit d'exiger d'autrui ? \\[0pt]
Qu'ai-je le droit d'exiger des autres ? \\[0pt]
Qu'aime-t-on ? \\[0pt]
Qu'aime-t-on dans l'amour ? \\[0pt]
Qu'aime-t-on quand on aime ? \\[0pt]
Qu'aime-t-on quand on aime une œuvre d'art ? \\[0pt]
Qu'aimons-nous dans l'amour ? \\[0pt]
Qualité et quantité \\[0pt]
Qualités premières et qualités secondes \\[0pt]
Qualités premières, qualités secondes \\[0pt]
Quand agit-on ? \\[0pt]
Quand est-on stupide ? \\[0pt]
Quand faut-il désobéir ? \\[0pt]
Quand faut-il désobéir aux lois ? \\[0pt]
Quand faut-il mentir ? \\[0pt]
Quand faut-il se taire ? \\[0pt]
Quand la guerre finira-t-elle ? \\[0pt]
Quand l'art est-il abstrait ? \\[0pt]
Quand la technique devient-elle art ? \\[0pt]
Quand le temps passe, que reste-t-il ? \\[0pt]
Quand manquons-nous à nos devoirs ? \\[0pt]
Quand pense-t-on ? \\[0pt]
Quand peut-on se passer de théories ? \\[0pt]
Quand suis-je en faute ? \\[0pt]
Quand suis-je moi-même ? \\[0pt]
Quand une autorité est-elle légitime ? \\[0pt]
Quand y a-t-il art ? \\[0pt]
Quand y a-t-il de l'art ? \\[0pt]
Quand y a-t-il métaphore ? \\[0pt]
Quand y a-t-il œuvre ? \\[0pt]
Quand y a-t-il paysage ? \\[0pt]
Quand y a-t-il peuple ? \\[0pt]
Qu'anticipent les romans d'anticipation ? \\[0pt]
Quantification et intelligibilité \\[0pt]
Quantification et pensée scientifique \\[0pt]
Quantifier \\[0pt]
Quantité et qualité \\[0pt]
Qu'a perdu le fou ? \\[0pt]
Qu'appelle-t-on chef-d'œuvre ? \\[0pt]
Qu'appelle-t-on destin ? \\[0pt]
Qu'appelle-t-on penser ? \\[0pt]
Qu'apporte au mathématicien l'histoire des mathématiques ? \\[0pt]
Qu'apporte la photographie aux arts ? \\[0pt]
Qu'apporte l'histoire à la vie politique ? \\[0pt]
Qu'apprend-on dans les livres ? \\[0pt]
Qu'apprend-on de ses erreurs ? \\[0pt]
Qu'apprend-on des romans ? \\[0pt]
Qu'apprend-on en apprenant une technique ? \\[0pt]
Qu'apprend-on en commettant une faute ? \\[0pt]
Qu'apprend-on quand on apprend à parler ? \\[0pt]
Qu'apprenons-nous de nos affects ? \\[0pt]
Qu'apprenons-nous en éduquant ? \\[0pt]
Qu'a-t-on le droit de pardonner ? \\[0pt]
Qu'a-t-on le droit d'exiger ? \\[0pt]
Qu'a-t-on le droit d'interdire ? \\[0pt]
Qu'a-t-on le droit d'interpréter ? \\[0pt]
Qu'attendons-nous de la science ? \\[0pt]
Qu'attendons-nous de la technique ? \\[0pt]
Qu'attendons-nous des œuvres d'art ? \\[0pt]
Qu'attendons-nous pour être heureux ? \\[0pt]
Qu'attendre de l'État ? \\[0pt]
Qu'avons-nous à apprendre de nos illusions ? \\[0pt]
Qu'avons-nous à apprendre des historiens ? \\[0pt]
 Qu'avons-nous à attendre de la vie politique ? \\[0pt]
Qu'avons-nous en commun ? \\[0pt]
Que célèbre l'art ? \\[0pt]
Que changent les révolutions ? \\[0pt]
Qu'échange-t-on ? \\[0pt]
Que cherchons-nous dans le regard des autres ? \\[0pt]
Que choisir ? \\[0pt]
Que comprend-on d'une œuvre d'art ? \\[0pt]
Que connaissons-nous du vivant ? \\[0pt]
Que connaît-on de ses passions ? \\[0pt]
Que construit le politique ? \\[0pt]
Que coûte une victoire ? \\[0pt]
Que crée l'artiste ? \\[0pt]
Que déduire d'une contradiction ? \\[0pt]
Que démontrent nos actions ? \\[0pt]
Que désire-t-on ? \\[0pt]
Que désirons-nous ? \\[0pt]
Que désirons-nous quand nous désirons savoir ? \\[0pt]
Que devons-nous à autrui ? \\[0pt]
Que devons-nous à l'animal ? \\[0pt]
Que devons-nous à l'État ? \\[0pt]
Que disent les légendes ? \\[0pt]
Que disent les tables de vérité ? \\[0pt]
Que dit la loi ? \\[0pt]
Que dit la musique ? \\[0pt]
Que dit l'interdit ? \\[0pt]
Que dois-je à autrui ? \\[0pt]
Que dois-je à l'État ? \\[0pt]
Que dois-je faire ? \\[0pt]
Que dois-je respecter en autrui ? \\[0pt]
Que doit la pensée à l'écriture ? \\[0pt]
Que doit l'art à la nature ? \\[0pt]
Que doit la science à la technique ? \\[0pt]
Que doit-on à autrui ? \\[0pt]
Que doit-on à l'État ? \\[0pt]
Que doit-on aux morts ? \\[0pt]
Que doit-on croire ? \\[0pt]
Que doit-on désirer pour ne pas être déçu ? \\[0pt]
Que doit-on faire de ses rêves ? \\[0pt]
Que doit-on protéger ? \\[0pt]
Que doit-on savoir avant d'agir ? \\[0pt]
Que faire ? \\[0pt]
Que faire de la diversité des arts ? \\[0pt]
Que faire de la violence ? \\[0pt]
Que faire de nos émotions ? \\[0pt]
Que faire de nos passions ? \\[0pt]
Que faire de notre cerveau ? \\[0pt]
Que faire des adversaires ? \\[0pt]
Que faire du vraisemblable ? \\[0pt]
Que faire quand la loi est injuste ? \\[0pt]
Que fait aux œuvres d'art leur reproductibilité ? \\[0pt]
Que fait la machine ? \\[0pt]
Que fait la police ? \\[0pt]
Que fait l'art à nos vies ? \\[0pt]
Que fait l'artiste ? \\[0pt]
Que fait le spectateur ? \\[0pt]
Que faut-il absolument savoir ? \\[0pt]
Que faut-il craindre ? \\[0pt]
Que faut-il pour faire un monde ? \\[0pt]
Que faut-il respecter ? \\[0pt]
Que faut-il savoir pour agir ? \\[0pt]
Que faut-il savoir pour bien agir ? \\[0pt]
Que faut-il savoir pour gouverner ? \\[0pt]
Que faut-il savoir pour pouvoir gouverner ? \\[0pt]
Que faut-il vouloir pour être heureux ? \\[0pt]
Que gagne l'art à devenir abstrait ? \\[0pt]
Que gagne l'art à se réfléchir ? \\[0pt]
Que gagne-t-on à se mettre à la place d'autrui ? \\[0pt]
Que gagne-t-on à travailler ? \\[0pt]
Que garantit la séparation des pouvoirs ? \\[0pt]
Que la nature soit explicable, est-ce explicable ? \\[0pt]
Quel besoin avons-nous de chercher la vérité ? \\[0pt]
Quel contrôle a-t-on sur son corps ? \\[0pt]
Quel est le bon nombre d'amis ? \\[0pt]
Quel est le but de la politique ? \\[0pt]
Quel est le but d'une théorie physique ? \\[0pt]
Quel est le but du travail scientifique ? \\[0pt]
Quel est le contraire du travail ? \\[0pt]
Quel est le fondement de la propriété ? \\[0pt]
Quel est le fondement de l'autorité ? \\[0pt]
Quel est le meilleur régime politique ? \\[0pt]
Quel est le poids du passé ? \\[0pt]
Quel est le pouvoir de la beauté ? \\[0pt]
Quel est le pouvoir de l'art ? \\[0pt]
Quel est le pouvoir des métaphores ? \\[0pt]
Quel est le pouvoir des mots ? \\[0pt]
Quel est le problème posé par la pluralité des langues ? \\[0pt]
Quel est le propre de la pensée technique ? \\[0pt]
Quel est le rôle de la créativité dans les sciences ? \\[0pt]
Quel est le rôle du concept en art ? \\[0pt]
Quel est le rôle du médecin ? \\[0pt]
Quel est le sens du progrès technique ? \\[0pt]
Quel est le statut de la preuve en sciences humaines et sociales ? \\[0pt]
Quel est le sujet de la pensée ? \\[0pt]
Quel est le sujet de l'histoire ? \\[0pt]
Quel est le sujet des sciences sociales ? \\[0pt]
Quel est le sujet du devenir ? \\[0pt]
Quel est l'être de l'illusion ? \\[0pt]
Quel est l'homme de l'ethnologie ? \\[0pt]
Quel est l'homme des Droits de l'homme ? \\[0pt]
Quel est l'humain des sciences humaines ? \\[0pt]
Quel est l'objectif des sciences humaines ? \\[0pt]
Quel est l'objet de la biologie ? \\[0pt]
Quel est l'objet de la conscience de soi ? \\[0pt]
Quel est l'objet de la géométrie ? \\[0pt]
Quel est l'objet de la logique ? \\[0pt]
Quel est l'objet de la métaphysique ? \\[0pt]
Quel est l'objet de l'amour ? \\[0pt]
Quel est l'objet de la perception ? \\[0pt]
Quel est l'objet de la philosophie politique ? \\[0pt]
Quel est l'objet de la physique ? \\[0pt]
Quel est l'objet de la psychologie ? \\[0pt]
Quel est l'objet de la psychologie collective ? \\[0pt]
Quel est l'objet de la psychologie sociale ? \\[0pt]
Quel est l'objet de la science ? \\[0pt]
Quel est l'objet de l'échange ? \\[0pt]
Quel est l'objet de l'esthétique ? \\[0pt]
Quel est l'objet de l'histoire ? \\[0pt]
Quel est l'objet de l'ontologie ? \\[0pt]
Quel est l'objet des mathématiques ? \\[0pt]
Quel est l'objet des « sciences de l'esprit » ? \\[0pt]
Quel est l'objet des sciences humaines ? \\[0pt]
Quel est l'objet des sciences politiques ? \\[0pt]
Quel est l'objet du désir ? \\[0pt]
Quel être peut être un sujet de droits ? \\[0pt]
Quel genre de conscience peut-on accorder à l'animal ? \\[0pt]
Quel intérêt y a-t-il à se connaître ? \\[0pt]
Quelle causalité pour le vivant ? \\[0pt]
« Quelle chimère est-ce donc que l'homme » ? \\[0pt]
Quelle confiance accorder au langage ? \\[0pt]
Quelle différence y a-t-il entre un corps vivant et un corps mort ? \\[0pt]
Quelle espèce de vivants sommes-nous ? \\[0pt]
Quelle est la cause du désir ? \\[0pt]
Quelle est la fin de la science ? \\[0pt]
Quelle est la fin de l'État ? \\[0pt]
Quelle est la fonction première de l'État ? \\[0pt]
Quelle est la force de la loi ? \\[0pt]
Quelle est la limite du pouvoir de l'État ? \\[0pt]
Quelle est la matière de l'œuvre d'art ? \\[0pt]
Quelle est la nature du droit naturel ? \\[0pt]
Quelle est la place de l'imagination dans la vie de l'esprit ? \\[0pt]
Quelle est la place du hasard dans l'histoire ? \\[0pt]
Quelle est la place du sujet dans les sciences humaines ? \\[0pt]
Quelle est la portée d'un exemple ? \\[0pt]
Quelle est la réalité de la matière ? \\[0pt]
Quelle est la réalité de l'avenir ? \\[0pt]
Quelle est la réalité des objets mathématiques ? \\[0pt]
Quelle est la réalité d'une idée ? \\[0pt]
Quelle est la réalité du passé ? \\[0pt]
Quelle est la réalité du temps ? \\[0pt]
Quelle est la source de nos devoirs ? \\[0pt]
Quelle est la spécificité de la communauté politique ? \\[0pt]
Quelle est la valeur culturelle de la science ? \\[0pt]
Quelle est la valeur de la connaissance ? \\[0pt]
Quelle est la valeur de l'expérience ? \\[0pt]
Quelle est la valeur des hypothèses ? \\[0pt]
Quelle est la valeur d'une expérimentation ? \\[0pt]
Quelle est la valeur d'une œuvre d'art ? \\[0pt]
Quelle est la valeur d'une promesse ? \\[0pt]
Quelle est la valeur du rêve ? \\[0pt]
Quelle est la valeur du témoignage ? \\[0pt]
Quelle est la valeur du temps ? \\[0pt]
Quelle est la valeur du vivant ? \\[0pt]
Quelle est l'unité du « je » ? \\[0pt]
Quelle expérience la politique requiert-elle ? \\[0pt]
Quelle idée le fanatique se fait-il de la vérité ? \\[0pt]
Quelle maîtrise avons-nous du temps ? \\[0pt]
Quelle peut être la force de nos idées ? \\[0pt]
Quelle place donner au vraisemblable ? \\[0pt]
Quelle place la raison peut-elle faire à la croyance ? \\[0pt]
Quelle place les sciences humaines doivent-elles accorder à la subjectivité ? \\[0pt]
Quelle place pour la mesure dans les sciences de l'homme ? \\[0pt]
Quelle politique fait-on avec les sciences humaines ? \\[0pt]
Quelle réalité attribuer à la matière ? \\[0pt]
Quelle réalité conférer aux fictions ? \\[0pt]
Quelle réalité l'art nous fait-il connaître ? \\[0pt]
Quelle réalité la science décrit-elle ? \\[0pt]
Quelle réalité l'imagination nous fait-elle connaître ? \\[0pt]
Quelle réalité peut-on accorder au temps ? \\[0pt]
Quelle réalité peut-on attribuer au temps ? \\[0pt]
Quelle responsabilité publique pour le chercheur en sciences sociales ? \\[0pt]
Quelles actions permettent d'être heureux ? \\[0pt]
Quelles limites assigner au pouvoir de la parole ? \\[0pt]
Quelles limites poser à l'autorité de l'État ? \\[0pt]
Quelles lois pour les sciences humaines ? \\[0pt]
Quelle sorte d'histoire ont les sciences ? \\[0pt]
Quelles règles la technique dicte-t-elle à l'art ? \\[0pt]
Quelles sont les caractéristiques d'une proposition morale ? \\[0pt]
Quelles sont les caractéristiques d'un être vivant ? \\[0pt]
Quelles sont les limites de la démonstration ? \\[0pt]
Quelles sont les limites de la souveraineté ? \\[0pt]
Quelles sont les limites de mon monde ? \\[0pt]
Quelle valeur accorder à l'expérience ? \\[0pt]
Quelle valeur devons accorder à l'expérience ? \\[0pt]
Quelle valeur devons-nous accorder à l'expérience ? \\[0pt]
Quelle valeur devons-nous accorder à l'intuition ? \\[0pt]
Quelle valeur donner à la notion de « corps social » ? \\[0pt]
Quelle valeur peut-on accorder à l'expérience ? \\[0pt]
« Quelle vanité que la peinture » \\[0pt]
Quelle vérité y a-t-il dans la perception ? \\[0pt]
Quel réel pour l'art ? \\[0pt]
Quel rôle attribuer à l'intuition \emph{a priori} dans une philosophie des mathématiques ? \\[0pt]
Quel rôle la logique joue-t-elle en mathématiques ? \\[0pt]
Quel rôle les statistiques jouent-elles dans les sciences humaines ? \\[0pt]
Quel rôle l'imagination joue-t-elle en mathématiques ? \\[0pt]
Quels critères de démarcation pour les sciences sociales ? \\[0pt]
Quels désirs dois-je m'interdire ? \\[0pt]
Quels devoirs les religions peuvent-elles énoncer ? \\[0pt]
Quel sens donner à l'expression « gagner sa vie » ? \\[0pt]
Quel sens donner au projet de forger une langue parfaite ? \\[0pt]
Quels enseignements peut-on tirer de l'histoire des sciences ? \\[0pt]
Quel sens y a-t-il à se demander si les sciences humaines sont vraiment des sciences ? \\[0pt]
Quels matériaux pour les sciences humaines et sociales ? \\[0pt]
Quels sont les droits de la conscience ? \\[0pt]
Quels sont les fondements de l'autorité ? \\[0pt]
Quels sont les moyens légitimes de la politique ? \\[0pt]
Quel statut philosophique pour l'opinion ? \\[0pt]
Quel type de réalité faut-il attribuer à l'inconscient en psychanalyse ? \\[0pt]
Quel type de réalité faut-il attribuer au temps ? \\[0pt]
Quel usage faire de la finalité ? \\[0pt]
Quel usage faut-il faire des exemples ? \\[0pt]
Quel usage peut-on faire des fictions ? \\[0pt]
Quel vivant sommes-nous ? \\[0pt]
Que manque-t-il à une machine pour être vivante ? \\[0pt]
Que manque-t-il aux machines pour être des organismes ? \\[0pt]
Que mesure-t-on du temps ? \\[0pt]
Que montre l'image ? \\[0pt]
Que montre une démonstration ? \\[0pt]
Que montre un tableau ? \\[0pt]
Que m'ordonne ma conscience ? \\[0pt]
Que ne peut-on pas expliquer ? \\[0pt]
Que nous append l'histoire ? \\[0pt]
Que nous apporte l'art ? \\[0pt]
Que nous apporte la technique ? \\[0pt]
Que nous apporte la vérité ? \\[0pt]
Que nous apporte le travail ? \\[0pt]
Que nous apprend la biologie sur l'homme ? \\[0pt]
Que nous apprend la définition de la vérité ? \\[0pt]
Que nous apprend la diversité des langues ? \\[0pt]
Que nous apprend la fiction sur la réalité ? \\[0pt]
Que nous apprend la géométrie ? \\[0pt]
Que nous apprend la grammaire ? \\[0pt]
Que nous apprend la littérature ? \\[0pt]
Que nous apprend la maladie ? \\[0pt]
Que nous apprend la maladie sur la santé ? \\[0pt]
Que nous apprend la métaphysique ? \\[0pt]
Que nous apprend la musique ? \\[0pt]
Que nous apprend l'animal ? \\[0pt]
Que nous apprend l'anthropologie ? \\[0pt]
Que nous apprend la poésie ? \\[0pt]
Que nous apprend l'approche sociologique de l'art ? \\[0pt]
Que nous apprend la psychanalyse de l'homme ? \\[0pt]
Que nous apprend la religion ? \\[0pt]
Que nous apprend la sociologie des sciences ? \\[0pt]
Que nous apprend la technique ? \\[0pt]
Que nous apprend la vie ? \\[0pt]
Que nous apprend le cinéma ? \\[0pt]
Que nous apprend le faux ? \\[0pt]
Que nous apprend le plaisir ? \\[0pt]
Que nous apprend l'erreur ? \\[0pt]
Que nous apprend le témoignage ? \\[0pt]
Que nous apprend le théâtre ? \\[0pt]
Que nous apprend le toucher ? \\[0pt]
Que nous apprend l'étude du cerveau ? \\[0pt]
Que nous apprend l'expérience ? \\[0pt]
Que nous apprend l'expérience esthétique ? \\[0pt]
Que nous apprend l'histoire de l'art ? \\[0pt]
Que nous apprend l'histoire des sciences ? \\[0pt]
Que nous apprend, sur la politique, l'utopie ? \\[0pt]
Que nous apprennent les algorithmes sur nos sociétés ? \\[0pt]
Que nous apprennent les Anciens ? \\[0pt]
Que nous apprennent les animaux ? \\[0pt]
Que nous apprennent les animaux sur nous-mêmes ? \\[0pt]
Que nous apprennent les controverses scientifiques ? \\[0pt]
Que nous apprennent les expériences de pensée ? \\[0pt]
Que nous apprennent les faits divers ? \\[0pt]
Que nous apprennent les illusions d'optique ? \\[0pt]
Que nous apprennent les jeux ? \\[0pt]
Que nous apprennent les langues étrangères ? \\[0pt]
Que nous apprennent les machines ? \\[0pt]
Que nous apprennent les métaphores ? \\[0pt]
Que nous apprennent les mythes ? \\[0pt]
Que nous apprennent les objets techniques ? \\[0pt]
Que nous apprennent les statistiques ? \\[0pt]
Que nous apprennent nos erreurs ? \\[0pt]
Que nous apprennent nos sentiments ? \\[0pt]
Que nous devons-nous ? \\[0pt]
Que nous doit l'État ? \\[0pt]
Que nous enseigne la lecture des romans ? \\[0pt]
Que nous enseigne la monstruosité ? \\[0pt]
Que nous enseigne l'expérience ? \\[0pt]
Que nous enseignent les œuvres d'art ? \\[0pt]
Que nous enseignent les sens ? \\[0pt]
Que nous enseignent nos peurs ? \\[0pt]
Que nous impose la nature ? \\[0pt]
Que nous impose le temps ? \\[0pt]
Que nous montre le cinéma ? \\[0pt]
Que nous montre l'œuvre d'art ? \\[0pt]
Que nous montrent les natures mortes ? \\[0pt]
Que nous réserve l'avenir ? \\[0pt]
« Que nul n'entre ici s'il n'est géomètre » \\[0pt]
Que nul n'entre ici s'il n'est géomètre \\[0pt]
Que partage-t-on avec les animaux ? \\[0pt]
Que peindre ? \\[0pt]
Que peint le peintre ? \\[0pt]
Que penser de l'adage : « Que la justice s'accomplisse, le monde dût-il périr » (Fiat justitia pereat mundus) ? \\[0pt]
Que penser de la formule : « il faut suivre la nature » ? \\[0pt]
Que penser de l'opposition travail manuel, travail intellectuel ? \\[0pt]
Que percevons-nous ? \\[0pt]
Que percevons-nous d'autrui ? \\[0pt]
Que percevons-nous du monde extérieur ? \\[0pt]
Que perçoit-on ? \\[0pt]
Que perd la pensée en perdant l'écriture ? \\[0pt]
Que perd-on quand on perd son temps ? \\[0pt]
Que perdrait la pensée en perdant l'écriture ? \\[0pt]
Que peut expliquer l'histoire ? \\[0pt]
Que peut la force ? \\[0pt]
Que peut la musique ? \\[0pt]
Que peut la pensée ? \\[0pt]
Que peut la philosophie ? \\[0pt]
Que peut la politique ? \\[0pt]
Que peut la raison ? \\[0pt]
Que peut la raison contre une croyance ? \\[0pt]
Que peut l'art ? \\[0pt]
Que peut la science ? \\[0pt]
Que peut la théorie ? \\[0pt]
Que peut la volonté ? \\[0pt]
Que peut la volonté contre les passions ? \\[0pt]
Que peut le corps ? \\[0pt]
Que peut le droit ? \\[0pt]
Que peut le politique ? \\[0pt]
Que peut l'esprit ? \\[0pt]
Que peut l'esprit sur la matière ? \\[0pt]
Que peut l'État ? \\[0pt]
Que peut l'intelligence artificielle ? \\[0pt]
Que peut-on apprendre des émotions esthétiques ? \\[0pt]
Que peut-on attendre de l'État ? \\[0pt]
Que peut-on attendre du droit international ? \\[0pt]
Que peut-on attendre d'une philosophie de l'art ? \\[0pt]
Que peut-on calculer ? \\[0pt]
Que peut-on comprendre immédiatement ? \\[0pt]
Que peut-on comprendre qu'on ne puisse expliquer ? \\[0pt]
Que peut-on contre un préjugé ? \\[0pt]
Que peut-on cultiver ? \\[0pt]
Que peut-on démontrer ? \\[0pt]
Que peut-on dire de l'être ? \\[0pt]
Que peut-on échanger ? \\[0pt]
Que peut-on enseigner ? \\[0pt]
Que peut-on espérer ? \\[0pt]
Que peut-on exprimer ? \\[0pt]
Que peut-on interdire ? \\[0pt]
Que peut-on partager ? \\[0pt]
Que peut-on perdre ? \\[0pt]
Que peut-on savoir de l'inconscient ? \\[0pt]
Que peut-on savoir de soi ? \\[0pt]
Que peut-on savoir du réel ? \\[0pt]
Que peut-on savoir par expérience ? \\[0pt]
Que peut-on sur autrui ? \\[0pt]
Que peut-on voir ? \\[0pt]
Que peut prétendre imposer une religion ? \\[0pt]
Que peut signifier « avoir le sens de la réalité » ? \\[0pt]
Que peut signifier : « gérer son temps » ? \\[0pt]
Que peut-signifier « tuer le temps » ? \\[0pt]
Que peut un corps ? \\[0pt]
Que peuvent les idées ? \\[0pt]
Que peuvent les images ? \\[0pt]
Que peuvent les lois ? \\[0pt]
Que peuvent nous apprendre les expériences de pensée ? \\[0pt]
Que pouvons-nous attendre de la technique ? \\[0pt]
Que pouvons-nous aujourd'hui apprendre des sciences d'autrefois ? \\[0pt]
Que pouvons-nous comprendre du monde ? \\[0pt]
Que pouvons-nous connaître ? \\[0pt]
Que pouvons-nous espérer ? \\[0pt]
Que pouvons-nous espérer de la connaissance du vivant ? \\[0pt]
Que pouvons-nous faire de notre passé ? \\[0pt]
Que pouvons-nous partager ? \\[0pt]
Que produit la poésie ? \\[0pt]
Que produit l'inconscient ? \\[0pt]
Que protège la loi ? \\[0pt]
Que prouvent les faits ? \\[0pt]
Que prouvent les preuves de l'existence de Dieu ? \\[0pt]
Que puis-je dire être mien ? \\[0pt]
Que recherche l'artiste ? \\[0pt]
Que rend visible l'art ? \\[0pt]
Que répondre au sceptique ? \\[0pt]
Que reste-t-il d'une existence ? \\[0pt]
Que reste-t-il du passé ? \\[0pt]
Que sais-je d'autrui ? \\[0pt]
Que sais-je de ma souffrance ? \\[0pt]
Que sait la conscience ? \\[0pt]
Que sait la science ? \\[0pt]
Que sait-on de soi ? \\[0pt]
Que sait-on du réel ? \\[0pt]
Que savons-nous de l'avenir ? \\[0pt]
Que savons-nous de l'inconscient ? \\[0pt]
Que savons-nous de notre vie ? \\[0pt]
Que savons-nous de nous-mêmes ? \\[0pt]
Que savons-nous des principes ? \\[0pt]
Que sent le corps ? \\[0pt]
Que serait la fin de l'histoire ? \\[0pt]
Que serait la vie sans l'art ? \\[0pt]
Que serait le meilleur des mondes ? \\[0pt]
Que serait un art total ? \\[0pt]
Que serait une démocratie parfaite ? \\[0pt]
Que serait une pure sensation ? \\[0pt]
Que serions-nous sans l'État ? \\[0pt]
Que signifie apprendre ? \\[0pt]
Que signifie : « avoir de l'esprit » ? \\[0pt]
Que signifie connaître ? \\[0pt]
Que signifie « donner le change » ? \\[0pt]
Que signifie : « échanger des arguments » ? \\[0pt]
Que signifie être dépendant ? \\[0pt]
Que signifie être en guerre ? \\[0pt]
Que signifie être mortel ? \\[0pt]
Que signifie la mort ? \\[0pt]
Que signifie l'angoisse ? \\[0pt]
Que signifie l'expression : « l'histoire jugera » ? \\[0pt]
Que signifie l'idée de technoscience ? \\[0pt]
Que signifient les mots ? \\[0pt]
Que signifie « politiquement correct » ? \\[0pt]
Que signifie pour l'homme être mortel ? \\[0pt]
Que signifie prouver en sciences sociales ? \\[0pt]
Que signifie refuser l'injustice ? \\[0pt]
Que signifier « juger en son âme et conscience » ? \\[0pt]
Que signifie : « se rendre à l'évidence » ? \\[0pt]
Que sondent les sondages d'opinion ? \\[0pt]
Que sont les apparences ? \\[0pt]
Que sont les institutions sans les individus ? \\[0pt]
Que sont les objets de la perception ? \\[0pt]
Qu'est-ce la vérité ? \\[0pt]
Qu'est-ce qu'abstraire ? \\[0pt]
Qu'est-ce qu'agir avec discernement ? \\[0pt]
Qu'est-ce qu'agir ensemble ? \\[0pt]
Qu'est-ce qu'aimer une œuvre d'art ? \\[0pt]
Qu'est-ce qu'apparaître ? \\[0pt]
Qu'est-ce qu'appliquer une règle de droit ? \\[0pt]
Qu'est-ce qu'apprécier une œuvre d'art ? \\[0pt]
Qu'est-ce qu'apprendre ? \\[0pt]
Qu'est-ce qu'argumenter ? \\[0pt]
Qu'est-ce qu'avoir conscience de soi \\[0pt]
Qu'est-ce qu'avoir conscience de soi ? \\[0pt]
Qu'est-ce qu'avoir de l'expérience ? \\[0pt]
Qu'est-ce qu'avoir du goût ? \\[0pt]
Qu'est-ce qu'avoir du jugement ? \\[0pt]
Qu'est-ce qu'avoir du style ? \\[0pt]
Qu'est-ce qu'avoir raison ? \\[0pt]
Qu'est-ce qu'avoir un droit ? \\[0pt]
Qu'est-ce qu'avoir une idée ? \\[0pt]
Qu'est-ce qu'avoir un esprit scientifique ? \\[0pt]
Qu'est-ce que bien vivre ? \\[0pt]
Qu'est-ce que calculer ? \\[0pt]
Qu'est-ce que catégoriser ? \\[0pt]
Qu'est-ce que classer ? \\[0pt]
Qu'est-ce que commencer ? \\[0pt]
Qu'est-ce que composer une œuvre ? \\[0pt]
Qu'est-ce que comprendre ? \\[0pt]
Qu'est-ce que comprendre une œuvre d'art ? \\[0pt]
Qu'est-ce que connaître ? \\[0pt]
Qu'est-ce que contempler ? \\[0pt]
Qu'est-ce que créer ? \\[0pt]
Qu'est-ce que croire ? \\[0pt]
Qu'est-ce que décider ? \\[0pt]
Qu'est-ce que définir ? \\[0pt]
Qu'est-ce que démontrer ? \\[0pt]
Qu'est-ce que déraisonner ? \\[0pt]
Qu'est-ce que Dieu pour athée ? \\[0pt]
Qu'est-ce que Dieu pour un athée ? \\[0pt]
Qu'est-ce que discuter ? \\[0pt]
Qu'est-ce que donner sa parole ? \\[0pt]
Qu'est-ce qu'éduquer ? \\[0pt]
Qu'est-ce qu'éduquer le sens esthétique ? \\[0pt]
Qu'est-ce que faire autorité ? \\[0pt]
Qu'est-ce que « faire de la politique » ? \\[0pt]
Qu'est-ce que faire la paix ? \\[0pt]
Qu'est-ce que faire preuve d'humanité ? \\[0pt]
Qu'est-ce que faire une expérience ? \\[0pt]
Qu'est-ce que « faire usage de sa raison » ? \\[0pt]
Qu'est-ce que gouverner ? \\[0pt]
Qu'est-ce que guérir ? \\[0pt]
Qu'est-ce que jouer ? \\[0pt]
Qu'est-ce que juger ? \\[0pt]
Qu'est-ce que la barbarie ? \\[0pt]
Qu'est-ce que la causalité ? \\[0pt]
Qu'est-ce que la critique ? \\[0pt]
Qu'est-ce que la culture générale \\[0pt]
Qu'est-ce que la démocratie ? \\[0pt]
Qu'est-ce que la folie ? \\[0pt]
Qu'est-ce que la normalité ? \\[0pt]
Qu'est-ce que l'anthropologie politique ? \\[0pt]
Qu'est-ce que la perception ? \\[0pt]
Qu'est-ce que la politique ? \\[0pt]
Qu'est-ce que la psychanalyse nous apprend sur les sociétés humaines ? \\[0pt]
Qu'est-ce que la psychologie ? \\[0pt]
Qu'est-ce que la raison d'État ? \\[0pt]
Qu'est-ce que la réalité ? \\[0pt]
Qu'est-ce que la religion nous donne à savoir ? \\[0pt]
Qu'est-ce que l'art contemporain ? \\[0pt]
Qu'est-ce que la science doit à l'expérience ? \\[0pt]
Qu'est-ce que la science saisit du vivant ? \\[0pt]
Qu'est-ce que la science, si elle inclut la psychanalyse ? \\[0pt]
Qu'est-ce que la scientificité ? \\[0pt]
Qu'est-ce que la sociologie nous apprend sur la culture ? \\[0pt]
Qu'est-ce que la souveraineté ? \\[0pt]
Qu'est-ce que la technique ? \\[0pt]
Qu'est-ce que la technique doit à la nature ? \\[0pt]
Qu'est-ce que la tragédie ? \\[0pt]
Qu'est-ce que la valeur marchande ? \\[0pt]
Qu'est-ce que la vérité ? \\[0pt]
Qu'est-ce que la vie ? \\[0pt]
Qu'est-ce que la vie bonne ? \\[0pt]
Qu'est-ce que le bonheur ? \\[0pt]
Qu'est-ce que le cinéma a changé dans l'idée que l'on se fait du temps ? \\[0pt]
Qu'est-ce que le cinéma donne à voir ? \\[0pt]
Qu'est-ce que le courage ? \\[0pt]
Qu'est-ce que le désordre ? \\[0pt]
Qu'est-ce que le dogmatisme ? \\[0pt]
Qu'est-ce que le génie ? \\[0pt]
Qu'est-ce que le hasard ? \\[0pt]
Qu'est-ce que le langage ordinaire ? \\[0pt]
Qu'est-ce que le malheur ? \\[0pt]
Qu'est-ce que le mal radical ? \\[0pt]
Qu'est-ce que le mauvais goût ? \\[0pt]
Qu'est-ce que le moi ? \\[0pt]
Qu'est-ce que le naturalisme ? \\[0pt]
Qu'est-ce que l'enfance ? \\[0pt]
Qu'est-ce que le nihilisme ? \\[0pt]
Qu'est-ce que le pathologique nous apprend sur le normal ? \\[0pt]
Qu'est-ce que le présent ? \\[0pt]
Qu'est-ce que le progrès technique ? \\[0pt]
Qu'est-ce que le réel ? \\[0pt]
Qu'est-ce que le sacré ? \\[0pt]
Qu'est-ce que le sens pratique ? \\[0pt]
Qu'est-ce que les sciences humaines apportent à la critique littéraire ? \\[0pt]
Qu'est-ce que les sciences religieuses nous apprennent de la religion ? \\[0pt]
Qu'est-ce que le style ? \\[0pt]
Qu'est-ce que le sublime ? \\[0pt]
Qu'est-ce que le symbolique ? \\[0pt]
Qu'est-ce que le temps ? \\[0pt]
Qu'est-ce que le travail ? \\[0pt]
Qu'est-ce que l'expérience pour un empiriste ? \\[0pt]
Qu'est-ce que l'harmonie ? \\[0pt]
Qu'est-ce que l'identité sociale des individus ? \\[0pt]
Qu'est-ce que l'inconscient ? \\[0pt]
Qu'est-ce que l'indifférence ? \\[0pt]
Qu'est-ce que l'individualisme méthodologique ? \\[0pt]
Qu'est-ce que l'intérêt général ? \\[0pt]
Qu'est-ce que l'interprétation des Écritures nous apprend sur les religions ? \\[0pt]
Qu'est-ce que l'intuition ? \\[0pt]
Qu'est-ce que lire ? \\[0pt]
Qu'est-ce que l'objectivité scientifique ? \\[0pt]
Qu'est-ce que l'ordinaire ? \\[0pt]
Qu'est-ce que maîtriser une technique ? \\[0pt]
Qu'est-ce que manquer de culture ? \\[0pt]
Qu'est-ce que méditer ? \\[0pt]
Qu'est-ce que mesurer le temps ? \\[0pt]
Qu'est-ce que mourir ? \\[0pt]
Qu'est-ce que nous apprennent les animaux sur nous-même ? \\[0pt]
Qu'est-ce qu'enquêter ? \\[0pt]
Qu'est-ce qu'enseigner ? \\[0pt]
Qu'est-ce que parler ? \\[0pt]
Qu'est-ce que parler à-propos ? \\[0pt]
Qu'est-ce que « parler le même langage » ? \\[0pt]
Qu'est-ce que parler le même langage ? \\[0pt]
Qu'est-ce que parler pour ne rien dire ? \\[0pt]
Qu'est-ce que parler veut dire ? \\[0pt]
Qu'est-ce que penser ? \\[0pt]
Qu'est-ce que percevoir ? \\[0pt]
Qu'est-ce que perdre la raison ? \\[0pt]
Qu'est-ce que perdre sa liberté ? \\[0pt]
Qu'est-ce que perdre son temps ? \\[0pt]
Qu'est-ce que peut un corps ? \\[0pt]
Qu'est-ce que prendre conscience ? \\[0pt]
Qu'est-ce que prendre le pouvoir ? \\[0pt]
Qu'est-ce que produire ? \\[0pt]
Qu'est-ce que promettre ? \\[0pt]
Qu'est-ce que prouver ? \\[0pt]
Qu'est-ce que raisonner ? \\[0pt]
Qu'est-ce que réfuter ? \\[0pt]
Qu'est-ce que réfuter une philosophie ? \\[0pt]
Qu'est-ce que rendre raison d'un effet ? \\[0pt]
Qu'est-ce que résister ? \\[0pt]
Qu'est-ce que résoudre une contradiction ? \\[0pt]
Qu'est-ce que « rester soi-même » ? \\[0pt]
Qu'est-ce que rester soi-même ? \\[0pt]
Qu'est-ce que réussir sa vie ? \\[0pt]
Qu'est-ce que savoir ? \\[0pt]
Qu'est-ce que se cultiver ? \\[0pt]
Qu'est-ce que « se rendre maître et possesseur de la nature » ? \\[0pt]
Qu'est-ce que se tromper ? \\[0pt]
Qu'est-ce que s'orienter ? \\[0pt]
Qu'est-ce que témoigner ? \\[0pt]
Qu'est-ce que traduire ? \\[0pt]
Qu'est-ce que travailler ? \\[0pt]
Qu'est-ce qu'être ? \\[0pt]
Qu'est-ce qu'être adulte ? \\[0pt]
Qu'est-ce qu'être artiste ? \\[0pt]
Qu'est-ce qu'être asocial ? \\[0pt]
Qu'est-ce qu'être barbare ? \\[0pt]
Qu'est-ce qu'être chez soi ? \\[0pt]
Qu'est-ce qu'être cohérent ? \\[0pt]
Qu'est-ce qu'être comportementaliste ? \\[0pt]
Qu'est-ce qu'être cultivé ? \\[0pt]
Qu'est-ce qu'être dans le vrai ? \\[0pt]
Qu'est-ce qu'être de bonne volonté ? \\[0pt]
Qu'est-ce qu'être démocrate ? \\[0pt]
Qu'est-ce qu'être de son temps ? \\[0pt]
Qu'est-ce qu'être efficace en politique ? \\[0pt]
Qu'est-ce qu'être ensemble ? \\[0pt]
Qu'est-ce qu'être en vie ? \\[0pt]
Qu'est-ce qu'être esclave ? \\[0pt]
Qu'est-ce qu'être fidèle à soi-même ? \\[0pt]
Qu'est-ce qu'être fou ? \\[0pt]
Qu'est-ce qu'être généreux ? \\[0pt]
Qu'est-ce qu'être hors de soi ? \\[0pt]
Qu'est-ce qu'être idéaliste ? \\[0pt]
Qu'est-ce qu'être inhumain ? \\[0pt]
Qu'est-ce qu'être l'auteur de son acte ? \\[0pt]
Qu'est-ce qu'être libéral ? \\[0pt]
Qu'est-ce qu'être libre ? \\[0pt]
Qu'est-ce qu'être maître de soi ? \\[0pt]
Qu'est-ce qu'être maître de soi-même ? \\[0pt]
Qu'est-ce qu'être malade ? \\[0pt]
Qu'est-ce qu'être matérialiste ? \\[0pt]
Qu'est-ce qu'être moderne ? \\[0pt]
Qu'est-ce qu'être nihiliste ? \\[0pt]
Qu'est-ce qu'être normal ? \\[0pt]
Qu'est-ce qu'être psychologue ? \\[0pt]
Qu'est-ce qu'être rationaliste ? \\[0pt]
Qu'est-ce qu'être rationnel ? \\[0pt]
Qu'est-ce qu'être réaliste ? \\[0pt]
Qu'est-ce qu'être républicain ? \\[0pt]
Qu'est-ce qu'être sceptique ? \\[0pt]
Qu'est-ce qu'être seul ? \\[0pt]
Qu'est-ce qu'être simple ? \\[0pt]
Qu'est-ce qu'être soi-même ? \\[0pt]
Qu'est-ce qu'être souverain ? \\[0pt]
Qu'est-ce qu'être spirituel ? \\[0pt]
Qu'est-ce qu'être témoin ? \\[0pt]
Qu'est-ce qu'être un bon citoyen ? \\[0pt]
Qu'est-ce qu'être un esclave ? \\[0pt]
Qu'est-ce qu'être un sujet ? \\[0pt]
Qu'est-ce qu'être vertueux ? \\[0pt]
Qu'est-ce qu'être visionnaire ? \\[0pt]
Qu'est-ce qu'être vivant ? \\[0pt]
Qu'est-ce que un individu \\[0pt]
Qu'est-ce que vérifier ? \\[0pt]
Qu'est-ce que vérifier une théorie ? \\[0pt]
Qu'est-ce que vieillir ? \\[0pt]
Qu'est-ce que vivre ? \\[0pt]
Qu'est-ce que vivre au présent ? \\[0pt]
Qu'est-ce que vivre bien ? \\[0pt]
Qu'est-ce que vivre vertueusement ? \\[0pt]
Qu'est-ce que voir ? \\[0pt]
Qu'est-ce qu'exercer un pouvoir ? \\[0pt]
Qu'est-ce qu'exister ? \\[0pt]
Qu'est-ce qu'exister pour un individu ? \\[0pt]
Qu'est-ce qu'expliquer ? \\[0pt]
Qu'est-ce qu'expliquer un comportement ? \\[0pt]
Qu'est-ce qu'habiter ? \\[0pt]
Qu'est-ce qui adoucit les mœurs ? \\[0pt]
Qu'est-ce qui a du sens ? \\[0pt]
Qu'est-ce qui agit ? \\[0pt]
Qu'est-ce qui anime l'animal ? \\[0pt]
Qu'est-ce qui apparaît ? \\[0pt]
Qu'est-ce qui a un sens ? \\[0pt]
Qu'est-ce qu'identifier ? \\[0pt]
Qu'est-ce qui dépend de nous ? \\[0pt]
Qu'est-ce qui distingue un argument d'une démonstration ? \\[0pt]
Qu'est-ce qui distingue un vivant d'une machine ? \\[0pt]
Qu'est-ce qui échappe aux sciences humaines ? \\[0pt]
Qu'est-ce qui est absurde ? \\[0pt]
Qu'est-ce qui est actuel ? \\[0pt]
Qu'est-ce qui est anormal ? \\[0pt]
Qu'est-ce qui est artificiel ? \\[0pt]
Qu'est-ce qui est beau ? \\[0pt]
Qu'est-ce qui est commun à toutes les langues ? \\[0pt]
Qu'est-ce qui est concret ? \\[0pt]
Qu'est-ce qui est contre nature ? \\[0pt]
Qu'est-ce qui est culturel ? \\[0pt]
Qu'est-ce qui est démonstratif ? \\[0pt]
Qu'est-ce qui est donné ? \\[0pt]
Qu'est-ce qui est essentiel ? \\[0pt]
Qu'est-ce qui est étonnant ? \\[0pt]
Qu'est-ce qui est extérieur à ma conscience, ? \\[0pt]
Qu'est-ce qui est historique ? \\[0pt]
Qu'est-ce qui est hors la loi ? \\[0pt]
Qu'est-ce qui est humain ? \\[0pt]
Qu'est-ce qui est immoral ? \\[0pt]
Qu'est-ce qui est impossible ? \\[0pt]
Qu'est-ce qui est indiscutable ? \\[0pt]
Qu'est-ce qui est injuste ? \\[0pt]
Qu'est-ce qui est insignifiant ? \\[0pt]
Qu'est-ce qui est intolérable ? \\[0pt]
Qu'est-ce qui est invérifiable ? \\[0pt]
Qu'est-ce qui est irrationnel ? \\[0pt]
Qu'est-ce qui est irréfutable ? \\[0pt]
Qu'est-ce qui est irréversible ? \\[0pt]
Qu'est-ce qui est le plus à craindre, l'ordre ou le désordre ? \\[0pt]
Qu'est-ce qui est mauvais dans l'égoïsme ? \\[0pt]
Qu'est-ce qui est mien ? \\[0pt]
Qu'est-ce qui est moderne ? \\[0pt]
Qu'est-ce qui est naturel ? \\[0pt]
Qu'est-ce qui est nécessaire ? \\[0pt]
Qu'est-ce qui est noble ? \\[0pt]
Qu'est-ce qui est normal ? \\[0pt]
Qu'est-ce qui est politique ? \\[0pt]
Qu'est-ce qui est possible ? \\[0pt]
Qu'est-ce qui est prométhéen ? \\[0pt]
Qu'est-ce qui est public ? \\[0pt]
Qu'est-ce qui est réel ? \\[0pt]
Qu'est-ce qui est respectable ? \\[0pt]
Qu'est-ce qui est sacré ? \\[0pt]
Qu'est-ce qui est sans raison ? \\[0pt]
Qu'est-ce qui est sauvage ? \\[0pt]
Qu'est-ce qui est scientifique ? \\[0pt]
Qu'est-ce qui est sensible ? \\[0pt]
Qu'est-ce qui est spectaculaire ? \\[0pt]
Qu'est-ce qui est sublime ? \\[0pt]
Qu'est-ce qui est tragique ? \\[0pt]
Qu'est-ce qui est transcendant ? \\[0pt]
Qu'est-ce qui est vital ? \\[0pt]
Qu'est-ce qui est vital pour le vivant ? \\[0pt]
Qu'est-ce qui est vivant ? \\[0pt]
Qu'est-ce qui exige d'être interprété ? \\[0pt]
Qu'est-ce qui existe ? \\[0pt]
Qu'est-ce qui fait autorité ? \\[0pt]
Qu'est-ce qui fait changer les sociétés ? \\[0pt]
Qu'est-ce qui fait des sciences humaines des sciences ? \\[0pt]
Qu'est-ce qui fait d'une activité un travail ? \\[0pt]
Qu'est-ce qui fait du scepticisme une philosophie ? \\[0pt]
Qu'est-ce qui fait la force de la loi ? \\[0pt]
Qu'est-ce qui fait la force des lois ? \\[0pt]
Qu'est-ce qui fait la justice des lois ? \\[0pt]
Qu'est-ce qui fait la légitimité d'une autorité politique ? \\[0pt]
Qu'est-ce qui fait la valeur de la technique ? \\[0pt]
Qu'est-ce qui fait la valeur de l'œuvre d'art ? \\[0pt]
Qu'est-ce qui fait la valeur des objets d'art ? \\[0pt]
Qu'est-ce qui fait la valeur d'une croyance ? \\[0pt]
Qu'est-ce qui fait la valeur d'une existence ? \\[0pt]
Qu'est-ce qui fait la valeur d'une œuvre ? \\[0pt]
Qu'est-ce qui fait la valeur d'une œuvre d'art ? \\[0pt]
Qu'est-ce qui fait le pouvoir des mots ? \\[0pt]
Qu'est-ce qui fait le propre d'un corps propre ? \\[0pt]
Qu'est-ce qui fait l'humanité d'un corps ? \\[0pt]
Qu'est-ce qui fait l'unité d'une science ? \\[0pt]
Qu'est-ce qui fait l'unité d'un organisme ? \\[0pt]
Qu'est-ce qui fait l'unité d'un peuple ? \\[0pt]
Qu'est-ce qui fait l'unité du vivant ? \\[0pt]
Qu'est-ce qui fait mon identité ? \\[0pt]
Qu'est-ce qui fait obstacle à la science ? \\[0pt]
Qu'est-ce qui fait qu'un corps est humain ? \\[0pt]
Qu'est-ce qui fait qu'une théorie est vraie ? \\[0pt]
Qu'est-ce qui fait qu'un peuple est un peuple ? \\[0pt]
Qu'est-ce qui fait un peuple ? \\[0pt]
Qu'est-ce qui fonde la croyance ? \\[0pt]
Qu'est-ce qui fonde le respect d'autrui ? \\[0pt]
Qu'est-ce qu'ignore la science ? \\[0pt]
Qu'est-ce qui importe ? \\[0pt]
Qu'est-ce qui importe dans l'éducation ? \\[0pt]
Qu'est-ce qui innocente le bourreau ? \\[0pt]
Qu'est-ce qui justifie l'hypothèse d'un inconscient ? \\[0pt]
Qu'est-ce qui justifie une croyance ? \\[0pt]
Qu'est-ce qu'imaginer ? \\[0pt]
Qu'est-ce qui m'appartient ? \\[0pt]
Qu'est-ce qui menace la liberté ? \\[0pt]
Qu'est-ce qui me rend plus fort ? \\[0pt]
Qu'est-ce qui mérite l'admiration ? \\[0pt]
Qu'est-ce qui m'est étranger ? \\[0pt]
Qu'est-ce qui mesure la valeur d'un travail ? \\[0pt]
Qu'est-ce qui n'a pas d'histoire ? \\[0pt]
Qu'est-ce qui n'appartient pas au monde ? \\[0pt]
Qu'est-ce qui ne change pas ? \\[0pt]
Qu'est-ce qui ne disparaît jamais ? \\[0pt]
Qu'est-ce qui ne s'achète pas ? \\[0pt]
Qu'est-ce qui ne s'échange pas ? \\[0pt]
Qu'est-ce qui n'est pas démontrable ? \\[0pt]
Qu'est-ce qui n'est pas en mouvement ? \\[0pt]
Qu'est-ce qui n'est pas maîtrisable ? \\[0pt]
Qu'est-ce qui n'est pas mathématisable ? \\[0pt]
Qu'est-ce qui n'est pas normal ? \\[0pt]
Qu'est-ce qui n'est pas politique ? \\[0pt]
Qu'est-ce qui n'est pas technique dans l'activité humaine ? \\[0pt]
Qu'est-ce qui n'existe pas ? \\[0pt]
Qu'est-ce qui nous échappe dans le temps ? \\[0pt]
Qu'est-ce qui nous fait danser ? \\[0pt]
Qu'est-ce qui nous oblige ? \\[0pt]
Qu'est-ce qu'interpréter ? \\[0pt]
Qu'est-ce qu'interpréter une œuvre d'art ? \\[0pt]
Qu'est-ce qu'interpréter un texte ? \\[0pt]
Qu'est-ce qui oppose les hommes ? \\[0pt]
Qu'est-ce qui peut être hors du temps ? \\[0pt]
Qu'est-ce qui peut se transformer ? \\[0pt]
Qu'est-ce qui plaît dans la musique ? \\[0pt]
Qu'est-ce qui rapproche le vivant de la machine ? \\[0pt]
Qu'est-ce qui rend l'objectivité difficile dans les sciences humaines ? \\[0pt]
Qu'est-ce qui rend une parole poétique ? \\[0pt]
Qu'est-ce qui rend vrai un énoncé ? \\[0pt]
Qu'est-ce qu'obéir ? \\[0pt]
Qu'est-ce qu'on attend ? \\[0pt]
Qu'est-ce qu'on ne peut comprendre ? \\[0pt]
Qu'est-ce qu'on ne peut pas partager ? \\[0pt]
Qu'est-ce qu'un abus de langage ? \\[0pt]
Qu'est-ce qu'un abus de pouvoir ? \\[0pt]
Qu'est-ce qu'un accident ? \\[0pt]
Qu'est-ce qu'un acte ? \\[0pt]
Qu'est-ce qu'un acte libre ? \\[0pt]
Qu'est-ce qu'un acte moral ? \\[0pt]
Qu'est-ce qu'un acte symbolique ? \\[0pt]
Qu'est-ce qu'un acteur ? \\[0pt]
Qu'est-ce qu'un adversaire en politique ? \\[0pt]
Qu'est-ce qu'un alter ego \\[0pt]
Qu'est-ce qu'un alter ego ? \\[0pt]
Qu'est-ce qu'un ami ? \\[0pt]
Qu'est-ce qu'un animal ? \\[0pt]
Qu'est-ce qu'un animal domestique ? \\[0pt]
Qu'est-ce qu'un argument ? \\[0pt]
Qu'est-ce qu'un art abstrait ? \\[0pt]
Qu'est-ce qu'un art de vivre ? \\[0pt]
Qu'est-ce qu'un artefact ? \\[0pt]
Qu'est-ce qu'un artiste ? \\[0pt]
Qu'est-ce qu'un art moral ? \\[0pt]
Qu'est-ce qu'un art populaire ? \\[0pt]
Qu'est-ce qu'un auteur ? \\[0pt]
Qu'est-ce qu'un axiome ? \\[0pt]
Qu'est-ce qu'un beau parleur ? \\[0pt]
Qu'est-ce qu'un beau paysage ? \\[0pt]
Qu'est-ce qu'un beau travail ? \\[0pt]
Qu'est-ce qu'un bon argument ? \\[0pt]
Qu'est-ce qu'un bon citoyen ? \\[0pt]
Qu'est-ce qu'un bon commentaire ? \\[0pt]
Qu'est-ce qu'un bon conseil ? \\[0pt]
Qu'est-ce qu'un bon gouvernement ? \\[0pt]
Qu'est-ce qu'un bon jugement ? \\[0pt]
Qu'est-ce qu'un capital culturel ? \\[0pt]
Qu'est-ce qu'un caractère ? \\[0pt]
Qu'est-ce qu'un cas de conscience ? \\[0pt]
Qu'est-ce qu'un cas en morale ? \\[0pt]
Qu'est-ce qu'un « champ artistique » ? \\[0pt]
Qu'est-ce qu'un châtiment ? \\[0pt]
Qu'est-ce qu'un chef ? \\[0pt]
Qu'est-ce qu'un chef-d'œuvre ? \\[0pt]
Qu'est-ce qu'un choix éclairé ? \\[0pt]
Qu'est-ce qu'un choix rationnel ? \\[0pt]
Qu'est-ce qu'un citoyen ? \\[0pt]
Qu'est-ce qu'un citoyen libre ? \\[0pt]
Qu'est-ce qu'un civilisé ? \\[0pt]
Qu'est-ce qu'un classique ? \\[0pt]
Qu'est-ce qu'un code ? \\[0pt]
Qu'est-ce qu'un collectif ? \\[0pt]
Qu'est-ce qu'un concept ? \\[0pt]
Qu'est-ce qu'un concept philosophique ? \\[0pt]
Qu'est-ce qu'un concept scientifique ? \\[0pt]
Qu'est-ce qu'un conflit de devoirs ? \\[0pt]
Qu'est-ce qu'un conflit de générations ? \\[0pt]
Qu'est-ce qu'un conflit de valeurs ? \\[0pt]
Qu'est-ce qu'un conflit politique ? \\[0pt]
Qu'est-ce qu'un consommateur ? \\[0pt]
Qu'est-ce qu'un contenu de conscience ? \\[0pt]
Qu'est-ce qu'un contrat ? \\[0pt]
Qu'est-ce qu'un contre-pouvoir ? \\[0pt]
Qu'est-ce qu'un corps ? \\[0pt]
Qu'est-ce qu'un corps politique ? \\[0pt]
Qu'est-ce qu'un corps social ? \\[0pt]
Qu'est-ce qu'un coup d'État ? \\[0pt]
Qu'est-ce qu'un créateur ? \\[0pt]
Qu'est-ce qu'un crime ? \\[0pt]
Qu'est-ce qu'un crime contre l'humanité ? \\[0pt]
Qu'est-ce qu'un crime politique ? \\[0pt]
Qu'est-ce qu'un critère de vérité ? \\[0pt]
Qu'est-ce qu'un déni ? \\[0pt]
Qu'est-ce qu'un désir satisfait ? \\[0pt]
Qu'est-ce qu'un détail ? \\[0pt]
Qu'est-ce qu'un dialogue ? \\[0pt]
Qu'est-ce qu'un dieu ? \\[0pt]
Qu'est-ce qu'un Dieu ? \\[0pt]
Qu'est-ce qu'un dilemme ? \\[0pt]
Qu'est-ce qu'un document ? \\[0pt]
Qu'est-ce qu'un dogme ? \\[0pt]
Qu'est-ce qu'un doute raisonnable ? \\[0pt]
Qu'est-ce qu'un droit naturel ? \\[0pt]
Qu'est-ce qu'une action intentionnelle ? \\[0pt]
Qu'est-ce qu'une action juste ? \\[0pt]
Qu'est-ce qu'une action politique ? \\[0pt]
Qu'est-ce qu'une action réussie ? \\[0pt]
Qu'est-ce qu'une alternative ? \\[0pt]
Qu'est-ce qu'une âme ? \\[0pt]
Qu'est-ce qu'une analyse ? \\[0pt]
Qu'est-ce qu'une anomalie ? \\[0pt]
Qu'est-ce qu'une aporie ? \\[0pt]
Qu'est-ce qu'une autorité légitime ? \\[0pt]
Qu'est-ce qu'une avant-garde ? \\[0pt]
Qu'est-ce qu'une belle action ? \\[0pt]
Qu'est-ce qu'une belle démonstration ? \\[0pt]
Qu'est-ce qu'une belle forme ? \\[0pt]
Qu'est-ce qu'une belle mort ? \\[0pt]
Qu'est-ce qu'une bête ? \\[0pt]
Qu'est-ce qu'une bonne définition ? \\[0pt]
Qu'est-ce qu'une bonne délibération ? \\[0pt]
Qu'est-ce qu'une bonne éducation ? \\[0pt]
Qu'est-ce qu'une bonne loi ? \\[0pt]
Qu'est-ce qu'une bonne méthode ? \\[0pt]
Qu'est-ce qu'une bonne théorie ? \\[0pt]
Qu'est-ce qu'une bonne traduction ? \\[0pt]
Qu'est-ce qu'une catastrophe ? \\[0pt]
Qu'est-ce qu'une catégorie ? \\[0pt]
Qu'est-ce qu'une catégorie de l'être ? \\[0pt]
Qu'est-ce qu'une cause ? \\[0pt]
Qu'est-ce qu'un échange juste ? \\[0pt]
Qu'est-ce qu'un échange réussi ? \\[0pt]
Qu'est-ce qu'une chimère ? \\[0pt]
Qu'est-ce qu'une chose ? \\[0pt]
Qu'est-ce qu'une chose matérielle ? \\[0pt]
Qu'est-ce qu'une civilisation ? \\[0pt]
Qu'est-ce qu'une classe sociale ? \\[0pt]
Qu'est-ce qu'une classification scientifique ? \\[0pt]
Qu'est-ce qu'une collectivité ? \\[0pt]
Qu'est-ce qu'une comédie ? \\[0pt]
Qu'est-ce qu'une communauté ? \\[0pt]
Qu'est-ce qu'une communauté humaine ? \\[0pt]
Qu'est-ce qu'une communauté politique ? \\[0pt]
Qu'est-ce qu'une communauté scientifique ? \\[0pt]
Qu'est-ce qu'une conception scientifique du monde ? \\[0pt]
Qu'est-ce qu'une condition de possibilité ? \\[0pt]
Qu'est-ce qu'une condition suffisante ? \\[0pt]
Qu'est-ce qu'une conduite irrationnelle ? \\[0pt]
Qu'est-ce qu'une connaissance a priori ? \\[0pt]
Qu'est-ce qu'une connaissance fiable ? \\[0pt]
Qu'est-ce qu'une connaissance métaphysique \\[0pt]
Qu'est-ce qu'une connaissance non scientifique ? \\[0pt]
Qu'est-ce qu'une connaissance par les faits ? \\[0pt]
Qu'est-ce qu'une conscience collective ? \\[0pt]
Qu'est-ce qu'une constitution ? \\[0pt]
Qu'est-ce qu'une contradiction ? \\[0pt]
Qu'est-ce qu'une contrainte ? \\[0pt]
Qu'est-ce qu'une convention ? \\[0pt]
Qu'est-ce qu'une conviction ? \\[0pt]
Qu'est-ce qu'une crise ? \\[0pt]
Qu'est-ce qu'une crise politique ? \\[0pt]
Qu'est-ce qu'une croyance ? \\[0pt]
Qu'est-ce qu'une croyance rationnelle ? \\[0pt]
Qu'est-ce qu'une croyance vraie ? \\[0pt]
Qu'est-ce qu'une culture ? \\[0pt]
Qu'est-ce qu'une décision politique ? \\[0pt]
Qu'est-ce qu'une décision rationnelle ? \\[0pt]
Qu'est-ce qu'une découverte ? \\[0pt]
Qu'est-ce qu'une découverte scientifique ? \\[0pt]
Qu'est-ce qu'une définition ? \\[0pt]
Qu'est-ce qu'une démocratie ? \\[0pt]
Qu'est-ce qu'une démonstration ? \\[0pt]
Qu'est-ce qu'une discipline savante ? \\[0pt]
Qu'est-ce qu'une disposition ? \\[0pt]
Qu'est-ce qu'une école philosophique ? \\[0pt]
Qu'est-ce qu'une éducation réussie ? \\[0pt]
Qu'est-ce qu'une éducation scientifique ? \\[0pt]
Qu'est-ce qu'une encyclopédie ? \\[0pt]
Qu'est-ce qu'une époque ? \\[0pt]
Qu'est-ce qu'une erreur ? \\[0pt]
Qu'est-ce qu'une espèce naturelle ? \\[0pt]
Qu'est-ce qu'une exception ? \\[0pt]
Qu'est-ce qu'une existence historique ? \\[0pt]
Qu'est-ce qu'une expérience ? \\[0pt]
Qu'est-ce qu'une expérience cruciale ? \\[0pt]
Qu'est-ce qu'une « expérience de pensée » ? \\[0pt]
Qu'est-ce qu'une expérience de pensée ? \\[0pt]
Qu'est-ce qu'une expérience esthétique ? \\[0pt]
Qu'est-ce qu'une expérience politique ? \\[0pt]
Qu'est-ce qu'une expérience religieuse ? \\[0pt]
Qu'est-ce qu'une expérience scientifique ? \\[0pt]
Qu'est-ce qu'une explication matérialiste ? \\[0pt]
Qu'est-ce qu'une exposition ? \\[0pt]
Qu'est-ce qu'une famille ? \\[0pt]
Qu'est-ce qu'une fausse science ? \\[0pt]
Qu'est-ce qu'une faute de goût ? \\[0pt]
Qu'est-ce qu'une fiction ? \\[0pt]
Qu'est-ce qu'une fonction ? \\[0pt]
Qu'est-ce qu'une forme ? \\[0pt]
Qu'est-ce qu'une grande cause ? \\[0pt]
Qu'est-ce qu'une guerre juste ? \\[0pt]
Qu'est-ce qu'une histoire vraie ? \\[0pt]
Qu'est-ce qu'une hypothèse ? \\[0pt]
Qu'est-ce qu'une hypothèse scientifique ? \\[0pt]
Qu'est-ce qu'une idée ? \\[0pt]
Qu'est-ce qu'une idée esthétique ? \\[0pt]
Qu'est-ce qu'une idée fausse ? \\[0pt]
Qu'est-ce qu'une idée incertaine ? \\[0pt]
Qu'est-ce qu'une idée morale ? \\[0pt]
Qu'est-ce qu'une idée vraie ? \\[0pt]
Qu'est-ce qu'une idéologie ? \\[0pt]
Qu'est-ce qu'une idéologie scientifique ? \\[0pt]
Qu'est-ce qu'une illusion ? \\[0pt]
Qu'est-ce qu'une image ? \\[0pt]
Qu'est-ce qu'une impression ? \\[0pt]
Qu'est-ce qu'une inégalité ? \\[0pt]
Qu'est-ce qu'une inférence ? \\[0pt]
Qu'est-ce qu'une injustice ? \\[0pt]
Qu'est-ce qu'une innovation technologique ? \\[0pt]
Qu'est-ce qu'une institution ? \\[0pt]
Qu'est-ce qu'une interprétation ? \\[0pt]
Qu'est-ce qu'une invention technique ? \\[0pt]
Qu'est-ce qu'une langue ? \\[0pt]
Qu'est-ce qu'une langue artificielle ? \\[0pt]
Qu'est-ce qu'une langue bien faite ? \\[0pt]
Qu'est-ce qu'une langue morte ? \\[0pt]
Qu'est-ce qu'un élément ? \\[0pt]
Qu'est-ce qu'une libération ? \\[0pt]
Qu'est-ce qu'une liberté fondamentale ? \\[0pt]
Qu'est-ce qu'une libre interprétation ? \\[0pt]
Qu'est-ce qu'une limite ? \\[0pt]
Qu'est-ce qu'une logique sociale ? \\[0pt]
Qu'est-ce qu'une loi ? \\[0pt]
Qu'est-ce qu'une loi de la nature ? \\[0pt]
Qu'est-ce qu'une loi de la pensée ? \\[0pt]
Qu'est-ce qu'une loi d'exception ? \\[0pt]
Qu'est-ce qu'une loi injuste ? \\[0pt]
Qu'est-ce qu'une loi scientifique ? \\[0pt]
Qu'est-ce qu'une machine ? \\[0pt]
Qu'est-ce qu'une machine ne peut pas faire ? \\[0pt]
Qu'est-ce qu'une maladie ? \\[0pt]
Qu'est-ce qu'une marchandise ? \\[0pt]
Qu'est-ce qu'une mauvaise idée ? \\[0pt]
Qu'est-ce qu'une mauvaise interprétation ? \\[0pt]
Qu'est-ce qu'une méditation ? \\[0pt]
Qu'est-ce qu'une méditation métaphysique ? \\[0pt]
Qu'est-ce qu'une mentalité collective ? \\[0pt]
Qu'est-ce qu'une métaphore ? \\[0pt]
Qu'est-ce qu'une méthode ? \\[0pt]
Qu'est-ce qu'une minorité ? \\[0pt]
Qu'est-ce qu'une morale de la communication ? \\[0pt]
Qu'est-ce qu'un empire ? \\[0pt]
Qu'est-ce qu'une nation ? \\[0pt]
Qu'est-ce qu'un enfant ? \\[0pt]
Qu'est-ce qu'un ennemi ? \\[0pt]
Qu'est-ce qu'un ennemi politique ? \\[0pt]
Qu'est-ce qu'un énoncé scientifique sur l'homme ? \\[0pt]
Qu'est-ce qu'une norme ? \\[0pt]
Qu'est-ce qu'une norme sociale ? \\[0pt]
Qu'est-ce qu'une nouveauté ? \\[0pt]
Qu'est-ce qu'une occasion ? \\[0pt]
Qu'est-ce qu'une œuvre ? \\[0pt]
Qu'est-ce qu'une œuvre classique ? \\[0pt]
Qu'est-ce qu'une œuvre d'art ? \\[0pt]
Qu'est-ce qu'une œuvre d'art authentique ? \\[0pt]
Qu'est-ce qu'une œuvre d'art réaliste ? \\[0pt]
Qu'est-ce qu'une œuvre d'art réussie ? \\[0pt]
Qu'est-ce qu'une œuvre « géniale » ? \\[0pt]
Qu'est-ce qu'une œuvre ratée ? \\[0pt]
Qu'est-ce qu'une parole en l'air ? \\[0pt]
Qu'est-ce qu'une parole juste ? \\[0pt]
Qu'est-ce qu'une parole libre ? \\[0pt]
Qu'est-ce qu'une parole vivante ? \\[0pt]
Qu'est-ce qu'une parole vraie ? \\[0pt]
Qu'est-ce qu'une passion ? \\[0pt]
Qu'est-ce qu'une patrie ? \\[0pt]
Qu'est-ce qu'une peinture abstraite ? \\[0pt]
Qu'est-ce qu'une pensée libre ? \\[0pt]
Qu'est-ce qu'une perception sensible ? \\[0pt]
Qu'est-ce qu'une « performance » ? \\[0pt]
Qu'est-ce qu'une période en histoire ? \\[0pt]
Qu'est-ce qu'une personne ? \\[0pt]
Qu'est-ce qu'une personne morale ? \\[0pt]
Qu'est-ce qu'une pétition de principe ? \\[0pt]
Qu'est-ce qu'une philosophie ? \\[0pt]
Qu'est-ce qu'une philosophie de la vie ? \\[0pt]
Qu'est-ce qu'une philosophie première ? \\[0pt]
Qu'est-ce qu'une phrase ? \\[0pt]
Qu'est-ce qu'une politique sociale ? \\[0pt]
Qu'est-ce qu'une preuve ? \\[0pt]
Qu'est-ce qu'une preuve scientifique ? \\[0pt]
Qu'est-ce qu'une promesse ? \\[0pt]
Qu'est-ce qu'une propriété ? \\[0pt]
Qu'est-ce qu'une propriété essentielle ? \\[0pt]
Qu'est-ce qu'une pseudoscience ? \\[0pt]
Qu'est-ce qu'une psychologie scientifique ? \\[0pt]
Qu'est-ce qu'une question ? \\[0pt]
Qu'est-ce qu'une question dénuée de sens ? \\[0pt]
Qu'est-ce qu'une question métaphysique ? \\[0pt]
Qu'est-ce qu'une raison d'agir ? \\[0pt]
Qu'est-ce qu'une réfutation ? \\[0pt]
Qu'est-ce qu'une règle ? \\[0pt]
Qu'est-ce qu'une règle de grammaire ? \\[0pt]
Qu'est-ce qu'une règle de vie ? \\[0pt]
Qu'est-ce qu'une règle sociale ? \\[0pt]
Qu'est-ce qu'une relation ? \\[0pt]
Qu'est-ce qu'une religion ? \\[0pt]
Qu'est-ce qu'une rencontre ? \\[0pt]
Qu'est-ce qu'une représentation ? \\[0pt]
Qu'est-ce qu'une représentation du monde ? \\[0pt]
Qu'est-ce qu'une représentation réussie ? \\[0pt]
Qu'est-ce qu'une république ? \\[0pt]
Qu'est-ce qu'une révélation ? \\[0pt]
Qu'est-ce qu'une révolution ? \\[0pt]
Qu'est-ce qu'une révolution artistique ? \\[0pt]
Qu'est-ce qu'une révolution politique ? \\[0pt]
Qu'est-ce qu'une révolution scientifique ? \\[0pt]
Qu'est-ce qu'une science exacte ? \\[0pt]
Qu'est-ce qu'une science expérimentale ? \\[0pt]
Qu'est-ce qu'une science humaine ? \\[0pt]
Qu'est-ce qu'une science rigoureuse ? \\[0pt]
Qu'est-ce qu'un esclave ? \\[0pt]
Qu'est-ce qu'une secte ? \\[0pt]
Qu'est-ce qu'une situation ? \\[0pt]
Qu'est-ce qu'une situation tragique ? \\[0pt]
Qu'est-ce qu'une société ? \\[0pt]
Qu'est-ce qu'une société juste ? \\[0pt]
Qu'est-ce qu'une société libre ? \\[0pt]
Qu'est-ce qu'une société mondialisée ? \\[0pt]
Qu'est-ce qu'une société ouverte ? \\[0pt]
Qu'est-ce qu'une société primitive ? \\[0pt]
Qu'est-ce qu'une solution ? \\[0pt]
Qu'est-ce qu'un esprit faux ? \\[0pt]
Qu'est-ce qu'un esprit juste ? \\[0pt]
Qu'est-ce qu'un esprit libre ? \\[0pt]
Qu'est-ce qu'un esprit profond ? \\[0pt]
Qu'est-ce qu'une structure ? \\[0pt]
Qu'est-ce qu'une substance ? \\[0pt]
Qu'est-ce qu'un État de droit ? \\[0pt]
Qu'est-ce qu'un État libre ? \\[0pt]
Qu'est-ce qu'un état mental ? \\[0pt]
Qu'est-ce qu'un État souverain ? \\[0pt]
Qu'est-ce qu'une théorie ? \\[0pt]
Qu'est-ce qu'une théorie scientifique ? \\[0pt]
Qu'est-ce qu'une tradition ? \\[0pt]
Qu'est-ce qu'une tragédie ? \\[0pt]
Qu'est-ce qu'une tragédie historique ? \\[0pt]
Qu'est-ce qu'un être cultivé ? \\[0pt]
Qu'est-ce qu'un « être dégénéré » ? \\[0pt]
Qu'est-ce qu'un être imaginaire ? \\[0pt]
Qu'est-ce qu'un être vivant ? \\[0pt]
Qu'est-ce qu'une valeur ? \\[0pt]
Qu'est-ce qu'une valeur esthétique ? \\[0pt]
Qu'est-ce qu'un événement ? \\[0pt]
Qu'est-ce qu'un événement fondateur ? \\[0pt]
Qu'est-ce qu'un événement historique ? \\[0pt]
Qu'est-ce qu'une vérité contingente ? \\[0pt]
Qu'est-ce qu'une vérité historique ? \\[0pt]
Qu'est-ce qu'une vérité scientifique ? \\[0pt]
Qu'est-ce qu'une vérité subjective ? \\[0pt]
Qu'est-ce qu'une vertu ? \\[0pt]
Qu'est-ce qu'une vie d'artiste ? \\[0pt]
Qu'est-ce qu'une vie heureuse ? \\[0pt]
Qu'est-ce qu'une vie humaine ? \\[0pt]
Qu'est-ce qu'une vie réussie ? \\[0pt]
Qu'est-ce qu'une ville ? \\[0pt]
Qu'est-ce qu'une violence symbolique ? \\[0pt]
Qu'est-ce qu'une vision du monde ? \\[0pt]
Qu'est-ce qu'une vision scientifique du monde ? \\[0pt]
Qu'est-ce qu'une volonté libre ? \\[0pt]
Qu'est-ce qu'une volonté raisonnable ? \\[0pt]
Qu'est-ce qu'un exemple ? \\[0pt]
Qu'est-ce qu'un expérimentateur ? \\[0pt]
Qu'est-ce qu'un expert ? \\[0pt]
Qu'est-ce qu'un fait ? \\[0pt]
Qu'est-ce qu'un fait culturel ? \\[0pt]
Qu'est-ce qu'un fait de culture ? \\[0pt]
Qu'est-ce qu'un fait de société ? \\[0pt]
Qu'est-ce qu'un fait divers ? \\[0pt]
Qu'est-ce qu'un fait historique ? \\[0pt]
Qu'est-ce qu'un fait moral ? \\[0pt]
Qu'est-ce qu'un fait religieux ? \\[0pt]
Qu'est-ce qu'un fait scientifique ? \\[0pt]
Qu'est-ce qu'un fait social ? \\[0pt]
Qu'est-ce qu'un faux ? \\[0pt]
Qu'est-ce qu'un faux problème ? \\[0pt]
Qu'est-ce qu'un faux sentiment ? \\[0pt]
Qu'est-ce qu'un film ? \\[0pt]
Qu'est-ce qu'un génie ? \\[0pt]
Qu'est-ce qu'un geste artistique ? \\[0pt]
Qu'est-ce qu'un geste technique ? \\[0pt]
Qu'est-ce qu'un gouvernement ? \\[0pt]
Qu'est-ce qu'un gouvernement démocratique ? \\[0pt]
Qu'est-ce qu'un gouvernement juste ? \\[0pt]
Qu'est-ce qu'un gouvernement républicain ? \\[0pt]
Qu'est-ce qu'un grand homme ? \\[0pt]
Qu'est-ce qu'un grand homme ou une grande femme ? \\[0pt]
Qu'est-ce qu'un grand philosophe ? \\[0pt]
Qu'est-ce qu'un héros ? \\[0pt]
Qu'est-ce qu'un homme bon ? \\[0pt]
Qu'est-ce qu'un homme d'action ? \\[0pt]
Qu'est-ce qu'un homme d'État ? \\[0pt]
Qu'est-ce qu'un homme d'expérience ? \\[0pt]
Qu'est-ce qu'un homme juste ? \\[0pt]
Qu'est-ce qu'un homme libre ? \\[0pt]
Qu'est-ce qu'un homme méchant ? \\[0pt]
Qu'est-ce qu'un homme normal ? \\[0pt]
Qu'est-ce qu'un homme politique ? \\[0pt]
Qu'est-ce qu'un homme sans éducation ? \\[0pt]
Qu'est-ce qu'un homme seul ? \\[0pt]
Qu'est-ce qu'un idéal ? \\[0pt]
Qu'est-ce qu'un idéaliste ? \\[0pt]
Qu'est-ce qu'un idéal moral ? \\[0pt]
Qu'est-ce qu'un imaginaire social ? \\[0pt]
Qu'est-ce qu'un impératif technique ? \\[0pt]
Qu'est-ce qu'un individu ? \\[0pt]
Qu'est-ce qu'un intellectuel ? \\[0pt]
Qu'est-ce qu'un jeu ? \\[0pt]
Qu'est-ce qu'un juge ? \\[0pt]
Qu'est-ce qu'un jugement analytique ? \\[0pt]
Qu'est-ce qu'un jugement de goût ? \\[0pt]
Qu'est-ce qu'un juste ? \\[0pt]
Qu'est-ce qu'un juste salaire ? \\[0pt]
Qu'est-ce qu'un justicier ? \\[0pt]
Qu'est-ce qu'un laboratoire ? \\[0pt]
Qu'est-ce qu'un langage technique ? \\[0pt]
Qu'est-ce qu'un législateur ? \\[0pt]
Qu'est-ce qu'un lien logique ? \\[0pt]
Qu'est-ce qu'un lieu ? \\[0pt]
Qu'est-ce qu'un lieu commun ? \\[0pt]
Qu'est-ce qu'un livre ? \\[0pt]
Qu'est-ce qu'un maître ? \\[0pt]
Qu'est-ce qu'un marginal ? \\[0pt]
Qu'est-ce qu'un mécanisme social ? \\[0pt]
Qu'est-ce qu'un métaphysicien ? \\[0pt]
Qu'est-ce qu'un métier ? \\[0pt]
Qu'est-ce qu'un mineur ? \\[0pt]
Qu'est-ce qu'un miracle ? \\[0pt]
Qu'est-ce qu'un modèle ? \\[0pt]
Qu'est-ce qu'un moderne ? \\[0pt]
Qu'est-ce qu'un moment ? \\[0pt]
Qu'est-ce qu'un monde \\[0pt]
Qu'est-ce qu'un monde ? \\[0pt]
Qu'est-ce qu'un monde commun ? \\[0pt]
Qu'est-ce qu'un monstre ? \\[0pt]
Qu'est-ce qu'un monument ? \\[0pt]
Qu'est-ce qu'un mouvement politique \\[0pt]
Qu'est-ce qu'un mouvement politique ? \\[0pt]
Qu'est-ce qu'un musée ? \\[0pt]
Qu'est-ce qu'un mythe ? \\[0pt]
Qu'est-ce qu'un nombre ? \\[0pt]
Qu'est-ce qu'un nom propre ? \\[0pt]
Qu'est-ce qu'un objet ? \\[0pt]
Qu'est-ce qu'un objet d'art ? \\[0pt]
Qu'est-ce qu'un objet de connaissance ? \\[0pt]
Qu'est-ce qu'un objet de science ? \\[0pt]
Qu'est-ce qu'un objet esthétique ? \\[0pt]
Qu'est-ce qu'un objet mathématique ? \\[0pt]
Qu'est-ce qu'un objet métaphysique ? \\[0pt]
Qu'est-ce qu'un objet technique ? \\[0pt]
Qu'est-ce qu'un œuvre d'art ? \\[0pt]
Qu'est-ce qu'un ordre ? \\[0pt]
Qu'est-ce qu'un organisme ? \\[0pt]
Qu'est-ce qu'un original ? \\[0pt]
Qu'est-ce qu'un outil ? \\[0pt]
Qu'est-ce qu'un paradoxe ? \\[0pt]
Qu'est-ce qu'un passe-temps ? \\[0pt]
Qu'est-ce qu'un patrimoine ? \\[0pt]
Qu'est-ce qu'un pauvre ? \\[0pt]
Qu'est-ce qu'un paysage ? \\[0pt]
Qu'est-ce qu'un pédant ? \\[0pt]
Qu'est-ce qu'un peuple \\[0pt]
Qu'est-ce qu'un peuple ? \\[0pt]
Qu'est-ce qu'un peuple libre ? \\[0pt]
Qu'est-ce qu'un phénomène ? \\[0pt]
Qu'est-ce qu'un philosophe ? \\[0pt]
Qu'est-ce qu'un plaisir pur ? \\[0pt]
Qu'est-ce qu'un point de vue ? \\[0pt]
Qu'est-ce qu'un portrait ? \\[0pt]
Qu'est-ce qu'un post-moderne ? \\[0pt]
Qu'est-ce qu'un postulat ? \\[0pt]
Qu'est-ce qu'un pouvoir despotique ? \\[0pt]
Qu'est-ce qu'un pouvoir légitime ? \\[0pt]
Qu'est-ce qu'un précurseur ? \\[0pt]
Qu'est-ce qu'un préjugé ? \\[0pt]
Qu'est-ce qu'un primitif ? \\[0pt]
Qu'est-ce qu'un prince juste ? \\[0pt]
Qu'est-ce qu'un principe ? \\[0pt]
Qu'est-ce qu'un problème ? \\[0pt]
Qu'est-ce qu'un problème éthique ? \\[0pt]
Qu'est-ce qu'un problème insoluble ? \\[0pt]
Qu'est-ce qu'un problème métaphysique ? \\[0pt]
Qu'est-ce qu'un problème philosophique ? \\[0pt]
Qu'est-ce qu'un problème politique ? \\[0pt]
Qu'est-ce qu'un problème scientifique ? \\[0pt]
Qu'est-ce qu'un problème technique ? \\[0pt]
Qu'est-ce qu'un produit culturel ? \\[0pt]
Qu'est-ce qu'un programme ? \\[0pt]
Qu'est-ce qu'un programme politique ? \\[0pt]
Qu'est-ce qu'un programmer ? \\[0pt]
Qu'est-ce qu'un progrès scientifique ? \\[0pt]
Qu'est-ce qu'un progrès technique ? \\[0pt]
Qu'est-ce qu'un projet ? \\[0pt]
Qu'est-ce qu'un prophète ? \\[0pt]
Qu'est-ce qu'un public ? \\[0pt]
Qu'est-ce qu'un pur esprit ? \\[0pt]
Qu'est-ce qu'un rapport de force ? \\[0pt]
Qu'est-ce qu'un récit ? \\[0pt]
Qu'est-ce qu'un récit véridique ? \\[0pt]
Qu'est-ce qu'un réfugié politique ? \\[0pt]
Qu'est-ce qu'un réfutation ? \\[0pt]
Qu'est-ce qu'un régime politique ? \\[0pt]
Qu'est-ce qu'un réseau ? \\[0pt]
Qu'est-ce qu'un rhéteur ? \\[0pt]
Qu'est-ce qu'un rite ? \\[0pt]
Qu'est-ce qu'un rival ? \\[0pt]
Qu'est-ce qu'un sage ? \\[0pt]
Qu'est-ce qu'un savoir-faire ? \\[0pt]
Qu'est-ce qu'un sceptique ? \\[0pt]
Qu'est-ce qu'un sentiment moral ? \\[0pt]
Qu'est-ce qu'un sentiment vrai ? \\[0pt]
Qu'est-ce qu'un signe ? \\[0pt]
Qu'est-ce qu'un sophisme ? \\[0pt]
Qu'est-ce qu'un sophiste ? \\[0pt]
Qu'est-ce qu'un souvenir ? \\[0pt]
Qu'est-ce qu'un spécialiste ? \\[0pt]
Qu'est-ce qu'un spectacle ? \\[0pt]
Qu'est-ce qu'un spectateur ? \\[0pt]
Qu'est-ce qu'un style ? \\[0pt]
Qu'est-ce qu'un sujet de droit ? \\[0pt]
Qu'est-ce qu'un symbole ? \\[0pt]
Qu'est-ce qu'un symptôme ? \\[0pt]
Qu'est-ce qu'un système ? \\[0pt]
Qu'est-ce qu'un système philosophique ? \\[0pt]
Qu'est-ce qu'un tableau \\[0pt]
Qu'est-ce qu'un tableau ? \\[0pt]
Qu'est-ce qu'un tabou ? \\[0pt]
Qu'est-ce qu'un technicien ? \\[0pt]
Qu'est-ce qu'un témoignage ? \\[0pt]
Qu'est-ce qu'un témoin ? \\[0pt]
Qu'est-ce qu'un temple ? \\[0pt]
Qu'est-ce qu'un territoire ? \\[0pt]
Qu'est-ce qu'un terroriste ? \\[0pt]
Qu'est-ce qu'un texte ? \\[0pt]
Qu'est-ce qu'un tout ? \\[0pt]
Qu'est-ce qu'un traître ? \\[0pt]
Qu'est-ce qu'un travail bien fait ? \\[0pt]
Qu'est-ce qu'un trouble social ? \\[0pt]
Qu'est-ce qu'un tyran ? \\[0pt]
Qu'est-ce qu'un vice ? \\[0pt]
Qu'est-ce qu'un visage ? \\[0pt]
Qu'est-ce qu'un vrai changement ? \\[0pt]
Qu'est-il légitime d'interdire ? \\[0pt]
Qu'est-il utile d'interdire ? \\[0pt]
Question et problème \\[0pt]
Qu'est que l'autorité ? \\[0pt]
Qu'est qu'une image ? \\[0pt]
Qu'est qu'un régime politique ? \\[0pt]
Que suis-je ? \\[0pt]
Que suppose le mouvement ? \\[0pt]
Que transmettons-nous ? \\[0pt]
Que trouve-t-on dans ce que l'on trouve beau ? \\[0pt]
Que valent les classifications en sciences humaines ? \\[0pt]
Que valent les décisions de la majorité ? \\[0pt]
Que valent les excuses ? \\[0pt]
Que valent les idées générales ? \\[0pt]
Que valent les mots ? \\[0pt]
Que valent les préjugés ? \\[0pt]
Que valent les preuves ? \\[0pt]
Que valent les théories ? \\[0pt]
« Que va-t-il se passer ? » \\[0pt]
Que vaut en morale la justification par l'utilité ? \\[0pt]
Que vaut la décision de la majorité ? \\[0pt]
Que vaut la définition de l'homme comme animal doué de raison ? \\[0pt]
Que vaut la distinction entre nature et culture ? \\[0pt]
Que vaut la fidélité ? \\[0pt]
Que vaut le conseil : « vivez avec votre temps » ? \\[0pt]
Que vaut le travail ? \\[0pt]
Que vaut l'excuse : c'est plus fort que moi ? \\[0pt]
Que vaut l'excuse : « C'est plus fort que moi » ? \\[0pt]
Que vaut l'excuse : « Je ne l'ai pas fait exprès » ? \\[0pt]
Que vaut l'incertain ? \\[0pt]
Que vaut un argument d'autorité ? \\[0pt]
Que vaut un consensus ? \\[0pt]
Que vaut une connaissance empirique de l'homme ? \\[0pt]
Que vaut une parole ? \\[0pt]
Que vaut une preuve contre un préjugé ? \\[0pt]
Que veut dire avoir raison ? \\[0pt]
Que veut dire « essentiel » ? \\[0pt]
Que veut dire : être athée ? \\[0pt]
Que veut dire : « être cultivé » ? \\[0pt]
Que veut dire introduire à la métaphysique ? \\[0pt]
Que veut dire « je t'aime » ? \\[0pt]
Que veut dire : « je t'aime » ? \\[0pt]
Que veut dire : « le temps passe » ? \\[0pt]
Que veut dire l'expression « aller au fond des choses » ? \\[0pt]
Que veut dire « réel » ? \\[0pt]
Que veut dire « respecter la nature » ? \\[0pt]
Que veut dire : « respecter la nature » ? \\[0pt]
Que veut-on dire quand on dit que « rien n'est sans raison » ? \\[0pt]
Que veut-on dire quand on dit « rien n'est sans raison » ? \\[0pt]
Que voit-on dans une image ? \\[0pt]
Que voit-on dans un miroir \\[0pt]
Que voit-on dans un miroir ? \\[0pt]
Que voit-on dans un tableau ? \\[0pt]
Que voulons-nous ? \\[0pt]
Que voulons-nous vraiment savoir ? \\[0pt]
Que voyons-nous ? \\[0pt]
Que voyons-nous sur un tableau ? \\[0pt]
Qu'expriment les mythes ? \\[0pt]
Qu'expriment les œuvres d'art ? \\[0pt]
Qu'exprime une œuvre d'art ? \\[0pt]
Qui accroît son savoir accroît sa douleur \\[0pt]
Qui a des droits ? \\[0pt]
Qui agit ? \\[0pt]
Qui a le droit de juger ? \\[0pt]
Qui a le droit de punir ? \\[0pt]
Qui a une histoire ? \\[0pt]
Qui a une parole politique ? \\[0pt]
Qui commande ? \\[0pt]
Qui connaît le mieux mon corps ? \\[0pt]
Qui croire ? \\[0pt]
Qui détient le pouvoir ? \\[0pt]
Qui doit dire le droit ? \\[0pt]
Qui doit éduquer ? \\[0pt]
Qui doit faire les lois ? \\[0pt]
Qui doit gouverner ? \\[0pt]
Qui domine ? \\[0pt]
Qui donne la norme du goût ? \\[0pt]
Qui écrit l'histoire ? \\[0pt]
Qui est autorisé à me dire « tu dois » ? \\[0pt]
Qui est citoyen ? \\[0pt]
Qui est comédien ? \\[0pt]
Qui est compétent en matière politique ? \\[0pt]
Qui est crédible ? \\[0pt]
Qui est cultivé ? \\[0pt]
Qui est digne du bonheur ? \\[0pt]
Qui est immoral ? \\[0pt]
Qui est l'autre ? \\[0pt]
Qui est le maître ? \\[0pt]
Qui est le peuple ? \\[0pt]
Qui est l'homme des sciences humaines ? \\[0pt]
Qui est libre ? \\[0pt]
Qui est méchant ? \\[0pt]
Qui est métaphysicien ? \\[0pt]
Qui est mon prochain ? \\[0pt]
Qui est mon semblable ? \\[0pt]
Qui est responsable ? \\[0pt]
Qui est riche ? \\[0pt]
Qui est sage ? \\[0pt]
Qui est souverain ? \\[0pt]
Qui est une personne ? \\[0pt]
Qui fait la loi ? \\[0pt]
Qui fait l'histoire ? \\[0pt]
Qui fait l'unité du peuple ? \\[0pt]
Qui faut-il protéger ? \\[0pt]
Qui gouverne ? \\[0pt]
Qui me dit ce que je dois faire ? \\[0pt]
Qui mérite d'être aimé ? \\[0pt]
Qui meurt ? \\[0pt]
« Qui ne dit mot consent » \\[0pt]
Qui nous dicte nos devoirs ? \\[0pt]
Qui parle ? \\[0pt]
Qui parle quand je dis « je » ? \\[0pt]
Qui pense ? \\[0pt]
Qui peut avoir des droits ? \\[0pt]
Qui peut me dire « tu ne dois pas » ? \\[0pt]
Qui peut obliger ? \\[0pt]
Qui peut parler ? \\[0pt]
Qui peut prétendre énoncer des devoirs ? \\[0pt]
Qui peut prétendre imposer des bornes à la technique ? \\[0pt]
Qui peut se passer de religion ? \\[0pt]
Qui pose les fins ? \\[0pt]
Qui sont mes amis ? \\[0pt]
Qui sont mes semblables ? \\[0pt]
Qui sont nos ennemis ? \\[0pt]
« Qui suis-je ? » \\[0pt]
Qui suis-je ? \\[0pt]
Qui suis-je et qui es-tu ? \\[0pt]
Qui suis-je pour me juger ? \\[0pt]
Qui travaille ? \\[0pt]
Qui veut la fin veut les moyens \\[0pt]
Qu'oppose-t-on à la vérité ? \\[0pt]
Qu'y a-t-il ? \\[0pt]
Qu'y a-t-il à comprendre dans une œuvre d'art ? \\[0pt]
Qu'y a-t-il à comprendre en histoire ? \\[0pt]
Qu'y a-t-il à craindre de la technique ? \\[0pt]
Qu'y a-t-il à l'origine de toutes choses ? \\[0pt]
Qu'y a-t-il au-delà de l'être ? \\[0pt]
Qu'y a-t-il au-delà du réel ? \\[0pt]
Qu'y a-t-il au-dessus des lois ? \\[0pt]
Qu'y a-t-il au fondement de l'objectivité ? \\[0pt]
Qu'y a-t-il de sacré ? \\[0pt]
Qu'y a-t-il de sérieux dans le jeu ? \\[0pt]
Qu'y a-t-il de technique dans l'art ? \\[0pt]
Qu'y a-t-il d'universel dans la culture ? \\[0pt]
Qu'y a-t-il que la nature fait en vain ? \\[0pt]
Qu'y a-t-il sinon le réel ? \\[0pt]
Race, classe, nation \\[0pt]
Raconter sa vie \\[0pt]
Raconter son histoire \\[0pt]
Raison et calcul \\[0pt]
Raison et démonstration \\[0pt]
Raison et déraison \\[0pt]
Raison et dialogue \\[0pt]
Raison et folie \\[0pt]
Raison et fondement \\[0pt]
Raison et langage \\[0pt]
Raison et politique \\[0pt]
Raison et religion sont-elles incompatibles ? \\[0pt]
Raison et révélation \\[0pt]
Raison et technique \\[0pt]
Raison et tradition \\[0pt]
Raisonnable et rationnel \\[0pt]
Raisonnement et expérimentation \\[0pt]
Raisonner \\[0pt]
Raisonner et argumenter \\[0pt]
Raisonner et calculer \\[0pt]
Raisonner et décider \\[0pt]
Raisonner par l'absurde \\[0pt]
Rapports de force, rapport de pouvoir \\[0pt]
Rapports de la science et de la politique \\[0pt]
Rassembler les hommes, est-ce les unir ? \\[0pt]
Rationalité et irrationalité du travail \\[0pt]
Rationnel et raisonnable \\[0pt]
Réalisme et idéalisme \\[0pt]
Réalité et apparence \\[0pt]
Réalité et idéal \\[0pt]
Réalité et perception \\[0pt]
Réalité et représentation \\[0pt]
Rebuts et objets quelconques : une matière pour l'art ? \\[0pt]
Recevoir \\[0pt]
Recherche et militantisme \\[0pt]
Rechercher la vérité, est-ce renoncer à toute opinion ? \\[0pt]
Récit et explication en histoire \\[0pt]
Récit et fiction \\[0pt]
Récit et histoire \\[0pt]
Récit et mémoire \\[0pt]
Reconnaissance et inégalité \\[0pt]
Reconnaissons-nous le bien comme nous reconnaissons le vrai ? \\[0pt]
Reconnaître la vérité \\[0pt]
Recourir au langage, est-ce renoncer à la violence ? \\[0pt]
Réécrire l'histoire \\[0pt]
Refaire le monde \\[0pt]
Refaire sa vie \\[0pt]
Réforme et révolution \\[0pt]
Refuser et réfuter \\[0pt]
Réfutation et confirmation \\[0pt]
Réfuter \\[0pt]
Réfuter une théorie \\[0pt]
Regarder \\[0pt]
Regarder et voir \\[0pt]
Regarder une œuvre d'art \\[0pt]
Regarder un tableau \\[0pt]
Règle et commandement \\[0pt]
Règle morale et norme juridique \\[0pt]
Règles logiques et lois psychologiques \\[0pt]
Règles sociales et loi morale \\[0pt]
Regrets et remords \\[0pt]
Régularité et singularité \\[0pt]
Relativisme et sciences humaines \\[0pt]
Religion et démocratie \\[0pt]
Religion et liberté \\[0pt]
Religion et métaphysique \\[0pt]
Religion et moralité \\[0pt]
Religion et politique \\[0pt]
Religion et raison \\[0pt]
Religion et superstition \\[0pt]
Religion et vérité \\[0pt]
Religion et violence \\[0pt]
Religion naturelle et religion révélée \\[0pt]
Religions et démocratie \\[0pt]
Rendre justice \\[0pt]
Rendre la justice \\[0pt]
Rendre raison \\[0pt]
Rendre visible l'invisible \\[0pt]
Renoncer à ses préjugés \\[0pt]
Renoncer au passé \\[0pt]
Rentrer en soi-même \\[0pt]
Réparer \\[0pt]
Répondre \\[0pt]
Répondre de soi \\[0pt]
Représentation et expression \\[0pt]
Représentation et illusion \\[0pt]
Représenter \\[0pt]
Reproduire \\[0pt]
Reproduire, copier, imiter \\[0pt]
Réprouver \\[0pt]
République et démocratie \\[0pt]
Résistance et obéissance \\[0pt]
Résistance et soumission \\[0pt]
Résister \\[0pt]
Résister à l'oppression \\[0pt]
Résister peut-il être un droit ? \\[0pt]
Respect de la volonté, respect de la vie \\[0pt]
Respecter autrui, est-ce l'aimer ? \\[0pt]
Respecter la nature, est-ce renoncer à l'exploiter ? \\[0pt]
Respecter la nature, est-ce renoncer à s'en emparer ? \\[0pt]
Respect et tolérance \\[0pt]
Responsabilité et culpabilité \\[0pt]
Ressemblance et identité \\[0pt]
Ressemblance et imitation \\[0pt]
Ressembler \\[0pt]
Ressentir \\[0pt]
Ressent-on ou apprécie-t-on l'art ? \\[0pt]
« Rester soi-même » \\[0pt]
Rester soi-même \\[0pt]
Retenons-nous le temps par le souvenir ? \\[0pt]
Réussir sa vie \\[0pt]
Rêve et réalité \\[0pt]
Révéler \\[0pt]
Revenir à la nature \\[0pt]
Rêver \\[0pt]
Rêver à un autre monde \\[0pt]
Rêver éloigne-t-il de la réalité ? \\[0pt]
Revient-il à l'État d'assurer le bonheur des citoyens ? \\[0pt]
Revient-il à l'État d'assurer votre bonheur ? \\[0pt]
Révolte et révolution \\[0pt]
Rêvons-nous ? \\[0pt]
Rhétorique et vérité \\[0pt]
Richesse et pauvreté \\[0pt]
Rien \\[0pt]
« Rien de ce qui est humain ne m'est étranger » \\[0pt]
« Rien de nouveau sous le soleil » \\[0pt]
Rien de nouveau sous le soleil \\[0pt]
« Rien n'est sans raison » \\[0pt]
Rien n'est sans raison \\[0pt]
« Rien n'est simple » \\[0pt]
Rire \\[0pt]
Rire et pleurer \\[0pt]
Rites et cérémonies \\[0pt]
Rituels et cérémonies \\[0pt]
Roman et vérité \\[0pt]
Rythmes sociaux, rythmes naturels \\[0pt]
S'adapter \\[0pt]
Sagesse et renoncement \\[0pt]
Sait-on ce que l'on veut ? \\[0pt]
Sait-on ce qu'on fait ? \\[0pt]
Sait-on ce qu'on veut ? \\[0pt]
Sait-on nécessairement ce que l'on désire ? \\[0pt]
Sait-on toujours ce que l'on fait ? \\[0pt]
Sait-on toujours ce que l'on veut ? \\[0pt]
Sait-on toujours ce qu'on veut ? \\[0pt]
Sait-on vivre au présent ? \\[0pt]
S'aliéner \\[0pt]
S'amuser \\[0pt]
Sans foi ni loi \\[0pt]
Sans justice, pas de liberté ? \\[0pt]
Sans l'art, parlerait-on de beauté ? \\[0pt]
Sans les mots, que seraient les choses ? \\[0pt]
« Sans l'ombre d'un doute » \\[0pt]
Sans référence à la beauté, peut-on rendre compte de l'art ? \\[0pt]
« Sans titre » \\[0pt]
Santé et politique \\[0pt]
S'approprier une œuvre d'art \\[0pt]
S'arracher à la nature \\[0pt]
S'associer \\[0pt]
« Sauver les apparences » \\[0pt]
Sauver les apparences \\[0pt]
« Sauver les phénomènes » \\[0pt]
Sauver les phénomènes \\[0pt]
Savoir ce qu'on dit \\[0pt]
Savoir ce qu'on fait \\[0pt]
Savoir ce qu'on veut \\[0pt]
Savoir démontrer \\[0pt]
Savoir de quoi on parle \\[0pt]
Savoir, est-ce cesser de croire ? \\[0pt]
Savoir, est-ce déjouer le réel ? \\[0pt]
Savoir, est-ce pouvoir ? \\[0pt]
Savoir, est-ce se libérer ? \\[0pt]
Savoir et croire \\[0pt]
Savoir et démontrer \\[0pt]
Savoir et faire \\[0pt]
Savoir et liberté \\[0pt]
Savoir et objectivité dans les sciences \\[0pt]
Savoir et pouvoir \\[0pt]
Savoir et rectification \\[0pt]
Savoir être heureux \\[0pt]
Savoir et savoir faire \\[0pt]
Savoir et savoir-faire \\[0pt]
Savoir et vérifier \\[0pt]
Savoir faire \\[0pt]
Savoir par cœur \\[0pt]
Savoir pour prévoir \\[0pt]
Savoir, pouvoir \\[0pt]
Savoir quoi faire \\[0pt]
Savoir renoncer \\[0pt]
Savoir s'arrêter \\[0pt]
Savoir se décider \\[0pt]
Savoir tout \\[0pt]
Savoir vivre \\[0pt]
Savons-nous ce que nous disons ? \\[0pt]
Savons-nous ce que peut un corps ? \\[0pt]
Savons-nous ce qui nous rend heureux ? \\[0pt]
Savons-nous de ce que nous disons ? \\[0pt]
Savons-nous nous souvenir ? \\[0pt]
Scepticisme et relativisme \\[0pt]
Science du vivant et finalisme \\[0pt]
Science du vivant, science de l'inerte \\[0pt]
Science et abstraction \\[0pt]
Science et certitude \\[0pt]
Science et complexité \\[0pt]
Science et conscience \\[0pt]
Science et croyance \\[0pt]
Science et démocratie \\[0pt]
Science et domination sociale \\[0pt]
Science et expérience \\[0pt]
Science et histoire \\[0pt]
Science et hypothèse \\[0pt]
Science et idéologie \\[0pt]
Science et imagination \\[0pt]
Science et invention \\[0pt]
Science et libération \\[0pt]
Science et magie \\[0pt]
Science et métaphysique \\[0pt]
Science et méthode \\[0pt]
Science et mythe \\[0pt]
Science et objectivité \\[0pt]
Science et opinion \\[0pt]
Science et persuasion \\[0pt]
Science et philosophie \\[0pt]
Science et pouvoir \\[0pt]
Science et réalité \\[0pt]
Science et religion \\[0pt]
Science et sagesse \\[0pt]
Science et société \\[0pt]
Science et subjectivité \\[0pt]
Science et technique \\[0pt]
Science et technologie \\[0pt]
Science pure et science appliquée \\[0pt]
Science sans conscience n'est-elle que ruine de l'âme ? \\[0pt]
Sciences de la nature et sciences de l'esprit \\[0pt]
Sciences de la nature et sciences humaines \\[0pt]
Sciences empiriques et critères du vrai \\[0pt]
Sciences et philosophie \\[0pt]
Sciences et vérité \\[0pt]
Sciences humaines et déterminisme \\[0pt]
Sciences humaines et herméneutique \\[0pt]
Sciences humaines et idéologie \\[0pt]
Sciences humaines et liberté sont-elles compatibles ? \\[0pt]
Sciences humaines et littérature \\[0pt]
Sciences humaines et naturalisme \\[0pt]
Sciences humaines et nature humaine \\[0pt]
Sciences humaines et objectivité \\[0pt]
Sciences humaines et philosophie \\[0pt]
Sciences humaines et vérité \\[0pt]
Sciences humaines ou sciences sociales ? \\[0pt]
Sciences humaines, sciences de l'homme \\[0pt]
Sciences sociales et humanités \\[0pt]
Sciences sociales et questions environnementales \\[0pt]
Sculpter \\[0pt]
Se chercher soi-même, est-ce devenir un autre ? \\[0pt]
Se connaître soi-même \\[0pt]
Se conserver \\[0pt]
Se convertir \\[0pt]
Sécularisation et laïcisation \\[0pt]
Se cultiver \\[0pt]
Se cultiver, est-ce s'affranchir de son appartenance culturelle ? \\[0pt]
Sécurité et liberté \\[0pt]
Se décider \\[0pt]
Se défendre \\[0pt]
Se détacher des sens \\[0pt]
Se distinguer \\[0pt]
Se divertir \\[0pt]
Se donner corps et âme \\[0pt]
Séduire \\[0pt]
Se faire comprendre \\[0pt]
Se faire justice \\[0pt]
Se force-t-on à faire son devoir ? \\[0pt]
S'émanciper \\[0pt]
S'émanciper de ses maîtres \\[0pt]
Se méfier des idées \\[0pt]
Se mentir à soi-même \\[0pt]
Se mentir à soi-même : est-ce possible ? \\[0pt]
Se mettre à la place d'autrui \\[0pt]
S'engager \\[0pt]
S'engager, est-ce perdre sa liberté ? \\[0pt]
S'ennuyer \\[0pt]
Se nourrir \\[0pt]
Sensation et perception \\[0pt]
Sens et existence \\[0pt]
Sens et fait \\[0pt]
Sens et limites de la notion de capital culturel \\[0pt]
Sens et non-sens \\[0pt]
Sens et sensibilité \\[0pt]
Sens et sensible \\[0pt]
Sens et signification \\[0pt]
Sens et structure \\[0pt]
Sens et vérité \\[0pt]
Sensible et intelligible \\[0pt]
Sens propre et sens figuré \\[0pt]
S'entendre \\[0pt]
Sentiment et émotion \\[0pt]
Sentiment et justice sont-ils compatibles ? \\[0pt]
Sentir \\[0pt]
Sentir et juger \\[0pt]
Sentir et penser \\[0pt]
Sentir et percevoir \\[0pt]
Se parler et s'entendre \\[0pt]
Se passer de philosophie \\[0pt]
Se peindre \\[0pt]
Se prendre au sérieux \\[0pt]
Se raconter des histoires \\[0pt]
Se raisonner \\[0pt]
Serait-il immoral d'autoriser le commerce des organes humains ? \\[0pt]
Se réfugier dans la croyance \\[0pt]
Se réfugier dans la croyance ? \\[0pt]
Se retirer dans la pensée ? \\[0pt]
Se retirer du monde \\[0pt]
Se révolter \\[0pt]
Serions-nous heureux dans un ordre politique parfait ? \\[0pt]
Serions-nous plus libres sans État ? \\[0pt]
Servir \\[0pt]
Servir, est-ce nécessairement renoncer à sa liberté ? \\[0pt]
Servir l'État \\[0pt]
Se savoir mortel \\[0pt]
Se sentir libre \\[0pt]
Se sentir libre implique-t-il qu'on le soit ? \\[0pt]
Se souvenir \\[0pt]
Se souvenir, est-ce préparer l'avenir ? \\[0pt]
Se suffire à soi-même \\[0pt]
Se taire \\[0pt]
S'étonner \\[0pt]
« Se trouver » a-t-il un sens ? \\[0pt]
Seul \\[0pt]
Seul le présent existe-t-il ? \\[0pt]
Seuls les humains sont-ils libres ? \\[0pt]
Se voiler la face \\[0pt]
Sexe et genre \\[0pt]
S'exercer \\[0pt]
S'exprimer \\[0pt]
Sexualité et féminité \\[0pt]
Sexualité et nature \\[0pt]
Si\ldots{} alors \\[0pt]
Si Dieu n'existe pas, tout est-il permis ? \\[0pt]
Si Dieu n'existe pas, tout est-il possible ? \\[0pt]
Signe et symbole \\[0pt]
Signes naturels, signes artificiels \\[0pt]
Signes, traces et indices \\[0pt]
Signification et expression \\[0pt]
Signification et vérité \\[0pt]
Si l'esprit n'est pas une table rase, qu'est-il ? \\[0pt]
Si l'État n'existait pas, faudrait-il l'inventer ? \\[0pt]
Simple / composé \\[0pt]
Sincérité et authenticité \\[0pt]
Sincérité et vérité \\[0pt]
S'indigner \\[0pt]
S'indigner, est-ce un devoir ? \\[0pt]
Si nous étions moraux, le droit serait-il inutile ? \\[0pt]
S'intéresser à l'art \\[0pt]
Si tout est historique, tout est-il relatif ? \\[0pt]
« Si tu veux la paix, prépare la guerre » \\[0pt]
Si tu veux, tu peux \\[0pt]
Société et biologie \\[0pt]
Société et communauté \\[0pt]
Société et contrat \\[0pt]
Société et liberté \\[0pt]
Société et mouvement \\[0pt]
Société et organisme \\[0pt]
Société et réciprocité \\[0pt]
Société et religion \\[0pt]
Société humaines, sociétés animales \\[0pt]
Sociologie et anthropologie \\[0pt]
Sociologie et économie \\[0pt]
Sociologie et ethnologie \\[0pt]
Socrate \\[0pt]
Soi \\[0pt]
Soigner \\[0pt]
« Sois naturel » : est-ce un bon conseil ? \\[0pt]
« Sois toi-même ! » : un impératif absurde ? \\[0pt]
Solitude et isolement \\[0pt]
Solitude et liberté \\[0pt]
Sommes-nous adaptés au monde de la technique ? \\[0pt]
Sommes-nous capables d'agir de manière désintéressée ? \\[0pt]
Sommes-nous condamnés à être libres ? \\[0pt]
Sommes-nous conscients de nos mobiles ? \\[0pt]
Sommes-nous dans le temps comme dans l'espace ? \\[0pt]
Sommes-nous des êtres métaphysiques ? \\[0pt]
Sommes-nous des sujets ? \\[0pt]
Sommes-nous déterminés par notre culture ? \\[0pt]
Sommes-nous dominés par la technique ? \\[0pt]
Sommes-nous égaux devant le bonheur ? \\[0pt]
Sommes-nous faits pour être heureux ? \\[0pt]
Sommes-nous faits pour la vérité ? \\[0pt]
Sommes-nous faits pour le bonheur ? \\[0pt]
Sommes-nous faits pour vivre en société ? \\[0pt]
Sommes-nous gouvernés par nos passions ? \\[0pt]
Sommes-nous jamais certains d'avoir choisi librement ? \\[0pt]
Sommes-nous les jouets de l'histoire ? \\[0pt]
Sommes-nous les jouets de nos pulsions ? \\[0pt]
Sommes-nous libres ? \\[0pt]
Sommes-nous libres de nos croyances ? \\[0pt]
Sommes-nous libres de nos pensées ? \\[0pt]
Sommes-nous libres de nos préférences morales ? \\[0pt]
Sommes-nous libres face à l'évidence ? \\[0pt]
Sommes-nous libres lorsque les actes émanent de notre personne ? \\[0pt]
Sommes-nous libres par nature ? \\[0pt]
Sommes-nous maîtres de la nature ? \\[0pt]
Sommes-nous maîtres de nos désirs ? \\[0pt]
Sommes-nous maîtres de nos paroles ? \\[0pt]
Sommes-nous maîtres de nos pensées ? \\[0pt]
Sommes-nous menacés par les progrès techniques ? \\[0pt]
Sommes-nous perfectibles ? \\[0pt]
Sommes-nous portés au bien ? \\[0pt]
Sommes-nous prisonniers de nos désirs ? \\[0pt]
Sommes-nous prisonniers de notre histoire ? \\[0pt]
Sommes-nous prisonniers du temps ? \\[0pt]
Sommes-nous responsables d'autrui ? \\[0pt]
Sommes-nous responsables de ce dont nous n'avons pas conscience ? \\[0pt]
Sommes-nous responsables de ce que nous sommes ? \\[0pt]
Sommes-nous responsables de nos désirs ? \\[0pt]
Sommes-nous responsables de nos erreurs ? \\[0pt]
Sommes-nous responsables de nos opinions ? \\[0pt]
Sommes-nous responsables de nos passions ? \\[0pt]
Sommes-nous responsables de notre être social ? \\[0pt]
Sommes-nous responsables du sens que prennent nos paroles ? \\[0pt]
Sommes-nous séparés du réel par les représentations que nous nous en faisons ? \\[0pt]
Sommes-nous soumis au temps ? \\[0pt]
Sommes-nous sujets de nos désirs ? \\[0pt]
Sommes-nous toujours conscients des causes de nos désirs ? \\[0pt]
Sommes-nous toujours dépendants d'autrui ? \\[0pt]
Sommes-nous tous artistes ? \\[0pt]
Sommes-nous tous contemporains ? \\[0pt]
Sommes-nous vulnérables toute notre vie ? \\[0pt]
Songe et réalité \\[0pt]
Sophisme et paralogisme \\[0pt]
Sophismes et paradoxes \\[0pt]
S'opposer \\[0pt]
S'opposer à l'autorité, est-ce toujours une marque de liberté ? \\[0pt]
S'orienter \\[0pt]
S'orienter dans la pensée \\[0pt]
S'orienter dans le monde \\[0pt]
Sortir de soi \\[0pt]
Souffrir \\[0pt]
Soumission et servitude \\[0pt]
Soupçonner \\[0pt]
Soutenir une thèse \\[0pt]
Souveraineté et liberté \\[0pt]
Soyez naturel ! \\[0pt]
Sphère privée et sphère publique \\[0pt]
Spontanéité et liberté \\[0pt]
Sport et politique \\[0pt]
Structure et événement \\[0pt]
Structure et histoire \\[0pt]
Subir \\[0pt]
Subissons-nous la langue que nous parlons ? \\[0pt]
Subjectivité et vérité \\[0pt]
Substance et accident \\[0pt]
Substance et sujet \\[0pt]
Suffit-il à Dieu d'être possible pour exister ? \\[0pt]
Suffit-il d'avoir raison ? \\[0pt]
Suffit-il de bien juger pour bien faire ? \\[0pt]
Suffit-il de bien penser pour bien agir ? \\[0pt]
Suffit-il de communiquer pour dialoguer ? \\[0pt]
Suffit-il de démontrer pour convaincre ? \\[0pt]
Suffit-il de faire son devoir ? \\[0pt]
Suffit-il de faire son devoir pour être vertueux ? \\[0pt]
Suffit-il de n'avoir rien fait pour être innocent ? \\[0pt]
Suffit-il de se parler pour éviter la discorde ? \\[0pt]
Suffit-il de suivre ses convictions pour bien agir ? \\[0pt]
Suffit-il de tenir ses promesses pour faire son devoir ? \\[0pt]
Suffit-il d'être conscient de ses actes pour en être responsable ? \\[0pt]
Suffit-il d'être dans le vrai pour savoir ? \\[0pt]
Suffit-il d'être informé pour comprendre ? \\[0pt]
Suffit-il d'être juste ? \\[0pt]
Suffit-il d'être raisonnable pour être moral ? \\[0pt]
Suffit-il d'être soi-même pour être libre ? \\[0pt]
Suffit-il d'être vertueux pour être heureux ? \\[0pt]
Suffit-il de vivre pour exister ? \\[0pt]
Suffit-il de voir le meilleur pour le suivre ? \\[0pt]
Suffit-il de voir pour savoir ? \\[0pt]
Suffit-il de vouloir pour bien faire ? \\[0pt]
Suffit-il, pour croire, de le vouloir ? \\[0pt]
Suffit-il, pour être juste, d'obéir aux lois et aux coutumes de son pays ? \\[0pt]
Suffit-il que nos intentions soient bonnes pour que nos actions le soient aussi ? \\[0pt]
Suicide et homicide \\[0pt]
Suis-ce que j'ai conscience d'être ? \\[0pt]
Suis-je au centre de l'espace ? \\[0pt]
Suis-je aussi ce que j'aurais pu être ? \\[0pt]
Suis-je ce que j'ai conscience d'être ? \\[0pt]
Suis-je ce que je fais ? \\[0pt]
Suis-je ce que les autres font de moi ? \\[0pt]
Suis-je ce que mon passé a fait de moi ? \\[0pt]
Suis-je dans le temps comme je suis dans l'espace ? \\[0pt]
Suis-je étranger à moi-même ? \\[0pt]
Suis-je l'auteur de ce que je dis ? \\[0pt]
Suis-je l'auteur de mes actions ? \\[0pt]
Suis-je le même en des temps différents ? \\[0pt]
Suis-je le mieux placé pour me connaître ? \\[0pt]
Suis-je le mieux placé pour savoir qui je suis ? \\[0pt]
Suis-je le sujet de mes pensées ? \\[0pt]
Suis-je libre ? \\[0pt]
Suis-je maître de ma conscience ? \\[0pt]
Suis-je maître de mes pensées ? \\[0pt]
Suis-je ma mémoire ? \\[0pt]
Suis-je mon cerveau ? \\[0pt]
Suis-je mon corps ? \\[0pt]
Suis-je mon passé ? \\[0pt]
Suis-je propriétaire de mon corps ? \\[0pt]
Suis-je responsable de ce dont je n'ai pas conscience ? \\[0pt]
Suis-je responsable de ce que je suis ? \\[0pt]
Suis-je seul au monde ? \\[0pt]
Suis-je toujours autre que moi-même ? \\[0pt]
Suis-je toujours le même ? \\[0pt]
Suivre la coutume \\[0pt]
Suivre la nature \\[0pt]
Suivre la règle \\[0pt]
Suivre le plus sûr \\[0pt]
Suivre sa nature \\[0pt]
Suivre son intuition \\[0pt]
Suivre une règle \\[0pt]
Sujet et citoyen \\[0pt]
Sujet et prédicat \\[0pt]
Sujet et substance \\[0pt]
Superstition et fanatisme sont-ils inhérents à la religion ? \\[0pt]
Superstition et religion \\[0pt]
Surface et profondeur \\[0pt]
Sur quels fondements distinguer ce qui est public et ce qui est privé ? \\[0pt]
Sur quoi fonder la justice ? \\[0pt]
Sur quoi fonder la légitimité de la loi ? \\[0pt]
Sur quoi fonder la propriété ? \\[0pt]
Sur quoi fonder la société ? \\[0pt]
Sur quoi fonder l'autorité ? \\[0pt]
Sur quoi fonder l'autorité politique ? \\[0pt]
Sur quoi fonder le devoir ? \\[0pt]
Sur quoi fonder le droit de punir ? \\[0pt]
Sur quoi le langage doit-il se régler ? \\[0pt]
Sur quoi l'historien travaille-t-il ? \\[0pt]
Sur quoi peut-on fonder la certitude d'avoir raison ? \\[0pt]
Sur quoi repose l'accord des esprits ? \\[0pt]
Sur quoi repose la croyance au progrès ? \\[0pt]
Sur quoi reposent nos certitudes ? \\[0pt]
Sur quoi se fonde la connaissance scientifique ? \\[0pt]
Sur quoi se fonde la vérité des énoncés ? \\[0pt]
Sur quoi sont fondées les mathématiques ? \\[0pt]
Surveillance et discipline \\[0pt]
Surveiller son comportement \\[0pt]
Survivre \\[0pt]
Suspendre son assentiment \\[0pt]
Suspendre son jugement \\[0pt]
Syllogisme et démonstration \\[0pt]
Sympathie et respect \\[0pt]
Système et liberté \\[0pt]
Système et structure \\[0pt]
Talent et génie \\[0pt]
Tantôt je pense, tantôt je suis \\[0pt]
Tautologie et contradiction \\[0pt]
Technique et apprentissage \\[0pt]
Technique et démocratie \\[0pt]
Technique et esthétique \\[0pt]
Technique et idéologie \\[0pt]
Technique et intérêt \\[0pt]
Technique et liberté \\[0pt]
Technique et nature \\[0pt]
Technique et pratiques scientifiques \\[0pt]
Technique et progrès \\[0pt]
Technique et responsabilité \\[0pt]
Technique et savoir-faire \\[0pt]
Technique et technologie \\[0pt]
Technique et violence \\[0pt]
Témoignage et vérité \\[0pt]
Témoigner \\[0pt]
Temps et commencement \\[0pt]
Temps et conscience \\[0pt]
Temps et création \\[0pt]
Temps et durée \\[0pt]
Temps et éternité \\[0pt]
Temps et histoire \\[0pt]
Temps et irréversibilité \\[0pt]
Temps et liberté \\[0pt]
Temps et mémoire \\[0pt]
Temps et musique \\[0pt]
Temps et négation \\[0pt]
Temps et réalité \\[0pt]
Temps et vérité \\[0pt]
Temps individuel et temps collectif \\[0pt]
Temps physique et temps biologique \\[0pt]
Tendances et besoins \\[0pt]
Tenir parole \\[0pt]
Tenir pour vrai \\[0pt]
Tenir sa parole \\[0pt]
Terres et territoires \\[0pt]
Territoire et peuplement \\[0pt]
Terroriser \\[0pt]
Tester une hypothèse permet-il de la prouver ? \\[0pt]
Texte, son, image \\[0pt]
Thème et variations \\[0pt]
Théologie et morale \\[0pt]
Théorie et expérience \\[0pt]
Théorie et modèle \\[0pt]
Théorie et modélisation \\[0pt]
Théorie et pratique \\[0pt]
Théorie et pratique en médecine \\[0pt]
Tirer des leçons du passé \\[0pt]
Tolérance et laïcité \\[0pt]
Tolérer \\[0pt]
Tomber amoureux \\[0pt]
Toucher \\[0pt]
Toucher, sentir, goûter \\[0pt]
Toujours plus vite ? \\[0pt]
Tous les arts se valent-ils ? \\[0pt]
Tous les conflits peuvent-ils être résolus par le dialogue ? \\[0pt]
Tous les désirs sont-ils naturels ? \\[0pt]
Tous les droits sont-ils formels ? \\[0pt]
Tous les hommes désirent-ils connaître ? \\[0pt]
Tous les hommes désirent-ils être heureux ? \\[0pt]
Tous les hommes désirent-ils naturellement être heureux ? \\[0pt]
Tous les hommes désirent-ils naturellement savoir ? \\[0pt]
Tous les hommes sont-ils égaux ? \\[0pt]
Tous les paradis sont-ils perdus ? \\[0pt]
Tous les plaisirs se valent-ils ? \\[0pt]
Tous les problèmes peuvent-ils recevoir une solution technique ? \\[0pt]
Tous les rapports humains sont-ils des échanges ? \\[0pt]
Tout art est-il abstrait ? \\[0pt]
Tout art est-il décoratif ? \\[0pt]
Tout art est-il poésie ? \\[0pt]
Tout art est-il symbolique ? \\[0pt]
Tout a-t-il une cause ? \\[0pt]
Tout a-t-il une fin ? \\[0pt]
Tout a-t-il une raison d'être ? \\[0pt]
Tout a-t-il un prix ? \\[0pt]
Tout a-t-il un sens ? \\[0pt]
Tout bien n'est-il qu'un moindre mal ? \\[0pt]
Tout ce qui est excessif est-il insignifiant ? \\[0pt]
Tout ce qui est humain est-il signifiant ? \\[0pt]
Tout ce qui est humain nous est-il familier ? \\[0pt]
Tout ce qui est naturel est-il normal ? \\[0pt]
Tout ce qui est rationnel est-il raisonnable ? \\[0pt]
Tout ce qui est vrai doit-il être prouvé ? \\[0pt]
Tout ce qui existe a-t-il un prix ? \\[0pt]
Tout ce qui existe est-il matériel ? \\[0pt]
Tout change-t-il avec le temps ? \\[0pt]
Tout comprendre, est-ce tout pardonner ? \\[0pt]
Tout définir, tout démontrer \\[0pt]
Tout démontrer \\[0pt]
Tout désir est-il désir de posséder ? \\[0pt]
Tout désir est-il désir de quelque chose ? \\[0pt]
Tout désir est-il égoïste ? \\[0pt]
Tout désir est-il manque ? \\[0pt]
Tout désir est-il nostalgique ? \\[0pt]
Tout désir est-il une souffrance ? \\[0pt]
Tout devoir est-il l'envers d'un droit ? \\[0pt]
Tout dire \\[0pt]
Tout droit est-il un pouvoir ? \\[0pt]
Toute action politique est-elle collective ? \\[0pt]
Toute chose a-t-elle une essence ? \\[0pt]
Toute chose a-t-elle une fin ? \\[0pt]
Toute communauté est-elle politique ? \\[0pt]
Toute compréhension implique-t-elle une interprétation ? \\[0pt]
Toute conception de l'humain est-elle particulière ? \\[0pt]
Toute connaissance autre que scientifique doit-elle être considérée comme une illusion ? \\[0pt]
Toute connaissance commence-t-elle avec l'expérience ? \\[0pt]
Toute connaissance consiste-t-elle en un savoir-faire ? \\[0pt]
Toute connaissance est-elle historique ? \\[0pt]
Toute connaissance est-elle hypothétique ? \\[0pt]
Toute connaissance est-elle relative ? \\[0pt]
Toute connaissance s'enracine-t-elle dans la perception ? \\[0pt]
Toute conscience est-elle conscience de quelque chose ? \\[0pt]
Toute conscience est-elle conscience de soi ? \\[0pt]
Toute conscience est-elle subjective ? \\[0pt]
Toute conscience n'est-elle pas implicitement morale ? \\[0pt]
Toute conversation est-elle futile ? \\[0pt]
Toute définition est-elle une convention ? \\[0pt]
Toute description est-elle une interprétation ? \\[0pt]
Toute existence est-elle indémontrable ? \\[0pt]
Toute expérience appelle-t-elle une interprétation ? \\[0pt]
Toute expression est-elle métaphorique ? \\[0pt]
Toute faute est-elle une erreur ? \\[0pt]
Toute hiérarchie est-elle inégalitaire ? \\[0pt]
Toute inégalité est-elle injuste ? \\[0pt]
Toute interprétation est-elle contestable ? \\[0pt]
Toute interprétation est-elle subjective ? \\[0pt]
Toute loi est-elle arbitraire ? \\[0pt]
Toute métaphysique implique-t-elle une transcendance ? \\[0pt]
Toute morale est-elle rationnelle ? \\[0pt]
Toute morale implique-t-elle l'effort ? \\[0pt]
Toute morale s'oppose-t-elle aux désirs ? \\[0pt]
Tout énoncé admet-il une interprétation ? \\[0pt]
Tout énoncé est-il nécessairement vrai ou faux ? \\[0pt]
Toute notre connaissance dérive-t-elle de l'expérience ? \\[0pt]
Toute œuvre d'art exige-t-elle une interprétation ? \\[0pt]
Toute origine est-elle mythique ? \\[0pt]
Toute passion fait-elle souffrir ? \\[0pt]
« Toute peine mérite salaire » \\[0pt]
Toute pensée revêt-elle nécessairement une forme linguistique ? \\[0pt]
Toute pensée revêt-elle une forme linguistique ? \\[0pt]
Toute peur est-elle irrationnelle ? \\[0pt]
Toute philosophie constitue-t-elle une doctrine ? \\[0pt]
Toute philosophie est-elle systématique ? \\[0pt]
Toute philosophie implique-t-elle une politique ? \\[0pt]
Toute physique exige-t-elle une métaphysique ? \\[0pt]
Toute polémique est-elle stérile ? \\[0pt]
Toute psychologie est-elle normalisatrice ? \\[0pt]
Toute relation humaine est-elle un échange ? \\[0pt]
Toute religion a-t-elle sa vérité ? \\[0pt]
Toute science est-elle naturelle ? \\[0pt]
Toutes les choses sont-elles singulières ? \\[0pt]
Toutes les convictions sont-elles respectables ? \\[0pt]
Toutes les croyances se valent-elles ? \\[0pt]
Toutes les fautes se valent-elles ? \\[0pt]
Toutes les inégalités ont-elles une importance politique ? \\[0pt]
Toutes les inégalités sont-elles des injustices ? \\[0pt]
Toutes les interprétations se valent-elles ? \\[0pt]
Toutes les opinions se valent-elles ? \\[0pt]
Toutes les opinions sont-elles bonnes à dire ? \\[0pt]
Toutes les vérités scientifiques sont-elles révisables ? \\[0pt]
Toute société a-t-elle besoin d'une religion ? \\[0pt]
Tout est corps \\[0pt]
Tout est-il affaire de point de vue ? \\[0pt]
Tout est-il à vendre ? \\[0pt]
Tout est-il connaissable ? \\[0pt]
Tout est-il contingent ? \\[0pt]
Tout est-il démontrable ? \\[0pt]
Tout est-il digne de mémoire ? \\[0pt]
Tout est-il faux dans la fiction ? \\[0pt]
Tout est-il historique ? \\[0pt]
Tout est-il matière ? \\[0pt]
Tout est-il mesurable ? \\[0pt]
Tout est-il nécessaire ? \\[0pt]
Tout est-il politique ? \\[0pt]
Tout est-il quantifiable ? \\[0pt]
Tout est-il relatif ? \\[0pt]
Tout est-il vraiment permis, si Dieu n'existe pas ? \\[0pt]
Tout est permis \\[0pt]
« Tout est politique » \\[0pt]
« Tout est possible » \\[0pt]
« Tout est relatif » \\[0pt]
Tout est relatif \\[0pt]
Tout est vanité \\[0pt]
Tout être est-il dans l'espace ? \\[0pt]
Tout événement a-t-il une cause ? \\[0pt]
Toute vérité doit-elle être dite ? \\[0pt]
Toute vérité est-elle bonne à dire ? \\[0pt]
Toute vérité est-elle démontrable ? \\[0pt]
Toute vérité est-elle nécessaire ? \\[0pt]
Toute vérité est-elle une erreur rectifiée ? \\[0pt]
Toute vérité est-elle un rapport ? \\[0pt]
Toute vérité est-elle vérifiable ? \\[0pt]
Toute vérité se laisse-t-elle démontrer ? \\[0pt]
Toute vie est-elle intrinsèquement respectable ? \\[0pt]
Toute violence est-elle contre nature ? \\[0pt]
Tout exercice du pouvoir politique implique-t-il l'usage de la violence ? \\[0pt]
Tout fondement de la connaissance est-il métaphysique ? \\[0pt]
Tout futur est-il contingent ? \\[0pt]
Tout malheur est-il une injustice ? \\[0pt]
Tout ordre est-il rationnel ? \\[0pt]
Tout ordre est-il une violence déguisée ? \\[0pt]
Tout ou rien \\[0pt]
Tout passe-t-il avec le temps ? \\[0pt]
Tout peut-il avoir une valeur marchande ? \\[0pt]
Tout peut-il avoir valeur marchande ? \\[0pt]
Tout peut-il être dit ? \\[0pt]
Tout peut-il être expliqué historiquement ? \\[0pt]
Tout peut-il être objet d'échange ? \\[0pt]
Tout peut-il être objet de jugement esthétique ? \\[0pt]
Tout peut-il être objet de science ? \\[0pt]
Tout peut-il n'être qu'apparence ? \\[0pt]
Tout peut-il s'acheter ? \\[0pt]
Tout peut-il se démontrer ? \\[0pt]
Tout peut-il se quantifier ? \\[0pt]
Tout peut-il se vendre ? \\[0pt]
Tout peut-il s'expliquer ? \\[0pt]
Tout pouvoir corrompt-il ? \\[0pt]
Tout pouvoir est-il oppresseur ? \\[0pt]
Tout pouvoir est-il politique ? \\[0pt]
Tout pouvoir est-il usurpé ? \\[0pt]
Tout pouvoir n'est-il pas abusif ? \\[0pt]
Tout pouvoir relève-t-il de la domination ? \\[0pt]
Tout principe est-il un fondement ? \\[0pt]
Tout savoir \\[0pt]
Tout savoir a-t-il une justification ? \\[0pt]
Tout savoir est-il fondé sur un savoir premier ? \\[0pt]
Tout savoir est-il pouvoir ? \\[0pt]
Tout savoir est-il transmissible ? \\[0pt]
Tout savoir est-il un pouvoir ? \\[0pt]
Tout savoir peut-il se transmettre ? \\[0pt]
Tout s'en va-t-il avec le temps ? \\[0pt]
Tout se prête-il à la mesure ? \\[0pt]
Tout texte est-il à interpréter ? \\[0pt]
Tout travail a-t-il un sens ? \\[0pt]
Tout travail est-il forcé ? \\[0pt]
Tout travail est-il social ? \\[0pt]
Tout travail mérite salaire \\[0pt]
Tout vouloir \\[0pt]
Tradition et innovation \\[0pt]
Tradition et liberté \\[0pt]
Tradition et nouveauté \\[0pt]
Tradition et raison \\[0pt]
Tradition et transmission \\[0pt]
Tradition et vérité \\[0pt]
« Tradition n'est pas raison » \\[0pt]
Traduction, Trahison \\[0pt]
Traduire \\[0pt]
Traduire, est-ce trahir ? \\[0pt]
Traduire et interpréter \\[0pt]
Tragédie et comédie \\[0pt]
Trahir \\[0pt]
Traiter autrui comme une chose \\[0pt]
Traiter des faits humains comme des choses, est-ce considérer l'homme comme une chose ? \\[0pt]
Traiter les faits humains comme des choses, est-ce réduire les hommes à des choses ? \\[0pt]
Transcendance et altérité \\[0pt]
Transcendance et immanence \\[0pt]
Transmettre \\[0pt]
Travail, besoin, désir \\[0pt]
Travail et aliénation \\[0pt]
Travail et besoin \\[0pt]
Travail et bonheur \\[0pt]
Travail et capital \\[0pt]
Travail et liberté \\[0pt]
Travail et loisir \\[0pt]
Travail et nécessité \\[0pt]
Travail et œuvre \\[0pt]
Travail et plaisir \\[0pt]
Travail et propriété \\[0pt]
Travail et subjectivité \\[0pt]
Travailler, est-ce faire œuvre ? \\[0pt]
Travailler, est-ce lutter contre soi-même ? \\[0pt]
Travailler, est-ce seulement produire ? \\[0pt]
Travailler et œuvrer \\[0pt]
Travailler par plaisir, est-ce encore travailler ? \\[0pt]
Travailler sans souffrir peut-il être un idéal ? \\[0pt]
Travaille-t-on pour soi-même ? \\[0pt]
Travail manuel et travail intellectuel \\[0pt]
Travail manuel, travail intellectuel \\[0pt]
Travail servile et travail salarié \\[0pt]
Tricher \\[0pt]
Tromper \\[0pt]
« Trop beau pour être vrai » \\[0pt]
Trouver et prouver \\[0pt]
Trouver sa voie \\[0pt]
Tu aimeras ton prochain comme toi-même \\[0pt]
« Tu dois, donc tu peux » \\[0pt]
Tuer et laisser mourir \\[0pt]
Tuer le temps \\[0pt]
« Tu ne tueras point » \\[0pt]
Tyrannie et dictature \\[0pt]
Un acte désintéressé est-il possible ? \\[0pt]
Un acte gratuit est-il possible ? \\[0pt]
Un acte inconscient est-il nécessairement un acte involontaire ? \\[0pt]
Un acte libre est-il un acte imprévisible ? \\[0pt]
Un acte peut-il être inhumain ? \\[0pt]
Un artiste doit-il être génial ? \\[0pt]
Un artiste doit-il être original ? \\[0pt]
Un art peut-il être populaire ? \\[0pt]
Un art peut-il se passer de règles ? \\[0pt]
Un art sans sublimation est-il possible ? \\[0pt]
Un autre monde est-il possible ? \\[0pt]
Un bien peut-il être commun ? \\[0pt]
Un bien peut-il sortir d'un mal ? \\[0pt]
Un chef-d'œuvre est-il immortel ? \\[0pt]
Un choix peut-il être rationnel ? \\[0pt]
Un choix rationnel est-il un choix libre ? \\[0pt]
Un contrat peut-il être injuste ? \\[0pt]
Un contrat peut-il être social ? \\[0pt]
Un corps n'est-il qu'un objet ? \\[0pt]
Un désir peut-il être coupable ? \\[0pt]
Un désir peut-il être inconscient ? \\[0pt]
Un devoir admet-il des exceptions ? \\[0pt]
Un devoir peut-il être absolu ? \\[0pt]
Un Dieu unique ? \\[0pt]
Une action peut-elle être désintéressée ? \\[0pt]
Une action peut-elle être machinale ? \\[0pt]
Une action se juge-t-elle à ses conséquences ? \\[0pt]
Une action vertueuse se reconnaît-elle à sa difficulté ? \\[0pt]
Une activité inutile est-elle sans valeur ? \\[0pt]
Une bonne cité peut-elle se passer du beau ? \\[0pt]
Une cause peut-elle être libre ? \\[0pt]
Une communauté politique n'est-elle qu'une communauté d'intérêt ? \\[0pt]
Une communauté politique n'est-elle qu'une communauté d'intérêts ? \\[0pt]
Une connaissance peut-elle ne pas être relative ? \\[0pt]
Une connaissance scientifique du vivant est-elle possible ? \\[0pt]
Une croyance infondée est-elle illégitime ? \\[0pt]
Une croyance peut-elle être libre ? \\[0pt]
Une croyance peut-elle être rationnelle ? \\[0pt]
Une culture constitue-t-elle un paradigme ? \\[0pt]
Une culture de masse est-elle une culture ? \\[0pt]
Une culture peut-elle être porteuse de valeurs universelles ? \\[0pt]
Une décision politique peut-elle être juste ? \\[0pt]
Une destruction peut-elle être créatrice ? \\[0pt]
Une d'œuvre peut-elle être achevée ? \\[0pt]
Une durée peut-elle être éternelle ? \\[0pt]
Une éducation des sentiments est-elle possible ? \\[0pt]
Une éducation esthétique est-elle possible ? \\[0pt]
Une éducation morale est-elle possible ? \\[0pt]
Une éthique sceptique est-elle possible ? \\[0pt]
Une existence se démontre-t-elle ? \\[0pt]
Une expérience peut-elle être fictive ? \\[0pt]
Une expérience se partage-t-elle ? \\[0pt]
Une explication peut-elle être réductrice ? \\[0pt]
Une fausse science est-elle une science qui commet des erreurs ? \\[0pt]
Une femme politique est-elle un homme politique comme les autres ? \\[0pt]
Une fiction peut-elle être vraie ? \\[0pt]
Une foi rationnelle \\[0pt]
Une guerre peut-elle être juste ? \\[0pt]
Une idée peut-elle être fausse ? \\[0pt]
Une idée peut-elle être générale ? \\[0pt]
Une imitation peut-elle être parfaite ? \\[0pt]
Une inégalité peut-elle être juste ? \\[0pt]
Une injustice vaut elle mieux qu'un désordre ? \\[0pt]
Une intention peut-elle être coupable ? \\[0pt]
Une interprétation est-elle nécessairement subjective ? \\[0pt]
Une interprétation peut-elle échapper à l'arbitraire ? \\[0pt]
Une interprétation peut-elle être définitive ? \\[0pt]
Une interprétation peut-elle être fausse ? \\[0pt]
Une interprétation peut-elle être objective ? \\[0pt]
Une interprétation peut-elle être vraie ? \\[0pt]
Une interprétation peut-elle prétendre à la vérité ? \\[0pt]
Une justice sans égalité est-elle possible ? \\[0pt]
Une langue n'est-elle faite que de mots ? \\[0pt]
Une ligne de conduite peut-elle tenir lieu de morale ? \\[0pt]
Une logique non-formelle est-elle possible ? \\[0pt]
Une loi n'est-elle qu'une règle ? \\[0pt]
Une loi peut-elle être injuste ? \\[0pt]
Une machine n'est-elle qu'un outil perfectionné ? \\[0pt]
Une machine peut-elle avoir une mémoire ? \\[0pt]
Une machine peut-elle être naturelle ? \\[0pt]
Une machine peut-elle penser ? \\[0pt]
Une machine pourrait-elle penser ? \\[0pt]
Une machine se réduit-elle à un mécanisme ? \\[0pt]
« Une main de fer dans un gant de velours » \\[0pt]
Une matière sans forme est-elle concevable ? \\[0pt]
Une métaphysique athée est-elle possible ? \\[0pt]
Une métaphysique n'est-elle qu'une ontologie ? \\[0pt]
Une métaphysique peut-elle être sceptique ? \\[0pt]
Une morale du plaisir est-elle concevable ? \\[0pt]
Une morale personnelle est-elle une contradiction dans les termes ? \\[0pt]
Une morale peut-elle être dépassée ? \\[0pt]
Une morale peut-elle être permissive ? \\[0pt]
Une morale peut-elle être provisoire ? \\[0pt]
Une morale peut-elle prétendre à l'universalité ? \\[0pt]
Une morale peut-elle se passer de la notion de vertu ? \\[0pt]
Une morale qui ne vise pas l'utilité est-elle rationnelle ? \\[0pt]
Une morale sans devoirs est-elle possible ? \\[0pt]
Une morale sans Dieu \\[0pt]
Une morale sans obligation est-elle possible ? \\[0pt]
Une morale sceptique est-elle possible ? \\[0pt]
Un enfant est-il un citoyen ? \\[0pt]
Une œuvre d'art a-t-elle toujours un sens ? \\[0pt]
Une œuvre d'art a-t-elle un prix ? \\[0pt]
Une œuvre d'art doit-elle avoir un sens ? \\[0pt]
Une œuvre d'art doit-elle nécessairement être belle ? \\[0pt]
Une œuvre d'art doit-elle plaire ? \\[0pt]
Une œuvre d'art est-elle immortelle ? \\[0pt]
Une œuvre d'art est-elle toujours originale ? \\[0pt]
Une œuvre d'art est-elle une marchandise ? \\[0pt]
Une œuvre d'art peut-elle être exemplaire ? \\[0pt]
Une œuvre d'art peut-elle être immorale ? \\[0pt]
Une œuvre d'art peut-elle être laide ? \\[0pt]
Une œuvre d'art produit-elle du beau ? \\[0pt]
Une œuvre d'art s'explique-t-elle à partir de ses influences ? \\[0pt]
Une œuvre doit-elle nécessairement être belle ? \\[0pt]
Une œuvre est-elle nécessairement singulière ? \\[0pt]
Une œuvre est-elle toujours de son temps ? \\[0pt]
Une passion peut-elle résister au temps ? \\[0pt]
Une pensée contradictoire est-elle dénuée de valeur ? \\[0pt]
Une perception peut-elle être illusoire ? \\[0pt]
Une philosophie de l'amour est-elle possible ? \\[0pt]
Une philosophie peut-elle être réactionnaire ? \\[0pt]
Une politique de la nature est-elle concevable ? \\[0pt]
Une politique morale est-elle possible ? \\[0pt]
Une politique peut-elle se réclamer de la vie ? \\[0pt]
Une psychologie peut-elle être matérialiste ? \\[0pt]
Une religion civile est-elle possible ? \\[0pt]
Une religion peut-elle être fausse ? \\[0pt]
Une religion peut-elle être rationnelle ? \\[0pt]
Une religion peut-elle être universelle ? \\[0pt]
Une religion peut-elle prétendre à la vérité ? \\[0pt]
Une religion peut-elle se passer de pratiques ? \\[0pt]
Une religion rationnelle est-elle possible ? \\[0pt]
Une science bien faite est-elle une langue bien faite ? \\[0pt]
Une science de la conscience est-elle possible ? \\[0pt]
Une science de la culture est-elle possible ? \\[0pt]
Une science de la morale est-elle possible ? \\[0pt]
Une science de l'éducation est-elle possible ? \\[0pt]
Une science de l'esprit est-elle possible ? \\[0pt]
Une science de l'homme est-elle possible ? \\[0pt]
Une science de l'imaginaire est-elle possible ? \\[0pt]
Une science des symboles est-elle possible ? \\[0pt]
Une science parfaite est-elle possible ? \\[0pt]
Une science peut-elle se passer d'expérimentation ? \\[0pt]
Une seconde nature \\[0pt]
Une sensation peut-elle être fausse ? \\[0pt]
Une société basée sur le don pourrait-elle être prospère ? \\[0pt]
Une société d'athées est-elle possible ? \\[0pt]
Une société juste, est-ce une société sans conflit ? \\[0pt]
Une société juste est-elle une société sans conflits ? \\[0pt]
Une société n'est-elle qu'un ensemble d'individus ? \\[0pt]
Une société peut-elle être juste ? \\[0pt]
Une société peut-elle se passer de religion ? \\[0pt]
Une société sans conflit est-elle pensable ? \\[0pt]
Une société sans conflit est-elle possible ? \\[0pt]
Une société sans conflits est-elle souhaitable ? \\[0pt]
Une société sans État est-elle possible ? \\[0pt]
Une société sans État est-elle une société sans politique ? \\[0pt]
Une société sans religion est-elle possible ? \\[0pt]
Une société sans travail est-elle souhaitable ? \\[0pt]
Un État mondial ? \\[0pt]
Un État peut-il être trop étendu ? \\[0pt]
Un État peut-il être « voyou » ? \\[0pt]
Une technique ne se réduit-elle pas toujours à une forme de bricolage ? \\[0pt]
Une théorie de la culture peut-elle être scientifique ? \\[0pt]
Une théorie peut-elle être vérifiée ? \\[0pt]
Une théorie sans expérience nous apprend-elle quelque chose ? \\[0pt]
Une théorie scientifique peut-elle devenir fausse ? \\[0pt]
Une théorie scientifique peut-elle être ramenée à des propositions empiriques élémentaires ? \\[0pt]
Une théorie scientifique peut-elle être vraie ? \\[0pt]
Un être vivant peut-il être comparé à une œuvre d'art ? \\[0pt]
Un événement historique est-il toujours imprévisible ? \\[0pt]
Une vérité approchée est-elle pour autant approximative ? \\[0pt]
Une vérité est-elle discutable ? \\[0pt]
Une vérité peut-elle être approximative ? \\[0pt]
Une vérité peut-elle être indicible ? \\[0pt]
Une vérité peut-elle être provisoire ? \\[0pt]
Une vérité sans preuve est-elle une vérité ? \\[0pt]
Une vie heureuse est-elle une vie de plaisirs ? \\[0pt]
Une vie juste est-elle une bonne vie ? \\[0pt]
Une vie libre exclut-elle le travail ? \\[0pt]
Une volonté peut-elle être générale ? \\[0pt]
Un fait existe-t-il sans interprétation ? \\[0pt]
Un fait scientifique doit-il être nécessairement démontré ? \\[0pt]
Un gouvernement de savants est-il souhaitable ? \\[0pt]
Un grand bonheur \\[0pt]
Un homme n'est-il que la somme de ses actes ? \\[0pt]
« Un instant d'éternité » \\[0pt]
Unité et unicité \\[0pt]
Universalité et nécessité dans les sciences \\[0pt]
Univocité et équivocité \\[0pt]
Un jeu peut-il être sérieux ? \\[0pt]
Un jugement de goût est-il culturel ? \\[0pt]
Un jugement moral peut-il relever de la vérité ? \\[0pt]
Un langage peut-il être privé ? \\[0pt]
Un langage universel est-il concevable ? \\[0pt]
Un mensonge peut-il avoir une valeur morale ? \\[0pt]
Un mensonge peut-il être légitime ? \\[0pt]
Un moment d'éternité \\[0pt]
Un monde meilleur \\[0pt]
Un monde sans beauté \\[0pt]
Un monde sans guerre serait-il un monde sans histoire ? \\[0pt]
Un monde sans nature est-il pensable ? \\[0pt]
Un monde sans travail est-il souhaitable ? \\[0pt]
Un objet technique peut-il être beau ? \\[0pt]
Un organisme n'est-il qu'un objet ? \\[0pt]
Un peuple en paix n'a-t-il plus d'histoire ? \\[0pt]
Un peuple est-il responsable de son histoire ? \\[0pt]
Un peuple est-il un rassemblement d'individus ? \\[0pt]
Un peuple peut-il être sans histoire ? \\[0pt]
Un peuple peut-il être souverain ? \\[0pt]
Un peuple se définit-il par son histoire ? \\[0pt]
Un philosophe a-t-il des devoirs envers la société ? \\[0pt]
Un plaisir peut-il être désintéressé ? \\[0pt]
Un pouvoir a-t-il besoin d'être légitime ? \\[0pt]
Un préjugé n'a-t-il aucun fondement ? \\[0pt]
Un problème moral peut-il recevoir une solution certaine ? \\[0pt]
Un problème philosophique peut-il être périmé ? \\[0pt]
Un problème scientifique peut-il être insoluble ? \\[0pt]
Un savoir peut-il être inconscient ? \\[0pt]
Un savoir scientifique sur l'homme est-il compatible avec l'idée de liberté ? \\[0pt]
Un savoir sur l'homme peut-il être séparé d'un pouvoir sur les hommes ? \\[0pt]
Un sceptique peut-il être logicien ? \\[0pt]
Un seul peut-il avoir raison contre tous ? \\[0pt]
Un souverain élu peut-il être illégitime ? \\[0pt]
Un tableau peut-il être une dénonciation ? \\[0pt]
Un témoin de moralité est-il possible ? \\[0pt]
Un travail bien fait est-il nécessairement un bon travail ? \\[0pt]
Un vice, est-ce un manque ? \\[0pt]
Urbanisme et politique \\[0pt]
User de violence peut-il être moral ? \\[0pt]
Utiliser les embryons humains ? \\[0pt]
Utilité et beauté \\[0pt]
Utopie et tradition \\[0pt]
Vaincre la mort \\[0pt]
Vainqueurs et vaincus \\[0pt]
Valeur artistique, valeur esthétique \\[0pt]
Valeur et évaluation \\[0pt]
Vanité des vanités \\[0pt]
Vaut-il mieux oublier ou pardonner ? \\[0pt]
Vaut-il mieux subir l'injustice que la commettre ? \\[0pt]
Vaut-il mieux subir ou commettre l'injustice ? \\[0pt]
Vendre son âme \\[0pt]
Vendre son corps \\[0pt]
Vérifier \\[0pt]
Vérité et apparence \\[0pt]
Vérité et certitude \\[0pt]
Vérité et cohérence \\[0pt]
Vérité et cohérence ? \\[0pt]
Vérité et convention \\[0pt]
Vérité et démonstration \\[0pt]
Vérité et efficacité \\[0pt]
Vérité et exactitude \\[0pt]
Vérité et fiction \\[0pt]
Vérité et histoire \\[0pt]
Vérité et illusion \\[0pt]
Vérité et liberté \\[0pt]
Vérité et poésie \\[0pt]
Vérité et réalité \\[0pt]
Vérité et référence \\[0pt]
Vérité et religion \\[0pt]
Vérité et sensibilité \\[0pt]
Vérité et signification \\[0pt]
Vérité et sincérité \\[0pt]
Vérité et subjectivité \\[0pt]
Vérité et totalité \\[0pt]
Vérité et validité \\[0pt]
Vérité et vérification \\[0pt]
Vérité et vraisemblance \\[0pt]
Vérités de fait et vérités de raison \\[0pt]
Vérités mathématiques, vérités philosophiques \\[0pt]
Vérité théorique, vérité pratique \\[0pt]
Vertu et habitude \\[0pt]
Vertu et perfection \\[0pt]
Vertus morales, vertus intellectuelles \\[0pt]
Veut-on toujours savoir ? \\[0pt]
Vice et délice \\[0pt]
Vices privés, vertus publiques \\[0pt]
Vie active, vie contemplative \\[0pt]
Vie et destin \\[0pt]
Vie et existence \\[0pt]
Vie et finalité \\[0pt]
Vie et matière \\[0pt]
Vie et matière : une distinction périmée ? \\[0pt]
Vie et mécanisme \\[0pt]
Vie et volonté \\[0pt]
Vie familiale, vie politique \\[0pt]
Vieillir \\[0pt]
Vie politique et vie contemplative \\[0pt]
Vie privée et vie publique \\[0pt]
Vie publique et vie privée \\[0pt]
Violence et discours \\[0pt]
Violence et force \\[0pt]
Violence et histoire \\[0pt]
Violence et politique \\[0pt]
Violence et pouvoir \\[0pt]
« Vis caché » \\[0pt]
Vitalisme et mécanique \\[0pt]
Vit-on au présent ? \\[0pt]
Vivons-nous au présent ? \\[0pt]
Vivons-nous tous dans le même monde ? \\[0pt]
Vivrait-on mieux sans désirs ? \\[0pt]
Vivrait-on mieux sans morale ? \\[0pt]
Vivre \\[0pt]
Vivre au jour le jour \\[0pt]
Vivre au présent \\[0pt]
Vivre caché \\[0pt]
Vivre comme si nous ne devions pas mourir \\[0pt]
Vivre dans son monde \\[0pt]
Vivre en immortel \\[0pt]
Vivre en société, est-ce seulement vivre ensemble ? \\[0pt]
Vivre, est-ce interpréter ? \\[0pt]
Vivre, est-ce lutter contre la mort ? \\[0pt]
Vivre, est-ce lutter pour survivre ? \\[0pt]
Vivre, est-ce résister à la mort ? \\[0pt]
Vivre, est-ce un droit ? \\[0pt]
Vivre et bien vivre \\[0pt]
Vivre et exister \\[0pt]
Vivre et exister, est-ce la même chose ? \\[0pt]
Vivre intensément \\[0pt]
Vivre le moment présent \\[0pt]
Vivre libre \\[0pt]
« Vivre libre ou mourir » \\[0pt]
Vivre pleinement \\[0pt]
Vivre pour les autres \\[0pt]
Vivre sans loi \\[0pt]
Vivre sans mémoire, est-ce être libre ? \\[0pt]
Vivre sans morale \\[0pt]
Vivre sans religion, est-ce vivre sans espoir ? \\[0pt]
Vivre sa vie \\[0pt]
Vivre selon la nature \\[0pt]
Vivre ses désirs \\[0pt]
Vivre sous la conduite de la raison \\[0pt]
Vivre vertueusement \\[0pt]
Voir \\[0pt]
Voir et concevoir \\[0pt]
Voir et entendre \\[0pt]
Voir et observer \\[0pt]
Voir et savoir \\[0pt]
Voir et toucher \\[0pt]
Voir la réalité en face \\[0pt]
Voir le bien, faire le mal \\[0pt]
Voir le meilleur et faire le pire \\[0pt]
Voir le meilleur, faire le pire \\[0pt]
Voir les choses comme elles sont \\[0pt]
Voir, observer, penser \\[0pt]
Voir un tableau \\[0pt]
Voit-on ce qu'on croit ? \\[0pt]
Volonté et désir \\[0pt]
Vouloir ce que l'on peut \\[0pt]
Vouloir croire, est-ce possible ? \\[0pt]
Vouloir dire \\[0pt]
Vouloir, est-ce encore désirer ? \\[0pt]
Vouloir et pouvoir \\[0pt]
Vouloir être heureux \\[0pt]
Vouloir être immortel \\[0pt]
Vouloir la paix \\[0pt]
Vouloir la paix sociale peut-il aller jusqu'à accepter l'injustice ? \\[0pt]
Vouloir la solitude \\[0pt]
Vouloir le bien \\[0pt]
Vouloir l'égalité \\[0pt]
Vouloir le mal \\[0pt]
Vouloir l'impossible \\[0pt]
Vouloir oublier \\[0pt]
Vouloir vivre \\[0pt]
Voulons-nous la vérité ? \\[0pt]
Voyager \\[0pt]
Voyager dans le temps \\[0pt]
Voyager entre les mondes \\[0pt]
Vulgariser la science ? \\[0pt]
Y a-t-il à la question « qu'est-ce que l'homme ? » une réponse scientifique ? \\[0pt]
Y a-t-il autant de mondes que de cultures ? \\[0pt]
Y a-t-il continuité entre l'expérience et la science ? \\[0pt]
Y a-t-il continuité ou discontinuité entre la pensée mythique et la science ? \\[0pt]
Y a-t-il d'autres moyens que la démonstration pour établir la vérité ? \\[0pt]
Y a-t-il de bons et de mauvais désirs ? \\[0pt]
Y a-t-il de bons préjugés ? \\[0pt]
Y a-t-il de fausses religions ? \\[0pt]
Y a-t-il de fausses sciences ? \\[0pt]
Y a-t-il de faux besoins ? \\[0pt]
Y a-t-il de faux problèmes ? \\[0pt]
Y a-t-il de justes inégalités ? \\[0pt]
Y a-t-il de la fatalité dans la vie de l'homme ? \\[0pt]
Y a-t-il de la grandeur à être libre ? \\[0pt]
Y a-t-il de la raison dans la perception ? \\[0pt]
Y a-t-il de l'impensable ? \\[0pt]
Y a-t-il de l'incommunicable ? \\[0pt]
Y a-t-il de l'incompréhensible ? \\[0pt]
Y a-t-il de l'inconcevable ? \\[0pt]
Y a-t-il de l'inconnaissable ? \\[0pt]
Y a-t-il de l'indémontrable ? \\[0pt]
Y a-t-il de l'indésirable ? \\[0pt]
Y a-t-il de l'indicible ? \\[0pt]
Y a-t-il de l'inexprimable ? \\[0pt]
Y a-t-il de l'intelligible dans l'art ? \\[0pt]
Y a-t-il de l'irréductible ? \\[0pt]
Y a-t-il de l'irréfutable ? \\[0pt]
Y a-t-il de l'irréparable ? \\[0pt]
Y a-t-il de l'universel ? \\[0pt]
Y a-t-il de mauvais désirs ? \\[0pt]
Y a-t-il de mauvaises raisons ? \\[0pt]
Y a-t-il de mauvais spectateurs ? \\[0pt]
Y a-t-il des acquis définitifs en science ? \\[0pt]
Y a-t-il des actes de pensée ? \\[0pt]
Y a-t-il des actes désintéressés ? \\[0pt]
Y a-t-il des actes gratuits ? \\[0pt]
Y a-t-il des actes intrinsèquement mauvais ? \\[0pt]
Y a-t-il des actes moralement indifférents ? \\[0pt]
Y a-t-il des actions collectives ? \\[0pt]
Y a-t-il des actions désintéressées ? \\[0pt]
Y a-t-il des actions libres ? \\[0pt]
Y a-t-il des aliénations heureuses ? \\[0pt]
Y a-t-il des arts du corps ? \\[0pt]
Y a-t-il des arts majeurs ? \\[0pt]
Y a-t-il des arts mineurs ? \\[0pt]
Y a-t-il des barbares ? \\[0pt]
Y a-t-il des biens communs ? \\[0pt]
Y a-t-il des biens inestimables ? \\[0pt]
Y a-t-il des canons de la beauté ? \\[0pt]
Y a-t-il des certitudes historiques ? \\[0pt]
Y a-t-il des certitudes morales ? \\[0pt]
Y a-t-il des choses à savoir ? \\[0pt]
Y a-t-il des choses dont on ne peut parler ? \\[0pt]
Y a-t-il des choses qui échappent au droit ? \\[0pt]
Y a-t-il des choses qu'on n'échange pas ? \\[0pt]
Y a-t-il des colères justes ? \\[0pt]
Y a-t-il des compétences politiques ? \\[0pt]
Y a-t-il des conflits de devoirs ? \\[0pt]
Y a-t-il des connaissances dangereuses ? \\[0pt]
Y a-t-il des connaissances désintéressées ? \\[0pt]
Y a-t-il des contraintes légitimes ? \\[0pt]
Y a-t-il des convictions philosophiques ? \\[0pt]
Y a-t-il des correspondances entre les arts ? \\[0pt]
Y a-t-il des critères de la scientificité ? \\[0pt]
Y a-t-il des critères de l'humanité ? \\[0pt]
Y a-t-il des critères du beau ? \\[0pt]
Y a-t-il des critères du goût ? \\[0pt]
Y a-t-il des croyances démocratiques ? \\[0pt]
Y a-t-il des croyances nécessaires ? \\[0pt]
Y a-t-il des croyances raisonnables ? \\[0pt]
Y a-t-il des croyances rationnelles ? \\[0pt]
Y a-t-il des degrés dans la certitude ? \\[0pt]
Y a-t-il des degrés de conscience ? \\[0pt]
Y a-t-il des degrés de liberté ? \\[0pt]
Y a-t-il des degrés de réalité ? \\[0pt]
Y a-t-il des degrés de vérité ? \\[0pt]
Y a-t-il des démonstrations en morale ? \\[0pt]
Y a-t-il des démonstrations en philosophie ? \\[0pt]
Y a-t-il des désirs moraux ? \\[0pt]
Y a-t-il des désordres naturels ? \\[0pt]
Y a-t-il des despotes éclairés ? \\[0pt]
Y a-t-il des déterminismes sociaux ? \\[0pt]
Y a-t-il des devoirs envers soi ? \\[0pt]
Y a-t-il des devoirs envers soi-même ? \\[0pt]
Y a-t-il des dilemmes moraux ? \\[0pt]
Y a-t-il des droits naturels ? \\[0pt]
Y a-t-il des droits sans devoirs ? \\[0pt]
Y a-t-il des erreurs de la nature ? \\[0pt]
Y a-t-il des erreurs en politique ? \\[0pt]
Y a-t-il des êtres de raison ? \\[0pt]
Y a-t-il des êtres mathématiques ? \\[0pt]
Y a-t-il des évidences morales ? \\[0pt]
Y a-t-il des excès en art ? \\[0pt]
Y a-t-il des expériences absolument certaines ? \\[0pt]
Y a-t-il des expériences cruciales ? \\[0pt]
Y a-t-il des expériences de la liberté ? \\[0pt]
Y a-t-il des expériences métaphysiques ? \\[0pt]
Y a-t-il des expériences sans théorie ? \\[0pt]
Y a-t-il des facultés dans l'esprit ? \\[0pt]
Y a-t-il des faits moraux ? \\[0pt]
Y a-t-il des faits religieux ? \\[0pt]
Y a-t-il des faits sans essence ? \\[0pt]
Y a-t-il des faits scientifiques ? \\[0pt]
Y a-t-il des faux problèmes ? \\[0pt]
Y a-t-il des fins dans la nature ? \\[0pt]
Y a-t-il des fins de la nature ? \\[0pt]
Y a-t-il des fins dernières ? \\[0pt]
Y a-t-il des fondements naturels à l'ordre social ? \\[0pt]
Y a-t-il des formes élémentaires de la société ? \\[0pt]
Y a-t-il des genres de plaisir ? \\[0pt]
Y a-t-il des genres du plaisir ? \\[0pt]
Y a-t-il des guerres injustes ? \\[0pt]
Y a-t-il des guerres justes ? \\[0pt]
Y a-t-il des héritages philosophiques ? \\[0pt]
Y a-t-il des idées générales ? \\[0pt]
Y a-t-il des idées innées ? \\[0pt]
Y a-t-il des illusions de la conscience ? \\[0pt]
Y a-t-il des illusions de la liberté ? \\[0pt]
Y a-t-il des illusions nécessaires ? \\[0pt]
Y a-t-il des inégalités justes ? \\[0pt]
Y a-t-il des injustices naturelles ? \\[0pt]
Y a-t-il des instincts propres à l'Homme ? \\[0pt]
Y a-t-il des interprétations fausses ? \\[0pt]
Y a-t-il des intuitions morales ? \\[0pt]
Y a-t-il des jours où je n'existe pas ? \\[0pt]
Y a-t-il des leçons de l'histoire ? \\[0pt]
Y a-t-il des liens qui libèrent ? \\[0pt]
Y a-t-il des limites à la connaissance ? \\[0pt]
Y a-t-il des limites à la connaissance de l'homme par les sciences ? \\[0pt]
Y a-t-il des limites à la conscience ? \\[0pt]
Y a-t-il des limites à la critique ? \\[0pt]
Y a-t-il des limites à la description ? \\[0pt]
Y a-t-il des limites à la pensée ? \\[0pt]
Y a-t-il des limites à la responsabilité ? \\[0pt]
Y a-t-il des limites à la tolérance ? \\[0pt]
Y a-t-il des limites à l'exprimable ? \\[0pt]
Y a-t-il des limites au droit ? \\[0pt]
Y a-t-il des limites au pouvoir de la technique ? \\[0pt]
Y a-t-il des limites proprement morales à la discussion ? \\[0pt]
Y a-t-il des lois de la pensée ? \\[0pt]
Y a-t-il des lois de l'histoire ? \\[0pt]
Y a-t-il des lois de l'Histoire ? \\[0pt]
Y a-t-il des lois du comportement humain ? \\[0pt]
Y a-t-il des lois du hasard ? \\[0pt]
Y a-t-il des lois du social ? \\[0pt]
Y a-t-il des lois du vivant ? \\[0pt]
Y a-t-il des lois en histoire ? \\[0pt]
Y a-t-il des lois injustes ? \\[0pt]
Y a-t-il des lois morales ? \\[0pt]
Y a-t-il des lois non écrites ? \\[0pt]
Y a-t-il des lois sociales ? \\[0pt]
Y a-t-il des mécanismes humains ? \\[0pt]
Y a-t-il des mentalités collectives ? \\[0pt]
Y a-t-il des méthodes pour être heureux ? \\[0pt]
Y a-t-il des modèles en morale ? \\[0pt]
Y a-t-il des mondes imaginaires ? \\[0pt]
Y a-t-il des mots vides de sens ? \\[0pt]
Y a-t-il des normes de la vie ? \\[0pt]
Y a-t-il des normes naturelles ? \\[0pt]
Y a-t-il des notions communes ? \\[0pt]
Y a-t-il des objets qui n'existent pas ? \\[0pt]
Y a-t-il des obstacles à la connaissance du vivant ? \\[0pt]
Y a-t-il de sots métiers ? \\[0pt]
Y a-t-il des passions collectives ? \\[0pt]
Y a-t-il des passions intraitables ? \\[0pt]
Y a-t-il des passions raisonnables ? \\[0pt]
Y a-t-il des pathologies du social ? \\[0pt]
Y a-t-il des pathologies sociales ? \\[0pt]
Y a-t-il des pensées folles ? \\[0pt]
Y a-t-il des pensées inconscientes ? \\[0pt]
Y a-t-il des perceptions insensibles ? \\[0pt]
Y a-t-il des petites vertus ? \\[0pt]
Y a-t-il des peuples sans histoire ? \\[0pt]
Y a-t-il des phénomènes psychiques ? \\[0pt]
Y a-t-il des plaisirs innocents ? \\[0pt]
Y a-t-il des plaisirs mauvais ? \\[0pt]
Y a-t-il des plaisirs meilleurs que d'autres ? \\[0pt]
Y a-t-il des plaisirs purs ? \\[0pt]
Y a-t-il des plaisirs simples ? \\[0pt]
Y a-t-il des points de vue en morale ? \\[0pt]
Y a-t-il des preuves d'amour ? \\[0pt]
Y a-t-il des preuves de la liberté ? \\[0pt]
Y a-t-il des preuves de la non-existence de Dieu ? \\[0pt]
Y a-t-il des preuves de l'existence de Dieu ? \\[0pt]
Y a-t-il des preuves en métaphysique ? \\[0pt]
Y a-t-il des preuves en philosophie ? \\[0pt]
Y a-t-il des principes de justice universels ? \\[0pt]
Y a-t-il des progrès dans l'art ? \\[0pt]
Y a-t-il des progrès en art ? \\[0pt]
Y a-t-il des progrès en philosophie ? \\[0pt]
Y a-t-il des propriétés singulières ? \\[0pt]
Y a-t-il des questions sans réponse ? \\[0pt]
Y a-t-il des questions sans réponses ? \\[0pt]
Y a-t-il des raisons d'aimer ? \\[0pt]
Y a-t-il des raisons de douter de la raison ? \\[0pt]
Y a-t-il des raisons de vivre ? \\[0pt]
Y a-t-il des réalités mathématiques ? \\[0pt]
Y a-t-il des règles de la guerre ? \\[0pt]
Y a-t-il des règles de l'art ? \\[0pt]
Y a-t-il des régressions historiques ? \\[0pt]
Y a-t-il des révolutions dans l'art ? \\[0pt]
Y a-t-il des révolutions dans les sciences ? \\[0pt]
Y a-t-il des révolutions en art ? \\[0pt]
Y a-t-il des révolutions scientifiques ? \\[0pt]
Y a-t-il des savoirs immédiats ? \\[0pt]
Y a-t-il des sciences de l'éducation ? \\[0pt]
Y a-t-il des sciences de l'homme ? \\[0pt]
Y a-t-il des sciences exactes ? \\[0pt]
Y a-t-il des secrets de la nature ? \\[0pt]
Y a-t-il des sentiments moraux ? \\[0pt]
Y a-t-il des signes naturels ? \\[0pt]
Y a-t-il des sociétés archaïques ? \\[0pt]
Y a-t-il des sociétés sans État ? \\[0pt]
Y a-t-il des sociétés sans histoire ? \\[0pt]
Y a-t-il des solutions en politique ? \\[0pt]
Y a-t-il des sots métiers ? \\[0pt]
Y a-t-il des substances incorporelles ? \\[0pt]
Y a-t-il des techniques de pensée ? \\[0pt]
Y a-t-il des techniques du corps ? \\[0pt]
Y a-t-il des techniques pour être heureux ? \\[0pt]
Y a-t-il des tyrannies douces ? \\[0pt]
Y a-t-il des universaux anthropologiques ? \\[0pt]
Y a-t-il des valeurs absolues ? \\[0pt]
Y a-t-il des valeurs naturelles ? \\[0pt]
Y a-t-il des valeurs objectives ? \\[0pt]
Y a-t-il des valeurs propres à la science ? \\[0pt]
Y a-t-il des valeurs universelles ? \\[0pt]
Y a-t-il des vérités de fait ? \\[0pt]
Y a-t-il des vérités définitives ? \\[0pt]
Y a-t-il des vérités en art ? \\[0pt]
Y a-t-il des vérités en politique ? \\[0pt]
Y a-t-il des vérités éternelles ? \\[0pt]
Y a-t-il des vérités indémontrables ? \\[0pt]
Y a-t-il des vérités indiscutables ? \\[0pt]
Y a-t-il des vérités métaphysiques ? \\[0pt]
Y a-t-il des vérités morales ? \\[0pt]
Y a-t-il des vérités philosophiques ? \\[0pt]
Y a-t-il des vérités plus importantes que d'autres ? \\[0pt]
Y a-t-il des vérités qui échappent à la raison ? \\[0pt]
Y a-t-il des vérités religieuses ? \\[0pt]
Y a-t-il des vérités sans preuve ? \\[0pt]
Y a-t-il des vertus mineures ? \\[0pt]
Y a-t-il des violences justifiées ? \\[0pt]
Y a-t-il des violences légitimes ? \\[0pt]
Y a-t-il différentes façons d'exister ? \\[0pt]
Y a-t-il différentes manières de connaître ? \\[0pt]
Y a-t-il du déterminisme dans les sciences humaines ? \\[0pt]
Y a-t-il du hasard objectif ? \\[0pt]
Y a-t-il du non-être ? \\[0pt]
Y a-t-il du nouveau dans l'histoire ? \\[0pt]
Y a-t-il du progrès en politique ? \\[0pt]
Y a-t-il du sacré dans la nature ? \\[0pt]
Y a-t-il du synthétique \emph{a priori} ? \\[0pt]
Y a-t-il du vrai dans le beau ? \\[0pt]
Y a-t-il encore des mythologies ? \\[0pt]
Y a-t-il encore une sphère privée ? \\[0pt]
Y a-t-il incompatibilité entre science et religion ? \\[0pt]
Y a-t-il lieu de distinguer entre éthique et morale ? \\[0pt]
Y a-t-il lieu de distinguer instinct et pulsion ? \\[0pt]
Y a-t-il lieu de distinguer le don et l'échange ? \\[0pt]
Y a-t-il lieu d'opposer la philosophie et les sciences humaines ? \\[0pt]
Y a-t-il lieu d'opposer matière et esprit ? \\[0pt]
Y a-t-il nécessairement du religieux dans l'art ? \\[0pt]
Y a-t-il place pour l'idée de vérité en morale ? \\[0pt]
Y a-t-il plusieurs libertés ? \\[0pt]
Y a-t-il plusieurs manières de définir ? \\[0pt]
Y a-t-il plusieurs métaphysiques ? \\[0pt]
Y a-t-il plusieurs morales ? \\[0pt]
Y a-t-il plusieurs nécessités ? \\[0pt]
Y a-t-il plusieurs sens du mot vie ? \\[0pt]
Y a-t-il plusieurs sortes de matières ? \\[0pt]
Y a-t-il plusieurs sortes de vérité ? \\[0pt]
Y a-t-il plusieurs temps ? \\[0pt]
Y a-t-il plusieurs vérités ? \\[0pt]
Y a-t-il progrès en art ? \\[0pt]
Y a-t-il quelque chose de commun aux œuvres d'art ? \\[0pt]
Y a-t-il quelque chose de subversif dans les sciences humaines ? \\[0pt]
Y a-t-il quoi que ce soit de nouveau dans l'histoire ? \\[0pt]
Y a-t-il sens à vouloir justifier le mal ? \\[0pt]
Y a-t-il trop d'images ? \\[0pt]
Y a-t-il un art de gouverner ? \\[0pt]
Y a-t-il un art de penser ? \\[0pt]
Y a-t-il un art de persuader ? \\[0pt]
Y a-t-il un art d'être heureux ? \\[0pt]
Y a-t-il un art de vivre ? \\[0pt]
Y a-t-il un art d'interpréter ? \\[0pt]
Y a-t-il un art d'inventer ? \\[0pt]
Y a-t-il un art du bonheur ? \\[0pt]
Y a-t-il un art populaire ? \\[0pt]
Y a-t-il un art total ? \\[0pt]
Y a-t-il un au-delà de la vérité ? \\[0pt]
Y a-t-il un au-delà du langage ? \\[0pt]
Y a-t-il un auteur de l'histoire ? \\[0pt]
Y a-t-il un autre monde ? \\[0pt]
Y a-t-il un beau idéal ? \\[0pt]
Y a-t-il un beau naturel ? \\[0pt]
Y a-t-il un besoin métaphysique ? \\[0pt]
Y a-t-il un bien commun ? \\[0pt]
Y a-t-il un bien plus précieux que la paix ? \\[0pt]
Y a-t-il un Bien suprême ? \\[0pt]
Y a-t-il un bonheur sans illusion ? \\[0pt]
Y a-t-il un bon usage de la rhétorique ? \\[0pt]
Y a-t-il un bon usage des passions ? \\[0pt]
Y a-t-il un bon usage du temps ? \\[0pt]
Y a-t-il un canon de la beauté ? \\[0pt]
Y a-t-il un commencement à tout ? \\[0pt]
Y a-t-il un critère de vérité ? \\[0pt]
Y a-t-il un critère du vrai ? \\[0pt]
Y a-t-il un désir de connaître ? \\[0pt]
Y a-t-il un déterminisme dans les sciences humaines ? \\[0pt]
Y a-t-il un devoir de désobéissance ? \\[0pt]
Y a-t-il un devoir d'émancipation ? \\[0pt]
Y a-t-il un devoir de mémoire ? \\[0pt]
Y a-t-il un devoir de travailler ? \\[0pt]
Y a-t-il un devoir d'être heureux ? \\[0pt]
Y a-t-il un devoir de vérité ? \\[0pt]
Y a-t-il un devoir d'indignation ? \\[0pt]
Y a-t-il un devoir d'oubli ? \\[0pt]
Y a-t-il un différend entre poésie et philosophie ? \\[0pt]
Y a-t-il un droit à la différence ? \\[0pt]
Y a-t-il un droit à la propriété ? \\[0pt]
Y a-t-il un droit au bonheur ? \\[0pt]
Y a-t-il un droit au travail ? \\[0pt]
Y a-t-il un droit de désobéissance ? \\[0pt]
Y a-t-il un droit de la guerre ? \\[0pt]
Y a-t-il un droit de mentir ? \\[0pt]
Y a-t-il un droit de mourir ? \\[0pt]
Y a-t-il un droit de résistance ? \\[0pt]
Y a-t-il un droit de révolte ? \\[0pt]
Y a-t-il un droit des peuples ? \\[0pt]
Y a-t-il un droit d'ingérence ? \\[0pt]
Y a-t-il un droit du plus faible ? \\[0pt]
Y a-t-il un droit du plus fort ? \\[0pt]
Y a-t-il un droit international ? \\[0pt]
Y a-t-il un droit naturel ? \\[0pt]
Y a-t-il un droit universel au mariage ? \\[0pt]
Y a-t-il une argumentation métaphysique ? \\[0pt]
Y a-t-il une beauté morale ? \\[0pt]
Y a-t-il une beauté naturelle ? \\[0pt]
Y a-t-il une beauté propre à la photographie ? \\[0pt]
Y a-t-il une beauté propre à l'objet technique ? \\[0pt]
Y a-t-il une bonne éducation ? \\[0pt]
Y a-t-il une bonne imitation ? \\[0pt]
Y a-t-il une causalité empirique ? \\[0pt]
Y a-t-il une causalité en histoire ? \\[0pt]
Y a-t-il une causalité historique ? \\[0pt]
Y a-t-il une cause première ? \\[0pt]
Y a-t-il une compétence en politique ? \\[0pt]
Y a-t-il une compétence politique ? \\[0pt]
Y a-t-il une condition humaine ? \\[0pt]
Y a-t-il une connaissance du probable ? \\[0pt]
Y a-t-il une connaissance du singulier ? \\[0pt]
Y a-t-il une connaissance historique ? \\[0pt]
Y a-t-il une connaissance logique ? \\[0pt]
Y a-t-il une connaissance métaphysique ? \\[0pt]
Y a-t-il une connaissance sensible ? \\[0pt]
Y a-t-il une conscience collective ? \\[0pt]
Y a-t-il une correspondance des arts ? \\[0pt]
Y a-t-il une définition du bonheur ? \\[0pt]
Y a-t-il une dimension métaphysique du désir ? \\[0pt]
Y a-t-il une éducation du goût ? \\[0pt]
Y a-t-il une enfance de l'humanité ? \\[0pt]
Y a-t-il une esthétique de la laideur ? \\[0pt]
Y a-t-il une esthétique du laid ? \\[0pt]
Y a-t-il une esthétique industrielle ? \\[0pt]
Y a-t-il une éthique de la technique ? \\[0pt]
Y a-t-il une éthique de l'authenticité ? \\[0pt]
Y a-t-il une éthique des moyens ? \\[0pt]
Y a-t-il une expérience de la liberté ? \\[0pt]
Y a-t-il une expérience de l'éternité ? \\[0pt]
Y a-t-il une expérience du néant ? \\[0pt]
Y a-t-il une expérience du temps ? \\[0pt]
Y a-t-il une expérience métaphysique ? \\[0pt]
Y a-t-il une finalité dans la nature ? \\[0pt]
Y a-t-il une fin de l'histoire ? \\[0pt]
Y a-t-il une fin dernière ? \\[0pt]
Y a-t-il une fonction propre à l'œuvre d'art ? \\[0pt]
Y a-t-il une force des faibles ? \\[0pt]
Y a-t-il une force du droit ? \\[0pt]
Y a-t-il une formation de la pensée logique ? \\[0pt]
Y a-t-il une forme morale de fanatisme ? \\[0pt]
Y a-t-il une hiérarchie des devoirs ? \\[0pt]
Y a-t-il une hiérarchie des êtres ? \\[0pt]
Y a-t-il une hiérarchie des sciences ? \\[0pt]
Y a-t-il une hiérarchie des sciences sociales ? \\[0pt]
Y a-t-il une hiérarchie du vivant ? \\[0pt]
Y a-t-il une histoire de la folie ? \\[0pt]
Y a-t-il une histoire de la nature ? \\[0pt]
Y a-t-il une histoire de la raison ? \\[0pt]
Y a-t-il une histoire de l'art ? \\[0pt]
Y a-t-il une histoire de la vérité ? \\[0pt]
Y a-t-il une histoire de la vie quotidienne ? \\[0pt]
Y a-t-il une histoire universelle ? \\[0pt]
Y a-t-il une intelligence du corps ? \\[0pt]
Y a-t-il une intentionnalité collective ? \\[0pt]
Y a-t-il une irréversibilité du temps ? \\[0pt]
Y a-t-il une justice naturelle ? \\[0pt]
Y a-t-il une justice sans morale ? \\[0pt]
Y a-t-il une langue de la philosophie ? \\[0pt]
Y a-t-il une limite à la connaissance du vivant ? \\[0pt]
Y a-t-il une limite au désir ? \\[0pt]
Y a-t-il une limite au développement scientifique ? \\[0pt]
Y a-t-il une logique dans l'histoire ? \\[0pt]
Y a-t-il une logique de la découverte ? \\[0pt]
Y a-t-il une logique de la découverte scientifique ? \\[0pt]
Y a-t-il une logique de la déraison ? \\[0pt]
Y a-t-il une logique de l'art ? \\[0pt]
Y a-t-il une logique de l'inconscient ? \\[0pt]
Y a-t-il une logique des événements historiques ? \\[0pt]
Y a-t-il une logique du désir ? \\[0pt]
Y a-t-il une logique du mythe ? \\[0pt]
Y a-t-il une loi suprême ? \\[0pt]
Y a-t-il une mathématique universelle ? \\[0pt]
Y a-t-il une mécanique des passions ? \\[0pt]
Y a-t-il une médecine de l'âme ? \\[0pt]
Y a-t-il une métaphysique de l'amour ? \\[0pt]
Y a-t-il une méthode de l'interprétation ? \\[0pt]
Y a-t-il une méthode propre aux sciences humaines ? \\[0pt]
Y a-t-il une morale universelle ? \\[0pt]
Y a-t-il un empire de la technique ? \\[0pt]
Y a-t-il une nature humaine ? \\[0pt]
Y a-t-il une nécessité de l'erreur ? \\[0pt]
Y a-t-il une nécessité de l'Histoire ? \\[0pt]
Y a-t-il une nécessité morale ? \\[0pt]
Y a-t-il une œuvre du temps ? \\[0pt]
Y a-t-il une opinion publique mondiale ? \\[0pt]
Y a-t-il une ou des morales ? \\[0pt]
Y a-t-il une ou plusieurs philosophies ? \\[0pt]
Y a-t-il une pensée sans signes ? \\[0pt]
Y a-t-il une pensée technique ? \\[0pt]
Y a-t-il une perception esthétique ? \\[0pt]
Y a-t-il une philosophie de la nature ? \\[0pt]
Y a-t-il une philosophie de la philosophie ? \\[0pt]
Y a-t-il une philosophie première ? \\[0pt]
Y a-t-il une place dans les sciences humaines pour la catégorie de personne ? \\[0pt]
Y a-t-il une place pour la morale dans l'économie ? \\[0pt]
Y a-t-il une positivité de l'erreur ? \\[0pt]
Y a-t-il une présence du passé ? \\[0pt]
Y a-t-il une primauté du devoir sur le droit ? \\[0pt]
Y a-t-il une psychologie animale ? \\[0pt]
Y a-t-il une raison à tout ? \\[0pt]
Y a-t-il une raison des choses ? \\[0pt]
Y a-t-il une rationalité dans la religion ? \\[0pt]
Y a-t-il une rationalité de la contingence ? \\[0pt]
Y a-t-il une rationalité des sentiments ? \\[0pt]
Y a-t-il une rationalité du hasard ? \\[0pt]
Y a-t-il une rationalité du marché ? \\[0pt]
Y a-t-il une rationalité propre de l'intérêt ? \\[0pt]
Y a-t-il une réalité des idées ? \\[0pt]
Y a-t-il une réalité du hasard ? \\[0pt]
Y a-t-il une réalité du mal ? \\[0pt]
Y a-t-il une religion civile ? \\[0pt]
Y a-t-il une responsabilité de l'artiste ? \\[0pt]
Y a-t-il une sagesse de l'inconscient ? \\[0pt]
Y a-t-il une sagesse populaire ? \\[0pt]
Y a-t-il une science de la vie ? \\[0pt]
Y a-t-il une science de la vie mentale ? \\[0pt]
Y a-t-il une science de l'esprit ? \\[0pt]
Y a-t-il une science de l'être ? \\[0pt]
Y a-t-il une science de l'homme ? \\[0pt]
Y a-t-il une science de l'individuel ? \\[0pt]
Y a-t-il une science des principes ? \\[0pt]
Y a-t-il une science du juste ? \\[0pt]
Y a-t-il une science du moi ? \\[0pt]
Y a-t-il une science du qualitatif ? \\[0pt]
Y a-t-il une science ou des sciences ? \\[0pt]
Y a-t-il une science politique ? \\[0pt]
Y a-t-il une sensibilité esthétique ? \\[0pt]
Y a-t-il une servitude volontaire ? \\[0pt]
Y a-t-il une singularité de l'histoire de l'art ? \\[0pt]
Y a-t-il une spécificité de la délibération politique ? \\[0pt]
Y a-t-il une spécificité des sciences humaines ? \\[0pt]
Y a-t-il une spécificité du vivant ? \\[0pt]
Y a-t-il un esprit scientifique ? \\[0pt]
Y a-t-il un État idéal ? \\[0pt]
Y a-t-il une technique de la nature ? \\[0pt]
Y a-t-il une technique pour tout ? \\[0pt]
Y a-t-il une temporalité proprement morale ? \\[0pt]
Y a-t-il une tyrannie du vrai ? \\[0pt]
Y a-t-il une unité de la science ? \\[0pt]
Y a-t-il une unité des devoirs ? \\[0pt]
Y a-t-il une unité des langages humains ? \\[0pt]
Y a-t-il une unité des sciences ? \\[0pt]
Y a-t-il une unité en psychologie ? \\[0pt]
Y a-t-il une universalité des mathématiques ? \\[0pt]
Y a-t-il une universalité du beau ? \\[0pt]
Y a-t-il une utilité des lieux communs ? \\[0pt]
Y a-t-il une valeur de l'inutile ? \\[0pt]
Y a-t-il une vérité absolue ? \\[0pt]
Y a-t-il une vérité dans les arts ? \\[0pt]
Y a-t-il une vérité de la fiction ? \\[0pt]
Y a-t-il une vérité de l'inconscient ? \\[0pt]
Y a-t-il une vérité de l'œuvre d'art ? \\[0pt]
Y a-t-il une vérité des apparences ? \\[0pt]
Y a-t-il une vérité des représentations ? \\[0pt]
Y a-t-il une vérité des sentiments ? \\[0pt]
Y a-t-il une vérité des symboles ? \\[0pt]
Y a-t-il une vérité du sensible ? \\[0pt]
Y a-t-il une vérité du sentiment ? \\[0pt]
Y a-t-il une vérité en art ? \\[0pt]
Y a-t-il une vérité en histoire ? \\[0pt]
Y a-t-il une vérité philosophique ? \\[0pt]
Y a-t-il une vérité sur le bonheur ? \\[0pt]
Y a-t-il une vertu de l'ignorance \\[0pt]
Y a-t-il une vertu de l'imitation ? \\[0pt]
Y a-t-il une vertu de l'oubli ? \\[0pt]
Y a-t-il une vie de l'esprit ? \\[0pt]
Y a-t-il une violence du droit ? \\[0pt]
Y a-t-il une vision du monde propre à la pensée technique ? \\[0pt]
Y a-t-il une volonté du mal ? \\[0pt]
Y a-t-il un fondement de la croyance ? \\[0pt]
Y a-t-il un fondement moral aux circonstances atténuantes ? \\[0pt]
Y a-t-il un inconscient collectif ? \\[0pt]
Y a-t-il un inconscient psychique ? \\[0pt]
Y a-t-il un inconscient social ? \\[0pt]
Y a t-il un instinct moral ? \\[0pt]
Y a-t-il un intérêt à faire son devoir ? \\[0pt]
Y a-t-il un jugement de l'histoire ? \\[0pt]
Y a-t-il un langage animal ? \\[0pt]
Y a-t-il un langage commun ? \\[0pt]
Y a-t-il un langage de la musique ? \\[0pt]
Y a-t-il un langage de l'art ? \\[0pt]
Y a-t-il un langage de l'inconscient ? \\[0pt]
Y a-t-il un langage des animaux ? \\[0pt]
Y a-t-il un langage du corps ? \\[0pt]
Y a-t-il un langage unifié de la science ? \\[0pt]
Y a-t-il un langage universel ? \\[0pt]
Y a-t-il un mal absolu ? \\[0pt]
Y a-t-il un monde de l'art ? \\[0pt]
Y a-t-il un monde extérieur ? \\[0pt]
Y a-t-il un monde technique ? \\[0pt]
Y a-t-il un moteur de l'histoire ? \\[0pt]
Y a-t-il un objet du désir ? \\[0pt]
Y a-t-il un ordre dans la nature ? \\[0pt]
Y a-t-il un ordre des choses ? \\[0pt]
Y a-t-il un ordre du monde ? \\[0pt]
Y a-t-il un pouvoir de la technique ? \\[0pt]
Y a-t-il un primat de la nature sur la culture ? \\[0pt]
Y a-t-il un principe du mal ? \\[0pt]
Y a-t-il un progrès dans l'art ? \\[0pt]
Y a-t-il un progrès des/en sciences humaines ? \\[0pt]
Y a-t-il un progrès du droit ? \\[0pt]
Y a-t-il un progrès en art ? \\[0pt]
Y a-t-il un progrès en philosophie ? \\[0pt]
Y a-t-il un progrès moral ? \\[0pt]
Y a-t-il un propre de l'homme ? \\[0pt]
Y a-t-il un rapport moral à soi-même ? \\[0pt]
Y a-t-il un rythme de l'histoire ? \\[0pt]
Y a-t-il un savoir de la justice ? \\[0pt]
Y a-t-il un savoir du bien ? \\[0pt]
Y a-t-il un savoir du contingent ? \\[0pt]
Y a-t-il un savoir du corps ? \\[0pt]
Y a-t-il un savoir du juste ? \\[0pt]
Y a-t-il un savoir du politique ? \\[0pt]
Y a-t-il un savoir immédiat ? \\[0pt]
Y a-t-il un savoir inconscient ? \\[0pt]
Y a-t-il un savoir politique ? \\[0pt]
Y a-t-il un savoir pratique \\[0pt]
Y a-t-il un sens à dire que Dieu comprend, veut, fait ? \\[0pt]
Y a-t-il un sens à instituer une langue universelle ? \\[0pt]
Y a-t-il un sens à ne plus rien désirer ? \\[0pt]
Y a-t-il un sens à penser un droit des générations futures ? \\[0pt]
Y a-t-il un sens à se dire « citoyen du monde » ? \\[0pt]
Y a-t-il un sens à se dire citoyen du monde ? \\[0pt]
Y a-t-il un sens à s'opposer à la technique ? \\[0pt]
Y a-t-il un sens du beau ? \\[0pt]
Y a-t-il un sens moral ? \\[0pt]
Y a-t-il un sentiment métaphysique ? \\[0pt]
Y a-t-il un sentiment naturel du beau ? \\[0pt]
Y a-t-il un souverain bien ? \\[0pt]
Y a-t-il un temps des choses ? \\[0pt]
Y a-t-il un temps pour l'action ? \\[0pt]
Y a-t-il un temps pour tout ? \\[0pt]
Y a-t-il un traitement scientifique du langage ? \\[0pt]
Y a-t-il un travail de la pensée ? \\[0pt]
Y a-t-il un tribunal de l'histoire ? \\[0pt]
Y a-t-il un usage moral des passions ? \\[0pt]
Y a-t-il un usage purement instrumental de la raison ? \\[0pt]
Y a-t-il vérité sans interprétation ? \\[0pt]
Y a-t-une science du comportement ? \\[0pt]
Y aura-t-il toujours des religions ? \\[0pt]

\section{Tri par concours}
\label{sec:org6b89ce5}

\subsection{Agrégation}
\label{sec:orgf9d8128}

\noindent
Abjection et sainteté \\[0pt]
Abolir la propriété \\[0pt]
Absolu / relatif \\[0pt]
Abstraire \\[0pt]
Accepter l'injustice \\[0pt]
Accuser et défendre \\[0pt]
À chacun sa morale \\[0pt]
À chacun ses goûts \\[0pt]
À chacun son dû \\[0pt]
Acteurs sociaux et usages sociaux \\[0pt]
Action et événement \\[0pt]
Action et fabrication \\[0pt]
Action et production \\[0pt]
Actions et pratiques \\[0pt]
Admettre le hasard, est-ce nier l'ordre de la nature ? \\[0pt]
Admettre une cause première, est-ce faire une pétition de principe ? \\[0pt]
Affirmation et négation \\[0pt]
Affirmer et nier \\[0pt]
Agir \\[0pt]
Agir en conscience \\[0pt]
Agir justement fait-il de moi un homme juste ? \\[0pt]
Agir librement \\[0pt]
Agir moralement, est-ce lutter contre ses idées ? \\[0pt]
Agir moralement, est-ce lutter contre soi-même ? \\[0pt]
Agir par devoir, est-ce évaluer les conséquences de ses actes \\[0pt]
Agir par devoir, est-ce renoncer à ses intérêts ? \\[0pt]
Agir sans raison \\[0pt]
Ai-je une âme ? \\[0pt]
« Aime, et fais ce que tu veux » \\[0pt]
Aimer la nature \\[0pt]
Aimer la vérité \\[0pt]
Aimer la vie \\[0pt]
Aimer les lois \\[0pt]
Aimer ses proches \\[0pt]
Aimer son prochain comme soi-même \\[0pt]
Aimer une œuvre d'art \\[0pt]
« Aimez vos ennemis » \\[0pt]
Aliénation et servitude \\[0pt]
« À l'impossible, nul n'est tenu » \\[0pt]
À l'impossible nul n'est tenu \\[0pt]
Aller au-delà des apparences \\[0pt]
Aller au fond des choses \\[0pt]
Aller au vrai \\[0pt]
Altérité, diversité, pluralité \\[0pt]
Âme, conscience, esprit \\[0pt]
Amitié et société \\[0pt]
Amour et respect \\[0pt]
Analogie et ressemblance \\[0pt]
Analyse et synthèse \\[0pt]
Analyser les mœurs \\[0pt]
Animal politique ou social ? \\[0pt]
Anomalie et anomie \\[0pt]
Antérieur / postérieur \\[0pt]
Anthropologie et humanisme \\[0pt]
Anthropologie et ontologie \\[0pt]
Anthropologie et politique \\[0pt]
Apparaître \\[0pt]
Apparence et réalité \\[0pt]
Apparition et manifestation \\[0pt]
Appartenons-nous à une culture ? \\[0pt]
Appliquer la loi \\[0pt]
Apprend-on à être libre ? \\[0pt]
Apprend-on à penser ? \\[0pt]
Apprend-on à percevoir la beauté ? \\[0pt]
Apprend-on à voir ? \\[0pt]
Apprendre \\[0pt]
Apprendre à aimer \\[0pt]
Apprendre à gouverner \\[0pt]
Apprendre à parler \\[0pt]
Apprendre à penser \\[0pt]
Apprendre à vivre \\[0pt]
Apprendre à voir \\[0pt]
Apprendre et devenir \\[0pt]
Apprendre s'apprend-il ? \\[0pt]
Apprentissage et conditionnement \\[0pt]
Approcher du vrai \\[0pt]
Après-coup \\[0pt]
« Après moi, le déluge » \\[0pt]
\emph{A priori} et \emph{a posteriori} \\[0pt]
À quelle généralité peuvent prétendre les sciences humaines ? \\[0pt]
À quelle objectivité peuvent prétendre les sciences sociales ? \\[0pt]
À quelles conditions est-il acceptable de travailler ? \\[0pt]
À quelles conditions penser une seconde nature ? \\[0pt]
À quelles conditions une démarche est-elle scientifique ? \\[0pt]
À quelles conditions une explication est-elle scientifique ? \\[0pt]
À quelles conditions une hypothèse est-elle scientifique ? \\[0pt]
À quelles conditions un énoncé est-il doué de sens ? \\[0pt]
À quelles conditions une pensée est-elle libre ? \\[0pt]
À quelles conditions une vérité s'impose-t-elle à nous ? \\[0pt]
À quelles conditions y a-t-il progrès technique ? \\[0pt]
À quelque chose le malheur peut-il être bon ? \\[0pt]
À quelque chose, le malheur peut-il être bon ? \\[0pt]
À qui appartient-il de décider du juste et de l'injuste ? \\[0pt]
À qui appartient la Terre ? \\[0pt]
À qui dois-je la vérité ? \\[0pt]
À qui la faute ? \\[0pt]
À qui le pouvoir appartient-il ? \\[0pt]
À qui peut-on faire confiance ? \\[0pt]
À qui profite le travail ? \\[0pt]
À qui se fier ? \\[0pt]
À quoi bon ? \\[0pt]
À quoi bon avoir mauvaise conscience ? \\[0pt]
À quoi bon chercher à prouver l'existence de Dieu ? \\[0pt]
À quoi bon critiquer les autres ? \\[0pt]
À quoi bon des philosophes ? \\[0pt]
À quoi bon discuter ? \\[0pt]
À quoi bon imaginer la cité idéale ? \\[0pt]
À quoi bon imiter la nature ? \\[0pt]
À quoi bon la morale ? \\[0pt]
À quoi bon les sciences humaines et sociales ? \\[0pt]
À quoi bon les systèmes ? \\[0pt]
À quoi bon rêver ? \\[0pt]
À quoi bon se parler ? \\[0pt]
À quoi est-il nécessaire d'obéir ? \\[0pt]
À quoi faut-il être fidèle ? \\[0pt]
À quoi faut-il renoncer ? \\[0pt]
À quoi juger l'action d'un gouvernement ? \\[0pt]
À quoi la conscience nous donne-t-elle accès ? \\[0pt]
À quoi la logique peut-elle servir dans les sciences ? \\[0pt]
À quoi l'art est-il bon ? \\[0pt]
À quoi l'art nous rend-il sensibles ? \\[0pt]
À quoi les sciences nous forment-elles ? \\[0pt]
À quoi reconnaît-on la rationalité ? \\[0pt]
À quoi reconnaît-on la vérité ? \\[0pt]
À quoi reconnaît-on l'injustice ? \\[0pt]
À quoi reconnaît-on qu'une politique est juste ? \\[0pt]
À quoi reconnaît-on qu'une théorie est scientifique ? \\[0pt]
À quoi reconnaît-on un bon gouvernement ? \\[0pt]
À quoi reconnaît-on un comportement humain ? \\[0pt]
À quoi reconnaît-on une attitude religieuse ? \\[0pt]
À quoi reconnaît-on une œuvre d'art ? \\[0pt]
À quoi reconnaît-on une science ? \\[0pt]
À quoi rêve-t-on ? \\[0pt]
À quoi sert la bioéthique ? \\[0pt]
À quoi sert la critique ? \\[0pt]
À quoi sert la dialectique ? \\[0pt]
À quoi sert la logique ? \\[0pt]
À quoi sert la métaphysique ? \\[0pt]
À quoi sert la négation ? \\[0pt]
À quoi sert la notion de contrat social ? \\[0pt]
À quoi sert la notion d'état de nature ? \\[0pt]
À quoi sert la technique ? \\[0pt]
À quoi sert la théodicée ? \\[0pt]
À quoi sert la vérité ? \\[0pt]
À quoi sert l'écriture ? \\[0pt]
À quoi sert l'État ? \\[0pt]
À quoi sert l'intuition ? \\[0pt]
À quoi sert l'ontologie ? \\[0pt]
À quoi sert une métaphore ? \\[0pt]
À quoi sert un exemple ? \\[0pt]
À quoi servent les cérémonies ? \\[0pt]
À quoi servent les constitutions ? \\[0pt]
À quoi servent les croyances ? \\[0pt]
À quoi servent les doctrines morales ? \\[0pt]
À quoi servent les élections ? \\[0pt]
À quoi servent les encyclopédies ? \\[0pt]
À quoi servent les hypothèses ? \\[0pt]
À quoi servent les preuves ? \\[0pt]
À quoi servent les règles ? \\[0pt]
À quoi servent les religions ? \\[0pt]
À quoi servent les sciences ? \\[0pt]
À quoi servent les sciences humaines ? \\[0pt]
À quoi servent les statistiques ? \\[0pt]
À quoi servent les utopies ? \\[0pt]
À quoi suis-je obligé ? \\[0pt]
À quoi tient la fermeté du vouloir ? \\[0pt]
À quoi tient la force des religions ? \\[0pt]
À quoi tient la vérité d'une interprétation ? \\[0pt]
À quoi tient l'efficacité d'une technique ? \\[0pt]
À quoi tient notre humanité ? \\[0pt]
Architecture et religion \\[0pt]
Architecture et utopie \\[0pt]
Argumenter \\[0pt]
Argumenter et démontrer \\[0pt]
Arriver à ses fins \\[0pt]
Arrive-t-il que l'impossible se produise ? \\[0pt]
Art et abstraction \\[0pt]
Art et apparence \\[0pt]
Art et artifice \\[0pt]
Art et authenticité \\[0pt]
Art et critique \\[0pt]
Art et décadence \\[0pt]
Art et divertissement \\[0pt]
Art et émotion \\[0pt]
Art et expression \\[0pt]
Art et finitude \\[0pt]
Art et folie \\[0pt]
Art et forme \\[0pt]
Art et illusion \\[0pt]
Art et image \\[0pt]
Art et industrie \\[0pt]
Art et interdit \\[0pt]
Art et jeu \\[0pt]
Art et langage \\[0pt]
Art et marchandise \\[0pt]
Art et mélancolie \\[0pt]
Art et mémoire \\[0pt]
Art et métaphysique \\[0pt]
Art et morale \\[0pt]
Art et politique \\[0pt]
Art et présence \\[0pt]
Art et propagande \\[0pt]
Art et religieux \\[0pt]
Art et religion \\[0pt]
Art et représentation \\[0pt]
Art et technique \\[0pt]
Art et tradition \\[0pt]
Art et transgression \\[0pt]
Art et utilité \\[0pt]
Art et vérité \\[0pt]
Artiste et artisan \\[0pt]
Artistes et ingénieurs \\[0pt]
Art populaire et art savant \\[0pt]
Arts de l'espace et arts du temps \\[0pt]
Arts du temps et arts de l'espace \\[0pt]
À science nouvelle, nouvelle philosophie ? \\[0pt]
A-t-on besoin de spécialistes ? \\[0pt]
A-t-on des droits contre l'État ? \\[0pt]
A-t-on des raisons de croire ? \\[0pt]
A-t-on des raisons de croire ce qu'on croit ? \\[0pt]
A-t-on expliqué un phénomène quand on l'a rapporté à des lois ? \\[0pt]
A-t-on le droit de faire tout ce qui est permis par la loi ? \\[0pt]
A-t-on le droit de se révolter ? \\[0pt]
A-t-on raison de tenir à la démocratie ? \\[0pt]
Attendre \\[0pt]
Attente et espérance \\[0pt]
Au-delà \\[0pt]
Au-delà de la nature ? \\[0pt]
Au nom de quoi le plaisir serait-il condamnable ? \\[0pt]
Au nom du peuple \\[0pt]
Aussitôt dit, aussitôt fait \\[0pt]
Autorité et autoritarisme \\[0pt]
Autorité et pouvoir \\[0pt]
Autrui \\[0pt]
Autrui, est-ce n'importe quel autre ? \\[0pt]
Autrui est-il mon prochain ? \\[0pt]
Avoir \\[0pt]
Avoir bonne conscience \\[0pt]
Avoir de la chance \\[0pt]
Avoir de l'autorité \\[0pt]
Avoir de l'esprit \\[0pt]
Avoir de l'expérience \\[0pt]
Avoir des ennemis \\[0pt]
Avoir des ennemis, est-ce le principe de la politique ? \\[0pt]
Avoir des principes \\[0pt]
Avoir des scrupules \\[0pt]
Avoir des valeurs \\[0pt]
Avoir du caractère \\[0pt]
Avoir du goût \\[0pt]
Avoir du mérite \\[0pt]
Avoir du style \\[0pt]
Avoir la conscience tranquille \\[0pt]
Avoir le droit \\[0pt]
Avoir le sens du devoir \\[0pt]
Avoir le temps \\[0pt]
Avoir mauvaise conscience \\[0pt]
Avoir peur \\[0pt]
Avoir raison \\[0pt]
Avoir raison de ses émotions \\[0pt]
Avoir ses raisons \\[0pt]
Avoir un corps \\[0pt]
Avoir une bonne mémoire \\[0pt]
Avoir une idée \\[0pt]
Avoir un sens \\[0pt]
Avons-nous à apprendre des images ? \\[0pt]
Avons-nous accès aux choses-mêmes ? \\[0pt]
Avons-nous besoin de l'État ? \\[0pt]
Avons-nous besoin de lois ? \\[0pt]
Avons-nous besoin de métaphysique ? \\[0pt]
Avons-nous besoin de méthodes ? \\[0pt]
Avons-nous besoin de partis politiques ? \\[0pt]
Avons-nous besoin de spectacles ? \\[0pt]
Avons-nous besoin de traditions ? \\[0pt]
Avons-nous besoin d'être gouvernés ? \\[0pt]
Avons-nous besoin d'experts en matière d'art ? \\[0pt]
Avons-nous besoin d'idéal \\[0pt]
Avons-nous besoin d'un chef ? \\[0pt]
Avons-nous besoin d'une conception métaphysique du monde ? \\[0pt]
Avons-nous besoin d'une définition de l'art ? \\[0pt]
Avons-nous besoin d'une philosophie politique ? \\[0pt]
Avons-nous besoin d'un libre arbitre ? \\[0pt]
Avons-nous besoin d'utopies ? \\[0pt]
Avons-nous des devoirs envers les animaux ? \\[0pt]
Avons-nous des devoirs envers les générations futures ? \\[0pt]
Avons-nous des devoirs envers les morts ? \\[0pt]
Avons-nous des devoirs envers le vivant ? \\[0pt]
Avons-nous des devoirs envers nous-mêmes ? \\[0pt]
Avons-nous encore besoin de la nature ? \\[0pt]
Avons-nous le devoir d'être heureux ? \\[0pt]
Avons-nous le droit de juger autrui ? \\[0pt]
Avons-nous un accès direct aux phénomènes ? \\[0pt]
Avons-nous une identité ? \\[0pt]
Avons-nous une intuition du temps ? \\[0pt]
Avons-nous une responsabilité envers le passé ? \\[0pt]
Avons-nous un monde commun ? \\[0pt]
Axiomatique et vérité \\[0pt]
Bâtir un monde \\[0pt]
Beauté et vérité \\[0pt]
Beauté naturelle et beauté artistique \\[0pt]
Beauté réelle, beauté idéale \\[0pt]
Bien agir \\[0pt]
Bien commun et bien public \\[0pt]
Bien commun et droits individuels \\[0pt]
Bien faire \\[0pt]
« Bienheureuse faute » \\[0pt]
Bien jouer son rôle \\[0pt]
Bien juger \\[0pt]
Bien parler \\[0pt]
Bien parler, est-ce bien penser ? \\[0pt]
Bien vivre \\[0pt]
Bonheur et autarcie \\[0pt]
Bon sens et sens commun \\[0pt]
Calculer \\[0pt]
Calculer et penser \\[0pt]
Cartographier \\[0pt]
Castes et classes \\[0pt]
Catégories de l'être, catégories de langue \\[0pt]
Catégories de pensée, catégories de langue \\[0pt]
Catégories logiques et catégories linguistiques \\[0pt]
Causalité et finalité \\[0pt]
Cause et loi \\[0pt]
Cause et motivation \\[0pt]
Cause et raison \\[0pt]
Causes et lois \\[0pt]
Causes et motivations \\[0pt]
Causes premières et causes secondes \\[0pt]
Ceci \\[0pt]
Cela a-t-il un sens de penser par soi-même ? \\[0pt]
Ce que la morale autorise, l'État peut-il légitimement l'interdire ? \\[0pt]
Ce que la technique rend possible, peut-on jamais en empêcher la réalisation ? \\[0pt]
Ce que sait le poète \\[0pt]
Ce qui dépend de moi \\[0pt]
Ce qui doit-être, est-ce autre chose que ce qui est ? \\[0pt]
Ce qui est à moi \\[0pt]
Ce qui est contingent peut-il être objet de science ? \\[0pt]
Ce qui est contradictoire peut-il exister ? \\[0pt]
Ce qui est démontré est-il nécessairement vrai ? \\[0pt]
Ce qui est faux est-il dénué de sens ? \\[0pt]
Ce qui est passé est-il dépassé ? \\[0pt]
Ce qui est subjectif est-il arbitraire ? \\[0pt]
Ce qui fut et ce qui sera \\[0pt]
Ce qui importe \\[0pt]
Ce qu'il y a \\[0pt]
Ce qui n'a pas de prix \\[0pt]
Ce qui n'a pas lieu d'être \\[0pt]
Ce qui ne dépend pas de nous \\[0pt]
Ce qui n'est pas \\[0pt]
Ce qui n'est pas démontré peut-il être vrai ? \\[0pt]
Ce qui n'est pas matériel peut-il être réel ? \\[0pt]
Ce qui n'est pas réel est-il impossible ? \\[0pt]
Ce qui n'existe pas \\[0pt]
Ce qui passe et ce qui demeure \\[0pt]
Ce qui subsiste et ce qui change \\[0pt]
Ce qu'on ne peut pas vendre \\[0pt]
Certaines œuvres d'art ont-elles plus de valeur que d'autres ? \\[0pt]
Certitude et pari \\[0pt]
Certitude et vérité \\[0pt]
Cesser d'espérer \\[0pt]
« C'est humain » \\[0pt]
« C'est la vie » \\[0pt]
C'est pour ton bien \\[0pt]
« C'est scientifique » \\[0pt]
« C'est tout un art » \\[0pt]
C'est trop beau pour être vrai ! \\[0pt]
Ceux qui savent doivent-ils gouverner ? \\[0pt]
Changement et mouvement \\[0pt]
Changer \\[0pt]
Changer le monde \\[0pt]
Changer ses désirs plutôt que l'ordre du monde \\[0pt]
Chanter et parler \\[0pt]
Chaos et cosmos \\[0pt]
Chaque science porte-t-elle une métaphysique qui lui est propre ? \\[0pt]
Chercher ses mots \\[0pt]
Chercher son intérêt, est-ce être immoral ? \\[0pt]
Choisir \\[0pt]
Choisir ses souvenirs ? \\[0pt]
Choisit-on ses souvenirs ? \\[0pt]
Choisit-on son corps ? \\[0pt]
Chose et objet \\[0pt]
Choses et personnes \\[0pt]
Cinéma et réalité \\[0pt]
Cinéma et vérité \\[0pt]
Cité juste ou citoyen juste ? \\[0pt]
Citoyen et soldat \\[0pt]
Citoyenneté et solidarité \\[0pt]
Citoyens du monde \\[0pt]
Civilisé, barbare, sauvage \\[0pt]
Classer \\[0pt]
Classer, est-ce penser ? \\[0pt]
Classer et ordonner \\[0pt]
Classes et essences \\[0pt]
Classes et histoire \\[0pt]
Cohérence et vérité \\[0pt]
Collectionner \\[0pt]
Commander \\[0pt]
Comme d'habitude \\[0pt]
Commémorer \\[0pt]
Commencer \\[0pt]
Commencer en philosophie \\[0pt]
Comment assumer les conséquences de ses actes ? \\[0pt]
Comment bien appliquer une règle ? \\[0pt]
Comment bien vivre ? \\[0pt]
Comment caractériser une idée confuse ? \\[0pt]
Comment comprendre une croyance qu'on ne partage pas ? \\[0pt]
Comment connaître l'autre ? \\[0pt]
Comment considérer le devenir ? \\[0pt]
Comment décider, sinon à la majorité ? \\[0pt]
Comment définir la démocratie ? \\[0pt]
Comment définir la raison ? \\[0pt]
Comment définir la responsabilité politique \\[0pt]
Comment définir la signification \\[0pt]
Comment définir le bien commun ? \\[0pt]
Comment définir le laid ? \\[0pt]
Comment définir le peuple ? \\[0pt]
Comment définir le style ? \\[0pt]
Comment définir l'État de droit ? \\[0pt]
Comment définir l'excès ? \\[0pt]
Comment définir l'infini ? \\[0pt]
Comment déterminer le sens d'une conduite ? \\[0pt]
Comment devient-on artiste ? \\[0pt]
Comment devient-on citoyen ? \\[0pt]
Comment devient-on raisonnable ? \\[0pt]
Comment dire la vérité ? \\[0pt]
Comment entrer dans l'histoire ? \\[0pt]
Comment évaluer l'art ? \\[0pt]
Comment expliquer l'erreur ? \\[0pt]
Comment expliquer les phénomènes mentaux ? \\[0pt]
Comment faire de l'homme un objet d'étude ? \\[0pt]
Comment fonder une psychologie sociale ? \\[0pt]
Comment juger de la politique ? \\[0pt]
Comment juger d'une œuvre d'art ? \\[0pt]
Comment juger son éducation ? \\[0pt]
Comment justifier l'autonomie des sciences de la vie ? \\[0pt]
Comment justifier une méthode ? \\[0pt]
Comment la psychanalyse use-t-elle de l'interprétation ? \\[0pt]
Comment la réalité peut-elle dépasser la fiction ? \\[0pt]
Comment la volonté peut-elle être indéterminée ? \\[0pt]
Comment les sciences sociales expliquent-elles l'irrationnel ? \\[0pt]
Comment les sociétés changent-elles ? \\[0pt]
Comment ne pas être libéral ? \\[0pt]
Comment penser la diversité des langues ? \\[0pt]
Comment penser l'écoulement du temps ? \\[0pt]
Comment penser les universaux ? \\[0pt]
Comment penser un pouvoir qui ne corrompe pas ? \\[0pt]
Comment percevons-nous l'espace ? \\[0pt]
Comment peut-on choisir entre différentes hypothèses ? \\[0pt]
« Comment peut-on être persan ? » \\[0pt]
Comment peut-on être sceptique ? \\[0pt]
Comment reconnaît-on une œuvre d'art ? \\[0pt]
Comment reconnaît-on un vivant ? \\[0pt]
Comment réfuter une thèse métaphysique ? \\[0pt]
Comment représenter la douleur ? \\[0pt]
Comment sait-on qu'on se comprend ? \\[0pt]
Comment s'assurer de ce qui est réel ? \\[0pt]
Comment s'assurer qu'on est libre ? \\[0pt]
Comment savoir ce qui échappe à la conscience \\[0pt]
Comment savoir ce qui existe ? \\[0pt]
Comment savoir quand nous sommes libres ? \\[0pt]
Comment se libérer du temps ? \\[0pt]
Comment s'entendre ? \\[0pt]
Comment traiter les animaux ? \\[0pt]
Comment trancher une controverse ? \\[0pt]
Comment vérifier les hypothèses ? \\[0pt]
Comment vivre ensemble ? \\[0pt]
Comme on dit \\[0pt]
Commettre une faute \\[0pt]
Communauté, collectivité, société \\[0pt]
Communauté et conflit \\[0pt]
Communauté et société \\[0pt]
Communiquer \\[0pt]
Communiquer et enseigner \\[0pt]
Comparer les cultures \\[0pt]
Comparer les cultures ? \\[0pt]
Comparer les langues \\[0pt]
Compatir \\[0pt]
Compétence et autorité \\[0pt]
Complet / incomplet \\[0pt]
Comportement et conduite \\[0pt]
Composer avec les circonstances \\[0pt]
Composition et construction \\[0pt]
Comprendre autrui \\[0pt]
Comprendre, est-ce excuser ? \\[0pt]
Comprendre, est-ce expliquer par les causes ? \\[0pt]
Comprendre, est-ce interpréter ? \\[0pt]
Comprendre l'inconscient \\[0pt]
Compter et mesurer \\[0pt]
Compter sur soi \\[0pt]
Concept et existence \\[0pt]
Concept et image \\[0pt]
Concept et métaphore \\[0pt]
Conception et perception \\[0pt]
Concevoir et juger \\[0pt]
Concevoir et percevoir \\[0pt]
Concevoir le possible \\[0pt]
Conduire sa vie \\[0pt]
Conduire ses pensées \\[0pt]
Conflit et démocratie \\[0pt]
Confondre \\[0pt]
Connaissance commune et connaissance scientifique \\[0pt]
Connaissance, croyance, conjecture \\[0pt]
Connaissance du futur et connaissance du passé \\[0pt]
Connaissance et croyance \\[0pt]
Connaissance et expérience \\[0pt]
Connaissance et liberté \\[0pt]
Connaissance et progrès \\[0pt]
Connaissance historique et action politique \\[0pt]
Connaissons-nous mieux le présent que le passé ? \\[0pt]
Connaît-on la vie ou le vivant ? \\[0pt]
Connaître, est-ce connaître par les causes ? \\[0pt]
Connaître et agir \\[0pt]
Connaître et comprendre \\[0pt]
Connaître et penser \\[0pt]
Connaître, expliquer, comprendre \\[0pt]
Connaître les causes \\[0pt]
Connaître les choses, en quoi est-ce déterminer leurs différences ? \\[0pt]
Connaître par les causes \\[0pt]
Connaître ses limites \\[0pt]
Conscience et attention \\[0pt]
Conscience et mémoire \\[0pt]
Conseiller le prince \\[0pt]
Consensus et conflit \\[0pt]
Consentir \\[0pt]
Conservatisme et tradition \\[0pt]
Considère-t-on jamais le temps en lui-même ? \\[0pt]
Consistance et précarité \\[0pt]
Constitution et lois \\[0pt]
Consumérisme et démocratie \\[0pt]
Contemplation et action \\[0pt]
Contemplation et création \\[0pt]
Contemplation et distraction \\[0pt]
Contempler \\[0pt]
Contempler une œuvre d'art \\[0pt]
Contingence et nécessité \\[0pt]
Contingence et rationalité \\[0pt]
Continuité et discontinuité \\[0pt]
Contradiction et opposition \\[0pt]
Contrainte et désobéissance \\[0pt]
Contrainte et obligation \\[0pt]
Convention et observation \\[0pt]
Conviction et responsabilité \\[0pt]
Corps et identité \\[0pt]
Correspondre \\[0pt]
Couleur et forme \\[0pt]
Création et critique \\[0pt]
Création et production \\[0pt]
Création et réception \\[0pt]
Créativité et contrainte \\[0pt]
Créer \\[0pt]
Crime et châtiment \\[0pt]
Crimes et châtiments \\[0pt]
Crise et création \\[0pt]
Crise et progrès \\[0pt]
Crise et révolution \\[0pt]
Critiquer \\[0pt]
Critiquer la démocratie \\[0pt]
Croire au bonheur \\[0pt]
Croire aux fictions \\[0pt]
Croire en Dieu \\[0pt]
Croire en soi \\[0pt]
Croire, est-ce être faible ? \\[0pt]
Croire, est-ce obéir ? \\[0pt]
Croire et savoir \\[0pt]
Croire pour savoir \\[0pt]
Croire savoir \\[0pt]
Croyance et certitude \\[0pt]
Croyance et passivité \\[0pt]
Croyance et probabilité \\[0pt]
Croyance et société \\[0pt]
Croyance et vérité \\[0pt]
Cultes et rituels \\[0pt]
Cultivons notre jardin \\[0pt]
Culture et civilisation \\[0pt]
Culture et conscience \\[0pt]
Culture et moralité \\[0pt]
Culture et sociétés \\[0pt]
Dans les sciences humaines et sociales, peut-on préconiser l'imitation des sciences de la nature ? \\[0pt]
Dans quelle mesure la traduction est-elle une activité cognitive ? \\[0pt]
Dans quelle mesure pouvons-nous échapper à notre passé ? \\[0pt]
Dans quelle mesure suis-je responsable de mon inconscient ? \\[0pt]
D'après nature \\[0pt]
Débattre \\[0pt]
Décider \\[0pt]
Décomposer les choses \\[0pt]
Décorer \\[0pt]
Découverte et invention \\[0pt]
Découverte et invention dans les sciences \\[0pt]
Découvrir \\[0pt]
Découvrir et inventer \\[0pt]
Décrire \\[0pt]
Décrire, est-ce déjà expliquer ? \\[0pt]
Décrire et expliquer \\[0pt]
Décrire une langue \\[0pt]
Déduction et expérience \\[0pt]
Défendre ses droits \\[0pt]
Défendre son honneur \\[0pt]
Définir \\[0pt]
Définir, est-ce déterminer l'essence ? \\[0pt]
Définir et démontrer : à quoi bon ? \\[0pt]
Définir l'art : à quoi bon ? \\[0pt]
Définir la vérité, est-ce la connaître ? \\[0pt]
Définition et démonstration \\[0pt]
Définition nominale et définition réelle \\[0pt]
Définitions, axiomes, postulats \\[0pt]
Dégager la structure, est-ce rendre compte de ce qu'est une chose ? \\[0pt]
Déjouer \\[0pt]
« De la musique avant toute chose » \\[0pt]
Délibérer \\[0pt]
Délibérer, est-ce être dans l'incertitude ? \\[0pt]
De l'utilité des voyages \\[0pt]
Démêler le vrai du faux \\[0pt]
Démériter \\[0pt]
Démocrates et démagogues \\[0pt]
Démocratie ancienne et démocratie moderne \\[0pt]
Démocratie et anarchie \\[0pt]
Démocratie et consensus \\[0pt]
Démocratie et démagogie \\[0pt]
Démocratie et impérialisme \\[0pt]
Démocratie et religion \\[0pt]
Démocratie et représentation \\[0pt]
Démocratie et république \\[0pt]
Démocratie et transparence \\[0pt]
Démocratie et vérité \\[0pt]
Démocratie représentative \\[0pt]
Démonstration et argumentation \\[0pt]
Démonstration et déduction \\[0pt]
Démontrer, argumenter, expérimenter \\[0pt]
Démontrer est-il le privilège du mathématicien ? \\[0pt]
Démontrer et argumenter \\[0pt]
Dénaturer \\[0pt]
Dépasser les apparences ? \\[0pt]
Dépasser les bornes \\[0pt]
Dépasser l'humain \\[0pt]
Déplaire \\[0pt]
De quel bonheur sommes-nous capables ? \\[0pt]
De quel droit ? \\[0pt]
De quelle certitude la science est-elle capable ? \\[0pt]
De quelle expertise les sciences humaines peuvent-elles se prévaloir ? \\[0pt]
De quelle liberté témoigne l'œuvre d'art ? \\[0pt]
De quelle réalité nos perceptions témoignent-elles ? \\[0pt]
De quelle réalité témoignent nos perceptions ? \\[0pt]
De quelle science humaine la folie peut-elle être l'objet ? \\[0pt]
De quelle transgression l'art est-il susceptible ? \\[0pt]
De quelle vérité l'art est-il capable ? \\[0pt]
De quelle vérité l'inconscient est-il porteur ? \\[0pt]
De quelle vérité l'opinion est-elle capable ? \\[0pt]
De qui sommes-nous les obligés ? \\[0pt]
De quoi a-t-on besoin ? \\[0pt]
De quoi doute un sceptique ? \\[0pt]
De quoi est-on conscient ? \\[0pt]
De quoi est-on malheureux ? \\[0pt]
De quoi faisons-nous l'expérience ? \\[0pt]
De quoi fait-on l'expérience face à une œuvre ? \\[0pt]
De quoi faut-il attendre son bonheur ? \\[0pt]
De quoi « humain » est-il le nom ? \\[0pt]
De quoi la forme est-elle la forme ? \\[0pt]
De quoi la logique est-elle la science ? \\[0pt]
De quoi la politique s'occupe-t-elle ? \\[0pt]
De quoi l'art nous console-t-il ? \\[0pt]
De quoi l'art nous délivre-t-il ? \\[0pt]
De quoi l'art peut-il être contemporain ? \\[0pt]
De quoi l'art peut-il nous libérer ? \\[0pt]
De quoi le phénomène est-il l'apparaître ? \\[0pt]
De quoi le réel est-il constitué ? \\[0pt]
De quoi les logiciens parlent-ils ? \\[0pt]
De quoi les métaphysiciens parlent-ils ? \\[0pt]
De quoi les sciences humaines nous instruisent-elles ? \\[0pt]
De quoi l'État doit-il être propriétaire ? \\[0pt]
De quoi le tissu social est-il fait ? \\[0pt]
De quoi l'expérience esthétique est-elle expérience ? \\[0pt]
De quoi l'expérience esthétique est-elle l'expérience ? \\[0pt]
De quoi l'historien fait-il l'histoire ? \\[0pt]
De quoi l'outil nous libère-t-il ? \\[0pt]
De quoi n'avons-nous pas conscience ? \\[0pt]
De quoi ne peut-on pas répondre ? \\[0pt]
De quoi parlent les mathématiques ? \\[0pt]
De quoi parlent les théories physiques ? \\[0pt]
De quoi pâtit-on ? \\[0pt]
De quoi peut-il y avoir science ? \\[0pt]
De quoi peut-on réclamer la propriété ? \\[0pt]
De quoi pouvons-nous être certains ? \\[0pt]
De quoi sommes-nous prisonniers ? \\[0pt]
De quoi sommes-nous responsables ? \\[0pt]
De quoi suis-je responsable ? \\[0pt]
De quoi y a-t-il expérience ? \\[0pt]
De quoi y a-t-il histoire ? \\[0pt]
Déraisonner \\[0pt]
Désacraliser \\[0pt]
Des comportements économiques peuvent-ils être rationnels ? \\[0pt]
Désespérer de l'homme \\[0pt]
Des événements aléatoires peuvent-ils obéir à des lois ? \\[0pt]
Des goûts et des couleurs \\[0pt]
Des hommes et des dieux \\[0pt]
Désintérêt et désintéressement \\[0pt]
Désirer \\[0pt]
Désirer, est-ce refuser de se satisfaire de la réalité \\[0pt]
Désire-t-on la reconnaissance ? \\[0pt]
Désire-t-on un autre que soi ? \\[0pt]
Désir et politique \\[0pt]
Désir et société \\[0pt]
Désir et temps \\[0pt]
Désir et volonté \\[0pt]
Des motivations peuvent-elles être sociales ? \\[0pt]
Des mythes peuvent-ils encore apparaître ? \\[0pt]
Des nations peuvent-elles former une société ? \\[0pt]
Désobéir \\[0pt]
Désobéir aux lois \\[0pt]
Désobéissance et résistance \\[0pt]
Désordre et chaos \\[0pt]
Des peuples sans histoire \\[0pt]
Des sciences molles ? \\[0pt]
Dessiner \\[0pt]
Des sociétés sans État sont-elles des sociétés politiques ? \\[0pt]
Des sociétés sans histoire \\[0pt]
Des statistiques et des hommes \\[0pt]
Des structures et des hommes \\[0pt]
Détermination, déterminisme, liberté \\[0pt]
Déterminer \\[0pt]
Déterminisme naturel et déterminisme social \\[0pt]
Déterminisme psychique et déterminisme physique \\[0pt]
Détruire pour reconstruire \\[0pt]
Deux et deux font quatre \\[0pt]
Devant qui sommes-nous responsables ? \\[0pt]
Devenir \\[0pt]
Devenir autre \\[0pt]
Devenir ce que l'on est \\[0pt]
Devenir citoyen \\[0pt]
Devenir et évolution \\[0pt]
« Deviens ce que tu es » \\[0pt]
« Deviens qui tu es » \\[0pt]
Devient-on raisonnable ? \\[0pt]
Devoir, est-ce pouvoir ? \\[0pt]
Devoir et conformisme \\[0pt]
Devoir et utilité \\[0pt]
Devoir mourir \\[0pt]
Devons-nous aimer notre destin ? \\[0pt]
Devons-nous douter de l'existence des choses ? \\[0pt]
Devons-nous nous faire confiance ? \\[0pt]
Diachronie et synchronie \\[0pt]
Dialectique et Philosophie \\[0pt]
Dialectique et vérité \\[0pt]
Dialoguer \\[0pt]
Dieu aurait-il pu mieux faire ? \\[0pt]
Dieu des philosophes et Dieu des croyants \\[0pt]
Dieu est-il l'être ? \\[0pt]
Dieu est-il un concept ? \\[0pt]
Dieu est-il une limite de la pensée ? \\[0pt]
« Dieu est mort » \\[0pt]
Dieu est mort \\[0pt]
Dieu et César \\[0pt]
Dieu et l'âme \\[0pt]
Dieu et le monde \\[0pt]
Dieu, l'âme, le monde \\[0pt]
Dieu pense-t-il ? \\[0pt]
Dieu peut-il tout faire ? \\[0pt]
Dieu, prouvé ou éprouvé ? \\[0pt]
Dieu tout-puissant \\[0pt]
Différer \\[0pt]
Dire ce qui est \\[0pt]
Dire, est-ce faire ? \\[0pt]
Dire et faire \\[0pt]
Dire et montrer \\[0pt]
Dire « je » \\[0pt]
Dire la vérité \\[0pt]
Dire la vérité, est-ce un devoir ? \\[0pt]
Dire le monde \\[0pt]
Dire le singulier \\[0pt]
Dire oui \\[0pt]
Dire que l'être humain est un animal, est-ce méconnaître sa dignité ? \\[0pt]
Diriger son esprit \\[0pt]
Discerner et juger \\[0pt]
Discret et continu \\[0pt]
Discussion et dialogue \\[0pt]
Discuter de la beauté d'une chose, est-ce discuter sur une réalité ? \\[0pt]
Disposer de son corps \\[0pt]
Distinguer \\[0pt]
Diversité des sciences humaines et unité de l'homme \\[0pt]
Diversité ou universalité \\[0pt]
Diviser pour mieux régner \\[0pt]
Division du travail et cohésion sociale \\[0pt]
Document, archive, témoignage \\[0pt]
Documents et monuments \\[0pt]
Doit-on bien juger pour bien faire ? \\[0pt]
Doit-on cesser de chercher à définir l'œuvre d'art ? \\[0pt]
Doit-on corriger les inégalités sociales ? \\[0pt]
Doit-on croire en l'humanité ? \\[0pt]
Doit-on déplorer le désaccord ? \\[0pt]
Doit-on distinguer devoir moral et obligation sociale ? \\[0pt]
Doit-on justifier les inégalités ? \\[0pt]
Doit-on préférer l'injustice au désordre ? \\[0pt]
Doit-on répondre de ce qu'on est devenu ? \\[0pt]
Doit-on respecter la nature ? \\[0pt]
Doit-on se faire l'avocat du diable ? \\[0pt]
Doit-on se justifier d'exister ? \\[0pt]
Doit-on toujours dire la vérité ? \\[0pt]
Doit-on tout calculer ? \\[0pt]
Don, échange, potlatch \\[0pt]
Donner \\[0pt]
Donner à chacun son dû \\[0pt]
Donner, à quoi bon ? \\[0pt]
Donner des exemples \\[0pt]
Donner des raisons \\[0pt]
Donner du sens \\[0pt]
Donner raison \\[0pt]
Donner raison, rendre raison \\[0pt]
Donner sa parole \\[0pt]
Donner sens \\[0pt]
Donner une représentation \\[0pt]
Donner un exemple \\[0pt]
D'où la politique tire-t-elle sa légitimité ? \\[0pt]
Douter \\[0pt]
Douter de tout \\[0pt]
D'où viennent les idées générales ? \\[0pt]
D'où vient aux objets techniques leur beauté ? \\[0pt]
D'où vient la certitude dans les sciences ? \\[0pt]
D'où vient la signification des mots ? \\[0pt]
D'où vient le mal ? \\[0pt]
D'où vient le plaisir de lire ? \\[0pt]
D'où vient le sens du devoir ? \\[0pt]
D'où vient que l'histoire soit autre chose qu'un chaos ? \\[0pt]
Droit et coutume \\[0pt]
Droit et démocratie \\[0pt]
Droit et devoir sont-ils liés ? \\[0pt]
Droit et morale \\[0pt]
Droit et nature \\[0pt]
Droit et protection \\[0pt]
Droit naturel et loi naturelle \\[0pt]
Droits de l'homme, droits du citoyen \\[0pt]
Droits de l'homme et droits du citoyen \\[0pt]
Droits et devoirs \\[0pt]
Droits et devoirs sont-ils réciproques ? \\[0pt]
Droits et obligations \\[0pt]
Du penser à l'être, la conséquence est-elle bonne ? \\[0pt]
Durer \\[0pt]
Échange et don \\[0pt]
Échange et partage \\[0pt]
Échange et valeur \\[0pt]
Échanger, est-ce risquer ? \\[0pt]
Éclairer \\[0pt]
Économie et politique \\[0pt]
Économie et prédiction \\[0pt]
Économie et société \\[0pt]
Économie politique et politique économique \\[0pt]
Écouter \\[0pt]
Écrire \\[0pt]
Écrire l'histoire \\[0pt]
Écrire l'histoire, est-ce la faire ? \\[0pt]
Éducation de l'homme, éducation du citoyen \\[0pt]
Éducation et citoyenneté \\[0pt]
Éduquer \\[0pt]
Éduquer à la liberté \\[0pt]
Éduquer, est-ce dénaturer ? \\[0pt]
Éduquer et instruire \\[0pt]
Éduquer le citoyen \\[0pt]
Éduquer le désir \\[0pt]
Éduquer le goût \\[0pt]
Égalité des droits, égalité des conditions \\[0pt]
Égalité et identité \\[0pt]
Égalité et liberté \\[0pt]
Égoïsme et altruisme \\[0pt]
Égoïsme et individualisme \\[0pt]
Égoïsme et méchanceté \\[0pt]
Élite et démocratie \\[0pt]
Empathie et sympathie \\[0pt]
Empirique et expérimental \\[0pt]
Empirique et transcendantal \\[0pt]
En apparence \\[0pt]
Enfance et moralité \\[0pt]
En histoire, tout est-il affaire d'interprétation ? \\[0pt]
En mathématiques, la notion d'existence a-t-elle un sens ? \\[0pt]
En morale, est-ce seulement l'intention qui compte ? \\[0pt]
En morale, y a-t-il quelque chose à savoir ? \\[0pt]
En politique, faut-il refuser l'utopie ? \\[0pt]
En politique, nécessité fait loi \\[0pt]
En politique, ne fait-on que défendre ses intérêts ? \\[0pt]
En politique n'y a-t-il que des rapports de force ? \\[0pt]
En politique, peut-on faire table rase du passé ? \\[0pt]
En politique, y a-t-il des modèles ? \\[0pt]
En quel sens la maladie peut-elle transformer notre vie ? \\[0pt]
En quel sens la métaphysique a-t-elle une histoire ? \\[0pt]
En quel sens la métaphysique est-elle une science ? \\[0pt]
En quel sens l'anthropologie peut-elle être historique ? \\[0pt]
En quel sens le vécu est-il l'objet des sciences humaines ? \\[0pt]
En quel sens parler de structure métaphysique ? \\[0pt]
En quel sens peut-on dire de l'écologie qu'elle est politique ? \\[0pt]
En quel sens peut-on parler de la vie sociale comme d'un jeu ? \\[0pt]
En quel sens peut-on parler de transcendance ? \\[0pt]
En quel sens peut-on parler d'expérience possible ? \\[0pt]
En quel sens une œuvre d'art est-elle un document ? \\[0pt]
En quels sens les corps peuvent-ils être objets de la métaphysique ? \\[0pt]
Enquêter \\[0pt]
En quoi la connaissance de la matière peut-elle relever de la métaphysique ? \\[0pt]
En quoi la linguistique est-elle une science ? \\[0pt]
En quoi la matière s'oppose-t-elle à l'esprit ? \\[0pt]
En quoi l'animal intéresse-t-il les sciences humaines ? \\[0pt]
En quoi la sexualité peut-elle être objet de science ? \\[0pt]
En quoi la technique fait-elle question ? \\[0pt]
En quoi le langage est-il objet de science ? \\[0pt]
En quoi les sciences humaines nous éclairent-elles sur la barbarie ? \\[0pt]
En quoi les sciences humaines sont-elles normatives ? \\[0pt]
En quoi l'histoire est-elle une science du réel ? \\[0pt]
En quoi l'œuvre d'art donne-t-elle à penser ? \\[0pt]
En quoi une discussion est-elle politique ? \\[0pt]
En quoi une insulte est-elle blessante ? \\[0pt]
Enseigner \\[0pt]
Enseigner, instruire, éduquer \\[0pt]
Enseigner l'art \\[0pt]
Entendement et raison \\[0pt]
Entendre \\[0pt]
Entendre raison \\[0pt]
Entité et identité \\[0pt]
Entre le vrai et le faux y a-t-il une place pour le probable ? \\[0pt]
Entre l'utile et l'honnête, peut-on choisir ? \\[0pt]
Entrer en scène \\[0pt]
Énumérer \\[0pt]
Épistémologie et psychologie \\[0pt]
Épistémologie générale et épistémologie des sciences particulières \\[0pt]
Éprouver \\[0pt]
Éprouver sa valeur \\[0pt]
Erreur et illusion \\[0pt]
Erreur, illusion, hallucination \\[0pt]
Espace et réalité \\[0pt]
Espace et structure sociale \\[0pt]
Espace mathématique et espace physique \\[0pt]
Espace privé, espace public \\[0pt]
Espace public et vie privée \\[0pt]
Espace vécu, espace géométrique \\[0pt]
Espèce et genre \\[0pt]
Esprit et intériorité \\[0pt]
Essayer \\[0pt]
Essence et existence \\[0pt]
Essence et nature \\[0pt]
Est beau ce qui ne sert à rien \\[0pt]
Est-ce la certitude qui fait la science ? \\[0pt]
Est-ce la démonstration qui fait la science ? \\[0pt]
Est-ce le corps qui perçoit ? \\[0pt]
Est-ce l'utilité qui définit un objet technique ? \\[0pt]
Est-ce par son objet ou par ses méthodes qu'une science peut se définir ? \\[0pt]
Est-ce pour des raisons morales qu'il faut protéger l'environnement ? \\[0pt]
Est-ce un devoir de rechercher la vérité ? \\[0pt]
Est-ce un devoir d'être sincère ? \\[0pt]
Est-ce une chance de naître humain ? \\[0pt]
Esthétique et éthique \\[0pt]
Esthétique et poétique \\[0pt]
Esthétique et politique \\[0pt]
Esthétisme et moralité \\[0pt]
Est-il bon qu'un seul commande ? \\[0pt]
Est-il difficile de savoir ce que l'on veut ? \\[0pt]
Est-il difficile d'être heureux ? \\[0pt]
Est-il impossible de moraliser la politique ? \\[0pt]
Est-il judicieux de revenir sur ses décisions ? \\[0pt]
Est-il légitime d'affirmer que seul le présent existe ? \\[0pt]
Est-il mauvais de suivre son désir ? \\[0pt]
Est-il parfois bon de mentir ? \\[0pt]
Est-il possible de croire en la vie éternelle ? \\[0pt]
Est-il possible de douter de tout ? \\[0pt]
Est-il possible de ne croire à rien ? \\[0pt]
Est-il possible de ne pas savoir ce que l'on pense ? \\[0pt]
Est-il possible d'être neutre politiquement ? \\[0pt]
Est-il raisonnable de lutter contre le temps ? \\[0pt]
Est-il si difficile d'accéder à la vérité ? \\[0pt]
Est-il toujours illicite d'inférer ce qui doit être à partir de ce qui est ? \\[0pt]
Est-il toujours possible de faire ce que l'on dit ? \\[0pt]
Est-il vrai qu'en science, « rien n'est donné, tout est construit » ? \\[0pt]
Est-il vrai qu'on apprenne de ses erreurs ? \\[0pt]
Estimer \\[0pt]
Est-on fondé à distinguer la justice et le droit ? \\[0pt]
Est-on fondé à parler d'une imperfection du langage ? \\[0pt]
Est-on le produit d'une culture ? \\[0pt]
Est-on propriétaire de son corps ? \\[0pt]
Est-on responsable de ce qu'on n'a pas voulu ? \\[0pt]
Est-on responsable de l'avenir de l'humanité \\[0pt]
Est-on responsable de ses rêves ? \\[0pt]
Est-on responsable de son passé ? \\[0pt]
Établir la vérité, est-ce nécessairement démontrer ? \\[0pt]
Établir les faits \\[0pt]
État et nation \\[0pt]
État et société \\[0pt]
État et vie privée \\[0pt]
État, peuple, nation \\[0pt]
Éternité et immortalité \\[0pt]
Éternité et perpétuité \\[0pt]
Éthique et authenticité \\[0pt]
Ethnologie et cinéma \\[0pt]
Ethnologie et ethnocentrisme \\[0pt]
Ethnologie et sociologie \\[0pt]
Être absolument libre \\[0pt]
Être acteur \\[0pt]
Être adulte \\[0pt]
Être affairé \\[0pt]
Être aliéné \\[0pt]
Être apolitique \\[0pt]
Être asocial \\[0pt]
Être au monde \\[0pt]
Être bien élevé \\[0pt]
Être bon juge \\[0pt]
Être cause de soi \\[0pt]
Être certain \\[0pt]
Être, c'est agir \\[0pt]
Être chez soi \\[0pt]
Être citoyen \\[0pt]
Être citoyen du monde \\[0pt]
Être civilisé \\[0pt]
Être compris \\[0pt]
Être conformiste \\[0pt]
Être conscient de soi, est-ce être maître de soi ? \\[0pt]
Être content de soi \\[0pt]
« Être contre » \\[0pt]
Être dans l'esprit \\[0pt]
Être dans le temps \\[0pt]
Être dans le vrai \\[0pt]
Être dans l'opposition \\[0pt]
Être dans son bon droit \\[0pt]
Être dans son droit \\[0pt]
Être de mauvaise humeur \\[0pt]
Être démocrate \\[0pt]
Être de son temps \\[0pt]
Être déterminé \\[0pt]
Être égal à soi-même \\[0pt]
Être égaux \\[0pt]
Être en bonne santé \\[0pt]
Être en désaccord \\[0pt]
Être en dette \\[0pt]
Être en puissance \\[0pt]
Être en règle avec soi-même \\[0pt]
Être ensemble \\[0pt]
Être, est-ce agir ? \\[0pt]
Être et avoir \\[0pt]
Être et devenir \\[0pt]
Être et devoir être \\[0pt]
Être et être pensé \\[0pt]
Être et être perçu \\[0pt]
Être et exister \\[0pt]
Être et ne plus être \\[0pt]
Être et paraître \\[0pt]
Être et représentation \\[0pt]
Être et sens \\[0pt]
Être exemplaire \\[0pt]
Être fidèle \\[0pt]
Être heureux \\[0pt]
Être hors de soi \\[0pt]
Être hors la loi \\[0pt]
Être hors-la-loi \\[0pt]
Être identique \\[0pt]
Être impossible \\[0pt]
Être inspiré \\[0pt]
Être juge et partie \\[0pt]
Être là \\[0pt]
Être l'entrepreneur de soi-même \\[0pt]
Être libre, est-ce n'obéir qu'à soi-même ? \\[0pt]
Être libre, même dans les fers \\[0pt]
Être logique \\[0pt]
Être maître de soi \\[0pt]
Être malade \\[0pt]
Être matérialiste \\[0pt]
Être méchant \\[0pt]
Être méchant volontairement \\[0pt]
Être mère \\[0pt]
Être mortel \\[0pt]
Être naturel \\[0pt]
Être né \\[0pt]
Être normal \\[0pt]
Être ou ne pas être ? \\[0pt]
Être par soi \\[0pt]
Être par soi / être par autre chose \\[0pt]
Être passionné \\[0pt]
Être patient \\[0pt]
Être pauvre \\[0pt]
Être père \\[0pt]
Être poli \\[0pt]
Être politique \\[0pt]
Être possible \\[0pt]
Être pragmatique \\[0pt]
Être présent \\[0pt]
Être quelqu'un \\[0pt]
Être réaliste \\[0pt]
Être religieux, est-ce nécessairement être dogmatique ? \\[0pt]
Être riche \\[0pt]
Être sans cause \\[0pt]
Être sans cœur \\[0pt]
Être sans foi ni loi \\[0pt]
Être sans scrupule \\[0pt]
Être sceptique \\[0pt]
« Être » se dit-il en plusieurs sens ? \\[0pt]
Être sensible \\[0pt]
Être seul avec sa conscience \\[0pt]
Être seul avec soi-même \\[0pt]
Être soi-même \\[0pt]
Être solidaire \\[0pt]
Être spirituel \\[0pt]
Être sujet \\[0pt]
Être systématique \\[0pt]
Être un artiste \\[0pt]
Être un corps \\[0pt]
Être une chose qui pense \\[0pt]
Être une personne \\[0pt]
Être, vie et pensée \\[0pt]
Étudier \\[0pt]
Évaluer \\[0pt]
Évaluer l'art \\[0pt]
Événement et structure \\[0pt]
Évidence et certitude \\[0pt]
Évolution et progrès \\[0pt]
Exactitude et objectivité \\[0pt]
Existe-il des mentalités primitives ? \\[0pt]
Existence et contingence \\[0pt]
Existence et essence \\[0pt]
Existence et transcendance \\[0pt]
Exister \\[0pt]
Existe-t-il dans le monde des réalités identiques ? \\[0pt]
Existe-t-il des autorités en matière d'art ? \\[0pt]
Existe-t-il des connaissances a priori ? \\[0pt]
Existe-t-il des degrés de vérité ? \\[0pt]
Existe-t-il des dilemmes moraux ? \\[0pt]
Existe-t-il des émotions proprement esthétiques ? \\[0pt]
Existe-t-il des états mentaux collectifs ? \\[0pt]
Existe-t-il des expériences métaphysiques ? \\[0pt]
Existe-t-il des identités collectives ? \\[0pt]
Existe-t-il des intuitions métaphysiques ? \\[0pt]
Existe-t-il des langages naturels ? \\[0pt]
Existe-t-il des mondes sociaux ? \\[0pt]
Existe-t-il des paroles vraies ? \\[0pt]
Existe-t-il des preuves irréfutables ? \\[0pt]
Existe-t-il des principes premiers ? \\[0pt]
Existe-t-il des problèmes insolubles ? \\[0pt]
Existe-t-il des questions sans réponse ? \\[0pt]
Existe-t-il des sciences de différentes natures ? \\[0pt]
Existe-t-il des sujets collectifs ? \\[0pt]
Existe-t-il des violences légitimes ? \\[0pt]
Existe-t-il différentes sortes de sciences ? \\[0pt]
Existe-t-il plusieurs déterminismes ? \\[0pt]
Existe-t-il plusieurs mondes ? \\[0pt]
Existe-t-il un bien commun qui soit la norme de la vie politique ? \\[0pt]
Existe-t-il un bien souverain ? \\[0pt]
Existe-t-il une autorité naturelle ? \\[0pt]
Existe-t-il une hiérarchie des êtres ? \\[0pt]
Existe-t-il une histoire du présent ? \\[0pt]
Existe-t-il une opinion publique ? \\[0pt]
Existe-t-il une réalité subjective ? \\[0pt]
Existe-t-il une réalité symbolique ? \\[0pt]
Existe-t-il une science de la morale ? \\[0pt]
Existe-t-il une science de l'être ? \\[0pt]
Existe-t-il une sensibilité philosophique ? \\[0pt]
Existe-t-il une unité des arts ? \\[0pt]
Existe-t-il une violence symbolique ? \\[0pt]
Expérience esthétique et sens commun \\[0pt]
Expérience et approximation \\[0pt]
Expérience et expérimentation \\[0pt]
Expérience et habitude \\[0pt]
Expérience et interprétation \\[0pt]
Expérience et jugement \\[0pt]
Expérience et phénomène \\[0pt]
Expérimentation et vérification \\[0pt]
Expérimenter \\[0pt]
Explication causale, explication génétique dans les sciences humaines \\[0pt]
Explication et compréhension \\[0pt]
Explication et prévision \\[0pt]
Expliquer \\[0pt]
Expliquer, est-ce interpréter ? \\[0pt]
Expliquer, est-ce justifier ? \\[0pt]
Expliquer et comprendre \\[0pt]
Expliquer et interpréter \\[0pt]
Expliquer et justifier \\[0pt]
« Expliquer les faits sociaux par des faits sociaux » \\[0pt]
Expression et création \\[0pt]
Expression et signification \\[0pt]
Extension et compréhension \\[0pt]
Faire ce qu'on dit \\[0pt]
Faire comme les autres \\[0pt]
Faire comme si \\[0pt]
Faire confiance \\[0pt]
Faire corps \\[0pt]
Faire croire \\[0pt]
Faire de la métaphysique, est-ce se détourner du monde ? \\[0pt]
Faire de la politique \\[0pt]
Faire de nécessité vertu \\[0pt]
Faire de sa vie une œuvre d'art \\[0pt]
Faire des choix \\[0pt]
Faire école \\[0pt]
Faire est-il nécessairement savoir faire ? \\[0pt]
Faire et agir \\[0pt]
Faire et laisser faire \\[0pt]
Faire justice \\[0pt]
Faire la loi \\[0pt]
Faire la morale \\[0pt]
Faire la paix \\[0pt]
Faire la révolution \\[0pt]
Faire la vérité \\[0pt]
Faire le bien \\[0pt]
Faire le mal \\[0pt]
Faire l'histoire \\[0pt]
Faire preuve d'humanité \\[0pt]
Faire sa vie \\[0pt]
Faire ses preuves \\[0pt]
Faire société \\[0pt]
Faire son devoir \\[0pt]
Faire son possible \\[0pt]
Faire table rase \\[0pt]
Faire une expérience \\[0pt]
Faisons-nous notre propre malheur ? \\[0pt]
Fait et essence \\[0pt]
Fait et valeur \\[0pt]
Faits et valeurs \\[0pt]
Falsifier \\[0pt]
Famille et tribu \\[0pt]
Faudrait-il ne rien oublier ? \\[0pt]
Faut-il accorder l'esprit aux bêtes ? \\[0pt]
Faut-il aimer la vie ? \\[0pt]
Faut-il aimer son prochain comme soi-même ? \\[0pt]
Faut-il aller au-delà des apparences ? \\[0pt]
Faut-il apprendre à voir ? \\[0pt]
Faut-il avoir des ennemis ? \\[0pt]
Faut-il avoir des principes ? \\[0pt]
Faut-il avoir foi en la raison ? \\[0pt]
Faut-il avoir peur de la liberté ? \\[0pt]
Faut-il avoir peur de la nature ? \\[0pt]
Faut-il avoir peur de l'avenir ? \\[0pt]
Faut-il avoir peur de la vérité ? \\[0pt]
Faut-il avoir peur des contradictions ? \\[0pt]
Faut-il avoir peur des habitudes ? \\[0pt]
Faut-il chasser les poètes ? \\[0pt]
Faut-il combattre ses illusions ? \\[0pt]
Faut-il concilier les contraires ? \\[0pt]
Faut-il condamner la rhétorique ? \\[0pt]
Faut-il condamner les illusions ? \\[0pt]
Faut-il connaître l'histoire des sciences ? \\[0pt]
Faut-il considérer le droit pénal comme instituant une violence légitime ? \\[0pt]
Faut-il considérer les faits sociaux comme des choses ? \\[0pt]
Faut-il craindre la démocratie ? \\[0pt]
Faut-il craindre la révolution ? \\[0pt]
Faut-il craindre le pire ? \\[0pt]
Faut-il craindre le pouvoir souverain ? \\[0pt]
Faut-il craindre les foules ? \\[0pt]
Faut-il craindre les masses ? \\[0pt]
Faut-il croire au progrès ? \\[0pt]
Faut-il croire en quelque chose ? \\[0pt]
Faut-il cultiver le naturel ? \\[0pt]
Faut-il défendre l'ordre à tout prix ? \\[0pt]
Faut-il des techniques du raisonnement ? \\[0pt]
Faut-il détruire l'État ? \\[0pt]
Faut-il dire tout haut ce que les autres pensent tout bas ? \\[0pt]
Faut-il diriger l'économie ? \\[0pt]
Faut-il distinguer ce qui est de ce qui doit être ? \\[0pt]
Faut-il distinguer esthétique et philosophie de l'art ? \\[0pt]
Faut-il écouter sa conscience ? \\[0pt]
Faut-il enfermer ? \\[0pt]
Faut-il enfermer les œuvres dans les musées ? \\[0pt]
Faut-il en finir avec l'esthétique ? \\[0pt]
Faut-il être bon ? \\[0pt]
Faut-il être cosmopolite ? \\[0pt]
Faut-il être discipliné ? \\[0pt]
Faut-il être fidèle à soi-même ? \\[0pt]
Faut-il être heureux ? \\[0pt]
Faut-il être mesuré ? \\[0pt]
Faut-il être mesuré en toutes choses ? \\[0pt]
Faut-il être naturel ? \\[0pt]
Faut-il être objectif ? \\[0pt]
Faut-il être réaliste ? \\[0pt]
Faut-il être réaliste en politique ? \\[0pt]
Faut-il expliquer la morale par son utilité ? \\[0pt]
Faut-il faire de la philosophie ? \\[0pt]
Faut-il faire de sa vie une œuvre d'art ? \\[0pt]
Faut-il fuir la politique ? \\[0pt]
Faut-il interpréter la loi ? \\[0pt]
Faut-il laisser parler la nature ? \\[0pt]
Faut-il libérer la parole ? \\[0pt]
Faut-il limiter la souveraineté ? \\[0pt]
Faut-il limiter les prétentions de la science ? \\[0pt]
Faut-il limiter l'exercice de la puissance publique ? \\[0pt]
Faut-il maîtriser ses émotions ? \\[0pt]
Faut-il mathématiser le réel pour le connaître ? \\[0pt]
Faut-il ménager les apparences ? \\[0pt]
Faut-il mépriser le luxe ? \\[0pt]
Faut-il mieux vivre comme si nous ne devions jamais mourir ? \\[0pt]
Faut-il n'être jamais méchant ? \\[0pt]
Faut-il opposer à la politique la souveraineté du droit ? \\[0pt]
Faut-il opposer éduquer et instruire ? \\[0pt]
Faut-il opposer la foi et la raison ? \\[0pt]
Faut-il opposer l'art à la connaissance ? \\[0pt]
Faut-il opposer la théorie et la pratique ? \\[0pt]
Faut-il opposer le réel et l'imaginaire ? \\[0pt]
Faut-il opposer l'esprit et la matière ? \\[0pt]
Faut-il opposer l'État et la société ? \\[0pt]
Faut-il opposer l'histoire et la fiction ? \\[0pt]
Faut-il opposer produire et créer ? \\[0pt]
Faut-il opposer rhétorique et philosophie ? \\[0pt]
Faut-il opposer science et croyance ? \\[0pt]
Faut-il opposer science et métaphysique ? \\[0pt]
Faut-il opposer sciences humaines et sciences de la nature ? \\[0pt]
Faut-il opposer sens et vérité ? \\[0pt]
Faut-il oublier le passé ? \\[0pt]
Faut-il pardonner ? \\[0pt]
Faut-il parler pour avoir des idées générales ? \\[0pt]
Faut-il penser l'État comme un corps ? \\[0pt]
Faut-il penser l'État comme un individu ? \\[0pt]
Faut-il perdre ses illusions ? \\[0pt]
Faut-il préférer la liberté à l'égalité ? \\[0pt]
Faut-il préférer le bonheur à la vérité ? \\[0pt]
Faut-il préférer l'injustice au désordre ? \\[0pt]
Faut-il préférer une injustice au désordre ? \\[0pt]
Faut-il prendre soin de soi ? \\[0pt]
Faut-il que le réel ait un sens ? \\[0pt]
Faut-il que les meilleurs gouvernent ? \\[0pt]
Faut-il rechercher la certitude ? \\[0pt]
Faut-il renoncer à connaître la nature des choses ? \\[0pt]
Faut-il renoncer à croire ? \\[0pt]
Faut-il renoncer à la distinction entre nature et culture ? \\[0pt]
Faut-il renoncer à l'idée de fondement ? \\[0pt]
Faut-il renoncer à son désir ? \\[0pt]
Faut-il respecter la nature ? \\[0pt]
Faut-il respecter les convenances ? \\[0pt]
Faut-il restaurer les œuvres d'art ? \\[0pt]
Faut-il rompre avec le passé ? \\[0pt]
Faut-il s'adapter ? \\[0pt]
Faut-il s'adapter aux événements ? \\[0pt]
Faut-il s'aimer pour vivre ensemble ? \\[0pt]
Faut-il s'attendre à l'inattendu ? \\[0pt]
Faut-il sauver les apparences ? \\[0pt]
Faut-il savoir ce que l'on veut ? \\[0pt]
Faut-il savoir désobéir ? \\[0pt]
Faut-il se chercher pour se trouver ? \\[0pt]
Faut-il se délivrer des passions ? \\[0pt]
Faut-il se demander si l'homme est bon ou méchant par nature \\[0pt]
Faut-il se fier à la majorité ? \\[0pt]
Faut-il se méfier de l'imagination ? \\[0pt]
Faut-il se méfier des images ? \\[0pt]
Faut-il se méfier du volontarisme politique ? \\[0pt]
Faut-il s'en tenir aux faits ? \\[0pt]
Faut-il séparer morale et politique ? \\[0pt]
Faut-il se résigner aux inégalités ? \\[0pt]
Faut-il s'intéresser aux œuvres mineures ? \\[0pt]
Faut-il suivre ses intuitions ? \\[0pt]
Faut-il supposer un énoncé vrai pour en comprendre le sens ? \\[0pt]
Faut-il tolérer les intolérants ? \\[0pt]
Faut-il toujours respecter ses engagements ? \\[0pt]
Faut-il tout démontrer ? \\[0pt]
Faut-il tout interpréter ? \\[0pt]
Faut-il vivre comme si l'on ne devait jamais mourir ? \\[0pt]
Faut-il vivre comme si nous étions immortels ? \\[0pt]
Faut-il vivre conformément à la nature ? \\[0pt]
Faut-il voir pour croire ? \\[0pt]
Faut-il vouloir changer le monde ? \\[0pt]
Faut-il vouloir la paix ? \\[0pt]
Féminité et citoyenneté \\[0pt]
Fiction et mensonge \\[0pt]
Fiction et réalité \\[0pt]
Fiction et vérité \\[0pt]
Figurer le pouvoir \\[0pt]
Finalité et causalité \\[0pt]
Fins et moyens \\[0pt]
Foi et bonne foi \\[0pt]
Foi et raison \\[0pt]
Folie et société \\[0pt]
Fonction et limites des modèles en sciences sociales \\[0pt]
Fonction et prédicat \\[0pt]
Fonder \\[0pt]
Fonder la justice \\[0pt]
Fonder la morale \\[0pt]
Fonder une cité \\[0pt]
Fonder une famille \\[0pt]
Force et violence \\[0pt]
Forcer à être libre \\[0pt]
Forger des hypothèses \\[0pt]
Formaliser et axiomatiser \\[0pt]
Forme et matière \\[0pt]
Forme et rythme \\[0pt]
Former les esprits \\[0pt]
Fuir le monde \\[0pt]
Gagner sa vie \\[0pt]
Garder en mémoire \\[0pt]
Garder la mesure \\[0pt]
Généralité de la règle, contingence des faits \\[0pt]
Généralité et universalité \\[0pt]
Genèse et structure \\[0pt]
Géographie et géométrie \\[0pt]
Gérer et gouverner \\[0pt]
Gestion et politique \\[0pt]
Gouvernement des hommes et administration des choses \\[0pt]
Gouverner \\[0pt]
Gouverner, administrer, gérer \\[0pt]
Gouverner, est-ce dominer ? \\[0pt]
Gouverner, est-ce prévoir ? \\[0pt]
Gouverner et se gouverner \\[0pt]
Gouverner la nature \\[0pt]
Gouverner les passions \\[0pt]
Grammaire et logique \\[0pt]
Grammaire et métaphysique \\[0pt]
Grammaire et philosophie \\[0pt]
Grandeur et décadence \\[0pt]
Grandir \\[0pt]
Grève et politique \\[0pt]
Groupe, classe, société \\[0pt]
Guérir \\[0pt]
Guerre et paix \\[0pt]
Guerre et politique \\[0pt]
Habiter \\[0pt]
Habiter le monde \\[0pt]
Habiter l'espace \\[0pt]
Haïr \\[0pt]
Haïr la raison \\[0pt]
Hasard et destin \\[0pt]
Hériter \\[0pt]
Herméneutique et sciences humaines \\[0pt]
Heureux les simples d'esprit \\[0pt]
Hiérarchiser les arts \\[0pt]
Histoire des techniques et histoire de l'art \\[0pt]
Histoire et anthropologie \\[0pt]
Histoire et écriture \\[0pt]
Histoire et ethnologie \\[0pt]
Histoire et fiction \\[0pt]
Histoire et géographie \\[0pt]
Histoire et géographie font-elles couple ? \\[0pt]
Histoire et mémoire \\[0pt]
Histoire et morale \\[0pt]
Histoire et narration \\[0pt]
Histoire et préhistoire \\[0pt]
Historicité et vérité \\[0pt]
Homo religiosus \\[0pt]
Humanisation et déshumanisation \\[0pt]
Humour et ironie \\[0pt]
Hypothèse et vérité \\[0pt]
Ici et maintenant \\[0pt]
Identité et communauté \\[0pt]
Identité et différence \\[0pt]
Identité et égalité \\[0pt]
Identité et indiscernabilité \\[0pt]
Identité et représentation politique \\[0pt]
Identité et responsabilité \\[0pt]
Idéologie et utopie \\[0pt]
« Il faudrait rester des années entières pour contempler une telle œuvre » \\[0pt]
« Il faut de tout pour faire un monde » \\[0pt]
Il faut de tout pour faire un monde \\[0pt]
Illégalité et injustice \\[0pt]
« Il n'y a pas de fait, il n'y a que des interprétations » \\[0pt]
Il y a \\[0pt]
Image, concept et signe \\[0pt]
Image et idée \\[0pt]
Imaginaire et politique \\[0pt]
Imagination et connaissance \\[0pt]
Imagination et fantasme \\[0pt]
Imagination et perception \\[0pt]
Imaginer \\[0pt]
Imaginer, est-ce créer ? \\[0pt]
Imitation et création \\[0pt]
Imitation et identification \\[0pt]
Imiter, est-ce copier ? \\[0pt]
Imiter, est-ce manquer d'imagination ? \\[0pt]
Immoralité et amoralité \\[0pt]
Immortalité et éternité \\[0pt]
Improviser \\[0pt]
Incarner le pouvoir \\[0pt]
Inconscient et identité \\[0pt]
Inconscient et inconscience \\[0pt]
Inconscient et langage \\[0pt]
Incréé / créé \\[0pt]
Indépendance et autonomie \\[0pt]
Indépendance et liberté \\[0pt]
Indépendant / dépendant \\[0pt]
Individualisme et égoïsme \\[0pt]
Individuation et identité \\[0pt]
Individu et société \\[0pt]
Inégalités et différences \\[0pt]
Inégalités et discriminations \\[0pt]
Infini et indéfini \\[0pt]
Information et communication \\[0pt]
Innocenter le devenir \\[0pt]
Instinct et morale \\[0pt]
Instruire et éduquer \\[0pt]
Intégration et assimilation \\[0pt]
Intégration et exclusion \\[0pt]
Interdire et prohiber \\[0pt]
Intériorité et extériorité \\[0pt]
Interprétation et création \\[0pt]
Interprétation et scientificité \\[0pt]
Interpréter \\[0pt]
Interpréter, est-ce renoncer à prouver ? \\[0pt]
Interpréter, est-ce savoir ? \\[0pt]
Interpréter, est-ce un art ? \\[0pt]
Interpréter et expliquer \\[0pt]
Interpréter et formaliser dans les sciences humaines \\[0pt]
Interpréter ou expliquer \\[0pt]
Interpréter une œuvre d'art \\[0pt]
Intuition et concept \\[0pt]
Intuition et déduction \\[0pt]
Invariance et pluralité des langues \\[0pt]
Invention et création \\[0pt]
« J'ai le droit » \\[0pt]
J'ai un corps \\[0pt]
« Je mens » \\[0pt]
Je mens \\[0pt]
« Je n'ai pas voulu cela » \\[0pt]
Je ne l'ai pas fait exprès \\[0pt]
« Je ne voulais pas cela » : en quoi les sciences humaines permettent-elles de comprendre cette excuse ? \\[0pt]
Je sens, donc je suis \\[0pt]
Je, tu, il \\[0pt]
Jouer \\[0pt]
Jouer son rôle \\[0pt]
Jouer un rôle \\[0pt]
Joue-t-on toujours un rôle ? \\[0pt]
Jouir sans entraves \\[0pt]
Jugement analytique et jugement synthétique \\[0pt]
Jugement esthétique et jugement de valeur \\[0pt]
Jugement esthétique et objectivité \\[0pt]
Jugement et proposition \\[0pt]
Juger \\[0pt]
Juger en conscience \\[0pt]
Juger et connaître \\[0pt]
Juger et raisonner \\[0pt]
Juge-t-on un acte ou son auteur ? \\[0pt]
Jusqu'à quel point pouvons-nous juger autrui ? \\[0pt]
Jusqu'où peut-on soigner ? \\[0pt]
Justice et bonheur \\[0pt]
Justice et égalité \\[0pt]
Justice et équité \\[0pt]
Justice et fiscalité \\[0pt]
Justice et force \\[0pt]
Justice et vengeance \\[0pt]
Justice et violence \\[0pt]
Justice sociale et charité publique \\[0pt]
Justifier \\[0pt]
Justifier et prouver \\[0pt]
Justifier le mensonge \\[0pt]
La banalité \\[0pt]
L'abandon \\[0pt]
La barbarie \\[0pt]
La barbarie de la technique \\[0pt]
La barbarie peut-elle avoir un visage humain ? \\[0pt]
La bassesse \\[0pt]
La béatitude \\[0pt]
La beauté \\[0pt]
La beauté a-t-elle une histoire ? \\[0pt]
La beauté comporte-t-elle des degrés ? \\[0pt]
La beauté de la nature \\[0pt]
La beauté demande-t-elle des preuves ? \\[0pt]
La beauté des corps \\[0pt]
La beauté des ruines \\[0pt]
La beauté des villes \\[0pt]
La beauté du diable \\[0pt]
La beauté du geste \\[0pt]
La beauté du monde \\[0pt]
La beauté du vrai \\[0pt]
La beauté est-elle dans le regard ou dans la chose vue ? \\[0pt]
La beauté est-elle l'objet d'une connaissance ? \\[0pt]
La beauté est-elle partout ? \\[0pt]
La beauté est-elle sensible ? \\[0pt]
La beauté est-elle une promesse de bonheur ? \\[0pt]
La beauté et la grâce \\[0pt]
La beauté et l'ennui \\[0pt]
La beauté idéale \\[0pt]
La beauté morale \\[0pt]
La beauté naturelle \\[0pt]
La beauté naturelle est-elle une catégorie esthétique périmée ? \\[0pt]
La beauté n'est-elle qu'un effet artistique ? \\[0pt]
La beauté n'est-elle qu'une idée ? \\[0pt]
La beauté peut-elle délivrer une vérité ? \\[0pt]
La beauté peut-elle être cachée ? \\[0pt]
La beauté peut-elle laisser indifférent ? \\[0pt]
La belle âme \\[0pt]
La belle nature \\[0pt]
La bestialité \\[0pt]
La bêtise \\[0pt]
La bêtise n'est-elle pas proprement humaine ? \\[0pt]
La bibliothèque \\[0pt]
La bienfaisance \\[0pt]
La bienveillance \\[0pt]
La bioéthique \\[0pt]
La biologie peut-elle se passer de causes finales ? \\[0pt]
L'abnégation \\[0pt]
L'abondance \\[0pt]
La bonne conscience \\[0pt]
La bonne volonté \\[0pt]
L'absence \\[0pt]
L'absence de fondement \\[0pt]
L'absence de générosité \\[0pt]
L'absence de preuves \\[0pt]
L'absence d'œuvre \\[0pt]
L'absolu \\[0pt]
L'absolu est-il connaissable ? \\[0pt]
L'Absolu est-il un objet de savoir ? \\[0pt]
L'abstention \\[0pt]
L'abstraction \\[0pt]
L'abstraction en art \\[0pt]
L'abstrait est-il en dehors de l'espace et du temps ? \\[0pt]
L'abstrait et le concret \\[0pt]
L'abstrait et l'immatériel \\[0pt]
L'absurde \\[0pt]
L'absurde peut-il avoir un sens pour une pensée 155 \\[0pt]
L'abus de langage \\[0pt]
L'abus de pouvoir \\[0pt]
L'académisme \\[0pt]
L'académisme dans l'art \\[0pt]
La calomnie \\[0pt]
La caricature \\[0pt]
La carte et le territoire \\[0pt]
La casuistique \\[0pt]
La catastrophe \\[0pt]
La catharsis \\[0pt]
La causalité \\[0pt]
La causalité dans les sciences humaines \\[0pt]
La causalité en histoire \\[0pt]
La causalité en sciences humaines \\[0pt]
La causalité historique \\[0pt]
La causalité suppose-t-elle des lois ? \\[0pt]
La cause \\[0pt]
La cause de soi \\[0pt]
La cause efficiente \\[0pt]
La cause et la raison \\[0pt]
La cause finale \\[0pt]
La cause formelle \\[0pt]
La cause motrice \\[0pt]
La cause première \\[0pt]
L'accélération du temps \\[0pt]
L'accident \\[0pt]
L'accidentel \\[0pt]
L'accomplissement \\[0pt]
L'accomplissement de soi \\[0pt]
L'accord \\[0pt]
L'accord des esprits \\[0pt]
L'accusation \\[0pt]
La censure \\[0pt]
La certitude \\[0pt]
La chair \\[0pt]
La chair et l'esprit \\[0pt]
La chance \\[0pt]
La charité \\[0pt]
La charité est-elle une vertu ? \\[0pt]
La chasse et la guerre \\[0pt]
L'achèvement de l'œuvre \\[0pt]
La chose \\[0pt]
La chose en soi \\[0pt]
La chose et l'objet \\[0pt]
La chose publique \\[0pt]
La chronologie \\[0pt]
La chute \\[0pt]
La circonspection \\[0pt]
La circonstance \\[0pt]
La citation \\[0pt]
La cité idéale \\[0pt]
La citoyenneté \\[0pt]
La civilisation \\[0pt]
La civilité \\[0pt]
La clarté \\[0pt]
La clarté suffit-elle au savoir ? \\[0pt]
La classification \\[0pt]
La classification des arts \\[0pt]
La classification des sciences \\[0pt]
La clause de conscience \\[0pt]
La clémence \\[0pt]
La cohérence \\[0pt]
La cohérence des attributs divins \\[0pt]
La cohérence est-elle un critère de la vérité ? \\[0pt]
La cohésion nationale \\[0pt]
La cohésion sociale \\[0pt]
La colère \\[0pt]
La comédie du pouvoir \\[0pt]
La comédie humaine \\[0pt]
La comédie sociale \\[0pt]
La communauté des individus \\[0pt]
La communauté des savants \\[0pt]
La communauté internationale \\[0pt]
La communauté morale \\[0pt]
La communauté scientifique \\[0pt]
La communication \\[0pt]
La communication est-elle nécessaire à la démocratie ? \\[0pt]
La comparaison \\[0pt]
La comparaison en sociologie \\[0pt]
La compassion \\[0pt]
La compassion risque-t-elle d'abolir l'exigence politique ? \\[0pt]
La compétence \\[0pt]
La compétence fonde-t-elle la compétence juridique ? \\[0pt]
La compétence fonde-t-elle l'autorité politique ? \\[0pt]
La compétence politique \\[0pt]
La compétence technique peut-elle fonder l'autorité publique ? \\[0pt]
La composition \\[0pt]
La compréhension \\[0pt]
La concorde \\[0pt]
La concurrence \\[0pt]
La condition \\[0pt]
La condition humaine \\[0pt]
La condition sociale \\[0pt]
La confiance \\[0pt]
La conflictualité politique est-elle indépassable ? \\[0pt]
La confusion \\[0pt]
La connaissance adéquate \\[0pt]
La connaissance animale \\[0pt]
La connaissance a-t-elle des limites ? \\[0pt]
La connaissance commune est-elle le point de départ de la science ? \\[0pt]
La connaissance de l'amour \\[0pt]
La connaissance de la nécessité a priori peut-elle évoluer ? \\[0pt]
La connaissance de l'animal \\[0pt]
La connaissance de la vie \\[0pt]
La connaissance de la vie se confond-elle avec celle du vivant ? \\[0pt]
La connaissance de l'histoire est-elle utile à l'action ? \\[0pt]
La connaissance de l'histoire nous rend-elle plus libres ? \\[0pt]
La connaissance de l'individuel \\[0pt]
La connaissance de l'infini \\[0pt]
La connaissance des causes \\[0pt]
La connaissance de soi \\[0pt]
La connaissance des principes \\[0pt]
La connaissance du futur \\[0pt]
La connaissance du singulier \\[0pt]
La connaissance du singulier dans les sciences humaines \\[0pt]
La connaissance du vivant \\[0pt]
La connaissance est-elle une croyance justifiée ? \\[0pt]
La connaissance mathématique \\[0pt]
La connaissance objective \\[0pt]
La connaissance scientifique abolit-elle toute croyance ? \\[0pt]
La connaissance scientifique est-elle désintéressée ? \\[0pt]
La connaissance scientifique n'est-elle qu'une croyance argumentée ? \\[0pt]
La connaissance suppose-t-elle une éthique ? \\[0pt]
La connexion des choses et la connexion des idées \\[0pt]
La conquête \\[0pt]
La conquête de l'espace \\[0pt]
La conscience de classe \\[0pt]
La conscience définit-elle l'homme en propre ? \\[0pt]
La conscience des autres \\[0pt]
La conscience de soi \\[0pt]
La conscience de soi de l'art \\[0pt]
La conscience entrave-t-elle l'action ? \\[0pt]
La conscience est-elle ce qui fait le sujet ? \\[0pt]
La conscience est-elle intrinsèquement morale ? \\[0pt]
La conscience est-elle le produit de la vie réelle ? \\[0pt]
La conscience est-elle ou n'est-elle pas ? \\[0pt]
La conscience est-elle temporelle ? \\[0pt]
La conscience historique \\[0pt]
La conscience malheureuse \\[0pt]
La conscience morale \\[0pt]
La conscience morale est-elle innée ? \\[0pt]
La conscience nous leurre-t-elle ? \\[0pt]
La conscience peut-elle être collective ? \\[0pt]
La conscience peut-elle être objet de science ? \\[0pt]
La conscience politique \\[0pt]
La conséquence \\[0pt]
La conservation de soi \\[0pt]
La considération \\[0pt]
La consolation \\[0pt]
La consommation \\[0pt]
La constance \\[0pt]
La constitution \\[0pt]
La contemplation \\[0pt]
La contestation \\[0pt]
La contingence \\[0pt]
La contingence des lois de la nature \\[0pt]
La contingence du futur \\[0pt]
La contingence du mal \\[0pt]
La continuité \\[0pt]
La contradiction \\[0pt]
La contradiction réside-t-elle dans les choses ? \\[0pt]
La contrainte \\[0pt]
La contrainte des lois est-elle une violence ? \\[0pt]
La contrainte morale \\[0pt]
La contrainte supprime-t-elle la responsabilité ? \\[0pt]
La contrôle social \\[0pt]
La controverse \\[0pt]
La convention \\[0pt]
La conversation \\[0pt]
La conversion \\[0pt]
La conviction \\[0pt]
La coopération \\[0pt]
La correspondance des arts \\[0pt]
La corruption \\[0pt]
La corruption politique \\[0pt]
La cosmogonie \\[0pt]
La couleur \\[0pt]
La couleur en peinture \\[0pt]
La couleur et le dessin \\[0pt]
La coutume \\[0pt]
L'acquis \\[0pt]
La crainte \\[0pt]
La crainte des Dieux \\[0pt]
La crainte et l'ignorance \\[0pt]
La création \\[0pt]
La création artistique \\[0pt]
La création dans l'art \\[0pt]
La création de l'humanité \\[0pt]
La créativité \\[0pt]
La criminalité \\[0pt]
La crise du sens \\[0pt]
La crise sociale \\[0pt]
La critique \\[0pt]
La critique d'art \\[0pt]
La critique de l'art \\[0pt]
La critique du pouvoir peut-elle conduire à la désobéissance ? \\[0pt]
La critique et la crise \\[0pt]
La croissance \\[0pt]
La croyance en Dieu est-elle irrationnelle ? \\[0pt]
La croyance est-elle l'asile de l'ignorance ? \\[0pt]
La croyance est-elle signe de faiblesse ? \\[0pt]
La croyance est-elle une opinion ? \\[0pt]
La croyance est-elle une opinion comme les autres ? \\[0pt]
La croyance peut-elle être rationnelle ? \\[0pt]
La cruauté \\[0pt]
La cruauté politique \\[0pt]
L'acte \\[0pt]
L'acte gratuit \\[0pt]
L'acteur \\[0pt]
L'acteur et l'observateur dans les sciences humaines \\[0pt]
L'acteur et son rôle \\[0pt]
L'action collective \\[0pt]
L'action du temps \\[0pt]
L'action et la passion \\[0pt]
L'action politique \\[0pt]
L'action politique a-t-elle un fondement rationnel ? \\[0pt]
L'action politique peut-elle se passer de mots ? \\[0pt]
L'action requiert-elle décision d'un sujet ? \\[0pt]
L'activité philosophique conduit-elle à la métaphysique ? \\[0pt]
L'actualité \\[0pt]
L'actuel \\[0pt]
La cuisine \\[0pt]
La culpabilité \\[0pt]
La culture artistique \\[0pt]
La culture de masse \\[0pt]
La culture démocratique \\[0pt]
La culture d'entreprise \\[0pt]
La culture est-elle affaire de politique ? \\[0pt]
La culture est-elle ce qui nous relie au passé ? \\[0pt]
La culture est-elle l'objet naturel des sciences humaines ? \\[0pt]
La culture est-elle nécessaire à l'appréciation d'une œuvre d'art ? \\[0pt]
La culture est-elle une seconde nature ? \\[0pt]
La culture est-elle un objet ? \\[0pt]
La culture et les cultures \\[0pt]
La culture générale \\[0pt]
La culture libère-t-elle des préjugés ? \\[0pt]
La culture morale \\[0pt]
La culture nous rend-elle meilleurs ? \\[0pt]
La culture peut-elle être instituée ? \\[0pt]
La culture peut-elle être objet de science ? \\[0pt]
La culture populaire \\[0pt]
La culture scientifique \\[0pt]
La curiosité \\[0pt]
La curiosité est-elle à l'origine du savoir ? \\[0pt]
La danse \\[0pt]
La danse est-elle l'œuvre du corps ? \\[0pt]
La danse et l'espace \\[0pt]
L'adaptation \\[0pt]
La décadence \\[0pt]
La décence \\[0pt]
La déception \\[0pt]
La décision \\[0pt]
La décision morale \\[0pt]
La décision politique \\[0pt]
La découverte scientifique a-t-elle une logique ? \\[0pt]
La déduction \\[0pt]
La défense nationale \\[0pt]
La déficience \\[0pt]
La définition \\[0pt]
La délibération \\[0pt]
La délibération en morale \\[0pt]
La délibération politique \\[0pt]
La démagogie \\[0pt]
La démence \\[0pt]
La démesure \\[0pt]
La démocratie conduit-elle au règne de l'opinion ? \\[0pt]
La démocratie est-elle le pire des régimes politiques ? \\[0pt]
La démocratie est-elle moyen ou fin ? \\[0pt]
La démocratie est-elle nécessairement libérale ? \\[0pt]
La démocratie est-elle possible ? \\[0pt]
La démocratie est-elle un mythe ? \\[0pt]
La démocratie et les experts \\[0pt]
La démocratie n'est-elle que la force des faibles ? \\[0pt]
La démocratie participative \\[0pt]
La démocratie peut-elle se passer de représentation ? \\[0pt]
La démonstration \\[0pt]
La démonstration obéit-elle à des lois ? \\[0pt]
La démonstration par l'absurde \\[0pt]
La démonstration suffit-elle à établir la vérité ? \\[0pt]
La déontologie \\[0pt]
La dépendance \\[0pt]
La déraison \\[0pt]
La descendance \\[0pt]
La description \\[0pt]
La désillusion \\[0pt]
La désinvolture \\[0pt]
La désobéissance \\[0pt]
La désobéissance civile \\[0pt]
La destruction \\[0pt]
La détermination \\[0pt]
La dette \\[0pt]
La déviance \\[0pt]
La dialectique \\[0pt]
La dialectique est-elle une science ? \\[0pt]
La dictature \\[0pt]
La différence \\[0pt]
La différence de l'être et de l'étant \\[0pt]
La différence des arts \\[0pt]
La différence des sexes \\[0pt]
La différence sexuelle \\[0pt]
La difformité \\[0pt]
La dignité \\[0pt]
La dignité de l'homme \\[0pt]
La dignité humaine \\[0pt]
La digression \\[0pt]
La discipline \\[0pt]
La discipline de vie \\[0pt]
La discrétion \\[0pt]
La discrimination \\[0pt]
La discursivité \\[0pt]
La discussion \\[0pt]
La disposition \\[0pt]
La disposition. Le possible et le réel \\[0pt]
La disposition morale \\[0pt]
La distance \\[0pt]
La distinction \\[0pt]
La distinction de genre \\[0pt]
La distinction de la nature et de la culture est-elle un fait de culture ? \\[0pt]
La distinction du sacré et du profane \\[0pt]
La distinction entre vraie et fausse sciences a-t-elle un sens ? \\[0pt]
La distinction sociale \\[0pt]
La distraction \\[0pt]
La diversion \\[0pt]
La diversité des cultures \\[0pt]
La diversité des langues \\[0pt]
La diversité des perceptions \\[0pt]
La diversité des religions \\[0pt]
La diversité des sciences \\[0pt]
La diversité humaine \\[0pt]
La division \\[0pt]
La division des pouvoirs \\[0pt]
La division des tâches \\[0pt]
La division du travail \\[0pt]
L'admiration \\[0pt]
La docilité est-elle un vice ou une vertu ? \\[0pt]
La domestication \\[0pt]
La domination \\[0pt]
La domination du corps \\[0pt]
La domination sociale \\[0pt]
La douceur \\[0pt]
L'adoucissement des mœurs \\[0pt]
La douleur \\[0pt]
La droit de conquête \\[0pt]
La droiture \\[0pt]
La dualité \\[0pt]
La duplicité \\[0pt]
La durée \\[0pt]
La faiblesse \\[0pt]
La faiblesse de la démocratie \\[0pt]
La faiblesse de la volonté \\[0pt]
La faiblesse d'esprit \\[0pt]
La familiarité \\[0pt]
La famille \\[0pt]
La famille est-elle le lieu de la formation morale ? \\[0pt]
La famille est-elle une invention culturelle ? \\[0pt]
La fatalité \\[0pt]
La fatigue \\[0pt]
La faute \\[0pt]
La faute de goût \\[0pt]
La faute et l'erreur \\[0pt]
La fécondité du raisonnement mathématique \\[0pt]
La fête \\[0pt]
La fête nationale \\[0pt]
L'affection \\[0pt]
La fiction \\[0pt]
La fidélité \\[0pt]
La fidélité à soi \\[0pt]
La figuration \\[0pt]
La figure de l'ennemi \\[0pt]
La fin \\[0pt]
La finalité \\[0pt]
La finalité des sciences humaines \\[0pt]
La fin de la métaphysique \\[0pt]
La fin de la politique \\[0pt]
La fin de la politique est-elle d'éliminer la violence ? \\[0pt]
La fin de la politique est-elle l'établissement de la justice ? \\[0pt]
La fin de l'art \\[0pt]
La fin de la vie \\[0pt]
La fin de l'État \\[0pt]
La fin de l'histoire \\[0pt]
La fin des désirs \\[0pt]
La fin des guerres \\[0pt]
La fin des temps \\[0pt]
La fin du monde \\[0pt]
La fin du travail \\[0pt]
La fin et les moyens \\[0pt]
La finitude \\[0pt]
La fin justifie-t-elle les moyens ? \\[0pt]
La foi \\[0pt]
La folie \\[0pt]
La folie des grandeurs \\[0pt]
La folie et l'étude du psychisme \\[0pt]
La fonction de l'art \\[0pt]
La fonction de penser peut-elle se déléguer ? \\[0pt]
La fonction des exemples \\[0pt]
La fonction des rêves \\[0pt]
La fonction du philosophe est-elle de diriger l'État ? \\[0pt]
La fonction première de l'État est-elle de durer ? \\[0pt]
La fonction sociale de l'écrivain \\[0pt]
La fonction symbolique \\[0pt]
La force \\[0pt]
La force d'âme \\[0pt]
La force de la loi \\[0pt]
La force de la morale \\[0pt]
La force de l'art \\[0pt]
La force de la vérité \\[0pt]
La force de l'expérience \\[0pt]
La force de l'habitude \\[0pt]
La force de l'idée \\[0pt]
La force de l'inconscient \\[0pt]
La force de l'obligation \\[0pt]
La force de l'oubli \\[0pt]
La force des choses \\[0pt]
La force des idées \\[0pt]
La force des institutions \\[0pt]
La force des lois \\[0pt]
La force des objections \\[0pt]
La force du droit \\[0pt]
La force du génie \\[0pt]
La force du pouvoir \\[0pt]
La force du social \\[0pt]
La force du vrai \\[0pt]
La force est-elle une vertu ? \\[0pt]
La force et le droit \\[0pt]
La force fait-elle le droit ? \\[0pt]
La force publique \\[0pt]
La formation de l'esprit \\[0pt]
La formation des citoyens \\[0pt]
La formation du goût \\[0pt]
La formation d'une conscience \\[0pt]
La forme \\[0pt]
La forme du droit \\[0pt]
La forme et la couleur \\[0pt]
La fortune \\[0pt]
La foule \\[0pt]
La fragilité \\[0pt]
La fragilité du beau \\[0pt]
La franchise \\[0pt]
La franchise est-elle une vertu ? \\[0pt]
La fraternité \\[0pt]
La fraternité a-t-elle un sens politique ? \\[0pt]
La fraternité est-elle un idéal moral ? \\[0pt]
La fraternité peut-elle se passer d'un fondement religieux ? \\[0pt]
La fraude \\[0pt]
La frivolité \\[0pt]
La frontière \\[0pt]
La fuite du temps \\[0pt]
La futilité \\[0pt]
L'âge d'or \\[0pt]
La généalogie \\[0pt]
La généalogie peut-elle être une méthode historique ? \\[0pt]
La généralité \\[0pt]
La générosité \\[0pt]
La genèse \\[0pt]
La genèse de l'œuvre \\[0pt]
L'agent \\[0pt]
La gentillesse \\[0pt]
La géographie \\[0pt]
La géographie est-elle une science humaine ? \\[0pt]
La géométrie \\[0pt]
La géopolitique \\[0pt]
La gloire \\[0pt]
La grâce \\[0pt]
La grammaire \\[0pt]
La grammaire contraint-elle la pensée ? \\[0pt]
La grammaire contraint-elle notre pensée ? \\[0pt]
La grammaire et la logique \\[0pt]
La grammaire générative \\[0pt]
La grammaire véhicule-t-elle une métaphysique ? \\[0pt]
La grandeur \\[0pt]
La grandeur d'âme \\[0pt]
La grandeur d'une culture \\[0pt]
La gratitude \\[0pt]
La gratuité \\[0pt]
La gravité \\[0pt]
L'agressivité \\[0pt]
La grève \\[0pt]
L'agriculture \\[0pt]
La guérison \\[0pt]
La guérison en psychanalyse \\[0pt]
La guerre civile \\[0pt]
La guerre est-elle la continuation de la politique ? \\[0pt]
La guerre est-elle la continuation de la politique par d'autres moyens ? \\[0pt]
La guerre est-elle la plus grande violence ? \\[0pt]
La guerre et la paix \\[0pt]
La guerre juste \\[0pt]
La guerre met-elle fin au droit ? \\[0pt]
La guerre peut-elle être juste ? \\[0pt]
La guerre totale \\[0pt]
La haine \\[0pt]
La haine de la pensée \\[0pt]
La haine de la raison \\[0pt]
La haine de la vérité \\[0pt]
La haine des images \\[0pt]
La haine des machines \\[0pt]
La haine de soi \\[0pt]
La hiérarchie \\[0pt]
La hiérarchie des arts \\[0pt]
La hiérarchie des cultures \\[0pt]
La hiérarchie des énoncés scientifiques \\[0pt]
La hiérarchie des êtres \\[0pt]
La honte \\[0pt]
Laisser faire \\[0pt]
La jalousie \\[0pt]
La jeunesse \\[0pt]
La joie \\[0pt]
La joie de vivre \\[0pt]
La jouissance \\[0pt]
La jurisprudence \\[0pt]
La juste colère \\[0pt]
La juste mesure \\[0pt]
La juste peine \\[0pt]
La justice \\[0pt]
La justice a-t-elle besoin des institutions ? \\[0pt]
La justice a-t-elle pour fonction de compenser les inégalités naturelles ? \\[0pt]
La justice comme avantage mutuel \\[0pt]
La justice consiste-t-elle à traiter tout le monde de la même manière ? \\[0pt]
La justice consiste-t-elle dans l'application de la loi ? \\[0pt]
La justice de l'État \\[0pt]
La justice divine \\[0pt]
La justice entre les générations \\[0pt]
La justice est-elle affaire de raison ? \\[0pt]
La justice est-elle de ce monde ? \\[0pt]
La justice est-elle le ciment de la paix sociale ? \\[0pt]
La justice est-elle une notion morale ? \\[0pt]
La justice est-elle une vertu ? \\[0pt]
La justice exclut-elle l'arbitraire ? \\[0pt]
La justice : moyen ou fin de la politique ? \\[0pt]
La justice peut-elle se passer de la force ? \\[0pt]
La justice peut-elle se passer d'institutions ? \\[0pt]
La justice requiert-elle toujours l'impartialité ? \\[0pt]
La justice sociale \\[0pt]
La justification \\[0pt]
La lâcheté \\[0pt]
La laïcité \\[0pt]
La laideur \\[0pt]
La laideur est-elle une valeur esthétique ? \\[0pt]
La langue de la raison \\[0pt]
La langue maternelle \\[0pt]
La langue nous parle-t-elle ? \\[0pt]
La langue officielle \\[0pt]
L'aléatoire \\[0pt]
La leçon des choses \\[0pt]
La lecture \\[0pt]
La légende \\[0pt]
La légèreté \\[0pt]
La légitimation \\[0pt]
La légitimité \\[0pt]
La légitimité de l'État \\[0pt]
La légitimité d'un clonage humain ? \\[0pt]
La lettre et l'esprit \\[0pt]
La liaison des idées \\[0pt]
La libération des mœurs \\[0pt]
La liberté artistique \\[0pt]
La liberté a-t-elle une histoire ? \\[0pt]
La liberté civile \\[0pt]
La liberté comporte-t-elle des degrés ? \\[0pt]
La liberté, concept moral ? \\[0pt]
La liberté créatrice \\[0pt]
La liberté de culte \\[0pt]
La liberté de l'artiste \\[0pt]
La liberté de la science \\[0pt]
La liberté de parole \\[0pt]
La liberté de penser \\[0pt]
La liberté des autres \\[0pt]
La liberté des citoyens \\[0pt]
La liberté d'expression \\[0pt]
La liberté d'expression a-t-elle des limites ? \\[0pt]
La liberté d'imaginer \\[0pt]
La liberté d'obéir \\[0pt]
La liberté doit-elle se conquérir ? \\[0pt]
La liberté d'opinion \\[0pt]
La liberté du citoyen \\[0pt]
La liberté du savant \\[0pt]
La liberté, est-ce l'indépendance à l'égard des passions ? \\[0pt]
La liberté est-elle la condition du bonheur ? \\[0pt]
La liberté est-elle menacée par l'égalité ? \\[0pt]
La liberté est-elle un fait ? \\[0pt]
La liberté et la grâce \\[0pt]
La liberté et la loi \\[0pt]
La liberté implique-t-elle l'indifférence ? \\[0pt]
La liberté individuelle \\[0pt]
La liberté intéresse-t-elle les sciences humaines ? \\[0pt]
La liberté morale \\[0pt]
La liberté peut-elle faire peur ? \\[0pt]
La liberté peut-elle se constater ? \\[0pt]
La liberté peut-elle se prouver ? \\[0pt]
La liberté peut-elle se refuser ? \\[0pt]
La liberté politique \\[0pt]
La liberté se prouve-t-elle ? \\[0pt]
La liberté s'éprouve-t-elle ? \\[0pt]
La liberté se réduit-elle au libre arbitre ? \\[0pt]
L'aliénation \\[0pt]
L'aliéné \\[0pt]
La limitation des créatures \\[0pt]
La limite \\[0pt]
La limite du politique \\[0pt]
La linguistique est-elle une science de l'information ? \\[0pt]
La linguistique fait-elle partie des sciences humaines ? \\[0pt]
La linguistique peut-elle se passer de la psychologie ? \\[0pt]
La littérature engagée \\[0pt]
La littérature est-elle la mémoire de l'humanité ? \\[0pt]
La littérature et le mythe \\[0pt]
La littérature peut-elle suppléer les sciences de l'homme ? \\[0pt]
L'allégorie \\[0pt]
La logique a-t-elle une histoire ? \\[0pt]
La logique a-t-elle un intérêt philosophique ? \\[0pt]
La logique : calcul ou langage ? \\[0pt]
La logique : découverte ou invention ? \\[0pt]
La logique décrit-elle le monde ? \\[0pt]
La logique du sens \\[0pt]
La logique est-elle indépendante de la psychologie ? \\[0pt]
La logique est-elle la norme du vrai ? \\[0pt]
La logique est-elle l'art de penser ? \\[0pt]
La logique est-elle un art de penser ? \\[0pt]
La logique est-elle un art de raisonner ? \\[0pt]
La logique est-elle une discipline normative ? \\[0pt]
La logique est-elle une forme de calcul ? \\[0pt]
La logique est-elle une science de la vérité ? \\[0pt]
La logique est-elle utile à la métaphysique ? \\[0pt]
La logique et le réel \\[0pt]
La logique nous apprend-elle quelque chose sur le langage ordinaire ? \\[0pt]
« La logique » ou bien « les logiques » ? \\[0pt]
La logique peut-elle se passer de la métaphysique ? \\[0pt]
La logique pourrait-elle nous surprendre ? \\[0pt]
La loi \\[0pt]
La loi du désir \\[0pt]
La loi du genre \\[0pt]
La loi du marché \\[0pt]
La loi éduque-t-elle ? \\[0pt]
La loi et le contrat \\[0pt]
La loi et le règlement \\[0pt]
La loi peut-elle changer les mœurs ? \\[0pt]
La louange et le blâme \\[0pt]
La loyauté \\[0pt]
L'alter ego \\[0pt]
L'altérité \\[0pt]
L'altérité culturelle \\[0pt]
L'alternative \\[0pt]
L'altruisme \\[0pt]
L'altruisme est-il une attitude morale ? \\[0pt]
La lumière \\[0pt]
La lumière de la vérité \\[0pt]
La lumière naturelle \\[0pt]
La lutte des classes \\[0pt]
La machine \\[0pt]
La magie \\[0pt]
La magnanimité \\[0pt]
La main \\[0pt]
La main et l'outil \\[0pt]
La maîtrise \\[0pt]
La maîtrise de la langue \\[0pt]
La maîtrise de la nature \\[0pt]
La maîtrise de soi \\[0pt]
La maîtrise du feu \\[0pt]
La maîtrise du geste \\[0pt]
La maîtrise du temps \\[0pt]
La majesté \\[0pt]
La majorité \\[0pt]
La majorité a-t-elle toujours raison ? \\[0pt]
La majorité peut-elle être tyrannique ? \\[0pt]
La maladie \\[0pt]
La maladie mentale a-t-elle une histoire ? \\[0pt]
La malchance \\[0pt]
La manière \\[0pt]
La manifestation \\[0pt]
La marchandisation \\[0pt]
La marginalité \\[0pt]
La masse \\[0pt]
L'amateur \\[0pt]
L'amateur d'art \\[0pt]
L'amateurisme \\[0pt]
La mathématique est-elle une ontologie ? \\[0pt]
La matière \\[0pt]
La matière de la pensée \\[0pt]
La matière de l'œuvre \\[0pt]
La matière, est-ce le mal ? \\[0pt]
La matière, est-ce l'informe ? \\[0pt]
La matière est-elle amorphe ? \\[0pt]
La matière est-elle une substance ? \\[0pt]
La matière est-elle une vue de l'esprit ? \\[0pt]
La matière et la forme \\[0pt]
La matière n'est-elle qu'une idée ? \\[0pt]
La matière n'est-elle qu'un obstacle ? \\[0pt]
La matière pense-t-elle ? \\[0pt]
La matière peut-elle être objet de connaissance ? \\[0pt]
La matière peut-elle penser ? \\[0pt]
La matière première \\[0pt]
La matière sensible \\[0pt]
La matière vivante \\[0pt]
La maturité \\[0pt]
La mauvaise conscience \\[0pt]
La mauvaise éducation \\[0pt]
La mauvaise foi \\[0pt]
La mauvaise volonté \\[0pt]
L'ambiguïté \\[0pt]
L'ambition \\[0pt]
L'ambition politique \\[0pt]
L'âme \\[0pt]
La méchanceté \\[0pt]
L'âme concerne-t-elle les sciences humaines ? \\[0pt]
La méconnaissance \\[0pt]
La méconnaissance de soi \\[0pt]
La médecine est-elle une science ? \\[0pt]
L'âme des bêtes \\[0pt]
La médiation \\[0pt]
La médiocrité artistique \\[0pt]
La médisance \\[0pt]
La méditation \\[0pt]
L'âme est-elle immortelle ? \\[0pt]
L'âme et le corps \\[0pt]
La méfiance \\[0pt]
La meilleure constitution \\[0pt]
La mélancolie \\[0pt]
L'âme, le monde et Dieu \\[0pt]
La mémoire \\[0pt]
La mémoire collective \\[0pt]
La mémoire et l'histoire \\[0pt]
La mémoire et l'individu \\[0pt]
La mémoire sélective \\[0pt]
La menace \\[0pt]
La mesure \\[0pt]
La mesure dans les sciences humaines \\[0pt]
La mesure de l'homme \\[0pt]
La mesure de l'intelligence \\[0pt]
La mesure des choses \\[0pt]
La mesure du temps \\[0pt]
La métaphore \\[0pt]
La métaphysique a-t-elle ses fictions ? \\[0pt]
La métaphysique est-elle affaire de raisonnement ? \\[0pt]
La métaphysique est-elle le fondement de la morale ? \\[0pt]
La métaphysique est-elle nécessairement une réflexion sur Dieu ? \\[0pt]
La métaphysique est-elle une discipline théorique ? \\[0pt]
La métaphysique est-elle une science ? \\[0pt]
La métaphysique peut-elle être autre chose qu'une science recherchée ? \\[0pt]
La métaphysique peut-elle faire appel à l'expérience ? \\[0pt]
La métaphysique procure-t-elle un savoir ? \\[0pt]
La métaphysique relève-t-elle de la philosophie ou de la poésie ? \\[0pt]
La métaphysique répond-elle à un besoin ? \\[0pt]
La métaphysique repose-t-elle sur des croyances ? \\[0pt]
La métaphysique se définit-elle par son objet ou sa démarche ? \\[0pt]
La méthode \\[0pt]
La méthode de la science \\[0pt]
La méthode en sciences sociales \\[0pt]
La méthode hypothético-déductive \\[0pt]
La méthode statistique \\[0pt]
L'ami et l'ennemi \\[0pt]
La minorité \\[0pt]
La misanthropie \\[0pt]
La mise à l'épreuve des hypothèses en sciences humaines et sociales \\[0pt]
La mise en scène \\[0pt]
La misère \\[0pt]
La miséricorde \\[0pt]
La misologie \\[0pt]
L'amitié \\[0pt]
L'amitié est-elle une vertu ? \\[0pt]
L'amitié est-elle un principe politique ? \\[0pt]
L'amitié peut-elle obliger ? \\[0pt]
La modalité \\[0pt]
La mode \\[0pt]
La modélisation en sciences sociales \\[0pt]
La modération \\[0pt]
La modération est-elle l'essence de la vertu ? \\[0pt]
La modération est-elle une vertu politique ? \\[0pt]
La modernité \\[0pt]
La modernité dans les arts \\[0pt]
La modestie \\[0pt]
La mondialisation \\[0pt]
La monnaie \\[0pt]
La monumentalité \\[0pt]
La morale a-t-elle besoin d'être fondée ? \\[0pt]
La morale a-t-elle besoin d'un au-delà ? \\[0pt]
La morale a-t-elle besoin d'un fondement ? \\[0pt]
La morale commune \\[0pt]
La morale consiste-t-elle à suivre la nature ? \\[0pt]
La morale de l'athée \\[0pt]
La morale de l'intérêt \\[0pt]
La morale des fables \\[0pt]
La morale doit-elle en appeler à la nature ? \\[0pt]
La morale doit-elle fournir des préceptes ? \\[0pt]
La morale du citoyen \\[0pt]
La morale du plus fort \\[0pt]
La morale est-elle affaire de jugement ? \\[0pt]
La morale est-elle affaire de sentiment ? \\[0pt]
La morale est-elle affaire de sentiments ? \\[0pt]
La morale est-elle affaire individuelle ? \\[0pt]
La morale est-elle émotion ? \\[0pt]
La morale est-elle ennemie du bonheur ? \\[0pt]
La morale est-elle fondée sur la liberté ? \\[0pt]
La morale est-elle incompatible avec le déterminisme ? \\[0pt]
La morale est-elle l'ennemie de la vie ? \\[0pt]
La morale est-elle nécessairement répressive ? \\[0pt]
La morale est-elle un art de vivre ? \\[0pt]
La morale est-elle un besoin ? \\[0pt]
La morale est-elle une affaire d'habitude ? \\[0pt]
La morale est-elle une médecine de l'âme ? \\[0pt]
La morale est-elle une preuve de la liberté ? \\[0pt]
La morale est-elle une science ? \\[0pt]
La morale est-elle un fait social ? \\[0pt]
La morale et le droit \\[0pt]
La morale peut-elle être fondée sur la science ? \\[0pt]
La morale peut-elle être naturelle ? \\[0pt]
La morale peut-elle être personnelle ? \\[0pt]
La morale peut-elle être un calcul ? \\[0pt]
La morale peut-elle être une science ? \\[0pt]
La morale peut-elle se passer d'un fondement religieux ? \\[0pt]
La morale politique \\[0pt]
La morale requiert-elle une casuistique ? \\[0pt]
La morale requiert-elle un fondement ? \\[0pt]
La morale se déduit-elle de la métaphysique ? \\[0pt]
La morale s'enracine-t-elle dans le sens commun ? \\[0pt]
La morale suppose-t-elle le libre arbitre ? \\[0pt]
La moralité des lois \\[0pt]
La moralité n'est-elle que dressage ? \\[0pt]
La moralité réside-t-elle dans l'intention ? \\[0pt]
La mort dans l'âme \\[0pt]
La mort de Dieu \\[0pt]
La mort de l'art \\[0pt]
La mort fait-elle partie de la vie ? \\[0pt]
La motivation et le jugement moral \\[0pt]
L'amour \\[0pt]
L'amour de la liberté \\[0pt]
L'amour de la nature \\[0pt]
L'amour de l'art \\[0pt]
L'amour de l'humanité \\[0pt]
L'amour des lois \\[0pt]
L'amour de soi \\[0pt]
L'amour est-il aveugle ? \\[0pt]
L'amour est-il désir ? \\[0pt]
L'amour est-il sans raison ? \\[0pt]
L'amour est-il une vertu ? \\[0pt]
L'amour et la haine \\[0pt]
L'amour et la justice \\[0pt]
L'amour et l'amitié \\[0pt]
L'amour et la mort \\[0pt]
L'amour et la philosophie \\[0pt]
L'amour et l'humanité \\[0pt]
L'amour maternel \\[0pt]
L'amour peut-il être absolu ? \\[0pt]
L'amour-propre \\[0pt]
L'amour vrai \\[0pt]
La multiplicité \\[0pt]
La multiplicité des acceptions de l'étant \\[0pt]
La multitude \\[0pt]
La musique a-t-elle une essence ? \\[0pt]
La musique de film \\[0pt]
La musique donne-t-elle à penser ? \\[0pt]
La musique est-elle un discours ? \\[0pt]
La musique est-elle un langage ? \\[0pt]
La musique et la danse \\[0pt]
La musique et le bruit \\[0pt]
La musique et le temps \\[0pt]
La musique exprime-t-elle quelque chose ? \\[0pt]
La musique peut-elle être narrative ? \\[0pt]
L'anachronisme \\[0pt]
La naissance \\[0pt]
La naissance de l'art \\[0pt]
La naissance de la science \\[0pt]
La naissance de l'homme \\[0pt]
La naïveté \\[0pt]
L'analogie \\[0pt]
L'analyse \\[0pt]
L'analyse des comportements humains permet-elle de réduire le hasard ? \\[0pt]
L'analyse des mythes doit-elle être uniquement linguistique \\[0pt]
L'analyse du vécu \\[0pt]
L'analyse économique rend-elle compte des comportements humains ? \\[0pt]
L'anarchie \\[0pt]
La nation \\[0pt]
La nation est-elle dépassée ? \\[0pt]
La nation et l'État \\[0pt]
La nature \\[0pt]
La nature artiste \\[0pt]
La nature a-t-elle des droits ? \\[0pt]
La nature a-t-elle une histoire ? \\[0pt]
La nature a-t-elle une valeur ? \\[0pt]
La nature des choses \\[0pt]
La nature des objets mathématiques \\[0pt]
La nature du bien \\[0pt]
La nature du droit \\[0pt]
La nature, est-ce le donné ? \\[0pt]
La nature est-elle artiste ? \\[0pt]
La nature est-elle belle ? \\[0pt]
La nature est-elle digne de respect ? \\[0pt]
La nature est-elle écrite en langage mathématique ? \\[0pt]
La nature est-elle muette ? \\[0pt]
La nature est-elle sacrée ? \\[0pt]
La nature est-elle sans histoire ? \\[0pt]
La nature est-elle sauvage ? \\[0pt]
La nature et la grâce \\[0pt]
La nature et l'artifice \\[0pt]
La nature et le monde \\[0pt]
« La nature fait bien les choses » \\[0pt]
La nature fait-elle bien les choses ? \\[0pt]
La nature humaine \\[0pt]
La nature imite-t-elle l'art ? \\[0pt]
La nature morte \\[0pt]
La nature nous dicte-t-elle des fins ? \\[0pt]
La nature parle-t-elle le langage des mathématiques ? \\[0pt]
La nature peut-elle être belle ? \\[0pt]
La nature peut-elle être un modèle ? \\[0pt]
La nature s'oppose-t-elle à l'esprit ? \\[0pt]
L'anecdotique \\[0pt]
La nécessité \\[0pt]
La nécessité des contradictions \\[0pt]
La nécessité des signes \\[0pt]
La nécessité historique \\[0pt]
La nécessité sociale des cérémonies \\[0pt]
La négation \\[0pt]
La négligence \\[0pt]
La négligence est-elle une faute ? \\[0pt]
La négociation \\[0pt]
La neutralité \\[0pt]
La neutralité axiologique \\[0pt]
La neutralité de l'État \\[0pt]
La névrose \\[0pt]
Langage et communication \\[0pt]
Langage et réalité \\[0pt]
Langage et réification \\[0pt]
Langage et société \\[0pt]
Langage, langue et parole \\[0pt]
Langage, langue, parole \\[0pt]
Langage ordinaire et langage de la science \\[0pt]
L'angélisme \\[0pt]
L'angoisse \\[0pt]
Langue et langage \\[0pt]
Langue et parole \\[0pt]
L'animal \\[0pt]
L'animal apprend-il à l'homme quelque chose sur l'homme ? \\[0pt]
L'animal a-t-il des droits ? \\[0pt]
L'animal domestique \\[0pt]
L'animalité \\[0pt]
L'animal nous apprend-il quelque chose sur l'homme ? \\[0pt]
L'animal peut-il être un sujet moral ? \\[0pt]
L'animal politique \\[0pt]
L'animisme \\[0pt]
La noblesse \\[0pt]
L'anomalie \\[0pt]
L'anomie \\[0pt]
La non-contradiction : principe logique ou ontologique ? \\[0pt]
L'anonymat \\[0pt]
L'anonyme \\[0pt]
L'anormal \\[0pt]
La normalité \\[0pt]
La normativité \\[0pt]
La norme du beau \\[0pt]
La norme du goût \\[0pt]
La norme et l'écart \\[0pt]
La norme et le fait \\[0pt]
La norme et son application \\[0pt]
La nostalgie \\[0pt]
La notion d'administration \\[0pt]
La notion d'art contemporain \\[0pt]
La notion d'art poétique \\[0pt]
La notion de barbarie a-t-elle un sens ? \\[0pt]
La notion de caractère relève-t- elle de la morale ou de la psychologie ? \\[0pt]
La notion de civilisation \\[0pt]
La notion de classe dominante \\[0pt]
La notion de classe sociale \\[0pt]
La notion de communauté morale \\[0pt]
La notion de corps social \\[0pt]
La notion de fait social \\[0pt]
La notion de genre littéraire \\[0pt]
La notion de loi a-t-elle une unité ? \\[0pt]
La notion de loi dans les sciences de la nature et dans les sciences de l'homme \\[0pt]
La notion de milieu \\[0pt]
La notion de paradis a-t-elle un sens exclusivement religieux ? \\[0pt]
La notion de peuple \\[0pt]
La notion de point de vue \\[0pt]
La notion d'époque \\[0pt]
La notion de possible \\[0pt]
La notion de progrès a-t-elle un sens en politique ? \\[0pt]
La notion de race a-t-elle un statut scientifique ? \\[0pt]
La notion de responsabilité est-elle suspecte ? \\[0pt]
La notion de signification est-elle indispensable aux sciences humaines ? \\[0pt]
La notion de souveraineté partagée \\[0pt]
La notion de sujet en politique \\[0pt]
La notion d'évolution \\[0pt]
La notion d'intérêt \\[0pt]
La notion d'ordre \\[0pt]
La notion économique d'abondance \\[0pt]
La notion physique de force \\[0pt]
La nouveauté \\[0pt]
La nouveauté en art \\[0pt]
L'antériorité \\[0pt]
L'anthropocentrisme \\[0pt]
L'anthropologie est-elle une ontologie ? \\[0pt]
L'anticipation \\[0pt]
La nuance \\[0pt]
La nudité \\[0pt]
La paix \\[0pt]
La paix civile \\[0pt]
La paix de la conscience \\[0pt]
La paix est-elle l'absence de guerre ? \\[0pt]
La paix est-elle la visée de la politique ? \\[0pt]
La paix est-elle moins naturelle que la guerre ? \\[0pt]
La paix est-elle possible ? \\[0pt]
La paix est-elle toujours préférable ? \\[0pt]
La paix n'est-elle que l'absence de conflit ? \\[0pt]
La paix n'est-elle que l'absence de guerre ? \\[0pt]
La paix n'est-elle qu'un idéal ? \\[0pt]
La paix perpétuelle \\[0pt]
La paix sociale est-elle la finalité de la politique ? \\[0pt]
La paix sociale est-elle une fin en soi ? \\[0pt]
La panne et la maladie \\[0pt]
La parenté \\[0pt]
La parenté et la famille \\[0pt]
La paresse \\[0pt]
La parole \\[0pt]
La parole comme acte social \\[0pt]
La parole donnée \\[0pt]
La parole peut-elle être étudiée comme un acte social ? \\[0pt]
La parole politique \\[0pt]
La parole publique \\[0pt]
La participation \\[0pt]
La participation des citoyens \\[0pt]
La participation politique \\[0pt]
La parure \\[0pt]
La passion de la vérité \\[0pt]
La passion de l'égalité \\[0pt]
La passion du juste \\[0pt]
La passion n'est-elle que souffrance ? \\[0pt]
La paternité \\[0pt]
L'apathie \\[0pt]
La patience \\[0pt]
La patience est-elle une vertu ? \\[0pt]
L'apatride \\[0pt]
La patrie \\[0pt]
La pauvreté \\[0pt]
La pédagogie, philosophie pratique ou psychologie appliquée ? \\[0pt]
La peine capitale \\[0pt]
La peinture apprend-elle à voir ? \\[0pt]
La peinture est-elle une poésie muette ? \\[0pt]
La peinture peut-elle être un art du temps ? \\[0pt]
La pensée a-t-elle une histoire ? \\[0pt]
La pensée collective \\[0pt]
La pensée de la mort a-t-elle un objet ? \\[0pt]
La pensée des machines \\[0pt]
La pensée est-elle en lutte avec le langage ? \\[0pt]
La pensée formelle \\[0pt]
La pensée formelle peut-elle avoir un contenu ? \\[0pt]
La pensée magique \\[0pt]
La pensée mythique \\[0pt]
La pensée peut-elle s'écrire ? \\[0pt]
La pensée scientifique \\[0pt]
La pensée technique \\[0pt]
La perception de la folie change-t-elle de nature avec les sciences humaines ? \\[0pt]
La perception est-elle l'interprétation du réel ? \\[0pt]
La perception musicale \\[0pt]
La perception peut-elle être désintéressée ? \\[0pt]
La perfectibilité \\[0pt]
La perfection \\[0pt]
La perfection artistique \\[0pt]
La perfection en art \\[0pt]
La perfection morale \\[0pt]
La persécution \\[0pt]
La persévérance \\[0pt]
La personnalité \\[0pt]
La personne \\[0pt]
La personne et la mémoire \\[0pt]
La personne politique \\[0pt]
La perspective \\[0pt]
La persuasion \\[0pt]
La pertinence \\[0pt]
La perversion \\[0pt]
La perversion morale \\[0pt]
La perversité \\[0pt]
La peur \\[0pt]
La peur de la mort \\[0pt]
La peur de la nature \\[0pt]
La peur de l'autre \\[0pt]
La peur de l'avenir \\[0pt]
La peur de la vérité \\[0pt]
La peur de l'inconnu \\[0pt]
La peur de manquer \\[0pt]
La peur des mots \\[0pt]
La peur du châtiment \\[0pt]
La peur du désordre \\[0pt]
La philanthropie \\[0pt]
La philosophie doit-elle être systématique ? \\[0pt]
La philosophie doit-elle se préoccuper du salut ? \\[0pt]
La philosophie est-elle la servante de la science ? \\[0pt]
La philosophie et le sens commun \\[0pt]
La philosophie peut-elle disparaître ? \\[0pt]
La philosophie peut-elle être expérimentale ? \\[0pt]
La philosophie peut-elle se passer de l'idée de vérité ? \\[0pt]
La philosophie peut-elle se passer de théologie ? \\[0pt]
La philosophie première \\[0pt]
La photographie est-elle un art ? \\[0pt]
La physique est-elle le modèle de toutes les sciences ? \\[0pt]
La physique et la chimie \\[0pt]
La pitié \\[0pt]
La pitié est-elle dangereuse ? \\[0pt]
La pitié est-elle morale ? \\[0pt]
La pitié est-elle un sentiment moral ? \\[0pt]
La pitié peut-elle fonder la morale ? \\[0pt]
La place d'autrui \\[0pt]
La place de l'art est-elle sur le marché de l'art ? \\[0pt]
La place du hasard dans la science \\[0pt]
La place du sujet dans la science \\[0pt]
La place du sujet dans les sciences humaines \\[0pt]
La plénitude \\[0pt]
La pluralité \\[0pt]
La pluralité des arts \\[0pt]
La pluralité des biens \\[0pt]
La pluralité des cultures \\[0pt]
La pluralité des langues \\[0pt]
La pluralité des méthodes en sociologie \\[0pt]
La pluralité des mondes \\[0pt]
La pluralité des opinions \\[0pt]
La pluralité des pouvoirs \\[0pt]
La pluralité des représentations \\[0pt]
La pluralité des sciences \\[0pt]
La pluralité des sciences de la nature \\[0pt]
La pluralité des sens de l'être \\[0pt]
La pluralité des valeurs \\[0pt]
La plus grande perfection \\[0pt]
La poésie \\[0pt]
La poésie dit-elle le réel ? \\[0pt]
La poésie est-elle comme une peinture ? \\[0pt]
La poésie et l'idée \\[0pt]
La poésie pense-t-elle ? \\[0pt]
La polémique \\[0pt]
La police et l'armée \\[0pt]
La politesse \\[0pt]
La politique a-t-elle besoin de héros ? \\[0pt]
La politique a-t-elle besoin de la fiction ? \\[0pt]
La politique a-t-elle besoin de la religion ? \\[0pt]
La politique a-t-elle besoin de modèles ? \\[0pt]
La politique a-t-elle besoin d'experts ? \\[0pt]
La politique a-t-elle pour fin d'éliminer la violence ? \\[0pt]
La politique consiste-t-elle à faire des compromis ? \\[0pt]
La politique de la santé \\[0pt]
La politique doit-elle être morale ? \\[0pt]
La politique doit-elle être rationnelle ? \\[0pt]
La politique doit-elle refuser l'utopie ? \\[0pt]
La politique doit-elle se mêler de l'art ? \\[0pt]
La politique doit-elle viser la concorde ? \\[0pt]
La politique doit-elle viser le consensus ? \\[0pt]
La politique du pire \\[0pt]
La politique échappe-t-elle à l'exigence de vérité ? \\[0pt]
La politique est-elle affaire de décision ? \\[0pt]
La politique est-elle affaire de jugement ? \\[0pt]
La politique est-elle affaire de science ? \\[0pt]
La politique est-elle architectonique ? \\[0pt]
La politique est-elle continuation de la guerre, par d'autres moyens ? \\[0pt]
La politique est-elle extérieure au droit ? \\[0pt]
La politique est-elle la continuation de la guerre ? \\[0pt]
La politique est-elle la continuation de la guerre par d'autres moyens ? \\[0pt]
La politique est-elle l'affaire de tous ? \\[0pt]
La politique est-elle l'art de la délibération ? \\[0pt]
La politique est-elle l'art des possibles ? \\[0pt]
La politique est-elle l'art du possible ? \\[0pt]
La politique est-elle le refus de la violence ? \\[0pt]
La politique est-elle le règne des passions ? \\[0pt]
La politique est-elle naturelle ? \\[0pt]
La politique est-elle par nature sujette à dispute ? \\[0pt]
La politique est-elle plus importante que tout ? \\[0pt]
La politique est-elle un art ? \\[0pt]
La politique est-elle une science ? \\[0pt]
La politique est-elle une technique ? \\[0pt]
La politique est-elle un métier ? \\[0pt]
La politique est-elle un moyen ou une fin ? \\[0pt]
La politique et la gloire \\[0pt]
La politique et la ville \\[0pt]
La politique et le mal \\[0pt]
La politique et le politique \\[0pt]
La politique et l'opinion \\[0pt]
La politique et l'utopie \\[0pt]
La politique exclut-elle le désordre ? \\[0pt]
La politique implique-t-elle la hiérarchie ? \\[0pt]
La politique peut-elle changer la société ? \\[0pt]
La politique peut-elle changer le monde ? \\[0pt]
La politique peut-elle disparaître ? \\[0pt]
La politique peut-elle échapper au mythe ? \\[0pt]
La politique peut-elle être indépendante de la morale ? \\[0pt]
La politique peut-elle être morale ? \\[0pt]
La politique peut-elle être objet de science ? \\[0pt]
La politique peut-elle être un objet de science ? \\[0pt]
La politique peut-elle n'être qu'une pratique ? \\[0pt]
La politique peut-elle se passer de croyances ? \\[0pt]
La politique peut-elle se passer des idéologies ? \\[0pt]
La politique peut-elle unir les hommes ? \\[0pt]
La politique peut-elle viser à autre chose qu'à l'efficacité ? \\[0pt]
La politique repose-t-elle sur un contrat ? \\[0pt]
La politique requière-t-elle le compromis \\[0pt]
La politique scientifique \\[0pt]
La politique se réduit-elle à des rapports de force ? \\[0pt]
La politique suppose-t-elle la morale ? \\[0pt]
La politique suppose-t-elle une idée de l'homme ? \\[0pt]
La politique vise-t-elle à dépasser les injustices ? \\[0pt]
L'apolitisme \\[0pt]
La populace \\[0pt]
La population \\[0pt]
L'aporie \\[0pt]
La position \\[0pt]
La possibilité \\[0pt]
La possibilité logique \\[0pt]
La possibilité métaphysique \\[0pt]
La possibilité réelle \\[0pt]
La potentialité \\[0pt]
L'apparence \\[0pt]
L'apparence fait-elle obstacle à la vérité ? \\[0pt]
L'appartenance sociale \\[0pt]
L'appel \\[0pt]
L'appréciation de la nature \\[0pt]
L'apprentissage \\[0pt]
L'apprentissage de la langue \\[0pt]
L'approche psychologique de l'homme met-elle en question la notion d'âme ? \\[0pt]
L'appropriation \\[0pt]
L'approximation \\[0pt]
La pratique \\[0pt]
La pratique de l'espace \\[0pt]
La pratique des sciences met-elle à l'abri des préjugés ? \\[0pt]
La précaution peut-elle être un principe ? \\[0pt]
La précision \\[0pt]
La première fois \\[0pt]
La première vérité \\[0pt]
La présence \\[0pt]
La présence à soi \\[0pt]
La présence d'autrui nous évite-t-elle la solitude ? \\[0pt]
La présence d'esprit \\[0pt]
La présence du passé \\[0pt]
La présomption \\[0pt]
La présomption d'innocence \\[0pt]
La preuve \\[0pt]
La preuve de l'existence de Dieu \\[0pt]
La prévision \\[0pt]
La prévision des conduites rationnelles \\[0pt]
L'\emph{a priori} \\[0pt]
L'a priori peut-il être historique ? \\[0pt]
La prise de parti est-elle essentielle en politique ? \\[0pt]
La prise en charge de l'individuel par la psychanalyse \\[0pt]
La prison \\[0pt]
La prison est-elle utile ? \\[0pt]
La privation \\[0pt]
La probabilité \\[0pt]
La probité \\[0pt]
La production \\[0pt]
La productivité de l'art \\[0pt]
La profondeur \\[0pt]
La prohibition de l'inceste \\[0pt]
La promenade \\[0pt]
La promesse \\[0pt]
La promesse et le contrat \\[0pt]
La promesse n'est-elle qu'un acte de langage ? \\[0pt]
La propagande \\[0pt]
La propagation des idées \\[0pt]
La proportion \\[0pt]
La proposition \\[0pt]
La propreté \\[0pt]
La propriété \\[0pt]
La propriété est-elle une garantie de liberté ? \\[0pt]
La protection \\[0pt]
La protection sociale \\[0pt]
La providence \\[0pt]
La prudence \\[0pt]
La prudence est-elle une vertu politique ? \\[0pt]
La psychanalyse donne-t-elle un sens au rêve ? \\[0pt]
La psychanalyse est-elle une science humaine ? \\[0pt]
La psychanalyse permet-elle la construction du sujet ? \\[0pt]
La psychanalyse s'applique-t-elle aux phénomènes sociaux ? \\[0pt]
La psychologie dit-elle ce qu'est le moi ? \\[0pt]
La psychologie est-elle l'étude de ce qui n'est pas social ? \\[0pt]
La psychologie est-elle l'étude des comportements individuels ? \\[0pt]
La psychologie est-elle une science ? \\[0pt]
La psychologie est-elle une science de la nature ? \\[0pt]
La psychologie est-elle une science humaine ? \\[0pt]
La publicité \\[0pt]
La pudeur \\[0pt]
La puissance \\[0pt]
La puissance de la technique \\[0pt]
La puissance de l'État \\[0pt]
La puissance des contraires \\[0pt]
La puissance des images \\[0pt]
La puissance du faux \\[0pt]
La puissance du langage \\[0pt]
La puissance du peuple \\[0pt]
La puissance et l'acte \\[0pt]
La puissance et le possible \\[0pt]
La pulsion \\[0pt]
La punition \\[0pt]
La pureté \\[0pt]
La qualité \\[0pt]
La question de la nature humaine relève-t-elle des sciences humaines ? \\[0pt]
La question de l'œuvre d'art \\[0pt]
La question de l'origine \\[0pt]
La question « qu'est-ce que l'homme ? » est-elle scientifique ? \\[0pt]
La question : « qui ? » \\[0pt]
La question sociale \\[0pt]
La quête des origines \\[0pt]
La racine du mal \\[0pt]
La radicalité \\[0pt]
La raison a-t-elle des limites ? \\[0pt]
La raison a-t-elle le droit d'expliquer ce que morale condamne ? \\[0pt]
La raison a-t-elle une histoire ? \\[0pt]
La raison d'État \\[0pt]
La raison doit-elle être cultivée ? \\[0pt]
La raison du plus fort \\[0pt]
La raison, est-ce la modération ? \\[0pt]
La raison est-elle le pouvoir de distinguer le vrai du faux ? \\[0pt]
La raison est-elle morale par elle-même ? \\[0pt]
La raison est-elle suffisante ? \\[0pt]
La raison est-elle un instrument ? \\[0pt]
La raison est-elle un obstacle au bonheur ? \\[0pt]
La raison et le calcul \\[0pt]
La raison gouverne-t-elle le monde ? \\[0pt]
La raison peut-elle errer ? \\[0pt]
La raison peut-elle être immédiatement pratique ? \\[0pt]
La raison peut-elle être pratique ? \\[0pt]
La raison peut-elle nous commander de croire ? \\[0pt]
La raison peut-elle rendre raison de tout ? \\[0pt]
La raison peut-elle s'opposer à elle-même ? \\[0pt]
La raison pratique \\[0pt]
La raison s'identifie-t-elle au calcul ? \\[0pt]
La raison suffisante \\[0pt]
La rationalité de la psychanalyse \\[0pt]
La rationalité de l'homo œconomicus \\[0pt]
La rationalité des choix politiques \\[0pt]
La rationalité des comportements économiques \\[0pt]
La rationalité du langage \\[0pt]
La rationalité du marché \\[0pt]
La rationalité économique \\[0pt]
La rationalité en sciences sociales \\[0pt]
La rationalité politique \\[0pt]
L'arbitraire \\[0pt]
L'arbitraire de la règle \\[0pt]
L'arbitraire du signe \\[0pt]
L'arbitraire en politique \\[0pt]
L'archéologie \\[0pt]
L'architecte : artiste ou ingénieur ? \\[0pt]
L'architecte : artiste ou l'ingénieur \\[0pt]
L'architecte dans la cité \\[0pt]
L'architecte et le législateur \\[0pt]
L'architecte et l'ingénieur \\[0pt]
L'architecture du monde \\[0pt]
L'architecture est-elle un art ? \\[0pt]
L'architecture et l'espace \\[0pt]
La réaction en politique \\[0pt]
La réalité \\[0pt]
La réalité a-t-elle une forme logique ? \\[0pt]
La réalité décrite par la science s'oppose-t-elle à la démonstration ? \\[0pt]
La réalité de la contradiction \\[0pt]
La réalité de la vie s'épuise-t-elle dans celle des vivants ? \\[0pt]
La réalité de l'idéal \\[0pt]
La réalité de l'idée \\[0pt]
La réalité de l'inconscient \\[0pt]
La réalité du beau \\[0pt]
La réalité du bien \\[0pt]
La réalité du corps \\[0pt]
La réalité du futur \\[0pt]
La réalité du mal \\[0pt]
La réalité du passé \\[0pt]
La réalité du possible \\[0pt]
La réalité du progrès \\[0pt]
La réalité du sensible \\[0pt]
La réalité du temps \\[0pt]
La réalité du temps se réduit-elle à la conscience que nous en avons ? \\[0pt]
La réalité est-elle une idée ? \\[0pt]
La réalité mentale \\[0pt]
La réalité peut-elle être virtuelle ? \\[0pt]
La réalité physique \\[0pt]
La réalité sociale \\[0pt]
La réalité symbolique \\[0pt]
La réception de l'œuvre d'art \\[0pt]
La recherche-action en sciences sociales \\[0pt]
La recherche de l'absolu \\[0pt]
La recherche de la vérité \\[0pt]
La recherche de la vérité dans les sciences humaines \\[0pt]
La recherche des invariants \\[0pt]
La recherche des origines \\[0pt]
La recherche du bonheur \\[0pt]
La recherche du bonheur suffit-elle à déterminer une morale ? \\[0pt]
La recherche scientifique est-elle désintéressée ? \\[0pt]
La réciprocité \\[0pt]
La réciprocité est-elle indispensable à la communauté politique ? \\[0pt]
La reconnaissance \\[0pt]
La rectitude \\[0pt]
La rectitude du droit \\[0pt]
La référence \\[0pt]
La référence à la mémoire en histoire \\[0pt]
La référence aux faits suffit-elle à garantir l'objectivité de la connaissance ? \\[0pt]
La réflexion \\[0pt]
La réflexion sur l'expérience participe-t-elle de l'expérience ? \\[0pt]
La réforme \\[0pt]
La réforme des institutions \\[0pt]
La réfutation \\[0pt]
La règle et l'exception \\[0pt]
La régression à l'infini \\[0pt]
La régulation \\[0pt]
La relation \\[0pt]
La relation à autrui est-elle symétrique ? \\[0pt]
La relation de causalité est-elle temporelle ? \\[0pt]
La relation de cause à effet \\[0pt]
La relation de nécessité \\[0pt]
La relation d'identité \\[0pt]
La religion \\[0pt]
La religion a-t-elle besoin d'un dieu ? \\[0pt]
La religion a-t-elle une fonction sociale ? \\[0pt]
La religion civile \\[0pt]
La religion de l'art \\[0pt]
La religion est-elle la sagesse des pauvres ? \\[0pt]
La religion est-elle simple affaire de croyance ? \\[0pt]
La religion est-elle source de conflit ? \\[0pt]
La religion est-elle une affaire privée ? \\[0pt]
La religion est-elle un obstacle à la liberté ? \\[0pt]
La religion peut-elle faire lien social ? \\[0pt]
La religion peut-elle se passer du recours au sacré ? \\[0pt]
La religion peut-elle suppléer la raison ? \\[0pt]
La religion rend-elle meilleur ? \\[0pt]
La réminiscence \\[0pt]
La renaissance \\[0pt]
La rencontre \\[0pt]
La rencontre d'autrui \\[0pt]
La réparation \\[0pt]
La répétition \\[0pt]
La représentation \\[0pt]
La représentation en politique \\[0pt]
La représentation politique \\[0pt]
La reproductibilité de l'œuvre d'art \\[0pt]
La reproduction \\[0pt]
La reproduction des œuvres d'art \\[0pt]
La reproduction des œuvres d'art nuit-elle à l'art ? \\[0pt]
La reproduction sociale \\[0pt]
La république \\[0pt]
La réputation \\[0pt]
La résignation \\[0pt]
La résilience \\[0pt]
La résistance à l'oppression \\[0pt]
La résistance de la matière \\[0pt]
La résistance du réel \\[0pt]
La résolution \\[0pt]
La responsabilité \\[0pt]
La responsabilité collective \\[0pt]
La responsabilité de l'artiste \\[0pt]
La responsabilité envers les générations futures \\[0pt]
La responsabilité implique-t-elle autrui ? \\[0pt]
La responsabilité politique \\[0pt]
La ressemblance \\[0pt]
La restauration des œuvres d'art \\[0pt]
La révélation \\[0pt]
La rêverie \\[0pt]
La révolte \\[0pt]
La révolte peut-elle être un droit ? \\[0pt]
La révolution \\[0pt]
L'argent \\[0pt]
L'argent et la valeur \\[0pt]
L'argumentation \\[0pt]
L'argumentation morale \\[0pt]
L'argument d'autorité \\[0pt]
La rhétorique \\[0pt]
La rhétorique a-t-elle une valeur ? \\[0pt]
La rhétorique est-elle un art ? \\[0pt]
La richesse \\[0pt]
La richesse du sensible \\[0pt]
La richesse intérieure \\[0pt]
La rigueur \\[0pt]
La rigueur de la loi \\[0pt]
La rigueur morale \\[0pt]
La rime et la raison \\[0pt]
L'aristocratie \\[0pt]
La rivalité \\[0pt]
L'arme rhétorique \\[0pt]
L'art abstrait \\[0pt]
L'art à l'épreuve du goût \\[0pt]
L'art apprend-il à percevoir ? \\[0pt]
L'art a-t-il besoin d'un discours sur l'art ? \\[0pt]
L'art a-t-il des règles ? \\[0pt]
L'art a-t-il des vertus thérapeutiques ? \\[0pt]
L'art a-t-il plus de valeur que la vérité ? \\[0pt]
L'art a-t-il une fin morale ? \\[0pt]
L'art a-t-il une histoire ? \\[0pt]
L'art a-t-il une responsabilité morale ? \\[0pt]
L'art a-t-il une valeur sociale ? \\[0pt]
L'art contre la beauté ? \\[0pt]
L'art d'écrire \\[0pt]
L'art décrit-il ? \\[0pt]
L'art de faire croire \\[0pt]
L'art de gouverner \\[0pt]
L'art de la guerre \\[0pt]
L'art de masse \\[0pt]
L'art de persuader \\[0pt]
L'art des images \\[0pt]
L'art de vivre est-il un art ? \\[0pt]
L'art doit-il être critique ? \\[0pt]
L'art doit-il nous étonner ? \\[0pt]
L'art doit-il refaire le monde ? \\[0pt]
L'art donne-t-il à voir l'invisible ? \\[0pt]
L'art dramatique \\[0pt]
L'art du comédien \\[0pt]
L'art du mensonge \\[0pt]
L'art échappe-t-il à la raison ? \\[0pt]
L'art éduque-t-il l'homme ? \\[0pt]
L'art engagé \\[0pt]
L'art, est-ce ce qui résiste à la certitude ? \\[0pt]
L'art est-il affaire de goût ? \\[0pt]
L'art est-il affaire d'imagination ? \\[0pt]
L'art est-il à lui-même son propre but ? \\[0pt]
L'art est-il ce qui permet de partager ses émotions ? \\[0pt]
L'art est-il destiné à embellir ? \\[0pt]
L'art est-il interprétation ? \\[0pt]
L'art est-il le miroir du monde ? \\[0pt]
L'art est-il le produit de l'inconscient ? \\[0pt]
L'art est-il le propre de l'homme ? \\[0pt]
L'art est-il le règne des apparences ? \\[0pt]
L'art est-il objet de compréhension ? \\[0pt]
L'art est-il politique ? \\[0pt]
L'art est-il révolutionnaire ? \\[0pt]
L'art est-il subversif ? \\[0pt]
L'art est-il un divertissement ? \\[0pt]
L'art est-il une connaissance ? \\[0pt]
L'art est-il une critique de la culture ? \\[0pt]
L'art est-il une expérience de la liberté ? \\[0pt]
L'art est-il une valeur ? \\[0pt]
L'art est-il un jeu ? \\[0pt]
L'art est-il un langage ? \\[0pt]
L'art est-il un langage universel ? \\[0pt]
L'art est-il un luxe ? \\[0pt]
L'art est-il un mode de connaissance ? \\[0pt]
L'art est-il un modèle pour la philosophie ? \\[0pt]
L'art est-il un monde ? \\[0pt]
L'art est par-delà beauté et laideur ? \\[0pt]
L'art et la manière \\[0pt]
L'art et la mort \\[0pt]
L'art et la nature \\[0pt]
L'art et l'artifice \\[0pt]
L'art et la tradition \\[0pt]
L'art et la vérité \\[0pt]
L'art et la vie \\[0pt]
L'art et le divin \\[0pt]
L'art et le mouvement \\[0pt]
L'art et le mythe \\[0pt]
L'art et l'éphémère \\[0pt]
L'art et le rêve \\[0pt]
L'art et le sacré \\[0pt]
L'art et les arts \\[0pt]
L'art et le temps \\[0pt]
L'art et le vivant \\[0pt]
L'art et l'immoralité \\[0pt]
L'art et l'interdit \\[0pt]
L'art et morale \\[0pt]
L'art et ses institutions \\[0pt]
L'art : expérience, exercice ou habitude ? \\[0pt]
L'art exprime-t-il ce que nous ne saurions dire ? \\[0pt]
L'art fait-il penser ? \\[0pt]
L'artifice \\[0pt]
L'artificiel \\[0pt]
L'art imite-t-il la nature ? \\[0pt]
L'art industriel \\[0pt]
L'artiste a-t-il besoin d'une idée de l'art ? \\[0pt]
L'artiste a-t-il besoin d'un public ? \\[0pt]
L'artiste a-t-il toujours raison ? \\[0pt]
L'artiste a-t-il une méthode ? \\[0pt]
L'artiste dans la cité \\[0pt]
L'artiste est-il le mieux placé pour comprendre son œuvre ? \\[0pt]
L'artiste est-il le seul travailleur libre ? \\[0pt]
L'artiste est-il le technicien du beau ? \\[0pt]
L'artiste est-il maître de son œuvre ? \\[0pt]
L'artiste est-il un métaphysicien ? \\[0pt]
L'artiste et l'artisan \\[0pt]
L'artiste et la société \\[0pt]
L'artiste et le savant \\[0pt]
L'artiste et son public \\[0pt]
L'artiste exprime-t-il quelque chose ? \\[0pt]
L'artiste peut-il se passer d'un maître ? \\[0pt]
L'artiste sait-il ce qu'il fait ? \\[0pt]
L'artiste travaille-t-il ? \\[0pt]
L'art modifie-t-il notre rapport à la réalité ? \\[0pt]
L'art modifie-t-il notre rapport au réel ? \\[0pt]
L'art n'est-il pas toujours politique ? \\[0pt]
L'art n'est-il pas toujours religieux ? \\[0pt]
L'art n'est-il qu'apparence ? \\[0pt]
L'art n'est-il qu'un artifice ? \\[0pt]
L'art n'est-il qu'une affaire d'esthétique ? \\[0pt]
L'art n'est-il qu'une question de sentiment ? \\[0pt]
L'art nous donne-t-il des raisons d'espérer ? \\[0pt]
L'art nous enseigne-t-il quelque chose ? \\[0pt]
L'art nous fait-il mieux percevoir le réel ? \\[0pt]
L'art nous libère-t-il de l'insignifiance ? \\[0pt]
L'art nous permet-il de lutter contre l'irréversibilité ? \\[0pt]
L'art nous ramène-t-il à la réalité ? \\[0pt]
L'art officiel \\[0pt]
L'art ou les arts \\[0pt]
L'art pense-t-il ? \\[0pt]
L'art peut-il changer le monde ? \\[0pt]
L'art peut-il contribuer à éduquer les hommes ? \\[0pt]
L'art peut-il encore imiter la nature ? \\[0pt]
L'art peut-il être abstrait ? \\[0pt]
L'art peut-il être brut ? \\[0pt]
L'art peut-il être enseigné ? \\[0pt]
L'art peut-il être révolutionnaire ? \\[0pt]
L'art peut-il être sans œuvre ? \\[0pt]
L'art peut-il être utile ? \\[0pt]
L'art peut-il finir ? \\[0pt]
L'art peut-il mourir ? \\[0pt]
L'art peut-il n'être pas conceptuel ? \\[0pt]
L'art peut-il nous rendre meilleurs ? \\[0pt]
L'art peut-il prétendre à la vérité ? \\[0pt]
L'art peut-il quelque chose contre la morale ? \\[0pt]
L'art peut-il quelque chose pour la morale ? \\[0pt]
L'art peut-il rendre le mouvement ? \\[0pt]
L'art peut-il s'affranchir des lois ? \\[0pt]
L'art peut-il s'enseigner ? \\[0pt]
L'art peut-il se passer de règles ? \\[0pt]
L'art peut-il se passer des critères du beau et du laid ? \\[0pt]
L'art peut-il se passer d'idéal ? \\[0pt]
L'art peut-il se passer d'œuvres ? \\[0pt]
L'art peut-il tenir lieu de métaphysique ? \\[0pt]
L'art politique \\[0pt]
L'art populaire \\[0pt]
L'art pour l'art \\[0pt]
L'art produit-il nécessairement des œuvres ? \\[0pt]
L'art s'adresse-t-il à la sensibilité ? \\[0pt]
L'art sait-il montrer ce que le langage ne peut pas dire ? \\[0pt]
L'art s'apparente-t-il à la philosophie ? \\[0pt]
L'art : une arithmétique sensible ? \\[0pt]
L'art vise-t-il nécessairement le beau ? \\[0pt]
La ruine \\[0pt]
La rumeur \\[0pt]
La rupture \\[0pt]
La ruse \\[0pt]
La ruse technique \\[0pt]
La sacralisation de l'œuvre \\[0pt]
La sagesse \\[0pt]
La sagesse du corps \\[0pt]
La sagesse et l'expérience \\[0pt]
La sagesse rend-elle heureux ? \\[0pt]
La sainteté \\[0pt]
La sanction \\[0pt]
La santé \\[0pt]
La santé est-elle un devoir ? \\[0pt]
La satisfaction des penchants \\[0pt]
La scène \\[0pt]
La scène théâtrale \\[0pt]
L'ascèse \\[0pt]
L'ascétisme \\[0pt]
La science admet-elle des degrés de croyance ? \\[0pt]
La science a-t-elle besoin du principe de causalité ? \\[0pt]
La science a-t-elle des limites ? \\[0pt]
La science a-t-elle le monopole de la vérité ? \\[0pt]
La science a-t-elle toujours raison ? \\[0pt]
La science a-t-elle une histoire ? \\[0pt]
La science commence-t-elle avec la perception ? \\[0pt]
La science connaît-elle ses limites ? \\[0pt]
La science découvre-t-elle ou construit-elle son objet ? \\[0pt]
La science de l'être \\[0pt]
La science de l'individuel \\[0pt]
La science dépend-elle nécessairement de l'expérience ? \\[0pt]
La science désenchante-t-elle le monde ? \\[0pt]
La science des mœurs \\[0pt]
La science dévoile-t-elle le réel ? \\[0pt]
La science doit-elle nous fournir des représentations du monde ? \\[0pt]
La science doit-elle se fonder sur une idée de la nature ? \\[0pt]
La science doit-elle se passer de l'idée de finalité ? \\[0pt]
La science est-elle indépendante de toute métaphysique ? \\[0pt]
La science est-elle une affaire politique ? \\[0pt]
La science est-elle une langue bien faite ? \\[0pt]
La science et le mythe \\[0pt]
La science et les sciences \\[0pt]
La science et l'irrationnel \\[0pt]
La science explique-t-elle le réel ? \\[0pt]
La science historique peut-elle être prédictive ? \\[0pt]
La science n'est-elle qu'une activité théorique ? \\[0pt]
La science n'est-elle qu'une fiction ? \\[0pt]
La science nous éloigne-t-elle des choses ? \\[0pt]
La science nous indique-t-elle ce que nous devons faire ? \\[0pt]
La science pense-t-elle ? \\[0pt]
La science permet-elle d'expliquer toute la réalité ? \\[0pt]
La science peut-elle errer ? \\[0pt]
La science peut-elle guider notre conduite ? \\[0pt]
La science peut-elle lutter contre les préjugés ? \\[0pt]
La science peut-elle se passer de fondement ? \\[0pt]
La science peut-elle se passer de métaphysique ? \\[0pt]
La science peut-elle se passer de méthode ? \\[0pt]
La science peut-elle se passer d'hypothèses ? \\[0pt]
La science peut-elle se passer d'institutions ? \\[0pt]
La science peut-elle tout expliquer ? \\[0pt]
La science politique \\[0pt]
La science porte-elle au scepticisme ? \\[0pt]
La science procède-t-elle par rectification ? \\[0pt]
La scientificité des sciences humaines \\[0pt]
La sculpture \\[0pt]
La sculpture et la peinture \\[0pt]
La sculpture et le mouvement \\[0pt]
La sculpture et l'espace \\[0pt]
La seconde nature \\[0pt]
La sécularisation \\[0pt]
La sécurité \\[0pt]
La sécurité nationale \\[0pt]
La sécurité publique \\[0pt]
La séduction \\[0pt]
La séduction en politique \\[0pt]
La ségrégation \\[0pt]
La sensation est-elle une connaissance ? \\[0pt]
La sensibilité \\[0pt]
La sensibilité animale \\[0pt]
La sensibilité peut-elle délivrer une vérité ? \\[0pt]
La séparation \\[0pt]
La séparation des pouvoirs \\[0pt]
La sérénité \\[0pt]
La servitude \\[0pt]
La servitude humaine \\[0pt]
La servitude peut-elle être volontaire ? \\[0pt]
La servitude volontaire \\[0pt]
La sévérité \\[0pt]
La sexualité \\[0pt]
La signification \\[0pt]
La signification dans l'œuvre \\[0pt]
La signification des mots \\[0pt]
La signification des rites \\[0pt]
La signification en musique \\[0pt]
La signification et l'usage \\[0pt]
La simplicité \\[0pt]
La simplicité du bien \\[0pt]
La sincérité \\[0pt]
La singularité \\[0pt]
La singularité du réel \\[0pt]
La singularité en morale \\[0pt]
La situation \\[0pt]
La sobriété \\[0pt]
La socialisation \\[0pt]
La socialisation des comportements \\[0pt]
La société civile \\[0pt]
La société civile et l'État \\[0pt]
La société contre l'État \\[0pt]
La société de consommation \\[0pt]
La société des individus \\[0pt]
La société des loisirs \\[0pt]
La société des nations \\[0pt]
La société des savants \\[0pt]
La société du genre humain \\[0pt]
La société est-elle concevable sans le travail ? \\[0pt]
La société et l'État \\[0pt]
La société existe-t-elle ? \\[0pt]
La société peut-elle être objet de connaissance ? \\[0pt]
La société peut-elle se passer de l'État ? \\[0pt]
La société sans l'État \\[0pt]
La sociologie de la religion \\[0pt]
La sociologie de l'art \\[0pt]
La sociologie de l'art nous permet-elle de comprendre l'art ? \\[0pt]
La sociologie est-elle déterministe ? \\[0pt]
La sociologie est-elle une physique sociale ? \\[0pt]
La sociologie relativise-t-elle la valeur des œuvres d'art ? \\[0pt]
La solidarité \\[0pt]
La solitude \\[0pt]
La solitude constitue-t-elle un obstacle à la citoyenneté ? \\[0pt]
La solitude de l'artiste \\[0pt]
La sollicitude \\[0pt]
La somme et le tout \\[0pt]
La souffrance \\[0pt]
La souffrance a-t-elle un sens ? \\[0pt]
La souffrance a-t-elle un sens moral ? \\[0pt]
La souffrance au travail \\[0pt]
La souffrance d'autrui \\[0pt]
La souffrance morale \\[0pt]
La souffrance peut-elle être partagée ? \\[0pt]
La soumission \\[0pt]
La souveraineté \\[0pt]
La souveraineté de l'État \\[0pt]
La souveraineté du peuple \\[0pt]
La souveraineté est-elle indivisible ? \\[0pt]
La souveraineté nationale \\[0pt]
La souveraineté peut-elle se partager ? \\[0pt]
La souveraineté populaire \\[0pt]
La spécificité des sciences humaines \\[0pt]
La spéculation \\[0pt]
La sphère privée échappe-t-elle au politique ? \\[0pt]
L'aspiration esthétique \\[0pt]
La spontanéité \\[0pt]
L'assassinat politique \\[0pt]
L'assentiment \\[0pt]
L'asservissement à l'utile \\[0pt]
L'asservissement de la pensée \\[0pt]
L'association \\[0pt]
L'association des idées \\[0pt]
La structure \\[0pt]
La structure et les éléments \\[0pt]
La structure et le sujet \\[0pt]
La subjectivité \\[0pt]
La subsistance \\[0pt]
La substance \\[0pt]
La substance et l'accident \\[0pt]
La substance et le substrat \\[0pt]
La succession des théories scientifiques \\[0pt]
La superstition \\[0pt]
La sûreté \\[0pt]
La surface et la profondeur \\[0pt]
La surveillance de la société \\[0pt]
La survie \\[0pt]
La sympathie \\[0pt]
La sympathie peut-elle tenir lieu de moralité ? \\[0pt]
La systématicité \\[0pt]
La table rase \\[0pt]
La technique a-t-elle une histoire ? \\[0pt]
La technique est-elle moralement neutre ? \\[0pt]
La technique fait-elle des miracles ? \\[0pt]
La technique fait-elle violence à la nature ? \\[0pt]
La technique n'est-elle qu'une application de la science ? \\[0pt]
La technique n'est-elle qu'un prolongement de nos organes ? \\[0pt]
La technique n'est-elle qu'un savoir-faire ? \\[0pt]
La technique permet-elle de réaliser tous les désirs ? \\[0pt]
La technique peut-elle améliorer l'homme ? \\[0pt]
La technocratie \\[0pt]
La technologie modifie-t-elle les rapports sociaux ? \\[0pt]
La téléologie \\[0pt]
La témérité \\[0pt]
La tempérance \\[0pt]
La temporalité de l'œuvre d'art \\[0pt]
La tendance \\[0pt]
La tentation \\[0pt]
La tentation réductionniste \\[0pt]
La terre \\[0pt]
La terre est-elle un bien commun ? \\[0pt]
La terre et le ciel \\[0pt]
La Terre et le Ciel \\[0pt]
La terreur \\[0pt]
La terreur morale \\[0pt]
L'athée peut-il être vertueux ? \\[0pt]
L'athéisme \\[0pt]
La théogonie \\[0pt]
La théologie est-elle une science ? \\[0pt]
La théologie peut-elle être rationnelle ? \\[0pt]
La théologie rationnelle \\[0pt]
La théorie et la pratique \\[0pt]
La théorie et l'expérience \\[0pt]
La tolérance \\[0pt]
La tolérance a-t-elle des limites ? \\[0pt]
La tolérance envers les intolérants \\[0pt]
La tolérance est-elle un concept politique ? \\[0pt]
La tolérance est-elle une vertu ? \\[0pt]
La tolérance peut-elle constituer un problème pour la démocratie ? \\[0pt]
L'atome \\[0pt]
La totalitarisme \\[0pt]
La totalité \\[0pt]
La toute-puissance \\[0pt]
La toute-puissance de la pensée \\[0pt]
La trace \\[0pt]
La trace et l'indice \\[0pt]
La tradition \\[0pt]
La traductibilité des langues \\[0pt]
La traduction \\[0pt]
La tragédie \\[0pt]
La trahison \\[0pt]
La tranquillité \\[0pt]
La transcendance \\[0pt]
La transe \\[0pt]
La transgression \\[0pt]
La transgression des règles \\[0pt]
La transmission \\[0pt]
La transparence \\[0pt]
La transparence est-elle un idéal démocratique ? \\[0pt]
La tristesse \\[0pt]
La tristesse du fini \\[0pt]
La tristesse est-elle mauvaise ? \\[0pt]
L'attachement \\[0pt]
L'attente \\[0pt]
L'attention \\[0pt]
L'attrait du beau \\[0pt]
La tyrannie \\[0pt]
La tyrannie de la majorité \\[0pt]
La tyrannie du bonheur \\[0pt]
L'audace \\[0pt]
L'audace politique \\[0pt]
L'au-delà \\[0pt]
L'au-delà de l'être \\[0pt]
L'autarcie \\[0pt]
L'auteur et le créateur \\[0pt]
L'authenticité \\[0pt]
L'authenticité artistique \\[0pt]
L'authenticité de l'œuvre d'art \\[0pt]
L'autobiographie \\[0pt]
L'automate \\[0pt]
L'autonomie \\[0pt]
L'autonomie de l'art \\[0pt]
L'autonomie de l'œuvre d'art \\[0pt]
L'autonomie du théorique \\[0pt]
L'autonomie du vivant \\[0pt]
L'autoportrait \\[0pt]
L'autorité \\[0pt]
L'autorité de la loi \\[0pt]
L'autorité de la parole \\[0pt]
L'autorité de la science \\[0pt]
L'autorité de l'écrit \\[0pt]
L'autorité de l'État \\[0pt]
L'autorité des savants \\[0pt]
L'autorité morale \\[0pt]
L'autorité politique \\[0pt]
L'autre est-il le fondement de la conscience morale ? \\[0pt]
L'autre et les autres \\[0pt]
L'autre monde \\[0pt]
La valeur artistique \\[0pt]
La valeur d'échange \\[0pt]
La valeur de la loi \\[0pt]
La valeur de l'argent \\[0pt]
La valeur de l'art \\[0pt]
La valeur de la science \\[0pt]
La valeur de la vérité \\[0pt]
La valeur de l'échange \\[0pt]
La valeur de l'exemple \\[0pt]
La valeur des arts \\[0pt]
La valeur des choses \\[0pt]
La valeur des conventions \\[0pt]
La valeur des exceptions \\[0pt]
La valeur des hypothèses \\[0pt]
La valeur des normes \\[0pt]
La valeur du beau \\[0pt]
La valeur du consensus \\[0pt]
La valeur du consentement \\[0pt]
La valeur d'une action se mesure-t-elle à sa réussite ? \\[0pt]
La valeur d'une action se mesure-t-elle à ses conséquences ? \\[0pt]
La valeur d'une théorie scientifique se mesure-t-elle à son efficacité ? \\[0pt]
La valeur du plaisir \\[0pt]
La valeur d'usage \\[0pt]
La valeur du témoignage \\[0pt]
La valeur du temps \\[0pt]
La valeur du travail \\[0pt]
La valeur et le prix \\[0pt]
La valeur morale \\[0pt]
« La valeur travail » \\[0pt]
La validité \\[0pt]
La vanité \\[0pt]
La vanité est-elle toujours sans objet ? \\[0pt]
L'avant-garde \\[0pt]
L'avarice \\[0pt]
La variété \\[0pt]
La vénalité \\[0pt]
La vénération \\[0pt]
La vengeance \\[0pt]
L'avenir \\[0pt]
L'avenir a-t-il une réalité ? \\[0pt]
L'avenir de l'humanité \\[0pt]
L'avenir est-il imaginable ? \\[0pt]
L'avenir est-il prévisible ? \\[0pt]
L'avenir est-il sans image ? \\[0pt]
L'avenir existe-t-il ? \\[0pt]
L'aventure \\[0pt]
La véracité \\[0pt]
La vérification \\[0pt]
La vérité admet-elle des degrés ? \\[0pt]
La vérité a-t-elle une histoire ? \\[0pt]
La vérité de la fiction \\[0pt]
La vérité de l'apparence \\[0pt]
La vérité de la religion \\[0pt]
La vérité demande-t-elle du courage ? \\[0pt]
La vérité des images \\[0pt]
La vérité des sciences \\[0pt]
La vérité des sentiments \\[0pt]
La vérité doit-elle toujours être démontrée ? \\[0pt]
La vérité du déterminisme \\[0pt]
La vérité d'une théorie dépend-elle de sa correspondance avec les faits ? \\[0pt]
La vérité du roman \\[0pt]
La vérité en politique \\[0pt]
La vérité est-elle affaire d'unanimité ? \\[0pt]
La vérité est-elle éternelle ? \\[0pt]
La vérité est-elle fille de son temps ? \\[0pt]
La vérité est-elle hors de notre portée ? \\[0pt]
La vérité est-elle morale ? \\[0pt]
La vérité est-elle relative ? \\[0pt]
La vérité est-elle une construction ? \\[0pt]
La vérité historique \\[0pt]
La vérité judiciaire \\[0pt]
La vérité n'est-elle qu'une erreur rectifiée ? \\[0pt]
La vérité nous contraint-elle ? \\[0pt]
La vérité peut-elle changer avec le temps ? \\[0pt]
La vérité peut-elle être équivoque ? \\[0pt]
La vérité peut-elle être indicible ? \\[0pt]
La vérité philosophique \\[0pt]
La vérité résiste-elle à sa révélation ? \\[0pt]
La vérité scientifique est-elle relative ? \\[0pt]
La vertu \\[0pt]
La vertu de l'abstraction \\[0pt]
La vertu de l'exemple \\[0pt]
La vertu de l'homme politique \\[0pt]
La vertu du citoyen \\[0pt]
La vertu du plaisir \\[0pt]
La vertu est-elle une forme de santé ? \\[0pt]
La vertu est-elle utile ? \\[0pt]
La vertu, les vertus \\[0pt]
La vertu peut-elle être excessive ? \\[0pt]
La vertu peut-elle être purement morale ? \\[0pt]
La vertu peut-elle s'enseigner ? \\[0pt]
La vertu politique \\[0pt]
L'aveu \\[0pt]
L'aveu diminue-t-il la faute ? \\[0pt]
L'aveuglement \\[0pt]
La vie active \\[0pt]
La vie brève \\[0pt]
La vie collective est-elle nécessairement frustrante ? \\[0pt]
La vie de la langue \\[0pt]
La vie de l'esprit \\[0pt]
« La vie des formes » \\[0pt]
La vie des machines \\[0pt]
La vie des signes \\[0pt]
La vie du droit \\[0pt]
La vie est-elle la valeur suprême ? \\[0pt]
La vie est-elle le bien le plus précieux ? \\[0pt]
La vie est-elle le bien suprême ? \\[0pt]
La vie est-elle trop courte ? \\[0pt]
La vie est-elle une notion métaphysique ? \\[0pt]
La vie est-elle une valeur ? \\[0pt]
La vie est-elle un songe ? \\[0pt]
« La vie est un songe » \\[0pt]
La vie éternelle \\[0pt]
La vie heureuse \\[0pt]
La vieillesse \\[0pt]
La vie intérieure \\[0pt]
La vie intérieure est-elle une illusion ? \\[0pt]
La vie ordinaire \\[0pt]
La vie peut-elle être éternelle ? \\[0pt]
La vie peut-elle être objet de science ? \\[0pt]
La vie peut-elle être sans histoire ? \\[0pt]
La vie politique \\[0pt]
La vie politique est-elle aliénante ? \\[0pt]
La vie privée \\[0pt]
La vie psychique \\[0pt]
La vie quotidienne \\[0pt]
La vie sauvage \\[0pt]
La vie sociale \\[0pt]
La vie sociale est-elle une comédie ? \\[0pt]
La vie sociale relève-t-elle des habitudes ? \\[0pt]
La vie suffit-elle à créer des valeurs ? \\[0pt]
La vigilance \\[0pt]
La ville \\[0pt]
La ville et la campagne \\[0pt]
La violence \\[0pt]
La violence a-t-elle des degrés ? \\[0pt]
La violence de l'art \\[0pt]
La violence de l'État \\[0pt]
La violence d'État \\[0pt]
La violence peut-elle être morale ? \\[0pt]
La violence politique \\[0pt]
La violence révolutionnaire \\[0pt]
La violence sociale \\[0pt]
La violence verbale \\[0pt]
La virtualité \\[0pt]
La virtuosité \\[0pt]
La vision scientifique du monde \\[0pt]
La vocation \\[0pt]
La vocation utopique de l'art \\[0pt]
La voix \\[0pt]
La voix de la conscience \\[0pt]
La voix du peuple \\[0pt]
La volonté constitue-t-elle le principe de la politique ? \\[0pt]
La volonté de croire \\[0pt]
La volonté du peuple \\[0pt]
La volonté générale \\[0pt]
La volonté peut-elle être collective ? \\[0pt]
La volonté peut-elle être indéterminée ? \\[0pt]
La volonté peut-elle être libre ? \\[0pt]
La volonté peut-elle nous manquer ? \\[0pt]
La volonté politique peut-elle être commune ? \\[0pt]
La volupté \\[0pt]
« La vraie morale se moque de la morale » \\[0pt]
La vraie morale se moque-t-elle de la morale ? \\[0pt]
La vraie vie \\[0pt]
La vraisemblance \\[0pt]
La vulgarisation \\[0pt]
La vulgarité \\[0pt]
La vulnérabilité \\[0pt]
L'axiome \\[0pt]
Le barbare \\[0pt]
Le baroque \\[0pt]
Le bavardage \\[0pt]
Le beau a-t-il une histoire ? \\[0pt]
Le beau est-il aimable ? \\[0pt]
Le beau est-il dicible ? \\[0pt]
Le beau est-il l'objet de l'esthétique ? \\[0pt]
Le beau est-il une valeur commune ? \\[0pt]
Le beau et l'agréable \\[0pt]
Le beau et le bien \\[0pt]
Le beau et le bon \\[0pt]
Le beau et le sublime \\[0pt]
Le beau existe-t-il indépendamment du bien ? \\[0pt]
Le beau naturel \\[0pt]
Le beau n'est-il qu'une idée ? \\[0pt]
Le beau peut-il être effrayant ? \\[0pt]
Le bénéfice du doute \\[0pt]
Le besoin \\[0pt]
Le besoin d'absolu \\[0pt]
Le besoin de beauté \\[0pt]
Le besoin de métaphysique est-il un besoin de connaissance ? \\[0pt]
Le besoin de philosophie \\[0pt]
Le besoin de rêve \\[0pt]
Le besoin de vérité \\[0pt]
Le besoin métaphysique \\[0pt]
Le bien commun \\[0pt]
Le bien détermine-t-il la volonté ? \\[0pt]
Le bien est-il conformité à la nature ? \\[0pt]
Le bien est-il l'utile ? \\[0pt]
Le bien et le beau \\[0pt]
Le bien et le mal \\[0pt]
Le bien et les biens \\[0pt]
Le bien-être \\[0pt]
Le bien public \\[0pt]
Le bien suppose-t-il la transcendance ? \\[0pt]
Le biologisme \\[0pt]
Le biologisme en sciences sociales \\[0pt]
Le bon et l'utile \\[0pt]
Le bon goût \\[0pt]
Le bonheur a-t-il nécessairement un objet ? \\[0pt]
Le bonheur comporte-t-il des degrés ? \\[0pt]
Le bonheur dans le mal \\[0pt]
Le bonheur de la passion est-il sans lendemain ? \\[0pt]
Le bonheur des autres \\[0pt]
Le bonheur des citoyens est-il un idéal politique ? \\[0pt]
Le bonheur des uns, le malheur des autres \\[0pt]
Le bonheur du citoyen \\[0pt]
Le bonheur est-il affaire de calcul ? \\[0pt]
Le bonheur est-il affaire de hasard ou de nécessité ? \\[0pt]
Le bonheur est-il la fin de la vie ? \\[0pt]
Le bonheur est-il nécessairement lié au plaisir ? \\[0pt]
Le bonheur est-il un accident ? \\[0pt]
Le bonheur est-il une affaire de circonstances ? \\[0pt]
Le bonheur est-il une fin morale ? \\[0pt]
Le bonheur est-il une fin politique ? \\[0pt]
Le bonheur est-il une valeur morale ? \\[0pt]
Le bonheur est-il un principe politique ? \\[0pt]
Le bonheur et la vertu \\[0pt]
Le bonheur n'est-il qu'une idée ? \\[0pt]
Le bonheur n'est-il qu'un idéal ? \\[0pt]
Le bonheur peut-il être un droit ? \\[0pt]
Le bonheur se fonde-t-il sur un savoir ? \\[0pt]
Le bon plaisir \\[0pt]
Le bon sens \\[0pt]
Le bon usage des passions \\[0pt]
Le bourgeois et le citoyen \\[0pt]
Le bricolage \\[0pt]
Le bruit \\[0pt]
Le cadavre \\[0pt]
Le cadre \\[0pt]
Le calcul \\[0pt]
Le calcul des plaisirs \\[0pt]
Le cannibalisme \\[0pt]
Le canon \\[0pt]
Le capitalisme \\[0pt]
Le capital social \\[0pt]
Le caractère \\[0pt]
L'écart \\[0pt]
Le cas de conscience \\[0pt]
Le catéchisme moral \\[0pt]
Le certain et le probable \\[0pt]
Le cerveau et la pensée \\[0pt]
L'échange \\[0pt]
L'échange des marchandises et les rapports humains \\[0pt]
L'échange des signes \\[0pt]
L'échange est-il un facteur de paix ? \\[0pt]
L'échange inégal \\[0pt]
Le changement \\[0pt]
L'échange symbolique \\[0pt]
Le chant \\[0pt]
Le chant et le cri \\[0pt]
Le chaos \\[0pt]
Le chaos du monde \\[0pt]
Le charisme en politique \\[0pt]
Le charisme politique \\[0pt]
Le charme et la grâce \\[0pt]
Le châtiment \\[0pt]
Le châtiment rend-il meilleur ? \\[0pt]
Le chef-d'œuvre \\[0pt]
Le chemin \\[0pt]
Le choix \\[0pt]
Le choix d'un destin \\[0pt]
Le choix en économie \\[0pt]
Le choix peut-il être éclairé ? \\[0pt]
Le ciel et la terre \\[0pt]
Le cinéma, art de la représentation ? \\[0pt]
Le cinéma est-il un art ? \\[0pt]
Le cinéma est-il un art comme les autres ? \\[0pt]
Le cinéma est-il un art ou une industrie ? \\[0pt]
Le cinéma est-il un art populaire ? \\[0pt]
Le citoyen \\[0pt]
Le citoyen a-t-il perdu toute naturalité ? \\[0pt]
Le citoyen peut-il être à la fois libre et soumis à l'État ? \\[0pt]
Le civisme \\[0pt]
Le clair et le distinct \\[0pt]
Le clair-obscur \\[0pt]
Le classicisme \\[0pt]
Le code \\[0pt]
Le cœur \\[0pt]
Le cœur et la raison \\[0pt]
L'école de la vie \\[0pt]
L'école des vertus \\[0pt]
L'écologie est-elle aussi une science humaine ? \\[0pt]
L'écologie est-elle une science de la nature ou une science sociale ? \\[0pt]
L'écologie est-elle un problème politique ? \\[0pt]
L'écologie politique \\[0pt]
L'écologie, une science humaine ? \\[0pt]
Le combat contre l'injustice a-t-il une source morale ? \\[0pt]
Le comédien \\[0pt]
Le comique et le tragique \\[0pt]
Le commencement \\[0pt]
Le commerce \\[0pt]
Le commerce adoucit-il les mœurs ? \\[0pt]
Le commerce des corps \\[0pt]
Le commerce des hommes \\[0pt]
Le commerce équitable \\[0pt]
Le commerce est-il pacificateur ? \\[0pt]
Le commerce peut-il être équitable ? \\[0pt]
Le commun \\[0pt]
Le comparatisme dans les sciences humaines \\[0pt]
Le complexe \\[0pt]
Le comportement \\[0pt]
Le compromis \\[0pt]
Le concept \\[0pt]
Le concept de civilisation \\[0pt]
Le concept de comportement rationnel dans les sciences humaines \\[0pt]
Le concept de nature est-il un concept scientifique ? \\[0pt]
Le concept de pulsion \\[0pt]
Le concept de race est-il anthropologique ? \\[0pt]
Le concept de société sans écriture est-il pertinent ? \\[0pt]
Le concept de structure sociale \\[0pt]
Le concept d'évolution \\[0pt]
Le concept d'inconscient est-il nécessaire en sciences humaines ? \\[0pt]
Le concept et la vie \\[0pt]
Le concept et l'image \\[0pt]
Le concret \\[0pt]
Le concret et l'abstrait \\[0pt]
Le conflit de devoirs \\[0pt]
Le conflit des devoirs \\[0pt]
Le conflit des interprétations \\[0pt]
Le conflit entre la science et la religion est-il inévitable ? \\[0pt]
Le conflit entre l'individu et la société est-il irréductible ? \\[0pt]
Le conflit esthétique \\[0pt]
Le conflit est-il au fondement de la vie sociale ? \\[0pt]
Le conflit est-il constitutif de la politique ? \\[0pt]
Le conflit est-il la raison d'être de la politique ? \\[0pt]
Le conflit social \\[0pt]
Le conformisme \\[0pt]
Le conformisme moral \\[0pt]
Le conformisme social \\[0pt]
L'économie a-t-elle des lois ? \\[0pt]
L'économie a-t-elle ses propres lois ? \\[0pt]
L'économie de la langue \\[0pt]
L'économie du don \\[0pt]
L'économie est-elle politique ? \\[0pt]
L'économie est-elle une science humaine ? \\[0pt]
L'économie et la valeur humaine \\[0pt]
L'économie fait-elle partie des sciences humaines ? \\[0pt]
L'économie politique \\[0pt]
L'économie primitive \\[0pt]
L'économie psychique \\[0pt]
L'économique et le politique \\[0pt]
Le conseil \\[0pt]
Le conseiller du prince \\[0pt]
Le consensus \\[0pt]
Le consensus social \\[0pt]
Le consentement \\[0pt]
Le consentement des gouvernés \\[0pt]
Le consentement du peuple \\[0pt]
Le contemporain \\[0pt]
Le contenu empirique \\[0pt]
Le contingent \\[0pt]
Le continu \\[0pt]
Le contradictoire peut-il exister ? \\[0pt]
Le contrat \\[0pt]
Le contrat social \\[0pt]
Le contrôle social \\[0pt]
Le convenable \\[0pt]
Le corps dansant \\[0pt]
Le corps dit-il quelque chose ? \\[0pt]
Le corps du prince \\[0pt]
Le corps du travailleur \\[0pt]
Le corps est-il porteur de valeurs ? \\[0pt]
Le corps est-il respectable ? \\[0pt]
Le corps est-il un enjeu des sciences humaines ? \\[0pt]
Le corps et la danse \\[0pt]
Le corps et la machine \\[0pt]
Le corps et l'âme \\[0pt]
Le corps et l'esprit \\[0pt]
Le corps et le temps \\[0pt]
Le corps et l'instrument \\[0pt]
Le corps humain \\[0pt]
Le corps humain est-il naturel ? \\[0pt]
Le corps n'est-il que matière ? \\[0pt]
Le corps pense-t-il ? \\[0pt]
Le corps politique \\[0pt]
Le corps propre \\[0pt]
Le cosmopolitisme \\[0pt]
Le cosmopolitisme peut-il devenir réalité ? \\[0pt]
Le cosmopolitisme peut-il être réaliste ? \\[0pt]
Le coupable \\[0pt]
Le coup d'État \\[0pt]
Le couple \\[0pt]
Le courage \\[0pt]
Le courage de la vérité \\[0pt]
Le courage est-il une vertu ? \\[0pt]
Le courage politique \\[0pt]
Le cours des choses \\[0pt]
Le cours du temps \\[0pt]
Le créé et l'incréé \\[0pt]
Le cri \\[0pt]
Le crime suppose-t-il la loi ? \\[0pt]
Le critère \\[0pt]
L'écrit et l'oral \\[0pt]
Le critique d'art \\[0pt]
L'écriture des lois \\[0pt]
L'écriture et la parole \\[0pt]
L'écriture et la pensée \\[0pt]
L'écriture ne sert-elle qu'à consigner la pensée ? \\[0pt]
Le cru et le cuit \\[0pt]
Le culte des ancêtres \\[0pt]
Le culte des images \\[0pt]
Le cynique \\[0pt]
Le cynisme \\[0pt]
Le dandysme \\[0pt]
Le danger \\[0pt]
Le débat \\[0pt]
Le débat politique \\[0pt]
Le décentrement ethnologique \\[0pt]
Le déclassement \\[0pt]
Le défaut \\[0pt]
Le déguisement \\[0pt]
Le dérèglement \\[0pt]
Le dernier mot \\[0pt]
Le désaccord \\[0pt]
Le désespoir \\[0pt]
Le désespoir est-il une faute morale ? \\[0pt]
Le déshonneur \\[0pt]
Le design \\[0pt]
Le désintéressement \\[0pt]
Le désintéressement esthétique \\[0pt]
Le désir de connaissance \\[0pt]
Le désir de domination \\[0pt]
Le désir de dominer \\[0pt]
Le désir de gloire \\[0pt]
Le désir de l'autre \\[0pt]
Le désir de pouvoir \\[0pt]
Le désir de reconnaissance \\[0pt]
Le désir de savoir \\[0pt]
Le désir d'éternité \\[0pt]
Le désir de vérité \\[0pt]
Le désir d'immortalité \\[0pt]
Le désir d'infini \\[0pt]
Le désir d'originalité \\[0pt]
Le désir est-il l'essence de l'homme ? \\[0pt]
Le désir est-il sans limite ? \\[0pt]
Le désir et la loi \\[0pt]
Le désir et le manque \\[0pt]
Le désir métaphysique \\[0pt]
Le désir n'est-il que l'épreuve d'un manque ? \\[0pt]
Le désir n'est-il qu'inquiétude ? \\[0pt]
Le désir peut-il atteindre son objet ? \\[0pt]
Le désir peut-il se satisfaire de la réalité ? \\[0pt]
Le désœuvrement \\[0pt]
Le désordre \\[0pt]
Le désordre des choses \\[0pt]
Le désordre est-il étonnant ? \\[0pt]
Le désordre est-il un mal ? \\[0pt]
Le despote peut-il être éclairé ? \\[0pt]
Le despotisme \\[0pt]
Le despotisme peut-il être éclairé ? \\[0pt]
Le dessin \\[0pt]
Le dessin et la couleur \\[0pt]
Le destin \\[0pt]
Le désuet \\[0pt]
Le détachement \\[0pt]
Le détail \\[0pt]
Le déterminisme \\[0pt]
Le déterminisme psychique \\[0pt]
Le déterminisme social \\[0pt]
Le deuil \\[0pt]
Le développement moral \\[0pt]
Le devenir \\[0pt]
Le devenir est-il pensable ? \\[0pt]
Le devoir d'aimer \\[0pt]
Le devoir d'assistance \\[0pt]
Le devoir de vérité \\[0pt]
Le devoir d'obéissance \\[0pt]
Le devoir et la dette \\[0pt]
Le devoir-être \\[0pt]
Le devoir s'apprend-il ? \\[0pt]
Le devoir se présente-t-il avec la force de l'évidence ? \\[0pt]
Le dévouement \\[0pt]
Le dialogue \\[0pt]
Le dialogue des philosophes \\[0pt]
Le dialogue entre les cultures \\[0pt]
Le dictionnaire \\[0pt]
Le dieu artiste \\[0pt]
Le dieu des philosophes \\[0pt]
Le Dieu des philosophes \\[0pt]
L'édification morale \\[0pt]
Le dilemme \\[0pt]
Le dire et le faire \\[0pt]
Le discernement \\[0pt]
Le discontinu \\[0pt]
Le discours politique \\[0pt]
Le divers \\[0pt]
Le divertissement \\[0pt]
Le divertissement est-il vain ? \\[0pt]
Le divin \\[0pt]
Le dogmatisme \\[0pt]
Le domestique \\[0pt]
Le don \\[0pt]
Le don de soi \\[0pt]
Le don et l'échange \\[0pt]
Le donné \\[0pt]
Le donné et le construit \\[0pt]
Le double \\[0pt]
Le doute dans les sciences \\[0pt]
Le doute est-il une faiblesse de la pensée ? \\[0pt]
Le doute métaphysique \\[0pt]
Le drame \\[0pt]
Le droit à la citoyenneté \\[0pt]
Le droit à l'erreur \\[0pt]
Le droit au bonheur \\[0pt]
Le droit au Bonheur \\[0pt]
Le droit au respect de la vie privée \\[0pt]
Le droit au travail \\[0pt]
Le droit d'asile \\[0pt]
Le droit de disposer de son corps \\[0pt]
Le droit de guerre \\[0pt]
Le droit de la guerre \\[0pt]
Le droit de la majorité \\[0pt]
Le droit de propriété \\[0pt]
Le droit de punir \\[0pt]
Le droit de résistance \\[0pt]
Le droit de résistance à l'oppression \\[0pt]
Le droit de révolte \\[0pt]
Le droit des animaux \\[0pt]
Le droit des gens \\[0pt]
Le droit des minorités \\[0pt]
Le droit des peuples \\[0pt]
Le droit des peuples à disposer d'eux-mêmes \\[0pt]
Le droit de veto \\[0pt]
Le droit de vie et de mort \\[0pt]
Le droit de vivre \\[0pt]
Le droit d'ingérence \\[0pt]
Le droit d'intervention \\[0pt]
Le droit divin \\[0pt]
Le droit doit-il être le seul régulateur de la vie sociale ? \\[0pt]
Le droit doit-il être moral ? \\[0pt]
Le droit du plus faible \\[0pt]
Le droit du plus fort \\[0pt]
Le droit du premier occupant \\[0pt]
Le droit est-il une science humaine ? \\[0pt]
Le droit et la violence \\[0pt]
Le droit exclut-il la force ? \\[0pt]
Le droit humanitaire \\[0pt]
Le droit international \\[0pt]
Le droit peut-il être naturel ? \\[0pt]
Le droit peut-il se fonder sur la force ? \\[0pt]
Le dualisme \\[0pt]
L'éducation artistique \\[0pt]
L'éducation civique \\[0pt]
L'éducation des esprits \\[0pt]
L'éducation du goût \\[0pt]
L'éducation est-elle affaire de politique ? \\[0pt]
L'éducation est-elle par nature conservatrice ? \\[0pt]
L'éducation esthétique \\[0pt]
L'éducation peut-elle être sentimentale ? \\[0pt]
L'éducation physique \\[0pt]
L'éducation politique \\[0pt]
L'éducation relève-t-elle du politique ? \\[0pt]
Le fait de vivre est-il un bien en soi ? \\[0pt]
Le fait d'exister \\[0pt]
Le fait divers \\[0pt]
Le fait du prince \\[0pt]
Le fait et le droit \\[0pt]
Le fait historique \\[0pt]
Le fait religieux \\[0pt]
Le fait scientifique \\[0pt]
Le fait social est-il une chose ? \\[0pt]
Le fanatisme \\[0pt]
Le fantastique \\[0pt]
Le faux en art \\[0pt]
Le faux et l'absurde \\[0pt]
Le faux et le fictif \\[0pt]
Le fédéralisme \\[0pt]
Le féminin et le masculin \\[0pt]
Le féminisme \\[0pt]
Le fétichisme \\[0pt]
Le fétichisme de la marchandise \\[0pt]
L'effet de la musique \\[0pt]
L'efficacité est-elle une vertu ? \\[0pt]
L'efficacité thérapeutique de la psychanalyse \\[0pt]
L'efficience \\[0pt]
L'effort \\[0pt]
Le finalisme \\[0pt]
Le fini et l'infini \\[0pt]
Le flegme \\[0pt]
Le folklore \\[0pt]
Le fonctionnaire \\[0pt]
Le fonctionnel et le beau \\[0pt]
Le fond \\[0pt]
Le fondement \\[0pt]
Le fondement de l'induction \\[0pt]
Le fond et la forme \\[0pt]
Le for intérieur \\[0pt]
Le formalisme \\[0pt]
Le formalisme dans les sciences humaines \\[0pt]
Le formalisme moral \\[0pt]
Le fou \\[0pt]
Le fragment \\[0pt]
Le frivole \\[0pt]
Le futur est-il contingent ? \\[0pt]
Le futur est-il nécessairement contingent ? \\[0pt]
L'égalité \\[0pt]
L'égalité civile \\[0pt]
L'égalité des chances \\[0pt]
L'égalité des citoyens \\[0pt]
L'égalité des conditions \\[0pt]
L'égalité des hommes est-elle un fait ou une idée ? \\[0pt]
L'égalité des hommes et des femmes est-elle une question politique ? \\[0pt]
L'égalité des sexes \\[0pt]
L'égalité devant la loi \\[0pt]
L'égalité est-elle souhaitable ? \\[0pt]
Légalité et légitimité \\[0pt]
Légalité et moralité \\[0pt]
L'égalité peut-elle être une menace pour la liberté ? \\[0pt]
Le génie \\[0pt]
Le génie du lieu \\[0pt]
Le génie du mal \\[0pt]
Le genre et l'espèce \\[0pt]
Le genre humain \\[0pt]
Le genre humain : unité ou pluralité ? \\[0pt]
Le geste \\[0pt]
Le geste créateur \\[0pt]
Le geste et la parole \\[0pt]
Légitimité et légalité \\[0pt]
L'égoïsme \\[0pt]
Le goût \\[0pt]
Le goût : certitude ou conviction ? \\[0pt]
Le goût de la fête \\[0pt]
Le goût de la polémique \\[0pt]
Le goût de l'artiste \\[0pt]
Le goût des autres \\[0pt]
Le goût du beau \\[0pt]
Le goût du pouvoir \\[0pt]
Le goût est-il nécessairement subjectif ? \\[0pt]
Le goût est-il une faculté ? \\[0pt]
Le goût est-il une question de classe ? \\[0pt]
Le goût est-il une vertu sociale ? \\[0pt]
Le goût et la distinction \\[0pt]
Le goût se forme-t-il ? \\[0pt]
Le gouvernement des experts \\[0pt]
Le gouvernement des hommes et l'administration des choses \\[0pt]
Le gouvernement des hommes libres \\[0pt]
Le gouvernement des juges \\[0pt]
Le gouvernement des meilleurs \\[0pt]
Le gouvernement de soi et des autres \\[0pt]
Le gouvernement par le peuple est-il nécessairement pour le peuple ? \\[0pt]
Le grandiose \\[0pt]
Le grand livre de la nature \\[0pt]
Le grotesque \\[0pt]
Le groupe \\[0pt]
Le hasard \\[0pt]
Le hasard existe-t-il ? \\[0pt]
Le hasard fait-il bien les choses ? \\[0pt]
Le hasard gouverne-t-il le monde ? \\[0pt]
Le hasard n'est-il que la mesure de notre ignorance ? \\[0pt]
Le hasard n'est-il que le nom de notre ignorance ? \\[0pt]
Le haut et le bas \\[0pt]
Le héros moral \\[0pt]
Le hors-la-loi \\[0pt]
Le je-ne-sais-quoi \\[0pt]
Le jeu \\[0pt]
Le jeu de mots \\[0pt]
Le jeu des apparences \\[0pt]
Le jeu des possibles \\[0pt]
Le jeu du monde \\[0pt]
Le jeu et la règle \\[0pt]
Le jeu social \\[0pt]
Le joli, le beau \\[0pt]
Le jugement \\[0pt]
Le jugement artistique se fait-il sans concept ? \\[0pt]
Le jugement critique peut-il s'exercer sans culture ? \\[0pt]
Le jugement de goût \\[0pt]
Le jugement de goût est-il universel ? \\[0pt]
Le jugement de l'histoire \\[0pt]
Le jugement dernier \\[0pt]
Le jugement de valeur est-il indifférent à la vérité ? \\[0pt]
Le jugement moral \\[0pt]
Le jugement politique \\[0pt]
Le jugement pratique \\[0pt]
Le juste et le bien \\[0pt]
Le juste et le légal \\[0pt]
Le juste et le vrai \\[0pt]
Le juste milieu \\[0pt]
Le laboratoire \\[0pt]
Le laid \\[0pt]
Le langage animal \\[0pt]
Le langage de la pensée \\[0pt]
Le langage de l'art \\[0pt]
Le langage des sciences \\[0pt]
Le langage du corps \\[0pt]
Le langage est-il assimilable à un outil ? \\[0pt]
Le langage est-il d'essence poétique ? \\[0pt]
Le langage est-il le miroir du monde ? \\[0pt]
Le langage est-il l'instrument de la pensée ? \\[0pt]
Le langage et les langues \\[0pt]
Le langage fait-il obstacle à la connaissance ? \\[0pt]
Le langage ne sert-il qu'à communiquer ? \\[0pt]
Le langage poétique \\[0pt]
Le langage scientifique \\[0pt]
Le latent et le manifeste \\[0pt]
L'élégance \\[0pt]
Le législateur \\[0pt]
Le législateur et le juge \\[0pt]
L'élément \\[0pt]
L'élémentaire \\[0pt]
Le libertin \\[0pt]
Le libertin, le libertaire \\[0pt]
Le libre arbitre \\[0pt]
Le libre échange \\[0pt]
Le libre jeu des formes \\[0pt]
Le lien politique \\[0pt]
Le lien social \\[0pt]
Le lien social peut-il être compassionnel ? \\[0pt]
Le lien social relève-t-il de la politique ? \\[0pt]
Le lieu \\[0pt]
Le lieu commun \\[0pt]
Le lieu de la pensée \\[0pt]
Le lieu de l'esprit \\[0pt]
Le littéral et le figuré \\[0pt]
L'éloge de la démesure \\[0pt]
Le loisir \\[0pt]
Le loisir caractérise-t-il l'homme libre ? \\[0pt]
Le luxe \\[0pt]
Le lyrisme \\[0pt]
Le magistrat \\[0pt]
Le maintien de l'ordre \\[0pt]
Le maître et le disciple \\[0pt]
Le mal \\[0pt]
Le mal apparaît-il toujours ? \\[0pt]
Le mal constitue-t-il une objection à l'existence de Dieu ? \\[0pt]
Le malentendu \\[0pt]
Le mal est-il une réalité objective ? \\[0pt]
Le malheur \\[0pt]
Le malin plaisir \\[0pt]
Le mal métaphysique \\[0pt]
Le mal peut-il être absolu ? \\[0pt]
L'émancipation \\[0pt]
L'émancipation des femmes \\[0pt]
Le maniérisme \\[0pt]
Le manifeste politique \\[0pt]
Le manque de culture \\[0pt]
Le manque de rigueur \\[0pt]
Le marché \\[0pt]
Le marché de l'art \\[0pt]
Le mariage \\[0pt]
Le masque \\[0pt]
Le matériau musical \\[0pt]
Le matériel \\[0pt]
Le mauvais goût \\[0pt]
L'embarras \\[0pt]
L'embarras du choix \\[0pt]
L'embryon humain est-il une personne ? \\[0pt]
L'embryon humain peut-il être l'objet d'expérimentation ? \\[0pt]
Le mécanisme \\[0pt]
Le mécanisme et la mécanique \\[0pt]
Le mécénat \\[0pt]
Le méchant peut-il être heureux ? \\[0pt]
Le meilleur \\[0pt]
Le meilleur des mondes possible \\[0pt]
Le meilleur régime \\[0pt]
Le meilleur régime politique \\[0pt]
Le mélange \\[0pt]
Le même et l'autre \\[0pt]
Le ménage \\[0pt]
Le mensonge \\[0pt]
Le mensonge de l'art ? \\[0pt]
Le mensonge en politique \\[0pt]
Le mensonge politique \\[0pt]
Le mépris \\[0pt]
Le mépris de classe \\[0pt]
Le mépris peut-il être justifié ? \\[0pt]
Le mérite \\[0pt]
Le mérite est-il le critère de la vertu ? \\[0pt]
Le mesurable \\[0pt]
Le métaphysicien est-il un physicien à sa façon ? \\[0pt]
Le métier \\[0pt]
Le métier de politique \\[0pt]
Le métier d'homme \\[0pt]
Le métier du politique \\[0pt]
Le mien et le tien \\[0pt]
Le mieux est-il l'ennemi du bien ? \\[0pt]
Le milieu \\[0pt]
Le milieu social \\[0pt]
Le miracle \\[0pt]
Le miroir \\[0pt]
Le mode \\[0pt]
Le mode d'existence de l'œuvre d'art \\[0pt]
Le modèle \\[0pt]
Le modèle du jeu dans les sciences humaines ? \\[0pt]
Le modèle en morale \\[0pt]
Le modèle et la copie \\[0pt]
Le modèle évolutionniste en histoire \\[0pt]
Le modèle organiciste \\[0pt]
Le modèle organiciste en sciences humaines \\[0pt]
Le moi \\[0pt]
Le moi est-il haïssable ? \\[0pt]
Le moi est-il une collection d'idées ? \\[0pt]
Le moi est-il une illusion ? \\[0pt]
Le moi et ses images \\[0pt]
Le moindre mal \\[0pt]
Le moi peut-il être l'objet d'un savoir ? \\[0pt]
Le monde à l'envers \\[0pt]
Le monde a-t-il une histoire ? \\[0pt]
Le monde aurait-il un sens si l'homme n'existait pas ? \\[0pt]
Le monde de la culture \\[0pt]
Le monde de l'animal \\[0pt]
Le monde de l'animal nous est-il étranger ? \\[0pt]
Le monde de l'art \\[0pt]
Le monde de l'art et l'ordinateur \\[0pt]
Le monde de la technique \\[0pt]
Le monde de la vie \\[0pt]
Le monde de l'entreprise \\[0pt]
Le monde de l'être humain \\[0pt]
Le monde des idées \\[0pt]
Le monde des machines \\[0pt]
Le monde des œuvres \\[0pt]
Le monde des physiciens \\[0pt]
Le monde des rêves \\[0pt]
Le monde des sens \\[0pt]
Le monde du rêve \\[0pt]
Le monde du travail \\[0pt]
Le monde est-il en progrès ? \\[0pt]
Le monde est-il éternel ? \\[0pt]
Le monde est-il notre représentation ? \\[0pt]
Le monde est-il un théâtre ? \\[0pt]
Le monde et la nature \\[0pt]
Le monde intérieur \\[0pt]
Le monde peut-il être nouveau ? \\[0pt]
Le monde politique \\[0pt]
Le monde se suffit-il à lui-même ? \\[0pt]
Le monde vrai \\[0pt]
Le monopole de la violence légitime \\[0pt]
Le monstre \\[0pt]
Le monstrueux \\[0pt]
Le moralisme \\[0pt]
Le moraliste \\[0pt]
Le mot d'esprit \\[0pt]
Le mot et la chose \\[0pt]
L'émotion \\[0pt]
L'émotion esthétique \\[0pt]
L'émotion esthétique peut-elle se communiquer ? \\[0pt]
L'émotion esthétique peut-elle se partager ? \\[0pt]
L'émotion morale \\[0pt]
Le mot juste \\[0pt]
Le mot vie a-t-il plusieurs sens ? \\[0pt]
Le mouvement \\[0pt]
Le mouvement de la pensée \\[0pt]
L'empathie \\[0pt]
L'empathie est-elle nécessaire aux sciences sociales ? \\[0pt]
L'empire \\[0pt]
L'empire sur soi \\[0pt]
L'empirisme des sciences humaines \\[0pt]
L'emploi du temps \\[0pt]
Le multiculturalisme \\[0pt]
Le multiple \\[0pt]
Le musée \\[0pt]
Le mystère \\[0pt]
Le mysticisme \\[0pt]
Le mythe \\[0pt]
Le mythe est-il objet de science ? \\[0pt]
Le naïf \\[0pt]
Le narcissisme \\[0pt]
Le naturalisme \\[0pt]
Le naturalisme des sciences humaines et sociales \\[0pt]
Le naturel \\[0pt]
Le naturel et l'artificiel \\[0pt]
Le naturel et le normal \\[0pt]
L'enchevêtrement des causes \\[0pt]
L'encyclopédie \\[0pt]
Le néant \\[0pt]
Le nécessaire et le contingent \\[0pt]
Le négatif \\[0pt]
Le néologisme \\[0pt]
L'énergie \\[0pt]
L'enfance \\[0pt]
L'enfance de l'art \\[0pt]
L'enfance est-elle ce qui doit être surmonté ? \\[0pt]
L'enfance peut-elle servir de modèle ? \\[0pt]
L'enfant \\[0pt]
L'enfant sauvage \\[0pt]
L'enfer est-il véritablement pavé de bonnes intentions ? \\[0pt]
« L'enfer est pavé de bonnes intentions » \\[0pt]
L'engagement \\[0pt]
L'engagement dans l'art \\[0pt]
L'engagement politique \\[0pt]
L'engendrement \\[0pt]
Le nihilisme \\[0pt]
L'ennemi \\[0pt]
L'ennemi intérieur \\[0pt]
L'ennui \\[0pt]
Le noble et le vil \\[0pt]
Le noir et blanc \\[0pt]
Le nomade \\[0pt]
Le nomadisme \\[0pt]
Le nombre \\[0pt]
Le nombre et la mesure \\[0pt]
Le nom propre \\[0pt]
Le nom propre et le nom commun \\[0pt]
Le non-être \\[0pt]
Le non-sens \\[0pt]
Le normal et le pathologique \\[0pt]
L'enquête \\[0pt]
L'enquête de terrain \\[0pt]
L'enquête sociale \\[0pt]
L'enthousiasme \\[0pt]
L'enthousiasme est-il moral ? \\[0pt]
L'entraide \\[0pt]
Le nu \\[0pt]
L'envie \\[0pt]
L'environnement est-il un nouvel objet pour les sciences humaines ? \\[0pt]
Le oui-dire \\[0pt]
Le pacifisme \\[0pt]
Le paradigme \\[0pt]
Le paradoxal \\[0pt]
Le paradoxe \\[0pt]
Le pardon \\[0pt]
Le pardon et l'oubli \\[0pt]
Le parfait \\[0pt]
Le pari \\[0pt]
Le partage \\[0pt]
Le partage des biens \\[0pt]
Le partage des connaissances \\[0pt]
Le partage des savoirs \\[0pt]
Le partage est-il une obligation morale ? \\[0pt]
Le particulier \\[0pt]
Le passage à l'acte \\[0pt]
Le passé \\[0pt]
Le passé a-t-il plus de réalité que l'avenir ? \\[0pt]
Le passé est-il indépassable ? \\[0pt]
Le passé est-il objet de science ? \\[0pt]
Le passé peut-il être un objet de connaissance ? \\[0pt]
Le passionné mérite-t-il d'être plaint ? \\[0pt]
Le paternalisme \\[0pt]
Le pathologique \\[0pt]
Le patriarcat \\[0pt]
Le patrimoine \\[0pt]
Le patrimoine artistique \\[0pt]
Le patriotisme \\[0pt]
Le paysage \\[0pt]
Le paysage urbain \\[0pt]
Le pays natal \\[0pt]
Le péché \\[0pt]
Le pédagogue \\[0pt]
Le peintre et le sculpteur \\[0pt]
Le peintre et son modèle \\[0pt]
Le père, la mère, l'enfant \\[0pt]
Le pessimisme \\[0pt]
Le peuple a-t-il toujours raison ? \\[0pt]
Le peuple est-il souverain ? \\[0pt]
Le peuple et la nation \\[0pt]
Le peuple et les élites \\[0pt]
Le peuple possède-t-il une volonté ? \\[0pt]
Le phantasme \\[0pt]
L'éphémère \\[0pt]
Le phénomène \\[0pt]
Le phénomène religieux \\[0pt]
Le philanthrope \\[0pt]
Le philosophe a-t-il besoin de l'histoire ? \\[0pt]
Le philosophe a-t-il des leçons à donner au politique ? \\[0pt]
Le philosophe est-il le vrai politique ? \\[0pt]
Le philosophe et l'écrivain \\[0pt]
Le philosophe et le médecin \\[0pt]
Le philosophe et l'enfant \\[0pt]
Le philosophe-roi \\[0pt]
Le philosophe se détourne-t-il du réel ? \\[0pt]
L'épistémologie des sciences humaines \\[0pt]
L'épistémologie est-elle une logique de la science ? \\[0pt]
Le plagiat \\[0pt]
Le plaisir \\[0pt]
Le plaisir artistique est-il affaire de jugement ? \\[0pt]
Le plaisir a-t-il un rôle à jouer dans la morale ? \\[0pt]
Le plaisir de l'art \\[0pt]
Le plaisir d'imiter \\[0pt]
Le plaisir esthétique \\[0pt]
Le plaisir esthétique est-il un plaisir ? \\[0pt]
Le plaisir esthétique peut-il se communiquer ? \\[0pt]
Le plaisir esthétique suppose-t-il une culture ? \\[0pt]
Le plaisir esthétique suppose-t-il une culture esthétique ? \\[0pt]
Le plaisir est-il immoral ? \\[0pt]
Le plaisir est-il la fin du désir ? \\[0pt]
Le plaisir est-il un bien ? \\[0pt]
Le plaisir et le bien \\[0pt]
Le plaisir se mérite-t-il ? \\[0pt]
Le pluralisme \\[0pt]
Le pluralisme politique \\[0pt]
Le plus et le moins \\[0pt]
Le plus grand des maux \\[0pt]
Le poète réinvente-t-il la langue ? \\[0pt]
Le poids des circonstances \\[0pt]
Le poids du passé \\[0pt]
Le poids du préjugé en politique \\[0pt]
Le point de vue \\[0pt]
Le point de vue de l'auteur \\[0pt]
Le point de vue du spectateur \\[0pt]
Le politique a-t-il à régler les passions ? \\[0pt]
Le politique a-t-il à régler les passions humaines ? \\[0pt]
Le politique doit-il être un technicien ? \\[0pt]
Le politique doit-il se soucier des émotions ? \\[0pt]
Le politique et le religieux \\[0pt]
Le politique et le social \\[0pt]
Le politique peut-il faire abstraction de la morale ? \\[0pt]
Le populaire \\[0pt]
Le populisme \\[0pt]
Le portrait \\[0pt]
Le possible \\[0pt]
Le possible et le contingent \\[0pt]
Le possible et le probable \\[0pt]
Le possible et le réel \\[0pt]
Le possible et l'existence \\[0pt]
Le possible et l'impossible \\[0pt]
Le postulat et l'axiome \\[0pt]
Le pour et le contre \\[0pt]
Le pouvoir absolu \\[0pt]
Le pouvoir causal de l'inconscient \\[0pt]
Le pouvoir corrompt-il ? \\[0pt]
Le pouvoir corrompt-il nécessairement ? \\[0pt]
Le pouvoir de la beauté \\[0pt]
Le pouvoir de la minorité \\[0pt]
Le pouvoir de l'argent \\[0pt]
Le pouvoir de la science \\[0pt]
Le pouvoir de l'image \\[0pt]
Le pouvoir de l'opinion \\[0pt]
Le pouvoir des idées \\[0pt]
Le pouvoir des images \\[0pt]
Le pouvoir des meilleurs \\[0pt]
Le pouvoir des mots \\[0pt]
Le pouvoir des savants \\[0pt]
Le pouvoir des sciences humaines et sociales \\[0pt]
Le pouvoir du peuple \\[0pt]
Le pouvoir législatif \\[0pt]
Le pouvoir peut-il arrêter le pouvoir ? \\[0pt]
Le pouvoir peut-il ignorer la morale ? \\[0pt]
Le pouvoir peut-il limiter le pouvoir ? \\[0pt]
Le pouvoir peut-il se déléguer ? \\[0pt]
Le pouvoir peut-il se passer de sa mise en scène ? \\[0pt]
Le pouvoir peut-il se transmettre ? \\[0pt]
Le pouvoir politique est-il nécessairement coercitif ? \\[0pt]
Le pouvoir politique peut-il échapper à l'arbitraire ? \\[0pt]
Le pouvoir politique repose-t-il sur la contrainte ? \\[0pt]
Le pouvoir politique repose-t-il sur un savoir ? \\[0pt]
Le pouvoir se partage-t-il ? \\[0pt]
Le pouvoir souverain \\[0pt]
Le pouvoir spirituel \\[0pt]
Le pouvoir tend-il toujours à l'excès ? \\[0pt]
Le pouvoir traditionnel \\[0pt]
Le préférable \\[0pt]
Le préjugé \\[0pt]
Le premier devoir de l'État est-il de se défendre ? \\[0pt]
Le premier et le primitif \\[0pt]
Le premier principe \\[0pt]
Le présent \\[0pt]
L'épreuve de la liberté \\[0pt]
L'épreuve de la réalité \\[0pt]
L'épreuve des faits \\[0pt]
L'épreuve du pouvoir \\[0pt]
Le primitivisme en art \\[0pt]
Le prince \\[0pt]
Le principe de causalité \\[0pt]
Le principe de contradiction \\[0pt]
Le principe d'égalité \\[0pt]
Le principe de précaution \\[0pt]
Le principe de raison \\[0pt]
Le principe de réalité \\[0pt]
Le principe de réciprocité \\[0pt]
Le principe de simplicité \\[0pt]
Le principe d'identité \\[0pt]
Le privé et le public \\[0pt]
Le privilège de l'original \\[0pt]
Le prix \\[0pt]
Le prix de la liberté \\[0pt]
Le prix de la vérité \\[0pt]
Le probabilisme \\[0pt]
Le probable \\[0pt]
Le problème de l'être \\[0pt]
Le processus \\[0pt]
Le processus de civilisation \\[0pt]
Le prochain \\[0pt]
Le proche et le lointain \\[0pt]
Le profane \\[0pt]
Le profit est-il la fin de l'échange ? \\[0pt]
Le profit est-il le moteur de la production économique ? \\[0pt]
Le progrès \\[0pt]
Le progrès des sciences \\[0pt]
Le progrès des sciences infirme-t-il les résultats anciens ? \\[0pt]
Le progrès en logique \\[0pt]
Le progrès moral \\[0pt]
Le progrès scientifique fait-il disparaître la superstition ? \\[0pt]
Le progrès technique \\[0pt]
Le projet \\[0pt]
Le projet d'une paix perpétuelle est-il insensé ? \\[0pt]
Le projet politique \\[0pt]
Le propre \\[0pt]
Le propre de la musique \\[0pt]
Le propre de l'homme \\[0pt]
Le propre et l'impropre \\[0pt]
Le propriétaire \\[0pt]
Le protectionnisme \\[0pt]
Le provisoire \\[0pt]
Le psychisme est-il objet de connaissance ? \\[0pt]
Le psychologisme \\[0pt]
Le public \\[0pt]
Le public et le privé \\[0pt]
Le pur et l'impur \\[0pt]
Lequel, de l'art ou du réel, est-il une imitation de l'autre ? \\[0pt]
Le quelque chose \\[0pt]
L'équilibre \\[0pt]
L'équilibre des pouvoirs \\[0pt]
L'équilibre économique \\[0pt]
L'équité \\[0pt]
L'équivalence \\[0pt]
L'équivalence des hypothèses \\[0pt]
L'équivocité \\[0pt]
L'équivocité du langage \\[0pt]
L'équivoque \\[0pt]
Le quotidien \\[0pt]
Le raffinement \\[0pt]
Le raisonnement par l'absurde \\[0pt]
Le raisonnement scientifique \\[0pt]
Le raisonnement suit-il des règles ? \\[0pt]
Le rapport de force \\[0pt]
Le rapport de l'homme à son milieu a-t-il une dimension morale ? \\[0pt]
Le rationnel et le raisonnable \\[0pt]
Le réalisme \\[0pt]
Le réalisme de la science \\[0pt]
Le réalisme politique \\[0pt]
Le récit \\[0pt]
Le récit en histoire \\[0pt]
Le récit national \\[0pt]
Le recours à la force signifie-t-il l'échec de la justice ? \\[0pt]
Le réel \\[0pt]
Le réel est-il ce qui résiste ? \\[0pt]
Le réel est-il rationnel ? \\[0pt]
Le réel et le virtuel \\[0pt]
Le réel et l'idéal \\[0pt]
Le réel et l'impossible \\[0pt]
Le réel excède-t-il l'intelligible ? \\[0pt]
Le réel peut-il être beau ? \\[0pt]
Le réel peut-il être contradictoire ? \\[0pt]
Le réel se donne-t-il à voir ? \\[0pt]
Le refoulement \\[0pt]
Le refus \\[0pt]
Le refus de l'autorité \\[0pt]
Le refus de la vérité \\[0pt]
Le regard \\[0pt]
Le regard de l'autre \\[0pt]
Le regard du photographe \\[0pt]
Le regard éloigné \\[0pt]
Le règlement politique des conflits ? \\[0pt]
Le règne de la technique \\[0pt]
Le règne de l'homme \\[0pt]
Le règne des experts \\[0pt]
Le règne des fins \\[0pt]
Le regret \\[0pt]
Le relativisme \\[0pt]
Le relativisme culturel \\[0pt]
Le relativisme culturel est-il indépassable ? \\[0pt]
Le relativisme dans les sciences humaines \\[0pt]
Le relativisme moral \\[0pt]
Le relativisme peut-il être un principe de méthode dans les sciences humaines ? \\[0pt]
Le remords \\[0pt]
Le renoncement \\[0pt]
Le repas \\[0pt]
Le repentir \\[0pt]
Le repos \\[0pt]
Le respect \\[0pt]
Le respect des convenances \\[0pt]
Le respect des institutions \\[0pt]
Le ressentiment \\[0pt]
Le retour à la nature est-il souhaitable ? \\[0pt]
Le retour à l'expérience \\[0pt]
Le rêve \\[0pt]
Le rêve et la veille \\[0pt]
Le révisionnisme \\[0pt]
Le ridicule \\[0pt]
Le rien \\[0pt]
Le rigorisme \\[0pt]
Le rire \\[0pt]
Le risque \\[0pt]
Le risque calculé \\[0pt]
Le risque technique \\[0pt]
Le rituel \\[0pt]
Le rôle de la théorie dans l'expérience scientifique \\[0pt]
Le rôle des institutions \\[0pt]
Le roman national \\[0pt]
L'érotisme \\[0pt]
Le royaume du possible \\[0pt]
L'erreur \\[0pt]
L'erreur et la faute \\[0pt]
L'erreur et l'ignorance \\[0pt]
L'erreur peut-elle donner un accès à la vérité ? \\[0pt]
L'erreur peut-elle jouer un rôle dans la connaissance scientifique ? \\[0pt]
L'erreur politique, la faute politique \\[0pt]
L'erreur scientifique \\[0pt]
L'érudition \\[0pt]
Le rythme \\[0pt]
Le rythme en peinture \\[0pt]
Le sacré \\[0pt]
Le sacré et le profane \\[0pt]
Le sacrifice \\[0pt]
Le sacrifice de soi \\[0pt]
Les affaires publiques \\[0pt]
Les affections \\[0pt]
Les affects sont-ils des objets sociologiques ? \\[0pt]
Les agents économiques ont-ils un comportement rationnel ? \\[0pt]
Les agents sociaux poursuivent-ils l'utilité ? \\[0pt]
Les agents sociaux sont-ils rationnels ? \\[0pt]
Les âges de la vie \\[0pt]
Le salut \\[0pt]
Le salut du peuple \\[0pt]
Les amis \\[0pt]
Les analogies dans les sciences humaines \\[0pt]
Les anciens et les modernes \\[0pt]
Les animaux échappent-ils à la moralité ? \\[0pt]
Les animaux ont-ils des droits ? \\[0pt]
Les animaux pensent-ils ? \\[0pt]
Les animaux révèlent-ils ce que nous sommes ? \\[0pt]
Les animaux sont-ils nos amis ? \\[0pt]
Les antagonismes sociaux \\[0pt]
Les apparences font-elles partie du monde ? \\[0pt]
Les apparences sont-elles toujours trompeuses ? \\[0pt]
Les archives \\[0pt]
Les artistes sont-ils sérieux ? \\[0pt]
Les arts appliqués \\[0pt]
Les arts communiquent-ils entre eux ? \\[0pt]
Les arts décoratifs \\[0pt]
Les arts de la mémoire \\[0pt]
Les arts industriels \\[0pt]
Les arts mineurs \\[0pt]
Les arts nobles \\[0pt]
Les arts ont-ils besoin de théorie ? \\[0pt]
Les arts populaires \\[0pt]
Les arts sont-ils des jeux ? \\[0pt]
Les arts vivants \\[0pt]
Les atomes \\[0pt]
Les autorités \\[0pt]
Les autres \\[0pt]
Le sauvage \\[0pt]
Le sauvage et le barbare \\[0pt]
Le sauvage et le civilisé \\[0pt]
Le sauvage et le cultivé \\[0pt]
Le savant et le politique \\[0pt]
Le savant, l'artiste et le politique \\[0pt]
Le savoir a-t-il besoin d'être fondé ? \\[0pt]
Le savoir du peintre \\[0pt]
Le savoir émancipe-t-il ? \\[0pt]
Le savoir est-il libérateur ? \\[0pt]
Le savoir-faire \\[0pt]
Le savoir peut-il être gratuit ? \\[0pt]
Le savoir se vulgarise-t-il ? \\[0pt]
Le savoir utile au citoyen \\[0pt]
Les beautés de la nature \\[0pt]
Les beaux-arts \\[0pt]
Les beaux-arts sont-ils compatibles entre eux ? \\[0pt]
Les belles actions \\[0pt]
Les bénéfices du doute \\[0pt]
Les bénéfices moraux \\[0pt]
Les bêtes \\[0pt]
Les biens communs \\[0pt]
Les biens culturels \\[0pt]
Les blessures de l'esprit \\[0pt]
Les bonnes intentions \\[0pt]
Les bonnes lois font-elles les bonnes mœurs ? \\[0pt]
Les bonnes mœurs \\[0pt]
Les bons sentiments \\[0pt]
Les bornes de la conscience \\[0pt]
Le scandale \\[0pt]
Les canons de la beauté \\[0pt]
Les caractères moraux \\[0pt]
Les cas de conscience \\[0pt]
Les catégories \\[0pt]
Les catégories sont-elles définitives ? \\[0pt]
Les catégories sont-elles des effets de langue ? \\[0pt]
Les causes \\[0pt]
Les causes et les conditions \\[0pt]
Les causes et les effets \\[0pt]
Les causes et les lois \\[0pt]
Les causes finales \\[0pt]
Les causes naturelles expliquent-elles la nature ? \\[0pt]
Les cérémonies \\[0pt]
Les changements scientifiques et la réalité \\[0pt]
Les chemins de traverse \\[0pt]
Les choses \\[0pt]
Les choses humaines \\[0pt]
Les choses ont-elles quelque chose en commun ? \\[0pt]
Les choses ont-elles une essence ? \\[0pt]
Les choses peuvent-elles avoir un sens ? \\[0pt]
Les choses sont-elles dans l'espace ? \\[0pt]
Les cinq sens \\[0pt]
Les circonstances \\[0pt]
Les classes sociales \\[0pt]
L'esclavage \\[0pt]
L'esclave \\[0pt]
L'esclave et son maître \\[0pt]
Les clichés \\[0pt]
Les commandements divins \\[0pt]
Les conditions de la démocratie \\[0pt]
Les conditions d'une science historique \\[0pt]
Les conflits politiques \\[0pt]
Les conflits politiques ne sont-ils que des conflits sociaux ? \\[0pt]
Les conflits sociaux \\[0pt]
Les conflits sociaux sont-ils des conflits de classe ? \\[0pt]
Les conflits sociaux sont-ils des conflits politiques ? \\[0pt]
Les connaissances scientifiques peuvent-elles être à la fois vraies et provisoires ? \\[0pt]
Les connaissances scientifiques peuvent-elles être vulgarisées ? \\[0pt]
Les conquêtes de la science \\[0pt]
Les conséquences de l'action \\[0pt]
Les contradictions de la raison \\[0pt]
Les contradictions sont-elles un échec de la raison ? \\[0pt]
Les contre-pouvoirs \\[0pt]
Les convenances \\[0pt]
Les conventions \\[0pt]
Les couleurs \\[0pt]
Les couleurs existent-elles en dehors de notre esprit ? \\[0pt]
Les coutumes \\[0pt]
Les critères de vérité dans les sciences humaines \\[0pt]
Les croyances politiques \\[0pt]
Les croyances sont-elles utiles ? \\[0pt]
Le scrupule \\[0pt]
Les cultures sont-elles incommensurables ? \\[0pt]
Les décisions politiques peuvent-elles être collectives ? \\[0pt]
Les degrés de conscience \\[0pt]
Les degrés de la beauté \\[0pt]
Les degrés du beau \\[0pt]
Les désaccords politiques peuvent-ils être surmontés ? \\[0pt]
Les devoirs à l'égard de la nature \\[0pt]
Les devoirs de l'État \\[0pt]
Les devoirs envers nos proches \\[0pt]
Les devoirs envers soi-même \\[0pt]
Les dictionnaires \\[0pt]
Les différences entre les sciences humaines sont-elles des différences d'objet ? \\[0pt]
Les différentes preuves de l'existence de Dieu prouvent-elles le même Dieu ? \\[0pt]
Les difficultés de l'observation \\[0pt]
Les dilemmes moraux \\[0pt]
Les dispositions sociales \\[0pt]
Les distinctions sociales \\[0pt]
Les divisions entre les hommes \\[0pt]
Les divisions sociales \\[0pt]
Les dogmes \\[0pt]
Les droits de la nature \\[0pt]
Les droits de l'enfant \\[0pt]
Les droits de l'homme \\[0pt]
Les droits de l'homme et ceux du citoyen \\[0pt]
Les droits de l'homme ont-ils un fondement moral ? \\[0pt]
Les droits de l'homme sont-ils ceux du citoyen ? \\[0pt]
Les droits de l'homme sont-ils une abstraction ? \\[0pt]
Les droits et les devoirs \\[0pt]
Les droits naturels imposent-ils une limite à la politique ? \\[0pt]
Les droits sociaux \\[0pt]
Les échanges \\[0pt]
Les échanges, facteurs de paix ? \\[0pt]
Le secret \\[0pt]
Le secret d'État \\[0pt]
Les effets de l'esclavage \\[0pt]
Les éléments \\[0pt]
Le semblable \\[0pt]
Les émotions politiques \\[0pt]
Le sens commun \\[0pt]
Le sens commun existe-t-il ? \\[0pt]
Le sens de la famille \\[0pt]
Le sens de la fête \\[0pt]
Le sens de la mesure \\[0pt]
Le sens de la réalité \\[0pt]
Le sens de la situation \\[0pt]
Le sens de la vie \\[0pt]
Le sens de l'État \\[0pt]
Le sens de l'être \\[0pt]
Le sens de l'existence \\[0pt]
Le sens de l'histoire \\[0pt]
Le sens de l'Histoire \\[0pt]
Le sens de l'humour \\[0pt]
Le sens de l'opportunité est-il moral ? \\[0pt]
Le sens des mots \\[0pt]
Le sens des mots dépend-il de notre connaissance des choses ? \\[0pt]
Le sens d'un texte \\[0pt]
Le sens du sacrifice \\[0pt]
Le sens du silence \\[0pt]
Le sens du travail \\[0pt]
Les ensembles \\[0pt]
Le sens et le non-sens \\[0pt]
Le sensible \\[0pt]
Le sensible est-il communicable ? \\[0pt]
Le sensible est-il irréductible à l'intelligible ? \\[0pt]
Le sensible peut-il fonder des certitudes ? \\[0pt]
Le sens interne \\[0pt]
Le sens moral \\[0pt]
Le sens musical \\[0pt]
Le sens pratique \\[0pt]
Le sens propre \\[0pt]
Le sentiment de la faute \\[0pt]
Le sentiment de l'existence \\[0pt]
Le sentiment de l'injustice \\[0pt]
Le sentiment d'injustice \\[0pt]
Le sentiment du libre arbitre \\[0pt]
Le sentiment esthétique \\[0pt]
Le sentiment moral \\[0pt]
Les entités mathématiques sont-elles des fictions ? \\[0pt]
Les envieux \\[0pt]
Les épreuves ont-elles un sens ? \\[0pt]
Le sérieux \\[0pt]
Le serment \\[0pt]
Le service public \\[0pt]
Les États ont-ils pour fin d'assurer la paix ? \\[0pt]
Les éthiques professionnelles \\[0pt]
Les études comparatives \\[0pt]
Les études de genre \\[0pt]
Les événements \\[0pt]
Les exceptions ont-elles une place en morale ? \\[0pt]
Les experts \\[0pt]
Les factions politiques \\[0pt]
Les faits \\[0pt]
Les faits et les valeurs \\[0pt]
Les faits parlent-ils d'eux-mêmes ? \\[0pt]
Les faits sociaux se répètent-ils ? \\[0pt]
Les faits sont-ils têtus ? \\[0pt]
Les fausses sciences \\[0pt]
Les fictions \\[0pt]
Les fins de l'art \\[0pt]
Les fins de l'éducation \\[0pt]
Les fins dernières \\[0pt]
Les fins naturelles et les fins morales \\[0pt]
Les fins sont-elles toujours intentionnelles ? \\[0pt]
Les fonctions de l'image \\[0pt]
Les fondements de l'État \\[0pt]
Les forces de l'ordre \\[0pt]
Les formes de vie \\[0pt]
Les forts et les faibles \\[0pt]
Les foules \\[0pt]
Les fous \\[0pt]
Les frontières \\[0pt]
Les frontières de la connaissance \\[0pt]
Les frontières de l'art \\[0pt]
Les frontières du moi \\[0pt]
Les fruits du travail \\[0pt]
Les genres de Dieu \\[0pt]
Les genres de l'être \\[0pt]
Les genres esthétiques \\[0pt]
Les genres naturels \\[0pt]
Les grands hommes \\[0pt]
Les habitudes de pensée \\[0pt]
Les hasards de la vie \\[0pt]
Les hommes croient-ils dans leurs mythes ? \\[0pt]
Les hommes de pouvoir \\[0pt]
Les hommes et les dieux \\[0pt]
Les hommes et les femmes \\[0pt]
Les hommes n'agissent-ils que par intérêt ? \\[0pt]
Les hommes sont-ils faits pour s'entendre ? \\[0pt]
Les hommes sont-ils naturellement sociables ? \\[0pt]
Les hors-la-loi \\[0pt]
Les idées et les choses \\[0pt]
Les idées existent-elles ? \\[0pt]
Les idées mènent-elles le monde ? \\[0pt]
Les idées ont-elles une histoire ? \\[0pt]
Les idées ont-elles une réalité ? \\[0pt]
Les idées politiques \\[0pt]
Les idées reçues \\[0pt]
Les idées sont-elles vivantes ? \\[0pt]
Les idoles \\[0pt]
Le signe et l'indice \\[0pt]
Le silence \\[0pt]
Le silence des lois \\[0pt]
Les images empêchent-elles de penser ? \\[0pt]
Les images nous égarent-elles ? \\[0pt]
Les images peuvent-elles instruire ? \\[0pt]
Le simple \\[0pt]
Le simple et le complexe \\[0pt]
Le simulacre \\[0pt]
Les individus \\[0pt]
Les individus se réduisent-ils à leurs qualités ? \\[0pt]
Les industries culturelles \\[0pt]
Les inégalités sociales \\[0pt]
Les inégalités sociales sont-elles inévitables ? \\[0pt]
Le singulier \\[0pt]
Le singulier est-il objet de connaissance ? \\[0pt]
Le singulier et le pluriel \\[0pt]
Les institutions artistiques \\[0pt]
Les institutions de l'art \\[0pt]
Les institutions politiques : qu'ont-elles en propre ? \\[0pt]
Les instruments de la pensée \\[0pt]
Les intentions de l'artiste \\[0pt]
Les intentions et les conséquences \\[0pt]
Les interdits \\[0pt]
Les intérêts particuliers peuvent-ils tempérer l'autorité politique ? \\[0pt]
Les invariants culturels \\[0pt]
Les jeux de hasard \\[0pt]
Les jeux du pouvoir \\[0pt]
Les jugements analytiques \\[0pt]
Les jugements de goût sont-ils susceptibles de vérité ? \\[0pt]
Les langages de l'art \\[0pt]
Les langages formels \\[0pt]
Les langues meurent-elles ? \\[0pt]
Les leçons de l'expérience \\[0pt]
Les leçons de morale \\[0pt]
Les libertés civiles \\[0pt]
Les libertés fondamentales \\[0pt]
Les libertés politiques \\[0pt]
Les libertés publiques \\[0pt]
Les liens \\[0pt]
Les liens sociaux \\[0pt]
Les lieux de mémoire \\[0pt]
Les lieux du pouvoir \\[0pt]
Les limites de la connaissance scientifique \\[0pt]
Les limites de la démocratie \\[0pt]
Les limites de la description \\[0pt]
Les limites de la raison \\[0pt]
Les limites de la tolérance \\[0pt]
Les limites de la vérité \\[0pt]
Les limites de la vertu \\[0pt]
Les limites de l'État \\[0pt]
Les limites de l'expérience \\[0pt]
Les limites de l'humain \\[0pt]
Les limites de l'imagination \\[0pt]
Les limites de l'interprétation \\[0pt]
Les limites de l'obéissance \\[0pt]
Les limites de ma langue sont-elles les limites de mon monde ? \\[0pt]
Les limites du corps \\[0pt]
Les limites du déterminisme \\[0pt]
Les limites du langage \\[0pt]
Les limites du plaisir \\[0pt]
Les limites du possible \\[0pt]
Les limites du pouvoir \\[0pt]
Les limites du pouvoir politique \\[0pt]
Les limites du réel \\[0pt]
Les limites du vivant \\[0pt]
Les lois causales \\[0pt]
Les lois de la guerre \\[0pt]
Les lois de la nature \\[0pt]
Les lois de la nature sont-elles contingentes ? \\[0pt]
Les lois de la nature sont-elles de simples régularités ? \\[0pt]
Les lois de la nature sont elles nécessaires ? \\[0pt]
Les lois de l'art \\[0pt]
Les lois de l'histoire \\[0pt]
Les lois de l'hospitalité \\[0pt]
Les lois du sang \\[0pt]
Les lois et les mœurs \\[0pt]
Les lois nous rendent-elles meilleurs ? \\[0pt]
Les lois scientifiques sont-elles des lois de la nature ? \\[0pt]
Les lois sont-elles sacrées ? \\[0pt]
Les lois sont-elles seulement utiles ? \\[0pt]
Les lois suffisent-elles à garantir nos libertés ? \\[0pt]
Les luttes politiques \\[0pt]
Les machines \\[0pt]
Les maladies de l'âme \\[0pt]
Les maladies de l'esprit \\[0pt]
Les maladies mentales sont-elles des maladies sociales ? \\[0pt]
Les marginaux \\[0pt]
Les masses \\[0pt]
Les matériaux \\[0pt]
Les mathématiques du mouvement \\[0pt]
Les mathématiques et la pensée de l'infini \\[0pt]
Les mathématiques et le monde sensible \\[0pt]
Les mathématiques peuvent-elles se passer de l'intuition ? \\[0pt]
Les mathématiques sont-elles le modèle de toute science ? \\[0pt]
Les mathématiques sont-elles réductibles à la logique ? \\[0pt]
Les mathématiques sont-elles une science de l'ordre ? \\[0pt]
Les mathématiques sont-elles un langage ? \\[0pt]
Les mathématiques sont-elles utiles au philosophe ? \\[0pt]
Les mauvaises habitudes \\[0pt]
Les mauvaises intentions \\[0pt]
Les mécanismes cérébraux \\[0pt]
Les mécanismes d'identification \\[0pt]
Les métamorphoses du goût \\[0pt]
Les miracles \\[0pt]
Les miracles de la technique \\[0pt]
Les modalités \\[0pt]
Les modèles \\[0pt]
Les mœurs \\[0pt]
Les mœurs et la morale \\[0pt]
Les mœurs sont-elles objet de science ? \\[0pt]
Les mœurs sont-ils l'objet d'une science ? \\[0pt]
Les mondes possibles \\[0pt]
Les mots disent-ils les choses ? \\[0pt]
Les mots et les choses \\[0pt]
Les mots et les concepts \\[0pt]
Les mots justes \\[0pt]
Les mots n'ont-ils que le pouvoir qu'on leur donne ? \\[0pt]
Les mots ont-ils un pouvoir ? \\[0pt]
Les moyens de l'art \\[0pt]
Les moyens de l'autorité \\[0pt]
Les moyens et la fin \\[0pt]
Les moyens et les fins \\[0pt]
Les moyens et les fins en art \\[0pt]
Les muses \\[0pt]
Les mythes collectifs \\[0pt]
Les mythes relèvent-ils de l'anthropologie ? \\[0pt]
Les nombres gouvernent-ils le monde ? \\[0pt]
Les noms \\[0pt]
Les noms divins \\[0pt]
Les noms propres \\[0pt]
Les normes \\[0pt]
Les normes du savoir sont-elles universelles ? \\[0pt]
Les normes du vivant \\[0pt]
Les normes esthétiques \\[0pt]
Les normes et les valeurs \\[0pt]
Les nouvelles technologies transforment-elles l'idée de l'art ? \\[0pt]
Les objets de pensée \\[0pt]
Les objets existent-ils ? \\[0pt]
Les objets impossibles \\[0pt]
Les objets scientifiques \\[0pt]
Les objets techniques nous imposent-ils une manière de vivre ? \\[0pt]
Le social et le politique \\[0pt]
Le sociologisme \\[0pt]
Le sociologue et l'historien \\[0pt]
Les œuvres d'art doivent-elles faire l'objet d'une évaluation morale ? \\[0pt]
Les œuvres d'art ont-elles besoin d'un commentaire ? \\[0pt]
Le soin \\[0pt]
Le soleil se lèvera demain \\[0pt]
Le solipsisme \\[0pt]
Le sommeil de la raison \\[0pt]
Le sommeil et la veille \\[0pt]
Les opérations de la pensée \\[0pt]
Les opinions politiques \\[0pt]
Les origines de l'humanité \\[0pt]
L'ésotérisme \\[0pt]
Le souci \\[0pt]
Le souci d'autrui résume-t-il la morale ? \\[0pt]
Le souci de l'avenir \\[0pt]
Le souci de l'ordre \\[0pt]
Le souci de soi \\[0pt]
Le souci de soi est-il une attitude morale ? \\[0pt]
Le souci du bien-être est-il politique ? \\[0pt]
Le souverain bien \\[0pt]
L'espace dans la peinture \\[0pt]
L'espace de la perception \\[0pt]
L'espace du tableau \\[0pt]
L'espace et le lieu \\[0pt]
L'espace et le territoire \\[0pt]
L'espace intérieur \\[0pt]
L'espace public \\[0pt]
L'espace social \\[0pt]
Les paroles et les actes \\[0pt]
« Les paroles s'envolent, les écrits restent » \\[0pt]
Les parties de l'âme \\[0pt]
Les partis politiques \\[0pt]
Les passions peuvent-elles être raisonnables ? \\[0pt]
Les passions politiques \\[0pt]
Les passions sont-elles mauvaises ? \\[0pt]
Les passions sont-elles un obstacle à la vie sociale ? \\[0pt]
Les pauvres \\[0pt]
L'espèce et l'individu \\[0pt]
L'espèce humaine \\[0pt]
Le spectacle \\[0pt]
Le spectacle de la nature \\[0pt]
Le spectacle de la pensée \\[0pt]
Le spectateur \\[0pt]
L'espérance \\[0pt]
L'espérance est-elle une vertu ? \\[0pt]
Les personnes et les choses \\[0pt]
Les peuples ont-ils les gouvernements qu'ils méritent ? \\[0pt]
Les phénomènes \\[0pt]
Les phénomènes de mode \\[0pt]
Les phénomènes inconscients sont-ils réductibles à une mécanique cérébrale ? \\[0pt]
Le spirituel et le temporel \\[0pt]
Les plaisirs \\[0pt]
Les plaisirs de l'amitié \\[0pt]
Les poètes et la cité \\[0pt]
L'espoir \\[0pt]
Les politiques publiques \\[0pt]
Les pouvoirs de la parole \\[0pt]
Les pouvoirs de la religion \\[0pt]
Les préjugés moraux \\[0pt]
Les prêtres \\[0pt]
Les preuves de la liberté \\[0pt]
Les principes de la démonstration \\[0pt]
Les principes d'une science sont-ils des conventions ? \\[0pt]
Les principes et les éléments \\[0pt]
Les principes moraux \\[0pt]
Les principes sont-ils indémontrables ? \\[0pt]
L'esprit appartient-il à la nature ? \\[0pt]
L'esprit critique \\[0pt]
L'esprit de finesse \\[0pt]
L'esprit de sérieux \\[0pt]
L'esprit de système \\[0pt]
L'esprit d'invention \\[0pt]
L'esprit est-il matériel ? \\[0pt]
L'esprit est-il objet de science ? \\[0pt]
L'esprit est-il plus aisé à connaître que le corps ? \\[0pt]
L'esprit est-il une machine ? \\[0pt]
L'esprit est-il un ensemble de facultés ? \\[0pt]
L'esprit et la machine \\[0pt]
L'esprit humain progresse-t-il ? \\[0pt]
L'esprit n'a-t-il jamais affaire qu'à lui-même ? \\[0pt]
L'esprit peut-il être malade ? \\[0pt]
L'esprit peut-il être mesuré ? \\[0pt]
L'esprit peut-il être objet de science ? \\[0pt]
L'esprit scientifique \\[0pt]
L'esprit s'explique-t-il par une activité cérébrale ? \\[0pt]
L'esprit tranquille \\[0pt]
Les problèmes politiques peuvent-ils se ramener à des problèmes techniques ? \\[0pt]
Les problèmes politiques sont-ils des problèmes techniques ? \\[0pt]
Les procédures de validation dans les sciences humaines ? \\[0pt]
Les propositions métaphysiques sont-elles des illusions ? \\[0pt]
Les proverbes \\[0pt]
Les proverbes enseignent-ils quelque chose ? \\[0pt]
Les proverbes nous instruisent-ils moralement ? \\[0pt]
Les qualités esthétiques \\[0pt]
Les questions métaphysiques ont-elles un sens ? \\[0pt]
L'esquisse \\[0pt]
Les raisons d'aimer \\[0pt]
Les raisons de vivre \\[0pt]
Les rapports sociaux sont-ils des rapports entre individus ? \\[0pt]
Les règles de l'art \\[0pt]
Les règles du jeu \\[0pt]
Les règles d'un bon gouvernement \\[0pt]
Les règles sociales \\[0pt]
Les règles sociales sont-elles répressives ? \\[0pt]
Les relations \\[0pt]
Les relations ont-elles une existence objective ? \\[0pt]
Les relations sociales \\[0pt]
Les religions sont-elles des illusions ? \\[0pt]
Les remords \\[0pt]
Les représentants du peuple \\[0pt]
Les représentations collectives \\[0pt]
Les reproductions \\[0pt]
Les réseaux sociaux \\[0pt]
Les ressources de l'analogie \\[0pt]
Les ressources humaines \\[0pt]
Les rêves ont-ils un sens ? \\[0pt]
Les révolutions scientifiques \\[0pt]
Les révolutions techniques suscitent-elles des révolutions dans l'art ? \\[0pt]
Les riches et les pauvres \\[0pt]
Les rites \\[0pt]
Les rites funéraires \\[0pt]
Les rituels \\[0pt]
Les rôles sociaux \\[0pt]
Les ruines \\[0pt]
Les sacrifices \\[0pt]
Les sans-papiers \\[0pt]
Les sauvages \\[0pt]
Les sciences décrivent-elles le réel ? \\[0pt]
Les sciences de la vie et de la Terre \\[0pt]
Les sciences de la vie visent-elles un objet irréductible à la matière ? \\[0pt]
Les sciences de l'éducation \\[0pt]
Les sciences de l'esprit \\[0pt]
Les sciences de l'homme et l'évolution \\[0pt]
Les sciences de l'homme ont-elles inventé leur objet ? \\[0pt]
Les sciences de l'homme permettent-elles d'affiner la notion de responsabilité ? \\[0pt]
Les sciences de l'homme peuvent-elles expliquer l'impuissance de la liberté ? \\[0pt]
Les sciences de l'homme rendent-elles l'homme prévisible ? \\[0pt]
Les sciences de l'homme sont-elles utiles à la société ? \\[0pt]
Les sciences doivent-elle prétendre à l'unification ? \\[0pt]
Les sciences du comportement \\[0pt]
Les sciences et le vivant \\[0pt]
Les sciences exactes \\[0pt]
Les sciences forment-elle un système ? \\[0pt]
Les sciences historiques \\[0pt]
Les sciences humaines doivent-elles être transdisciplinaires ? \\[0pt]
Les sciences humaines échappent-elles à l'abstraction ? \\[0pt]
Les sciences humaines éliminent-elles la contingence du futur ? \\[0pt]
Les sciences humaines, est-ce la mort de l'homme ? \\[0pt]
Les sciences humaines et le droit \\[0pt]
Les sciences humaines et l'expérience \\[0pt]
Les sciences humaines et sociales n'ont-elles de sciences que le nom ? \\[0pt]
Les sciences humaines finissent-elles de ruiner l'idée de liberté ? \\[0pt]
Les sciences humaines jouent elles un rôle dans l'exploitation de l'homme par l'homme ? \\[0pt]
Les sciences humaines nient-elles la liberté de l'homme ? \\[0pt]
Les sciences humaines nient-elles la liberté du sujet humain ? \\[0pt]
Les sciences humaines nous font-elles connaître l'homme ? \\[0pt]
Les sciences humaines nous obligent-elles à renoncer à l'idée de libre arbitre ? \\[0pt]
Les sciences humaines nous protègent-elles de l'essentialisme ? \\[0pt]
Les sciences humaines ont-elles une utilité ? \\[0pt]
Les sciences humaines ont-elles un objet commun ? \\[0pt]
Les sciences humaines opposent-elles histoire et système ? \\[0pt]
Les sciences humaines permettent-elles de comprendre la vie d'un homme ? \\[0pt]
Les sciences humaines peuvent-elles adopter les méthodes des sciences de la nature ? \\[0pt]
Les sciences humaines peuvent-elles éclairer la réflexion éthique ? \\[0pt]
Les sciences humaines peuvent-elles être objectives ? \\[0pt]
Les sciences humaines peuvent-elles faire abstraction de la nature ? \\[0pt]
Les sciences humaines peuvent-elles s'affranchir de l'ethnocentrisme ? \\[0pt]
Les sciences humaines peuvent-elles se passer de la notion d'inconscient ? \\[0pt]
Les sciences humaines présupposent-elles une définition de l'homme ? \\[0pt]
Les sciences humaines recherchent-elles des invariants ? \\[0pt]
Les sciences humaines rendent-elle l'homme prévisible ? \\[0pt]
Les sciences humaines reposent-elles sur des faits ? \\[0pt]
Les sciences humaines sont-elles des sciences ? \\[0pt]
Les sciences humaines sont-elles des sciences de la nature humaine ? \\[0pt]
Les sciences humaines sont-elles des sciences de la vie humaine ? \\[0pt]
Les sciences humaines sont-elles des sciences de l'esprit ? \\[0pt]
Les sciences humaines sont-elles des sciences de l'interprétation ? \\[0pt]
Les sciences humaines sont-elles des sciences d'interprétation ? \\[0pt]
Les sciences humaines sont-elles explicatives ou compréhensives ? \\[0pt]
Les sciences humaines sont-elles les sciences de l'esprit ? \\[0pt]
Les sciences humaines sont-elles nécessairement empiriques ? \\[0pt]
Les sciences humaines sont-elles normatives ? \\[0pt]
Les sciences humaines sont-elles relativistes ? \\[0pt]
Les sciences humaines sont-elles subversives ? \\[0pt]
Les sciences humaines suffisent-elles à connaître l'homme ? \\[0pt]
Les sciences humaines supposent-elles une définition de l'homme ? \\[0pt]
Les sciences humaines supposent-elles une rupture avec l'ordre naturel ? \\[0pt]
Les sciences humaines traitent-elles de l'homme ? \\[0pt]
Les sciences humaines traitent-elles de l'individu ? \\[0pt]
Les sciences humaines transforment-elles la notion de causalité ? \\[0pt]
Les sciences naturelles \\[0pt]
Les sciences ne sont-elles qu'une description du monde ? \\[0pt]
Les sciences ont-elles à penser leurs fondements ? \\[0pt]
Les sciences ont-elles besoin d'une fondation ? \\[0pt]
Les sciences ont-elles besoin d'une fondation métaphysique ? \\[0pt]
Les sciences peuvent-elles penser leurs fondements ? \\[0pt]
Les sciences peuvent-elles penser l'individu ? \\[0pt]
Les sciences produisent-elles des normes ? \\[0pt]
Les sciences sociales \\[0pt]
Les sciences sociales peuvent-elles être expérimentales ? \\[0pt]
Les sciences sociales peuvent-elles servir à gouverner ? \\[0pt]
Les sciences sociales sont-elles nécessairement inexactes ? \\[0pt]
Les secrets \\[0pt]
L'essence \\[0pt]
L'essence et l'existence \\[0pt]
Les sensations peuvent-elles se partager ? \\[0pt]
Les sens humains \\[0pt]
Les sens nous trompent-ils ? \\[0pt]
Les sens peuvent-ils nous tromper ? \\[0pt]
Les sentiments \\[0pt]
Les sentiments peuvent-ils s'apprendre ? \\[0pt]
Les services publics \\[0pt]
Les signes \\[0pt]
Les signes de l'humanité \\[0pt]
Les signes de l'intelligence \\[0pt]
Les sociétés animales \\[0pt]
Les sociétés évoluent-elles ? \\[0pt]
Les sociétés humaines se construisent-elles contre la nature ? \\[0pt]
Les sociétés ont-elles besoin de théâtre ? \\[0pt]
Les sociétés ont-elles un inconscient ? \\[0pt]
Les sociétés sans histoire \\[0pt]
Les sociétés savantes \\[0pt]
Les sociétés sont-elles hiérarchisables ? \\[0pt]
Les sociétés sont-elles imprévisibles ? \\[0pt]
Les statistiques comme outil dans les sciences humaines \\[0pt]
Les structures expliquent-elles tout ? \\[0pt]
Les structures que dégagent les sciences humaines agissent-elles ? \\[0pt]
Les structures sociales sont-elles économiques ? \\[0pt]
Les styles \\[0pt]
Les substances \\[0pt]
Les symboles du pouvoir \\[0pt]
Les systèmes \\[0pt]
Les tabous \\[0pt]
Le statut de l'axiome \\[0pt]
Le statut des hypothèses dans la démarche scientifique \\[0pt]
Les techniques artistiques \\[0pt]
Les techniques de manipulation des hommes peuvent-elles s'appuyer sur les sciences humaines ? \\[0pt]
Les techniques du corps \\[0pt]
Les théories scientifiques décrivent-elles la réalité ? \\[0pt]
Les théories scientifiques sont-elles vraies ? \\[0pt]
L'esthète \\[0pt]
L'esthète et l'artiste \\[0pt]
L'esthétique des ruines \\[0pt]
L'esthétique est-elle une métaphysique de l'art ? \\[0pt]
L'esthétisme \\[0pt]
L'estime \\[0pt]
L'estime de soi \\[0pt]
Les traditions \\[0pt]
Les transcendantaux \\[0pt]
Le style \\[0pt]
Le sublime \\[0pt]
Le substrat \\[0pt]
Le succès \\[0pt]
Le suicide \\[0pt]
Le sujet \\[0pt]
Le sujet de droit \\[0pt]
Le sujet de l'action \\[0pt]
Le sujet de la pensée \\[0pt]
Le sujet et l'objet \\[0pt]
Le sujet moral \\[0pt]
Les universaux \\[0pt]
Les universaux existent-ils ? \\[0pt]
Le superflu \\[0pt]
Les usages \\[0pt]
Les usages de l'art \\[0pt]
Les valeurs de la République \\[0pt]
Les valeurs ont-elles une histoire ? \\[0pt]
Les valeurs ont-elles une origine ? \\[0pt]
Les vérités éternelles \\[0pt]
Les vérités scientifiques sont-elles relatives ? \\[0pt]
Les vérités sont-elles toujours démontrables ? \\[0pt]
Les vertus \\[0pt]
Les vertus de la formalisation \\[0pt]
Les vertus de l'amour \\[0pt]
Les vertus politiques \\[0pt]
Les vertus sont-elles au principe de la morale ? \\[0pt]
Les vices de l'esprit \\[0pt]
Les visages du mal \\[0pt]
Les vivants \\[0pt]
Les vivants et les morts \\[0pt]
Le syllogisme \\[0pt]
Le symbole \\[0pt]
Le symbole, comme concept, dans les sciences humaines \\[0pt]
Le symbolique \\[0pt]
Le symbolisme \\[0pt]
Le symbolisme mathématique \\[0pt]
Le système des arts \\[0pt]
Le système des beaux-arts \\[0pt]
Le système des besoins \\[0pt]
Le tableau \\[0pt]
Le tableau vivant \\[0pt]
Le tact \\[0pt]
Le talent \\[0pt]
Le talent et le génie \\[0pt]
Le talent s'enseigne-t-il ? \\[0pt]
L'étant est-il le premier pensable ? \\[0pt]
L'État a-t-il pour finalité de maintenir l'ordre ? \\[0pt]
L'État a-t-il un fondement rationnel ? \\[0pt]
« L'État, c'est moi » \\[0pt]
L'État crée-t-il la liberté ? \\[0pt]
L'État de droit \\[0pt]
L'État de droit ? \\[0pt]
L'état de la nature \\[0pt]
L'état de nature \\[0pt]
L'état d'exception \\[0pt]
L'État doit-il disparaître ? \\[0pt]
L'État doit-il éduquer le citoyen ? \\[0pt]
L'État doit-il éduquer les citoyens ? \\[0pt]
L'État doit-il faire le bonheur des citoyens ? \\[0pt]
L'État est-il appelé à disparaître ? \\[0pt]
L'État est-il ennemi de la liberté ? \\[0pt]
L'État est-il fin ou moyen ? \\[0pt]
L'État est-il le garant de la propriété privée ? \\[0pt]
L'État est-il l'ennemi de la liberté ? \\[0pt]
L'État est-il un moindre mal ? \\[0pt]
L'État et la culture \\[0pt]
L'État et la guerre \\[0pt]
L'État et la Nation \\[0pt]
L'État et la violence \\[0pt]
L'État et le marché \\[0pt]
L'État et les Églises \\[0pt]
L'État libéral \\[0pt]
L'État mondial \\[0pt]
L'État n'est-il qu'un moyen ? \\[0pt]
L'État peut-il avoir tous les droits ? \\[0pt]
L'État peut-il créer la liberté ? \\[0pt]
L'État peut-il disparaître ? \\[0pt]
L'État peut-il être fondé sur un contrat ? \\[0pt]
L'État peut-il être indifférent à la religion ? \\[0pt]
L'État peut-il être libéral ? \\[0pt]
L'État-Providence \\[0pt]
L'État universel \\[0pt]
Le témoignage \\[0pt]
Le témoignage des sens \\[0pt]
Le temps de la liberté \\[0pt]
Le temps de l'art \\[0pt]
Le temps de la science \\[0pt]
Le temps de vivre \\[0pt]
Le temps du désir \\[0pt]
Le temps du monde \\[0pt]
Le temps est-il essentiellement destructeur ? \\[0pt]
Le temps est-il notre allié ? \\[0pt]
Le temps est-il notre ennemi ? \\[0pt]
Le temps est-il une dimension de la nature ? \\[0pt]
Le temps est-il une réalité ? \\[0pt]
Le temps ne fait-il que passer ? \\[0pt]
Le temps passe-t-il ? \\[0pt]
Le temps perdu \\[0pt]
Le temps peut-il s'arrêter ? \\[0pt]
Le temps politique \\[0pt]
Le temps rend-il tout vain ? \\[0pt]
Le temps se laisse-t-il décrire logiquement ? \\[0pt]
L'éternel présent \\[0pt]
L'éternité \\[0pt]
L'éternité n'est-elle qu'une illusion ? \\[0pt]
Le « terrain » \\[0pt]
Le terrain \\[0pt]
Le territoire \\[0pt]
Le théâtre du monde \\[0pt]
Le théâtre et les mœurs \\[0pt]
L'éthique à l'épreuve du tragique \\[0pt]
L'éthique des plaisirs \\[0pt]
L'éthique du spectateur \\[0pt]
L'éthique est-elle affaire de choix ? \\[0pt]
L'éthique suppose-t-elle la liberté ? \\[0pt]
L'ethnocentrisme \\[0pt]
L'ethnocentrisme comme problème pour l'ethnologie \\[0pt]
L'ethnologie est-elle une science mélancolique ? \\[0pt]
Le tiers exclu \\[0pt]
L'étonnement \\[0pt]
Le totalitarisme \\[0pt]
Le totémisme \\[0pt]
Le toucher \\[0pt]
Le tourment moral \\[0pt]
Le tout est-il la somme de ses parties ? \\[0pt]
Le tout et la partie \\[0pt]
Le tout et les parties \\[0pt]
Le tragique \\[0pt]
Le trait d'esprit \\[0pt]
L'étranger \\[0pt]
L'étrangeté \\[0pt]
Le transcendantal \\[0pt]
Le travail \\[0pt]
Le travail artistique \\[0pt]
Le travail artistique doit-il demeurer caché ? \\[0pt]
Le travail bien fait \\[0pt]
Le travail est-il une fin ? \\[0pt]
Le travail est-il une valeur morale ? \\[0pt]
Le travail et l'œuvre \\[0pt]
Le travail intellectuel \\[0pt]
Le travail nous libère-t-il ? \\[0pt]
Le travail nous rend-il heureux ? \\[0pt]
Le travail nous rend-il solidaires ? \\[0pt]
Le travail rapproche-t-il les hommes ? \\[0pt]
Le travail sur le terrain \\[0pt]
Le travail sur soi \\[0pt]
L'être de la conscience \\[0pt]
L'être de la vérité \\[0pt]
L'être de l'image \\[0pt]
L'être du possible \\[0pt]
L'être en tant qu'être \\[0pt]
L'être en tant qu'être est-il connaissable ? \\[0pt]
L'être est-il un genre ? \\[0pt]
L'être et l'apparence \\[0pt]
L'être et la volonté \\[0pt]
L'être et le bien \\[0pt]
L'être et le devenir \\[0pt]
L'être et le devoir-être \\[0pt]
L'être et le néant \\[0pt]
L'être et les êtres \\[0pt]
L'être et l'esprit \\[0pt]
L'être et l'essence \\[0pt]
L'être et l'étant \\[0pt]
L'être et le temps \\[0pt]
L'être intelligent \\[0pt]
L'être se confond-il avec l'être perçu ? \\[0pt]
L'Être suprême \\[0pt]
Le tribunal de l'histoire \\[0pt]
Le trompe-l'œil \\[0pt]
Le tyran \\[0pt]
L'eugénisme \\[0pt]
Le vainqueur a-t-il tous les droits ? \\[0pt]
L'évaluation de l'action politique \\[0pt]
L'évaluation des œuvres d'art \\[0pt]
Le vécu \\[0pt]
L'événement \\[0pt]
L'événement et le fait divers \\[0pt]
L'événement manque-t-il d'être ? \\[0pt]
Le verbalisme \\[0pt]
Le verbe \\[0pt]
Le vertige \\[0pt]
Le vide \\[0pt]
L'évidence \\[0pt]
L'évidence a-t-elle une valeur absolue ? \\[0pt]
L'évidence est-elle toujours un critère de vérité ? \\[0pt]
L'évidence est-elle un critère de vérité ? \\[0pt]
Le village global \\[0pt]
Le virtuel \\[0pt]
Le virtuel existe-t-il ? \\[0pt]
Le visage \\[0pt]
Le visible et l'invisible \\[0pt]
Le vivant \\[0pt]
Le vivant a-t-il des droits ? \\[0pt]
Le vivant a-t-il une temporalité propre ? \\[0pt]
Le vivant comme problème pour la philosophie des sciences \\[0pt]
Le vivant échappe-t-il au déterminisme ? \\[0pt]
Le vivant est-il entièrement connaissable ? \\[0pt]
Le vivant est-il l'affaire du politique ? \\[0pt]
Le vivant et la technique \\[0pt]
Le vivant et ses lois \\[0pt]
Le vol \\[0pt]
Le volontaire et l'involontaire \\[0pt]
L'évolution \\[0pt]
L'évolution des langues \\[0pt]
L'évolution des sociétés dépend-elle du progrès technique \\[0pt]
Le voyage \\[0pt]
Le vrai a-t-il une histoire ? \\[0pt]
Le vrai doit-il être démontré ? \\[0pt]
Le vrai est-il à lui-même sa propre marque ? \\[0pt]
Le vrai et le réel \\[0pt]
Le vrai et le vraisemblable \\[0pt]
Le vrai et l'imaginaire \\[0pt]
Le vrai peut-il rester invérifiable ? \\[0pt]
Le vraisemblable \\[0pt]
Le vraisemblable présente-t-il un intérêt pour la pensée ? \\[0pt]
Le vrai se perçoit-il ? \\[0pt]
Le vrai se réduit-il à l'utile ? \\[0pt]
Le vulgaire \\[0pt]
L'exactitude \\[0pt]
L'excellence \\[0pt]
L'exception \\[0pt]
L'exception peut-elle confirmer la règle ? \\[0pt]
L'excès \\[0pt]
L'excès et le défaut \\[0pt]
L'exclusion \\[0pt]
L'excuse \\[0pt]
L'exécution d'une œuvre d'art est-elle toujours une œuvre d'art ? \\[0pt]
L'exemplaire \\[0pt]
L'exemplarité \\[0pt]
L'exemple \\[0pt]
L'exemple en morale \\[0pt]
L'exercice de la raison \\[0pt]
L'exercice de la vertu \\[0pt]
L'exercice du pouvoir \\[0pt]
L'exercice du pouvoir est-il nécessairement solitaire ? \\[0pt]
L'exercice solitaire du pouvoir \\[0pt]
L'exigence de vérité a-t-elle un sens moral ? \\[0pt]
L'exigence morale \\[0pt]
L'exil \\[0pt]
L'existence \\[0pt]
L'existence a-t-elle un sens ? \\[0pt]
L'existence de Dieu \\[0pt]
L'existence de l'État dépend-elle d'un contrat ? \\[0pt]
L'existence du mal \\[0pt]
L'existence est-elle un jeu ? \\[0pt]
L'existence se démontre-t-elle ? \\[0pt]
L'existence se prouve-t-elle ? \\[0pt]
L'expérience \\[0pt]
L'expérience artistique \\[0pt]
L'expérience cruciale \\[0pt]
L'expérience dans les sciences humaines \\[0pt]
L'expérience de la beauté est-elle naturelle ? \\[0pt]
L'expérience de la liberté \\[0pt]
L'expérience de l'injustice \\[0pt]
L'expérience de l'irrationnel \\[0pt]
L'expérience directe est-elle une connaissance ? \\[0pt]
L'expérience du beau peut-elle être une expérience métaphysique ? \\[0pt]
L'expérience du mal \\[0pt]
L'expérience du présent \\[0pt]
L'expérience du sociologue \\[0pt]
L'expérience en psychologie \\[0pt]
L'expérience en sciences humaines \\[0pt]
L'expérience enseigne-elle quelque chose ? \\[0pt]
L'expérience, est-ce l'observation ? \\[0pt]
L'expérience est-elle la seule source de nos connaissances ? \\[0pt]
L'expérience esthétique \\[0pt]
L'expérience esthétique peut-elle se communiquer ? \\[0pt]
L'expérience et l'expérimentation \\[0pt]
L'expérience instruit-elle ? \\[0pt]
L'expérience métaphysique \\[0pt]
L'expérience morale \\[0pt]
L'expérience sensible est-elle la seule source légitime de connaissance ? \\[0pt]
L'expérimentation \\[0pt]
L'expérimentation en art \\[0pt]
L'expérimentation en psychologie \\[0pt]
L'expérimentation en sciences sociales \\[0pt]
L'expérimentation est-elle facteur de vérité ? \\[0pt]
L'expérimentation médicale \\[0pt]
L'expérimentation sur l'être humain \\[0pt]
L'expérimentation sur le vivant \\[0pt]
L'expert et l'amateur \\[0pt]
L'expertise \\[0pt]
L'expertise politique \\[0pt]
L'explication causale dans les sciences sociales \\[0pt]
L'explication de l'œuvre d'art \\[0pt]
L'explication scientifique \\[0pt]
L'exploitation de l'homme par l'homme \\[0pt]
L'exposition \\[0pt]
L'exposition de l'œuvre d'art \\[0pt]
L'expression \\[0pt]
L'expression artistique \\[0pt]
L'expression de l'inconscient \\[0pt]
L'expression des citoyens dans le vote suffit-elle à faire la démocratie ? \\[0pt]
L'expression des citoyens dans le vote suffit-elle à faire vivre la démocratie ? \\[0pt]
L'expression des sentiments \\[0pt]
L'expression du désir \\[0pt]
L'expression du mouvement \\[0pt]
L'expression peut-elle être libre ? \\[0pt]
L'expressivité musicale \\[0pt]
L'extériorité \\[0pt]
L'extrémisme \\[0pt]
L'habileté \\[0pt]
L'habileté et la prudence \\[0pt]
L'habitation \\[0pt]
L'habitude \\[0pt]
L'habitude est-elle une force ? \\[0pt]
L'habitude est-elle une seconde nature ? \\[0pt]
L'habitude morale \\[0pt]
L'habitus \\[0pt]
L'harmonie \\[0pt]
L'hégémonie politique \\[0pt]
L'hérésie \\[0pt]
L'héritage \\[0pt]
L'héroïsme \\[0pt]
L'hésitation \\[0pt]
L'hétérogénéité sociale \\[0pt]
L'hétéronomie \\[0pt]
L'hétéronomie de l'art \\[0pt]
L'histoire a-t-elle un sens ? \\[0pt]
L'histoire de l'art \\[0pt]
L'histoire de l'art est-elle celle des styles ? \\[0pt]
L'histoire de l'art est-elle finie ? \\[0pt]
L'histoire des arts est-elle liée à l'histoire des techniques ? \\[0pt]
L'histoire des civilisations \\[0pt]
L'histoire des sciences \\[0pt]
L'histoire des sciences est-elle une histoire ? \\[0pt]
L'histoire : enquête ou science ? \\[0pt]
L'histoire est-elle avant tout mémoire ? \\[0pt]
L'histoire est-elle cyclique ? \\[0pt]
L'histoire est-elle déterministe ? \\[0pt]
L'histoire est-elle la science de l'événementiel ? \\[0pt]
L'histoire est-elle le règne du hasard ? \\[0pt]
L'histoire est-elle sans fin ? \\[0pt]
L'histoire est-elle tragique ? \\[0pt]
L'histoire est-elle une science de l'individuel ? \\[0pt]
L'histoire est-elle un genre littéraire ? \\[0pt]
L'histoire est-elle un récit ? \\[0pt]
L'histoire est-elle un roman vrai ? \\[0pt]
L'histoire est-elle utile ? \\[0pt]
L'histoire est-elle utile à la politique ? \\[0pt]
L'histoire et la géographie \\[0pt]
« L'histoire jugera » \\[0pt]
L'histoire n'est-elle que bruit et fureur ? \\[0pt]
L'histoire n'est-elle qu'une suite d'événements ? \\[0pt]
L'histoire n'est-elle qu'un récit ? \\[0pt]
L'histoire peut-elle être objective ? \\[0pt]
L'histoire peut-elle être universelle ? \\[0pt]
L'histoire peut-elle s'écrire au présent ? \\[0pt]
L'histoire peut-elle se répéter ? \\[0pt]
L'histoire : science ou récit ? \\[0pt]
L'histoire se réduit-elle à l'étude du passé ? \\[0pt]
L'histoire universelle est-elle l'histoire des guerres ? \\[0pt]
L'historicisme \\[0pt]
L'historicité des sciences \\[0pt]
L'historien est-il un homme qui se souvient du passé ? \\[0pt]
L'historien peut-il se passer du concept de causalité ? \\[0pt]
L'homicide \\[0pt]
L'hominisation \\[0pt]
L'homme a-t-il une nature ? \\[0pt]
L'homme cultivé \\[0pt]
L'homme de la rue \\[0pt]
L'homme des droits de l'homme n'est-il qu'une fiction ? \\[0pt]
L'homme des foules \\[0pt]
L'homme des sciences de l'homme \\[0pt]
L'homme des sciences humaines \\[0pt]
L'homme d'État \\[0pt]
L'homme est-il la mesure de toutes choses ? \\[0pt]
L'homme est-il meilleur seul ou en société ? \\[0pt]
L'homme est-il objet de science ? \\[0pt]
L'homme est-il prisonnier du temps ? \\[0pt]
L'homme est-il religieux par nature ? \\[0pt]
L'homme est-il un animal dénaturé ? \\[0pt]
L'homme est-il un animal métaphysique ? \\[0pt]
L'homme est-il un animal politique ? \\[0pt]
L'homme est-il un animal symbolique ? \\[0pt]
L'homme est-il un être de devoir ? \\[0pt]
L'homme est-il un être inachevé ? \\[0pt]
L'homme est-il un être social par nature ? \\[0pt]
L'homme est-il vraiment un animal politique ? \\[0pt]
« L'homme est la mesure de toute chose » \\[0pt]
L'homme et la bête \\[0pt]
L'homme et la machine \\[0pt]
L'homme et la nature sont-ils commensurables ? \\[0pt]
L'homme et le citoyen \\[0pt]
L'homme et le sacré \\[0pt]
L'homme fait-il partie de la nature ? \\[0pt]
L'homme injuste peut-il être heureux ? \\[0pt]
L'homme, le citoyen, le soldat \\[0pt]
L'homme libre est-il un homme seul ? \\[0pt]
L'homme moyen \\[0pt]
L'homme peut-il changer ? \\[0pt]
L'homme peut-il être objet d'expérience ? \\[0pt]
L'homo œconomicus est-il rationnel ? \\[0pt]
L'honnêteté \\[0pt]
L'honneur \\[0pt]
L'horizon \\[0pt]
L'horizon de la mort \\[0pt]
L'horreur \\[0pt]
L'horreur du vide \\[0pt]
L'horrible \\[0pt]
L'hospitalité \\[0pt]
L'hospitalité a-t-elle un sens politique ? \\[0pt]
L'hospitalité est-elle un devoir ? \\[0pt]
L'hospitalité est-elle une vertu politique ? \\[0pt]
L'humain est-il un animal irrationnel ? \\[0pt]
L'humanité : un concept normatif ? \\[0pt]
L'humiliation \\[0pt]
L'humilité \\[0pt]
L'humilité est-elle une vertu ? \\[0pt]
L'humour \\[0pt]
L'humour et l'ironie \\[0pt]
L'hybridation des arts \\[0pt]
L'hypocrisie \\[0pt]
L'hypothèse \\[0pt]
L'hypothèse de l'inconscient \\[0pt]
Libéral et libertaire \\[0pt]
Libéralisme et démocratie \\[0pt]
Liberté, égalité, fraternité \\[0pt]
Liberté, égalité, solidarité \\[0pt]
Liberté et égalité \\[0pt]
Liberté et habitude \\[0pt]
Liberté et libération \\[0pt]
Liberté et nécessité \\[0pt]
Liberté et négativité \\[0pt]
Liberté et propriété \\[0pt]
Liberté et responsabilité \\[0pt]
Liberté et vérité \\[0pt]
Liberté humaine et liberté divine \\[0pt]
Liberté réelle, liberté formelle \\[0pt]
Libertés publiques et culture politique \\[0pt]
Libre arbitre et liberté \\[0pt]
Libre-arbitre, impulsion, contrainte \\[0pt]
L'idéal \\[0pt]
L'idéal dans l'art \\[0pt]
L'idéal de l'art \\[0pt]
L'idéal démonstratif \\[0pt]
L'idéal de vérité \\[0pt]
L'idéal et le réel \\[0pt]
L'idéalisme \\[0pt]
L'idéaliste \\[0pt]
L'idéalité \\[0pt]
L'idéal moral est-il vain ? \\[0pt]
L'idéal-type \\[0pt]
L'idée d'anthropologie \\[0pt]
L'idée de beaux arts \\[0pt]
L'idée de communauté \\[0pt]
L'idée de connaissance approchée \\[0pt]
L'idée de conscience collective \\[0pt]
L'idée de continuité \\[0pt]
L'idée de contrat \\[0pt]
L'idée de contrat social \\[0pt]
L'idée de création \\[0pt]
L'idée de crise \\[0pt]
L'idée de critique \\[0pt]
L'idée de destin \\[0pt]
L'idée de destin a-t-elle un sens ? \\[0pt]
L'idée de Dieu \\[0pt]
L'idée de domination \\[0pt]
L'idée de forme sociale \\[0pt]
L'idée d'égalité \\[0pt]
L'idée de guerre juste \\[0pt]
L'idée de langue parfaite \\[0pt]
L'idée de langue universelle \\[0pt]
L'idée de logique \\[0pt]
L'idée de logique transcendantale \\[0pt]
L'idée de logique universelle \\[0pt]
L'idée de loi logique \\[0pt]
L'idée de loi naturelle \\[0pt]
L'idée de mathesis universalis \\[0pt]
L'idée de morale appliquée \\[0pt]
L'idée de nation \\[0pt]
L'idée de nature humaine \\[0pt]
L'idée d'encyclopédie \\[0pt]
L'idée de norme \\[0pt]
L'idée de perfection \\[0pt]
L'idée de philosophie première \\[0pt]
L'idée de préhistoire \\[0pt]
L'idée de progrès \\[0pt]
L'idée de réduction \\[0pt]
L'idée de république \\[0pt]
L'idée de République \\[0pt]
L'idée de rétribution est-elle nécessaire à la morale ? \\[0pt]
L'idée de révolution \\[0pt]
L'idée de science expérimentale \\[0pt]
L'idée de « sciences exactes » \\[0pt]
L'idée de société primitive \\[0pt]
L'idée de substance \\[0pt]
L'idée d'éternité \\[0pt]
L'idée d'exactitude \\[0pt]
L'idée d'humanité \\[0pt]
L'idée d'un commencement absolu \\[0pt]
L'idée d'une langue universelle \\[0pt]
L'idée d'une science bien faite \\[0pt]
L'idée d'une science générale des signes \\[0pt]
L'idée esthétique \\[0pt]
L'idée et l'idéal \\[0pt]
L'identité \\[0pt]
L'identité collective \\[0pt]
L'identité et la différence \\[0pt]
L'identité nationale \\[0pt]
L'identité personnelle \\[0pt]
L'identité personnelle est-elle donnée ou construite ? \\[0pt]
L'identité relève-t-elle du champ politique ? \\[0pt]
L'idéologie \\[0pt]
L'idolâtrie \\[0pt]
L'idole \\[0pt]
L'ignoble \\[0pt]
L'ignorance \\[0pt]
L'ignorance des savants \\[0pt]
L'ignorance est-elle la raison du mal ? \\[0pt]
L'ignorance est-elle une excuse ? \\[0pt]
L'ignorance nous excuse-t-elle ? \\[0pt]
L'illimité \\[0pt]
L'illusion \\[0pt]
L'illusoire \\[0pt]
L'illustration \\[0pt]
L'image \\[0pt]
L'imaginaire \\[0pt]
L'imaginaire et le réel \\[0pt]
L'imaginaire politique \\[0pt]
L'imaginaire social \\[0pt]
L'imagination \\[0pt]
L'imagination a-t-elle des limites ? \\[0pt]
L'imagination dans l'art \\[0pt]
L'imagination dans les sciences \\[0pt]
L'imagination est-elle dangereuse ? \\[0pt]
L'imagination esthétique \\[0pt]
L'imagination et la raison \\[0pt]
L'imagination nous éloigne-t-elle du réel ? \\[0pt]
L'imagination nous instruit-elle ? \\[0pt]
L'imagination politique \\[0pt]
L'imbécile heureux \\[0pt]
L'imitation \\[0pt]
L'imitation a-t-elle une fonction morale ? \\[0pt]
L'immanence \\[0pt]
L'immatériel \\[0pt]
L'immédiat \\[0pt]
L'immémorial \\[0pt]
L'immensité \\[0pt]
L'immoraliste \\[0pt]
L'immoralité \\[0pt]
L'immortalité \\[0pt]
L'immortalité de l'âme \\[0pt]
L'immortalité des œuvres d'art \\[0pt]
L'immuable \\[0pt]
L'immutabilité \\[0pt]
L'impardonnable \\[0pt]
L'imparfait \\[0pt]
L'impartialité \\[0pt]
L'impartialité des historiens \\[0pt]
L'impensable \\[0pt]
L'impératif \\[0pt]
L'imperceptible \\[0pt]
L'impersonnel \\[0pt]
L'implicite \\[0pt]
L'implicite dans la communication \\[0pt]
L'importance des détails \\[0pt]
L'impossible \\[0pt]
L'impossible est-il concevable ? \\[0pt]
L'imposteur \\[0pt]
L'imprescriptible \\[0pt]
L'impression \\[0pt]
L'imprévisible \\[0pt]
L'imprévu \\[0pt]
L'improbable \\[0pt]
L'improvisation \\[0pt]
L'improvisation dans l'art \\[0pt]
L'imprudence \\[0pt]
L'impuissance \\[0pt]
L'impuissance de la raison \\[0pt]
L'impuissance de l'art \\[0pt]
L'impuissance de l'État \\[0pt]
L'impunité \\[0pt]
L'inacceptable \\[0pt]
L'inachevé \\[0pt]
L'inachèvement \\[0pt]
L'inaction \\[0pt]
L'inapparent \\[0pt]
L'inattendu \\[0pt]
L'incarnation \\[0pt]
L'incarnation du pouvoir \\[0pt]
L'incertitude \\[0pt]
L'incertitude est-elle dans les choses ou dans les idées ? \\[0pt]
L'incommensurabilité \\[0pt]
L'incommensurable \\[0pt]
L'incompréhensible \\[0pt]
L'inconcevable \\[0pt]
L'inconnu \\[0pt]
L'inconscience \\[0pt]
L'inconscient \\[0pt]
L'inconscient a-t-il son propre langage ? \\[0pt]
L'inconscient a-t-il une histoire ? \\[0pt]
L'inconscient collectif \\[0pt]
L'inconscient de l'art \\[0pt]
L'inconscient est-il l'animal en nous ? \\[0pt]
L'inconscient est-il pure négation de la conscience ? \\[0pt]
L'inconscient est-il une découverte ou une invention ? \\[0pt]
L'inconscient est-il une dimension de la conscience ? \\[0pt]
L'inconscient n'est-il qu'un défaut de conscience ? \\[0pt]
L'inconscient nous révèle-t-il à nous-même ? \\[0pt]
L'inconscient psychanalytique réfute-t-il la notion philosophique de conscience ? \\[0pt]
L'inconséquence \\[0pt]
L'inconstance \\[0pt]
L'incorporel \\[0pt]
L'incrédulité \\[0pt]
L'inculture \\[0pt]
L'indécidable \\[0pt]
L'indécision \\[0pt]
L'indéfini \\[0pt]
L'indémontrable \\[0pt]
L'indépassable \\[0pt]
L'indépendance \\[0pt]
L'indésirable \\[0pt]
L'indétermination \\[0pt]
L'indéterminé \\[0pt]
L'indice \\[0pt]
L'indicible \\[0pt]
L'indifférence \\[0pt]
L'indifférence à la politique \\[0pt]
L'indigène et l'étranger \\[0pt]
L'indignation \\[0pt]
L'indiscernable \\[0pt]
L'indiscutable \\[0pt]
L'indistinct \\[0pt]
L'individu \\[0pt]
L'individualisme \\[0pt]
L'individualisme a-t-il sa place en politique ? \\[0pt]
L'individualisme méthodologique \\[0pt]
L'individualisme rend-il la politique impossible ? \\[0pt]
L'individu a-t-il des droits ? \\[0pt]
L'individu échappe-t-il aux sciences humaines ? \\[0pt]
L'individuel \\[0pt]
L'individuel et le collectif \\[0pt]
L'individu est-il définissable ? \\[0pt]
L'individu et la multitude \\[0pt]
L'individu et la société \\[0pt]
L'individu et le groupe \\[0pt]
L'individu et les masses \\[0pt]
L'individu face à L'État \\[0pt]
L'individu particulier peut-il être l'objet d'une connaissance sociologique ? \\[0pt]
L'indivisible \\[0pt]
L'indubitable \\[0pt]
L'induction \\[0pt]
L'induction et la déduction \\[0pt]
L'indulgence \\[0pt]
L'industrie culturelle \\[0pt]
L'industrie du beau \\[0pt]
L'ineffable \\[0pt]
L'inégalité a-t-elle des vertus ? \\[0pt]
L'inégalité des chances \\[0pt]
L'inégalité entre les hommes \\[0pt]
L'inégalité est-elle nécessairement injuste ? \\[0pt]
L'inégalité naturelle \\[0pt]
L'inégalité sociale \\[0pt]
L'inertie \\[0pt]
L'inesthétique \\[0pt]
L'inexactitude et le savoir scientifique \\[0pt]
L'inexact peut-il avoir une valeur ? \\[0pt]
L'infâme \\[0pt]
L'infamie \\[0pt]
L'inférence \\[0pt]
L'infériorité \\[0pt]
L'infini \\[0pt]
L'infini est-il la négation du fini ? \\[0pt]
L'infini et l'indéfini \\[0pt]
L'infinité de l'espace \\[0pt]
L'infinité du monde \\[0pt]
L'influence \\[0pt]
L'information \\[0pt]
L'informe \\[0pt]
L'informe et le difforme \\[0pt]
L'ingénieur et l'artiste \\[0pt]
L'ingéniosité \\[0pt]
L'ingérence \\[0pt]
L'ingratitude \\[0pt]
Linguistique et sens \\[0pt]
L'inhibition \\[0pt]
L'inhumain \\[0pt]
L'inimaginable \\[0pt]
L'inimitié \\[0pt]
L'inintelligible \\[0pt]
L'iniquité \\[0pt]
L'initiation \\[0pt]
L'injonction \\[0pt]
L'injure \\[0pt]
L'injustice \\[0pt]
L'injustice de la loi \\[0pt]
L'injustice est-elle préférable au désordre ? \\[0pt]
L'injustifiable \\[0pt]
L'innocence \\[0pt]
L'innommable \\[0pt]
L'inobservable \\[0pt]
L'inquiétant \\[0pt]
L'inquiétude \\[0pt]
L'insatisfaction \\[0pt]
L'insensé \\[0pt]
L'insensibilité \\[0pt]
L'insignifiant \\[0pt]
« L'insociable sociabilité » \\[0pt]
L'insociable sociabilité \\[0pt]
L'insoluble \\[0pt]
L'insouciance \\[0pt]
L'insoumission \\[0pt]
L'insoutenable \\[0pt]
L'inspiration \\[0pt]
L'instant \\[0pt]
L'instant a-t-il une réalité ? \\[0pt]
L'instinct \\[0pt]
L'institution \\[0pt]
L'institution de la société \\[0pt]
L'institution du mariage \\[0pt]
L'institutionnalisation des conduites \\[0pt]
L'institution scientifique \\[0pt]
L'institution scolaire \\[0pt]
L'instruction est-elle facteur de moralité ? \\[0pt]
L'instrument \\[0pt]
L'instrument, l'outil, le jouet \\[0pt]
L'instrument mathématique en sciences humaines \\[0pt]
L'instrument scientifique \\[0pt]
L'insulte \\[0pt]
L'insurrection \\[0pt]
L'intangible \\[0pt]
L'intégration sociale \\[0pt]
L'intégrité \\[0pt]
L'intellectuel \\[0pt]
L'intelligence \\[0pt]
L'intelligence de la machine \\[0pt]
L'intelligence de la main \\[0pt]
L'intelligence de la matière \\[0pt]
L'intelligence des bêtes \\[0pt]
L'intelligence des foules \\[0pt]
L'intelligence du lointain \\[0pt]
L'intelligence du sensible \\[0pt]
L'intelligence du vivant \\[0pt]
L'intelligence politique \\[0pt]
L'intelligible \\[0pt]
L'intempérance \\[0pt]
L'intemporel \\[0pt]
L'intention \\[0pt]
L'intention morale \\[0pt]
L'intention morale suffit-elle à constituer la valeur morale de l'action ? \\[0pt]
L'intentionnalité \\[0pt]
L'interdépendance \\[0pt]
L'interdépendance des faits humains \\[0pt]
L'interdiction \\[0pt]
L'interdit \\[0pt]
L'interdit a-t-il une origine sociale ? \\[0pt]
L'interdit et le sacré \\[0pt]
L'intérêt \\[0pt]
L'intérêt bien compris \\[0pt]
L'intérêt commun \\[0pt]
L'intérêt des machines \\[0pt]
L'intérêt général \\[0pt]
L'intérêt général est-il le bien commun ? \\[0pt]
L'intérêt général est-il un mythe ? \\[0pt]
L'intérêt général n'est-il qu'un mythe ? \\[0pt]
L'intérêt peut-il avoir une valeur morale ? \\[0pt]
L'intérêt peut-il être général ? \\[0pt]
L'intérêt peut-il être une valeur morale ? \\[0pt]
L'intérêt public est-il une illusion ? \\[0pt]
L'intérieur et l'extérieur \\[0pt]
L'intériorisation des normes \\[0pt]
L'intériorité \\[0pt]
L'intériorité de l'œuvre \\[0pt]
L'intériorité est-elle un mythe ? \\[0pt]
L'interprétation \\[0pt]
L'interprétation de la loi \\[0pt]
L'interprétation de la nature \\[0pt]
L'interprétation des œuvres \\[0pt]
L'interprétation est-elle sans fin ? \\[0pt]
L'interprétation est-elle un art ? \\[0pt]
L'interprète est-il un créateur ? \\[0pt]
L'interrogation humaine \\[0pt]
L'intersectionnalité \\[0pt]
L'intime \\[0pt]
L'intime conviction \\[0pt]
L'intime est-il politique ? \\[0pt]
L'intimité \\[0pt]
L'intolérable \\[0pt]
L'intolérance \\[0pt]
L'intraduisible \\[0pt]
L'intransigeance \\[0pt]
L'intransmissible \\[0pt]
L'introspection \\[0pt]
L'intuition \\[0pt]
L'intuition a-t-elle une place en logique ? \\[0pt]
L'intuition en mathématiques \\[0pt]
L'intuition morale \\[0pt]
L'inutile \\[0pt]
L'invention \\[0pt]
L'invention de soi \\[0pt]
L'invention des valeurs \\[0pt]
L'invention en politique \\[0pt]
L'invérifiable \\[0pt]
L'invisibilité \\[0pt]
L'invisible \\[0pt]
L'involontaire \\[0pt]
L'invraisemblable \\[0pt]
Lire \\[0pt]
Lire et écrire \\[0pt]
L'ironie \\[0pt]
L'irrationalité \\[0pt]
L'irrationnel \\[0pt]
L'irrationnel et le politique \\[0pt]
L'irréel \\[0pt]
L'irréfutable \\[0pt]
L'irrégularité \\[0pt]
L'irréparable \\[0pt]
L'irreprésentable \\[0pt]
L'irrésolution \\[0pt]
L'irresponsabilité \\[0pt]
L'irréversibilité \\[0pt]
L'irréversible \\[0pt]
L'irrévocable \\[0pt]
Littérature et philosophie \\[0pt]
Littérature et réalité \\[0pt]
L'ivresse \\[0pt]
L'obéissance \\[0pt]
L'obéissance à l'autorité \\[0pt]
L'obéissance peut-elle être un acte de liberté ? \\[0pt]
L'objectivité \\[0pt]
L'objectivité de l'art \\[0pt]
L'objectivité de l'œuvre d'art \\[0pt]
L'objectivité des sciences humaines \\[0pt]
L'objectivité des sciences sociales \\[0pt]
L'objectivité des valeurs \\[0pt]
L'objectivité en histoire \\[0pt]
L'objectivité existe-t-elle ? \\[0pt]
L'objectivité historique \\[0pt]
L'objet \\[0pt]
L'objet d'amour \\[0pt]
L'objet de culte \\[0pt]
L'objet de la littérature \\[0pt]
L'objet de l'amour \\[0pt]
L'objet de la politique \\[0pt]
L'objet de la psychologie \\[0pt]
L'objet de la réflexion \\[0pt]
L'objet de l'art \\[0pt]
L'objet du désir \\[0pt]
L'objet du désir en est-il la cause ? \\[0pt]
L'objet du jugement esthétique \\[0pt]
L'obligation \\[0pt]
L'obligation d'échanger \\[0pt]
L'obligation morale \\[0pt]
L'obligation morale peut-elle se réduire à une obligation sociale ? \\[0pt]
L'obscène \\[0pt]
L'obscénité \\[0pt]
L'obscur \\[0pt]
L'obscurité \\[0pt]
L'observable et l'inobservable \\[0pt]
L'observation \\[0pt]
L'observation de l'homme \\[0pt]
L'observation participante \\[0pt]
L'obsession \\[0pt]
L'obstacle \\[0pt]
L'obstacle épistémologique \\[0pt]
L'occasion \\[0pt]
L'œil du photographe \\[0pt]
L'œil et l'oreille \\[0pt]
L'œuvre anonyme \\[0pt]
L'œuvre d'art échappe-t-elle au monde ? \\[0pt]
L'œuvre d'art est-elle anhistorique ? \\[0pt]
L'œuvre d'art est-elle intemporelle ? \\[0pt]
L'œuvre d'art est-elle l'expression d'une idée ? \\[0pt]
L'œuvre d'art est-elle toujours destinée à un public ? \\[0pt]
L'œuvre d'art est-elle une belle apparence ? \\[0pt]
L'œuvre d'art est-elle une marchandise ? \\[0pt]
L'œuvre d'art et le plaisir \\[0pt]
L'œuvre d'art et sa reproduction \\[0pt]
L'œuvre d'art et son auteur \\[0pt]
L'œuvre d'art nous apprend-elle quelque chose ? \\[0pt]
L'œuvre d'art représente-t-elle quelque chose ? \\[0pt]
L'œuvre d'art totale \\[0pt]
L'œuvre d'art traduit-elle une vision du monde ? \\[0pt]
L'œuvre de fiction \\[0pt]
L'œuvre de l'historien \\[0pt]
L'œuvre du temps \\[0pt]
L'œuvre et le produit \\[0pt]
L'œuvre inachevée \\[0pt]
L'œuvre : l'universel et le particulier \\[0pt]
L'œuvre totale \\[0pt]
L'offense \\[0pt]
Logique et dialectique \\[0pt]
Logique et écriture \\[0pt]
Logique et existence \\[0pt]
Logique et logiques \\[0pt]
Logique et mathématique \\[0pt]
Logique et mathématiques \\[0pt]
Logique et métaphysique \\[0pt]
Logique et méthode \\[0pt]
Logique et ontologie \\[0pt]
Logique et psychologie \\[0pt]
Logique et réalité \\[0pt]
Logique et vérité \\[0pt]
Logique générale et logique transcendantale \\[0pt]
Loi et commandement \\[0pt]
Loi morale et loi politique \\[0pt]
Loi naturelle et loi morale \\[0pt]
Loi naturelle et loi politique \\[0pt]
Lois et normes \\[0pt]
Lois et règles en logique \\[0pt]
Lois et valeurs \\[0pt]
L'oisiveté \\[0pt]
Lois naturelles et lois civiles \\[0pt]
L'oligarchie \\[0pt]
L'ombre \\[0pt]
L'ombre et la lumière \\[0pt]
L'omniscience \\[0pt]
L'ontologie peut-elle être relative ? \\[0pt]
L'opéra est-il un art complet ? \\[0pt]
L'opinion \\[0pt]
L'opinion droite \\[0pt]
L'opinion du citoyen \\[0pt]
L'opinion publique \\[0pt]
L'opinion vraie \\[0pt]
L'opportunisme \\[0pt]
L'opposant \\[0pt]
L'opposition \\[0pt]
L'optimisme \\[0pt]
L'oral et l'écrit \\[0pt]
L'ordinaire est-il ennuyeux ? \\[0pt]
L'ordinateur \\[0pt]
L'ordre \\[0pt]
L'ordre des choses \\[0pt]
L'ordre des raisons \\[0pt]
L'ordre du monde \\[0pt]
L'ordre du temps \\[0pt]
L'ordre est-il dans les choses ? \\[0pt]
L'ordre établi \\[0pt]
L'ordre et la beauté \\[0pt]
L'ordre et la mesure \\[0pt]
L'ordre international \\[0pt]
L'ordre, le nombre, la mesure \\[0pt]
L'ordre moral \\[0pt]
L'ordre naturel \\[0pt]
L'ordre politique \\[0pt]
L'ordre politique exclut-il la violence ? \\[0pt]
L'ordre politique peut-il exclure la violence ? \\[0pt]
L'ordre public \\[0pt]
L'ordre social \\[0pt]
L'ordre symbolique \\[0pt]
L'organique et le mécanique \\[0pt]
L'organisation \\[0pt]
L'organisation du vivant \\[0pt]
L'orgueil \\[0pt]
L'orientation \\[0pt]
L'original et la copie \\[0pt]
L'originalité \\[0pt]
L'originalité en art \\[0pt]
L'origine \\[0pt]
L'origine de la culpabilité \\[0pt]
L'origine de la famille \\[0pt]
L'origine de la négation \\[0pt]
L'origine de l'art \\[0pt]
L'origine des croyances \\[0pt]
L'origine des langues \\[0pt]
L'origine des langues est-elle un faux problème ? \\[0pt]
L'origine des valeurs \\[0pt]
L'origine des vertus \\[0pt]
L'origine du mal \\[0pt]
L'origine du symbolisme \\[0pt]
L'origine et le fondement \\[0pt]
L'ornement \\[0pt]
L'orthodoxie \\[0pt]
L'oubli \\[0pt]
L'oubli des fautes \\[0pt]
L'oubli est-il nécessaire à la vie ? \\[0pt]
L'oubli est-il un échec de la mémoire ? \\[0pt]
L'oubli n'est-il qu'une perte de mémoire ? \\[0pt]
L'outil \\[0pt]
L'outil et la machine \\[0pt]
Lumière et couleur \\[0pt]
L'un \\[0pt]
L'Un \\[0pt]
L'unanimité \\[0pt]
L'unanimité est-elle un critère de légitimité ? \\[0pt]
L'un et le multiple \\[0pt]
L'un et le tout \\[0pt]
L'un et l'être \\[0pt]
L'uniformité \\[0pt]
L'unité \\[0pt]
L'unité dans le beau \\[0pt]
L'unité de l'art \\[0pt]
L'unité de la science \\[0pt]
L'unité de l'État \\[0pt]
L'unité de l'œuvre d'art \\[0pt]
L'unité des contraires \\[0pt]
L'unité des langues \\[0pt]
L'unité des sciences \\[0pt]
L'unité des sciences humaines \\[0pt]
L'unité des sciences humaines ? \\[0pt]
L'unité des vices \\[0pt]
L'unité du corps politique \\[0pt]
L'unité du moi \\[0pt]
L'unité du vivant \\[0pt]
L'univers \\[0pt]
L'universalité du vrai \\[0pt]
L'universel \\[0pt]
L'universel en histoire \\[0pt]
L'universel est-il une illusion ? \\[0pt]
L'universel et le particulier \\[0pt]
L'universel et le singulier \\[0pt]
L'universel implique-t-il l'indifférence à la singularité ? \\[0pt]
L'univers infini \\[0pt]
L'univocité de l'étant \\[0pt]
L'un, le vrai, le bien \\[0pt]
L'urbanité \\[0pt]
L'urgence \\[0pt]
L'usage \\[0pt]
L'usage de la langue \\[0pt]
L'usage de l'analogie \\[0pt]
L'usage de l'analogie dans les sciences humaines \\[0pt]
L'usage des fictions \\[0pt]
L'usage des généalogies \\[0pt]
L'usage des modèles dans les sciences humaines \\[0pt]
L'usage des mots \\[0pt]
L'usage des passions \\[0pt]
L'usage des principes \\[0pt]
L'usage du concept de race \\[0pt]
L'usage du monde \\[0pt]
L'usure des mots \\[0pt]
L'usure du pouvoir \\[0pt]
L'utile et l'agréable \\[0pt]
L'utilité \\[0pt]
L'utilité de la peine \\[0pt]
L'utilité de la poésie \\[0pt]
L'utilité de l'art \\[0pt]
L'utilité des préjugés \\[0pt]
L'utilité des sciences humaines \\[0pt]
L'utilité est-elle étrangère à la morale ? \\[0pt]
L'utilité publique \\[0pt]
L'utopie \\[0pt]
L'utopie a-t-elle une signification politique ? \\[0pt]
L'utopie a-t-elle un lien avec la pratique ? \\[0pt]
L'utopie en politique \\[0pt]
Machine et organisme \\[0pt]
Machines et liberté \\[0pt]
Machines et mémoire \\[0pt]
Magie et religion \\[0pt]
Maître et disciple \\[0pt]
Maître et serviteur \\[0pt]
Maîtriser l'absence \\[0pt]
Maîtriser la nature \\[0pt]
Maîtriser le vivant \\[0pt]
Mal faire \\[0pt]
« Malheur aux vaincus » \\[0pt]
Manger \\[0pt]
Manquer de jugement \\[0pt]
Ma parole m'engage-t-elle ? \\[0pt]
Masculin et féminin \\[0pt]
Masculin, féminin \\[0pt]
Mass-media et création artistique \\[0pt]
Matérialisme et mécanisme \\[0pt]
Matériel / immatériel \\[0pt]
Mathématique et imagination \\[0pt]
Mathématiques et réalité \\[0pt]
Mathématiques et sciences humaines \\[0pt]
Mathématiques et universel \\[0pt]
Mathématiques pures et mathématiques appliquées \\[0pt]
Matière et corps \\[0pt]
Matière et forme \\[0pt]
Matière et matériaux \\[0pt]
Ma vraie nature \\[0pt]
Mécanisme et finalité \\[0pt]
Médecine et morale \\[0pt]
Méditer \\[0pt]
Mémoire et anticipation \\[0pt]
Mémoire et fiction \\[0pt]
Mémoire et identité \\[0pt]
Mémoire et imagination \\[0pt]
Mémoire et responsabilité \\[0pt]
Ménager les apparences \\[0pt]
Mensonge et politique \\[0pt]
Mensonge, vérité, véracité \\[0pt]
Mentir \\[0pt]
Mérite et démocratie \\[0pt]
Mesurer \\[0pt]
Mesurer le risque \\[0pt]
Métaphysique et histoire \\[0pt]
Métaphysique et mystique \\[0pt]
Métaphysique et ontologie \\[0pt]
Métaphysique et poésie \\[0pt]
Métaphysique et psychologie \\[0pt]
Métaphysique et religion \\[0pt]
Métaphysique et théologie \\[0pt]
Métaphysique spéciale, métaphysique générale \\[0pt]
Méthode et métaphysique \\[0pt]
Métier et vocation \\[0pt]
Mettre en ordre \\[0pt]
Microscope et télescope \\[0pt]
Milieux naturels et milieux sociaux \\[0pt]
Misère et pauvreté \\[0pt]
Modèle, type et paradigme \\[0pt]
Mœurs, coutumes, lois \\[0pt]
Moi d'abord \\[0pt]
Mon corps \\[0pt]
Mon corps est-il ma propriété ? \\[0pt]
Mon corps m'appartient-il ? \\[0pt]
Monde et nature \\[0pt]
Monde naturel et monde humain \\[0pt]
Monde perçu et monde conçu \\[0pt]
Mon existence est-elle contingente ? \\[0pt]
Monologue et dialogue \\[0pt]
Montrer et démontrer \\[0pt]
Montrer et dire \\[0pt]
Morale et calcul \\[0pt]
Morale et convention \\[0pt]
Morale et dispositions \\[0pt]
Morale et éducation \\[0pt]
Morale et histoire \\[0pt]
Morale et identité \\[0pt]
Morale et intérêt \\[0pt]
Morale et liberté \\[0pt]
Morale et politique sont-elles indépendantes ? \\[0pt]
Morale et pratique \\[0pt]
Morale et prudence \\[0pt]
Morale et religion \\[0pt]
Morale et science des mœurs \\[0pt]
Morale et sentiments \\[0pt]
Morale et sexualité \\[0pt]
Morale et société \\[0pt]
Morale et technique \\[0pt]
Morale et violence \\[0pt]
Mourir \\[0pt]
Mourir dans la dignité \\[0pt]
Mourir pour des principes \\[0pt]
Mourir pour la patrie \\[0pt]
Mouvement et changement \\[0pt]
Murs et frontières \\[0pt]
Musique et architecture \\[0pt]
Musique et bruit \\[0pt]
Musique et émotion \\[0pt]
Musique et poésie \\[0pt]
Musique et rhétorique \\[0pt]
Mythe et histoire \\[0pt]
Mythe et philosophie \\[0pt]
Mythe et symbole \\[0pt]
Mythes et idéologies \\[0pt]
N'aimer que soi \\[0pt]
Naître \\[0pt]
Naître et mourir \\[0pt]
Nation et richesse \\[0pt]
Naturaliser l'esprit \\[0pt]
Nature et fonction du sacrifice \\[0pt]
Nature et histoire \\[0pt]
Nature et institution \\[0pt]
Nature et institutions \\[0pt]
Nature et liberté \\[0pt]
Nature et mesure du temps \\[0pt]
Nature et monde \\[0pt]
Nature et morale \\[0pt]
Nature et nature humaine \\[0pt]
Naturel et artificiel \\[0pt]
Nature, monde, univers \\[0pt]
Naviguer \\[0pt]
N'avons-nous affaire qu'au réel ? \\[0pt]
Nécessaire / contingent \\[0pt]
Nécessité et contingence \\[0pt]
Nécessité fait loi \\[0pt]
N'échange-t-on que ce qui a de la valeur ? \\[0pt]
N'échange-t-on que des biens ? \\[0pt]
N'échange-t-on que des symboles ? \\[0pt]
« Ne fais pas à autrui ce que tu ne voudrais pas qu'on te fasse » \\[0pt]
Négation et privation \\[0pt]
Ne lèse personne \\[0pt]
N'en faire qu'à sa tête \\[0pt]
Ne pas choisir \\[0pt]
Ne pas comprendre \\[0pt]
Ne pas multiplier en vain les entités \\[0pt]
Ne pas raconter d'histoires \\[0pt]
Ne pas savoir ce que l'on fait \\[0pt]
Ne pas tuer \\[0pt]
Ne penser à rien \\[0pt]
Ne penser qu'à soi \\[0pt]
Ne prêche-t-on que les convertis ? \\[0pt]
Ne rien devoir à personne \\[0pt]
Ne sait-on rien que par expérience ? \\[0pt]
Ne sommes-nous véritablement maîtres que de nos pensées ? \\[0pt]
N'est-on juste que par crainte du châtiment ? \\[0pt]
Ne travaille-t- on que pour un salaire ? \\[0pt]
Névroses et psychoses \\[0pt]
N'existe-t-il que des individus ? \\[0pt]
N'existe-t-il qu'un seul temps ? \\[0pt]
N'exprime-t-on que ce dont on a conscience ? \\[0pt]
« Ni Dieu, ni maître ! » \\[0pt]
Ni Dieu ni maître \\[0pt]
Ni Dieu, ni maître \\[0pt]
Nier la vérité \\[0pt]
Nier le monde \\[0pt]
Nier l'évidence \\[0pt]
Ni regrets, ni remords \\[0pt]
Nomade et sédentaire \\[0pt]
Nommer \\[0pt]
Non-sens et absurdité \\[0pt]
Normalité et normativité \\[0pt]
Norme et moyenne \\[0pt]
Normes morales et normes vitales \\[0pt]
Nos devoirs nous motivent-ils ? \\[0pt]
Nos goûts sont-ils justifiables ? \\[0pt]
Nos sens peuvent-ils s'éduquer ? \\[0pt]
Notre besoin de fictions \\[0pt]
Notre connaissance du réel se limite-t-elle au savoir scientifique ? \\[0pt]
Notre corps pense-t-il ? \\[0pt]
Notre ignorance nous excuse-t-elle ? \\[0pt]
Notre rapport au monde peut-il n'être que technique ? \\[0pt]
Nul n'est censé ignorer la loi \\[0pt]
N'y a-t-il d'amitié qu'entre égaux ? \\[0pt]
N'y a-t-il de beauté qu'artistique ? \\[0pt]
N'y a-t-il de bonheur que dans l'instant ? \\[0pt]
N'y a-t-il de connaissance que par les causes ? \\[0pt]
N'y a-t-il de connaissance que scientifique ? \\[0pt]
N'y a-t-il de droit que de l'homme ? \\[0pt]
N'y a-t-il de propriété que privée ? \\[0pt]
N'y a-t-il de rationalité que scientifique ? \\[0pt]
N'y a-t-il de réel que le présent ? \\[0pt]
N'y a-t-il de science qu'autant qu'il s'y trouve de mathématique ? \\[0pt]
N'y a-t-il de science que du général ? \\[0pt]
N'y a-t-il de science que du nécessaire ? \\[0pt]
N'y a-t-il de science que provisoire ? \\[0pt]
N'y a-t-il de science que systématique ? \\[0pt]
N'y a-t-il de sens que par le langage ? \\[0pt]
N'y a-t-il d'être que sensible ? \\[0pt]
N'y a-t-il de vérité que scientifique ? \\[0pt]
N'y a-t-il d'origine que mythique ? \\[0pt]
N'y a-t-il que du mesurable ? \\[0pt]
N'y a-t-il qu'une substance ? \\[0pt]
N'y a-t-il qu'un seul monde ? \\[0pt]
N'y a-t-il rien au monde de certain ? \\[0pt]
Obéir \\[0pt]
Obéir, est-ce se soumettre ? \\[0pt]
Obéir et désobéir \\[0pt]
Obéissance et esclavage \\[0pt]
Obéissance et servitude \\[0pt]
Objecter et répondre \\[0pt]
Objectivité et subjectivité \\[0pt]
Objet d'art, objet d'échange \\[0pt]
Observation et expérimentation \\[0pt]
Observer \\[0pt]
Observer, décrire et expliquer \\[0pt]
Observer et expérimenter \\[0pt]
« Œil pour œil, dent pour dent » \\[0pt]
Œuvre et événement \\[0pt]
Œuvrer \\[0pt]
Offre et demande \\[0pt]
Ontologie et théologie \\[0pt]
Opinion, croyance, jugement \\[0pt]
Opinion publique et conscience universelle \\[0pt]
Ordre et chaos \\[0pt]
Ordre et désordre \\[0pt]
Ordre et liberté \\[0pt]
Organe et fonction \\[0pt]
Organiser \\[0pt]
Organisme et milieu \\[0pt]
Origine des hommes et principe de l'humanité \\[0pt]
Origine et commencement \\[0pt]
Origine et fondement \\[0pt]
Où commence la servitude ? \\[0pt]
Où commence la vie privée ? \\[0pt]
Où commence le territoire de la psychologie ? \\[0pt]
Où commence l'interprétation ? \\[0pt]
Où est le danger ? \\[0pt]
Où est le passé ? \\[0pt]
Où est le pouvoir ? \\[0pt]
Où est-on quand on pense ? \\[0pt]
Où s'arrête la responsabilité ? \\[0pt]
Où s'arrête l'espace public ? \\[0pt]
Où sont les relations ? \\[0pt]
Où suis-je ? \\[0pt]
Où suis-je quand je pense ? \\[0pt]
Outil et machine \\[0pt]
Par-delà beauté et laideur \\[0pt]
Pardonner \\[0pt]
Pardonner et oublier \\[0pt]
Parfaire \\[0pt]
Parier \\[0pt]
Parler de Dieu, est-ce nécessairement tenir un discours religieux ? \\[0pt]
Parler de soi \\[0pt]
Parler de soi est-il intéressant ? \\[0pt]
Parler, écrire, penser \\[0pt]
Parler, est-ce communiquer ? \\[0pt]
Parler, est-ce ne pas agir ? \\[0pt]
Parler et agir \\[0pt]
Parler et penser \\[0pt]
Parler pour ne rien dire \\[0pt]
Parler pour quelqu'un \\[0pt]
Par où commencer ? \\[0pt]
Par quoi un individu diffère-t-il réellement d'un autre ? \\[0pt]
Par quoi un individu se distingue-t-il d'un autre ? \\[0pt]
Partager \\[0pt]
Partager les richesses \\[0pt]
Partager sa vie \\[0pt]
Partager ses sentiments \\[0pt]
Participation et imitation \\[0pt]
Passer du fait au droit \\[0pt]
Pâtir \\[0pt]
Peindre \\[0pt]
Peindre d'après nature \\[0pt]
Peindre, est-ce traduire ? \\[0pt]
Peindre la présence \\[0pt]
Peindre un paysage, est-ce représenter la nature ? \\[0pt]
Peinture et abstraction \\[0pt]
Peinture et histoire \\[0pt]
Peinture et photographie \\[0pt]
Peinture et réalité \\[0pt]
Peinture et vérité \\[0pt]
Pensée et calcul \\[0pt]
Pensée et réalité \\[0pt]
Penser, est-ce aller contre l'évidence ? \\[0pt]
Penser, est-ce calculer ? \\[0pt]
Penser, est-ce comparer ? \\[0pt]
Penser, est-ce dire non ? \\[0pt]
Penser et calculer \\[0pt]
Penser et être \\[0pt]
Penser et parler \\[0pt]
Penser la matière \\[0pt]
Penser la technique \\[0pt]
Penser l'avenir \\[0pt]
Penser le devenir \\[0pt]
Penser le réel \\[0pt]
Penser le rien, est-ce ne rien penser ? \\[0pt]
Penser les sociétés comme des organismes \\[0pt]
Penser par soi-même \\[0pt]
Penser requiert-il d'avoir un corps ? \\[0pt]
Penser requiert-il un corps ? \\[0pt]
Penser sans corps \\[0pt]
Perception et aperception \\[0pt]
Perception et connaissance \\[0pt]
Perception et création artistique \\[0pt]
Perception et jugement \\[0pt]
Perception et mémoire \\[0pt]
Perception et mouvement \\[0pt]
Perception et observation \\[0pt]
Perception et passivité \\[0pt]
Perception et souvenir \\[0pt]
Perception et vérité \\[0pt]
Percer les apparences \\[0pt]
Percevoir \\[0pt]
Percevoir, est-ce connaître ? \\[0pt]
Percevoir, est-ce interpréter ? \\[0pt]
Percevoir, est-ce juger ? \\[0pt]
Percevoir, est-ce reconnaître ? \\[0pt]
Percevoir, est-ce savoir ? \\[0pt]
Percevoir et imaginer \\[0pt]
Percevoir et juger \\[0pt]
Percevoir et sentir \\[0pt]
Percevoir s'apprend-il ? \\[0pt]
Percevons-nous le monde extérieur ? \\[0pt]
Perçoit-on les choses comme elles sont ? \\[0pt]
Perdre connaissance \\[0pt]
Perdre la face \\[0pt]
Perdre la mémoire \\[0pt]
Perdre la raison \\[0pt]
Perdre ses habitudes \\[0pt]
Perdre ses illusions \\[0pt]
Perdre son âme \\[0pt]
Permanent / successif \\[0pt]
Permettre \\[0pt]
Persévérer dans son être \\[0pt]
Persister \\[0pt]
Persuader \\[0pt]
Persuader et convaincre \\[0pt]
« Petites causes, grands effets » \\[0pt]
Peuple et culture \\[0pt]
Peuple et masse \\[0pt]
Peuple et société \\[0pt]
Peuples et masses \\[0pt]
Peut-il être moral de tuer ? \\[0pt]
Peut-il y avoir de bons tyrans ? \\[0pt]
Peut-il y avoir de la politique sans conflit ? \\[0pt]
Peut-il y avoir des degrés dans la vérité ? \\[0pt]
Peut-il y avoir des nations sans État ? \\[0pt]
Peut-il y avoir science sans intuition du vrai ? \\[0pt]
Peut-il y avoir un droit à désobéir ? \\[0pt]
Peut-il y avoir une anthropologie non anthropocentriste ? \\[0pt]
Peut-il y avoir une ethnologie non ethnocentriste ? \\[0pt]
Peut-il y avoir une méthode commune à toutes les sciences ? \\[0pt]
Peut-il y avoir une morale dans les échanges ? \\[0pt]
Peut-il y avoir une philosophie applicable ? \\[0pt]
Peut-il y avoir une philosophie politique sans Dieu ? \\[0pt]
Peut-il y avoir une politique de la langue ? \\[0pt]
Peut-il y avoir une pratique sans théorie ? \\[0pt]
Peut-il y avoir une science de l'éducation ? \\[0pt]
Peut-il y avoir une science de l'homme ? \\[0pt]
Peut-il y avoir une science des signes ? \\[0pt]
Peut-il y avoir une science politique ? \\[0pt]
Peut-il y avoir une servitude volontaire ? \\[0pt]
Peut-il y avoir une société des nations ? \\[0pt]
Peut-il y avoir une société sans État ? \\[0pt]
Peut-il y avoir une vérité en politique ? \\[0pt]
Peut-il y avoir un intérêt collectif ? \\[0pt]
Peut-on admettre ce qui n'est pas vérifié ? \\[0pt]
Peut-on admettre un droit à la révolte ? \\[0pt]
Peut-on affirmer qu'être, c'est être perçu ou percevoir ? \\[0pt]
Peut-on agir sans prendre de risques ? \\[0pt]
Peut-on agir sans raison ? \\[0pt]
Peut-on aimer ce qu'on ne connaît pas ? \\[0pt]
Peut-on aimer les animaux ? \\[0pt]
Peut-on aimer l'humanité ? \\[0pt]
Peut-on aimer son prochain comme soi-même ? \\[0pt]
Peut-on à la fois préserver et dominer la nature ? \\[0pt]
Peut-on appréhender les choses telles qu'elles sont ? \\[0pt]
Peut-on apprendre à être heureux ? \\[0pt]
Peut-on apprendre à être juste ? \\[0pt]
Peut-on apprendre à être libre ? \\[0pt]
Peut-on apprendre à mourir ? \\[0pt]
Peut-on apprendre à penser ? \\[0pt]
Peut-on apprendre à vivre ? \\[0pt]
Peut-on arrêter le progrès ? \\[0pt]
Peut-on avoir le droit de se révolter ? \\[0pt]
Peut-on avoir raison contre les faits ? \\[0pt]
Peut-on avoir raison contre tous ? \\[0pt]
Peut-on avoir raison tout seul ? \\[0pt]
Peut-on avoir trop d'imagination ? \\[0pt]
Peut-on bien agir sans le savoir ? \\[0pt]
Peut-on changer de culture ? \\[0pt]
Peut-on changer de logique ? \\[0pt]
Peut-on changer le monde ? \\[0pt]
Peut-on changer le passé ? \\[0pt]
Peut-on comparer deux philosophies ? \\[0pt]
Peut-on comprendre ce qui est illogique ? \\[0pt]
Peut-on comprendre les actions humaines ? \\[0pt]
Peut-on concevoir la science achevée ? \\[0pt]
Peut-on concevoir une morale sans sanction ni obligation ? \\[0pt]
Peut-on concevoir une science qui ne soit pas démonstrative ? \\[0pt]
Peut-on concevoir une société qui n'aurait plus besoin du droit ? \\[0pt]
Peut-on concevoir un État mondial ? \\[0pt]
Peut-on concilier nécessité et liberté ? \\[0pt]
Peut-on conclure de l'être au devoir-être ? \\[0pt]
Peut-on connaître autrui ? \\[0pt]
Peut-on connaître les causes ? \\[0pt]
Peut-on connaître une vie humaine ? \\[0pt]
Peut-on considérer la conscience comme une chose ? \\[0pt]
Peut-on considérer l'art comme un langage ? \\[0pt]
Peut-on considérer les hommes de science comme des autorités morales ? \\[0pt]
Peut-on convaincre quelqu'un de la beauté d'une œuvre d'art ? \\[0pt]
Peut-on critiquer la démocratie ? \\[0pt]
Peut-on croire ce qu'on veut ? \\[0pt]
Peut-on croire sans être crédule ? \\[0pt]
Peut-on croire sans savoir pourquoi ? \\[0pt]
Peut-on décider de croire ? \\[0pt]
Peut-on définir ce qu'est un progrès technique ? \\[0pt]
Peut-on définir la santé ? \\[0pt]
Peut-on définir la vérité ? \\[0pt]
Peut-on définir la vérité par la correspondance ? \\[0pt]
Peut-on définir la vie ? \\[0pt]
Peut-on définir le bien ? \\[0pt]
Peut-on démontrer qu'on ne rêve pas ? \\[0pt]
Peut-on dériver des concepts de l'expérience ? \\[0pt]
Peut-on désirer ce qu'on possède ? \\[0pt]
Peut-on dialoguer avec soi-même ? \\[0pt]
Peut-on dire ce qui n'est pas ? \\[0pt]
Peut-on dire de la connaissance scientifique qu'elle procède par approximation ? \\[0pt]
Peut-on dire de l'art qu'il donne un monde en partage ? \\[0pt]
Peut-on dire d'une image qu'elle parle ? \\[0pt]
Peut-on dire d'une théorie scientifique qu'elle n'est jamais plus vraie qu'une autre mais seulement plus commode ? \\[0pt]
Peut-on dire l'être ? \\[0pt]
Peut-on dire que la science ne nous fait pas connaître les choses mais les rapports entre les choses ? \\[0pt]
Peut-on dire que rien n'échappe à la technique ? \\[0pt]
Peut-on dire qu'est vrai ce qui correspond aux faits ? \\[0pt]
Peut-on dire que tout est politique ? \\[0pt]
Peut-on dire qu'une théorie physique en contredit une autre ? \\[0pt]
Peut-on dire toute la vérité ? \\[0pt]
Peut-on disposer de son corps ? \\[0pt]
Peut-on distinguer différents types de causes ? \\[0pt]
Peut-on distinguer le réel de l'imaginaire ? \\[0pt]
Peut-on distinguer les faits de leurs interprétations ? \\[0pt]
Peut-on distinguer le vrai du faux ? \\[0pt]
Peut-on donner un sens à la souffrance ? \\[0pt]
Peut-on douter de la raison ? \\[0pt]
Peut-on douter de l'humanité d'un homme ? \\[0pt]
Peut-on douter de sa propre existence ? \\[0pt]
Peut-on douter de tout ? \\[0pt]
Peut-on douter du sens des mots ? \\[0pt]
Peut-on éclairer la liberté ? \\[0pt]
Peut-on en appeler à la conscience contre la loi ? \\[0pt]
Peut-on encore être cosmopolite ? \\[0pt]
Peut-on encore parler d'une nature humaine ? \\[0pt]
Peut-on en finir avec les préjugés ? \\[0pt]
Peut-on en savoir trop ? \\[0pt]
Peut-on entreprendre d'éliminer la métaphysique ? \\[0pt]
Peut-on établir une hiérarchie des arts ? \\[0pt]
Peut-on être aliéné sans le savoir ? \\[0pt]
Peut-on être amoral ? \\[0pt]
Peut-on être apolitique ? \\[0pt]
Peut-on être citoyen du monde ? \\[0pt]
Peut-on être citoyen sans être soldat ? \\[0pt]
Peut-on être content de soi ? \\[0pt]
Peut-on être désintéressé ? \\[0pt]
Peut-on être en conflit avec soi-même ? \\[0pt]
Peut-on être fidèle à soi-même ? \\[0pt]
Peut-on être heureux sans le savoir ? \\[0pt]
Peut-on être heureux tout seul ? \\[0pt]
Peut-on être homme sans être citoyen ? \\[0pt]
Peut-on être hors de soi ? \\[0pt]
Peut-on être injuste et heureux ? \\[0pt]
Peut-on être insensible à l'art ? \\[0pt]
Peut-on être juste dans une société injuste ? \\[0pt]
Peut-on être libre sans le savoir ? \\[0pt]
Peut-on être objectif en matière d'art ? \\[0pt]
Peut-on être plus ou moins libre ? \\[0pt]
Peut-on être prisonnier de soi-même ? \\[0pt]
Peut-on être sage tout seul ? \\[0pt]
Peut-on être sans opinion ? \\[0pt]
Peut-on être seul ? \\[0pt]
Peut-on être soi-même en société ? \\[0pt]
Peut-on être sûr d'avoir raison ? \\[0pt]
Peut-on être trop religieux ? \\[0pt]
Peut-on être trop sage ? \\[0pt]
Peut-on être trop sensible ? \\[0pt]
Peut-on étudier l'homme de l'extérieur ? \\[0pt]
Peut-on expliquer le mal ? \\[0pt]
Peut-on expliquer le monde par la matière ? \\[0pt]
Peut-on expliquer une œuvre d'art ? \\[0pt]
Peut-on faire ce qu'on ne veut pas ? \\[0pt]
Peut-on faire comme si le passé n'existait pas ? \\[0pt]
Peut-on faire de l'art avec tout ? \\[0pt]
Peut-on faire de sa vie une œuvre d'art ? \\[0pt]
Peut-on faire du dialogue un modèle de relation morale ? \\[0pt]
Peut-on faire le bien de quelqu'un malgré lui ? \\[0pt]
Peut-on faire l'économie de la notion de forme ? \\[0pt]
Peut-on faire le mal en vue du bien ? \\[0pt]
Peut-on faire l'inventaire du monde ? \\[0pt]
Peut-on fixer des limites à la science ? \\[0pt]
Peut-on fonder la morale sur la pitié ? \\[0pt]
Peut-on fonder le droit de propriété ? \\[0pt]
Peut-on fonder les droits de l'homme ? \\[0pt]
Peut-on fonder les mathématiques ? \\[0pt]
Peut-on fonder une morale sur la nature ? \\[0pt]
Peut-on gouverner sans lois ? \\[0pt]
Peut-on guérir par la parole ? \\[0pt]
Peut-on hiérarchiser les œuvres d'art ? \\[0pt]
Peut-on hiérarchiser les sociétés ? \\[0pt]
Peut-on innover en politique ? \\[0pt]
Peut-on interpréter la nature ? \\[0pt]
Peut-on inventer en morale ? \\[0pt]
Peut-on jamais aimer son prochain ? \\[0pt]
Peut-on juger de la valeur d'une vie humaine ? \\[0pt]
Peut-on juger des œuvres d'art sans recourir à l'idée de beauté ? \\[0pt]
Peut-on justifier la discrimination ? \\[0pt]
Peut-on justifier la guerre ? \\[0pt]
Peut-on justifier la raison d'État ? \\[0pt]
Peut-on justifier le mensonge ? \\[0pt]
Peut-on justifier le recours à l'idée de finalité ? \\[0pt]
Peut-on lutter contre le destin ? \\[0pt]
Peut-on lutter contre soi-même ? \\[0pt]
Peut-on maîtriser l'inconscient ? \\[0pt]
Peut-on manquer de culture ? \\[0pt]
Peut-on mesurer les phénomènes sociaux ? \\[0pt]
Peut-on montrer en cachant ? \\[0pt]
Peut-on ne croire en rien \\[0pt]
Peut-on ne croire en rien ? \\[0pt]
Peut-on ne pas être de son temps ? \\[0pt]
Peut-on ne pas être matérialiste ? \\[0pt]
Peut-on ne pas être soi-même ? \\[0pt]
Peut-on ne pas interpréter ? \\[0pt]
Peut-on ne pas savoir ce que l'on fait ? \\[0pt]
Peut-on ne pas savoir ce que l'on veut ? \\[0pt]
Peut-on ne pas vouloir être heureux ? \\[0pt]
Peut-on ne penser à rien ? \\[0pt]
Peut-on ne rien aimer ? \\[0pt]
Peut-on ne rien devoir à personne ? \\[0pt]
Peut-on ne rien vouloir ? \\[0pt]
Peut-on nier la réalité ? \\[0pt]
Peut-on nier l'évidence ? \\[0pt]
Peut-on nier l'existence de la matière ? \\[0pt]
Peut-on objectiver le psychisme ? \\[0pt]
Peut-on opposer justice et liberté ? \\[0pt]
Peut-on opposer morale et technique ? \\[0pt]
Peut-on opposer nature et culture ? \\[0pt]
Peut-on oublier ? \\[0pt]
Peut-on pactiser avec un brigand ? \\[0pt]
Peut-on parler d'art primitif ? \\[0pt]
Peut-on parler de capital culturel ? \\[0pt]
Peut-on parler de corruption des mœurs ? \\[0pt]
Peut-on parler de despotisme démocratique ? \\[0pt]
Peut-on parler de droits des animaux ? \\[0pt]
Peut-on parler de progrès en art ? \\[0pt]
Peut-on parler des œuvres d'art ? \\[0pt]
Peut-on parler de vérités métaphysiques ? \\[0pt]
Peut-on parler de vérité théâtrale ? \\[0pt]
Peut-on parler de vertu politique ? \\[0pt]
Peut-on parler de violence d'État ? \\[0pt]
Peut-on parler d'un droit de la guerre ? \\[0pt]
Peut-on parler d'un droit de résistance ? \\[0pt]
Peut-on parler d'une science de l'art ? \\[0pt]
Peut-on parler d'un progrès moral ? \\[0pt]
Peut-on parler d'un savoir poétique ? \\[0pt]
Peut-on parler d'un sens moral ? \\[0pt]
Peut-on parler d'un style moral ? \\[0pt]
Peut-on parler d'un travail intellectuel ? \\[0pt]
Peut-on partager ses goûts ? \\[0pt]
Peut-on partager une expérience ? \\[0pt]
Peut-on penser contre l'expérience ? \\[0pt]
Peut-on penser illogiquement ? \\[0pt]
Peut-on penser la création ? \\[0pt]
Peut-on penser la douleur ? \\[0pt]
Peut-on penser la fin de toute chose ? \\[0pt]
Peut-on penser la matière ? \\[0pt]
Peut-on penser la mort ? \\[0pt]
Peut-on penser l'art comme un langage ? \\[0pt]
Peut-on penser l'art sans référence au beau ? \\[0pt]
Peut-on penser l'avenir ? \\[0pt]
Peut-on penser le divin ? \\[0pt]
Peut-on penser le monde sans la technique ? \\[0pt]
Peut-on penser le réel comme un tout ? \\[0pt]
Peut-on penser le social sans la politique ? \\[0pt]
Peut-on penser le temps ? \\[0pt]
Peut-on penser le temps sans l'espace ? \\[0pt]
Peut-on penser l'extériorité ? \\[0pt]
Peut-on penser l'irrationnel ? \\[0pt]
Peut-on penser l'œuvre d'art sans référence à l'idée de beauté ? \\[0pt]
Peut-on penser sans concept ? \\[0pt]
Peut-on penser sans concepts ? \\[0pt]
Peut-on penser sans les mots ? \\[0pt]
Peut-on penser sans méthode ? \\[0pt]
Peut-on penser sans règles ? \\[0pt]
Peut-on penser sans signes ? \\[0pt]
Peut-on penser sans son corps ? \\[0pt]
Peut-on penser un art sans œuvres ? \\[0pt]
Peut-on penser une conscience sans mémoire ? \\[0pt]
Peut-on penser une conscience sans objet ? \\[0pt]
Peut-on penser une métaphysique sans Dieu ? \\[0pt]
Peut-on penser une religion sans le recours au divin ? \\[0pt]
Peut-on penser une volonté diabolique ? \\[0pt]
Peut-on penser un pouvoir absolu ? \\[0pt]
Peut-on percevoir sans s'en apercevoir ? \\[0pt]
Peut-on perdre la raison ? \\[0pt]
Peut-on perdre sa dignité ? \\[0pt]
Peut-on perdre sa liberté ? \\[0pt]
Peut-on perdre son identité ? \\[0pt]
Peut-on permettre le mensonge ? \\[0pt]
Peut-on préconiser, dans les sciences humaines et sociales, l'imitation des sciences de la nature ? \\[0pt]
Peut-on préférer l'ordre à la justice ? \\[0pt]
Peut-on prendre le risque de donner la mort en voulant soulager la souffrance ? \\[0pt]
Peut-on prévoir l'avenir ? \\[0pt]
Peut-on prévoir les hommes ? \\[0pt]
Peut-on prouver l'existence ? \\[0pt]
Peut-on prouver l'existence du monde ? \\[0pt]
Peut-on recommencer sa vie ? \\[0pt]
Peut-on reconnaître un sens à l'histoire sans lui assigner une fin ? \\[0pt]
Peut-on réduire la pensée à une espèce de comportement ? \\[0pt]
Peut-on réduire l'esprit à la matière ? \\[0pt]
Peut-on réduire une métaphysique à une conception du monde ? \\[0pt]
Peut-on réduire un homme à la somme de ses actes ? \\[0pt]
Peut-on refaire le monde ? \\[0pt]
Peut-on refuser la loi ? \\[0pt]
Peut-on refuser l'évidence ? \\[0pt]
Peut-on régner innocemment ? \\[0pt]
Peut-on renoncer à ses droits ? \\[0pt]
Peut-on renoncer au bonheur ? \\[0pt]
Peut-on réparer une injustice ? \\[0pt]
Peut-on représenter l'espace ? \\[0pt]
Peut-on représenter l'invisible ? \\[0pt]
Peut-on reprocher à la morale d'être abstraite ? \\[0pt]
Peut-on résister à la vérité ? \\[0pt]
Peut-on rester insensible à la beauté ? \\[0pt]
Peut-on rester sceptique ? \\[0pt]
Peut-on restreindre la logique à la pensée formelle ? \\[0pt]
Peut-on résumer une philosophie ? \\[0pt]
Peut-on réunir des arts différents dans une même œuvre ? \\[0pt]
Peut-on revendiquer la paix comme un droit ? \\[0pt]
Peut-on rire de tout ? \\[0pt]
Peut-on s'abstenir de penser politiquement ? \\[0pt]
Peut-on s'accorder sur des vérités morales ? \\[0pt]
Peut-on s'affranchir de sa propre nature ? \\[0pt]
Peut-on savoir ce qui est bien ? \\[0pt]
Peut-on se connaître soi-même ? \\[0pt]
Peut-on se désintéresser de la politique ? \\[0pt]
Peut-on se faire une idée de tout ? \\[0pt]
Peut-on se fier à sa propre raison ? \\[0pt]
Peut-on se fier à son intuition ? \\[0pt]
Peut-on se gouverner soi-même ? \\[0pt]
Peut-on se mentir à soi-même ? \\[0pt]
Peut-on se mettre à la place d'autrui ? \\[0pt]
Peut-on séparer politique et économie ? \\[0pt]
Peut-on se passer de chef ? \\[0pt]
Peut-on se passer de croire ? \\[0pt]
Peut-on se passer de Dieu ? \\[0pt]
Peut-on se passer de l'État ? \\[0pt]
Peut-on se passer de l'idée de cause finale ? \\[0pt]
Peut-on se passer de l'idée de Dieu ? \\[0pt]
Peut-on se passer de l'idée de nature humaine ? \\[0pt]
Peut-on se passer de métaphysique ? \\[0pt]
Peut-on se passer de représentants ? \\[0pt]
Peut-on se passer des causes finales ? \\[0pt]
Peut-on se passer de spiritualité ? \\[0pt]
Peut-on se passer des relations ? \\[0pt]
Peut-on se passer d'ontologie ? \\[0pt]
Peut-on se passer d'un maître ? \\[0pt]
Peut-on se peindre soi-même ? \\[0pt]
Peut-on se punir soi-même ? \\[0pt]
Peut-on se régler sur des exemples en politique ? \\[0pt]
Peut-on se retirer du monde ? \\[0pt]
Peut-on souhaiter le gouvernement des meilleurs ? \\[0pt]
Peut-on suspendre le temps ? \\[0pt]
Peut-on toujours savoir entièrement ce que l'on dit ? \\[0pt]
Peut-on tout attendre de l'État ? \\[0pt]
Peut-on tout définir ? \\[0pt]
Peut-on tout démontrer ? \\[0pt]
Peut-on tout dire ? \\[0pt]
Peut-on tout échanger ? \\[0pt]
Peut-on tout justifier ? \\[0pt]
Peut-on tout mesurer ? \\[0pt]
Peut-on tout prévoir ? \\[0pt]
Peut-on tout prouver ? \\[0pt]
Peut-on tout soumettre à la discussion ? \\[0pt]
Peut-on transformer le réel ? \\[0pt]
Peut-on transiger avec les principes ? \\[0pt]
Peut-on trop penser ? \\[0pt]
Peut-on trouver du plaisir à l'ennui ? \\[0pt]
Peut-on vivre avec les autres ? \\[0pt]
Peut-on vivre dans le doute ? \\[0pt]
Peut-on vivre en marge de la société ? \\[0pt]
Peut-on vivre en paix avec son inconscient ? \\[0pt]
Peut-on vivre hors du temps ? \\[0pt]
Peut-on vivre sans art ? \\[0pt]
Peut-on vivre sans aucune certitude ? \\[0pt]
Peut-on vivre sans croyance ? \\[0pt]
Peut-on vivre sans foi ni loi ? \\[0pt]
Peut-on vivre sans opinions ? \\[0pt]
Peut-on vivre sans ressentiment ? \\[0pt]
Peut-on vivre sans rien espérer ? \\[0pt]
Peut-on vivre sans travailler ? \\[0pt]
Peut-on vivre sans valeurs ? \\[0pt]
Peut-on vouloir le bien sans le faire ? \\[0pt]
Peut-on vouloir le mal ? \\[0pt]
Peut-on vouloir le mal sachant que c'est le mal ? \\[0pt]
Peut-on vouloir sans désirer ? \\[0pt]
Peut-on vraiment tirer des leçons du passé ? \\[0pt]
Phénomène ou fait social ? \\[0pt]
Philosopher, est-ce apprendre à vivre ? \\[0pt]
Philosophie et mathématiques \\[0pt]
Philosophie et métaphysique \\[0pt]
Philosophie et religion \\[0pt]
Philosophie et théologie \\[0pt]
Philosophie, psychologie, sociologie \\[0pt]
Photographier le réel \\[0pt]
Physique et cosmologie \\[0pt]
Physique et mathématique \\[0pt]
Physique et mathématiques \\[0pt]
Physique et métaphysique \\[0pt]
Plaider \\[0pt]
Plaisir et préférences \\[0pt]
Poésie et philosophie \\[0pt]
Poésie et vérité \\[0pt]
Point de vue du créateur et point de vue du spectateur \\[0pt]
Police et politique \\[0pt]
Politique et esthétique \\[0pt]
Politique et médias \\[0pt]
Politique et mémoire \\[0pt]
Politique et parole \\[0pt]
Politique et participation \\[0pt]
Politique et passions \\[0pt]
Politique et propagande \\[0pt]
Politique et religion \\[0pt]
Politique et secret \\[0pt]
Politique et technique \\[0pt]
Politique et technologie \\[0pt]
Politique et territoire \\[0pt]
Politique et trahison \\[0pt]
Politique et vertu \\[0pt]
Position, rôle, fonction \\[0pt]
Possibilité et intelligibilité \\[0pt]
Pour apprécier une œuvre, faut-il être cultivé ? \\[0pt]
Pour dialoguer faut-il parler le même langage ? \\[0pt]
Pour être heureux, faut-il renoncer à la perfection ? \\[0pt]
Pourquoi a-t-on peur de la folie ? \\[0pt]
Pourquoi avoir recours à la notion d'inconscient ? \\[0pt]
Pourquoi châtier ? \\[0pt]
Pourquoi chercher un sens à l'histoire ? \\[0pt]
Pourquoi commémorer ? \\[0pt]
Pourquoi conserver des œuvres d'art ? \\[0pt]
Pourquoi conserver les œuvres d'art ? \\[0pt]
Pourquoi croyons-nous ? \\[0pt]
Pourquoi définir ? \\[0pt]
Pourquoi démontrer ? \\[0pt]
Pourquoi démontrer ce que l'on sait être vrai ? \\[0pt]
Pourquoi des artifices ? \\[0pt]
Pourquoi des artistes ? \\[0pt]
Pourquoi des cérémonies ? \\[0pt]
Pourquoi des châtiments ? \\[0pt]
Pourquoi des classifications ? \\[0pt]
Pourquoi des comités d'éthique ? \\[0pt]
Pourquoi des conflits ? \\[0pt]
Pourquoi des constitutions ? \\[0pt]
Pourquoi des corps intermédiaires ? \\[0pt]
Pourquoi des définitions ? \\[0pt]
Pourquoi des élections ? \\[0pt]
Pourquoi des exemples ? \\[0pt]
Pourquoi des fictions ? \\[0pt]
Pourquoi des géométries ? \\[0pt]
Pourquoi des historiens ? \\[0pt]
Pourquoi des hypothèses ? \\[0pt]
Pourquoi des institutions ? \\[0pt]
Pourquoi désirer ? \\[0pt]
Pourquoi des logiciens ? \\[0pt]
Pourquoi des lois ? \\[0pt]
Pourquoi des métaphores ? \\[0pt]
Pourquoi des modèles ? \\[0pt]
Pourquoi des musées ? \\[0pt]
Pourquoi des œuvres d'art ? \\[0pt]
Pourquoi des poètes ? \\[0pt]
Pourquoi des rites ? \\[0pt]
Pourquoi des sociologues ? \\[0pt]
Pourquoi des syllogismes ? \\[0pt]
Pourquoi des symboles ? \\[0pt]
Pourquoi des utopies ? \\[0pt]
Pourquoi Dieu se soucierait-il des affaires humaines ? \\[0pt]
Pourquoi dire la vérité ? \\[0pt]
Pourquoi donner ? \\[0pt]
Pourquoi douter ? \\[0pt]
Pourquoi écrire ? \\[0pt]
Pourquoi écrit-on des lois ? \\[0pt]
Pourquoi écrit-on les lois ? \\[0pt]
Pourquoi est-il difficile de rectifier une erreur ? \\[0pt]
Pourquoi être exigeant ? \\[0pt]
Pourquoi être moral ? \\[0pt]
Pourquoi exiger la cohérence \\[0pt]
Pourquoi faire de la métaphysique ? \\[0pt]
Pourquoi faire de la politique ? \\[0pt]
Pourquoi faire de l'histoire ? \\[0pt]
Pourquoi faire la guerre ? \\[0pt]
Pourquoi faire l'hypothèse de l'inconscient ? \\[0pt]
Pourquoi fait-on le mal ? \\[0pt]
Pourquoi faudrait-il être cohérent ? \\[0pt]
Pourquoi faudrait-il sauver les apparences ? \\[0pt]
Pourquoi faut-il agir ? \\[0pt]
Pourquoi faut-il être cohérent ? \\[0pt]
Pourquoi formaliser des arguments ? \\[0pt]
Pourquoi il n'y a pas de société sans art ? \\[0pt]
Pourquoi la guerre ? \\[0pt]
Pourquoi la musique intéresse-t-elle le philosophe ? \\[0pt]
Pourquoi la raison recourt-elle à l'hypothèse ? \\[0pt]
Pourquoi la réalité peut-elle dépasser la fiction ? \\[0pt]
Pourquoi l'art intéresse-t-il les philosophes ? \\[0pt]
Pourquoi l'économie est-elle politique ? \\[0pt]
Pourquoi le droit international est-il imparfait ? \\[0pt]
Pourquoi les États se font-ils la guerre ? \\[0pt]
Pourquoi les hommes jouent-ils ? \\[0pt]
Pourquoi les hommes mentent-ils ? \\[0pt]
Pourquoi les mathématiques s'appliquent-elles à la réalité ? \\[0pt]
Pourquoi les œuvres d'art résistent-elles au temps ? \\[0pt]
Pourquoi les sciences humaines s'intéressent-elles à la parenté ? \\[0pt]
Pourquoi l'ethnologue s'intéresse-t-il à la vie urbaine ? \\[0pt]
Pourquoi l'homme est-il l'objet de plusieurs sciences ? \\[0pt]
Pourquoi lire des romans ? \\[0pt]
Pourquoi mentir ? \\[0pt]
Pourquoi nous souvenons-nous ? \\[0pt]
Pourquoi nous trompons-nous ? \\[0pt]
Pourquoi obéir ? \\[0pt]
Pourquoi obéir aux lois ? \\[0pt]
Pourquoi obéit-on aux lois ? \\[0pt]
Pourquoi oublie-t-on ? \\[0pt]
Pourquoi pardonner ? \\[0pt]
Pourquoi parler de fautes de goût ? \\[0pt]
Pourquoi parler de jugement de goût ? \\[0pt]
Pourquoi parlons-nous ? \\[0pt]
Pourquoi peindre la nature ? \\[0pt]
Pourquoi penser à la mort ? \\[0pt]
Pourquoi penser l'impossible ? \\[0pt]
Pourquoi pensons-nous ? \\[0pt]
Pourquoi plusieurs sciences ? \\[0pt]
Pourquoi préférer l'original à sa reproduction ? \\[0pt]
Pourquoi promettre ? \\[0pt]
Pourquoi punir ? \\[0pt]
Pourquoi punit-on ? \\[0pt]
Pourquoi rechercher le bonheur ? \\[0pt]
Pourquoi sauver les phénomènes ? \\[0pt]
Pourquoi se cultiver ? \\[0pt]
Pourquoi se mettre à la place d'autrui ? \\[0pt]
Pourquoi séparer les pouvoirs ? \\[0pt]
Pourquoi se référer au destin ? \\[0pt]
Pourquoi s'étonner ? \\[0pt]
Pourquoi s'exprimer ? \\[0pt]
Pourquoi s'inspirer de l'art antique ? \\[0pt]
Pourquoi sommes-nous déçus par les œuvres d'un faussaire ? \\[0pt]
Pourquoi sommes-nous moraux ? \\[0pt]
Pourquoi transformer le monde ? \\[0pt]
Pourquoi travailler ? \\[0pt]
Pourquoi travaille-t-on ? \\[0pt]
Pourquoi un droit du travail ? \\[0pt]
Pourquoi une instruction publique ? \\[0pt]
Pourquoi vouloir gagner ? \\[0pt]
Pourquoi vouloir la vérité ? \\[0pt]
Pourquoi vouloir s'abstraire du quotidien ? \\[0pt]
Pourquoi voyager ? \\[0pt]
Pourquoi y a-t-il des conflits insolubles ? \\[0pt]
Pourquoi y a-t-il des religions ? \\[0pt]
Pourquoi y a-t-il plusieurs philosophies ? \\[0pt]
Pourquoi y a-t-il quelque chose plutôt que rien ? \\[0pt]
Pourrait-on vivre sans art ? \\[0pt]
Pourrait-on vivre sans idéal ? \\[0pt]
Pourrions-nous nous passer des musées ? \\[0pt]
Pourrions-nous vivre sans religion ? \\[0pt]
Pouvoir et autorité \\[0pt]
Pouvoir et contre-pouvoir \\[0pt]
Pouvoir et devoir \\[0pt]
Pouvoir et politique \\[0pt]
Pouvoir et puissance \\[0pt]
Pouvoir et savoir \\[0pt]
Pouvoirs et libertés \\[0pt]
Pouvoir temporel et pouvoir spirituel \\[0pt]
Pouvons-nous communiquer ce que nous sentons ? \\[0pt]
Pouvons-nous devenir meilleurs ? \\[0pt]
Pouvons-nous réellement tirer des leçons du passé ? \\[0pt]
Pouvons-nous vivre sans préjugés ? \\[0pt]
Prédicats et relations \\[0pt]
Prémisses et conclusions \\[0pt]
Prendre des risques \\[0pt]
Prendre le pouvoir \\[0pt]
Prendre les armes \\[0pt]
« Prendre ses désirs pour des réalités » \\[0pt]
Prendre soin \\[0pt]
Prendre son temps \\[0pt]
Prendre sur soi \\[0pt]
Prendre une décision \\[0pt]
Prendre une décision politique \\[0pt]
Prescrire et décrire \\[0pt]
Prescrire et interdire \\[0pt]
Présence et absence \\[0pt]
Présence et représentation \\[0pt]
Preuve et démonstration \\[0pt]
Preuve et résultat dans les sciences humaines \\[0pt]
Prévoir \\[0pt]
Prévoir les comportements humains \\[0pt]
Primitif ou premier ? \\[0pt]
Prince : nature ou fonction ? \\[0pt]
Principe et cause \\[0pt]
Principe et commencement \\[0pt]
Principe et conséquence \\[0pt]
Principe et fondement \\[0pt]
Privation et négation \\[0pt]
Probabilité et explication scientifique \\[0pt]
Problème, énigme, mystère \\[0pt]
Production et création \\[0pt]
Production et réception de l'œuvre d'art \\[0pt]
Produire et consommer \\[0pt]
Promettre \\[0pt]
Prophétie, utopie, pronostic \\[0pt]
Proposition et jugement \\[0pt]
Propriété privée, propriété collective \\[0pt]
Propriétés artistiques, propriétés esthétiques \\[0pt]
Propriétés et dispositions \\[0pt]
Prose et poésie \\[0pt]
Prospérité et sécurité \\[0pt]
Protester \\[0pt]
Prouver \\[0pt]
Prouver en métaphysique \\[0pt]
Prouver et éprouver \\[0pt]
Prouver et justifier \\[0pt]
Prouvez-le ! \\[0pt]
Providence et destin \\[0pt]
Psychanalyse et critique littéraire \\[0pt]
Psychanalyse et esthétique \\[0pt]
Psychanalyse et interprétation \\[0pt]
Psychologie et contrôle des comportements \\[0pt]
Psychologie et ethnologie \\[0pt]
Psychologie et métaphysique \\[0pt]
Psychologie et neurosciences \\[0pt]
Psychologie et philosophie \\[0pt]
Publier \\[0pt]
Puis-je comprendre autrui ? \\[0pt]
Puis-je devenir moi-même ? \\[0pt]
Puis-je douter de ma propre existence ? \\[0pt]
Puis-je être sûr de bien agir ? \\[0pt]
Puis-je être universel ? \\[0pt]
Puis-je juger une culture à laquelle je n'appartiens pas ? \\[0pt]
Puis-je me mettre à la place d'un autre ? \\[0pt]
Puis-je ne rien croire ? \\[0pt]
Puissance et pouvoir \\[0pt]
Pulsion et instinct \\[0pt]
Pulsions et passions \\[0pt]
Punir \\[0pt]
Qu'admire-t-on dans une œuvre d'art ? \\[0pt]
Qu'aime-t-on dans l'amour ? \\[0pt]
Qu'aime-t-on quand on aime une œuvre d'art ? \\[0pt]
Qu'aimons-nous dans l'amour ? \\[0pt]
Qualités premières et qualités secondes \\[0pt]
Qualités premières, qualités secondes \\[0pt]
Quand agit-on ? \\[0pt]
Quand faut-il désobéir ? \\[0pt]
Quand faut-il mentir ? \\[0pt]
Quand faut-il se taire ? \\[0pt]
Quand la guerre finira-t-elle ? \\[0pt]
Quand l'art est-il abstrait ? \\[0pt]
Quand la technique devient-elle art ? \\[0pt]
Quand pense-t-on ? \\[0pt]
Quand peut-on se passer de théories ? \\[0pt]
Quand suis-je en faute ? \\[0pt]
Quand suis-je moi-même ? \\[0pt]
Quand y a-t-il art ? \\[0pt]
Quand y a-t-il de l'art ? \\[0pt]
Quand y a-t-il métaphore ? \\[0pt]
Quand y a-t-il œuvre ? \\[0pt]
Quand y a-t-il paysage ? \\[0pt]
Quand y a-t-il peuple ? \\[0pt]
Quantification et intelligibilité \\[0pt]
Quantification et pensée scientifique \\[0pt]
Quantifier \\[0pt]
Quantité et qualité \\[0pt]
Qu'a perdu le fou ? \\[0pt]
Qu'appelle-t-on chef-d'œuvre ? \\[0pt]
Qu'apporte au mathématicien l'histoire des mathématiques ? \\[0pt]
Qu'apporte la photographie aux arts ? \\[0pt]
Qu'apporte l'histoire à la vie politique ? \\[0pt]
Qu'apprend-on dans les livres ? \\[0pt]
Qu'apprenons-nous de nos affects ? \\[0pt]
Qu'a-t-on le droit de pardonner ? \\[0pt]
Qu'a-t-on le droit d'interpréter ? \\[0pt]
Qu'attendons-nous des œuvres d'art ? \\[0pt]
Qu'attendre de l'État ? \\[0pt]
Qu'avons-nous à apprendre de nos illusions ? \\[0pt]
Qu'avons-nous à apprendre des historiens ? \\[0pt]
 Qu'avons-nous à attendre de la vie politique ? \\[0pt]
Que changent les révolutions ? \\[0pt]
Qu'échange-t-on ? \\[0pt]
Que cherchons-nous dans le regard des autres ? \\[0pt]
Que comprend-on d'une œuvre d'art ? \\[0pt]
Que connaissons-nous du vivant ? \\[0pt]
Que connaît-on de ses passions ? \\[0pt]
Que construit le politique ? \\[0pt]
Que crée l'artiste ? \\[0pt]
Que déduire d'une contradiction ? \\[0pt]
Que désirons-nous ? \\[0pt]
Que devons-nous à l'animal ? \\[0pt]
Que devons-nous à l'État ? \\[0pt]
Que disent les tables de vérité ? \\[0pt]
Que dit la loi ? \\[0pt]
Que dit la musique ? \\[0pt]
Que dit l'interdit ? \\[0pt]
Que dois-je à autrui ? \\[0pt]
Que dois-je à l'État ? \\[0pt]
Que doit l'art à la nature ? \\[0pt]
Que doit-on à autrui ? \\[0pt]
Que doit-on aux morts ? \\[0pt]
Que doit-on faire de ses rêves ? \\[0pt]
Que doit-on protéger ? \\[0pt]
Que faire de nos passions ? \\[0pt]
Que faire des adversaires ? \\[0pt]
Que faire du vraisemblable ? \\[0pt]
Que fait aux œuvres d'art leur reproductibilité ? \\[0pt]
Que fait la machine ? \\[0pt]
Que fait la police ? \\[0pt]
Que fait l'art à nos vies ? \\[0pt]
Que fait l'artiste ? \\[0pt]
Que fait le spectateur ? \\[0pt]
Que faut-il craindre ? \\[0pt]
Que faut-il pour faire un monde ? \\[0pt]
Que faut-il savoir pour agir ? \\[0pt]
Que faut-il savoir pour gouverner ? \\[0pt]
Que faut-il vouloir pour être heureux ? \\[0pt]
Que gagne l'art à devenir abstrait ? \\[0pt]
Que gagne l'art à se réfléchir ? \\[0pt]
Que gagne-t-on à se mettre à la place d'autrui ? \\[0pt]
Que garantit la séparation des pouvoirs ? \\[0pt]
Quel est le but de la politique ? \\[0pt]
Quel est le but d'une théorie physique ? \\[0pt]
Quel est le but du travail scientifique ? \\[0pt]
Quel est le fondement de la propriété ? \\[0pt]
Quel est le fondement de l'autorité ? \\[0pt]
Quel est le meilleur régime politique ? \\[0pt]
Quel est le pouvoir de la beauté ? \\[0pt]
Quel est le pouvoir de l'art ? \\[0pt]
Quel est le pouvoir des métaphores ? \\[0pt]
Quel est le problème posé par la pluralité des langues ? \\[0pt]
Quel est le rôle de la créativité dans les sciences ? \\[0pt]
Quel est le rôle du médecin ? \\[0pt]
Quel est le statut de la preuve en sciences humaines et sociales ? \\[0pt]
Quel est le sujet de l'histoire ? \\[0pt]
Quel est le sujet des sciences sociales ? \\[0pt]
Quel est le sujet du devenir ? \\[0pt]
Quel est l'être de l'illusion ? \\[0pt]
Quel est l'homme de l'ethnologie ? \\[0pt]
Quel est l'homme des Droits de l'homme ? \\[0pt]
Quel est l'humain des sciences humaines ? \\[0pt]
Quel est l'objectif des sciences humaines ? \\[0pt]
Quel est l'objet de la géométrie ? \\[0pt]
Quel est l'objet de la logique ? \\[0pt]
Quel est l'objet de la métaphysique ? \\[0pt]
Quel est l'objet de la perception ? \\[0pt]
Quel est l'objet de la philosophie politique ? \\[0pt]
Quel est l'objet de la psychologie ? \\[0pt]
Quel est l'objet de la psychologie collective ? \\[0pt]
Quel est l'objet de la psychologie sociale ? \\[0pt]
Quel est l'objet de la science ? \\[0pt]
Quel est l'objet de l'échange ? \\[0pt]
Quel est l'objet de l'esthétique ? \\[0pt]
Quel est l'objet de l'ontologie ? \\[0pt]
Quel est l'objet des sciences politiques ? \\[0pt]
Quel est l'objet du désir ? \\[0pt]
Quel être peut être un sujet de droits ? \\[0pt]
« Quelle chimère est-ce donc que l'homme » ? \\[0pt]
Quelle confiance accorder au langage ? \\[0pt]
Quelle espèce de vivants sommes-nous ? \\[0pt]
Quelle est la fin de la science ? \\[0pt]
Quelle est la limite du pouvoir de l'État ? \\[0pt]
Quelle est la matière de l'œuvre d'art ? \\[0pt]
Quelle est la nature du droit naturel ? \\[0pt]
Quelle est la place du sujet dans les sciences humaines ? \\[0pt]
Quelle est la réalité de la matière ? \\[0pt]
Quelle est la réalité des objets mathématiques ? \\[0pt]
Quelle est la spécificité de la communauté politique ? \\[0pt]
Quelle est la valeur de la connaissance ? \\[0pt]
Quelle est la valeur du témoignage ? \\[0pt]
Quelle expérience la politique requiert-elle ? \\[0pt]
Quelle idée le fanatique se fait-il de la vérité ? \\[0pt]
Quelle place donner au vraisemblable ? \\[0pt]
Quelle place la raison peut-elle faire à la croyance ? \\[0pt]
Quelle place les sciences humaines doivent-elles accorder à la subjectivité ? \\[0pt]
Quelle place pour la mesure dans les sciences de l'homme ? \\[0pt]
Quelle politique fait-on avec les sciences humaines ? \\[0pt]
Quelle réalité conférer aux fictions ? \\[0pt]
Quelle réalité la science décrit-elle ? \\[0pt]
Quelle réalité l'imagination nous fait-elle connaître ? \\[0pt]
Quelle réalité peut-on attribuer au temps ? \\[0pt]
Quelle responsabilité publique pour le chercheur en sciences sociales ? \\[0pt]
Quelles actions permettent d'être heureux ? \\[0pt]
Quelles lois pour les sciences humaines ? \\[0pt]
Quelles règles la technique dicte-t-elle à l'art ? \\[0pt]
Quelles sont les caractéristiques d'une proposition morale ? \\[0pt]
Quelle valeur donner à la notion de « corps social » ? \\[0pt]
Quelle vérité y a-t-il dans la perception ? \\[0pt]
Quel réel pour l'art ? \\[0pt]
Quel rôle attribuer à l'intuition \emph{a priori} dans une philosophie des mathématiques ? \\[0pt]
Quel rôle la logique joue-t-elle en mathématiques ? \\[0pt]
Quel rôle les statistiques jouent-elles dans les sciences humaines ? \\[0pt]
Quel rôle l'imagination joue-t-elle en mathématiques ? \\[0pt]
Quels critères de démarcation pour les sciences sociales ? \\[0pt]
Quel sens donner au projet de forger une langue parfaite ? \\[0pt]
Quel sens y a-t-il à se demander si les sciences humaines sont vraiment des sciences ? \\[0pt]
Quels matériaux pour les sciences humaines et sociales ? \\[0pt]
Quels sont les moyens légitimes de la politique ? \\[0pt]
Quel statut philosophique pour l'opinion ? \\[0pt]
Quel type de réalité faut-il attribuer à l'inconscient en psychanalyse ? \\[0pt]
Quel type de réalité faut-il attribuer au temps ? \\[0pt]
Quel usage faire de la finalité ? \\[0pt]
Quel vivant sommes-nous ? \\[0pt]
Que montre l'image ? \\[0pt]
Que montre un tableau ? \\[0pt]
Que ne peut-on pas expliquer ? \\[0pt]
Que nous apporte l'art ? \\[0pt]
Que nous apprend la biologie sur l'homme ? \\[0pt]
Que nous apprend la géométrie ? \\[0pt]
Que nous apprend la littérature ? \\[0pt]
Que nous apprend la métaphysique ? \\[0pt]
Que nous apprend l'animal ? \\[0pt]
Que nous apprend l'anthropologie ? \\[0pt]
Que nous apprend la poésie ? \\[0pt]
Que nous apprend l'approche sociologique de l'art ? \\[0pt]
Que nous apprend la psychanalyse de l'homme ? \\[0pt]
Que nous apprend la religion ? \\[0pt]
Que nous apprend la sociologie des sciences ? \\[0pt]
Que nous apprend le plaisir ? \\[0pt]
Que nous apprend l'erreur ? \\[0pt]
Que nous apprend le témoignage ? \\[0pt]
Que nous apprend le théâtre ? \\[0pt]
Que nous apprend le toucher ? \\[0pt]
Que nous apprend l'expérience ? \\[0pt]
Que nous apprend l'expérience esthétique ? \\[0pt]
Que nous apprend l'histoire de l'art ? \\[0pt]
Que nous apprend l'histoire des sciences ? \\[0pt]
Que nous apprend, sur la politique, l'utopie ? \\[0pt]
Que nous apprennent les algorithmes sur nos sociétés ? \\[0pt]
Que nous apprennent les Anciens ? \\[0pt]
Que nous apprennent les controverses scientifiques ? \\[0pt]
Que nous apprennent les faits divers ? \\[0pt]
Que nous apprennent les jeux ? \\[0pt]
Que nous apprennent les langues étrangères ? \\[0pt]
Que nous apprennent les objets techniques ? \\[0pt]
Que nous apprennent les statistiques ? \\[0pt]
Que nous apprennent nos sentiments ? \\[0pt]
Que nous devons-nous ? \\[0pt]
Que nous doit l'État ? \\[0pt]
Que nous enseigne la lecture des romans ? \\[0pt]
Que nous enseigne la monstruosité ? \\[0pt]
Que nous enseignent nos peurs ? \\[0pt]
Que nous impose la nature ? \\[0pt]
Que nous montrent les natures mortes ? \\[0pt]
« Que nul n'entre ici s'il n'est géomètre » \\[0pt]
Que peindre ? \\[0pt]
Que peint le peintre ? \\[0pt]
Que percevons-nous ? \\[0pt]
Que percevons-nous du monde extérieur ? \\[0pt]
Que perçoit-on ? \\[0pt]
Que perd la pensée en perdant l'écriture ? \\[0pt]
Que peut expliquer l'histoire ? \\[0pt]
Que peut la force ? \\[0pt]
Que peut la politique ? \\[0pt]
Que peut la raison ? \\[0pt]
Que peut la raison contre une croyance ? \\[0pt]
Que peut l'art ? \\[0pt]
Que peut la volonté contre les passions ? \\[0pt]
Que peut le droit ? \\[0pt]
Que peut le politique ? \\[0pt]
Que peut l'État ? \\[0pt]
Que peut l'intelligence artificielle ? \\[0pt]
Que peut-on apprendre des émotions esthétiques ? \\[0pt]
Que peut-on attendre de l'État ? \\[0pt]
Que peut-on attendre du droit international ? \\[0pt]
Que peut-on attendre d'une philosophie de l'art ? \\[0pt]
Que peut-on calculer ? \\[0pt]
Que peut-on comprendre qu'on ne puisse expliquer ? \\[0pt]
Que peut-on cultiver ? \\[0pt]
Que peut-on démontrer ? \\[0pt]
Que peut-on dire de l'être ? \\[0pt]
Que peut-on enseigner ? \\[0pt]
Que peut-on espérer ? \\[0pt]
Que peut-on exprimer ? \\[0pt]
Que peut-on interdire ? \\[0pt]
Que peut-on partager ? \\[0pt]
Que peut-on perdre ? \\[0pt]
Que peut-on sur autrui ? \\[0pt]
Que peut signifier « avoir le sens de la réalité » ? \\[0pt]
Que peut un corps ? \\[0pt]
Que peuvent les idées ? \\[0pt]
Que peuvent les images ? \\[0pt]
Que peuvent les lois ? \\[0pt]
Que peuvent nous apprendre les expériences de pensée ? \\[0pt]
Que pouvons-nous aujourd'hui apprendre des sciences d'autrefois ? \\[0pt]
Que pouvons-nous comprendre du monde ? \\[0pt]
Que pouvons-nous espérer ? \\[0pt]
Que pouvons-nous partager ? \\[0pt]
Que produit la poésie ? \\[0pt]
Que produit l'inconscient ? \\[0pt]
Que protège la loi ? \\[0pt]
Que prouvent les preuves de l'existence de Dieu ? \\[0pt]
Que puis-je dire être mien ? \\[0pt]
Que rend visible l'art ? \\[0pt]
Que sais-je de ma souffrance ? \\[0pt]
Que sait la science ? \\[0pt]
Que sait-on de soi ? \\[0pt]
Que savons-nous de l'avenir ? \\[0pt]
Que savons-nous de l'inconscient ? \\[0pt]
Que savons-nous de nous-mêmes ? \\[0pt]
Que savons-nous des principes ? \\[0pt]
Que sent le corps ? \\[0pt]
Que serait la fin de l'histoire ? \\[0pt]
Que serait le meilleur des mondes ? \\[0pt]
Que serait un art total ? \\[0pt]
Que serait une démocratie parfaite ? \\[0pt]
Que serions-nous sans l'État ? \\[0pt]
Que signifie apprendre ? \\[0pt]
Que signifie : « avoir de l'esprit » ? \\[0pt]
Que signifie être dépendant ? \\[0pt]
Que signifie l'angoisse ? \\[0pt]
Que signifie « politiquement correct » ? \\[0pt]
Que signifie prouver en sciences sociales ? \\[0pt]
Que sondent les sondages d'opinion ? \\[0pt]
Que sont les institutions sans les individus ? \\[0pt]
Qu'est-ce la vérité ? \\[0pt]
Qu'est-ce qu'abstraire ? \\[0pt]
Qu'est-ce qu'agir avec discernement ? \\[0pt]
Qu'est-ce qu'apparaître ? \\[0pt]
Qu'est-ce qu'appliquer une règle de droit ? \\[0pt]
Qu'est-ce qu'apprécier une œuvre d'art ? \\[0pt]
Qu'est-ce qu'apprendre ? \\[0pt]
Qu'est-ce qu'avoir conscience de soi ? \\[0pt]
Qu'est-ce qu'avoir de l'expérience ? \\[0pt]
Qu'est-ce qu'avoir du goût ? \\[0pt]
Qu'est-ce qu'avoir du style ? \\[0pt]
Qu'est-ce qu'avoir raison ? \\[0pt]
Qu'est-ce qu'avoir un droit ? \\[0pt]
Qu'est-ce qu'avoir une idée ? \\[0pt]
Qu'est-ce qu'avoir un esprit scientifique ? \\[0pt]
Qu'est-ce que bien vivre ? \\[0pt]
Qu'est-ce que calculer ? \\[0pt]
Qu'est-ce que classer ? \\[0pt]
Qu'est-ce que comprendre une œuvre d'art ? \\[0pt]
Qu'est-ce que connaître ? \\[0pt]
Qu'est-ce que démontrer ? \\[0pt]
Qu'est-ce que déraisonner ? \\[0pt]
Qu'est-ce que Dieu pour un athée ? \\[0pt]
Qu'est-ce que discuter ? \\[0pt]
Qu'est-ce qu'éduquer le sens esthétique ? \\[0pt]
Qu'est-ce que faire autorité ? \\[0pt]
Qu'est-ce que « faire de la politique » ? \\[0pt]
Qu'est-ce que faire la paix ? \\[0pt]
Qu'est-ce que faire preuve d'humanité ? \\[0pt]
Qu'est-ce que gouverner ? \\[0pt]
Qu'est-ce que guérir ? \\[0pt]
Qu'est-ce que juger ? \\[0pt]
Qu'est-ce que la culture générale \\[0pt]
Qu'est-ce que l'anthropologie politique ? \\[0pt]
Qu'est-ce que la psychanalyse nous apprend sur les sociétés humaines ? \\[0pt]
Qu'est-ce que la psychologie ? \\[0pt]
Qu'est-ce que l'art contemporain ? \\[0pt]
Qu'est-ce que la sociologie nous apprend sur la culture ? \\[0pt]
Qu'est-ce que le cinéma a changé dans l'idée que l'on se fait du temps ? \\[0pt]
Qu'est-ce que le désordre ? \\[0pt]
Qu'est-ce que le génie ? \\[0pt]
Qu'est-ce que le mal radical ? \\[0pt]
Qu'est-ce que le moi ? \\[0pt]
Qu'est-ce que le naturalisme ? \\[0pt]
Qu'est-ce que l'enfance ? \\[0pt]
Qu'est-ce que les sciences humaines apportent à la critique littéraire ? \\[0pt]
Qu'est-ce que les sciences religieuses nous apprennent de la religion ? \\[0pt]
Qu'est-ce que le style ? \\[0pt]
Qu'est-ce que le symbolique ? \\[0pt]
Qu'est-ce que le temps ? \\[0pt]
Qu'est-ce que l'expérience pour un empiriste ? \\[0pt]
Qu'est-ce que l'harmonie ? \\[0pt]
Qu'est-ce que l'identité sociale des individus ? \\[0pt]
Qu'est-ce que l'individualisme méthodologique ? \\[0pt]
Qu'est-ce que l'interprétation des Écritures nous apprend sur les religions ? \\[0pt]
Qu'est-ce que lire ? \\[0pt]
Qu'est-ce que méditer ? \\[0pt]
Qu'est-ce qu'enquêter ? \\[0pt]
Qu'est-ce que parler ? \\[0pt]
Qu'est-ce que perdre son temps ? \\[0pt]
Qu'est-ce que peut un corps ? \\[0pt]
Qu'est-ce que prendre conscience ? \\[0pt]
Qu'est-ce que prendre le pouvoir ? \\[0pt]
Qu'est-ce que raisonner ? \\[0pt]
Qu'est-ce que rendre raison d'un effet ? \\[0pt]
Qu'est-ce que résoudre une contradiction ? \\[0pt]
Qu'est-ce que réussir sa vie ? \\[0pt]
Qu'est-ce que savoir ? \\[0pt]
Qu'est-ce que se tromper ? \\[0pt]
Qu'est-ce que s'orienter ? \\[0pt]
Qu'est-ce que traduire ? \\[0pt]
Qu'est-ce qu'être asocial ? \\[0pt]
Qu'est-ce qu'être chez soi ? \\[0pt]
Qu'est-ce qu'être comportementaliste ? \\[0pt]
Qu'est-ce qu'être de bonne volonté ? \\[0pt]
Qu'est-ce qu'être démocrate ? \\[0pt]
Qu'est-ce qu'être ensemble ? \\[0pt]
Qu'est-ce qu'être fou ? \\[0pt]
Qu'est-ce qu'être libéral ? \\[0pt]
Qu'est-ce qu'être malade ? \\[0pt]
Qu'est-ce qu'être moderne ? \\[0pt]
Qu'est-ce qu'être réaliste ? \\[0pt]
Qu'est-ce qu'être républicain ? \\[0pt]
Qu'est-ce qu'être sceptique ? \\[0pt]
Qu'est-ce qu'être seul ? \\[0pt]
Qu'est-ce qu'être souverain ? \\[0pt]
Qu'est-ce qu'être un esclave ? \\[0pt]
Qu'est-ce qu'être un sujet ? \\[0pt]
Qu'est-ce qu'être vertueux ? \\[0pt]
Qu'est-ce qu'être visionnaire ? \\[0pt]
Qu'est-ce qu'être vivant ? \\[0pt]
Qu'est-ce que un individu \\[0pt]
Qu'est-ce que vieillir ? \\[0pt]
Qu'est-ce que vivre ? \\[0pt]
Qu'est-ce que vivre au présent ? \\[0pt]
Qu'est-ce qu'exercer un pouvoir ? \\[0pt]
Qu'est-ce qu'expliquer un comportement ? \\[0pt]
Qu'est-ce qu'habiter ? \\[0pt]
Qu'est-ce qui adoucit les mœurs ? \\[0pt]
Qu'est-ce qui a du sens ? \\[0pt]
Qu'est-ce qui agit ? \\[0pt]
Qu'est-ce qui apparaît ? \\[0pt]
Qu'est-ce qu'identifier ? \\[0pt]
Qu'est-ce qui dépend de nous ? \\[0pt]
Qu'est-ce qui échappe aux sciences humaines ? \\[0pt]
Qu'est-ce qui est anormal ? \\[0pt]
Qu'est-ce qui est artificiel ? \\[0pt]
Qu'est-ce qui est beau ? \\[0pt]
Qu'est-ce qui est commun à toutes les langues ? \\[0pt]
Qu'est-ce qui est concret ? \\[0pt]
Qu'est-ce qui est contre nature ? \\[0pt]
Qu'est-ce qui est culturel ? \\[0pt]
Qu'est-ce qui est démonstratif ? \\[0pt]
Qu'est-ce qui est donné ? \\[0pt]
Qu'est-ce qui est essentiel ? \\[0pt]
Qu'est-ce qui est historique ? \\[0pt]
Qu'est-ce qui est hors la loi ? \\[0pt]
Qu'est-ce qui est humain ? \\[0pt]
Qu'est-ce qui est impossible ? \\[0pt]
Qu'est-ce qui est indiscutable ? \\[0pt]
Qu'est-ce qui est invérifiable ? \\[0pt]
Qu'est-ce qui est irréfutable ? \\[0pt]
Qu'est-ce qui est mien ? \\[0pt]
Qu'est-ce qui est moderne ? \\[0pt]
Qu'est-ce qui est nécessaire ? \\[0pt]
Qu'est-ce qui est noble ? \\[0pt]
Qu'est-ce qui est normal ? \\[0pt]
Qu'est-ce qui est politique ? \\[0pt]
Qu'est-ce qui est public ? \\[0pt]
Qu'est-ce qui est réel ? \\[0pt]
Qu'est-ce qui est respectable ? \\[0pt]
Qu'est-ce qui est sacré ? \\[0pt]
Qu'est-ce qui est sans raison ? \\[0pt]
Qu'est-ce qui est sensible ? \\[0pt]
Qu'est-ce qui est spectaculaire ? \\[0pt]
Qu'est-ce qui est sublime ? \\[0pt]
Qu'est-ce qui est transcendant ? \\[0pt]
Qu'est-ce qui est vital pour le vivant ? \\[0pt]
Qu'est-ce qui existe ? \\[0pt]
Qu'est-ce qui fait des sciences humaines des sciences ? \\[0pt]
Qu'est-ce qui fait la force de la loi ? \\[0pt]
Qu'est-ce qui fait la force des lois ? \\[0pt]
Qu'est-ce qui fait la justice des lois ? \\[0pt]
Qu'est-ce qui fait la légitimité d'une autorité politique ? \\[0pt]
Qu'est-ce qui fait la valeur de l'œuvre d'art ? \\[0pt]
Qu'est-ce qui fait la valeur des objets d'art ? \\[0pt]
Qu'est-ce qui fait la valeur d'une croyance ? \\[0pt]
Qu'est-ce qui fait la valeur d'une œuvre d'art ? \\[0pt]
Qu'est-ce qui fait le propre d'un corps propre ? \\[0pt]
Qu'est-ce qui fait l'humanité d'un corps ? \\[0pt]
Qu'est-ce qui fait l'unité d'une science ? \\[0pt]
Qu'est-ce qui fait l'unité d'un peuple ? \\[0pt]
Qu'est-ce qui fait obstacle à la science ? \\[0pt]
Qu'est-ce qui fait qu'une théorie est vraie ? \\[0pt]
Qu'est-ce qui fait qu'un peuple est un peuple ? \\[0pt]
Qu'est-ce qui fait un peuple ? \\[0pt]
Qu'est-ce qui fonde la croyance ? \\[0pt]
Qu'est-ce qu'ignore la science ? \\[0pt]
Qu'est-ce qui importe dans l'éducation ? \\[0pt]
Qu'est-ce qui justifie l'hypothèse d'un inconscient ? \\[0pt]
Qu'est-ce qui me rend plus fort ? \\[0pt]
Qu'est-ce qui mérite l'admiration ? \\[0pt]
Qu'est-ce qui m'est étranger ? \\[0pt]
Qu'est-ce qui n'appartient pas au monde ? \\[0pt]
Qu'est-ce qui ne change pas ? \\[0pt]
Qu'est-ce qui ne s'achète pas ? \\[0pt]
Qu'est-ce qui ne s'échange pas ? \\[0pt]
Qu'est-ce qui n'est pas démontrable ? \\[0pt]
Qu'est-ce qui n'est pas en mouvement ? \\[0pt]
Qu'est-ce qui n'est pas maîtrisable ? \\[0pt]
Qu'est-ce qui n'est pas mathématisable ? \\[0pt]
Qu'est-ce qui n'est pas normal ? \\[0pt]
Qu'est-ce qui n'est pas politique ? \\[0pt]
Qu'est-ce qui n'existe pas ? \\[0pt]
Qu'est-ce qui nous oblige ? \\[0pt]
Qu'est-ce qu'interpréter ? \\[0pt]
Qu'est-ce qu'interpréter une œuvre d'art ? \\[0pt]
Qu'est-ce qui peut être hors du temps ? \\[0pt]
Qu'est-ce qui rend l'objectivité difficile dans les sciences humaines ? \\[0pt]
Qu'est-ce qui rend vrai un énoncé ? \\[0pt]
Qu'est-ce qu'obéir ? \\[0pt]
Qu'est-ce qu'on attend ? \\[0pt]
Qu'est-ce qu'on ne peut pas partager ? \\[0pt]
Qu'est-ce qu'un abus de langage ? \\[0pt]
Qu'est-ce qu'un abus de pouvoir ? \\[0pt]
Qu'est-ce qu'un accident ? \\[0pt]
Qu'est-ce qu'un acte libre ? \\[0pt]
Qu'est-ce qu'un acte moral ? \\[0pt]
Qu'est-ce qu'un acte symbolique ? \\[0pt]
Qu'est-ce qu'un acteur ? \\[0pt]
Qu'est-ce qu'un adversaire en politique ? \\[0pt]
Qu'est-ce qu'un alter ego \\[0pt]
Qu'est-ce qu'un ami ? \\[0pt]
Qu'est-ce qu'un animal ? \\[0pt]
Qu'est-ce qu'un animal domestique ? \\[0pt]
Qu'est-ce qu'un argument ? \\[0pt]
Qu'est-ce qu'un artefact ? \\[0pt]
Qu'est-ce qu'un artiste ? \\[0pt]
Qu'est-ce qu'un art moral ? \\[0pt]
Qu'est-ce qu'un auteur ? \\[0pt]
Qu'est-ce qu'un axiome ? \\[0pt]
Qu'est-ce qu'un beau parleur ? \\[0pt]
Qu'est-ce qu'un beau paysage ? \\[0pt]
Qu'est-ce qu'un beau travail ? \\[0pt]
Qu'est-ce qu'un bon citoyen ? \\[0pt]
Qu'est-ce qu'un bon commentaire ? \\[0pt]
Qu'est-ce qu'un bon conseil ? \\[0pt]
Qu'est-ce qu'un capital culturel ? \\[0pt]
Qu'est-ce qu'un cas de conscience ? \\[0pt]
Qu'est-ce qu'un cas en morale ? \\[0pt]
Qu'est-ce qu'un « champ artistique » ? \\[0pt]
Qu'est-ce qu'un chef ? \\[0pt]
Qu'est-ce qu'un chef-d'œuvre ? \\[0pt]
Qu'est-ce qu'un choix éclairé ? \\[0pt]
Qu'est-ce qu'un choix rationnel ? \\[0pt]
Qu'est-ce qu'un citoyen ? \\[0pt]
Qu'est-ce qu'un civilisé ? \\[0pt]
Qu'est-ce qu'un classique ? \\[0pt]
Qu'est-ce qu'un collectif ? \\[0pt]
Qu'est-ce qu'un concept ? \\[0pt]
Qu'est-ce qu'un concept scientifique ? \\[0pt]
Qu'est-ce qu'un conflit de valeurs ? \\[0pt]
Qu'est-ce qu'un conflit politique ? \\[0pt]
Qu'est-ce qu'un contenu de conscience ? \\[0pt]
Qu'est-ce qu'un contrat ? \\[0pt]
Qu'est-ce qu'un contre-pouvoir ? \\[0pt]
Qu'est-ce qu'un corps social ? \\[0pt]
Qu'est-ce qu'un coup d'État ? \\[0pt]
Qu'est-ce qu'un crime contre l'humanité ? \\[0pt]
Qu'est-ce qu'un crime politique ? \\[0pt]
Qu'est-ce qu'un déni ? \\[0pt]
Qu'est-ce qu'un dieu ? \\[0pt]
Qu'est-ce qu'un Dieu ? \\[0pt]
Qu'est-ce qu'un document ? \\[0pt]
Qu'est-ce qu'un dogme ? \\[0pt]
Qu'est-ce qu'une alternative ? \\[0pt]
Qu'est-ce qu'une âme ? \\[0pt]
Qu'est-ce qu'une aporie ? \\[0pt]
Qu'est-ce qu'une belle démonstration ? \\[0pt]
Qu'est-ce qu'une bonne loi ? \\[0pt]
Qu'est-ce qu'une bonne méthode ? \\[0pt]
Qu'est-ce qu'une bonne théorie ? \\[0pt]
Qu'est-ce qu'une catégorie ? \\[0pt]
Qu'est-ce qu'une catégorie de l'être ? \\[0pt]
Qu'est-ce qu'une cause ? \\[0pt]
Qu'est-ce qu'un échange juste ? \\[0pt]
Qu'est-ce qu'une chimère ? \\[0pt]
Qu'est-ce qu'une chose ? \\[0pt]
Qu'est-ce qu'une civilisation ? \\[0pt]
Qu'est-ce qu'une classe sociale ? \\[0pt]
Qu'est-ce qu'une classification scientifique ? \\[0pt]
Qu'est-ce qu'une collectivité ? \\[0pt]
Qu'est-ce qu'une communauté ? \\[0pt]
Qu'est-ce qu'une communauté humaine ? \\[0pt]
Qu'est-ce qu'une communauté politique ? \\[0pt]
Qu'est-ce qu'une communauté scientifique ? \\[0pt]
Qu'est-ce qu'une conception scientifique du monde ? \\[0pt]
Qu'est-ce qu'une condition de possibilité ? \\[0pt]
Qu'est-ce qu'une condition suffisante ? \\[0pt]
Qu'est-ce qu'une conduite irrationnelle ? \\[0pt]
Qu'est-ce qu'une connaissance a priori ? \\[0pt]
Qu'est-ce qu'une connaissance métaphysique \\[0pt]
Qu'est-ce qu'une connaissance non scientifique ? \\[0pt]
Qu'est-ce qu'une conscience collective ? \\[0pt]
Qu'est-ce qu'une constitution ? \\[0pt]
Qu'est-ce qu'une crise ? \\[0pt]
Qu'est-ce qu'une crise politique ? \\[0pt]
Qu'est-ce qu'une croyance rationnelle ? \\[0pt]
Qu'est-ce qu'une croyance vraie ? \\[0pt]
Qu'est-ce qu'une culture ? \\[0pt]
Qu'est-ce qu'une découverte ? \\[0pt]
Qu'est-ce qu'une découverte scientifique ? \\[0pt]
Qu'est-ce qu'une discipline savante ? \\[0pt]
Qu'est-ce qu'une disposition ? \\[0pt]
Qu'est-ce qu'une école philosophique ? \\[0pt]
Qu'est-ce qu'une éducation réussie ? \\[0pt]
Qu'est-ce qu'une éducation scientifique ? \\[0pt]
Qu'est-ce qu'une époque ? \\[0pt]
Qu'est-ce qu'une erreur ? \\[0pt]
Qu'est-ce qu'une espèce naturelle ? \\[0pt]
Qu'est-ce qu'une existence historique ? \\[0pt]
Qu'est-ce qu'une expérience cruciale ? \\[0pt]
Qu'est-ce qu'une expérience de pensée ? \\[0pt]
Qu'est-ce qu'une expérience esthétique ? \\[0pt]
Qu'est-ce qu'une expérience politique ? \\[0pt]
Qu'est-ce qu'une explication matérialiste ? \\[0pt]
Qu'est-ce qu'une exposition ? \\[0pt]
Qu'est-ce qu'une famille ? \\[0pt]
Qu'est-ce qu'une fonction ? \\[0pt]
Qu'est-ce qu'une forme ? \\[0pt]
Qu'est-ce qu'une guerre juste ? \\[0pt]
Qu'est-ce qu'une histoire vraie ? \\[0pt]
Qu'est-ce qu'une hypothèse ? \\[0pt]
Qu'est-ce qu'une hypothèse scientifique ? \\[0pt]
Qu'est-ce qu'une idée ? \\[0pt]
Qu'est-ce qu'une idée esthétique ? \\[0pt]
Qu'est-ce qu'une idée incertaine ? \\[0pt]
Qu'est-ce qu'une idée morale ? \\[0pt]
Qu'est-ce qu'une idée vraie ? \\[0pt]
Qu'est-ce qu'une idéologie ? \\[0pt]
Qu'est-ce qu'une idéologie scientifique ? \\[0pt]
Qu'est-ce qu'une image ? \\[0pt]
Qu'est-ce qu'une impression ? \\[0pt]
Qu'est-ce qu'une inférence ? \\[0pt]
Qu'est-ce qu'une injustice ? \\[0pt]
Qu'est-ce qu'une innovation technologique ? \\[0pt]
Qu'est-ce qu'une institution ? \\[0pt]
Qu'est-ce qu'une langue bien faite ? \\[0pt]
Qu'est-ce qu'une langue morte ? \\[0pt]
Qu'est-ce qu'un élément ? \\[0pt]
Qu'est-ce qu'une limite ? \\[0pt]
Qu'est-ce qu'une logique sociale ? \\[0pt]
Qu'est-ce qu'une loi ? \\[0pt]
Qu'est-ce qu'une loi d'exception ? \\[0pt]
Qu'est-ce qu'une loi injuste ? \\[0pt]
Qu'est-ce qu'une loi scientifique ? \\[0pt]
Qu'est-ce qu'une machine ? \\[0pt]
Qu'est-ce qu'une machine ne peut pas faire ? \\[0pt]
Qu'est-ce qu'une marchandise ? \\[0pt]
Qu'est-ce qu'une mauvaise interprétation ? \\[0pt]
Qu'est-ce qu'une méditation ? \\[0pt]
Qu'est-ce qu'une méditation métaphysique ? \\[0pt]
Qu'est-ce qu'une mentalité collective ? \\[0pt]
Qu'est-ce qu'une méthode ? \\[0pt]
Qu'est-ce qu'une minorité ? \\[0pt]
Qu'est-ce qu'un empire ? \\[0pt]
Qu'est-ce qu'une nation ? \\[0pt]
Qu'est-ce qu'un enfant ? \\[0pt]
Qu'est-ce qu'un énoncé scientifique sur l'homme ? \\[0pt]
Qu'est-ce qu'une norme ? \\[0pt]
Qu'est-ce qu'une norme sociale ? \\[0pt]
Qu'est-ce qu'une œuvre ? \\[0pt]
Qu'est-ce qu'une œuvre classique ? \\[0pt]
Qu'est-ce qu'une œuvre d'art ? \\[0pt]
Qu'est-ce qu'une œuvre d'art authentique ? \\[0pt]
Qu'est-ce qu'une œuvre d'art réussie ? \\[0pt]
Qu'est-ce qu'une œuvre « géniale » ? \\[0pt]
Qu'est-ce qu'une œuvre ratée ? \\[0pt]
Qu'est-ce qu'une parole en l'air ? \\[0pt]
Qu'est-ce qu'une parole vivante ? \\[0pt]
Qu'est-ce qu'une patrie ? \\[0pt]
Qu'est-ce qu'une peinture abstraite ? \\[0pt]
Qu'est-ce qu'une « performance » ? \\[0pt]
Qu'est-ce qu'une période en histoire ? \\[0pt]
Qu'est-ce qu'une personne morale ? \\[0pt]
Qu'est-ce qu'une philosophie première ? \\[0pt]
Qu'est-ce qu'une phrase ? \\[0pt]
Qu'est-ce qu'une politique sociale ? \\[0pt]
Qu'est-ce qu'une preuve ? \\[0pt]
Qu'est-ce qu'une preuve scientifique ? \\[0pt]
Qu'est-ce qu'une promesse ? \\[0pt]
Qu'est-ce qu'une propriété ? \\[0pt]
Qu'est-ce qu'une propriété essentielle ? \\[0pt]
Qu'est-ce qu'une psychologie scientifique ? \\[0pt]
Qu'est-ce qu'une question dénuée de sens ? \\[0pt]
Qu'est-ce qu'une question métaphysique ? \\[0pt]
Qu'est-ce qu'une réfutation ? \\[0pt]
Qu'est-ce qu'une règle de grammaire ? \\[0pt]
Qu'est-ce qu'une règle de vie ? \\[0pt]
Qu'est-ce qu'une règle sociale ? \\[0pt]
Qu'est-ce qu'une religion ? \\[0pt]
Qu'est-ce qu'une représentation réussie ? \\[0pt]
Qu'est-ce qu'une révolution ? \\[0pt]
Qu'est-ce qu'une révolution artistique ? \\[0pt]
Qu'est-ce qu'une révolution politique ? \\[0pt]
Qu'est-ce qu'une révolution scientifique ? \\[0pt]
Qu'est-ce qu'une science exacte ? \\[0pt]
Qu'est-ce qu'une science rigoureuse ? \\[0pt]
Qu'est-ce qu'une secte ? \\[0pt]
Qu'est-ce qu'une situation ? \\[0pt]
Qu'est-ce qu'une situation tragique ? \\[0pt]
Qu'est-ce qu'une société ? \\[0pt]
Qu'est-ce qu'une société juste ? \\[0pt]
Qu'est-ce qu'une société mondialisée ? \\[0pt]
Qu'est-ce qu'une société ouverte ? \\[0pt]
Qu'est-ce qu'une société primitive ? \\[0pt]
Qu'est-ce qu'un esprit faux ? \\[0pt]
Qu'est-ce qu'un esprit libre ? \\[0pt]
Qu'est-ce qu'une structure ? \\[0pt]
Qu'est-ce qu'une substance ? \\[0pt]
Qu'est-ce qu'un état mental ? \\[0pt]
Qu'est-ce qu'un État souverain ? \\[0pt]
Qu'est-ce qu'une théorie ? \\[0pt]
Qu'est-ce qu'une théorie scientifique ? \\[0pt]
Qu'est-ce qu'une tradition ? \\[0pt]
Qu'est-ce qu'une tragédie historique ? \\[0pt]
Qu'est-ce qu'un être cultivé ? \\[0pt]
Qu'est-ce qu'un être imaginaire ? \\[0pt]
Qu'est-ce qu'une valeur ? \\[0pt]
Qu'est-ce qu'une valeur esthétique ? \\[0pt]
Qu'est-ce qu'un événement ? \\[0pt]
Qu'est-ce qu'un événement historique ? \\[0pt]
Qu'est-ce qu'une vérité scientifique ? \\[0pt]
Qu'est-ce qu'une vie d'artiste ? \\[0pt]
Qu'est-ce qu'une vie humaine ? \\[0pt]
Qu'est-ce qu'une vie réussie ? \\[0pt]
Qu'est-ce qu'une ville ? \\[0pt]
Qu'est-ce qu'une violence symbolique ? \\[0pt]
Qu'est-ce qu'une vision du monde ? \\[0pt]
Qu'est-ce qu'une vision scientifique du monde ? \\[0pt]
Qu'est-ce qu'une volonté libre ? \\[0pt]
Qu'est-ce qu'un exemple ? \\[0pt]
Qu'est-ce qu'un expert ? \\[0pt]
Qu'est-ce qu'un fait ? \\[0pt]
Qu'est-ce qu'un fait de société ? \\[0pt]
Qu'est-ce qu'un fait divers ? \\[0pt]
Qu'est-ce qu'un fait historique ? \\[0pt]
Qu'est-ce qu'un fait moral ? \\[0pt]
Qu'est-ce qu'un fait scientifique ? \\[0pt]
Qu'est-ce qu'un fait social ? \\[0pt]
Qu'est-ce qu'un faux ? \\[0pt]
Qu'est-ce qu'un faux problème ? \\[0pt]
Qu'est-ce qu'un faux sentiment ? \\[0pt]
Qu'est-ce qu'un film ? \\[0pt]
Qu'est-ce qu'un geste artistique ? \\[0pt]
Qu'est-ce qu'un geste technique ? \\[0pt]
Qu'est-ce qu'un gouvernement ? \\[0pt]
Qu'est-ce qu'un gouvernement juste ? \\[0pt]
Qu'est-ce qu'un gouvernement républicain ? \\[0pt]
Qu'est-ce qu'un grand philosophe ? \\[0pt]
Qu'est-ce qu'un héros ? \\[0pt]
Qu'est-ce qu'un homme bon ? \\[0pt]
Qu'est-ce qu'un homme juste ? \\[0pt]
Qu'est-ce qu'un homme politique ? \\[0pt]
Qu'est-ce qu'un homme seul ? \\[0pt]
Qu'est-ce qu'un idéal moral ? \\[0pt]
Qu'est-ce qu'un imaginaire social ? \\[0pt]
Qu'est-ce qu'un individu ? \\[0pt]
Qu'est-ce qu'un jeu ? \\[0pt]
Qu'est-ce qu'un juge ? \\[0pt]
Qu'est-ce qu'un jugement analytique ? \\[0pt]
Qu'est-ce qu'un laboratoire ? \\[0pt]
Qu'est-ce qu'un langage technique ? \\[0pt]
Qu'est-ce qu'un législateur ? \\[0pt]
Qu'est-ce qu'un lien logique ? \\[0pt]
Qu'est-ce qu'un lieu commun ? \\[0pt]
Qu'est-ce qu'un livre ? \\[0pt]
Qu'est-ce qu'un maître ? \\[0pt]
Qu'est-ce qu'un marginal ? \\[0pt]
Qu'est-ce qu'un mécanisme social ? \\[0pt]
Qu'est-ce qu'un métaphysicien ? \\[0pt]
Qu'est-ce qu'un métier ? \\[0pt]
Qu'est-ce qu'un mineur ? \\[0pt]
Qu'est-ce qu'un modèle ? \\[0pt]
Qu'est-ce qu'un moderne ? \\[0pt]
Qu'est-ce qu'un monde ? \\[0pt]
Qu'est-ce qu'un monde commun ? \\[0pt]
Qu'est-ce qu'un monstre ? \\[0pt]
Qu'est-ce qu'un monument ? \\[0pt]
Qu'est-ce qu'un mouvement politique \\[0pt]
Qu'est-ce qu'un mouvement politique ? \\[0pt]
Qu'est-ce qu'un mythe ? \\[0pt]
Qu'est-ce qu'un nombre ? \\[0pt]
Qu'est-ce qu'un nom propre ? \\[0pt]
Qu'est-ce qu'un objet ? \\[0pt]
Qu'est-ce qu'un objet d'art ? \\[0pt]
Qu'est-ce qu'un objet de connaissance ? \\[0pt]
Qu'est-ce qu'un objet esthétique ? \\[0pt]
Qu'est-ce qu'un objet mathématique ? \\[0pt]
Qu'est-ce qu'un objet métaphysique ? \\[0pt]
Qu'est-ce qu'un organisme ? \\[0pt]
Qu'est-ce qu'un original ? \\[0pt]
Qu'est-ce qu'un outil ? \\[0pt]
Qu'est-ce qu'un paradoxe ? \\[0pt]
Qu'est-ce qu'un patrimoine ? \\[0pt]
Qu'est-ce qu'un paysage ? \\[0pt]
Qu'est-ce qu'un pédant ? \\[0pt]
Qu'est-ce qu'un peuple \\[0pt]
Qu'est-ce qu'un peuple ? \\[0pt]
Qu'est-ce qu'un peuple libre ? \\[0pt]
Qu'est-ce qu'un phénomène ? \\[0pt]
Qu'est-ce qu'un plaisir pur ? \\[0pt]
Qu'est-ce qu'un point de vue ? \\[0pt]
Qu'est-ce qu'un postulat ? \\[0pt]
Qu'est-ce qu'un pouvoir despotique ? \\[0pt]
Qu'est-ce qu'un pouvoir légitime ? \\[0pt]
Qu'est-ce qu'un primitif ? \\[0pt]
Qu'est-ce qu'un prince juste ? \\[0pt]
Qu'est-ce qu'un principe ? \\[0pt]
Qu'est-ce qu'un problème ? \\[0pt]
Qu'est-ce qu'un problème éthique ? \\[0pt]
Qu'est-ce qu'un problème métaphysique ? \\[0pt]
Qu'est-ce qu'un problème philosophique ? \\[0pt]
Qu'est-ce qu'un problème politique ? \\[0pt]
Qu'est-ce qu'un problème scientifique ? \\[0pt]
Qu'est-ce qu'un produit culturel ? \\[0pt]
Qu'est-ce qu'un programme politique ? \\[0pt]
Qu'est-ce qu'un projet ? \\[0pt]
Qu'est-ce qu'un rapport de force ? \\[0pt]
Qu'est-ce qu'un réfugié politique ? \\[0pt]
Qu'est-ce qu'un rite ? \\[0pt]
Qu'est-ce qu'un rival ? \\[0pt]
Qu'est-ce qu'un sage ? \\[0pt]
Qu'est-ce qu'un sentiment moral ? \\[0pt]
Qu'est-ce qu'un sentiment vrai ? \\[0pt]
Qu'est-ce qu'un signe ? \\[0pt]
Qu'est-ce qu'un sophisme ? \\[0pt]
Qu'est-ce qu'un sophiste ? \\[0pt]
Qu'est-ce qu'un souvenir ? \\[0pt]
Qu'est-ce qu'un spécialiste ? \\[0pt]
Qu'est-ce qu'un spectacle ? \\[0pt]
Qu'est-ce qu'un spectateur ? \\[0pt]
Qu'est-ce qu'un style ? \\[0pt]
Qu'est-ce qu'un sujet de droit ? \\[0pt]
Qu'est-ce qu'un symbole ? \\[0pt]
Qu'est-ce qu'un symptôme ? \\[0pt]
Qu'est-ce qu'un système ? \\[0pt]
Qu'est-ce qu'un système philosophique ? \\[0pt]
Qu'est-ce qu'un tableau \\[0pt]
Qu'est-ce qu'un tableau ? \\[0pt]
Qu'est-ce qu'un témoin ? \\[0pt]
Qu'est-ce qu'un territoire ? \\[0pt]
Qu'est-ce qu'un texte ? \\[0pt]
Qu'est-ce qu'un tout ? \\[0pt]
Qu'est-ce qu'un trouble social ? \\[0pt]
Qu'est-ce qu'un visage ? \\[0pt]
Qu'est-ce qu'un vrai changement ? \\[0pt]
Qu'est-il légitime d'interdire ? \\[0pt]
Qu'est-il utile d'interdire ? \\[0pt]
Question et problème \\[0pt]
Qu'est qu'un régime politique ? \\[0pt]
Que suppose le mouvement ? \\[0pt]
Que transmettons-nous ? \\[0pt]
Que trouve-t-on dans ce que l'on trouve beau ? \\[0pt]
Que valent les classifications en sciences humaines ? \\[0pt]
Que valent les décisions de la majorité ? \\[0pt]
Que valent les excuses ? \\[0pt]
Que valent les idées générales ? \\[0pt]
Que valent les preuves ? \\[0pt]
Que vaut en morale la justification par l'utilité ? \\[0pt]
Que vaut la distinction entre nature et culture ? \\[0pt]
Que vaut la fidélité ? \\[0pt]
Que vaut le travail ? \\[0pt]
Que vaut l'excuse : « C'est plus fort que moi » ? \\[0pt]
Que vaut l'incertain ? \\[0pt]
Que vaut un argument d'autorité ? \\[0pt]
Que vaut un consensus ? \\[0pt]
Que vaut une connaissance empirique de l'homme ? \\[0pt]
Que vaut une preuve contre un préjugé ? \\[0pt]
Que veut dire avoir raison ? \\[0pt]
Que veut dire : « être cultivé » ? \\[0pt]
Que veut dire introduire à la métaphysique ? \\[0pt]
Que veut dire l'expression « aller au fond des choses » ? \\[0pt]
Que veut-on dire quand on dit que « rien n'est sans raison » ? \\[0pt]
Que veut-on dire quand on dit « rien n'est sans raison » ? \\[0pt]
Que voit-on dans une image ? \\[0pt]
Que voit-on dans un miroir ? \\[0pt]
Que voulons-nous ? \\[0pt]
Que voyons-nous sur un tableau ? \\[0pt]
Qu'expriment les œuvres d'art ? \\[0pt]
Qu'exprime une œuvre d'art ? \\[0pt]
Qui a des droits ? \\[0pt]
Qui agit ? \\[0pt]
Qui a le droit de juger ? \\[0pt]
Qui a une histoire ? \\[0pt]
Qui a une parole politique ? \\[0pt]
Qui commande ? \\[0pt]
Qui connaît le mieux mon corps ? \\[0pt]
Qui croire ? \\[0pt]
Qui détient le pouvoir ? \\[0pt]
Qui doit dire le droit ? \\[0pt]
Qui doit éduquer ? \\[0pt]
Qui doit faire les lois ? \\[0pt]
Qui doit gouverner ? \\[0pt]
Qui domine ? \\[0pt]
Qui donne la norme du goût ? \\[0pt]
Qui écrit l'histoire ? \\[0pt]
Qui est autorisé à me dire « tu dois » ? \\[0pt]
Qui est compétent en matière politique ? \\[0pt]
Qui est crédible ? \\[0pt]
Qui est cultivé ? \\[0pt]
Qui est le maître ? \\[0pt]
Qui est l'homme des sciences humaines ? \\[0pt]
Qui est libre ? \\[0pt]
Qui est méchant ? \\[0pt]
Qui est métaphysicien ? \\[0pt]
Qui est mon prochain ? \\[0pt]
Qui est responsable ? \\[0pt]
Qui est sage ? \\[0pt]
Qui est souverain ? \\[0pt]
Qui est une personne ? \\[0pt]
Qui fait la loi ? \\[0pt]
Qui fait l'unité du peuple ? \\[0pt]
Qui faut-il protéger ? \\[0pt]
Qui gouverne ? \\[0pt]
Qui mérite d'être aimé ? \\[0pt]
Qui meurt ? \\[0pt]
« Qui ne dit mot consent » \\[0pt]
Qui parle ? \\[0pt]
Qui pense ? \\[0pt]
Qui peut obliger ? \\[0pt]
Qui peut parler ? \\[0pt]
Qui pose les fins ? \\[0pt]
Qui sont mes amis ? \\[0pt]
Qui sont mes semblables ? \\[0pt]
Qui sont nos ennemis ? \\[0pt]
Qui suis-je ? \\[0pt]
Qui suis-je pour me juger ? \\[0pt]
Qui travaille ? \\[0pt]
Qu'oppose-t-on à la vérité ? \\[0pt]
Qu'y a-t-il à comprendre dans une œuvre d'art ? \\[0pt]
Qu'y a-t-il à comprendre en histoire ? \\[0pt]
Qu'y a-t-il à l'origine de toutes choses ? \\[0pt]
Qu'y a-t-il au-delà de l'être ? \\[0pt]
Qu'y a-t-il au-delà du réel ? \\[0pt]
Qu'y a-t-il au-dessus des lois ? \\[0pt]
Qu'y a-t-il au fondement de l'objectivité ? \\[0pt]
Qu'y a-t-il de technique dans l'art ? \\[0pt]
Qu'y a-t-il d'universel dans la culture ? \\[0pt]
Race, classe, nation \\[0pt]
Raconter sa vie \\[0pt]
Raconter son histoire \\[0pt]
Raison et déraison \\[0pt]
Raison et langage \\[0pt]
Raison et politique \\[0pt]
Raison et révélation \\[0pt]
Raisonner et argumenter \\[0pt]
Raisonner et calculer \\[0pt]
Raisonner et décider \\[0pt]
Raisonner par l'absurde \\[0pt]
Rapports de force, rapport de pouvoir \\[0pt]
Rapports de la science et de la politique \\[0pt]
Rassembler les hommes, est-ce les unir ? \\[0pt]
Réalisme et idéalisme \\[0pt]
Réalité et idéal \\[0pt]
Rebuts et objets quelconques : une matière pour l'art ? \\[0pt]
Recevoir \\[0pt]
Recherche et militantisme \\[0pt]
Récit et explication en histoire \\[0pt]
Récit et fiction \\[0pt]
Récit et mémoire \\[0pt]
Reconnaissons-nous le bien comme nous reconnaissons le vrai ? \\[0pt]
Réécrire l'histoire \\[0pt]
Refaire le monde \\[0pt]
Réforme et révolution \\[0pt]
Réfutation et confirmation \\[0pt]
Réfuter \\[0pt]
Réfuter une théorie \\[0pt]
Regarder \\[0pt]
Regarder une œuvre d'art \\[0pt]
Regarder un tableau \\[0pt]
Règle et commandement \\[0pt]
Règles logiques et lois psychologiques \\[0pt]
Régularité et singularité \\[0pt]
Relativisme et sciences humaines \\[0pt]
Religion et liberté \\[0pt]
Religion et métaphysique \\[0pt]
Religion et superstition \\[0pt]
Religion naturelle et religion révélée \\[0pt]
Rendre justice \\[0pt]
Rendre la justice \\[0pt]
Rendre raison \\[0pt]
Rendre visible l'invisible \\[0pt]
Renoncer à ses préjugés \\[0pt]
Renoncer au passé \\[0pt]
Rentrer en soi-même \\[0pt]
Réparer \\[0pt]
Répondre \\[0pt]
Représentation et illusion \\[0pt]
Reproduire \\[0pt]
Reproduire, copier, imiter \\[0pt]
Réprouver \\[0pt]
République et démocratie \\[0pt]
Résistance et soumission \\[0pt]
Résister \\[0pt]
Résister à l'oppression \\[0pt]
Résister peut-il être un droit ? \\[0pt]
Respect de la volonté, respect de la vie \\[0pt]
Respecter la nature, est-ce renoncer à l'exploiter ? \\[0pt]
Responsabilité et culpabilité \\[0pt]
Ressemblance et identité \\[0pt]
Ressemblance et imitation \\[0pt]
Ressembler \\[0pt]
Ressentir \\[0pt]
Ressent-on ou apprécie-t-on l'art ? \\[0pt]
Rester soi-même \\[0pt]
Rêve et réalité \\[0pt]
Révéler \\[0pt]
Revenir à la nature \\[0pt]
Rêver \\[0pt]
Rêver éloigne-t-il de la réalité ? \\[0pt]
Revient-il à l'État d'assurer votre bonheur ? \\[0pt]
Révolte et révolution \\[0pt]
Rhétorique et vérité \\[0pt]
Richesse et pauvreté \\[0pt]
Rien \\[0pt]
Rien de nouveau sous le soleil \\[0pt]
« Rien n'est sans raison » \\[0pt]
Rien n'est sans raison \\[0pt]
« Rien n'est simple » \\[0pt]
Rire \\[0pt]
Rire et pleurer \\[0pt]
Rites et cérémonies \\[0pt]
Roman et vérité \\[0pt]
Rythmes sociaux, rythmes naturels \\[0pt]
S'adapter \\[0pt]
Sagesse et renoncement \\[0pt]
Sait-on toujours ce que l'on fait ? \\[0pt]
Sait-on toujours ce qu'on veut ? \\[0pt]
S'aliéner \\[0pt]
Sans foi ni loi \\[0pt]
Sans les mots, que seraient les choses ? \\[0pt]
« Sans titre » \\[0pt]
S'approprier une œuvre d'art \\[0pt]
S'arracher à la nature \\[0pt]
S'associer \\[0pt]
Sauver les apparences \\[0pt]
« Sauver les phénomènes » \\[0pt]
Sauver les phénomènes \\[0pt]
Savoir ce qu'on dit \\[0pt]
Savoir ce qu'on fait \\[0pt]
Savoir ce qu'on veut \\[0pt]
Savoir démontrer \\[0pt]
Savoir de quoi on parle \\[0pt]
Savoir, est-ce déjouer le réel ? \\[0pt]
Savoir, est-ce pouvoir ? \\[0pt]
Savoir et faire \\[0pt]
Savoir et liberté \\[0pt]
Savoir et objectivité dans les sciences \\[0pt]
Savoir et pouvoir \\[0pt]
Savoir et rectification \\[0pt]
Savoir être heureux \\[0pt]
Savoir et vérifier \\[0pt]
Savoir faire \\[0pt]
Savoir pour prévoir \\[0pt]
Savoir, pouvoir \\[0pt]
Savoir quoi faire \\[0pt]
Savoir s'arrêter \\[0pt]
Savoir se décider \\[0pt]
Savoir vivre \\[0pt]
Savons-nous ce que nous disons ? \\[0pt]
Savons-nous ce que peut un corps ? \\[0pt]
Savons-nous ce qui nous rend heureux ? \\[0pt]
Savons-nous de ce que nous disons ? \\[0pt]
Savons-nous nous souvenir ? \\[0pt]
Scepticisme et relativisme \\[0pt]
Science et complexité \\[0pt]
Science et conscience \\[0pt]
Science et démocratie \\[0pt]
Science et expérience \\[0pt]
Science et histoire \\[0pt]
Science et hypothèse \\[0pt]
Science et idéologie \\[0pt]
Science et imagination \\[0pt]
Science et magie \\[0pt]
Science et mythe \\[0pt]
Science et opinion \\[0pt]
Science et persuasion \\[0pt]
Science et philosophie \\[0pt]
Science et pouvoir \\[0pt]
Science et réalité \\[0pt]
Science et religion \\[0pt]
Science et sagesse \\[0pt]
Science et société \\[0pt]
Science et subjectivité \\[0pt]
Science et technique \\[0pt]
Science et technologie \\[0pt]
Science pure et science appliquée \\[0pt]
Science sans conscience n'est-elle que ruine de l'âme ? \\[0pt]
Sciences de la nature et sciences de l'esprit \\[0pt]
Sciences de la nature et sciences humaines \\[0pt]
Sciences et philosophie \\[0pt]
Sciences humaines et déterminisme \\[0pt]
Sciences humaines et herméneutique \\[0pt]
Sciences humaines et idéologie \\[0pt]
Sciences humaines et liberté sont-elles compatibles ? \\[0pt]
Sciences humaines et littérature \\[0pt]
Sciences humaines et naturalisme \\[0pt]
Sciences humaines et nature humaine \\[0pt]
Sciences humaines et objectivité \\[0pt]
Sciences humaines et philosophie \\[0pt]
Sciences humaines et vérité \\[0pt]
Sciences humaines ou sciences sociales ? \\[0pt]
Sciences humaines, sciences de l'homme \\[0pt]
Sciences sociales et humanités \\[0pt]
Sciences sociales et questions environnementales \\[0pt]
Sculpter \\[0pt]
Se connaître soi-même \\[0pt]
Se conserver \\[0pt]
Sécularisation et laïcisation \\[0pt]
Se cultiver \\[0pt]
Sécurité et liberté \\[0pt]
Se défendre \\[0pt]
Se détacher des sens \\[0pt]
Se distinguer \\[0pt]
Se divertir \\[0pt]
Se donner corps et âme \\[0pt]
Séduire \\[0pt]
Se faire justice \\[0pt]
S'émanciper de ses maîtres \\[0pt]
Se méfier des idées \\[0pt]
Se mentir à soi-même \\[0pt]
Se mettre à la place d'autrui \\[0pt]
S'ennuyer \\[0pt]
Sensation et perception \\[0pt]
Sens et fait \\[0pt]
Sens et limites de la notion de capital culturel \\[0pt]
Sens et sensibilité \\[0pt]
Sens et sensible \\[0pt]
Sens et structure \\[0pt]
Sens et vérité \\[0pt]
Sensible et intelligible \\[0pt]
S'entendre \\[0pt]
Sentiment et émotion \\[0pt]
Sentir \\[0pt]
Se parler et s'entendre \\[0pt]
Se passer de philosophie \\[0pt]
Se peindre \\[0pt]
Se prendre au sérieux \\[0pt]
Se raisonner \\[0pt]
Se réfugier dans la croyance ? \\[0pt]
Se retirer dans la pensée ? \\[0pt]
Se retirer du monde \\[0pt]
Se révolter \\[0pt]
Servir \\[0pt]
Servir l'État \\[0pt]
Se savoir mortel \\[0pt]
Se sentir libre implique-t-il qu'on le soit ? \\[0pt]
Se souvenir \\[0pt]
Se souvenir, est-ce préparer l'avenir ? \\[0pt]
Se taire \\[0pt]
S'étonner \\[0pt]
Seul le présent existe-t-il ? \\[0pt]
Se voiler la face \\[0pt]
Sexe et genre \\[0pt]
S'exercer \\[0pt]
S'exprimer \\[0pt]
Sexualité et nature \\[0pt]
Si Dieu n'existe pas, tout est-il permis ? \\[0pt]
Signe et symbole \\[0pt]
Signes naturels, signes artificiels \\[0pt]
Signes, traces et indices \\[0pt]
Signification et expression \\[0pt]
Signification et vérité \\[0pt]
Si l'esprit n'est pas une table rase, qu'est-il ? \\[0pt]
Simple / composé \\[0pt]
Sincérité et authenticité \\[0pt]
S'indigner \\[0pt]
S'indigner, est-ce un devoir ? \\[0pt]
S'intéresser à l'art \\[0pt]
« Si tu veux la paix, prépare la guerre » \\[0pt]
Si tu veux, tu peux \\[0pt]
Société et communauté \\[0pt]
Société et liberté \\[0pt]
Société et mouvement \\[0pt]
Société et organisme \\[0pt]
Société et réciprocité \\[0pt]
Sociologie et économie \\[0pt]
Sociologie et ethnologie \\[0pt]
Soi \\[0pt]
Soigner \\[0pt]
Solitude et liberté \\[0pt]
Sommes-nous capables d'agir de manière désintéressée ? \\[0pt]
Sommes-nous conscients de nos mobiles ? \\[0pt]
Sommes-nous des êtres métaphysiques ? \\[0pt]
Sommes-nous faits pour la vérité ? \\[0pt]
Sommes-nous faits pour vivre en société ? \\[0pt]
Sommes-nous les jouets de l'histoire ? \\[0pt]
Sommes-nous libres de nos croyances ? \\[0pt]
Sommes-nous libres de nos pensées ? \\[0pt]
Sommes-nous libres de nos préférences morales ? \\[0pt]
Sommes-nous libres par nature ? \\[0pt]
Sommes-nous maîtres de la nature ? \\[0pt]
Sommes-nous maîtres de nos paroles ? \\[0pt]
Sommes-nous responsables de ce que nous sommes ? \\[0pt]
Sommes-nous responsables de nos erreurs ? \\[0pt]
Sommes-nous responsables de notre être social ? \\[0pt]
Sommes-nous responsables du sens que prennent nos paroles ? \\[0pt]
Sommes-nous toujours dépendants d'autrui ? \\[0pt]
Sommes-nous tous artistes ? \\[0pt]
Sommes-nous tous contemporains ? \\[0pt]
Sophisme et paralogisme \\[0pt]
Sophismes et paradoxes \\[0pt]
S'orienter \\[0pt]
S'orienter dans la pensée \\[0pt]
S'orienter dans le monde \\[0pt]
Sortir de soi \\[0pt]
Souffrir \\[0pt]
Soupçonner \\[0pt]
Soutenir une thèse \\[0pt]
Souveraineté et liberté \\[0pt]
Sphère privée et sphère publique \\[0pt]
Spontanéité et liberté \\[0pt]
Structure et événement \\[0pt]
Structure et histoire \\[0pt]
Subir \\[0pt]
Subjectivité et vérité \\[0pt]
Substance et sujet \\[0pt]
Suffit-il à Dieu d'être possible pour exister ? \\[0pt]
Suffit-il de bien juger pour bien faire ? \\[0pt]
Suffit-il de faire son devoir ? \\[0pt]
Suffit-il de suivre ses convictions pour bien agir ? \\[0pt]
Suffit-il d'être conscient de ses actes pour en être responsable ? \\[0pt]
Suffit-il d'être dans le vrai pour savoir ? \\[0pt]
Suffit-il d'être juste ? \\[0pt]
Suffit-il d'être raisonnable pour être moral ? \\[0pt]
Suffit-il de vouloir pour bien faire ? \\[0pt]
Suffit-il, pour croire, de le vouloir ? \\[0pt]
Suffit-il, pour être juste, d'obéir aux lois et aux coutumes de son pays ? \\[0pt]
Suicide et homicide \\[0pt]
Suis-je au centre de l'espace ? \\[0pt]
Suis-je aussi ce que j'aurais pu être ? \\[0pt]
Suis-je l'auteur de mes actions ? \\[0pt]
Suis-je le sujet de mes pensées ? \\[0pt]
Suis-je maître de ma conscience ? \\[0pt]
Suis-je maître de mes pensées ? \\[0pt]
Suis-je ma mémoire ? \\[0pt]
Suis-je mon corps ? \\[0pt]
Suivre la nature \\[0pt]
Suivre la règle \\[0pt]
Suivre le plus sûr \\[0pt]
Suivre sa nature \\[0pt]
Sujet et citoyen \\[0pt]
Sujet et prédicat \\[0pt]
Sujet et substance \\[0pt]
Sur quels fondements distinguer ce qui est public et ce qui est privé ? \\[0pt]
Sur quoi fonder la légitimité de la loi ? \\[0pt]
Sur quoi fonder la propriété ? \\[0pt]
Sur quoi fonder la société ? \\[0pt]
Sur quoi fonder l'autorité ? \\[0pt]
Sur quoi fonder le droit de punir ? \\[0pt]
Sur quoi reposent nos certitudes ? \\[0pt]
Sur quoi se fonde la connaissance scientifique ? \\[0pt]
Sur quoi se fonde la vérité des énoncés ? \\[0pt]
Surveillance et discipline \\[0pt]
Surveiller son comportement \\[0pt]
Survivre \\[0pt]
Suspendre son assentiment \\[0pt]
Suspendre son jugement \\[0pt]
Syllogisme et démonstration \\[0pt]
Système et liberté \\[0pt]
Système et structure \\[0pt]
Tantôt je pense, tantôt je suis \\[0pt]
Tautologie et contradiction \\[0pt]
Technique et démocratie \\[0pt]
Technique et esthétique \\[0pt]
Technique et pratiques scientifiques \\[0pt]
Technique et technologie \\[0pt]
Témoigner \\[0pt]
Temps et conscience \\[0pt]
Temps et durée \\[0pt]
Temps et éternité \\[0pt]
Temps et musique \\[0pt]
Temps et réalité \\[0pt]
Temps individuel et temps collectif \\[0pt]
Tenir parole \\[0pt]
Terres et territoires \\[0pt]
Territoire et peuplement \\[0pt]
Tester une hypothèse permet-il de la prouver ? \\[0pt]
Texte, son, image \\[0pt]
Thème et variations \\[0pt]
Théologie et morale \\[0pt]
Théorie et modélisation \\[0pt]
Théorie et pratique \\[0pt]
Théorie et pratique en médecine \\[0pt]
Tirer des leçons du passé \\[0pt]
Tolérer \\[0pt]
Tomber amoureux \\[0pt]
Toucher \\[0pt]
Tous les arts se valent-ils ? \\[0pt]
Tous les droits sont-ils formels ? \\[0pt]
Tous les hommes désirent-ils connaître ? \\[0pt]
Tous les hommes désirent-ils être heureux ? \\[0pt]
Tout art est-il abstrait ? \\[0pt]
Tout art est-il décoratif ? \\[0pt]
Tout art est-il poésie ? \\[0pt]
Tout art est-il symbolique ? \\[0pt]
Tout a-t-il une fin ? \\[0pt]
Tout a-t-il une raison d'être ? \\[0pt]
Tout a-t-il un sens ? \\[0pt]
Tout bien n'est-il qu'un moindre mal ? \\[0pt]
Tout ce qui est humain est-il signifiant ? \\[0pt]
Tout ce qui est humain nous est-il familier ? \\[0pt]
Tout ce qui existe est-il matériel ? \\[0pt]
Tout définir, tout démontrer \\[0pt]
Tout désir est-il nostalgique ? \\[0pt]
Tout devoir est-il l'envers d'un droit ? \\[0pt]
Toute action politique est-elle collective ? \\[0pt]
Toute chose a-t-elle une essence ? \\[0pt]
Toute chose a-t-elle une fin ? \\[0pt]
Toute communauté est-elle politique ? \\[0pt]
Toute connaissance autre que scientifique doit-elle être considérée comme une illusion ? \\[0pt]
Toute connaissance est-elle historique ? \\[0pt]
Toute connaissance est-elle relative ? \\[0pt]
Toute conscience n'est-elle pas implicitement morale ? \\[0pt]
Toute conversation est-elle futile ? \\[0pt]
Toute définition est-elle une convention ? \\[0pt]
Toute description est-elle une interprétation ? \\[0pt]
Toute existence est-elle indémontrable ? \\[0pt]
Toute expérience appelle-t-elle une interprétation ? \\[0pt]
Toute expression est-elle métaphorique ? \\[0pt]
Toute hiérarchie est-elle inégalitaire ? \\[0pt]
Toute inégalité est-elle injuste ? \\[0pt]
Toute loi est-elle arbitraire ? \\[0pt]
Toute métaphysique implique-t-elle une transcendance ? \\[0pt]
Toute morale est-elle rationnelle ? \\[0pt]
Toute morale implique-t-elle l'effort ? \\[0pt]
Tout énoncé est-il nécessairement vrai ou faux ? \\[0pt]
Toute origine est-elle mythique ? \\[0pt]
Toute passion fait-elle souffrir ? \\[0pt]
« Toute peine mérite salaire » \\[0pt]
Toute peur est-elle irrationnelle ? \\[0pt]
Toute philosophie est-elle systématique ? \\[0pt]
Toute philosophie implique-t-elle une politique ? \\[0pt]
Toute physique exige-t-elle une métaphysique ? \\[0pt]
Toute psychologie est-elle normalisatrice ? \\[0pt]
Toutes les choses sont-elles singulières ? \\[0pt]
Toutes les croyances se valent-elles ? \\[0pt]
Toutes les fautes se valent-elles ? \\[0pt]
Toutes les vérités scientifiques sont-elles révisables ? \\[0pt]
Tout est corps \\[0pt]
Tout est-il à vendre ? \\[0pt]
Tout est-il connaissable ? \\[0pt]
Tout est-il contingent ? \\[0pt]
Tout est-il digne de mémoire ? \\[0pt]
Tout est-il mesurable ? \\[0pt]
Tout est-il nécessaire ? \\[0pt]
Tout est-il politique ? \\[0pt]
Tout est-il relatif ? \\[0pt]
Tout est permis \\[0pt]
« Tout est politique » \\[0pt]
Tout être est-il dans l'espace ? \\[0pt]
Tout événement a-t-il une cause ? \\[0pt]
Toute vérité est-elle démontrable ? \\[0pt]
Toute vérité est-elle un rapport ? \\[0pt]
Toute vérité est-elle vérifiable ? \\[0pt]
Toute violence est-elle contre nature ? \\[0pt]
Tout fondement de la connaissance est-il métaphysique ? \\[0pt]
Tout malheur est-il une injustice ? \\[0pt]
Tout ou rien \\[0pt]
Tout peut-il être expliqué historiquement ? \\[0pt]
Tout peut-il être objet de jugement esthétique ? \\[0pt]
Tout peut-il n'être qu'apparence ? \\[0pt]
Tout peut-il se quantifier ? \\[0pt]
Tout peut-il se vendre ? \\[0pt]
Tout pouvoir est-il oppresseur ? \\[0pt]
Tout pouvoir est-il politique ? \\[0pt]
Tout pouvoir est-il usurpé ? \\[0pt]
Tout pouvoir n'est-il pas abusif ? \\[0pt]
Tout pouvoir relève-t-il de la domination ? \\[0pt]
Tout savoir \\[0pt]
Tout savoir est-il fondé sur un savoir premier ? \\[0pt]
Tout savoir est-il pouvoir ? \\[0pt]
Tout savoir est-il transmissible ? \\[0pt]
Tout travail mérite salaire \\[0pt]
Tradition et innovation \\[0pt]
Tradition et raison \\[0pt]
Traduire \\[0pt]
Tragédie et comédie \\[0pt]
Trahir \\[0pt]
Traiter autrui comme une chose \\[0pt]
Traiter des faits humains comme des choses, est-ce considérer l'homme comme une chose ? \\[0pt]
Traiter les faits humains comme des choses, est-ce réduire les hommes à des choses ? \\[0pt]
Transcendance et altérité \\[0pt]
Transmettre \\[0pt]
Travail et plaisir \\[0pt]
Travail et subjectivité \\[0pt]
Travailler par plaisir, est-ce encore travailler ? \\[0pt]
Travailler sans souffrir peut-il être un idéal ? \\[0pt]
Travail manuel et travail intellectuel \\[0pt]
Travail manuel, travail intellectuel \\[0pt]
Tricher \\[0pt]
« Trop beau pour être vrai » \\[0pt]
Tuer le temps \\[0pt]
« Tu ne tueras point » \\[0pt]
Un acte désintéressé est-il possible ? \\[0pt]
Un acte inconscient est-il nécessairement un acte involontaire ? \\[0pt]
Un acte libre est-il un acte imprévisible ? \\[0pt]
Un artiste doit-il être génial ? \\[0pt]
Un art peut-il se passer de règles ? \\[0pt]
Un art sans sublimation est-il possible ? \\[0pt]
Un bien peut-il sortir d'un mal ? \\[0pt]
Un contrat peut-il être social ? \\[0pt]
Un corps n'est-il qu'un objet ? \\[0pt]
Un Dieu unique ? \\[0pt]
Une action se juge-t-elle à ses conséquences ? \\[0pt]
Une action vertueuse se reconnaît-elle à sa difficulté ? \\[0pt]
Une bonne cité peut-elle se passer du beau ? \\[0pt]
Une cause peut-elle être libre ? \\[0pt]
Une croyance infondée est-elle illégitime ? \\[0pt]
Une culture constitue-t-elle un paradigme ? \\[0pt]
Une culture de masse est-elle une culture ? \\[0pt]
Une décision politique peut-elle être juste ? \\[0pt]
Une d'œuvre peut-elle être achevée ? \\[0pt]
Une éducation des sentiments est-elle possible ? \\[0pt]
Une éducation esthétique est-elle possible ? \\[0pt]
Une éthique sceptique est-elle possible ? \\[0pt]
Une existence se démontre-t-elle ? \\[0pt]
Une explication peut-elle être réductrice ? \\[0pt]
Une femme politique est-elle un homme politique comme les autres ? \\[0pt]
Une fiction peut-elle être vraie ? \\[0pt]
Une foi rationnelle \\[0pt]
Une guerre peut-elle être juste ? \\[0pt]
Une idée peut-elle être fausse ? \\[0pt]
Une imitation peut-elle être parfaite ? \\[0pt]
Une inégalité peut-elle être juste ? \\[0pt]
Une injustice vaut elle mieux qu'un désordre ? \\[0pt]
Une interprétation est-elle nécessairement subjective ? \\[0pt]
Une interprétation peut-elle être vraie ? \\[0pt]
Une justice sans égalité est-elle possible ? \\[0pt]
Une ligne de conduite peut-elle tenir lieu de morale ? \\[0pt]
Une logique non-formelle est-elle possible ? \\[0pt]
Une loi n'est-elle qu'une règle ? \\[0pt]
Une machine peut-elle être naturelle ? \\[0pt]
Une machine peut-elle penser ? \\[0pt]
Une machine pourrait-elle penser ? \\[0pt]
Une machine se réduit-elle à un mécanisme ? \\[0pt]
« Une main de fer dans un gant de velours » \\[0pt]
Une métaphysique athée est-elle possible ? \\[0pt]
Une métaphysique n'est-elle qu'une ontologie ? \\[0pt]
Une métaphysique peut-elle être sceptique ? \\[0pt]
Une morale du plaisir est-elle concevable ? \\[0pt]
Une morale personnelle est-elle une contradiction dans les termes ? \\[0pt]
Une morale peut-elle être dépassée ? \\[0pt]
Une morale peut-elle être provisoire ? \\[0pt]
Une morale peut-elle prétendre à l'universalité ? \\[0pt]
Une morale peut-elle se passer de la notion de vertu ? \\[0pt]
Une morale qui ne vise pas l'utilité est-elle rationnelle ? \\[0pt]
Une morale sans Dieu \\[0pt]
Une morale sans obligation est-elle possible ? \\[0pt]
Un enfant est-il un citoyen ? \\[0pt]
Une œuvre d'art a-t-elle un prix ? \\[0pt]
Une œuvre d'art doit-elle avoir un sens ? \\[0pt]
Une œuvre d'art est-elle immortelle ? \\[0pt]
Une œuvre d'art est-elle toujours originale ? \\[0pt]
Une œuvre d'art est-elle une marchandise ? \\[0pt]
Une œuvre d'art peut-elle être exemplaire ? \\[0pt]
Une œuvre d'art peut-elle être immorale ? \\[0pt]
Une œuvre d'art peut-elle être laide ? \\[0pt]
Une œuvre d'art s'explique-t-elle à partir de ses influences ? \\[0pt]
Une œuvre est-elle toujours de son temps ? \\[0pt]
Une passion peut-elle résister au temps ? \\[0pt]
Une perception peut-elle être illusoire ? \\[0pt]
Une philosophie de l'amour est-elle possible ? \\[0pt]
Une politique de la nature est-elle concevable ? \\[0pt]
Une politique morale est-elle possible ? \\[0pt]
Une politique peut-elle se réclamer de la vie ? \\[0pt]
Une religion civile est-elle possible ? \\[0pt]
Une religion peut-elle être fausse ? \\[0pt]
Une religion peut-elle être rationnelle ? \\[0pt]
Une science bien faite est-elle une langue bien faite ? \\[0pt]
Une science de la conscience est-elle possible ? \\[0pt]
Une science de la culture est-elle possible ? \\[0pt]
Une science de la morale est-elle possible ? \\[0pt]
Une science de l'imaginaire est-elle possible ? \\[0pt]
Une science des symboles est-elle possible ? \\[0pt]
Une seconde nature \\[0pt]
Une société juste est-elle une société sans conflits ? \\[0pt]
Une société n'est-elle qu'un ensemble d'individus ? \\[0pt]
Une société sans conflit est-elle possible ? \\[0pt]
Une société sans État est-elle une société sans politique ? \\[0pt]
Une société sans religion est-elle possible ? \\[0pt]
Un État mondial ? \\[0pt]
Un État peut-il être trop étendu ? \\[0pt]
Un État peut-il être « voyou » ? \\[0pt]
Une théorie de la culture peut-elle être scientifique ? \\[0pt]
Une théorie sans expérience nous apprend-elle quelque chose ? \\[0pt]
Une théorie scientifique peut-elle devenir fausse ? \\[0pt]
Une théorie scientifique peut-elle être ramenée à des propositions empiriques élémentaires ? \\[0pt]
Une vérité est-elle discutable ? \\[0pt]
Une vérité sans preuve est-elle une vérité ? \\[0pt]
Une volonté peut-elle être générale ? \\[0pt]
Un fait scientifique doit-il être nécessairement démontré ? \\[0pt]
Un homme n'est-il que la somme de ses actes ? \\[0pt]
Unité et unicité \\[0pt]
Universalité et nécessité dans les sciences \\[0pt]
Univocité et équivocité \\[0pt]
Un jugement de goût est-il culturel ? \\[0pt]
Un jugement moral peut-il relever de la vérité ? \\[0pt]
Un langage peut-il être privé ? \\[0pt]
Un mensonge peut-il être légitime ? \\[0pt]
Un moment d'éternité \\[0pt]
Un monde sans beauté \\[0pt]
Un monde sans nature est-il pensable ? \\[0pt]
Un objet technique peut-il être beau ? \\[0pt]
Un organisme n'est-il qu'un objet ? \\[0pt]
Un peuple est-il responsable de son histoire ? \\[0pt]
Un peuple peut-il être sans histoire ? \\[0pt]
Un pouvoir a-t-il besoin d'être légitime ? \\[0pt]
Un préjugé n'a-t-il aucun fondement ? \\[0pt]
Un problème philosophique peut-il être périmé ? \\[0pt]
Un problème scientifique peut-il être insoluble ? \\[0pt]
Un savoir sur l'homme peut-il être séparé d'un pouvoir sur les hommes ? \\[0pt]
Un sceptique peut-il être logicien ? \\[0pt]
Un souverain élu peut-il être illégitime ? \\[0pt]
Un tableau peut-il être une dénonciation ? \\[0pt]
Un témoin de moralité est-il possible ? \\[0pt]
Un travail bien fait est-il nécessairement un bon travail ? \\[0pt]
Un vice, est-ce un manque ? \\[0pt]
Urbanisme et politique \\[0pt]
Utiliser les embryons humains ? \\[0pt]
Utopie et tradition \\[0pt]
Vainqueurs et vaincus \\[0pt]
Valeur artistique, valeur esthétique \\[0pt]
Valeur et évaluation \\[0pt]
Vanité des vanités \\[0pt]
Vendre son âme \\[0pt]
Vérité et certitude \\[0pt]
Vérité et cohérence \\[0pt]
Vérité et cohérence ? \\[0pt]
Vérité et convention \\[0pt]
Vérité et fiction \\[0pt]
Vérité et histoire \\[0pt]
Vérité et réalité \\[0pt]
Vérité et référence \\[0pt]
Vérité et sensibilité \\[0pt]
Vérité et signification \\[0pt]
Vérité et totalité \\[0pt]
Vérités de fait et vérités de raison \\[0pt]
Vérités mathématiques, vérités philosophiques \\[0pt]
Vertu et habitude \\[0pt]
Vertus morales, vertus intellectuelles \\[0pt]
Vices privés, vertus publiques \\[0pt]
Vie et destin \\[0pt]
Vie et existence \\[0pt]
Vie et matière \\[0pt]
Vie et matière : une distinction périmée ? \\[0pt]
Vie et mécanisme \\[0pt]
Vie et volonté \\[0pt]
Vie familiale, vie politique \\[0pt]
Vieillir \\[0pt]
Violence et discours \\[0pt]
Violence et histoire \\[0pt]
Violence et politique \\[0pt]
Vitalisme et mécanique \\[0pt]
Vit-on au présent ? \\[0pt]
Vivons-nous tous dans le même monde ? \\[0pt]
Vivrait-on mieux sans morale ? \\[0pt]
Vivre au jour le jour \\[0pt]
Vivre au présent \\[0pt]
Vivre caché \\[0pt]
Vivre comme si nous ne devions pas mourir \\[0pt]
Vivre dans son monde \\[0pt]
Vivre, est-ce interpréter ? \\[0pt]
Vivre, est-ce lutter contre la mort ? \\[0pt]
Vivre et bien vivre \\[0pt]
Vivre et exister \\[0pt]
Vivre intensément \\[0pt]
Vivre pleinement \\[0pt]
Vivre pour les autres \\[0pt]
Vivre sans mémoire, est-ce être libre ? \\[0pt]
Vivre sans morale \\[0pt]
Vivre sans religion, est-ce vivre sans espoir ? \\[0pt]
Vivre sa vie \\[0pt]
Vivre selon la nature \\[0pt]
Vivre sous la conduite de la raison \\[0pt]
Vivre vertueusement \\[0pt]
Voir \\[0pt]
Voir et concevoir \\[0pt]
Voir et entendre \\[0pt]
Voir et savoir \\[0pt]
Voir et toucher \\[0pt]
Voir la réalité en face \\[0pt]
Voir le bien, faire le mal \\[0pt]
Voir le meilleur et faire le pire \\[0pt]
Voir le meilleur, faire le pire \\[0pt]
Voir les choses comme elles sont \\[0pt]
Vouloir ce que l'on peut \\[0pt]
Vouloir et pouvoir \\[0pt]
Vouloir être heureux \\[0pt]
Vouloir la paix \\[0pt]
Vouloir le bien \\[0pt]
Vouloir l'égalité \\[0pt]
Vouloir le mal \\[0pt]
Voulons-nous la vérité ? \\[0pt]
Voyager \\[0pt]
Voyager dans le temps \\[0pt]
Voyager entre les mondes \\[0pt]
Vulgariser la science ? \\[0pt]
Y a-t-il autant de mondes que de cultures ? \\[0pt]
Y a-t-il continuité entre l'expérience et la science ? \\[0pt]
Y a-t-il continuité ou discontinuité entre la pensée mythique et la science ? \\[0pt]
Y a-t-il de bons préjugés ? \\[0pt]
Y a-t-il de fausses religions ? \\[0pt]
Y a-t-il de faux besoins ? \\[0pt]
Y a-t-il de la grandeur à être libre ? \\[0pt]
Y a-t-il de la raison dans la perception ? \\[0pt]
Y a-t-il de l'impensable ? \\[0pt]
Y a-t-il de l'incompréhensible ? \\[0pt]
Y a-t-il de l'inconcevable ? \\[0pt]
Y a-t-il de l'inconnaissable ? \\[0pt]
Y a-t-il de l'indémontrable ? \\[0pt]
Y a-t-il de l'indicible ? \\[0pt]
Y a-t-il de l'inexprimable ? \\[0pt]
Y a-t-il de l'intelligible dans l'art ? \\[0pt]
Y a-t-il de l'irréparable ? \\[0pt]
Y a-t-il de l'universel ? \\[0pt]
Y a-t-il de mauvaises raisons ? \\[0pt]
Y a-t-il des actes désintéressés ? \\[0pt]
Y a-t-il des actes intrinsèquement mauvais ? \\[0pt]
Y a-t-il des actes moralement indifférents ? \\[0pt]
Y a-t-il des actions collectives ? \\[0pt]
Y a-t-il des actions désintéressées ? \\[0pt]
Y a-t-il des aliénations heureuses ? \\[0pt]
Y a-t-il des arts du corps ? \\[0pt]
Y a-t-il des arts majeurs ? \\[0pt]
Y a-t-il des arts mineurs ? \\[0pt]
Y a-t-il des biens communs ? \\[0pt]
Y a-t-il des canons de la beauté ? \\[0pt]
Y a-t-il des certitudes historiques ? \\[0pt]
Y a-t-il des choses à savoir ? \\[0pt]
Y a-t-il des choses qui échappent au droit ? \\[0pt]
Y a-t-il des compétences politiques ? \\[0pt]
Y a-t-il des conflits de devoirs ? \\[0pt]
Y a-t-il des critères de la scientificité ? \\[0pt]
Y a-t-il des critères de l'humanité ? \\[0pt]
Y a-t-il des critères du beau ? \\[0pt]
Y a-t-il des croyances démocratiques ? \\[0pt]
Y a-t-il des croyances raisonnables ? \\[0pt]
Y a-t-il des degrés dans la certitude ? \\[0pt]
Y a-t-il des degrés de liberté ? \\[0pt]
Y a-t-il des degrés de réalité ? \\[0pt]
Y a-t-il des degrés de vérité ? \\[0pt]
Y a-t-il des démonstrations en morale ? \\[0pt]
Y a-t-il des démonstrations en philosophie ? \\[0pt]
Y a-t-il des devoirs envers soi-même ? \\[0pt]
Y a-t-il des droits naturels ? \\[0pt]
Y a-t-il des erreurs en politique ? \\[0pt]
Y a-t-il des êtres de raison ? \\[0pt]
Y a-t-il des êtres mathématiques ? \\[0pt]
Y a-t-il des excès en art ? \\[0pt]
Y a-t-il des expériences absolument certaines ? \\[0pt]
Y a-t-il des expériences cruciales ? \\[0pt]
Y a-t-il des expériences métaphysiques ? \\[0pt]
Y a-t-il des faits moraux ? \\[0pt]
Y a-t-il des faits sans essence ? \\[0pt]
Y a-t-il des fins de la nature ? \\[0pt]
Y a-t-il des fins dernières ? \\[0pt]
Y a-t-il des fondements naturels à l'ordre social ? \\[0pt]
Y a-t-il des formes élémentaires de la société ? \\[0pt]
Y a-t-il des guerres justes ? \\[0pt]
Y a-t-il des héritages philosophiques ? \\[0pt]
Y a-t-il des idées générales ? \\[0pt]
Y a-t-il des idées innées ? \\[0pt]
Y a-t-il des inégalités justes ? \\[0pt]
Y a-t-il des jours où je n'existe pas ? \\[0pt]
Y a-t-il des leçons de l'histoire ? \\[0pt]
Y a-t-il des limites à la connaissance de l'homme par les sciences ? \\[0pt]
Y a-t-il des limites à la critique ? \\[0pt]
Y a-t-il des limites à la description ? \\[0pt]
Y a-t-il des limites à la responsabilité ? \\[0pt]
Y a-t-il des limites à l'exprimable ? \\[0pt]
Y a-t-il des limites au droit ? \\[0pt]
Y a-t-il des limites proprement morales à la discussion ? \\[0pt]
Y a-t-il des lois de l'histoire ? \\[0pt]
Y a-t-il des lois du comportement humain ? \\[0pt]
Y a-t-il des lois du hasard ? \\[0pt]
Y a-t-il des lois du social ? \\[0pt]
Y a-t-il des lois du vivant ? \\[0pt]
Y a-t-il des lois en histoire ? \\[0pt]
Y a-t-il des lois injustes ? \\[0pt]
Y a-t-il des lois morales ? \\[0pt]
Y a-t-il des lois non écrites ? \\[0pt]
Y a-t-il des lois sociales ? \\[0pt]
Y a-t-il des mécanismes humains ? \\[0pt]
Y a-t-il des mentalités collectives ? \\[0pt]
Y a-t-il des méthodes pour être heureux ? \\[0pt]
Y a-t-il des modèles en morale ? \\[0pt]
Y a-t-il des normes de la vie ? \\[0pt]
Y a-t-il des normes naturelles ? \\[0pt]
Y a-t-il des notions communes ? \\[0pt]
Y a-t-il de sots métiers ? \\[0pt]
Y a-t-il des passions collectives ? \\[0pt]
Y a-t-il des pathologies du social ? \\[0pt]
Y a-t-il des pathologies sociales ? \\[0pt]
Y a-t-il des pensées folles ? \\[0pt]
Y a-t-il des pensées inconscientes ? \\[0pt]
Y a-t-il des petites vertus ? \\[0pt]
Y a-t-il des peuples sans histoire ? \\[0pt]
Y a-t-il des phénomènes psychiques ? \\[0pt]
Y a-t-il des plaisirs innocents ? \\[0pt]
Y a-t-il des plaisirs mauvais ? \\[0pt]
Y a-t-il des plaisirs purs ? \\[0pt]
Y a-t-il des plaisirs simples ? \\[0pt]
Y a-t-il des points de vue en morale ? \\[0pt]
Y a-t-il des preuves d'amour ? \\[0pt]
Y a-t-il des preuves de la non-existence de Dieu ? \\[0pt]
Y a-t-il des preuves de l'existence de Dieu ? \\[0pt]
Y a-t-il des preuves en métaphysique ? \\[0pt]
Y a-t-il des preuves en philosophie ? \\[0pt]
Y a-t-il des progrès en art ? \\[0pt]
Y a-t-il des progrès en philosophie ? \\[0pt]
Y a-t-il des propriétés singulières ? \\[0pt]
Y a-t-il des questions sans réponse ? \\[0pt]
Y a-t-il des questions sans réponses ? \\[0pt]
Y a-t-il des raisons d'aimer ? \\[0pt]
Y a-t-il des raisons de vivre ? \\[0pt]
Y a-t-il des réalités mathématiques ? \\[0pt]
Y a-t-il des règles de l'art ? \\[0pt]
Y a-t-il des révolutions en art ? \\[0pt]
Y a-t-il des révolutions scientifiques ? \\[0pt]
Y a-t-il des savoirs immédiats ? \\[0pt]
Y a-t-il des sciences de l'éducation ? \\[0pt]
Y a-t-il des secrets de la nature ? \\[0pt]
Y a-t-il des sociétés archaïques ? \\[0pt]
Y a-t-il des sociétés sans État ? \\[0pt]
Y a-t-il des sociétés sans histoire ? \\[0pt]
Y a-t-il des substances incorporelles ? \\[0pt]
Y a-t-il des universaux anthropologiques ? \\[0pt]
Y a-t-il des vérités morales ? \\[0pt]
Y a-t-il des vérités philosophiques ? \\[0pt]
Y a-t-il des vérités sans preuve ? \\[0pt]
Y a-t-il des vertus mineures ? \\[0pt]
Y a-t-il des violences justifiées ? \\[0pt]
Y a-t-il des violences légitimes ? \\[0pt]
Y a-t-il différentes manières de connaître ? \\[0pt]
Y a-t-il du déterminisme dans les sciences humaines ? \\[0pt]
Y a-t-il du hasard objectif ? \\[0pt]
Y a-t-il du non-être ? \\[0pt]
Y a-t-il du progrès en politique ? \\[0pt]
Y a-t-il du sacré dans la nature ? \\[0pt]
Y a-t-il du synthétique \emph{a priori} ? \\[0pt]
Y a-t-il du vrai dans le beau ? \\[0pt]
Y a-t-il encore des mythologies ? \\[0pt]
Y a-t-il encore une sphère privée ? \\[0pt]
Y a-t-il lieu de distinguer entre éthique et morale ? \\[0pt]
Y a-t-il lieu de distinguer instinct et pulsion ? \\[0pt]
Y a-t-il lieu d'opposer la philosophie et les sciences humaines ? \\[0pt]
Y a-t-il lieu d'opposer matière et esprit ? \\[0pt]
Y a-t-il place pour l'idée de vérité en morale ? \\[0pt]
Y a-t-il plusieurs manières de définir ? \\[0pt]
Y a-t-il plusieurs métaphysiques ? \\[0pt]
Y a-t-il plusieurs morales ? \\[0pt]
Y a-t-il plusieurs nécessités ? \\[0pt]
Y a-t-il quelque chose de commun aux œuvres d'art ? \\[0pt]
Y a-t-il quelque chose de subversif dans les sciences humaines ? \\[0pt]
Y a-t-il un art de gouverner ? \\[0pt]
Y a-t-il un art de penser ? \\[0pt]
Y a-t-il un art de persuader ? \\[0pt]
Y a-t-il un art d'inventer ? \\[0pt]
Y a-t-il un art populaire ? \\[0pt]
Y a-t-il un art total ? \\[0pt]
Y a-t-il un au-delà de la vérité ? \\[0pt]
Y a-t-il un auteur de l'histoire ? \\[0pt]
Y a-t-il un autre monde ? \\[0pt]
Y a-t-il un beau idéal ? \\[0pt]
Y a-t-il un besoin métaphysique ? \\[0pt]
Y a-t-il un bien commun ? \\[0pt]
Y a-t-il un bien plus précieux que la paix ? \\[0pt]
Y a-t-il un bon usage de la rhétorique ? \\[0pt]
Y a-t-il un canon de la beauté ? \\[0pt]
Y a-t-il un commencement à tout ? \\[0pt]
Y a-t-il un critère du vrai ? \\[0pt]
Y a-t-il un désir de connaître ? \\[0pt]
Y a-t-il un déterminisme dans les sciences humaines ? \\[0pt]
Y a-t-il un devoir de travailler ? \\[0pt]
Y a-t-il un devoir d'être heureux ? \\[0pt]
Y a-t-il un devoir d'indignation ? \\[0pt]
Y a-t-il un devoir d'oubli ? \\[0pt]
Y a-t-il un différend entre poésie et philosophie ? \\[0pt]
Y a-t-il un droit à la différence ? \\[0pt]
Y a-t-il un droit de la guerre ? \\[0pt]
Y a-t-il un droit de mourir ? \\[0pt]
Y a-t-il un droit de résistance ? \\[0pt]
Y a-t-il un droit du plus fort ? \\[0pt]
Y a-t-il un droit international ? \\[0pt]
Y a-t-il un droit naturel ? \\[0pt]
Y a-t-il une argumentation métaphysique ? \\[0pt]
Y a-t-il une beauté morale ? \\[0pt]
Y a-t-il une beauté naturelle ? \\[0pt]
Y a-t-il une beauté propre à la photographie ? \\[0pt]
Y a-t-il une bonne éducation ? \\[0pt]
Y a-t-il une bonne imitation ? \\[0pt]
Y a-t-il une causalité historique ? \\[0pt]
Y a-t-il une compétence en politique ? \\[0pt]
Y a-t-il une connaissance du singulier ? \\[0pt]
Y a-t-il une connaissance logique ? \\[0pt]
Y a-t-il une connaissance métaphysique ? \\[0pt]
Y a-t-il une connaissance sensible ? \\[0pt]
Y a-t-il une correspondance des arts ? \\[0pt]
Y a-t-il une esthétique du laid ? \\[0pt]
Y a-t-il une esthétique industrielle ? \\[0pt]
Y a-t-il une éthique de l'authenticité ? \\[0pt]
Y a-t-il une expérience de la liberté ? \\[0pt]
Y a-t-il une expérience de l'éternité ? \\[0pt]
Y a-t-il une expérience du néant ? \\[0pt]
Y a-t-il une expérience métaphysique ? \\[0pt]
Y a-t-il une fin de l'histoire ? \\[0pt]
Y a-t-il une fin dernière ? \\[0pt]
Y a-t-il une force du droit ? \\[0pt]
Y a-t-il une formation de la pensée logique ? \\[0pt]
Y a-t-il une forme morale de fanatisme ? \\[0pt]
Y a-t-il une hiérarchie des êtres ? \\[0pt]
Y a-t-il une hiérarchie des sciences ? \\[0pt]
Y a-t-il une hiérarchie des sciences sociales ? \\[0pt]
Y a-t-il une histoire de la folie ? \\[0pt]
Y a-t-il une histoire de l'art ? \\[0pt]
Y a-t-il une histoire de la vérité ? \\[0pt]
Y a-t-il une histoire de la vie quotidienne ? \\[0pt]
Y a-t-il une histoire universelle ? \\[0pt]
Y a-t-il une intentionnalité collective ? \\[0pt]
Y a-t-il une justice sans morale ? \\[0pt]
Y a-t-il une logique de la découverte ? \\[0pt]
Y a-t-il une logique de la découverte scientifique ? \\[0pt]
Y a-t-il une logique de la déraison ? \\[0pt]
Y a-t-il une logique de l'art ? \\[0pt]
Y a-t-il une logique de l'inconscient ? \\[0pt]
Y a-t-il une logique des événements historiques ? \\[0pt]
Y a-t-il une logique du mythe ? \\[0pt]
Y a-t-il une loi suprême ? \\[0pt]
Y a-t-il une mathématique universelle ? \\[0pt]
Y a-t-il une métaphysique de l'amour ? \\[0pt]
Y a-t-il une méthode propre aux sciences humaines ? \\[0pt]
Y a-t-il une nature humaine ? \\[0pt]
Y a-t-il une nécessité de l'erreur ? \\[0pt]
Y a-t-il une opinion publique mondiale ? \\[0pt]
Y a-t-il une ou plusieurs philosophies ? \\[0pt]
Y a-t-il une pensée sans signes ? \\[0pt]
Y a-t-il une pensée technique ? \\[0pt]
Y a-t-il une perception esthétique ? \\[0pt]
Y a-t-il une philosophie de la philosophie ? \\[0pt]
Y a-t-il une philosophie première ? \\[0pt]
Y a-t-il une place dans les sciences humaines pour la catégorie de personne ? \\[0pt]
Y a-t-il une place pour la morale dans l'économie ? \\[0pt]
Y a-t-il une psychologie animale ? \\[0pt]
Y a-t-il une raison à tout ? \\[0pt]
Y a-t-il une raison des choses ? \\[0pt]
Y a-t-il une rationalité dans la religion ? \\[0pt]
Y a-t-il une rationalité du marché ? \\[0pt]
Y a-t-il une rationalité propre de l'intérêt ? \\[0pt]
Y a-t-il une réalité des idées ? \\[0pt]
Y a-t-il une réalité du hasard ? \\[0pt]
Y a-t-il une religion civile ? \\[0pt]
Y a-t-il une science de la vie ? \\[0pt]
Y a-t-il une science de la vie mentale ? \\[0pt]
Y a-t-il une science de l'être ? \\[0pt]
Y a-t-il une science de l'homme ? \\[0pt]
Y a-t-il une science de l'individuel ? \\[0pt]
Y a-t-il une science des principes ? \\[0pt]
Y a-t-il une science du qualitatif ? \\[0pt]
Y a-t-il une science politique ? \\[0pt]
Y a-t-il une sensibilité esthétique ? \\[0pt]
Y a-t-il une singularité de l'histoire de l'art ? \\[0pt]
Y a-t-il une spécificité de la délibération politique ? \\[0pt]
Y a-t-il une spécificité des sciences humaines ? \\[0pt]
Y a-t-il un esprit scientifique ? \\[0pt]
Y a-t-il une technique de la nature ? \\[0pt]
Y a-t-il une temporalité proprement morale ? \\[0pt]
Y a-t-il une unité de la science ? \\[0pt]
Y a-t-il une unité des sciences ? \\[0pt]
Y a-t-il une unité en psychologie ? \\[0pt]
Y a-t-il une universalité des mathématiques ? \\[0pt]
Y a-t-il une utilité des lieux communs ? \\[0pt]
Y a-t-il une vérité absolue ? \\[0pt]
Y a-t-il une vérité dans les arts ? \\[0pt]
Y a-t-il une vérité de la fiction ? \\[0pt]
Y a-t-il une vérité de l'inconscient ? \\[0pt]
Y a-t-il une vérité de l'œuvre d'art ? \\[0pt]
Y a-t-il une vérité des sentiments ? \\[0pt]
Y a-t-il une vérité des symboles ? \\[0pt]
Y a-t-il une vérité du sentiment ? \\[0pt]
Y a-t-il une vérité en histoire ? \\[0pt]
Y a-t-il une vérité philosophique ? \\[0pt]
Y a-t-il une vertu de l'ignorance \\[0pt]
Y a-t-il une vertu de l'imitation ? \\[0pt]
Y a-t-il une vie de l'esprit ? \\[0pt]
Y a-t-il un fondement moral aux circonstances atténuantes ? \\[0pt]
Y a-t-il un inconscient collectif ? \\[0pt]
Y a-t-il un inconscient psychique ? \\[0pt]
Y a-t-il un inconscient social ? \\[0pt]
Y a t-il un instinct moral ? \\[0pt]
Y a-t-il un langage commun ? \\[0pt]
Y a-t-il un langage de l'inconscient ? \\[0pt]
Y a-t-il un langage du corps ? \\[0pt]
Y a-t-il un mal absolu ? \\[0pt]
Y a-t-il un monde extérieur ? \\[0pt]
Y a-t-il un ordre des choses ? \\[0pt]
Y a-t-il un principe du mal ? \\[0pt]
Y a-t-il un progrès dans l'art ? \\[0pt]
Y a-t-il un progrès des/en sciences humaines ? \\[0pt]
Y a-t-il un progrès en art ? \\[0pt]
Y a-t-il un progrès moral ? \\[0pt]
Y a-t-il un savoir du bien ? \\[0pt]
Y a-t-il un savoir du contingent ? \\[0pt]
Y a-t-il un savoir du corps ? \\[0pt]
Y a-t-il un savoir du politique ? \\[0pt]
Y a-t-il un savoir immédiat ? \\[0pt]
Y a-t-il un savoir inconscient ? \\[0pt]
Y a-t-il un sens à dire que Dieu comprend, veut, fait ? \\[0pt]
Y a-t-il un sens à penser un droit des générations futures ? \\[0pt]
Y a-t-il un sens à se dire « citoyen du monde » ? \\[0pt]
Y a-t-il un sens du beau ? \\[0pt]
Y a-t-il un sens moral ? \\[0pt]
Y a-t-il un sentiment métaphysique ? \\[0pt]
Y a-t-il un sentiment naturel du beau ? \\[0pt]
Y a-t-il un temps des choses ? \\[0pt]
Y a-t-il un temps pour l'action ? \\[0pt]
Y a-t-il un temps pour tout ? \\[0pt]
Y a-t-il un traitement scientifique du langage ? \\[0pt]
Y a-t-il un usage moral des passions ? \\[0pt]
Y a-t-une science du comportement ? \\[0pt]


\subsection{Agrégation externe}
\label{sec:orgbd1b840}

\noindent
Abjection et sainteté \\[0pt]
Abolir la propriété \\[0pt]
Absolu / relatif \\[0pt]
Abstraire \\[0pt]
Accepter l'injustice \\[0pt]
À chacun ses goûts \\[0pt]
À chacun son dû \\[0pt]
Acteurs sociaux et usages sociaux \\[0pt]
Action et événement \\[0pt]
Action et fabrication \\[0pt]
Actions et pratiques \\[0pt]
Admettre le hasard, est-ce nier l'ordre de la nature ? \\[0pt]
Admettre une cause première, est-ce faire une pétition de principe ? \\[0pt]
Affirmation et négation \\[0pt]
Affirmer et nier \\[0pt]
Agir \\[0pt]
Agir en conscience \\[0pt]
Agir librement \\[0pt]
Agir moralement, est-ce lutter contre ses idées ? \\[0pt]
Ai-je une âme ? \\[0pt]
« Aime, et fais ce que tu veux » \\[0pt]
Aimer la nature \\[0pt]
Aimer la vérité \\[0pt]
Aimer la vie \\[0pt]
Aimer les lois \\[0pt]
Aimer ses proches \\[0pt]
Aimer son prochain comme soi-même \\[0pt]
Aimer une œuvre d'art \\[0pt]
« Aimez vos ennemis » \\[0pt]
Aliénation et servitude \\[0pt]
« À l'impossible, nul n'est tenu » \\[0pt]
À l'impossible nul n'est tenu \\[0pt]
Aller au-delà des apparences \\[0pt]
Aller au fond des choses \\[0pt]
Aller au vrai \\[0pt]
Altérité, diversité, pluralité \\[0pt]
Âme, conscience, esprit \\[0pt]
Amitié et société \\[0pt]
Amour et respect \\[0pt]
Analogie et ressemblance \\[0pt]
Analyse et synthèse \\[0pt]
Analyser les mœurs \\[0pt]
Animal politique ou social ? \\[0pt]
Anomalie et anomie \\[0pt]
Antérieur / postérieur \\[0pt]
Anthropologie et humanisme \\[0pt]
Anthropologie et ontologie \\[0pt]
Anthropologie et politique \\[0pt]
Apparaître \\[0pt]
Apparence et réalité \\[0pt]
Apparition et manifestation \\[0pt]
Appliquer la loi \\[0pt]
Apprend-on à penser ? \\[0pt]
Apprend-on à percevoir la beauté ? \\[0pt]
Apprend-on à voir ? \\[0pt]
Apprendre \\[0pt]
Apprendre à aimer \\[0pt]
Apprendre à gouverner \\[0pt]
Apprendre à parler \\[0pt]
Apprendre à penser \\[0pt]
Apprendre à vivre \\[0pt]
Apprendre à voir \\[0pt]
Apprendre et devenir \\[0pt]
Apprendre s'apprend-il ? \\[0pt]
Apprentissage et conditionnement \\[0pt]
Approcher du vrai \\[0pt]
Après-coup \\[0pt]
« Après moi, le déluge » \\[0pt]
\emph{A priori} et \emph{a posteriori} \\[0pt]
À quelle généralité peuvent prétendre les sciences humaines ? \\[0pt]
À quelle objectivité peuvent prétendre les sciences sociales ? \\[0pt]
À quelles conditions est-il acceptable de travailler ? \\[0pt]
À quelles conditions penser une seconde nature ? \\[0pt]
À quelles conditions une démarche est-elle scientifique ? \\[0pt]
À quelles conditions une explication est-elle scientifique ? \\[0pt]
À quelles conditions une hypothèse est-elle scientifique ? \\[0pt]
À quelles conditions un énoncé est-il doué de sens ? \\[0pt]
À quelles conditions une pensée est-elle libre ? \\[0pt]
À quelque chose le malheur peut-il être bon ? \\[0pt]
À quelque chose, le malheur peut-il être bon ? \\[0pt]
À qui appartient-il de décider du juste et de l'injuste ? \\[0pt]
À qui appartient la Terre ? \\[0pt]
À qui la faute ? \\[0pt]
À qui peut-on faire confiance ? \\[0pt]
À qui se fier ? \\[0pt]
À quoi bon ? \\[0pt]
À quoi bon chercher à prouver l'existence de Dieu ? \\[0pt]
À quoi bon des philosophes ? \\[0pt]
À quoi bon discuter ? \\[0pt]
À quoi bon imaginer la cité idéale ? \\[0pt]
À quoi bon la morale ? \\[0pt]
À quoi bon les sciences humaines et sociales ? \\[0pt]
À quoi bon les systèmes ? \\[0pt]
À quoi bon rêver ? \\[0pt]
À quoi est-il nécessaire d'obéir ? \\[0pt]
À quoi faut-il être fidèle ? \\[0pt]
À quoi faut-il renoncer ? \\[0pt]
À quoi juger l'action d'un gouvernement ? \\[0pt]
À quoi la conscience nous donne-t-elle accès ? \\[0pt]
À quoi la logique peut-elle servir dans les sciences ? \\[0pt]
À quoi l'art est-il bon ? \\[0pt]
À quoi les sciences nous forment-elles ? \\[0pt]
À quoi reconnaît-on la vérité ? \\[0pt]
À quoi reconnaît-on l'injustice ? \\[0pt]
À quoi reconnaît-on qu'une politique est juste ? \\[0pt]
À quoi reconnaît-on qu'une théorie est scientifique ? \\[0pt]
À quoi reconnaît-on un bon gouvernement ? \\[0pt]
À quoi reconnaît-on un comportement humain ? \\[0pt]
À quoi reconnaît-on une attitude religieuse ? \\[0pt]
À quoi reconnaît-on une œuvre d'art ? \\[0pt]
À quoi reconnaît-on une science ? \\[0pt]
À quoi rêve-t-on ? \\[0pt]
À quoi sert la bioéthique ? \\[0pt]
À quoi sert la critique ? \\[0pt]
À quoi sert la dialectique ? \\[0pt]
À quoi sert la logique ? \\[0pt]
À quoi sert la métaphysique ? \\[0pt]
À quoi sert la négation ? \\[0pt]
À quoi sert la notion de contrat social ? \\[0pt]
À quoi sert la notion d'état de nature ? \\[0pt]
À quoi sert la technique ? \\[0pt]
À quoi sert la théodicée ? \\[0pt]
À quoi sert l'écriture ? \\[0pt]
À quoi sert l'État ? \\[0pt]
À quoi sert l'intuition ? \\[0pt]
À quoi sert l'ontologie ? \\[0pt]
À quoi sert une métaphore ? \\[0pt]
À quoi sert un exemple ? \\[0pt]
À quoi servent les cérémonies ? \\[0pt]
À quoi servent les constitutions ? \\[0pt]
À quoi servent les croyances ? \\[0pt]
À quoi servent les doctrines morales ? \\[0pt]
À quoi servent les élections ? \\[0pt]
À quoi servent les encyclopédies ? \\[0pt]
À quoi servent les hypothèses ? \\[0pt]
À quoi servent les preuves ? \\[0pt]
À quoi servent les règles ? \\[0pt]
À quoi servent les religions ? \\[0pt]
À quoi servent les sciences ? \\[0pt]
À quoi servent les sciences humaines ? \\[0pt]
À quoi servent les statistiques ? \\[0pt]
À quoi servent les utopies ? \\[0pt]
À quoi suis-je obligé ? \\[0pt]
À quoi tient la fermeté du vouloir ? \\[0pt]
À quoi tient la vérité d'une interprétation ? \\[0pt]
À quoi tient notre humanité ? \\[0pt]
Architecture et religion \\[0pt]
Architecture et utopie \\[0pt]
Argumenter \\[0pt]
Argumenter et démontrer \\[0pt]
Arriver à ses fins \\[0pt]
Arrive-t-il que l'impossible se produise ? \\[0pt]
Art et abstraction \\[0pt]
Art et apparence \\[0pt]
Art et artifice \\[0pt]
Art et authenticité \\[0pt]
Art et critique \\[0pt]
Art et décadence \\[0pt]
Art et divertissement \\[0pt]
Art et émotion \\[0pt]
Art et expression \\[0pt]
Art et finitude \\[0pt]
Art et folie \\[0pt]
Art et forme \\[0pt]
Art et illusion \\[0pt]
Art et image \\[0pt]
Art et industrie \\[0pt]
Art et interdit \\[0pt]
Art et jeu \\[0pt]
Art et langage \\[0pt]
Art et marchandise \\[0pt]
Art et mélancolie \\[0pt]
Art et mémoire \\[0pt]
Art et métaphysique \\[0pt]
Art et morale \\[0pt]
Art et politique \\[0pt]
Art et présence \\[0pt]
Art et propagande \\[0pt]
Art et religieux \\[0pt]
Art et religion \\[0pt]
Art et représentation \\[0pt]
Art et technique \\[0pt]
Art et tradition \\[0pt]
Art et transgression \\[0pt]
Art et utilité \\[0pt]
Art et vérité \\[0pt]
Artiste et artisan \\[0pt]
Artistes et ingénieurs \\[0pt]
Art populaire et art savant \\[0pt]
Arts de l'espace et arts du temps \\[0pt]
Arts du temps et arts de l'espace \\[0pt]
À science nouvelle, nouvelle philosophie ? \\[0pt]
A-t-on besoin de spécialistes ? \\[0pt]
A-t-on des droits contre l'État ? \\[0pt]
A-t-on des raisons de croire ? \\[0pt]
A-t-on des raisons de croire ce qu'on croit ? \\[0pt]
A-t-on expliqué un phénomène quand on l'a rapporté à des lois ? \\[0pt]
A-t-on raison de tenir à la démocratie ? \\[0pt]
Attendre \\[0pt]
Attente et espérance \\[0pt]
Au-delà \\[0pt]
Au-delà de la nature ? \\[0pt]
Au nom de quoi le plaisir serait-il condamnable ? \\[0pt]
Au nom du peuple \\[0pt]
Aussitôt dit, aussitôt fait \\[0pt]
Autorité et autoritarisme \\[0pt]
Autorité et pouvoir \\[0pt]
Autrui \\[0pt]
Autrui est-il mon prochain ? \\[0pt]
Avoir \\[0pt]
Avoir de la chance \\[0pt]
Avoir de l'autorité \\[0pt]
Avoir de l'esprit \\[0pt]
Avoir de l'expérience \\[0pt]
Avoir des ennemis \\[0pt]
Avoir des ennemis, est-ce le principe de la politique ? \\[0pt]
Avoir des principes \\[0pt]
Avoir du caractère \\[0pt]
Avoir du goût \\[0pt]
Avoir du mérite \\[0pt]
Avoir du style \\[0pt]
Avoir la conscience tranquille \\[0pt]
Avoir le droit \\[0pt]
Avoir mauvaise conscience \\[0pt]
Avoir peur \\[0pt]
Avoir raison \\[0pt]
Avoir ses raisons \\[0pt]
Avoir un corps \\[0pt]
Avoir une bonne mémoire \\[0pt]
Avoir une idée \\[0pt]
Avoir un sens \\[0pt]
Avons-nous à apprendre des images ? \\[0pt]
Avons-nous accès aux choses-mêmes ? \\[0pt]
Avons-nous besoin de l'État ? \\[0pt]
Avons-nous besoin de lois ? \\[0pt]
Avons-nous besoin de métaphysique ? \\[0pt]
Avons-nous besoin de méthodes ? \\[0pt]
Avons-nous besoin de partis politiques ? \\[0pt]
Avons-nous besoin de spectacles ? \\[0pt]
Avons-nous besoin de traditions ? \\[0pt]
Avons-nous besoin d'être gouvernés ? \\[0pt]
Avons-nous besoin d'experts en matière d'art ? \\[0pt]
Avons-nous besoin d'idéal \\[0pt]
Avons-nous besoin d'un chef ? \\[0pt]
Avons-nous besoin d'une conception métaphysique du monde ? \\[0pt]
Avons-nous besoin d'une définition de l'art ? \\[0pt]
Avons-nous besoin d'une philosophie politique ? \\[0pt]
Avons-nous besoin d'un libre arbitre ? \\[0pt]
Avons-nous besoin d'utopies ? \\[0pt]
Avons-nous des devoirs envers les animaux ? \\[0pt]
Avons-nous des devoirs envers les morts ? \\[0pt]
Avons-nous des devoirs envers le vivant ? \\[0pt]
Avons-nous des devoirs envers nous-mêmes ? \\[0pt]
Avons-nous encore besoin de la nature ? \\[0pt]
Avons-nous le devoir d'être heureux ? \\[0pt]
Avons-nous le droit de juger autrui ? \\[0pt]
Avons-nous un accès direct aux phénomènes ? \\[0pt]
Avons-nous une identité ? \\[0pt]
Avons-nous une responsabilité envers le passé ? \\[0pt]
Avons-nous un monde commun ? \\[0pt]
Axiomatique et vérité \\[0pt]
Bâtir un monde \\[0pt]
Beauté et vérité \\[0pt]
Beauté naturelle et beauté artistique \\[0pt]
Beauté réelle, beauté idéale \\[0pt]
Bien agir \\[0pt]
Bien commun et droits individuels \\[0pt]
Bien faire \\[0pt]
« Bienheureuse faute » \\[0pt]
Bien jouer son rôle \\[0pt]
Bien juger \\[0pt]
Bien parler \\[0pt]
Bien parler, est-ce bien penser ? \\[0pt]
Bien vivre \\[0pt]
Bon sens et sens commun \\[0pt]
Calculer \\[0pt]
Calculer et penser \\[0pt]
Cartographier \\[0pt]
Castes et classes \\[0pt]
Catégories de l'être, catégories de langue \\[0pt]
Catégories logiques et catégories linguistiques \\[0pt]
Causalité et finalité \\[0pt]
Cause et loi \\[0pt]
Cause et motivation \\[0pt]
Cause et raison \\[0pt]
Causes et lois \\[0pt]
Causes et motivations \\[0pt]
Causes premières et causes secondes \\[0pt]
Ceci \\[0pt]
Cela a-t-il un sens de penser par soi-même ? \\[0pt]
Ce que la morale autorise, l'État peut-il légitimement l'interdire ? \\[0pt]
Ce que sait le poète \\[0pt]
Ce qui dépend de moi \\[0pt]
Ce qui doit-être, est-ce autre chose que ce qui est ? \\[0pt]
Ce qui est à moi \\[0pt]
Ce qui est contradictoire peut-il exister ? \\[0pt]
Ce qui est faux est-il dénué de sens ? \\[0pt]
Ce qui est subjectif est-il arbitraire ? \\[0pt]
Ce qui fut et ce qui sera \\[0pt]
Ce qui importe \\[0pt]
Ce qu'il y a \\[0pt]
Ce qui n'a pas de prix \\[0pt]
Ce qui n'a pas lieu d'être \\[0pt]
Ce qui ne dépend pas de nous \\[0pt]
Ce qui n'est pas \\[0pt]
Ce qui n'est pas réel est-il impossible ? \\[0pt]
Ce qui n'existe pas \\[0pt]
Ce qui passe et ce qui demeure \\[0pt]
Ce qui subsiste et ce qui change \\[0pt]
Ce qu'on ne peut pas vendre \\[0pt]
Certaines œuvres d'art ont-elles plus de valeur que d'autres ? \\[0pt]
Certitude et pari \\[0pt]
Certitude et vérité \\[0pt]
Cesser d'espérer \\[0pt]
« C'est humain » \\[0pt]
« C'est la vie » \\[0pt]
C'est pour ton bien \\[0pt]
« C'est tout un art » \\[0pt]
C'est trop beau pour être vrai ! \\[0pt]
Ceux qui savent doivent-ils gouverner ? \\[0pt]
Changement et mouvement \\[0pt]
Changer \\[0pt]
Changer le monde \\[0pt]
Changer ses désirs plutôt que l'ordre du monde \\[0pt]
Chanter et parler \\[0pt]
Chaos et cosmos \\[0pt]
Chaque science porte-t-elle une métaphysique qui lui est propre ? \\[0pt]
Chercher ses mots \\[0pt]
Chercher son intérêt, est-ce être immoral ? \\[0pt]
Choisir \\[0pt]
Choisir ses souvenirs ? \\[0pt]
Choisit-on ses souvenirs ? \\[0pt]
Choisit-on son corps ? \\[0pt]
Chose et objet \\[0pt]
Choses et personnes \\[0pt]
Cinéma et réalité \\[0pt]
Cinéma et vérité \\[0pt]
Cité juste ou citoyen juste ? \\[0pt]
Citoyen et soldat \\[0pt]
Citoyenneté et solidarité \\[0pt]
Citoyens du monde \\[0pt]
Civilisé, barbare, sauvage \\[0pt]
Classer \\[0pt]
Classer, est-ce penser ? \\[0pt]
Classer et ordonner \\[0pt]
Classes et essences \\[0pt]
Classes et histoire \\[0pt]
Cohérence et vérité \\[0pt]
Collectionner \\[0pt]
Commander \\[0pt]
Comme d'habitude \\[0pt]
Commémorer \\[0pt]
Commencer \\[0pt]
Commencer en philosophie \\[0pt]
Comment assumer les conséquences de ses actes ? \\[0pt]
Comment bien appliquer une règle ? \\[0pt]
Comment bien vivre ? \\[0pt]
Comment caractériser une idée confuse ? \\[0pt]
Comment comprendre une croyance qu'on ne partage pas ? \\[0pt]
Comment connaître l'autre ? \\[0pt]
Comment considérer le devenir ? \\[0pt]
Comment décider, sinon à la majorité ? \\[0pt]
Comment définir la démocratie ? \\[0pt]
Comment définir la raison ? \\[0pt]
Comment définir la responsabilité politique \\[0pt]
Comment définir la signification \\[0pt]
Comment définir le bien commun ? \\[0pt]
Comment définir le laid ? \\[0pt]
Comment définir le peuple ? \\[0pt]
Comment définir le style ? \\[0pt]
Comment définir l'État de droit ? \\[0pt]
Comment définir l'excès ? \\[0pt]
Comment définir l'infini ? \\[0pt]
Comment déterminer le sens d'une conduite ? \\[0pt]
Comment devient-on artiste ? \\[0pt]
Comment devient-on citoyen ? \\[0pt]
Comment devient-on raisonnable ? \\[0pt]
Comment dire la vérité ? \\[0pt]
Comment entrer dans l'histoire ? \\[0pt]
Comment évaluer l'art ? \\[0pt]
Comment expliquer l'erreur ? \\[0pt]
Comment faire de l'homme un objet d'étude ? \\[0pt]
Comment fonder une psychologie sociale ? \\[0pt]
Comment juger de la politique ? \\[0pt]
Comment juger d'une œuvre d'art ? \\[0pt]
Comment juger son éducation ? \\[0pt]
Comment justifier l'autonomie des sciences de la vie ? \\[0pt]
Comment justifier une méthode ? \\[0pt]
Comment la psychanalyse use-t-elle de l'interprétation ? \\[0pt]
Comment la réalité peut-elle dépasser la fiction ? \\[0pt]
Comment la volonté peut-elle être indéterminée ? \\[0pt]
Comment les sciences sociales expliquent-elles l'irrationnel ? \\[0pt]
Comment les sociétés changent-elles ? \\[0pt]
Comment ne pas être libéral ? \\[0pt]
Comment penser la diversité des langues ? \\[0pt]
Comment penser les universaux ? \\[0pt]
Comment penser un pouvoir qui ne corrompe pas ? \\[0pt]
Comment peut-on choisir entre différentes hypothèses ? \\[0pt]
« Comment peut-on être persan ? » \\[0pt]
Comment peut-on être sceptique ? \\[0pt]
Comment reconnaît-on une œuvre d'art ? \\[0pt]
Comment réfuter une thèse métaphysique ? \\[0pt]
Comment représenter la douleur ? \\[0pt]
Comment sait-on qu'on se comprend ? \\[0pt]
Comment s'assurer de ce qui est réel ? \\[0pt]
Comment s'assurer qu'on est libre ? \\[0pt]
Comment savoir ce qui existe ? \\[0pt]
Comment s'entendre ? \\[0pt]
Comment traiter les animaux ? \\[0pt]
Comment trancher une controverse ? \\[0pt]
Comment vérifier les hypothèses ? \\[0pt]
Comment vivre ensemble ? \\[0pt]
Comme on dit \\[0pt]
Communauté, collectivité, société \\[0pt]
Communauté et conflit \\[0pt]
Communauté et société \\[0pt]
Communiquer \\[0pt]
Communiquer et enseigner \\[0pt]
Comparer les cultures \\[0pt]
Comparer les cultures ? \\[0pt]
Comparer les langues \\[0pt]
Compatir \\[0pt]
Compétence et autorité \\[0pt]
Complet / incomplet \\[0pt]
Comportement et conduite \\[0pt]
Composer avec les circonstances \\[0pt]
Composition et construction \\[0pt]
Comprendre, est-ce excuser ? \\[0pt]
Comprendre, est-ce expliquer par les causes ? \\[0pt]
Compter et mesurer \\[0pt]
Compter sur soi \\[0pt]
Concept et existence \\[0pt]
Concept et image \\[0pt]
Concept et métaphore \\[0pt]
Conception et perception \\[0pt]
Concevoir et juger \\[0pt]
Concevoir et percevoir \\[0pt]
Concevoir le possible \\[0pt]
Conduire sa vie \\[0pt]
Conduire ses pensées \\[0pt]
Confondre \\[0pt]
Connaissance commune et connaissance scientifique \\[0pt]
Connaissance, croyance, conjecture \\[0pt]
Connaissance du futur et connaissance du passé \\[0pt]
Connaissance et croyance \\[0pt]
Connaissance et expérience \\[0pt]
Connaissance et liberté \\[0pt]
Connaissance et progrès \\[0pt]
Connaissance historique et action politique \\[0pt]
Connaît-on la vie ou le vivant ? \\[0pt]
Connaître, est-ce connaître par les causes ? \\[0pt]
Connaître et agir \\[0pt]
Connaître et comprendre \\[0pt]
Connaître et penser \\[0pt]
Connaître, expliquer, comprendre \\[0pt]
Connaître les causes \\[0pt]
Connaître les choses, en quoi est-ce déterminer leurs différences ? \\[0pt]
Connaître par les causes \\[0pt]
Connaître ses limites \\[0pt]
Conscience et mémoire \\[0pt]
Conseiller le prince \\[0pt]
Consensus et conflit \\[0pt]
Consentir \\[0pt]
Conservatisme et tradition \\[0pt]
Considère-t-on jamais le temps en lui-même ? \\[0pt]
Consistance et précarité \\[0pt]
Constitution et lois \\[0pt]
Consumérisme et démocratie \\[0pt]
Contemplation et action \\[0pt]
Contemplation et création \\[0pt]
Contemplation et distraction \\[0pt]
Contempler \\[0pt]
Contempler une œuvre d'art \\[0pt]
Contingence et nécessité \\[0pt]
Contingence et rationalité \\[0pt]
Continuité et discontinuité \\[0pt]
Contradiction et opposition \\[0pt]
Contrainte et désobéissance \\[0pt]
Contrainte et obligation \\[0pt]
Convention et observation \\[0pt]
Conviction et responsabilité \\[0pt]
Corps et identité \\[0pt]
Correspondre \\[0pt]
Couleur et forme \\[0pt]
Création et critique \\[0pt]
Création et production \\[0pt]
Création et réception \\[0pt]
Créativité et contrainte \\[0pt]
Créer \\[0pt]
Crime et châtiment \\[0pt]
Crimes et châtiments \\[0pt]
Crise et création \\[0pt]
Crise et progrès \\[0pt]
Crise et révolution \\[0pt]
Critiquer \\[0pt]
Critiquer la démocratie \\[0pt]
Croire au bonheur \\[0pt]
Croire aux fictions \\[0pt]
Croire en Dieu \\[0pt]
Croire en soi \\[0pt]
Croire, est-ce être faible ? \\[0pt]
Croire et savoir \\[0pt]
Croire savoir \\[0pt]
Croyance et certitude \\[0pt]
Croyance et passivité \\[0pt]
Croyance et probabilité \\[0pt]
Croyance et société \\[0pt]
Croyance et vérité \\[0pt]
Cultes et rituels \\[0pt]
Cultivons notre jardin \\[0pt]
Culture et civilisation \\[0pt]
Culture et conscience \\[0pt]
Culture et moralité \\[0pt]
Culture et sociétés \\[0pt]
Dans les sciences humaines et sociales, peut-on préconiser l'imitation des sciences de la nature ? \\[0pt]
Dans quelle mesure la traduction est-elle une activité cognitive ? \\[0pt]
D'après nature \\[0pt]
Débattre \\[0pt]
Décider \\[0pt]
Décomposer les choses \\[0pt]
Décorer \\[0pt]
Découverte et invention \\[0pt]
Découverte et invention dans les sciences \\[0pt]
Découvrir \\[0pt]
Découvrir et inventer \\[0pt]
Décrire \\[0pt]
Décrire, est-ce déjà expliquer ? \\[0pt]
Décrire une langue \\[0pt]
Déduction et expérience \\[0pt]
Défendre ses droits \\[0pt]
Défendre son honneur \\[0pt]
Définir \\[0pt]
Définir, est-ce déterminer l'essence ? \\[0pt]
Définir et démontrer : à quoi bon ? \\[0pt]
Définir l'art : à quoi bon ? \\[0pt]
Définir la vérité, est-ce la connaître ? \\[0pt]
Définition et démonstration \\[0pt]
Définition nominale et définition réelle \\[0pt]
Définitions, axiomes, postulats \\[0pt]
Dégager la structure, est-ce rendre compte de ce qu'est une chose ? \\[0pt]
Déjouer \\[0pt]
« De la musique avant toute chose » \\[0pt]
Délibérer \\[0pt]
Délibérer, est-ce être dans l'incertitude ? \\[0pt]
De l'utilité des voyages \\[0pt]
Démêler le vrai du faux \\[0pt]
Démériter \\[0pt]
Démocrates et démagogues \\[0pt]
Démocratie ancienne et démocratie moderne \\[0pt]
Démocratie et anarchie \\[0pt]
Démocratie et consensus \\[0pt]
Démocratie et démagogie \\[0pt]
Démocratie et impérialisme \\[0pt]
Démocratie et représentation \\[0pt]
Démocratie et république \\[0pt]
Démocratie et transparence \\[0pt]
Démocratie et vérité \\[0pt]
Démocratie représentative \\[0pt]
Démonstration et argumentation \\[0pt]
Démonstration et déduction \\[0pt]
Démontrer et argumenter \\[0pt]
Dénaturer \\[0pt]
Dépasser les apparences ? \\[0pt]
Dépasser les bornes \\[0pt]
Dépasser l'humain \\[0pt]
Déplaire \\[0pt]
De quel bonheur sommes-nous capables ? \\[0pt]
De quel droit ? \\[0pt]
De quelle certitude la science est-elle capable ? \\[0pt]
De quelle expertise les sciences humaines peuvent-elles se prévaloir ? \\[0pt]
De quelle réalité témoignent nos perceptions ? \\[0pt]
De quelle science humaine la folie peut-elle être l'objet ? \\[0pt]
De quelle transgression l'art est-il susceptible ? \\[0pt]
De quelle vérité l'art est-il capable ? \\[0pt]
De quelle vérité l'opinion est-elle capable ? \\[0pt]
De qui sommes-nous les obligés ? \\[0pt]
De quoi a-t-on besoin ? \\[0pt]
De quoi doute un sceptique ? \\[0pt]
De quoi est-on conscient ? \\[0pt]
De quoi est-on malheureux ? \\[0pt]
De quoi faisons-nous l'expérience ? \\[0pt]
De quoi fait-on l'expérience face à une œuvre ? \\[0pt]
De quoi « humain » est-il le nom ? \\[0pt]
De quoi la forme est-elle la forme ? \\[0pt]
De quoi la logique est-elle la science ? \\[0pt]
De quoi la politique s'occupe-t-elle ? \\[0pt]
De quoi l'art nous console-t-il ? \\[0pt]
De quoi l'art nous délivre-t-il ? \\[0pt]
De quoi l'art peut-il nous libérer ? \\[0pt]
De quoi le phénomène est-il l'apparaître ? \\[0pt]
De quoi le réel est-il constitué ? \\[0pt]
De quoi les logiciens parlent-ils ? \\[0pt]
De quoi les métaphysiciens parlent-ils ? \\[0pt]
De quoi les sciences humaines nous instruisent-elles ? \\[0pt]
De quoi l'État doit-il être propriétaire ? \\[0pt]
De quoi le tissu social est-il fait ? \\[0pt]
De quoi l'expérience esthétique est-elle expérience ? \\[0pt]
De quoi l'expérience esthétique est-elle l'expérience ? \\[0pt]
De quoi l'historien fait-il l'histoire ? \\[0pt]
De quoi n'avons-nous pas conscience ? \\[0pt]
De quoi ne peut-on pas répondre ? \\[0pt]
De quoi parlent les mathématiques ? \\[0pt]
De quoi parlent les théories physiques ? \\[0pt]
De quoi pâtit-on ? \\[0pt]
De quoi peut-on réclamer la propriété ? \\[0pt]
De quoi pouvons-nous être certains ? \\[0pt]
De quoi sommes-nous prisonniers ? \\[0pt]
De quoi sommes-nous responsables ? \\[0pt]
De quoi suis-je responsable ? \\[0pt]
De quoi y a-t-il expérience ? \\[0pt]
De quoi y a-t-il histoire ? \\[0pt]
Déraisonner \\[0pt]
Désacraliser \\[0pt]
Des comportements économiques peuvent-ils être rationnels ? \\[0pt]
Désespérer de l'homme \\[0pt]
Des événements aléatoires peuvent-ils obéir à des lois ? \\[0pt]
Des goûts et des couleurs \\[0pt]
Des hommes et des dieux \\[0pt]
Désintérêt et désintéressement \\[0pt]
Désirer \\[0pt]
Désirer, est-ce refuser de se satisfaire de la réalité \\[0pt]
Désire-t-on la reconnaissance ? \\[0pt]
Désire-t-on un autre que soi ? \\[0pt]
Désir et politique \\[0pt]
Désir et société \\[0pt]
Désir et temps \\[0pt]
Désir et volonté \\[0pt]
Des motivations peuvent-elles être sociales ? \\[0pt]
Des mythes peuvent-ils encore apparaître ? \\[0pt]
Des nations peuvent-elles former une société ? \\[0pt]
Désobéir \\[0pt]
Désobéir aux lois \\[0pt]
Désobéissance et résistance \\[0pt]
Désordre et chaos \\[0pt]
Des peuples sans histoire \\[0pt]
Des sciences molles ? \\[0pt]
Dessiner \\[0pt]
Des sociétés sans État sont-elles des sociétés politiques ? \\[0pt]
Des sociétés sans histoire \\[0pt]
Des statistiques et des hommes \\[0pt]
Des structures et des hommes \\[0pt]
Détermination, déterminisme, liberté \\[0pt]
Déterminer \\[0pt]
Déterminisme naturel et déterminisme social \\[0pt]
Déterminisme psychique et déterminisme physique \\[0pt]
Détruire pour reconstruire \\[0pt]
Deux et deux font quatre \\[0pt]
Devant qui sommes-nous responsables ? \\[0pt]
Devenir \\[0pt]
Devenir autre \\[0pt]
Devenir ce que l'on est \\[0pt]
Devenir citoyen \\[0pt]
Devenir et évolution \\[0pt]
« Deviens ce que tu es » \\[0pt]
« Deviens qui tu es » \\[0pt]
Devient-on raisonnable ? \\[0pt]
Devoir et utilité \\[0pt]
Devoir mourir \\[0pt]
Devons-nous aimer notre destin ? \\[0pt]
Devons-nous douter de l'existence des choses ? \\[0pt]
Devons-nous nous faire confiance ? \\[0pt]
Diachronie et synchronie \\[0pt]
Dialectique et Philosophie \\[0pt]
Dialectique et vérité \\[0pt]
Dialoguer \\[0pt]
Dieu aurait-il pu mieux faire ? \\[0pt]
Dieu des philosophes et Dieu des croyants \\[0pt]
Dieu est-il l'être ? \\[0pt]
Dieu est-il un concept ? \\[0pt]
Dieu est-il une limite de la pensée ? \\[0pt]
« Dieu est mort » \\[0pt]
Dieu est mort \\[0pt]
Dieu et César \\[0pt]
Dieu et l'âme \\[0pt]
Dieu et le monde \\[0pt]
Dieu, l'âme, le monde \\[0pt]
Dieu pense-t-il ? \\[0pt]
Dieu peut-il tout faire ? \\[0pt]
Dieu, prouvé ou éprouvé ? \\[0pt]
Différer \\[0pt]
Dire ce qui est \\[0pt]
Dire, est-ce faire ? \\[0pt]
Dire et faire \\[0pt]
Dire et montrer \\[0pt]
Dire « je » \\[0pt]
Dire la vérité \\[0pt]
Dire la vérité, est-ce un devoir ? \\[0pt]
Dire le monde \\[0pt]
Dire oui \\[0pt]
Dire que l'être humain est un animal, est-ce méconnaître sa dignité ? \\[0pt]
Diriger son esprit \\[0pt]
Discerner et juger \\[0pt]
Discret et continu \\[0pt]
Discussion et dialogue \\[0pt]
Discuter de la beauté d'une chose, est-ce discuter sur une réalité ? \\[0pt]
Disposer de son corps \\[0pt]
Distinguer \\[0pt]
Diversité des sciences humaines et unité de l'homme \\[0pt]
Diversité ou universalité \\[0pt]
Diviser pour mieux régner \\[0pt]
Division du travail et cohésion sociale \\[0pt]
Document, archive, témoignage \\[0pt]
Documents et monuments \\[0pt]
Doit-on bien juger pour bien faire ? \\[0pt]
Doit-on cesser de chercher à définir l'œuvre d'art ? \\[0pt]
Doit-on déplorer le désaccord ? \\[0pt]
Doit-on préférer l'injustice au désordre ? \\[0pt]
Doit-on répondre de ce qu'on est devenu ? \\[0pt]
Doit-on respecter la nature ? \\[0pt]
Doit-on se faire l'avocat du diable ? \\[0pt]
Doit-on toujours dire la vérité ? \\[0pt]
Doit-on tout calculer ? \\[0pt]
Don, échange, potlatch \\[0pt]
Donner \\[0pt]
Donner des exemples \\[0pt]
Donner des raisons \\[0pt]
Donner du sens \\[0pt]
Donner raison \\[0pt]
Donner raison, rendre raison \\[0pt]
Donner sa parole \\[0pt]
Donner sens \\[0pt]
Donner une représentation \\[0pt]
Donner un exemple \\[0pt]
D'où la politique tire-t-elle sa légitimité ? \\[0pt]
Douter \\[0pt]
Douter de tout \\[0pt]
D'où viennent les idées générales ? \\[0pt]
D'où vient aux objets techniques leur beauté ? \\[0pt]
D'où vient la certitude dans les sciences ? \\[0pt]
D'où vient la signification des mots ? \\[0pt]
D'où vient le mal ? \\[0pt]
D'où vient le plaisir de lire ? \\[0pt]
D'où vient que l'histoire soit autre chose qu'un chaos ? \\[0pt]
Droit et coutume \\[0pt]
Droit et morale \\[0pt]
Droit et nature \\[0pt]
Droit naturel et loi naturelle \\[0pt]
Droits de l'homme, droits du citoyen \\[0pt]
Droits de l'homme et droits du citoyen \\[0pt]
Droits et devoirs \\[0pt]
Droits et devoirs sont-ils réciproques ? \\[0pt]
Droits et obligations \\[0pt]
Du penser à l'être, la conséquence est-elle bonne ? \\[0pt]
Durer \\[0pt]
Échange et don \\[0pt]
Éclairer \\[0pt]
Économie et politique \\[0pt]
Économie et prédiction \\[0pt]
Économie et société \\[0pt]
Économie politique et politique économique \\[0pt]
Écouter \\[0pt]
Écrire \\[0pt]
Écrire l'histoire \\[0pt]
Écrire l'histoire, est-ce la faire ? \\[0pt]
Éducation de l'homme, éducation du citoyen \\[0pt]
Éducation et citoyenneté \\[0pt]
Éduquer \\[0pt]
Éduquer à la liberté \\[0pt]
Éduquer et instruire \\[0pt]
Éduquer le citoyen \\[0pt]
Éduquer le désir \\[0pt]
Éduquer le goût \\[0pt]
Égalité des droits, égalité des conditions \\[0pt]
Égalité et identité \\[0pt]
Égalité et liberté \\[0pt]
Égoïsme et altruisme \\[0pt]
Égoïsme et individualisme \\[0pt]
Égoïsme et méchanceté \\[0pt]
Élite et démocratie \\[0pt]
Empathie et sympathie \\[0pt]
Empirique et expérimental \\[0pt]
Empirique et transcendantal \\[0pt]
En apparence \\[0pt]
Enfance et moralité \\[0pt]
En mathématiques, la notion d'existence a-t-elle un sens ? \\[0pt]
En morale, y a-t-il quelque chose à savoir ? \\[0pt]
En politique, ne fait-on que défendre ses intérêts ? \\[0pt]
En politique n'y a-t-il que des rapports de force ? \\[0pt]
En politique, peut-on faire table rase du passé ? \\[0pt]
En politique, y a-t-il des modèles ? \\[0pt]
En quel sens la métaphysique a-t-elle une histoire ? \\[0pt]
En quel sens la métaphysique est-elle une science ? \\[0pt]
En quel sens l'anthropologie peut-elle être historique ? \\[0pt]
En quel sens le vécu est-il l'objet des sciences humaines ? \\[0pt]
En quel sens parler de structure métaphysique ? \\[0pt]
En quel sens peut-on dire de l'écologie qu'elle est politique ? \\[0pt]
En quel sens peut-on parler de la vie sociale comme d'un jeu ? \\[0pt]
En quel sens peut-on parler de transcendance ? \\[0pt]
En quel sens une œuvre d'art est-elle un document ? \\[0pt]
En quels sens les corps peuvent-ils être objets de la métaphysique ? \\[0pt]
Enquêter \\[0pt]
En quoi la connaissance de la matière peut-elle relever de la métaphysique ? \\[0pt]
En quoi la linguistique est-elle une science ? \\[0pt]
En quoi la matière s'oppose-t-elle à l'esprit ? \\[0pt]
En quoi l'animal intéresse-t-il les sciences humaines ? \\[0pt]
En quoi la sexualité peut-elle être objet de science ? \\[0pt]
En quoi la technique fait-elle question ? \\[0pt]
En quoi le langage est-il objet de science ? \\[0pt]
En quoi les sciences humaines nous éclairent-elles sur la barbarie ? \\[0pt]
En quoi les sciences humaines sont-elles normatives ? \\[0pt]
En quoi l'histoire est-elle une science du réel ? \\[0pt]
En quoi l'œuvre d'art donne-t-elle à penser ? \\[0pt]
En quoi une discussion est-elle politique ? \\[0pt]
En quoi une insulte est-elle blessante ? \\[0pt]
Enseigner \\[0pt]
Enseigner, instruire, éduquer \\[0pt]
Enseigner l'art \\[0pt]
Entendement et raison \\[0pt]
Entendre \\[0pt]
Entendre raison \\[0pt]
Entité et identité \\[0pt]
Entre l'utile et l'honnête, peut-on choisir ? \\[0pt]
Entrer en scène \\[0pt]
Énumérer \\[0pt]
Épistémologie et psychologie \\[0pt]
Épistémologie générale et épistémologie des sciences particulières \\[0pt]
Éprouver \\[0pt]
Éprouver sa valeur \\[0pt]
Erreur et illusion \\[0pt]
Erreur, illusion, hallucination \\[0pt]
Espace et réalité \\[0pt]
Espace et structure sociale \\[0pt]
Espace mathématique et espace physique \\[0pt]
Espace privé, espace public \\[0pt]
Espace public et vie privée \\[0pt]
Espace vécu, espace géométrique \\[0pt]
Espèce et genre \\[0pt]
Essayer \\[0pt]
Essence et existence \\[0pt]
Essence et nature \\[0pt]
Est beau ce qui ne sert à rien \\[0pt]
Est-ce par son objet ou par ses méthodes qu'une science peut se définir ? \\[0pt]
Est-ce pour des raisons morales qu'il faut protéger l'environnement ? \\[0pt]
Esthétique et poétique \\[0pt]
Esthétique et politique \\[0pt]
Esthétisme et moralité \\[0pt]
Est-il bon qu'un seul commande ? \\[0pt]
Est-il difficile de savoir ce que l'on veut ? \\[0pt]
Est-il difficile d'être heureux ? \\[0pt]
Est-il impossible de moraliser la politique ? \\[0pt]
Est-il judicieux de revenir sur ses décisions ? \\[0pt]
Est-il mauvais de suivre son désir ? \\[0pt]
Est-il parfois bon de mentir ? \\[0pt]
Est-il possible de croire en la vie éternelle ? \\[0pt]
Est-il possible de douter de tout ? \\[0pt]
Est-il possible de ne croire à rien ? \\[0pt]
Est-il possible d'être neutre politiquement ? \\[0pt]
Est-il toujours illicite d'inférer ce qui doit être à partir de ce qui est ? \\[0pt]
Est-il vrai qu'en science, « rien n'est donné, tout est construit » ? \\[0pt]
Est-il vrai qu'on apprenne de ses erreurs ? \\[0pt]
Estimer \\[0pt]
Est-on fondé à distinguer la justice et le droit ? \\[0pt]
Est-on fondé à parler d'une imperfection du langage ? \\[0pt]
Est-on le produit d'une culture ? \\[0pt]
Est-on propriétaire de son corps ? \\[0pt]
Est-on responsable de ce qu'on n'a pas voulu ? \\[0pt]
Est-on responsable de l'avenir de l'humanité \\[0pt]
Est-on responsable de ses rêves ? \\[0pt]
Est-on responsable de son passé ? \\[0pt]
Établir les faits \\[0pt]
État et nation \\[0pt]
État et société \\[0pt]
État et vie privée \\[0pt]
État, peuple, nation \\[0pt]
Éternité et immortalité \\[0pt]
Éternité et perpétuité \\[0pt]
Éthique et authenticité \\[0pt]
Ethnologie et cinéma \\[0pt]
Ethnologie et ethnocentrisme \\[0pt]
Ethnologie et sociologie \\[0pt]
Être absolument libre \\[0pt]
Être acteur \\[0pt]
Être adulte \\[0pt]
Être affairé \\[0pt]
Être apolitique \\[0pt]
Être asocial \\[0pt]
Être bien élevé \\[0pt]
Être bon juge \\[0pt]
Être cause de soi \\[0pt]
Être certain \\[0pt]
Être, c'est agir \\[0pt]
Être chez soi \\[0pt]
Être citoyen du monde \\[0pt]
Être civilisé \\[0pt]
Être compris \\[0pt]
Être conformiste \\[0pt]
Être content de soi \\[0pt]
« Être contre » \\[0pt]
Être dans l'esprit \\[0pt]
Être dans le temps \\[0pt]
Être dans le vrai \\[0pt]
Être dans l'opposition \\[0pt]
Être dans son bon droit \\[0pt]
Être dans son droit \\[0pt]
Être de mauvaise humeur \\[0pt]
Être démocrate \\[0pt]
Être de son temps \\[0pt]
Être déterminé \\[0pt]
Être égal à soi-même \\[0pt]
Être égaux \\[0pt]
Être en bonne santé \\[0pt]
Être en désaccord \\[0pt]
Être en dette \\[0pt]
Être en puissance \\[0pt]
Être en règle avec soi-même \\[0pt]
Être ensemble \\[0pt]
Être, est-ce agir ? \\[0pt]
Être et avoir \\[0pt]
Être et devenir \\[0pt]
Être et devoir être \\[0pt]
Être et être pensé \\[0pt]
Être et être perçu \\[0pt]
Être et exister \\[0pt]
Être et ne plus être \\[0pt]
Être et paraître \\[0pt]
Être et représentation \\[0pt]
Être et sens \\[0pt]
Être exemplaire \\[0pt]
Être fidèle \\[0pt]
Être hors de soi \\[0pt]
Être hors la loi \\[0pt]
Être hors-la-loi \\[0pt]
Être identique \\[0pt]
Être impossible \\[0pt]
Être inspiré \\[0pt]
Être juge et partie \\[0pt]
Être l'entrepreneur de soi-même \\[0pt]
Être logique \\[0pt]
Être maître de soi \\[0pt]
Être malade \\[0pt]
Être matérialiste \\[0pt]
Être méchant \\[0pt]
Être méchant volontairement \\[0pt]
Être mère \\[0pt]
Être mortel \\[0pt]
Être naturel \\[0pt]
Être né \\[0pt]
Être normal \\[0pt]
Être par soi \\[0pt]
Être par soi / être par autre chose \\[0pt]
Être passionné \\[0pt]
Être patient \\[0pt]
Être pauvre \\[0pt]
Être père \\[0pt]
Être poli \\[0pt]
Être politique \\[0pt]
Être possible \\[0pt]
Être pragmatique \\[0pt]
Être présent \\[0pt]
Être réaliste \\[0pt]
Être riche \\[0pt]
Être sans cause \\[0pt]
Être sans cœur \\[0pt]
Être sans foi ni loi \\[0pt]
Être sans scrupule \\[0pt]
Être sceptique \\[0pt]
« Être » se dit-il en plusieurs sens ? \\[0pt]
Être sensible \\[0pt]
Être seul avec sa conscience \\[0pt]
Être seul avec soi-même \\[0pt]
Être soi-même \\[0pt]
Être solidaire \\[0pt]
Être spirituel \\[0pt]
Être sujet \\[0pt]
Être systématique \\[0pt]
Être un artiste \\[0pt]
Être un corps \\[0pt]
Être une chose qui pense \\[0pt]
Être une personne \\[0pt]
Être, vie et pensée \\[0pt]
Étudier \\[0pt]
Évaluer \\[0pt]
Évaluer l'art \\[0pt]
Événement et structure \\[0pt]
Évidence et certitude \\[0pt]
Évolution et progrès \\[0pt]
Exactitude et objectivité \\[0pt]
Existe-il des mentalités primitives ? \\[0pt]
Existence et contingence \\[0pt]
Existence et essence \\[0pt]
Existence et transcendance \\[0pt]
Exister \\[0pt]
Existe-t-il dans le monde des réalités identiques ? \\[0pt]
Existe-t-il des autorités en matière d'art ? \\[0pt]
Existe-t-il des connaissances a priori ? \\[0pt]
Existe-t-il des degrés de vérité ? \\[0pt]
Existe-t-il des dilemmes moraux ? \\[0pt]
Existe-t-il des émotions proprement esthétiques ? \\[0pt]
Existe-t-il des états mentaux collectifs ? \\[0pt]
Existe-t-il des expériences métaphysiques ? \\[0pt]
Existe-t-il des identités collectives ? \\[0pt]
Existe-t-il des intuitions métaphysiques ? \\[0pt]
Existe-t-il des langages naturels ? \\[0pt]
Existe-t-il des mondes sociaux ? \\[0pt]
Existe-t-il des paroles vraies ? \\[0pt]
Existe-t-il des preuves irréfutables ? \\[0pt]
Existe-t-il des principes premiers ? \\[0pt]
Existe-t-il des problèmes insolubles ? \\[0pt]
Existe-t-il des questions sans réponse ? \\[0pt]
Existe-t-il des sciences de différentes natures ? \\[0pt]
Existe-t-il des sujets collectifs ? \\[0pt]
Existe-t-il des violences légitimes ? \\[0pt]
Existe-t-il différentes sortes de sciences ? \\[0pt]
Existe-t-il plusieurs déterminismes ? \\[0pt]
Existe-t-il plusieurs mondes ? \\[0pt]
Existe-t-il un bien commun qui soit la norme de la vie politique ? \\[0pt]
Existe-t-il un bien souverain ? \\[0pt]
Existe-t-il une autorité naturelle ? \\[0pt]
Existe-t-il une hiérarchie des êtres ? \\[0pt]
Existe-t-il une histoire du présent ? \\[0pt]
Existe-t-il une opinion publique ? \\[0pt]
Existe-t-il une réalité subjective ? \\[0pt]
Existe-t-il une réalité symbolique ? \\[0pt]
Existe-t-il une science de l'être ? \\[0pt]
Existe-t-il une sensibilité philosophique ? \\[0pt]
Existe-t-il une unité des arts ? \\[0pt]
Existe-t-il une violence symbolique ? \\[0pt]
Expérience esthétique et sens commun \\[0pt]
Expérience et approximation \\[0pt]
Expérience et expérimentation \\[0pt]
Expérimenter \\[0pt]
Explication causale, explication génétique dans les sciences humaines \\[0pt]
Explication et compréhension \\[0pt]
Explication et prévision \\[0pt]
Expliquer \\[0pt]
Expliquer, est-ce justifier ? \\[0pt]
Expliquer et comprendre \\[0pt]
Expliquer et interpréter \\[0pt]
Expliquer et justifier \\[0pt]
« Expliquer les faits sociaux par des faits sociaux » \\[0pt]
Expression et création \\[0pt]
Expression et signification \\[0pt]
Extension et compréhension \\[0pt]
Faire ce qu'on dit \\[0pt]
Faire comme les autres \\[0pt]
Faire comme si \\[0pt]
Faire corps \\[0pt]
Faire croire \\[0pt]
Faire de la métaphysique, est-ce se détourner du monde ? \\[0pt]
Faire de la politique \\[0pt]
Faire de nécessité vertu \\[0pt]
Faire de sa vie une œuvre d'art \\[0pt]
Faire des choix \\[0pt]
Faire école \\[0pt]
Faire est-il nécessairement savoir faire ? \\[0pt]
Faire et agir \\[0pt]
Faire et laisser faire \\[0pt]
Faire justice \\[0pt]
Faire la loi \\[0pt]
Faire la morale \\[0pt]
Faire la paix \\[0pt]
Faire la révolution \\[0pt]
Faire la vérité \\[0pt]
Faire le bien \\[0pt]
Faire le mal \\[0pt]
Faire l'histoire \\[0pt]
Faire sa vie \\[0pt]
Faire ses preuves \\[0pt]
Faire société \\[0pt]
Faire son possible \\[0pt]
Faire table rase \\[0pt]
Faire une expérience \\[0pt]
Fait et essence \\[0pt]
Fait et valeur \\[0pt]
Faits et valeurs \\[0pt]
Falsifier \\[0pt]
Famille et tribu \\[0pt]
Faut-il aimer la vie ? \\[0pt]
Faut-il aimer son prochain comme soi-même ? \\[0pt]
Faut-il aller au-delà des apparences ? \\[0pt]
Faut-il avoir des ennemis ? \\[0pt]
Faut-il avoir des principes ? \\[0pt]
Faut-il avoir peur de la liberté ? \\[0pt]
Faut-il avoir peur de l'avenir ? \\[0pt]
Faut-il avoir peur de la vérité ? \\[0pt]
Faut-il avoir peur des contradictions ? \\[0pt]
Faut-il avoir peur des habitudes ? \\[0pt]
Faut-il chasser les poètes ? \\[0pt]
Faut-il combattre ses illusions ? \\[0pt]
Faut-il concilier les contraires ? \\[0pt]
Faut-il condamner la rhétorique ? \\[0pt]
Faut-il condamner les illusions ? \\[0pt]
Faut-il connaître l'histoire des sciences ? \\[0pt]
Faut-il considérer le droit pénal comme instituant une violence légitime ? \\[0pt]
Faut-il considérer les faits sociaux comme des choses ? \\[0pt]
Faut-il craindre la démocratie ? \\[0pt]
Faut-il craindre la révolution ? \\[0pt]
Faut-il craindre le pire ? \\[0pt]
Faut-il craindre le pouvoir souverain ? \\[0pt]
Faut-il craindre les foules ? \\[0pt]
Faut-il craindre les masses ? \\[0pt]
Faut-il croire au progrès ? \\[0pt]
Faut-il croire en quelque chose ? \\[0pt]
Faut-il des techniques du raisonnement ? \\[0pt]
Faut-il détruire l'État ? \\[0pt]
Faut-il diriger l'économie ? \\[0pt]
Faut-il distinguer ce qui est de ce qui doit être ? \\[0pt]
Faut-il distinguer esthétique et philosophie de l'art ? \\[0pt]
Faut-il enfermer ? \\[0pt]
Faut-il enfermer les œuvres dans les musées ? \\[0pt]
Faut-il en finir avec l'esthétique ? \\[0pt]
Faut-il être bon ? \\[0pt]
Faut-il être cosmopolite ? \\[0pt]
Faut-il être discipliné ? \\[0pt]
Faut-il être heureux ? \\[0pt]
Faut-il être mesuré ? \\[0pt]
Faut-il être mesuré en toutes choses ? \\[0pt]
Faut-il être naturel ? \\[0pt]
Faut-il être objectif ? \\[0pt]
Faut-il être réaliste en politique ? \\[0pt]
Faut-il expliquer la morale par son utilité ? \\[0pt]
Faut-il faire de la philosophie ? \\[0pt]
Faut-il faire de sa vie une œuvre d'art ? \\[0pt]
Faut-il fuir la politique ? \\[0pt]
Faut-il laisser parler la nature ? \\[0pt]
Faut-il limiter la souveraineté ? \\[0pt]
Faut-il limiter l'exercice de la puissance publique ? \\[0pt]
Faut-il maîtriser ses émotions ? \\[0pt]
Faut-il mathématiser le réel pour le connaître ? \\[0pt]
Faut-il ménager les apparences ? \\[0pt]
Faut-il mépriser le luxe ? \\[0pt]
Faut-il mieux vivre comme si nous ne devions jamais mourir ? \\[0pt]
Faut-il n'être jamais méchant ? \\[0pt]
Faut-il opposer à la politique la souveraineté du droit ? \\[0pt]
Faut-il opposer éduquer et instruire ? \\[0pt]
Faut-il opposer l'art à la connaissance ? \\[0pt]
Faut-il opposer le réel et l'imaginaire ? \\[0pt]
Faut-il opposer l'esprit et la matière ? \\[0pt]
Faut-il opposer l'État et la société ? \\[0pt]
Faut-il opposer l'histoire et la fiction ? \\[0pt]
Faut-il opposer produire et créer ? \\[0pt]
Faut-il opposer rhétorique et philosophie ? \\[0pt]
Faut-il opposer science et métaphysique ? \\[0pt]
Faut-il opposer sciences humaines et sciences de la nature ? \\[0pt]
Faut-il opposer sens et vérité ? \\[0pt]
Faut-il pardonner ? \\[0pt]
Faut-il parler pour avoir des idées générales ? \\[0pt]
Faut-il penser l'État comme un corps ? \\[0pt]
Faut-il penser l'État comme un individu ? \\[0pt]
Faut-il perdre ses illusions ? \\[0pt]
Faut-il préférer le bonheur à la vérité ? \\[0pt]
Faut-il préférer l'injustice au désordre ? \\[0pt]
Faut-il préférer une injustice au désordre ? \\[0pt]
Faut-il prendre soin de soi ? \\[0pt]
Faut-il rechercher la certitude ? \\[0pt]
Faut-il renoncer à croire ? \\[0pt]
Faut-il renoncer à la distinction entre nature et culture ? \\[0pt]
Faut-il renoncer à l'idée de fondement ? \\[0pt]
Faut-il renoncer à son désir ? \\[0pt]
Faut-il respecter la nature ? \\[0pt]
Faut-il respecter les convenances ? \\[0pt]
Faut-il restaurer les œuvres d'art ? \\[0pt]
Faut-il rompre avec le passé ? \\[0pt]
Faut-il s'adapter aux événements ? \\[0pt]
Faut-il s'aimer pour vivre ensemble ? \\[0pt]
Faut-il s'attendre à l'inattendu ? \\[0pt]
Faut-il savoir ce que l'on veut ? \\[0pt]
Faut-il savoir désobéir ? \\[0pt]
Faut-il se délivrer des passions ? \\[0pt]
Faut-il se fier à la majorité ? \\[0pt]
Faut-il se méfier de l'imagination ? \\[0pt]
Faut-il se méfier des images ? \\[0pt]
Faut-il se méfier du volontarisme politique ? \\[0pt]
Faut-il séparer morale et politique ? \\[0pt]
Faut-il se résigner aux inégalités ? \\[0pt]
Faut-il s'intéresser aux œuvres mineures ? \\[0pt]
Faut-il suivre ses intuitions ? \\[0pt]
Faut-il supposer un énoncé vrai pour en comprendre le sens ? \\[0pt]
Faut-il tolérer les intolérants ? \\[0pt]
Faut-il toujours respecter ses engagements ? \\[0pt]
Faut-il vivre comme si l'on ne devait jamais mourir ? \\[0pt]
Faut-il vivre conformément à la nature ? \\[0pt]
Faut-il voir pour croire ? \\[0pt]
Faut-il vouloir changer le monde ? \\[0pt]
Faut-il vouloir la paix ? \\[0pt]
Féminité et citoyenneté \\[0pt]
Fiction et mensonge \\[0pt]
Fiction et réalité \\[0pt]
Fiction et vérité \\[0pt]
Finalité et causalité \\[0pt]
Fins et moyens \\[0pt]
Foi et raison \\[0pt]
Folie et société \\[0pt]
Fonction et limites des modèles en sciences sociales \\[0pt]
Fonction et prédicat \\[0pt]
Fonder \\[0pt]
Fonder la justice \\[0pt]
Fonder la morale \\[0pt]
Fonder une cité \\[0pt]
Fonder une famille \\[0pt]
Force et violence \\[0pt]
Forger des hypothèses \\[0pt]
Formaliser et axiomatiser \\[0pt]
Forme et matière \\[0pt]
Forme et rythme \\[0pt]
Fuir le monde \\[0pt]
Gagner sa vie \\[0pt]
Garder en mémoire \\[0pt]
Garder la mesure \\[0pt]
Généralité de la règle, contingence des faits \\[0pt]
Généralité et universalité \\[0pt]
Genèse et structure \\[0pt]
Géographie et géométrie \\[0pt]
Gérer et gouverner \\[0pt]
Gestion et politique \\[0pt]
Gouvernement des hommes et administration des choses \\[0pt]
Gouverner \\[0pt]
Gouverner, administrer, gérer \\[0pt]
Gouverner, est-ce dominer ? \\[0pt]
Gouverner, est-ce prévoir ? \\[0pt]
Gouverner et se gouverner \\[0pt]
Gouverner la nature \\[0pt]
Gouverner les passions \\[0pt]
Grammaire et logique \\[0pt]
Grammaire et métaphysique \\[0pt]
Grammaire et philosophie \\[0pt]
Grandeur et décadence \\[0pt]
Grandir \\[0pt]
Grève et politique \\[0pt]
Groupe, classe, société \\[0pt]
Guérir \\[0pt]
Guerre et paix \\[0pt]
Guerre et politique \\[0pt]
Habiter \\[0pt]
Habiter le monde \\[0pt]
Habiter l'espace \\[0pt]
Haïr \\[0pt]
Haïr la raison \\[0pt]
Hasard et destin \\[0pt]
Hériter \\[0pt]
Herméneutique et sciences humaines \\[0pt]
Hiérarchiser les arts \\[0pt]
Histoire des techniques et histoire de l'art \\[0pt]
Histoire et anthropologie \\[0pt]
Histoire et ethnologie \\[0pt]
Histoire et fiction \\[0pt]
Histoire et géographie \\[0pt]
Histoire et géographie font-elles couple ? \\[0pt]
Histoire et mémoire \\[0pt]
Histoire et narration \\[0pt]
Histoire et préhistoire \\[0pt]
Historicité et vérité \\[0pt]
Homo religiosus \\[0pt]
Humanisation et déshumanisation \\[0pt]
Humour et ironie \\[0pt]
Hypothèse et vérité \\[0pt]
Ici et maintenant \\[0pt]
Identité et communauté \\[0pt]
Identité et différence \\[0pt]
Identité et égalité \\[0pt]
Identité et indiscernabilité \\[0pt]
Identité et représentation politique \\[0pt]
Identité et responsabilité \\[0pt]
Idéologie et utopie \\[0pt]
« Il faudrait rester des années entières pour contempler une telle œuvre » \\[0pt]
« Il faut de tout pour faire un monde » \\[0pt]
Il faut de tout pour faire un monde \\[0pt]
Illégalité et injustice \\[0pt]
« Il n'y a pas de fait, il n'y a que des interprétations » \\[0pt]
Il y a \\[0pt]
Image, concept et signe \\[0pt]
Image et idée \\[0pt]
Imaginaire et politique \\[0pt]
Imagination et connaissance \\[0pt]
Imagination et fantasme \\[0pt]
Imagination et perception \\[0pt]
Imaginer \\[0pt]
Imaginer, est-ce créer ? \\[0pt]
Imitation et création \\[0pt]
Imitation et identification \\[0pt]
Imiter, est-ce copier ? \\[0pt]
Immoralité et amoralité \\[0pt]
Immortalité et éternité \\[0pt]
Improviser \\[0pt]
Inconscient et langage \\[0pt]
Incréé / créé \\[0pt]
Indépendance et autonomie \\[0pt]
Indépendance et liberté \\[0pt]
Indépendant / dépendant \\[0pt]
Individualisme et égoïsme \\[0pt]
Individuation et identité \\[0pt]
Individu et société \\[0pt]
Inégalités et différences \\[0pt]
Inégalités et discriminations \\[0pt]
Infini et indéfini \\[0pt]
Information et communication \\[0pt]
Innocenter le devenir \\[0pt]
Instinct et morale \\[0pt]
Instruire et éduquer \\[0pt]
Intégration et assimilation \\[0pt]
Intégration et exclusion \\[0pt]
Interdire et prohiber \\[0pt]
Intériorité et extériorité \\[0pt]
Interprétation et scientificité \\[0pt]
Interpréter \\[0pt]
Interpréter, est-ce un art ? \\[0pt]
Interpréter et expliquer \\[0pt]
Interpréter et formaliser dans les sciences humaines \\[0pt]
Interpréter une œuvre d'art \\[0pt]
Intuition et concept \\[0pt]
Intuition et déduction \\[0pt]
Invariance et pluralité des langues \\[0pt]
Invention et création \\[0pt]
« J'ai le droit » \\[0pt]
J'ai un corps \\[0pt]
Je mens \\[0pt]
« Je n'ai pas voulu cela » \\[0pt]
Je ne l'ai pas fait exprès \\[0pt]
« Je ne voulais pas cela » : en quoi les sciences humaines permettent-elles de comprendre cette excuse ? \\[0pt]
Je sens, donc je suis \\[0pt]
Je, tu, il \\[0pt]
Jouer \\[0pt]
Jouer son rôle \\[0pt]
Jouer un rôle \\[0pt]
Jouir sans entraves \\[0pt]
Jugement analytique et jugement synthétique \\[0pt]
Jugement esthétique et jugement de valeur \\[0pt]
Jugement esthétique et objectivité \\[0pt]
Jugement et proposition \\[0pt]
Juger \\[0pt]
Juger en conscience \\[0pt]
Juger et raisonner \\[0pt]
Juge-t-on un acte ou son auteur ? \\[0pt]
Jusqu'à quel point pouvons-nous juger autrui ? \\[0pt]
Jusqu'où peut-on soigner ? \\[0pt]
Justice et bonheur \\[0pt]
Justice et égalité \\[0pt]
Justice et équité \\[0pt]
Justice et fiscalité \\[0pt]
Justice et vengeance \\[0pt]
Justice et violence \\[0pt]
Justice sociale et charité publique \\[0pt]
Justifier \\[0pt]
Justifier et prouver \\[0pt]
La banalité \\[0pt]
L'abandon \\[0pt]
La barbarie \\[0pt]
La bassesse \\[0pt]
La béatitude \\[0pt]
La beauté \\[0pt]
La beauté a-t-elle une histoire ? \\[0pt]
La beauté comporte-t-elle des degrés ? \\[0pt]
La beauté de la nature \\[0pt]
La beauté demande-t-elle des preuves ? \\[0pt]
La beauté des corps \\[0pt]
La beauté des ruines \\[0pt]
La beauté des villes \\[0pt]
La beauté du diable \\[0pt]
La beauté du geste \\[0pt]
La beauté du monde \\[0pt]
La beauté du vrai \\[0pt]
La beauté est-elle dans le regard ou dans la chose vue ? \\[0pt]
La beauté est-elle l'objet d'une connaissance ? \\[0pt]
La beauté est-elle partout ? \\[0pt]
La beauté est-elle sensible ? \\[0pt]
La beauté est-elle une promesse de bonheur ? \\[0pt]
La beauté et la grâce \\[0pt]
La beauté et l'ennui \\[0pt]
La beauté idéale \\[0pt]
La beauté morale \\[0pt]
La beauté naturelle \\[0pt]
La beauté naturelle est-elle une catégorie esthétique périmée ? \\[0pt]
La beauté n'est-elle qu'un effet artistique ? \\[0pt]
La beauté n'est-elle qu'une idée ? \\[0pt]
La beauté peut-elle délivrer une vérité ? \\[0pt]
La beauté peut-elle être cachée ? \\[0pt]
La belle âme \\[0pt]
La belle nature \\[0pt]
La bestialité \\[0pt]
La bêtise \\[0pt]
La bêtise n'est-elle pas proprement humaine ? \\[0pt]
La bibliothèque \\[0pt]
La bienfaisance \\[0pt]
La bienveillance \\[0pt]
La bioéthique \\[0pt]
La biologie peut-elle se passer de causes finales ? \\[0pt]
L'abnégation \\[0pt]
L'abondance \\[0pt]
La bonne conscience \\[0pt]
La bonne volonté \\[0pt]
L'absence \\[0pt]
L'absence de fondement \\[0pt]
L'absence d'œuvre \\[0pt]
L'absolu \\[0pt]
L'absolu est-il connaissable ? \\[0pt]
L'Absolu est-il un objet de savoir ? \\[0pt]
L'abstention \\[0pt]
L'abstraction \\[0pt]
L'abstraction en art \\[0pt]
L'abstrait est-il en dehors de l'espace et du temps ? \\[0pt]
L'abstrait et le concret \\[0pt]
L'abstrait et l'immatériel \\[0pt]
L'absurde \\[0pt]
L'absurde peut-il avoir un sens pour une pensée 155 \\[0pt]
L'abus de langage \\[0pt]
L'abus de pouvoir \\[0pt]
L'académisme \\[0pt]
L'académisme dans l'art \\[0pt]
La calomnie \\[0pt]
La caricature \\[0pt]
La carte et le territoire \\[0pt]
La casuistique \\[0pt]
La catastrophe \\[0pt]
La catharsis \\[0pt]
La causalité \\[0pt]
La causalité dans les sciences humaines \\[0pt]
La causalité en histoire \\[0pt]
La causalité en sciences humaines \\[0pt]
La causalité historique \\[0pt]
La causalité suppose-t-elle des lois ? \\[0pt]
La cause \\[0pt]
La cause de soi \\[0pt]
La cause efficiente \\[0pt]
La cause et la raison \\[0pt]
La cause finale \\[0pt]
La cause formelle \\[0pt]
La cause motrice \\[0pt]
La cause première \\[0pt]
L'accident \\[0pt]
L'accidentel \\[0pt]
L'accomplissement \\[0pt]
L'accord \\[0pt]
L'accord des esprits \\[0pt]
L'accusation \\[0pt]
La censure \\[0pt]
La certitude \\[0pt]
La chair \\[0pt]
La chair et l'esprit \\[0pt]
La chance \\[0pt]
La charité \\[0pt]
La charité est-elle une vertu ? \\[0pt]
La chasse et la guerre \\[0pt]
L'achèvement de l'œuvre \\[0pt]
La chose \\[0pt]
La chose en soi \\[0pt]
La chose et l'objet \\[0pt]
La chose publique \\[0pt]
La chronologie \\[0pt]
La circonspection \\[0pt]
La circonstance \\[0pt]
La citation \\[0pt]
La cité idéale \\[0pt]
La citoyenneté \\[0pt]
La civilisation \\[0pt]
La civilité \\[0pt]
La clarté \\[0pt]
La clarté suffit-elle au savoir ? \\[0pt]
La classification \\[0pt]
La classification des arts \\[0pt]
La classification des sciences \\[0pt]
La clause de conscience \\[0pt]
La clémence \\[0pt]
La cohérence des attributs divins \\[0pt]
La cohérence est-elle un critère de la vérité ? \\[0pt]
La cohésion nationale \\[0pt]
La cohésion sociale \\[0pt]
La colère \\[0pt]
La comédie du pouvoir \\[0pt]
La comédie humaine \\[0pt]
La comédie sociale \\[0pt]
La communauté des individus \\[0pt]
La communauté des savants \\[0pt]
La communauté internationale \\[0pt]
La communauté morale \\[0pt]
La communauté scientifique \\[0pt]
La communication \\[0pt]
La communication est-elle nécessaire à la démocratie ? \\[0pt]
La comparaison \\[0pt]
La comparaison en sociologie \\[0pt]
La compassion \\[0pt]
La compassion risque-t-elle d'abolir l'exigence politique ? \\[0pt]
La compétence \\[0pt]
La compétence fonde-t-elle la compétence juridique ? \\[0pt]
La compétence fonde-t-elle l'autorité politique ? \\[0pt]
La compétence politique \\[0pt]
La compétence technique peut-elle fonder l'autorité publique ? \\[0pt]
La composition \\[0pt]
La compréhension \\[0pt]
La concorde \\[0pt]
La concurrence \\[0pt]
La condition \\[0pt]
La condition humaine \\[0pt]
La condition sociale \\[0pt]
La confiance \\[0pt]
La conflictualité politique est-elle indépassable ? \\[0pt]
La confusion \\[0pt]
La connaissance adéquate \\[0pt]
La connaissance animale \\[0pt]
La connaissance a-t-elle des limites ? \\[0pt]
La connaissance commune est-elle le point de départ de la science ? \\[0pt]
La connaissance de l'amour \\[0pt]
La connaissance de la nécessité a priori peut-elle évoluer ? \\[0pt]
La connaissance de l'animal \\[0pt]
La connaissance de la vie \\[0pt]
La connaissance de l'histoire nous rend-elle plus libres ? \\[0pt]
La connaissance de l'individuel \\[0pt]
La connaissance de l'infini \\[0pt]
La connaissance des causes \\[0pt]
La connaissance de soi \\[0pt]
La connaissance des principes \\[0pt]
La connaissance du futur \\[0pt]
La connaissance du singulier \\[0pt]
La connaissance du singulier dans les sciences humaines \\[0pt]
La connaissance du vivant \\[0pt]
La connaissance est-elle une croyance justifiée ? \\[0pt]
La connaissance mathématique \\[0pt]
La connaissance objective \\[0pt]
La connaissance scientifique abolit-elle toute croyance ? \\[0pt]
La connaissance scientifique n'est-elle qu'une croyance argumentée ? \\[0pt]
La connaissance suppose-t-elle une éthique ? \\[0pt]
La connexion des choses et la connexion des idées \\[0pt]
La conquête \\[0pt]
La conquête de l'espace \\[0pt]
La conscience de soi \\[0pt]
La conscience de soi de l'art \\[0pt]
La conscience est-elle intrinsèquement morale ? \\[0pt]
La conscience est-elle le produit de la vie réelle ? \\[0pt]
La conscience est-elle temporelle ? \\[0pt]
La conscience historique \\[0pt]
La conscience morale \\[0pt]
La conscience morale est-elle innée ? \\[0pt]
La conscience peut-elle être collective ? \\[0pt]
La conscience politique \\[0pt]
La conséquence \\[0pt]
La conservation de soi \\[0pt]
La considération \\[0pt]
La consolation \\[0pt]
La consommation \\[0pt]
La constance \\[0pt]
La constitution \\[0pt]
La contemplation \\[0pt]
La contestation \\[0pt]
La contingence \\[0pt]
La contingence des lois de la nature \\[0pt]
La contingence du futur \\[0pt]
La contingence du mal \\[0pt]
La continuité \\[0pt]
La contradiction \\[0pt]
La contradiction réside-t-elle dans les choses ? \\[0pt]
La contrainte \\[0pt]
La contrainte morale \\[0pt]
La contrôle social \\[0pt]
La controverse \\[0pt]
La convention \\[0pt]
La conversation \\[0pt]
La conversion \\[0pt]
La conviction \\[0pt]
La coopération \\[0pt]
La correspondance des arts \\[0pt]
La corruption \\[0pt]
La corruption politique \\[0pt]
La cosmogonie \\[0pt]
La couleur \\[0pt]
La couleur en peinture \\[0pt]
La couleur et le dessin \\[0pt]
La coutume \\[0pt]
L'acquis \\[0pt]
La crainte \\[0pt]
La crainte des Dieux \\[0pt]
La crainte et l'ignorance \\[0pt]
La création \\[0pt]
La création artistique \\[0pt]
La création dans l'art \\[0pt]
La création de l'humanité \\[0pt]
La créativité \\[0pt]
La criminalité \\[0pt]
La crise du sens \\[0pt]
La crise sociale \\[0pt]
La critique \\[0pt]
La critique d'art \\[0pt]
La critique de l'art \\[0pt]
La critique et la crise \\[0pt]
La croissance \\[0pt]
La croyance en Dieu est-elle irrationnelle ? \\[0pt]
La croyance est-elle l'asile de l'ignorance ? \\[0pt]
La cruauté \\[0pt]
La cruauté politique \\[0pt]
L'acte \\[0pt]
L'acteur \\[0pt]
L'acteur et l'observateur dans les sciences humaines \\[0pt]
L'acteur et son rôle \\[0pt]
L'action collective \\[0pt]
L'action du temps \\[0pt]
L'action et la passion \\[0pt]
L'action politique \\[0pt]
L'action politique a-t-elle un fondement rationnel ? \\[0pt]
L'action politique peut-elle se passer de mots ? \\[0pt]
L'action requiert-elle décision d'un sujet ? \\[0pt]
L'activité philosophique conduit-elle à la métaphysique ? \\[0pt]
L'actualité \\[0pt]
L'actuel \\[0pt]
La cuisine \\[0pt]
La culpabilité \\[0pt]
La culture artistique \\[0pt]
La culture de masse \\[0pt]
La culture démocratique \\[0pt]
La culture d'entreprise \\[0pt]
La culture est-elle affaire de politique ? \\[0pt]
La culture est-elle ce qui nous relie au passé ? \\[0pt]
La culture est-elle l'objet naturel des sciences humaines ? \\[0pt]
La culture est-elle nécessaire à l'appréciation d'une œuvre d'art ? \\[0pt]
La culture est-elle un objet ? \\[0pt]
La culture et les cultures \\[0pt]
La culture générale \\[0pt]
La culture morale \\[0pt]
La culture populaire \\[0pt]
La culture scientifique \\[0pt]
La curiosité \\[0pt]
La curiosité est-elle à l'origine du savoir ? \\[0pt]
La danse \\[0pt]
La danse est-elle l'œuvre du corps ? \\[0pt]
La danse et l'espace \\[0pt]
L'adaptation \\[0pt]
La décadence \\[0pt]
La décence \\[0pt]
La déception \\[0pt]
La décision \\[0pt]
La décision morale \\[0pt]
La décision politique \\[0pt]
La déduction \\[0pt]
La défense nationale \\[0pt]
La déficience \\[0pt]
La définition \\[0pt]
La délibération \\[0pt]
La délibération en morale \\[0pt]
La délibération politique \\[0pt]
La démagogie \\[0pt]
La démence \\[0pt]
La démesure \\[0pt]
La démocratie conduit-elle au règne de l'opinion ? \\[0pt]
La démocratie est-elle le pire des régimes politiques ? \\[0pt]
La démocratie est-elle moyen ou fin ? \\[0pt]
La démocratie est-elle nécessairement libérale ? \\[0pt]
La démocratie est-elle possible ? \\[0pt]
La démocratie est-elle un mythe ? \\[0pt]
La démocratie et les experts \\[0pt]
La démocratie n'est-elle que la force des faibles ? \\[0pt]
La démocratie participative \\[0pt]
La démonstration \\[0pt]
La démonstration par l'absurde \\[0pt]
La démonstration suffit-elle à établir la vérité ? \\[0pt]
La déontologie \\[0pt]
La dépendance \\[0pt]
La déraison \\[0pt]
La descendance \\[0pt]
La description \\[0pt]
La désillusion \\[0pt]
La désinvolture \\[0pt]
La désobéissance \\[0pt]
La désobéissance civile \\[0pt]
La destruction \\[0pt]
La détermination \\[0pt]
La dette \\[0pt]
La déviance \\[0pt]
La dialectique \\[0pt]
La dialectique est-elle une science ? \\[0pt]
La dictature \\[0pt]
La différence \\[0pt]
La différence de l'être et de l'étant \\[0pt]
La différence des arts \\[0pt]
La différence des sexes \\[0pt]
La différence sexuelle \\[0pt]
La difformité \\[0pt]
La dignité \\[0pt]
La dignité de l'homme \\[0pt]
La dignité humaine \\[0pt]
La digression \\[0pt]
La discipline \\[0pt]
La discipline de vie \\[0pt]
La discrétion \\[0pt]
La discrimination \\[0pt]
La discussion \\[0pt]
La disposition \\[0pt]
La disposition. Le possible et le réel \\[0pt]
La disposition morale \\[0pt]
La distance \\[0pt]
La distinction \\[0pt]
La distinction de genre \\[0pt]
La distinction de la nature et de la culture est-elle un fait de culture ? \\[0pt]
La distinction du sacré et du profane \\[0pt]
La distinction entre vraie et fausse sciences a-t-elle un sens ? \\[0pt]
La distraction \\[0pt]
La diversion \\[0pt]
La diversité des cultures \\[0pt]
La diversité des langues \\[0pt]
La diversité des perceptions \\[0pt]
La diversité des religions \\[0pt]
La diversité des sciences \\[0pt]
La diversité humaine \\[0pt]
La division \\[0pt]
La division des pouvoirs \\[0pt]
La division des tâches \\[0pt]
La division du travail \\[0pt]
L'admiration \\[0pt]
La docilité est-elle un vice ou une vertu ? \\[0pt]
La domestication \\[0pt]
La domination \\[0pt]
La domination du corps \\[0pt]
La domination sociale \\[0pt]
La douceur \\[0pt]
L'adoucissement des mœurs \\[0pt]
La douleur \\[0pt]
La droit de conquête \\[0pt]
La droiture \\[0pt]
La dualité \\[0pt]
La duplicité \\[0pt]
La faiblesse \\[0pt]
La faiblesse de la démocratie \\[0pt]
La faiblesse de la volonté \\[0pt]
La familiarité \\[0pt]
La famille \\[0pt]
La famille est-elle le lieu de la formation morale ? \\[0pt]
La famille est-elle une invention culturelle ? \\[0pt]
La fatalité \\[0pt]
La fatigue \\[0pt]
La faute \\[0pt]
La faute de goût \\[0pt]
La faute et l'erreur \\[0pt]
La fécondité du raisonnement mathématique \\[0pt]
La fête \\[0pt]
La fête nationale \\[0pt]
L'affection \\[0pt]
La fiction \\[0pt]
La fidélité \\[0pt]
La fidélité à soi \\[0pt]
La figuration \\[0pt]
La fin \\[0pt]
La finalité \\[0pt]
La finalité des sciences humaines \\[0pt]
La fin de la métaphysique \\[0pt]
La fin de la politique \\[0pt]
La fin de la politique est-elle d'éliminer la violence ? \\[0pt]
La fin de la politique est-elle l'établissement de la justice ? \\[0pt]
La fin de l'art \\[0pt]
La fin de la vie \\[0pt]
La fin de l'État \\[0pt]
La fin de l'histoire \\[0pt]
La fin du monde \\[0pt]
La fin et les moyens \\[0pt]
La finitude \\[0pt]
La fin justifie-t-elle les moyens ? \\[0pt]
La folie \\[0pt]
La folie des grandeurs \\[0pt]
La folie et l'étude du psychisme \\[0pt]
La fonction de l'art \\[0pt]
La fonction de penser peut-elle se déléguer ? \\[0pt]
La fonction des exemples \\[0pt]
La fonction des rêves \\[0pt]
La fonction première de l'État est-elle de durer ? \\[0pt]
La fonction sociale de l'écrivain \\[0pt]
La fonction symbolique \\[0pt]
La force \\[0pt]
La force d'âme \\[0pt]
La force de la loi \\[0pt]
La force de la morale \\[0pt]
La force de l'art \\[0pt]
La force de l'habitude \\[0pt]
La force de l'idée \\[0pt]
La force de l'obligation \\[0pt]
La force de l'oubli \\[0pt]
La force des choses \\[0pt]
La force des idées \\[0pt]
La force des institutions \\[0pt]
La force des lois \\[0pt]
La force des objections \\[0pt]
La force du droit \\[0pt]
La force du génie \\[0pt]
La force du pouvoir \\[0pt]
La force du social \\[0pt]
La force est-elle une vertu ? \\[0pt]
La force et le droit \\[0pt]
La force fait-elle le droit ? \\[0pt]
La force publique \\[0pt]
La formation des citoyens \\[0pt]
La formation du goût \\[0pt]
La formation d'une conscience \\[0pt]
La forme \\[0pt]
La forme et la couleur \\[0pt]
La fortune \\[0pt]
La foule \\[0pt]
La fragilité \\[0pt]
La fragilité du beau \\[0pt]
La franchise \\[0pt]
La franchise est-elle une vertu ? \\[0pt]
La fraternité \\[0pt]
La fraternité a-t-elle un sens politique ? \\[0pt]
La fraternité est-elle un idéal moral ? \\[0pt]
La fraude \\[0pt]
La frivolité \\[0pt]
La frontière \\[0pt]
La fuite du temps \\[0pt]
La futilité \\[0pt]
L'âge d'or \\[0pt]
La généalogie \\[0pt]
La généalogie peut-elle être une méthode historique ? \\[0pt]
La générosité \\[0pt]
La genèse \\[0pt]
La genèse de l'œuvre \\[0pt]
L'agent \\[0pt]
La gentillesse \\[0pt]
La géographie \\[0pt]
La géographie est-elle une science humaine ? \\[0pt]
La géométrie \\[0pt]
La géopolitique \\[0pt]
La gloire \\[0pt]
La grâce \\[0pt]
La grammaire \\[0pt]
La grammaire contraint-elle notre pensée ? \\[0pt]
La grammaire et la logique \\[0pt]
La grammaire générative \\[0pt]
La grammaire véhicule-t-elle une métaphysique ? \\[0pt]
La grandeur \\[0pt]
La grandeur d'âme \\[0pt]
La gratitude \\[0pt]
La gratuité \\[0pt]
La gravité \\[0pt]
L'agressivité \\[0pt]
La grève \\[0pt]
L'agriculture \\[0pt]
La guérison \\[0pt]
La guérison en psychanalyse \\[0pt]
La guerre civile \\[0pt]
La guerre est-elle la continuation de la politique ? \\[0pt]
La guerre est-elle la continuation de la politique par d'autres moyens ? \\[0pt]
La guerre est-elle la plus grande violence ? \\[0pt]
La guerre et la paix \\[0pt]
La guerre juste \\[0pt]
La guerre met-elle fin au droit ? \\[0pt]
La guerre peut-elle être juste ? \\[0pt]
La guerre totale \\[0pt]
La haine \\[0pt]
La haine de la pensée \\[0pt]
La haine de la raison \\[0pt]
La haine de la vérité \\[0pt]
La haine des images \\[0pt]
La haine de soi \\[0pt]
La hiérarchie \\[0pt]
La hiérarchie des arts \\[0pt]
La hiérarchie des cultures \\[0pt]
La hiérarchie des énoncés scientifiques \\[0pt]
La hiérarchie des êtres \\[0pt]
La honte \\[0pt]
Laisser faire \\[0pt]
La jalousie \\[0pt]
La jeunesse \\[0pt]
La joie \\[0pt]
La jouissance \\[0pt]
La jurisprudence \\[0pt]
La juste colère \\[0pt]
La juste mesure \\[0pt]
La juste peine \\[0pt]
La justice \\[0pt]
La justice a-t-elle besoin des institutions ? \\[0pt]
La justice a-t-elle pour fonction de compenser les inégalités naturelles ? \\[0pt]
La justice comme avantage mutuel \\[0pt]
La justice consiste-t-elle à traiter tout le monde de la même manière ? \\[0pt]
La justice de l'État \\[0pt]
La justice divine \\[0pt]
La justice entre les générations \\[0pt]
La justice est-elle affaire de raison ? \\[0pt]
La justice est-elle le ciment de la paix sociale ? \\[0pt]
La justice est-elle une notion morale ? \\[0pt]
La justice est-elle une vertu ? \\[0pt]
La justice : moyen ou fin de la politique ? \\[0pt]
La justice peut-elle se passer d'institutions ? \\[0pt]
La justice requiert-elle toujours l'impartialité ? \\[0pt]
La justice sociale \\[0pt]
La justification \\[0pt]
La lâcheté \\[0pt]
La laïcité \\[0pt]
La laideur \\[0pt]
La laideur est-elle une valeur esthétique ? \\[0pt]
La langue maternelle \\[0pt]
La langue nous parle-t-elle ? \\[0pt]
La langue officielle \\[0pt]
L'aléatoire \\[0pt]
La leçon des choses \\[0pt]
La lecture \\[0pt]
La légende \\[0pt]
La légèreté \\[0pt]
La légitimation \\[0pt]
La légitimité \\[0pt]
La légitimité d'un clonage humain ? \\[0pt]
La lettre et l'esprit \\[0pt]
La liaison des idées \\[0pt]
La libération des mœurs \\[0pt]
La liberté a-t-elle une histoire ? \\[0pt]
La liberté civile \\[0pt]
La liberté comporte-t-elle des degrés ? \\[0pt]
La liberté, concept moral ? \\[0pt]
La liberté créatrice \\[0pt]
La liberté de culte \\[0pt]
La liberté de l'artiste \\[0pt]
La liberté de la science \\[0pt]
La liberté de parole \\[0pt]
La liberté de penser \\[0pt]
La liberté des citoyens \\[0pt]
La liberté d'expression \\[0pt]
La liberté d'expression a-t-elle des limites ? \\[0pt]
La liberté d'imaginer \\[0pt]
La liberté d'opinion \\[0pt]
La liberté du citoyen \\[0pt]
La liberté, est-ce l'indépendance à l'égard des passions ? \\[0pt]
La liberté est-elle menacée par l'égalité ? \\[0pt]
La liberté et la grâce \\[0pt]
La liberté et la loi \\[0pt]
La liberté intéresse-t-elle les sciences humaines ? \\[0pt]
La liberté morale \\[0pt]
La liberté politique \\[0pt]
La liberté se prouve-t-elle ? \\[0pt]
La liberté s'éprouve-t-elle ? \\[0pt]
L'aliénation \\[0pt]
La limitation des créatures \\[0pt]
La limite \\[0pt]
La limite du politique \\[0pt]
La linguistique est-elle une science de l'information ? \\[0pt]
La linguistique fait-elle partie des sciences humaines ? \\[0pt]
La linguistique peut-elle se passer de la psychologie ? \\[0pt]
La littérature engagée \\[0pt]
La littérature est-elle la mémoire de l'humanité ? \\[0pt]
La littérature et le mythe \\[0pt]
La littérature peut-elle suppléer les sciences de l'homme ? \\[0pt]
L'allégorie \\[0pt]
La logique a-t-elle une histoire ? \\[0pt]
La logique a-t-elle un intérêt philosophique ? \\[0pt]
La logique : calcul ou langage ? \\[0pt]
La logique : découverte ou invention ? \\[0pt]
La logique décrit-elle le monde ? \\[0pt]
La logique est-elle indépendante de la psychologie ? \\[0pt]
La logique est-elle un art de penser ? \\[0pt]
La logique est-elle un art de raisonner ? \\[0pt]
La logique est-elle une discipline normative ? \\[0pt]
La logique est-elle une forme de calcul ? \\[0pt]
La logique est-elle une science de la vérité ? \\[0pt]
La logique est-elle utile à la métaphysique ? \\[0pt]
La logique et le réel \\[0pt]
La logique nous apprend-elle quelque chose sur le langage ordinaire ? \\[0pt]
« La logique » ou bien « les logiques » ? \\[0pt]
La logique peut-elle se passer de la métaphysique ? \\[0pt]
La logique pourrait-elle nous surprendre ? \\[0pt]
La loi \\[0pt]
La loi du genre \\[0pt]
La loi du marché \\[0pt]
La loi et le contrat \\[0pt]
La loi et le règlement \\[0pt]
La loi peut-elle changer les mœurs ? \\[0pt]
La louange et le blâme \\[0pt]
La loyauté \\[0pt]
L'alter ego \\[0pt]
L'altérité \\[0pt]
L'altérité culturelle \\[0pt]
L'alternative \\[0pt]
L'altruisme \\[0pt]
L'altruisme est-il une attitude morale ? \\[0pt]
La lumière \\[0pt]
La lumière de la vérité \\[0pt]
La lumière naturelle \\[0pt]
La lutte des classes \\[0pt]
La machine \\[0pt]
La magie \\[0pt]
La magnanimité \\[0pt]
La main \\[0pt]
La main et l'outil \\[0pt]
La maîtrise \\[0pt]
La maîtrise de la langue \\[0pt]
La maîtrise de la nature \\[0pt]
La maîtrise de soi \\[0pt]
La maîtrise du feu \\[0pt]
La maîtrise du geste \\[0pt]
La maîtrise du temps \\[0pt]
La majesté \\[0pt]
La majorité \\[0pt]
La majorité a-t-elle toujours raison ? \\[0pt]
La majorité peut-elle être tyrannique ? \\[0pt]
La maladie \\[0pt]
La maladie mentale a-t-elle une histoire ? \\[0pt]
La malchance \\[0pt]
La manière \\[0pt]
La manifestation \\[0pt]
La marchandisation \\[0pt]
La marginalité \\[0pt]
La masse \\[0pt]
L'amateur \\[0pt]
L'amateur d'art \\[0pt]
L'amateurisme \\[0pt]
La mathématique est-elle une ontologie ? \\[0pt]
La matière \\[0pt]
La matière, est-ce l'informe ? \\[0pt]
La matière est-elle une substance ? \\[0pt]
La matière et la forme \\[0pt]
La matière n'est-elle qu'une idée ? \\[0pt]
La matière peut-elle penser ? \\[0pt]
La matière première \\[0pt]
La matière sensible \\[0pt]
La maturité \\[0pt]
La mauvaise conscience \\[0pt]
La mauvaise éducation \\[0pt]
La mauvaise foi \\[0pt]
La mauvaise volonté \\[0pt]
L'ambiguïté \\[0pt]
L'ambition \\[0pt]
L'ambition politique \\[0pt]
L'âme \\[0pt]
La méchanceté \\[0pt]
L'âme concerne-t-elle les sciences humaines ? \\[0pt]
La méconnaissance \\[0pt]
La médecine est-elle une science ? \\[0pt]
L'âme des bêtes \\[0pt]
La médiation \\[0pt]
La médiocrité artistique \\[0pt]
La médisance \\[0pt]
La méditation \\[0pt]
L'âme est-elle immortelle ? \\[0pt]
L'âme et le corps \\[0pt]
La meilleure constitution \\[0pt]
La mélancolie \\[0pt]
L'âme, le monde et Dieu \\[0pt]
La mémoire \\[0pt]
La mémoire collective \\[0pt]
La mémoire et l'individu \\[0pt]
La mémoire sélective \\[0pt]
La menace \\[0pt]
La mesure \\[0pt]
La mesure dans les sciences humaines \\[0pt]
La mesure de l'homme \\[0pt]
La mesure de l'intelligence \\[0pt]
La mesure des choses \\[0pt]
La mesure du temps \\[0pt]
La métaphore \\[0pt]
La métaphysique a-t-elle ses fictions ? \\[0pt]
La métaphysique est-elle affaire de raisonnement ? \\[0pt]
La métaphysique est-elle le fondement de la morale ? \\[0pt]
La métaphysique est-elle nécessairement une réflexion sur Dieu ? \\[0pt]
La métaphysique est-elle une discipline théorique ? \\[0pt]
La métaphysique est-elle une science ? \\[0pt]
La métaphysique peut-elle être autre chose qu'une science recherchée ? \\[0pt]
La métaphysique peut-elle faire appel à l'expérience ? \\[0pt]
La métaphysique procure-t-elle un savoir ? \\[0pt]
La métaphysique relève-t-elle de la philosophie ou de la poésie ? \\[0pt]
La métaphysique répond-elle à un besoin ? \\[0pt]
La métaphysique repose-t-elle sur des croyances ? \\[0pt]
La métaphysique se définit-elle par son objet ou sa démarche ? \\[0pt]
La méthode \\[0pt]
La méthode de la science \\[0pt]
La méthode en sciences sociales \\[0pt]
La méthode hypothético-déductive \\[0pt]
La méthode statistique \\[0pt]
L'ami et l'ennemi \\[0pt]
La minorité \\[0pt]
La misanthropie \\[0pt]
La mise à l'épreuve des hypothèses en sciences humaines et sociales \\[0pt]
La mise en scène \\[0pt]
La misère \\[0pt]
La miséricorde \\[0pt]
La misologie \\[0pt]
L'amitié \\[0pt]
L'amitié est-elle une vertu ? \\[0pt]
L'amitié est-elle un principe politique ? \\[0pt]
La modalité \\[0pt]
La mode \\[0pt]
La modélisation en sciences sociales \\[0pt]
La modération \\[0pt]
La modération est-elle l'essence de la vertu ? \\[0pt]
La modération est-elle une vertu politique ? \\[0pt]
La modernité \\[0pt]
La modernité dans les arts \\[0pt]
La modestie \\[0pt]
La mondialisation \\[0pt]
La monnaie \\[0pt]
La monumentalité \\[0pt]
La morale a-t-elle besoin d'être fondée ? \\[0pt]
La morale a-t-elle besoin d'un au-delà ? \\[0pt]
La morale commune \\[0pt]
La morale consiste-t-elle à suivre la nature ? \\[0pt]
La morale de l'athée \\[0pt]
La morale de l'intérêt \\[0pt]
La morale des fables \\[0pt]
La morale doit-elle en appeler à la nature ? \\[0pt]
La morale doit-elle fournir des préceptes ? \\[0pt]
La morale du citoyen \\[0pt]
La morale du plus fort \\[0pt]
La morale est-elle affaire de jugement ? \\[0pt]
La morale est-elle affaire de sentiment ? \\[0pt]
La morale est-elle affaire de sentiments ? \\[0pt]
La morale est-elle affaire individuelle ? \\[0pt]
La morale est-elle émotion ? \\[0pt]
La morale est-elle ennemie du bonheur ? \\[0pt]
La morale est-elle fondée sur la liberté ? \\[0pt]
La morale est-elle incompatible avec le déterminisme ? \\[0pt]
La morale est-elle l'ennemie de la vie ? \\[0pt]
La morale est-elle nécessairement répressive ? \\[0pt]
La morale est-elle un art de vivre ? \\[0pt]
La morale est-elle un besoin ? \\[0pt]
La morale est-elle une affaire d'habitude ? \\[0pt]
La morale est-elle une médecine de l'âme ? \\[0pt]
La morale est-elle une preuve de la liberté ? \\[0pt]
La morale est-elle une science ? \\[0pt]
La morale est-elle un fait social ? \\[0pt]
La morale et le droit \\[0pt]
La morale peut-elle être fondée sur la science ? \\[0pt]
La morale peut-elle être naturelle ? \\[0pt]
La morale peut-elle être personnelle ? \\[0pt]
La morale peut-elle être un calcul ? \\[0pt]
La morale peut-elle être une science ? \\[0pt]
La morale peut-elle se passer d'un fondement religieux ? \\[0pt]
La morale politique \\[0pt]
La morale requiert-elle une casuistique ? \\[0pt]
La morale requiert-elle un fondement ? \\[0pt]
La morale se déduit-elle de la métaphysique ? \\[0pt]
La morale s'enracine-t-elle dans le sens commun ? \\[0pt]
La morale suppose-t-elle le libre arbitre ? \\[0pt]
La moralité des lois \\[0pt]
La moralité n'est-elle que dressage ? \\[0pt]
La moralité réside-t-elle dans l'intention ? \\[0pt]
La mort dans l'âme \\[0pt]
La mort de l'art \\[0pt]
La mort fait-elle partie de la vie ? \\[0pt]
La motivation et le jugement moral \\[0pt]
L'amour \\[0pt]
L'amour de la liberté \\[0pt]
L'amour de la nature \\[0pt]
L'amour de l'art \\[0pt]
L'amour de l'humanité \\[0pt]
L'amour des lois \\[0pt]
L'amour de soi \\[0pt]
L'amour est-il aveugle ? \\[0pt]
L'amour est-il sans raison ? \\[0pt]
L'amour et la haine \\[0pt]
L'amour et la justice \\[0pt]
L'amour et l'amitié \\[0pt]
L'amour et la mort \\[0pt]
L'amour et la philosophie \\[0pt]
L'amour et l'humanité \\[0pt]
L'amour peut-il être absolu ? \\[0pt]
L'amour-propre \\[0pt]
L'amour vrai \\[0pt]
La multiplicité \\[0pt]
La multiplicité des acceptions de l'étant \\[0pt]
La multitude \\[0pt]
La musique a-t-elle une essence ? \\[0pt]
La musique de film \\[0pt]
La musique donne-t-elle à penser ? \\[0pt]
La musique est-elle un discours ? \\[0pt]
La musique est-elle un langage ? \\[0pt]
La musique et la danse \\[0pt]
La musique et le bruit \\[0pt]
La musique et le temps \\[0pt]
La musique exprime-t-elle quelque chose ? \\[0pt]
La musique peut-elle être narrative ? \\[0pt]
L'anachronisme \\[0pt]
La naissance \\[0pt]
La naissance de l'art \\[0pt]
La naissance de la science \\[0pt]
La naissance de l'homme \\[0pt]
La naïveté \\[0pt]
L'analogie \\[0pt]
L'analyse \\[0pt]
L'analyse des comportements humains permet-elle de réduire le hasard ? \\[0pt]
L'analyse des mythes doit-elle être uniquement linguistique \\[0pt]
L'analyse du vécu \\[0pt]
L'analyse économique rend-elle compte des comportements humains ? \\[0pt]
L'anarchie \\[0pt]
La nation \\[0pt]
La nation est-elle dépassée ? \\[0pt]
La nation et l'État \\[0pt]
La nature \\[0pt]
La nature artiste \\[0pt]
La nature a-t-elle des droits ? \\[0pt]
La nature a-t-elle une histoire ? \\[0pt]
La nature a-t-elle une valeur ? \\[0pt]
La nature des choses \\[0pt]
La nature des objets mathématiques \\[0pt]
La nature du bien \\[0pt]
La nature, est-ce le donné ? \\[0pt]
La nature est-elle artiste ? \\[0pt]
La nature est-elle belle ? \\[0pt]
La nature est-elle digne de respect ? \\[0pt]
La nature est-elle écrite en langage mathématique ? \\[0pt]
La nature est-elle muette ? \\[0pt]
La nature est-elle sacrée ? \\[0pt]
La nature est-elle sauvage ? \\[0pt]
La nature et la grâce \\[0pt]
La nature et l'artifice \\[0pt]
La nature et le monde \\[0pt]
« La nature fait bien les choses » \\[0pt]
La nature humaine \\[0pt]
La nature imite-t-elle l'art ? \\[0pt]
La nature morte \\[0pt]
La nature nous dicte-t-elle des fins ? \\[0pt]
La nature parle-t-elle le langage des mathématiques ? \\[0pt]
La nature peut-elle être belle ? \\[0pt]
La nature peut-elle être un modèle ? \\[0pt]
La nature s'oppose-t-elle à l'esprit ? \\[0pt]
L'anecdotique \\[0pt]
La nécessité \\[0pt]
La nécessité des contradictions \\[0pt]
La nécessité des signes \\[0pt]
La nécessité historique \\[0pt]
La nécessité sociale des cérémonies \\[0pt]
La négation \\[0pt]
La négligence \\[0pt]
La négligence est-elle une faute ? \\[0pt]
La négociation \\[0pt]
La neutralité \\[0pt]
La neutralité axiologique \\[0pt]
La neutralité de l'État \\[0pt]
La névrose \\[0pt]
Langage et communication \\[0pt]
Langage et réalité \\[0pt]
Langage et réification \\[0pt]
Langage et société \\[0pt]
Langage, langue et parole \\[0pt]
Langage, langue, parole \\[0pt]
Langage ordinaire et langage de la science \\[0pt]
L'angélisme \\[0pt]
L'angoisse \\[0pt]
Langue et langage \\[0pt]
Langue et parole \\[0pt]
L'animal \\[0pt]
L'animal apprend-il à l'homme quelque chose sur l'homme ? \\[0pt]
L'animal domestique \\[0pt]
L'animalité \\[0pt]
L'animal nous apprend-il quelque chose sur l'homme ? \\[0pt]
L'animal peut-il être un sujet moral ? \\[0pt]
L'animal politique \\[0pt]
L'animisme \\[0pt]
La noblesse \\[0pt]
L'anomalie \\[0pt]
L'anomie \\[0pt]
La non-contradiction : principe logique ou ontologique ? \\[0pt]
L'anonymat \\[0pt]
L'anonyme \\[0pt]
L'anormal \\[0pt]
La normalité \\[0pt]
La normativité \\[0pt]
La norme du beau \\[0pt]
La norme du goût \\[0pt]
La norme et l'écart \\[0pt]
La norme et le fait \\[0pt]
La norme et son application \\[0pt]
La nostalgie \\[0pt]
La notion d'administration \\[0pt]
La notion d'art contemporain \\[0pt]
La notion d'art poétique \\[0pt]
La notion de barbarie a-t-elle un sens ? \\[0pt]
La notion de caractère relève-t- elle de la morale ou de la psychologie ? \\[0pt]
La notion de civilisation \\[0pt]
La notion de classe dominante \\[0pt]
La notion de classe sociale \\[0pt]
La notion de communauté morale \\[0pt]
La notion de corps social \\[0pt]
La notion de fait social \\[0pt]
La notion de genre littéraire \\[0pt]
La notion de loi a-t-elle une unité ? \\[0pt]
La notion de loi dans les sciences de la nature et dans les sciences de l'homme \\[0pt]
La notion de milieu \\[0pt]
La notion de paradis a-t-elle un sens exclusivement religieux ? \\[0pt]
La notion de peuple \\[0pt]
La notion de point de vue \\[0pt]
La notion d'époque \\[0pt]
La notion de possible \\[0pt]
La notion de progrès a-t-elle un sens en politique ? \\[0pt]
La notion de race a-t-elle un statut scientifique ? \\[0pt]
La notion de responsabilité est-elle suspecte ? \\[0pt]
La notion de signification est-elle indispensable aux sciences humaines ? \\[0pt]
La notion de souveraineté partagée \\[0pt]
La notion de sujet en politique \\[0pt]
La notion d'évolution \\[0pt]
La notion d'intérêt \\[0pt]
La notion d'ordre \\[0pt]
La notion économique d'abondance \\[0pt]
La notion physique de force \\[0pt]
La nouveauté \\[0pt]
La nouveauté en art \\[0pt]
L'antériorité \\[0pt]
L'anthropocentrisme \\[0pt]
L'anthropologie est-elle une ontologie ? \\[0pt]
L'anticipation \\[0pt]
La nuance \\[0pt]
La nudité \\[0pt]
La paix \\[0pt]
La paix civile \\[0pt]
La paix est-elle l'absence de guerre ? \\[0pt]
La paix est-elle la visée de la politique ? \\[0pt]
La paix est-elle moins naturelle que la guerre ? \\[0pt]
La paix est-elle possible ? \\[0pt]
La paix est-elle toujours préférable ? \\[0pt]
La paix n'est-elle que l'absence de conflit ? \\[0pt]
La paix n'est-elle que l'absence de guerre ? \\[0pt]
La paix n'est-elle qu'un idéal ? \\[0pt]
La paix perpétuelle \\[0pt]
La paix sociale est-elle une fin en soi ? \\[0pt]
La parenté \\[0pt]
La parenté et la famille \\[0pt]
La paresse \\[0pt]
La parole \\[0pt]
La parole comme acte social \\[0pt]
La parole donnée \\[0pt]
La parole peut-elle être étudiée comme un acte social ? \\[0pt]
La parole politique \\[0pt]
La parole publique \\[0pt]
La participation \\[0pt]
La participation des citoyens \\[0pt]
La participation politique \\[0pt]
La parure \\[0pt]
La passion de la vérité \\[0pt]
La passion de l'égalité \\[0pt]
La passion du juste \\[0pt]
La paternité \\[0pt]
L'apathie \\[0pt]
La patience \\[0pt]
La patience est-elle une vertu ? \\[0pt]
L'apatride \\[0pt]
La patrie \\[0pt]
La pauvreté \\[0pt]
La pédagogie, philosophie pratique ou psychologie appliquée ? \\[0pt]
La peine capitale \\[0pt]
La peinture apprend-elle à voir ? \\[0pt]
La peinture est-elle une poésie muette ? \\[0pt]
La peinture peut-elle être un art du temps ? \\[0pt]
La pensée a-t-elle une histoire ? \\[0pt]
La pensée collective \\[0pt]
La pensée de la mort a-t-elle un objet ? \\[0pt]
La pensée des machines \\[0pt]
La pensée est-elle en lutte avec le langage ? \\[0pt]
La pensée formelle \\[0pt]
La pensée formelle peut-elle avoir un contenu ? \\[0pt]
La pensée magique \\[0pt]
La pensée mythique \\[0pt]
La pensée peut-elle s'écrire ? \\[0pt]
La pensée scientifique \\[0pt]
La pensée technique \\[0pt]
La perception de la folie change-t-elle de nature avec les sciences humaines ? \\[0pt]
La perception musicale \\[0pt]
La perfectibilité \\[0pt]
La perfection \\[0pt]
La perfection artistique \\[0pt]
La perfection en art \\[0pt]
La perfection morale \\[0pt]
La persécution \\[0pt]
La persévérance \\[0pt]
La personnalité \\[0pt]
La personne \\[0pt]
La personne et la mémoire \\[0pt]
La perspective \\[0pt]
La pertinence \\[0pt]
La perversion \\[0pt]
La perversion morale \\[0pt]
La perversité \\[0pt]
La peur \\[0pt]
La peur de la mort \\[0pt]
La peur de la nature \\[0pt]
La peur de l'autre \\[0pt]
La peur de l'avenir \\[0pt]
La peur de la vérité \\[0pt]
La peur de l'inconnu \\[0pt]
La peur de manquer \\[0pt]
La peur du châtiment \\[0pt]
La peur du désordre \\[0pt]
La philanthropie \\[0pt]
La philosophie doit-elle être systématique ? \\[0pt]
La philosophie doit-elle se préoccuper du salut ? \\[0pt]
La philosophie est-elle la servante de la science ? \\[0pt]
La philosophie et le sens commun \\[0pt]
La philosophie peut-elle disparaître ? \\[0pt]
La philosophie peut-elle se passer de l'idée de vérité ? \\[0pt]
La philosophie peut-elle se passer de théologie ? \\[0pt]
La philosophie première \\[0pt]
La photographie est-elle un art ? \\[0pt]
La physique est-elle le modèle de toutes les sciences ? \\[0pt]
La physique et la chimie \\[0pt]
La pitié \\[0pt]
La pitié est-elle dangereuse ? \\[0pt]
La pitié est-elle morale ? \\[0pt]
La pitié est-elle un sentiment moral ? \\[0pt]
La pitié peut-elle fonder la morale ? \\[0pt]
La place d'autrui \\[0pt]
La place de l'art est-elle sur le marché de l'art ? \\[0pt]
La place du hasard dans la science \\[0pt]
La place du sujet dans la science \\[0pt]
La place du sujet dans les sciences humaines \\[0pt]
La plénitude \\[0pt]
La pluralité \\[0pt]
La pluralité des arts \\[0pt]
La pluralité des biens \\[0pt]
La pluralité des cultures \\[0pt]
La pluralité des langues \\[0pt]
La pluralité des méthodes en sociologie \\[0pt]
La pluralité des mondes \\[0pt]
La pluralité des opinions \\[0pt]
La pluralité des pouvoirs \\[0pt]
La pluralité des représentations \\[0pt]
La pluralité des sciences \\[0pt]
La pluralité des sciences de la nature \\[0pt]
La pluralité des sens de l'être \\[0pt]
La pluralité des valeurs \\[0pt]
La plus grande perfection \\[0pt]
La poésie \\[0pt]
La poésie dit-elle le réel ? \\[0pt]
La poésie est-elle comme une peinture ? \\[0pt]
La poésie et l'idée \\[0pt]
La poésie pense-t-elle ? \\[0pt]
La polémique \\[0pt]
La police et l'armée \\[0pt]
La politesse \\[0pt]
La politique a-t-elle besoin de héros ? \\[0pt]
La politique a-t-elle besoin de la fiction ? \\[0pt]
La politique a-t-elle besoin de la religion ? \\[0pt]
La politique a-t-elle besoin de modèles ? \\[0pt]
La politique a-t-elle besoin d'experts ? \\[0pt]
La politique a-t-elle pour fin d'éliminer la violence ? \\[0pt]
La politique consiste-t-elle à faire des compromis ? \\[0pt]
La politique de la santé \\[0pt]
La politique doit-elle être morale ? \\[0pt]
La politique doit-elle être rationnelle ? \\[0pt]
La politique doit-elle se mêler de l'art ? \\[0pt]
La politique doit-elle viser la concorde ? \\[0pt]
La politique doit-elle viser le consensus ? \\[0pt]
La politique du pire \\[0pt]
La politique échappe-t-elle à l'exigence de vérité ? \\[0pt]
La politique est-elle affaire de décision ? \\[0pt]
La politique est-elle affaire de jugement ? \\[0pt]
La politique est-elle affaire de science ? \\[0pt]
La politique est-elle architectonique ? \\[0pt]
La politique est-elle continuation de la guerre, par d'autres moyens ? \\[0pt]
La politique est-elle la continuation de la guerre ? \\[0pt]
La politique est-elle la continuation de la guerre par d'autres moyens ? \\[0pt]
La politique est-elle l'affaire de tous ? \\[0pt]
La politique est-elle l'art de la délibération ? \\[0pt]
La politique est-elle l'art des possibles ? \\[0pt]
La politique est-elle l'art du possible ? \\[0pt]
La politique est-elle le refus de la violence ? \\[0pt]
La politique est-elle le règne des passions ? \\[0pt]
La politique est-elle naturelle ? \\[0pt]
La politique est-elle par nature sujette à dispute ? \\[0pt]
La politique est-elle plus importante que tout ? \\[0pt]
La politique est-elle un art ? \\[0pt]
La politique est-elle une science ? \\[0pt]
La politique est-elle une technique ? \\[0pt]
La politique est-elle un métier ? \\[0pt]
La politique est-elle un moyen ou une fin ? \\[0pt]
La politique et la gloire \\[0pt]
La politique et la ville \\[0pt]
La politique et le mal \\[0pt]
La politique et le politique \\[0pt]
La politique et l'opinion \\[0pt]
La politique et l'utopie \\[0pt]
La politique exclut-elle le désordre ? \\[0pt]
La politique implique-t-elle la hiérarchie ? \\[0pt]
La politique peut-elle changer la société ? \\[0pt]
La politique peut-elle changer le monde ? \\[0pt]
La politique peut-elle disparaître ? \\[0pt]
La politique peut-elle échapper au mythe ? \\[0pt]
La politique peut-elle être indépendante de la morale ? \\[0pt]
La politique peut-elle être morale ? \\[0pt]
La politique peut-elle être objet de science ? \\[0pt]
La politique peut-elle être un objet de science ? \\[0pt]
La politique peut-elle n'être qu'une pratique ? \\[0pt]
La politique peut-elle se passer de croyances ? \\[0pt]
La politique peut-elle se passer des idéologies ? \\[0pt]
La politique peut-elle unir les hommes ? \\[0pt]
La politique peut-elle viser à autre chose qu'à l'efficacité ? \\[0pt]
La politique repose-t-elle sur un contrat ? \\[0pt]
La politique requière-t-elle le compromis \\[0pt]
La politique scientifique \\[0pt]
La politique se réduit-elle à des rapports de force ? \\[0pt]
La politique suppose-t-elle la morale ? \\[0pt]
La politique suppose-t-elle une idée de l'homme ? \\[0pt]
La politique vise-t-elle à dépasser les injustices ? \\[0pt]
L'apolitisme \\[0pt]
La populace \\[0pt]
La population \\[0pt]
L'aporie \\[0pt]
La position \\[0pt]
La possibilité \\[0pt]
La possibilité logique \\[0pt]
La possibilité métaphysique \\[0pt]
La possibilité réelle \\[0pt]
La potentialité \\[0pt]
L'apparence \\[0pt]
L'appartenance sociale \\[0pt]
L'appel \\[0pt]
L'appréciation de la nature \\[0pt]
L'apprentissage de la langue \\[0pt]
L'approche psychologique de l'homme met-elle en question la notion d'âme ? \\[0pt]
L'appropriation \\[0pt]
L'approximation \\[0pt]
La pratique \\[0pt]
La pratique de l'espace \\[0pt]
La pratique des sciences met-elle à l'abri des préjugés ? \\[0pt]
La précaution peut-elle être un principe ? \\[0pt]
La précision \\[0pt]
La première fois \\[0pt]
La première vérité \\[0pt]
La présence \\[0pt]
La présence à soi \\[0pt]
La présence d'autrui nous évite-t-elle la solitude ? \\[0pt]
La présence d'esprit \\[0pt]
La présence du passé \\[0pt]
La présomption \\[0pt]
La preuve \\[0pt]
La preuve de l'existence de Dieu \\[0pt]
La prévision \\[0pt]
La prévision des conduites rationnelles \\[0pt]
L'\emph{a priori} \\[0pt]
L'a priori peut-il être historique ? \\[0pt]
La prise de parti est-elle essentielle en politique ? \\[0pt]
La prise en charge de l'individuel par la psychanalyse \\[0pt]
La prison \\[0pt]
La prison est-elle utile ? \\[0pt]
La privation \\[0pt]
La probabilité \\[0pt]
La probité \\[0pt]
La production \\[0pt]
La productivité de l'art \\[0pt]
La profondeur \\[0pt]
La prohibition de l'inceste \\[0pt]
La promenade \\[0pt]
La promesse \\[0pt]
La promesse et le contrat \\[0pt]
La promesse n'est-elle qu'un acte de langage ? \\[0pt]
La propagande \\[0pt]
La propagation des idées \\[0pt]
La proportion \\[0pt]
La proposition \\[0pt]
La propreté \\[0pt]
La propriété \\[0pt]
La propriété est-elle une garantie de liberté ? \\[0pt]
La protection \\[0pt]
La protection sociale \\[0pt]
La providence \\[0pt]
La prudence \\[0pt]
La prudence est-elle une vertu politique ? \\[0pt]
La psychanalyse donne-t-elle un sens au rêve ? \\[0pt]
La psychanalyse est-elle une science humaine ? \\[0pt]
La psychanalyse permet-elle la construction du sujet ? \\[0pt]
La psychanalyse s'applique-t-elle aux phénomènes sociaux ? \\[0pt]
La psychologie dit-elle ce qu'est le moi ? \\[0pt]
La psychologie est-elle l'étude de ce qui n'est pas social ? \\[0pt]
La psychologie est-elle l'étude des comportements individuels ? \\[0pt]
La psychologie est-elle une science ? \\[0pt]
La psychologie est-elle une science de la nature ? \\[0pt]
La psychologie est-elle une science humaine ? \\[0pt]
La publicité \\[0pt]
La pudeur \\[0pt]
La puissance \\[0pt]
La puissance de la technique \\[0pt]
La puissance de l'État \\[0pt]
La puissance des contraires \\[0pt]
La puissance des images \\[0pt]
La puissance du faux \\[0pt]
La puissance du langage \\[0pt]
La puissance et l'acte \\[0pt]
La puissance et le possible \\[0pt]
La pulsion \\[0pt]
La punition \\[0pt]
La pureté \\[0pt]
La qualité \\[0pt]
La question de la nature humaine relève-t-elle des sciences humaines ? \\[0pt]
La question de l'œuvre d'art \\[0pt]
La question de l'origine \\[0pt]
La question « qu'est-ce que l'homme ? » est-elle scientifique ? \\[0pt]
La question : « qui ? » \\[0pt]
La question sociale \\[0pt]
La quête des origines \\[0pt]
La racine du mal \\[0pt]
La radicalité \\[0pt]
La raison a-t-elle des limites ? \\[0pt]
La raison a-t-elle le droit d'expliquer ce que morale condamne ? \\[0pt]
La raison a-t-elle une histoire ? \\[0pt]
La raison d'État \\[0pt]
La raison du plus fort \\[0pt]
La raison, est-ce la modération ? \\[0pt]
La raison est-elle morale par elle-même ? \\[0pt]
La raison est-elle suffisante ? \\[0pt]
La raison est-elle un instrument ? \\[0pt]
La raison et le calcul \\[0pt]
La raison peut-elle errer ? \\[0pt]
La raison peut-elle être immédiatement pratique ? \\[0pt]
La raison peut-elle être pratique ? \\[0pt]
La raison peut-elle s'opposer à elle-même ? \\[0pt]
La raison s'identifie-t-elle au calcul ? \\[0pt]
La raison suffisante \\[0pt]
La rationalité de la psychanalyse \\[0pt]
La rationalité de l'homo œconomicus \\[0pt]
La rationalité des choix politiques \\[0pt]
La rationalité des comportements économiques \\[0pt]
La rationalité du langage \\[0pt]
La rationalité du marché \\[0pt]
La rationalité économique \\[0pt]
La rationalité en sciences sociales \\[0pt]
La rationalité politique \\[0pt]
L'arbitraire \\[0pt]
L'arbitraire de la règle \\[0pt]
L'arbitraire du signe \\[0pt]
L'arbitraire en politique \\[0pt]
L'archéologie \\[0pt]
L'architecte : artiste ou ingénieur ? \\[0pt]
L'architecte : artiste ou l'ingénieur \\[0pt]
L'architecte dans la cité \\[0pt]
L'architecte et le législateur \\[0pt]
L'architecte et l'ingénieur \\[0pt]
L'architecture du monde \\[0pt]
L'architecture est-elle un art ? \\[0pt]
L'architecture et l'espace \\[0pt]
La réaction en politique \\[0pt]
La réalité \\[0pt]
La réalité a-t-elle une forme logique ? \\[0pt]
La réalité décrite par la science s'oppose-t-elle à la démonstration ? \\[0pt]
La réalité de la contradiction \\[0pt]
La réalité de l'idéal \\[0pt]
La réalité de l'idée \\[0pt]
La réalité de l'inconscient \\[0pt]
La réalité du beau \\[0pt]
La réalité du bien \\[0pt]
La réalité du corps \\[0pt]
La réalité du futur \\[0pt]
La réalité du mal \\[0pt]
La réalité du passé \\[0pt]
La réalité du possible \\[0pt]
La réalité du progrès \\[0pt]
La réalité du sensible \\[0pt]
La réalité du temps \\[0pt]
La réalité mentale \\[0pt]
La réalité peut-elle être virtuelle ? \\[0pt]
La réalité physique \\[0pt]
La réalité sociale \\[0pt]
La réalité symbolique \\[0pt]
La réception de l'œuvre d'art \\[0pt]
La recherche-action en sciences sociales \\[0pt]
La recherche de l'absolu \\[0pt]
La recherche de la vérité \\[0pt]
La recherche de la vérité dans les sciences humaines \\[0pt]
La recherche des invariants \\[0pt]
La recherche des origines \\[0pt]
La recherche du bonheur \\[0pt]
La recherche du bonheur suffit-elle à déterminer une morale ? \\[0pt]
La recherche scientifique est-elle désintéressée ? \\[0pt]
La réciprocité \\[0pt]
La réciprocité est-elle indispensable à la communauté politique ? \\[0pt]
La reconnaissance \\[0pt]
La rectitude \\[0pt]
La référence \\[0pt]
La référence à la mémoire en histoire \\[0pt]
La réflexion \\[0pt]
La réflexion sur l'expérience participe-t-elle de l'expérience ? \\[0pt]
La réforme \\[0pt]
La réforme des institutions \\[0pt]
La réfutation \\[0pt]
La règle et l'exception \\[0pt]
La régression à l'infini \\[0pt]
La régulation \\[0pt]
La relation \\[0pt]
La relation à autrui est-elle symétrique ? \\[0pt]
La relation de causalité est-elle temporelle ? \\[0pt]
La relation de cause à effet \\[0pt]
La relation de nécessité \\[0pt]
La relation d'identité \\[0pt]
La religion \\[0pt]
La religion civile \\[0pt]
La religion de l'art \\[0pt]
La religion peut-elle faire lien social ? \\[0pt]
La religion peut-elle se passer du recours au sacré ? \\[0pt]
La religion peut-elle suppléer la raison ? \\[0pt]
La réminiscence \\[0pt]
La renaissance \\[0pt]
La rencontre \\[0pt]
La réparation \\[0pt]
La répétition \\[0pt]
La représentation \\[0pt]
La représentation en politique \\[0pt]
La représentation politique \\[0pt]
La reproductibilité de l'œuvre d'art \\[0pt]
La reproduction \\[0pt]
La reproduction des œuvres d'art \\[0pt]
La reproduction des œuvres d'art nuit-elle à l'art ? \\[0pt]
La reproduction sociale \\[0pt]
La république \\[0pt]
La réputation \\[0pt]
La résignation \\[0pt]
La résilience \\[0pt]
La résistance à l'oppression \\[0pt]
La résistance du réel \\[0pt]
La résolution \\[0pt]
La responsabilité \\[0pt]
La responsabilité collective \\[0pt]
La responsabilité de l'artiste \\[0pt]
La responsabilité envers les générations futures \\[0pt]
La responsabilité implique-t-elle autrui ? \\[0pt]
La responsabilité politique \\[0pt]
La ressemblance \\[0pt]
La restauration des œuvres d'art \\[0pt]
La révélation \\[0pt]
La rêverie \\[0pt]
La révolte \\[0pt]
La révolte peut-elle être un droit ? \\[0pt]
La révolution \\[0pt]
L'argent \\[0pt]
L'argent et la valeur \\[0pt]
L'argumentation \\[0pt]
L'argumentation morale \\[0pt]
L'argument d'autorité \\[0pt]
La rhétorique \\[0pt]
La rhétorique a-t-elle une valeur ? \\[0pt]
La rhétorique est-elle un art ? \\[0pt]
La richesse \\[0pt]
La richesse du sensible \\[0pt]
La richesse intérieure \\[0pt]
La rigueur \\[0pt]
La rigueur de la loi \\[0pt]
La rigueur morale \\[0pt]
La rime et la raison \\[0pt]
L'aristocratie \\[0pt]
L'art abstrait \\[0pt]
L'art à l'épreuve du goût \\[0pt]
L'art apprend-il à percevoir ? \\[0pt]
L'art a-t-il besoin d'un discours sur l'art ? \\[0pt]
L'art a-t-il des règles ? \\[0pt]
L'art a-t-il des vertus thérapeutiques ? \\[0pt]
L'art a-t-il plus de valeur que la vérité ? \\[0pt]
L'art a-t-il une fin morale ? \\[0pt]
L'art a-t-il une histoire ? \\[0pt]
L'art a-t-il une responsabilité morale ? \\[0pt]
L'art a-t-il une valeur sociale ? \\[0pt]
L'art contre la beauté ? \\[0pt]
L'art d'écrire \\[0pt]
L'art décrit-il ? \\[0pt]
L'art de gouverner \\[0pt]
L'art de la guerre \\[0pt]
L'art de masse \\[0pt]
L'art de persuader \\[0pt]
L'art des images \\[0pt]
L'art de vivre est-il un art ? \\[0pt]
L'art doit-il être critique ? \\[0pt]
L'art doit-il nous étonner ? \\[0pt]
L'art donne-t-il à voir l'invisible ? \\[0pt]
L'art dramatique \\[0pt]
L'art du comédien \\[0pt]
L'art du mensonge \\[0pt]
L'art échappe-t-il à la raison ? \\[0pt]
L'art engagé \\[0pt]
L'art, est-ce ce qui résiste à la certitude ? \\[0pt]
L'art est-il affaire de goût ? \\[0pt]
L'art est-il affaire d'imagination ? \\[0pt]
L'art est-il à lui-même son propre but ? \\[0pt]
L'art est-il ce qui permet de partager ses émotions ? \\[0pt]
L'art est-il destiné à embellir ? \\[0pt]
L'art est-il interprétation ? \\[0pt]
L'art est-il le miroir du monde ? \\[0pt]
L'art est-il objet de compréhension ? \\[0pt]
L'art est-il politique ? \\[0pt]
L'art est-il révolutionnaire ? \\[0pt]
L'art est-il subversif ? \\[0pt]
L'art est-il un divertissement ? \\[0pt]
L'art est-il une connaissance ? \\[0pt]
L'art est-il une critique de la culture ? \\[0pt]
L'art est-il une expérience de la liberté ? \\[0pt]
L'art est-il une valeur ? \\[0pt]
L'art est-il un langage ? \\[0pt]
L'art est-il un langage universel ? \\[0pt]
L'art est-il un luxe ? \\[0pt]
L'art est-il un mode de connaissance ? \\[0pt]
L'art est-il un modèle pour la philosophie ? \\[0pt]
L'art est-il un monde ? \\[0pt]
L'art est par-delà beauté et laideur ? \\[0pt]
L'art et la manière \\[0pt]
L'art et la mort \\[0pt]
L'art et la nature \\[0pt]
L'art et l'artifice \\[0pt]
L'art et la tradition \\[0pt]
L'art et la vérité \\[0pt]
L'art et la vie \\[0pt]
L'art et le divin \\[0pt]
L'art et le mouvement \\[0pt]
L'art et le mythe \\[0pt]
L'art et l'éphémère \\[0pt]
L'art et le rêve \\[0pt]
L'art et le sacré \\[0pt]
L'art et les arts \\[0pt]
L'art et le temps \\[0pt]
L'art et le vivant \\[0pt]
L'art et l'immoralité \\[0pt]
L'art et l'interdit \\[0pt]
L'art et morale \\[0pt]
L'art et ses institutions \\[0pt]
L'art : expérience, exercice ou habitude ? \\[0pt]
L'art fait-il penser ? \\[0pt]
L'artifice \\[0pt]
L'artificiel \\[0pt]
L'art imite-t-il la nature ? \\[0pt]
L'art industriel \\[0pt]
L'artiste a-t-il besoin d'une idée de l'art ? \\[0pt]
L'artiste a-t-il besoin d'un public ? \\[0pt]
L'artiste a-t-il toujours raison ? \\[0pt]
L'artiste a-t-il une méthode ? \\[0pt]
L'artiste dans la cité \\[0pt]
L'artiste est-il le mieux placé pour comprendre son œuvre ? \\[0pt]
L'artiste est-il le technicien du beau ? \\[0pt]
L'artiste est-il maître de son œuvre ? \\[0pt]
L'artiste est-il un métaphysicien ? \\[0pt]
L'artiste et l'artisan \\[0pt]
L'artiste et son public \\[0pt]
L'artiste exprime-t-il quelque chose ? \\[0pt]
L'artiste peut-il se passer d'un maître ? \\[0pt]
L'artiste sait-il ce qu'il fait ? \\[0pt]
L'artiste travaille-t-il ? \\[0pt]
L'art modifie-t-il notre rapport au réel ? \\[0pt]
L'art n'est-il pas toujours politique ? \\[0pt]
L'art n'est-il pas toujours religieux ? \\[0pt]
L'art n'est-il qu'apparence ? \\[0pt]
L'art n'est-il qu'un artifice ? \\[0pt]
L'art n'est-il qu'une affaire d'esthétique ? \\[0pt]
L'art n'est-il qu'une question de sentiment ? \\[0pt]
L'art nous donne-t-il des raisons d'espérer ? \\[0pt]
L'art nous enseigne-t-il quelque chose ? \\[0pt]
L'art nous libère-t-il de l'insignifiance ? \\[0pt]
L'art nous permet-il de lutter contre l'irréversibilité ? \\[0pt]
L'art nous ramène-t-il à la réalité ? \\[0pt]
L'art officiel \\[0pt]
L'art ou les arts \\[0pt]
L'art peut-il changer le monde ? \\[0pt]
L'art peut-il encore imiter la nature ? \\[0pt]
L'art peut-il être brut ? \\[0pt]
L'art peut-il être enseigné ? \\[0pt]
L'art peut-il être révolutionnaire ? \\[0pt]
L'art peut-il être sans œuvre ? \\[0pt]
L'art peut-il être utile ? \\[0pt]
L'art peut-il finir ? \\[0pt]
L'art peut-il mourir ? \\[0pt]
L'art peut-il n'être pas conceptuel ? \\[0pt]
L'art peut-il nous rendre meilleurs ? \\[0pt]
L'art peut-il prétendre à la vérité ? \\[0pt]
L'art peut-il quelque chose contre la morale ? \\[0pt]
L'art peut-il quelque chose pour la morale ? \\[0pt]
L'art peut-il rendre le mouvement ? \\[0pt]
L'art peut-il s'affranchir des lois ? \\[0pt]
L'art peut-il s'enseigner ? \\[0pt]
L'art peut-il se passer de règles ? \\[0pt]
L'art peut-il se passer des critères du beau et du laid ? \\[0pt]
L'art peut-il se passer d'idéal ? \\[0pt]
L'art peut-il se passer d'œuvres ? \\[0pt]
L'art peut-il tenir lieu de métaphysique ? \\[0pt]
L'art politique \\[0pt]
L'art populaire \\[0pt]
L'art pour l'art \\[0pt]
L'art produit-il nécessairement des œuvres ? \\[0pt]
L'art s'adresse-t-il à la sensibilité ? \\[0pt]
L'art sait-il montrer ce que le langage ne peut pas dire ? \\[0pt]
L'art s'apparente-t-il à la philosophie ? \\[0pt]
L'art : une arithmétique sensible ? \\[0pt]
L'art vise-t-il nécessairement le beau ? \\[0pt]
La ruine \\[0pt]
La rumeur \\[0pt]
La rupture \\[0pt]
La ruse \\[0pt]
La ruse technique \\[0pt]
La sacralisation de l'œuvre \\[0pt]
La sagesse \\[0pt]
La sagesse du corps \\[0pt]
La sagesse et l'expérience \\[0pt]
La sagesse rend-elle heureux ? \\[0pt]
La sainteté \\[0pt]
La sanction \\[0pt]
La santé \\[0pt]
La satisfaction des penchants \\[0pt]
La scène \\[0pt]
La scène théâtrale \\[0pt]
L'ascèse \\[0pt]
L'ascétisme \\[0pt]
La science admet-elle des degrés de croyance ? \\[0pt]
La science a-t-elle besoin du principe de causalité ? \\[0pt]
La science a-t-elle des limites ? \\[0pt]
La science a-t-elle le monopole de la vérité ? \\[0pt]
La science a-t-elle toujours raison ? \\[0pt]
La science a-t-elle une histoire ? \\[0pt]
La science commence-t-elle avec la perception ? \\[0pt]
La science connaît-elle ses limites ? \\[0pt]
La science découvre-t-elle ou construit-elle son objet ? \\[0pt]
La science de l'être \\[0pt]
La science de l'individuel \\[0pt]
La science désenchante-t-elle le monde ? \\[0pt]
La science des mœurs \\[0pt]
La science dévoile-t-elle le réel ? \\[0pt]
La science doit-elle nous fournir des représentations du monde ? \\[0pt]
La science doit-elle se fonder sur une idée de la nature ? \\[0pt]
La science doit-elle se passer de l'idée de finalité ? \\[0pt]
La science est-elle indépendante de toute métaphysique ? \\[0pt]
La science est-elle une affaire politique ? \\[0pt]
La science est-elle une langue bien faite ? \\[0pt]
La science et le mythe \\[0pt]
La science et les sciences \\[0pt]
La science et l'irrationnel \\[0pt]
La science explique-t-elle le réel ? \\[0pt]
La science historique peut-elle être prédictive ? \\[0pt]
La science nous éloigne-t-elle des choses ? \\[0pt]
La science nous indique-t-elle ce que nous devons faire ? \\[0pt]
La science pense-t-elle ? \\[0pt]
La science peut-elle errer ? \\[0pt]
La science peut-elle guider notre conduite ? \\[0pt]
La science peut-elle lutter contre les préjugés ? \\[0pt]
La science peut-elle se passer de fondement ? \\[0pt]
La science peut-elle se passer de métaphysique ? \\[0pt]
La science peut-elle se passer de méthode ? \\[0pt]
La science peut-elle se passer d'hypothèses ? \\[0pt]
La science peut-elle se passer d'institutions ? \\[0pt]
La science peut-elle tout expliquer ? \\[0pt]
La science politique \\[0pt]
La science porte-elle au scepticisme ? \\[0pt]
La science procède-t-elle par rectification ? \\[0pt]
La scientificité des sciences humaines \\[0pt]
La sculpture \\[0pt]
La sculpture et la peinture \\[0pt]
La sculpture et le mouvement \\[0pt]
La sculpture et l'espace \\[0pt]
La sécularisation \\[0pt]
La sécurité \\[0pt]
La sécurité nationale \\[0pt]
La sécurité publique \\[0pt]
La séduction \\[0pt]
La séduction en politique \\[0pt]
La ségrégation \\[0pt]
La sensation est-elle une connaissance ? \\[0pt]
La sensibilité \\[0pt]
La sensibilité peut-elle délivrer une vérité ? \\[0pt]
La séparation \\[0pt]
La séparation des pouvoirs \\[0pt]
La sérénité \\[0pt]
La servitude \\[0pt]
La servitude humaine \\[0pt]
La servitude volontaire \\[0pt]
La sévérité \\[0pt]
La sexualité \\[0pt]
La signification \\[0pt]
La signification dans l'œuvre \\[0pt]
La signification des mots \\[0pt]
La signification des rites \\[0pt]
La signification en musique \\[0pt]
La signification et l'usage \\[0pt]
La simplicité \\[0pt]
La sincérité \\[0pt]
La singularité \\[0pt]
La singularité du réel \\[0pt]
La singularité en morale \\[0pt]
La situation \\[0pt]
La sobriété \\[0pt]
La socialisation \\[0pt]
La socialisation des comportements \\[0pt]
La société civile \\[0pt]
La société civile et l'État \\[0pt]
La société de consommation \\[0pt]
La société des individus \\[0pt]
La société des loisirs \\[0pt]
La société des nations \\[0pt]
La société des savants \\[0pt]
La société et l'État \\[0pt]
La société existe-t-elle ? \\[0pt]
La société peut-elle être objet de connaissance ? \\[0pt]
La société peut-elle se passer de l'État ? \\[0pt]
La sociologie de la religion \\[0pt]
La sociologie de l'art \\[0pt]
La sociologie de l'art nous permet-elle de comprendre l'art ? \\[0pt]
La sociologie est-elle déterministe ? \\[0pt]
La sociologie est-elle une physique sociale ? \\[0pt]
La sociologie relativise-t-elle la valeur des œuvres d'art ? \\[0pt]
La solidarité \\[0pt]
La solitude \\[0pt]
La solitude constitue-t-elle un obstacle à la citoyenneté ? \\[0pt]
La solitude de l'artiste \\[0pt]
La sollicitude \\[0pt]
La somme et le tout \\[0pt]
La souffrance \\[0pt]
La souffrance a-t-elle un sens ? \\[0pt]
La souffrance a-t-elle un sens moral ? \\[0pt]
La souffrance au travail \\[0pt]
La souffrance d'autrui \\[0pt]
La souffrance morale \\[0pt]
La souffrance peut-elle être partagée ? \\[0pt]
La soumission \\[0pt]
La souveraineté \\[0pt]
La souveraineté de l'État \\[0pt]
La souveraineté du peuple \\[0pt]
La souveraineté est-elle indivisible ? \\[0pt]
La souveraineté nationale \\[0pt]
La souveraineté peut-elle se partager ? \\[0pt]
La souveraineté populaire \\[0pt]
La spécificité des sciences humaines \\[0pt]
La spéculation \\[0pt]
La sphère privée échappe-t-elle au politique ? \\[0pt]
L'aspiration esthétique \\[0pt]
La spontanéité \\[0pt]
L'assassinat politique \\[0pt]
L'assentiment \\[0pt]
L'asservissement de la pensée \\[0pt]
L'association \\[0pt]
L'association des idées \\[0pt]
La structure \\[0pt]
La structure et les éléments \\[0pt]
La structure et le sujet \\[0pt]
La subsistance \\[0pt]
La substance \\[0pt]
La substance et l'accident \\[0pt]
La substance et le substrat \\[0pt]
La superstition \\[0pt]
La sûreté \\[0pt]
La surface et la profondeur \\[0pt]
La surveillance de la société \\[0pt]
La survie \\[0pt]
La sympathie \\[0pt]
La sympathie peut-elle tenir lieu de moralité ? \\[0pt]
La systématicité \\[0pt]
La table rase \\[0pt]
La technique est-elle moralement neutre ? \\[0pt]
La technique fait-elle des miracles ? \\[0pt]
La technique n'est-elle qu'une application de la science ? \\[0pt]
La technique n'est-elle qu'un prolongement de nos organes ? \\[0pt]
La technique peut-elle améliorer l'homme ? \\[0pt]
La technocratie \\[0pt]
La technologie modifie-t-elle les rapports sociaux ? \\[0pt]
La téléologie \\[0pt]
La témérité \\[0pt]
La tempérance \\[0pt]
La temporalité de l'œuvre d'art \\[0pt]
La tendance \\[0pt]
La tentation \\[0pt]
La tentation réductionniste \\[0pt]
La terre \\[0pt]
La terre est-elle un bien commun ? \\[0pt]
La terre et le ciel \\[0pt]
La Terre et le Ciel \\[0pt]
La terreur \\[0pt]
La terreur morale \\[0pt]
L'athée peut-il être vertueux ? \\[0pt]
L'athéisme \\[0pt]
La théogonie \\[0pt]
La théologie est-elle une science ? \\[0pt]
La théologie peut-elle être rationnelle ? \\[0pt]
La théologie rationnelle \\[0pt]
La théorie et l'expérience \\[0pt]
La tolérance \\[0pt]
La tolérance a-t-elle des limites ? \\[0pt]
La tolérance envers les intolérants \\[0pt]
La tolérance est-elle un concept politique ? \\[0pt]
La tolérance est-elle une vertu ? \\[0pt]
La tolérance peut-elle constituer un problème pour la démocratie ? \\[0pt]
L'atome \\[0pt]
La totalitarisme \\[0pt]
La totalité \\[0pt]
La toute-puissance \\[0pt]
La toute-puissance de la pensée \\[0pt]
La trace \\[0pt]
La trace et l'indice \\[0pt]
La tradition \\[0pt]
La traductibilité des langues \\[0pt]
La traduction \\[0pt]
La tragédie \\[0pt]
La trahison \\[0pt]
La tranquillité \\[0pt]
La transcendance \\[0pt]
La transe \\[0pt]
La transgression \\[0pt]
La transgression des règles \\[0pt]
La transmission \\[0pt]
La transparence \\[0pt]
La transparence est-elle un idéal démocratique ? \\[0pt]
La tristesse \\[0pt]
La tristesse du fini \\[0pt]
La tristesse est-elle mauvaise ? \\[0pt]
L'attachement \\[0pt]
L'attente \\[0pt]
L'attention \\[0pt]
L'attrait du beau \\[0pt]
La tyrannie \\[0pt]
La tyrannie de la majorité \\[0pt]
La tyrannie du bonheur \\[0pt]
L'audace \\[0pt]
L'audace politique \\[0pt]
L'au-delà \\[0pt]
L'au-delà de l'être \\[0pt]
L'autarcie \\[0pt]
L'auteur et le créateur \\[0pt]
L'authenticité \\[0pt]
L'authenticité artistique \\[0pt]
L'authenticité de l'œuvre d'art \\[0pt]
L'autobiographie \\[0pt]
L'automate \\[0pt]
L'autonomie \\[0pt]
L'autonomie de l'art \\[0pt]
L'autonomie de l'œuvre d'art \\[0pt]
L'autonomie du théorique \\[0pt]
L'autonomie du vivant \\[0pt]
L'autoportrait \\[0pt]
L'autorité \\[0pt]
L'autorité de la loi \\[0pt]
L'autorité de la parole \\[0pt]
L'autorité de la science \\[0pt]
L'autorité de l'écrit \\[0pt]
L'autorité de l'État \\[0pt]
L'autorité des savants \\[0pt]
L'autorité morale \\[0pt]
L'autorité politique \\[0pt]
L'autre est-il le fondement de la conscience morale ? \\[0pt]
L'autre monde \\[0pt]
La valeur artistique \\[0pt]
La valeur d'échange \\[0pt]
La valeur de la loi \\[0pt]
La valeur de l'argent \\[0pt]
La valeur de l'art \\[0pt]
La valeur de la science \\[0pt]
La valeur de la vérité \\[0pt]
La valeur de l'échange \\[0pt]
La valeur de l'exemple \\[0pt]
La valeur des arts \\[0pt]
La valeur des choses \\[0pt]
La valeur des conventions \\[0pt]
La valeur des exceptions \\[0pt]
La valeur des hypothèses \\[0pt]
La valeur des normes \\[0pt]
La valeur du beau \\[0pt]
La valeur du consensus \\[0pt]
La valeur du consentement \\[0pt]
La valeur d'une action se mesure-t-elle à ses conséquences ? \\[0pt]
La valeur d'une théorie scientifique se mesure-t-elle à son efficacité ? \\[0pt]
La valeur d'usage \\[0pt]
La valeur du témoignage \\[0pt]
La valeur du temps \\[0pt]
La valeur du travail \\[0pt]
La valeur et le prix \\[0pt]
La valeur morale \\[0pt]
« La valeur travail » \\[0pt]
La validité \\[0pt]
La vanité \\[0pt]
La vanité est-elle toujours sans objet ? \\[0pt]
L'avant-garde \\[0pt]
L'avarice \\[0pt]
La variété \\[0pt]
La vénalité \\[0pt]
La vénération \\[0pt]
La vengeance \\[0pt]
L'avenir \\[0pt]
L'avenir de l'humanité \\[0pt]
L'avenir est-il imaginable ? \\[0pt]
L'avenir est-il prévisible ? \\[0pt]
L'avenir est-il sans image ? \\[0pt]
L'avenir existe-t-il ? \\[0pt]
L'aventure \\[0pt]
La véracité \\[0pt]
La vérification \\[0pt]
La vérité admet-elle des degrés ? \\[0pt]
La vérité a-t-elle une histoire ? \\[0pt]
La vérité de la fiction \\[0pt]
La vérité de la religion \\[0pt]
La vérité demande-t-elle du courage ? \\[0pt]
La vérité des images \\[0pt]
La vérité des sentiments \\[0pt]
La vérité du déterminisme \\[0pt]
La vérité d'une théorie dépend-elle de sa correspondance avec les faits ? \\[0pt]
La vérité du roman \\[0pt]
La vérité en politique \\[0pt]
La vérité est-elle fille de son temps ? \\[0pt]
La vérité est-elle hors de notre portée ? \\[0pt]
La vérité est-elle morale ? \\[0pt]
La vérité est-elle relative ? \\[0pt]
La vérité scientifique est-elle relative ? \\[0pt]
La vertu \\[0pt]
La vertu de l'abstraction \\[0pt]
La vertu de l'homme politique \\[0pt]
La vertu du citoyen \\[0pt]
La vertu est-elle une forme de santé ? \\[0pt]
La vertu est-elle utile ? \\[0pt]
La vertu, les vertus \\[0pt]
La vertu peut-elle être excessive ? \\[0pt]
La vertu peut-elle être purement morale ? \\[0pt]
La vertu peut-elle s'enseigner ? \\[0pt]
La vertu politique \\[0pt]
L'aveu \\[0pt]
L'aveu diminue-t-il la faute ? \\[0pt]
L'aveuglement \\[0pt]
La vie active \\[0pt]
La vie brève \\[0pt]
La vie collective est-elle nécessairement frustrante ? \\[0pt]
La vie de la langue \\[0pt]
La vie de l'esprit \\[0pt]
« La vie des formes » \\[0pt]
La vie des signes \\[0pt]
La vie du droit \\[0pt]
La vie est-elle la valeur suprême ? \\[0pt]
La vie est-elle le bien suprême ? \\[0pt]
La vie est-elle une notion métaphysique ? \\[0pt]
La vie est-elle une valeur ? \\[0pt]
La vie est-elle un songe ? \\[0pt]
« La vie est un songe » \\[0pt]
La vie éternelle \\[0pt]
La vie heureuse \\[0pt]
La vieillesse \\[0pt]
La vie intérieure \\[0pt]
La vie ordinaire \\[0pt]
La vie peut-elle être éternelle ? \\[0pt]
La vie peut-elle être sans histoire ? \\[0pt]
La vie politique \\[0pt]
La vie politique est-elle aliénante ? \\[0pt]
La vie privée \\[0pt]
La vie psychique \\[0pt]
La vie quotidienne \\[0pt]
La vie sauvage \\[0pt]
La vie sociale \\[0pt]
La vie sociale est-elle une comédie ? \\[0pt]
La vie sociale relève-t-elle des habitudes ? \\[0pt]
La vie suffit-elle à créer des valeurs ? \\[0pt]
La vigilance \\[0pt]
La ville \\[0pt]
La ville et la campagne \\[0pt]
La violence \\[0pt]
La violence a-t-elle des degrés ? \\[0pt]
La violence de l'art \\[0pt]
La violence de l'État \\[0pt]
La violence politique \\[0pt]
La violence révolutionnaire \\[0pt]
La violence sociale \\[0pt]
La violence verbale \\[0pt]
La virtualité \\[0pt]
La virtuosité \\[0pt]
La vision scientifique du monde \\[0pt]
La vocation \\[0pt]
La vocation utopique de l'art \\[0pt]
La voix \\[0pt]
La voix de la conscience \\[0pt]
La voix du peuple \\[0pt]
La volonté constitue-t-elle le principe de la politique ? \\[0pt]
La volonté de croire \\[0pt]
La volonté du peuple \\[0pt]
La volonté générale \\[0pt]
La volonté peut-elle être collective ? \\[0pt]
La volonté peut-elle être indéterminée ? \\[0pt]
La volonté peut-elle être libre ? \\[0pt]
La volonté politique peut-elle être commune ? \\[0pt]
La volupté \\[0pt]
La vraie morale se moque-t-elle de la morale ? \\[0pt]
La vraisemblance \\[0pt]
La vulgarisation \\[0pt]
La vulgarité \\[0pt]
La vulnérabilité \\[0pt]
L'axiome \\[0pt]
Le barbare \\[0pt]
Le baroque \\[0pt]
Le bavardage \\[0pt]
Le beau a-t-il une histoire ? \\[0pt]
Le beau est-il aimable ? \\[0pt]
Le beau est-il dicible ? \\[0pt]
Le beau est-il l'objet de l'esthétique ? \\[0pt]
Le beau est-il une valeur commune ? \\[0pt]
Le beau et l'agréable \\[0pt]
Le beau et le bien \\[0pt]
Le beau et le bon \\[0pt]
Le beau et le sublime \\[0pt]
Le beau existe-t-il indépendamment du bien ? \\[0pt]
Le beau naturel \\[0pt]
Le beau n'est-il qu'une idée ? \\[0pt]
Le beau peut-il être effrayant ? \\[0pt]
Le besoin \\[0pt]
Le besoin d'absolu \\[0pt]
Le besoin de beauté \\[0pt]
Le besoin de métaphysique est-il un besoin de connaissance ? \\[0pt]
Le besoin de philosophie \\[0pt]
Le besoin de rêve \\[0pt]
Le besoin de vérité \\[0pt]
Le besoin métaphysique \\[0pt]
Le bien commun \\[0pt]
Le bien détermine-t-il la volonté ? \\[0pt]
Le bien est-il conformité à la nature ? \\[0pt]
Le bien est-il l'utile ? \\[0pt]
Le bien et le beau \\[0pt]
Le bien et le mal \\[0pt]
Le bien et les biens \\[0pt]
Le bien public \\[0pt]
Le bien suppose-t-il la transcendance ? \\[0pt]
Le biologisme \\[0pt]
Le biologisme en sciences sociales \\[0pt]
Le bon et l'utile \\[0pt]
Le bon goût \\[0pt]
Le bonheur comporte-t-il des degrés ? \\[0pt]
Le bonheur dans le mal \\[0pt]
Le bonheur de la passion est-il sans lendemain ? \\[0pt]
Le bonheur des citoyens est-il un idéal politique ? \\[0pt]
Le bonheur des uns, le malheur des autres \\[0pt]
Le bonheur du citoyen \\[0pt]
Le bonheur est-il affaire de calcul ? \\[0pt]
Le bonheur est-il une affaire de circonstances ? \\[0pt]
Le bonheur est-il une fin morale ? \\[0pt]
Le bonheur est-il une fin politique ? \\[0pt]
Le bonheur est-il une valeur morale ? \\[0pt]
Le bonheur est-il un principe politique ? \\[0pt]
Le bonheur et la vertu \\[0pt]
Le bonheur n'est-il qu'une idée ? \\[0pt]
Le bon plaisir \\[0pt]
Le bon sens \\[0pt]
Le bon usage des passions \\[0pt]
Le bourgeois et le citoyen \\[0pt]
Le bruit \\[0pt]
Le cadavre \\[0pt]
Le cadre \\[0pt]
Le calcul \\[0pt]
Le calcul des plaisirs \\[0pt]
Le cannibalisme \\[0pt]
Le canon \\[0pt]
Le capitalisme \\[0pt]
Le capital social \\[0pt]
Le caractère \\[0pt]
L'écart \\[0pt]
Le cas de conscience \\[0pt]
Le catéchisme moral \\[0pt]
Le certain et le probable \\[0pt]
Le cerveau et la pensée \\[0pt]
L'échange \\[0pt]
L'échange des marchandises et les rapports humains \\[0pt]
L'échange des signes \\[0pt]
L'échange inégal \\[0pt]
Le changement \\[0pt]
L'échange symbolique \\[0pt]
Le chant \\[0pt]
Le chant et le cri \\[0pt]
Le chaos \\[0pt]
Le chaos du monde \\[0pt]
Le charisme en politique \\[0pt]
Le charisme politique \\[0pt]
Le charme et la grâce \\[0pt]
Le châtiment \\[0pt]
Le châtiment rend-il meilleur ? \\[0pt]
Le chef-d'œuvre \\[0pt]
Le chemin \\[0pt]
Le choix \\[0pt]
Le choix d'un destin \\[0pt]
Le choix en économie \\[0pt]
Le choix peut-il être éclairé ? \\[0pt]
Le ciel et la terre \\[0pt]
Le cinéma, art de la représentation ? \\[0pt]
Le cinéma est-il un art ? \\[0pt]
Le cinéma est-il un art comme les autres ? \\[0pt]
Le cinéma est-il un art ou une industrie ? \\[0pt]
Le cinéma est-il un art populaire ? \\[0pt]
Le citoyen \\[0pt]
Le citoyen peut-il être à la fois libre et soumis à l'État ? \\[0pt]
Le civisme \\[0pt]
Le clair et le distinct \\[0pt]
Le clair-obscur \\[0pt]
Le classicisme \\[0pt]
Le code \\[0pt]
Le cœur \\[0pt]
Le cœur et la raison \\[0pt]
L'école de la vie \\[0pt]
L'école des vertus \\[0pt]
L'écologie est-elle aussi une science humaine ? \\[0pt]
L'écologie est-elle une science de la nature ou une science sociale ? \\[0pt]
L'écologie est-elle un problème politique ? \\[0pt]
L'écologie politique \\[0pt]
L'écologie, une science humaine ? \\[0pt]
Le combat contre l'injustice a-t-il une source morale ? \\[0pt]
Le comédien \\[0pt]
Le comique et le tragique \\[0pt]
Le commencement \\[0pt]
Le commerce \\[0pt]
Le commerce des corps \\[0pt]
Le commerce des hommes \\[0pt]
Le commerce est-il pacificateur ? \\[0pt]
Le commun \\[0pt]
Le comparatisme dans les sciences humaines \\[0pt]
Le comportement \\[0pt]
Le compromis \\[0pt]
Le concept \\[0pt]
Le concept de civilisation \\[0pt]
Le concept de comportement rationnel dans les sciences humaines \\[0pt]
Le concept de nature est-il un concept scientifique ? \\[0pt]
Le concept de pulsion \\[0pt]
Le concept de race est-il anthropologique ? \\[0pt]
Le concept de société sans écriture est-il pertinent ? \\[0pt]
Le concept de structure sociale \\[0pt]
Le concept d'évolution \\[0pt]
Le concept d'inconscient est-il nécessaire en sciences humaines ? \\[0pt]
Le concept et la vie \\[0pt]
Le concept et l'image \\[0pt]
Le concret \\[0pt]
Le concret et l'abstrait \\[0pt]
Le conflit de devoirs \\[0pt]
Le conflit des devoirs \\[0pt]
Le conflit des interprétations \\[0pt]
Le conflit esthétique \\[0pt]
Le conflit est-il au fondement de la vie sociale ? \\[0pt]
Le conflit est-il constitutif de la politique ? \\[0pt]
Le conflit est-il la raison d'être de la politique ? \\[0pt]
Le conflit social \\[0pt]
Le conformisme \\[0pt]
Le conformisme moral \\[0pt]
Le conformisme social \\[0pt]
L'économie a-t-elle des lois ? \\[0pt]
L'économie a-t-elle ses propres lois ? \\[0pt]
L'économie de la langue \\[0pt]
L'économie du don \\[0pt]
L'économie est-elle politique ? \\[0pt]
L'économie est-elle une science humaine ? \\[0pt]
L'économie et la valeur humaine \\[0pt]
L'économie fait-elle partie des sciences humaines ? \\[0pt]
L'économie politique \\[0pt]
L'économie primitive \\[0pt]
L'économie psychique \\[0pt]
L'économique et le politique \\[0pt]
Le conseil \\[0pt]
Le conseiller du prince \\[0pt]
Le consensus \\[0pt]
Le consensus social \\[0pt]
Le consentement \\[0pt]
Le consentement des gouvernés \\[0pt]
Le consentement du peuple \\[0pt]
Le contemporain \\[0pt]
Le contenu empirique \\[0pt]
Le contingent \\[0pt]
Le continu \\[0pt]
Le contradictoire peut-il exister ? \\[0pt]
Le contrat \\[0pt]
Le contrat social \\[0pt]
Le contrôle social \\[0pt]
Le convenable \\[0pt]
Le corps dansant \\[0pt]
Le corps dit-il quelque chose ? \\[0pt]
Le corps du prince \\[0pt]
Le corps est-il porteur de valeurs ? \\[0pt]
Le corps est-il respectable ? \\[0pt]
Le corps est-il un enjeu des sciences humaines ? \\[0pt]
Le corps et la danse \\[0pt]
Le corps et la machine \\[0pt]
Le corps et l'âme \\[0pt]
Le corps et l'esprit \\[0pt]
Le corps humain \\[0pt]
Le corps humain est-il naturel ? \\[0pt]
Le corps politique \\[0pt]
Le corps propre \\[0pt]
Le cosmopolitisme \\[0pt]
Le cosmopolitisme peut-il devenir réalité ? \\[0pt]
Le cosmopolitisme peut-il être réaliste ? \\[0pt]
Le coupable \\[0pt]
Le coup d'État \\[0pt]
Le couple \\[0pt]
Le courage \\[0pt]
Le courage est-il une vertu ? \\[0pt]
Le courage politique \\[0pt]
Le cours du temps \\[0pt]
Le créé et l'incréé \\[0pt]
Le cri \\[0pt]
Le crime suppose-t-il la loi ? \\[0pt]
Le critère \\[0pt]
L'écrit et l'oral \\[0pt]
Le critique d'art \\[0pt]
L'écriture des lois \\[0pt]
L'écriture et la parole \\[0pt]
Le cru et le cuit \\[0pt]
Le culte des ancêtres \\[0pt]
Le culte des images \\[0pt]
Le cynique \\[0pt]
Le cynisme \\[0pt]
Le dandysme \\[0pt]
Le danger \\[0pt]
Le débat \\[0pt]
Le débat politique \\[0pt]
Le décentrement ethnologique \\[0pt]
Le déclassement \\[0pt]
Le défaut \\[0pt]
Le déguisement \\[0pt]
Le dérèglement \\[0pt]
Le dernier mot \\[0pt]
Le désaccord \\[0pt]
Le désespoir \\[0pt]
Le désespoir est-il une faute morale ? \\[0pt]
Le déshonneur \\[0pt]
Le design \\[0pt]
Le désintéressement \\[0pt]
Le désintéressement esthétique \\[0pt]
Le désir de connaissance \\[0pt]
Le désir de domination \\[0pt]
Le désir de dominer \\[0pt]
Le désir de gloire \\[0pt]
Le désir de l'autre \\[0pt]
Le désir de pouvoir \\[0pt]
Le désir de savoir \\[0pt]
Le désir d'éternité \\[0pt]
Le désir de vérité \\[0pt]
Le désir d'infini \\[0pt]
Le désir d'originalité \\[0pt]
Le désir est-il sans limite ? \\[0pt]
Le désir et la loi \\[0pt]
Le désir et le manque \\[0pt]
Le désir métaphysique \\[0pt]
Le désir n'est-il qu'inquiétude ? \\[0pt]
Le désir peut-il atteindre son objet ? \\[0pt]
Le désœuvrement \\[0pt]
Le désordre \\[0pt]
Le désordre des choses \\[0pt]
Le désordre est-il étonnant ? \\[0pt]
Le désordre est-il un mal ? \\[0pt]
Le despote peut-il être éclairé ? \\[0pt]
Le despotisme \\[0pt]
Le despotisme peut-il être éclairé ? \\[0pt]
Le dessin \\[0pt]
Le dessin et la couleur \\[0pt]
Le destin \\[0pt]
Le désuet \\[0pt]
Le détachement \\[0pt]
Le détail \\[0pt]
Le déterminisme \\[0pt]
Le déterminisme psychique \\[0pt]
Le déterminisme social \\[0pt]
Le deuil \\[0pt]
Le développement moral \\[0pt]
Le devenir \\[0pt]
Le devenir est-il pensable ? \\[0pt]
Le devoir d'assistance \\[0pt]
Le devoir de vérité \\[0pt]
Le devoir d'obéissance \\[0pt]
Le devoir-être \\[0pt]
Le devoir s'apprend-il ? \\[0pt]
Le dévouement \\[0pt]
Le dialogue \\[0pt]
Le dialogue des philosophes \\[0pt]
Le dialogue entre les cultures \\[0pt]
Le dictionnaire \\[0pt]
Le dieu artiste \\[0pt]
Le dieu des philosophes \\[0pt]
Le Dieu des philosophes \\[0pt]
L'édification morale \\[0pt]
Le dilemme \\[0pt]
Le dire et le faire \\[0pt]
Le discernement \\[0pt]
Le discontinu \\[0pt]
Le discours politique \\[0pt]
Le divers \\[0pt]
Le divertissement \\[0pt]
Le divertissement est-il vain ? \\[0pt]
Le divin \\[0pt]
Le dogmatisme \\[0pt]
Le domestique \\[0pt]
Le don \\[0pt]
Le don de soi \\[0pt]
Le don et l'échange \\[0pt]
Le donné \\[0pt]
Le donné et le construit \\[0pt]
Le double \\[0pt]
Le doute dans les sciences \\[0pt]
Le doute est-il une faiblesse de la pensée ? \\[0pt]
Le doute métaphysique \\[0pt]
Le drame \\[0pt]
Le droit à l'erreur \\[0pt]
Le droit au bonheur \\[0pt]
Le droit au Bonheur \\[0pt]
Le droit au travail \\[0pt]
Le droit d'asile \\[0pt]
Le droit de disposer de son corps \\[0pt]
Le droit de guerre \\[0pt]
Le droit de la guerre \\[0pt]
Le droit de la majorité \\[0pt]
Le droit de propriété \\[0pt]
Le droit de punir \\[0pt]
Le droit de résistance \\[0pt]
Le droit de résistance à l'oppression \\[0pt]
Le droit de révolte \\[0pt]
Le droit des animaux \\[0pt]
Le droit des gens \\[0pt]
Le droit des minorités \\[0pt]
Le droit des peuples \\[0pt]
Le droit des peuples à disposer d'eux-mêmes \\[0pt]
Le droit de veto \\[0pt]
Le droit de vie et de mort \\[0pt]
Le droit de vivre \\[0pt]
Le droit d'ingérence \\[0pt]
Le droit d'intervention \\[0pt]
Le droit divin \\[0pt]
Le droit doit-il être le seul régulateur de la vie sociale ? \\[0pt]
Le droit du plus faible \\[0pt]
Le droit du plus fort \\[0pt]
Le droit du premier occupant \\[0pt]
Le droit est-il une science humaine ? \\[0pt]
Le droit et la violence \\[0pt]
Le droit humanitaire \\[0pt]
Le droit international \\[0pt]
Le droit peut-il être naturel ? \\[0pt]
Le dualisme \\[0pt]
L'éducation artistique \\[0pt]
L'éducation civique \\[0pt]
L'éducation des esprits \\[0pt]
L'éducation du goût \\[0pt]
L'éducation est-elle affaire de politique ? \\[0pt]
L'éducation est-elle par nature conservatrice ? \\[0pt]
L'éducation esthétique \\[0pt]
L'éducation peut-elle être sentimentale ? \\[0pt]
L'éducation physique \\[0pt]
L'éducation politique \\[0pt]
L'éducation relève-t-elle du politique ? \\[0pt]
Le fait de vivre est-il un bien en soi ? \\[0pt]
Le fait divers \\[0pt]
Le fait du prince \\[0pt]
Le fait et le droit \\[0pt]
Le fait historique \\[0pt]
Le fait religieux \\[0pt]
Le fait scientifique \\[0pt]
Le fanatisme \\[0pt]
Le fantastique \\[0pt]
Le faux en art \\[0pt]
Le faux et l'absurde \\[0pt]
Le faux et le fictif \\[0pt]
Le fédéralisme \\[0pt]
Le féminin et le masculin \\[0pt]
Le féminisme \\[0pt]
Le fétichisme \\[0pt]
Le fétichisme de la marchandise \\[0pt]
L'effet de la musique \\[0pt]
L'efficacité est-elle une vertu ? \\[0pt]
L'efficacité thérapeutique de la psychanalyse \\[0pt]
L'efficience \\[0pt]
L'effort \\[0pt]
Le finalisme \\[0pt]
Le fini et l'infini \\[0pt]
Le flegme \\[0pt]
Le folklore \\[0pt]
Le fonctionnaire \\[0pt]
Le fonctionnel et le beau \\[0pt]
Le fond \\[0pt]
Le fondement \\[0pt]
Le fondement de l'induction \\[0pt]
Le fond et la forme \\[0pt]
Le for intérieur \\[0pt]
Le formalisme \\[0pt]
Le formalisme dans les sciences humaines \\[0pt]
Le formalisme moral \\[0pt]
Le fou \\[0pt]
Le fragment \\[0pt]
Le frivole \\[0pt]
Le futur est-il contingent ? \\[0pt]
Le futur est-il nécessairement contingent ? \\[0pt]
L'égalité \\[0pt]
L'égalité civile \\[0pt]
L'égalité des chances \\[0pt]
L'égalité des citoyens \\[0pt]
L'égalité des conditions \\[0pt]
L'égalité des hommes est-elle un fait ou une idée ? \\[0pt]
L'égalité des hommes et des femmes est-elle une question politique ? \\[0pt]
L'égalité des sexes \\[0pt]
L'égalité est-elle souhaitable ? \\[0pt]
Légalité et légitimité \\[0pt]
Légalité et moralité \\[0pt]
Le génie \\[0pt]
Le génie du lieu \\[0pt]
Le génie du mal \\[0pt]
Le genre et l'espèce \\[0pt]
Le genre humain \\[0pt]
Le genre humain : unité ou pluralité ? \\[0pt]
Le geste \\[0pt]
Le geste créateur \\[0pt]
Le geste et la parole \\[0pt]
Légitimité et légalité \\[0pt]
L'égoïsme \\[0pt]
Le goût \\[0pt]
Le goût : certitude ou conviction ? \\[0pt]
Le goût de la fête \\[0pt]
Le goût de la polémique \\[0pt]
Le goût de l'artiste \\[0pt]
Le goût des autres \\[0pt]
Le goût du beau \\[0pt]
Le goût du pouvoir \\[0pt]
Le goût est-il nécessairement subjectif ? \\[0pt]
Le goût est-il une faculté ? \\[0pt]
Le goût est-il une question de classe ? \\[0pt]
Le goût est-il une vertu sociale ? \\[0pt]
Le goût et la distinction \\[0pt]
Le goût se forme-t-il ? \\[0pt]
Le gouvernement des experts \\[0pt]
Le gouvernement des hommes et l'administration des choses \\[0pt]
Le gouvernement des hommes libres \\[0pt]
Le gouvernement des juges \\[0pt]
Le gouvernement des meilleurs \\[0pt]
Le gouvernement par le peuple est-il nécessairement pour le peuple ? \\[0pt]
Le grandiose \\[0pt]
Le grand livre de la nature \\[0pt]
Le grotesque \\[0pt]
Le groupe \\[0pt]
Le hasard \\[0pt]
Le hasard existe-t-il ? \\[0pt]
Le hasard fait-il bien les choses ? \\[0pt]
Le hasard gouverne-t-il le monde ? \\[0pt]
Le hasard n'est-il que la mesure de notre ignorance ? \\[0pt]
Le hasard n'est-il que le nom de notre ignorance ? \\[0pt]
Le haut et le bas \\[0pt]
Le héros moral \\[0pt]
Le hors-la-loi \\[0pt]
Le je-ne-sais-quoi \\[0pt]
Le jeu \\[0pt]
Le jeu de mots \\[0pt]
Le jeu des apparences \\[0pt]
Le jeu des possibles \\[0pt]
Le jeu du monde \\[0pt]
Le jeu et la règle \\[0pt]
Le jeu social \\[0pt]
Le joli, le beau \\[0pt]
Le jugement \\[0pt]
Le jugement artistique se fait-il sans concept ? \\[0pt]
Le jugement de goût \\[0pt]
Le jugement de goût est-il universel ? \\[0pt]
Le jugement de l'histoire \\[0pt]
Le jugement dernier \\[0pt]
Le jugement de valeur est-il indifférent à la vérité ? \\[0pt]
Le jugement moral \\[0pt]
Le jugement politique \\[0pt]
Le jugement pratique \\[0pt]
Le juste et le bien \\[0pt]
Le juste et le légal \\[0pt]
Le juste milieu \\[0pt]
Le laboratoire \\[0pt]
Le laid \\[0pt]
Le langage de la pensée \\[0pt]
Le langage de l'art \\[0pt]
Le langage des sciences \\[0pt]
Le langage du corps \\[0pt]
Le langage est-il assimilable à un outil ? \\[0pt]
Le langage est-il d'essence poétique ? \\[0pt]
Le langage est-il l'instrument de la pensée ? \\[0pt]
Le langage et les langues \\[0pt]
Le langage fait-il obstacle à la connaissance ? \\[0pt]
Le langage poétique \\[0pt]
Le langage scientifique \\[0pt]
Le latent et le manifeste \\[0pt]
L'élégance \\[0pt]
Le législateur \\[0pt]
Le législateur et le juge \\[0pt]
L'élément \\[0pt]
L'élémentaire \\[0pt]
Le libertin \\[0pt]
Le libertin, le libertaire \\[0pt]
Le libre arbitre \\[0pt]
Le libre jeu des formes \\[0pt]
Le lien politique \\[0pt]
Le lien social \\[0pt]
Le lien social peut-il être compassionnel ? \\[0pt]
Le lien social relève-t-il de la politique ? \\[0pt]
Le lieu \\[0pt]
Le lieu commun \\[0pt]
Le lieu de la pensée \\[0pt]
L'éloge de la démesure \\[0pt]
Le loisir \\[0pt]
Le loisir caractérise-t-il l'homme libre ? \\[0pt]
Le luxe \\[0pt]
Le lyrisme \\[0pt]
Le magistrat \\[0pt]
Le maintien de l'ordre \\[0pt]
Le maître et le disciple \\[0pt]
Le mal \\[0pt]
Le mal apparaît-il toujours ? \\[0pt]
Le mal constitue-t-il une objection à l'existence de Dieu ? \\[0pt]
Le malentendu \\[0pt]
Le mal est-il une réalité objective ? \\[0pt]
Le malheur \\[0pt]
Le malin plaisir \\[0pt]
Le mal métaphysique \\[0pt]
Le mal peut-il être absolu ? \\[0pt]
L'émancipation \\[0pt]
L'émancipation des femmes \\[0pt]
Le maniérisme \\[0pt]
Le manifeste politique \\[0pt]
Le manque de culture \\[0pt]
Le marché \\[0pt]
Le marché de l'art \\[0pt]
Le mariage \\[0pt]
Le masque \\[0pt]
Le matériau musical \\[0pt]
Le mauvais goût \\[0pt]
L'embarras \\[0pt]
L'embarras du choix \\[0pt]
L'embryon humain est-il une personne ? \\[0pt]
L'embryon humain peut-il être l'objet d'expérimentation ? \\[0pt]
Le mécanisme \\[0pt]
Le mécanisme et la mécanique \\[0pt]
Le mécénat \\[0pt]
Le méchant peut-il être heureux ? \\[0pt]
Le meilleur \\[0pt]
Le meilleur des mondes possible \\[0pt]
Le meilleur régime \\[0pt]
Le meilleur régime politique \\[0pt]
Le mélange \\[0pt]
Le même et l'autre \\[0pt]
Le ménage \\[0pt]
Le mensonge \\[0pt]
Le mensonge de l'art ? \\[0pt]
Le mensonge en politique \\[0pt]
Le mensonge politique \\[0pt]
Le mépris \\[0pt]
Le mépris de classe \\[0pt]
Le mépris peut-il être justifié ? \\[0pt]
Le mérite \\[0pt]
Le mérite est-il le critère de la vertu ? \\[0pt]
Le mesurable \\[0pt]
Le métaphysicien est-il un physicien à sa façon ? \\[0pt]
Le métier \\[0pt]
Le métier de politique \\[0pt]
Le métier d'homme \\[0pt]
Le métier du politique \\[0pt]
Le mien et le tien \\[0pt]
Le mieux est-il l'ennemi du bien ? \\[0pt]
Le milieu \\[0pt]
Le milieu social \\[0pt]
Le miracle \\[0pt]
Le miroir \\[0pt]
Le mode \\[0pt]
Le mode d'existence de l'œuvre d'art \\[0pt]
Le modèle \\[0pt]
Le modèle du jeu dans les sciences humaines ? \\[0pt]
Le modèle en morale \\[0pt]
Le modèle et la copie \\[0pt]
Le modèle évolutionniste en histoire \\[0pt]
Le modèle organiciste \\[0pt]
Le modèle organiciste en sciences humaines \\[0pt]
Le moi \\[0pt]
Le moi est-il une collection d'idées ? \\[0pt]
Le moi est-il une illusion ? \\[0pt]
Le moi et ses images \\[0pt]
Le moindre mal \\[0pt]
Le monde à l'envers \\[0pt]
Le monde a-t-il une histoire ? \\[0pt]
Le monde aurait-il un sens si l'homme n'existait pas ? \\[0pt]
Le monde de la culture \\[0pt]
Le monde de l'animal \\[0pt]
Le monde de l'animal nous est-il étranger ? \\[0pt]
Le monde de l'art \\[0pt]
Le monde de l'art et l'ordinateur \\[0pt]
Le monde de la vie \\[0pt]
Le monde de l'entreprise \\[0pt]
Le monde de l'être humain \\[0pt]
Le monde des idées \\[0pt]
Le monde des machines \\[0pt]
Le monde des œuvres \\[0pt]
Le monde des physiciens \\[0pt]
Le monde des rêves \\[0pt]
Le monde des sens \\[0pt]
Le monde du rêve \\[0pt]
Le monde est-il en progrès ? \\[0pt]
Le monde est-il éternel ? \\[0pt]
Le monde est-il notre représentation ? \\[0pt]
Le monde est-il un théâtre ? \\[0pt]
Le monde et la nature \\[0pt]
Le monde intérieur \\[0pt]
Le monde politique \\[0pt]
Le monde se suffit-il à lui-même ? \\[0pt]
Le monde vrai \\[0pt]
Le monopole de la violence légitime \\[0pt]
Le monstre \\[0pt]
Le monstrueux \\[0pt]
Le moralisme \\[0pt]
Le moraliste \\[0pt]
Le mot d'esprit \\[0pt]
Le mot et la chose \\[0pt]
L'émotion \\[0pt]
L'émotion esthétique \\[0pt]
L'émotion esthétique peut-elle se communiquer ? \\[0pt]
L'émotion esthétique peut-elle se partager ? \\[0pt]
L'émotion morale \\[0pt]
Le mot juste \\[0pt]
Le mot vie a-t-il plusieurs sens ? \\[0pt]
Le mouvement \\[0pt]
Le mouvement de la pensée \\[0pt]
L'empathie \\[0pt]
L'empathie est-elle nécessaire aux sciences sociales ? \\[0pt]
L'empire \\[0pt]
L'empire sur soi \\[0pt]
L'empirisme des sciences humaines \\[0pt]
L'emploi du temps \\[0pt]
Le multiculturalisme \\[0pt]
Le multiple \\[0pt]
Le musée \\[0pt]
Le mystère \\[0pt]
Le mysticisme \\[0pt]
Le mythe \\[0pt]
Le mythe est-il objet de science ? \\[0pt]
Le naïf \\[0pt]
Le narcissisme \\[0pt]
Le naturalisme \\[0pt]
Le naturalisme des sciences humaines et sociales \\[0pt]
Le naturel \\[0pt]
Le naturel et l'artificiel \\[0pt]
Le naturel et le normal \\[0pt]
L'enchevêtrement des causes \\[0pt]
L'encyclopédie \\[0pt]
Le néant \\[0pt]
Le nécessaire et le contingent \\[0pt]
Le négatif \\[0pt]
Le néologisme \\[0pt]
L'énergie \\[0pt]
L'enfance \\[0pt]
L'enfance de l'art \\[0pt]
L'enfance est-elle ce qui doit être surmonté ? \\[0pt]
L'enfance peut-elle servir de modèle ? \\[0pt]
L'enfant sauvage \\[0pt]
L'enfer est-il véritablement pavé de bonnes intentions ? \\[0pt]
« L'enfer est pavé de bonnes intentions » \\[0pt]
L'engagement \\[0pt]
L'engagement dans l'art \\[0pt]
L'engagement politique \\[0pt]
Le nihilisme \\[0pt]
L'ennemi \\[0pt]
L'ennemi intérieur \\[0pt]
L'ennui \\[0pt]
Le noble et le vil \\[0pt]
Le noir et blanc \\[0pt]
Le nomade \\[0pt]
Le nomadisme \\[0pt]
Le nombre \\[0pt]
Le nombre et la mesure \\[0pt]
Le nom propre \\[0pt]
Le nom propre et le nom commun \\[0pt]
Le non-être \\[0pt]
Le non-sens \\[0pt]
Le normal et le pathologique \\[0pt]
L'enquête \\[0pt]
L'enquête de terrain \\[0pt]
L'enquête sociale \\[0pt]
L'enthousiasme \\[0pt]
L'enthousiasme est-il moral ? \\[0pt]
L'entraide \\[0pt]
Le nu \\[0pt]
L'envie \\[0pt]
L'environnement est-il un nouvel objet pour les sciences humaines ? \\[0pt]
Le oui-dire \\[0pt]
Le pacifisme \\[0pt]
Le paradigme \\[0pt]
Le paradoxal \\[0pt]
Le paradoxe \\[0pt]
Le pardon \\[0pt]
Le pardon et l'oubli \\[0pt]
Le parfait \\[0pt]
Le pari \\[0pt]
Le partage \\[0pt]
Le partage des biens \\[0pt]
Le partage des connaissances \\[0pt]
Le partage des savoirs \\[0pt]
Le partage est-il une obligation morale ? \\[0pt]
Le particulier \\[0pt]
Le passage à l'acte \\[0pt]
Le passé \\[0pt]
Le passé est-il objet de science ? \\[0pt]
Le passionné mérite-t-il d'être plaint ? \\[0pt]
Le pathologique \\[0pt]
Le patriarcat \\[0pt]
Le patrimoine \\[0pt]
Le patrimoine artistique \\[0pt]
Le patriotisme \\[0pt]
Le paysage \\[0pt]
Le paysage urbain \\[0pt]
Le pays natal \\[0pt]
Le péché \\[0pt]
Le pédagogue \\[0pt]
Le peintre et le sculpteur \\[0pt]
Le peintre et son modèle \\[0pt]
Le père, la mère, l'enfant \\[0pt]
Le pessimisme \\[0pt]
Le peuple a-t-il toujours raison ? \\[0pt]
Le peuple est-il souverain ? \\[0pt]
Le peuple et la nation \\[0pt]
Le peuple et les élites \\[0pt]
Le peuple possède-t-il une volonté ? \\[0pt]
Le phantasme \\[0pt]
L'éphémère \\[0pt]
Le phénomène \\[0pt]
Le phénomène religieux \\[0pt]
Le philanthrope \\[0pt]
Le philosophe a-t-il des leçons à donner au politique ? \\[0pt]
Le philosophe est-il le vrai politique ? \\[0pt]
Le philosophe et l'écrivain \\[0pt]
Le philosophe et le médecin \\[0pt]
Le philosophe et l'enfant \\[0pt]
Le philosophe-roi \\[0pt]
Le philosophe se détourne-t-il du réel ? \\[0pt]
L'épistémologie des sciences humaines \\[0pt]
L'épistémologie est-elle une logique de la science ? \\[0pt]
Le plagiat \\[0pt]
Le plaisir \\[0pt]
Le plaisir artistique est-il affaire de jugement ? \\[0pt]
Le plaisir a-t-il un rôle à jouer dans la morale ? \\[0pt]
Le plaisir de l'art \\[0pt]
Le plaisir d'imiter \\[0pt]
Le plaisir esthétique \\[0pt]
Le plaisir esthétique est-il un plaisir ? \\[0pt]
Le plaisir esthétique peut-il se communiquer ? \\[0pt]
Le plaisir esthétique suppose-t-il une culture ? \\[0pt]
Le plaisir esthétique suppose-t-il une culture esthétique ? \\[0pt]
Le plaisir est-il immoral ? \\[0pt]
Le plaisir est-il un bien ? \\[0pt]
Le plaisir et le bien \\[0pt]
Le plaisir se mérite-t-il ? \\[0pt]
Le pluralisme \\[0pt]
Le pluralisme politique \\[0pt]
Le plus et le moins \\[0pt]
Le plus grand des maux \\[0pt]
Le poète réinvente-t-il la langue ? \\[0pt]
Le poids des circonstances \\[0pt]
Le poids du passé \\[0pt]
Le poids du préjugé en politique \\[0pt]
Le point de vue \\[0pt]
Le point de vue de l'auteur \\[0pt]
Le point de vue du spectateur \\[0pt]
Le politique a-t-il à régler les passions ? \\[0pt]
Le politique a-t-il à régler les passions humaines ? \\[0pt]
Le politique doit-il être un technicien ? \\[0pt]
Le politique doit-il se soucier des émotions ? \\[0pt]
Le politique et le religieux \\[0pt]
Le politique et le social \\[0pt]
Le politique peut-il faire abstraction de la morale ? \\[0pt]
Le populaire \\[0pt]
Le populisme \\[0pt]
Le portrait \\[0pt]
Le possible \\[0pt]
Le possible et le contingent \\[0pt]
Le possible et le probable \\[0pt]
Le possible et le réel \\[0pt]
Le possible et l'existence \\[0pt]
Le possible et l'impossible \\[0pt]
Le postulat et l'axiome \\[0pt]
Le pour et le contre \\[0pt]
Le pouvoir absolu \\[0pt]
Le pouvoir causal de l'inconscient \\[0pt]
Le pouvoir corrompt-il ? \\[0pt]
Le pouvoir corrompt-il nécessairement ? \\[0pt]
Le pouvoir de la beauté \\[0pt]
Le pouvoir de la minorité \\[0pt]
Le pouvoir de l'argent \\[0pt]
Le pouvoir de la science \\[0pt]
Le pouvoir de l'image \\[0pt]
Le pouvoir de l'opinion \\[0pt]
Le pouvoir des idées \\[0pt]
Le pouvoir des images \\[0pt]
Le pouvoir des meilleurs \\[0pt]
Le pouvoir des mots \\[0pt]
Le pouvoir des savants \\[0pt]
Le pouvoir des sciences humaines et sociales \\[0pt]
Le pouvoir du peuple \\[0pt]
Le pouvoir législatif \\[0pt]
Le pouvoir peut-il arrêter le pouvoir ? \\[0pt]
Le pouvoir peut-il ignorer la morale ? \\[0pt]
Le pouvoir peut-il limiter le pouvoir ? \\[0pt]
Le pouvoir peut-il se déléguer ? \\[0pt]
Le pouvoir peut-il se passer de sa mise en scène ? \\[0pt]
Le pouvoir peut-il se transmettre ? \\[0pt]
Le pouvoir politique est-il nécessairement coercitif ? \\[0pt]
Le pouvoir politique repose-t-il sur la contrainte ? \\[0pt]
Le pouvoir politique repose-t-il sur un savoir ? \\[0pt]
Le pouvoir se partage-t-il ? \\[0pt]
Le pouvoir souverain \\[0pt]
Le pouvoir spirituel \\[0pt]
Le pouvoir tend-il toujours à l'excès ? \\[0pt]
Le pouvoir traditionnel \\[0pt]
Le préférable \\[0pt]
Le préjugé \\[0pt]
Le premier devoir de l'État est-il de se défendre ? \\[0pt]
Le premier et le primitif \\[0pt]
Le premier principe \\[0pt]
Le présent \\[0pt]
L'épreuve de la réalité \\[0pt]
L'épreuve du pouvoir \\[0pt]
Le primitivisme en art \\[0pt]
Le prince \\[0pt]
Le principe de causalité \\[0pt]
Le principe de contradiction \\[0pt]
Le principe d'égalité \\[0pt]
Le principe de précaution \\[0pt]
Le principe de raison \\[0pt]
Le principe de réalité \\[0pt]
Le principe de réciprocité \\[0pt]
Le principe de simplicité \\[0pt]
Le principe d'identité \\[0pt]
Le privilège de l'original \\[0pt]
Le prix \\[0pt]
Le probabilisme \\[0pt]
Le probable \\[0pt]
Le problème de l'être \\[0pt]
Le processus \\[0pt]
Le processus de civilisation \\[0pt]
Le prochain \\[0pt]
Le proche et le lointain \\[0pt]
Le profane \\[0pt]
Le profit est-il le moteur de la production économique ? \\[0pt]
Le progrès \\[0pt]
Le progrès des sciences \\[0pt]
Le progrès des sciences infirme-t-il les résultats anciens ? \\[0pt]
Le progrès en logique \\[0pt]
Le progrès moral \\[0pt]
Le progrès scientifique fait-il disparaître la superstition ? \\[0pt]
Le progrès technique \\[0pt]
Le projet \\[0pt]
Le projet d'une paix perpétuelle est-il insensé ? \\[0pt]
Le projet politique \\[0pt]
Le propre \\[0pt]
Le propre de la musique \\[0pt]
Le propre de l'homme \\[0pt]
Le propre et l'impropre \\[0pt]
Le propriétaire \\[0pt]
Le protectionnisme \\[0pt]
Le provisoire \\[0pt]
Le psychisme est-il objet de connaissance ? \\[0pt]
Le psychologisme \\[0pt]
Le public \\[0pt]
Le public et le privé \\[0pt]
Le pur et l'impur \\[0pt]
Lequel, de l'art ou du réel, est-il une imitation de l'autre ? \\[0pt]
Le quelque chose \\[0pt]
L'équilibre \\[0pt]
L'équilibre des pouvoirs \\[0pt]
L'équilibre économique \\[0pt]
L'équité \\[0pt]
L'équivalence \\[0pt]
L'équivalence des hypothèses \\[0pt]
L'équivocité \\[0pt]
L'équivoque \\[0pt]
Le quotidien \\[0pt]
Le raffinement \\[0pt]
Le raisonnement par l'absurde \\[0pt]
Le raisonnement scientifique \\[0pt]
Le raisonnement suit-il des règles ? \\[0pt]
Le rapport de force \\[0pt]
Le rapport de l'homme à son milieu a-t-il une dimension morale ? \\[0pt]
Le rationnel et le raisonnable \\[0pt]
Le réalisme \\[0pt]
Le réalisme de la science \\[0pt]
Le réalisme politique \\[0pt]
Le récit \\[0pt]
Le récit en histoire \\[0pt]
Le récit national \\[0pt]
Le réel \\[0pt]
Le réel est-il ce qui résiste ? \\[0pt]
Le réel est-il rationnel ? \\[0pt]
Le réel et le virtuel \\[0pt]
Le réel et l'idéal \\[0pt]
Le réel et l'impossible \\[0pt]
Le réel excède-t-il l'intelligible ? \\[0pt]
Le réel peut-il être beau ? \\[0pt]
Le réel peut-il être contradictoire ? \\[0pt]
Le réel se donne-t-il à voir ? \\[0pt]
Le refoulement \\[0pt]
Le refus \\[0pt]
Le refus de l'autorité \\[0pt]
Le refus de la vérité \\[0pt]
Le regard \\[0pt]
Le regard de l'autre \\[0pt]
Le regard du photographe \\[0pt]
Le regard éloigné \\[0pt]
Le règlement politique des conflits ? \\[0pt]
Le règne de l'homme \\[0pt]
Le règne des experts \\[0pt]
Le règne des fins \\[0pt]
Le regret \\[0pt]
Le relativisme \\[0pt]
Le relativisme culturel \\[0pt]
Le relativisme culturel est-il indépassable ? \\[0pt]
Le relativisme dans les sciences humaines \\[0pt]
Le relativisme moral \\[0pt]
Le relativisme peut-il être un principe de méthode dans les sciences humaines ? \\[0pt]
Le remords \\[0pt]
Le renoncement \\[0pt]
Le repas \\[0pt]
Le repentir \\[0pt]
Le repos \\[0pt]
Le respect \\[0pt]
Le respect des convenances \\[0pt]
Le respect des institutions \\[0pt]
Le ressentiment \\[0pt]
Le rêve \\[0pt]
Le rêve et la veille \\[0pt]
Le révisionnisme \\[0pt]
Le ridicule \\[0pt]
Le rien \\[0pt]
Le rigorisme \\[0pt]
Le rire \\[0pt]
Le risque \\[0pt]
Le risque calculé \\[0pt]
Le rituel \\[0pt]
Le rôle de la théorie dans l'expérience scientifique \\[0pt]
Le rôle des institutions \\[0pt]
Le roman national \\[0pt]
L'érotisme \\[0pt]
Le royaume du possible \\[0pt]
L'erreur et la faute \\[0pt]
L'erreur et l'ignorance \\[0pt]
L'erreur peut-elle jouer un rôle dans la connaissance scientifique ? \\[0pt]
L'erreur politique, la faute politique \\[0pt]
L'erreur scientifique \\[0pt]
L'érudition \\[0pt]
Le rythme \\[0pt]
Le rythme en peinture \\[0pt]
Le sacré \\[0pt]
Le sacré et le profane \\[0pt]
Le sacrifice \\[0pt]
Les affaires publiques \\[0pt]
Les affections \\[0pt]
Les affects sont-ils des objets sociologiques ? \\[0pt]
Les agents économiques ont-ils un comportement rationnel ? \\[0pt]
Les agents sociaux poursuivent-ils l'utilité ? \\[0pt]
Les agents sociaux sont-ils rationnels ? \\[0pt]
Les âges de la vie \\[0pt]
Le salut \\[0pt]
Le salut du peuple \\[0pt]
Les amis \\[0pt]
Les analogies dans les sciences humaines \\[0pt]
Les anciens et les modernes \\[0pt]
Les animaux échappent-ils à la moralité ? \\[0pt]
Les animaux ont-ils des droits ? \\[0pt]
Les animaux pensent-ils ? \\[0pt]
Les animaux révèlent-ils ce que nous sommes ? \\[0pt]
Les animaux sont-ils nos amis ? \\[0pt]
Les antagonismes sociaux \\[0pt]
Les apparences font-elles partie du monde ? \\[0pt]
Les apparences sont-elles toujours trompeuses ? \\[0pt]
Les archives \\[0pt]
Les artistes sont-ils sérieux ? \\[0pt]
Les arts appliqués \\[0pt]
Les arts communiquent-ils entre eux ? \\[0pt]
Les arts décoratifs \\[0pt]
Les arts de la mémoire \\[0pt]
Les arts industriels \\[0pt]
Les arts mineurs \\[0pt]
Les arts nobles \\[0pt]
Les arts ont-ils besoin de théorie ? \\[0pt]
Les arts populaires \\[0pt]
Les arts sont-ils des jeux ? \\[0pt]
Les arts vivants \\[0pt]
Les atomes \\[0pt]
Les autorités \\[0pt]
Les autres \\[0pt]
Le sauvage \\[0pt]
Le sauvage et le barbare \\[0pt]
Le sauvage et le civilisé \\[0pt]
Le sauvage et le cultivé \\[0pt]
Le savant et le politique \\[0pt]
Le savant, l'artiste et le politique \\[0pt]
Le savoir a-t-il besoin d'être fondé ? \\[0pt]
Le savoir du peintre \\[0pt]
Le savoir émancipe-t-il ? \\[0pt]
Le savoir est-il libérateur ? \\[0pt]
Le savoir-faire \\[0pt]
Le savoir peut-il être gratuit ? \\[0pt]
Le savoir se vulgarise-t-il ? \\[0pt]
Le savoir utile au citoyen \\[0pt]
Les beautés de la nature \\[0pt]
Les beaux-arts sont-ils compatibles entre eux ? \\[0pt]
Les belles actions \\[0pt]
Les bénéfices du doute \\[0pt]
Les bénéfices moraux \\[0pt]
Les bêtes \\[0pt]
Les biens communs \\[0pt]
Les biens culturels \\[0pt]
Les blessures de l'esprit \\[0pt]
Les bonnes intentions \\[0pt]
Les bonnes lois font-elles les bonnes mœurs ? \\[0pt]
Les bonnes mœurs \\[0pt]
Les bons sentiments \\[0pt]
Le scandale \\[0pt]
Les canons de la beauté \\[0pt]
Les caractères moraux \\[0pt]
Les cas de conscience \\[0pt]
Les catégories \\[0pt]
Les catégories sont-elles définitives ? \\[0pt]
Les catégories sont-elles des effets de langue ? \\[0pt]
Les causes \\[0pt]
Les causes et les conditions \\[0pt]
Les causes et les effets \\[0pt]
Les causes et les lois \\[0pt]
Les causes finales \\[0pt]
Les causes naturelles expliquent-elles la nature ? \\[0pt]
Les changements scientifiques et la réalité \\[0pt]
Les chemins de traverse \\[0pt]
Les choses \\[0pt]
Les choses humaines \\[0pt]
Les choses ont-elles quelque chose en commun ? \\[0pt]
Les choses ont-elles une essence ? \\[0pt]
Les choses peuvent-elles avoir un sens ? \\[0pt]
Les choses sont-elles dans l'espace ? \\[0pt]
Les cinq sens \\[0pt]
Les circonstances \\[0pt]
Les classes sociales \\[0pt]
L'esclavage \\[0pt]
L'esclave \\[0pt]
L'esclave et son maître \\[0pt]
Les conditions de la démocratie \\[0pt]
Les conditions d'une science historique \\[0pt]
Les conflits politiques \\[0pt]
Les conflits politiques ne sont-ils que des conflits sociaux ? \\[0pt]
Les conflits sociaux \\[0pt]
Les conflits sociaux sont-ils des conflits de classe ? \\[0pt]
Les conflits sociaux sont-ils des conflits politiques ? \\[0pt]
Les connaissances scientifiques peuvent-elles être à la fois vraies et provisoires ? \\[0pt]
Les connaissances scientifiques peuvent-elles être vulgarisées ? \\[0pt]
Les conquêtes de la science \\[0pt]
Les conséquences de l'action \\[0pt]
Les contradictions de la raison \\[0pt]
Les contradictions sont-elles un échec de la raison ? \\[0pt]
Les contre-pouvoirs \\[0pt]
Les convenances \\[0pt]
Les conventions \\[0pt]
Les couleurs \\[0pt]
Les couleurs existent-elles en dehors de notre esprit ? \\[0pt]
Les coutumes \\[0pt]
Les critères de vérité dans les sciences humaines \\[0pt]
Les croyances politiques \\[0pt]
Les croyances sont-elles utiles ? \\[0pt]
Le scrupule \\[0pt]
Les cultures sont-elles incommensurables ? \\[0pt]
Les décisions politiques peuvent-elles être collectives ? \\[0pt]
Les degrés de conscience \\[0pt]
Les degrés de la beauté \\[0pt]
Les degrés du beau \\[0pt]
Les désaccords politiques peuvent-ils être surmontés ? \\[0pt]
Les devoirs à l'égard de la nature \\[0pt]
Les devoirs de l'État \\[0pt]
Les devoirs envers nos proches \\[0pt]
Les devoirs envers soi-même \\[0pt]
Les dictionnaires \\[0pt]
Les différences entre les sciences humaines sont-elles des différences d'objet ? \\[0pt]
Les différentes preuves de l'existence de Dieu prouvent-elles le même Dieu ? \\[0pt]
Les difficultés de l'observation \\[0pt]
Les dilemmes moraux \\[0pt]
Les dispositions sociales \\[0pt]
Les distinctions sociales \\[0pt]
Les divisions entre les hommes \\[0pt]
Les divisions sociales \\[0pt]
Les dogmes \\[0pt]
Les droits de la nature \\[0pt]
Les droits de l'enfant \\[0pt]
Les droits de l'homme \\[0pt]
Les droits de l'homme et ceux du citoyen \\[0pt]
Les droits de l'homme ont-ils un fondement moral ? \\[0pt]
Les droits de l'homme sont-ils ceux du citoyen ? \\[0pt]
Les droits de l'homme sont-ils une abstraction ? \\[0pt]
Les droits et les devoirs \\[0pt]
Les droits naturels imposent-ils une limite à la politique ? \\[0pt]
Les droits sociaux \\[0pt]
Les échanges \\[0pt]
Le secret \\[0pt]
Le secret d'État \\[0pt]
Les effets de l'esclavage \\[0pt]
Les éléments \\[0pt]
Le semblable \\[0pt]
Les émotions politiques \\[0pt]
Le sens commun \\[0pt]
Le sens commun existe-t-il ? \\[0pt]
Le sens de la famille \\[0pt]
Le sens de la fête \\[0pt]
Le sens de la mesure \\[0pt]
Le sens de la réalité \\[0pt]
Le sens de la situation \\[0pt]
Le sens de la vie \\[0pt]
Le sens de l'État \\[0pt]
Le sens de l'être \\[0pt]
Le sens de l'histoire \\[0pt]
Le sens de l'Histoire \\[0pt]
Le sens de l'humour \\[0pt]
Le sens de l'opportunité est-il moral ? \\[0pt]
Le sens des mots \\[0pt]
Le sens des mots dépend-il de notre connaissance des choses ? \\[0pt]
Le sens d'un texte \\[0pt]
Le sens du sacrifice \\[0pt]
Le sens du travail \\[0pt]
Les ensembles \\[0pt]
Le sens et le non-sens \\[0pt]
Le sensible \\[0pt]
Le sensible peut-il fonder des certitudes ? \\[0pt]
Le sens interne \\[0pt]
Le sens moral \\[0pt]
Le sens musical \\[0pt]
Le sens pratique \\[0pt]
Le sens propre \\[0pt]
Le sentiment de la faute \\[0pt]
Le sentiment de l'existence \\[0pt]
Le sentiment de l'injustice \\[0pt]
Le sentiment d'injustice \\[0pt]
Le sentiment du libre arbitre \\[0pt]
Le sentiment esthétique \\[0pt]
Le sentiment moral \\[0pt]
Les entités mathématiques sont-elles des fictions ? \\[0pt]
Les envieux \\[0pt]
Les épreuves ont-elles un sens ? \\[0pt]
Le sérieux \\[0pt]
Le serment \\[0pt]
Le service public \\[0pt]
Les États ont-ils pour fin d'assurer la paix ? \\[0pt]
Les éthiques professionnelles \\[0pt]
Les études comparatives \\[0pt]
Les études de genre \\[0pt]
Les événements \\[0pt]
Les exceptions ont-elles une place en morale ? \\[0pt]
Les experts \\[0pt]
Les factions politiques \\[0pt]
Les faits \\[0pt]
Les faits et les valeurs \\[0pt]
Les faits parlent-ils d'eux-mêmes ? \\[0pt]
Les faits sociaux se répètent-ils ? \\[0pt]
Les faits sont-ils têtus ? \\[0pt]
Les fausses sciences \\[0pt]
Les fictions \\[0pt]
Les fins de l'art \\[0pt]
Les fins de l'éducation \\[0pt]
Les fins dernières \\[0pt]
Les fins naturelles et les fins morales \\[0pt]
Les fins sont-elles toujours intentionnelles ? \\[0pt]
Les fonctions de l'image \\[0pt]
Les fondements de l'État \\[0pt]
Les forces de l'ordre \\[0pt]
Les formes de vie \\[0pt]
Les forts et les faibles \\[0pt]
Les foules \\[0pt]
Les fous \\[0pt]
Les frontières \\[0pt]
Les frontières de la connaissance \\[0pt]
Les frontières de l'art \\[0pt]
Les frontières du moi \\[0pt]
Les fruits du travail \\[0pt]
Les genres de Dieu \\[0pt]
Les genres de l'être \\[0pt]
Les genres esthétiques \\[0pt]
Les genres naturels \\[0pt]
Les grands hommes \\[0pt]
Les habitudes de pensée \\[0pt]
Les hasards de la vie \\[0pt]
Les hommes croient-ils dans leurs mythes ? \\[0pt]
Les hommes de pouvoir \\[0pt]
Les hommes et les dieux \\[0pt]
Les hommes et les femmes \\[0pt]
Les hommes n'agissent-ils que par intérêt ? \\[0pt]
Les hommes sont-ils naturellement sociables ? \\[0pt]
Les hors-la-loi \\[0pt]
Les idées et les choses \\[0pt]
Les idées existent-elles ? \\[0pt]
Les idées mènent-elles le monde ? \\[0pt]
Les idées ont-elles une histoire ? \\[0pt]
Les idées ont-elles une réalité ? \\[0pt]
Les idées politiques \\[0pt]
Les idées sont-elles vivantes ? \\[0pt]
Les idoles \\[0pt]
Le signe et l'indice \\[0pt]
Le silence \\[0pt]
Le silence des lois \\[0pt]
Les images empêchent-elles de penser ? \\[0pt]
Les images nous égarent-elles ? \\[0pt]
Les images peuvent-elles instruire ? \\[0pt]
Le simple \\[0pt]
Le simple et le complexe \\[0pt]
Le simulacre \\[0pt]
Les individus \\[0pt]
Les individus se réduisent-ils à leurs qualités ? \\[0pt]
Les industries culturelles \\[0pt]
Les inégalités sociales \\[0pt]
Les inégalités sociales sont-elles inévitables ? \\[0pt]
Le singulier \\[0pt]
Le singulier est-il objet de connaissance ? \\[0pt]
Le singulier et le pluriel \\[0pt]
Les institutions artistiques \\[0pt]
Les institutions de l'art \\[0pt]
Les institutions politiques : qu'ont-elles en propre ? \\[0pt]
Les instruments de la pensée \\[0pt]
Les intentions de l'artiste \\[0pt]
Les intentions et les conséquences \\[0pt]
Les interdits \\[0pt]
Les intérêts particuliers peuvent-ils tempérer l'autorité politique ? \\[0pt]
Les invariants culturels \\[0pt]
Les jeux de hasard \\[0pt]
Les jeux du pouvoir \\[0pt]
Les jugements analytiques \\[0pt]
Les jugements de goût sont-ils susceptibles de vérité ? \\[0pt]
Les langages de l'art \\[0pt]
Les langages formels \\[0pt]
Les langues meurent-elles ? \\[0pt]
Les leçons de l'expérience \\[0pt]
Les leçons de morale \\[0pt]
Les libertés civiles \\[0pt]
Les libertés fondamentales \\[0pt]
Les libertés politiques \\[0pt]
Les libertés publiques \\[0pt]
Les liens \\[0pt]
Les liens sociaux \\[0pt]
Les lieux de mémoire \\[0pt]
Les lieux du pouvoir \\[0pt]
Les limites de la connaissance scientifique \\[0pt]
Les limites de la démocratie \\[0pt]
Les limites de la description \\[0pt]
Les limites de la raison \\[0pt]
Les limites de la tolérance \\[0pt]
Les limites de la vertu \\[0pt]
Les limites de l'État \\[0pt]
Les limites de l'expérience \\[0pt]
Les limites de l'humain \\[0pt]
Les limites de l'imagination \\[0pt]
Les limites de l'interprétation \\[0pt]
Les limites de ma langue sont-elles les limites de mon monde ? \\[0pt]
Les limites du corps \\[0pt]
Les limites du déterminisme \\[0pt]
Les limites du langage \\[0pt]
Les limites du plaisir \\[0pt]
Les limites du possible \\[0pt]
Les limites du pouvoir \\[0pt]
Les limites du pouvoir politique \\[0pt]
Les lois causales \\[0pt]
Les lois de la guerre \\[0pt]
Les lois de la nature \\[0pt]
Les lois de la nature sont-elles contingentes ? \\[0pt]
Les lois de la nature sont-elles de simples régularités ? \\[0pt]
Les lois de la nature sont elles nécessaires ? \\[0pt]
Les lois de l'art \\[0pt]
Les lois de l'histoire \\[0pt]
Les lois de l'hospitalité \\[0pt]
Les lois du sang \\[0pt]
Les lois et les mœurs \\[0pt]
Les lois scientifiques sont-elles des lois de la nature ? \\[0pt]
Les lois sont-elles sacrées ? \\[0pt]
Les lois sont-elles seulement utiles ? \\[0pt]
Les luttes politiques \\[0pt]
Les machines \\[0pt]
Les maladies de l'âme \\[0pt]
Les maladies de l'esprit \\[0pt]
Les maladies mentales sont-elles des maladies sociales ? \\[0pt]
Les marginaux \\[0pt]
Les masses \\[0pt]
Les mathématiques du mouvement \\[0pt]
Les mathématiques et la pensée de l'infini \\[0pt]
Les mathématiques et le monde sensible \\[0pt]
Les mathématiques peuvent-elles se passer de l'intuition ? \\[0pt]
Les mathématiques sont-elles le modèle de toute science ? \\[0pt]
Les mathématiques sont-elles réductibles à la logique ? \\[0pt]
Les mathématiques sont-elles une science de l'ordre ? \\[0pt]
Les mathématiques sont-elles un langage ? \\[0pt]
Les mathématiques sont-elles utiles au philosophe ? \\[0pt]
Les mauvaises habitudes \\[0pt]
Les mauvaises intentions \\[0pt]
Les mécanismes cérébraux \\[0pt]
Les mécanismes d'identification \\[0pt]
Les métamorphoses du goût \\[0pt]
Les miracles \\[0pt]
Les miracles de la technique \\[0pt]
Les modalités \\[0pt]
Les modèles \\[0pt]
Les mœurs \\[0pt]
Les mœurs et la morale \\[0pt]
Les mœurs sont-elles objet de science ? \\[0pt]
Les mœurs sont-ils l'objet d'une science ? \\[0pt]
Les mondes possibles \\[0pt]
Les mots et les choses \\[0pt]
Les mots justes \\[0pt]
Les mots ont-ils un pouvoir ? \\[0pt]
Les moyens de l'art \\[0pt]
Les moyens de l'autorité \\[0pt]
Les moyens et la fin \\[0pt]
Les moyens et les fins \\[0pt]
Les moyens et les fins en art \\[0pt]
Les muses \\[0pt]
Les mythes collectifs \\[0pt]
Les mythes relèvent-ils de l'anthropologie ? \\[0pt]
Les nombres gouvernent-ils le monde ? \\[0pt]
Les noms \\[0pt]
Les noms divins \\[0pt]
Les noms propres \\[0pt]
Les normes \\[0pt]
Les normes du savoir sont-elles universelles ? \\[0pt]
Les normes du vivant \\[0pt]
Les normes esthétiques \\[0pt]
Les normes et les valeurs \\[0pt]
Les nouvelles technologies transforment-elles l'idée de l'art ? \\[0pt]
Les objets de pensée \\[0pt]
Les objets existent-ils ? \\[0pt]
Les objets impossibles \\[0pt]
Les objets scientifiques \\[0pt]
Le social et le politique \\[0pt]
Le sociologisme \\[0pt]
Le sociologue et l'historien \\[0pt]
Les œuvres d'art doivent-elles faire l'objet d'une évaluation morale ? \\[0pt]
Les œuvres d'art ont-elles besoin d'un commentaire ? \\[0pt]
Le soin \\[0pt]
Le soleil se lèvera demain \\[0pt]
Le solipsisme \\[0pt]
Le sommeil et la veille \\[0pt]
Les opérations de la pensée \\[0pt]
Les opinions politiques \\[0pt]
Les origines de l'humanité \\[0pt]
L'ésotérisme \\[0pt]
Le souci \\[0pt]
Le souci d'autrui résume-t-il la morale ? \\[0pt]
Le souci de l'ordre \\[0pt]
Le souci de soi \\[0pt]
Le souci du bien-être est-il politique ? \\[0pt]
Le souverain bien \\[0pt]
L'espace dans la peinture \\[0pt]
L'espace du tableau \\[0pt]
L'espace et le lieu \\[0pt]
L'espace et le territoire \\[0pt]
L'espace intérieur \\[0pt]
L'espace public \\[0pt]
L'espace social \\[0pt]
Les paroles et les actes \\[0pt]
« Les paroles s'envolent, les écrits restent » \\[0pt]
Les parties de l'âme \\[0pt]
Les partis politiques \\[0pt]
Les passions peuvent-elles être raisonnables ? \\[0pt]
Les passions politiques \\[0pt]
Les passions sont-elles mauvaises ? \\[0pt]
Les pauvres \\[0pt]
L'espèce et l'individu \\[0pt]
L'espèce humaine \\[0pt]
Le spectacle \\[0pt]
Le spectacle de la nature \\[0pt]
Le spectacle de la pensée \\[0pt]
Le spectateur \\[0pt]
L'espérance \\[0pt]
L'espérance est-elle une vertu ? \\[0pt]
Les personnes et les choses \\[0pt]
Les peuples ont-ils les gouvernements qu'ils méritent ? \\[0pt]
Les phénomènes \\[0pt]
Les phénomènes de mode \\[0pt]
Les plaisirs \\[0pt]
Les plaisirs de l'amitié \\[0pt]
Les poètes et la cité \\[0pt]
L'espoir \\[0pt]
Les politiques publiques \\[0pt]
Les pouvoirs de la parole \\[0pt]
Les pouvoirs de la religion \\[0pt]
Les préjugés moraux \\[0pt]
Les prêtres \\[0pt]
Les preuves de la liberté \\[0pt]
Les principes de la démonstration \\[0pt]
Les principes d'une science sont-ils des conventions ? \\[0pt]
Les principes moraux \\[0pt]
L'esprit appartient-il à la nature ? \\[0pt]
L'esprit critique \\[0pt]
L'esprit de finesse \\[0pt]
L'esprit de sérieux \\[0pt]
L'esprit de système \\[0pt]
L'esprit est-il matériel ? \\[0pt]
L'esprit est-il objet de science ? \\[0pt]
L'esprit et la machine \\[0pt]
L'esprit humain progresse-t-il ? \\[0pt]
L'esprit peut-il être malade ? \\[0pt]
L'esprit peut-il être mesuré ? \\[0pt]
L'esprit scientifique \\[0pt]
Les problèmes politiques peuvent-ils se ramener à des problèmes techniques ? \\[0pt]
Les problèmes politiques sont-ils des problèmes techniques ? \\[0pt]
Les procédures de validation dans les sciences humaines ? \\[0pt]
Les propositions métaphysiques sont-elles des illusions ? \\[0pt]
Les proverbes \\[0pt]
Les proverbes enseignent-ils quelque chose ? \\[0pt]
Les proverbes nous instruisent-ils moralement ? \\[0pt]
Les qualités esthétiques \\[0pt]
Les questions métaphysiques ont-elles un sens ? \\[0pt]
L'esquisse \\[0pt]
Les raisons d'aimer \\[0pt]
Les raisons de vivre \\[0pt]
Les rapports sociaux sont-ils des rapports entre individus ? \\[0pt]
Les règles de l'art \\[0pt]
Les règles du jeu \\[0pt]
Les règles d'un bon gouvernement \\[0pt]
Les règles sociales \\[0pt]
Les règles sociales sont-elles répressives ? \\[0pt]
Les relations \\[0pt]
Les relations ont-elles une existence objective ? \\[0pt]
Les relations sociales \\[0pt]
Les remords \\[0pt]
Les représentants du peuple \\[0pt]
Les représentations collectives \\[0pt]
Les reproductions \\[0pt]
Les réseaux sociaux \\[0pt]
Les ressources de l'analogie \\[0pt]
Les ressources humaines \\[0pt]
Les révolutions scientifiques \\[0pt]
Les révolutions techniques suscitent-elles des révolutions dans l'art ? \\[0pt]
Les riches et les pauvres \\[0pt]
Les rites \\[0pt]
Les rites funéraires \\[0pt]
Les rituels \\[0pt]
Les rôles sociaux \\[0pt]
Les ruines \\[0pt]
Les sacrifices \\[0pt]
Les sans-papiers \\[0pt]
Les sauvages \\[0pt]
Les sciences décrivent-elles le réel ? \\[0pt]
Les sciences de la vie et de la Terre \\[0pt]
Les sciences de la vie visent-elles un objet irréductible à la matière ? \\[0pt]
Les sciences de l'éducation \\[0pt]
Les sciences de l'esprit \\[0pt]
Les sciences de l'homme et l'évolution \\[0pt]
Les sciences de l'homme ont-elles inventé leur objet ? \\[0pt]
Les sciences de l'homme permettent-elles d'affiner la notion de responsabilité ? \\[0pt]
Les sciences de l'homme peuvent-elles expliquer l'impuissance de la liberté ? \\[0pt]
Les sciences de l'homme rendent-elles l'homme prévisible ? \\[0pt]
Les sciences de l'homme sont-elles utiles à la société ? \\[0pt]
Les sciences doivent-elle prétendre à l'unification ? \\[0pt]
Les sciences du comportement \\[0pt]
Les sciences et le vivant \\[0pt]
Les sciences exactes \\[0pt]
Les sciences forment-elle un système ? \\[0pt]
Les sciences historiques \\[0pt]
Les sciences humaines doivent-elles être transdisciplinaires ? \\[0pt]
Les sciences humaines échappent-elles à l'abstraction ? \\[0pt]
Les sciences humaines éliminent-elles la contingence du futur ? \\[0pt]
Les sciences humaines, est-ce la mort de l'homme ? \\[0pt]
Les sciences humaines et le droit \\[0pt]
Les sciences humaines et l'expérience \\[0pt]
Les sciences humaines et sociales n'ont-elles de sciences que le nom ? \\[0pt]
Les sciences humaines finissent-elles de ruiner l'idée de liberté ? \\[0pt]
Les sciences humaines jouent elles un rôle dans l'exploitation de l'homme par l'homme ? \\[0pt]
Les sciences humaines nient-elles la liberté de l'homme ? \\[0pt]
Les sciences humaines nient-elles la liberté du sujet humain ? \\[0pt]
Les sciences humaines nous font-elles connaître l'homme ? \\[0pt]
Les sciences humaines nous obligent-elles à renoncer à l'idée de libre arbitre ? \\[0pt]
Les sciences humaines nous protègent-elles de l'essentialisme ? \\[0pt]
Les sciences humaines ont-elles une utilité ? \\[0pt]
Les sciences humaines ont-elles un objet commun ? \\[0pt]
Les sciences humaines opposent-elles histoire et système ? \\[0pt]
Les sciences humaines permettent-elles de comprendre la vie d'un homme ? \\[0pt]
Les sciences humaines peuvent-elles adopter les méthodes des sciences de la nature ? \\[0pt]
Les sciences humaines peuvent-elles éclairer la réflexion éthique ? \\[0pt]
Les sciences humaines peuvent-elles être objectives ? \\[0pt]
Les sciences humaines peuvent-elles faire abstraction de la nature ? \\[0pt]
Les sciences humaines peuvent-elles s'affranchir de l'ethnocentrisme ? \\[0pt]
Les sciences humaines peuvent-elles se passer de la notion d'inconscient ? \\[0pt]
Les sciences humaines présupposent-elles une définition de l'homme ? \\[0pt]
Les sciences humaines recherchent-elles des invariants ? \\[0pt]
Les sciences humaines rendent-elle l'homme prévisible ? \\[0pt]
Les sciences humaines reposent-elles sur des faits ? \\[0pt]
Les sciences humaines sont-elles des sciences ? \\[0pt]
Les sciences humaines sont-elles des sciences de la nature humaine ? \\[0pt]
Les sciences humaines sont-elles des sciences de la vie humaine ? \\[0pt]
Les sciences humaines sont-elles des sciences de l'esprit ? \\[0pt]
Les sciences humaines sont-elles des sciences de l'interprétation ? \\[0pt]
Les sciences humaines sont-elles des sciences d'interprétation ? \\[0pt]
Les sciences humaines sont-elles explicatives ou compréhensives ? \\[0pt]
Les sciences humaines sont-elles les sciences de l'esprit ? \\[0pt]
Les sciences humaines sont-elles nécessairement empiriques ? \\[0pt]
Les sciences humaines sont-elles normatives ? \\[0pt]
Les sciences humaines sont-elles relativistes ? \\[0pt]
Les sciences humaines sont-elles subversives ? \\[0pt]
Les sciences humaines suffisent-elles à connaître l'homme ? \\[0pt]
Les sciences humaines supposent-elles une définition de l'homme ? \\[0pt]
Les sciences humaines supposent-elles une rupture avec l'ordre naturel ? \\[0pt]
Les sciences humaines traitent-elles de l'homme ? \\[0pt]
Les sciences humaines traitent-elles de l'individu ? \\[0pt]
Les sciences humaines transforment-elles la notion de causalité ? \\[0pt]
Les sciences naturelles \\[0pt]
Les sciences ont-elles à penser leurs fondements ? \\[0pt]
Les sciences ont-elles besoin d'une fondation ? \\[0pt]
Les sciences ont-elles besoin d'une fondation métaphysique ? \\[0pt]
Les sciences peuvent-elles penser leurs fondements ? \\[0pt]
Les sciences peuvent-elles penser l'individu ? \\[0pt]
Les sciences produisent-elles des normes ? \\[0pt]
Les sciences sociales \\[0pt]
Les sciences sociales peuvent-elles être expérimentales ? \\[0pt]
Les sciences sociales peuvent-elles servir à gouverner ? \\[0pt]
Les sciences sociales sont-elles nécessairement inexactes ? \\[0pt]
Les secrets \\[0pt]
L'essence \\[0pt]
L'essence et l'existence \\[0pt]
Les sensations peuvent-elles se partager ? \\[0pt]
Les sens humains \\[0pt]
Les sens nous trompent-ils ? \\[0pt]
Les sens peuvent-ils nous tromper ? \\[0pt]
Les sentiments \\[0pt]
Les sentiments peuvent-ils s'apprendre ? \\[0pt]
Les services publics \\[0pt]
Les signes \\[0pt]
Les signes de l'humanité \\[0pt]
Les signes de l'intelligence \\[0pt]
Les sociétés animales \\[0pt]
Les sociétés évoluent-elles ? \\[0pt]
Les sociétés humaines se construisent-elles contre la nature ? \\[0pt]
Les sociétés ont-elles besoin de théâtre ? \\[0pt]
Les sociétés ont-elles un inconscient ? \\[0pt]
Les sociétés sans histoire \\[0pt]
Les sociétés savantes \\[0pt]
Les sociétés sont-elles hiérarchisables ? \\[0pt]
Les sociétés sont-elles imprévisibles ? \\[0pt]
Les statistiques comme outil dans les sciences humaines \\[0pt]
Les structures expliquent-elles tout ? \\[0pt]
Les structures que dégagent les sciences humaines agissent-elles ? \\[0pt]
Les structures sociales sont-elles économiques ? \\[0pt]
Les styles \\[0pt]
Les substances \\[0pt]
Les symboles du pouvoir \\[0pt]
Les systèmes \\[0pt]
Les tabous \\[0pt]
Le statut de l'axiome \\[0pt]
Le statut des hypothèses dans la démarche scientifique \\[0pt]
Les techniques artistiques \\[0pt]
Les techniques de manipulation des hommes peuvent-elles s'appuyer sur les sciences humaines ? \\[0pt]
Les théories scientifiques sont-elles vraies ? \\[0pt]
L'esthète \\[0pt]
L'esthète et l'artiste \\[0pt]
L'esthétique des ruines \\[0pt]
L'esthétique est-elle une métaphysique de l'art ? \\[0pt]
L'esthétisme \\[0pt]
L'estime \\[0pt]
L'estime de soi \\[0pt]
Les traditions \\[0pt]
Les transcendantaux \\[0pt]
Le style \\[0pt]
Le sublime \\[0pt]
Le substrat \\[0pt]
Le succès \\[0pt]
Le suicide \\[0pt]
Le sujet \\[0pt]
Le sujet de l'action \\[0pt]
Le sujet de la pensée \\[0pt]
Le sujet et l'objet \\[0pt]
Le sujet moral \\[0pt]
Les universaux \\[0pt]
Les universaux existent-ils ? \\[0pt]
Le superflu \\[0pt]
Les usages \\[0pt]
Les usages de l'art \\[0pt]
Les valeurs de la République \\[0pt]
Les valeurs ont-elles une histoire ? \\[0pt]
Les valeurs ont-elles une origine ? \\[0pt]
Les vérités éternelles \\[0pt]
Les vérités scientifiques sont-elles relatives ? \\[0pt]
Les vérités sont-elles toujours démontrables ? \\[0pt]
Les vertus \\[0pt]
Les vertus de la formalisation \\[0pt]
Les vertus politiques \\[0pt]
Les vertus sont-elles au principe de la morale ? \\[0pt]
Les vices de l'esprit \\[0pt]
Les visages du mal \\[0pt]
Les vivants et les morts \\[0pt]
Le syllogisme \\[0pt]
Le symbole \\[0pt]
Le symbole, comme concept, dans les sciences humaines \\[0pt]
Le symbolique \\[0pt]
Le symbolisme \\[0pt]
Le symbolisme mathématique \\[0pt]
Le système des arts \\[0pt]
Le système des beaux-arts \\[0pt]
Le système des besoins \\[0pt]
Le tableau \\[0pt]
Le tableau vivant \\[0pt]
Le tact \\[0pt]
Le talent \\[0pt]
Le talent et le génie \\[0pt]
Le talent s'enseigne-t-il ? \\[0pt]
L'étant est-il le premier pensable ? \\[0pt]
L'État a-t-il pour finalité de maintenir l'ordre ? \\[0pt]
L'État a-t-il un fondement rationnel ? \\[0pt]
« L'État, c'est moi » \\[0pt]
L'État crée-t-il la liberté ? \\[0pt]
L'État de droit \\[0pt]
L'État de droit ? \\[0pt]
L'état de la nature \\[0pt]
L'état de nature \\[0pt]
L'état d'exception \\[0pt]
L'État doit-il disparaître ? \\[0pt]
L'État doit-il éduquer le citoyen ? \\[0pt]
L'État doit-il éduquer les citoyens ? \\[0pt]
L'État doit-il faire le bonheur des citoyens ? \\[0pt]
L'État est-il ennemi de la liberté ? \\[0pt]
L'État est-il fin ou moyen ? \\[0pt]
L'État est-il le garant de la propriété privée ? \\[0pt]
L'État est-il l'ennemi de la liberté ? \\[0pt]
L'État et la culture \\[0pt]
L'État et la Nation \\[0pt]
L'État et la violence \\[0pt]
L'État et le marché \\[0pt]
L'État et les Églises \\[0pt]
L'État libéral \\[0pt]
L'État n'est-il qu'un moyen ? \\[0pt]
L'État peut-il avoir tous les droits ? \\[0pt]
L'État peut-il créer la liberté ? \\[0pt]
L'État peut-il disparaître ? \\[0pt]
L'État peut-il être fondé sur un contrat ? \\[0pt]
L'État peut-il être indifférent à la religion ? \\[0pt]
L'État-Providence \\[0pt]
L'État universel \\[0pt]
Le témoignage \\[0pt]
Le témoignage des sens \\[0pt]
Le temps de l'art \\[0pt]
Le temps du monde \\[0pt]
Le temps est-il une dimension de la nature ? \\[0pt]
Le temps est-il une réalité ? \\[0pt]
Le temps ne fait-il que passer ? \\[0pt]
Le temps perdu \\[0pt]
Le temps politique \\[0pt]
Le temps rend-il tout vain ? \\[0pt]
Le temps se laisse-t-il décrire logiquement ? \\[0pt]
L'éternel présent \\[0pt]
L'éternité \\[0pt]
L'éternité n'est-elle qu'une illusion ? \\[0pt]
Le « terrain » \\[0pt]
Le terrain \\[0pt]
Le territoire \\[0pt]
Le théâtre du monde \\[0pt]
Le théâtre et les mœurs \\[0pt]
L'éthique à l'épreuve du tragique \\[0pt]
L'éthique des plaisirs \\[0pt]
L'éthique du spectateur \\[0pt]
L'éthique est-elle affaire de choix ? \\[0pt]
L'éthique suppose-t-elle la liberté ? \\[0pt]
L'ethnocentrisme \\[0pt]
L'ethnocentrisme comme problème pour l'ethnologie \\[0pt]
L'ethnologie est-elle une science mélancolique ? \\[0pt]
Le tiers exclu \\[0pt]
L'étonnement \\[0pt]
Le totalitarisme \\[0pt]
Le totémisme \\[0pt]
Le toucher \\[0pt]
Le tourment moral \\[0pt]
Le tout est-il la somme de ses parties ? \\[0pt]
Le tout et la partie \\[0pt]
Le tout et les parties \\[0pt]
Le tragique \\[0pt]
Le trait d'esprit \\[0pt]
L'étranger \\[0pt]
L'étrangeté \\[0pt]
Le transcendantal \\[0pt]
Le travail \\[0pt]
Le travail artistique \\[0pt]
Le travail artistique doit-il demeurer caché ? \\[0pt]
Le travail bien fait \\[0pt]
Le travail est-il une valeur morale ? \\[0pt]
Le travail rapproche-t-il les hommes ? \\[0pt]
Le travail sur le terrain \\[0pt]
L'être de la conscience \\[0pt]
L'être de la vérité \\[0pt]
L'être de l'image \\[0pt]
L'être du possible \\[0pt]
L'être en tant qu'être \\[0pt]
L'être en tant qu'être est-il connaissable ? \\[0pt]
L'être est-il un genre ? \\[0pt]
L'être et l'apparence \\[0pt]
L'être et la volonté \\[0pt]
L'être et le bien \\[0pt]
L'être et le devenir \\[0pt]
L'être et le néant \\[0pt]
L'être et les êtres \\[0pt]
L'être et l'esprit \\[0pt]
L'être et l'essence \\[0pt]
L'être et l'étant \\[0pt]
L'être et le temps \\[0pt]
L'être intelligent \\[0pt]
L'être se confond-il avec l'être perçu ? \\[0pt]
L'Être suprême \\[0pt]
Le tribunal de l'histoire \\[0pt]
Le trompe-l'œil \\[0pt]
Le tyran \\[0pt]
Le vainqueur a-t-il tous les droits ? \\[0pt]
L'évaluation de l'action politique \\[0pt]
L'évaluation des œuvres d'art \\[0pt]
Le vécu \\[0pt]
L'événement \\[0pt]
L'événement et le fait divers \\[0pt]
L'événement manque-t-il d'être ? \\[0pt]
Le verbalisme \\[0pt]
Le verbe \\[0pt]
Le vertige \\[0pt]
Le vide \\[0pt]
L'évidence \\[0pt]
L'évidence est-elle un critère de vérité ? \\[0pt]
Le village global \\[0pt]
Le virtuel \\[0pt]
Le virtuel existe-t-il ? \\[0pt]
Le visage \\[0pt]
Le visible et l'invisible \\[0pt]
Le vivant \\[0pt]
Le vivant a-t-il une temporalité propre ? \\[0pt]
Le vivant comme problème pour la philosophie des sciences \\[0pt]
Le vivant échappe-t-il au déterminisme ? \\[0pt]
Le vivant est-il l'affaire du politique ? \\[0pt]
Le vivant et la technique \\[0pt]
Le vivant et ses lois \\[0pt]
Le vol \\[0pt]
Le volontaire et l'involontaire \\[0pt]
L'évolution \\[0pt]
L'évolution des langues \\[0pt]
Le voyage \\[0pt]
Le vrai a-t-il une histoire ? \\[0pt]
Le vrai est-il à lui-même sa propre marque ? \\[0pt]
Le vrai et le vraisemblable \\[0pt]
Le vrai et l'imaginaire \\[0pt]
Le vrai peut-il rester invérifiable ? \\[0pt]
Le vraisemblable \\[0pt]
Le vraisemblable présente-t-il un intérêt pour la pensée ? \\[0pt]
Le vrai se réduit-il à l'utile ? \\[0pt]
Le vulgaire \\[0pt]
L'exactitude \\[0pt]
L'excellence \\[0pt]
L'exception \\[0pt]
L'exception peut-elle confirmer la règle ? \\[0pt]
L'excès \\[0pt]
L'excès et le défaut \\[0pt]
L'exclusion \\[0pt]
L'excuse \\[0pt]
L'exécution d'une œuvre d'art est-elle toujours une œuvre d'art ? \\[0pt]
L'exemplaire \\[0pt]
L'exemplarité \\[0pt]
L'exemple \\[0pt]
L'exemple en morale \\[0pt]
L'exercice de la raison \\[0pt]
L'exercice de la vertu \\[0pt]
L'exercice du pouvoir \\[0pt]
L'exercice du pouvoir est-il nécessairement solitaire ? \\[0pt]
L'exercice solitaire du pouvoir \\[0pt]
L'exigence de vérité a-t-elle un sens moral ? \\[0pt]
L'exigence morale \\[0pt]
L'exil \\[0pt]
L'existence \\[0pt]
L'existence a-t-elle un sens ? \\[0pt]
L'existence de l'État dépend-elle d'un contrat ? \\[0pt]
L'existence du mal \\[0pt]
L'existence se démontre-t-elle ? \\[0pt]
L'existence se prouve-t-elle ? \\[0pt]
L'expérience \\[0pt]
L'expérience artistique \\[0pt]
L'expérience cruciale \\[0pt]
L'expérience dans les sciences humaines \\[0pt]
L'expérience de la beauté est-elle naturelle ? \\[0pt]
L'expérience de l'injustice \\[0pt]
L'expérience de l'irrationnel \\[0pt]
L'expérience directe est-elle une connaissance ? \\[0pt]
L'expérience du beau peut-elle être une expérience métaphysique ? \\[0pt]
L'expérience du mal \\[0pt]
L'expérience du sociologue \\[0pt]
L'expérience en psychologie \\[0pt]
L'expérience en sciences humaines \\[0pt]
L'expérience enseigne-elle quelque chose ? \\[0pt]
L'expérience est-elle la seule source de nos connaissances ? \\[0pt]
L'expérience esthétique \\[0pt]
L'expérience esthétique peut-elle se communiquer ? \\[0pt]
L'expérience et l'expérimentation \\[0pt]
L'expérience instruit-elle ? \\[0pt]
L'expérience métaphysique \\[0pt]
L'expérience morale \\[0pt]
L'expérience sensible est-elle la seule source légitime de connaissance ? \\[0pt]
L'expérimentation \\[0pt]
L'expérimentation en art \\[0pt]
L'expérimentation en psychologie \\[0pt]
L'expérimentation en sciences sociales \\[0pt]
L'expérimentation est-elle facteur de vérité ? \\[0pt]
L'expérimentation médicale \\[0pt]
L'expérimentation sur le vivant \\[0pt]
L'expert et l'amateur \\[0pt]
L'expertise \\[0pt]
L'expertise politique \\[0pt]
L'explication causale dans les sciences sociales \\[0pt]
L'explication de l'œuvre d'art \\[0pt]
L'explication scientifique \\[0pt]
L'exploitation de l'homme par l'homme \\[0pt]
L'exposition \\[0pt]
L'exposition de l'œuvre d'art \\[0pt]
L'expression \\[0pt]
L'expression artistique \\[0pt]
L'expression de l'inconscient \\[0pt]
L'expression des citoyens dans le vote suffit-elle à faire la démocratie ? \\[0pt]
L'expression des citoyens dans le vote suffit-elle à faire vivre la démocratie ? \\[0pt]
L'expression des sentiments \\[0pt]
L'expression du mouvement \\[0pt]
L'expression peut-elle être libre ? \\[0pt]
L'expressivité musicale \\[0pt]
L'extériorité \\[0pt]
L'extrémisme \\[0pt]
L'habileté \\[0pt]
L'habitation \\[0pt]
L'habitude \\[0pt]
L'habitude est-elle une force ? \\[0pt]
L'habitude est-elle une seconde nature ? \\[0pt]
L'habitude morale \\[0pt]
L'habitus \\[0pt]
L'harmonie \\[0pt]
L'hégémonie politique \\[0pt]
L'hérésie \\[0pt]
L'héritage \\[0pt]
L'héroïsme \\[0pt]
L'hésitation \\[0pt]
L'hétérogénéité sociale \\[0pt]
L'hétéronomie \\[0pt]
L'hétéronomie de l'art \\[0pt]
L'histoire a-t-elle un sens ? \\[0pt]
L'histoire de l'art \\[0pt]
L'histoire de l'art est-elle celle des styles ? \\[0pt]
L'histoire de l'art est-elle finie ? \\[0pt]
L'histoire des arts est-elle liée à l'histoire des techniques ? \\[0pt]
L'histoire des civilisations \\[0pt]
L'histoire des sciences \\[0pt]
L'histoire des sciences est-elle une histoire ? \\[0pt]
L'histoire : enquête ou science ? \\[0pt]
L'histoire est-elle cyclique ? \\[0pt]
L'histoire est-elle déterministe ? \\[0pt]
L'histoire est-elle la science de l'événementiel ? \\[0pt]
L'histoire est-elle sans fin ? \\[0pt]
L'histoire est-elle une science de l'individuel ? \\[0pt]
L'histoire est-elle un genre littéraire ? \\[0pt]
L'histoire est-elle un récit ? \\[0pt]
L'histoire est-elle un roman vrai ? \\[0pt]
L'histoire est-elle utile à la politique ? \\[0pt]
L'histoire et la géographie \\[0pt]
« L'histoire jugera » \\[0pt]
L'histoire n'est-elle qu'une suite d'événements ? \\[0pt]
L'histoire n'est-elle qu'un récit ? \\[0pt]
L'histoire peut-elle être objective ? \\[0pt]
L'histoire peut-elle s'écrire au présent ? \\[0pt]
L'histoire peut-elle se répéter ? \\[0pt]
L'histoire : science ou récit ? \\[0pt]
L'histoire se réduit-elle à l'étude du passé ? \\[0pt]
L'histoire universelle est-elle l'histoire des guerres ? \\[0pt]
L'historicisme \\[0pt]
L'historicité des sciences \\[0pt]
L'historien est-il un homme qui se souvient du passé ? \\[0pt]
L'homicide \\[0pt]
L'hominisation \\[0pt]
L'homme a-t-il une nature ? \\[0pt]
L'homme cultivé \\[0pt]
L'homme de la rue \\[0pt]
L'homme des droits de l'homme n'est-il qu'une fiction ? \\[0pt]
L'homme des foules \\[0pt]
L'homme des sciences de l'homme \\[0pt]
L'homme des sciences humaines \\[0pt]
L'homme d'État \\[0pt]
L'homme est-il la mesure de toutes choses ? \\[0pt]
L'homme est-il meilleur seul ou en société ? \\[0pt]
L'homme est-il objet de science ? \\[0pt]
L'homme est-il prisonnier du temps ? \\[0pt]
L'homme est-il un animal dénaturé ? \\[0pt]
L'homme est-il un animal métaphysique ? \\[0pt]
L'homme est-il un animal politique ? \\[0pt]
L'homme est-il un animal symbolique ? \\[0pt]
L'homme est-il un être inachevé ? \\[0pt]
« L'homme est la mesure de toute chose » \\[0pt]
L'homme et la bête \\[0pt]
L'homme et la machine \\[0pt]
L'homme et la nature sont-ils commensurables ? \\[0pt]
L'homme et le citoyen \\[0pt]
L'homme et le sacré \\[0pt]
L'homme fait-il partie de la nature ? \\[0pt]
L'homme injuste peut-il être heureux ? \\[0pt]
L'homme, le citoyen, le soldat \\[0pt]
L'homme libre est-il un homme seul ? \\[0pt]
L'homme moyen \\[0pt]
L'homme peut-il changer ? \\[0pt]
L'homme peut-il être objet d'expérience ? \\[0pt]
L'homo œconomicus est-il rationnel ? \\[0pt]
L'honnêteté \\[0pt]
L'honneur \\[0pt]
L'horizon \\[0pt]
L'horreur \\[0pt]
L'horreur du vide \\[0pt]
L'horrible \\[0pt]
L'hospitalité \\[0pt]
L'hospitalité a-t-elle un sens politique ? \\[0pt]
L'hospitalité est-elle un devoir ? \\[0pt]
L'hospitalité est-elle une vertu politique ? \\[0pt]
L'humain est-il un animal irrationnel ? \\[0pt]
L'humanité : un concept normatif ? \\[0pt]
L'humiliation \\[0pt]
L'humilité \\[0pt]
L'humilité est-elle une vertu ? \\[0pt]
L'humour \\[0pt]
L'humour et l'ironie \\[0pt]
L'hybridation des arts \\[0pt]
L'hypocrisie \\[0pt]
L'hypothèse \\[0pt]
L'hypothèse de l'inconscient \\[0pt]
Libéralisme et démocratie \\[0pt]
Liberté, égalité, fraternité \\[0pt]
Liberté, égalité, solidarité \\[0pt]
Liberté et égalité \\[0pt]
Liberté et libération \\[0pt]
Liberté et nécessité \\[0pt]
Liberté et négativité \\[0pt]
Liberté et propriété \\[0pt]
Liberté et vérité \\[0pt]
Liberté humaine et liberté divine \\[0pt]
Liberté réelle, liberté formelle \\[0pt]
Libertés publiques et culture politique \\[0pt]
Libre-arbitre, impulsion, contrainte \\[0pt]
L'idéal \\[0pt]
L'idéal dans l'art \\[0pt]
L'idéal de l'art \\[0pt]
L'idéal de vérité \\[0pt]
L'idéal et le réel \\[0pt]
L'idéalisme \\[0pt]
L'idéaliste \\[0pt]
L'idéalité \\[0pt]
L'idéal moral est-il vain ? \\[0pt]
L'idéal-type \\[0pt]
L'idée d'anthropologie \\[0pt]
L'idée de beaux arts \\[0pt]
L'idée de communauté \\[0pt]
L'idée de connaissance approchée \\[0pt]
L'idée de conscience collective \\[0pt]
L'idée de continuité \\[0pt]
L'idée de contrat \\[0pt]
L'idée de contrat social \\[0pt]
L'idée de création \\[0pt]
L'idée de crise \\[0pt]
L'idée de critique \\[0pt]
L'idée de destin \\[0pt]
L'idée de destin a-t-elle un sens ? \\[0pt]
L'idée de Dieu \\[0pt]
L'idée de domination \\[0pt]
L'idée de forme sociale \\[0pt]
L'idée d'égalité \\[0pt]
L'idée de guerre juste \\[0pt]
L'idée de langue parfaite \\[0pt]
L'idée de langue universelle \\[0pt]
L'idée de logique \\[0pt]
L'idée de logique transcendantale \\[0pt]
L'idée de logique universelle \\[0pt]
L'idée de loi logique \\[0pt]
L'idée de loi naturelle \\[0pt]
L'idée de mathesis universalis \\[0pt]
L'idée de morale appliquée \\[0pt]
L'idée de nation \\[0pt]
L'idée de nature humaine \\[0pt]
L'idée d'encyclopédie \\[0pt]
L'idée de norme \\[0pt]
L'idée de perfection \\[0pt]
L'idée de philosophie première \\[0pt]
L'idée de préhistoire \\[0pt]
L'idée de progrès \\[0pt]
L'idée de réduction \\[0pt]
L'idée de république \\[0pt]
L'idée de République \\[0pt]
L'idée de rétribution est-elle nécessaire à la morale ? \\[0pt]
L'idée de révolution \\[0pt]
L'idée de science expérimentale \\[0pt]
L'idée de « sciences exactes » \\[0pt]
L'idée de société primitive \\[0pt]
L'idée de substance \\[0pt]
L'idée d'éternité \\[0pt]
L'idée d'exactitude \\[0pt]
L'idée d'humanité \\[0pt]
L'idée d'un commencement absolu \\[0pt]
L'idée d'une langue universelle \\[0pt]
L'idée d'une science bien faite \\[0pt]
L'idée d'une science générale des signes \\[0pt]
L'idée esthétique \\[0pt]
L'idée et l'idéal \\[0pt]
L'identité \\[0pt]
L'identité collective \\[0pt]
L'identité et la différence \\[0pt]
L'identité nationale \\[0pt]
L'identité personnelle \\[0pt]
L'idéologie \\[0pt]
L'idolâtrie \\[0pt]
L'idole \\[0pt]
L'ignoble \\[0pt]
L'ignorance \\[0pt]
L'ignorance des savants \\[0pt]
L'ignorance est-elle la raison du mal ? \\[0pt]
L'ignorance est-elle une excuse ? \\[0pt]
L'ignorance nous excuse-t-elle ? \\[0pt]
L'illimité \\[0pt]
L'illusion \\[0pt]
L'illusoire \\[0pt]
L'illustration \\[0pt]
L'image \\[0pt]
L'imaginaire \\[0pt]
L'imaginaire et le réel \\[0pt]
L'imaginaire politique \\[0pt]
L'imaginaire social \\[0pt]
L'imagination \\[0pt]
L'imagination a-t-elle des limites ? \\[0pt]
L'imagination dans l'art \\[0pt]
L'imagination dans les sciences \\[0pt]
L'imagination est-elle dangereuse ? \\[0pt]
L'imagination esthétique \\[0pt]
L'imagination nous éloigne-t-elle du réel ? \\[0pt]
L'imagination nous instruit-elle ? \\[0pt]
L'imagination politique \\[0pt]
L'imbécile heureux \\[0pt]
L'imitation \\[0pt]
L'imitation a-t-elle une fonction morale ? \\[0pt]
L'immanence \\[0pt]
L'immatériel \\[0pt]
L'immédiat \\[0pt]
L'immémorial \\[0pt]
L'immensité \\[0pt]
L'immoraliste \\[0pt]
L'immoralité \\[0pt]
L'immortalité \\[0pt]
L'immortalité de l'âme \\[0pt]
L'immortalité des œuvres d'art \\[0pt]
L'immuable \\[0pt]
L'immutabilité \\[0pt]
L'impardonnable \\[0pt]
L'imparfait \\[0pt]
L'impartialité \\[0pt]
L'impensable \\[0pt]
L'impératif \\[0pt]
L'imperceptible \\[0pt]
L'impersonnel \\[0pt]
L'implicite \\[0pt]
L'implicite dans la communication \\[0pt]
L'importance des détails \\[0pt]
L'impossible \\[0pt]
L'impossible est-il concevable ? \\[0pt]
L'imposteur \\[0pt]
L'imprescriptible \\[0pt]
L'impression \\[0pt]
L'imprévisible \\[0pt]
L'imprévu \\[0pt]
L'improbable \\[0pt]
L'improvisation \\[0pt]
L'improvisation dans l'art \\[0pt]
L'imprudence \\[0pt]
L'impuissance \\[0pt]
L'impuissance de la raison \\[0pt]
L'impuissance de l'art \\[0pt]
L'impuissance de l'État \\[0pt]
L'impunité \\[0pt]
L'inacceptable \\[0pt]
L'inachevé \\[0pt]
L'inachèvement \\[0pt]
L'inaction \\[0pt]
L'inapparent \\[0pt]
L'inattendu \\[0pt]
L'incarnation \\[0pt]
L'incarnation du pouvoir \\[0pt]
L'incertitude \\[0pt]
L'incertitude est-elle dans les choses ou dans les idées ? \\[0pt]
L'incommensurabilité \\[0pt]
L'incommensurable \\[0pt]
L'incompréhensible \\[0pt]
L'inconcevable \\[0pt]
L'inconnu \\[0pt]
L'inconscience \\[0pt]
L'inconscient \\[0pt]
L'inconscient a-t-il une histoire ? \\[0pt]
L'inconscient collectif \\[0pt]
L'inconscient de l'art \\[0pt]
L'inconscient psychanalytique réfute-t-il la notion philosophique de conscience ? \\[0pt]
L'inconséquence \\[0pt]
L'inconstance \\[0pt]
L'incorporel \\[0pt]
L'incrédulité \\[0pt]
L'inculture \\[0pt]
L'indécidable \\[0pt]
L'indécision \\[0pt]
L'indéfini \\[0pt]
L'indémontrable \\[0pt]
L'indépassable \\[0pt]
L'indépendance \\[0pt]
L'indétermination \\[0pt]
L'indéterminé \\[0pt]
L'indicible \\[0pt]
L'indifférence \\[0pt]
L'indifférence à la politique \\[0pt]
L'indigène et l'étranger \\[0pt]
L'indignation \\[0pt]
L'indiscernable \\[0pt]
L'indiscutable \\[0pt]
L'indistinct \\[0pt]
L'individu \\[0pt]
L'individualisme \\[0pt]
L'individualisme a-t-il sa place en politique ? \\[0pt]
L'individualisme méthodologique \\[0pt]
L'individualisme rend-il la politique impossible ? \\[0pt]
L'individu a-t-il des droits ? \\[0pt]
L'individu échappe-t-il aux sciences humaines ? \\[0pt]
L'individuel \\[0pt]
L'individuel et le collectif \\[0pt]
L'individu est-il définissable ? \\[0pt]
L'individu et la multitude \\[0pt]
L'individu et la société \\[0pt]
L'individu et le groupe \\[0pt]
L'individu et les masses \\[0pt]
L'individu particulier peut-il être l'objet d'une connaissance sociologique ? \\[0pt]
L'indivisible \\[0pt]
L'indubitable \\[0pt]
L'induction \\[0pt]
L'induction et la déduction \\[0pt]
L'indulgence \\[0pt]
L'industrie culturelle \\[0pt]
L'industrie du beau \\[0pt]
L'ineffable \\[0pt]
L'inégalité a-t-elle des vertus ? \\[0pt]
L'inégalité des chances \\[0pt]
L'inégalité entre les hommes \\[0pt]
L'inégalité naturelle \\[0pt]
L'inégalité sociale \\[0pt]
L'inertie \\[0pt]
L'inesthétique \\[0pt]
L'inexactitude et le savoir scientifique \\[0pt]
L'inexact peut-il avoir une valeur ? \\[0pt]
L'infâme \\[0pt]
L'infamie \\[0pt]
L'inférence \\[0pt]
L'infériorité \\[0pt]
L'infini \\[0pt]
L'infini est-il la négation du fini ? \\[0pt]
L'infini et l'indéfini \\[0pt]
L'infinité de l'espace \\[0pt]
L'infinité du monde \\[0pt]
L'influence \\[0pt]
L'information \\[0pt]
L'informe \\[0pt]
L'informe et le difforme \\[0pt]
L'ingénieur et l'artiste \\[0pt]
L'ingéniosité \\[0pt]
L'ingérence \\[0pt]
L'ingratitude \\[0pt]
Linguistique et sens \\[0pt]
L'inhibition \\[0pt]
L'inhumain \\[0pt]
L'inimaginable \\[0pt]
L'inimitié \\[0pt]
L'inintelligible \\[0pt]
L'iniquité \\[0pt]
L'initiation \\[0pt]
L'injonction \\[0pt]
L'injure \\[0pt]
L'injustice \\[0pt]
L'injustice de la loi \\[0pt]
L'injustice est-elle préférable au désordre ? \\[0pt]
L'injustifiable \\[0pt]
L'innocence \\[0pt]
L'innommable \\[0pt]
L'inobservable \\[0pt]
L'inquiétant \\[0pt]
L'inquiétude \\[0pt]
L'insensé \\[0pt]
L'insensibilité \\[0pt]
L'insignifiant \\[0pt]
« L'insociable sociabilité » \\[0pt]
L'insociable sociabilité \\[0pt]
L'insouciance \\[0pt]
L'insoumission \\[0pt]
L'insoutenable \\[0pt]
L'inspiration \\[0pt]
L'instant \\[0pt]
L'instant a-t-il une réalité ? \\[0pt]
L'instinct \\[0pt]
L'institution \\[0pt]
L'institution de la société \\[0pt]
L'institution du mariage \\[0pt]
L'institutionnalisation des conduites \\[0pt]
L'institution scientifique \\[0pt]
L'institution scolaire \\[0pt]
L'instruction est-elle facteur de moralité ? \\[0pt]
L'instrument, l'outil, le jouet \\[0pt]
L'instrument mathématique en sciences humaines \\[0pt]
L'instrument scientifique \\[0pt]
L'insulte \\[0pt]
L'insurrection \\[0pt]
L'intangible \\[0pt]
L'intégration sociale \\[0pt]
L'intégrité \\[0pt]
L'intellectuel \\[0pt]
L'intelligence \\[0pt]
L'intelligence de la machine \\[0pt]
L'intelligence de la main \\[0pt]
L'intelligence de la matière \\[0pt]
L'intelligence des bêtes \\[0pt]
L'intelligence des foules \\[0pt]
L'intelligence du lointain \\[0pt]
L'intelligence du sensible \\[0pt]
L'intelligence du vivant \\[0pt]
L'intelligence politique \\[0pt]
L'intelligible \\[0pt]
L'intempérance \\[0pt]
L'intemporel \\[0pt]
L'intention \\[0pt]
L'intention morale \\[0pt]
L'intention morale suffit-elle à constituer la valeur morale de l'action ? \\[0pt]
L'intentionnalité \\[0pt]
L'interdépendance \\[0pt]
L'interdépendance des faits humains \\[0pt]
L'interdiction \\[0pt]
L'interdit \\[0pt]
L'interdit a-t-il une origine sociale ? \\[0pt]
L'interdit et le sacré \\[0pt]
L'intérêt \\[0pt]
L'intérêt bien compris \\[0pt]
L'intérêt commun \\[0pt]
L'intérêt des machines \\[0pt]
L'intérêt général \\[0pt]
L'intérêt général est-il le bien commun ? \\[0pt]
L'intérêt général est-il un mythe ? \\[0pt]
L'intérêt général n'est-il qu'un mythe ? \\[0pt]
L'intérêt peut-il avoir une valeur morale ? \\[0pt]
L'intérêt peut-il être général ? \\[0pt]
L'intérêt peut-il être une valeur morale ? \\[0pt]
L'intérêt public est-il une illusion ? \\[0pt]
L'intérieur et l'extérieur \\[0pt]
L'intériorisation des normes \\[0pt]
L'intériorité \\[0pt]
L'intériorité de l'œuvre \\[0pt]
L'interprétation \\[0pt]
L'interprétation de la loi \\[0pt]
L'interprétation des œuvres \\[0pt]
L'interprétation est-elle un art ? \\[0pt]
L'interprète est-il un créateur ? \\[0pt]
L'interrogation humaine \\[0pt]
L'intersectionnalité \\[0pt]
L'intime \\[0pt]
L'intime conviction \\[0pt]
L'intime est-il politique ? \\[0pt]
L'intimité \\[0pt]
L'intolérable \\[0pt]
L'intolérance \\[0pt]
L'intraduisible \\[0pt]
L'intransigeance \\[0pt]
L'intransmissible \\[0pt]
L'introspection \\[0pt]
L'intuition \\[0pt]
L'intuition a-t-elle une place en logique ? \\[0pt]
L'intuition en mathématiques \\[0pt]
L'intuition morale \\[0pt]
L'inutile \\[0pt]
L'invention \\[0pt]
L'invention de soi \\[0pt]
L'invention des valeurs \\[0pt]
L'invention en politique \\[0pt]
L'invérifiable \\[0pt]
L'invisibilité \\[0pt]
L'invisible \\[0pt]
L'involontaire \\[0pt]
L'invraisemblable \\[0pt]
Lire \\[0pt]
Lire et écrire \\[0pt]
L'ironie \\[0pt]
L'irrationalité \\[0pt]
L'irrationnel \\[0pt]
L'irrationnel et le politique \\[0pt]
L'irréel \\[0pt]
L'irréfutable \\[0pt]
L'irrégularité \\[0pt]
L'irréparable \\[0pt]
L'irreprésentable \\[0pt]
L'irrésolution \\[0pt]
L'irresponsabilité \\[0pt]
L'irréversibilité \\[0pt]
L'irréversible \\[0pt]
L'irrévocable \\[0pt]
Littérature et philosophie \\[0pt]
Littérature et réalité \\[0pt]
L'ivresse \\[0pt]
L'obéissance \\[0pt]
L'obéissance à l'autorité \\[0pt]
L'objectivité \\[0pt]
L'objectivité de l'art \\[0pt]
L'objectivité de l'œuvre d'art \\[0pt]
L'objectivité des sciences humaines \\[0pt]
L'objectivité des sciences sociales \\[0pt]
L'objectivité des valeurs \\[0pt]
L'objectivité en histoire \\[0pt]
L'objectivité existe-t-elle ? \\[0pt]
L'objectivité historique \\[0pt]
L'objet \\[0pt]
L'objet d'amour \\[0pt]
L'objet de culte \\[0pt]
L'objet de la littérature \\[0pt]
L'objet de l'amour \\[0pt]
L'objet de la politique \\[0pt]
L'objet de la réflexion \\[0pt]
L'objet de l'art \\[0pt]
L'objet du jugement esthétique \\[0pt]
L'obligation \\[0pt]
L'obligation d'échanger \\[0pt]
L'obligation morale \\[0pt]
L'obligation morale peut-elle se réduire à une obligation sociale ? \\[0pt]
L'obscène \\[0pt]
L'obscénité \\[0pt]
L'obscur \\[0pt]
L'obscurité \\[0pt]
L'observable et l'inobservable \\[0pt]
L'observation \\[0pt]
L'observation de l'homme \\[0pt]
L'observation participante \\[0pt]
L'obsession \\[0pt]
L'obstacle \\[0pt]
L'obstacle épistémologique \\[0pt]
L'occasion \\[0pt]
L'œil du photographe \\[0pt]
L'œil et l'oreille \\[0pt]
L'œuvre anonyme \\[0pt]
L'œuvre d'art échappe-t-elle au monde ? \\[0pt]
L'œuvre d'art est-elle anhistorique ? \\[0pt]
L'œuvre d'art est-elle l'expression d'une idée ? \\[0pt]
L'œuvre d'art est-elle toujours destinée à un public ? \\[0pt]
L'œuvre d'art est-elle une belle apparence ? \\[0pt]
L'œuvre d'art est-elle une marchandise ? \\[0pt]
L'œuvre d'art et sa reproduction \\[0pt]
L'œuvre d'art et son auteur \\[0pt]
L'œuvre d'art nous apprend-elle quelque chose ? \\[0pt]
L'œuvre d'art représente-t-elle quelque chose ? \\[0pt]
L'œuvre d'art totale \\[0pt]
L'œuvre d'art traduit-elle une vision du monde ? \\[0pt]
L'œuvre de fiction \\[0pt]
L'œuvre de l'historien \\[0pt]
L'œuvre et le produit \\[0pt]
L'œuvre inachevée \\[0pt]
L'œuvre : l'universel et le particulier \\[0pt]
L'œuvre totale \\[0pt]
L'offense \\[0pt]
Logique et dialectique \\[0pt]
Logique et écriture \\[0pt]
Logique et existence \\[0pt]
Logique et logiques \\[0pt]
Logique et mathématique \\[0pt]
Logique et mathématiques \\[0pt]
Logique et métaphysique \\[0pt]
Logique et méthode \\[0pt]
Logique et ontologie \\[0pt]
Logique et psychologie \\[0pt]
Logique et vérité \\[0pt]
Logique générale et logique transcendantale \\[0pt]
Loi et commandement \\[0pt]
Loi morale et loi politique \\[0pt]
Loi naturelle et loi morale \\[0pt]
Loi naturelle et loi politique \\[0pt]
Lois et normes \\[0pt]
Lois et règles en logique \\[0pt]
Lois et valeurs \\[0pt]
L'oisiveté \\[0pt]
L'oligarchie \\[0pt]
L'ombre \\[0pt]
L'ombre et la lumière \\[0pt]
L'omniscience \\[0pt]
L'ontologie peut-elle être relative ? \\[0pt]
L'opéra est-il un art complet ? \\[0pt]
L'opinion \\[0pt]
L'opinion droite \\[0pt]
L'opinion du citoyen \\[0pt]
L'opinion publique \\[0pt]
L'opinion vraie \\[0pt]
L'opportunisme \\[0pt]
L'opposant \\[0pt]
L'opposition \\[0pt]
L'optimisme \\[0pt]
L'ordinaire est-il ennuyeux ? \\[0pt]
L'ordinateur \\[0pt]
L'ordre \\[0pt]
L'ordre des choses \\[0pt]
L'ordre des raisons \\[0pt]
L'ordre du monde \\[0pt]
L'ordre du temps \\[0pt]
L'ordre est-il dans les choses ? \\[0pt]
L'ordre établi \\[0pt]
L'ordre et la beauté \\[0pt]
L'ordre et la mesure \\[0pt]
L'ordre international \\[0pt]
L'ordre, le nombre, la mesure \\[0pt]
L'ordre moral \\[0pt]
L'ordre naturel \\[0pt]
L'ordre politique \\[0pt]
L'ordre politique peut-il exclure la violence ? \\[0pt]
L'ordre public \\[0pt]
L'ordre social \\[0pt]
L'ordre symbolique \\[0pt]
L'organique et le mécanique \\[0pt]
L'organisation \\[0pt]
L'orgueil \\[0pt]
L'orientation \\[0pt]
L'original et la copie \\[0pt]
L'originalité \\[0pt]
L'originalité en art \\[0pt]
L'origine \\[0pt]
L'origine de la culpabilité \\[0pt]
L'origine de la famille \\[0pt]
L'origine de la négation \\[0pt]
L'origine de l'art \\[0pt]
L'origine des croyances \\[0pt]
L'origine des langues \\[0pt]
L'origine des langues est-elle un faux problème ? \\[0pt]
L'origine des vertus \\[0pt]
L'origine du mal \\[0pt]
L'origine du symbolisme \\[0pt]
L'origine et le fondement \\[0pt]
L'ornement \\[0pt]
L'orthodoxie \\[0pt]
L'oubli \\[0pt]
L'oubli des fautes \\[0pt]
L'oubli est-il un échec de la mémoire ? \\[0pt]
L'oubli n'est-il qu'une perte de mémoire ? \\[0pt]
L'outil \\[0pt]
Lumière et couleur \\[0pt]
L'un \\[0pt]
L'Un \\[0pt]
L'unanimité \\[0pt]
L'unanimité est-elle un critère de légitimité ? \\[0pt]
L'un et le multiple \\[0pt]
L'un et le tout \\[0pt]
L'un et l'être \\[0pt]
L'uniformité \\[0pt]
L'unité \\[0pt]
L'unité dans le beau \\[0pt]
L'unité de l'art \\[0pt]
L'unité de la science \\[0pt]
L'unité de l'œuvre d'art \\[0pt]
L'unité des contraires \\[0pt]
L'unité des langues \\[0pt]
L'unité des sciences \\[0pt]
L'unité des sciences humaines \\[0pt]
L'unité des sciences humaines ? \\[0pt]
L'unité des vices \\[0pt]
L'unité du corps politique \\[0pt]
L'unité du moi \\[0pt]
L'univers \\[0pt]
L'universel \\[0pt]
L'universel en histoire \\[0pt]
L'universel est-il une illusion ? \\[0pt]
L'universel et le particulier \\[0pt]
L'universel et le singulier \\[0pt]
L'universel implique-t-il l'indifférence à la singularité ? \\[0pt]
L'univers infini \\[0pt]
L'univocité de l'étant \\[0pt]
L'un, le vrai, le bien \\[0pt]
L'urbanité \\[0pt]
L'urgence \\[0pt]
L'usage \\[0pt]
L'usage de la langue \\[0pt]
L'usage de l'analogie \\[0pt]
L'usage de l'analogie dans les sciences humaines \\[0pt]
L'usage des fictions \\[0pt]
L'usage des généalogies \\[0pt]
L'usage des modèles dans les sciences humaines \\[0pt]
L'usage des mots \\[0pt]
L'usage des passions \\[0pt]
L'usage des principes \\[0pt]
L'usage du concept de race \\[0pt]
L'usage du monde \\[0pt]
L'usure du pouvoir \\[0pt]
L'utile et l'agréable \\[0pt]
L'utilité \\[0pt]
L'utilité de la peine \\[0pt]
L'utilité de la poésie \\[0pt]
L'utilité de l'art \\[0pt]
L'utilité des préjugés \\[0pt]
L'utilité des sciences humaines \\[0pt]
L'utilité est-elle étrangère à la morale ? \\[0pt]
L'utilité publique \\[0pt]
L'utopie \\[0pt]
L'utopie a-t-elle une signification politique ? \\[0pt]
L'utopie a-t-elle un lien avec la pratique ? \\[0pt]
L'utopie en politique \\[0pt]
Machine et organisme \\[0pt]
Machines et liberté \\[0pt]
Machines et mémoire \\[0pt]
Magie et religion \\[0pt]
Maître et disciple \\[0pt]
Maître et serviteur \\[0pt]
Maîtriser l'absence \\[0pt]
Maîtriser la nature \\[0pt]
Mal faire \\[0pt]
« Malheur aux vaincus » \\[0pt]
Manger \\[0pt]
Manquer de jugement \\[0pt]
Ma parole m'engage-t-elle ? \\[0pt]
Masculin et féminin \\[0pt]
Masculin, féminin \\[0pt]
Mass-media et création artistique \\[0pt]
Matérialisme et mécanisme \\[0pt]
Matériel / immatériel \\[0pt]
Mathématique et imagination \\[0pt]
Mathématiques et réalité \\[0pt]
Mathématiques et sciences humaines \\[0pt]
Mathématiques et universel \\[0pt]
Mathématiques pures et mathématiques appliquées \\[0pt]
Matière et corps \\[0pt]
Matière et forme \\[0pt]
Matière et matériaux \\[0pt]
Ma vraie nature \\[0pt]
Mécanisme et finalité \\[0pt]
Médecine et morale \\[0pt]
Méditer \\[0pt]
Mémoire et anticipation \\[0pt]
Mémoire et fiction \\[0pt]
Mémoire et identité \\[0pt]
Mémoire et imagination \\[0pt]
Mémoire et responsabilité \\[0pt]
Ménager les apparences \\[0pt]
Mensonge et politique \\[0pt]
Mensonge, vérité, véracité \\[0pt]
Mentir \\[0pt]
Mérite et démocratie \\[0pt]
Mesurer \\[0pt]
Mesurer le risque \\[0pt]
Métaphysique et histoire \\[0pt]
Métaphysique et mystique \\[0pt]
Métaphysique et ontologie \\[0pt]
Métaphysique et poésie \\[0pt]
Métaphysique et psychologie \\[0pt]
Métaphysique et religion \\[0pt]
Métaphysique et théologie \\[0pt]
Métaphysique spéciale, métaphysique générale \\[0pt]
Méthode et métaphysique \\[0pt]
Métier et vocation \\[0pt]
Mettre en ordre \\[0pt]
Microscope et télescope \\[0pt]
Milieux naturels et milieux sociaux \\[0pt]
Misère et pauvreté \\[0pt]
Modèle, type et paradigme \\[0pt]
Mœurs, coutumes, lois \\[0pt]
Moi d'abord \\[0pt]
Mon corps \\[0pt]
Mon corps est-il ma propriété ? \\[0pt]
Mon corps m'appartient-il ? \\[0pt]
Monde et nature \\[0pt]
Monde perçu et monde conçu \\[0pt]
Mon existence est-elle contingente ? \\[0pt]
Monologue et dialogue \\[0pt]
Montrer et démontrer \\[0pt]
Montrer et dire \\[0pt]
Morale et calcul \\[0pt]
Morale et convention \\[0pt]
Morale et dispositions \\[0pt]
Morale et éducation \\[0pt]
Morale et histoire \\[0pt]
Morale et intérêt \\[0pt]
Morale et liberté \\[0pt]
Morale et politique sont-elles indépendantes ? \\[0pt]
Morale et pratique \\[0pt]
Morale et prudence \\[0pt]
Morale et religion \\[0pt]
Morale et science des mœurs \\[0pt]
Morale et sentiments \\[0pt]
Morale et sexualité \\[0pt]
Morale et société \\[0pt]
Morale et technique \\[0pt]
Morale et violence \\[0pt]
Mourir \\[0pt]
Mourir dans la dignité \\[0pt]
Mourir pour des principes \\[0pt]
Mourir pour la patrie \\[0pt]
Mouvement et changement \\[0pt]
Murs et frontières \\[0pt]
Musique et architecture \\[0pt]
Musique et bruit \\[0pt]
Musique et émotion \\[0pt]
Musique et poésie \\[0pt]
Musique et rhétorique \\[0pt]
Mythe et histoire \\[0pt]
Mythe et philosophie \\[0pt]
Mythe et symbole \\[0pt]
Mythes et idéologies \\[0pt]
N'aimer que soi \\[0pt]
Naître \\[0pt]
Nation et richesse \\[0pt]
Naturaliser l'esprit \\[0pt]
Nature et fonction du sacrifice \\[0pt]
Nature et histoire \\[0pt]
Nature et institution \\[0pt]
Nature et institutions \\[0pt]
Nature et liberté \\[0pt]
Nature et mesure du temps \\[0pt]
Nature et monde \\[0pt]
Nature et morale \\[0pt]
Nature et nature humaine \\[0pt]
Nature, monde, univers \\[0pt]
Naviguer \\[0pt]
Nécessaire / contingent \\[0pt]
Nécessité et contingence \\[0pt]
Nécessité fait loi \\[0pt]
N'échange-t-on que ce qui a de la valeur ? \\[0pt]
N'échange-t-on que des biens ? \\[0pt]
N'échange-t-on que des symboles ? \\[0pt]
« Ne fais pas à autrui ce que tu ne voudrais pas qu'on te fasse » \\[0pt]
Négation et privation \\[0pt]
Ne lèse personne \\[0pt]
Ne pas choisir \\[0pt]
Ne pas comprendre \\[0pt]
Ne pas multiplier en vain les entités \\[0pt]
Ne pas raconter d'histoires \\[0pt]
Ne pas savoir ce que l'on fait \\[0pt]
Ne pas tuer \\[0pt]
Ne penser à rien \\[0pt]
Ne penser qu'à soi \\[0pt]
Ne prêche-t-on que les convertis ? \\[0pt]
Ne rien devoir à personne \\[0pt]
N'est-on juste que par crainte du châtiment ? \\[0pt]
Ne travaille-t- on que pour un salaire ? \\[0pt]
Névroses et psychoses \\[0pt]
N'existe-t-il que des individus ? \\[0pt]
N'exprime-t-on que ce dont on a conscience ? \\[0pt]
« Ni Dieu, ni maître ! » \\[0pt]
Ni Dieu ni maître \\[0pt]
Ni Dieu, ni maître \\[0pt]
Nier le monde \\[0pt]
Nier l'évidence \\[0pt]
Ni regrets, ni remords \\[0pt]
Nomade et sédentaire \\[0pt]
Nommer \\[0pt]
Non-sens et absurdité \\[0pt]
Normalité et normativité \\[0pt]
Norme et moyenne \\[0pt]
Normes morales et normes vitales \\[0pt]
Nos devoirs nous motivent-ils ? \\[0pt]
Nos goûts sont-ils justifiables ? \\[0pt]
Nos sens peuvent-ils s'éduquer ? \\[0pt]
Notre besoin de fictions \\[0pt]
Notre connaissance du réel se limite-t-elle au savoir scientifique ? \\[0pt]
Notre corps pense-t-il ? \\[0pt]
Notre ignorance nous excuse-t-elle ? \\[0pt]
Nul n'est censé ignorer la loi \\[0pt]
N'y a-t-il d'amitié qu'entre égaux ? \\[0pt]
N'y a-t-il de bonheur que dans l'instant ? \\[0pt]
N'y a-t-il de connaissance que par les causes ? \\[0pt]
N'y a-t-il de connaissance que scientifique ? \\[0pt]
N'y a-t-il de droit que de l'homme ? \\[0pt]
N'y a-t-il de propriété que privée ? \\[0pt]
N'y a-t-il de rationalité que scientifique ? \\[0pt]
N'y a-t-il de réel que le présent ? \\[0pt]
N'y a-t-il de science qu'autant qu'il s'y trouve de mathématique ? \\[0pt]
N'y a-t-il de science que du général ? \\[0pt]
N'y a-t-il de science que du nécessaire ? \\[0pt]
N'y a-t-il de science que provisoire ? \\[0pt]
N'y a-t-il de science que systématique ? \\[0pt]
N'y a-t-il de sens que par le langage ? \\[0pt]
N'y a-t-il d'être que sensible ? \\[0pt]
N'y a-t-il de vérité que scientifique ? \\[0pt]
N'y a-t-il d'origine que mythique ? \\[0pt]
N'y a-t-il que du mesurable ? \\[0pt]
N'y a-t-il qu'une substance ? \\[0pt]
N'y a-t-il qu'un seul monde ? \\[0pt]
N'y a-t-il rien au monde de certain ? \\[0pt]
Obéir \\[0pt]
Obéir, est-ce se soumettre ? \\[0pt]
Obéir et désobéir \\[0pt]
Obéissance et esclavage \\[0pt]
Objecter et répondre \\[0pt]
Objectivité et subjectivité \\[0pt]
Objet d'art, objet d'échange \\[0pt]
Observation et expérimentation \\[0pt]
Observer \\[0pt]
Observer, décrire et expliquer \\[0pt]
Observer et expérimenter \\[0pt]
« Œil pour œil, dent pour dent » \\[0pt]
Œuvre et événement \\[0pt]
Œuvrer \\[0pt]
Offre et demande \\[0pt]
Ontologie et théologie \\[0pt]
Opinion, croyance, jugement \\[0pt]
Opinion publique et conscience universelle \\[0pt]
Ordre et chaos \\[0pt]
Ordre et désordre \\[0pt]
Ordre et liberté \\[0pt]
Organe et fonction \\[0pt]
Organiser \\[0pt]
Organisme et milieu \\[0pt]
Origine des hommes et principe de l'humanité \\[0pt]
Origine et commencement \\[0pt]
Origine et fondement \\[0pt]
Où commence la servitude ? \\[0pt]
Où commence la vie privée ? \\[0pt]
Où commence le territoire de la psychologie ? \\[0pt]
Où est le danger ? \\[0pt]
Où est le passé ? \\[0pt]
Où est le pouvoir ? \\[0pt]
Où est-on quand on pense ? \\[0pt]
Où s'arrête la responsabilité ? \\[0pt]
Où s'arrête l'espace public ? \\[0pt]
Où sont les relations ? \\[0pt]
Où suis-je ? \\[0pt]
Où suis-je quand je pense ? \\[0pt]
Outil et machine \\[0pt]
Par-delà beauté et laideur \\[0pt]
Pardonner \\[0pt]
Pardonner et oublier \\[0pt]
Parfaire \\[0pt]
Parier \\[0pt]
Parler de Dieu, est-ce nécessairement tenir un discours religieux ? \\[0pt]
Parler de soi \\[0pt]
Parler de soi est-il intéressant ? \\[0pt]
Parler, écrire, penser \\[0pt]
Parler, est-ce communiquer ? \\[0pt]
Parler et agir \\[0pt]
Parler et penser \\[0pt]
Parler pour ne rien dire \\[0pt]
Parler pour quelqu'un \\[0pt]
Par où commencer ? \\[0pt]
Par quoi un individu diffère-t-il réellement d'un autre ? \\[0pt]
Par quoi un individu se distingue-t-il d'un autre ? \\[0pt]
Partager \\[0pt]
Partager les richesses \\[0pt]
Participation et imitation \\[0pt]
Passer du fait au droit \\[0pt]
Pâtir \\[0pt]
Peindre \\[0pt]
Peindre d'après nature \\[0pt]
Peindre, est-ce traduire ? \\[0pt]
Peindre la présence \\[0pt]
Peindre un paysage, est-ce représenter la nature ? \\[0pt]
Peinture et abstraction \\[0pt]
Peinture et histoire \\[0pt]
Peinture et photographie \\[0pt]
Peinture et réalité \\[0pt]
Peinture et vérité \\[0pt]
Pensée et calcul \\[0pt]
Pensée et réalité \\[0pt]
Penser, est-ce aller contre l'évidence ? \\[0pt]
Penser, est-ce calculer ? \\[0pt]
Penser, est-ce comparer ? \\[0pt]
Penser et calculer \\[0pt]
Penser et être \\[0pt]
Penser et parler \\[0pt]
Penser la technique \\[0pt]
Penser l'avenir \\[0pt]
Penser le devenir \\[0pt]
Penser le réel \\[0pt]
Penser le rien, est-ce ne rien penser ? \\[0pt]
Penser les sociétés comme des organismes \\[0pt]
Penser par soi-même \\[0pt]
Penser requiert-il d'avoir un corps ? \\[0pt]
Penser requiert-il un corps ? \\[0pt]
Penser sans corps \\[0pt]
Perception et aperception \\[0pt]
Perception et jugement \\[0pt]
Perception et mémoire \\[0pt]
Perception et mouvement \\[0pt]
Perception et observation \\[0pt]
Percer les apparences \\[0pt]
Percevoir \\[0pt]
Percevoir, est-ce connaître ? \\[0pt]
Percevoir, est-ce savoir ? \\[0pt]
Percevoir et imaginer \\[0pt]
Percevoir et sentir \\[0pt]
Percevons-nous le monde extérieur ? \\[0pt]
Perdre la face \\[0pt]
Perdre la mémoire \\[0pt]
Perdre la raison \\[0pt]
Perdre ses habitudes \\[0pt]
Perdre ses illusions \\[0pt]
Perdre son âme \\[0pt]
Permanent / successif \\[0pt]
Permettre \\[0pt]
Persévérer dans son être \\[0pt]
Persister \\[0pt]
Persuader \\[0pt]
Persuader et convaincre \\[0pt]
« Petites causes, grands effets » \\[0pt]
Peuple et culture \\[0pt]
Peuple et masse \\[0pt]
Peuple et société \\[0pt]
Peuples et masses \\[0pt]
Peut-il être moral de tuer ? \\[0pt]
Peut-il y avoir de bons tyrans ? \\[0pt]
Peut-il y avoir de la politique sans conflit ? \\[0pt]
Peut-il y avoir des degrés dans la vérité ? \\[0pt]
Peut-il y avoir des nations sans État ? \\[0pt]
Peut-il y avoir science sans intuition du vrai ? \\[0pt]
Peut-il y avoir un droit à désobéir ? \\[0pt]
Peut-il y avoir une anthropologie non anthropocentriste ? \\[0pt]
Peut-il y avoir une ethnologie non ethnocentriste ? \\[0pt]
Peut-il y avoir une méthode commune à toutes les sciences ? \\[0pt]
Peut-il y avoir une morale dans les échanges ? \\[0pt]
Peut-il y avoir une philosophie applicable ? \\[0pt]
Peut-il y avoir une philosophie politique sans Dieu ? \\[0pt]
Peut-il y avoir une politique de la langue ? \\[0pt]
Peut-il y avoir une pratique sans théorie ? \\[0pt]
Peut-il y avoir une science de l'éducation ? \\[0pt]
Peut-il y avoir une science de l'homme ? \\[0pt]
Peut-il y avoir une science des signes ? \\[0pt]
Peut-il y avoir une science politique ? \\[0pt]
Peut-il y avoir une servitude volontaire ? \\[0pt]
Peut-il y avoir une société des nations ? \\[0pt]
Peut-il y avoir une société sans État ? \\[0pt]
Peut-il y avoir une vérité en politique ? \\[0pt]
Peut-il y avoir un intérêt collectif ? \\[0pt]
Peut-on admettre ce qui n'est pas vérifié ? \\[0pt]
Peut-on admettre un droit à la révolte ? \\[0pt]
Peut-on affirmer qu'être, c'est être perçu ou percevoir ? \\[0pt]
Peut-on agir sans prendre de risques ? \\[0pt]
Peut-on aimer ce qu'on ne connaît pas ? \\[0pt]
Peut-on aimer les animaux ? \\[0pt]
Peut-on aimer l'humanité ? \\[0pt]
Peut-on aimer son prochain comme soi-même ? \\[0pt]
Peut-on à la fois préserver et dominer la nature ? \\[0pt]
Peut-on appréhender les choses telles qu'elles sont ? \\[0pt]
Peut-on apprendre à être juste ? \\[0pt]
Peut-on apprendre à mourir ? \\[0pt]
Peut-on apprendre à penser ? \\[0pt]
Peut-on apprendre à vivre ? \\[0pt]
Peut-on avoir raison contre les faits ? \\[0pt]
Peut-on avoir raison contre tous ? \\[0pt]
Peut-on avoir raison tout seul ? \\[0pt]
Peut-on avoir trop d'imagination ? \\[0pt]
Peut-on bien agir sans le savoir ? \\[0pt]
Peut-on changer de culture ? \\[0pt]
Peut-on changer de logique ? \\[0pt]
Peut-on changer le monde ? \\[0pt]
Peut-on changer le passé ? \\[0pt]
Peut-on comparer deux philosophies ? \\[0pt]
Peut-on comprendre ce qui est illogique ? \\[0pt]
Peut-on comprendre les actions humaines ? \\[0pt]
Peut-on concevoir la science achevée ? \\[0pt]
Peut-on concevoir une morale sans sanction ni obligation ? \\[0pt]
Peut-on concevoir une société qui n'aurait plus besoin du droit ? \\[0pt]
Peut-on concevoir un État mondial ? \\[0pt]
Peut-on concilier nécessité et liberté ? \\[0pt]
Peut-on conclure de l'être au devoir-être ? \\[0pt]
Peut-on connaître autrui ? \\[0pt]
Peut-on connaître les causes ? \\[0pt]
Peut-on connaître une vie humaine ? \\[0pt]
Peut-on considérer l'art comme un langage ? \\[0pt]
Peut-on considérer les hommes de science comme des autorités morales ? \\[0pt]
Peut-on critiquer la démocratie ? \\[0pt]
Peut-on décider de croire ? \\[0pt]
Peut-on définir ce qu'est un progrès technique ? \\[0pt]
Peut-on définir la santé ? \\[0pt]
Peut-on définir la vérité ? \\[0pt]
Peut-on définir la vérité par la correspondance ? \\[0pt]
Peut-on définir la vie ? \\[0pt]
Peut-on définir le bien ? \\[0pt]
Peut-on dériver des concepts de l'expérience ? \\[0pt]
Peut-on dire ce qui n'est pas ? \\[0pt]
Peut-on dire de la connaissance scientifique qu'elle procède par approximation ? \\[0pt]
Peut-on dire de l'art qu'il donne un monde en partage ? \\[0pt]
Peut-on dire d'une image qu'elle parle ? \\[0pt]
Peut-on dire d'une théorie scientifique qu'elle n'est jamais plus vraie qu'une autre mais seulement plus commode ? \\[0pt]
Peut-on dire l'être ? \\[0pt]
Peut-on dire que la science ne nous fait pas connaître les choses mais les rapports entre les choses ? \\[0pt]
Peut-on dire qu'est vrai ce qui correspond aux faits ? \\[0pt]
Peut-on dire qu'une théorie physique en contredit une autre ? \\[0pt]
Peut-on dire toute la vérité ? \\[0pt]
Peut-on disposer de son corps ? \\[0pt]
Peut-on distinguer différents types de causes ? \\[0pt]
Peut-on distinguer le vrai du faux ? \\[0pt]
Peut-on donner un sens à la souffrance ? \\[0pt]
Peut-on douter de la raison ? \\[0pt]
Peut-on douter de l'humanité d'un homme ? \\[0pt]
Peut-on douter de sa propre existence ? \\[0pt]
Peut-on douter de tout ? \\[0pt]
Peut-on douter du sens des mots ? \\[0pt]
Peut-on éclairer la liberté ? \\[0pt]
Peut-on en appeler à la conscience contre la loi ? \\[0pt]
Peut-on encore être cosmopolite ? \\[0pt]
Peut-on encore parler d'une nature humaine ? \\[0pt]
Peut-on en finir avec les préjugés ? \\[0pt]
Peut-on en savoir trop ? \\[0pt]
Peut-on entreprendre d'éliminer la métaphysique ? \\[0pt]
Peut-on établir une hiérarchie des arts ? \\[0pt]
Peut-on être amoral ? \\[0pt]
Peut-on être apolitique ? \\[0pt]
Peut-on être citoyen du monde ? \\[0pt]
Peut-on être citoyen sans être soldat ? \\[0pt]
Peut-on être content de soi ? \\[0pt]
Peut-on être désintéressé ? \\[0pt]
Peut-on être en conflit avec soi-même ? \\[0pt]
Peut-on être fidèle à soi-même ? \\[0pt]
Peut-on être heureux tout seul ? \\[0pt]
Peut-on être homme sans être citoyen ? \\[0pt]
Peut-on être hors de soi ? \\[0pt]
Peut-on être insensible à l'art ? \\[0pt]
Peut-on être juste dans une société injuste ? \\[0pt]
Peut-on être objectif en matière d'art ? \\[0pt]
Peut-on être plus ou moins libre ? \\[0pt]
Peut-on être prisonnier de soi-même ? \\[0pt]
Peut-on être sage tout seul ? \\[0pt]
Peut-on être sans opinion ? \\[0pt]
Peut-on être sûr d'avoir raison ? \\[0pt]
Peut-on être trop sage ? \\[0pt]
Peut-on être trop sensible ? \\[0pt]
Peut-on étudier l'homme de l'extérieur ? \\[0pt]
Peut-on expliquer une œuvre d'art ? \\[0pt]
Peut-on faire ce qu'on ne veut pas ? \\[0pt]
Peut-on faire comme si le passé n'existait pas ? \\[0pt]
Peut-on faire de l'art avec tout ? \\[0pt]
Peut-on faire de sa vie une œuvre d'art ? \\[0pt]
Peut-on faire du dialogue un modèle de relation morale ? \\[0pt]
Peut-on faire le bien de quelqu'un malgré lui ? \\[0pt]
Peut-on faire l'économie de la notion de forme ? \\[0pt]
Peut-on faire le mal en vue du bien ? \\[0pt]
Peut-on faire l'inventaire du monde ? \\[0pt]
Peut-on fixer des limites à la science ? \\[0pt]
Peut-on fonder le droit de propriété ? \\[0pt]
Peut-on fonder les droits de l'homme ? \\[0pt]
Peut-on fonder les mathématiques ? \\[0pt]
Peut-on fonder une morale sur la nature ? \\[0pt]
Peut-on gouverner sans lois ? \\[0pt]
Peut-on hiérarchiser les œuvres d'art ? \\[0pt]
Peut-on hiérarchiser les sociétés ? \\[0pt]
Peut-on innover en politique ? \\[0pt]
Peut-on inventer en morale ? \\[0pt]
Peut-on jamais aimer son prochain ? \\[0pt]
Peut-on juger de la valeur d'une vie humaine ? \\[0pt]
Peut-on juger des œuvres d'art sans recourir à l'idée de beauté ? \\[0pt]
Peut-on justifier la discrimination ? \\[0pt]
Peut-on justifier la guerre ? \\[0pt]
Peut-on justifier la raison d'État ? \\[0pt]
Peut-on justifier le mensonge ? \\[0pt]
Peut-on justifier le recours à l'idée de finalité ? \\[0pt]
Peut-on mesurer les phénomènes sociaux ? \\[0pt]
Peut-on montrer en cachant ? \\[0pt]
Peut-on ne croire en rien \\[0pt]
Peut-on ne croire en rien ? \\[0pt]
Peut-on ne pas être de son temps ? \\[0pt]
Peut-on ne pas être matérialiste ? \\[0pt]
Peut-on ne pas être soi-même ? \\[0pt]
Peut-on ne pas savoir ce que l'on fait ? \\[0pt]
Peut-on ne penser à rien ? \\[0pt]
Peut-on ne rien aimer ? \\[0pt]
Peut-on ne rien vouloir ? \\[0pt]
Peut-on objectiver le psychisme ? \\[0pt]
Peut-on opposer justice et liberté ? \\[0pt]
Peut-on opposer morale et technique ? \\[0pt]
Peut-on oublier ? \\[0pt]
Peut-on pactiser avec un brigand ? \\[0pt]
Peut-on parler d'art primitif ? \\[0pt]
Peut-on parler de capital culturel ? \\[0pt]
Peut-on parler de corruption des mœurs ? \\[0pt]
Peut-on parler de despotisme démocratique ? \\[0pt]
Peut-on parler de droits des animaux ? \\[0pt]
Peut-on parler de progrès en art ? \\[0pt]
Peut-on parler des œuvres d'art ? \\[0pt]
Peut-on parler de vérités métaphysiques ? \\[0pt]
Peut-on parler de vérité théâtrale ? \\[0pt]
Peut-on parler de vertu politique ? \\[0pt]
Peut-on parler d'un droit de la guerre ? \\[0pt]
Peut-on parler d'une science de l'art ? \\[0pt]
Peut-on parler d'un progrès moral ? \\[0pt]
Peut-on parler d'un savoir poétique ? \\[0pt]
Peut-on parler d'un sens moral ? \\[0pt]
Peut-on parler d'un style moral ? \\[0pt]
Peut-on parler d'un travail intellectuel ? \\[0pt]
Peut-on partager ses goûts ? \\[0pt]
Peut-on penser illogiquement ? \\[0pt]
Peut-on penser la création ? \\[0pt]
Peut-on penser la douleur ? \\[0pt]
Peut-on penser la fin de toute chose ? \\[0pt]
Peut-on penser la mort ? \\[0pt]
Peut-on penser l'art comme un langage ? \\[0pt]
Peut-on penser l'avenir ? \\[0pt]
Peut-on penser le divin ? \\[0pt]
Peut-on penser le réel comme un tout ? \\[0pt]
Peut-on penser le social sans la politique ? \\[0pt]
Peut-on penser le temps ? \\[0pt]
Peut-on penser l'extériorité ? \\[0pt]
Peut-on penser l'irrationnel ? \\[0pt]
Peut-on penser sans concept ? \\[0pt]
Peut-on penser sans concepts ? \\[0pt]
Peut-on penser sans les mots ? \\[0pt]
Peut-on penser sans méthode ? \\[0pt]
Peut-on penser sans règles ? \\[0pt]
Peut-on penser sans signes ? \\[0pt]
Peut-on penser un art sans œuvres ? \\[0pt]
Peut-on penser une métaphysique sans Dieu ? \\[0pt]
Peut-on penser une volonté diabolique ? \\[0pt]
Peut-on penser un pouvoir absolu ? \\[0pt]
Peut-on percevoir sans s'en apercevoir ? \\[0pt]
Peut-on perdre la raison ? \\[0pt]
Peut-on perdre sa liberté ? \\[0pt]
Peut-on perdre son identité ? \\[0pt]
Peut-on permettre le mensonge ? \\[0pt]
Peut-on préconiser, dans les sciences humaines et sociales, l'imitation des sciences de la nature ? \\[0pt]
Peut-on prendre le risque de donner la mort en voulant soulager la souffrance ? \\[0pt]
Peut-on prévoir l'avenir ? \\[0pt]
Peut-on prévoir les hommes ? \\[0pt]
Peut-on prouver l'existence ? \\[0pt]
Peut-on prouver l'existence du monde ? \\[0pt]
Peut-on recommencer sa vie ? \\[0pt]
Peut-on réduire la pensée à une espèce de comportement ? \\[0pt]
Peut-on réduire une métaphysique à une conception du monde ? \\[0pt]
Peut-on refuser la loi ? \\[0pt]
Peut-on refuser l'évidence ? \\[0pt]
Peut-on régner innocemment ? \\[0pt]
Peut-on réparer une injustice ? \\[0pt]
Peut-on représenter l'espace ? \\[0pt]
Peut-on représenter l'invisible ? \\[0pt]
Peut-on reprocher à la morale d'être abstraite ? \\[0pt]
Peut-on rester insensible à la beauté ? \\[0pt]
Peut-on restreindre la logique à la pensée formelle ? \\[0pt]
Peut-on résumer une philosophie ? \\[0pt]
Peut-on réunir des arts différents dans une même œuvre ? \\[0pt]
Peut-on revendiquer la paix comme un droit ? \\[0pt]
Peut-on rire de tout ? \\[0pt]
Peut-on s'abstenir de penser politiquement ? \\[0pt]
Peut-on s'accorder sur des vérités morales ? \\[0pt]
Peut-on savoir ce qui est bien ? \\[0pt]
Peut-on se connaître soi-même ? \\[0pt]
Peut-on se désintéresser de la politique ? \\[0pt]
Peut-on se faire une idée de tout ? \\[0pt]
Peut-on se fier à son intuition ? \\[0pt]
Peut-on se gouverner soi-même ? \\[0pt]
Peut-on se mettre à la place d'autrui ? \\[0pt]
Peut-on séparer politique et économie ? \\[0pt]
Peut-on se passer de chef ? \\[0pt]
Peut-on se passer de Dieu ? \\[0pt]
Peut-on se passer de l'État ? \\[0pt]
Peut-on se passer de l'idée de cause finale ? \\[0pt]
Peut-on se passer de l'idée de Dieu ? \\[0pt]
Peut-on se passer de l'idée de nature humaine ? \\[0pt]
Peut-on se passer de métaphysique ? \\[0pt]
Peut-on se passer de représentants ? \\[0pt]
Peut-on se passer des causes finales ? \\[0pt]
Peut-on se passer des relations ? \\[0pt]
Peut-on se passer d'ontologie ? \\[0pt]
Peut-on se passer d'un maître ? \\[0pt]
Peut-on se peindre soi-même ? \\[0pt]
Peut-on se punir soi-même ? \\[0pt]
Peut-on se régler sur des exemples en politique ? \\[0pt]
Peut-on se retirer du monde ? \\[0pt]
Peut-on souhaiter le gouvernement des meilleurs ? \\[0pt]
Peut-on tout attendre de l'État ? \\[0pt]
Peut-on tout définir ? \\[0pt]
Peut-on tout démontrer ? \\[0pt]
Peut-on tout dire ? \\[0pt]
Peut-on tout justifier ? \\[0pt]
Peut-on tout mesurer ? \\[0pt]
Peut-on tout prouver ? \\[0pt]
Peut-on tout soumettre à la discussion ? \\[0pt]
Peut-on transiger avec les principes ? \\[0pt]
Peut-on trop penser ? \\[0pt]
Peut-on trouver du plaisir à l'ennui ? \\[0pt]
Peut-on vivre avec les autres ? \\[0pt]
Peut-on vivre dans le doute ? \\[0pt]
Peut-on vivre sans art ? \\[0pt]
Peut-on vivre sans aucune certitude ? \\[0pt]
Peut-on vivre sans croyance ? \\[0pt]
Peut-on vivre sans foi ni loi ? \\[0pt]
Peut-on vivre sans opinions ? \\[0pt]
Peut-on vivre sans ressentiment ? \\[0pt]
Peut-on vivre sans travailler ? \\[0pt]
Peut-on vivre sans valeurs ? \\[0pt]
Peut-on vouloir le bien sans le faire ? \\[0pt]
Peut-on vouloir le mal ? \\[0pt]
Phénomène ou fait social ? \\[0pt]
Philosopher, est-ce apprendre à vivre ? \\[0pt]
Philosophie et mathématiques \\[0pt]
Philosophie et métaphysique \\[0pt]
Philosophie et religion \\[0pt]
Philosophie et théologie \\[0pt]
Philosophie, psychologie, sociologie \\[0pt]
Physique et cosmologie \\[0pt]
Physique et mathématique \\[0pt]
Physique et mathématiques \\[0pt]
Physique et métaphysique \\[0pt]
Plaider \\[0pt]
Plaisir et préférences \\[0pt]
Poésie et philosophie \\[0pt]
Poésie et vérité \\[0pt]
Point de vue du créateur et point de vue du spectateur \\[0pt]
Police et politique \\[0pt]
Politique et esthétique \\[0pt]
Politique et médias \\[0pt]
Politique et mémoire \\[0pt]
Politique et parole \\[0pt]
Politique et participation \\[0pt]
Politique et passions \\[0pt]
Politique et propagande \\[0pt]
Politique et religion \\[0pt]
Politique et secret \\[0pt]
Politique et technique \\[0pt]
Politique et technologie \\[0pt]
Politique et territoire \\[0pt]
Politique et trahison \\[0pt]
Politique et vertu \\[0pt]
Position, rôle, fonction \\[0pt]
Possibilité et intelligibilité \\[0pt]
Pour apprécier une œuvre, faut-il être cultivé ? \\[0pt]
Pour dialoguer faut-il parler le même langage ? \\[0pt]
Pourquoi a-t-on peur de la folie ? \\[0pt]
Pourquoi avoir recours à la notion d'inconscient ? \\[0pt]
Pourquoi châtier ? \\[0pt]
Pourquoi chercher un sens à l'histoire ? \\[0pt]
Pourquoi commémorer ? \\[0pt]
Pourquoi conserver des œuvres d'art ? \\[0pt]
Pourquoi conserver les œuvres d'art ? \\[0pt]
Pourquoi croyons-nous ? \\[0pt]
Pourquoi définir ? \\[0pt]
Pourquoi des artifices ? \\[0pt]
Pourquoi des artistes ? \\[0pt]
Pourquoi des cérémonies ? \\[0pt]
Pourquoi des châtiments ? \\[0pt]
Pourquoi des classifications ? \\[0pt]
Pourquoi des comités d'éthique ? \\[0pt]
Pourquoi des conflits ? \\[0pt]
Pourquoi des constitutions ? \\[0pt]
Pourquoi des corps intermédiaires ? \\[0pt]
Pourquoi des définitions ? \\[0pt]
Pourquoi des élections ? \\[0pt]
Pourquoi des exemples ? \\[0pt]
Pourquoi des fictions ? \\[0pt]
Pourquoi des géométries ? \\[0pt]
Pourquoi des historiens ? \\[0pt]
Pourquoi des hypothèses ? \\[0pt]
Pourquoi des institutions ? \\[0pt]
Pourquoi désirer ? \\[0pt]
Pourquoi des logiciens ? \\[0pt]
Pourquoi des lois ? \\[0pt]
Pourquoi des métaphores ? \\[0pt]
Pourquoi des modèles ? \\[0pt]
Pourquoi des musées ? \\[0pt]
Pourquoi des œuvres d'art ? \\[0pt]
Pourquoi des poètes ? \\[0pt]
Pourquoi des rites ? \\[0pt]
Pourquoi des sociologues ? \\[0pt]
Pourquoi des syllogismes ? \\[0pt]
Pourquoi des symboles ? \\[0pt]
Pourquoi des utopies ? \\[0pt]
Pourquoi Dieu se soucierait-il des affaires humaines ? \\[0pt]
Pourquoi dire la vérité ? \\[0pt]
Pourquoi donner ? \\[0pt]
Pourquoi douter ? \\[0pt]
Pourquoi écrire ? \\[0pt]
Pourquoi écrit-on des lois ? \\[0pt]
Pourquoi est-il difficile de rectifier une erreur ? \\[0pt]
Pourquoi être exigeant ? \\[0pt]
Pourquoi être moral ? \\[0pt]
Pourquoi exiger la cohérence \\[0pt]
Pourquoi faire de la métaphysique ? \\[0pt]
Pourquoi faire de la politique ? \\[0pt]
Pourquoi faire de l'histoire ? \\[0pt]
Pourquoi faire la guerre ? \\[0pt]
Pourquoi fait-on le mal ? \\[0pt]
Pourquoi faudrait-il être cohérent ? \\[0pt]
Pourquoi faudrait-il sauver les apparences ? \\[0pt]
Pourquoi faut-il agir ? \\[0pt]
Pourquoi faut-il être cohérent ? \\[0pt]
Pourquoi formaliser des arguments ? \\[0pt]
Pourquoi il n'y a pas de société sans art ? \\[0pt]
Pourquoi la musique intéresse-t-elle le philosophe ? \\[0pt]
Pourquoi la raison recourt-elle à l'hypothèse ? \\[0pt]
Pourquoi l'art intéresse-t-il les philosophes ? \\[0pt]
Pourquoi le droit international est-il imparfait ? \\[0pt]
Pourquoi les États se font-ils la guerre ? \\[0pt]
Pourquoi les hommes jouent-ils ? \\[0pt]
Pourquoi les mathématiques s'appliquent-elles à la réalité ? \\[0pt]
Pourquoi les œuvres d'art résistent-elles au temps ? \\[0pt]
Pourquoi les sciences humaines s'intéressent-elles à la parenté ? \\[0pt]
Pourquoi l'ethnologue s'intéresse-t-il à la vie urbaine ? \\[0pt]
Pourquoi l'homme est-il l'objet de plusieurs sciences ? \\[0pt]
Pourquoi lire des romans ? \\[0pt]
Pourquoi mentir ? \\[0pt]
Pourquoi nous souvenons-nous ? \\[0pt]
Pourquoi nous trompons-nous ? \\[0pt]
Pourquoi obéir ? \\[0pt]
Pourquoi obéir aux lois ? \\[0pt]
Pourquoi oublie-t-on ? \\[0pt]
Pourquoi pardonner ? \\[0pt]
Pourquoi parler de jugement de goût ? \\[0pt]
Pourquoi parlons-nous ? \\[0pt]
Pourquoi peindre la nature ? \\[0pt]
Pourquoi penser l'impossible ? \\[0pt]
Pourquoi pensons-nous ? \\[0pt]
Pourquoi plusieurs sciences ? \\[0pt]
Pourquoi préférer l'original à sa reproduction ? \\[0pt]
Pourquoi promettre ? \\[0pt]
Pourquoi punir ? \\[0pt]
Pourquoi punit-on ? \\[0pt]
Pourquoi sauver les phénomènes ? \\[0pt]
Pourquoi se mettre à la place d'autrui ? \\[0pt]
Pourquoi séparer les pouvoirs ? \\[0pt]
Pourquoi se référer au destin ? \\[0pt]
Pourquoi s'étonner ? \\[0pt]
Pourquoi s'inspirer de l'art antique ? \\[0pt]
Pourquoi sommes-nous déçus par les œuvres d'un faussaire ? \\[0pt]
Pourquoi sommes-nous moraux ? \\[0pt]
Pourquoi travailler ? \\[0pt]
Pourquoi travaille-t-on ? \\[0pt]
Pourquoi un droit du travail ? \\[0pt]
Pourquoi une instruction publique ? \\[0pt]
Pourquoi vouloir gagner ? \\[0pt]
Pourquoi vouloir la vérité ? \\[0pt]
Pourquoi vouloir s'abstraire du quotidien ? \\[0pt]
Pourquoi voyager ? \\[0pt]
Pourquoi y a-t-il des conflits insolubles ? \\[0pt]
Pourquoi y a-t-il plusieurs philosophies ? \\[0pt]
Pourquoi y a-t-il quelque chose plutôt que rien ? \\[0pt]
Pourrait-on vivre sans art ? \\[0pt]
Pourrait-on vivre sans idéal ? \\[0pt]
Pourrions-nous nous passer des musées ? \\[0pt]
Pourrions-nous vivre sans religion ? \\[0pt]
Pouvoir et autorité \\[0pt]
Pouvoir et contre-pouvoir \\[0pt]
Pouvoir et devoir \\[0pt]
Pouvoir et politique \\[0pt]
Pouvoir et puissance \\[0pt]
Pouvoir et savoir \\[0pt]
Pouvoirs et libertés \\[0pt]
Pouvoir temporel et pouvoir spirituel \\[0pt]
Pouvons-nous devenir meilleurs ? \\[0pt]
Pouvons-nous vivre sans préjugés ? \\[0pt]
Prédicats et relations \\[0pt]
Prémisses et conclusions \\[0pt]
Prendre des risques \\[0pt]
Prendre le pouvoir \\[0pt]
Prendre les armes \\[0pt]
« Prendre ses désirs pour des réalités » \\[0pt]
Prendre soin \\[0pt]
Prendre son temps \\[0pt]
Prendre sur soi \\[0pt]
Prendre une décision \\[0pt]
Prendre une décision politique \\[0pt]
Prescrire et décrire \\[0pt]
Prescrire et interdire \\[0pt]
Présence et absence \\[0pt]
Présence et représentation \\[0pt]
Preuve et démonstration \\[0pt]
Preuve et résultat dans les sciences humaines \\[0pt]
Prévoir \\[0pt]
Prévoir les comportements humains \\[0pt]
Primitif ou premier ? \\[0pt]
Prince : nature ou fonction ? \\[0pt]
Principe et cause \\[0pt]
Principe et commencement \\[0pt]
Principe et conséquence \\[0pt]
Principe et fondement \\[0pt]
Privation et négation \\[0pt]
Probabilité et explication scientifique \\[0pt]
Problème, énigme, mystère \\[0pt]
Production et création \\[0pt]
Production et réception de l'œuvre d'art \\[0pt]
Produire et consommer \\[0pt]
Promettre \\[0pt]
Prophétie, utopie, pronostic \\[0pt]
Proposition et jugement \\[0pt]
Propriété privée, propriété collective \\[0pt]
Propriétés artistiques, propriétés esthétiques \\[0pt]
Propriétés et dispositions \\[0pt]
Prose et poésie \\[0pt]
Prospérité et sécurité \\[0pt]
Protester \\[0pt]
Prouver \\[0pt]
Prouver en métaphysique \\[0pt]
Prouver et éprouver \\[0pt]
Providence et destin \\[0pt]
Psychanalyse et critique littéraire \\[0pt]
Psychanalyse et esthétique \\[0pt]
Psychanalyse et interprétation \\[0pt]
Psychologie et contrôle des comportements \\[0pt]
Psychologie et ethnologie \\[0pt]
Psychologie et métaphysique \\[0pt]
Psychologie et neurosciences \\[0pt]
Psychologie et philosophie \\[0pt]
Publier \\[0pt]
Puis-je être sûr de bien agir ? \\[0pt]
Puis-je être universel ? \\[0pt]
Puis-je juger une culture à laquelle je n'appartiens pas ? \\[0pt]
Puis-je me mettre à la place d'un autre ? \\[0pt]
Puis-je ne rien croire ? \\[0pt]
Puissance et pouvoir \\[0pt]
Pulsion et instinct \\[0pt]
Pulsions et passions \\[0pt]
Punir \\[0pt]
Qu'admire-t-on dans une œuvre d'art ? \\[0pt]
Qu'aime-t-on dans l'amour ? \\[0pt]
Qu'aime-t-on quand on aime une œuvre d'art ? \\[0pt]
Qu'aimons-nous dans l'amour ? \\[0pt]
Qualités premières et qualités secondes \\[0pt]
Qualités premières, qualités secondes \\[0pt]
Quand agit-on ? \\[0pt]
Quand faut-il désobéir ? \\[0pt]
Quand faut-il se taire ? \\[0pt]
Quand la guerre finira-t-elle ? \\[0pt]
Quand l'art est-il abstrait ? \\[0pt]
Quand la technique devient-elle art ? \\[0pt]
Quand pense-t-on ? \\[0pt]
Quand suis-je en faute ? \\[0pt]
Quand suis-je moi-même ? \\[0pt]
Quand y a-t-il art ? \\[0pt]
Quand y a-t-il métaphore ? \\[0pt]
Quand y a-t-il œuvre ? \\[0pt]
Quand y a-t-il paysage ? \\[0pt]
Quand y a-t-il peuple ? \\[0pt]
Quantification et intelligibilité \\[0pt]
Quantification et pensée scientifique \\[0pt]
Quantifier \\[0pt]
Quantité et qualité \\[0pt]
Qu'a perdu le fou ? \\[0pt]
Qu'appelle-t-on chef-d'œuvre ? \\[0pt]
Qu'apporte au mathématicien l'histoire des mathématiques ? \\[0pt]
Qu'apporte la photographie aux arts ? \\[0pt]
Qu'apporte l'histoire à la vie politique ? \\[0pt]
Qu'apprend-on dans les livres ? \\[0pt]
Qu'apprenons-nous de nos affects ? \\[0pt]
Qu'a-t-on le droit de pardonner ? \\[0pt]
Qu'a-t-on le droit d'interpréter ? \\[0pt]
Qu'attendons-nous des œuvres d'art ? \\[0pt]
Qu'avons-nous à apprendre de nos illusions ? \\[0pt]
Qu'avons-nous à apprendre des historiens ? \\[0pt]
 Qu'avons-nous à attendre de la vie politique ? \\[0pt]
Que changent les révolutions ? \\[0pt]
Qu'échange-t-on ? \\[0pt]
Que cherchons-nous dans le regard des autres ? \\[0pt]
Que comprend-on d'une œuvre d'art ? \\[0pt]
Que connaissons-nous du vivant ? \\[0pt]
Que connaît-on de ses passions ? \\[0pt]
Que construit le politique ? \\[0pt]
Que crée l'artiste ? \\[0pt]
Que déduire d'une contradiction ? \\[0pt]
Que désirons-nous ? \\[0pt]
Que devons-nous à l'animal ? \\[0pt]
Que disent les tables de vérité ? \\[0pt]
Que dit la loi ? \\[0pt]
Que dit la musique ? \\[0pt]
Que dit l'interdit ? \\[0pt]
Que dois-je à l'État ? \\[0pt]
Que doit l'art à la nature ? \\[0pt]
Que doit-on à autrui ? \\[0pt]
Que doit-on aux morts ? \\[0pt]
Que doit-on faire de ses rêves ? \\[0pt]
Que doit-on protéger ? \\[0pt]
Que faire de nos passions ? \\[0pt]
Que faire des adversaires ? \\[0pt]
Que faire du vraisemblable ? \\[0pt]
Que fait aux œuvres d'art leur reproductibilité ? \\[0pt]
Que fait la police ? \\[0pt]
Que fait l'art à nos vies ? \\[0pt]
Que fait l'artiste ? \\[0pt]
Que fait le spectateur ? \\[0pt]
Que faut-il craindre ? \\[0pt]
Que faut-il pour faire un monde ? \\[0pt]
Que faut-il savoir pour agir ? \\[0pt]
Que faut-il savoir pour gouverner ? \\[0pt]
Que gagne l'art à devenir abstrait ? \\[0pt]
Que gagne l'art à se réfléchir ? \\[0pt]
Que gagne-t-on à se mettre à la place d'autrui ? \\[0pt]
Que garantit la séparation des pouvoirs ? \\[0pt]
Quel est le but de la politique ? \\[0pt]
Quel est le but d'une théorie physique ? \\[0pt]
Quel est le but du travail scientifique ? \\[0pt]
Quel est le fondement de la propriété ? \\[0pt]
Quel est le fondement de l'autorité ? \\[0pt]
Quel est le meilleur régime politique ? \\[0pt]
Quel est le pouvoir de la beauté ? \\[0pt]
Quel est le pouvoir de l'art ? \\[0pt]
Quel est le pouvoir des métaphores ? \\[0pt]
Quel est le problème posé par la pluralité des langues ? \\[0pt]
Quel est le rôle de la créativité dans les sciences ? \\[0pt]
Quel est le rôle du médecin ? \\[0pt]
Quel est le statut de la preuve en sciences humaines et sociales ? \\[0pt]
Quel est le sujet de l'histoire ? \\[0pt]
Quel est le sujet des sciences sociales ? \\[0pt]
Quel est l'être de l'illusion ? \\[0pt]
Quel est l'homme de l'ethnologie ? \\[0pt]
Quel est l'homme des Droits de l'homme ? \\[0pt]
Quel est l'humain des sciences humaines ? \\[0pt]
Quel est l'objectif des sciences humaines ? \\[0pt]
Quel est l'objet de la géométrie ? \\[0pt]
Quel est l'objet de la logique ? \\[0pt]
Quel est l'objet de la métaphysique ? \\[0pt]
Quel est l'objet de la philosophie politique ? \\[0pt]
Quel est l'objet de la psychologie ? \\[0pt]
Quel est l'objet de la psychologie collective ? \\[0pt]
Quel est l'objet de la psychologie sociale ? \\[0pt]
Quel est l'objet de la science ? \\[0pt]
Quel est l'objet de l'échange ? \\[0pt]
Quel est l'objet de l'esthétique ? \\[0pt]
Quel est l'objet de l'ontologie ? \\[0pt]
Quel est l'objet des sciences politiques ? \\[0pt]
« Quelle chimère est-ce donc que l'homme » ? \\[0pt]
Quelle espèce de vivants sommes-nous ? \\[0pt]
Quelle est la matière de l'œuvre d'art ? \\[0pt]
Quelle est la nature du droit naturel ? \\[0pt]
Quelle est la place du sujet dans les sciences humaines ? \\[0pt]
Quelle est la réalité de la matière ? \\[0pt]
Quelle est la réalité des objets mathématiques ? \\[0pt]
Quelle est la spécificité de la communauté politique ? \\[0pt]
Quelle est la valeur de la connaissance ? \\[0pt]
Quelle est la valeur du témoignage ? \\[0pt]
Quelle expérience la politique requiert-elle ? \\[0pt]
Quelle idée le fanatique se fait-il de la vérité ? \\[0pt]
Quelle place donner au vraisemblable ? \\[0pt]
Quelle place la raison peut-elle faire à la croyance ? \\[0pt]
Quelle place les sciences humaines doivent-elles accorder à la subjectivité ? \\[0pt]
Quelle place pour la mesure dans les sciences de l'homme ? \\[0pt]
Quelle politique fait-on avec les sciences humaines ? \\[0pt]
Quelle réalité conférer aux fictions ? \\[0pt]
Quelle réalité la science décrit-elle ? \\[0pt]
Quelle réalité l'imagination nous fait-elle connaître ? \\[0pt]
Quelle responsabilité publique pour le chercheur en sciences sociales ? \\[0pt]
Quelles actions permettent d'être heureux ? \\[0pt]
Quelles lois pour les sciences humaines ? \\[0pt]
Quelles règles la technique dicte-t-elle à l'art ? \\[0pt]
Quelles sont les caractéristiques d'une proposition morale ? \\[0pt]
Quelle valeur donner à la notion de « corps social » ? \\[0pt]
Quel réel pour l'art ? \\[0pt]
Quel rôle attribuer à l'intuition \emph{a priori} dans une philosophie des mathématiques ? \\[0pt]
Quel rôle la logique joue-t-elle en mathématiques ? \\[0pt]
Quel rôle les statistiques jouent-elles dans les sciences humaines ? \\[0pt]
Quel rôle l'imagination joue-t-elle en mathématiques ? \\[0pt]
Quels critères de démarcation pour les sciences sociales ? \\[0pt]
Quel sens donner au projet de forger une langue parfaite ? \\[0pt]
Quel sens y a-t-il à se demander si les sciences humaines sont vraiment des sciences ? \\[0pt]
Quels matériaux pour les sciences humaines et sociales ? \\[0pt]
Quels sont les moyens légitimes de la politique ? \\[0pt]
Quel statut philosophique pour l'opinion ? \\[0pt]
Quel type de réalité faut-il attribuer à l'inconscient en psychanalyse ? \\[0pt]
Quel type de réalité faut-il attribuer au temps ? \\[0pt]
Quel usage faire de la finalité ? \\[0pt]
Quel vivant sommes-nous ? \\[0pt]
Que montre l'image ? \\[0pt]
Que montre un tableau ? \\[0pt]
Que ne peut-on pas expliquer ? \\[0pt]
Que nous apporte l'art ? \\[0pt]
Que nous apprend la biologie sur l'homme ? \\[0pt]
Que nous apprend la géométrie ? \\[0pt]
Que nous apprend la littérature ? \\[0pt]
Que nous apprend la métaphysique ? \\[0pt]
Que nous apprend l'animal ? \\[0pt]
Que nous apprend l'anthropologie ? \\[0pt]
Que nous apprend l'approche sociologique de l'art ? \\[0pt]
Que nous apprend la psychanalyse de l'homme ? \\[0pt]
Que nous apprend la sociologie des sciences ? \\[0pt]
Que nous apprend le plaisir ? \\[0pt]
Que nous apprend l'erreur ? \\[0pt]
Que nous apprend le témoignage ? \\[0pt]
Que nous apprend le théâtre ? \\[0pt]
Que nous apprend le toucher ? \\[0pt]
Que nous apprend l'expérience esthétique ? \\[0pt]
Que nous apprend l'histoire de l'art ? \\[0pt]
Que nous apprend l'histoire des sciences ? \\[0pt]
Que nous apprend, sur la politique, l'utopie ? \\[0pt]
Que nous apprennent les algorithmes sur nos sociétés ? \\[0pt]
Que nous apprennent les Anciens ? \\[0pt]
Que nous apprennent les controverses scientifiques ? \\[0pt]
Que nous apprennent les faits divers ? \\[0pt]
Que nous apprennent les jeux ? \\[0pt]
Que nous apprennent les langues étrangères ? \\[0pt]
Que nous apprennent les objets techniques ? \\[0pt]
Que nous apprennent les statistiques ? \\[0pt]
Que nous apprennent nos sentiments ? \\[0pt]
Que nous devons-nous ? \\[0pt]
Que nous doit l'État ? \\[0pt]
Que nous enseigne la lecture des romans ? \\[0pt]
Que nous enseigne la monstruosité ? \\[0pt]
Que nous montrent les natures mortes ? \\[0pt]
« Que nul n'entre ici s'il n'est géomètre » \\[0pt]
Que peindre ? \\[0pt]
Que peint le peintre ? \\[0pt]
Que perd la pensée en perdant l'écriture ? \\[0pt]
Que peut la force ? \\[0pt]
Que peut la politique ? \\[0pt]
Que peut l'art ? \\[0pt]
Que peut la volonté contre les passions ? \\[0pt]
Que peut le droit ? \\[0pt]
Que peut le politique ? \\[0pt]
Que peut l'État ? \\[0pt]
Que peut l'intelligence artificielle ? \\[0pt]
Que peut-on apprendre des émotions esthétiques ? \\[0pt]
Que peut-on attendre de l'État ? \\[0pt]
Que peut-on attendre du droit international ? \\[0pt]
Que peut-on attendre d'une philosophie de l'art ? \\[0pt]
Que peut-on calculer ? \\[0pt]
Que peut-on comprendre qu'on ne puisse expliquer ? \\[0pt]
Que peut-on démontrer ? \\[0pt]
Que peut-on dire de l'être ? \\[0pt]
Que peut-on enseigner ? \\[0pt]
Que peut-on espérer ? \\[0pt]
Que peut-on exprimer ? \\[0pt]
Que peut-on interdire ? \\[0pt]
Que peut-on partager ? \\[0pt]
Que peut-on perdre ? \\[0pt]
Que peut signifier « avoir le sens de la réalité » ? \\[0pt]
Que peut un corps ? \\[0pt]
Que peuvent les idées ? \\[0pt]
Que peuvent les images ? \\[0pt]
Que peuvent les lois ? \\[0pt]
Que peuvent nous apprendre les expériences de pensée ? \\[0pt]
Que pouvons-nous aujourd'hui apprendre des sciences d'autrefois ? \\[0pt]
Que pouvons-nous comprendre du monde ? \\[0pt]
Que pouvons-nous espérer ? \\[0pt]
Que pouvons-nous partager ? \\[0pt]
Que protège la loi ? \\[0pt]
Que prouvent les preuves de l'existence de Dieu ? \\[0pt]
Que puis-je dire être mien ? \\[0pt]
Que rend visible l'art ? \\[0pt]
Que sais-je de ma souffrance ? \\[0pt]
Que sait la science ? \\[0pt]
Que savons-nous de l'avenir ? \\[0pt]
Que savons-nous de nous-mêmes ? \\[0pt]
Que savons-nous des principes ? \\[0pt]
Que sent le corps ? \\[0pt]
Que serait le meilleur des mondes ? \\[0pt]
Que serait un art total ? \\[0pt]
Que serait une démocratie parfaite ? \\[0pt]
Que serions-nous sans l'État ? \\[0pt]
Que signifie apprendre ? \\[0pt]
Que signifie : « avoir de l'esprit » ? \\[0pt]
Que signifie être dépendant ? \\[0pt]
Que signifie l'angoisse ? \\[0pt]
Que signifie « politiquement correct » ? \\[0pt]
Que signifie prouver en sciences sociales ? \\[0pt]
Que sondent les sondages d'opinion ? \\[0pt]
Que sont les institutions sans les individus ? \\[0pt]
Qu'est-ce la vérité ? \\[0pt]
Qu'est-ce qu'abstraire ? \\[0pt]
Qu'est-ce qu'agir avec discernement ? \\[0pt]
Qu'est-ce qu'apparaître ? \\[0pt]
Qu'est-ce qu'appliquer une règle de droit ? \\[0pt]
Qu'est-ce qu'apprécier une œuvre d'art ? \\[0pt]
Qu'est-ce qu'apprendre ? \\[0pt]
Qu'est-ce qu'avoir conscience de soi ? \\[0pt]
Qu'est-ce qu'avoir de l'expérience ? \\[0pt]
Qu'est-ce qu'avoir du goût ? \\[0pt]
Qu'est-ce qu'avoir du style ? \\[0pt]
Qu'est-ce qu'avoir raison ? \\[0pt]
Qu'est-ce qu'avoir un droit ? \\[0pt]
Qu'est-ce qu'avoir une idée ? \\[0pt]
Qu'est-ce que bien vivre ? \\[0pt]
Qu'est-ce que calculer ? \\[0pt]
Qu'est-ce que classer ? \\[0pt]
Qu'est-ce que comprendre une œuvre d'art ? \\[0pt]
Qu'est-ce que connaître ? \\[0pt]
Qu'est-ce que démontrer ? \\[0pt]
Qu'est-ce que déraisonner ? \\[0pt]
Qu'est-ce que Dieu pour un athée ? \\[0pt]
Qu'est-ce que discuter ? \\[0pt]
Qu'est-ce qu'éduquer le sens esthétique ? \\[0pt]
Qu'est-ce que faire autorité ? \\[0pt]
Qu'est-ce que « faire de la politique » ? \\[0pt]
Qu'est-ce que faire la paix ? \\[0pt]
Qu'est-ce que faire preuve d'humanité ? \\[0pt]
Qu'est-ce que gouverner ? \\[0pt]
Qu'est-ce que guérir ? \\[0pt]
Qu'est-ce que juger ? \\[0pt]
Qu'est-ce que la culture générale \\[0pt]
Qu'est-ce que l'anthropologie politique ? \\[0pt]
Qu'est-ce que la psychanalyse nous apprend sur les sociétés humaines ? \\[0pt]
Qu'est-ce que la psychologie ? \\[0pt]
Qu'est-ce que l'art contemporain ? \\[0pt]
Qu'est-ce que la sociologie nous apprend sur la culture ? \\[0pt]
Qu'est-ce que le désordre ? \\[0pt]
Qu'est-ce que le génie ? \\[0pt]
Qu'est-ce que le mal radical ? \\[0pt]
Qu'est-ce que le moi ? \\[0pt]
Qu'est-ce que le naturalisme ? \\[0pt]
Qu'est-ce que l'enfance ? \\[0pt]
Qu'est-ce que les sciences humaines apportent à la critique littéraire ? \\[0pt]
Qu'est-ce que les sciences religieuses nous apprennent de la religion ? \\[0pt]
Qu'est-ce que le style ? \\[0pt]
Qu'est-ce que le symbolique ? \\[0pt]
Qu'est-ce que le temps ? \\[0pt]
Qu'est-ce que l'expérience pour un empiriste ? \\[0pt]
Qu'est-ce que l'harmonie ? \\[0pt]
Qu'est-ce que l'identité sociale des individus ? \\[0pt]
Qu'est-ce que l'individualisme méthodologique ? \\[0pt]
Qu'est-ce que l'interprétation des Écritures nous apprend sur les religions ? \\[0pt]
Qu'est-ce que lire ? \\[0pt]
Qu'est-ce que méditer ? \\[0pt]
Qu'est-ce qu'enquêter ? \\[0pt]
Qu'est-ce que parler ? \\[0pt]
Qu'est-ce que perdre son temps ? \\[0pt]
Qu'est-ce que peut un corps ? \\[0pt]
Qu'est-ce que prendre conscience ? \\[0pt]
Qu'est-ce que prendre le pouvoir ? \\[0pt]
Qu'est-ce que raisonner ? \\[0pt]
Qu'est-ce que rendre raison d'un effet ? \\[0pt]
Qu'est-ce que résoudre une contradiction ? \\[0pt]
Qu'est-ce que réussir sa vie ? \\[0pt]
Qu'est-ce que se tromper ? \\[0pt]
Qu'est-ce que s'orienter ? \\[0pt]
Qu'est-ce que traduire ? \\[0pt]
Qu'est-ce qu'être chez soi ? \\[0pt]
Qu'est-ce qu'être comportementaliste ? \\[0pt]
Qu'est-ce qu'être de bonne volonté ? \\[0pt]
Qu'est-ce qu'être démocrate ? \\[0pt]
Qu'est-ce qu'être fou ? \\[0pt]
Qu'est-ce qu'être libéral ? \\[0pt]
Qu'est-ce qu'être malade ? \\[0pt]
Qu'est-ce qu'être réaliste ? \\[0pt]
Qu'est-ce qu'être républicain ? \\[0pt]
Qu'est-ce qu'être sceptique ? \\[0pt]
Qu'est-ce qu'être seul ? \\[0pt]
Qu'est-ce qu'être souverain ? \\[0pt]
Qu'est-ce qu'être un esclave ? \\[0pt]
Qu'est-ce qu'être vertueux ? \\[0pt]
Qu'est-ce qu'être visionnaire ? \\[0pt]
Qu'est-ce qu'être vivant ? \\[0pt]
Qu'est-ce que un individu \\[0pt]
Qu'est-ce que vieillir ? \\[0pt]
Qu'est-ce que vivre ? \\[0pt]
Qu'est-ce que vivre au présent ? \\[0pt]
Qu'est-ce qu'exercer un pouvoir ? \\[0pt]
Qu'est-ce qu'expliquer un comportement ? \\[0pt]
Qu'est-ce qu'habiter ? \\[0pt]
Qu'est-ce qui adoucit les mœurs ? \\[0pt]
Qu'est-ce qui a du sens ? \\[0pt]
Qu'est-ce qui agit ? \\[0pt]
Qu'est-ce qui apparaît ? \\[0pt]
Qu'est-ce qu'identifier ? \\[0pt]
Qu'est-ce qui dépend de nous ? \\[0pt]
Qu'est-ce qui échappe aux sciences humaines ? \\[0pt]
Qu'est-ce qui est anormal ? \\[0pt]
Qu'est-ce qui est artificiel ? \\[0pt]
Qu'est-ce qui est beau ? \\[0pt]
Qu'est-ce qui est commun à toutes les langues ? \\[0pt]
Qu'est-ce qui est concret ? \\[0pt]
Qu'est-ce qui est contre nature ? \\[0pt]
Qu'est-ce qui est démonstratif ? \\[0pt]
Qu'est-ce qui est donné ? \\[0pt]
Qu'est-ce qui est essentiel ? \\[0pt]
Qu'est-ce qui est historique ? \\[0pt]
Qu'est-ce qui est humain ? \\[0pt]
Qu'est-ce qui est impossible ? \\[0pt]
Qu'est-ce qui est indiscutable ? \\[0pt]
Qu'est-ce qui est invérifiable ? \\[0pt]
Qu'est-ce qui est irréfutable ? \\[0pt]
Qu'est-ce qui est mien ? \\[0pt]
Qu'est-ce qui est moderne ? \\[0pt]
Qu'est-ce qui est nécessaire ? \\[0pt]
Qu'est-ce qui est noble ? \\[0pt]
Qu'est-ce qui est normal ? \\[0pt]
Qu'est-ce qui est politique ? \\[0pt]
Qu'est-ce qui est réel ? \\[0pt]
Qu'est-ce qui est respectable ? \\[0pt]
Qu'est-ce qui est sacré ? \\[0pt]
Qu'est-ce qui est sans raison ? \\[0pt]
Qu'est-ce qui est spectaculaire ? \\[0pt]
Qu'est-ce qui est sublime ? \\[0pt]
Qu'est-ce qui est transcendant ? \\[0pt]
Qu'est-ce qui est vital pour le vivant ? \\[0pt]
Qu'est-ce qui existe ? \\[0pt]
Qu'est-ce qui fait des sciences humaines des sciences ? \\[0pt]
Qu'est-ce qui fait la force de la loi ? \\[0pt]
Qu'est-ce qui fait la force des lois ? \\[0pt]
Qu'est-ce qui fait la justice des lois ? \\[0pt]
Qu'est-ce qui fait la légitimité d'une autorité politique ? \\[0pt]
Qu'est-ce qui fait la valeur de l'œuvre d'art ? \\[0pt]
Qu'est-ce qui fait la valeur des objets d'art ? \\[0pt]
Qu'est-ce qui fait la valeur d'une croyance ? \\[0pt]
Qu'est-ce qui fait le propre d'un corps propre ? \\[0pt]
Qu'est-ce qui fait l'humanité d'un corps ? \\[0pt]
Qu'est-ce qui fait l'unité d'une science ? \\[0pt]
Qu'est-ce qui fait l'unité d'un peuple ? \\[0pt]
Qu'est-ce qui fait obstacle à la science ? \\[0pt]
Qu'est-ce qui fait qu'un peuple est un peuple ? \\[0pt]
Qu'est-ce qui fait un peuple ? \\[0pt]
Qu'est-ce qu'ignore la science ? \\[0pt]
Qu'est-ce qui importe dans l'éducation ? \\[0pt]
Qu'est-ce qui me rend plus fort ? \\[0pt]
Qu'est-ce qui mérite l'admiration ? \\[0pt]
Qu'est-ce qui m'est étranger ? \\[0pt]
Qu'est-ce qui n'appartient pas au monde ? \\[0pt]
Qu'est-ce qui ne change pas ? \\[0pt]
Qu'est-ce qui n'est pas en mouvement ? \\[0pt]
Qu'est-ce qui n'est pas maîtrisable ? \\[0pt]
Qu'est-ce qui n'est pas mathématisable ? \\[0pt]
Qu'est-ce qui n'est pas normal ? \\[0pt]
Qu'est-ce qui n'est pas politique ? \\[0pt]
Qu'est-ce qui n'existe pas ? \\[0pt]
Qu'est-ce qui nous oblige ? \\[0pt]
Qu'est-ce qu'interpréter ? \\[0pt]
Qu'est-ce qu'interpréter une œuvre d'art ? \\[0pt]
Qu'est-ce qui peut être hors du temps ? \\[0pt]
Qu'est-ce qui rend l'objectivité difficile dans les sciences humaines ? \\[0pt]
Qu'est-ce qui rend vrai un énoncé ? \\[0pt]
Qu'est-ce qu'obéir ? \\[0pt]
Qu'est-ce qu'on attend ? \\[0pt]
Qu'est-ce qu'on ne peut pas partager ? \\[0pt]
Qu'est-ce qu'un abus de langage ? \\[0pt]
Qu'est-ce qu'un abus de pouvoir ? \\[0pt]
Qu'est-ce qu'un accident ? \\[0pt]
Qu'est-ce qu'un acte moral ? \\[0pt]
Qu'est-ce qu'un acte symbolique ? \\[0pt]
Qu'est-ce qu'un acteur ? \\[0pt]
Qu'est-ce qu'un adversaire en politique ? \\[0pt]
Qu'est-ce qu'un alter ego \\[0pt]
Qu'est-ce qu'un ami ? \\[0pt]
Qu'est-ce qu'un animal ? \\[0pt]
Qu'est-ce qu'un animal domestique ? \\[0pt]
Qu'est-ce qu'un argument ? \\[0pt]
Qu'est-ce qu'un artefact ? \\[0pt]
Qu'est-ce qu'un artiste ? \\[0pt]
Qu'est-ce qu'un art moral ? \\[0pt]
Qu'est-ce qu'un auteur ? \\[0pt]
Qu'est-ce qu'un axiome ? \\[0pt]
Qu'est-ce qu'un beau paysage ? \\[0pt]
Qu'est-ce qu'un beau travail ? \\[0pt]
Qu'est-ce qu'un bon citoyen ? \\[0pt]
Qu'est-ce qu'un bon commentaire ? \\[0pt]
Qu'est-ce qu'un bon conseil ? \\[0pt]
Qu'est-ce qu'un capital culturel ? \\[0pt]
Qu'est-ce qu'un cas de conscience ? \\[0pt]
Qu'est-ce qu'un cas en morale ? \\[0pt]
Qu'est-ce qu'un « champ artistique » ? \\[0pt]
Qu'est-ce qu'un chef ? \\[0pt]
Qu'est-ce qu'un chef-d'œuvre ? \\[0pt]
Qu'est-ce qu'un choix éclairé ? \\[0pt]
Qu'est-ce qu'un choix rationnel ? \\[0pt]
Qu'est-ce qu'un citoyen ? \\[0pt]
Qu'est-ce qu'un civilisé ? \\[0pt]
Qu'est-ce qu'un classique ? \\[0pt]
Qu'est-ce qu'un collectif ? \\[0pt]
Qu'est-ce qu'un concept ? \\[0pt]
Qu'est-ce qu'un concept scientifique ? \\[0pt]
Qu'est-ce qu'un conflit de valeurs ? \\[0pt]
Qu'est-ce qu'un conflit politique ? \\[0pt]
Qu'est-ce qu'un contenu de conscience ? \\[0pt]
Qu'est-ce qu'un contrat ? \\[0pt]
Qu'est-ce qu'un contre-pouvoir ? \\[0pt]
Qu'est-ce qu'un corps social ? \\[0pt]
Qu'est-ce qu'un coup d'État ? \\[0pt]
Qu'est-ce qu'un crime contre l'humanité ? \\[0pt]
Qu'est-ce qu'un crime politique ? \\[0pt]
Qu'est-ce qu'un déni ? \\[0pt]
Qu'est-ce qu'un dieu ? \\[0pt]
Qu'est-ce qu'un Dieu ? \\[0pt]
Qu'est-ce qu'un document ? \\[0pt]
Qu'est-ce qu'un dogme ? \\[0pt]
Qu'est-ce qu'une alternative ? \\[0pt]
Qu'est-ce qu'une âme ? \\[0pt]
Qu'est-ce qu'une aporie ? \\[0pt]
Qu'est-ce qu'une belle démonstration ? \\[0pt]
Qu'est-ce qu'une bonne loi ? \\[0pt]
Qu'est-ce qu'une bonne méthode ? \\[0pt]
Qu'est-ce qu'une bonne théorie ? \\[0pt]
Qu'est-ce qu'une catégorie ? \\[0pt]
Qu'est-ce qu'une catégorie de l'être ? \\[0pt]
Qu'est-ce qu'une cause ? \\[0pt]
Qu'est-ce qu'une chimère ? \\[0pt]
Qu'est-ce qu'une chose ? \\[0pt]
Qu'est-ce qu'une civilisation ? \\[0pt]
Qu'est-ce qu'une classe sociale ? \\[0pt]
Qu'est-ce qu'une classification scientifique ? \\[0pt]
Qu'est-ce qu'une collectivité ? \\[0pt]
Qu'est-ce qu'une communauté ? \\[0pt]
Qu'est-ce qu'une communauté humaine ? \\[0pt]
Qu'est-ce qu'une communauté politique ? \\[0pt]
Qu'est-ce qu'une communauté scientifique ? \\[0pt]
Qu'est-ce qu'une conception scientifique du monde ? \\[0pt]
Qu'est-ce qu'une condition de possibilité ? \\[0pt]
Qu'est-ce qu'une condition suffisante ? \\[0pt]
Qu'est-ce qu'une conduite irrationnelle ? \\[0pt]
Qu'est-ce qu'une connaissance a priori ? \\[0pt]
Qu'est-ce qu'une connaissance métaphysique \\[0pt]
Qu'est-ce qu'une connaissance non scientifique ? \\[0pt]
Qu'est-ce qu'une constitution ? \\[0pt]
Qu'est-ce qu'une crise ? \\[0pt]
Qu'est-ce qu'une crise politique ? \\[0pt]
Qu'est-ce qu'une croyance rationnelle ? \\[0pt]
Qu'est-ce qu'une croyance vraie ? \\[0pt]
Qu'est-ce qu'une culture ? \\[0pt]
Qu'est-ce qu'une découverte ? \\[0pt]
Qu'est-ce qu'une découverte scientifique ? \\[0pt]
Qu'est-ce qu'une discipline savante ? \\[0pt]
Qu'est-ce qu'une disposition ? \\[0pt]
Qu'est-ce qu'une école philosophique ? \\[0pt]
Qu'est-ce qu'une éducation réussie ? \\[0pt]
Qu'est-ce qu'une éducation scientifique ? \\[0pt]
Qu'est-ce qu'une époque ? \\[0pt]
Qu'est-ce qu'une erreur ? \\[0pt]
Qu'est-ce qu'une espèce naturelle ? \\[0pt]
Qu'est-ce qu'une expérience cruciale ? \\[0pt]
Qu'est-ce qu'une expérience de pensée ? \\[0pt]
Qu'est-ce qu'une expérience esthétique ? \\[0pt]
Qu'est-ce qu'une expérience politique ? \\[0pt]
Qu'est-ce qu'une exposition ? \\[0pt]
Qu'est-ce qu'une famille ? \\[0pt]
Qu'est-ce qu'une fonction ? \\[0pt]
Qu'est-ce qu'une forme ? \\[0pt]
Qu'est-ce qu'une guerre juste ? \\[0pt]
Qu'est-ce qu'une hypothèse ? \\[0pt]
Qu'est-ce qu'une hypothèse scientifique ? \\[0pt]
Qu'est-ce qu'une idée ? \\[0pt]
Qu'est-ce qu'une idée esthétique ? \\[0pt]
Qu'est-ce qu'une idée incertaine ? \\[0pt]
Qu'est-ce qu'une idée morale ? \\[0pt]
Qu'est-ce qu'une idée vraie ? \\[0pt]
Qu'est-ce qu'une idéologie ? \\[0pt]
Qu'est-ce qu'une idéologie scientifique ? \\[0pt]
Qu'est-ce qu'une image ? \\[0pt]
Qu'est-ce qu'une impression ? \\[0pt]
Qu'est-ce qu'une inférence ? \\[0pt]
Qu'est-ce qu'une injustice ? \\[0pt]
Qu'est-ce qu'une innovation technologique ? \\[0pt]
Qu'est-ce qu'une institution ? \\[0pt]
Qu'est-ce qu'une langue morte ? \\[0pt]
Qu'est-ce qu'un élément ? \\[0pt]
Qu'est-ce qu'une limite ? \\[0pt]
Qu'est-ce qu'une logique sociale ? \\[0pt]
Qu'est-ce qu'une loi ? \\[0pt]
Qu'est-ce qu'une loi d'exception ? \\[0pt]
Qu'est-ce qu'une loi injuste ? \\[0pt]
Qu'est-ce qu'une loi scientifique ? \\[0pt]
Qu'est-ce qu'une machine ? \\[0pt]
Qu'est-ce qu'une machine ne peut pas faire ? \\[0pt]
Qu'est-ce qu'une marchandise ? \\[0pt]
Qu'est-ce qu'une mauvaise interprétation ? \\[0pt]
Qu'est-ce qu'une méditation ? \\[0pt]
Qu'est-ce qu'une méditation métaphysique ? \\[0pt]
Qu'est-ce qu'une mentalité collective ? \\[0pt]
Qu'est-ce qu'une méthode ? \\[0pt]
Qu'est-ce qu'une minorité ? \\[0pt]
Qu'est-ce qu'un empire ? \\[0pt]
Qu'est-ce qu'une nation ? \\[0pt]
Qu'est-ce qu'un enfant ? \\[0pt]
Qu'est-ce qu'un énoncé scientifique sur l'homme ? \\[0pt]
Qu'est-ce qu'une norme ? \\[0pt]
Qu'est-ce qu'une norme sociale ? \\[0pt]
Qu'est-ce qu'une œuvre ? \\[0pt]
Qu'est-ce qu'une œuvre classique ? \\[0pt]
Qu'est-ce qu'une œuvre d'art ? \\[0pt]
Qu'est-ce qu'une œuvre d'art authentique ? \\[0pt]
Qu'est-ce qu'une œuvre d'art réussie ? \\[0pt]
Qu'est-ce qu'une œuvre « géniale » ? \\[0pt]
Qu'est-ce qu'une œuvre ratée ? \\[0pt]
Qu'est-ce qu'une parole vivante ? \\[0pt]
Qu'est-ce qu'une peinture abstraite ? \\[0pt]
Qu'est-ce qu'une « performance » ? \\[0pt]
Qu'est-ce qu'une période en histoire ? \\[0pt]
Qu'est-ce qu'une philosophie première ? \\[0pt]
Qu'est-ce qu'une phrase ? \\[0pt]
Qu'est-ce qu'une politique sociale ? \\[0pt]
Qu'est-ce qu'une preuve ? \\[0pt]
Qu'est-ce qu'une promesse ? \\[0pt]
Qu'est-ce qu'une propriété ? \\[0pt]
Qu'est-ce qu'une propriété essentielle ? \\[0pt]
Qu'est-ce qu'une psychologie scientifique ? \\[0pt]
Qu'est-ce qu'une question dénuée de sens ? \\[0pt]
Qu'est-ce qu'une question métaphysique ? \\[0pt]
Qu'est-ce qu'une réfutation ? \\[0pt]
Qu'est-ce qu'une règle de grammaire ? \\[0pt]
Qu'est-ce qu'une règle de vie ? \\[0pt]
Qu'est-ce qu'une règle sociale ? \\[0pt]
Qu'est-ce qu'une religion ? \\[0pt]
Qu'est-ce qu'une représentation réussie ? \\[0pt]
Qu'est-ce qu'une révolution ? \\[0pt]
Qu'est-ce qu'une révolution artistique ? \\[0pt]
Qu'est-ce qu'une révolution scientifique ? \\[0pt]
Qu'est-ce qu'une science exacte ? \\[0pt]
Qu'est-ce qu'une science rigoureuse ? \\[0pt]
Qu'est-ce qu'une secte ? \\[0pt]
Qu'est-ce qu'une situation ? \\[0pt]
Qu'est-ce qu'une situation tragique ? \\[0pt]
Qu'est-ce qu'une société ? \\[0pt]
Qu'est-ce qu'une société juste ? \\[0pt]
Qu'est-ce qu'une société mondialisée ? \\[0pt]
Qu'est-ce qu'une société primitive ? \\[0pt]
Qu'est-ce qu'un esprit faux ? \\[0pt]
Qu'est-ce qu'une structure ? \\[0pt]
Qu'est-ce qu'une substance ? \\[0pt]
Qu'est-ce qu'un état mental ? \\[0pt]
Qu'est-ce qu'un État souverain ? \\[0pt]
Qu'est-ce qu'une théorie scientifique ? \\[0pt]
Qu'est-ce qu'une tradition ? \\[0pt]
Qu'est-ce qu'une tragédie historique ? \\[0pt]
Qu'est-ce qu'un être cultivé ? \\[0pt]
Qu'est-ce qu'un être imaginaire ? \\[0pt]
Qu'est-ce qu'une valeur ? \\[0pt]
Qu'est-ce qu'une valeur esthétique ? \\[0pt]
Qu'est-ce qu'un événement ? \\[0pt]
Qu'est-ce qu'un événement historique ? \\[0pt]
Qu'est-ce qu'une vérité scientifique ? \\[0pt]
Qu'est-ce qu'une vie humaine ? \\[0pt]
Qu'est-ce qu'une vie réussie ? \\[0pt]
Qu'est-ce qu'une ville ? \\[0pt]
Qu'est-ce qu'une violence symbolique ? \\[0pt]
Qu'est-ce qu'une vision du monde ? \\[0pt]
Qu'est-ce qu'une vision scientifique du monde ? \\[0pt]
Qu'est-ce qu'une volonté libre ? \\[0pt]
Qu'est-ce qu'un exemple ? \\[0pt]
Qu'est-ce qu'un expert ? \\[0pt]
Qu'est-ce qu'un fait ? \\[0pt]
Qu'est-ce qu'un fait de société ? \\[0pt]
Qu'est-ce qu'un fait divers ? \\[0pt]
Qu'est-ce qu'un fait historique ? \\[0pt]
Qu'est-ce qu'un fait moral ? \\[0pt]
Qu'est-ce qu'un fait scientifique ? \\[0pt]
Qu'est-ce qu'un fait social ? \\[0pt]
Qu'est-ce qu'un faux ? \\[0pt]
Qu'est-ce qu'un faux problème ? \\[0pt]
Qu'est-ce qu'un faux sentiment ? \\[0pt]
Qu'est-ce qu'un film ? \\[0pt]
Qu'est-ce qu'un geste artistique ? \\[0pt]
Qu'est-ce qu'un geste technique ? \\[0pt]
Qu'est-ce qu'un gouvernement ? \\[0pt]
Qu'est-ce qu'un grand philosophe ? \\[0pt]
Qu'est-ce qu'un héros ? \\[0pt]
Qu'est-ce qu'un homme bon ? \\[0pt]
Qu'est-ce qu'un homme juste ? \\[0pt]
Qu'est-ce qu'un homme politique ? \\[0pt]
Qu'est-ce qu'un homme seul ? \\[0pt]
Qu'est-ce qu'un idéal moral ? \\[0pt]
Qu'est-ce qu'un imaginaire social ? \\[0pt]
Qu'est-ce qu'un individu ? \\[0pt]
Qu'est-ce qu'un jeu ? \\[0pt]
Qu'est-ce qu'un juge ? \\[0pt]
Qu'est-ce qu'un jugement analytique ? \\[0pt]
Qu'est-ce qu'un laboratoire ? \\[0pt]
Qu'est-ce qu'un législateur ? \\[0pt]
Qu'est-ce qu'un lien logique ? \\[0pt]
Qu'est-ce qu'un lieu commun ? \\[0pt]
Qu'est-ce qu'un livre ? \\[0pt]
Qu'est-ce qu'un maître ? \\[0pt]
Qu'est-ce qu'un marginal ? \\[0pt]
Qu'est-ce qu'un mécanisme social ? \\[0pt]
Qu'est-ce qu'un métaphysicien ? \\[0pt]
Qu'est-ce qu'un modèle ? \\[0pt]
Qu'est-ce qu'un moderne ? \\[0pt]
Qu'est-ce qu'un monde ? \\[0pt]
Qu'est-ce qu'un monde commun ? \\[0pt]
Qu'est-ce qu'un monstre ? \\[0pt]
Qu'est-ce qu'un monument ? \\[0pt]
Qu'est-ce qu'un mouvement politique \\[0pt]
Qu'est-ce qu'un mouvement politique ? \\[0pt]
Qu'est-ce qu'un mythe ? \\[0pt]
Qu'est-ce qu'un nombre ? \\[0pt]
Qu'est-ce qu'un nom propre ? \\[0pt]
Qu'est-ce qu'un objet ? \\[0pt]
Qu'est-ce qu'un objet d'art ? \\[0pt]
Qu'est-ce qu'un objet de connaissance ? \\[0pt]
Qu'est-ce qu'un objet esthétique ? \\[0pt]
Qu'est-ce qu'un objet métaphysique ? \\[0pt]
Qu'est-ce qu'un organisme ? \\[0pt]
Qu'est-ce qu'un original ? \\[0pt]
Qu'est-ce qu'un outil ? \\[0pt]
Qu'est-ce qu'un paradoxe ? \\[0pt]
Qu'est-ce qu'un patrimoine ? \\[0pt]
Qu'est-ce qu'un paysage ? \\[0pt]
Qu'est-ce qu'un pédant ? \\[0pt]
Qu'est-ce qu'un peuple \\[0pt]
Qu'est-ce qu'un peuple ? \\[0pt]
Qu'est-ce qu'un peuple libre ? \\[0pt]
Qu'est-ce qu'un phénomène ? \\[0pt]
Qu'est-ce qu'un plaisir pur ? \\[0pt]
Qu'est-ce qu'un point de vue ? \\[0pt]
Qu'est-ce qu'un postulat ? \\[0pt]
Qu'est-ce qu'un pouvoir despotique ? \\[0pt]
Qu'est-ce qu'un pouvoir légitime ? \\[0pt]
Qu'est-ce qu'un primitif ? \\[0pt]
Qu'est-ce qu'un prince juste ? \\[0pt]
Qu'est-ce qu'un principe ? \\[0pt]
Qu'est-ce qu'un problème ? \\[0pt]
Qu'est-ce qu'un problème éthique ? \\[0pt]
Qu'est-ce qu'un problème métaphysique ? \\[0pt]
Qu'est-ce qu'un problème philosophique ? \\[0pt]
Qu'est-ce qu'un problème politique ? \\[0pt]
Qu'est-ce qu'un problème scientifique ? \\[0pt]
Qu'est-ce qu'un produit culturel ? \\[0pt]
Qu'est-ce qu'un programme politique ? \\[0pt]
Qu'est-ce qu'un projet ? \\[0pt]
Qu'est-ce qu'un rapport de force ? \\[0pt]
Qu'est-ce qu'un réfugié politique ? \\[0pt]
Qu'est-ce qu'un rite ? \\[0pt]
Qu'est-ce qu'un sage ? \\[0pt]
Qu'est-ce qu'un sentiment moral ? \\[0pt]
Qu'est-ce qu'un signe ? \\[0pt]
Qu'est-ce qu'un sophisme ? \\[0pt]
Qu'est-ce qu'un sophiste ? \\[0pt]
Qu'est-ce qu'un souvenir ? \\[0pt]
Qu'est-ce qu'un spécialiste ? \\[0pt]
Qu'est-ce qu'un spectacle ? \\[0pt]
Qu'est-ce qu'un spectateur ? \\[0pt]
Qu'est-ce qu'un style ? \\[0pt]
Qu'est-ce qu'un sujet de droit ? \\[0pt]
Qu'est-ce qu'un symbole ? \\[0pt]
Qu'est-ce qu'un symptôme ? \\[0pt]
Qu'est-ce qu'un système ? \\[0pt]
Qu'est-ce qu'un système philosophique ? \\[0pt]
Qu'est-ce qu'un tableau \\[0pt]
Qu'est-ce qu'un tableau ? \\[0pt]
Qu'est-ce qu'un témoin ? \\[0pt]
Qu'est-ce qu'un territoire ? \\[0pt]
Qu'est-ce qu'un texte ? \\[0pt]
Qu'est-ce qu'un tout ? \\[0pt]
Qu'est-ce qu'un trouble social ? \\[0pt]
Qu'est-il légitime d'interdire ? \\[0pt]
Qu'est-il utile d'interdire ? \\[0pt]
Question et problème \\[0pt]
Qu'est qu'un régime politique ? \\[0pt]
Que suppose le mouvement ? \\[0pt]
Que transmettons-nous ? \\[0pt]
Que trouve-t-on dans ce que l'on trouve beau ? \\[0pt]
Que valent les classifications en sciences humaines ? \\[0pt]
Que valent les décisions de la majorité ? \\[0pt]
Que valent les excuses ? \\[0pt]
Que valent les idées générales ? \\[0pt]
Que valent les preuves ? \\[0pt]
Que vaut en morale la justification par l'utilité ? \\[0pt]
Que vaut la distinction entre nature et culture ? \\[0pt]
Que vaut la fidélité ? \\[0pt]
Que vaut l'excuse : « C'est plus fort que moi » ? \\[0pt]
Que vaut l'incertain ? \\[0pt]
Que vaut un argument d'autorité ? \\[0pt]
Que vaut un consensus ? \\[0pt]
Que vaut une connaissance empirique de l'homme ? \\[0pt]
Que vaut une preuve contre un préjugé ? \\[0pt]
Que veut dire avoir raison ? \\[0pt]
Que veut dire introduire à la métaphysique ? \\[0pt]
Que veut dire l'expression « aller au fond des choses » ? \\[0pt]
Que veut-on dire quand on dit que « rien n'est sans raison » ? \\[0pt]
Que veut-on dire quand on dit « rien n'est sans raison » ? \\[0pt]
Que voit-on dans une image ? \\[0pt]
Que voit-on dans un miroir ? \\[0pt]
Que voulons-nous ? \\[0pt]
Que voyons-nous sur un tableau ? \\[0pt]
Qu'expriment les œuvres d'art ? \\[0pt]
Qu'exprime une œuvre d'art ? \\[0pt]
Qui a des droits ? \\[0pt]
Qui agit ? \\[0pt]
Qui a le droit de juger ? \\[0pt]
Qui a une histoire ? \\[0pt]
Qui a une parole politique ? \\[0pt]
Qui commande ? \\[0pt]
Qui connaît le mieux mon corps ? \\[0pt]
Qui croire ? \\[0pt]
Qui détient le pouvoir ? \\[0pt]
Qui doit dire le droit ? \\[0pt]
Qui doit éduquer ? \\[0pt]
Qui doit faire les lois ? \\[0pt]
Qui doit gouverner ? \\[0pt]
Qui domine ? \\[0pt]
Qui donne la norme du goût ? \\[0pt]
Qui est compétent en matière politique ? \\[0pt]
Qui est crédible ? \\[0pt]
Qui est cultivé ? \\[0pt]
Qui est le maître ? \\[0pt]
Qui est l'homme des sciences humaines ? \\[0pt]
Qui est libre ? \\[0pt]
Qui est méchant ? \\[0pt]
Qui est métaphysicien ? \\[0pt]
Qui est mon prochain ? \\[0pt]
Qui est responsable ? \\[0pt]
Qui est sage ? \\[0pt]
Qui est souverain ? \\[0pt]
Qui est une personne ? \\[0pt]
Qui fait la loi ? \\[0pt]
Qui faut-il protéger ? \\[0pt]
Qui gouverne ? \\[0pt]
Qui mérite d'être aimé ? \\[0pt]
Qui meurt ? \\[0pt]
« Qui ne dit mot consent » \\[0pt]
Qui parle ? \\[0pt]
Qui pense ? \\[0pt]
Qui peut obliger ? \\[0pt]
Qui peut parler ? \\[0pt]
Qui pose les fins ? \\[0pt]
Qui sont mes amis ? \\[0pt]
Qui sont mes semblables ? \\[0pt]
Qui sont nos ennemis ? \\[0pt]
Qui suis-je ? \\[0pt]
Qui suis-je pour me juger ? \\[0pt]
Qu'y a-t-il à comprendre dans une œuvre d'art ? \\[0pt]
Qu'y a-t-il à comprendre en histoire ? \\[0pt]
Qu'y a-t-il à l'origine de toutes choses ? \\[0pt]
Qu'y a-t-il au-delà de l'être ? \\[0pt]
Qu'y a-t-il au-dessus des lois ? \\[0pt]
Qu'y a-t-il au fondement de l'objectivité ? \\[0pt]
Qu'y a-t-il de technique dans l'art ? \\[0pt]
Race, classe, nation \\[0pt]
Raconter sa vie \\[0pt]
Raconter son histoire \\[0pt]
Raison et déraison \\[0pt]
Raison et politique \\[0pt]
Raison et révélation \\[0pt]
Raisonner et argumenter \\[0pt]
Raisonner et calculer \\[0pt]
Raisonner et décider \\[0pt]
Rapports de force, rapport de pouvoir \\[0pt]
Rapports de la science et de la politique \\[0pt]
Rassembler les hommes, est-ce les unir ? \\[0pt]
Réalisme et idéalisme \\[0pt]
Réalité et idéal \\[0pt]
Rebuts et objets quelconques : une matière pour l'art ? \\[0pt]
Recevoir \\[0pt]
Recherche et militantisme \\[0pt]
Récit et explication en histoire \\[0pt]
Récit et fiction \\[0pt]
Récit et mémoire \\[0pt]
Reconnaissons-nous le bien comme nous reconnaissons le vrai ? \\[0pt]
Réécrire l'histoire \\[0pt]
Refaire le monde \\[0pt]
Réforme et révolution \\[0pt]
Réfutation et confirmation \\[0pt]
Réfuter \\[0pt]
Regarder \\[0pt]
Regarder une œuvre d'art \\[0pt]
Regarder un tableau \\[0pt]
Règle et commandement \\[0pt]
Règles logiques et lois psychologiques \\[0pt]
Régularité et singularité \\[0pt]
Relativisme et sciences humaines \\[0pt]
Religion et métaphysique \\[0pt]
Religion et superstition \\[0pt]
Religion naturelle et religion révélée \\[0pt]
Rendre justice \\[0pt]
Rendre la justice \\[0pt]
Rendre raison \\[0pt]
Rendre visible l'invisible \\[0pt]
Renoncer à ses préjugés \\[0pt]
Renoncer au passé \\[0pt]
Rentrer en soi-même \\[0pt]
Réparer \\[0pt]
Répondre \\[0pt]
Représentation et illusion \\[0pt]
Reproduire \\[0pt]
Reproduire, copier, imiter \\[0pt]
Réprouver \\[0pt]
République et démocratie \\[0pt]
Résistance et soumission \\[0pt]
Résister \\[0pt]
Résister à l'oppression \\[0pt]
Résister peut-il être un droit ? \\[0pt]
Respect de la volonté, respect de la vie \\[0pt]
Respecter la nature, est-ce renoncer à l'exploiter ? \\[0pt]
Ressemblance et identité \\[0pt]
Ressemblance et imitation \\[0pt]
Ressembler \\[0pt]
Ressentir \\[0pt]
Ressent-on ou apprécie-t-on l'art ? \\[0pt]
Rêve et réalité \\[0pt]
Révéler \\[0pt]
Rêver \\[0pt]
Rêver éloigne-t-il de la réalité ? \\[0pt]
Revient-il à l'État d'assurer votre bonheur ? \\[0pt]
Révolte et révolution \\[0pt]
Rhétorique et vérité \\[0pt]
Richesse et pauvreté \\[0pt]
Rien \\[0pt]
Rien de nouveau sous le soleil \\[0pt]
« Rien n'est sans raison » \\[0pt]
Rien n'est sans raison \\[0pt]
« Rien n'est simple » \\[0pt]
Rire \\[0pt]
Rire et pleurer \\[0pt]
Rites et cérémonies \\[0pt]
Roman et vérité \\[0pt]
Rythmes sociaux, rythmes naturels \\[0pt]
S'adapter \\[0pt]
Sagesse et renoncement \\[0pt]
Sait-on toujours ce que l'on fait ? \\[0pt]
Sait-on toujours ce qu'on veut ? \\[0pt]
S'aliéner \\[0pt]
Sans foi ni loi \\[0pt]
Sans les mots, que seraient les choses ? \\[0pt]
« Sans titre » \\[0pt]
S'approprier une œuvre d'art \\[0pt]
S'associer \\[0pt]
Sauver les apparences \\[0pt]
« Sauver les phénomènes » \\[0pt]
Sauver les phénomènes \\[0pt]
Savoir ce qu'on dit \\[0pt]
Savoir ce qu'on fait \\[0pt]
Savoir ce qu'on veut \\[0pt]
Savoir de quoi on parle \\[0pt]
Savoir, est-ce déjouer le réel ? \\[0pt]
Savoir, est-ce pouvoir ? \\[0pt]
Savoir et liberté \\[0pt]
Savoir et objectivité dans les sciences \\[0pt]
Savoir et pouvoir \\[0pt]
Savoir et rectification \\[0pt]
Savoir être heureux \\[0pt]
Savoir et vérifier \\[0pt]
Savoir faire \\[0pt]
Savoir pour prévoir \\[0pt]
Savoir, pouvoir \\[0pt]
Savoir s'arrêter \\[0pt]
Savoir se décider \\[0pt]
Savoir vivre \\[0pt]
Savons-nous ce que nous disons ? \\[0pt]
Savons-nous ce que peut un corps ? \\[0pt]
Savons-nous de ce que nous disons ? \\[0pt]
Savons-nous nous souvenir ? \\[0pt]
Scepticisme et relativisme \\[0pt]
Science et complexité \\[0pt]
Science et conscience \\[0pt]
Science et démocratie \\[0pt]
Science et histoire \\[0pt]
Science et hypothèse \\[0pt]
Science et idéologie \\[0pt]
Science et imagination \\[0pt]
Science et magie \\[0pt]
Science et mythe \\[0pt]
Science et opinion \\[0pt]
Science et persuasion \\[0pt]
Science et philosophie \\[0pt]
Science et pouvoir \\[0pt]
Science et réalité \\[0pt]
Science et religion \\[0pt]
Science et sagesse \\[0pt]
Science et société \\[0pt]
Science et subjectivité \\[0pt]
Science et technique \\[0pt]
Science et technologie \\[0pt]
Science pure et science appliquée \\[0pt]
Sciences de la nature et sciences de l'esprit \\[0pt]
Sciences de la nature et sciences humaines \\[0pt]
Sciences et philosophie \\[0pt]
Sciences humaines et déterminisme \\[0pt]
Sciences humaines et herméneutique \\[0pt]
Sciences humaines et idéologie \\[0pt]
Sciences humaines et liberté sont-elles compatibles ? \\[0pt]
Sciences humaines et littérature \\[0pt]
Sciences humaines et naturalisme \\[0pt]
Sciences humaines et nature humaine \\[0pt]
Sciences humaines et objectivité \\[0pt]
Sciences humaines et philosophie \\[0pt]
Sciences humaines et vérité \\[0pt]
Sciences humaines ou sciences sociales ? \\[0pt]
Sciences humaines, sciences de l'homme \\[0pt]
Sciences sociales et humanités \\[0pt]
Sciences sociales et questions environnementales \\[0pt]
Sculpter \\[0pt]
Se conserver \\[0pt]
Sécularisation et laïcisation \\[0pt]
Se cultiver \\[0pt]
Sécurité et liberté \\[0pt]
Se défendre \\[0pt]
Se détacher des sens \\[0pt]
Se distinguer \\[0pt]
Se divertir \\[0pt]
Se donner corps et âme \\[0pt]
Séduire \\[0pt]
Se faire justice \\[0pt]
Se méfier des idées \\[0pt]
Se mentir à soi-même \\[0pt]
Se mettre à la place d'autrui \\[0pt]
S'ennuyer \\[0pt]
Sensation et perception \\[0pt]
Sens et fait \\[0pt]
Sens et limites de la notion de capital culturel \\[0pt]
Sens et sensibilité \\[0pt]
Sens et sensible \\[0pt]
Sens et structure \\[0pt]
Sensible et intelligible \\[0pt]
S'entendre \\[0pt]
Sentiment et émotion \\[0pt]
Sentir \\[0pt]
Se parler et s'entendre \\[0pt]
Se passer de philosophie \\[0pt]
Se peindre \\[0pt]
Se prendre au sérieux \\[0pt]
Se raisonner \\[0pt]
Se réfugier dans la croyance ? \\[0pt]
Se retirer dans la pensée ? \\[0pt]
Se retirer du monde \\[0pt]
Se révolter \\[0pt]
Servir \\[0pt]
Servir l'État \\[0pt]
Se savoir mortel \\[0pt]
Se souvenir \\[0pt]
Se souvenir, est-ce préparer l'avenir ? \\[0pt]
Se taire \\[0pt]
S'étonner \\[0pt]
Seul le présent existe-t-il ? \\[0pt]
Se voiler la face \\[0pt]
Sexe et genre \\[0pt]
S'exercer \\[0pt]
S'exprimer \\[0pt]
Sexualité et nature \\[0pt]
Si Dieu n'existe pas, tout est-il permis ? \\[0pt]
Signes naturels, signes artificiels \\[0pt]
Signes, traces et indices \\[0pt]
Signification et expression \\[0pt]
Signification et vérité \\[0pt]
Si l'esprit n'est pas une table rase, qu'est-il ? \\[0pt]
Simple / composé \\[0pt]
Sincérité et authenticité \\[0pt]
S'indigner \\[0pt]
S'indigner, est-ce un devoir ? \\[0pt]
S'intéresser à l'art \\[0pt]
« Si tu veux la paix, prépare la guerre » \\[0pt]
Si tu veux, tu peux \\[0pt]
Société et communauté \\[0pt]
Société et liberté \\[0pt]
Société et mouvement \\[0pt]
Société et organisme \\[0pt]
Société et réciprocité \\[0pt]
Sociologie et économie \\[0pt]
Sociologie et ethnologie \\[0pt]
Soi \\[0pt]
Solitude et liberté \\[0pt]
Sommes-nous capables d'agir de manière désintéressée ? \\[0pt]
Sommes-nous des êtres métaphysiques ? \\[0pt]
Sommes-nous libres de nos croyances ? \\[0pt]
Sommes-nous libres de nos pensées ? \\[0pt]
Sommes-nous libres de nos préférences morales ? \\[0pt]
Sommes-nous libres par nature ? \\[0pt]
Sommes-nous maîtres de la nature ? \\[0pt]
Sommes-nous responsables de ce que nous sommes ? \\[0pt]
Sommes-nous responsables de nos erreurs ? \\[0pt]
Sommes-nous responsables de notre être social ? \\[0pt]
Sommes-nous responsables du sens que prennent nos paroles ? \\[0pt]
Sommes-nous toujours dépendants d'autrui ? \\[0pt]
Sommes-nous tous artistes ? \\[0pt]
Sommes-nous tous contemporains ? \\[0pt]
Sophisme et paralogisme \\[0pt]
Sophismes et paradoxes \\[0pt]
S'orienter \\[0pt]
S'orienter dans la pensée \\[0pt]
S'orienter dans le monde \\[0pt]
Sortir de soi \\[0pt]
Soupçonner \\[0pt]
Soutenir une thèse \\[0pt]
Souveraineté et liberté \\[0pt]
Sphère privée et sphère publique \\[0pt]
Spontanéité et liberté \\[0pt]
Structure et événement \\[0pt]
Structure et histoire \\[0pt]
Subir \\[0pt]
Subjectivité et vérité \\[0pt]
Substance et sujet \\[0pt]
Suffit-il à Dieu d'être possible pour exister ? \\[0pt]
Suffit-il de bien juger pour bien faire ? \\[0pt]
Suffit-il d'être conscient de ses actes pour en être responsable ? \\[0pt]
Suffit-il d'être dans le vrai pour savoir ? \\[0pt]
Suffit-il d'être juste ? \\[0pt]
Suffit-il d'être raisonnable pour être moral ? \\[0pt]
Suffit-il de vouloir pour bien faire ? \\[0pt]
Suffit-il, pour être juste, d'obéir aux lois et aux coutumes de son pays ? \\[0pt]
Suicide et homicide \\[0pt]
Suis-je au centre de l'espace ? \\[0pt]
Suis-je aussi ce que j'aurais pu être ? \\[0pt]
Suis-je l'auteur de mes actions ? \\[0pt]
Suis-je le sujet de mes pensées ? \\[0pt]
Suis-je maître de mes pensées ? \\[0pt]
Suis-je ma mémoire ? \\[0pt]
Suis-je mon corps ? \\[0pt]
Suivre la nature \\[0pt]
Suivre la règle \\[0pt]
Suivre le plus sûr \\[0pt]
Suivre sa nature \\[0pt]
Sujet et citoyen \\[0pt]
Sujet et prédicat \\[0pt]
Sur quels fondements distinguer ce qui est public et ce qui est privé ? \\[0pt]
Sur quoi fonder la propriété ? \\[0pt]
Sur quoi fonder l'autorité ? \\[0pt]
Sur quoi reposent nos certitudes ? \\[0pt]
Sur quoi se fonde la connaissance scientifique ? \\[0pt]
Sur quoi se fonde la vérité des énoncés ? \\[0pt]
Surveillance et discipline \\[0pt]
Surveiller son comportement \\[0pt]
Survivre \\[0pt]
Suspendre son assentiment \\[0pt]
Suspendre son jugement \\[0pt]
Syllogisme et démonstration \\[0pt]
Système et liberté \\[0pt]
Tantôt je pense, tantôt je suis \\[0pt]
Tautologie et contradiction \\[0pt]
Technique et démocratie \\[0pt]
Technique et esthétique \\[0pt]
Technique et pratiques scientifiques \\[0pt]
Technique et technologie \\[0pt]
Témoigner \\[0pt]
Temps et durée \\[0pt]
Temps et éternité \\[0pt]
Temps et musique \\[0pt]
Temps et réalité \\[0pt]
Tenir parole \\[0pt]
Terres et territoires \\[0pt]
Territoire et peuplement \\[0pt]
Tester une hypothèse permet-il de la prouver ? \\[0pt]
Texte, son, image \\[0pt]
Thème et variations \\[0pt]
Théologie et morale \\[0pt]
Théorie et modélisation \\[0pt]
Théorie et pratique \\[0pt]
Théorie et pratique en médecine \\[0pt]
Tolérer \\[0pt]
Tomber amoureux \\[0pt]
Toucher \\[0pt]
Tous les arts se valent-ils ? \\[0pt]
Tous les droits sont-ils formels ? \\[0pt]
Tous les hommes désirent-ils connaître ? \\[0pt]
Tous les hommes désirent-ils être heureux ? \\[0pt]
Tout art est-il abstrait ? \\[0pt]
Tout art est-il décoratif ? \\[0pt]
Tout art est-il poésie ? \\[0pt]
Tout art est-il symbolique ? \\[0pt]
Tout a-t-il une fin ? \\[0pt]
Tout a-t-il une raison d'être ? \\[0pt]
Tout a-t-il un sens ? \\[0pt]
Tout bien n'est-il qu'un moindre mal ? \\[0pt]
Tout ce qui est humain est-il signifiant ? \\[0pt]
Tout ce qui est humain nous est-il familier ? \\[0pt]
Tout ce qui existe est-il matériel ? \\[0pt]
Tout définir, tout démontrer \\[0pt]
Tout désir est-il nostalgique ? \\[0pt]
Tout devoir est-il l'envers d'un droit ? \\[0pt]
Toute action politique est-elle collective ? \\[0pt]
Toute chose a-t-elle une essence ? \\[0pt]
Toute chose a-t-elle une fin ? \\[0pt]
Toute communauté est-elle politique ? \\[0pt]
Toute connaissance autre que scientifique doit-elle être considérée comme une illusion ? \\[0pt]
Toute connaissance est-elle historique ? \\[0pt]
Toute connaissance est-elle relative ? \\[0pt]
Toute conversation est-elle futile ? \\[0pt]
Toute définition est-elle une convention ? \\[0pt]
Toute description est-elle une interprétation ? \\[0pt]
Toute expression est-elle métaphorique ? \\[0pt]
Toute hiérarchie est-elle inégalitaire ? \\[0pt]
Toute inégalité est-elle injuste ? \\[0pt]
Toute loi est-elle arbitraire ? \\[0pt]
Toute métaphysique implique-t-elle une transcendance ? \\[0pt]
Toute morale est-elle rationnelle ? \\[0pt]
Toute morale implique-t-elle l'effort ? \\[0pt]
Tout énoncé est-il nécessairement vrai ou faux ? \\[0pt]
Toute origine est-elle mythique ? \\[0pt]
« Toute peine mérite salaire » \\[0pt]
Toute peur est-elle irrationnelle ? \\[0pt]
Toute philosophie est-elle systématique ? \\[0pt]
Toute philosophie implique-t-elle une politique ? \\[0pt]
Toute physique exige-t-elle une métaphysique ? \\[0pt]
Toute psychologie est-elle normalisatrice ? \\[0pt]
Toutes les choses sont-elles singulières ? \\[0pt]
Toutes les fautes se valent-elles ? \\[0pt]
Toutes les vérités scientifiques sont-elles révisables ? \\[0pt]
Tout est corps \\[0pt]
Tout est-il à vendre ? \\[0pt]
Tout est-il connaissable ? \\[0pt]
Tout est-il contingent ? \\[0pt]
Tout est-il digne de mémoire ? \\[0pt]
Tout est-il mesurable ? \\[0pt]
Tout est-il nécessaire ? \\[0pt]
Tout est-il politique ? \\[0pt]
Tout est-il relatif ? \\[0pt]
Tout est permis \\[0pt]
« Tout est politique » \\[0pt]
Tout être est-il dans l'espace ? \\[0pt]
Tout événement a-t-il une cause ? \\[0pt]
Toute vérité est-elle démontrable ? \\[0pt]
Toute vérité est-elle un rapport ? \\[0pt]
Toute violence est-elle contre nature ? \\[0pt]
Tout fondement de la connaissance est-il métaphysique ? \\[0pt]
Tout malheur est-il une injustice ? \\[0pt]
Tout ou rien \\[0pt]
Tout peut-il être expliqué historiquement ? \\[0pt]
Tout peut-il être objet de jugement esthétique ? \\[0pt]
Tout peut-il n'être qu'apparence ? \\[0pt]
Tout peut-il se quantifier ? \\[0pt]
Tout pouvoir est-il oppresseur ? \\[0pt]
Tout pouvoir est-il politique ? \\[0pt]
Tout pouvoir est-il usurpé ? \\[0pt]
Tout pouvoir n'est-il pas abusif ? \\[0pt]
Tout pouvoir relève-t-il de la domination ? \\[0pt]
Tout savoir \\[0pt]
Tout savoir est-il fondé sur un savoir premier ? \\[0pt]
Tout savoir est-il transmissible ? \\[0pt]
Tradition et innovation \\[0pt]
Tradition et raison \\[0pt]
Traduire \\[0pt]
Tragédie et comédie \\[0pt]
Trahir \\[0pt]
Traiter autrui comme une chose \\[0pt]
Transcendance et altérité \\[0pt]
Transmettre \\[0pt]
Travail et subjectivité \\[0pt]
Travail manuel, travail intellectuel \\[0pt]
Tricher \\[0pt]
« Trop beau pour être vrai » \\[0pt]
Tuer le temps \\[0pt]
« Tu ne tueras point » \\[0pt]
Un acte désintéressé est-il possible ? \\[0pt]
Un artiste doit-il être génial ? \\[0pt]
Un art peut-il se passer de règles ? \\[0pt]
Un art sans sublimation est-il possible ? \\[0pt]
Un bien peut-il sortir d'un mal ? \\[0pt]
Un corps n'est-il qu'un objet ? \\[0pt]
Un Dieu unique ? \\[0pt]
Une action se juge-t-elle à ses conséquences ? \\[0pt]
Une action vertueuse se reconnaît-elle à sa difficulté ? \\[0pt]
Une bonne cité peut-elle se passer du beau ? \\[0pt]
Une cause peut-elle être libre ? \\[0pt]
Une culture constitue-t-elle un paradigme ? \\[0pt]
Une culture de masse est-elle une culture ? \\[0pt]
Une décision politique peut-elle être juste ? \\[0pt]
Une d'œuvre peut-elle être achevée ? \\[0pt]
Une éducation des sentiments est-elle possible ? \\[0pt]
Une éducation esthétique est-elle possible ? \\[0pt]
Une éthique sceptique est-elle possible ? \\[0pt]
Une explication peut-elle être réductrice ? \\[0pt]
Une femme politique est-elle un homme politique comme les autres ? \\[0pt]
Une fiction peut-elle être vraie ? \\[0pt]
Une guerre peut-elle être juste ? \\[0pt]
Une idée peut-elle être fausse ? \\[0pt]
Une imitation peut-elle être parfaite ? \\[0pt]
Une inégalité peut-elle être juste ? \\[0pt]
Une interprétation peut-elle être vraie ? \\[0pt]
Une justice sans égalité est-elle possible ? \\[0pt]
Une ligne de conduite peut-elle tenir lieu de morale ? \\[0pt]
Une logique non-formelle est-elle possible ? \\[0pt]
Une loi n'est-elle qu'une règle ? \\[0pt]
Une machine peut-elle être naturelle ? \\[0pt]
Une machine peut-elle penser ? \\[0pt]
Une machine pourrait-elle penser ? \\[0pt]
Une machine se réduit-elle à un mécanisme ? \\[0pt]
« Une main de fer dans un gant de velours » \\[0pt]
Une métaphysique athée est-elle possible ? \\[0pt]
Une métaphysique n'est-elle qu'une ontologie ? \\[0pt]
Une métaphysique peut-elle être sceptique ? \\[0pt]
Une morale du plaisir est-elle concevable ? \\[0pt]
Une morale personnelle est-elle une contradiction dans les termes ? \\[0pt]
Une morale peut-elle être dépassée ? \\[0pt]
Une morale peut-elle être provisoire ? \\[0pt]
Une morale peut-elle prétendre à l'universalité ? \\[0pt]
Une morale peut-elle se passer de la notion de vertu ? \\[0pt]
Une morale qui ne vise pas l'utilité est-elle rationnelle ? \\[0pt]
Une morale sans Dieu \\[0pt]
Une morale sans obligation est-elle possible ? \\[0pt]
Un enfant est-il un citoyen ? \\[0pt]
Une œuvre d'art a-t-elle un prix ? \\[0pt]
Une œuvre d'art doit-elle avoir un sens ? \\[0pt]
Une œuvre d'art est-elle immortelle ? \\[0pt]
Une œuvre d'art est-elle toujours originale ? \\[0pt]
Une œuvre d'art est-elle une marchandise ? \\[0pt]
Une œuvre d'art peut-elle être exemplaire ? \\[0pt]
Une œuvre d'art peut-elle être laide ? \\[0pt]
Une œuvre d'art s'explique-t-elle à partir de ses influences ? \\[0pt]
Une œuvre est-elle toujours de son temps ? \\[0pt]
Une passion peut-elle résister au temps ? \\[0pt]
Une perception peut-elle être illusoire ? \\[0pt]
Une philosophie de l'amour est-elle possible ? \\[0pt]
Une politique de la nature est-elle concevable ? \\[0pt]
Une politique morale est-elle possible ? \\[0pt]
Une politique peut-elle se réclamer de la vie ? \\[0pt]
Une religion civile est-elle possible ? \\[0pt]
Une religion peut-elle être rationnelle ? \\[0pt]
Une science bien faite est-elle une langue bien faite ? \\[0pt]
Une science de la conscience est-elle possible ? \\[0pt]
Une science de la culture est-elle possible ? \\[0pt]
Une science de la morale est-elle possible ? \\[0pt]
Une science de l'imaginaire est-elle possible ? \\[0pt]
Une science des symboles est-elle possible ? \\[0pt]
Une seconde nature \\[0pt]
Une société juste est-elle une société sans conflits ? \\[0pt]
Une société sans conflit est-elle possible ? \\[0pt]
Une société sans État est-elle une société sans politique ? \\[0pt]
Un État peut-il être trop étendu ? \\[0pt]
Un État peut-il être « voyou » ? \\[0pt]
Une théorie de la culture peut-elle être scientifique ? \\[0pt]
Une théorie sans expérience nous apprend-elle quelque chose ? \\[0pt]
Une théorie scientifique peut-elle devenir fausse ? \\[0pt]
Une théorie scientifique peut-elle être ramenée à des propositions empiriques élémentaires ? \\[0pt]
Une vérité est-elle discutable ? \\[0pt]
Une volonté peut-elle être générale ? \\[0pt]
Un homme n'est-il que la somme de ses actes ? \\[0pt]
Unité et unicité \\[0pt]
Universalité et nécessité dans les sciences \\[0pt]
Univocité et équivocité \\[0pt]
Un jugement de goût est-il culturel ? \\[0pt]
Un jugement moral peut-il relever de la vérité ? \\[0pt]
Un langage peut-il être privé ? \\[0pt]
Un mensonge peut-il être légitime ? \\[0pt]
Un moment d'éternité \\[0pt]
Un monde sans beauté \\[0pt]
Un monde sans nature est-il pensable ? \\[0pt]
Un objet technique peut-il être beau ? \\[0pt]
Un organisme n'est-il qu'un objet ? \\[0pt]
Un peuple est-il responsable de son histoire ? \\[0pt]
Un pouvoir a-t-il besoin d'être légitime ? \\[0pt]
Un préjugé n'a-t-il aucun fondement ? \\[0pt]
Un problème philosophique peut-il être périmé ? \\[0pt]
Un problème scientifique peut-il être insoluble ? \\[0pt]
Un savoir sur l'homme peut-il être séparé d'un pouvoir sur les hommes ? \\[0pt]
Un sceptique peut-il être logicien ? \\[0pt]
Un souverain élu peut-il être illégitime ? \\[0pt]
Un tableau peut-il être une dénonciation ? \\[0pt]
Un témoin de moralité est-il possible ? \\[0pt]
Un vice, est-ce un manque ? \\[0pt]
Urbanisme et politique \\[0pt]
Utiliser les embryons humains ? \\[0pt]
Utopie et tradition \\[0pt]
Vainqueurs et vaincus \\[0pt]
Valeur et évaluation \\[0pt]
Vanité des vanités \\[0pt]
Vendre son âme \\[0pt]
Vérité et cohérence \\[0pt]
Vérité et cohérence ? \\[0pt]
Vérité et convention \\[0pt]
Vérité et fiction \\[0pt]
Vérité et histoire \\[0pt]
Vérité et référence \\[0pt]
Vérité et sensibilité \\[0pt]
Vérité et signification \\[0pt]
Vérité et totalité \\[0pt]
Vérités de fait et vérités de raison \\[0pt]
Vérités mathématiques, vérités philosophiques \\[0pt]
Vertu et habitude \\[0pt]
Vertus morales, vertus intellectuelles \\[0pt]
Vices privés, vertus publiques \\[0pt]
Vie et destin \\[0pt]
Vie et existence \\[0pt]
Vie et matière \\[0pt]
Vie et matière : une distinction périmée ? \\[0pt]
Vie et mécanisme \\[0pt]
Vie et volonté \\[0pt]
Vie familiale, vie politique \\[0pt]
Vieillir \\[0pt]
Violence et discours \\[0pt]
Violence et politique \\[0pt]
Vitalisme et mécanique \\[0pt]
Vit-on au présent ? \\[0pt]
Vivons-nous tous dans le même monde ? \\[0pt]
Vivrait-on mieux sans morale ? \\[0pt]
Vivre au présent \\[0pt]
Vivre caché \\[0pt]
Vivre comme si nous ne devions pas mourir \\[0pt]
Vivre et bien vivre \\[0pt]
Vivre et exister \\[0pt]
Vivre intensément \\[0pt]
Vivre pour les autres \\[0pt]
Vivre sans morale \\[0pt]
Vivre sa vie \\[0pt]
Vivre selon la nature \\[0pt]
Vivre sous la conduite de la raison \\[0pt]
Vivre vertueusement \\[0pt]
Voir \\[0pt]
Voir et concevoir \\[0pt]
Voir et entendre \\[0pt]
Voir et savoir \\[0pt]
Voir et toucher \\[0pt]
Voir la réalité en face \\[0pt]
Voir le meilleur et faire le pire \\[0pt]
Voir les choses comme elles sont \\[0pt]
Vouloir ce que l'on peut \\[0pt]
Vouloir être heureux \\[0pt]
Vouloir la paix \\[0pt]
Vouloir le bien \\[0pt]
Vouloir l'égalité \\[0pt]
Vouloir le mal \\[0pt]
Voyager \\[0pt]
Voyager entre les mondes \\[0pt]
Vulgariser la science ? \\[0pt]
Y a-t-il continuité entre l'expérience et la science ? \\[0pt]
Y a-t-il continuité ou discontinuité entre la pensée mythique et la science ? \\[0pt]
Y a-t-il de bons préjugés ? \\[0pt]
Y a-t-il de fausses religions ? \\[0pt]
Y a-t-il de faux besoins ? \\[0pt]
Y a-t-il de la grandeur à être libre ? \\[0pt]
Y a-t-il de l'incompréhensible ? \\[0pt]
Y a-t-il de l'inconcevable ? \\[0pt]
Y a-t-il de l'inconnaissable ? \\[0pt]
Y a-t-il de l'indémontrable ? \\[0pt]
Y a-t-il de l'indicible ? \\[0pt]
Y a-t-il de l'intelligible dans l'art ? \\[0pt]
Y a-t-il de l'irréparable ? \\[0pt]
Y a-t-il de l'universel ? \\[0pt]
Y a-t-il de mauvaises raisons ? \\[0pt]
Y a-t-il des actes intrinsèquement mauvais ? \\[0pt]
Y a-t-il des actes moralement indifférents ? \\[0pt]
Y a-t-il des actions collectives ? \\[0pt]
Y a-t-il des actions désintéressées ? \\[0pt]
Y a-t-il des aliénations heureuses ? \\[0pt]
Y a-t-il des arts du corps ? \\[0pt]
Y a-t-il des arts majeurs ? \\[0pt]
Y a-t-il des arts mineurs ? \\[0pt]
Y a-t-il des biens communs ? \\[0pt]
Y a-t-il des canons de la beauté ? \\[0pt]
Y a-t-il des certitudes historiques ? \\[0pt]
Y a-t-il des choses à savoir ? \\[0pt]
Y a-t-il des choses qui échappent au droit ? \\[0pt]
Y a-t-il des compétences politiques ? \\[0pt]
Y a-t-il des conflits de devoirs ? \\[0pt]
Y a-t-il des critères de la scientificité ? \\[0pt]
Y a-t-il des critères de l'humanité ? \\[0pt]
Y a-t-il des critères du beau ? \\[0pt]
Y a-t-il des croyances démocratiques ? \\[0pt]
Y a-t-il des croyances raisonnables ? \\[0pt]
Y a-t-il des degrés dans la certitude ? \\[0pt]
Y a-t-il des degrés de liberté ? \\[0pt]
Y a-t-il des degrés de réalité ? \\[0pt]
Y a-t-il des degrés de vérité ? \\[0pt]
Y a-t-il des démonstrations en morale ? \\[0pt]
Y a-t-il des démonstrations en philosophie ? \\[0pt]
Y a-t-il des devoirs envers soi-même ? \\[0pt]
Y a-t-il des droits naturels ? \\[0pt]
Y a-t-il des erreurs en politique ? \\[0pt]
Y a-t-il des êtres de raison ? \\[0pt]
Y a-t-il des êtres mathématiques ? \\[0pt]
Y a-t-il des excès en art ? \\[0pt]
Y a-t-il des expériences cruciales ? \\[0pt]
Y a-t-il des expériences métaphysiques ? \\[0pt]
Y a-t-il des faits moraux ? \\[0pt]
Y a-t-il des faits sans essence ? \\[0pt]
Y a-t-il des fins de la nature ? \\[0pt]
Y a-t-il des fins dernières ? \\[0pt]
Y a-t-il des fondements naturels à l'ordre social ? \\[0pt]
Y a-t-il des formes élémentaires de la société ? \\[0pt]
Y a-t-il des guerres justes ? \\[0pt]
Y a-t-il des héritages philosophiques ? \\[0pt]
Y a-t-il des idées générales ? \\[0pt]
Y a-t-il des idées innées ? \\[0pt]
Y a-t-il des leçons de l'histoire ? \\[0pt]
Y a-t-il des limites à la connaissance de l'homme par les sciences ? \\[0pt]
Y a-t-il des limites à la critique ? \\[0pt]
Y a-t-il des limites à la description ? \\[0pt]
Y a-t-il des limites à la responsabilité ? \\[0pt]
Y a-t-il des limites à l'exprimable ? \\[0pt]
Y a-t-il des limites au droit ? \\[0pt]
Y a-t-il des limites proprement morales à la discussion ? \\[0pt]
Y a-t-il des lois de l'histoire ? \\[0pt]
Y a-t-il des lois du comportement humain ? \\[0pt]
Y a-t-il des lois du hasard ? \\[0pt]
Y a-t-il des lois du social ? \\[0pt]
Y a-t-il des lois en histoire ? \\[0pt]
Y a-t-il des lois injustes ? \\[0pt]
Y a-t-il des lois morales ? \\[0pt]
Y a-t-il des lois non écrites ? \\[0pt]
Y a-t-il des lois sociales ? \\[0pt]
Y a-t-il des mécanismes humains ? \\[0pt]
Y a-t-il des mentalités collectives ? \\[0pt]
Y a-t-il des normes de la vie ? \\[0pt]
Y a-t-il des normes naturelles ? \\[0pt]
Y a-t-il des notions communes ? \\[0pt]
Y a-t-il des passions collectives ? \\[0pt]
Y a-t-il des pathologies du social ? \\[0pt]
Y a-t-il des pathologies sociales ? \\[0pt]
Y a-t-il des pensées folles ? \\[0pt]
Y a-t-il des pensées inconscientes ? \\[0pt]
Y a-t-il des petites vertus ? \\[0pt]
Y a-t-il des phénomènes psychiques ? \\[0pt]
Y a-t-il des plaisirs innocents ? \\[0pt]
Y a-t-il des plaisirs mauvais ? \\[0pt]
Y a-t-il des plaisirs purs ? \\[0pt]
Y a-t-il des plaisirs simples ? \\[0pt]
Y a-t-il des points de vue en morale ? \\[0pt]
Y a-t-il des preuves d'amour ? \\[0pt]
Y a-t-il des preuves de la non-existence de Dieu ? \\[0pt]
Y a-t-il des preuves de l'existence de Dieu ? \\[0pt]
Y a-t-il des preuves en métaphysique ? \\[0pt]
Y a-t-il des preuves en philosophie ? \\[0pt]
Y a-t-il des progrès en art ? \\[0pt]
Y a-t-il des progrès en philosophie ? \\[0pt]
Y a-t-il des propriétés singulières ? \\[0pt]
Y a-t-il des questions sans réponse ? \\[0pt]
Y a-t-il des questions sans réponses ? \\[0pt]
Y a-t-il des raisons d'aimer ? \\[0pt]
Y a-t-il des raisons de vivre ? \\[0pt]
Y a-t-il des révolutions en art ? \\[0pt]
Y a-t-il des révolutions scientifiques ? \\[0pt]
Y a-t-il des savoirs immédiats ? \\[0pt]
Y a-t-il des sciences de l'éducation ? \\[0pt]
Y a-t-il des secrets de la nature ? \\[0pt]
Y a-t-il des sociétés archaïques ? \\[0pt]
Y a-t-il des sociétés sans État ? \\[0pt]
Y a-t-il des sociétés sans histoire ? \\[0pt]
Y a-t-il des substances incorporelles ? \\[0pt]
Y a-t-il des universaux anthropologiques ? \\[0pt]
Y a-t-il des vérités morales ? \\[0pt]
Y a-t-il des vertus mineures ? \\[0pt]
Y a-t-il des violences justifiées ? \\[0pt]
Y a-t-il des violences légitimes ? \\[0pt]
Y a-t-il différentes manières de connaître ? \\[0pt]
Y a-t-il du déterminisme dans les sciences humaines ? \\[0pt]
Y a-t-il du hasard objectif ? \\[0pt]
Y a-t-il du progrès en politique ? \\[0pt]
Y a-t-il du sacré dans la nature ? \\[0pt]
Y a-t-il du synthétique \emph{a priori} ? \\[0pt]
Y a-t-il du vrai dans le beau ? \\[0pt]
Y a-t-il encore des mythologies ? \\[0pt]
Y a-t-il encore une sphère privée ? \\[0pt]
Y a-t-il lieu de distinguer entre éthique et morale ? \\[0pt]
Y a-t-il lieu de distinguer instinct et pulsion ? \\[0pt]
Y a-t-il lieu d'opposer la philosophie et les sciences humaines ? \\[0pt]
Y a-t-il place pour l'idée de vérité en morale ? \\[0pt]
Y a-t-il plusieurs manières de définir ? \\[0pt]
Y a-t-il plusieurs métaphysiques ? \\[0pt]
Y a-t-il plusieurs morales ? \\[0pt]
Y a-t-il plusieurs nécessités ? \\[0pt]
Y a-t-il quelque chose de commun aux œuvres d'art ? \\[0pt]
Y a-t-il quelque chose de subversif dans les sciences humaines ? \\[0pt]
Y a-t-il un art de gouverner ? \\[0pt]
Y a-t-il un art de penser ? \\[0pt]
Y a-t-il un art de persuader ? \\[0pt]
Y a-t-il un art d'inventer ? \\[0pt]
Y a-t-il un art populaire ? \\[0pt]
Y a-t-il un autre monde ? \\[0pt]
Y a-t-il un beau idéal ? \\[0pt]
Y a-t-il un besoin métaphysique ? \\[0pt]
Y a-t-il un bien commun ? \\[0pt]
Y a-t-il un bien plus précieux que la paix ? \\[0pt]
Y a-t-il un bon usage de la rhétorique ? \\[0pt]
Y a-t-il un canon de la beauté ? \\[0pt]
Y a-t-il un commencement à tout ? \\[0pt]
Y a-t-il un critère du vrai ? \\[0pt]
Y a-t-il un désir de connaître ? \\[0pt]
Y a-t-il un déterminisme dans les sciences humaines ? \\[0pt]
Y a-t-il un devoir d'être heureux ? \\[0pt]
Y a-t-il un devoir d'indignation ? \\[0pt]
Y a-t-il un devoir d'oubli ? \\[0pt]
Y a-t-il un différend entre poésie et philosophie ? \\[0pt]
Y a-t-il un droit de la guerre ? \\[0pt]
Y a-t-il un droit de mourir ? \\[0pt]
Y a-t-il un droit de résistance ? \\[0pt]
Y a-t-il un droit international ? \\[0pt]
Y a-t-il un droit naturel ? \\[0pt]
Y a-t-il une argumentation métaphysique ? \\[0pt]
Y a-t-il une beauté morale ? \\[0pt]
Y a-t-il une beauté naturelle ? \\[0pt]
Y a-t-il une beauté propre à la photographie ? \\[0pt]
Y a-t-il une bonne éducation ? \\[0pt]
Y a-t-il une causalité historique ? \\[0pt]
Y a-t-il une compétence en politique ? \\[0pt]
Y a-t-il une connaissance du singulier ? \\[0pt]
Y a-t-il une connaissance logique ? \\[0pt]
Y a-t-il une connaissance métaphysique ? \\[0pt]
Y a-t-il une connaissance sensible ? \\[0pt]
Y a-t-il une correspondance des arts ? \\[0pt]
Y a-t-il une esthétique du laid ? \\[0pt]
Y a-t-il une esthétique industrielle ? \\[0pt]
Y a-t-il une éthique de l'authenticité ? \\[0pt]
Y a-t-il une expérience de la liberté ? \\[0pt]
Y a-t-il une expérience de l'éternité ? \\[0pt]
Y a-t-il une expérience du néant ? \\[0pt]
Y a-t-il une expérience métaphysique ? \\[0pt]
Y a-t-il une fin de l'histoire ? \\[0pt]
Y a-t-il une fin dernière ? \\[0pt]
Y a-t-il une formation de la pensée logique ? \\[0pt]
Y a-t-il une forme morale de fanatisme ? \\[0pt]
Y a-t-il une hiérarchie des êtres ? \\[0pt]
Y a-t-il une hiérarchie des sciences ? \\[0pt]
Y a-t-il une hiérarchie des sciences sociales ? \\[0pt]
Y a-t-il une histoire de la folie ? \\[0pt]
Y a-t-il une histoire de l'art ? \\[0pt]
Y a-t-il une histoire de la vérité ? \\[0pt]
Y a-t-il une histoire de la vie quotidienne ? \\[0pt]
Y a-t-il une histoire universelle ? \\[0pt]
Y a-t-il une intentionnalité collective ? \\[0pt]
Y a-t-il une justice sans morale ? \\[0pt]
Y a-t-il une logique de la découverte ? \\[0pt]
Y a-t-il une logique de la découverte scientifique ? \\[0pt]
Y a-t-il une logique de la déraison ? \\[0pt]
Y a-t-il une logique de l'art ? \\[0pt]
Y a-t-il une logique du mythe ? \\[0pt]
Y a-t-il une loi suprême ? \\[0pt]
Y a-t-il une mathématique universelle ? \\[0pt]
Y a-t-il une métaphysique de l'amour ? \\[0pt]
Y a-t-il une méthode propre aux sciences humaines ? \\[0pt]
Y a-t-il une nature humaine ? \\[0pt]
Y a-t-il une nécessité de l'erreur ? \\[0pt]
Y a-t-il une opinion publique mondiale ? \\[0pt]
Y a-t-il une ou plusieurs philosophies ? \\[0pt]
Y a-t-il une pensée sans signes ? \\[0pt]
Y a-t-il une pensée technique ? \\[0pt]
Y a-t-il une perception esthétique ? \\[0pt]
Y a-t-il une philosophie de la philosophie ? \\[0pt]
Y a-t-il une philosophie première ? \\[0pt]
Y a-t-il une place dans les sciences humaines pour la catégorie de personne ? \\[0pt]
Y a-t-il une place pour la morale dans l'économie ? \\[0pt]
Y a-t-il une psychologie animale ? \\[0pt]
Y a-t-il une raison des choses ? \\[0pt]
Y a-t-il une rationalité du marché ? \\[0pt]
Y a-t-il une rationalité propre de l'intérêt ? \\[0pt]
Y a-t-il une réalité des idées ? \\[0pt]
Y a-t-il une religion civile ? \\[0pt]
Y a-t-il une science de la vie mentale ? \\[0pt]
Y a-t-il une science de l'être ? \\[0pt]
Y a-t-il une science de l'individuel ? \\[0pt]
Y a-t-il une science des principes ? \\[0pt]
Y a-t-il une science du qualitatif ? \\[0pt]
Y a-t-il une sensibilité esthétique ? \\[0pt]
Y a-t-il une singularité de l'histoire de l'art ? \\[0pt]
Y a-t-il une spécificité de la délibération politique ? \\[0pt]
Y a-t-il une spécificité des sciences humaines ? \\[0pt]
Y a-t-il une temporalité proprement morale ? \\[0pt]
Y a-t-il une unité de la science ? \\[0pt]
Y a-t-il une unité des sciences ? \\[0pt]
Y a-t-il une unité en psychologie ? \\[0pt]
Y a-t-il une universalité des mathématiques ? \\[0pt]
Y a-t-il une utilité des lieux communs ? \\[0pt]
Y a-t-il une vérité absolue ? \\[0pt]
Y a-t-il une vérité dans les arts ? \\[0pt]
Y a-t-il une vérité de l'œuvre d'art ? \\[0pt]
Y a-t-il une vérité des symboles ? \\[0pt]
Y a-t-il une vérité du sentiment ? \\[0pt]
Y a-t-il une vérité philosophique ? \\[0pt]
Y a-t-il une vertu de l'imitation ? \\[0pt]
Y a-t-il une vie de l'esprit ? \\[0pt]
Y a-t-il un fondement moral aux circonstances atténuantes ? \\[0pt]
Y a-t-il un inconscient collectif ? \\[0pt]
Y a t-il un instinct moral ? \\[0pt]
Y a-t-il un langage commun ? \\[0pt]
Y a-t-il un langage du corps ? \\[0pt]
Y a-t-il un mal absolu ? \\[0pt]
Y a-t-il un monde extérieur ? \\[0pt]
Y a-t-il un ordre des choses ? \\[0pt]
Y a-t-il un principe du mal ? \\[0pt]
Y a-t-il un progrès dans l'art ? \\[0pt]
Y a-t-il un progrès des/en sciences humaines ? \\[0pt]
Y a-t-il un progrès en art ? \\[0pt]
Y a-t-il un progrès moral ? \\[0pt]
Y a-t-il un savoir du bien ? \\[0pt]
Y a-t-il un savoir du contingent ? \\[0pt]
Y a-t-il un savoir du corps ? \\[0pt]
Y a-t-il un savoir du politique ? \\[0pt]
Y a-t-il un savoir immédiat ? \\[0pt]
Y a-t-il un sens à dire que Dieu comprend, veut, fait ? \\[0pt]
Y a-t-il un sens à penser un droit des générations futures ? \\[0pt]
Y a-t-il un sens à se dire « citoyen du monde » ? \\[0pt]
Y a-t-il un sentiment métaphysique ? \\[0pt]
Y a-t-il un sentiment naturel du beau ? \\[0pt]
Y a-t-il un temps des choses ? \\[0pt]
Y a-t-il un temps pour l'action ? \\[0pt]
Y a-t-il un temps pour tout ? \\[0pt]
Y a-t-il un traitement scientifique du langage ? \\[0pt]
Y a-t-il un usage moral des passions ? \\[0pt]
Y a-t-une science du comportement ? \\[0pt]


\subsection{Agrégation interne}
\label{sec:org47b14ec}

\noindent
Accuser et défendre \\[0pt]
À chacun sa morale \\[0pt]
Action et production \\[0pt]
Admettre le hasard, est-ce nier l'ordre de la nature ? \\[0pt]
Agir justement fait-il de moi un homme juste ? \\[0pt]
Agir moralement, est-ce lutter contre soi-même ? \\[0pt]
Agir par devoir, est-ce évaluer les conséquences de ses actes \\[0pt]
Agir par devoir, est-ce renoncer à ses intérêts ? \\[0pt]
Agir sans raison \\[0pt]
Amitié et société \\[0pt]
Analyse et synthèse \\[0pt]
Appartenons-nous à une culture ? \\[0pt]
Apprend-on à être libre ? \\[0pt]
Apprendre à parler \\[0pt]
À quelles conditions une vérité s'impose-t-elle à nous ? \\[0pt]
À quelles conditions y a-t-il progrès technique ? \\[0pt]
À qui dois-je la vérité ? \\[0pt]
À qui le pouvoir appartient-il ? \\[0pt]
À qui profite le travail ? \\[0pt]
À quoi bon avoir mauvaise conscience ? \\[0pt]
À quoi bon critiquer les autres ? \\[0pt]
À quoi bon imiter la nature ? \\[0pt]
À quoi bon se parler ? \\[0pt]
À quoi l'art nous rend-il sensibles ? \\[0pt]
À quoi reconnaît-on la rationalité ? \\[0pt]
À quoi reconnaît-on une œuvre d'art ? \\[0pt]
À quoi sert la technique ? \\[0pt]
À quoi sert la vérité ? \\[0pt]
À quoi tient la force des religions ? \\[0pt]
À quoi tient l'efficacité d'une technique ? \\[0pt]
A-t-on le droit de faire tout ce qui est permis par la loi ? \\[0pt]
A-t-on le droit de se révolter ? \\[0pt]
Autrui, est-ce n'importe quel autre ? \\[0pt]
Avoir bonne conscience \\[0pt]
Avoir des principes \\[0pt]
Avoir des scrupules \\[0pt]
Avoir des valeurs \\[0pt]
Avoir le sens du devoir \\[0pt]
Avoir le temps \\[0pt]
Avoir raison de ses émotions \\[0pt]
Avons-nous des devoirs envers les animaux ? \\[0pt]
Avons-nous des devoirs envers les générations futures ? \\[0pt]
Avons-nous le devoir d'être heureux ? \\[0pt]
Avons-nous une intuition du temps ? \\[0pt]
Avons-nous un monde commun ? \\[0pt]
Bien commun et bien public \\[0pt]
Bonheur et autarcie \\[0pt]
Catégories de pensée, catégories de langue \\[0pt]
Cause et loi \\[0pt]
Ce que la technique rend possible, peut-on jamais en empêcher la réalisation ? \\[0pt]
Ce qui est contingent peut-il être objet de science ? \\[0pt]
Ce qui est démontré est-il nécessairement vrai ? \\[0pt]
Ce qui est passé est-il dépassé ? \\[0pt]
Ce qui n'a pas de prix \\[0pt]
Ce qui n'est pas démontré peut-il être vrai ? \\[0pt]
Ce qui n'est pas matériel peut-il être réel ? \\[0pt]
« C'est scientifique » \\[0pt]
Changer \\[0pt]
Changer le monde \\[0pt]
Chercher ses mots \\[0pt]
Choisir \\[0pt]
Comment expliquer les phénomènes mentaux ? \\[0pt]
Comment penser l'écoulement du temps ? \\[0pt]
Comment percevons-nous l'espace ? \\[0pt]
Comment reconnaît-on un vivant ? \\[0pt]
Comment savoir ce qui échappe à la conscience \\[0pt]
Comment savoir quand nous sommes libres ? \\[0pt]
Comment se libérer du temps ? \\[0pt]
Commettre une faute \\[0pt]
Comprendre autrui \\[0pt]
Comprendre, est-ce interpréter ? \\[0pt]
Comprendre l'inconscient \\[0pt]
Concept et image \\[0pt]
Conduire sa vie \\[0pt]
Conflit et démocratie \\[0pt]
Connaissons-nous mieux le présent que le passé ? \\[0pt]
Connaître, est-ce connaître par les causes ? \\[0pt]
Conscience et attention \\[0pt]
Conscience et mémoire \\[0pt]
Contempler \\[0pt]
Croire, est-ce obéir ? \\[0pt]
Croire pour savoir \\[0pt]
Croyance et certitude \\[0pt]
Croyance et vérité \\[0pt]
Culture et civilisation \\[0pt]
Dans quelle mesure pouvons-nous échapper à notre passé ? \\[0pt]
Dans quelle mesure suis-je responsable de mon inconscient ? \\[0pt]
Découvrir et inventer \\[0pt]
Décrire et expliquer \\[0pt]
Démocratie et religion \\[0pt]
Démontrer, argumenter, expérimenter \\[0pt]
Démontrer est-il le privilège du mathématicien ? \\[0pt]
Démontrer et argumenter \\[0pt]
Dénaturer \\[0pt]
De quelle liberté témoigne l'œuvre d'art ? \\[0pt]
De quelle réalité nos perceptions témoignent-elles ? \\[0pt]
De quelle vérité l'inconscient est-il porteur ? \\[0pt]
De quoi faut-il attendre son bonheur ? \\[0pt]
De quoi l'art peut-il être contemporain ? \\[0pt]
De quoi l'outil nous libère-t-il ? \\[0pt]
De quoi peut-il y avoir science ? \\[0pt]
Déraisonner \\[0pt]
Désobéir \\[0pt]
Devant qui sommes-nous responsables ? \\[0pt]
Devoir, est-ce pouvoir ? \\[0pt]
Devoir et conformisme \\[0pt]
Dieu aurait-il pu mieux faire ? \\[0pt]
Dieu tout-puissant \\[0pt]
Dire et faire \\[0pt]
Dire et montrer \\[0pt]
Dire le singulier \\[0pt]
Doit-on corriger les inégalités sociales ? \\[0pt]
Doit-on croire en l'humanité ? \\[0pt]
Doit-on distinguer devoir moral et obligation sociale ? \\[0pt]
Doit-on justifier les inégalités ? \\[0pt]
Doit-on se justifier d'exister ? \\[0pt]
Donner à chacun son dû \\[0pt]
Donner, à quoi bon ? \\[0pt]
Donner sa parole \\[0pt]
D'où vient le sens du devoir ? \\[0pt]
Droit et démocratie \\[0pt]
Droit et devoir sont-ils liés ? \\[0pt]
Droit et morale \\[0pt]
Droit et protection \\[0pt]
Échange et partage \\[0pt]
Échange et valeur \\[0pt]
Échanger, est-ce risquer ? \\[0pt]
Éduquer, est-ce dénaturer ? \\[0pt]
En histoire, tout est-il affaire d'interprétation ? \\[0pt]
En morale, est-ce seulement l'intention qui compte ? \\[0pt]
En politique, faut-il refuser l'utopie ? \\[0pt]
En politique, nécessité fait loi \\[0pt]
En quel sens la maladie peut-elle transformer notre vie ? \\[0pt]
En quel sens peut-on parler d'expérience possible ? \\[0pt]
Entendement et raison \\[0pt]
Entendre raison \\[0pt]
Entre le vrai et le faux y a-t-il une place pour le probable ? \\[0pt]
Esprit et intériorité \\[0pt]
Essence et existence \\[0pt]
Est-ce la certitude qui fait la science ? \\[0pt]
Est-ce la démonstration qui fait la science ? \\[0pt]
Est-ce le corps qui perçoit ? \\[0pt]
Est-ce l'utilité qui définit un objet technique ? \\[0pt]
Est-ce un devoir de rechercher la vérité ? \\[0pt]
Est-ce un devoir d'être sincère ? \\[0pt]
Est-ce une chance de naître humain ? \\[0pt]
Esthétique et éthique \\[0pt]
Est-il légitime d'affirmer que seul le présent existe ? \\[0pt]
Est-il possible de ne pas savoir ce que l'on pense ? \\[0pt]
Est-il raisonnable de lutter contre le temps ? \\[0pt]
Est-il si difficile d'accéder à la vérité ? \\[0pt]
Est-il toujours possible de faire ce que l'on dit ? \\[0pt]
Établir la vérité, est-ce nécessairement démontrer ? \\[0pt]
Être aliéné \\[0pt]
Être au monde \\[0pt]
Être citoyen \\[0pt]
Être conscient de soi, est-ce être maître de soi ? \\[0pt]
Être dans le temps \\[0pt]
Être dans son droit \\[0pt]
Être de son temps \\[0pt]
Être déterminé \\[0pt]
Être heureux \\[0pt]
Être là \\[0pt]
Être libre, est-ce n'obéir qu'à soi-même ? \\[0pt]
Être libre, même dans les fers \\[0pt]
Être malade \\[0pt]
Être matérialiste \\[0pt]
Être ou ne pas être ? \\[0pt]
Être présent \\[0pt]
Être quelqu'un \\[0pt]
Être religieux, est-ce nécessairement être dogmatique ? \\[0pt]
Être soi-même \\[0pt]
Existe-t-il une science de la morale ? \\[0pt]
Expérience et habitude \\[0pt]
Expérience et interprétation \\[0pt]
Expérience et jugement \\[0pt]
Expérience et phénomène \\[0pt]
Expérimentation et vérification \\[0pt]
Expérimenter \\[0pt]
Expliquer, est-ce interpréter ? \\[0pt]
Faire confiance \\[0pt]
Faire des choix \\[0pt]
Faire école \\[0pt]
Faire justice \\[0pt]
Faire preuve d'humanité \\[0pt]
Faire son devoir \\[0pt]
Faisons-nous notre propre malheur ? \\[0pt]
Faudrait-il ne rien oublier ? \\[0pt]
Faut-il accorder l'esprit aux bêtes ? \\[0pt]
Faut-il apprendre à voir ? \\[0pt]
Faut-il avoir foi en la raison ? \\[0pt]
Faut-il avoir peur de la nature ? \\[0pt]
Faut-il cultiver le naturel ? \\[0pt]
Faut-il défendre l'ordre à tout prix ? \\[0pt]
Faut-il dire tout haut ce que les autres pensent tout bas ? \\[0pt]
Faut-il écouter sa conscience ? \\[0pt]
Faut-il être fidèle à soi-même ? \\[0pt]
Faut-il être réaliste ? \\[0pt]
Faut-il interpréter la loi ? \\[0pt]
Faut-il libérer la parole ? \\[0pt]
Faut-il limiter les prétentions de la science ? \\[0pt]
Faut-il opposer la foi et la raison ? \\[0pt]
Faut-il opposer la théorie et la pratique ? \\[0pt]
Faut-il opposer science et croyance ? \\[0pt]
Faut-il oublier le passé ? \\[0pt]
Faut-il préférer la liberté à l'égalité ? \\[0pt]
Faut-il que le réel ait un sens ? \\[0pt]
Faut-il que les meilleurs gouvernent ? \\[0pt]
Faut-il renoncer à connaître la nature des choses ? \\[0pt]
Faut-il respecter la nature ? \\[0pt]
Faut-il s'adapter ? \\[0pt]
Faut-il sauver les apparences ? \\[0pt]
Faut-il se chercher pour se trouver ? \\[0pt]
Faut-il se demander si l'homme est bon ou méchant par nature \\[0pt]
Faut-il s'en tenir aux faits ? \\[0pt]
Faut-il séparer morale et politique ? \\[0pt]
Faut-il tout démontrer ? \\[0pt]
Faut-il tout interpréter ? \\[0pt]
Faut-il vivre comme si nous étions immortels ? \\[0pt]
Figurer le pouvoir \\[0pt]
Foi et bonne foi \\[0pt]
Forcer à être libre \\[0pt]
Former les esprits \\[0pt]
Heureux les simples d'esprit \\[0pt]
Histoire et écriture \\[0pt]
Histoire et morale \\[0pt]
Imiter, est-ce manquer d'imagination ? \\[0pt]
Incarner le pouvoir \\[0pt]
Inconscient et identité \\[0pt]
Inconscient et inconscience \\[0pt]
Interprétation et création \\[0pt]
Interpréter, est-ce renoncer à prouver ? \\[0pt]
Interpréter, est-ce savoir ? \\[0pt]
Interpréter ou expliquer \\[0pt]
Invention et création \\[0pt]
« Je mens » \\[0pt]
Joue-t-on toujours un rôle ? \\[0pt]
Juger et connaître \\[0pt]
Justice et égalité \\[0pt]
Justice et force \\[0pt]
Justice et vengeance \\[0pt]
Justifier le mensonge \\[0pt]
La barbarie \\[0pt]
La barbarie de la technique \\[0pt]
La barbarie peut-elle avoir un visage humain ? \\[0pt]
La beauté n'est-elle qu'une idée ? \\[0pt]
La beauté peut-elle laisser indifférent ? \\[0pt]
La bienveillance \\[0pt]
L'absence de générosité \\[0pt]
L'absence de preuves \\[0pt]
L'abstraction \\[0pt]
L'absurde \\[0pt]
La causalité historique \\[0pt]
La cause première \\[0pt]
L'accélération du temps \\[0pt]
L'accomplissement de soi \\[0pt]
La certitude \\[0pt]
La chair \\[0pt]
La chute \\[0pt]
La civilité \\[0pt]
La cohérence \\[0pt]
La colère \\[0pt]
La comparaison \\[0pt]
La concorde \\[0pt]
La connaissance de la vie se confond-elle avec celle du vivant ? \\[0pt]
La connaissance de l'histoire est-elle utile à l'action ? \\[0pt]
La connaissance scientifique est-elle désintéressée ? \\[0pt]
La conscience de classe \\[0pt]
La conscience définit-elle l'homme en propre ? \\[0pt]
La conscience des autres \\[0pt]
La conscience entrave-t-elle l'action ? \\[0pt]
La conscience est-elle ce qui fait le sujet ? \\[0pt]
La conscience est-elle ou n'est-elle pas ? \\[0pt]
La conscience malheureuse \\[0pt]
La conscience nous leurre-t-elle ? \\[0pt]
La conscience peut-elle être collective ? \\[0pt]
La conscience peut-elle être objet de science ? \\[0pt]
La contingence \\[0pt]
La contradiction \\[0pt]
La contrainte des lois est-elle une violence ? \\[0pt]
La contrainte supprime-t-elle la responsabilité ? \\[0pt]
La conversation \\[0pt]
La création \\[0pt]
La critique \\[0pt]
La critique du pouvoir peut-elle conduire à la désobéissance ? \\[0pt]
La croyance est-elle signe de faiblesse ? \\[0pt]
La croyance est-elle une opinion ? \\[0pt]
La croyance est-elle une opinion comme les autres ? \\[0pt]
La croyance peut-elle être rationnelle ? \\[0pt]
L'acte gratuit \\[0pt]
L'action du temps \\[0pt]
L'action politique \\[0pt]
L'actualité \\[0pt]
La culture est-elle une seconde nature ? \\[0pt]
La culture et les cultures \\[0pt]
La culture libère-t-elle des préjugés ? \\[0pt]
La culture nous rend-elle meilleurs ? \\[0pt]
La culture peut-elle être instituée ? \\[0pt]
La culture peut-elle être objet de science ? \\[0pt]
La curiosité \\[0pt]
La découverte scientifique a-t-elle une logique ? \\[0pt]
La définition \\[0pt]
La démocratie peut-elle se passer de représentation ? \\[0pt]
La démonstration obéit-elle à des lois ? \\[0pt]
La déraison \\[0pt]
La discipline \\[0pt]
La discursivité \\[0pt]
La distinction sociale \\[0pt]
La division du travail \\[0pt]
La domination \\[0pt]
La durée \\[0pt]
La faiblesse d'esprit \\[0pt]
La fidélité \\[0pt]
La figure de l'ennemi \\[0pt]
La fin des désirs \\[0pt]
La fin des guerres \\[0pt]
La fin des temps \\[0pt]
La fin du monde \\[0pt]
La fin du travail \\[0pt]
La finitude \\[0pt]
La foi \\[0pt]
La folie des grandeurs \\[0pt]
La fonction du philosophe est-elle de diriger l'État ? \\[0pt]
La force de la vérité \\[0pt]
La force de l'expérience \\[0pt]
La force de l'habitude \\[0pt]
La force de l'inconscient \\[0pt]
La force des lois \\[0pt]
La force du vrai \\[0pt]
La formation de l'esprit \\[0pt]
La forme du droit \\[0pt]
La fraternité \\[0pt]
La fraternité peut-elle se passer d'un fondement religieux ? \\[0pt]
La généralité \\[0pt]
La générosité \\[0pt]
La grammaire contraint-elle la pensée ? \\[0pt]
La grandeur d'une culture \\[0pt]
La guerre civile \\[0pt]
La haine de la raison \\[0pt]
La haine des machines \\[0pt]
La honte \\[0pt]
La joie \\[0pt]
La joie de vivre \\[0pt]
La jouissance \\[0pt]
La juste mesure \\[0pt]
La justice consiste-t-elle dans l'application de la loi ? \\[0pt]
La justice divine \\[0pt]
La justice est-elle de ce monde ? \\[0pt]
La justice exclut-elle l'arbitraire ? \\[0pt]
La justice peut-elle se passer de la force ? \\[0pt]
La justice sociale \\[0pt]
La laïcité \\[0pt]
La laideur \\[0pt]
La langue de la raison \\[0pt]
La légitimité de l'État \\[0pt]
La lettre et l'esprit \\[0pt]
La liberté artistique \\[0pt]
La liberté civile \\[0pt]
La liberté des autres \\[0pt]
La liberté d'expression \\[0pt]
La liberté d'obéir \\[0pt]
La liberté doit-elle se conquérir ? \\[0pt]
La liberté du savant \\[0pt]
La liberté est-elle la condition du bonheur ? \\[0pt]
La liberté est-elle un fait ? \\[0pt]
La liberté implique-t-elle l'indifférence ? \\[0pt]
La liberté individuelle \\[0pt]
La liberté peut-elle faire peur ? \\[0pt]
La liberté peut-elle se constater ? \\[0pt]
La liberté peut-elle se prouver ? \\[0pt]
La liberté peut-elle se refuser ? \\[0pt]
La liberté se prouve-t-elle ? \\[0pt]
La liberté se réduit-elle au libre arbitre ? \\[0pt]
L'aliéné \\[0pt]
La logique du sens \\[0pt]
La logique est-elle la norme du vrai ? \\[0pt]
La logique est-elle l'art de penser ? \\[0pt]
La loi du désir \\[0pt]
La loi éduque-t-elle ? \\[0pt]
L'altérité \\[0pt]
L'altruisme \\[0pt]
La main \\[0pt]
La maîtrise de soi \\[0pt]
La maîtrise du temps \\[0pt]
La maladie \\[0pt]
La matière de la pensée \\[0pt]
La matière de l'œuvre \\[0pt]
La matière, est-ce le mal ? \\[0pt]
La matière est-elle amorphe ? \\[0pt]
La matière est-elle une vue de l'esprit ? \\[0pt]
La matière et la forme \\[0pt]
La matière n'est-elle qu'un obstacle ? \\[0pt]
La matière pense-t-elle ? \\[0pt]
La matière peut-elle être objet de connaissance ? \\[0pt]
La matière vivante \\[0pt]
La mauvaise conscience \\[0pt]
La mauvaise éducation \\[0pt]
La mauvaise foi \\[0pt]
La méconnaissance de soi \\[0pt]
La médecine est-elle une science ? \\[0pt]
La méfiance \\[0pt]
La mémoire et l'histoire \\[0pt]
La mesure du temps \\[0pt]
La métaphore \\[0pt]
L'amitié \\[0pt]
L'amitié peut-elle obliger ? \\[0pt]
La morale a-t-elle besoin d'un fondement ? \\[0pt]
La morale est-elle affaire de sentiments ? \\[0pt]
La mort de Dieu \\[0pt]
L'amour de soi \\[0pt]
L'amour est-il désir ? \\[0pt]
L'amour est-il une vertu ? \\[0pt]
L'amour maternel \\[0pt]
L'anarchie \\[0pt]
La nature a-t-elle des droits ? \\[0pt]
La nature du droit \\[0pt]
La nature est-elle écrite en langage mathématique ? \\[0pt]
La nature est-elle sans histoire ? \\[0pt]
La nature fait-elle bien les choses ? \\[0pt]
La négligence \\[0pt]
La neutralité \\[0pt]
L'angoisse \\[0pt]
L'animal a-t-il des droits ? \\[0pt]
L'anormal \\[0pt]
La nostalgie \\[0pt]
La nouveauté \\[0pt]
La paix \\[0pt]
La paix de la conscience \\[0pt]
La paix n'est-elle qu'un idéal ? \\[0pt]
La paix sociale est-elle la finalité de la politique ? \\[0pt]
La panne et la maladie \\[0pt]
La passion n'est-elle que souffrance ? \\[0pt]
L'apathie \\[0pt]
La pauvreté \\[0pt]
La perception est-elle l'interprétation du réel ? \\[0pt]
La perception peut-elle être désintéressée ? \\[0pt]
La personne politique \\[0pt]
La persuasion \\[0pt]
La peur des mots \\[0pt]
La philosophie peut-elle être expérimentale ? \\[0pt]
La politesse \\[0pt]
La politique doit-elle refuser l'utopie ? \\[0pt]
La politique est-elle extérieure au droit ? \\[0pt]
La politique est-elle l'affaire de tous ? \\[0pt]
La politique est-elle une technique ? \\[0pt]
L'apparence fait-elle obstacle à la vérité ? \\[0pt]
L'apprentissage \\[0pt]
La première fois \\[0pt]
La présence \\[0pt]
La présomption d'innocence \\[0pt]
L'\emph{a priori} \\[0pt]
La promesse \\[0pt]
La propriété \\[0pt]
La protection sociale \\[0pt]
La prudence \\[0pt]
La pudeur \\[0pt]
La puissance du peuple \\[0pt]
La raison a-t-elle une histoire ? \\[0pt]
La raison d'État \\[0pt]
La raison doit-elle être cultivée ? \\[0pt]
La raison du plus fort \\[0pt]
La raison est-elle le pouvoir de distinguer le vrai du faux ? \\[0pt]
La raison est-elle un obstacle au bonheur ? \\[0pt]
La raison gouverne-t-elle le monde ? \\[0pt]
La raison peut-elle errer ? \\[0pt]
La raison peut-elle nous commander de croire ? \\[0pt]
La raison peut-elle rendre raison de tout ? \\[0pt]
La raison pratique \\[0pt]
La réalité de la vie s'épuise-t-elle dans celle des vivants ? \\[0pt]
La réalité du temps se réduit-elle à la conscience que nous en avons ? \\[0pt]
La réalité est-elle une idée ? \\[0pt]
La réalité sociale \\[0pt]
La reconnaissance \\[0pt]
La rectitude du droit \\[0pt]
La référence aux faits suffit-elle à garantir l'objectivité de la connaissance ? \\[0pt]
La religion a-t-elle besoin d'un dieu ? \\[0pt]
La religion a-t-elle une fonction sociale ? \\[0pt]
La religion civile \\[0pt]
La religion est-elle la sagesse des pauvres ? \\[0pt]
La religion est-elle simple affaire de croyance ? \\[0pt]
La religion est-elle source de conflit ? \\[0pt]
La religion est-elle une affaire privée ? \\[0pt]
La religion est-elle un obstacle à la liberté ? \\[0pt]
La religion rend-elle meilleur ? \\[0pt]
La rencontre d'autrui \\[0pt]
La répétition \\[0pt]
La résistance de la matière \\[0pt]
La responsabilité de l'artiste \\[0pt]
L'argent \\[0pt]
La rigueur \\[0pt]
La rivalité \\[0pt]
L'arme rhétorique \\[0pt]
L'art de faire croire \\[0pt]
L'art doit-il refaire le monde ? \\[0pt]
L'art éduque-t-il l'homme ? \\[0pt]
L'art est-il le produit de l'inconscient ? \\[0pt]
L'art est-il le propre de l'homme ? \\[0pt]
L'art est-il le règne des apparences ? \\[0pt]
L'art est-il un jeu ? \\[0pt]
L'art et la manière \\[0pt]
L'art et le temps \\[0pt]
L'art exprime-t-il ce que nous ne saurions dire ? \\[0pt]
L'artiste est-il le seul travailleur libre ? \\[0pt]
L'artiste et la société \\[0pt]
L'artiste et le savant \\[0pt]
L'art modifie-t-il notre rapport à la réalité ? \\[0pt]
L'art nous fait-il mieux percevoir le réel ? \\[0pt]
L'art pense-t-il ? \\[0pt]
L'art peut-il contribuer à éduquer les hommes ? \\[0pt]
L'art peut-il être abstrait ? \\[0pt]
La santé est-elle un devoir ? \\[0pt]
La science a-t-elle toujours raison ? \\[0pt]
La science commence-t-elle avec la perception ? \\[0pt]
La science dépend-elle nécessairement de l'expérience ? \\[0pt]
La science est-elle indépendante de toute métaphysique ? \\[0pt]
La science n'est-elle qu'une activité théorique ? \\[0pt]
La science n'est-elle qu'une fiction ? \\[0pt]
La science permet-elle d'expliquer toute la réalité ? \\[0pt]
La seconde nature \\[0pt]
La sécurité \\[0pt]
La sensibilité \\[0pt]
La sensibilité animale \\[0pt]
La servitude peut-elle être volontaire ? \\[0pt]
La servitude volontaire \\[0pt]
La sexualité \\[0pt]
La simplicité du bien \\[0pt]
La sincérité \\[0pt]
La société civile \\[0pt]
La société contre l'État \\[0pt]
La société du genre humain \\[0pt]
La société est-elle concevable sans le travail ? \\[0pt]
La société peut-elle se passer de l'État ? \\[0pt]
La société sans l'État \\[0pt]
La solidarité \\[0pt]
La solitude \\[0pt]
La souffrance \\[0pt]
La souffrance a-t-elle un sens ? \\[0pt]
La soumission \\[0pt]
La souveraineté \\[0pt]
La spontanéité \\[0pt]
L'asservissement à l'utile \\[0pt]
La subjectivité \\[0pt]
La succession des théories scientifiques \\[0pt]
La superstition \\[0pt]
La technique a-t-elle une histoire ? \\[0pt]
La technique fait-elle violence à la nature ? \\[0pt]
La technique n'est-elle qu'une application de la science ? \\[0pt]
La technique n'est-elle qu'un savoir-faire ? \\[0pt]
La technique permet-elle de réaliser tous les désirs ? \\[0pt]
La tempérance \\[0pt]
La théorie et la pratique \\[0pt]
La totalité \\[0pt]
La tradition \\[0pt]
L'attention \\[0pt]
La tyrannie \\[0pt]
L'autorité de l'État \\[0pt]
L'autre et les autres \\[0pt]
La valeur de l'échange \\[0pt]
La valeur d'une action se mesure-t-elle à sa réussite ? \\[0pt]
La valeur du plaisir \\[0pt]
La valeur du travail \\[0pt]
L'avenir a-t-il une réalité ? \\[0pt]
La vérité a-t-elle une histoire ? \\[0pt]
La vérité de l'apparence \\[0pt]
La vérité des images \\[0pt]
La vérité des sciences \\[0pt]
La vérité doit-elle toujours être démontrée ? \\[0pt]
La vérité est-elle affaire d'unanimité ? \\[0pt]
La vérité est-elle éternelle ? \\[0pt]
La vérité est-elle une construction ? \\[0pt]
La vérité historique \\[0pt]
La vérité judiciaire \\[0pt]
La vérité n'est-elle qu'une erreur rectifiée ? \\[0pt]
La vérité nous contraint-elle ? \\[0pt]
La vérité peut-elle changer avec le temps ? \\[0pt]
La vérité peut-elle être équivoque ? \\[0pt]
La vérité peut-elle être indicible ? \\[0pt]
La vérité philosophique \\[0pt]
La vérité résiste-elle à sa révélation ? \\[0pt]
La vertu de l'exemple \\[0pt]
La vertu du plaisir \\[0pt]
La vertu peut-elle s'enseigner ? \\[0pt]
L'aveu \\[0pt]
La vie de l'esprit \\[0pt]
La vie des machines \\[0pt]
La vie est-elle le bien le plus précieux ? \\[0pt]
La vie est-elle trop courte ? \\[0pt]
La vie intérieure \\[0pt]
La vie intérieure est-elle une illusion ? \\[0pt]
La vie peut-elle être éternelle ? \\[0pt]
La vie peut-elle être objet de science ? \\[0pt]
La violence d'État \\[0pt]
La violence peut-elle être morale ? \\[0pt]
La voix de la conscience \\[0pt]
La volonté peut-elle nous manquer ? \\[0pt]
« La vraie morale se moque de la morale » \\[0pt]
La vraie vie \\[0pt]
L'axiome \\[0pt]
Le bénéfice du doute \\[0pt]
Le bien-être \\[0pt]
Le bon goût \\[0pt]
Le bonheur a-t-il nécessairement un objet ? \\[0pt]
Le bonheur des autres \\[0pt]
Le bonheur est-il affaire de hasard ou de nécessité ? \\[0pt]
Le bonheur est-il la fin de la vie ? \\[0pt]
Le bonheur est-il nécessairement lié au plaisir ? \\[0pt]
Le bonheur est-il un accident ? \\[0pt]
Le bonheur n'est-il qu'un idéal ? \\[0pt]
Le bonheur peut-il être un droit ? \\[0pt]
Le bonheur se fonde-t-il sur un savoir ? \\[0pt]
Le bricolage \\[0pt]
L'échange est-il un facteur de paix ? \\[0pt]
Le citoyen a-t-il perdu toute naturalité ? \\[0pt]
Le commencement \\[0pt]
Le commerce adoucit-il les mœurs ? \\[0pt]
Le commerce équitable \\[0pt]
Le commerce peut-il être équitable ? \\[0pt]
Le complexe \\[0pt]
Le conflit de devoirs \\[0pt]
Le conflit des devoirs \\[0pt]
Le conflit des interprétations \\[0pt]
Le conflit entre la science et la religion est-il inévitable ? \\[0pt]
Le conflit entre l'individu et la société est-il irréductible ? \\[0pt]
Le consentement \\[0pt]
Le contrat \\[0pt]
Le corps du travailleur \\[0pt]
Le corps et le temps \\[0pt]
Le corps et l'instrument \\[0pt]
Le corps n'est-il que matière ? \\[0pt]
Le corps pense-t-il ? \\[0pt]
Le corps politique \\[0pt]
Le cosmopolitisme \\[0pt]
Le courage de la vérité \\[0pt]
Le cours des choses \\[0pt]
L'écriture et la parole \\[0pt]
L'écriture et la pensée \\[0pt]
L'écriture ne sert-elle qu'à consigner la pensée ? \\[0pt]
Le désir de reconnaissance \\[0pt]
Le désir d'éternité \\[0pt]
Le désir de vérité \\[0pt]
Le désir d'immortalité \\[0pt]
Le désir est-il l'essence de l'homme ? \\[0pt]
Le désir n'est-il que l'épreuve d'un manque ? \\[0pt]
Le désir peut-il se satisfaire de la réalité ? \\[0pt]
Le désœuvrement \\[0pt]
Le despotisme peut-il être éclairé ? \\[0pt]
Le destin \\[0pt]
Le devenir \\[0pt]
Le devoir d'aimer \\[0pt]
Le devoir et la dette \\[0pt]
Le devoir se présente-t-il avec la force de l'évidence ? \\[0pt]
Le dilemme \\[0pt]
Le droit à la citoyenneté \\[0pt]
Le droit à l'erreur \\[0pt]
Le droit au bonheur \\[0pt]
Le droit au respect de la vie privée \\[0pt]
Le droit au travail \\[0pt]
Le droit de vivre \\[0pt]
Le droit doit-il être moral ? \\[0pt]
Le droit exclut-il la force ? \\[0pt]
Le droit peut-il se fonder sur la force ? \\[0pt]
Le fait d'exister \\[0pt]
Le fait social est-il une chose ? \\[0pt]
Le fanatisme \\[0pt]
Le féminisme \\[0pt]
Le fond et la forme \\[0pt]
Le futur est-il contingent ? \\[0pt]
L'égalité devant la loi \\[0pt]
L'égalité peut-elle être une menace pour la liberté ? \\[0pt]
Le gouvernement de soi et des autres \\[0pt]
Le hasard \\[0pt]
Le hasard n'est-il que la mesure de notre ignorance ? \\[0pt]
Le jugement critique peut-il s'exercer sans culture ? \\[0pt]
Le juste et le vrai \\[0pt]
Le laboratoire \\[0pt]
Le langage animal \\[0pt]
Le langage du corps \\[0pt]
Le langage est-il le miroir du monde ? \\[0pt]
Le langage ne sert-il qu'à communiquer ? \\[0pt]
Le libre arbitre \\[0pt]
Le libre échange \\[0pt]
Le lieu de l'esprit \\[0pt]
Le littéral et le figuré \\[0pt]
Le loisir \\[0pt]
L'émancipation \\[0pt]
Le manque de rigueur \\[0pt]
Le matériel \\[0pt]
Le mépris \\[0pt]
Le mien et le tien \\[0pt]
Le mieux est-il l'ennemi du bien ? \\[0pt]
Le miracle \\[0pt]
Le moi \\[0pt]
Le moi est-il haïssable ? \\[0pt]
Le moi peut-il être l'objet d'un savoir ? \\[0pt]
Le monde de la technique \\[0pt]
Le monde du travail \\[0pt]
Le monde peut-il être nouveau ? \\[0pt]
Le mot d'esprit \\[0pt]
L'empathie \\[0pt]
L'emploi du temps \\[0pt]
Le mystère \\[0pt]
L'enfant \\[0pt]
L'engendrement \\[0pt]
L'ennui \\[0pt]
Le non-sens \\[0pt]
Le passé a-t-il plus de réalité que l'avenir ? \\[0pt]
Le passé est-il indépassable ? \\[0pt]
Le passé peut-il être un objet de connaissance ? \\[0pt]
Le paternalisme \\[0pt]
Le patrimoine \\[0pt]
Le peuple a-t-il toujours raison ? \\[0pt]
Le philosophe a-t-il besoin de l'histoire ? \\[0pt]
Le plaisir est-il la fin du désir ? \\[0pt]
Le point de vue \\[0pt]
Le possible \\[0pt]
Le possible et le réel \\[0pt]
Le pouvoir politique peut-il échapper à l'arbitraire ? \\[0pt]
Le présent \\[0pt]
L'épreuve de la liberté \\[0pt]
L'épreuve des faits \\[0pt]
Le principe de raison \\[0pt]
Le privé et le public \\[0pt]
Le prix de la liberté \\[0pt]
Le prix de la vérité \\[0pt]
Le profit est-il la fin de l'échange ? \\[0pt]
Le progrès \\[0pt]
Le provisoire \\[0pt]
L'équité \\[0pt]
L'équivocité \\[0pt]
L'équivocité du langage \\[0pt]
Le quotidien \\[0pt]
Le recours à la force signifie-t-il l'échec de la justice ? \\[0pt]
Le réel et l'idéal \\[0pt]
Le règne de la technique \\[0pt]
Le relativisme \\[0pt]
Le remords \\[0pt]
Le retour à la nature est-il souhaitable ? \\[0pt]
Le retour à l'expérience \\[0pt]
Le risque \\[0pt]
Le risque technique \\[0pt]
L'erreur \\[0pt]
L'erreur et la faute \\[0pt]
L'erreur peut-elle donner un accès à la vérité ? \\[0pt]
Le sacrifice \\[0pt]
Le sacrifice de soi \\[0pt]
Le savoir-faire \\[0pt]
Les beaux-arts \\[0pt]
Les bornes de la conscience \\[0pt]
Les cérémonies \\[0pt]
Les cinq sens \\[0pt]
Les clichés \\[0pt]
Les commandements divins \\[0pt]
Le scrupule \\[0pt]
Les devoirs envers soi-même \\[0pt]
Les échanges, facteurs de paix ? \\[0pt]
Le secret d'État \\[0pt]
Le sens de l'État \\[0pt]
Le sens de l'existence \\[0pt]
Le sens de l'histoire \\[0pt]
Le sens du silence \\[0pt]
Le sensible est-il communicable ? \\[0pt]
Le sensible est-il irréductible à l'intelligible ? \\[0pt]
Les faits parlent-ils d'eux-mêmes ? \\[0pt]
Les faits sont-ils têtus ? \\[0pt]
Les grands hommes \\[0pt]
Les hommes sont-ils faits pour s'entendre ? \\[0pt]
Les idées ont-elles une histoire ? \\[0pt]
Les idées ont-elles une réalité ? \\[0pt]
Les idées reçues \\[0pt]
Le silence \\[0pt]
Les leçons de l'expérience \\[0pt]
Les limites de la vérité \\[0pt]
Les limites de l'interprétation \\[0pt]
Les limites de l'obéissance \\[0pt]
Les limites du réel \\[0pt]
Les limites du vivant \\[0pt]
Les lois nous rendent-elles meilleurs ? \\[0pt]
Les lois suffisent-elles à garantir nos libertés ? \\[0pt]
Les maladies de l'esprit \\[0pt]
Les matériaux \\[0pt]
Les mots disent-ils les choses ? \\[0pt]
Les mots et les concepts \\[0pt]
Les mots n'ont-ils que le pouvoir qu'on leur donne ? \\[0pt]
Les objets techniques nous imposent-ils une manière de vivre ? \\[0pt]
Le sommeil de la raison \\[0pt]
Le souci de l'avenir \\[0pt]
Le souci de soi est-il une attitude morale ? \\[0pt]
L'espace de la perception \\[0pt]
Les passions sont-elles un obstacle à la vie sociale ? \\[0pt]
L'espérance \\[0pt]
Les phénomènes inconscients sont-ils réductibles à une mécanique cérébrale ? \\[0pt]
Le spirituel et le temporel \\[0pt]
Les principes et les éléments \\[0pt]
Les principes sont-ils indémontrables ? \\[0pt]
L'esprit d'invention \\[0pt]
L'esprit est-il plus aisé à connaître que le corps ? \\[0pt]
L'esprit est-il une machine ? \\[0pt]
L'esprit est-il un ensemble de facultés ? \\[0pt]
L'esprit n'a-t-il jamais affaire qu'à lui-même ? \\[0pt]
L'esprit peut-il être objet de science ? \\[0pt]
L'esprit s'explique-t-il par une activité cérébrale ? \\[0pt]
L'esprit tranquille \\[0pt]
Les règles de l'art \\[0pt]
Les religions sont-elles des illusions ? \\[0pt]
Les rêves ont-ils un sens ? \\[0pt]
Les riches et les pauvres \\[0pt]
Les sciences ne sont-elles qu'une description du monde ? \\[0pt]
Les techniques du corps \\[0pt]
Les théories scientifiques décrivent-elles la réalité ? \\[0pt]
Les théories scientifiques sont-elles vraies ? \\[0pt]
L'estime de soi \\[0pt]
Le sujet de droit \\[0pt]
Le sujet moral \\[0pt]
Les vertus de l'amour \\[0pt]
Les vivants \\[0pt]
Le symbole \\[0pt]
Le talent et le génie \\[0pt]
L'État est-il appelé à disparaître ? \\[0pt]
L'État est-il un moindre mal ? \\[0pt]
L'État et la guerre \\[0pt]
L'État mondial \\[0pt]
L'État peut-il être libéral ? \\[0pt]
L'État-Providence \\[0pt]
Le témoignage \\[0pt]
Le temps de la liberté \\[0pt]
Le temps de la science \\[0pt]
Le temps de vivre \\[0pt]
Le temps du désir \\[0pt]
Le temps est-il essentiellement destructeur ? \\[0pt]
Le temps est-il notre allié ? \\[0pt]
Le temps est-il notre ennemi ? \\[0pt]
Le temps passe-t-il ? \\[0pt]
Le temps peut-il s'arrêter ? \\[0pt]
L'éternité \\[0pt]
L'ethnocentrisme \\[0pt]
L'étranger \\[0pt]
Le travail est-il une fin ? \\[0pt]
Le travail et l'œuvre \\[0pt]
Le travail intellectuel \\[0pt]
Le travail nous libère-t-il ? \\[0pt]
Le travail nous rend-il heureux ? \\[0pt]
Le travail nous rend-il solidaires ? \\[0pt]
Le travail sur soi \\[0pt]
L'être et le devoir-être \\[0pt]
Le tribunal de l'histoire \\[0pt]
L'eugénisme \\[0pt]
L'événement \\[0pt]
L'évidence \\[0pt]
L'évidence a-t-elle une valeur absolue ? \\[0pt]
L'évidence est-elle toujours un critère de vérité ? \\[0pt]
Le virtuel \\[0pt]
Le vivant a-t-il des droits ? \\[0pt]
Le vivant est-il entièrement connaissable ? \\[0pt]
L'évolution des sociétés dépend-elle du progrès technique \\[0pt]
Le vrai doit-il être démontré ? \\[0pt]
Le vrai et le réel \\[0pt]
Le vrai se perçoit-il ? \\[0pt]
L'exactitude \\[0pt]
L'existence de Dieu \\[0pt]
L'existence est-elle un jeu ? \\[0pt]
L'expérience de la liberté \\[0pt]
L'expérience du présent \\[0pt]
L'expérience, est-ce l'observation ? \\[0pt]
L'expérimentation \\[0pt]
L'expérimentation sur l'être humain \\[0pt]
L'expression du désir \\[0pt]
L'habileté et la prudence \\[0pt]
L'histoire est-elle avant tout mémoire ? \\[0pt]
L'histoire est-elle le règne du hasard ? \\[0pt]
L'histoire est-elle tragique ? \\[0pt]
L'histoire est-elle un genre littéraire ? \\[0pt]
L'histoire est-elle utile ? \\[0pt]
L'histoire n'est-elle que bruit et fureur ? \\[0pt]
L'histoire peut-elle être universelle ? \\[0pt]
L'historien peut-il se passer du concept de causalité ? \\[0pt]
L'homme est-il religieux par nature ? \\[0pt]
L'homme est-il un animal dénaturé ? \\[0pt]
L'homme est-il un être de devoir ? \\[0pt]
L'homme est-il un être social par nature ? \\[0pt]
L'homme est-il vraiment un animal politique ? \\[0pt]
L'homme injuste peut-il être heureux ? \\[0pt]
L'homme libre est-il un homme seul ? \\[0pt]
L'horizon de la mort \\[0pt]
L'hospitalité \\[0pt]
Libéral et libertaire \\[0pt]
Liberté et habitude \\[0pt]
Liberté et responsabilité \\[0pt]
Libre arbitre et liberté \\[0pt]
L'idéal démonstratif \\[0pt]
L'idée de continuité \\[0pt]
L'idée d'éternité \\[0pt]
L'identité personnelle est-elle donnée ou construite ? \\[0pt]
L'identité relève-t-elle du champ politique ? \\[0pt]
L'image \\[0pt]
L'imaginaire \\[0pt]
L'imagination et la raison \\[0pt]
L'imitation \\[0pt]
L'immatériel \\[0pt]
L'immoralité \\[0pt]
L'impartialité des historiens \\[0pt]
L'impossible \\[0pt]
L'imprévu \\[0pt]
L'impuissance de l'État \\[0pt]
L'incommensurable \\[0pt]
L'inconscience \\[0pt]
L'inconscient a-t-il son propre langage ? \\[0pt]
L'inconscient a-t-il une histoire ? \\[0pt]
L'inconscient est-il l'animal en nous ? \\[0pt]
L'inconscient est-il pure négation de la conscience ? \\[0pt]
L'inconscient est-il une découverte ou une invention ? \\[0pt]
L'inconscient est-il une dimension de la conscience ? \\[0pt]
L'inconscient n'est-il qu'un défaut de conscience ? \\[0pt]
L'inconscient nous révèle-t-il à nous-même ? \\[0pt]
L'incrédulité \\[0pt]
L'inculture \\[0pt]
L'indépendance \\[0pt]
L'indésirable \\[0pt]
L'indice \\[0pt]
L'indicible \\[0pt]
L'indifférence \\[0pt]
L'indiscutable \\[0pt]
L'individu \\[0pt]
L'individualisme \\[0pt]
L'individu face à L'État \\[0pt]
L'inégalité est-elle nécessairement injuste ? \\[0pt]
L'innocence \\[0pt]
L'inquiétude \\[0pt]
L'insatisfaction \\[0pt]
L'insoluble \\[0pt]
L'inspiration \\[0pt]
L'instant \\[0pt]
L'instrument \\[0pt]
L'intemporel \\[0pt]
L'intention \\[0pt]
L'intérêt peut-il être une valeur morale ? \\[0pt]
L'intériorité est-elle un mythe ? \\[0pt]
L'interprétation de la nature \\[0pt]
L'interprétation est-elle sans fin ? \\[0pt]
L'introspection \\[0pt]
L'intuition \\[0pt]
L'irrationnel \\[0pt]
L'irréfutable \\[0pt]
L'irresponsabilité \\[0pt]
L'irréversible \\[0pt]
L'obéissance peut-elle être un acte de liberté ? \\[0pt]
L'objectivité \\[0pt]
L'objet de la psychologie \\[0pt]
L'objet du désir \\[0pt]
L'objet du désir en est-il la cause ? \\[0pt]
L'obligation morale \\[0pt]
L'œuvre d'art est-elle intemporelle ? \\[0pt]
L'œuvre d'art et le plaisir \\[0pt]
L'œuvre du temps \\[0pt]
Logique et réalité \\[0pt]
Lois naturelles et lois civiles \\[0pt]
L'opinion publique \\[0pt]
L'opinion vraie \\[0pt]
L'oral et l'écrit \\[0pt]
L'ordre des raisons \\[0pt]
L'ordre établi \\[0pt]
L'ordre politique exclut-il la violence ? \\[0pt]
L'organisation \\[0pt]
L'organisation du vivant \\[0pt]
L'origine \\[0pt]
L'origine des valeurs \\[0pt]
L'oubli \\[0pt]
L'oubli est-il nécessaire à la vie ? \\[0pt]
L'outil et la machine \\[0pt]
L'unité de l'État \\[0pt]
L'unité du vivant \\[0pt]
L'universalité du vrai \\[0pt]
L'usure des mots \\[0pt]
L'utopie \\[0pt]
Maîtriser le vivant \\[0pt]
Matière et corps \\[0pt]
Monde naturel et monde humain \\[0pt]
Morale et identité \\[0pt]
Naître et mourir \\[0pt]
Naturel et artificiel \\[0pt]
N'avons-nous affaire qu'au réel ? \\[0pt]
Nécessité fait loi \\[0pt]
N'en faire qu'à sa tête \\[0pt]
Ne sait-on rien que par expérience ? \\[0pt]
Ne sommes-nous véritablement maîtres que de nos pensées ? \\[0pt]
N'existe-t-il qu'un seul temps ? \\[0pt]
Nier la vérité \\[0pt]
Notre rapport au monde peut-il n'être que technique ? \\[0pt]
N'y a-t-il de beauté qu'artistique ? \\[0pt]
Obéir, est-ce se soumettre ? \\[0pt]
Obéissance et servitude \\[0pt]
Œuvrer \\[0pt]
Où commence l'interprétation ? \\[0pt]
Où est-on quand on pense ? \\[0pt]
Parler de soi \\[0pt]
Parler, est-ce ne pas agir ? \\[0pt]
Par où commencer ? \\[0pt]
Partager sa vie \\[0pt]
Partager ses sentiments \\[0pt]
Penser, est-ce dire non ? \\[0pt]
Penser la matière \\[0pt]
Penser par soi-même \\[0pt]
Perception et connaissance \\[0pt]
Perception et création artistique \\[0pt]
Perception et passivité \\[0pt]
Perception et souvenir \\[0pt]
Perception et vérité \\[0pt]
Percevoir, est-ce connaître ? \\[0pt]
Percevoir, est-ce interpréter ? \\[0pt]
Percevoir, est-ce juger ? \\[0pt]
Percevoir, est-ce reconnaître ? \\[0pt]
Percevoir et juger \\[0pt]
Percevoir s'apprend-il ? \\[0pt]
Perçoit-on les choses comme elles sont ? \\[0pt]
Perdre connaissance \\[0pt]
Perdre la raison \\[0pt]
Peut-on agir sans raison ? \\[0pt]
Peut-on apprendre à être heureux ? \\[0pt]
Peut-on apprendre à être libre ? \\[0pt]
Peut-on arrêter le progrès ? \\[0pt]
Peut-on avoir le droit de se révolter ? \\[0pt]
Peut-on concevoir une science qui ne soit pas démonstrative ? \\[0pt]
Peut-on considérer la conscience comme une chose ? \\[0pt]
Peut-on convaincre quelqu'un de la beauté d'une œuvre d'art ? \\[0pt]
Peut-on croire ce qu'on veut ? \\[0pt]
Peut-on croire sans être crédule ? \\[0pt]
Peut-on croire sans savoir pourquoi ? \\[0pt]
Peut-on définir la vie ? \\[0pt]
Peut-on démontrer qu'on ne rêve pas ? \\[0pt]
Peut-on désirer ce qu'on possède ? \\[0pt]
Peut-on dialoguer avec soi-même ? \\[0pt]
Peut-on dire que rien n'échappe à la technique ? \\[0pt]
Peut-on dire que tout est politique ? \\[0pt]
Peut-on distinguer le réel de l'imaginaire ? \\[0pt]
Peut-on distinguer les faits de leurs interprétations ? \\[0pt]
Peut-on être aliéné sans le savoir ? \\[0pt]
Peut-on être citoyen du monde ? \\[0pt]
Peut-on être en conflit avec soi-même ? \\[0pt]
Peut-on être heureux sans le savoir ? \\[0pt]
Peut-on être heureux tout seul ? \\[0pt]
Peut-on être injuste et heureux ? \\[0pt]
Peut-on être libre sans le savoir ? \\[0pt]
Peut-on être seul ? \\[0pt]
Peut-on être soi-même en société ? \\[0pt]
Peut-on être trop religieux ? \\[0pt]
Peut-on expliquer le mal ? \\[0pt]
Peut-on expliquer le monde par la matière ? \\[0pt]
Peut-on expliquer une œuvre d'art ? \\[0pt]
Peut-on faire de sa vie une œuvre d'art ? \\[0pt]
Peut-on fonder la morale sur la pitié ? \\[0pt]
Peut-on guérir par la parole ? \\[0pt]
Peut-on interpréter la nature ? \\[0pt]
Peut-on lutter contre le destin ? \\[0pt]
Peut-on lutter contre soi-même ? \\[0pt]
Peut-on maîtriser l'inconscient ? \\[0pt]
Peut-on manquer de culture ? \\[0pt]
Peut-on ne pas interpréter ? \\[0pt]
Peut-on ne pas savoir ce que l'on veut ? \\[0pt]
Peut-on ne pas vouloir être heureux ? \\[0pt]
Peut-on ne rien devoir à personne ? \\[0pt]
Peut-on nier la réalité ? \\[0pt]
Peut-on nier l'évidence ? \\[0pt]
Peut-on nier l'existence de la matière ? \\[0pt]
Peut-on opposer nature et culture ? \\[0pt]
Peut-on parler de violence d'État ? \\[0pt]
Peut-on parler d'un droit de résistance ? \\[0pt]
Peut-on partager une expérience ? \\[0pt]
Peut-on penser contre l'expérience ? \\[0pt]
Peut-on penser la matière ? \\[0pt]
Peut-on penser l'art sans référence au beau ? \\[0pt]
Peut-on penser le monde sans la technique ? \\[0pt]
Peut-on penser le temps sans l'espace ? \\[0pt]
Peut-on penser l'œuvre d'art sans référence à l'idée de beauté ? \\[0pt]
Peut-on penser sans son corps ? \\[0pt]
Peut-on penser une conscience sans mémoire ? \\[0pt]
Peut-on penser une conscience sans objet ? \\[0pt]
Peut-on penser une religion sans le recours au divin ? \\[0pt]
Peut-on perdre sa dignité ? \\[0pt]
Peut-on préférer l'ordre à la justice ? \\[0pt]
Peut-on prouver l'existence ? \\[0pt]
Peut-on reconnaître un sens à l'histoire sans lui assigner une fin ? \\[0pt]
Peut-on réduire l'esprit à la matière ? \\[0pt]
Peut-on réduire un homme à la somme de ses actes ? \\[0pt]
Peut-on refaire le monde ? \\[0pt]
Peut-on renoncer à ses droits ? \\[0pt]
Peut-on renoncer au bonheur ? \\[0pt]
Peut-on résister à la vérité ? \\[0pt]
Peut-on rester sceptique ? \\[0pt]
Peut-on s'affranchir de sa propre nature ? \\[0pt]
Peut-on se connaître soi-même ? \\[0pt]
Peut-on se fier à sa propre raison ? \\[0pt]
Peut-on se mentir à soi-même ? \\[0pt]
Peut-on se passer de croire ? \\[0pt]
Peut-on se passer de spiritualité ? \\[0pt]
Peut-on se passer d'un maître ? \\[0pt]
Peut-on suspendre le temps ? \\[0pt]
Peut-on toujours savoir entièrement ce que l'on dit ? \\[0pt]
Peut-on tout échanger ? \\[0pt]
Peut-on tout prévoir ? \\[0pt]
Peut-on transformer le réel ? \\[0pt]
Peut-on vivre en marge de la société ? \\[0pt]
Peut-on vivre en paix avec son inconscient ? \\[0pt]
Peut-on vivre hors du temps ? \\[0pt]
Peut-on vivre sans rien espérer ? \\[0pt]
Peut-on vouloir le mal ? \\[0pt]
Peut-on vouloir le mal sachant que c'est le mal ? \\[0pt]
Peut-on vouloir sans désirer ? \\[0pt]
Peut-on vraiment tirer des leçons du passé ? \\[0pt]
Photographier le réel \\[0pt]
Pour être heureux, faut-il renoncer à la perfection ? \\[0pt]
Pourquoi démontrer ? \\[0pt]
Pourquoi démontrer ce que l'on sait être vrai ? \\[0pt]
Pourquoi des musées ? \\[0pt]
Pourquoi des rites ? \\[0pt]
Pourquoi des utopies ? \\[0pt]
Pourquoi écrire ? \\[0pt]
Pourquoi écrit-on les lois ? \\[0pt]
Pourquoi faire l'hypothèse de l'inconscient ? \\[0pt]
Pourquoi la guerre ? \\[0pt]
Pourquoi la réalité peut-elle dépasser la fiction ? \\[0pt]
Pourquoi l'économie est-elle politique ? \\[0pt]
Pourquoi les hommes mentent-ils ? \\[0pt]
Pourquoi obéir ? \\[0pt]
Pourquoi obéit-on aux lois ? \\[0pt]
Pourquoi parler de fautes de goût ? \\[0pt]
Pourquoi penser à la mort ? \\[0pt]
Pourquoi punir ? \\[0pt]
Pourquoi rechercher le bonheur ? \\[0pt]
Pourquoi se cultiver ? \\[0pt]
Pourquoi s'exprimer ? \\[0pt]
Pourquoi transformer le monde ? \\[0pt]
Pourquoi travailler ? \\[0pt]
Pourquoi y a-t-il des religions ? \\[0pt]
Pouvons-nous communiquer ce que nous sentons ? \\[0pt]
Pouvons-nous réellement tirer des leçons du passé ? \\[0pt]
Prévoir \\[0pt]
Prouver et justifier \\[0pt]
Prouvez-le ! \\[0pt]
Puis-je comprendre autrui ? \\[0pt]
Puis-je devenir moi-même ? \\[0pt]
Puis-je douter de ma propre existence ? \\[0pt]
Quand faut-il désobéir ? \\[0pt]
Quand faut-il mentir ? \\[0pt]
Quand peut-on se passer de théories ? \\[0pt]
Quand y a-t-il de l'art ? \\[0pt]
Qu'attendre de l'État ? \\[0pt]
Que devons-nous à l'État ? \\[0pt]
Que dois-je à autrui ? \\[0pt]
Que fait la machine ? \\[0pt]
Que faut-il vouloir pour être heureux ? \\[0pt]
Quel est le sujet du devenir ? \\[0pt]
Quel est l'objet de la perception ? \\[0pt]
Quel est l'objet du désir ? \\[0pt]
Quel être peut être un sujet de droits ? \\[0pt]
Quelle confiance accorder au langage ? \\[0pt]
Quelle est la fin de la science ? \\[0pt]
Quelle est la limite du pouvoir de l'État ? \\[0pt]
Quelle réalité peut-on attribuer au temps ? \\[0pt]
Quelle vérité y a-t-il dans la perception ? \\[0pt]
Que nous apprend la poésie ? \\[0pt]
Que nous apprend la religion ? \\[0pt]
Que nous apprend l'expérience ? \\[0pt]
Que nous enseignent nos peurs ? \\[0pt]
Que nous impose la nature ? \\[0pt]
Que percevons-nous ? \\[0pt]
Que percevons-nous du monde extérieur ? \\[0pt]
Que perçoit-on ? \\[0pt]
Que peut expliquer l'histoire ? \\[0pt]
Que peut la raison ? \\[0pt]
Que peut la raison contre une croyance ? \\[0pt]
Que peut-on cultiver ? \\[0pt]
Que peut-on sur autrui ? \\[0pt]
Que produit la poésie ? \\[0pt]
Que produit l'inconscient ? \\[0pt]
Que sait-on de soi ? \\[0pt]
Que savons-nous de l'inconscient ? \\[0pt]
Que serait la fin de l'histoire ? \\[0pt]
Qu'est-ce qu'avoir un esprit scientifique ? \\[0pt]
Qu'est-ce que le cinéma a changé dans l'idée que l'on se fait du temps ? \\[0pt]
Qu'est-ce que savoir ? \\[0pt]
Qu'est-ce que traduire ? \\[0pt]
Qu'est-ce qu'être asocial ? \\[0pt]
Qu'est-ce qu'être ensemble ? \\[0pt]
Qu'est-ce qu'être moderne ? \\[0pt]
Qu'est-ce qu'être un sujet ? \\[0pt]
Qu'est-ce qui est contre nature ? \\[0pt]
Qu'est-ce qui est culturel ? \\[0pt]
Qu'est-ce qui est historique ? \\[0pt]
Qu'est-ce qui est hors la loi ? \\[0pt]
Qu'est-ce qui est public ? \\[0pt]
Qu'est-ce qui est sensible ? \\[0pt]
Qu'est-ce qui fait la valeur d'une œuvre d'art ? \\[0pt]
Qu'est-ce qui fait qu'une théorie est vraie ? \\[0pt]
Qu'est-ce qui fonde la croyance ? \\[0pt]
Qu'est-ce qui justifie l'hypothèse d'un inconscient ? \\[0pt]
Qu'est-ce qui ne s'achète pas ? \\[0pt]
Qu'est-ce qui ne s'échange pas ? \\[0pt]
Qu'est-ce qui n'est pas démontrable ? \\[0pt]
Qu'est-ce qui n'existe pas ? \\[0pt]
Qu'est-ce qu'un acte libre ? \\[0pt]
Qu'est-ce qu'un artiste ? \\[0pt]
Qu'est-ce qu'un axiome ? \\[0pt]
Qu'est-ce qu'un beau parleur ? \\[0pt]
Qu'est-ce qu'un cas de conscience ? \\[0pt]
Qu'est-ce qu'un chef-d'œuvre ? \\[0pt]
Qu'est-ce qu'un échange juste ? \\[0pt]
Qu'est-ce qu'une conscience collective ? \\[0pt]
Qu'est-ce qu'une constitution ? \\[0pt]
Qu'est-ce qu'une croyance rationnelle ? \\[0pt]
Qu'est-ce qu'une époque ? \\[0pt]
Qu'est-ce qu'une erreur ? \\[0pt]
Qu'est-ce qu'une existence historique ? \\[0pt]
Qu'est-ce qu'une expérience de pensée ? \\[0pt]
Qu'est-ce qu'une explication matérialiste ? \\[0pt]
Qu'est-ce qu'une guerre juste ? \\[0pt]
Qu'est-ce qu'une histoire vraie ? \\[0pt]
Qu'est-ce qu'une hypothèse ? \\[0pt]
Qu'est-ce qu'une idée vraie ? \\[0pt]
Qu'est-ce qu'une langue bien faite ? \\[0pt]
Qu'est-ce qu'une marchandise ? \\[0pt]
Qu'est-ce qu'une parole en l'air ? \\[0pt]
Qu'est-ce qu'une patrie ? \\[0pt]
Qu'est-ce qu'une personne morale ? \\[0pt]
Qu'est-ce qu'une preuve ? \\[0pt]
Qu'est-ce qu'une preuve scientifique ? \\[0pt]
Qu'est-ce qu'une révolution artistique ? \\[0pt]
Qu'est-ce qu'une révolution politique ? \\[0pt]
Qu'est-ce qu'une science exacte ? \\[0pt]
Qu'est-ce qu'une société juste ? \\[0pt]
Qu'est-ce qu'une société ouverte ? \\[0pt]
Qu'est-ce qu'un esprit faux ? \\[0pt]
Qu'est-ce qu'un esprit libre ? \\[0pt]
Qu'est-ce qu'une théorie ? \\[0pt]
Qu'est-ce qu'un événement ? \\[0pt]
Qu'est-ce qu'une vie d'artiste ? \\[0pt]
Qu'est-ce qu'un fait scientifique ? \\[0pt]
Qu'est-ce qu'un geste technique ? \\[0pt]
Qu'est-ce qu'un gouvernement juste ? \\[0pt]
Qu'est-ce qu'un gouvernement républicain ? \\[0pt]
Qu'est-ce qu'un langage technique ? \\[0pt]
Qu'est-ce qu'un métier ? \\[0pt]
Qu'est-ce qu'un mineur ? \\[0pt]
Qu'est-ce qu'un monstre ? \\[0pt]
Qu'est-ce qu'un objet mathématique ? \\[0pt]
Qu'est-ce qu'un peuple ? \\[0pt]
Qu'est-ce qu'un rival ? \\[0pt]
Qu'est-ce qu'un sentiment vrai ? \\[0pt]
Qu'est-ce qu'un tableau ? \\[0pt]
Qu'est-ce qu'un visage ? \\[0pt]
Qu'est-ce qu'un vrai changement ? \\[0pt]
Que vaut le travail ? \\[0pt]
Que veut dire : « être cultivé » ? \\[0pt]
Qui croire ? \\[0pt]
Qui doit faire les lois ? \\[0pt]
Qui écrit l'histoire ? \\[0pt]
Qui est autorisé à me dire « tu dois » ? \\[0pt]
Qui est mon prochain ? \\[0pt]
Qui fait l'unité du peuple ? \\[0pt]
Qui pense ? \\[0pt]
Qui suis-je ? \\[0pt]
Qui travaille ? \\[0pt]
Qu'oppose-t-on à la vérité ? \\[0pt]
Qu'y a-t-il au-delà du réel ? \\[0pt]
Qu'y a-t-il d'universel dans la culture ? \\[0pt]
Raison et langage \\[0pt]
Raisonner par l'absurde \\[0pt]
Réfuter une théorie \\[0pt]
Regarder un tableau \\[0pt]
Religion et liberté \\[0pt]
Rendre raison \\[0pt]
Responsabilité et culpabilité \\[0pt]
Rester soi-même \\[0pt]
Revenir à la nature \\[0pt]
Rêver \\[0pt]
« Rien n'est sans raison » \\[0pt]
S'arracher à la nature \\[0pt]
Savoir démontrer \\[0pt]
Savoir et faire \\[0pt]
Savoir et pouvoir \\[0pt]
Savoir quoi faire \\[0pt]
Savons-nous ce que nous disons ? \\[0pt]
Savons-nous ce qui nous rend heureux ? \\[0pt]
Science et expérience \\[0pt]
Science et religion \\[0pt]
Science sans conscience n'est-elle que ruine de l'âme ? \\[0pt]
Sciences de la nature et sciences de l'esprit \\[0pt]
Se connaître soi-même \\[0pt]
Se cultiver \\[0pt]
S'émanciper de ses maîtres \\[0pt]
Sensation et perception \\[0pt]
Sens et vérité \\[0pt]
Se révolter \\[0pt]
Se savoir mortel \\[0pt]
Se sentir libre implique-t-il qu'on le soit ? \\[0pt]
Se taire \\[0pt]
Signe et symbole \\[0pt]
Soigner \\[0pt]
Sommes-nous conscients de nos mobiles ? \\[0pt]
Sommes-nous faits pour la vérité ? \\[0pt]
Sommes-nous faits pour vivre en société ? \\[0pt]
Sommes-nous les jouets de l'histoire ? \\[0pt]
Sommes-nous maîtres de nos paroles ? \\[0pt]
Souffrir \\[0pt]
Suffit-il de bien juger pour bien faire ? \\[0pt]
Suffit-il de faire son devoir ? \\[0pt]
Suffit-il de suivre ses convictions pour bien agir ? \\[0pt]
Suffit-il, pour croire, de le vouloir ? \\[0pt]
Suis-je maître de ma conscience ? \\[0pt]
Sujet et substance \\[0pt]
Sur quoi fonder la légitimité de la loi ? \\[0pt]
Sur quoi fonder la société ? \\[0pt]
Sur quoi fonder le droit de punir ? \\[0pt]
Suspendre son jugement \\[0pt]
Système et structure \\[0pt]
Temps et conscience \\[0pt]
Temps individuel et temps collectif \\[0pt]
Tirer des leçons du passé \\[0pt]
Tous les hommes désirent-ils être heureux ? \\[0pt]
Toute conscience n'est-elle pas implicitement morale ? \\[0pt]
Toute existence est-elle indémontrable ? \\[0pt]
Toute expérience appelle-t-elle une interprétation ? \\[0pt]
Toute passion fait-elle souffrir ? \\[0pt]
Toutes les croyances se valent-elles ? \\[0pt]
Toute vérité est-elle vérifiable ? \\[0pt]
Tout peut-il se vendre ? \\[0pt]
Tout savoir est-il pouvoir ? \\[0pt]
Tout savoir est-il transmissible ? \\[0pt]
Tout travail mérite salaire \\[0pt]
Tradition et raison \\[0pt]
Traiter des faits humains comme des choses, est-ce considérer l'homme comme une chose ? \\[0pt]
Traiter les faits humains comme des choses, est-ce réduire les hommes à des choses ? \\[0pt]
Travail et plaisir \\[0pt]
Travailler par plaisir, est-ce encore travailler ? \\[0pt]
Travailler sans souffrir peut-il être un idéal ? \\[0pt]
Travail manuel et travail intellectuel \\[0pt]
Travail manuel, travail intellectuel \\[0pt]
Tuer le temps \\[0pt]
Un acte inconscient est-il nécessairement un acte involontaire ? \\[0pt]
Un acte libre est-il un acte imprévisible ? \\[0pt]
Un contrat peut-il être social ? \\[0pt]
Une croyance infondée est-elle illégitime ? \\[0pt]
Une existence se démontre-t-elle ? \\[0pt]
Une fiction peut-elle être vraie ? \\[0pt]
Une foi rationnelle \\[0pt]
Une idée peut-elle être fausse ? \\[0pt]
Une injustice vaut elle mieux qu'un désordre ? \\[0pt]
Une interprétation est-elle nécessairement subjective ? \\[0pt]
Une œuvre d'art peut-elle être immorale ? \\[0pt]
Une religion peut-elle être fausse ? \\[0pt]
Une société n'est-elle qu'un ensemble d'individus ? \\[0pt]
Une société sans religion est-elle possible ? \\[0pt]
Un État mondial ? \\[0pt]
Une vérité sans preuve est-elle une vérité ? \\[0pt]
Un fait scientifique doit-il être nécessairement démontré ? \\[0pt]
Un peuple peut-il être sans histoire ? \\[0pt]
Un travail bien fait est-il nécessairement un bon travail ? \\[0pt]
Valeur artistique, valeur esthétique \\[0pt]
Vérité et certitude \\[0pt]
Vérité et réalité \\[0pt]
Vieillir \\[0pt]
Violence et histoire \\[0pt]
Vivre au jour le jour \\[0pt]
Vivre au présent \\[0pt]
Vivre dans son monde \\[0pt]
Vivre, est-ce interpréter ? \\[0pt]
Vivre, est-ce lutter contre la mort ? \\[0pt]
Vivre pleinement \\[0pt]
Vivre sans mémoire, est-ce être libre ? \\[0pt]
Vivre sans religion, est-ce vivre sans espoir ? \\[0pt]
Vivre sa vie \\[0pt]
Voir le bien, faire le mal \\[0pt]
Voir le meilleur et faire le pire \\[0pt]
Voir le meilleur, faire le pire \\[0pt]
Vouloir et pouvoir \\[0pt]
Vouloir le mal \\[0pt]
Voulons-nous la vérité ? \\[0pt]
Voyager dans le temps \\[0pt]
Y a-t-il autant de mondes que de cultures ? \\[0pt]
Y a-t-il de la raison dans la perception ? \\[0pt]
Y a-t-il de l'impensable ? \\[0pt]
Y a-t-il de l'indémontrable ? \\[0pt]
Y a-t-il de l'inexprimable ? \\[0pt]
Y a-t-il des actes désintéressés ? \\[0pt]
Y a-t-il des arts mineurs ? \\[0pt]
Y a-t-il des devoirs envers soi-même ? \\[0pt]
Y a-t-il des expériences absolument certaines ? \\[0pt]
Y a-t-il des guerres justes ? \\[0pt]
Y a-t-il des inégalités justes ? \\[0pt]
Y a-t-il des jours où je n'existe pas ? \\[0pt]
Y a-t-il des lois du vivant ? \\[0pt]
Y a-t-il des méthodes pour être heureux ? \\[0pt]
Y a-t-il des modèles en morale ? \\[0pt]
Y a-t-il de sots métiers ? \\[0pt]
Y a-t-il des peuples sans histoire ? \\[0pt]
Y a-t-il des réalités mathématiques ? \\[0pt]
Y a-t-il des règles de l'art ? \\[0pt]
Y a-t-il des vérités philosophiques ? \\[0pt]
Y a-t-il des vérités sans preuve ? \\[0pt]
Y a-t-il du non-être ? \\[0pt]
Y a-t-il lieu d'opposer matière et esprit ? \\[0pt]
Y a-t-il un art total ? \\[0pt]
Y a-t-il un au-delà de la vérité ? \\[0pt]
Y a-t-il un auteur de l'histoire ? \\[0pt]
Y a-t-il un devoir de travailler ? \\[0pt]
Y a-t-il un droit à la différence ? \\[0pt]
Y a-t-il un droit du plus fort ? \\[0pt]
Y a-t-il une bonne imitation ? \\[0pt]
Y a-t-il une connaissance du singulier ? \\[0pt]
Y a-t-il une expérience de l'éternité ? \\[0pt]
Y a-t-il une expérience du néant ? \\[0pt]
Y a-t-il une force du droit ? \\[0pt]
Y a-t-il une histoire de la vérité ? \\[0pt]
Y a-t-il une logique de l'inconscient ? \\[0pt]
Y a-t-il une logique des événements historiques ? \\[0pt]
Y a-t-il une raison à tout ? \\[0pt]
Y a-t-il une rationalité dans la religion ? \\[0pt]
Y a-t-il une réalité du hasard ? \\[0pt]
Y a-t-il une science de la vie ? \\[0pt]
Y a-t-il une science de l'homme ? \\[0pt]
Y a-t-il une science politique ? \\[0pt]
Y a-t-il un esprit scientifique ? \\[0pt]
Y a-t-il une technique de la nature ? \\[0pt]
Y a-t-il une vérité de la fiction ? \\[0pt]
Y a-t-il une vérité de l'inconscient ? \\[0pt]
Y a-t-il une vérité des sentiments ? \\[0pt]
Y a-t-il une vérité en histoire ? \\[0pt]
Y a-t-il une vertu de l'ignorance \\[0pt]
Y a-t-il un inconscient psychique ? \\[0pt]
Y a-t-il un inconscient social ? \\[0pt]
Y a-t-il un langage de l'inconscient ? \\[0pt]
Y a-t-il un savoir inconscient ? \\[0pt]
Y a-t-il un sens du beau ? \\[0pt]
Y a-t-il un sens moral ? \\[0pt]
Y a-t-il un temps pour tout ? \\[0pt]


\subsection{CAPES}
\label{sec:orgcebdb34}

\noindent
Abstraire, est-ce se couper du réel ? \\[0pt]
« À chacun sa morale » \\[0pt]
À chacun sa vérité \\[0pt]
À chacun sa vérité ? \\[0pt]
À chacun selon son mérite \\[0pt]
Action et contemplation \\[0pt]
Activité et passivité \\[0pt]
Agir \\[0pt]
Agir en politique, est-ce agir dans l'incertain ? \\[0pt]
Agir et faire \\[0pt]
Agir et réagir \\[0pt]
Agir par devoir, est-ce agir contre son intérêt ? \\[0pt]
Agir volontairement, est-ce nécessairement savoir ce que l'on fait ? \\[0pt]
Ai-je des devoirs envers moi-même ? \\[0pt]
Ai-je intérêt à la liberté d'autrui ? \\[0pt]
Ai-je le pouvoir de me changer ? \\[0pt]
Ai-je un corps ou suis-je mon corps ? \\[0pt]
Aimer la vérité \\[0pt]
Aimer le monde tel qu'il est. \\[0pt]
Aimer l'éphémère \\[0pt]
Aimer peut-il être un devoir ? \\[0pt]
« Aimer » se dit-il en un seul sens ? \\[0pt]
Ami et ennemi \\[0pt]
Amour et inconscient \\[0pt]
Analyse et synthèse \\[0pt]
Analyser \\[0pt]
Apparence et réalité \\[0pt]
Appartenons-nous à une culture ? \\[0pt]
Apprend-on à percevoir ? \\[0pt]
Apprendre à vivre \\[0pt]
Apprendre à voir \\[0pt]
Apprendre à voir ? \\[0pt]
Apprendre et enseigner \\[0pt]
À quelle condition un travail est-il humain ? \\[0pt]
À quelles conditions le vivant peut-il être objet de science ? \\[0pt]
À quelles conditions une activité est-elle un travail ? \\[0pt]
À quelles conditions une croyance devient-elle religieuse ? \\[0pt]
À quelles conditions une démarche est-elle scientifique ? \\[0pt]
À quelles conditions une expérience est-elle possible ? \\[0pt]
À quelles conditions une hypothèse est-elle scientifique ? \\[0pt]
À quelles conditions un État peut-il être juste ? \\[0pt]
À qui doit-on la vérité ? \\[0pt]
À qui doit-on le respect ? \\[0pt]
À qui doit-on obéir ? \\[0pt]
À qui faut-il obéir ? \\[0pt]
À qui la faute ? \\[0pt]
À quoi bon culpabiliser ? \\[0pt]
À quoi bon démontrer ? \\[0pt]
À quoi bon des héros ? \\[0pt]
À quoi bon imiter la nature ? \\[0pt]
À quoi bon la science ? \\[0pt]
À quoi bon parler ? \\[0pt]
À quoi bon se parler ? \\[0pt]
À quoi faut-il renoncer pour dialoguer ? \\[0pt]
À quoi la perception donne-t-elle accès ? \\[0pt]
À quoi la religion sert-elle ? \\[0pt]
À quoi l'art nous rend-il sensibles ? \\[0pt]
À quoi nos illusions tiennent-elles ? \\[0pt]
À quoi peut-on reconnaître la vérité ? \\[0pt]
À quoi peut-on reconnaître une œuvre d'art ? \\[0pt]
À quoi reconnaît-on la vérité ? \\[0pt]
À quoi reconnaît-on le faux ? \\[0pt]
À quoi reconnaît-on le réel ? \\[0pt]
À quoi reconnaît-on l'humanité en chaque homme ? \\[0pt]
À quoi reconnaît-on qu'une activité est un travail ? \\[0pt]
À quoi reconnaît-on qu'une expérience est scientifique ? \\[0pt]
À quoi reconnaît-on qu'une pensée est vraie ? \\[0pt]
À quoi reconnaît-on qu'une science est une science ? \\[0pt]
À quoi reconnaît-on qu'un événement est historique ? \\[0pt]
À quoi reconnaît-on un acte libre ? \\[0pt]
À quoi reconnaît-on un acte vraiment libre ? \\[0pt]
À quoi reconnaît-on un bon artisan ? \\[0pt]
À quoi reconnaît-on un devoir ? \\[0pt]
À quoi reconnaît-on une action morale ? \\[0pt]
À quoi reconnaît-on une attitude religieuse ? \\[0pt]
À quoi reconnaît-on une bonne interprétation ? \\[0pt]
À quoi reconnaît-on une idéologie ? \\[0pt]
À quoi reconnaît-on une œuvre d'art ? \\[0pt]
À quoi reconnaît-on une religion ? \\[0pt]
À quoi reconnaît-on une théorie scientifique ? \\[0pt]
À quoi reconnaît-on un être vivant ? \\[0pt]
À quoi renonçons-nous quand nous obéissons ? \\[0pt]
À quoi sert la notion d'état de nature ? \\[0pt]
À quoi sert la technique ? \\[0pt]
À quoi sert la vérité ? \\[0pt]
À quoi sert l'État ? \\[0pt]
À quoi sert l'histoire ? \\[0pt]
À quoi sert un critique d'art ? \\[0pt]
À quoi servent les images ? \\[0pt]
À quoi servent les lois ? \\[0pt]
À quoi servent les machines ? \\[0pt]
À quoi servent les œuvres d'art ? \\[0pt]
À quoi servent les preuves ? \\[0pt]
À quoi servent les preuves de l'existence de Dieu ? \\[0pt]
À quoi servent les religions ? \\[0pt]
À quoi servent les symboles ? \\[0pt]
À quoi servent les théories ? \\[0pt]
À quoi servent les utopies ? \\[0pt]
À quoi servent les voyages ? \\[0pt]
À quoi tient la force de l'État ? \\[0pt]
À quoi tient la force des religions ? \\[0pt]
À quoi tient la valeur d'une pensée ? \\[0pt]
À quoi tient le pouvoir des mots ? \\[0pt]
À quoi tient le pouvoir des sciences ? \\[0pt]
À quoi tient notre humanité ? \\[0pt]
Argent et liberté \\[0pt]
Argumenter et démontrer \\[0pt]
Art et beauté \\[0pt]
Art et création \\[0pt]
Art et illusion \\[0pt]
Art et imagination \\[0pt]
Art et jeu \\[0pt]
Art et matière \\[0pt]
Art et pouvoir \\[0pt]
Art et représentation \\[0pt]
Art et société \\[0pt]
Art et Société \\[0pt]
Art et symbole \\[0pt]
Art et technique \\[0pt]
Art et vérité \\[0pt]
Aspect et apparence \\[0pt]
A-t-on besoin de certitudes ? \\[0pt]
A-t-on besoin d'experts ? \\[0pt]
A-t-on besoin d'un chef ? \\[0pt]
A-t-on des devoirs envers les animaux ? \\[0pt]
A-t-on des devoirs envers l'humanité ? \\[0pt]
A-t-on des devoirs envers qui n'a aucun droit ? \\[0pt]
A-t-on des devoirs envers soi-même ? \\[0pt]
A-t-on le droit de se désintéresser de la politique ? \\[0pt]
A-t-on le droit de se révolter ? \\[0pt]
A-t-on raison de mettre la technique en accusation ? \\[0pt]
Au nom de qui rend-on justice ? \\[0pt]
Au nom de quoi rend-on justice ? \\[0pt]
Autorité et domination \\[0pt]
Autorité et souveraineté \\[0pt]
Autrui \\[0pt]
Autrui, est-ce n'importe quel autre ? \\[0pt]
Autrui est-il pour moi un mystère ? \\[0pt]
Autrui est-il un autre moi-même ? \\[0pt]
Autrui m'est-il étranger ? \\[0pt]
Avoir de l'expérience \\[0pt]
Avoir des remords \\[0pt]
Avoir du jugement \\[0pt]
Avoir du métier \\[0pt]
Avoir du pouvoir \\[0pt]
Avoir et être \\[0pt]
Avoir la parole, est-ce avoir le pouvoir ? \\[0pt]
Avoir mauvaise conscience \\[0pt]
Avoir raison \\[0pt]
Avoir raison, est-ce nécessairement être raisonnable ? \\[0pt]
Avoir tout pour être heureux \\[0pt]
Avoir un corps \\[0pt]
Avoir un destin \\[0pt]
Avoir un passé, avoir un avenir \\[0pt]
Avons-nous besoin de cérémonies ? \\[0pt]
Avons-nous besoin de Dieu ? \\[0pt]
Avons-nous besoin de héros ? \\[0pt]
Avons-nous besoin de maîtres ? \\[0pt]
Avons-nous besoin de nous tromper nous-même ? \\[0pt]
Avons-nous des devoirs à l'égard de la vérité ? \\[0pt]
Avons-nous des devoirs envers la nature ? \\[0pt]
Avons-nous des devoirs envers le passé ? \\[0pt]
Avons-nous des devoirs envers les autres êtres vivants ? \\[0pt]
Avons-nous des devoirs envers les générations futures ? \\[0pt]
Avons-nous des devoirs envers nous-mêmes ? \\[0pt]
Avons-nous des devoirs envers tous les vivants ? \\[0pt]
Avons-nous des droits sur la nature ? \\[0pt]
Avons-nous intérêt à la liberté d'autrui ? \\[0pt]
Avons-nous le devoir de dire la vérité ? \\[0pt]
Avons-nous le devoir de vivre ? \\[0pt]
Avons-nous le temps d'apprendre à vivre ? \\[0pt]
Avons-nous peur de la liberté ? \\[0pt]
Avons-nous raison de croire ? \\[0pt]
Avons-nous un devoir de vérité ? \\[0pt]
Avons-nous une identité ? \\[0pt]
Beauté et moralité \\[0pt]
Beauté et vérité \\[0pt]
Bêtise et méchanceté \\[0pt]
Bien agir, est-ce nécessairement faire son devoir ? \\[0pt]
Bien agir, est-ce toujours être moral ? \\[0pt]
Bien commun et intérêt particulier \\[0pt]
Bien parler exige-t-il des qualités morales ? \\[0pt]
Bonheur et bien-être \\[0pt]
Bonheur et plaisir \\[0pt]
Bonheur et satisfaction \\[0pt]
Bonheur et société \\[0pt]
Bonheur et technique \\[0pt]
Bonheur et vertu \\[0pt]
Calculer et penser \\[0pt]
Cause et condition \\[0pt]
Cause et effet \\[0pt]
Cause et loi \\[0pt]
Cause et raison \\[0pt]
Ce que je pense est-il nécessairement vrai ? \\[0pt]
Ce que la technique rend possible, peut-on jamais en empêcher la réalisation ? \\[0pt]
Ce que nous avons le devoir de faire peut-il toujours s'exprimer sous forme de loi ? \\[0pt]
Ce qui dépasse la raison est-il nécessairement irréel ? \\[0pt]
Ce qui est ordinaire est-il normal ? \\[0pt]
Ce qui est vrai est-il toujours vérifiable ? \\[0pt]
Ce qui ne peut s'acheter est-il dépourvu de valeur ? \\[0pt]
Ce qui n'est pas matériel peut-il être réel ? \\[0pt]
Certitude et conviction \\[0pt]
« C'est plus fort que moi » \\[0pt]
Ceux qui oppriment sont-ils libres ? \\[0pt]
Chacun a-t-il le droit d'invoquer sa vérité ? \\[0pt]
Chacun a-t-il sa propre vérité ? \\[0pt]
Chance et bonheur \\[0pt]
Changer, est-ce devenir un autre ? \\[0pt]
Changer, est-ce se trahir ? \\[0pt]
Changer le monde \\[0pt]
Change-t-on avec le temps ? \\[0pt]
Châtier, est ce faire honneur au criminel ? \\[0pt]
Choisir \\[0pt]
Choisir, est-ce renoncer ? \\[0pt]
Choisissons-nous qui nous sommes ? \\[0pt]
Choisit-on ses souvenirs ? \\[0pt]
Choix et raison \\[0pt]
Chose et objet \\[0pt]
Chose et personne \\[0pt]
Choses naturelles, choses fabriquées \\[0pt]
Civilisation et barbarie \\[0pt]
Classer \\[0pt]
Colère et indignation \\[0pt]
Commander \\[0pt]
Comment autrui peut-il m'aider à rechercher le bonheur ? \\[0pt]
Comment chercher ce qu'on ignore ? \\[0pt]
Comment comprendre les faits sociaux ? \\[0pt]
Comment connaître le passé ? \\[0pt]
Comment connaître l'inconscient ? \\[0pt]
Comment connaître nos devoirs ? \\[0pt]
Comment dire la vérité ? \\[0pt]
Comment distinguer culture et civilisation ? \\[0pt]
Comment distinguer le rêvé du perçu ? \\[0pt]
Comment distingue-t-on le vrai du réel ? \\[0pt]
Comment fonder nos devoirs ? \\[0pt]
Comment juger de la justesse d'une interprétation ? \\[0pt]
Comment juger une œuvre d'art ? \\[0pt]
Comment la volonté peut-elle être faible ? \\[0pt]
Comment le devoir peut-il déterminer l'action ? \\[0pt]
Comment le passé nous est-il présent ? \\[0pt]
Comment le passé peut-il demeurer présent ? \\[0pt]
Comment l'erreur est-elle possible ? \\[0pt]
Comment les mathématiques peuvent-elles s'appliquer au réel ? \\[0pt]
Comment l'homme peut-il se représenter le temps ? \\[0pt]
Comment l'inconscient peut-il se manifester ? \\[0pt]
Comment penser le futur ? \\[0pt]
Comment penser le hasard ? \\[0pt]
Comment penser l'éternel ? \\[0pt]
Comment peut-on définir un être vivant ? \\[0pt]
Comment peut-on être heureux ? \\[0pt]
Comment peut-on être sceptique ? \\[0pt]
Comment peut-on savoir que l'on a raison ? \\[0pt]
Comment prend-on connaissance de ses devoirs ? \\[0pt]
Comment prouver la liberté ? \\[0pt]
Comment puis-je devenir ce que je suis ? \\[0pt]
Comment rendre la justice ? \\[0pt]
Comment savoir quels sont nos devoirs ? \\[0pt]
Communauté et société \\[0pt]
Comprendre \\[0pt]
Comprendre, est-ce interpréter ? \\[0pt]
Comprendre le réel, est-ce le dominer ? \\[0pt]
Comprendre une démonstration \\[0pt]
Comprendre une œuvre d'art \\[0pt]
Concept et métaphore \\[0pt]
Concurrence et égalité \\[0pt]
Connaissance de soi et conscience de soi \\[0pt]
Connaissance et perception \\[0pt]
Connaissons-nous la réalité des choses ? \\[0pt]
Connaissons-nous mieux le présent que le passé ? \\[0pt]
« Connais-toi toi-même » \\[0pt]
Connaît-on jamais pour le plaisir ? \\[0pt]
Connaît-on la vie ou bien connaît-on le vivant ? \\[0pt]
Connaît-on la vie ou le vivant ? \\[0pt]
Connaît-on les choses telles qu'elles sont ? \\[0pt]
Connaît-on pour le plaisir ? \\[0pt]
Connaître, est-ce découvrir le réel ? \\[0pt]
Connaître, est-ce dépasser les apparences ? \\[0pt]
Connaître la vie ou le vivant ? \\[0pt]
Connaître une chose, est-ce en connaître la cause ? \\[0pt]
Conquérir \\[0pt]
Conscience de soi et amour de soi \\[0pt]
Conscience de soi et connaissance de soi \\[0pt]
Conscience et attention \\[0pt]
Conscience et connaissance \\[0pt]
Conscience et conscience de soi \\[0pt]
Conscience et liberté \\[0pt]
Conscience et responsabilité \\[0pt]
Conscience et subjectivité \\[0pt]
Conscience et volonté \\[0pt]
Construire la vérité \\[0pt]
Contempler \\[0pt]
Contrainte et obligation \\[0pt]
Convaincre \\[0pt]
Convaincre et persuader \\[0pt]
Convient-il d'opposer explication et interprétation ? \\[0pt]
Convient-il d'opposer liberté et nécessité ? \\[0pt]
Corps et espace \\[0pt]
Corps et matière \\[0pt]
Corps et nature \\[0pt]
Crainte et espoir \\[0pt]
Création et production \\[0pt]
Créer et produire \\[0pt]
Critiquer \\[0pt]
Croire, est-ce renoncer à l'usage de la raison ? \\[0pt]
Croire, est-ce renoncer à savoir ? \\[0pt]
Croire, est-ce renoncer au savoir ? \\[0pt]
Croire et savoir \\[0pt]
Croire que Dieu existe, est-ce croire en lui ? \\[0pt]
Croyance et confiance \\[0pt]
Croyance et vérité \\[0pt]
Culpabilité et responsabilité \\[0pt]
Culture et artifice \\[0pt]
Culture et civilisation \\[0pt]
Culture et différence \\[0pt]
Culture et éducation \\[0pt]
Culture et identité \\[0pt]
Culture et langage \\[0pt]
Culture et savoir \\[0pt]
Culture et technique \\[0pt]
Culture et violence \\[0pt]
Dans l'action, est-ce l'intention qui compte ? \\[0pt]
Dans quel but les hommes se donnent-ils des lois ? \\[0pt]
Dans quelle mesure est-on l'auteur de sa propre vie ? \\[0pt]
Dans quelle mesure la technique nous libère-t-elle de la nature ? \\[0pt]
Dans quelle mesure les sciences ne sont-elles pas à l'abri de l'erreur ? \\[0pt]
Dans quelle mesure le temps nous appartient-il ? \\[0pt]
Dans quelle mesure toute philosophie est-elle critique du langage ? \\[0pt]
« Dans un bois aussi courbe que celui dont l'homme est fait on ne peut rien tailler de tout à fait droit » \\[0pt]
Débat et démocratie \\[0pt]
Débattre et dialoguer \\[0pt]
Déchiffrer \\[0pt]
Décider \\[0pt]
Découverte et justification \\[0pt]
Déduction et démonstration \\[0pt]
Délibérer \\[0pt]
Dématérialiser \\[0pt]
Démocratie et opinion \\[0pt]
Démocratie et représentation \\[0pt]
Démocratie et technocratie \\[0pt]
Démonstration et intuition \\[0pt]
Démontrer et argumenter \\[0pt]
Démontrer et expérimenter \\[0pt]
Démontrer par l'absurde \\[0pt]
Démontre-t-on ailleurs qu'en mathématiques ? \\[0pt]
Dépend-il de nous d'être heureux ? \\[0pt]
Dépend-il de soi d'être heureux ? \\[0pt]
De quel droit l'État exerce-t-il un pouvoir ? \\[0pt]
De quel droit punit-on ? \\[0pt]
De quel homme parlent les sciences humaines ? \\[0pt]
De quelle réalité parlons-nous lorsque nous nous exprimons ? \\[0pt]
De quelle vérité l'art est-il capable ? \\[0pt]
De quoi avons-nous vraiment besoin ? \\[0pt]
De quoi dépend le bonheur ? \\[0pt]
De quoi dépend notre bonheur ? \\[0pt]
De quoi est fait mon présent ? \\[0pt]
De quoi est fait notre présent ? \\[0pt]
De quoi la philosophie est-elle le désir ? \\[0pt]
De quoi la religion sauve-t-elle ? \\[0pt]
De quoi la technique ne peut-elle pas nous libérer ? \\[0pt]
De quoi la vérité libère-t-elle ? \\[0pt]
De quoi le devoir libère-t-il ? \\[0pt]
De quoi les logiciens parlent-ils ? \\[0pt]
De quoi l'État ne doit-il pas se mêler ? \\[0pt]
De quoi le tyran est-il libre ? \\[0pt]
De quoi l'inconscient peut-il être la cause ? \\[0pt]
De quoi parlent les mathématiques ? \\[0pt]
De quoi peut-on être certain ? \\[0pt]
De quoi peut-on être inconscient ? \\[0pt]
De quoi peut-on faire l'expérience ? \\[0pt]
De quoi pouvons-nous être sûrs ? \\[0pt]
De quoi puis-je répondre ? \\[0pt]
De quoi sommes-nous responsables ? \\[0pt]
De quoi suis-je inconscient ? \\[0pt]
De quoi suis-je responsable ? \\[0pt]
De quoi une œuvre d'art nous instruit-elle ? \\[0pt]
De quoi y a-t-il histoire ? \\[0pt]
Déraisonner, est-ce perdre de vue le réel ? \\[0pt]
Désirer, est-ce être aliéné ? \\[0pt]
Désir et besoin \\[0pt]
Désir et bonheur \\[0pt]
Désir et interdit \\[0pt]
Désir et langage \\[0pt]
Désir et manque \\[0pt]
Désir et ordre \\[0pt]
Désir et plaisir \\[0pt]
Désir et pouvoir \\[0pt]
Désir et raison \\[0pt]
Désir et réalité \\[0pt]
Désir et volonté \\[0pt]
Désirons-nous les choses parce qu'elles sont bonnes ? \\[0pt]
Des lois justes suffisent-elles à assurer la justice ? \\[0pt]
Devant qui sommes-nous responsables ? \\[0pt]
Devenir humain \\[0pt]
Devoir, est-ce avoir une dette envers quelqu'un ? \\[0pt]
Devoir, est-ce vouloir ? \\[0pt]
Devoir et bonheur \\[0pt]
Devoir et contrainte \\[0pt]
Devoir et intérêt \\[0pt]
Devoir et liberté \\[0pt]
Devoir et plaisir \\[0pt]
Devoir et prudence \\[0pt]
Devoir et vertu \\[0pt]
Devoirs envers les autres et devoirs envers soi-même \\[0pt]
Devoirs et passions \\[0pt]
Devons-nous apprendre à vivre ? \\[0pt]
Devons-nous dire la vérité ? \\[0pt]
Devons-nous espérer vivre sans travailler ? \\[0pt]
Devons-nous être heureux ? \\[0pt]
Devons-nous être obéissants ? \\[0pt]
Devons-nous toujours dire la vérité ? \\[0pt]
Dialoguer et s'entendre \\[0pt]
Dieu est-il une hypothèse dont le savant et le philosophe peuvent se passer ? \\[0pt]
Dire, est-ce autre chose que vouloir dire ? \\[0pt]
Dire, est-ce faire ? \\[0pt]
Dire, est-ce trahir ? \\[0pt]
Dire et exprimer \\[0pt]
Dire et faire \\[0pt]
Dire je \\[0pt]
Dire la vérité \\[0pt]
Dogme et opinion \\[0pt]
Dois-je admettre tout ce que je ne peux réfuter ? \\[0pt]
Dois-je tenir compte de ce que font les autres pour orienter ma conduite ? \\[0pt]
Doit-on apprendre à devenir soi-même ? \\[0pt]
Doit-on apprendre à percevoir ? \\[0pt]
Doit-on apprendre à vivre ? \\[0pt]
Doit-on attendre de la technique qu'elle mette fin au travail ? \\[0pt]
Doit-on bien juger pour bien faire ? \\[0pt]
Doit-on changer ses désirs, plutôt que l'ordre du monde ? \\[0pt]
Doit-on chercher à être heureux ? \\[0pt]
Doit-on distinguer devoir moral et obligation sociale ? \\[0pt]
Doit-on identifier l'âme à la conscience ? \\[0pt]
Doit-on interpréter les rêves ? \\[0pt]
Doit-on le respect au vivant ? \\[0pt]
Doit-on mûrir pour la liberté ? \\[0pt]
Doit-on rechercher le bonheur ? \\[0pt]
Doit-on refuser d'interpréter ? \\[0pt]
Doit-on respecter les êtres vivants ? \\[0pt]
Doit-on se passer des utopies ? \\[0pt]
Doit-on tenir le plaisir pour une fin ? \\[0pt]
Doit-on toujours dire la vérité ? \\[0pt]
Doit-on tout accepter de l'État ? \\[0pt]
Doit-on tout attendre de l'État ? \\[0pt]
Doit-on tout pardonner ? \\[0pt]
Doit-on vraiment tout pardonner ? \\[0pt]
Don et échange \\[0pt]
Donner, à quoi bon ? \\[0pt]
Donner et recevoir \\[0pt]
Donner l'exemple ? \\[0pt]
Donner sa parole \\[0pt]
Doute et raison \\[0pt]
D'où vient la certitude ? \\[0pt]
D'où vient la force d'une religion ? \\[0pt]
D'où vient l'amour de Dieu ? \\[0pt]
D'où vient la servitude ? \\[0pt]
Droit et coutume \\[0pt]
Droit et devoir \\[0pt]
Droit et interprétation \\[0pt]
Droit et morale \\[0pt]
Droit et violence \\[0pt]
« Droits de\ldots{} » et « droits à\ldots{} » \\[0pt]
Droits de l'homme ou droits du citoyen ? \\[0pt]
Droits et devoirs \\[0pt]
« Du passé, faisons table rase » \\[0pt]
Durée et instant \\[0pt]
Durer \\[0pt]
Échange et don \\[0pt]
Échanger des idées \\[0pt]
Échanger, est-ce créer de la valeur ? \\[0pt]
Échanger, est-ce partager ? \\[0pt]
Échappe-t-on à la nécessité ? \\[0pt]
Écouter et entendre \\[0pt]
Écrire et parler \\[0pt]
Écrire l'histoire, est-ce donner un sens à la violence ? \\[0pt]
Égalité et différence \\[0pt]
En histoire, tout est-il affaire d'interprétation ? \\[0pt]
En morale, est-ce seulement l'intention qui compte ? \\[0pt]
En morale, peut-on dire : « C'est l'intention qui compte » ? \\[0pt]
En politique, faut-il refuser l'utopie ? \\[0pt]
En quel sens la perception engage-t-elle le corps ? \\[0pt]
En quel sens l'État est-il rationnel ? \\[0pt]
En quel sens le vivant a-t-il une histoire ? \\[0pt]
En quel sens parler de lois de la pensée ? \\[0pt]
En quel sens parler d'identité culturelle ? \\[0pt]
En quel sens parler d'illusion religieuse ? \\[0pt]
En quel sens peut-on appeler les sciences humaines « sciences de l'esprit » ? \\[0pt]
En quel sens peut-on dire que la vérité s'impose ? \\[0pt]
En quel sens peut-on dire que le langage est la condition de la pensée ? \\[0pt]
En quel sens peut-on dire que l'homme est un animal politique ? \\[0pt]
En quel sens peut-on dire que nos paroles nous trahissent ? \\[0pt]
En quel sens peut-on dire qu'« on expérimente avec sa raison » ? \\[0pt]
En quel sens peut-on parler de la mort de l'art ? \\[0pt]
En quel sens peut-on parler de responsabilité collective ? \\[0pt]
En quel sens peut-on parler d'une « culture technique » ? \\[0pt]
En quel sens peut-on parler d'une culture technique ? \\[0pt]
En quel sens peut-on parler d'une interprétation de la nature ? \\[0pt]
En quoi la connaissance du vivant contribue-t-elle à la connaissance de l'homme ? \\[0pt]
En quoi la croyance religieuse se distingue-t-elle des autres croyances ? \\[0pt]
En quoi la liberté n'est-elle pas une illusion ? \\[0pt]
En quoi la méthode est-elle un art de penser ? \\[0pt]
En quoi la nature peut-elle être source de création ? \\[0pt]
En quoi la question de Dieu est-elle philosophique ? \\[0pt]
En quoi l'art peut-il intéresser le philosophe ? \\[0pt]
En quoi le bien d'autrui m'importe-t-il ? \\[0pt]
En quoi le bonheur est-il l'affaire de l'État ? \\[0pt]
En quoi les vivants témoignent-ils d'une histoire ? \\[0pt]
En quoi une culture peut-elle être la mienne ? \\[0pt]
Entre croire et savoir, y a-t-il une différence de nature ? \\[0pt]
Entre la sécurité de tous et la liberté de chacun, le politique doit-il choisir ? \\[0pt]
Entre l'opinion et la science, n'y a-t-il qu'une différence de degré ? \\[0pt]
Envers qui avons-nous des devoirs ? \\[0pt]
Errer \\[0pt]
Erreur et faute \\[0pt]
Erreur et illusion \\[0pt]
Essence et existence \\[0pt]
Est-ce à la raison de déterminer ce qui est réel ? \\[0pt]
Est-ce à l'État de faire le bonheur du peuple ? \\[0pt]
Est-ce de la force que l'État tient son autorité ? \\[0pt]
Est-ce la majorité qui doit décider ? \\[0pt]
Est-ce l'autorité qui fait la loi ? \\[0pt]
Est-ce le cerveau qui pense ? \\[0pt]
Est-ce l'échange utilitaire qui fait le lien social ? \\[0pt]
Est-ce l'ignorance qui nous fait croire ? \\[0pt]
Est-ce l'ignorance qui rend les hommes croyants ? \\[0pt]
Est-ce l'intérêt qui fonde le lien social ? \\[0pt]
Est-ce l'utilité qui définit un objet technique ? \\[0pt]
Est-ce un devoir d'aimer son prochain ? \\[0pt]
Est-il bien vrai qu'« on n'arrête pas le progrès » ? \\[0pt]
Est-il dans mon intérêt d'accomplir mes devoirs ? \\[0pt]
Est-il facile d'être libre ? \\[0pt]
Est-il immoral de se rendre heureux ? \\[0pt]
Est-il juste d'interpréter la loi ? \\[0pt]
Est-il justifié de parler de « corps social » ? \\[0pt]
Est-il légitime d'affirmer que seul le présent existe ? \\[0pt]
Est-il légitime d'opposer liberté et nécessité ? \\[0pt]
Est-il méritoire de ne faire que son devoir ? \\[0pt]
Est-il naturel à l'homme de parler ? \\[0pt]
Est-il pertinent d'opposer les actes et les paroles ? \\[0pt]
Est-il pertinent d'opposer l'individuel et le collectif ? \\[0pt]
Est-il possible de tout avoir pour être heureux ? \\[0pt]
Est-il possible d'être immoral sans le savoir ? \\[0pt]
Est-il possible d'ignorer toute vérité ? \\[0pt]
Est-il raisonnable d'aimer ? \\[0pt]
Est-il raisonnable d'avoir des certitudes ? \\[0pt]
Est-il raisonnable de chercher à interpréter les rêves ? \\[0pt]
Est-il raisonnable de se quereller pour des mots ? \\[0pt]
Est-il raisonnable d'être rationnel ? \\[0pt]
Est-il raisonnable de vouloir maîtriser la nature ? \\[0pt]
Est-il toujours moral de faire son devoir ? \\[0pt]
Est-il toujours possible de savoir ce que l'on doit faire ? \\[0pt]
Est-il vrai que les animaux ne pensent pas ? \\[0pt]
Est-il vrai que l'ignorant n'est pas libre ? \\[0pt]
Est-il vrai que ma liberté s'arrête là où commence celle des autres ? \\[0pt]
Est-il vrai que plus on échange, moins on se bat ? \\[0pt]
Estime et respect \\[0pt]
Estimer \\[0pt]
Est-on l'auteur de sa propre vie ? \\[0pt]
Est-on libre de travailler ? \\[0pt]
Est-on libre face à la vérité ? \\[0pt]
Est-on libre par hasard ? \\[0pt]
Est-on propriétaire de son corps ? \\[0pt]
Est-on responsable de ses croyances ? \\[0pt]
Est-on sociable par nature ? \\[0pt]
Est-on vraiment libre d'exprimer ce que l'on veut ? \\[0pt]
État et gouvernement \\[0pt]
État et institutions \\[0pt]
État et nation \\[0pt]
État et société \\[0pt]
État et Société \\[0pt]
État et société civile \\[0pt]
État et violence \\[0pt]
Éthique et Morale \\[0pt]
Être à l'écoute de son désir, est-ce nier le désir de l'autre ? \\[0pt]
Être bon juge \\[0pt]
Être citoyen du monde \\[0pt]
Être conscient, est-ce être maître de soi ? \\[0pt]
Être cultivé rend-il meilleur ? \\[0pt]
Être dans le vrai \\[0pt]
Être de son temps \\[0pt]
Être dogmatique \\[0pt]
Être éduqué, est-ce devenir soi-même ? \\[0pt]
Être en paix \\[0pt]
Être et apparaître \\[0pt]
Être et avoir \\[0pt]
Être et avoir été \\[0pt]
Être et devenir \\[0pt]
Être et devoir être \\[0pt]
Être et devoir-être \\[0pt]
Être et exister \\[0pt]
Être et paraître \\[0pt]
Être exemplaire \\[0pt]
Être heureux, est-ce devoir ? \\[0pt]
Être libre, cela s'apprend-il ? \\[0pt]
Être libre, est-ce avoir le choix ? \\[0pt]
Être libre, est-ce dire non ? \\[0pt]
Être libre, est-ce échapper aux prévisions ? \\[0pt]
Être libre, est-ce faire ce que l'on veut ? \\[0pt]
Être libre, est-ce n'avoir aucun maître ? \\[0pt]
Être libre, est-ce n'obéir qu'à soi-même ? \\[0pt]
Être libre, est-ce pouvoir choisir ? \\[0pt]
Être libre, est-ce se suffire à soi-même ? \\[0pt]
Être libre, est-ce une question de volonté ? \\[0pt]
Être libre, est-ce vivre comme on l'entend ? \\[0pt]
Être membre de L'État \\[0pt]
Être naturel \\[0pt]
Être raisonnable, est-ce renoncer à ses désirs ? \\[0pt]
Être réaliste \\[0pt]
Être sceptique \\[0pt]
Être soi-même \\[0pt]
« Être soi-même », cela a-t-il un sens ? \\[0pt]
Être spectateur \\[0pt]
Être spontané, est-ce être libre ? \\[0pt]
Être un sujet, est-ce être maître de soi ? \\[0pt]
Être vertueux \\[0pt]
Évidence et raison \\[0pt]
Évidence et vérité \\[0pt]
Évidences et préjugés \\[0pt]
Évolution biologique et culture \\[0pt]
Évolution et progrès \\[0pt]
Excuser et pardonner \\[0pt]
Existence et contingence \\[0pt]
Existence et religion \\[0pt]
Exister \\[0pt]
Exister, est-ce simplement vivre ? \\[0pt]
Existe-t-il au moins un devoir universel ? \\[0pt]
Existe-t-il des choses en soi ? \\[0pt]
Existe-t-il des choses réellement sublimes ? \\[0pt]
Existe-t-il des comportements contraires à la nature ? \\[0pt]
Existe-t-il des désirs coupables ? \\[0pt]
Existe-t-il des devoirs envers soi-même ? \\[0pt]
Existe-t-il des plaisirs purs ? \\[0pt]
Existe-t-il un art de la parole ? \\[0pt]
Existe-t-il un déterminisme social ? \\[0pt]
Existe-t-il une méthode pour rechercher la vérité ? \\[0pt]
Existe-t-il une méthode pour trouver la vérité ? \\[0pt]
Expérience et expérimentation \\[0pt]
Expérience immédiate et expérimentation scientifique \\[0pt]
Expliquer, est-ce interpréter ? \\[0pt]
Expliquer et comprendre \\[0pt]
Exploiter la nature, est-ce lui vouloir du mal ? \\[0pt]
Fabriquer et créer \\[0pt]
Faire autorité \\[0pt]
Faire confiance \\[0pt]
Faire de nécessité vertu \\[0pt]
Faire la paix \\[0pt]
Faire le mal \\[0pt]
Faire l'histoire \\[0pt]
Faire son devoir, est-ce faire son propre malheur ? \\[0pt]
Faire son devoir, est-ce là toute la morale ? \\[0pt]
Faire son devoir, est-ce obéir ? \\[0pt]
Faire son devoir, est-ce payer une dette ? \\[0pt]
Faire son devoir, est-ce perdre toute sensibilité ? \\[0pt]
Faire son devoir, est-ce se soumettre ? \\[0pt]
Faisons-nous l'histoire ? \\[0pt]
Faisons-nous l'Histoire ? \\[0pt]
Fait et fiction \\[0pt]
Fait et preuve \\[0pt]
Fait et valeur \\[0pt]
Faits et preuves \\[0pt]
Famille et société \\[0pt]
Faudrait-il bannir la polysémie du langage ? \\[0pt]
Faudrait-il ne rien oublier ? \\[0pt]
Faudrait-il vivre sans passion ? \\[0pt]
Faut-avoir peur de la technique ? \\[0pt]
Faut-il accepter sa condition ? \\[0pt]
Faut-il accorder de l'importance aux mots ? \\[0pt]
Faut-il affirmer son identité ? \\[0pt]
Faut-il aimer autrui pour le respecter ? \\[0pt]
Faut-il aimer la nature plus que soi-même ? \\[0pt]
Faut-il aimer son prochain comme soi-même ? \\[0pt]
Faut-il apprendre à être libre ? \\[0pt]
Faut-il apprendre à obéir ? \\[0pt]
Faut-il apprendre à vivre en renonçant au bonheur ? \\[0pt]
Faut-il assimiler la pensée au langage ? \\[0pt]
Faut-il avoir foi en la technique ? \\[0pt]
Faut-il avoir peur de la technique ? \\[0pt]
Faut-il avoir peur des machines ? \\[0pt]
Faut-il avoir peur d'être libre ? \\[0pt]
Faut-il avoir peur du désordre ? \\[0pt]
Faut-il changer ses désirs plutôt que l'ordre du monde ? \\[0pt]
Faut-il chasser le naturel ? \\[0pt]
Faut-il chercher à être heureux ? \\[0pt]
Faut-il chercher à satisfaire tous nos désirs ? \\[0pt]
Faut-il chercher à se connaître ? \\[0pt]
Faut-il chercher à tout démontrer ? \\[0pt]
Faut-il chercher la paix à tout prix ? \\[0pt]
Faut-il chercher le bonheur à tout prix ? \\[0pt]
Faut-il chercher un sens à l'histoire ? \\[0pt]
Faut-il choisir entre être heureux et être libre ? \\[0pt]
Faut-il comprendre pour croire ? \\[0pt]
Faut-il connaître la vérité pour pouvoir mentir ? \\[0pt]
Faut-il connaître l'Histoire pour gouverner ? \\[0pt]
Faut-il craindre de perdre son temps ? \\[0pt]
Faut-il craindre la tyrannie du bonheur ? \\[0pt]
Faut-il craindre le développement des techniques ? \\[0pt]
Faut-il craindre le pouvoir des machines ? \\[0pt]
Faut-il craindre les machines ? \\[0pt]
Faut-il craindre l'État ? \\[0pt]
Faut-il craindre l'ordre ? \\[0pt]
Faut-il croire en la science ? \\[0pt]
Faut-il croire les historiens ? \\[0pt]
Faut-il croire que l'histoire a un sens ? \\[0pt]
Faut-il défendre les faibles ? \\[0pt]
Faut-il dire de la justice qu'elle n'existe pas ? \\[0pt]
Faut-il distinguer culture et civilisation ? \\[0pt]
Faut-il distinguer désir et besoin ? \\[0pt]
Faut-il distinguer devoir moral et obligation sociale ? \\[0pt]
Faut-il douter de ce qu'on ne peut pas démontrer ? \\[0pt]
Faut-il douter de tout ? \\[0pt]
Faut-il du passé faire table rase ? \\[0pt]
Faut-il espérer pour agir ? \\[0pt]
Faut-il être bon ? \\[0pt]
Faut-il être cohérent ? \\[0pt]
Faut-il être courageux pour être libre ? \\[0pt]
Faut-il être libre pour être heureux ? \\[0pt]
Faut-il être modéré ? \\[0pt]
Faut-il être pragmatique ? \\[0pt]
Faut-il faire confiance au progrès technique ? \\[0pt]
Faut-il faire de nécessité vertu ? \\[0pt]
Faut-il faire des efforts pour être soi-même ? \\[0pt]
Faut-il faire table rase du passé ? \\[0pt]
Faut-il faire une place à la croyance ? \\[0pt]
Faut-il hiérarchiser les désirs ? \\[0pt]
Faut-il hiérarchiser les formes de vie ? \\[0pt]
Faut-il hiérarchiser ses désirs ? \\[0pt]
Faut-il laisser faire la nature ? \\[0pt]
Faut-il libérer l'humanité du travail ? \\[0pt]
Faut-il limiter la souveraineté de l'État ? \\[0pt]
Faut-il limiter le pouvoir de l'État ? \\[0pt]
Faut-il mériter son bonheur ? \\[0pt]
Faut-il ne manquer de rien pour être heureux ? \\[0pt]
Faut-il obéir à la voix de sa conscience ? \\[0pt]
Faut-il opposer culture et nature ? \\[0pt]
Faut-il opposer droits et devoirs ? \\[0pt]
Faut-il opposer histoire et mémoire ? \\[0pt]
Faut-il opposer la matière et l'esprit ? \\[0pt]
Faut-il opposer la société et l'État ? \\[0pt]
Faut-il opposer la vie à la mort ? \\[0pt]
Faut-il opposer le devoir-être à l'être ? \\[0pt]
Faut-il opposer le don et l'échange ? \\[0pt]
Faut-il opposer le temps vécu et le temps des choses ? \\[0pt]
Faut-il opposer raison et sensation ? \\[0pt]
Faut-il opposer subjectivité et objectivité ? \\[0pt]
Faut-il opposer technique et nature ? \\[0pt]
Faut-il opposer théorie et expérience ? \\[0pt]
Faut-il oublier le passé pour se donner un avenir ? \\[0pt]
Faut-il parfois désobéir aux lois ? \\[0pt]
Faut-il parfois sacrifier la vérité ? \\[0pt]
Faut-il pour le connaître faire du vivant un objet ? \\[0pt]
Faut-il préférer l'art à la nature ? \\[0pt]
Faut-il préférer le bonheur à la vérité ? \\[0pt]
Faut-il privilégier le naturel ? \\[0pt]
Faut-il rechercher la perfection dans le langage ? \\[0pt]
Faut-il rechercher le bonheur ? \\[0pt]
Faut-il regretter l'équivocité du langage ? \\[0pt]
Faut-il réguler la technique ? \\[0pt]
Faut-il rejeter tous les préjugés ? \\[0pt]
Faut-il rejeter toute norme ? \\[0pt]
Faut-il renoncer à faire du travail une valeur ? \\[0pt]
Faut-il renoncer à la vérité ? \\[0pt]
Faut-il renoncer à l'idée d'âme ? \\[0pt]
Faut-il renoncer à rechercher la vérité ? \\[0pt]
Faut-il respecter la nature ? \\[0pt]
Faut-il respecter la vie ? \\[0pt]
Faut-il respecter le vivant ? \\[0pt]
Faut-il rester soi-même ? \\[0pt]
Faut-il rompre avec le passé ? \\[0pt]
Faut-il s'adapter ? \\[0pt]
Faut-il s'affranchir des désirs ? \\[0pt]
Faut-il s'aimer soi-même ? \\[0pt]
Faut-il sauver les apparences ? \\[0pt]
Faut-il savoir mentir ? \\[0pt]
Faut-il savoir pour agir ? \\[0pt]
Faut-il se connaître pour être maître de soi ? \\[0pt]
Faut-il se cultiver ? \\[0pt]
Faut-il se détacher du monde ? \\[0pt]
Faut-il se fier à sa propre raison ? \\[0pt]
Faut-il se fier aux apparences ? \\[0pt]
Faut-il se libérer de soi-même ? \\[0pt]
Faut-il se libérer du travail ? \\[0pt]
Faut-il se libérer pour être libre ? \\[0pt]
Faut-il se méfier de la technique ? \\[0pt]
Faut-il se méfier de l'intuition ? \\[0pt]
Faut-il se méfier de ses désirs ? \\[0pt]
Faut-il se méfier de soi-même ? \\[0pt]
Faut-il se méfier du progrès technique ? \\[0pt]
Faut-il s'en tenir aux faits ? \\[0pt]
Faut-il se rendre à l'évidence ? \\[0pt]
Faut-il se ressembler pour former une société ? \\[0pt]
Faut-il souhaiter la fin du travail ? \\[0pt]
Faut-il souhaiter une langue universelle ? \\[0pt]
Faut-il suivre la nature ? \\[0pt]
Faut-il toujours chercher à savoir ? \\[0pt]
Faut-il toujours dire la vérité ? \\[0pt]
Faut-il toujours éviter de se contredire ? \\[0pt]
Faut-il toujours faire son devoir ? \\[0pt]
Faut-il tout critiquer ? \\[0pt]
Faut-il un corps pour penser ? \\[0pt]
Faut-il une méthode pour découvrir la vérité ? \\[0pt]
Faut-il vivre avec son temps ? \\[0pt]
Faut-il vivre comme si nous ne devions jamais mourir ? \\[0pt]
Faut-il vivre comme si on ne devait jamais mourir ? \\[0pt]
Faut-il vivre hors de la société pour être heureux ? \\[0pt]
Faut-il vouloir changer le monde ? \\[0pt]
Faut-il vouloir être heureux ? \\[0pt]
Foi et raison \\[0pt]
Foi et savoir \\[0pt]
Foi et superstition \\[0pt]
Force et violence \\[0pt]
Forme et matière \\[0pt]
Former et éduquer \\[0pt]
Forme-t-on son esprit en transformant la matière ? \\[0pt]
Génération, fabrication, création \\[0pt]
Gouvernement des hommes et administration des choses \\[0pt]
Gouverner \\[0pt]
Gouverner, est-ce régner ? \\[0pt]
Guerre et politique \\[0pt]
Habiter \\[0pt]
Hasard et destin \\[0pt]
Hier a-t-il plus de réalité que demain ? \\[0pt]
Histoire et causalité \\[0pt]
Histoire et mémoire \\[0pt]
Histoire et progrès \\[0pt]
Histoire et violence \\[0pt]
Histoire individuelle et histoire collective \\[0pt]
Hypothèse et vérité \\[0pt]
Ici et maintenant \\[0pt]
Idéal et utopie \\[0pt]
Idée et concept \\[0pt]
Idée et réalité \\[0pt]
Identité et appartenance \\[0pt]
Identité et différence \\[0pt]
Illusion et hallucination \\[0pt]
« Il ne lui manque que la parole » \\[0pt]
« Il y a un temps pour tout » \\[0pt]
Image et concept \\[0pt]
Image et idée \\[0pt]
Image, fiction, illusion \\[0pt]
Imagination et culture \\[0pt]
Imagination et pouvoir \\[0pt]
Imitation et création \\[0pt]
Imitation et représentation \\[0pt]
Imiter et créer \\[0pt]
Inconscient et déterminisme \\[0pt]
Inconscient et inconscience \\[0pt]
Inconscient et instinct \\[0pt]
Inconscient et liberté \\[0pt]
Inconscient et mythes \\[0pt]
Indépendance et autonomie \\[0pt]
Indépendance et liberté \\[0pt]
Individu et citoyen \\[0pt]
Individu et communauté \\[0pt]
Individu et société \\[0pt]
Innocence et ignorance \\[0pt]
Instruction et éducation \\[0pt]
Instruire et éduquer \\[0pt]
Intérêt général et bien commun \\[0pt]
Interprétation et création \\[0pt]
Interprétation et perception \\[0pt]
Interpréter \\[0pt]
Interpréter, est-ce connaître ? \\[0pt]
Interpréter, est-ce renoncer à prouver ? \\[0pt]
Interpréter, est-ce renoncer à toute objectivité ? \\[0pt]
Interpréter est-il subjectif ? \\[0pt]
Interpréter et comprendre \\[0pt]
Interpréter et traduire \\[0pt]
Interprète-t-on à défaut de connaître ? \\[0pt]
Interroger \\[0pt]
Intuition et déduction \\[0pt]
Inventer et découvrir \\[0pt]
Invention, création, découverte \\[0pt]
Invention et découverte \\[0pt]
« Je ne l'ai pas fait exprès » \\[0pt]
Jugement et réflexion \\[0pt]
Jugement et vérité \\[0pt]
Juger et sentir \\[0pt]
Jusqu'à quel point la nature est-elle objet de science ? \\[0pt]
Jusqu'où s'étend le domaine de la science ? \\[0pt]
Justice et charité \\[0pt]
Justice et égalité \\[0pt]
Justice et équité \\[0pt]
Justice et impartialité \\[0pt]
Justice et pardon \\[0pt]
Justice et ressentiment \\[0pt]
Justice et vengeance \\[0pt]
Justice et violence \\[0pt]
La barbarie \\[0pt]
La barrière de la langue \\[0pt]
La beauté du monde \\[0pt]
La beauté est-elle dans les choses ? \\[0pt]
La beauté est-elle intemporelle ? \\[0pt]
La beauté est-elle une promesse de bonheur ? \\[0pt]
La beauté et le mal \\[0pt]
La beauté morale \\[0pt]
La beauté nous rend-elle meilleurs ? \\[0pt]
La beauté s'explique-t-elle ? \\[0pt]
La bête \\[0pt]
La bête et l'animal \\[0pt]
La bêtise \\[0pt]
La bienveillance \\[0pt]
La bonne conscience \\[0pt]
La bonne intention \\[0pt]
La bonne volonté \\[0pt]
La bonté \\[0pt]
L'absence \\[0pt]
L'absolu \\[0pt]
L'absolu et le relatif \\[0pt]
L'abstraction \\[0pt]
L'abstrait et le concret \\[0pt]
L'absurde \\[0pt]
L'abus de pouvoir \\[0pt]
L'abus du droit \\[0pt]
L'académisme \\[0pt]
La causalité \\[0pt]
La causalité en histoire \\[0pt]
La cause efficiente \\[0pt]
La cause et l'effet \\[0pt]
L'accord des esprits est-il un critère de vérité ? \\[0pt]
L'accord est-il le terme ou le principe du dialogue ? \\[0pt]
La certitude de mourir \\[0pt]
La certitude est-elle une marque de vérité ? \\[0pt]
La certitude ne peut-elle naître que de l'évidence ? \\[0pt]
La chair \\[0pt]
La chance \\[0pt]
La charité est-elle la négation de la justice ? \\[0pt]
La cité \\[0pt]
La classification des arts \\[0pt]
La coexistence des libertés \\[0pt]
La cohérence est-elle la norme du vrai ? \\[0pt]
La cohérence logique est-elle une condition suffisante de la démonstration ? \\[0pt]
La cohérence suffit-elle à la vérité ? \\[0pt]
La colère \\[0pt]
La communauté scientifique \\[0pt]
La compétence \\[0pt]
La compétence en politique \\[0pt]
La compréhension \\[0pt]
La condition humaine \\[0pt]
La confiance \\[0pt]
La confiance est-elle une vertu ? \\[0pt]
La connaissance commune fait-elle obstacle à la vérité ? \\[0pt]
La connaissance de soi est-elle évidente ? \\[0pt]
La connaissance des principes \\[0pt]
La connaissance du passé est-elle nécessaire à la compréhension du présent ? \\[0pt]
La connaissance du vivant est-elle désintéressée ? \\[0pt]
La connaissance du vivant peut-elle être désintéressée ? \\[0pt]
La connaissance est-elle une contemplation ? \\[0pt]
La connaissance et la croyance \\[0pt]
La connaissance historique est-elle une interprétation des faits ? \\[0pt]
La connaissance historique est-elle utile à l'homme ? \\[0pt]
La connaissance intuitive \\[0pt]
La connaissance objective doit-elle s'interdire toute interprétation ? \\[0pt]
La connaissance objective exclut-elle toute forme de subjectivité ? \\[0pt]
La connaissance scientifique \\[0pt]
La connaissance scientifique est-elle désintéressée ? \\[0pt]
La connaissance sensible \\[0pt]
La conquête du pouvoir \\[0pt]
La conscience \\[0pt]
La conscience animale \\[0pt]
La conscience a-t-elle des degrés ? \\[0pt]
La conscience collective \\[0pt]
La conscience d'agir suffit-elle à garantir notre liberté ? \\[0pt]
La conscience d'autrui est-elle impénétrable ? \\[0pt]
La conscience de la mort est-elle une condition de la sagesse ? \\[0pt]
La conscience de soi \\[0pt]
La conscience de soi est-elle une donnée immédiate ? \\[0pt]
La conscience de soi est-elle une méconnaissance de soi ? \\[0pt]
La conscience de soi et l'identité personnelle \\[0pt]
La conscience de soi suppose-t-elle autrui ? \\[0pt]
La conscience du temps rend-elle l'existence tragique ? \\[0pt]
La conscience est-elle nécessairement malheureuse ? \\[0pt]
La conscience est-elle source d'illusions ? \\[0pt]
La conscience est-elle temporelle ? \\[0pt]
La conscience est-elle toujours morale ? \\[0pt]
La conscience est-elle une activité ? \\[0pt]
La conscience est-elle une connaissance ? \\[0pt]
La conscience est-elle une illusion ? \\[0pt]
La conscience et l'inconscient \\[0pt]
La conscience morale \\[0pt]
La conscience morale est-elle naturelle ? \\[0pt]
La conscience morale est-elle une illusion ? \\[0pt]
La conscience morale n'est-elle que le fruit de l'éducation ? \\[0pt]
La conscience morale n'est-elle que le produit de l'éducation ? \\[0pt]
La conscience peut-elle nous tromper ? \\[0pt]
La contemplation \\[0pt]
La contemplation de la nature \\[0pt]
La contingence de l'existence \\[0pt]
La contingence est-elle une condition de la liberté ? \\[0pt]
La contradiction \\[0pt]
La contrainte des lois est-elle une violence ? \\[0pt]
La contrainte peut-elle être légitime ? \\[0pt]
La contrainte s'oppose-t-elle à la liberté ? \\[0pt]
La controverse scientifique \\[0pt]
La convention et l'arbitraire \\[0pt]
La conversion \\[0pt]
La conviction \\[0pt]
La coopération \\[0pt]
La copie \\[0pt]
La corruption \\[0pt]
La courtoisie \\[0pt]
La coutume \\[0pt]
La création \\[0pt]
La création artistique \\[0pt]
La création de valeur \\[0pt]
La crédulité \\[0pt]
La critique \\[0pt]
La croyance \\[0pt]
La croyance et la foi \\[0pt]
La croyance et la raison \\[0pt]
La croyance peut-elle être rationnelle ? \\[0pt]
La croyance peut-elle s'accompagner de certitude ? \\[0pt]
La croyance peut-elle tenir lieu de savoir ? \\[0pt]
La croyance religieuse échappe-t-elle à toute logique ? \\[0pt]
L'acte et la parole \\[0pt]
L'acte manqué \\[0pt]
L'action \\[0pt]
L'action et le risque \\[0pt]
L'action morale dépend-elle des circonstances ? \\[0pt]
L'action politique \\[0pt]
L'activité technique dévalorise-t-elle l'homme ? \\[0pt]
L'actualité \\[0pt]
La culture \\[0pt]
La culture commence-t-elle avec le refus de la nature ? \\[0pt]
La culture engendre-t-elle le progrès ? \\[0pt]
La culture est-elle ce qui unit ou ce qui sépare les hommes ? \\[0pt]
La culture est-elle la négation de la nature ? \\[0pt]
La culture est-elle une seconde nature ? \\[0pt]
La culture est-elle un luxe ? \\[0pt]
La culture et le langage \\[0pt]
La culture et les cultures \\[0pt]
La culture garantit-elle l'excellence humaine ? \\[0pt]
La culture générale \\[0pt]
La culture historique forge-t-elle le jugement moral ? \\[0pt]
La culture nous rend-elle plus humains ? \\[0pt]
La culture nous unit-elle ? \\[0pt]
La culture populaire \\[0pt]
La culture rend-elle plus humain ? \\[0pt]
La culture technique \\[0pt]
La curiosité \\[0pt]
La décision \\[0pt]
La défense de l'intérêt général est-il la fin dernière de la politique ? \\[0pt]
La démarche scientifique exclut-elle tout recours à l'imagination ? \\[0pt]
La démesure \\[0pt]
La démocratie, est-ce le pouvoir du plus grand nombre ? \\[0pt]
La démocratie est-elle la loi du plus fort ? \\[0pt]
La démocratie est-elle le règne de l'opinion ? \\[0pt]
La démocratie peut-elle échapper à la démagogie ? \\[0pt]
La démocratie peut-elle être représentative ? \\[0pt]
La démonstration \\[0pt]
La démonstration mathématique peut-elle servir de modèle à toute démonstration ? \\[0pt]
La démonstration n'est-elle qu'une méthode d'exposition des découvertes ? \\[0pt]
La démonstration nous garantit-elle l'accès à la vérité ? \\[0pt]
La démonstration supprime-t-elle le doute ? \\[0pt]
L'adéquation aux choses suffit-elle à définir la vérité ? \\[0pt]
La déraison \\[0pt]
La désobéissance \\[0pt]
La détermination \\[0pt]
La dialectique \\[0pt]
La dialectique remet-elle en question le principe de contradiction ? \\[0pt]
La différence des sexes est-elle un problème philosophique ? \\[0pt]
La dignité \\[0pt]
La discipline \\[0pt]
La discorde \\[0pt]
La discrétion \\[0pt]
La disharmonie \\[0pt]
La distinction \\[0pt]
La diversité \\[0pt]
La diversité des cultures fait-elle obstacle à l'unité du genre humain ? \\[0pt]
La diversité des langues est-elle un obstacle à l'entente entre les hommes ? \\[0pt]
La diversité des opinions conduit-elle à douter de tout ? \\[0pt]
La division du travail \\[0pt]
L'admiration \\[0pt]
La domination technique de la nature doit-elle susciter la crainte ou l'espoir ? \\[0pt]
La douleur \\[0pt]
La douleur nous apprend-elle quelque chose ? \\[0pt]
La durée \\[0pt]
La durée et l'instant \\[0pt]
La faiblesse \\[0pt]
La familiarité \\[0pt]
La famille \\[0pt]
La famille est-elle un modèle de société ? \\[0pt]
La fatigue \\[0pt]
La fermeté \\[0pt]
La fête \\[0pt]
La fiction \\[0pt]
La fidélité \\[0pt]
La figuration sensible de l'intelligible \\[0pt]
La finalité \\[0pt]
La finalité est-elle nécessaire pour penser le vivant ? \\[0pt]
La fin de la nature \\[0pt]
La fin de la technique se résume-t-elle à son utilité ? \\[0pt]
La fin de l'État \\[0pt]
La fin de l'histoire \\[0pt]
La fin des temps \\[0pt]
La fin du travail \\[0pt]
La fin et les moyens \\[0pt]
La finitude \\[0pt]
La fin justifie-t-elle les moyens ? \\[0pt]
La foi \\[0pt]
La foi est-elle aveugle ? \\[0pt]
La foi est-elle rationnelle ? \\[0pt]
La folie \\[0pt]
La fonction \\[0pt]
La fonction et l'organe \\[0pt]
La force de la loi \\[0pt]
La force de la nature \\[0pt]
La force de la vérité \\[0pt]
La force de l'esprit \\[0pt]
La force de l'État est-elle nécessaire à la liberté des citoyens ? \\[0pt]
La force de l'habitude \\[0pt]
La force des idées \\[0pt]
La force des lois \\[0pt]
La force donne-t-elle des droits ? \\[0pt]
La force du droit \\[0pt]
La force et la violence \\[0pt]
La force et le droit \\[0pt]
La force et le droit s'opposent-ils nécessairement ? \\[0pt]
La franchise \\[0pt]
La fraternité \\[0pt]
La fraternité a-t-elle un sens politique ? \\[0pt]
La fuite du temps \\[0pt]
La fuite du temps est-elle nécessairement un malheur ? \\[0pt]
La généralisation empirique \\[0pt]
La générosité \\[0pt]
La gloire \\[0pt]
La grandeur \\[0pt]
La guerre \\[0pt]
La guerre et la paix \\[0pt]
La guerre peut-elle être juste ? \\[0pt]
La guerre peut-elle être justifiée ? \\[0pt]
La guerre peut-elle être un droit ? \\[0pt]
La haine \\[0pt]
La haine de la raison \\[0pt]
La hiérarchie \\[0pt]
La honte \\[0pt]
La joie \\[0pt]
La jurisprudence \\[0pt]
La juste mesure \\[0pt]
La justice \\[0pt]
La justice a-t-elle un fondement rationnel ? \\[0pt]
La justice distributive \\[0pt]
La justice est-elle de ce monde ? \\[0pt]
La justice est-elle de l'ordre du sentiment ? \\[0pt]
La justice est-elle l'affaire de l'État ? \\[0pt]
La justice est-elle une idée ? \\[0pt]
La justice est-elle une vertu ? \\[0pt]
La justice est-elle un idéal rationnel ? \\[0pt]
La justice et la force \\[0pt]
La justice et la loi \\[0pt]
La justice et la paix \\[0pt]
La justice et le droit \\[0pt]
La justice et l'égalité \\[0pt]
La justice n'est-elle qu'une institution ? \\[0pt]
La justice n'est-elle qu'un idéal ? \\[0pt]
La justice peut-elle être fondée en nature ? \\[0pt]
La justice peut-elle se passer d'institutions ? \\[0pt]
La justice requiert-elle l'infaillibilité des juges ? \\[0pt]
La justice sociale \\[0pt]
La justice suppose-t-elle l'égalité ? \\[0pt]
La lâcheté \\[0pt]
La laideur \\[0pt]
La langue et la parole \\[0pt]
La langue et la pensée \\[0pt]
La légèreté \\[0pt]
La légitime défense \\[0pt]
La lettre et l'esprit \\[0pt]
La liberté \\[0pt]
La liberté a-t-elle un prix ? \\[0pt]
La liberté comporte-t-elle des degrés ? \\[0pt]
La liberté connaît-elle des excès ? \\[0pt]
La liberté de croire \\[0pt]
La liberté de l'artiste \\[0pt]
La liberté de l'interprète \\[0pt]
La liberté de penser \\[0pt]
La liberté d'expression est-elle nécessaire à la liberté de pensée ? \\[0pt]
La liberté d'expression est-elle nécessaire à la liberté de penser ? \\[0pt]
La liberté d'indifférence \\[0pt]
La liberté doit-elle être limitée ? \\[0pt]
La liberté du choix \\[0pt]
La liberté est-elle ce qui définit l'homme ? \\[0pt]
La liberté est-elle contraire au principe de causalité ? \\[0pt]
La liberté est-elle innée ? \\[0pt]
La liberté est-elle le fondement de la responsabilité ? \\[0pt]
La liberté est-elle le pouvoir de refuser ? \\[0pt]
La liberté est-elle une illusion ? \\[0pt]
La liberté est-elle une illusion nécessaire ? \\[0pt]
La liberté est-elle un fait ? \\[0pt]
La liberté et la nécessité \\[0pt]
La liberté et l'égalité sont-elles compatibles ? \\[0pt]
La liberté et le hasard \\[0pt]
La liberté et les libertés \\[0pt]
La liberté et le temps \\[0pt]
La liberté fait-elle de nous des êtres meilleurs ? \\[0pt]
La liberté implique-t-elle l'indifférence ? \\[0pt]
La liberté impose-t-elle des devoirs ? \\[0pt]
La liberté ne s'éprouve-t-elle que dans la solitude ? \\[0pt]
La liberté n'est-elle qu'un droit ? \\[0pt]
La liberté n'est-elle qu'une illusion ? \\[0pt]
La liberté nous rend-elle inexcusables ? \\[0pt]
La liberté peut-elle être prouvée ? \\[0pt]
La liberté peut-elle faire peur ? \\[0pt]
La liberté peut-elle se constater ? \\[0pt]
La liberté peut-elle se prouver ? \\[0pt]
La liberté peut-elle se refuser ? \\[0pt]
La liberté requiert-elle le libre échange ? \\[0pt]
La liberté s'achète-t-elle ? \\[0pt]
La liberté s'apprend-elle ? \\[0pt]
La liberté se mérite-t-elle ? \\[0pt]
La liberté se réduit-elle au libre arbitre ? \\[0pt]
La liberté suppose-t-elle l'absence de déterminisme ? \\[0pt]
La libre interprétation \\[0pt]
L'aliénation \\[0pt]
La limite \\[0pt]
La logique des sens \\[0pt]
La logique est-elle la norme du vrai ? \\[0pt]
La loi \\[0pt]
La loi dit-elle ce qui est juste ? \\[0pt]
La loi doit-elle être la même pour tous ? \\[0pt]
La loi est-elle une garantie contre l'injustice ? \\[0pt]
La loi et la coutume \\[0pt]
La loi et l'ordre \\[0pt]
La loi peut-elle changer les mœurs ? \\[0pt]
La loyauté \\[0pt]
L'altérité \\[0pt]
L'altruisme \\[0pt]
L'altruisme n'est-il qu'un égoïsme bien compris ? \\[0pt]
La machine \\[0pt]
La machine fournit-elle un modèle pour penser le vivant ? \\[0pt]
La main \\[0pt]
La main et l'esprit \\[0pt]
La maîtrise de soi \\[0pt]
La majorité doit-elle toujours l'emporter ? \\[0pt]
La majorité, force ou droit ? \\[0pt]
La majorité peut-elle être tyrannique ? \\[0pt]
La maladie \\[0pt]
La maladie est-elle à l'organisme vivant ce que la panne est à la machine ? \\[0pt]
La maladie est-elle indispensable à la connaissance du vivant ? \\[0pt]
La marginalité \\[0pt]
La mathématisation du réel \\[0pt]
La matière \\[0pt]
La matière est-elle connaissable ? \\[0pt]
La matière est-elle plus aisée à connaître que l'esprit ? \\[0pt]
La matière est-elle plus facile à connaître que l'esprit ? \\[0pt]
La matière et la vie \\[0pt]
La matière et l'esprit \\[0pt]
La matière et l'esprit peuvent-ils constituer une seule chose ? \\[0pt]
La matière n'est-elle que ce que l'on perçoit ? \\[0pt]
La matière n'est-elle qu'une idée ? \\[0pt]
La matière peut-elle agir ? \\[0pt]
La maturité \\[0pt]
La mauvaise conscience \\[0pt]
La mauvaise foi \\[0pt]
La mauvaise volonté \\[0pt]
L'ambiguïté \\[0pt]
L'ambiguïté des mots peut-elle être heureuse ? \\[0pt]
La méchanceté \\[0pt]
L'âme des bêtes \\[0pt]
L'âme et le corps sont-ils une seule et même chose ? \\[0pt]
La méfiance \\[0pt]
L'âme jouit-elle d'une vie propre ? \\[0pt]
La mélancolie \\[0pt]
La mémoire \\[0pt]
La mémoire collective \\[0pt]
La mémoire est-elle une fonction vitale ? \\[0pt]
La mémoire et l'oubli \\[0pt]
La mesure \\[0pt]
La mesure du temps \\[0pt]
La métamorphose \\[0pt]
La métaphore n'est-elle qu'un ornement du discours ? \\[0pt]
La métaphysique est-elle un discours sans objet ? \\[0pt]
La méthode \\[0pt]
La méthode expérimentale est-elle appropriée à l'étude du vivant ? \\[0pt]
La misère \\[0pt]
L'amitié \\[0pt]
L'amitié est-elle une vertu ? \\[0pt]
L'amitié peut-elle obliger ? \\[0pt]
La modération \\[0pt]
La modestie \\[0pt]
La monnaie n'est-elle qu'un moyen d'échange ? \\[0pt]
La morale \\[0pt]
La morale a-t-elle à décider de la sexualité ? \\[0pt]
La morale a-t-elle besoin de la notion de sainteté ? \\[0pt]
La morale a-t-elle besoin d'un fondement ? \\[0pt]
La morale a-t-elle sa place dans l'économie ? \\[0pt]
La morale consiste-t-elle à respecter le droit ? \\[0pt]
La morale dépend-elle de la culture ? \\[0pt]
La morale doit-elle être rationnelle ? \\[0pt]
La morale est-elle affaire de convention ? \\[0pt]
La morale est-elle affaire de sentiment ? \\[0pt]
La morale est-elle condamnée à n'être qu'un champ de bataille ? \\[0pt]
La morale est-elle désintéressée ? \\[0pt]
La morale est-elle en conflit avec le désir ? \\[0pt]
La morale est-elle une affaire de raison ? \\[0pt]
La morale est-elle une affaire solitaire ? \\[0pt]
La morale est-elle un fait de culture ? \\[0pt]
La morale et la politique \\[0pt]
La morale et la religion visent-elles les mêmes fins ? \\[0pt]
La morale et les mœurs \\[0pt]
La morale n'est-elle qu'un art de vivre ? \\[0pt]
La morale n'est-elle qu'un dressage ? \\[0pt]
La morale n'est-elle qu'un ensemble de conventions ? \\[0pt]
La morale peut-elle se définir comme l'art d'être heureux ? \\[0pt]
La morale peut-elle se fonder sur les sentiments ? \\[0pt]
La morale peut-elle s'enseigner ? \\[0pt]
La morale s'enseigne-t-elle ? \\[0pt]
La morale s'oppose-t-elle à la politique ? \\[0pt]
La moralité est-elle utile à la vie sociale ? \\[0pt]
La moralité se juge-t-elle aux actes ? \\[0pt]
La mort \\[0pt]
La mort d'autrui \\[0pt]
La mort de Dieu \\[0pt]
L'amour de la liberté \\[0pt]
L'amour de l'art \\[0pt]
L'amour de la vie \\[0pt]
L'amour de soi \\[0pt]
L'amour de soi est-il immoral ? \\[0pt]
L'amour du pouvoir est-il compatible avec le dévouement à la chose publique ? \\[0pt]
L'amour du travail \\[0pt]
L'amour est-il toujours une passion ? \\[0pt]
L'amour et la haine \\[0pt]
L'amour et l'amitié \\[0pt]
L'amour et le devoir \\[0pt]
L'amour et le respect \\[0pt]
L'amour fou \\[0pt]
L'amour peut-il être raisonnable ? \\[0pt]
L'amour peut-il être un devoir ? \\[0pt]
L'amour-propre \\[0pt]
La multitude \\[0pt]
L'anachronisme \\[0pt]
L'analogie \\[0pt]
L'analyse du langage ordinaire peut-elle avoir un intérêt philosophique ? \\[0pt]
L'anarchie \\[0pt]
La nation \\[0pt]
La nature \\[0pt]
La nature a-t-elle des droits ? \\[0pt]
La nature a-t-elle une histoire ? \\[0pt]
La nature a-t-elle un ordre ? \\[0pt]
La nature donne-t-elle un sens à notre existence ? \\[0pt]
La nature est-elle écrite en langage mathématique ? \\[0pt]
La nature est-elle prévisible ? \\[0pt]
La nature est-elle une grande artiste ? \\[0pt]
La nature est-elle une idée ? \\[0pt]
La nature est-elle une invention de l'homme ? \\[0pt]
La nature est-elle une ressource ? \\[0pt]
La nature est-elle un guide ? \\[0pt]
La nature est-elle un modèle ? \\[0pt]
La nature existe-t-elle ? \\[0pt]
La nature fait-elle bien les choses ? \\[0pt]
La nature ne fait-elle rien en vain ? \\[0pt]
La nature nous domine-t-elle ? \\[0pt]
La nature nous donne-t-elle des commandements ? \\[0pt]
La nature nous indique-t-elle ce qui est bon ? \\[0pt]
La nature peut-elle avoir des droits ? \\[0pt]
La nature peut-elle constituer une norme ? \\[0pt]
La nature peut-elle être détruite ? \\[0pt]
La nature peut-elle être un modèle ? \\[0pt]
La nature peut-elle nous indiquer ce que nous devons faire ? \\[0pt]
La nature reprend-elle toujours ses droits ? \\[0pt]
La nécessité \\[0pt]
La négation \\[0pt]
La neutralité de l'État \\[0pt]
Langage et communication \\[0pt]
Langage et logique \\[0pt]
Langage et passions \\[0pt]
Langage et pensée \\[0pt]
Langage et pouvoir \\[0pt]
Langage et société \\[0pt]
L'angoisse \\[0pt]
L'angoisse et la peur \\[0pt]
L'animal \\[0pt]
L'animal a-t-il des droits ? \\[0pt]
L'animal est-il une personne ? \\[0pt]
L'animal et l'homme \\[0pt]
L'animal nous apprend-il quelque chose sur l'homme ? \\[0pt]
L'anomalie \\[0pt]
La non-violence \\[0pt]
La norme \\[0pt]
La nostalgie \\[0pt]
La notion de démocratie représentative est-elle contradictoire ? \\[0pt]
La notion de droit international est-elle contradictoire ? \\[0pt]
La notion de progrès moral a-t-elle encore un sens ? \\[0pt]
La notion d'histoire commune a-t-elle un sens ? \\[0pt]
La nouveauté \\[0pt]
La paix \\[0pt]
La paix est-elle l'absence de guerres ? \\[0pt]
La paix est-elle le plus grand des biens ? \\[0pt]
La paix met-elle fin à l'histoire ? \\[0pt]
La paix n'est-elle que l'absence de guerre ? \\[0pt]
La paix sociale \\[0pt]
La paix sociale est-elle le but de la politique ? \\[0pt]
La parole \\[0pt]
La parole donnée \\[0pt]
La parole et l'action \\[0pt]
La parole et l'écriture \\[0pt]
La parole et le geste \\[0pt]
La parole intérieure \\[0pt]
La parole nous libère-t-elle ? \\[0pt]
La parole peut-elle agir ? \\[0pt]
La passion de la connaissance \\[0pt]
La passion de la liberté \\[0pt]
La passion de la vérité \\[0pt]
La passion de l'égalité \\[0pt]
La passion est-elle immorale ? \\[0pt]
La passion est-elle l'ennemi de la raison ? \\[0pt]
La passion exclut-elle la lucidité ? \\[0pt]
La passion n'est-elle que souffrance ? \\[0pt]
La passivité \\[0pt]
La patience \\[0pt]
La patience de la pensée \\[0pt]
La patience est-elle une vertu ? \\[0pt]
La pauvreté \\[0pt]
La pauvreté est-elle une injustice ? \\[0pt]
La peine \\[0pt]
La pénibilité du travail \\[0pt]
La pensée \\[0pt]
La pensée a-t-elle besoin d'images ? \\[0pt]
La pensée doit-elle se soumettre aux règles de la logique ? \\[0pt]
La pensée et la conscience sont-elles une seule et même chose ? \\[0pt]
La pensée peut-elle devenir une technique ? \\[0pt]
La pensée peut-elle se passer de mots ? \\[0pt]
La pensée philosophique doit-elle être une pensée systématique ? \\[0pt]
L'aperception \\[0pt]
La perception \\[0pt]
La perception construit-elle son objet ? \\[0pt]
La perception de l'espace est-elle innée ou acquise ? \\[0pt]
La perception est-elle le premier degré de la connaissance ? \\[0pt]
La perception est-elle l'interprétation du réel ? \\[0pt]
La perception est-elle une interprétation ? \\[0pt]
La perception est-elle un mode de connaissance ? \\[0pt]
La perception me donne-t-elle le réel ? \\[0pt]
La perception peut-elle s'éduquer ? \\[0pt]
La perfection \\[0pt]
La perfection est-elle désirable ? \\[0pt]
La permanence \\[0pt]
La personne et l'individu \\[0pt]
La perspective \\[0pt]
La peur \\[0pt]
La peur de la science \\[0pt]
La peur de la technique \\[0pt]
La philosophie a-t-elle un objet qui lui soit propre ? \\[0pt]
La philosophie est-elle une méditation de la mort ? \\[0pt]
La philosophie est-elle une méditation de la vie ou de la mort ? \\[0pt]
La philosophie et son histoire \\[0pt]
La philosophie peut-elle ne pas parler de Dieu ? \\[0pt]
La philosophie rend-elle inefficace la propagande ? \\[0pt]
La pitié \\[0pt]
La pitié fonde-t-elle la morale ? \\[0pt]
La place de l'animal \\[0pt]
La plaisanterie \\[0pt]
La pluralité des arts \\[0pt]
La pluralité des interprétations \\[0pt]
La pluralité des langues \\[0pt]
La pluralité des religions \\[0pt]
La pluralité des sciences \\[0pt]
La pluralité des vérités condamne-t-elle l'idée de vérité ? \\[0pt]
La police \\[0pt]
La politesse \\[0pt]
La politesse est-elle une vertu ? \\[0pt]
La politique \\[0pt]
La politique : affaire de compétence ? \\[0pt]
La politique a-t-elle pour but de nous faire vivre dans un monde meilleur ? \\[0pt]
La politique consiste-t-elle à faire cause commune ? \\[0pt]
La politique doit-elle avoir pour visée le bonheur ? \\[0pt]
La politique doit-elle protéger la liberté des citoyens ? \\[0pt]
La politique doit-elle refuser l'utopie ? \\[0pt]
La politique est-elle affaire de compétence ? \\[0pt]
La politique est-elle affaire d'expérience ou de théorie ? \\[0pt]
La politique est-elle l'affaire des spécialistes ? \\[0pt]
La politique est-elle l'affaire de tous ? \\[0pt]
La politique est-elle l'art de convaincre le peuple ? \\[0pt]
La politique est-elle un art ? \\[0pt]
La politique est-elle une affaire d'experts ? \\[0pt]
La politique est-elle une science ? \\[0pt]
La politique est-elle une technique ? \\[0pt]
La politique est-elle un métier ? \\[0pt]
La politique et la guerre \\[0pt]
La politique et le bonheur \\[0pt]
La politique et le pouvoir \\[0pt]
La politique n'est-elle que l'art de conquérir et de conserver le pouvoir ? \\[0pt]
La postérité \\[0pt]
La poursuite de mon intérêt m'oppose-t-elle aux autres ? \\[0pt]
L'apparence \\[0pt]
L'apparence est-elle toujours trompeuse ? \\[0pt]
L'apparence fait-elle partie de la réalité ? \\[0pt]
L'apprentissage \\[0pt]
L'apprentissage de la liberté \\[0pt]
La précarité \\[0pt]
La présence d'esprit \\[0pt]
La prière \\[0pt]
L'\emph{a priori} \\[0pt]
La privation \\[0pt]
La promesse \\[0pt]
La proportion \\[0pt]
L'à propos \\[0pt]
La propriété \\[0pt]
La propriété et le travail \\[0pt]
La prudence \\[0pt]
La pudeur \\[0pt]
La puissance \\[0pt]
La puissance de la raison \\[0pt]
La puissance du désir \\[0pt]
La puissance technique est- elle sans limites ? \\[0pt]
La punition \\[0pt]
La pureté \\[0pt]
La quantité et la qualité \\[0pt]
La question du sens de la vie peut-elle être sans réponses ? \\[0pt]
La question « qui suis-je » admet-elle une réponse exacte ? \\[0pt]
La raison \\[0pt]
La raison a-t-elle des limites ? \\[0pt]
La raison a-t-elle pour fin la connaissance ? \\[0pt]
La raison a-t-elle toujours raison ? \\[0pt]
La raison a-t-elle une histoire ? \\[0pt]
La raison d'État \\[0pt]
La raison d'État peut-elle être justifiée ? \\[0pt]
La raison d'être \\[0pt]
La raison doit-elle critiquer la croyance ? \\[0pt]
La raison doit-elle être cultivée ? \\[0pt]
La raison doit-elle être notre guide ? \\[0pt]
La raison doit-elle faire une place à la croyance ? \\[0pt]
La raison doit-elle se soumettre au réel ? \\[0pt]
La raison engendre-t-elle des illusions ? \\[0pt]
La raison épuise-t-elle le réel ? \\[0pt]
La raison est-elle impersonnelle ? \\[0pt]
La raison est-elle l'esclave du désir ? \\[0pt]
La raison est-elle plus fiable que l'expérience ? \\[0pt]
La raison est-elle seulement affaire de logique ? \\[0pt]
La raison est-elle une valeur ? \\[0pt]
La raison et la démonstration \\[0pt]
La raison et la liberté \\[0pt]
La raison et le bonheur \\[0pt]
La raison et le réel \\[0pt]
La raison et l'existence \\[0pt]
La raison et l'expérience \\[0pt]
La raison et l'irrationnel \\[0pt]
La raison ne connaît-elle du réel que ce qu'elle y met d'elle-même ? \\[0pt]
La raison ne veut-elle que connaître ? \\[0pt]
La raison peut-elle entrer en conflit avec elle-même ? \\[0pt]
La raison peut-elle entrer en contradiction avec elle-même ? \\[0pt]
La raison peut-elle errer ? \\[0pt]
La raison peut-elle nous commander de croire ? \\[0pt]
La raison peut-elle nous induire en erreur ? \\[0pt]
La raison peut-elle rendre compte du réel entièrement ? \\[0pt]
La raison peut-elle s'aveugler elle-même ? \\[0pt]
La raison peut-elle se contredire ? \\[0pt]
La raison peut-elle servir le mal ? \\[0pt]
La raison peut-elle s'opposer à elle-même ? \\[0pt]
La raison suffisante \\[0pt]
La raison transforme-t-elle le réel ? \\[0pt]
La rationalité \\[0pt]
La rationalité se confond-elle avec la systématicité ? \\[0pt]
La rationalité technique \\[0pt]
L'arbitraire \\[0pt]
La réalisation du devoir exclut-elle toute forme de plaisir ? \\[0pt]
La réalité des idées \\[0pt]
La réalité des phénomènes \\[0pt]
La réalité du désordre \\[0pt]
La réalité du passé \\[0pt]
La réalité du temps \\[0pt]
La réalité du temps se réduit-elle à la conscience que nous en avons ? \\[0pt]
La réalité n'est-elle qu'une construction ? \\[0pt]
La réalité nourrit-elle la fiction ? \\[0pt]
La réalité sensible \\[0pt]
La recherche de la vérité a-t- elle une fin ? \\[0pt]
La recherche de la vérité peut-elle être désintéressée ? \\[0pt]
La recherche de la vérité peut-elle être une passion ? \\[0pt]
La recherche du bonheur \\[0pt]
La recherche du bonheur est-elle égoïste ? \\[0pt]
La recherche du bonheur est-elle un idéal égoïste ? \\[0pt]
La recherche du bonheur est-il un guide suffisant dans la vie ? \\[0pt]
La recherche du bonheur peut-elle être un devoir ? \\[0pt]
La réciprocité \\[0pt]
La reconnaissance \\[0pt]
La reconnaissance de la diversité des cultures condamne-t-elle au relativisme ? \\[0pt]
La réflexion \\[0pt]
La réfutation \\[0pt]
La règle et l'exception \\[0pt]
La régression \\[0pt]
La religion \\[0pt]
La religion a-t-elle des vertus ? \\[0pt]
La religion a-t-elle les mêmes fins que la morale ? \\[0pt]
La religion a-t-elle une fonction de cohésion sociale ? \\[0pt]
La religion conduit-elle l'homme au-delà de lui-même ? \\[0pt]
La religion divise-t-elle les hommes ? \\[0pt]
La religion est-elle à craindre ? \\[0pt]
La religion est-elle au fondement de la morale ? \\[0pt]
La religion est-elle contraire à la raison ? \\[0pt]
La religion est-elle fondée sur la peur de la mort ? \\[0pt]
La religion est-elle l'asile de l'ignorance ? \\[0pt]
La religion est-elle relation à l'absolu ? \\[0pt]
La religion est-elle une affaire privée ? \\[0pt]
La religion est-elle une consolation pour les hommes ? \\[0pt]
La religion est-elle une production culturelle comme les autres ? \\[0pt]
La religion est-elle un facteur de lien social ? \\[0pt]
La religion est-elle un instrument de pouvoir ? \\[0pt]
La religion et la croyance \\[0pt]
La religion implique-t-elle la croyance en un être divin ? \\[0pt]
La religion impose t-elle un joug salutaire à l'intelligence ? \\[0pt]
La religion naturelle \\[0pt]
La religion n'est-elle que l'affaire des croyants ? \\[0pt]
La religion n'est-elle qu'une affaire privée ? \\[0pt]
La religion n'est-elle qu'un fait de culture ? \\[0pt]
La religion outrepasse-t-elle les limites de la raison ? \\[0pt]
La religion peut-elle n'être qu'une affaire privée ? \\[0pt]
La religion peut-elle se passer de transcendant ? \\[0pt]
La religion relève-t-elle de l'irrationnel ? \\[0pt]
La religion relève-t-elle de l'opinion ? \\[0pt]
La religion relie-t-elle les hommes ? \\[0pt]
La religion rend-elle l'homme heureux ? \\[0pt]
La religion rend-elle meilleur ? \\[0pt]
La religion repose-t-elle sur une illusion ? \\[0pt]
La religion se distingue-t-elle de la superstition ? \\[0pt]
La religion se réduit-elle à la foi ? \\[0pt]
La répétition \\[0pt]
La représentation \\[0pt]
La représentation artistique \\[0pt]
La représentation politique \\[0pt]
La reproduction \\[0pt]
La responsabilité \\[0pt]
La responsabilité collective \\[0pt]
La responsabilité est-elle une fiction juridique ? \\[0pt]
La responsabilité politique \\[0pt]
La responsabilité politique n'est-elle le fait que de ceux qui gouvernent ? \\[0pt]
La réussite \\[0pt]
La révolte \\[0pt]
La révolution \\[0pt]
L'argent \\[0pt]
L'argent est-il la mesure de tout échange ? \\[0pt]
La rigueur \\[0pt]
La rigueur des lois \\[0pt]
La rigueur des lois ? \\[0pt]
L'art \\[0pt]
L'art ajoute-t-il quelque chose à la vie ? \\[0pt]
L'art a-t-il à être populaire ? \\[0pt]
L'art a-t-il besoin de théorie ? \\[0pt]
L'art a-t-il pour fin la vérité ? \\[0pt]
L'art a-t-il pour fin le plaisir ? \\[0pt]
L'art a-t-il pour fonction de sublimer le réel ? \\[0pt]
L'art a-t-il une histoire ? \\[0pt]
L'art a-t-il un rôle à jouer dans l'éducation ? \\[0pt]
L'art change-t-il la vie ? \\[0pt]
L'art de gouverner \\[0pt]
L'art de juger \\[0pt]
L'art de persuader \\[0pt]
L'art de vivre \\[0pt]
L'art d'interpréter \\[0pt]
L'art doit-il nécessairement représenter la réalité ? \\[0pt]
L'art donne-t-il à penser ? \\[0pt]
L'art donne-t-il à rêver ou à penser ? \\[0pt]
L'art donne-t-il nécessairement lieu à la production d'une œuvre ? \\[0pt]
L'art éduque-t-il la perception ? \\[0pt]
L'art est-il affaire d'apparence ? \\[0pt]
L'art est-il désintéressé ? \\[0pt]
L'art est-il hors du temps ? \\[0pt]
L'art est-il le règne des apparences ? \\[0pt]
L'art est-il moins nécessaire que la science ? \\[0pt]
L'art est-il subversif ? \\[0pt]
L'art est-il une affaire sérieuse ? \\[0pt]
L'art est-il une forme de langage ? \\[0pt]
L'art est-il une histoire ? \\[0pt]
L'art est-il universel ? \\[0pt]
L'art est-il un jeu ? \\[0pt]
L'art est-il un langage ? \\[0pt]
L'art est-il un luxe ? \\[0pt]
L'art est-il un moyen de connaître ? \\[0pt]
L'art est-il un refuge ? \\[0pt]
L'art et la manière \\[0pt]
L'art et la morale \\[0pt]
L'art et la raison \\[0pt]
L'art et la société \\[0pt]
L'art et la technique \\[0pt]
L'art et la vie \\[0pt]
L'art et le beau \\[0pt]
L'art et le jeu \\[0pt]
L'art et le réel \\[0pt]
L'art et le sacré \\[0pt]
L'art et le travail \\[0pt]
L'art et l'illusion \\[0pt]
L'art et l'interprétation \\[0pt]
L'art et l'invisible \\[0pt]
L'art éveille-t-il des émotions spécifiques ? \\[0pt]
L'art exprime-t-il ce que nous ne saurions dire ? \\[0pt]
L'artifice \\[0pt]
L'artificiel \\[0pt]
L'artisan et l'ingénieur \\[0pt]
L'artiste a-t-il besoin de modèle ? \\[0pt]
L'artiste doit-il être de son temps ? \\[0pt]
L'artiste doit-il être original ? \\[0pt]
L'artiste doit-il se donner des modèles ? \\[0pt]
L'artiste doit-il se soucier du goût du public ? \\[0pt]
L'artiste est-il souverain ? \\[0pt]
L'artiste est-il un travailleur ? \\[0pt]
L'artiste et l'artisan \\[0pt]
L'artiste sait-il ce qu'il fait ? \\[0pt]
L'artiste travaille-t-il ? \\[0pt]
L'art n'est-il qu'un mode d'expression subjectif ? \\[0pt]
L'art n'est qu'une affaire de goût ? \\[0pt]
L'art nous détourne-t-il de la réalité ? \\[0pt]
L'art nous fait-il mieux percevoir le réel ? \\[0pt]
L'art nous mène-t-il au vrai ? \\[0pt]
L'art nous réconcilie-t-il avec le monde ? \\[0pt]
L'art parachève-t-il la nature ? \\[0pt]
L'art participe-t-il à la vie politique ? \\[0pt]
L'art peut-il être conceptuel ? \\[0pt]
L'art peut-il être réaliste \\[0pt]
L'art peut-il être sans œuvre ? \\[0pt]
L'art peut-il mourir ? \\[0pt]
L'art peut-il ne pas être sacré ? \\[0pt]
L'art peut-il n'être aucunement mimétique ? \\[0pt]
L'art peut-il s'enseigner ? \\[0pt]
L'art peut-il se passer de la beauté ? \\[0pt]
L'art peut-il se passer de règles ? \\[0pt]
L'art peut-il se passer d'œuvres ? \\[0pt]
L'art pour l'art \\[0pt]
L'art progresse-t-il ? \\[0pt]
L'art prolonge-t-il la nature ? \\[0pt]
L'art rend-il heureux ? \\[0pt]
L'art rend-il les hommes meilleurs ? \\[0pt]
L'art s'adresse-t-il à tous ? \\[0pt]
L'art s'apprend-il ? \\[0pt]
La ruse \\[0pt]
La sagesse et la passion \\[0pt]
La santé \\[0pt]
La santé n'est-elle que l'absence de maladie ? \\[0pt]
La satisfaction \\[0pt]
L'ascétisme est-il une vertu ? \\[0pt]
La science a-t-elle besoin d'imagination ? \\[0pt]
La science a-t-elle besoin d'une méthode ? \\[0pt]
La science a-t-elle le monopole de la raison ? \\[0pt]
La science a-t-elle pour fin de prévoir ? \\[0pt]
La science a-t-elle réponse à tout ? \\[0pt]
La science commence-t-elle avec la perception ? \\[0pt]
La science du vivant peut-elle se passer de l'idée de finalité ? \\[0pt]
La science est-elle indépendante de toute métaphysique ? \\[0pt]
La science est-elle inhumaine ? \\[0pt]
La science est-elle le lieu de la vérité ? \\[0pt]
La science est-elle une connaissance du réel ? \\[0pt]
La science nous éloigne-t-elle de la religion ? \\[0pt]
La science nous indique-t-elle ce que nous devons faire ? \\[0pt]
La science permet-elle de comprendre le monde ? \\[0pt]
La science permet-elle de mieux comprendre la religion ? \\[0pt]
La science peut-elle être une métaphysique ? \\[0pt]
La science peut-elle produire des croyances ? \\[0pt]
La science peut-elle se passer de l'idée de finalité ? \\[0pt]
La science peut-elle se passer d'hypothèses ? \\[0pt]
La science politique \\[0pt]
La science rend-elle la religion caduque ? \\[0pt]
La science se limite-t-elle à constater les faits ? \\[0pt]
La science s'oppose-t-elle à la religion ? \\[0pt]
La sensibilité \\[0pt]
La sensibilité se cultive-t-elle ? \\[0pt]
La sérénité \\[0pt]
La servitude \\[0pt]
La servitude peut-elle être volontaire ? \\[0pt]
La servitude volontaire \\[0pt]
La simplicité \\[0pt]
La sincérité \\[0pt]
La société \\[0pt]
La société civile \\[0pt]
La société doit-elle reconnaître les désirs individuels ? \\[0pt]
La société est-elle un grand marché ? \\[0pt]
La société est-elle un organisme ? \\[0pt]
La société et le droit \\[0pt]
La société et les échanges \\[0pt]
La société et l'État \\[0pt]
La société et l'individu \\[0pt]
La société fait-elle l'homme ? \\[0pt]
La société peut-elle être l'objet d'une science ? \\[0pt]
La société peut-elle se passer de l'État ? \\[0pt]
La société repose-t-elle sur l'altruisme ? \\[0pt]
La solidarité \\[0pt]
La solidarité est-elle naturelle ? \\[0pt]
La solitude \\[0pt]
La sollicitude \\[0pt]
La souffrance au travail \\[0pt]
La souffrance d'autrui \\[0pt]
La souffrance d'autrui m'importe-t-elle ? \\[0pt]
La souffrance peut-elle être un mode de connaissance ? \\[0pt]
La soumission à l'autorité \\[0pt]
La soumission volontaire \\[0pt]
La souveraineté \\[0pt]
La souveraineté de l'État \\[0pt]
La souveraineté peut-elle être limitée ? \\[0pt]
La souveraineté peut-elle se partager ? \\[0pt]
La spontanéité \\[0pt]
L'association des idées \\[0pt]
La superstition \\[0pt]
La sympathie \\[0pt]
La sympathie peut-elle tenir lieu de morale ? \\[0pt]
La tautologie \\[0pt]
La technique \\[0pt]
La technique accroît-elle notre liberté ? \\[0pt]
La technique a-t-elle sa place en politique ? \\[0pt]
La technique a-t-elle une finalité ? \\[0pt]
La technique augmente-t-elle notre puissance d'agir ? \\[0pt]
La technique change-t-elle l'homme ? \\[0pt]
La technique change-t-elle notre rapport au monde ? \\[0pt]
La technique déshumanise-t-elle le monde ? \\[0pt]
La technique détermine-t-elle les rapports sociaux ? \\[0pt]
La technique doit-elle nous libérer du travail ? \\[0pt]
La technique doit-elle permettre de dépasser les limites de l'humain ? \\[0pt]
La technique donne-t-elle une illusion de pouvoir ? \\[0pt]
La technique est-elle civilisatrice ? \\[0pt]
La technique est-elle contre nature ? \\[0pt]
La technique est-elle le propre de l'homme ? \\[0pt]
La technique est-elle libératrice ? \\[0pt]
La technique est-elle neutre ? \\[0pt]
La technique est-elle une forme de savoir ? \\[0pt]
La technique est-elle un savoir ? \\[0pt]
La technique et la liberté \\[0pt]
La technique et la morale \\[0pt]
La technique et le corps \\[0pt]
La technique et le travail \\[0pt]
La technique facilite-t-elle la vie ? \\[0pt]
La technique fait-elle des miracles ? \\[0pt]
La technique imite-t-elle la nature ? \\[0pt]
La technique invente-t-elle des formes ? \\[0pt]
La technique libère-t-elle les hommes ? \\[0pt]
La technique met-elle le monde à notre disposition ? \\[0pt]
La technique ne fait-elle qu'appliquer la science ? \\[0pt]
La technique ne pose-t-elle que des problèmes techniques ? \\[0pt]
La technique n'est-elle pour l'homme qu'un moyen ? \\[0pt]
La technique n'est-elle qu'une application de la science ? \\[0pt]
La technique n'est-elle qu'une ruse ? \\[0pt]
La technique n'est-elle qu'un moyen ? \\[0pt]
La technique n'est-elle qu'un moyen de domination ? \\[0pt]
La technique n'est-elle qu'un outil au service de l'homme ? \\[0pt]
La technique n'existe-elle que pour satisfaire des besoins ? \\[0pt]
La technique nous délivre-t-elle d'un rapport irrationnel au monde ? \\[0pt]
La technique nous éloigne-t-elle de la nature ? \\[0pt]
La technique nous éloigne-t-elle de la réalité ? \\[0pt]
La technique nous libère-t-elle ? \\[0pt]
La technique nous libère-t-elle du travail ? \\[0pt]
La technique nous oppose-t-elle à la nature ? \\[0pt]
La technique nous permet-elle de comprendre la nature ? \\[0pt]
La technique pense-t-elle ? \\[0pt]
La technique permet-elle de réaliser tous les désirs ? \\[0pt]
La technique peut-elle être tenue pour la forme moderne de la culture ? \\[0pt]
La technique peut-elle respecter la nature ? \\[0pt]
La technique peut-elle se déduire de la science ? \\[0pt]
La technique peut-elle se passer de la science ? \\[0pt]
La technique pose-t-elle plus de problèmes qu'elle n'en résout ? \\[0pt]
La technique produit-elle autre chose que condition de la pensée ? \\[0pt]
La technique produit-elle son propre savoir ? \\[0pt]
La technique provoque-t-elle inévitablement des catastrophes ? \\[0pt]
La technique sert-elle nos désirs ? \\[0pt]
La technique s'oppose-t-elle à la nature ? \\[0pt]
La technocratie \\[0pt]
La tentation \\[0pt]
La terre est-elle un bien commun ? \\[0pt]
L'athéisme condamne-t-il l'existence à l'absurdité ? \\[0pt]
L'athéisme est-il une croyance ? \\[0pt]
La théorie \\[0pt]
La théorie et la pratique \\[0pt]
La théorie et l'expérience \\[0pt]
La théorie nous éloigne-t-elle de la réalité ? \\[0pt]
La théorie scientifique \\[0pt]
La tolérance \\[0pt]
La tolérance est-elle une vertu ? \\[0pt]
La tradition \\[0pt]
La traduction \\[0pt]
La transgression \\[0pt]
L'attente \\[0pt]
L'attention \\[0pt]
L'attention caractérise-t-elle la conscience ? \\[0pt]
L'attitude religieuse \\[0pt]
La tyrannie \\[0pt]
La tyrannie des désirs \\[0pt]
L'audace \\[0pt]
L'au-delà \\[0pt]
L'authenticité \\[0pt]
L'autobiographie \\[0pt]
L'automatisation \\[0pt]
L'autonomie \\[0pt]
L'autoportrait \\[0pt]
L'autorité \\[0pt]
L'autorité de la chose jugée \\[0pt]
L'autorité de la loi \\[0pt]
L'autorité de la science \\[0pt]
L'autorité du droit \\[0pt]
La valeur \\[0pt]
La valeur de la culture \\[0pt]
La valeur de la raison \\[0pt]
La valeur de la vérité \\[0pt]
La valeur du don \\[0pt]
La valeur d'une action se mesure-t-elle à sa réussite ? \\[0pt]
La valeur du travail \\[0pt]
La valeur morale de l'amour \\[0pt]
La valeur morale d'une action se juge-t-elle à ses conséquences ? \\[0pt]
La vanité \\[0pt]
L'avant-garde \\[0pt]
La vengeance \\[0pt]
L'avenir \\[0pt]
L'avenir a-t-il une réalité ? \\[0pt]
L'avenir est-il incertain ? \\[0pt]
L'avenir est-il l'affaire de la pensée ? \\[0pt]
L'avenir est-il une page blanche ? \\[0pt]
L'avenir peut-il être objet de connaissance ? \\[0pt]
La vérification \\[0pt]
La vérification expérimentale \\[0pt]
La vérification fait-elle la vérité ? \\[0pt]
La vérité \\[0pt]
La vérité a-t-elle une histoire ? \\[0pt]
La vérité doit-elle toujours être démontrée ? \\[0pt]
La vérité donne-t-elle le droit d'être injuste ? \\[0pt]
La vérité échappe-t-elle au temps ? \\[0pt]
La vérité est-elle affaire de cohérence ? \\[0pt]
La vérité est-elle contraignante ? \\[0pt]
La vérité est-elle la valeur suprême ? \\[0pt]
La vérité est-elle libératrice ? \\[0pt]
La vérité est-elle préférable à l'illusion ? \\[0pt]
La vérité est-elle une ? \\[0pt]
La vérité est-elle une idole ? \\[0pt]
La vérité est-elle une valeur ? \\[0pt]
La vérité et les vérités \\[0pt]
La vérité historique \\[0pt]
La vérité n'est-elle qu'un idéal inaccessible ? \\[0pt]
La vérité nous appartient-elle ? \\[0pt]
La vérité nous contraint-elle ? \\[0pt]
La vérité peut-elle être relative ? \\[0pt]
La vérité peut-elle faire peur ? \\[0pt]
La vérité peut-elle laisser indifférent ? \\[0pt]
La vérité peut-elle se définir par le consensus ? \\[0pt]
La vérité peut-elle se discuter ? \\[0pt]
La vérité rend-elle heureux ? \\[0pt]
La vérité rend-elle libre ? \\[0pt]
La vérité requiert-elle du courage ? \\[0pt]
La vérité se discute-t-elle ? \\[0pt]
La vertu \\[0pt]
La vertu peut-elle être excessive ? \\[0pt]
La vertu peut-elle s'enseigner ? \\[0pt]
La vertu requiert-elle l'idée de bien ? \\[0pt]
La vie de l'esprit \\[0pt]
La vie de plaisirs \\[0pt]
La vie en société impose-t-elle de n'être pas soi-même ? \\[0pt]
La vie en société menace-t-elle la liberté ? \\[0pt]
La vie est-elle le bien le plus précieux ? \\[0pt]
La vie est-elle sacrée ? \\[0pt]
La vie et le vivant \\[0pt]
La vie heureuse \\[0pt]
« La vieillesse est un naufrage » \\[0pt]
La vie intérieure \\[0pt]
La vie morale \\[0pt]
La vie peut-elle être objet de science ? \\[0pt]
La vie psychique \\[0pt]
La vie sauvage \\[0pt]
La vie sociale \\[0pt]
La vie sociale est-elle toujours conflictuelle ? \\[0pt]
La violence \\[0pt]
La violence a-t-elle un sens dans l'histoire ? \\[0pt]
La violence de l'État \\[0pt]
La violence de l'interprétation \\[0pt]
La violence du désir \\[0pt]
La violence engendre-t-elle toujours la violence ? \\[0pt]
La violence est-elle la condition du progrès ? \\[0pt]
La violence est-elle le fondement du droit ? \\[0pt]
La violence est-elle le prix du progrès ? \\[0pt]
La violence est-elle toujours destructrice ? \\[0pt]
La violence est-elle toujours injustifiable ? \\[0pt]
La violence peut-elle avoir raison ? \\[0pt]
La violence peut-elle être gratuite ? \\[0pt]
La violence verbale \\[0pt]
La virtuosité \\[0pt]
La vision peut-elle être le modèle de toute connaissance ? \\[0pt]
La vocation \\[0pt]
La voix de la raison \\[0pt]
La volonté est-elle la somme des inclinations ? \\[0pt]
La volonté est-elle libre ? \\[0pt]
La volonté et le désir \\[0pt]
La volonté peut-elle être générale ? \\[0pt]
La volonté peut-elle nous manquer ? \\[0pt]
La vraie vie \\[0pt]
La vue et le toucher \\[0pt]
Le bavardage \\[0pt]
Le beau est-il toujours moral ? \\[0pt]
Le beau et l'agréable \\[0pt]
Le beau et le bien \\[0pt]
Le beau et le sublime \\[0pt]
Le beau et l'utile \\[0pt]
Le beau geste \\[0pt]
Le bénéfice du doute \\[0pt]
Le besoin de métaphysique est-il un besoin de connaissance ? \\[0pt]
Le besoin de reconnaissance \\[0pt]
Le besoin de théorie \\[0pt]
Le besoin et le désir \\[0pt]
Le bien commun \\[0pt]
Le bien commun est-il une illusion ? \\[0pt]
Le bien commun et l'intérêt de tous \\[0pt]
Le bien, est-ce l'utile ? \\[0pt]
Le bien est-il relatif ? \\[0pt]
Le bien et le beau \\[0pt]
Le bien et les biens \\[0pt]
Le bien n'est-il réalisable que comme moindre mal ? \\[0pt]
Le bien public \\[0pt]
Le Bien public \\[0pt]
Le bon goût \\[0pt]
Le bonheur collectif \\[0pt]
Le bonheur de la passion est-il sans lendemain ? \\[0pt]
Le bonheur des sens \\[0pt]
Le bonheur du plus grand nombre \\[0pt]
Le bonheur est-il au nombre de nos devoirs ? \\[0pt]
Le bonheur est-il dans l'inconscience ? \\[0pt]
Le bonheur est-il l'absence de maux ? \\[0pt]
Le bonheur est-il l'affaire du politique ? \\[0pt]
Le bonheur est-il la fin de la vie ? \\[0pt]
Le bonheur est-il le bien suprême ? \\[0pt]
Le bonheur est-il le but de la politique ? \\[0pt]
Le bonheur est-il le prix de la vertu ? \\[0pt]
Le bonheur est-il un but politique ? \\[0pt]
Le bonheur est-il un droit ? \\[0pt]
Le bonheur est-il une affaire privée ? \\[0pt]
Le bonheur est-il une récompense ? \\[0pt]
Le bonheur est-il un idéal ? \\[0pt]
Le bonheur et la raison \\[0pt]
Le bonheur et la technique \\[0pt]
Le bonheur et le plaisir \\[0pt]
Le bonheur n'est-il qu'une idée ? \\[0pt]
Le bonheur n'est-il qu'un idéal ? \\[0pt]
Le bonheur peut-il être collectif ? \\[0pt]
Le bonheur peut-il être le but de la politique ? \\[0pt]
Le bonheur peut-il être un droit ? \\[0pt]
Le bonheur s'apprend-il ? \\[0pt]
Le bonheur se calcule-t-il ? \\[0pt]
Le bonheur se mérite-t-il ? \\[0pt]
Le bon sens \\[0pt]
Le bricolage \\[0pt]
Le calendrier \\[0pt]
Le capital et le travail \\[0pt]
Le caractère \\[0pt]
Le caractère sacré de la vie \\[0pt]
Le cas de conscience \\[0pt]
Le cerveau et la pensée \\[0pt]
Le cerveau pense-t-il ? \\[0pt]
L'échange constitue-t-il un lien social ? \\[0pt]
L'échange économique fonde-t-il la société humaine \\[0pt]
L'échange et l'usage \\[0pt]
Le changement \\[0pt]
L'échange n'a-t-il de fondement qu'économique ? \\[0pt]
L'échange ne porte-t-il que sur les choses ? \\[0pt]
L'échange peut-il être désintéressé ? \\[0pt]
Le chaos \\[0pt]
Le châtiment \\[0pt]
Le châtiment peut-il ne rien devoir au désir de vengeance ? \\[0pt]
Le chef \\[0pt]
Le chef-d'œuvre \\[0pt]
Le choc des idées \\[0pt]
Le choix \\[0pt]
Le choix et la liberté \\[0pt]
Le cinéma est-il un art narratif ? \\[0pt]
Le citoyen \\[0pt]
Le commencement \\[0pt]
Le commerce \\[0pt]
Le commerce adoucit-il les mœurs ? \\[0pt]
Le commerce des idées \\[0pt]
Le commerce est-il un facteur de paix ? \\[0pt]
Le commerce peut-il être équitable ? \\[0pt]
Le commerce unit-il les hommes ? \\[0pt]
Le commun et le propre \\[0pt]
Le concept \\[0pt]
Le concept d'histoire universelle \\[0pt]
Le concept et l'exemple \\[0pt]
Le concret \\[0pt]
Le conflit est-il une maladie sociale ? \\[0pt]
L'économie et la politique \\[0pt]
L'économie et le politique obéissent-ils à une même rationalité ? \\[0pt]
Le consensus peut-il être critère de vérité ? \\[0pt]
Le consensus peut-il faire le vrai ? \\[0pt]
Le consentement \\[0pt]
Le contentement \\[0pt]
Le contrat \\[0pt]
Le contrat de travail \\[0pt]
Le contrat est-il au fondement de la politique ? \\[0pt]
Le contrat peut-il remplacer la loi ? \\[0pt]
Le corps est-il négociable ? \\[0pt]
Le corps est-il une entrave pour l'esprit ? \\[0pt]
Le corps est-il un tout ? \\[0pt]
Le corps et l'âme \\[0pt]
Le corps et l'esprit \\[0pt]
Le corps impose-t-il des perspectives ? \\[0pt]
Le corps n'est-il qu'un mécanisme ? \\[0pt]
Le corps obéit-il à l'esprit ? \\[0pt]
Le corps politique \\[0pt]
Le cosmopolitisme \\[0pt]
Le courage \\[0pt]
Le cours de l'histoire \\[0pt]
Le crime \\[0pt]
L'écriture \\[0pt]
L'écriture ne sert-elle qu'à consigner la pensée ? \\[0pt]
Le cynisme \\[0pt]
Le dedans et le dehors \\[0pt]
Le défaut \\[0pt]
Le dégoût \\[0pt]
Le déni et la loi \\[0pt]
Le désespoir \\[0pt]
Le désespoir peut-il passer pour une sagesse ? \\[0pt]
Le désintéressement \\[0pt]
Le désir \\[0pt]
Le désir a-t-il un objet ? \\[0pt]
Le désir d'absolu \\[0pt]
Le désir de l'autre \\[0pt]
Le désir de reconnaissance \\[0pt]
Le désir de savoir \\[0pt]
Le désir de savoir est-il comblé par la science ? \\[0pt]
Le désir de savoir est-il naturel ? \\[0pt]
Le désir d'éternité \\[0pt]
Le désir de vérité \\[0pt]
Le désir de vérité peut-il être interprété comme un désir de pouvoir ? \\[0pt]
Le désir du bonheur est-il universel ? \\[0pt]
Le désir est-il aveugle ? \\[0pt]
Le désir est-il ce qui nous fait vivre ? \\[0pt]
Le désir est-il désir de l'autre ? \\[0pt]
Le désir est-il le signe d'un manque ? \\[0pt]
Le désir est-il l'essence de l'homme ? \\[0pt]
Le désir est-il nécessairement l'expression d'un manque ? \\[0pt]
Le désir est-il par nature illimité ? \\[0pt]
Le désir et la culpabilité \\[0pt]
Le désir et la liberté \\[0pt]
Le désir et la loi \\[0pt]
Le désir et le besoin \\[0pt]
Le désir et le mal \\[0pt]
Le désir et le manque \\[0pt]
Le désir et le réel \\[0pt]
Le désir et le rêve \\[0pt]
Le désir et le temps \\[0pt]
Le désir et le travail \\[0pt]
Le désir et l'instinct \\[0pt]
Le désir et l'interdit \\[0pt]
Le désir n'est-il que l'épreuve d'un manque ? \\[0pt]
Le désir n'est-il que manque ? \\[0pt]
Le désir n'est-il qu'inquiétude ? \\[0pt]
Le désir peut-il être désintéressé ? \\[0pt]
Le désir peut-il ne pas avoir d'objet ? \\[0pt]
Le désir peut-il nous rendre libre ? \\[0pt]
Le désir peut-il se satisfaire de la réalité ? \\[0pt]
Le désordre \\[0pt]
Le despotisme \\[0pt]
Le destin \\[0pt]
Le développement de la technique est-il toujours facteur de progrès ? \\[0pt]
Le développement des techniques fait-il reculer la croyance ? \\[0pt]
Le devenir \\[0pt]
Le devoir \\[0pt]
Le devoir de loyauté \\[0pt]
Le devoir de vérité \\[0pt]
Le devoir est-il l'expression de la contrainte sociale ? \\[0pt]
Le devoir et le bonheur \\[0pt]
Le devoir nous apparaît-il d'autant mieux qu'on ne l'a pas accompli ? \\[0pt]
Le devoir rend-il libre ? \\[0pt]
Le devoir se présente-t-il avec la force de l'évidence ? \\[0pt]
Le devoir supprime-t-il la liberté ? \\[0pt]
Le diable \\[0pt]
Le dialogue \\[0pt]
Le dialogue conduit-il à la vérité ? \\[0pt]
Le dialogue suffit-il à rompre la solitude ? \\[0pt]
Le discernement \\[0pt]
Le divertissement \\[0pt]
Le don \\[0pt]
Le don de soi \\[0pt]
Le don est-il une modalité de l'échange ? \\[0pt]
Le don et la dette \\[0pt]
Le don et l'échange \\[0pt]
Le don exclut-il le calcul ? \\[0pt]
Le doute \\[0pt]
Le droit \\[0pt]
Le droit à la différence met-il en péril l'égalité des droits ? \\[0pt]
Le droit à la paresse \\[0pt]
Le droit à la vérité \\[0pt]
Le droit au bonheur \\[0pt]
Le droit au travail \\[0pt]
Le droit de mentir \\[0pt]
Le droit de propriété \\[0pt]
Le droit de résistance \\[0pt]
Le droit divin \\[0pt]
Le droit doit-il être indépendant de la morale ? \\[0pt]
Le droit du plus faible \\[0pt]
Le droit est-il facteur de paix ? \\[0pt]
Le droit est-il le fondement de l'État ? \\[0pt]
Le droit est-il raison du plus fort ? \\[0pt]
Le droit et la convention \\[0pt]
Le droit et la force \\[0pt]
Le droit et la liberté \\[0pt]
Le droit et la loi \\[0pt]
Le droit et la morale \\[0pt]
Le droit et le devoir \\[0pt]
Le droit naturel \\[0pt]
Le droit ne peut-il se fonder sur des faits ? \\[0pt]
Le droit n'est-il qu'une justice par défaut ? \\[0pt]
Le droit n'est-il qu'un ensemble de conventions ? \\[0pt]
Le droit peut-il échapper à l'histoire ? \\[0pt]
Le droit peut-il être naturel ? \\[0pt]
Le droit peut-il se passer de la morale ? \\[0pt]
Le droit positif \\[0pt]
Le droit sert-il à établir l'ordre ou la justice ? \\[0pt]
L'éducation artistique \\[0pt]
L'éducation du goût est-elle la condition de l'expérience esthétique ? \\[0pt]
L'éducation esthétique \\[0pt]
Le fait \\[0pt]
Le fait divers \\[0pt]
Le fait et l'événement \\[0pt]
Le fantasme \\[0pt]
L'efficacité \\[0pt]
L'efficacité technique est-elle la norme de toute réussite ? \\[0pt]
L'effort \\[0pt]
L'effort moral \\[0pt]
Le fini et l'infini \\[0pt]
Le for intérieur \\[0pt]
Le futur est-il contingent ? \\[0pt]
Le futur existe-t-il ? \\[0pt]
Le futur nous appartient-il ? \\[0pt]
L'égalité \\[0pt]
L'égalité est-elle contre nature ? \\[0pt]
L'égalité est-elle l'ennemie de la liberté ? \\[0pt]
L'égalité est-elle toujours juste ? \\[0pt]
Légalité et légitimité \\[0pt]
Légalité et moralité \\[0pt]
L'égalité peut-elle être une menace pour la liberté ? \\[0pt]
Le génie \\[0pt]
Le génie est-il la marque de l'excellence artistique ? \\[0pt]
Le génie et la règle \\[0pt]
Le génie et le savant \\[0pt]
Le génie n'a-t-il sa place que dans les arts ? \\[0pt]
Le geste et la parole \\[0pt]
Le geste technique exprime t-il une liberté sans fin ? \\[0pt]
Le goût \\[0pt]
Le goût de la liberté \\[0pt]
Le goût des autres \\[0pt]
Le goût s'éduque-t-il ? \\[0pt]
Le gouvernement des hommes et l'administration des choses \\[0pt]
Le gouvernement des meilleurs \\[0pt]
Le hasard \\[0pt]
Le hasard et la nécessité \\[0pt]
Le hasard n'est-il que la mesure de notre ignorance ? \\[0pt]
Le hasard peut-il être un concept explicatif ?La morale doit-elle s'adapter à la réalité ? \\[0pt]
Le hors-la-loi \\[0pt]
Le je et le tu \\[0pt]
Le jeu \\[0pt]
Le jeu et le divertissement \\[0pt]
Le jeu et le hasard \\[0pt]
Le juge \\[0pt]
Le jugement \\[0pt]
Le jugement dernier \\[0pt]
Le jugement moral \\[0pt]
Le juste et le légal \\[0pt]
Le laid \\[0pt]
Le langage \\[0pt]
Le langage a-t-il une prétention à la vérité ? \\[0pt]
Le langage des sciences \\[0pt]
Le langage du corps \\[0pt]
Le langage est-il d'essence poétique ? \\[0pt]
Le langage est-il le lieu de la vérité ? \\[0pt]
Le langage est-il le moyen d'expression le plus efficace ? \\[0pt]
Le langage est-il le propre de l'homme ? \\[0pt]
Le langage est-il logique ? \\[0pt]
Le langage est-il naturel ? \\[0pt]
Le langage est-il une prise de possession des choses ? \\[0pt]
Le langage est-il un instrument de connaissance ? \\[0pt]
Le langage est-il un obstacle pour la pensée ? \\[0pt]
Le langage et la pensée \\[0pt]
Le langage et la raison \\[0pt]
Le langage et la vérité \\[0pt]
Le langage et l'image \\[0pt]
Le langage masque-t-il la pensée ? \\[0pt]
Le langage ne sert-il qu'à communiquer ? \\[0pt]
Le langage n'est-il qu'un instrument de communication ? \\[0pt]
Le langage n'est-il qu'un outil ? \\[0pt]
Le langage nous donne-t-il une connaissance des choses ? \\[0pt]
Le langage nous éloigne-t-il de la réalité ? \\[0pt]
Le langage permet-il seulement de communiquer ? \\[0pt]
Le langage peut-il être un obstacle à la recherche de la vérité ? \\[0pt]
Le langage rend-il l'homme plus puissant ? \\[0pt]
Le langage reste-t-il toujours à la surface des choses ? \\[0pt]
Le langage traduit-il la pensée ? \\[0pt]
Le langage trahit-il la pensée ? \\[0pt]
Le langage usuel est-il un obstacle à la connaissance ? \\[0pt]
Le langage voile-t-il les choses ? \\[0pt]
L'élection \\[0pt]
Le législateur \\[0pt]
L'élémentaire \\[0pt]
Le libre arbitre \\[0pt]
Le libre cours de l'imagination est-il libérateur ? \\[0pt]
Le libre échange \\[0pt]
Le lien social \\[0pt]
Le livre de la nature \\[0pt]
Le loisir \\[0pt]
Le luxe \\[0pt]
Le maintien de l'ordre \\[0pt]
Le maître \\[0pt]
Le mal \\[0pt]
Le mal a-t-il des raisons ? \\[0pt]
Le malentendu \\[0pt]
Le mal être \\[0pt]
Le mal existe-t-il ? \\[0pt]
Le malheur \\[0pt]
Le malheur donne-t-il le droit d'être injuste ? \\[0pt]
Le malheur est-il injuste ? \\[0pt]
Le mal peut-il être involontaire ? \\[0pt]
L'émancipation \\[0pt]
Le marché \\[0pt]
Le marché du travail \\[0pt]
Le mariage \\[0pt]
Le matérialisme \\[0pt]
Le mauvais goût \\[0pt]
L'embarras du choix \\[0pt]
Le méchant peut-il être heureux ? \\[0pt]
Le médiat et l'immédiat \\[0pt]
Le meilleur est-il l'ennemi du bien ? \\[0pt]
Le meilleur gouvernement est-il le gouvernement des meilleurs ? \\[0pt]
Le mensonge \\[0pt]
Le mensonge est-il une forme d'indifférence à la vérité ? \\[0pt]
Le mensonge peut-il être au service de la vérité ? \\[0pt]
Le mépris \\[0pt]
Le mérite \\[0pt]
Le métier \\[0pt]
Le métier de politique \\[0pt]
Le mien et le tien \\[0pt]
Le modèle \\[0pt]
Le moi \\[0pt]
Le moi est-il haïssable ? \\[0pt]
Le moi est-il insaisissable ? \\[0pt]
Le moi est-il une fiction ? \\[0pt]
Le moi est-il une illusion ? \\[0pt]
Le moi et la conscience \\[0pt]
Le moindre mal \\[0pt]
Le moi n'est-il qu'une fiction ? \\[0pt]
Le moi reste-t-il identique à lui-même au cours du temps ? \\[0pt]
Le monde de l'art \\[0pt]
Le monde de la technique \\[0pt]
Le monde du travail \\[0pt]
Le monde est-il autre chose que ce que j'en dis ? \\[0pt]
Le monde et la technique \\[0pt]
Le monde n'est-il que l'ensemble de ce qui existe ? \\[0pt]
Le monde n'est-il que ma représentation ? \\[0pt]
Le monde n'est-il qu'une représentation ? \\[0pt]
Le monde se réduit-il à ce que nous en voyons ? \\[0pt]
Le monstre \\[0pt]
Le monstrueux \\[0pt]
Le mot et le geste \\[0pt]
L'émotion \\[0pt]
Le mouvement \\[0pt]
Le multiple et l'un \\[0pt]
Le musée \\[0pt]
Le mystère \\[0pt]
Le naturel \\[0pt]
Le naturel a-t-il toutes les vertus ? \\[0pt]
Le naturel et le fabriqué \\[0pt]
L'encyclopédie \\[0pt]
L'enfance \\[0pt]
L'enfance est-elle en nous ce qui doit être abandonné ? \\[0pt]
L'enfant \\[0pt]
L'engagement \\[0pt]
L'énigme, le mystère et le secret \\[0pt]
L'ennemi \\[0pt]
L'ennui \\[0pt]
Le non-être \\[0pt]
Le non-sens \\[0pt]
L'enquête empirique rend-elle la métaphysique inutile ? \\[0pt]
L'enseignement de la vérité \\[0pt]
L'entendement et la volonté \\[0pt]
L'envie \\[0pt]
Le pardon \\[0pt]
Le pardon et l'oubli \\[0pt]
Le passé a-t-il plus de réalité que l'avenir ? \\[0pt]
Le passé a-t-il une réalité ? \\[0pt]
Le passé détermine-t-il notre présent ? \\[0pt]
Le passé, est-ce du passé ? \\[0pt]
Le passé est-il ce qui a disparu ? \\[0pt]
Le passé et le présent \\[0pt]
Le passé existe-t-il ? \\[0pt]
Le passé peut-il être un objet de connaissance ? \\[0pt]
Le paysage \\[0pt]
Le personnage et la personne \\[0pt]
Le pessimisme \\[0pt]
Le peuple \\[0pt]
Le peuple est-il un acteur politique ? \\[0pt]
Le peuple et la nation \\[0pt]
Le peuple peut-il se tromper ? \\[0pt]
L'éphémère \\[0pt]
L'éphémère a-t-il une valeur ? \\[0pt]
Le phénomène \\[0pt]
Le physique et le mental \\[0pt]
Le plaisir \\[0pt]
Le plaisir de parler \\[0pt]
Le plaisir des sens \\[0pt]
Le plaisir esthétique \\[0pt]
Le plaisir esthétique est-il affaire de jugement ? \\[0pt]
Le plaisir esthétique peut-il se partager ? \\[0pt]
Le plaisir est-il tout le bonheur ? \\[0pt]
Le plaisir et la joie \\[0pt]
Le plaisir et la peine \\[0pt]
Le plaisir et le contentement \\[0pt]
Le plaisir peut-il être partagé ? \\[0pt]
Le plaisir suffit-il au bonheur ? \\[0pt]
Le poids de la société \\[0pt]
Le poids du passé \\[0pt]
Le politique et l'économique \\[0pt]
Le possible et le réel \\[0pt]
Le pouvoir \\[0pt]
Le pouvoir corrompt-il toujours ? \\[0pt]
Le pouvoir de la science \\[0pt]
Le pouvoir de l'État est-il arbitraire ? \\[0pt]
Le pouvoir des images \\[0pt]
Le pouvoir des mots \\[0pt]
Le pouvoir est-il une forme de domination ? \\[0pt]
Le pouvoir et l'autorité \\[0pt]
Le préscientifique \\[0pt]
Le présent \\[0pt]
Le présent historique \\[0pt]
L'épreuve du réel \\[0pt]
Le prince \\[0pt]
Le principe \\[0pt]
Le principe de raison \\[0pt]
Le principe de raison suffisante \\[0pt]
Le privé et le public \\[0pt]
Le prix du travail \\[0pt]
Le probable \\[0pt]
Le prochain et le semblable \\[0pt]
Le profit \\[0pt]
Le progrès \\[0pt]
Le progrès de l'humanité se réduit-il au progrès technique ? \\[0pt]
Le progrès est-il un mythe ? \\[0pt]
Le progrès moral \\[0pt]
Le progrès technique a-t-il une fin ? \\[0pt]
Le progrès technique est-il source de bonheur ? \\[0pt]
Le progrès technique peut-il être aliénant ? \\[0pt]
Le projet \\[0pt]
Le propre du vivant est-il de tomber malade ? \\[0pt]
Le provisoire \\[0pt]
Le public et le privé \\[0pt]
Le pur et l'impur \\[0pt]
L'équité \\[0pt]
L'équivocité \\[0pt]
L'équivoque \\[0pt]
Le quotidien \\[0pt]
Le raffinement \\[0pt]
Le rationnel et le raisonnable \\[0pt]
Le rationnel et l'irrationnel \\[0pt]
Le réalisme \\[0pt]
Le récit historique \\[0pt]
Le reconnaissance \\[0pt]
Le recours au symbole est-il la marque de la finitude de la pensée ? \\[0pt]
Le réel \\[0pt]
Le réel est-il ce que l'on croit ? \\[0pt]
Le réel est-il ce que nous expérimentons ? \\[0pt]
Le réel est-il ce que nous percevons ? \\[0pt]
Le réel est-il ce qui apparaît ? \\[0pt]
Le réel est-il ce qui est perçu ? \\[0pt]
Le réel est-il inaccessible ? \\[0pt]
Le réel est-il l'objet de la science ? \\[0pt]
Le réel est-il objet d'interprétation ? \\[0pt]
Le réel est-il rationnel ? \\[0pt]
Le réel et la fiction \\[0pt]
Le réel et le matériel \\[0pt]
Le réel et le possible \\[0pt]
Le réel et le virtuel \\[0pt]
Le réel et le vrai \\[0pt]
Le réel et l'imaginaire \\[0pt]
Le réel et l'irréel \\[0pt]
Le réel n'est-il qu'un ensemble de contraintes ? \\[0pt]
Le réel obéit-il à la raison ? \\[0pt]
Le réel résiste-t-il à la connaissance ? \\[0pt]
Le réel se limite-t-il à ce que font connaître les théories scientifiques ? \\[0pt]
Le réel se limite-t-il à ce que nous percevons ? \\[0pt]
Le réel se réduit-il à ce que l'on perçoit ? \\[0pt]
Le réel se réduit-il à l'objectivité ? \\[0pt]
Le regard \\[0pt]
Le relativisme \\[0pt]
Le remords est-il stérile ? \\[0pt]
Le renoncement \\[0pt]
Le respect \\[0pt]
Le respect n'est-il dû qu'aux personnes ? \\[0pt]
Le ressentiment \\[0pt]
Le retour à la nature \\[0pt]
Le rien \\[0pt]
Le risque \\[0pt]
Le risque de la liberté \\[0pt]
Le risque technologique \\[0pt]
Le rôle de l'État est-il de faire régner la justice ? \\[0pt]
Le rôle de l'État est-il de préserver la liberté de l'individu ? \\[0pt]
Le rôle de l'historien est-il de juger ? \\[0pt]
Le rôle des théories est-il d'expliquer ou de décrire ? \\[0pt]
L'erreur \\[0pt]
L'erreur et la faute \\[0pt]
L'erreur et l'illusion \\[0pt]
Le rythme \\[0pt]
Le sacré \\[0pt]
Le sacré et le profane \\[0pt]
Le sacrifice \\[0pt]
Les acteurs de l'histoire en sont-ils les auteurs ? \\[0pt]
Les affects sont-ils déraisonnables ? \\[0pt]
Le sage a-t-il besoin d'autrui ? \\[0pt]
Le sage est-il insensible ? \\[0pt]
Les âges de la vie \\[0pt]
Le salaire \\[0pt]
Le salut vient-il de la raison ? \\[0pt]
Les animaux ont-ils des droits ? \\[0pt]
Les apparences sont-elles toujours trompeuses ? \\[0pt]
Les arts ont-ils pour fonction de divertir ? \\[0pt]
Le sauvage \\[0pt]
Le savant et l'ignorant \\[0pt]
Le savant et l'ingénieur \\[0pt]
Le savoir du corps \\[0pt]
Le savoir est-il une condition du bonheur ? \\[0pt]
Le savoir exclut-il toute forme de croyance ? \\[0pt]
Le savoir-faire \\[0pt]
Le savoir rend-il libre ? \\[0pt]
Le savoir total \\[0pt]
Les bêtes travaillent-elles ? \\[0pt]
« Les bons comptes font les bons amis » \\[0pt]
Les catégories \\[0pt]
Les causes et les raisons \\[0pt]
Les causes et les signes \\[0pt]
Les changements de théorie s'expliquent-ils seulement par le résultat des expériences ? \\[0pt]
Les choses \\[0pt]
Les choses ont-elles l'air de ce qu'elles sont ? \\[0pt]
Les classes sociales \\[0pt]
L'esclavage des passions \\[0pt]
Les coïncidences ont-elles des causes ? \\[0pt]
Les commencements \\[0pt]
Les comportements humains s'expliquent-il par l'instinct naturel ? \\[0pt]
Les concepts sont-ils des fictions ? \\[0pt]
Les concitoyens doivent-ils être des amis ? \\[0pt]
Les conditions d'existence \\[0pt]
Les conflits de devoirs \\[0pt]
Les conflits menacent-ils la société ? \\[0pt]
Les connaissances scientifiques sont-elles vraies ? \\[0pt]
Les considérations morales ont-elles leur place en politique ? \\[0pt]
Les croyances religieuses sont-elles indiscutables ? \\[0pt]
Les démonstrations ont-elles des propriétés esthétiques ? \\[0pt]
Les désirs ont-ils nécessairement un objet ? \\[0pt]
Les devoirs de l'homme varient-ils selon la culture ? \\[0pt]
Les devoirs de l'homme varient-ils selon les cultures ? \\[0pt]
Les devoirs du citoyen \\[0pt]
Les divers sens du mot culture peuvent-ils être ramenés à l'unité ? \\[0pt]
Les droits de l'homme ont-ils besoin d'être fondés ? \\[0pt]
Les droits de l'individu \\[0pt]
Les échanges \\[0pt]
Les échanges économiques sont-ils facteurs de paix ? \\[0pt]
Les échanges favorisent-ils la paix ? \\[0pt]
Les échanges sont-ils facteurs de paix ? \\[0pt]
Les écrans \\[0pt]
Le secret \\[0pt]
Le sens caché \\[0pt]
Le sens commun \\[0pt]
Le sens de la justice \\[0pt]
Le sens de l'Histoire \\[0pt]
Le sens des mots \\[0pt]
Le sens de tout énoncé dépend-il de son interprétation ? \\[0pt]
Le sens du devoir \\[0pt]
Le sens du devoir nous fait-il connaître le bien de façon indiscutable ? \\[0pt]
Le sens du problème \\[0pt]
Le sensible \\[0pt]
Le sensible et l'intelligible \\[0pt]
Le sensible peut-il être connu ? \\[0pt]
Le sens interne \\[0pt]
Le sentiment \\[0pt]
Le sentiment de liberté \\[0pt]
Le sentiment d'injustice \\[0pt]
Le sentiment d'injustice est-il naturel ? \\[0pt]
Le sentiment du devoir accompli \\[0pt]
Le sentiment du juste et de l'injuste \\[0pt]
Le serment \\[0pt]
Les États ont-ils pour fin d'assurer la paix ? \\[0pt]
Les êtres vivants sont-ils des machines ? \\[0pt]
Les événements historiques sont-ils de nature imprévisible ? \\[0pt]
Les faits et les valeurs \\[0pt]
Les faits existent-ils indépendamment de leur établissement par l'esprit humain ? \\[0pt]
Les faits parlent-ils d'eux-mêmes ? \\[0pt]
Les faits peuvent-ils faire autorité ? \\[0pt]
Les faits sont-ils des preuves ? \\[0pt]
Les faits sont-ils têtus ? \\[0pt]
« Les faits sont là » \\[0pt]
Les fins de la culture \\[0pt]
Les fins dernières \\[0pt]
Les formes du vivant \\[0pt]
Les frontières de l'art \\[0pt]
Les frontières du vivant \\[0pt]
Les générations \\[0pt]
Les générations futures ont-elles des droits ? \\[0pt]
Les habitudes nous forment-elles ? \\[0pt]
Les hommes naissent-ils libres ? \\[0pt]
Les hommes ont-ils besoin de maîtres ? \\[0pt]
Les hommes peuvent-ils s'associer sans renoncer à leur liberté ? \\[0pt]
Les hommes savent-ils ce qu'ils désirent ? \\[0pt]
Les hommes sont-ils faits pour s'entendre ? \\[0pt]
Les hommes sont-ils seulement le produit de leur culture ? \\[0pt]
Les idées et les choses \\[0pt]
Les idées ont-elles une existence éternelle ? \\[0pt]
Le signe \\[0pt]
Le signe et l'indice \\[0pt]
Le silence \\[0pt]
Le silence a-t-il un sens ? \\[0pt]
Le silence est-il signifiant ? \\[0pt]
Les images peuvent-elles mentir ? \\[0pt]
Le simple \\[0pt]
Le simple et le complexe \\[0pt]
Les inégalités menacent-elles la société ? \\[0pt]
Les inégalités sociales sont-elles naturelles ? \\[0pt]
Les intentions et les actes \\[0pt]
Les langues sont-elles condamnées à l'approximation ? \\[0pt]
Les leçons de l'histoire \\[0pt]
Les limites de la connaissance \\[0pt]
Les limites de la raison \\[0pt]
Les limites de la science \\[0pt]
Les limites de la vérité \\[0pt]
Les limites de l'expérience \\[0pt]
Les limites de l'interprétation \\[0pt]
Les limites du pouvoir \\[0pt]
Les lois \\[0pt]
Les lois de la nature \\[0pt]
Les lois naturelles \\[0pt]
Les machines nous rendent-elles libres ? \\[0pt]
Les machines permettent-elles de mieux connaître le corps humain ? \\[0pt]
Les maîtres de vérité \\[0pt]
Les maladies de l'âme \\[0pt]
Les mathématiques parlent-elles du réel ? \\[0pt]
Les mathématiques sont-elles une science ? \\[0pt]
Les mathématiques sont-elles un instrument ? \\[0pt]
Les mécanismes de la vie \\[0pt]
Les mœurs \\[0pt]
Les monstres \\[0pt]
Les mots disent-ils les choses ? \\[0pt]
Les mots et les concepts \\[0pt]
Les mots expriment-ils les choses ? \\[0pt]
Les mots nous éloignent-ils des autres ? \\[0pt]
Les mots nous éloignent-ils des choses ? \\[0pt]
Les mots parviennent-ils à tout exprimer ? \\[0pt]
Les mots rendent-ils compte de la nature des choses ? \\[0pt]
Les mots sont-ils trompeurs ? \\[0pt]
Les moyens et les fins \\[0pt]
Les objets du désir \\[0pt]
Les œuvres d'art sont-elles des réalités comme les autres ? \\[0pt]
Le soi est-il inaccessible ? \\[0pt]
Le soi et le je \\[0pt]
Le soleil se lèvera-t-il demain ? \\[0pt]
Le solipsisme \\[0pt]
Le sommeil \\[0pt]
Le souci de soi \\[0pt]
Les outils \\[0pt]
Le souverain bien \\[0pt]
L'espace nous sépare-t-il ? \\[0pt]
Les paroles et les actes \\[0pt]
Le spectacle du monde \\[0pt]
L'espérance \\[0pt]
Les perceptions sont-elles des jugements ? \\[0pt]
Les personnages de fiction peuvent-ils avoir une réalité ? \\[0pt]
Les peuples font-ils l'histoire ? \\[0pt]
Les preuves de la liberté \\[0pt]
Les principes \\[0pt]
Les principes de la morale dépendent-ils de la culture ? \\[0pt]
L'esprit \\[0pt]
L'esprit critique \\[0pt]
L'esprit dépend-il du corps ? \\[0pt]
L'esprit de système \\[0pt]
L'esprit domine-t-il la matière ? \\[0pt]
L'esprit est-il chose ? \\[0pt]
L'esprit est-il irréductible à la matière ? \\[0pt]
L'esprit est-il libre ? \\[0pt]
L'esprit est-il mieux connu que le corps ? \\[0pt]
L'esprit est-il objet de science ? \\[0pt]
L'esprit est-il plus aisé à connaître que le corps ? \\[0pt]
L'esprit est-il plus difficile à connaître que la matière ? \\[0pt]
L'esprit est-il plus facile à connaître que le corps ? \\[0pt]
L'esprit est-il réductible à la matière ? \\[0pt]
L'esprit est-il une partie du corps ? \\[0pt]
L'esprit et la matière peuvent-ils agir l'un sur l'autre ? \\[0pt]
L'esprit humain progresse-t-il ? \\[0pt]
L'esprit n'a-t-il jamais affaire qu'à lui-même ? \\[0pt]
Les progrès de la technique sont-ils nécessairement des progrès de la raison ? \\[0pt]
Les progrès techniques constituent-ils des progrès de la civilisation ? \\[0pt]
Les querelles de mots sont-elles toujours stériles ? \\[0pt]
Les raisons de croire \\[0pt]
Les règles de l'art \\[0pt]
Les règles de l'interprétation \\[0pt]
Les règles du langage limitent-elles notre liberté d'expression ? \\[0pt]
Les règles du langage sont- elles un obstacle à la communication ? \\[0pt]
Les religions naissent-elles du besoin de justice ? \\[0pt]
Les religions peuvent-elles être objets de science ? \\[0pt]
Les religions peuvent-elles prétendre libérer les hommes ? \\[0pt]
Les religions sont-elles affaire de foi ? \\[0pt]
Les rêves sont-ils des pensées ? \\[0pt]
Les scélérats peuvent-ils être heureux ? \\[0pt]
Les sciences décrivent-elles le réel ? \\[0pt]
Les sciences de l'homme sont-elles des sciences ? \\[0pt]
Les sciences ont-elles le monopole de la vérité ? \\[0pt]
Les sciences pensent-elles leurs fondements ? \\[0pt]
Les sciences permettent-elles de connaître la réalité-même ? \\[0pt]
Les sciences peuvent-elles se passer de fondements métaphysiques ? \\[0pt]
Les sciences peuvent-elles se passer d'images ? \\[0pt]
L'essence et l'existence \\[0pt]
Les sens jugent-ils ? \\[0pt]
Les sens sont-ils source d'illusion ? \\[0pt]
Les sens sont-ils trompeurs ? \\[0pt]
Les théories scientifiques décrivent-elles la réalité ? \\[0pt]
L'estime de soi \\[0pt]
Le style \\[0pt]
Le sublime \\[0pt]
Le sublime est-il dans les choses ? \\[0pt]
Le succès peut-il être un critère de vérité ? \\[0pt]
Le sujet \\[0pt]
Le sujet du droit \\[0pt]
Le sujet et la personne \\[0pt]
Le sujet et l'individu \\[0pt]
Le sujet moral \\[0pt]
Le sujet n'est-il qu'une fiction ? \\[0pt]
Le sujet peut-il s'aliéner par un libre choix ? \\[0pt]
Les valeurs morales ont-elles leur origine dans la raison ? \\[0pt]
Les valeurs universelles \\[0pt]
Les vérités empiriques \\[0pt]
Les vérités éternelles \\[0pt]
Les vérités sont-elles intemporelles ? \\[0pt]
Les vertus du commerce \\[0pt]
Les vivants peuvent-ils se passer des morts ? \\[0pt]
Le symbole \\[0pt]
Le système \\[0pt]
Le système des arts \\[0pt]
Le tact \\[0pt]
Le talent \\[0pt]
L'État \\[0pt]
L'État a-t-il des intérêts propres ? \\[0pt]
L'État a-t-il pour but de maintenir l'ordre ? \\[0pt]
L'État a-t-il tous les droits ? \\[0pt]
L'État contribue-t-il à pacifier les relations entre les hommes ? \\[0pt]
L'État de droit \\[0pt]
L'état de nature \\[0pt]
L'État doit-il éduquer le peuple ? \\[0pt]
L'État doit-il être fort ? \\[0pt]
L'État doit-il être le plus fort ? \\[0pt]
L'État doit-il être sans pitié ? \\[0pt]
L'État doit-il préférer l'injustice au désordre ? \\[0pt]
L'État doit-il reconnaître des limites à sa puissance ? \\[0pt]
L'État doit-il se mêler de religion ? \\[0pt]
L'État doit-il se préoccuper des arts ? \\[0pt]
L'État doit-il se préoccuper du bonheur des citoyens ? \\[0pt]
L'État doit-il se soucier de la morale ? \\[0pt]
L'État doit-il veiller au bonheur des individus ? \\[0pt]
L'État est-il au service de la société ? \\[0pt]
L'État est-il la seule source du droit ? \\[0pt]
L'État est-il le garant du bien commun ? \\[0pt]
L'État est-il l'ennemi de la liberté ? \\[0pt]
L'État est-il l'ennemi de l'individu ? \\[0pt]
L'État est-il libérateur ? \\[0pt]
L'État est-il nécessaire ? \\[0pt]
L'État est-il notre ennemi ? \\[0pt]
L'État est-il oppresseur ? \\[0pt]
L'État est-il souverain ? \\[0pt]
L'État est-il toujours juste ? \\[0pt]
L'État est-il un mal nécessaire ? \\[0pt]
L'État est-il un moindre mal ? \\[0pt]
L'État est-il un « monstre froid » ? \\[0pt]
L'État est-il un tiers impartial ? \\[0pt]
L'État et la justice \\[0pt]
L'État et la nation \\[0pt]
L'État et la société \\[0pt]
L'État et le droit \\[0pt]
L'État et le peuple \\[0pt]
L'État et les communautés \\[0pt]
L'État et l'individu \\[0pt]
L'État n'a-t-il d'existence que contractuelle ? \\[0pt]
L'État n'est-il qu'un instrument de domination ? \\[0pt]
L'État nous rend-il meilleurs ? \\[0pt]
L'État peut-il demeurer indifférent à la religion ? \\[0pt]
L'État peut-il être impartial ? \\[0pt]
L'État peut-il fonder sa propre légitimité ? \\[0pt]
L'État peut-il limiter son pouvoir ? \\[0pt]
L'État peut-il poursuivre une autre fin que sa propre puissance ? \\[0pt]
L'État peut-il renoncer à la violence ? \\[0pt]
L'État peut-il se désintéresser de la religion ? \\[0pt]
L'État sert-il l'intérêt général ? \\[0pt]
Le technicien n'est-il qu'un exécutant ? \\[0pt]
Le témoignage \\[0pt]
Le témoignage des sens \\[0pt]
Le temporel et le spirituel \\[0pt]
Le temps \\[0pt]
Le temps de la mémoire \\[0pt]
Le temps de la vie et le temps de l'existence \\[0pt]
Le temps dépend-il de la mémoire ? \\[0pt]
Le temps des origines \\[0pt]
Le temps détruit-il tout ? \\[0pt]
Le temps du bonheur \\[0pt]
Le temps du monde et le temps de la conscience \\[0pt]
Le temps est-il destructeur ? \\[0pt]
Le temps est-il en nous ou hors de nous ? \\[0pt]
Le temps est-il essentiellement destructeur ? \\[0pt]
Le temps est-il la marque de notre impuissance ? \\[0pt]
Le temps est-il notre allié ? \\[0pt]
Le temps est-il notre ennemi ? \\[0pt]
Le temps est-il une contrainte ? \\[0pt]
Le temps est-il une expérience de la continuité ou de la discontinuité ? \\[0pt]
Le temps est-il une réalité ? \\[0pt]
Le temps et la liberté \\[0pt]
Le temps et le bonheur \\[0pt]
Le temps et l'espace \\[0pt]
Le temps et l'éternité \\[0pt]
Le temps libre \\[0pt]
Le temps n'est-il pour l'homme que ce qui le limite ? \\[0pt]
Le temps n'existe-t-il que subjectivement ? \\[0pt]
Le temps nous appartient-il ? \\[0pt]
Le temps nous est-il compté ? \\[0pt]
Le temps passe-t-il ? \\[0pt]
Le temps perdu \\[0pt]
Le temps peut-il s'arrêter ? \\[0pt]
Le temps physique est-il comparable au temps psychique ? \\[0pt]
Le temps se mesure-t-il ? \\[0pt]
L'éternel retour \\[0pt]
L'éternité \\[0pt]
L'éternité des œuvres d'art \\[0pt]
L'éternité est-elle désirable ? \\[0pt]
L'étonnement \\[0pt]
Le toucher \\[0pt]
Le tout est-il la somme de ses parties ? \\[0pt]
Le travail \\[0pt]
Le travail a-t-il une valeur morale ? \\[0pt]
Le travail de la pensée \\[0pt]
Le travail du droit \\[0pt]
Le travail est-il le propre de l'homme ? \\[0pt]
Le travail est-il libérateur ? \\[0pt]
Le travail est-il nécessaire au bonheur ? \\[0pt]
Le travail est-il toujours une activité productrice ? \\[0pt]
Le travail est-il un besoin ? \\[0pt]
Le travail est-il une marchandise ? \\[0pt]
Le travail est-il une valeur ? \\[0pt]
Le travail est-il un rapport naturel de l'homme à la nature ? \\[0pt]
Le travail et la propriété \\[0pt]
Le travail et la technique \\[0pt]
Le travail et le bonheur \\[0pt]
Le travail et le jeu \\[0pt]
Le travail et le loisir \\[0pt]
Le travail et le temps \\[0pt]
Le travail fait-il de l'homme un être moral ? \\[0pt]
Le travail fonde-t-il la propriété ? \\[0pt]
Le travaille libère-t-il ? \\[0pt]
Le travail manuel \\[0pt]
Le travail manuel est-il sans pensée ? \\[0pt]
Le travail peut-il être solitaire ? \\[0pt]
Le travail unit-il ou sépare-t-il les hommes ? \\[0pt]
L'être et le néant \\[0pt]
L'être humain est-il la mesure de toute chose ? \\[0pt]
L'être humain est-il par nature un être religieux \\[0pt]
L'être imaginaire et l'être de raison \\[0pt]
Le tribunal de la raison \\[0pt]
Le tribunal de l'histoire \\[0pt]
Le troc \\[0pt]
L'étude de l'histoire conduit-elle à désespérer l'homme ? \\[0pt]
Le tyran \\[0pt]
L'événement \\[0pt]
L'événement historique a-t-il un sens par lui-même ? \\[0pt]
Le vertige de la liberté \\[0pt]
Le vice \\[0pt]
Le vice et la vertu \\[0pt]
Le vide et le plein \\[0pt]
L'évidence \\[0pt]
L'évidence a-t-elle une valeur absolue ? \\[0pt]
L'évidence est-elle le signe de la vérité ? \\[0pt]
L'évidence est-elle un critère de vérité ? \\[0pt]
L'évidence est-elle un obstacle ou un instrument de la recherche de la vérité ? \\[0pt]
L'évidence et la démonstration \\[0pt]
L'évidence se passe-t-elle de démonstration ? \\[0pt]
Le visage n'est-il qu'un masque ? \\[0pt]
Le visible et l'invisible \\[0pt]
Le vivant \\[0pt]
Le vivant définit-il ses propres normes ? \\[0pt]
Le vivant est-il entièrement connaissable ? \\[0pt]
Le vivant est-il entièrement explicable ? \\[0pt]
Le vivant est-il réductible au physico-chimique ? \\[0pt]
Le vivant est-il un objet de science comme un autre ? \\[0pt]
Le vivant et la machine \\[0pt]
Le vivant et la mort \\[0pt]
Le vivant et la sensibilité \\[0pt]
Le vivant et la technique \\[0pt]
Le vivant et le vécu \\[0pt]
Le vivant et l'expérimentation \\[0pt]
Le vivant et l'inerte \\[0pt]
Le vivant et son milieu \\[0pt]
Le vivant n'est-il que matière ? \\[0pt]
Le vivant n'est-il qu'une machine ingénieuse ? \\[0pt]
Le vivant obéit-il à des lois ? \\[0pt]
Le vivant obéit-il à une nécessité ? \\[0pt]
Le vivant se définit-il par sa résistance à la mort ? \\[0pt]
Le voile des mots \\[0pt]
Le volontaire et l'involontaire \\[0pt]
Le voyage \\[0pt]
Le vrai admet-il des degrés ? \\[0pt]
Le vrai a-t-il une histoire ? \\[0pt]
Le vrai et le bien \\[0pt]
Le vrai et le vraisemblable \\[0pt]
Le vrai n'est-il que ce sur quoi l'on s'accorde ? \\[0pt]
Le vraisemblable et le probable \\[0pt]
Le vrai se réduit-il à ce qui est vérifiable ? \\[0pt]
L'exactitude \\[0pt]
L'excellence des sens \\[0pt]
L'exception \\[0pt]
L'excès \\[0pt]
L'excuse \\[0pt]
L'exigence morale \\[0pt]
L'existence \\[0pt]
L'existence a-t-elle un sens ? \\[0pt]
L'existence du mal met-elle en échec la raison ? \\[0pt]
L'existence du passé \\[0pt]
L'existence du possible \\[0pt]
L'existence est-elle pensable ? \\[0pt]
L'existence est-elle vaine ? \\[0pt]
L'existence et le temps \\[0pt]
L'existence se laisse-t-elle penser ? \\[0pt]
L'existence se prouve-t-elle ? \\[0pt]
L'existence suppose-t-elle la conscience du temps ? \\[0pt]
L'expérience \\[0pt]
L'expérience a-t-elle le même sens dans toutes les sciences ? \\[0pt]
L'expérience d'autrui nous est-elle utile ? \\[0pt]
L'expérience de la vie produit-elle un savoir ? \\[0pt]
L'expérience de l'injustice \\[0pt]
L'expérience démontre-t-elle quelque chose ? \\[0pt]
L'expérience de pensée \\[0pt]
L'expérience de soi-même \\[0pt]
L'expérience du temps \\[0pt]
L'expérience est-elle un guide suffisant ? \\[0pt]
L'expérience esthétique relève-t-elle de la contemplation ? \\[0pt]
L'expérience et la sensation \\[0pt]
L'expérience imaginaire \\[0pt]
L'expérience instruit-elle ? \\[0pt]
L'expérience morale \\[0pt]
L'expérience permet-elle de départager des théories concurrentes ? \\[0pt]
L'expérience peut-elle avoir raison des principes ? \\[0pt]
L'expérience peut-elle contredire la théorie ? \\[0pt]
L'expérience religieuse \\[0pt]
L'expérience rend-elle raisonnable ? \\[0pt]
L'expérience rend-elle responsable ? \\[0pt]
L'expérience scientifique \\[0pt]
L'expérience suffit-elle pour établir une vérité ? \\[0pt]
L'expression \\[0pt]
L'expression « perdre son temps » a-t-elle un sens ? \\[0pt]
L'extinction du désir \\[0pt]
L'extinction du désir est-elle le commencement du bonheur ? \\[0pt]
L'extrémité \\[0pt]
L'habitude \\[0pt]
L'habitude est-elle notre guide dans la vie ? \\[0pt]
L'harmonie \\[0pt]
L'hérédité \\[0pt]
L'héritage \\[0pt]
L'histoire \\[0pt]
L'Histoire \\[0pt]
L'histoire a-t-elle un commencement et une fin ? \\[0pt]
L'histoire a-t-elle une fin ? \\[0pt]
L'histoire a-t-elle un sens ? \\[0pt]
L'histoire comme science \\[0pt]
L'histoire des sciences \\[0pt]
L'histoire du droit est-elle celle du progrès de la justice ? \\[0pt]
L'histoire est-elle la connaissance du passé humain ? \\[0pt]
L'histoire est-elle la mémoire de l'humanité ? \\[0pt]
L'histoire est-elle la science du passé ? \\[0pt]
L'histoire est-elle la vérité de la mémoire ? \\[0pt]
L'histoire est-elle le récit objectif des faits passés ? \\[0pt]
L'histoire est-elle le règne du hasard ? \\[0pt]
L'histoire est-elle le théâtre des passions ? \\[0pt]
L'histoire est-elle rationnelle ? \\[0pt]
L'histoire est-elle une explication ou une justification du passé ? \\[0pt]
L'histoire est-elle une science ? \\[0pt]
L'histoire est-elle une science comme les autres ? \\[0pt]
L'histoire est-elle un roman ? \\[0pt]
L'histoire et la raison \\[0pt]
« L'histoire jugera » : quel sens faut-il accorder à cette expression ? \\[0pt]
L'histoire jugera-t-elle ? \\[0pt]
L'histoire n'a-t-elle pour objet que le passé ? \\[0pt]
L'histoire n'est-elle que la connaissance du passé ? \\[0pt]
L'histoire n'est-elle que le produit de rapports de force ? \\[0pt]
L'histoire n'est-elle qu'un déchaînement de violence ? \\[0pt]
L'histoire n'est-elle qu'un récit ? \\[0pt]
L'histoire nous appartient-elle ? \\[0pt]
L'histoire obéit-elle à des lois ? \\[0pt]
L'histoire pèse-t-elle comme un destin sur les individus ? \\[0pt]
L'histoire peut-elle être contemporaine ? \\[0pt]
L'histoire peut-elle ne pas avoir de sens ? \\[0pt]
L'histoire se répète-t-elle ? \\[0pt]
L'histoire sert-elle le présent ? \\[0pt]
L'historicité \\[0pt]
L'historien \\[0pt]
L'historien et le politique \\[0pt]
L'historien peut-il être impartial ? \\[0pt]
L'homme aime-t-il la justice pour elle-même ? \\[0pt]
L'homme a-t-il besoin de l'art ? \\[0pt]
L'homme a-t-il une place dans la nature ? \\[0pt]
L'homme des droits de l'homme \\[0pt]
L'homme d'État \\[0pt]
L'homme est-il chez lui dans l'univers ? \\[0pt]
L'homme est-il face à la nature ? \\[0pt]
L'homme est-il la mesure de toute chose ? \\[0pt]
L'homme est-il l'artisan de sa dignité ? \\[0pt]
L'homme est-il le seul être à avoir une histoire ? \\[0pt]
L'homme est-il le sujet de son histoire ? \\[0pt]
L'homme est-il libre par nature ? \\[0pt]
L'homme est-il naturellement artiste ? \\[0pt]
L'homme est-il par nature un être religieux ? \\[0pt]
L'homme est-il un animal ? \\[0pt]
L'homme est-il un animal comme un autre ? \\[0pt]
L'homme est-il un animal dénaturé ? \\[0pt]
L'homme est-il un animal politique ? \\[0pt]
L'homme est-il un animal rationnel ? \\[0pt]
L'homme est-il un animal religieux ? \\[0pt]
L'homme est-il un animal social ? \\[0pt]
L'homme est-il un animal technicien ? \\[0pt]
L'homme est-il un corps pensant ? \\[0pt]
L'homme est-il un être social par nature ? \\[0pt]
L'homme est-il un loup pour l'homme ? \\[0pt]
L'homme et la machine \\[0pt]
L'homme et l'animal \\[0pt]
L'homme et le citoyen \\[0pt]
L'homme injuste est-il heureux ? \\[0pt]
L'homme injuste peut-il être heureux ? \\[0pt]
L'homme injuste peut-il être malheureux ? \\[0pt]
L'homme peut-il être inhumain ? \\[0pt]
L'homme peut-il faire l'objet d'une science ? \\[0pt]
L'homme se réalise-t-il dans le travail ? \\[0pt]
L'homme trouble-t-il l'ordre de la nature ? \\[0pt]
L'homme vertueux doit-il s'attendre à être malheureux ? \\[0pt]
L'honneur \\[0pt]
L'hospitalité \\[0pt]
L'humanité \\[0pt]
L'humanité a-t-elle une histoire ? \\[0pt]
L'humanité est-elle aimable ? \\[0pt]
L'humilité \\[0pt]
L'hypothèse \\[0pt]
L'hypothèse de la liberté est-elle compatible avec les exigences de la raison ? \\[0pt]
L'hypothèse de l'inconscient \\[0pt]
Liberté d'agir, liberté de penser \\[0pt]
« Liberté, égalité, fraternité » \\[0pt]
Liberté et courage \\[0pt]
Liberté et déterminisme \\[0pt]
Liberté et éducation \\[0pt]
Liberté et égalité \\[0pt]
Liberté et engagement \\[0pt]
Liberté et existence \\[0pt]
Liberté et indépendance \\[0pt]
Liberté et libération \\[0pt]
Liberté et licence \\[0pt]
Liberté et nécessité \\[0pt]
Liberté et pouvoir \\[0pt]
Liberté et responsabilité \\[0pt]
Liberté et savoir \\[0pt]
Liberté et sécurité \\[0pt]
Liberté et solitude \\[0pt]
Liberté et transgression \\[0pt]
Liberté et vérité \\[0pt]
Libre arbitre et déterminisme sont-ils compatibles ? \\[0pt]
Libre et heureux \\[0pt]
L'idéal \\[0pt]
L'idéal et le réel \\[0pt]
L'idée de bioéthique \\[0pt]
L'idée de bonheur collectif a-t-elle un sens ? \\[0pt]
L'idée de décadence \\[0pt]
L'idée de devoir requiert-elle l'idée de liberté ? \\[0pt]
L'idée de langue universelle \\[0pt]
L'idée de milieu \\[0pt]
L'idée de « nature » n'est-elle qu'un mythe ? \\[0pt]
L'idée de progrès \\[0pt]
L'idée de progrès historique est-elle devenue désuète ? \\[0pt]
L'idée de république \\[0pt]
L'idée d'inconscient \\[0pt]
L'idée d'organisme \\[0pt]
L'idée d'une liberté totale a-t- elle un sens ? \\[0pt]
L'idée d'une pensée inconsciente est-elle contradictoire ? \\[0pt]
L'idée d'une religion personnelle a-t-elle un sens ? \\[0pt]
L'identité \\[0pt]
L'identité personnelle \\[0pt]
L'idéologie \\[0pt]
L'idiot \\[0pt]
L'ignorance \\[0pt]
L'ignorance est-elle préférable à l'erreur ? \\[0pt]
L'ignorance peut-elle être une excuse ? \\[0pt]
L'illusion \\[0pt]
L'illusion est-elle nécessaire au bonheur des hommes ? \\[0pt]
L'image \\[0pt]
L'imagination \\[0pt]
L'imagination a-t-elle sa place dans la connaissance scientifique ? \\[0pt]
L'imagination dans les sciences \\[0pt]
L'imagination enrichit-elle la connaissance ? \\[0pt]
L'imagination est-elle le refuge de la liberté ? \\[0pt]
L'imagination scientifique \\[0pt]
L'imitation \\[0pt]
Limite, borne, frontière \\[0pt]
L'immatériel \\[0pt]
L'immédiat \\[0pt]
L'immoralisme \\[0pt]
L'immortalité \\[0pt]
L'immortalité de l'âme \\[0pt]
L'impartialité \\[0pt]
L'impensable \\[0pt]
L'imperceptible \\[0pt]
L'impiété \\[0pt]
L'impossible \\[0pt]
L'imprévisible \\[0pt]
L'inaperçu \\[0pt]
L'inattendu \\[0pt]
L'incertitude \\[0pt]
L'incertitude interdit-elle de raisonner ? \\[0pt]
L'inconscience \\[0pt]
L'inconscience peut-elle être coupable ? \\[0pt]
L'inconscient \\[0pt]
L'inconscient est-il dans l'âme ou dans le corps ? \\[0pt]
L'inconscient est-il une excuse ? \\[0pt]
L'inconscient est-il un obstacle à la liberté ? \\[0pt]
L'inconscient et l'involontaire \\[0pt]
L'inconscient et l'oubli \\[0pt]
L'inconscient n'est-il qu'un défaut de conscience ? \\[0pt]
L'inconscient n'est-il qu'une hypothèse ? \\[0pt]
L'inconscient peut-il se manifester ? \\[0pt]
L'indécidable \\[0pt]
L'indécision \\[0pt]
L'indéfini \\[0pt]
L'indémontrable \\[0pt]
L'indépendance \\[0pt]
L'indescriptible \\[0pt]
L'indésirable \\[0pt]
L'indice et la preuve \\[0pt]
L'indicible \\[0pt]
L'indicible et l'impensable \\[0pt]
L'indicible et l'ineffable \\[0pt]
L'indifférence \\[0pt]
L'indifférence peut-elle être une vertu ? \\[0pt]
L'indignation \\[0pt]
L'indignité \\[0pt]
L'indiscutable \\[0pt]
L'individu \\[0pt]
L'individu a-t-il des droits ? \\[0pt]
L'individu est-il ineffable ? \\[0pt]
L'individu et l'espèce \\[0pt]
L'induction \\[0pt]
L'indulgence \\[0pt]
L'ineffable et l'innommable \\[0pt]
L'inestimable \\[0pt]
L'inexistant \\[0pt]
L'inexpérience \\[0pt]
L'infini et l'indéfini \\[0pt]
L'infinité de l'univers a-t-elle de quoi nous effrayer ? \\[0pt]
L'ingénieur et le politique \\[0pt]
L'ingratitude \\[0pt]
L'inhumain \\[0pt]
L'inimaginable \\[0pt]
L'injustifiable \\[0pt]
L'innocence \\[0pt]
L'innovation \\[0pt]
L'innovation technique nous impose-t-elle de nouveaux devoirs ? \\[0pt]
L'inoubliable \\[0pt]
L'inquiétude \\[0pt]
L'inquiétude de l'avenir \\[0pt]
L'inquiétude peut-elle définir l'existence humaine ? \\[0pt]
L'inquiétude peut-elle devenir l'existence humaine ? \\[0pt]
L'insatisfaction \\[0pt]
L'insouciance \\[0pt]
L'instant \\[0pt]
L'instant et la durée \\[0pt]
L'instruction est-elle facteur de moralité ? \\[0pt]
L'instrument \\[0pt]
L'instrument et la machine \\[0pt]
L'intellect \\[0pt]
L'intelligence \\[0pt]
L'intelligence artificielle \\[0pt]
L'intelligence de la technique \\[0pt]
L'intelligence peut-elle être artificielle ? \\[0pt]
L'intelligence peut-elle être inhumaine ? \\[0pt]
L'intemporel \\[0pt]
L'interdit \\[0pt]
L'interdit est-il au fondement de la culture ? \\[0pt]
L'intérêt constitue-t-il l'unique lien social ? \\[0pt]
L'intérêt de la justice \\[0pt]
L'intérêt de la société l'emporte-t-il sur celui des individus ? \\[0pt]
L'intérêt de l'État \\[0pt]
L'intérêt des machines \\[0pt]
L'intérêt est-il le principe de tout échange ? \\[0pt]
L'intérêt général est-il la somme des intérêts particuliers ? \\[0pt]
L'intériorité \\[0pt]
L'interprétation \\[0pt]
L'interprétation est-elle sans fin ? \\[0pt]
L'interprétation est-elle un art ? \\[0pt]
L'interprétation est-elle un art ou une science ? \\[0pt]
L'interprétation est-elle une activité sans fin ? \\[0pt]
L'interprétation est-elle une forme de connaissance ? \\[0pt]
L'interprétation : mode d'être ou mode de connaissance ? \\[0pt]
L'interprète et le créateur \\[0pt]
L'interprète sait-il ce qu'il cherche ? \\[0pt]
L'intersubjectivité \\[0pt]
L'intimité \\[0pt]
L'intolérable \\[0pt]
L'introspection \\[0pt]
L'intuition \\[0pt]
L'intuition intellectuelle \\[0pt]
L'inutile \\[0pt]
L'inutile est-il sans valeur ? \\[0pt]
L'invention \\[0pt]
L'invention de soi \\[0pt]
L'invention et la découverte \\[0pt]
L'invention technique \\[0pt]
L'invérifiable \\[0pt]
L'invisible \\[0pt]
L'involontaire \\[0pt]
Lire et écrire \\[0pt]
L'ironie \\[0pt]
L'irrationnel \\[0pt]
L'irrationnel est-il nécessairement absurde ? \\[0pt]
L'irrationnel est-il pensable ? \\[0pt]
L'irrationnel est-il toujours absurde ? \\[0pt]
L'irrationnel et l'absurde \\[0pt]
L'irrationnel existe-t-il ? \\[0pt]
L'irréel \\[0pt]
L'irréfléchi \\[0pt]
L'irréfutable \\[0pt]
L'irréparable \\[0pt]
L'irrésolution \\[0pt]
L'irrespect \\[0pt]
L'irresponsabilité \\[0pt]
L'irréversibilité \\[0pt]
L'irréversible \\[0pt]
L'obéissance \\[0pt]
L'obéissance est-elle compatible avec la liberté ? \\[0pt]
L'objectivité \\[0pt]
L'objectivité de l'historien \\[0pt]
L'objectivité historique est-elle synonyme de neutralité ? \\[0pt]
L'objet d'amour \\[0pt]
L'objet de la physique \\[0pt]
L'objet du désir \\[0pt]
L'objet du désir en est-il la cause ? \\[0pt]
L'objet et la chose \\[0pt]
L'objet technique \\[0pt]
L'obligation \\[0pt]
L'obscur \\[0pt]
L'observation \\[0pt]
L'occasion \\[0pt]
Locke \\[0pt]
L'œuvre \\[0pt]
L'œuvre d'art a-t-elle un sens ? \\[0pt]
L'œuvre d'art doit-elle être belle ? \\[0pt]
L'œuvre d'art donne-t-elle à penser ? \\[0pt]
L'œuvre d'art échappe-t-elle au temps ? \\[0pt]
L'œuvre d'art échappe-t-elle nécessairement au temps ? \\[0pt]
L'œuvre d'art est-elle une marchandise ? \\[0pt]
L'œuvre d'art est-elle un monde ? \\[0pt]
L'œuvre d'art est-elle un objet d'échange ? \\[0pt]
L'œuvre d'art est-elle un symbole ? \\[0pt]
L'œuvre d'art instruit-elle ? \\[0pt]
L'œuvre d'art nous apprend-elle quelque chose ? \\[0pt]
Loi scientifique et causalité \\[0pt]
Loisir et oisiveté \\[0pt]
L'oisiveté \\[0pt]
L'omniscience \\[0pt]
L'opinion a-t-elle nécessairement tort ? \\[0pt]
L'opinion est-elle un savoir ? \\[0pt]
L'ordre des choses \\[0pt]
L'ordre des échanges est-il un ordre rationnel ? \\[0pt]
L'ordre du monde \\[0pt]
L'ordre du vivant est-il façonné par le hasard ? \\[0pt]
L'ordre et le désordre \\[0pt]
L'ordre social \\[0pt]
L'ordre social peut-il être juste ? \\[0pt]
L'organique \\[0pt]
L'organique et l'inorganique \\[0pt]
L'organisme \\[0pt]
L'originalité \\[0pt]
L'origine des idées \\[0pt]
L'oubli \\[0pt]
L'oubli et le pardon \\[0pt]
L'outil \\[0pt]
L'outil et la machine \\[0pt]
L'outil, l'instrument, la machine \\[0pt]
L'ouverture d'esprit \\[0pt]
L'unanimité est-elle un critère de vérité ? \\[0pt]
L'unité de la nature \\[0pt]
L'unité de l'État \\[0pt]
L'unité des sciences \\[0pt]
L'universel \\[0pt]
L'universel et le particulier \\[0pt]
L'universel et le singulier \\[0pt]
L'urbanité \\[0pt]
L'urgence \\[0pt]
L'urgence de vivre \\[0pt]
L'usage du doute \\[0pt]
L'utile et l'agréable \\[0pt]
L'utile et le beau \\[0pt]
L'utile et le bon \\[0pt]
L'utile et l'inutile \\[0pt]
L'utilité \\[0pt]
L'utopie et l'idéologie \\[0pt]
Machine et organisme \\[0pt]
Ma conscience est-elle digne de confiance ? \\[0pt]
Maître et disciple \\[0pt]
Maîtrise et puissance \\[0pt]
Maladie de l'âme et maladie du corps \\[0pt]
Mal et liberté \\[0pt]
Ma liberté s'arrête-t-elle où commence celle des autres ? \\[0pt]
Mémoire et souvenir \\[0pt]
Mentir \\[0pt]
Mérite-t-on d'être heureux ? \\[0pt]
Mesurer \\[0pt]
Mettre en commun \\[0pt]
Mettre en ordre \\[0pt]
Modèle et copie \\[0pt]
Mon corps \\[0pt]
Mon corps est-il naturel ? \\[0pt]
Mon corps fait-il obstacle à ma liberté ? \\[0pt]
Mon devoir dépend-il de moi ? \\[0pt]
Mon passé m'appartient-il ? \\[0pt]
Mon prochain est-il mon semblable ? \\[0pt]
Mon semblable \\[0pt]
Montrer et démontrer \\[0pt]
Morale et calcul \\[0pt]
Morale et économie \\[0pt]
Morale et liberté \\[0pt]
Moralité et utilité \\[0pt]
Naît-on sujet ou le devient-on ? \\[0pt]
N'apprend-on que par l'expérience ? \\[0pt]
Narration et identité \\[0pt]
N'a-t-on des devoirs qu'envers autrui ? \\[0pt]
Nature et artifice \\[0pt]
Nature et convention \\[0pt]
Nature et histoire \\[0pt]
Nature et loi \\[0pt]
Nature et morale \\[0pt]
N'avons-nous affaire qu'au réel ? \\[0pt]
N'avons-nous de devoirs qu'envers autrui ? \\[0pt]
Nécessité et contingence \\[0pt]
Nécessité et fatalité \\[0pt]
N'échange-t-on que ce qui a de la valeur ? \\[0pt]
N'échange-t-on que par intérêt ? \\[0pt]
Ne désire-t-on que ce dont on manque ? \\[0pt]
Ne désirons-nous que ce qui est bon pour nous ? \\[0pt]
Ne désirons-nous que les choses que nous estimons bonnes ? \\[0pt]
Ne doit-on tenir pour vrai que ce qui est démontré ? \\[0pt]
Ne faire que son devoir \\[0pt]
Ne faut-il pas craindre la liberté ? \\[0pt]
Ne faut-il vivre que dans le présent ? \\[0pt]
Ne faut-il vivre que pour faire son devoir ? \\[0pt]
Ne rien devoir à personne \\[0pt]
Ne sait-on rien que par expérience ? \\[0pt]
Ne sommes-nous justes que par intérêt ? \\[0pt]
Ne sommes-nous véritablement maîtres que de nos pensées ? \\[0pt]
N'est-il d'inférence que démonstrative ? \\[0pt]
Ne travaille-t-on que par nécessité ? \\[0pt]
Ne veut-on que ce qui est désirable ? \\[0pt]
Ne vit-on bien qu'avec ses amis ? \\[0pt]
N'exprime-t-on que ce dont on a conscience ? \\[0pt]
Nier la liberté de l'homme, est-ce ôter tout sens à la morale ? \\[0pt]
Nier le libre arbitre, est-ce nier la liberté ? \\[0pt]
N'interprète-t-on que ce qui est équivoque ? \\[0pt]
Nommer \\[0pt]
Normes et valeurs \\[0pt]
Nos convictions morales sont-elles le simple reflet de notre temps ? \\[0pt]
Nos désirs nous appartiennent-ils ? \\[0pt]
Nos désirs nous opposent-ils ? \\[0pt]
Nos devoirs se ramènent-ils aux conventions sociales ? \\[0pt]
Nos habitudes compromettent-elles notre moralité ? \\[0pt]
Nos paroles peuvent-elles dépasser notre pensée ? \\[0pt]
Nos pensées dépendent-elles de nous ? \\[0pt]
Nos pensées sont-elles entièrement en notre pouvoir ? \\[0pt]
Nos sens nous font-ils connaître la matière ? \\[0pt]
Notre liberté de pensée a-t-elle des limites ? \\[0pt]
Notre liberté est-elle toujours relative ? \\[0pt]
Notre pensée est-elle prisonnière de la langue que nous parlons ? \\[0pt]
Notre rapport au monde est-il essentiellement technique ? \\[0pt]
Notre rapport au monde peut-il être exclusivement technique ? \\[0pt]
Notre rapport au monde peut-il n'être que technique ? \\[0pt]
Nous trouvons-nous nous-mêmes dans l'animal ? \\[0pt]
Nouveauté et tradition \\[0pt]
N'y a-t-il de beauté qu'artistique ? \\[0pt]
N'y a-t-il de bonheur que dans l'instant ? \\[0pt]
N'y a-t-il de bonheur qu'éphémère ? \\[0pt]
N'y a-t-il de devoirs qu'envers autrui ? \\[0pt]
N'y a-t-il de droit qu'écrit ? \\[0pt]
N'y a-t-il de foi que religieuse ? \\[0pt]
N'y a-t-il de liberté qu'individuelle ? \\[0pt]
N'y a-t-il de rationalité que scientifique ? \\[0pt]
N'y a-t-il de réalité que de l'individuel ? \\[0pt]
N'y a-t-il de savoir que livresque ? \\[0pt]
N'y a-t-il de science que de ce qui est mathématisable ? \\[0pt]
N'y a-t-il de vérité que scientifique ? \\[0pt]
N'y a-t-il de vérité que vérifiable ? \\[0pt]
N'y a-t-il de vérités que scientifiques ? \\[0pt]
N'y a-t-il de vrai que le vérifiable ? \\[0pt]
N'y a-t-il que des individus ? \\[0pt]
N'y a-t-il que l'intention qui compte ? \\[0pt]
Obéir aux lois, est-ce renoncer à la liberté ? \\[0pt]
Obéissance et liberté \\[0pt]
Obéissance et soumission \\[0pt]
Objectivé et subjectivité \\[0pt]
Objectivité et impartialité \\[0pt]
Observation et expérience \\[0pt]
Observer et comprendre \\[0pt]
Observer et expérimenter \\[0pt]
Observer et interpréter \\[0pt]
Ontologie et métaphysique \\[0pt]
Opinion et ignorance \\[0pt]
Ordre et désordre \\[0pt]
Ordre et justice \\[0pt]
Ordre et liberté \\[0pt]
Ordre et progrès \\[0pt]
Organisme et milieu \\[0pt]
Origine et fondement \\[0pt]
Où commence la liberté ? \\[0pt]
Où commence la violence ? \\[0pt]
Où commence ma liberté ? \\[0pt]
Où est l'esprit ? \\[0pt]
Outil et machine \\[0pt]
Outil et organe \\[0pt]
Paraître \\[0pt]
Parier \\[0pt]
Par le langage, peut-on agir sur la réalité ? \\[0pt]
Parler, est-ce agir ? \\[0pt]
Parler, est-ce communiquer ? \\[0pt]
Parler, est-ce donner sa parole ? \\[0pt]
Parler, est-ce ne pas agir ? \\[0pt]
Parler et agir \\[0pt]
Parler, n'est-ce que désigner ? \\[0pt]
Parler pour ne rien dire \\[0pt]
Parler vrai \\[0pt]
Parole et pouvoir \\[0pt]
Passions et intérêts \\[0pt]
Penser, est-ce calculer ? \\[0pt]
Penser, est-ce désobéir ? \\[0pt]
Penser, est-ce dire non ? \\[0pt]
Penser, est-ce se parler à soi-même ? \\[0pt]
Penser et connaître \\[0pt]
Penser et imaginer \\[0pt]
Penser et parler \\[0pt]
Penser et savoir \\[0pt]
Penser et sentir \\[0pt]
Penser l'avenir \\[0pt]
Penser le changement \\[0pt]
Penser l'infini \\[0pt]
Penser par soi-même \\[0pt]
Penser par soi-même, est-ce être l'auteur de ses pensées ? \\[0pt]
Penser peut-il nous rendre heureux ? \\[0pt]
Pense-t-on jamais seul ? \\[0pt]
Perception et connaissance \\[0pt]
Perception et imagination \\[0pt]
Perception et interprétation \\[0pt]
Perception et sensation \\[0pt]
Percevoir \\[0pt]
Percevoir, est-ce interpréter ? \\[0pt]
Percevoir, est-ce juger ? \\[0pt]
Percevoir, est-ce nécessaire pour penser ? \\[0pt]
Percevoir, est-ce savoir ? \\[0pt]
Percevoir, est-ce s'ouvrir au monde ? \\[0pt]
Percevoir et concevoir \\[0pt]
Percevoir et imaginer \\[0pt]
Percevoir et témoigner \\[0pt]
Percevoir s'apprend-il ? \\[0pt]
Perçoit-on le réel ? \\[0pt]
Perçoit-on le réel tel qu'il est ? \\[0pt]
Perçoit-on les choses comme elles sont ? \\[0pt]
Perdre la raison \\[0pt]
Perdre ses illusions \\[0pt]
Perdre son identité \\[0pt]
Permanence et identité \\[0pt]
Personne et individu \\[0pt]
Peuple et multitude \\[0pt]
Peut-il exister une action désintéressée ? \\[0pt]
Peut-il exister une morale sceptique ? \\[0pt]
Peut-il y avoir conflit entre nos devoirs ? \\[0pt]
Peut-il y avoir de bonnes raisons de croire ? \\[0pt]
Peut-il y avoir de bons tyrans ? \\[0pt]
Peut-il y avoir des choix collectifs ? \\[0pt]
Peut-il y avoir des conflits de devoirs ? \\[0pt]
Peut-il y avoir des échanges équitables ? \\[0pt]
Peut-il y avoir des lois de l'histoire ? \\[0pt]
Peut-il y avoir des lois injustes ? \\[0pt]
Peut-il y avoir des modèles en morale ? \\[0pt]
Peut-il y avoir des vérités partielles ? \\[0pt]
Peut-il y avoir esprit sans corps ? \\[0pt]
Peut-il y avoir plusieurs vérités religieuses ? \\[0pt]
Peut-il y avoir savoir-faire sans savoir ? \\[0pt]
Peut-il y avoir un bien commun ? \\[0pt]
Peut-il y avoir une pensée sans langage ? \\[0pt]
Peut-il y avoir une science de la morale ? \\[0pt]
Peut-il y avoir une science de l'inconscient ? \\[0pt]
Peut-il y avoir une science du désordre ? \\[0pt]
Peut-il y avoir une science du moi ? \\[0pt]
Peut-il y avoir une société sans État ? \\[0pt]
Peut-il y avoir un État mondial ? \\[0pt]
Peut-il y avoir une vérité en art ? \\[0pt]
Peut-il y avoir une vérité religieuse ? \\[0pt]
Peut-il y avoir un langage universel ? \\[0pt]
Peut-on abolir la religion ? \\[0pt]
Peut-on affirmer que le monde a un ordre ? \\[0pt]
Peut-on agir de manière désintéressée ? \\[0pt]
Peut-on aimer l'autre tel qu'il est ? \\[0pt]
Peut-on aimer sans perdre sa liberté ? \\[0pt]
Peut-on aimer son prochain comme soi-même ? \\[0pt]
Peut-on aimer une œuvre d'art sans la comprendre ? \\[0pt]
Peut-on aller à l'encontre de la nature ? \\[0pt]
Peut-on apprendre à être libre ? \\[0pt]
Peut-on apprendre à mourir ? \\[0pt]
Peut-on apprendre à vivre ? \\[0pt]
Peut-on assimiler le vivant à une machine ? \\[0pt]
Peut-on atteindre une certitude ? \\[0pt]
Peut-on attendre de la politique qu'elle soit conforme aux exigences de la raison ? \\[0pt]
Peut-on attribuer à chacun son dû ? \\[0pt]
Peut-on avoir de bonnes raisons de ne pas dire la vérité ? \\[0pt]
Peut-on avoir des droits sans avoir de devoirs ? \\[0pt]
Peut-on avoir le droit de se révolter ? \\[0pt]
Peut-on avoir peur de la liberté ? \\[0pt]
Peut-on avoir peur de soi-même ? \\[0pt]
Peut-on avoir raison contre la science ? \\[0pt]
Peut-on avoir raison contre les faits ? \\[0pt]
Peut-on avoir raison contre tout le monde ? \\[0pt]
Peut-on avoir raisons contre les faits ? \\[0pt]
Peut-on avoir raison tout seul ? \\[0pt]
Peut-on cesser de croire ? \\[0pt]
Peut-on cesser de désirer ? \\[0pt]
Peut-on changer au point de ne plus être le même ? \\[0pt]
Peut-on changer le cours de l'histoire ? \\[0pt]
Peut-on changer le monde ? \\[0pt]
Peut-on changer ses désirs ? \\[0pt]
Peut-on changer ses désirs à volonté ? \\[0pt]
Peut-on choisir de renoncer à sa liberté ? \\[0pt]
Peut-on choisir ses désirs ? \\[0pt]
Peut-on combattre une croyance par le raisonnement ? \\[0pt]
Peut-on commander à la nature ? \\[0pt]
Peut-on communiquer ses perceptions à autrui ? \\[0pt]
Peut-on communiquer son expérience ? \\[0pt]
Peut-on comparer les cultures ? \\[0pt]
Peut-on comparer l'organisme à une machine ? \\[0pt]
Peut-on comprendre le présent ? \\[0pt]
Peut-on comprendre un acte que l'on désapprouve ? \\[0pt]
Peut-on comprendre un auteur mieux qu'il ne s'est compris lui-même ? \\[0pt]
Peut-on concevoir une humanité sans art ? \\[0pt]
Peut-on concevoir une religion dans les limites de la simple raison ? \\[0pt]
Peut-on concevoir une science sans expérience ? \\[0pt]
Peut-on concevoir une société juste sans que les hommes ne le soient ? \\[0pt]
Peut-on concevoir une société sans État ? \\[0pt]
Peut-on concilier bonheur et liberté ? \\[0pt]
Peut-on concilier le droit à la liberté et l'égalité des droits ? \\[0pt]
Peut-on connaître les choses telles qu'elles sont ? \\[0pt]
Peut-on connaître l'esprit ? \\[0pt]
Peut-on connaître le vivant sans le dénaturer ? \\[0pt]
Peut-on connaître le vivant sans recourir à la notion de finalité ? \\[0pt]
Peut-on connaître l'individuel ? \\[0pt]
Peut-on connaître par intuition ? \\[0pt]
Peut-on connaître sans concept ? \\[0pt]
Peut-on considérer la main comme un outil ? \\[0pt]
Peut-on contredire l'expérience ? \\[0pt]
Peut-on convaincre quelqu'un de la beauté d'une œuvre d'art ? \\[0pt]
Peut-on craindre la liberté ? \\[0pt]
Peut-on critiquer la démocratie ? \\[0pt]
Peut-on croire en rien ? \\[0pt]
Peut-on croire librement ? \\[0pt]
Peut-on décider d'être heureux ? \\[0pt]
Peut-on définir la matière ? \\[0pt]
Peut-on définir la morale comme l'art d'être heureux ? \\[0pt]
Peut-on définir la vie ? \\[0pt]
Peut-on définir le bonheur ? \\[0pt]
Peut-on délimiter le réel ? \\[0pt]
Peut-on démontrer l'existence de la matière ? \\[0pt]
Peut-on démontrer que l'on est libre ? \\[0pt]
Peut-on démontrer une théorie ? \\[0pt]
Peut-on dépasser la subjectivité ? \\[0pt]
Peut-on désirer autre chose qu'être heureux ? \\[0pt]
Peut-on désirer ce qui est ? \\[0pt]
Peut-on désirer ce qu'on sait être impossible ? \\[0pt]
Peut-on désirer l'absolu ? \\[0pt]
Peut-on désirer l'impossible ? \\[0pt]
Peut-on désirer sans souffrir ? \\[0pt]
Peut-on désobéir à l'État ? \\[0pt]
Peut-on désobéir aux lois ? \\[0pt]
Peut-on désobéir par devoir ? \\[0pt]
Peut-on dire ce que l'on pense ? \\[0pt]
Peut-on dire d'un homme qu'il est supérieur à un autre homme ? \\[0pt]
Peut-on dire la vérité ? \\[0pt]
Peut-on dire le singulier ? \\[0pt]
Peut-on dire que la science désenchante le monde ? \\[0pt]
Peut-on dire que les hommes font l'histoire ? \\[0pt]
Peut-on dire que les machines travaillent pour nous ? \\[0pt]
Peut-on dire que les mots pensent pour nous ? \\[0pt]
Peut-on dire que l'humanité progresse ? \\[0pt]
Peut-on dire que rien n'échappe à la technique ? \\[0pt]
Peut-on dire que toutes les croyances se valent ? \\[0pt]
Peut-on discuter des goûts et des couleurs ? \\[0pt]
Peut-on distinguer entre de bons et de mauvais désirs ? \\[0pt]
Peut-on distinguer entre les bons et les mauvais désirs ? \\[0pt]
Peut-on distinguer la vie et l'existence ? \\[0pt]
Peut-on distinguer le réel de l'imaginaire ? \\[0pt]
Peut-on distinguer les faits de leurs interprétations ? \\[0pt]
Peut-on donner un sens à son existence ? \\[0pt]
Peut-on douter de sa propre existence ? \\[0pt]
Peut-on douter de soi ? \\[0pt]
Peut-on douter de tout ? \\[0pt]
Peut-on douter de toute vérité ? \\[0pt]
Peut-on échanger des idées ? \\[0pt]
Peut-on échapper à ses désirs ? \\[0pt]
Peut-on échapper à son temps ? \\[0pt]
Peut-on échapper au cours de l'histoire ? \\[0pt]
Peut-on échapper au temps ? \\[0pt]
Peut-on échapper aux relations de pouvoir ? \\[0pt]
Peut-on éduquer la conscience ? \\[0pt]
Peut-on éduquer la sensibilité ? \\[0pt]
Peut-on en appeler à la conscience contre l'État ? \\[0pt]
Peut-on encore soutenir que l'homme est un animal rationnel ? \\[0pt]
Peut-on en finir avec la violence ? \\[0pt]
Peut-on espérer être libéré du travail ? \\[0pt]
Peut-on être à la fois lucide et heureux ? \\[0pt]
Peut-on être apolitique ? \\[0pt]
Peut-on être assuré d'avoir raison ? \\[0pt]
Peut-on être athée ? \\[0pt]
Peut-on être citoyen du monde ? \\[0pt]
Peut-on être dans le présent ? \\[0pt]
Peut-on être en conflit avec soi-même ? \\[0pt]
Peut-on être esclave de soi-même ? \\[0pt]
Peut-on être étranger à soi-même ? \\[0pt]
Peut-on être heureux dans la solitude ? \\[0pt]
Peut-on être heureux sans être sage ? \\[0pt]
Peut-on être heureux sans philosophie ? \\[0pt]
Peut-on être heureux sans s'en rendre compte ? \\[0pt]
Peut-on être ignorant ? \\[0pt]
Peut-on être indifférent à l'histoire ? \\[0pt]
Peut-on être injuste avec soi-même ? \\[0pt]
Peut-on être injuste envers soi-même ? \\[0pt]
Peut-on être insensible au vrai ? \\[0pt]
Peut-on être juste sans être impartial ? \\[0pt]
Peut-on être méchant volontairement ? \\[0pt]
Peut-on être moral sans religion ? \\[0pt]
Peut-on être obligé d'aimer ? \\[0pt]
Peut-on être plus ou moins libre ? \\[0pt]
Peut-on être sage inconsciemment ? \\[0pt]
Peut-on être sûr d'avoir raison ? \\[0pt]
Peut-on être sûr de bien agir ? \\[0pt]
Peut-on être sûr de ne pas se tromper ? \\[0pt]
Peut-on être trop sensible ? \\[0pt]
Peut-on étudier le passé de façon objective ? \\[0pt]
Peut-on expérimenter sur le vivant ? \\[0pt]
Peut-on expliquer le vivant ? \\[0pt]
Peut-on expliquer un acte libre ? \\[0pt]
Peut-on expliquer une œuvre d'art ? \\[0pt]
Peut-on faire de la politique sans supposer les hommes méchants ? \\[0pt]
Peut-on faire de l'esprit un objet de science ? \\[0pt]
Peut-on faire la philosophie de l'histoire ? \\[0pt]
Peut-on faire le bien d'autrui malgré lui ? \\[0pt]
Peut-on faire l'éloge de l'illusion ? \\[0pt]
Peut-on faire le mal innocemment ? \\[0pt]
Peut-on faire l'expérience de la nécessité ? \\[0pt]
Peut-on faire son devoir par habitude ? \\[0pt]
Peut-on faire table rase du passé ? \\[0pt]
Peut-on faire un mauvais usage de la raison ? \\[0pt]
Peut-on feindre la vertu ? \\[0pt]
Peut-on fonder la liberté ? \\[0pt]
Peut-on fonder la morale ? \\[0pt]
Peut-on fonder la morale sur la pitié ? \\[0pt]
Peut-on fonder le droit sur la morale ? \\[0pt]
Peut-on fonder un droit de désobéir ? \\[0pt]
Peut-on fonder une éthique sur la biologie ? \\[0pt]
Peut-on fonder une morale sur le plaisir ? \\[0pt]
Peut-on fuir la société ? \\[0pt]
Peut-on gouverner le développement des innovations techniques ? \\[0pt]
Peut-on haïr la raison ? \\[0pt]
Peut-on haïr la vie ? \\[0pt]
Peut-on haïr les images ? \\[0pt]
Peut-on hiérarchiser les arts ? \\[0pt]
Peut-on hiérarchiser les devoirs ? \\[0pt]
Peut-on identifier le désir au besoin ? \\[0pt]
Peut-on ignorer sa propre liberté ? \\[0pt]
Peut-on ignorer volontairement la vérité ? \\[0pt]
Peut-on imaginer l'avenir ? \\[0pt]
Peut-on imaginer un langage universel ? \\[0pt]
Peut-on invoquer la nature comme une excuse ? \\[0pt]
Peut-on jamais avoir la conscience tranquille ? \\[0pt]
Peut-on juger autrui ? \\[0pt]
Peut-on justifier la violence ? \\[0pt]
Peut-on justifier le mal ? \\[0pt]
Peut-on maîtriser la nature ? \\[0pt]
Peut-on maîtriser le temps ? \\[0pt]
Peut-on maîtriser l'évolution de la technique ? \\[0pt]
Peut-on maîtriser ses désirs ? \\[0pt]
Peut-on manquer de volonté ? \\[0pt]
Peut-on me contraindre à faire mon devoir ? \\[0pt]
Peut-on mentir par humanité ? \\[0pt]
Peut-on mesurer le temps ? \\[0pt]
Peut-on moraliser la guerre ? \\[0pt]
Peut-on ne croire en rien ? \\[0pt]
Peut-on ne pas connaître son bonheur ? \\[0pt]
Peut-on ne pas croire ? \\[0pt]
Peut-on ne pas croire à la science ? \\[0pt]
Peut-on ne pas croire au progrès ? \\[0pt]
Peut-on ne pas être égoïste ? \\[0pt]
Peut-on ne pas être soi-même ? \\[0pt]
Peut-on ne pas manquer de temps ? \\[0pt]
Peut-on ne pas perdre son temps ? \\[0pt]
Peut-on ne pas rechercher le bonheur ? \\[0pt]
Peut-on ne pas savoir ce que l'on dit ? \\[0pt]
Peut-on ne pas savoir ce que l'on fait ? \\[0pt]
Peut-on ne pas savoir ce que l'on veut ? \\[0pt]
Peut-on ne penser à rien ? \\[0pt]
Peut-on ne rien devoir à personne ? \\[0pt]
Peut-on nier le réel ? \\[0pt]
Peut-on nier l'évidence ? \\[0pt]
Peut-on nier l'existence de la matière ? \\[0pt]
Peut-on nier l'existence du temps ? \\[0pt]
Peut-on opposer connaissance scientifique et création artistique ? \\[0pt]
Peut-on opposer le loisir au travail ? \\[0pt]
Peut-on opposer nature et culture ? \\[0pt]
Peut-on opposer religion et raison ? \\[0pt]
Peut-on ôter à l'homme sa liberté ? \\[0pt]
Peut-on parler de conscience collective ? \\[0pt]
Peut-on parler de dialogue des cultures ? \\[0pt]
Peut-on parler de mondes imaginaires ? \\[0pt]
Peut-on parler de « nature humaine » ? \\[0pt]
Peut-on parler de nourriture spirituelle ? \\[0pt]
Peut-on parler de problèmes techniques ? \\[0pt]
Peut-on parler des miracles de la technique ? \\[0pt]
Peut-on parler de travail intellectuel ? \\[0pt]
Peut-on parler de vérité en art ? \\[0pt]
Peut-on parler de vérité subjective ? \\[0pt]
Peut-on parler de violence d'État ? \\[0pt]
Peut-on parler d'une morale collective ? \\[0pt]
Peut-on parler d'une religion de l'humanité ? \\[0pt]
Peut-on parler d'univers symbolique ? \\[0pt]
Peut-on parler d'un progrès dans l'histoire ? \\[0pt]
Peut-on parler d'un progrès de la liberté ? \\[0pt]
Peut-on parler pour ne rien dire ? \\[0pt]
Peut-on parler sans incohérence d'une libre obéissance ? \\[0pt]
Peut-on penser ce qu'on ne peut dire ? \\[0pt]
Peut-on penser contre l'expérience ? \\[0pt]
Peut-on penser la justice comme une compétence ? \\[0pt]
Peut-on penser la matière ? \\[0pt]
Peut-on penser l'art sans référence au beau ? \\[0pt]
Peut-on penser la vie ? \\[0pt]
Peut-on penser la vie sans penser la mort ? \\[0pt]
Peut-on penser le temps ? \\[0pt]
Peut-on penser l'homme à partir de la nature ? \\[0pt]
Peut-on penser l'infini ? \\[0pt]
Peut-on penser l'œuvre d'art sans référence à l'idée de beauté ? \\[0pt]
Peut-on penser sans image ? \\[0pt]
Peut-on penser sans images ? \\[0pt]
Peut-on penser sans les mots ? \\[0pt]
Peut-on penser sans méthode ? \\[0pt]
Peut-on penser sans préjugés ? \\[0pt]
Peut-on penser une fin de l'histoire ? \\[0pt]
Peut-on penser un État sans violence ? \\[0pt]
Peut-on percevoir le temps ? \\[0pt]
Peut-on percevoir sans juger ? \\[0pt]
Peut-on perdre la raison ? \\[0pt]
Peut-on perdre sa liberté ? \\[0pt]
Peut-on perdre son temps ? \\[0pt]
Peut-on prédire les événements ? \\[0pt]
Peut-on prédire l'histoire ? \\[0pt]
Peut-on préférer le bonheur à la vérité ? \\[0pt]
Peut-on préférer l'injustice au désordre ? \\[0pt]
Peut-on préférer l'ordre à la justice ? \\[0pt]
Peut-on prendre les moyens pour la fin ? \\[0pt]
Peut-on promettre le bonheur ? \\[0pt]
Peut-on protéger les libertés sans les réduire ? \\[0pt]
Peut-on prouver la réalité de l'esprit ? \\[0pt]
Peut-on prouver l'existence ? \\[0pt]
Peut-on prouver une existence ? \\[0pt]
Peut-on raconter sa vie ? \\[0pt]
Peut-on ralentir la course du temps ? \\[0pt]
Peut-on recommencer sa vie ? \\[0pt]
Peut-on reconnaître un sens à l'histoire sans lui assigner une fin ? \\[0pt]
Peut-on réduire la vie à la matière ? \\[0pt]
Peut-on réduire le raisonnement au calcul ? \\[0pt]
Peut-on réduire l'esprit à la matière ? \\[0pt]
Peut-on refuser la vérité ? \\[0pt]
Peut-on refuser la violence ? \\[0pt]
Peut-on refuser le bonheur ? \\[0pt]
Peut-on refuser l'évidence ? \\[0pt]
Peut-on refuser le vrai ? \\[0pt]
Peut-on réfuter le scepticisme ? \\[0pt]
Peut-on rejeter le principe de non-contradiction ? \\[0pt]
Peut-on rendre raison des émotions ? \\[0pt]
Peut-on rendre raison de tout ? \\[0pt]
Peut-on rendre raison du réel ? \\[0pt]
Peut-on renoncer à être libre ? \\[0pt]
Peut-on renoncer à la liberté ? \\[0pt]
Peut-on renoncer à la vérité ? \\[0pt]
Peut-on renoncer au bonheur ? \\[0pt]
Peut-on réparer le vivant ? \\[0pt]
Peut-on répondre au sceptique ? \\[0pt]
Peut-on répondre d'autrui ? \\[0pt]
Peut-on reprocher au langage d'être équivoque ? \\[0pt]
Peut-on reprocher au langage d'être parfait ? \\[0pt]
Peut-on résister au vrai ? \\[0pt]
Peut-on retenir le temps ? \\[0pt]
Peut-on revendiquer le droit de désobéir ? \\[0pt]
Peut-on rompre avec la société ? \\[0pt]
Peut-on rompre avec le passé ? \\[0pt]
Peut-on s'affranchir de la subjectivité ? \\[0pt]
Peut-on s'affranchir des lois ? \\[0pt]
Peut-on saisir le temps ? \\[0pt]
Peut-on s'attendre à tout ? \\[0pt]
Peut-on savoir sans croire ? \\[0pt]
Peut-on se choisir un destin ? \\[0pt]
Peut-on se connaître soi-même ? \\[0pt]
Peut-on se fier à la technique ? \\[0pt]
Peut-on se gouverner soi-même ? \\[0pt]
Peut-on se mentir à soi-même ? \\[0pt]
Peut-on se mettre à la place d'autrui ? \\[0pt]
Peut-on se mettre à la place de l'autre ? \\[0pt]
Peut-on sentir le temps qui passe ? \\[0pt]
Peut-on séparer la politique de l'exigence de vérité ? \\[0pt]
Peut-on se passer de croire ? \\[0pt]
Peut-on se passer de croyances ? \\[0pt]
Peut-on se passer de la religion ? \\[0pt]
Peut-on se passer de la technique ? \\[0pt]
Peut-on se passer de l'État ? \\[0pt]
Peut-on se passer de l'idée de droit naturel ? \\[0pt]
Peut-on se passer de l'idée de finalité ? \\[0pt]
Peut-on se passer de maître ? \\[0pt]
Peut-on se passer de métaphysique ? \\[0pt]
Peut-on se passer de religion ? \\[0pt]
Peut-on se passer d'État ? \\[0pt]
Peut-on se passer de technique ? \\[0pt]
Peut-on se passer de toute religion ? \\[0pt]
Peut-on se passer d'idéal ? \\[0pt]
Peut-on se passer d'un maître ? \\[0pt]
Peut-on se prescrire une loi ? \\[0pt]
Peut-on se rendre maître de la technique ? \\[0pt]
Peut-on servir deux maîtres à la fois ? \\[0pt]
Peut-on se soustraire à son devoir ? \\[0pt]
Peut-on se tromper en se croyant heureux ? \\[0pt]
Peut-on se tromper sur son devoir ? \\[0pt]
Peut-on s'exprimer par le silence ? \\[0pt]
Peut-on s'obliger soi-même ? \\[0pt]
Peut-on s'oublier ? \\[0pt]
Peut-on souhaiter ne pas être libre ? \\[0pt]
Peut-on sympathiser avec l'ennemi ? \\[0pt]
Peut-on tirer des leçons de l'histoire ? \\[0pt]
Peut-on tolérer l'injustice ? \\[0pt]
Peut-on toujours faire ce qu'on doit ? \\[0pt]
Peut-on toujours raison garder ? \\[0pt]
Peut-on toujours savoir entièrement ce que l'on dit ? \\[0pt]
Peut-on tout analyser ? \\[0pt]
Peut-on tout attendre de l'État ? \\[0pt]
Peut-on tout démontrer ? \\[0pt]
Peut-on tout dire ? \\[0pt]
Peut-on tout donner ? \\[0pt]
Peut-on tout échanger ? \\[0pt]
Peut-on tout enseigner ? \\[0pt]
Peut-on tout interpréter ? \\[0pt]
Peut-on tout ordonner ? \\[0pt]
Peut-on traiter un être vivant comme une machine ? \\[0pt]
Peut-on trancher les conflits entre interprétations ? \\[0pt]
Peut-on vivre en dehors de l'histoire ? \\[0pt]
Peut-on vivre en sceptique ? \\[0pt]
Peut-on vivre hors du temps ? \\[0pt]
Peut-on vivre pour la vérité ? \\[0pt]
Peut-on vivre sans désir ? \\[0pt]
Peut-on vivre sans échange ? \\[0pt]
Peut-on vivre sans foi ni loi ? \\[0pt]
Peut-on vivre sans le plaisir de vivre ? \\[0pt]
Peut-on vivre sans lois ? \\[0pt]
Peut-on vivre sans peur ? \\[0pt]
Peut-on vivre sans réfléchir ? \\[0pt]
Peut-on vivre sans sacré ? \\[0pt]
Peut-on vouloir ce qu'on ne désire pas ? \\[0pt]
Peut-on vouloir le bien sans le faire ? \\[0pt]
Peut-on vouloir le bonheur d'autrui ? \\[0pt]
Peut-on vouloir le mal ? \\[0pt]
Peut-on vouloir ne pas être libre ? \\[0pt]
Peut-on vraiment créer ? \\[0pt]
Peut-on vraiment dire ce qu'on pense ? \\[0pt]
Peut-on vraiment parler de soi ? \\[0pt]
Philosophe-t-on pour être heureux ? \\[0pt]
Physique et mathématiques \\[0pt]
Physique et métaphysique \\[0pt]
Pitié et compassion \\[0pt]
Pitié et cruauté \\[0pt]
Pitié et mépris \\[0pt]
Plaisir et bonheur \\[0pt]
Plusieurs religions valent-elles mieux qu'une seule ? \\[0pt]
Poésie et philosophie \\[0pt]
Poésie et vérité \\[0pt]
Politique et vérité \\[0pt]
Politique et vertu \\[0pt]
Possession et propriété \\[0pt]
Pour agir moralement, faut-il ne pas se soucier de soi ? \\[0pt]
Pour bien agir, faut-il vouloir le bien d'autrui ? \\[0pt]
Pour bien penser, faut-il renoncer au langage courant ? \\[0pt]
Pour connaître, suffit-il de démontrer ? \\[0pt]
Pour être heureux, faut-il ne rien désirer ? \\[0pt]
Pour être homme, faut-il être citoyen ? \\[0pt]
Pour être libre, faut-il renoncer à être heureux ? \\[0pt]
Pour être un bon observateur faut-il être un bon théoricien ? \\[0pt]
Pour juger, faut-il seulement apprendre à raisonner ? \\[0pt]
Pour penser rigoureusement, faut-il renoncer au langage ordinaire ? \\[0pt]
Pourquoi accomplir son devoir ? \\[0pt]
Pourquoi aimer la liberté ? \\[0pt]
Pourquoi aller contre son désir ? \\[0pt]
Pourquoi avons-nous besoin des autres pour être heureux ? \\[0pt]
Pourquoi avons-nous du mal à reconnaître la vérité ? \\[0pt]
Pourquoi avons-nous tant de mal à renoncer à nos illusions ? \\[0pt]
Pourquoi chercher à connaître le passé ? \\[0pt]
Pourquoi chercher à se connaître soi-même ? \\[0pt]
Pourquoi chercher à se distinguer ? \\[0pt]
Pourquoi chercher à vivre libre ? \\[0pt]
Pourquoi chercher la vérité ? \\[0pt]
Pourquoi cherche-t-on à connaître ? \\[0pt]
Pourquoi défendre le faible ? \\[0pt]
Pourquoi délibérer ? \\[0pt]
Pourquoi des artistes ? \\[0pt]
Pourquoi des cérémonies ? \\[0pt]
Pourquoi des devoirs ? \\[0pt]
Pourquoi des hommes politiques ? \\[0pt]
Pourquoi des idoles ? \\[0pt]
Pourquoi désire-t-on ce dont on n'a nul besoin ? \\[0pt]
Pourquoi désirons-nous ? \\[0pt]
Pourquoi des lois ? \\[0pt]
Pourquoi des maîtres ? \\[0pt]
Pourquoi des poètes ? \\[0pt]
Pourquoi des symboles ? \\[0pt]
Pourquoi des utopies ? \\[0pt]
Pourquoi dialogue-t-on ? \\[0pt]
Pourquoi distinguer nature et culture ? \\[0pt]
Pourquoi donner ? \\[0pt]
Pourquoi donner des leçons de morale ? \\[0pt]
Pourquoi échanger des idées ? \\[0pt]
Pourquoi écrit-on ? \\[0pt]
Pourquoi écrit-on les lois ? \\[0pt]
Pourquoi écrit-on l'Histoire ? \\[0pt]
Pourquoi être moral ? \\[0pt]
Pourquoi étudier le vivant ? \\[0pt]
Pourquoi étudier l'Histoire ? \\[0pt]
Pourquoi faire confiance ? \\[0pt]
Pourquoi faire confiance au dialogue ? \\[0pt]
Pourquoi faire son devoir ? \\[0pt]
Pourquoi faudrait-il avoir peur de la technique ? \\[0pt]
Pourquoi faut-il diviser le travail ? \\[0pt]
Pourquoi faut-il être juste ? \\[0pt]
Pourquoi faut-il travailler ? \\[0pt]
Pourquoi interprète-t-on ? \\[0pt]
Pourquoi joue-t-on ? \\[0pt]
Pourquoi la justice a-t-elle besoin d'institutions ? \\[0pt]
Pourquoi les États se font-ils la guerre ? \\[0pt]
Pourquoi les hommes se soumettent-ils à l'autorité ? \\[0pt]
Pourquoi les sciences ont-elles une histoire ? \\[0pt]
Pourquoi les sociétés ont-elles besoin de lois ? \\[0pt]
Pourquoi l'Homme se pose-t-il des questions métaphysiques ? \\[0pt]
Pourquoi l'homme travaille-t-il ? \\[0pt]
Pourquoi lire les poètes ? \\[0pt]
Pourquoi ne peut-on concevoir la science comme achevée ? \\[0pt]
Pourquoi nous trompons-nous ? \\[0pt]
Pourquoi n'y aurait-il pas de sots métiers ? \\[0pt]
Pourquoi parle-t-on ? \\[0pt]
Pourquoi préserver la nature ? \\[0pt]
Pourquoi punir ? \\[0pt]
Pourquoi rechercher la vérité ? \\[0pt]
Pourquoi refuser de faire son devoir ? \\[0pt]
Pourquoi refuse-t-on la conscience à l'animal ? \\[0pt]
Pourquoi respecter autrui ? \\[0pt]
Pourquoi respecter le droit ? \\[0pt]
Pourquoi s'ennuie-t-on ? \\[0pt]
Pourquoi se taire ? \\[0pt]
Pourquoi s'intéresser à l'histoire ? \\[0pt]
Pourquoi s'interroger sur l'origine du langage ? \\[0pt]
Pourquoi sommes-nous des êtres moraux ? \\[0pt]
Pourquoi sommes-nous moraux ? \\[0pt]
Pourquoi suspendre son jugement ? \\[0pt]
Pourquoi théoriser ? \\[0pt]
Pourquoi transmettre ? \\[0pt]
Pourquoi travailler ? \\[0pt]
Pourquoi un fait devrait-il être établi ? \\[0pt]
Pourquoi vivre ensemble ? \\[0pt]
Pourquoi vouloir devenir « comme maîtres et possesseurs de la nature » ? \\[0pt]
Pourquoi vouloir être libre ? \\[0pt]
Pourquoi vouloir la vérité ? \\[0pt]
Pourquoi vouloir se connaître ? \\[0pt]
Pourquoi y a-t-il des institutions ? \\[0pt]
Pourquoi y a-t-il plusieurs sciences ? \\[0pt]
Pourrait-on se passer de l'argent ? \\[0pt]
Pour s'aimer, faut-il se connaître ? \\[0pt]
Pour se trouver, faut-il savoir se perdre ? \\[0pt]
Pouvoir et autorité \\[0pt]
Pouvoir et devoir \\[0pt]
Pouvoir et domination \\[0pt]
Pouvoir et puissance \\[0pt]
Pouvoir et savoir \\[0pt]
Pouvons-nous connaître sans interpréter ? \\[0pt]
Pouvons-nous désirer ce qui nous nuit ? \\[0pt]
Pouvons-nous dissocier le réel de nos interprétations ? \\[0pt]
Pouvons-nous être certains que nous ne rêvons pas ? \\[0pt]
Pouvons-nous faire l'expérience de la liberté ? \\[0pt]
Prédire et expliquer \\[0pt]
Prendre conscience \\[0pt]
Prendre la parole \\[0pt]
Prendre la parole, est-ce prendre le pouvoir ? \\[0pt]
Prendre ses responsabilités \\[0pt]
Prendre soin \\[0pt]
Prendre son temps \\[0pt]
Prendre son temps, est-ce le perdre ? \\[0pt]
Prêter attention \\[0pt]
Preuve et démonstration \\[0pt]
Principe et axiome \\[0pt]
Privation et négation \\[0pt]
Production et action \\[0pt]
Production et création \\[0pt]
Produire et créer \\[0pt]
Promettre, est-ce renoncer à sa liberté ? \\[0pt]
Prose et poésie \\[0pt]
Prouver \\[0pt]
Prouver et démontrer \\[0pt]
Prouver et éprouver \\[0pt]
Prouver et réfuter \\[0pt]
Prudence et liberté \\[0pt]
Puis-je avoir la certitude que mes choix sont libres ? \\[0pt]
Puis-je devenir un autre ? \\[0pt]
Puis-je être dans le vrai sans le savoir ? \\[0pt]
Puis-je être libre sans être responsable ? \\[0pt]
Puis-je faire confiance à mes sens ? \\[0pt]
Puis-je invoquer l'inconscient sans ruiner la morale ? \\[0pt]
Puis-je me connaître sans l'aide des autres ? \\[0pt]
Puis-je me connaître si je change à travers le temps ? \\[0pt]
Puis-je me passer d'imiter autrui ? \\[0pt]
Puis-je me trouver moi-même sans l'aide d'autrui ? \\[0pt]
Puis-je ne pas vouloir ce que je désire ? \\[0pt]
Puis-je ne rien devoir à personne ? \\[0pt]
Puis-je répondre des autres ? \\[0pt]
Puis-je savoir avec assurance en quoi consiste mon devoir ? \\[0pt]
Puis-je savoir ce qui m'est propre ? \\[0pt]
Punir \\[0pt]
Punition et vengeance \\[0pt]
Qu'ai-je le droit d'exiger d'autrui ? \\[0pt]
Qu'ai-je le droit d'exiger des autres ? \\[0pt]
Qu'aime-t-on dans l'amour ? \\[0pt]
Qualité et quantité \\[0pt]
Quand manquons-nous à nos devoirs ? \\[0pt]
Quand une autorité est-elle légitime ? \\[0pt]
Qu'apprend-on des romans ? \\[0pt]
Qu'apprend-on en apprenant une technique ? \\[0pt]
Qu'apprend-on en commettant une faute ? \\[0pt]
Qu'apprend-on quand on apprend à parler ? \\[0pt]
Qu'apprenons-nous en éduquant ? \\[0pt]
Qu'a-t-on le droit d'interdire ? \\[0pt]
Qu'attendons-nous de la science ? \\[0pt]
Qu'attendons-nous de la technique ? \\[0pt]
Qu'attendons-nous pour être heureux ? \\[0pt]
Que célèbre l'art ? \\[0pt]
Que démontrent nos actions ? \\[0pt]
Que désirons-nous ? \\[0pt]
Que désirons-nous quand nous désirons savoir ? \\[0pt]
Que devons-nous à autrui ? \\[0pt]
Que devons-nous à l'État ? \\[0pt]
Que dois-je faire ? \\[0pt]
Que dois-je respecter en autrui ? \\[0pt]
Que doit la pensée à l'écriture ? \\[0pt]
Que doit la science à la technique ? \\[0pt]
Que doit-on à l'État ? \\[0pt]
Que doit-on croire ? \\[0pt]
Que doit-on désirer pour ne pas être déçu ? \\[0pt]
Que faire de nos passions ? \\[0pt]
Que faire des adversaires ? \\[0pt]
Que faire quand la loi est injuste ? \\[0pt]
Que faut-il absolument savoir ? \\[0pt]
Que faut-il respecter ? \\[0pt]
Que faut-il savoir pour bien agir ? \\[0pt]
Que faut-il savoir pour pouvoir gouverner ? \\[0pt]
Que gagne-t-on à travailler ? \\[0pt]
Quel besoin avons-nous de chercher la vérité ? \\[0pt]
Quel est le contraire du travail ? \\[0pt]
Quel est le poids du passé ? \\[0pt]
Quel est le propre de la pensée technique ? \\[0pt]
Quel est le sens du progrès technique ? \\[0pt]
Quel est l'objet de la biologie ? \\[0pt]
Quel est l'objet de la conscience de soi ? \\[0pt]
Quel est l'objet de la métaphysique ? \\[0pt]
Quel est l'objet de la physique ? \\[0pt]
Quel est l'objet des mathématiques ? \\[0pt]
Quel est l'objet des « sciences de l'esprit » ? \\[0pt]
Quel est l'objet du désir ? \\[0pt]
Quel genre de conscience peut-on accorder à l'animal ? \\[0pt]
Quel intérêt y a-t-il à se connaître ? \\[0pt]
Quelle causalité pour le vivant ? \\[0pt]
Quelle confiance accorder au langage ? \\[0pt]
Quelle différence y a-t-il entre un corps vivant et un corps mort ? \\[0pt]
Quelle est la cause du désir ? \\[0pt]
Quelle est la fin de la science ? \\[0pt]
Quelle est la fonction première de l'État ? \\[0pt]
Quelle est la force de la loi ? \\[0pt]
Quelle est la place de l'imagination dans la vie de l'esprit ? \\[0pt]
Quelle est la place du hasard dans l'histoire ? \\[0pt]
Quelle est la réalité de l'avenir ? \\[0pt]
Quelle est la réalité d'une idée ? \\[0pt]
Quelle est la réalité du passé ? \\[0pt]
Quelle est la réalité du temps ? \\[0pt]
Quelle est la source de nos devoirs ? \\[0pt]
Quelle est la valeur culturelle de la science ? \\[0pt]
Quelle est la valeur d'une expérimentation ? \\[0pt]
Quelle est la valeur d'une promesse ? \\[0pt]
Quelle est la valeur du rêve ? \\[0pt]
Quelle est la valeur du temps ? \\[0pt]
Quelle est la valeur du vivant ? \\[0pt]
Quelle est l'unité du « je » ? \\[0pt]
Quelle maîtrise avons-nous du temps ? \\[0pt]
Quelle réalité attribuer à la matière ? \\[0pt]
Quelle réalité l'art nous fait-il connaître ? \\[0pt]
Quelles limites assigner au pouvoir de la parole ? \\[0pt]
Quelles limites poser à l'autorité de l'État ? \\[0pt]
Quelle sorte d'histoire ont les sciences ? \\[0pt]
Quelles sont les caractéristiques d'un être vivant ? \\[0pt]
Quelles sont les limites de mon monde ? \\[0pt]
Quelle valeur devons-nous accorder à l'expérience ? \\[0pt]
Quels devoirs les religions peuvent-elles énoncer ? \\[0pt]
Quel sens donner à l'expression « gagner sa vie » ? \\[0pt]
Quels enseignements peut-on tirer de l'histoire des sciences ? \\[0pt]
Quel usage faut-il faire des exemples ? \\[0pt]
Que manque-t-il à une machine pour être vivante ? \\[0pt]
Que mesure-t-on du temps ? \\[0pt]
Que montre une démonstration ? \\[0pt]
Que m'ordonne ma conscience ? \\[0pt]
Que nous append l'histoire ? \\[0pt]
Que nous apporte la technique ? \\[0pt]
Que nous apporte la vérité ? \\[0pt]
Que nous apporte le travail ? \\[0pt]
Que nous apprend la définition de la vérité ? \\[0pt]
Que nous apprend la diversité des langues ? \\[0pt]
Que nous apprend la fiction sur la réalité ? \\[0pt]
Que nous apprend la maladie ? \\[0pt]
Que nous apprend la maladie sur la santé ? \\[0pt]
Que nous apprend la musique ? \\[0pt]
Que nous apprend la technique ? \\[0pt]
Que nous apprend la vie ? \\[0pt]
Que nous apprend l'expérience ? \\[0pt]
Que nous apprennent les animaux ? \\[0pt]
Que nous apprennent les animaux sur nous-mêmes ? \\[0pt]
Que nous apprennent les machines ? \\[0pt]
Que nous apprennent les métaphores ? \\[0pt]
Que nous apprennent nos erreurs ? \\[0pt]
Que nous enseigne l'expérience ? \\[0pt]
Que nous enseignent les sens ? \\[0pt]
Que nous impose le temps ? \\[0pt]
Que nous réserve l'avenir ? \\[0pt]
Que peint le peintre ? \\[0pt]
Que penser de l'adage : « Que la justice s'accomplisse, le monde dût-il périr » (Fiat justitia pereat mundus) ? \\[0pt]
Que percevons-nous d'autrui ? \\[0pt]
Que perd-on quand on perd son temps ? \\[0pt]
Que perdrait la pensée en perdant l'écriture ? \\[0pt]
Que peut la musique ? \\[0pt]
Que peut la volonté ? \\[0pt]
Que peut le corps ? \\[0pt]
Que peut l'esprit sur la matière ? \\[0pt]
Que peut l'État ? \\[0pt]
Que peut-on contre un préjugé ? \\[0pt]
Que peut-on savoir de l'inconscient ? \\[0pt]
Que peut-on savoir de soi ? \\[0pt]
Que peut-on savoir du réel ? \\[0pt]
Que peut-on savoir par expérience ? \\[0pt]
Que peut prétendre imposer une religion ? \\[0pt]
Que peut signifier : « gérer son temps » ? \\[0pt]
Que peut-signifier « tuer le temps » ? \\[0pt]
Que pouvons-nous attendre de la technique ? \\[0pt]
Que pouvons-nous espérer de la connaissance du vivant ? \\[0pt]
Que pouvons-nous faire de notre passé ? \\[0pt]
Que produit l'inconscient ? \\[0pt]
Que prouvent les preuves de l'existence de Dieu ? \\[0pt]
Que reste-t-il d'une existence ? \\[0pt]
Que sait la conscience ? \\[0pt]
Que sait-on du réel ? \\[0pt]
Que savons-nous de notre vie ? \\[0pt]
Que serait une pure sensation ? \\[0pt]
Que serions-nous sans l'État ? \\[0pt]
Que signifie : « échanger des arguments » ? \\[0pt]
Que signifie être en guerre ? \\[0pt]
Que signifie l'expression : « l'histoire jugera » ? \\[0pt]
Que signifie l'idée de technoscience ? \\[0pt]
Que signifie refuser l'injustice ? \\[0pt]
Que signifier « juger en son âme et conscience » ? \\[0pt]
Que signifie : « se rendre à l'évidence » ? \\[0pt]
Que sont les apparences ? \\[0pt]
Que sont les objets de la perception ? \\[0pt]
Qu'est-ce qu'apprendre ? \\[0pt]
Qu'est-ce qu'argumenter ? \\[0pt]
Qu'est-ce qu'avoir du jugement ? \\[0pt]
Qu'est-ce que bien vivre ? \\[0pt]
Qu'est-ce que commencer ? \\[0pt]
Qu'est-ce que composer une œuvre ? \\[0pt]
Qu'est-ce que comprendre une œuvre d'art ? \\[0pt]
Qu'est-ce que contempler ? \\[0pt]
Qu'est-ce que créer ? \\[0pt]
Qu'est-ce que croire ? \\[0pt]
Qu'est-ce que définir ? \\[0pt]
Qu'est-ce que Dieu pour un athée ? \\[0pt]
Qu'est-ce que donner sa parole ? \\[0pt]
Qu'est-ce que faire une expérience ? \\[0pt]
Qu'est-ce que « faire usage de sa raison » ? \\[0pt]
Qu'est-ce que gouverner ? \\[0pt]
Qu'est-ce que juger ? \\[0pt]
Qu'est-ce que la causalité ? \\[0pt]
Qu'est-ce que la religion nous donne à savoir ? \\[0pt]
Qu'est-ce que la science doit à l'expérience ? \\[0pt]
Qu'est-ce que la science saisit du vivant ? \\[0pt]
Qu'est-ce que la technique doit à la nature ? \\[0pt]
Qu'est-ce que le langage ordinaire ? \\[0pt]
Qu'est-ce que le malheur ? \\[0pt]
Qu'est-ce que le moi ? \\[0pt]
Qu'est-ce que le présent ? \\[0pt]
Qu'est-ce que le progrès technique ? \\[0pt]
Qu'est-ce que le réel ? \\[0pt]
Qu'est-ce que le sacré ? \\[0pt]
Qu'est-ce que l'inconscient ? \\[0pt]
Qu'est-ce que l'intérêt général ? \\[0pt]
Qu'est-ce que l'objectivité scientifique ? \\[0pt]
Qu'est-ce que manquer de culture ? \\[0pt]
Qu'est-ce que mesurer le temps ? \\[0pt]
Qu'est-ce que nous apprennent les animaux sur nous-même ? \\[0pt]
Qu'est-ce que parler ? \\[0pt]
Qu'est-ce que parler à-propos ? \\[0pt]
Qu'est-ce que « parler le même langage » ? \\[0pt]
Qu'est-ce que parler le même langage ? \\[0pt]
Qu'est-ce que parler pour ne rien dire ? \\[0pt]
Qu'est-ce que prendre conscience ? \\[0pt]
Qu'est-ce que produire ? \\[0pt]
Qu'est-ce que prouver ? \\[0pt]
Qu'est-ce que « rester soi-même » ? \\[0pt]
Qu'est-ce que se cultiver ? \\[0pt]
Qu'est-ce que traduire ? \\[0pt]
Qu'est-ce qu'être artiste ? \\[0pt]
Qu'est-ce qu'être en vie ? \\[0pt]
Qu'est-ce qu'être esclave ? \\[0pt]
Qu'est-ce qu'être inhumain ? \\[0pt]
Qu'est-ce qu'être l'auteur de son acte ? \\[0pt]
Qu'est-ce qu'être maître de soi ? \\[0pt]
Qu'est-ce qu'être malade ? \\[0pt]
Qu'est-ce qu'être matérialiste ? \\[0pt]
Qu'est-ce qu'être normal ? \\[0pt]
Qu'est-ce qu'être rationaliste ? \\[0pt]
Qu'est-ce qu'être réaliste ? \\[0pt]
Qu'est-ce qu'être spirituel ? \\[0pt]
Qu'est-ce que vérifier une théorie ? \\[0pt]
Qu'est-ce que vivre bien ? \\[0pt]
Qu'est-ce que vivre vertueusement ? \\[0pt]
Qu'est-ce que voir ? \\[0pt]
Qu'est-ce qu'exister ? \\[0pt]
Qu'est-ce qu'exister pour un individu ? \\[0pt]
Qu'est-ce qui distingue un argument d'une démonstration ? \\[0pt]
Qu'est-ce qui distingue un vivant d'une machine ? \\[0pt]
Qu'est-ce qui est absurde ? \\[0pt]
Qu'est-ce qui est intolérable ? \\[0pt]
Qu'est-ce qui est irrationnel ? \\[0pt]
Qu'est-ce qui est irréversible ? \\[0pt]
Qu'est-ce qui est naturel ? \\[0pt]
Qu'est-ce qui est nécessaire ? \\[0pt]
Qu'est-ce qui est possible ? \\[0pt]
Qu'est-ce qui est réel ? \\[0pt]
Qu'est-ce qui est respectable ? \\[0pt]
Qu'est-ce qui est scientifique ? \\[0pt]
Qu'est-ce qui est sublime ? \\[0pt]
Qu'est-ce qui est vital ? \\[0pt]
Qu'est-ce qui est vivant ? \\[0pt]
Qu'est-ce qui exige d'être interprété ? \\[0pt]
Qu'est-ce qui fait autorité ? \\[0pt]
Qu'est-ce qui fait changer les sociétés ? \\[0pt]
Qu'est-ce qui fait d'une activité un travail ? \\[0pt]
Qu'est-ce qui fait du scepticisme une philosophie ? \\[0pt]
Qu'est-ce qui fait la valeur de la technique ? \\[0pt]
Qu'est-ce qui fait la valeur d'une existence ? \\[0pt]
Qu'est-ce qui fait la valeur d'une œuvre d'art ? \\[0pt]
Qu'est-ce qui fait le pouvoir des mots ? \\[0pt]
Qu'est-ce qui fait l'unité d'une science ? \\[0pt]
Qu'est-ce qui fait l'unité d'un organisme ? \\[0pt]
Qu'est-ce qui fait l'unité du vivant ? \\[0pt]
Qu'est-ce qui fait qu'un corps est humain ? \\[0pt]
Qu'est-ce qui fait un peuple ? \\[0pt]
Qu'est-ce qui fonde le respect d'autrui ? \\[0pt]
Qu'est-ce qui importe ? \\[0pt]
Qu'est-ce qui m'appartient ? \\[0pt]
Qu'est-ce qui menace la liberté ? \\[0pt]
Qu'est-ce qui mesure la valeur d'un travail ? \\[0pt]
Qu'est-ce qui n'a pas d'histoire ? \\[0pt]
Qu'est-ce qui n'est pas technique dans l'activité humaine ? \\[0pt]
Qu'est-ce qui nous échappe dans le temps ? \\[0pt]
Qu'est-ce qu'interpréter ? \\[0pt]
Qu'est-ce qu'interpréter une œuvre d'art ? \\[0pt]
Qu'est-ce qu'interpréter un texte ? \\[0pt]
Qu'est-ce qui oppose les hommes ? \\[0pt]
Qu'est-ce qui peut se transformer ? \\[0pt]
Qu'est-ce qui rapproche le vivant de la machine ? \\[0pt]
Qu'est-ce qui rend une parole poétique ? \\[0pt]
Qu'est-ce qu'on ne peut comprendre ? \\[0pt]
Qu'est-ce qu'un abus de langage ? \\[0pt]
Qu'est-ce qu'un acte libre ? \\[0pt]
Qu'est-ce qu'un alter ego ? \\[0pt]
Qu'est-ce qu'un ami ? \\[0pt]
Qu'est-ce qu'un animal ? \\[0pt]
Qu'est-ce qu'un art abstrait ? \\[0pt]
Qu'est-ce qu'un art populaire ? \\[0pt]
Qu'est-ce qu'un bon argument ? \\[0pt]
Qu'est-ce qu'un bon citoyen ? \\[0pt]
Qu'est-ce qu'un cas de conscience ? \\[0pt]
Qu'est-ce qu'un chef-d'œuvre ? \\[0pt]
Qu'est-ce qu'un choix éclairé ? \\[0pt]
Qu'est-ce qu'un choix rationnel ? \\[0pt]
Qu'est-ce qu'un citoyen libre ? \\[0pt]
Qu'est-ce qu'un classique ? \\[0pt]
Qu'est-ce qu'un concept ? \\[0pt]
Qu'est-ce qu'un conflit de devoirs ? \\[0pt]
Qu'est-ce qu'un consommateur ? \\[0pt]
Qu'est-ce qu'un corps politique ? \\[0pt]
Qu'est-ce qu'un désir satisfait ? \\[0pt]
Qu'est-ce qu'un dieu ? \\[0pt]
Qu'est-ce qu'un doute raisonnable ? \\[0pt]
Qu'est-ce qu'un droit naturel ? \\[0pt]
Qu'est-ce qu'une action juste ? \\[0pt]
Qu'est-ce qu'une action politique ? \\[0pt]
Qu'est-ce qu'une autorité légitime ? \\[0pt]
Qu'est-ce qu'une belle action ? \\[0pt]
Qu'est-ce qu'une belle forme ? \\[0pt]
Qu'est-ce qu'une bonne délibération ? \\[0pt]
Qu'est-ce qu'une bonne traduction ? \\[0pt]
Qu'est-ce qu'un échange juste ? \\[0pt]
Qu'est-ce qu'un échange réussi ? \\[0pt]
Qu'est-ce qu'une chose matérielle ? \\[0pt]
Qu'est-ce qu'une communauté ? \\[0pt]
Qu'est-ce qu'une connaissance fiable ? \\[0pt]
Qu'est-ce qu'une constitution ? \\[0pt]
Qu'est-ce qu'une crise ? \\[0pt]
Qu'est-ce qu'une décision rationnelle ? \\[0pt]
Qu'est-ce qu'une encyclopédie ? \\[0pt]
Qu'est-ce qu'une erreur ? \\[0pt]
Qu'est-ce qu'une expérience scientifique ? \\[0pt]
Qu'est-ce qu'une fausse science ? \\[0pt]
Qu'est-ce qu'une faute de goût ? \\[0pt]
Qu'est-ce qu'une fiction ? \\[0pt]
Qu'est-ce qu'une hypothèse scientifique ? \\[0pt]
Qu'est-ce qu'une idée fausse ? \\[0pt]
Qu'est-ce qu'une image ? \\[0pt]
Qu'est-ce qu'une injustice ? \\[0pt]
Qu'est-ce qu'une institution ? \\[0pt]
Qu'est-ce qu'une langue artificielle ? \\[0pt]
Qu'est-ce qu'une liberté fondamentale ? \\[0pt]
Qu'est-ce qu'une libre interprétation ? \\[0pt]
Qu'est-ce qu'une mauvaise idée ? \\[0pt]
Qu'est-ce qu'une méthode ? \\[0pt]
Qu'est-ce qu'une nation ? \\[0pt]
Qu'est-ce qu'un ennemi politique ? \\[0pt]
Qu'est-ce qu'une occasion ? \\[0pt]
Qu'est-ce qu'une œuvre d'art réaliste ? \\[0pt]
Qu'est-ce qu'une œuvre ratée ? \\[0pt]
Qu'est-ce qu'une parole juste ? \\[0pt]
Qu'est-ce qu'une parole vraie ? \\[0pt]
Qu'est-ce qu'une pensée libre ? \\[0pt]
Qu'est-ce qu'une perception sensible ? \\[0pt]
Qu'est-ce qu'une philosophie de la vie ? \\[0pt]
Qu'est-ce qu'une preuve ? \\[0pt]
Qu'est-ce qu'une pseudoscience ? \\[0pt]
Qu'est-ce qu'une question métaphysique ? \\[0pt]
Qu'est-ce qu'une république ? \\[0pt]
Qu'est-ce qu'une révolution ? \\[0pt]
Qu'est-ce qu'une révolution scientifique ? \\[0pt]
Qu'est-ce qu'une science expérimentale ? \\[0pt]
Qu'est-ce qu'une société juste ? \\[0pt]
Qu'est-ce qu'une solution ? \\[0pt]
Qu'est-ce qu'un esprit juste ? \\[0pt]
Qu'est-ce qu'un esprit libre ? \\[0pt]
Qu'est-ce qu'un État de droit ? \\[0pt]
Qu'est-ce qu'un État libre ? \\[0pt]
Qu'est-ce qu'une théorie ? \\[0pt]
Qu'est-ce qu'une théorie scientifique ? \\[0pt]
Qu'est-ce qu'une tradition ? \\[0pt]
Qu'est-ce qu'un événement ? \\[0pt]
Qu'est-ce qu'un événement fondateur ? \\[0pt]
Qu'est-ce qu'un événement historique ? \\[0pt]
Qu'est-ce qu'une vérité contingente ? \\[0pt]
Qu'est-ce qu'une vérité historique ? \\[0pt]
Qu'est-ce qu'une vérité subjective ? \\[0pt]
Qu'est-ce qu'une vie heureuse ? \\[0pt]
Qu'est-ce qu'une vie humaine ? \\[0pt]
Qu'est-ce qu'un exemple ? \\[0pt]
Qu'est-ce qu'un expérimentateur ? \\[0pt]
Qu'est-ce qu'un fait ? \\[0pt]
Qu'est-ce qu'un fait culturel ? \\[0pt]
Qu'est-ce qu'un fait de culture ? \\[0pt]
Qu'est-ce qu'un fait historique ? \\[0pt]
Qu'est-ce qu'un fait religieux ? \\[0pt]
Qu'est-ce qu'un fait scientifique ? \\[0pt]
Qu'est-ce qu'un faux ? \\[0pt]
Qu'est-ce qu'un faux problème ? \\[0pt]
Qu'est-ce qu'un gouvernement démocratique ? \\[0pt]
Qu'est-ce qu'un homme d'action ? \\[0pt]
Qu'est-ce qu'un homme d'État ? \\[0pt]
Qu'est-ce qu'un homme d'expérience ? \\[0pt]
Qu'est-ce qu'un homme juste ? \\[0pt]
Qu'est-ce qu'un homme méchant ? \\[0pt]
Qu'est-ce qu'un homme politique ? \\[0pt]
Qu'est-ce qu'un impératif technique ? \\[0pt]
Qu'est-ce qu'un juste ? \\[0pt]
Qu'est-ce qu'un juste salaire ? \\[0pt]
Qu'est-ce qu'un justicier ? \\[0pt]
Qu'est-ce qu'un maître ? \\[0pt]
Qu'est-ce qu'un modèle ? \\[0pt]
Qu'est-ce qu'un moment ? \\[0pt]
Qu'est-ce qu'un monstre ? \\[0pt]
Qu'est-ce qu'un musée ? \\[0pt]
Qu'est-ce qu'un mythe ? \\[0pt]
Qu'est-ce qu'un objet de science ? \\[0pt]
Qu'est-ce qu'un objet technique ? \\[0pt]
Qu'est-ce qu'un organisme ? \\[0pt]
Qu'est-ce qu'un outil ? \\[0pt]
Qu'est-ce qu'un paradoxe ? \\[0pt]
Qu'est-ce qu'un pauvre ? \\[0pt]
Qu'est-ce qu'un peuple ? \\[0pt]
Qu'est-ce qu'un principe ? \\[0pt]
Qu'est-ce qu'un problème ? \\[0pt]
Qu'est-ce qu'un problème scientifique ? \\[0pt]
Qu'est-ce qu'un problème technique ? \\[0pt]
Qu'est-ce qu'un progrès technique ? \\[0pt]
Qu'est-ce qu'un public ? \\[0pt]
Qu'est-ce qu'un pur esprit ? \\[0pt]
Qu'est-ce qu'un récit véridique ? \\[0pt]
Qu'est-ce qu'un savoir-faire ? \\[0pt]
Qu'est-ce qu'un style ? \\[0pt]
Qu'est-ce qu'un tableau ? \\[0pt]
Qu'est-ce qu'un tabou ? \\[0pt]
Qu'est-ce qu'un technicien ? \\[0pt]
Qu'est-ce qu'un témoin ? \\[0pt]
Qu'est-ce qu'un tyran ? \\[0pt]
Que valent les mots ? \\[0pt]
Que valent les théories ? \\[0pt]
Que vaut la définition de l'homme comme animal doué de raison ? \\[0pt]
Que vaut le conseil : « vivez avec votre temps » ? \\[0pt]
Que vaut l'excuse : c'est plus fort que moi ? \\[0pt]
Que vaut une parole ? \\[0pt]
Que vaut une preuve contre un préjugé ? \\[0pt]
Que veut dire : être athée ? \\[0pt]
Que veut dire : « être cultivé » ? \\[0pt]
Que veut dire : « le temps passe » ? \\[0pt]
Que voulons-nous vraiment savoir ? \\[0pt]
Qui accroît son savoir accroît sa douleur \\[0pt]
Qui a le droit de punir ? \\[0pt]
Qui commande ? \\[0pt]
Qui croire ? \\[0pt]
Qui doit faire les lois ? \\[0pt]
Qui est digne du bonheur ? \\[0pt]
Qui est le peuple ? \\[0pt]
Qui est libre ? \\[0pt]
Qui est mon prochain ? \\[0pt]
Qui est mon semblable ? \\[0pt]
Qui est riche ? \\[0pt]
Qui est sage ? \\[0pt]
Qui fait la loi ? \\[0pt]
Qui gouverne ? \\[0pt]
Qui me dit ce que je dois faire ? \\[0pt]
Qui nous dicte nos devoirs ? \\[0pt]
Qui parle ? \\[0pt]
Qui parle quand je dis « je » ? \\[0pt]
Qui peut avoir des droits ? \\[0pt]
Qui peut me dire « tu ne dois pas » ? \\[0pt]
Qui peut prétendre énoncer des devoirs ? \\[0pt]
Qui peut prétendre imposer des bornes à la technique ? \\[0pt]
Qui peut se passer de religion ? \\[0pt]
Qui travaille ? \\[0pt]
Qu'y a-t-il à craindre de la technique ? \\[0pt]
Qu'y a-t-il de sacré ? \\[0pt]
Qu'y a-t-il que la nature fait en vain ? \\[0pt]
Qu'y a-t-il sinon le réel ? \\[0pt]
Raison et calcul \\[0pt]
Raison et démonstration \\[0pt]
Raison et dialogue \\[0pt]
Raison et folie \\[0pt]
Raison et fondement \\[0pt]
Raison et langage \\[0pt]
Raison et religion sont-elles incompatibles ? \\[0pt]
Raison et tradition \\[0pt]
Raisonnable et rationnel \\[0pt]
Raisonner \\[0pt]
Rationalité et irrationalité du travail \\[0pt]
Réalité et apparence \\[0pt]
Réalité et perception \\[0pt]
Réalité et représentation \\[0pt]
Rechercher la vérité, est-ce renoncer à toute opinion ? \\[0pt]
Récit et histoire \\[0pt]
Reconnaître la vérité \\[0pt]
Recourir au langage, est-ce renoncer à la violence ? \\[0pt]
Réfuter \\[0pt]
Regarder et voir \\[0pt]
Règles sociales et loi morale \\[0pt]
Regrets et remords \\[0pt]
Religion et démocratie \\[0pt]
Religion et moralité \\[0pt]
Religion et politique \\[0pt]
Religion et raison \\[0pt]
Religion et superstition \\[0pt]
Religion et vérité \\[0pt]
Religions et démocratie \\[0pt]
Rendre justice \\[0pt]
Répondre de soi \\[0pt]
Représentation et expression \\[0pt]
Représenter \\[0pt]
République et démocratie \\[0pt]
Résistance et obéissance \\[0pt]
Résister peut-il être un droit ? \\[0pt]
Respecter autrui, est-ce l'aimer ? \\[0pt]
Respecter la nature, est-ce renoncer à l'exploiter ? \\[0pt]
Respecter la nature, est-ce renoncer à s'en emparer ? \\[0pt]
Respect et tolérance \\[0pt]
Rester soi-même \\[0pt]
Retenons-nous le temps par le souvenir ? \\[0pt]
Réussir sa vie \\[0pt]
Rêver \\[0pt]
Revient-il à l'État d'assurer le bonheur des citoyens ? \\[0pt]
Révolte et révolution \\[0pt]
Rhétorique et vérité \\[0pt]
Richesse et pauvreté \\[0pt]
« Rien de ce qui est humain ne m'est étranger » \\[0pt]
« Rien de nouveau sous le soleil » \\[0pt]
Rire \\[0pt]
Sait-on ce que l'on veut ? \\[0pt]
Sait-on ce qu'on fait ? \\[0pt]
Sait-on nécessairement ce que l'on désire ? \\[0pt]
Sait-on toujours ce qu'on veut ? \\[0pt]
Sait-on vivre au présent ? \\[0pt]
S'amuser \\[0pt]
Sans justice, pas de liberté ? \\[0pt]
Sans référence à la beauté, peut-on rendre compte de l'art ? \\[0pt]
Savoir, est-ce cesser de croire ? \\[0pt]
Savoir, est-ce pouvoir ? \\[0pt]
Savoir et croire \\[0pt]
Savoir et démontrer \\[0pt]
Savoir et pouvoir \\[0pt]
Savoir et savoir faire \\[0pt]
Savoir par cœur \\[0pt]
Savons-nous ce que nous disons ? \\[0pt]
Science du vivant et finalisme \\[0pt]
Science du vivant, science de l'inerte \\[0pt]
Science et croyance \\[0pt]
Science et imagination \\[0pt]
Science et métaphysique \\[0pt]
Science et méthode \\[0pt]
Science et mythe \\[0pt]
Science et religion \\[0pt]
Science et sagesse \\[0pt]
Sciences et vérité \\[0pt]
Se chercher soi-même, est-ce devenir un autre ? \\[0pt]
Se cultiver \\[0pt]
Se cultiver, est-ce s'affranchir de son appartenance culturelle ? \\[0pt]
Sécurité et liberté \\[0pt]
Se décider \\[0pt]
Se faire comprendre \\[0pt]
Se force-t-on à faire son devoir ? \\[0pt]
S'émanciper \\[0pt]
Se mentir à soi-même : est-ce possible ? \\[0pt]
S'engager, est-ce perdre sa liberté ? \\[0pt]
Se nourrir \\[0pt]
Sensation et perception \\[0pt]
Sens et existence \\[0pt]
Sens et non-sens \\[0pt]
Sens et signification \\[0pt]
Sens propre et sens figuré \\[0pt]
Sentiment et justice sont-ils compatibles ? \\[0pt]
Sentir et juger \\[0pt]
Sentir et penser \\[0pt]
Sentir et percevoir \\[0pt]
Serions-nous heureux dans un ordre politique parfait ? \\[0pt]
Serions-nous plus libres sans État ? \\[0pt]
Servir, est-ce nécessairement renoncer à sa liberté ? \\[0pt]
Se suffire à soi-même \\[0pt]
« Se trouver » a-t-il un sens ? \\[0pt]
Seuls les humains sont-ils libres ? \\[0pt]
S'exprimer \\[0pt]
Signe et symbole \\[0pt]
Sincérité et vérité \\[0pt]
Si nous étions moraux, le droit serait-il inutile ? \\[0pt]
Si tout est historique, tout est-il relatif ? \\[0pt]
Société et communauté \\[0pt]
Société et contrat \\[0pt]
Société et organisme \\[0pt]
Société humaines, sociétés animales \\[0pt]
« Sois naturel » : est-ce un bon conseil ? \\[0pt]
« Sois toi-même ! » : un impératif absurde ? \\[0pt]
Solitude et liberté \\[0pt]
Sommes-nous adaptés au monde de la technique ? \\[0pt]
Sommes-nous dans le temps comme dans l'espace ? \\[0pt]
Sommes-nous des sujets ? \\[0pt]
Sommes-nous déterminés par notre culture ? \\[0pt]
Sommes-nous égaux devant le bonheur ? \\[0pt]
Sommes-nous faits pour être heureux ? \\[0pt]
Sommes-nous faits pour le bonheur ? \\[0pt]
Sommes-nous jamais certains d'avoir choisi librement ? \\[0pt]
Sommes-nous les jouets de l'histoire ? \\[0pt]
Sommes-nous libres face à l'évidence ? \\[0pt]
Sommes-nous libres lorsque les actes émanent de notre personne ? \\[0pt]
Sommes-nous maîtres de nos désirs ? \\[0pt]
Sommes-nous maîtres de nos paroles ? \\[0pt]
Sommes-nous maîtres de nos pensées ? \\[0pt]
Sommes-nous menacés par les progrès techniques ? \\[0pt]
Sommes-nous portés au bien ? \\[0pt]
Sommes-nous prisonniers de nos désirs ? \\[0pt]
Sommes-nous prisonniers de notre histoire ? \\[0pt]
Sommes-nous prisonniers du temps ? \\[0pt]
Sommes-nous responsables de ce dont nous n'avons pas conscience ? \\[0pt]
Sommes-nous responsables de nos désirs ? \\[0pt]
Sommes-nous responsables de nos opinions ? \\[0pt]
Sommes-nous séparés du réel par les représentations que nous nous en faisons ? \\[0pt]
Sommes-nous sujets de nos désirs ? \\[0pt]
Sommes-nous toujours conscients des causes de nos désirs ? \\[0pt]
Sommes-nous vulnérables toute notre vie ? \\[0pt]
S'opposer \\[0pt]
S'opposer à l'autorité, est-ce toujours une marque de liberté ? \\[0pt]
Soumission et servitude \\[0pt]
Subissons-nous la langue que nous parlons ? \\[0pt]
Substance et accident \\[0pt]
Suffit-il d'avoir raison ? \\[0pt]
Suffit-il de bien juger pour bien faire ? \\[0pt]
Suffit-il de bien penser pour bien agir ? \\[0pt]
Suffit-il de communiquer pour dialoguer ? \\[0pt]
Suffit-il de démontrer pour convaincre ? \\[0pt]
Suffit-il de faire son devoir ? \\[0pt]
Suffit-il de faire son devoir pour être vertueux ? \\[0pt]
Suffit-il de n'avoir rien fait pour être innocent ? \\[0pt]
Suffit-il de se parler pour éviter la discorde ? \\[0pt]
Suffit-il de tenir ses promesses pour faire son devoir ? \\[0pt]
Suffit-il d'être soi-même pour être libre ? \\[0pt]
Suffit-il d'être vertueux pour être heureux ? \\[0pt]
Suffit-il de vivre pour exister ? \\[0pt]
Suffit-il de voir le meilleur pour le suivre ? \\[0pt]
Suffit-il de voir pour savoir ? \\[0pt]
Suffit-il, pour être juste, d'obéir aux lois et aux coutumes de son pays ? \\[0pt]
Suffit-il que nos intentions soient bonnes pour que nos actions le soient aussi ? \\[0pt]
Suis-ce que j'ai conscience d'être ? \\[0pt]
Suis-je ce que j'ai conscience d'être ? \\[0pt]
Suis-je ce que je fais ? \\[0pt]
Suis-je ce que les autres font de moi ? \\[0pt]
Suis-je ce que mon passé a fait de moi ? \\[0pt]
Suis-je dans le temps comme je suis dans l'espace ? \\[0pt]
Suis-je étranger à moi-même ? \\[0pt]
Suis-je l'auteur de ce que je dis ? \\[0pt]
Suis-je le mieux placé pour me connaître ? \\[0pt]
Suis-je le mieux placé pour savoir qui je suis ? \\[0pt]
Suis-je libre ? \\[0pt]
Suis-je mon corps ? \\[0pt]
Suis-je mon passé ? \\[0pt]
Suis-je propriétaire de mon corps ? \\[0pt]
Suis-je responsable de ce dont je n'ai pas conscience ? \\[0pt]
Suis-je responsable de ce que je suis ? \\[0pt]
Suis-je toujours autre que moi-même ? \\[0pt]
Suis-je toujours le même ? \\[0pt]
Suivre son intuition \\[0pt]
Superstition et fanatisme sont-ils inhérents à la religion ? \\[0pt]
Surface et profondeur \\[0pt]
Sur quoi fonder la justice ? \\[0pt]
Sur quoi fonder l'autorité ? \\[0pt]
Sur quoi fonder l'autorité politique ? \\[0pt]
Sur quoi fonder le devoir ? \\[0pt]
Sur quoi fonder le droit de punir ? \\[0pt]
Sur quoi le langage doit-il se régler ? \\[0pt]
Sur quoi peut-on fonder la certitude d'avoir raison ? \\[0pt]
Sur quoi repose la croyance au progrès ? \\[0pt]
Sur quoi sont fondées les mathématiques ? \\[0pt]
Survivre \\[0pt]
Suspendre son jugement \\[0pt]
Sympathie et respect \\[0pt]
Talent et génie \\[0pt]
Technique et idéologie \\[0pt]
Technique et liberté \\[0pt]
Technique et nature \\[0pt]
Technique et progrès \\[0pt]
Technique et savoir-faire \\[0pt]
Technique et violence \\[0pt]
Témoignage et vérité \\[0pt]
Temps et commencement \\[0pt]
Temps et création \\[0pt]
Temps et histoire \\[0pt]
Temps et irréversibilité \\[0pt]
Temps et liberté \\[0pt]
Temps et mémoire \\[0pt]
Temps et vérité \\[0pt]
Temps physique et temps biologique \\[0pt]
Théorie et expérience \\[0pt]
Tolérance et laïcité \\[0pt]
Toucher, sentir, goûter \\[0pt]
Tous les conflits peuvent-ils être résolus par le dialogue ? \\[0pt]
Tous les hommes désirent-ils être heureux ? \\[0pt]
Tous les hommes désirent-ils naturellement être heureux ? \\[0pt]
Tous les hommes désirent-ils naturellement savoir ? \\[0pt]
Tous les paradis sont-ils perdus ? \\[0pt]
Tous les problèmes peuvent-ils recevoir une solution technique ? \\[0pt]
Tous les rapports humains sont-ils des échanges ? \\[0pt]
Tout a-t-il une raison d'être ? \\[0pt]
Tout ce qui est naturel est-il normal ? \\[0pt]
Tout ce qui est rationnel est-il raisonnable ? \\[0pt]
Tout ce qui est vrai doit-il être prouvé ? \\[0pt]
Tout change-t-il avec le temps ? \\[0pt]
Tout démontrer \\[0pt]
Tout désir est-il désir de posséder ? \\[0pt]
Tout désir est-il désir de quelque chose ? \\[0pt]
Tout désir est-il égoïste ? \\[0pt]
Tout désir est-il manque ? \\[0pt]
Tout désir est-il une souffrance ? \\[0pt]
Tout dire \\[0pt]
Tout droit est-il un pouvoir ? \\[0pt]
Toute compréhension implique-t-elle une interprétation ? \\[0pt]
Toute connaissance est-elle hypothétique ? \\[0pt]
Toute connaissance s'enracine-t-elle dans la perception ? \\[0pt]
Toute conscience est-elle conscience de quelque chose ? \\[0pt]
Toute conscience est-elle subjective ? \\[0pt]
Toute description est-elle une interprétation ? \\[0pt]
Toute faute est-elle une erreur ? \\[0pt]
Toute inégalité est-elle injuste ? \\[0pt]
Toute interprétation est-elle contestable ? \\[0pt]
Toute interprétation est-elle subjective ? \\[0pt]
Toute morale est-elle rationnelle ? \\[0pt]
Toute morale s'oppose-t-elle aux désirs ? \\[0pt]
Tout énoncé admet-il une interprétation ? \\[0pt]
Toute œuvre d'art exige-t-elle une interprétation ? \\[0pt]
Toute passion fait-elle souffrir ? \\[0pt]
Toute pensée revêt-elle une forme linguistique ? \\[0pt]
Toute polémique est-elle stérile ? \\[0pt]
Toute relation humaine est-elle un échange ? \\[0pt]
Toute religion a-t-elle sa vérité ? \\[0pt]
Toutes les croyances se valent-elles ? \\[0pt]
Toutes les fautes se valent-elles ? \\[0pt]
Toutes les inégalités sont-elles des injustices ? \\[0pt]
Toutes les interprétations se valent-elles ? \\[0pt]
Toute société a-t-elle besoin d'une religion ? \\[0pt]
Tout est-il historique ? \\[0pt]
Tout est-il matière ? \\[0pt]
Tout est-il vraiment permis, si Dieu n'existe pas ? \\[0pt]
« Tout est relatif » \\[0pt]
Tout événement a-t-il une cause ? \\[0pt]
Toute vérité doit-elle être dite ? \\[0pt]
Toute vérité est-elle démontrable ? \\[0pt]
Toute vérité est-elle nécessaire ? \\[0pt]
Toute vérité est-elle une erreur rectifiée ? \\[0pt]
Toute vérité se laisse-t-elle démontrer ? \\[0pt]
Toute vie est-elle intrinsèquement respectable ? \\[0pt]
Tout exercice du pouvoir politique implique-t-il l'usage de la violence ? \\[0pt]
Tout futur est-il contingent ? \\[0pt]
Tout ordre est-il rationnel ? \\[0pt]
Tout ordre est-il une violence déguisée ? \\[0pt]
Tout passe-t-il avec le temps ? \\[0pt]
Tout peut-il avoir une valeur marchande ? \\[0pt]
Tout peut-il avoir valeur marchande ? \\[0pt]
Tout peut-il être objet d'échange ? \\[0pt]
Tout peut-il être objet de science ? \\[0pt]
Tout peut-il s'acheter ? \\[0pt]
Tout peut-il se démontrer ? \\[0pt]
Tout savoir est-il pouvoir ? \\[0pt]
Tout savoir est-il transmissible ? \\[0pt]
Tout s'en va-t-il avec le temps ? \\[0pt]
Tout se prête-il à la mesure ? \\[0pt]
Tout texte est-il à interpréter ? \\[0pt]
Tout travail a-t-il un sens ? \\[0pt]
Tout travail est-il forcé ? \\[0pt]
Tout travail est-il social ? \\[0pt]
Tout vouloir \\[0pt]
Tradition et liberté \\[0pt]
Tradition et nouveauté \\[0pt]
Tradition et transmission \\[0pt]
« Tradition n'est pas raison » \\[0pt]
Traduire, est-ce trahir ? \\[0pt]
Traiter des faits humains comme des choses, est-ce considérer l'homme comme une chose ? \\[0pt]
Traiter les faits humains comme des choses, est-ce réduire les hommes à des choses ? \\[0pt]
Transcendance et immanence \\[0pt]
Transmettre \\[0pt]
Travail, besoin, désir \\[0pt]
Travail et aliénation \\[0pt]
Travail et besoin \\[0pt]
Travail et bonheur \\[0pt]
Travail et capital \\[0pt]
Travail et liberté \\[0pt]
Travail et loisir \\[0pt]
Travail et nécessité \\[0pt]
Travail et œuvre \\[0pt]
Travail et propriété \\[0pt]
Travailler, est-ce faire œuvre ? \\[0pt]
Travailler, est-ce lutter contre soi-même ? \\[0pt]
Travailler, est-ce seulement produire ? \\[0pt]
Travailler et œuvrer \\[0pt]
Travaille-t-on pour soi-même ? \\[0pt]
Travail manuel et travail intellectuel \\[0pt]
Travail manuel, travail intellectuel \\[0pt]
Travail servile et travail salarié \\[0pt]
Trouver et prouver \\[0pt]
« Tu dois, donc tu peux » \\[0pt]
Tyrannie et dictature \\[0pt]
Un acte gratuit est-il possible ? \\[0pt]
Un acte libre est-il un acte imprévisible ? \\[0pt]
Un acte peut-il être inhumain ? \\[0pt]
Un artiste doit-il être original ? \\[0pt]
Un bien peut-il être commun ? \\[0pt]
Un chef-d'œuvre est-il immortel ? \\[0pt]
Un choix peut-il être rationnel ? \\[0pt]
Un choix rationnel est-il un choix libre ? \\[0pt]
Un désir peut-il être coupable ? \\[0pt]
Un désir peut-il être inconscient ? \\[0pt]
Un devoir admet-il des exceptions ? \\[0pt]
Un devoir peut-il être absolu ? \\[0pt]
Une activité inutile est-elle sans valeur ? \\[0pt]
Une communauté politique n'est-elle qu'une communauté d'intérêt ? \\[0pt]
Une communauté politique n'est-elle qu'une communauté d'intérêts ? \\[0pt]
Une connaissance peut-elle ne pas être relative ? \\[0pt]
Une connaissance scientifique du vivant est-elle possible ? \\[0pt]
Une croyance peut-elle être libre ? \\[0pt]
Une croyance peut-elle être rationnelle ? \\[0pt]
Une culture peut-elle être porteuse de valeurs universelles ? \\[0pt]
Une destruction peut-elle être créatrice ? \\[0pt]
Une durée peut-elle être éternelle ? \\[0pt]
Une éducation morale est-elle possible ? \\[0pt]
Une expérience peut-elle être fictive ? \\[0pt]
Une fausse science est-elle une science qui commet des erreurs ? \\[0pt]
Une fiction peut-elle être vraie ? \\[0pt]
Une idée peut-elle être générale ? \\[0pt]
Une imitation peut-elle être parfaite ? \\[0pt]
Une interprétation est-elle nécessairement subjective ? \\[0pt]
Une interprétation peut-elle échapper à l'arbitraire ? \\[0pt]
Une interprétation peut-elle être définitive ? \\[0pt]
Une interprétation peut-elle être fausse ? \\[0pt]
Une interprétation peut-elle être objective ? \\[0pt]
Une interprétation peut-elle prétendre à la vérité ? \\[0pt]
Une langue n'est-elle faite que de mots ? \\[0pt]
Une loi peut-elle être injuste ? \\[0pt]
Une machine n'est-elle qu'un outil perfectionné ? \\[0pt]
Une machine peut-elle penser ? \\[0pt]
Une matière sans forme est-elle concevable ? \\[0pt]
Une morale peut-elle être permissive ? \\[0pt]
Une morale sans devoirs est-elle possible ? \\[0pt]
Une morale sans obligation est-elle possible ? \\[0pt]
Une morale sceptique est-elle possible ? \\[0pt]
Une œuvre d'art a-t-elle toujours un sens ? \\[0pt]
Une œuvre d'art doit-elle nécessairement être belle ? \\[0pt]
Une œuvre d'art doit-elle plaire ? \\[0pt]
Une œuvre d'art peut-elle être immorale ? \\[0pt]
Une pensée contradictoire est-elle dénuée de valeur ? \\[0pt]
Une perception peut-elle être illusoire ? \\[0pt]
Une psychologie peut-elle être matérialiste ? \\[0pt]
Une religion peut-elle être rationnelle ? \\[0pt]
Une religion peut-elle être universelle ? \\[0pt]
Une religion peut-elle prétendre à la vérité ? \\[0pt]
Une religion peut-elle se passer de pratiques ? \\[0pt]
Une science de l'esprit est-elle possible ? \\[0pt]
Une science peut-elle se passer d'expérimentation ? \\[0pt]
Une sensation peut-elle être fausse ? \\[0pt]
Une société basée sur le don pourrait-elle être prospère ? \\[0pt]
Une société juste, est-ce une société sans conflit ? \\[0pt]
Une société n'est-elle qu'un ensemble d'individus ? \\[0pt]
Une société peut-elle être juste ? \\[0pt]
Une société peut-elle se passer de religion ? \\[0pt]
Une société sans conflit est-elle pensable ? \\[0pt]
Une société sans conflits est-elle souhaitable ? \\[0pt]
Une société sans État est-elle possible ? \\[0pt]
Une société sans religion est-elle possible ? \\[0pt]
Une société sans travail est-elle souhaitable ? \\[0pt]
Une technique ne se réduit-elle pas toujours à une forme de bricolage ? \\[0pt]
Une théorie peut-elle être vérifiée ? \\[0pt]
Un être vivant peut-il être comparé à une œuvre d'art ? \\[0pt]
Un événement historique est-il toujours imprévisible ? \\[0pt]
Une vérité approchée est-elle pour autant approximative ? \\[0pt]
Une vérité peut-elle être approximative ? \\[0pt]
Une vérité peut-elle être indicible ? \\[0pt]
Une vérité peut-elle être provisoire ? \\[0pt]
Une vie heureuse est-elle une vie de plaisirs ? \\[0pt]
Une vie juste est-elle une bonne vie ? \\[0pt]
Une vie libre exclut-elle le travail ? \\[0pt]
Un fait existe-t-il sans interprétation ? \\[0pt]
Un gouvernement de savants est-il souhaitable ? \\[0pt]
Un grand bonheur \\[0pt]
« Un instant d'éternité » \\[0pt]
Un mensonge peut-il avoir une valeur morale ? \\[0pt]
Un monde meilleur \\[0pt]
Un monde sans guerre serait-il un monde sans histoire ? \\[0pt]
Un monde sans travail est-il souhaitable ? \\[0pt]
Un peuple en paix n'a-t-il plus d'histoire ? \\[0pt]
Un peuple est-il responsable de son histoire ? \\[0pt]
Un peuple est-il un rassemblement d'individus ? \\[0pt]
Un peuple se définit-il par son histoire ? \\[0pt]
Un plaisir peut-il être désintéressé ? \\[0pt]
Un problème moral peut-il recevoir une solution certaine ? \\[0pt]
Un savoir peut-il être inconscient ? \\[0pt]
Un savoir scientifique sur l'homme est-il compatible avec l'idée de liberté ? \\[0pt]
User de violence peut-il être moral ? \\[0pt]
Utilité et beauté \\[0pt]
Vaincre la mort \\[0pt]
Vaut-il mieux subir ou commettre l'injustice ? \\[0pt]
Vérité et apparence \\[0pt]
Vérité et certitude \\[0pt]
Vérité et démonstration \\[0pt]
Vérité et efficacité \\[0pt]
Vérité et exactitude \\[0pt]
Vérité et illusion \\[0pt]
Vérité et liberté \\[0pt]
Vérité et réalité \\[0pt]
Vérité et religion \\[0pt]
Vérité et sincérité \\[0pt]
Vérité et validité \\[0pt]
Vérité et vérification \\[0pt]
Vérité et vraisemblance \\[0pt]
Vérité théorique, vérité pratique \\[0pt]
Veut-on toujours savoir ? \\[0pt]
Vice et délice \\[0pt]
Vie et finalité \\[0pt]
Vie politique et vie contemplative \\[0pt]
Vie privée et vie publique \\[0pt]
Vie publique et vie privée \\[0pt]
Violence et force \\[0pt]
Violence et pouvoir \\[0pt]
« Vis caché » \\[0pt]
Vivons-nous au présent ? \\[0pt]
Vivrait-on mieux sans désirs ? \\[0pt]
Vivre au présent \\[0pt]
Vivre en société, est-ce seulement vivre ensemble ? \\[0pt]
Vivre, est-ce interpréter ? \\[0pt]
Vivre, est-ce lutter pour survivre ? \\[0pt]
Vivre, est-ce résister à la mort ? \\[0pt]
Vivre et exister \\[0pt]
Vivre et exister, est-ce la même chose ? \\[0pt]
Vivre libre \\[0pt]
Vivre sa vie \\[0pt]
Vivre ses désirs \\[0pt]
Voir et entendre \\[0pt]
Voir et observer \\[0pt]
Voir et savoir \\[0pt]
Voir et toucher \\[0pt]
Voir le meilleur et faire le pire \\[0pt]
Voit-on ce qu'on croit ? \\[0pt]
Volonté et désir \\[0pt]
Vouloir dire \\[0pt]
Vouloir et pouvoir \\[0pt]
Vouloir être heureux \\[0pt]
Vouloir la paix sociale peut-il aller jusqu'à accepter l'injustice ? \\[0pt]
Vouloir la solitude \\[0pt]
Vouloir oublier \\[0pt]
Vouloir vivre \\[0pt]
Y a-t-il à la question « qu'est-ce que l'homme ? » une réponse scientifique ? \\[0pt]
Y a-t-il d'autres moyens que la démonstration pour établir la vérité ? \\[0pt]
Y a-t-il de bons et de mauvais désirs ? \\[0pt]
Y a-t-il de bons préjugés ? \\[0pt]
Y a-t-il de justes inégalités ? \\[0pt]
Y a-t-il de la fatalité dans la vie de l'homme ? \\[0pt]
Y a-t-il de la raison dans la perception ? \\[0pt]
Y a-t-il de l'impensable ? \\[0pt]
Y a-t-il de l'inconnaissable ? \\[0pt]
Y a-t-il de l'indémontrable ? \\[0pt]
Y a-t-il de l'indésirable ? \\[0pt]
Y a-t-il de mauvais désirs ? \\[0pt]
Y a-t-il des actions libres ? \\[0pt]
Y a-t-il des arts mineurs ? \\[0pt]
Y a-t-il des biens inestimables ? \\[0pt]
Y a-t-il des certitudes morales ? \\[0pt]
Y a-t-il des choses dont on ne peut parler ? \\[0pt]
Y a-t-il des choses qu'on n'échange pas ? \\[0pt]
Y a-t-il des connaissances dangereuses ? \\[0pt]
Y a-t-il des contraintes légitimes ? \\[0pt]
Y a-t-il des convictions philosophiques ? \\[0pt]
Y a-t-il des correspondances entre les arts ? \\[0pt]
Y a-t-il des critères de l'humanité ? \\[0pt]
Y a-t-il des croyances nécessaires ? \\[0pt]
Y a-t-il des croyances rationnelles ? \\[0pt]
Y a-t-il des degrés de conscience ? \\[0pt]
Y a-t-il des degrés de vérité ? \\[0pt]
Y a-t-il des démonstrations en philosophie ? \\[0pt]
Y a-t-il des désirs moraux ? \\[0pt]
Y a-t-il des désordres naturels ? \\[0pt]
Y a-t-il des devoirs envers soi ? \\[0pt]
Y a-t-il des erreurs de la nature ? \\[0pt]
Y a-t-il des évidences morales ? \\[0pt]
Y a-t-il des expériences sans théorie ? \\[0pt]
Y a-t-il des faits religieux ? \\[0pt]
Y a-t-il des faits scientifiques ? \\[0pt]
Y a-t-il des fins de la nature ? \\[0pt]
Y a-t-il des guerres injustes ? \\[0pt]
Y a-t-il des guerres justes ? \\[0pt]
Y a-t-il des illusions de la conscience ? \\[0pt]
Y a-t-il des illusions de la liberté ? \\[0pt]
Y a-t-il des inégalités justes ? \\[0pt]
Y a-t-il des injustices naturelles ? \\[0pt]
Y a-t-il des leçons de l'histoire ? \\[0pt]
Y a-t-il des liens qui libèrent ? \\[0pt]
Y a-t-il des limites à la connaissance ? \\[0pt]
Y a-t-il des limites à la tolérance ? \\[0pt]
Y a-t-il des limites au pouvoir de la technique ? \\[0pt]
Y a-t-il des mondes imaginaires ? \\[0pt]
Y a-t-il des mots vides de sens ? \\[0pt]
Y a-t-il des normes naturelles ? \\[0pt]
Y a-t-il des objets qui n'existent pas ? \\[0pt]
Y a-t-il des obstacles à la connaissance du vivant ? \\[0pt]
Y a-t-il des perceptions insensibles ? \\[0pt]
Y a-t-il des peuples sans histoire ? \\[0pt]
Y a-t-il des plaisirs meilleurs que d'autres ? \\[0pt]
Y a-t-il des preuves de la liberté ? \\[0pt]
Y a-t-il des principes de justice universels ? \\[0pt]
Y a-t-il des progrès dans l'art ? \\[0pt]
Y a-t-il des progrès en art ? \\[0pt]
Y a-t-il des questions sans réponse ? \\[0pt]
Y a-t-il des raisons de douter de la raison ? \\[0pt]
Y a-t-il des révolutions dans l'art ? \\[0pt]
Y a-t-il des révolutions dans les sciences ? \\[0pt]
Y a-t-il des révolutions scientifiques ? \\[0pt]
Y a-t-il des secrets de la nature ? \\[0pt]
Y a-t-il des solutions en politique ? \\[0pt]
Y a-t-il des sots métiers ? \\[0pt]
Y a-t-il des techniques de pensée ? \\[0pt]
Y a-t-il des techniques pour être heureux ? \\[0pt]
Y a-t-il des tyrannies douces ? \\[0pt]
Y a-t-il des valeurs absolues ? \\[0pt]
Y a-t-il des valeurs propres à la science ? \\[0pt]
Y a-t-il des vérités de fait ? \\[0pt]
Y a-t-il des vérités définitives ? \\[0pt]
Y a-t-il des vérités en art ? \\[0pt]
Y a-t-il des vérités en politique ? \\[0pt]
Y a-t-il des vérités éternelles ? \\[0pt]
Y a-t-il des vérités indémontrables ? \\[0pt]
Y a-t-il des vérités indiscutables ? \\[0pt]
Y a-t-il des vérités métaphysiques ? \\[0pt]
Y a-t-il des vérités morales ? \\[0pt]
Y a-t-il des vérités plus importantes que d'autres ? \\[0pt]
Y a-t-il des vérités qui échappent à la raison ? \\[0pt]
Y a-t-il des vérités religieuses ? \\[0pt]
Y a-t-il des vérités sans preuve ? \\[0pt]
Y a-t-il différentes façons d'exister ? \\[0pt]
Y a-t-il du non-être ? \\[0pt]
Y a-t-il du nouveau dans l'histoire ? \\[0pt]
Y a-t-il incompatibilité entre science et religion ? \\[0pt]
Y a-t-il nécessairement du religieux dans l'art ? \\[0pt]
Y a-t-il plusieurs sens du mot vie ? \\[0pt]
Y a-t-il plusieurs sortes de matières ? \\[0pt]
Y a-t-il plusieurs temps ? \\[0pt]
Y a-t-il plusieurs vérités ? \\[0pt]
Y a-t-il progrès en art ? \\[0pt]
Y a-t-il quoi que ce soit de nouveau dans l'histoire ? \\[0pt]
Y a-t-il sens à vouloir justifier le mal ? \\[0pt]
Y a-t-il un art d'être heureux ? \\[0pt]
Y a-t-il un art de vivre ? \\[0pt]
Y a-t-il un art d'interpréter ? \\[0pt]
Y a-t-il un art du bonheur ? \\[0pt]
Y a-t-il un auteur de l'histoire ? \\[0pt]
Y a-t-il un Bien suprême ? \\[0pt]
Y a-t-il un bonheur sans illusion ? \\[0pt]
Y a-t-il un bon usage du temps ? \\[0pt]
Y a-t-il un devoir de désobéissance ? \\[0pt]
Y a-t-il un devoir d'émancipation ? \\[0pt]
Y a-t-il un devoir de mémoire ? \\[0pt]
Y a-t-il un devoir d'être heureux ? \\[0pt]
Y a-t-il un devoir de vérité ? \\[0pt]
Y a-t-il un droit à la propriété ? \\[0pt]
Y a-t-il un droit au bonheur ? \\[0pt]
Y a-t-il un droit au travail ? \\[0pt]
Y a-t-il un droit de désobéissance ? \\[0pt]
Y a-t-il un droit de mentir ? \\[0pt]
Y a-t-il un droit de révolte ? \\[0pt]
Y a-t-il un droit des peuples ? \\[0pt]
Y a-t-il un droit du plus fort ? \\[0pt]
Y a-t-il un droit naturel ? \\[0pt]
Y a-t-il une beauté propre à l'objet technique ? \\[0pt]
Y a-t-il une bonne imitation ? \\[0pt]
Y a-t-il une causalité en histoire ? \\[0pt]
Y a-t-il une compétence politique ? \\[0pt]
Y a-t-il une condition humaine ? \\[0pt]
Y a-t-il une connaissance du singulier ? \\[0pt]
Y a-t-il une conscience collective ? \\[0pt]
Y a-t-il une définition du bonheur ? \\[0pt]
Y a-t-il une dimension métaphysique du désir ? \\[0pt]
Y a-t-il une enfance de l'humanité ? \\[0pt]
Y a-t-il une éthique de la technique ? \\[0pt]
Y a-t-il une expérience de la liberté ? \\[0pt]
Y a-t-il une expérience de l'éternité ? \\[0pt]
Y a-t-il une expérience du temps ? \\[0pt]
Y a-t-il une finalité dans la nature ? \\[0pt]
Y a-t-il une fonction propre à l'œuvre d'art ? \\[0pt]
Y a-t-il une force des faibles ? \\[0pt]
Y a-t-il une force du droit ? \\[0pt]
Y a-t-il une hiérarchie des devoirs ? \\[0pt]
Y a-t-il une hiérarchie du vivant ? \\[0pt]
Y a-t-il une histoire de la raison ? \\[0pt]
Y a-t-il une histoire de la vérité ? \\[0pt]
Y a-t-il une histoire universelle ? \\[0pt]
Y a-t-il une irréversibilité du temps ? \\[0pt]
Y a-t-il une justice naturelle ? \\[0pt]
Y a-t-il une limite à la connaissance du vivant ? \\[0pt]
Y a-t-il une limite au désir ? \\[0pt]
Y a-t-il une logique dans l'histoire ? \\[0pt]
Y a-t-il une logique des événements historiques ? \\[0pt]
Y a-t-il une logique du désir ? \\[0pt]
Y a-t-il une méthode de l'interprétation ? \\[0pt]
Y a-t-il une morale universelle ? \\[0pt]
Y a-t-il une nature humaine ? \\[0pt]
Y a-t-il une nécessité de l'erreur ? \\[0pt]
Y a-t-il une nécessité morale ? \\[0pt]
Y a-t-il une œuvre du temps ? \\[0pt]
Y a-t-il une ou des morales ? \\[0pt]
Y a-t-il une pensée technique ? \\[0pt]
Y a-t-il une primauté du devoir sur le droit ? \\[0pt]
Y a-t-il une rationalité de la contingence ? \\[0pt]
Y a-t-il une rationalité du hasard ? \\[0pt]
Y a-t-il une réalité du hasard ? \\[0pt]
Y a-t-il une réalité du mal ? \\[0pt]
Y a-t-il une responsabilité de l'artiste ? \\[0pt]
Y a-t-il une sagesse de l'inconscient ? \\[0pt]
Y a-t-il une science du juste ? \\[0pt]
Y a-t-il une science ou des sciences ? \\[0pt]
Y a-t-il une science politique ? \\[0pt]
Y a-t-il une servitude volontaire ? \\[0pt]
Y a-t-il une spécificité du vivant ? \\[0pt]
Y a-t-il un État idéal ? \\[0pt]
Y a-t-il une technique pour tout ? \\[0pt]
Y a-t-il une tyrannie du vrai ? \\[0pt]
Y a-t-il une unité des devoirs ? \\[0pt]
Y a-t-il une unité des sciences ? \\[0pt]
Y a-t-il une valeur de l'inutile ? \\[0pt]
Y a-t-il une vérité de l'œuvre d'art ? \\[0pt]
Y a-t-il une vérité des apparences ? \\[0pt]
Y a-t-il une vérité des représentations ? \\[0pt]
Y a-t-il une vérité des sentiments ? \\[0pt]
Y a-t-il une vérité en art ? \\[0pt]
Y a-t-il une vérité en histoire ? \\[0pt]
Y a-t-il une vérité sur le bonheur ? \\[0pt]
Y a-t-il une vertu de l'imitation ? \\[0pt]
Y a-t-il une violence du droit ? \\[0pt]
Y a-t-il une vision du monde propre à la pensée technique ? \\[0pt]
Y a-t-il une volonté du mal ? \\[0pt]
Y a-t-il un fondement de la croyance ? \\[0pt]
Y a-t-il un intérêt à faire son devoir ? \\[0pt]
Y a-t-il un jugement de l'histoire ? \\[0pt]
Y a-t-il un langage des animaux ? \\[0pt]
Y a-t-il un langage du corps ? \\[0pt]
Y a-t-il un moteur de l'histoire ? \\[0pt]
Y a-t-il un objet du désir ? \\[0pt]
Y a-t-il un ordre dans la nature ? \\[0pt]
Y a-t-il un ordre des choses ? \\[0pt]
Y a-t-il un ordre du monde ? \\[0pt]
Y a-t-il un pouvoir de la technique ? \\[0pt]
Y a-t-il un primat de la nature sur la culture ? \\[0pt]
Y a-t-il un progrès du droit ? \\[0pt]
Y a-t-il un progrès en art ? \\[0pt]
Y a-t-il un propre de l'homme ? \\[0pt]
Y a-t-il un rapport moral à soi-même ? \\[0pt]
Y a-t-il un savoir de la justice ? \\[0pt]
Y a-t-il un savoir du juste ? \\[0pt]
Y a-t-il un sens à instituer une langue universelle ? \\[0pt]
Y a-t-il un sens à ne plus rien désirer ? \\[0pt]
Y a-t-il un sens à s'opposer à la technique ? \\[0pt]
Y a-t-il un sens moral ? \\[0pt]
Y a-t-il un temps pour tout ? \\[0pt]
Y a-t-il vérité sans interprétation ? \\[0pt]
Y aura-t-il toujours des religions ? \\[0pt]


\subsection{CAPES externe}
\label{sec:org90a3b9c}

\noindent
Abstraire, est-ce se couper du réel ? \\[0pt]
« À chacun sa morale » \\[0pt]
À chacun sa vérité \\[0pt]
À chacun selon son mérite \\[0pt]
Action et contemplation \\[0pt]
Activité et passivité \\[0pt]
Agir en politique, est-ce agir dans l'incertain ? \\[0pt]
Agir et faire \\[0pt]
Agir et réagir \\[0pt]
Agir volontairement, est-ce nécessairement savoir ce que l'on fait ? \\[0pt]
Ai-je des devoirs envers moi-même ? \\[0pt]
Ai-je un corps ou suis-je mon corps ? \\[0pt]
Aimer le monde tel qu'il est. \\[0pt]
Aimer l'éphémère \\[0pt]
Aimer peut-il être un devoir ? \\[0pt]
« Aimer » se dit-il en un seul sens ? \\[0pt]
Ami et ennemi \\[0pt]
Amour et inconscient \\[0pt]
Analyse et synthèse \\[0pt]
Analyser \\[0pt]
Apparence et réalité \\[0pt]
Apprend-on à percevoir ? \\[0pt]
Apprendre à vivre \\[0pt]
Apprendre à voir \\[0pt]
Apprendre à voir ? \\[0pt]
Apprendre et enseigner \\[0pt]
À quelle condition un travail est-il humain ? \\[0pt]
À quelles conditions le vivant peut-il être objet de science ? \\[0pt]
À quelles conditions une activité est-elle un travail ? \\[0pt]
À quelles conditions une démarche est-elle scientifique ? \\[0pt]
À quelles conditions une expérience est-elle possible ? \\[0pt]
À quelles conditions une hypothèse est-elle scientifique ? \\[0pt]
À quelles conditions un État peut-il être juste ? \\[0pt]
À qui doit-on le respect ? \\[0pt]
À qui doit-on obéir ? \\[0pt]
À qui faut-il obéir ? \\[0pt]
À qui la faute ? \\[0pt]
À quoi bon démontrer ? \\[0pt]
À quoi bon imiter la nature ? \\[0pt]
À quoi bon la science ? \\[0pt]
À quoi la perception donne-t-elle accès ? \\[0pt]
À quoi nos illusions tiennent-elles ? \\[0pt]
À quoi peut-on reconnaître la vérité ? \\[0pt]
À quoi peut-on reconnaître une œuvre d'art ? \\[0pt]
À quoi reconnaît-on la vérité ? \\[0pt]
À quoi reconnaît-on le faux ? \\[0pt]
À quoi reconnaît-on le réel ? \\[0pt]
À quoi reconnaît-on l'humanité en chaque homme ? \\[0pt]
À quoi reconnaît-on qu'une activité est un travail ? \\[0pt]
À quoi reconnaît-on qu'une expérience est scientifique ? \\[0pt]
À quoi reconnaît-on qu'une pensée est vraie ? \\[0pt]
À quoi reconnaît-on qu'une science est une science ? \\[0pt]
À quoi reconnaît-on un acte libre ? \\[0pt]
À quoi reconnaît-on un bon artisan ? \\[0pt]
À quoi reconnaît-on un devoir ? \\[0pt]
À quoi reconnaît-on une action morale ? \\[0pt]
À quoi reconnaît-on une attitude religieuse ? \\[0pt]
À quoi reconnaît-on une bonne interprétation ? \\[0pt]
À quoi reconnaît-on une idéologie ? \\[0pt]
À quoi reconnaît-on une œuvre d'art ? \\[0pt]
À quoi reconnaît-on une théorie scientifique ? \\[0pt]
À quoi sert la notion d'état de nature ? \\[0pt]
À quoi sert la technique ? \\[0pt]
À quoi sert la vérité ? \\[0pt]
À quoi sert un critique d'art ? \\[0pt]
À quoi servent les images ? \\[0pt]
À quoi servent les lois ? \\[0pt]
À quoi servent les machines ? \\[0pt]
À quoi servent les œuvres d'art ? \\[0pt]
À quoi servent les preuves ? \\[0pt]
À quoi servent les preuves de l'existence de Dieu ? \\[0pt]
À quoi servent les symboles ? \\[0pt]
À quoi servent les théories ? \\[0pt]
À quoi servent les utopies ? \\[0pt]
À quoi servent les voyages ? \\[0pt]
À quoi tient la force des religions ? \\[0pt]
À quoi tient la valeur d'une pensée ? \\[0pt]
À quoi tient le pouvoir des mots ? \\[0pt]
À quoi tient le pouvoir des sciences ? \\[0pt]
À quoi tient notre humanité ? \\[0pt]
Argent et liberté \\[0pt]
Argumenter et démontrer \\[0pt]
Art et beauté \\[0pt]
Art et création \\[0pt]
Art et illusion \\[0pt]
Art et imagination \\[0pt]
Art et jeu \\[0pt]
Art et matière \\[0pt]
Art et pouvoir \\[0pt]
Art et représentation \\[0pt]
Art et société \\[0pt]
Art et Société \\[0pt]
Art et symbole \\[0pt]
Art et technique \\[0pt]
Art et vérité \\[0pt]
Aspect et apparence \\[0pt]
A-t-on besoin de certitudes ? \\[0pt]
A-t-on besoin d'experts ? \\[0pt]
A-t-on besoin d'un chef ? \\[0pt]
A-t-on des devoirs envers l'humanité ? \\[0pt]
A-t-on des devoirs envers soi-même ? \\[0pt]
A-t-on le droit de se désintéresser de la politique ? \\[0pt]
A-t-on le droit de se révolter ? \\[0pt]
Au nom de qui rend-on justice ? \\[0pt]
Au nom de quoi rend-on justice ? \\[0pt]
Autorité et domination \\[0pt]
Autorité et souveraineté \\[0pt]
Autrui \\[0pt]
Autrui, est-ce n'importe quel autre ? \\[0pt]
Autrui est-il pour moi un mystère ? \\[0pt]
Autrui est-il un autre moi-même ? \\[0pt]
Autrui m'est-il étranger ? \\[0pt]
Avoir de l'expérience \\[0pt]
Avoir du jugement \\[0pt]
Avoir du métier \\[0pt]
Avoir du pouvoir \\[0pt]
Avoir et être \\[0pt]
Avoir la parole, est-ce avoir le pouvoir ? \\[0pt]
Avoir mauvaise conscience \\[0pt]
Avoir raison \\[0pt]
Avoir raison, est-ce nécessairement être raisonnable ? \\[0pt]
Avoir tout pour être heureux \\[0pt]
Avoir un corps \\[0pt]
Avoir un destin \\[0pt]
Avoir un passé, avoir un avenir \\[0pt]
Avons-nous besoin de cérémonies ? \\[0pt]
Avons-nous besoin de héros ? \\[0pt]
Avons-nous besoin de maîtres ? \\[0pt]
Avons-nous besoin de nous tromper nous-même ? \\[0pt]
Avons-nous des devoirs à l'égard de la vérité ? \\[0pt]
Avons-nous des devoirs envers la nature ? \\[0pt]
Avons-nous des devoirs envers nous-mêmes ? \\[0pt]
Avons-nous des droits sur la nature ? \\[0pt]
Avons-nous intérêt à la liberté d'autrui ? \\[0pt]
Avons-nous le devoir de vivre ? \\[0pt]
Avons-nous peur de la liberté ? \\[0pt]
Avons-nous raison de croire ? \\[0pt]
Avons-nous un devoir de vérité ? \\[0pt]
Avons-nous une identité ? \\[0pt]
Bêtise et méchanceté \\[0pt]
Bien agir, est-ce toujours être moral ? \\[0pt]
Bien commun et intérêt particulier \\[0pt]
Bonheur et bien-être \\[0pt]
Bonheur et plaisir \\[0pt]
Bonheur et satisfaction \\[0pt]
Bonheur et technique \\[0pt]
Bonheur et vertu \\[0pt]
Calculer et penser \\[0pt]
Cause et condition \\[0pt]
Cause et effet \\[0pt]
Cause et loi \\[0pt]
Cause et raison \\[0pt]
Ce que je pense est-il nécessairement vrai ? \\[0pt]
Ce qui est ordinaire est-il normal ? \\[0pt]
Ce qui est vrai est-il toujours vérifiable ? \\[0pt]
Ce qui ne peut s'acheter est-il dépourvu de valeur ? \\[0pt]
Certitude et conviction \\[0pt]
« C'est plus fort que moi » \\[0pt]
Changer, est-ce devenir un autre ? \\[0pt]
Changer le monde \\[0pt]
Change-t-on avec le temps ? \\[0pt]
Châtier, est ce faire honneur au criminel ? \\[0pt]
Choisir \\[0pt]
Choisir, est-ce renoncer ? \\[0pt]
Choisissons-nous qui nous sommes ? \\[0pt]
Choisit-on ses souvenirs ? \\[0pt]
Choix et raison \\[0pt]
Chose et objet \\[0pt]
Chose et personne \\[0pt]
Choses naturelles, choses fabriquées \\[0pt]
Civilisation et barbarie \\[0pt]
Classer \\[0pt]
Colère et indignation \\[0pt]
Commander \\[0pt]
Comment autrui peut-il m'aider à rechercher le bonheur ? \\[0pt]
Comment chercher ce qu'on ignore ? \\[0pt]
Comment comprendre les faits sociaux ? \\[0pt]
Comment connaître l'inconscient ? \\[0pt]
Comment connaître nos devoirs ? \\[0pt]
Comment dire la vérité ? \\[0pt]
Comment distinguer culture et civilisation ? \\[0pt]
Comment distinguer le rêvé du perçu ? \\[0pt]
Comment juger de la justesse d'une interprétation ? \\[0pt]
Comment juger une œuvre d'art ? \\[0pt]
Comment la volonté peut-elle être faible ? \\[0pt]
Comment le passé nous est-il présent ? \\[0pt]
Comment le passé peut-il demeurer présent ? \\[0pt]
Comment l'erreur est-elle possible ? \\[0pt]
Comment les mathématiques peuvent-elles s'appliquer au réel ? \\[0pt]
Comment l'inconscient peut-il se manifester ? \\[0pt]
Comment penser le hasard ? \\[0pt]
Comment penser l'éternel ? \\[0pt]
Comment peut-on être sceptique ? \\[0pt]
Comment peut-on savoir que l'on a raison ? \\[0pt]
Comment puis-je devenir ce que je suis ? \\[0pt]
Comment rendre la justice ? \\[0pt]
Communauté et société \\[0pt]
Comprendre \\[0pt]
Comprendre, est-ce interpréter ? \\[0pt]
Comprendre le réel, est-ce le dominer ? \\[0pt]
Comprendre une démonstration \\[0pt]
Comprendre une œuvre d'art \\[0pt]
Concept et métaphore \\[0pt]
Concurrence et égalité \\[0pt]
Connaissance de soi et conscience de soi \\[0pt]
Connaissance et perception \\[0pt]
Connaissons-nous la réalité des choses ? \\[0pt]
« Connais-toi toi-même » \\[0pt]
Connaît-on la vie ou le vivant ? \\[0pt]
Connaît-on les choses telles qu'elles sont ? \\[0pt]
Connaître, est-ce découvrir le réel ? \\[0pt]
Connaître, est-ce dépasser les apparences ? \\[0pt]
Connaître la vie ou le vivant ? \\[0pt]
Connaître une chose, est-ce en connaître la cause ? \\[0pt]
Conquérir \\[0pt]
Conscience de soi et amour de soi \\[0pt]
Conscience et attention \\[0pt]
Conscience et connaissance \\[0pt]
Conscience et conscience de soi \\[0pt]
Conscience et responsabilité \\[0pt]
Conscience et subjectivité \\[0pt]
Conscience et volonté \\[0pt]
Construire la vérité \\[0pt]
Contempler \\[0pt]
Contrainte et obligation \\[0pt]
Convaincre et persuader \\[0pt]
Convient-il d'opposer explication et interprétation ? \\[0pt]
Corps et espace \\[0pt]
Corps et matière \\[0pt]
Corps et nature \\[0pt]
Crainte et espoir \\[0pt]
Création et production \\[0pt]
Créer et produire \\[0pt]
Critiquer \\[0pt]
Croire, est-ce renoncer à l'usage de la raison ? \\[0pt]
Croire, est-ce renoncer à savoir ? \\[0pt]
Croire, est-ce renoncer au savoir ? \\[0pt]
Croire et savoir \\[0pt]
Croire que Dieu existe, est-ce croire en lui ? \\[0pt]
Croyance et confiance \\[0pt]
Croyance et vérité \\[0pt]
Culpabilité et responsabilité \\[0pt]
Culture et artifice \\[0pt]
Culture et civilisation \\[0pt]
Culture et différence \\[0pt]
Culture et éducation \\[0pt]
Culture et identité \\[0pt]
Culture et langage \\[0pt]
Culture et savoir \\[0pt]
Culture et technique \\[0pt]
Culture et violence \\[0pt]
Dans l'action, est-ce l'intention qui compte ? \\[0pt]
Dans quel but les hommes se donnent-ils des lois ? \\[0pt]
Dans quelle mesure est-on l'auteur de sa propre vie ? \\[0pt]
Dans quelle mesure la technique nous libère-t-elle de la nature ? \\[0pt]
Dans quelle mesure les sciences ne sont-elles pas à l'abri de l'erreur ? \\[0pt]
Dans quelle mesure toute philosophie est-elle critique du langage ? \\[0pt]
« Dans un bois aussi courbe que celui dont l'homme est fait on ne peut rien tailler de tout à fait droit » \\[0pt]
Débat et démocratie \\[0pt]
Débattre et dialoguer \\[0pt]
Déchiffrer \\[0pt]
Décider \\[0pt]
Découverte et justification \\[0pt]
Déduction et démonstration \\[0pt]
Délibérer \\[0pt]
Dématérialiser \\[0pt]
Démocratie et opinion \\[0pt]
Démocratie et représentation \\[0pt]
Démocratie et technocratie \\[0pt]
Démonstration et intuition \\[0pt]
Démontrer et argumenter \\[0pt]
Démontrer et expérimenter \\[0pt]
Démontrer par l'absurde \\[0pt]
Démontre-t-on ailleurs qu'en mathématiques ? \\[0pt]
Dépend-il de nous d'être heureux ? \\[0pt]
De quel droit l'État exerce-t-il un pouvoir ? \\[0pt]
De quel droit punit-on ? \\[0pt]
De quel homme parlent les sciences humaines ? \\[0pt]
De quelle vérité l'art est-il capable ? \\[0pt]
De quoi avons-nous vraiment besoin ? \\[0pt]
De quoi dépend notre bonheur ? \\[0pt]
De quoi est fait mon présent ? \\[0pt]
De quoi est fait notre présent ? \\[0pt]
De quoi la philosophie est-elle le désir ? \\[0pt]
De quoi la religion sauve-t-elle ? \\[0pt]
De quoi la vérité libère-t-elle ? \\[0pt]
De quoi le devoir libère-t-il ? \\[0pt]
De quoi les logiciens parlent-ils ? \\[0pt]
De quoi l'inconscient peut-il être la cause ? \\[0pt]
De quoi parlent les mathématiques ? \\[0pt]
De quoi peut-on être certain ? \\[0pt]
De quoi peut-on être inconscient ? \\[0pt]
De quoi peut-on faire l'expérience ? \\[0pt]
De quoi pouvons-nous être sûrs ? \\[0pt]
De quoi puis-je répondre ? \\[0pt]
De quoi sommes-nous responsables ? \\[0pt]
De quoi suis-je inconscient ? \\[0pt]
De quoi suis-je responsable ? \\[0pt]
De quoi une œuvre d'art nous instruit-elle ? \\[0pt]
De quoi y a-t-il histoire ? \\[0pt]
Désirer, est-ce être aliéné ? \\[0pt]
Désir et besoin \\[0pt]
Désir et bonheur \\[0pt]
Désir et interdit \\[0pt]
Désir et langage \\[0pt]
Désir et manque \\[0pt]
Désir et ordre \\[0pt]
Désir et plaisir \\[0pt]
Désir et pouvoir \\[0pt]
Désir et raison \\[0pt]
Désir et réalité \\[0pt]
Désir et volonté \\[0pt]
Des lois justes suffisent-elles à assurer la justice ? \\[0pt]
Devant qui sommes-nous responsables ? \\[0pt]
Devenir humain \\[0pt]
Devoir et bonheur \\[0pt]
Devoir et contrainte \\[0pt]
Devoir et intérêt \\[0pt]
Devoir et liberté \\[0pt]
Devoir et prudence \\[0pt]
Devoir et vertu \\[0pt]
Devoirs et passions \\[0pt]
Devons-nous dire la vérité ? \\[0pt]
Devons-nous espérer vivre sans travailler ? \\[0pt]
Devons-nous être heureux ? \\[0pt]
Dialoguer et s'entendre \\[0pt]
Dieu est-il une hypothèse dont le savant et le philosophe peuvent se passer ? \\[0pt]
Dire, est-ce autre chose que vouloir dire ? \\[0pt]
Dire, est-ce faire ? \\[0pt]
Dire et exprimer \\[0pt]
Dire et faire \\[0pt]
Dire je \\[0pt]
Dogme et opinion \\[0pt]
Dois-je tenir compte de ce que font les autres pour orienter ma conduite ? \\[0pt]
Doit-on apprendre à devenir soi-même ? \\[0pt]
Doit-on apprendre à percevoir ? \\[0pt]
Doit-on apprendre à vivre ? \\[0pt]
Doit-on bien juger pour bien faire ? \\[0pt]
Doit-on changer ses désirs, plutôt que l'ordre du monde ? \\[0pt]
Doit-on identifier l'âme à la conscience ? \\[0pt]
Doit-on interpréter les rêves ? \\[0pt]
Doit-on le respect au vivant ? \\[0pt]
Doit-on mûrir pour la liberté ? \\[0pt]
Doit-on rechercher le bonheur ? \\[0pt]
Doit-on refuser d'interpréter ? \\[0pt]
Doit-on se passer des utopies ? \\[0pt]
Doit-on tenir le plaisir pour une fin ? \\[0pt]
Doit-on toujours dire la vérité ? \\[0pt]
Don et échange \\[0pt]
Donner, à quoi bon ? \\[0pt]
Donner et recevoir \\[0pt]
Donner l'exemple ? \\[0pt]
Donner sa parole \\[0pt]
Doute et raison \\[0pt]
D'où vient la certitude ? \\[0pt]
D'où vient la servitude ? \\[0pt]
Droit et coutume \\[0pt]
Droit et devoir \\[0pt]
Droit et interprétation \\[0pt]
Droit et morale \\[0pt]
Droit et violence \\[0pt]
« Droits de\ldots{} » et « droits à\ldots{} » \\[0pt]
Droits de l'homme ou droits du citoyen ? \\[0pt]
Droits et devoirs \\[0pt]
« Du passé, faisons table rase » \\[0pt]
Durée et instant \\[0pt]
Durer \\[0pt]
Échange et don \\[0pt]
Échanger des idées \\[0pt]
Échanger, est-ce créer de la valeur ? \\[0pt]
Échanger, est-ce partager ? \\[0pt]
Écouter et entendre \\[0pt]
Écrire et parler \\[0pt]
Égalité et différence \\[0pt]
En morale, peut-on dire : « C'est l'intention qui compte » ? \\[0pt]
En quel sens l'État est-il rationnel ? \\[0pt]
En quel sens parler de lois de la pensée ? \\[0pt]
En quel sens parler d'identité culturelle ? \\[0pt]
En quel sens parler d'illusion religieuse ? \\[0pt]
En quel sens peut-on appeler les sciences humaines « sciences de l'esprit » ? \\[0pt]
En quel sens peut-on dire que la vérité s'impose ? \\[0pt]
En quel sens peut-on dire que l'homme est un animal politique ? \\[0pt]
En quel sens peut-on dire qu'« on expérimente avec sa raison » ? \\[0pt]
En quel sens peut-on parler de la mort de l'art ? \\[0pt]
En quel sens peut-on parler de responsabilité collective ? \\[0pt]
En quel sens peut-on parler d'une culture technique ? \\[0pt]
En quel sens peut-on parler d'une interprétation de la nature ? \\[0pt]
En quoi la croyance religieuse se distingue-t-elle des autres croyances ? \\[0pt]
En quoi la méthode est-elle un art de penser ? \\[0pt]
En quoi la question de Dieu est-elle philosophique ? \\[0pt]
En quoi l'art peut-il intéresser le philosophe ? \\[0pt]
En quoi le bien d'autrui m'importe-t-il ? \\[0pt]
En quoi les vivants témoignent-ils d'une histoire ? \\[0pt]
En quoi une culture peut-elle être la mienne ? \\[0pt]
Entre la sécurité de tous et la liberté de chacun, le politique doit-il choisir ? \\[0pt]
Entre l'opinion et la science, n'y a-t-il qu'une différence de degré ? \\[0pt]
Errer \\[0pt]
Erreur et faute \\[0pt]
Erreur et illusion \\[0pt]
Essence et existence \\[0pt]
Est-ce à la raison de déterminer ce qui est réel ? \\[0pt]
Est-ce à l'État de faire le bonheur du peuple ? \\[0pt]
Est-ce la majorité qui doit décider ? \\[0pt]
Est-ce l'autorité qui fait la loi ? \\[0pt]
Est-ce le cerveau qui pense ? \\[0pt]
Est-ce l'échange utilitaire qui fait le lien social ? \\[0pt]
Est-ce l'ignorance qui nous fait croire ? \\[0pt]
Est-ce l'ignorance qui rend les hommes croyants ? \\[0pt]
Est-ce l'intérêt qui fonde le lien social ? \\[0pt]
Est-ce un devoir d'aimer son prochain ? \\[0pt]
Est-il immoral de se rendre heureux ? \\[0pt]
Est-il juste d'interpréter la loi ? \\[0pt]
Est-il justifié de parler de « corps social » ? \\[0pt]
Est-il légitime d'opposer liberté et nécessité ? \\[0pt]
Est-il naturel à l'homme de parler ? \\[0pt]
Est-il possible de tout avoir pour être heureux ? \\[0pt]
Est-il possible d'être immoral sans le savoir ? \\[0pt]
Est-il raisonnable d'aimer ? \\[0pt]
Est-il raisonnable d'avoir des certitudes ? \\[0pt]
Est-il raisonnable de chercher à interpréter les rêves ? \\[0pt]
Est-il raisonnable d'être rationnel ? \\[0pt]
Est-il raisonnable de vouloir maîtriser la nature ? \\[0pt]
Est-il toujours possible de savoir ce que l'on doit faire ? \\[0pt]
Est-il vrai que les animaux ne pensent pas ? \\[0pt]
Est-il vrai que l'ignorant n'est pas libre ? \\[0pt]
Est-il vrai que ma liberté s'arrête là où commence celle des autres ? \\[0pt]
Est-il vrai que plus on échange, moins on se bat ? \\[0pt]
Estime et respect \\[0pt]
Estimer \\[0pt]
Est-on l'auteur de sa propre vie ? \\[0pt]
Est-on libre de travailler ? \\[0pt]
Est-on libre face à la vérité ? \\[0pt]
Est-on propriétaire de son corps ? \\[0pt]
Est-on responsable de ses croyances ? \\[0pt]
Est-on sociable par nature ? \\[0pt]
État et gouvernement \\[0pt]
État et institutions \\[0pt]
État et nation \\[0pt]
État et société \\[0pt]
État et Société \\[0pt]
État et société civile \\[0pt]
État et violence \\[0pt]
Éthique et Morale \\[0pt]
Être bon juge \\[0pt]
Être citoyen du monde \\[0pt]
Être cultivé rend-il meilleur ? \\[0pt]
Être de son temps \\[0pt]
Être dogmatique \\[0pt]
Être en paix \\[0pt]
Être et apparaître \\[0pt]
Être et avoir \\[0pt]
Être et avoir été \\[0pt]
Être et devenir \\[0pt]
Être et devoir être \\[0pt]
Être et devoir-être \\[0pt]
Être et exister \\[0pt]
Être et paraître \\[0pt]
Être exemplaire \\[0pt]
Être libre, est-ce dire non ? \\[0pt]
Être libre, est-ce échapper aux prévisions ? \\[0pt]
Être libre, est-ce faire ce que l'on veut ? \\[0pt]
Être libre, est-ce pouvoir choisir ? \\[0pt]
Être libre, est-ce se suffire à soi-même ? \\[0pt]
Être membre de L'État \\[0pt]
Être naturel \\[0pt]
Être réaliste \\[0pt]
Être sceptique \\[0pt]
Être soi-même \\[0pt]
« Être soi-même », cela a-t-il un sens ? \\[0pt]
Être spectateur \\[0pt]
Être un sujet, est-ce être maître de soi ? \\[0pt]
Être vertueux \\[0pt]
Évidence et raison \\[0pt]
Évidence et vérité \\[0pt]
Évidences et préjugés \\[0pt]
Évolution biologique et culture \\[0pt]
Évolution et progrès \\[0pt]
Excuser et pardonner \\[0pt]
Existence et contingence \\[0pt]
Existence et religion \\[0pt]
Exister, est-ce simplement vivre ? \\[0pt]
Existe-t-il des choses en soi ? \\[0pt]
Existe-t-il des choses réellement sublimes ? \\[0pt]
Existe-t-il des comportements contraires à la nature ? \\[0pt]
Existe-t-il des désirs coupables ? \\[0pt]
Existe-t-il des plaisirs purs ? \\[0pt]
Existe-t-il un art de la parole ? \\[0pt]
Existe-t-il un déterminisme social ? \\[0pt]
Existe-t-il une méthode pour rechercher la vérité ? \\[0pt]
Existe-t-il une méthode pour trouver la vérité ? \\[0pt]
Expérience et expérimentation \\[0pt]
Expliquer et comprendre \\[0pt]
Faire autorité \\[0pt]
Faire confiance \\[0pt]
Faire de nécessité vertu \\[0pt]
Faire la paix \\[0pt]
Faire le mal \\[0pt]
Faire l'histoire \\[0pt]
Faisons-nous l'Histoire ? \\[0pt]
Fait et fiction \\[0pt]
Fait et preuve \\[0pt]
Fait et valeur \\[0pt]
Faits et preuves \\[0pt]
Faudrait-il bannir la polysémie du langage ? \\[0pt]
Faudrait-il ne rien oublier ? \\[0pt]
Faudrait-il vivre sans passion ? \\[0pt]
Faut-avoir peur de la technique ? \\[0pt]
Faut-il accepter sa condition ? \\[0pt]
Faut-il affirmer son identité ? \\[0pt]
Faut-il aimer autrui pour le respecter ? \\[0pt]
Faut-il aimer son prochain comme soi-même ? \\[0pt]
Faut-il apprendre à être libre ? \\[0pt]
Faut-il apprendre à obéir ? \\[0pt]
Faut-il avoir peur des machines ? \\[0pt]
Faut-il avoir peur d'être libre ? \\[0pt]
Faut-il avoir peur du désordre ? \\[0pt]
Faut-il chercher à être heureux ? \\[0pt]
Faut-il chercher à se connaître ? \\[0pt]
Faut-il chercher à tout démontrer ? \\[0pt]
Faut-il chercher la paix à tout prix ? \\[0pt]
Faut-il chercher le bonheur à tout prix ? \\[0pt]
Faut-il chercher un sens à l'histoire ? \\[0pt]
Faut-il choisir entre être heureux et être libre ? \\[0pt]
Faut-il connaître la vérité pour pouvoir mentir ? \\[0pt]
Faut-il craindre la tyrannie du bonheur ? \\[0pt]
Faut-il craindre le développement des techniques ? \\[0pt]
Faut-il craindre les machines ? \\[0pt]
Faut-il craindre l'ordre ? \\[0pt]
Faut-il croire en la science ? \\[0pt]
Faut-il défendre les faibles ? \\[0pt]
Faut-il dire de la justice qu'elle n'existe pas ? \\[0pt]
Faut-il douter de ce qu'on ne peut pas démontrer ? \\[0pt]
Faut-il douter de tout ? \\[0pt]
Faut-il du passé faire table rase ? \\[0pt]
Faut-il espérer pour agir ? \\[0pt]
Faut-il être bon ? \\[0pt]
Faut-il être cohérent ? \\[0pt]
Faut-il être modéré ? \\[0pt]
Faut-il être pragmatique ? \\[0pt]
Faut-il faire confiance au progrès technique ? \\[0pt]
Faut-il faire de nécessité vertu ? \\[0pt]
Faut-il hiérarchiser les formes de vie ? \\[0pt]
Faut-il hiérarchiser ses désirs ? \\[0pt]
Faut-il libérer l'humanité du travail ? \\[0pt]
Faut-il limiter la souveraineté de l'État ? \\[0pt]
Faut-il mériter son bonheur ? \\[0pt]
Faut-il ne manquer de rien pour être heureux ? \\[0pt]
Faut-il opposer la matière et l'esprit ? \\[0pt]
Faut-il opposer la société et l'État ? \\[0pt]
Faut-il opposer la vie à la mort ? \\[0pt]
Faut-il opposer le devoir-être à l'être ? \\[0pt]
Faut-il opposer le don et l'échange ? \\[0pt]
Faut-il opposer le temps vécu et le temps des choses ? \\[0pt]
Faut-il opposer raison et sensation ? \\[0pt]
Faut-il opposer subjectivité et objectivité ? \\[0pt]
Faut-il parfois désobéir aux lois ? \\[0pt]
Faut-il parfois sacrifier la vérité ? \\[0pt]
Faut-il pour le connaître faire du vivant un objet ? \\[0pt]
Faut-il préférer l'art à la nature ? \\[0pt]
Faut-il regretter l'équivocité du langage ? \\[0pt]
Faut-il réguler la technique ? \\[0pt]
Faut-il rejeter tous les préjugés ? \\[0pt]
Faut-il rejeter toute norme ? \\[0pt]
Faut-il renoncer à faire du travail une valeur ? \\[0pt]
Faut-il renoncer à l'idée d'âme ? \\[0pt]
Faut-il respecter la nature ? \\[0pt]
Faut-il respecter la vie ? \\[0pt]
Faut-il rompre avec le passé ? \\[0pt]
Faut-il s'adapter ? \\[0pt]
Faut-il s'affranchir des désirs ? \\[0pt]
Faut-il s'aimer soi-même ? \\[0pt]
Faut-il savoir mentir ? \\[0pt]
Faut-il savoir pour agir ? \\[0pt]
Faut-il se cultiver ? \\[0pt]
Faut-il se détacher du monde ? \\[0pt]
Faut-il se fier aux apparences ? \\[0pt]
Faut-il se méfier de la technique ? \\[0pt]
Faut-il se méfier de l'intuition ? \\[0pt]
Faut-il s'en tenir aux faits ? \\[0pt]
Faut-il se rendre à l'évidence ? \\[0pt]
Faut-il se ressembler pour former une société ? \\[0pt]
Faut-il souhaiter la fin du travail ? \\[0pt]
Faut-il souhaiter une langue universelle ? \\[0pt]
Faut-il suivre la nature ? \\[0pt]
Faut-il toujours chercher à savoir ? \\[0pt]
Faut-il toujours éviter de se contredire ? \\[0pt]
Faut-il toujours faire son devoir ? \\[0pt]
Faut-il tout critiquer ? \\[0pt]
Faut-il un corps pour penser ? \\[0pt]
Faut-il une méthode pour découvrir la vérité ? \\[0pt]
Faut-il vivre avec son temps ? \\[0pt]
Faut-il vivre comme si nous ne devions jamais mourir ? \\[0pt]
Faut-il vivre comme si on ne devait jamais mourir ? \\[0pt]
Faut-il vivre hors de la société pour être heureux ? \\[0pt]
Faut-il vouloir changer le monde ? \\[0pt]
Faut-il vouloir être heureux ? \\[0pt]
Foi et raison \\[0pt]
Foi et savoir \\[0pt]
Foi et superstition \\[0pt]
Force et violence \\[0pt]
Forme et matière \\[0pt]
Former et éduquer \\[0pt]
Forme-t-on son esprit en transformant la matière ? \\[0pt]
Génération, fabrication, création \\[0pt]
Gouvernement des hommes et administration des choses \\[0pt]
Gouverner \\[0pt]
Gouverner, est-ce régner ? \\[0pt]
Guerre et politique \\[0pt]
Habiter \\[0pt]
Hasard et destin \\[0pt]
Hier a-t-il plus de réalité que demain ? \\[0pt]
Histoire et causalité \\[0pt]
Histoire et mémoire \\[0pt]
Histoire et violence \\[0pt]
Histoire individuelle et histoire collective \\[0pt]
Hypothèse et vérité \\[0pt]
Ici et maintenant \\[0pt]
Idéal et utopie \\[0pt]
Idée et concept \\[0pt]
Idée et réalité \\[0pt]
Identité et différence \\[0pt]
Illusion et hallucination \\[0pt]
« Il ne lui manque que la parole » \\[0pt]
« Il y a un temps pour tout » \\[0pt]
Image et concept \\[0pt]
Image et idée \\[0pt]
Image, fiction, illusion \\[0pt]
Imagination et culture \\[0pt]
Imagination et pouvoir \\[0pt]
Imitation et création \\[0pt]
Imitation et représentation \\[0pt]
Imiter et créer \\[0pt]
Inconscient et déterminisme \\[0pt]
Inconscient et inconscience \\[0pt]
Inconscient et instinct \\[0pt]
Inconscient et liberté \\[0pt]
Inconscient et mythes \\[0pt]
Indépendance et autonomie \\[0pt]
Indépendance et liberté \\[0pt]
Individu et citoyen \\[0pt]
Individu et communauté \\[0pt]
Individu et société \\[0pt]
Innocence et ignorance \\[0pt]
Instruction et éducation \\[0pt]
Instruire et éduquer \\[0pt]
Intérêt général et bien commun \\[0pt]
Interprétation et création \\[0pt]
Interprétation et perception \\[0pt]
Interpréter \\[0pt]
Interpréter, est-ce connaître ? \\[0pt]
Interpréter, est-ce renoncer à prouver ? \\[0pt]
Interpréter est-il subjectif ? \\[0pt]
Interpréter et comprendre \\[0pt]
Interpréter et traduire \\[0pt]
Interprète-t-on à défaut de connaître ? \\[0pt]
Interroger \\[0pt]
Intuition et déduction \\[0pt]
Inventer et découvrir \\[0pt]
Invention, création, découverte \\[0pt]
Invention et découverte \\[0pt]
« Je ne l'ai pas fait exprès » \\[0pt]
Jugement et réflexion \\[0pt]
Jugement et vérité \\[0pt]
Juger et sentir \\[0pt]
Jusqu'à quel point la nature est-elle objet de science ? \\[0pt]
Jusqu'où s'étend le domaine de la science ? \\[0pt]
Justice et charité \\[0pt]
Justice et égalité \\[0pt]
Justice et équité \\[0pt]
Justice et impartialité \\[0pt]
Justice et ressentiment \\[0pt]
Justice et vengeance \\[0pt]
Justice et violence \\[0pt]
La barbarie \\[0pt]
La beauté du monde \\[0pt]
La beauté est-elle dans les choses ? \\[0pt]
La beauté est-elle intemporelle ? \\[0pt]
La beauté est-elle une promesse de bonheur ? \\[0pt]
La beauté et le mal \\[0pt]
La beauté morale \\[0pt]
La beauté nous rend-elle meilleurs ? \\[0pt]
La beauté s'explique-t-elle ? \\[0pt]
La bête \\[0pt]
La bête et l'animal \\[0pt]
La bêtise \\[0pt]
La bienveillance \\[0pt]
La bonne conscience \\[0pt]
La bonne intention \\[0pt]
La bonne volonté \\[0pt]
La bonté \\[0pt]
L'absence \\[0pt]
L'absolu \\[0pt]
L'absolu et le relatif \\[0pt]
L'abstraction \\[0pt]
L'abstrait et le concret \\[0pt]
L'absurde \\[0pt]
L'abus de pouvoir \\[0pt]
L'abus du droit \\[0pt]
L'académisme \\[0pt]
La causalité en histoire \\[0pt]
La cause efficiente \\[0pt]
La cause et l'effet \\[0pt]
L'accord des esprits est-il un critère de vérité ? \\[0pt]
L'accord est-il le terme ou le principe du dialogue ? \\[0pt]
La certitude de mourir \\[0pt]
La certitude ne peut-elle naître que de l'évidence ? \\[0pt]
La chair \\[0pt]
La chance \\[0pt]
La charité est-elle la négation de la justice ? \\[0pt]
La cité \\[0pt]
La classification des arts \\[0pt]
La coexistence des libertés \\[0pt]
La cohérence est-elle la norme du vrai ? \\[0pt]
La cohérence logique est-elle une condition suffisante de la démonstration ? \\[0pt]
La colère \\[0pt]
La communauté scientifique \\[0pt]
La compétence \\[0pt]
La compétence en politique \\[0pt]
La compréhension \\[0pt]
La condition humaine \\[0pt]
La confiance \\[0pt]
La confiance est-elle une vertu ? \\[0pt]
La connaissance commune fait-elle obstacle à la vérité ? \\[0pt]
La connaissance des principes \\[0pt]
La connaissance est-elle une contemplation ? \\[0pt]
La connaissance et la croyance \\[0pt]
La connaissance intuitive \\[0pt]
La connaissance objective doit-elle s'interdire toute interprétation ? \\[0pt]
La connaissance objective exclut-elle toute forme de subjectivité ? \\[0pt]
La connaissance sensible \\[0pt]
La conquête du pouvoir \\[0pt]
La conscience \\[0pt]
La conscience animale \\[0pt]
La conscience a-t-elle des degrés ? \\[0pt]
La conscience d'agir suffit-elle à garantir notre liberté ? \\[0pt]
La conscience d'autrui est-elle impénétrable ? \\[0pt]
La conscience de la mort est-elle une condition de la sagesse ? \\[0pt]
La conscience de soi est-elle une donnée immédiate ? \\[0pt]
La conscience de soi est-elle une méconnaissance de soi ? \\[0pt]
La conscience de soi suppose-t-elle autrui ? \\[0pt]
La conscience du temps rend-elle l'existence tragique ? \\[0pt]
La conscience est-elle nécessairement malheureuse ? \\[0pt]
La conscience est-elle temporelle ? \\[0pt]
La conscience est-elle une connaissance ? \\[0pt]
La conscience est-elle une illusion ? \\[0pt]
La conscience et l'inconscient \\[0pt]
La conscience morale \\[0pt]
La conscience morale est-elle naturelle ? \\[0pt]
La conscience morale n'est-elle que le produit de l'éducation ? \\[0pt]
La contemplation \\[0pt]
La contemplation de la nature \\[0pt]
La contingence de l'existence \\[0pt]
La contradiction \\[0pt]
La contrainte peut-elle être légitime ? \\[0pt]
La controverse scientifique \\[0pt]
La convention et l'arbitraire \\[0pt]
La conversion \\[0pt]
La conviction \\[0pt]
La coopération \\[0pt]
La copie \\[0pt]
La corruption \\[0pt]
La courtoisie \\[0pt]
La coutume \\[0pt]
La création \\[0pt]
La création artistique \\[0pt]
La création de valeur \\[0pt]
La crédulité \\[0pt]
La critique \\[0pt]
La croyance \\[0pt]
La croyance et la foi \\[0pt]
La croyance et la raison \\[0pt]
La croyance peut-elle être rationnelle ? \\[0pt]
La croyance peut-elle s'accompagner de certitude ? \\[0pt]
La croyance peut-elle tenir lieu de savoir ? \\[0pt]
La croyance religieuse échappe-t-elle à toute logique ? \\[0pt]
L'acte manqué \\[0pt]
L'action \\[0pt]
L'action et le risque \\[0pt]
L'action politique \\[0pt]
L'actualité \\[0pt]
La culture \\[0pt]
La culture commence-t-elle avec le refus de la nature ? \\[0pt]
La culture engendre-t-elle le progrès ? \\[0pt]
La culture est-elle la négation de la nature ? \\[0pt]
La culture est-elle une seconde nature ? \\[0pt]
La culture est-elle un luxe ? \\[0pt]
La culture et le langage \\[0pt]
La culture et les cultures \\[0pt]
La culture garantit-elle l'excellence humaine ? \\[0pt]
La culture générale \\[0pt]
La culture nous rend-elle plus humains ? \\[0pt]
La culture nous unit-elle ? \\[0pt]
La culture populaire \\[0pt]
La culture rend-elle plus humain ? \\[0pt]
La culture technique \\[0pt]
La curiosité \\[0pt]
La décision \\[0pt]
La défense de l'intérêt général est-il la fin dernière de la politique ? \\[0pt]
La démarche scientifique exclut-elle tout recours à l'imagination ? \\[0pt]
La démesure \\[0pt]
La démocratie, est-ce le pouvoir du plus grand nombre ? \\[0pt]
La démocratie est-elle la loi du plus fort ? \\[0pt]
La démocratie est-elle le règne de l'opinion ? \\[0pt]
La démocratie peut-elle échapper à la démagogie ? \\[0pt]
La démocratie peut-elle être représentative ? \\[0pt]
La démonstration \\[0pt]
La démonstration mathématique peut-elle servir de modèle à toute démonstration ? \\[0pt]
La démonstration n'est-elle qu'une méthode d'exposition des découvertes ? \\[0pt]
La démonstration nous garantit-elle l'accès à la vérité ? \\[0pt]
La démonstration supprime-t-elle le doute ? \\[0pt]
L'adéquation aux choses suffit-elle à définir la vérité ? \\[0pt]
La déraison \\[0pt]
La désobéissance \\[0pt]
La détermination \\[0pt]
La dialectique \\[0pt]
La dialectique remet-elle en question le principe de contradiction ? \\[0pt]
La différence des sexes est-elle un problème philosophique ? \\[0pt]
La dignité \\[0pt]
La discipline \\[0pt]
La discorde \\[0pt]
La discrétion \\[0pt]
La disharmonie \\[0pt]
La distinction \\[0pt]
La diversité \\[0pt]
La diversité des cultures fait-elle obstacle à l'unité du genre humain ? \\[0pt]
La diversité des langues est-elle un obstacle à l'entente entre les hommes ? \\[0pt]
La diversité des opinions conduit-elle à douter de tout ? \\[0pt]
La division du travail \\[0pt]
L'admiration \\[0pt]
La douleur \\[0pt]
La douleur nous apprend-elle quelque chose ? \\[0pt]
La durée \\[0pt]
La durée et l'instant \\[0pt]
La faiblesse \\[0pt]
La familiarité \\[0pt]
La famille \\[0pt]
La famille est-elle un modèle de société ? \\[0pt]
La fatigue \\[0pt]
La fermeté \\[0pt]
La fête \\[0pt]
La fiction \\[0pt]
La fidélité \\[0pt]
La figuration sensible de l'intelligible \\[0pt]
La finalité \\[0pt]
La fin de la nature \\[0pt]
La fin de l'État \\[0pt]
La fin de l'histoire \\[0pt]
La fin des temps \\[0pt]
La fin du travail \\[0pt]
La fin et les moyens \\[0pt]
La finitude \\[0pt]
La fin justifie-t-elle les moyens ? \\[0pt]
La foi \\[0pt]
La foi est-elle aveugle ? \\[0pt]
La foi est-elle rationnelle ? \\[0pt]
La folie \\[0pt]
La fonction \\[0pt]
La fonction et l'organe \\[0pt]
La force de la loi \\[0pt]
La force de la nature \\[0pt]
La force de la vérité \\[0pt]
La force de l'esprit \\[0pt]
La force de l'État est-elle nécessaire à la liberté des citoyens ? \\[0pt]
La force de l'habitude \\[0pt]
La force des idées \\[0pt]
La force des lois \\[0pt]
La force donne-t-elle des droits ? \\[0pt]
La force du droit \\[0pt]
La force et la violence \\[0pt]
La force et le droit \\[0pt]
La force et le droit s'opposent-ils nécessairement ? \\[0pt]
La franchise \\[0pt]
La fraternité \\[0pt]
La fuite du temps \\[0pt]
La généralisation empirique \\[0pt]
La générosité \\[0pt]
La gloire \\[0pt]
La grandeur \\[0pt]
La guerre \\[0pt]
La guerre et la paix \\[0pt]
La guerre peut-elle être juste ? \\[0pt]
La guerre peut-elle être justifiée ? \\[0pt]
La guerre peut-elle être un droit ? \\[0pt]
La haine \\[0pt]
La haine de la raison \\[0pt]
La hiérarchie \\[0pt]
La honte \\[0pt]
La joie \\[0pt]
La jurisprudence \\[0pt]
La juste mesure \\[0pt]
La justice \\[0pt]
La justice a-t-elle un fondement rationnel ? \\[0pt]
La justice distributive \\[0pt]
La justice est-elle de ce monde ? \\[0pt]
La justice est-elle de l'ordre du sentiment ? \\[0pt]
La justice est-elle l'affaire de l'État ? \\[0pt]
La justice est-elle une idée ? \\[0pt]
La justice est-elle une vertu ? \\[0pt]
La justice est-elle un idéal rationnel ? \\[0pt]
La justice et la force \\[0pt]
La justice et la loi \\[0pt]
La justice et la paix \\[0pt]
La justice et le droit \\[0pt]
La justice et l'égalité \\[0pt]
La justice n'est-elle qu'une institution ? \\[0pt]
La justice n'est-elle qu'un idéal ? \\[0pt]
La justice peut-elle être fondée en nature ? \\[0pt]
La justice peut-elle se passer d'institutions ? \\[0pt]
La justice requiert-elle l'infaillibilité des juges ? \\[0pt]
La justice sociale \\[0pt]
La justice suppose-t-elle l'égalité ? \\[0pt]
La lâcheté \\[0pt]
La laideur \\[0pt]
La langue et la parole \\[0pt]
La langue et la pensée \\[0pt]
La légèreté \\[0pt]
La légitime défense \\[0pt]
La lettre et l'esprit \\[0pt]
La liberté \\[0pt]
La liberté a-t-elle un prix ? \\[0pt]
La liberté comporte-t-elle des degrés ? \\[0pt]
La liberté connaît-elle des excès ? \\[0pt]
La liberté de croire \\[0pt]
La liberté de l'artiste \\[0pt]
La liberté de l'interprète \\[0pt]
La liberté de penser \\[0pt]
La liberté d'expression est-elle nécessaire à la liberté de pensée ? \\[0pt]
La liberté d'expression est-elle nécessaire à la liberté de penser ? \\[0pt]
La liberté d'indifférence \\[0pt]
La liberté du choix \\[0pt]
La liberté est-elle ce qui définit l'homme ? \\[0pt]
La liberté est-elle le pouvoir de refuser ? \\[0pt]
La liberté est-elle une illusion ? \\[0pt]
La liberté et la nécessité \\[0pt]
La liberté et l'égalité sont-elles compatibles ? \\[0pt]
La liberté et le hasard \\[0pt]
La liberté et les libertés \\[0pt]
La liberté et le temps \\[0pt]
La liberté fait-elle de nous des êtres meilleurs ? \\[0pt]
La liberté implique-t-elle l'indifférence ? \\[0pt]
La liberté ne s'éprouve-t-elle que dans la solitude ? \\[0pt]
La liberté n'est-elle qu'une illusion ? \\[0pt]
La liberté nous rend-elle inexcusables ? \\[0pt]
La liberté peut-elle être prouvée ? \\[0pt]
La liberté peut-elle faire peur ? \\[0pt]
La liberté peut-elle se refuser ? \\[0pt]
La liberté requiert-elle le libre échange ? \\[0pt]
La liberté s'achète-t-elle ? \\[0pt]
La liberté s'apprend-elle ? \\[0pt]
La liberté se mérite-t-elle ? \\[0pt]
La libre interprétation \\[0pt]
L'aliénation \\[0pt]
La logique des sens \\[0pt]
La logique est-elle la norme du vrai ? \\[0pt]
La loi \\[0pt]
La loi dit-elle ce qui est juste ? \\[0pt]
La loi doit-elle être la même pour tous ? \\[0pt]
La loi est-elle une garantie contre l'injustice ? \\[0pt]
La loi et la coutume \\[0pt]
La loi et l'ordre \\[0pt]
La loi peut-elle changer les mœurs ? \\[0pt]
La loyauté \\[0pt]
L'altérité \\[0pt]
L'altruisme \\[0pt]
L'altruisme n'est-il qu'un égoïsme bien compris ? \\[0pt]
La machine \\[0pt]
La main \\[0pt]
La main et l'esprit \\[0pt]
La maîtrise de soi \\[0pt]
La majorité doit-elle toujours l'emporter ? \\[0pt]
La majorité, force ou droit ? \\[0pt]
La majorité peut-elle être tyrannique ? \\[0pt]
La maladie \\[0pt]
La marginalité \\[0pt]
La mathématisation du réel \\[0pt]
La matière \\[0pt]
La matière est-elle connaissable ? \\[0pt]
La matière est-elle plus facile à connaître que l'esprit ? \\[0pt]
La matière et la vie \\[0pt]
La matière et l'esprit \\[0pt]
La matière et l'esprit peuvent-ils constituer une seule chose ? \\[0pt]
La matière n'est-elle que ce que l'on perçoit ? \\[0pt]
La matière n'est-elle qu'une idée ? \\[0pt]
La matière peut-elle agir ? \\[0pt]
La maturité \\[0pt]
La mauvaise conscience \\[0pt]
La mauvaise foi \\[0pt]
La mauvaise volonté \\[0pt]
L'ambiguïté \\[0pt]
L'ambiguïté des mots peut-elle être heureuse ? \\[0pt]
La méchanceté \\[0pt]
L'âme des bêtes \\[0pt]
L'âme et le corps sont-ils une seule et même chose ? \\[0pt]
La méfiance \\[0pt]
L'âme jouit-elle d'une vie propre ? \\[0pt]
La mélancolie \\[0pt]
La mémoire \\[0pt]
La mémoire collective \\[0pt]
La mémoire est-elle une fonction vitale ? \\[0pt]
La mémoire et l'oubli \\[0pt]
La mesure \\[0pt]
La mesure du temps \\[0pt]
La métamorphose \\[0pt]
La métaphore n'est-elle qu'un ornement du discours ? \\[0pt]
La métaphysique est-elle un discours sans objet ? \\[0pt]
La méthode \\[0pt]
La misère \\[0pt]
L'amitié \\[0pt]
L'amitié est-elle une vertu ? \\[0pt]
La modération \\[0pt]
La modestie \\[0pt]
La monnaie n'est-elle qu'un moyen d'échange ? \\[0pt]
La morale \\[0pt]
La morale a-t-elle à décider de la sexualité ? \\[0pt]
La morale a-t-elle besoin de la notion de sainteté ? \\[0pt]
La morale a-t-elle sa place dans l'économie ? \\[0pt]
La morale consiste-t-elle à respecter le droit ? \\[0pt]
La morale doit-elle être rationnelle ? \\[0pt]
La morale est-elle affaire de convention ? \\[0pt]
La morale est-elle affaire de sentiment ? \\[0pt]
La morale est-elle condamnée à n'être qu'un champ de bataille ? \\[0pt]
La morale est-elle désintéressée ? \\[0pt]
La morale est-elle en conflit avec le désir ? \\[0pt]
La morale est-elle une affaire de raison ? \\[0pt]
La morale est-elle une affaire solitaire ? \\[0pt]
La morale est-elle un fait de culture ? \\[0pt]
La morale et la politique \\[0pt]
La morale et la religion visent-elles les mêmes fins ? \\[0pt]
La morale et les mœurs \\[0pt]
La morale n'est-elle qu'un art de vivre ? \\[0pt]
La morale n'est-elle qu'un ensemble de conventions ? \\[0pt]
La morale peut-elle se définir comme l'art d'être heureux ? \\[0pt]
La morale peut-elle se fonder sur les sentiments ? \\[0pt]
La morale peut-elle s'enseigner ? \\[0pt]
La morale s'enseigne-t-elle ? \\[0pt]
La morale s'oppose-t-elle à la politique ? \\[0pt]
La moralité est-elle utile à la vie sociale ? \\[0pt]
La mort \\[0pt]
La mort d'autrui \\[0pt]
La mort de Dieu \\[0pt]
L'amour de la liberté \\[0pt]
L'amour de l'art \\[0pt]
L'amour de la vie \\[0pt]
L'amour de soi \\[0pt]
L'amour de soi est-il immoral ? \\[0pt]
L'amour du pouvoir est-il compatible avec le dévouement à la chose publique ? \\[0pt]
L'amour du travail \\[0pt]
L'amour est-il toujours une passion ? \\[0pt]
L'amour et la haine \\[0pt]
L'amour et l'amitié \\[0pt]
L'amour et le devoir \\[0pt]
L'amour et le respect \\[0pt]
L'amour fou \\[0pt]
L'amour peut-il être raisonnable ? \\[0pt]
L'amour peut-il être un devoir ? \\[0pt]
L'amour-propre \\[0pt]
La multitude \\[0pt]
L'anachronisme \\[0pt]
L'analogie \\[0pt]
L'analyse du langage ordinaire peut-elle avoir un intérêt philosophique ? \\[0pt]
L'anarchie \\[0pt]
La nation \\[0pt]
La nature \\[0pt]
La nature a-t-elle des droits ? \\[0pt]
La nature a-t-elle une histoire ? \\[0pt]
La nature a-t-elle un ordre ? \\[0pt]
La nature est-elle écrite en langage mathématique ? \\[0pt]
La nature est-elle prévisible ? \\[0pt]
La nature est-elle une grande artiste ? \\[0pt]
La nature est-elle une idée ? \\[0pt]
La nature est-elle une ressource ? \\[0pt]
La nature est-elle un modèle ? \\[0pt]
La nature existe-t-elle ? \\[0pt]
La nature fait-elle bien les choses ? \\[0pt]
La nature ne fait-elle rien en vain ? \\[0pt]
La nature nous donne-t-elle des commandements ? \\[0pt]
La nature nous indique-t-elle ce qui est bon ? \\[0pt]
La nature peut-elle avoir des droits ? \\[0pt]
La nature peut-elle constituer une norme ? \\[0pt]
La nature peut-elle être détruite ? \\[0pt]
La nature peut-elle être un modèle ? \\[0pt]
La nature peut-elle nous indiquer ce que nous devons faire ? \\[0pt]
La nature reprend-elle toujours ses droits ? \\[0pt]
La nécessité \\[0pt]
La négation \\[0pt]
La neutralité de l'État \\[0pt]
Langage et communication \\[0pt]
Langage et logique \\[0pt]
Langage et passions \\[0pt]
Langage et pensée \\[0pt]
Langage et pouvoir \\[0pt]
Langage et société \\[0pt]
L'angoisse \\[0pt]
L'angoisse et la peur \\[0pt]
L'animal \\[0pt]
L'animal a-t-il des droits ? \\[0pt]
L'animal est-il une personne ? \\[0pt]
L'animal et l'homme \\[0pt]
La non-violence \\[0pt]
La norme \\[0pt]
La nostalgie \\[0pt]
La notion de démocratie représentative est-elle contradictoire ? \\[0pt]
La notion de droit international est-elle contradictoire ? \\[0pt]
La notion de progrès moral a-t-elle encore un sens ? \\[0pt]
La notion d'histoire commune a-t-elle un sens ? \\[0pt]
La nouveauté \\[0pt]
La paix \\[0pt]
La paix est-elle l'absence de guerres ? \\[0pt]
La paix est-elle le plus grand des biens ? \\[0pt]
La paix n'est-elle que l'absence de guerre ? \\[0pt]
La paix sociale \\[0pt]
La paix sociale est-elle le but de la politique ? \\[0pt]
La parole \\[0pt]
La parole donnée \\[0pt]
La parole et l'action \\[0pt]
La parole et l'écriture \\[0pt]
La parole et le geste \\[0pt]
La parole intérieure \\[0pt]
La parole peut-elle agir ? \\[0pt]
La passion de la connaissance \\[0pt]
La passion de la liberté \\[0pt]
La passion de la vérité \\[0pt]
La passion de l'égalité \\[0pt]
La passion est-elle immorale ? \\[0pt]
La passion est-elle l'ennemi de la raison ? \\[0pt]
La passion exclut-elle la lucidité ? \\[0pt]
La passivité \\[0pt]
La patience \\[0pt]
La patience de la pensée \\[0pt]
La patience est-elle une vertu ? \\[0pt]
La pauvreté \\[0pt]
La pauvreté est-elle une injustice ? \\[0pt]
La peine \\[0pt]
La pensée \\[0pt]
La pensée a-t-elle besoin d'images ? \\[0pt]
La pensée doit-elle se soumettre aux règles de la logique ? \\[0pt]
La pensée et la conscience sont-elles une seule et même chose ? \\[0pt]
La pensée peut-elle se passer de mots ? \\[0pt]
La pensée philosophique doit-elle être une pensée systématique ? \\[0pt]
L'aperception \\[0pt]
La perception \\[0pt]
La perception construit-elle son objet ? \\[0pt]
La perception de l'espace est-elle innée ou acquise ? \\[0pt]
La perception est-elle le premier degré de la connaissance ? \\[0pt]
La perception est-elle une interprétation ? \\[0pt]
La perception est-elle un mode de connaissance ? \\[0pt]
La perception me donne-t-elle le réel ? \\[0pt]
La perception peut-elle s'éduquer ? \\[0pt]
La perfection \\[0pt]
La perfection est-elle désirable ? \\[0pt]
La permanence \\[0pt]
La personne et l'individu \\[0pt]
La perspective \\[0pt]
La peur \\[0pt]
La peur de la science \\[0pt]
La peur de la technique \\[0pt]
La philosophie a-t-elle un objet qui lui soit propre ? \\[0pt]
La philosophie est-elle une méditation de la mort ? \\[0pt]
La philosophie est-elle une méditation de la vie ou de la mort ? \\[0pt]
La philosophie et son histoire \\[0pt]
La philosophie peut-elle ne pas parler de Dieu ? \\[0pt]
La philosophie rend-elle inefficace la propagande ? \\[0pt]
La pitié \\[0pt]
La pitié fonde-t-elle la morale ? \\[0pt]
La place de l'animal \\[0pt]
La plaisanterie \\[0pt]
La pluralité des arts \\[0pt]
La pluralité des interprétations \\[0pt]
La pluralité des langues \\[0pt]
La pluralité des religions \\[0pt]
La pluralité des sciences \\[0pt]
La police \\[0pt]
La politesse \\[0pt]
La politesse est-elle une vertu ? \\[0pt]
La politique \\[0pt]
La politique a-t-elle pour but de nous faire vivre dans un monde meilleur ? \\[0pt]
La politique consiste-t-elle à faire cause commune ? \\[0pt]
La politique doit-elle avoir pour visée le bonheur ? \\[0pt]
La politique est-elle affaire de compétence ? \\[0pt]
La politique est-elle affaire d'expérience ou de théorie ? \\[0pt]
La politique est-elle l'affaire des spécialistes ? \\[0pt]
La politique est-elle l'affaire de tous ? \\[0pt]
La politique est-elle l'art de convaincre le peuple ? \\[0pt]
La politique est-elle un art ? \\[0pt]
La politique est-elle une affaire d'experts ? \\[0pt]
La politique est-elle une science ? \\[0pt]
La politique est-elle un métier ? \\[0pt]
La politique et la guerre \\[0pt]
La politique et le bonheur \\[0pt]
La politique et le pouvoir \\[0pt]
La politique n'est-elle que l'art de conquérir et de conserver le pouvoir ? \\[0pt]
La postérité \\[0pt]
La poursuite de mon intérêt m'oppose-t-elle aux autres ? \\[0pt]
L'apparence \\[0pt]
L'apparence est-elle toujours trompeuse ? \\[0pt]
L'apparence fait-elle partie de la réalité ? \\[0pt]
L'apprentissage \\[0pt]
L'apprentissage de la liberté \\[0pt]
La précarité \\[0pt]
La présence d'esprit \\[0pt]
La prière \\[0pt]
L'\emph{a priori} \\[0pt]
La privation \\[0pt]
La promesse \\[0pt]
L'à propos \\[0pt]
La propriété \\[0pt]
La propriété et le travail \\[0pt]
La prudence \\[0pt]
La pudeur \\[0pt]
La puissance \\[0pt]
La puissance de la raison \\[0pt]
La puissance du désir \\[0pt]
La punition \\[0pt]
La pureté \\[0pt]
La quantité et la qualité \\[0pt]
La question du sens de la vie peut-elle être sans réponses ? \\[0pt]
La question « qui suis-je » admet-elle une réponse exacte ? \\[0pt]
La raison \\[0pt]
La raison a-t-elle pour fin la connaissance ? \\[0pt]
La raison a-t-elle toujours raison ? \\[0pt]
La raison a-t-elle une histoire ? \\[0pt]
La raison d'État \\[0pt]
La raison d'État peut-elle être justifiée ? \\[0pt]
La raison d'être \\[0pt]
La raison doit-elle critiquer la croyance ? \\[0pt]
La raison doit-elle être notre guide ? \\[0pt]
La raison doit-elle faire une place à la croyance ? \\[0pt]
La raison doit-elle se soumettre au réel ? \\[0pt]
La raison est-elle impersonnelle ? \\[0pt]
La raison est-elle seulement affaire de logique ? \\[0pt]
La raison est-elle une valeur ? \\[0pt]
La raison et la démonstration \\[0pt]
La raison et la liberté \\[0pt]
La raison et le bonheur \\[0pt]
La raison et le réel \\[0pt]
La raison et l'existence \\[0pt]
La raison et l'expérience \\[0pt]
La raison et l'irrationnel \\[0pt]
La raison ne veut-elle que connaître ? \\[0pt]
La raison peut-elle entrer en conflit avec elle-même ? \\[0pt]
La raison peut-elle entrer en contradiction avec elle-même ? \\[0pt]
La raison peut-elle errer ? \\[0pt]
La raison peut-elle nous induire en erreur ? \\[0pt]
La raison peut-elle s'aveugler elle-même ? \\[0pt]
La raison peut-elle se contredire ? \\[0pt]
La raison peut-elle servir le mal ? \\[0pt]
La raison peut-elle s'opposer à elle-même ? \\[0pt]
La raison suffisante \\[0pt]
La rationalité \\[0pt]
La rationalité se confond-elle avec la systématicité ? \\[0pt]
L'arbitraire \\[0pt]
La réalisation du devoir exclut-elle toute forme de plaisir ? \\[0pt]
La réalité des idées \\[0pt]
La réalité des phénomènes \\[0pt]
La réalité du désordre \\[0pt]
La réalité du passé \\[0pt]
La réalité du temps \\[0pt]
La réalité n'est-elle qu'une construction ? \\[0pt]
La réalité nourrit-elle la fiction ? \\[0pt]
La réalité sensible \\[0pt]
La recherche de la vérité peut-elle être désintéressée ? \\[0pt]
La recherche du bonheur \\[0pt]
La recherche du bonheur est-elle égoïste ? \\[0pt]
La recherche du bonheur est-elle un idéal égoïste ? \\[0pt]
La recherche du bonheur est-il un guide suffisant dans la vie ? \\[0pt]
La recherche du bonheur peut-elle être un devoir ? \\[0pt]
La réciprocité \\[0pt]
La reconnaissance \\[0pt]
La reconnaissance de la diversité des cultures condamne-t-elle au relativisme ? \\[0pt]
La réflexion \\[0pt]
La réfutation \\[0pt]
La règle et l'exception \\[0pt]
La régression \\[0pt]
La religion \\[0pt]
La religion a-t-elle une fonction de cohésion sociale ? \\[0pt]
La religion conduit-elle l'homme au-delà de lui-même ? \\[0pt]
La religion divise-t-elle les hommes ? \\[0pt]
La religion est-elle à craindre ? \\[0pt]
La religion est-elle au fondement de la morale ? \\[0pt]
La religion est-elle fondée sur la peur de la mort ? \\[0pt]
La religion est-elle l'asile de l'ignorance ? \\[0pt]
La religion est-elle une affaire privée ? \\[0pt]
La religion est-elle un instrument de pouvoir ? \\[0pt]
La religion et la croyance \\[0pt]
La religion naturelle \\[0pt]
La religion n'est-elle que l'affaire des croyants ? \\[0pt]
La religion outrepasse-t-elle les limites de la raison ? \\[0pt]
La religion peut-elle n'être qu'une affaire privée ? \\[0pt]
La religion peut-elle se passer de transcendant ? \\[0pt]
La religion relève-t-elle de l'opinion ? \\[0pt]
La répétition \\[0pt]
La représentation \\[0pt]
La représentation artistique \\[0pt]
La représentation politique \\[0pt]
La reproduction \\[0pt]
La responsabilité \\[0pt]
La responsabilité collective \\[0pt]
La responsabilité est-elle une fiction juridique ? \\[0pt]
La responsabilité politique \\[0pt]
La responsabilité politique n'est-elle le fait que de ceux qui gouvernent ? \\[0pt]
La réussite \\[0pt]
La révolte \\[0pt]
La révolution \\[0pt]
L'argent \\[0pt]
L'argent est-il la mesure de tout échange ? \\[0pt]
La rigueur \\[0pt]
La rigueur des lois \\[0pt]
La rigueur des lois ? \\[0pt]
L'art \\[0pt]
L'art ajoute-t-il quelque chose à la vie ? \\[0pt]
L'art a-t-il à être populaire ? \\[0pt]
L'art a-t-il besoin de théorie ? \\[0pt]
L'art a-t-il pour fonction de sublimer le réel ? \\[0pt]
L'art a-t-il une histoire ? \\[0pt]
L'art a-t-il un rôle à jouer dans l'éducation ? \\[0pt]
L'art change-t-il la vie ? \\[0pt]
L'art de gouverner \\[0pt]
L'art de juger \\[0pt]
L'art de persuader \\[0pt]
L'art de vivre \\[0pt]
L'art d'interpréter \\[0pt]
L'art doit-il nécessairement représenter la réalité ? \\[0pt]
L'art donne-t-il à penser ? \\[0pt]
L'art donne-t-il à rêver ou à penser ? \\[0pt]
L'art donne-t-il nécessairement lieu à la production d'une œuvre ? \\[0pt]
L'art éduque-t-il la perception ? \\[0pt]
L'art est-il affaire d'apparence ? \\[0pt]
L'art est-il désintéressé ? \\[0pt]
L'art est-il hors du temps ? \\[0pt]
L'art est-il le règne des apparences ? \\[0pt]
L'art est-il moins nécessaire que la science ? \\[0pt]
L'art est-il subversif ? \\[0pt]
L'art est-il une histoire ? \\[0pt]
L'art est-il universel ? \\[0pt]
L'art est-il un langage ? \\[0pt]
L'art est-il un luxe ? \\[0pt]
L'art est-il un moyen de connaître ? \\[0pt]
L'art est-il un refuge ? \\[0pt]
L'art et la manière \\[0pt]
L'art et la morale \\[0pt]
L'art et la raison \\[0pt]
L'art et la société \\[0pt]
L'art et la technique \\[0pt]
L'art et la vie \\[0pt]
L'art et le jeu \\[0pt]
L'art et le réel \\[0pt]
L'art et le sacré \\[0pt]
L'art et le travail \\[0pt]
L'art et l'illusion \\[0pt]
L'art et l'interprétation \\[0pt]
L'art et l'invisible \\[0pt]
L'art éveille-t-il des émotions spécifiques ? \\[0pt]
L'artifice \\[0pt]
L'artificiel \\[0pt]
L'artisan et l'ingénieur \\[0pt]
L'artiste a-t-il besoin de modèle ? \\[0pt]
L'artiste doit-il être de son temps ? \\[0pt]
L'artiste doit-il se soucier du goût du public ? \\[0pt]
L'artiste est-il souverain ? \\[0pt]
L'artiste est-il un travailleur ? \\[0pt]
L'artiste et l'artisan \\[0pt]
L'artiste sait-il ce qu'il fait ? \\[0pt]
L'artiste travaille-t-il ? \\[0pt]
L'art n'est-il qu'un mode d'expression subjectif ? \\[0pt]
L'art nous détourne-t-il de la réalité ? \\[0pt]
L'art nous mène-t-il au vrai ? \\[0pt]
L'art parachève-t-il la nature ? \\[0pt]
L'art participe-t-il à la vie politique ? \\[0pt]
L'art peut-il être conceptuel ? \\[0pt]
L'art peut-il être réaliste \\[0pt]
L'art peut-il être sans œuvre ? \\[0pt]
L'art peut-il ne pas être sacré ? \\[0pt]
L'art peut-il n'être aucunement mimétique ? \\[0pt]
L'art peut-il se passer de la beauté ? \\[0pt]
L'art peut-il se passer de règles ? \\[0pt]
L'art peut-il se passer d'œuvres ? \\[0pt]
L'art pour l'art \\[0pt]
L'art progresse-t-il ? \\[0pt]
L'art prolonge-t-il la nature ? \\[0pt]
L'art rend-il heureux ? \\[0pt]
L'art s'adresse-t-il à tous ? \\[0pt]
L'art s'apprend-il ? \\[0pt]
La ruse \\[0pt]
La sagesse et la passion \\[0pt]
La santé \\[0pt]
La santé n'est-elle que l'absence de maladie ? \\[0pt]
La satisfaction \\[0pt]
L'ascétisme est-il une vertu ? \\[0pt]
La science a-t-elle besoin d'imagination ? \\[0pt]
La science a-t-elle besoin d'une méthode ? \\[0pt]
La science a-t-elle le monopole de la raison ? \\[0pt]
La science a-t-elle pour fin de prévoir ? \\[0pt]
La science a-t-elle réponse à tout ? \\[0pt]
La science commence-t-elle avec la perception ? \\[0pt]
La science du vivant peut-elle se passer de l'idée de finalité ? \\[0pt]
La science est-elle inhumaine ? \\[0pt]
La science est-elle le lieu de la vérité ? \\[0pt]
La science est-elle une connaissance du réel ? \\[0pt]
La science nous indique-t-elle ce que nous devons faire ? \\[0pt]
La science permet-elle de comprendre le monde ? \\[0pt]
La science permet-elle de mieux comprendre la religion ? \\[0pt]
La science peut-elle être une métaphysique ? \\[0pt]
La science peut-elle produire des croyances ? \\[0pt]
La science peut-elle se passer de l'idée de finalité ? \\[0pt]
La science peut-elle se passer d'hypothèses ? \\[0pt]
La science politique \\[0pt]
La science rend-elle la religion caduque ? \\[0pt]
La science se limite-t-elle à constater les faits ? \\[0pt]
La science s'oppose-t-elle à la religion ? \\[0pt]
La sensibilité \\[0pt]
La sérénité \\[0pt]
La servitude \\[0pt]
La servitude peut-elle être volontaire ? \\[0pt]
La servitude volontaire \\[0pt]
La simplicité \\[0pt]
La sincérité \\[0pt]
La société \\[0pt]
La société civile \\[0pt]
La société doit-elle reconnaître les désirs individuels ? \\[0pt]
La société est-elle un grand marché ? \\[0pt]
La société est-elle un organisme ? \\[0pt]
La société et le droit \\[0pt]
La société et les échanges \\[0pt]
La société et l'État \\[0pt]
La société et l'individu \\[0pt]
La société fait-elle l'homme ? \\[0pt]
La société peut-elle être l'objet d'une science ? \\[0pt]
La société repose-t-elle sur l'altruisme ? \\[0pt]
La solidarité \\[0pt]
La solidarité est-elle naturelle ? \\[0pt]
La solitude \\[0pt]
La sollicitude \\[0pt]
La souffrance au travail \\[0pt]
La souffrance d'autrui \\[0pt]
La souffrance d'autrui m'importe-t-elle ? \\[0pt]
La souffrance peut-elle être un mode de connaissance ? \\[0pt]
La soumission à l'autorité \\[0pt]
La soumission volontaire \\[0pt]
La souveraineté \\[0pt]
La souveraineté de l'État \\[0pt]
La souveraineté peut-elle être limitée ? \\[0pt]
La souveraineté peut-elle se partager ? \\[0pt]
La spontanéité \\[0pt]
L'association des idées \\[0pt]
La superstition \\[0pt]
La sympathie \\[0pt]
La sympathie peut-elle tenir lieu de morale ? \\[0pt]
La tautologie \\[0pt]
La technique \\[0pt]
La technique accroît-elle notre liberté ? \\[0pt]
La technique a-t-elle sa place en politique ? \\[0pt]
La technique est-elle civilisatrice ? \\[0pt]
La technique est-elle contre nature ? \\[0pt]
La technique est-elle le propre de l'homme ? \\[0pt]
La technique est-elle neutre ? \\[0pt]
La technique est-elle un savoir ? \\[0pt]
La technique et la liberté \\[0pt]
La technique et la morale \\[0pt]
La technique et le corps \\[0pt]
La technique et le travail \\[0pt]
La technique invente-t-elle des formes ? \\[0pt]
La technique libère-t-elle les hommes ? \\[0pt]
La technique ne fait-elle qu'appliquer la science ? \\[0pt]
La technique ne pose-t-elle que des problèmes techniques ? \\[0pt]
La technique n'est-elle pour l'homme qu'un moyen ? \\[0pt]
La technique n'est-elle qu'un outil au service de l'homme ? \\[0pt]
La technique n'existe-elle que pour satisfaire des besoins ? \\[0pt]
La technique nous éloigne-t-elle de la nature ? \\[0pt]
La technique nous libère-t-elle ? \\[0pt]
La technique nous libère-t-elle du travail ? \\[0pt]
La technique nous oppose-t-elle à la nature ? \\[0pt]
La technique nous permet-elle de comprendre la nature ? \\[0pt]
La technique peut-elle se déduire de la science ? \\[0pt]
La technique peut-elle se passer de la science ? \\[0pt]
La technique produit-elle son propre savoir ? \\[0pt]
La technique sert-elle nos désirs ? \\[0pt]
La technocratie \\[0pt]
La tentation \\[0pt]
La théorie \\[0pt]
La théorie et la pratique \\[0pt]
La théorie et l'expérience \\[0pt]
La théorie nous éloigne-t-elle de la réalité ? \\[0pt]
La théorie scientifique \\[0pt]
La tolérance \\[0pt]
La tolérance est-elle une vertu ? \\[0pt]
La tradition \\[0pt]
La traduction \\[0pt]
La transgression \\[0pt]
L'attente \\[0pt]
L'attention \\[0pt]
L'attention caractérise-t-elle la conscience ? \\[0pt]
L'attitude religieuse \\[0pt]
La tyrannie \\[0pt]
La tyrannie des désirs \\[0pt]
L'audace \\[0pt]
L'au-delà \\[0pt]
L'authenticité \\[0pt]
L'autobiographie \\[0pt]
L'automatisation \\[0pt]
L'autonomie \\[0pt]
L'autoportrait \\[0pt]
L'autorité \\[0pt]
L'autorité de la chose jugée \\[0pt]
L'autorité de la loi \\[0pt]
L'autorité de la science \\[0pt]
L'autorité du droit \\[0pt]
La valeur \\[0pt]
La valeur de la culture \\[0pt]
La valeur de la raison \\[0pt]
La valeur de la vérité \\[0pt]
La valeur du don \\[0pt]
La valeur du travail \\[0pt]
La valeur morale de l'amour \\[0pt]
La valeur morale d'une action se juge-t-elle à ses conséquences ? \\[0pt]
L'avant-garde \\[0pt]
La vengeance \\[0pt]
L'avenir \\[0pt]
L'avenir a-t-il une réalité ? \\[0pt]
L'avenir est-il incertain ? \\[0pt]
L'avenir est-il l'affaire de la pensée ? \\[0pt]
L'avenir est-il une page blanche ? \\[0pt]
L'avenir peut-il être objet de connaissance ? \\[0pt]
La vérification \\[0pt]
La vérification fait-elle la vérité ? \\[0pt]
La vérité \\[0pt]
La vérité a-t-elle une histoire ? \\[0pt]
La vérité donne-t-elle le droit d'être injuste ? \\[0pt]
La vérité échappe-t-elle au temps ? \\[0pt]
La vérité est-elle affaire de cohérence ? \\[0pt]
La vérité est-elle contraignante ? \\[0pt]
La vérité est-elle la valeur suprême ? \\[0pt]
La vérité est-elle libératrice ? \\[0pt]
La vérité est-elle une ? \\[0pt]
La vérité est-elle une valeur ? \\[0pt]
La vérité et les vérités \\[0pt]
La vérité historique \\[0pt]
La vérité peut-elle être relative ? \\[0pt]
La vérité peut-elle laisser indifférent ? \\[0pt]
La vérité peut-elle se définir par le consensus ? \\[0pt]
La vérité rend-elle heureux ? \\[0pt]
La vérité rend-elle libre ? \\[0pt]
La vertu \\[0pt]
La vertu peut-elle être excessive ? \\[0pt]
La vertu requiert-elle l'idée de bien ? \\[0pt]
La vie de l'esprit \\[0pt]
La vie de plaisirs \\[0pt]
La vie en société impose-t-elle de n'être pas soi-même ? \\[0pt]
La vie en société menace-t-elle la liberté ? \\[0pt]
La vie est-elle sacrée ? \\[0pt]
La vie et le vivant \\[0pt]
La vie heureuse \\[0pt]
« La vieillesse est un naufrage » \\[0pt]
La vie intérieure \\[0pt]
La vie morale \\[0pt]
La vie peut-elle être objet de science ? \\[0pt]
La vie psychique \\[0pt]
La vie sauvage \\[0pt]
La vie sociale \\[0pt]
La vie sociale est-elle toujours conflictuelle ? \\[0pt]
La violence \\[0pt]
La violence de l'interprétation \\[0pt]
La violence du désir \\[0pt]
La violence est-elle le fondement du droit ? \\[0pt]
La violence est-elle toujours destructrice ? \\[0pt]
La violence peut-elle avoir raison ? \\[0pt]
La violence peut-elle être gratuite ? \\[0pt]
La violence verbale \\[0pt]
La virtuosité \\[0pt]
La vision peut-elle être le modèle de toute connaissance ? \\[0pt]
La vocation \\[0pt]
La voix de la raison \\[0pt]
La volonté est-elle la somme des inclinations ? \\[0pt]
La volonté est-elle libre ? \\[0pt]
La volonté peut-elle être générale ? \\[0pt]
La volonté peut-elle nous manquer ? \\[0pt]
La vraie vie \\[0pt]
La vue et le toucher \\[0pt]
Le bavardage \\[0pt]
Le beau est-il toujours moral ? \\[0pt]
Le beau et l'agréable \\[0pt]
Le beau et le bien \\[0pt]
Le beau et le sublime \\[0pt]
Le beau et l'utile \\[0pt]
Le beau geste \\[0pt]
Le bénéfice du doute \\[0pt]
Le besoin de métaphysique est-il un besoin de connaissance ? \\[0pt]
Le besoin de reconnaissance \\[0pt]
Le besoin de théorie \\[0pt]
Le besoin et le désir \\[0pt]
Le bien commun \\[0pt]
Le bien commun est-il une illusion ? \\[0pt]
Le bien commun et l'intérêt de tous \\[0pt]
Le bien, est-ce l'utile ? \\[0pt]
Le bien est-il relatif ? \\[0pt]
Le bien et le beau \\[0pt]
Le bien et les biens \\[0pt]
Le bien n'est-il réalisable que comme moindre mal ? \\[0pt]
Le bien public \\[0pt]
Le Bien public \\[0pt]
Le bon goût \\[0pt]
Le bonheur collectif \\[0pt]
Le bonheur de la passion est-il sans lendemain ? \\[0pt]
Le bonheur des sens \\[0pt]
Le bonheur du plus grand nombre \\[0pt]
Le bonheur est-il au nombre de nos devoirs ? \\[0pt]
Le bonheur est-il dans l'inconscience ? \\[0pt]
Le bonheur est-il l'affaire du politique ? \\[0pt]
Le bonheur est-il la fin de la vie ? \\[0pt]
Le bonheur est-il le but de la politique ? \\[0pt]
Le bonheur est-il un but politique ? \\[0pt]
Le bonheur est-il une affaire privée ? \\[0pt]
Le bonheur est-il une récompense ? \\[0pt]
Le bonheur est-il un idéal ? \\[0pt]
Le bonheur et la raison \\[0pt]
Le bonheur et la technique \\[0pt]
Le bonheur et le plaisir \\[0pt]
Le bonheur n'est-il qu'une idée ? \\[0pt]
Le bonheur peut-il être collectif ? \\[0pt]
Le bonheur s'apprend-il ? \\[0pt]
Le bonheur se calcule-t-il ? \\[0pt]
Le bricolage \\[0pt]
Le calendrier \\[0pt]
Le capital et le travail \\[0pt]
Le caractère \\[0pt]
Le caractère sacré de la vie \\[0pt]
Le cas de conscience \\[0pt]
Le cerveau et la pensée \\[0pt]
Le cerveau pense-t-il ? \\[0pt]
L'échange constitue-t-il un lien social ? \\[0pt]
L'échange économique fonde-t-il la société humaine \\[0pt]
L'échange et l'usage \\[0pt]
Le changement \\[0pt]
L'échange n'a-t-il de fondement qu'économique ? \\[0pt]
L'échange ne porte-t-il que sur les choses ? \\[0pt]
L'échange peut-il être désintéressé ? \\[0pt]
Le chaos \\[0pt]
Le châtiment \\[0pt]
Le châtiment peut-il ne rien devoir au désir de vengeance ? \\[0pt]
Le chef \\[0pt]
Le chef-d'œuvre \\[0pt]
Le choc des idées \\[0pt]
Le choix \\[0pt]
Le choix et la liberté \\[0pt]
Le cinéma est-il un art narratif ? \\[0pt]
Le citoyen \\[0pt]
Le commencement \\[0pt]
Le commerce \\[0pt]
Le commerce adoucit-il les mœurs ? \\[0pt]
Le commerce des idées \\[0pt]
Le commerce est-il un facteur de paix ? \\[0pt]
Le commerce peut-il être équitable ? \\[0pt]
Le commerce unit-il les hommes ? \\[0pt]
Le commun et le propre \\[0pt]
Le concept \\[0pt]
Le concept d'histoire universelle \\[0pt]
Le concept et l'exemple \\[0pt]
Le concret \\[0pt]
Le conflit est-il une maladie sociale ? \\[0pt]
L'économie et la politique \\[0pt]
L'économie et le politique obéissent-ils à une même rationalité ? \\[0pt]
Le consensus peut-il être critère de vérité ? \\[0pt]
Le consentement \\[0pt]
Le contentement \\[0pt]
Le contrat \\[0pt]
Le contrat de travail \\[0pt]
Le contrat est-il au fondement de la politique ? \\[0pt]
Le contrat peut-il remplacer la loi ? \\[0pt]
Le corps est-il négociable ? \\[0pt]
Le corps est-il une entrave pour l'esprit ? \\[0pt]
Le corps et l'âme \\[0pt]
Le corps et l'esprit \\[0pt]
Le corps impose-t-il des perspectives ? \\[0pt]
Le corps n'est-il qu'un mécanisme ? \\[0pt]
Le corps obéit-il à l'esprit ? \\[0pt]
Le corps politique \\[0pt]
Le cosmopolitisme \\[0pt]
Le courage \\[0pt]
Le cours de l'histoire \\[0pt]
Le crime \\[0pt]
L'écriture \\[0pt]
Le cynisme \\[0pt]
Le dedans et le dehors \\[0pt]
Le défaut \\[0pt]
Le dégoût \\[0pt]
Le déni et la loi \\[0pt]
Le désespoir \\[0pt]
Le désespoir peut-il passer pour une sagesse ? \\[0pt]
Le désintéressement \\[0pt]
Le désir \\[0pt]
Le désir d'absolu \\[0pt]
Le désir de l'autre \\[0pt]
Le désir de reconnaissance \\[0pt]
Le désir de savoir \\[0pt]
Le désir de savoir est-il comblé par la science ? \\[0pt]
Le désir de savoir est-il naturel ? \\[0pt]
Le désir d'éternité \\[0pt]
Le désir de vérité \\[0pt]
Le désir de vérité peut-il être interprété comme un désir de pouvoir ? \\[0pt]
Le désir du bonheur est-il universel ? \\[0pt]
Le désir est-il aveugle ? \\[0pt]
Le désir est-il le signe d'un manque ? \\[0pt]
Le désir est-il nécessairement l'expression d'un manque ? \\[0pt]
Le désir et la culpabilité \\[0pt]
Le désir et la liberté \\[0pt]
Le désir et la loi \\[0pt]
Le désir et le besoin \\[0pt]
Le désir et le mal \\[0pt]
Le désir et le manque \\[0pt]
Le désir et le réel \\[0pt]
Le désir et le rêve \\[0pt]
Le désir et le temps \\[0pt]
Le désir et le travail \\[0pt]
Le désir et l'instinct \\[0pt]
Le désir et l'interdit \\[0pt]
Le désir n'est-il que manque ? \\[0pt]
Le désir n'est-il qu'inquiétude ? \\[0pt]
Le désir peut-il ne pas avoir d'objet ? \\[0pt]
Le désir peut-il nous rendre libre ? \\[0pt]
Le désir peut-il se satisfaire de la réalité ? \\[0pt]
Le désordre \\[0pt]
Le despotisme \\[0pt]
Le destin \\[0pt]
Le développement des techniques fait-il reculer la croyance ? \\[0pt]
Le devenir \\[0pt]
Le devoir \\[0pt]
Le devoir de loyauté \\[0pt]
Le devoir de vérité \\[0pt]
Le devoir est-il l'expression de la contrainte sociale ? \\[0pt]
Le devoir et le bonheur \\[0pt]
Le devoir rend-il libre ? \\[0pt]
Le devoir se présente-t-il avec la force de l'évidence ? \\[0pt]
Le devoir supprime-t-il la liberté ? \\[0pt]
Le diable \\[0pt]
Le dialogue \\[0pt]
Le dialogue conduit-il à la vérité ? \\[0pt]
Le dialogue suffit-il à rompre la solitude ? \\[0pt]
Le discernement \\[0pt]
Le divertissement \\[0pt]
Le don \\[0pt]
Le don de soi \\[0pt]
Le don est-il une modalité de l'échange ? \\[0pt]
Le don et la dette \\[0pt]
Le don et l'échange \\[0pt]
Le don exclut-il le calcul ? \\[0pt]
Le doute \\[0pt]
Le droit \\[0pt]
Le droit à la différence met-il en péril l'égalité des droits ? \\[0pt]
Le droit à la paresse \\[0pt]
Le droit à la vérité \\[0pt]
Le droit au bonheur \\[0pt]
Le droit au travail \\[0pt]
Le droit de mentir \\[0pt]
Le droit de propriété \\[0pt]
Le droit de résistance \\[0pt]
Le droit divin \\[0pt]
Le droit doit-il être indépendant de la morale ? \\[0pt]
Le droit du plus faible \\[0pt]
Le droit est-il facteur de paix ? \\[0pt]
Le droit est-il le fondement de l'État ? \\[0pt]
Le droit est-il raison du plus fort ? \\[0pt]
Le droit et la convention \\[0pt]
Le droit et la force \\[0pt]
Le droit et la liberté \\[0pt]
Le droit et la loi \\[0pt]
Le droit et la morale \\[0pt]
Le droit et le devoir \\[0pt]
Le droit naturel \\[0pt]
Le droit ne peut-il se fonder sur des faits ? \\[0pt]
Le droit n'est-il qu'une justice par défaut ? \\[0pt]
Le droit n'est-il qu'un ensemble de conventions ? \\[0pt]
Le droit peut-il échapper à l'histoire ? \\[0pt]
Le droit peut-il être naturel ? \\[0pt]
Le droit peut-il se passer de la morale ? \\[0pt]
Le droit positif \\[0pt]
Le droit sert-il à établir l'ordre ou la justice ? \\[0pt]
L'éducation artistique \\[0pt]
L'éducation du goût est-elle la condition de l'expérience esthétique ? \\[0pt]
L'éducation esthétique \\[0pt]
Le fait \\[0pt]
Le fait divers \\[0pt]
Le fait et l'événement \\[0pt]
Le fantasme \\[0pt]
L'efficacité \\[0pt]
L'effort \\[0pt]
L'effort moral \\[0pt]
Le fini et l'infini \\[0pt]
Le for intérieur \\[0pt]
Le futur existe-t-il ? \\[0pt]
L'égalité \\[0pt]
L'égalité est-elle contre nature ? \\[0pt]
L'égalité est-elle l'ennemie de la liberté ? \\[0pt]
L'égalité est-elle toujours juste ? \\[0pt]
Légalité et légitimité \\[0pt]
Légalité et moralité \\[0pt]
Le génie \\[0pt]
Le génie est-il la marque de l'excellence artistique ? \\[0pt]
Le génie et la règle \\[0pt]
Le génie et le savant \\[0pt]
Le génie n'a-t-il sa place que dans les arts ? \\[0pt]
Le goût \\[0pt]
Le goût de la liberté \\[0pt]
Le goût des autres \\[0pt]
Le goût s'éduque-t-il ? \\[0pt]
Le gouvernement des hommes et l'administration des choses \\[0pt]
Le gouvernement des meilleurs \\[0pt]
Le hasard \\[0pt]
Le hasard et la nécessité \\[0pt]
Le hasard peut-il être un concept explicatif ?La morale doit-elle s'adapter à la réalité ? \\[0pt]
Le hors-la-loi \\[0pt]
Le je et le tu \\[0pt]
Le jeu \\[0pt]
Le jeu et le divertissement \\[0pt]
Le jeu et le hasard \\[0pt]
Le juge \\[0pt]
Le jugement \\[0pt]
Le jugement dernier \\[0pt]
Le juste et le légal \\[0pt]
Le laid \\[0pt]
Le langage \\[0pt]
Le langage des sciences \\[0pt]
Le langage du corps \\[0pt]
Le langage est-il d'essence poétique ? \\[0pt]
Le langage est-il le lieu de la vérité ? \\[0pt]
Le langage est-il le propre de l'homme ? \\[0pt]
Le langage est-il logique ? \\[0pt]
Le langage est-il naturel ? \\[0pt]
Le langage est-il une prise de possession des choses ? \\[0pt]
Le langage est-il un instrument de connaissance ? \\[0pt]
Le langage est-il un obstacle pour la pensée ? \\[0pt]
Le langage et la raison \\[0pt]
Le langage et la vérité \\[0pt]
Le langage et l'image \\[0pt]
Le langage masque-t-il la pensée ? \\[0pt]
Le langage n'est-il qu'un instrument de communication ? \\[0pt]
Le langage n'est-il qu'un outil ? \\[0pt]
Le langage permet-il seulement de communiquer ? \\[0pt]
Le langage rend-il l'homme plus puissant ? \\[0pt]
Le langage traduit-il la pensée ? \\[0pt]
Le langage trahit-il la pensée ? \\[0pt]
Le langage usuel est-il un obstacle à la connaissance ? \\[0pt]
L'élection \\[0pt]
Le législateur \\[0pt]
L'élémentaire \\[0pt]
Le libre arbitre \\[0pt]
Le libre échange \\[0pt]
Le lien social \\[0pt]
Le livre de la nature \\[0pt]
Le loisir \\[0pt]
Le luxe \\[0pt]
Le maintien de l'ordre \\[0pt]
Le maître \\[0pt]
Le mal \\[0pt]
Le mal a-t-il des raisons ? \\[0pt]
Le malentendu \\[0pt]
Le mal être \\[0pt]
Le mal existe-t-il ? \\[0pt]
Le malheur \\[0pt]
Le malheur donne-t-il le droit d'être injuste ? \\[0pt]
Le malheur est-il injuste ? \\[0pt]
Le mal peut-il être involontaire ? \\[0pt]
L'émancipation \\[0pt]
Le marché \\[0pt]
Le marché du travail \\[0pt]
Le mariage \\[0pt]
Le matérialisme \\[0pt]
Le mauvais goût \\[0pt]
L'embarras du choix \\[0pt]
Le méchant peut-il être heureux ? \\[0pt]
Le médiat et l'immédiat \\[0pt]
Le meilleur est-il l'ennemi du bien ? \\[0pt]
Le meilleur gouvernement est-il le gouvernement des meilleurs ? \\[0pt]
Le mensonge \\[0pt]
Le mensonge peut-il être au service de la vérité ? \\[0pt]
Le mépris \\[0pt]
Le mérite \\[0pt]
Le métier \\[0pt]
Le mien et le tien \\[0pt]
Le modèle \\[0pt]
Le moi \\[0pt]
Le moi est-il haïssable ? \\[0pt]
Le moi est-il une fiction ? \\[0pt]
Le moi est-il une illusion ? \\[0pt]
Le moindre mal \\[0pt]
Le moi n'est-il qu'une fiction ? \\[0pt]
Le monde de l'art \\[0pt]
Le monde de la technique \\[0pt]
Le monde du travail \\[0pt]
Le monde est-il autre chose que ce que j'en dis ? \\[0pt]
Le monde et la technique \\[0pt]
Le monde n'est-il que l'ensemble de ce qui existe ? \\[0pt]
Le monde n'est-il que ma représentation ? \\[0pt]
Le monde se réduit-il à ce que nous en voyons ? \\[0pt]
Le monstre \\[0pt]
Le monstrueux \\[0pt]
Le mot et le geste \\[0pt]
L'émotion \\[0pt]
Le mouvement \\[0pt]
Le multiple et l'un \\[0pt]
Le musée \\[0pt]
Le mystère \\[0pt]
Le naturel \\[0pt]
Le naturel et le fabriqué \\[0pt]
L'encyclopédie \\[0pt]
L'enfance \\[0pt]
L'enfance est-elle en nous ce qui doit être abandonné ? \\[0pt]
L'enfant \\[0pt]
L'engagement \\[0pt]
L'énigme, le mystère et le secret \\[0pt]
L'ennemi \\[0pt]
L'ennui \\[0pt]
Le non-être \\[0pt]
Le non-sens \\[0pt]
L'enquête empirique rend-elle la métaphysique inutile ? \\[0pt]
L'entendement et la volonté \\[0pt]
L'envie \\[0pt]
Le pardon \\[0pt]
Le pardon et l'oubli \\[0pt]
Le passé a-t-il plus de réalité que l'avenir ? \\[0pt]
Le passé a-t-il une réalité ? \\[0pt]
Le passé est-il ce qui a disparu ? \\[0pt]
Le passé et le présent \\[0pt]
Le passé existe-t-il ? \\[0pt]
Le paysage \\[0pt]
Le personnage et la personne \\[0pt]
Le pessimisme \\[0pt]
Le peuple \\[0pt]
Le peuple est-il un acteur politique ? \\[0pt]
Le peuple et la nation \\[0pt]
Le peuple peut-il se tromper ? \\[0pt]
L'éphémère \\[0pt]
L'éphémère a-t-il une valeur ? \\[0pt]
Le phénomène \\[0pt]
Le physique et le mental \\[0pt]
Le plaisir \\[0pt]
Le plaisir de parler \\[0pt]
Le plaisir des sens \\[0pt]
Le plaisir esthétique \\[0pt]
Le plaisir esthétique peut-il se partager ? \\[0pt]
Le plaisir est-il tout le bonheur ? \\[0pt]
Le plaisir et la peine \\[0pt]
Le plaisir et le contentement \\[0pt]
Le plaisir peut-il être partagé ? \\[0pt]
Le plaisir suffit-il au bonheur ? \\[0pt]
Le poids de la société \\[0pt]
Le poids du passé \\[0pt]
Le politique et l'économique \\[0pt]
Le possible et le réel \\[0pt]
Le pouvoir \\[0pt]
Le pouvoir corrompt-il toujours ? \\[0pt]
Le pouvoir de la science \\[0pt]
Le pouvoir des images \\[0pt]
Le pouvoir des mots \\[0pt]
Le pouvoir est-il une forme de domination ? \\[0pt]
Le pouvoir et l'autorité \\[0pt]
Le préscientifique \\[0pt]
Le présent \\[0pt]
Le présent historique \\[0pt]
L'épreuve du réel \\[0pt]
Le prince \\[0pt]
Le principe \\[0pt]
Le principe de raison \\[0pt]
Le principe de raison suffisante \\[0pt]
Le privé et le public \\[0pt]
Le prix du travail \\[0pt]
Le probable \\[0pt]
Le prochain et le semblable \\[0pt]
Le profit \\[0pt]
Le progrès \\[0pt]
Le progrès est-il un mythe ? \\[0pt]
Le progrès moral \\[0pt]
Le progrès technique est-il source de bonheur ? \\[0pt]
Le progrès technique peut-il être aliénant ? \\[0pt]
Le projet \\[0pt]
Le propre du vivant est-il de tomber malade ? \\[0pt]
Le provisoire \\[0pt]
Le public et le privé \\[0pt]
Le pur et l'impur \\[0pt]
L'équité \\[0pt]
L'équivocité \\[0pt]
L'équivoque \\[0pt]
Le quotidien \\[0pt]
Le raffinement \\[0pt]
Le rationnel et le raisonnable \\[0pt]
Le réalisme \\[0pt]
Le reconnaissance \\[0pt]
Le recours au symbole est-il la marque de la finitude de la pensée ? \\[0pt]
Le réel \\[0pt]
Le réel est-il ce que nous expérimentons ? \\[0pt]
Le réel est-il ce que nous percevons ? \\[0pt]
Le réel est-il ce qui apparaît ? \\[0pt]
Le réel est-il ce qui est perçu ? \\[0pt]
Le réel est-il inaccessible ? \\[0pt]
Le réel est-il l'objet de la science ? \\[0pt]
Le réel est-il objet d'interprétation ? \\[0pt]
Le réel est-il rationnel ? \\[0pt]
Le réel et la fiction \\[0pt]
Le réel et le matériel \\[0pt]
Le réel et le possible \\[0pt]
Le réel et le virtuel \\[0pt]
Le réel et le vrai \\[0pt]
Le réel et l'imaginaire \\[0pt]
Le réel et l'irréel \\[0pt]
Le réel n'est-il qu'un ensemble de contraintes ? \\[0pt]
Le réel résiste-t-il à la connaissance ? \\[0pt]
Le réel se limite-t-il à ce que font connaître les théories scientifiques ? \\[0pt]
Le regard \\[0pt]
Le relativisme \\[0pt]
Le renoncement \\[0pt]
Le respect \\[0pt]
Le respect n'est-il dû qu'aux personnes ? \\[0pt]
Le ressentiment \\[0pt]
Le retour à la nature \\[0pt]
Le rien \\[0pt]
Le risque \\[0pt]
Le risque de la liberté \\[0pt]
Le risque technologique \\[0pt]
Le rôle des théories est-il d'expliquer ou de décrire ? \\[0pt]
L'erreur \\[0pt]
L'erreur et la faute \\[0pt]
L'erreur et l'illusion \\[0pt]
Le rythme \\[0pt]
Le sacré \\[0pt]
Le sacré et le profane \\[0pt]
Le sacrifice \\[0pt]
Les acteurs de l'histoire en sont-ils les auteurs ? \\[0pt]
Les affects sont-ils déraisonnables ? \\[0pt]
Le sage a-t-il besoin d'autrui ? \\[0pt]
Le sage est-il insensible ? \\[0pt]
Les âges de la vie \\[0pt]
Le salaire \\[0pt]
Le salut vient-il de la raison ? \\[0pt]
Les animaux ont-ils des droits ? \\[0pt]
Les apparences sont-elles toujours trompeuses ? \\[0pt]
Les arts ont-ils pour fonction de divertir ? \\[0pt]
Le sauvage \\[0pt]
Le savant et l'ignorant \\[0pt]
Le savant et l'ingénieur \\[0pt]
Le savoir du corps \\[0pt]
Le savoir est-il une condition du bonheur ? \\[0pt]
Le savoir exclut-il toute forme de croyance ? \\[0pt]
Le savoir-faire \\[0pt]
Le savoir rend-il libre ? \\[0pt]
Le savoir total \\[0pt]
Les bêtes travaillent-elles ? \\[0pt]
« Les bons comptes font les bons amis » \\[0pt]
Les catégories \\[0pt]
Les causes et les raisons \\[0pt]
Les causes et les signes \\[0pt]
Les changements de théorie s'expliquent-ils seulement par le résultat des expériences ? \\[0pt]
Les choses ont-elles l'air de ce qu'elles sont ? \\[0pt]
Les classes sociales \\[0pt]
L'esclavage des passions \\[0pt]
Les coïncidences ont-elles des causes ? \\[0pt]
Les commencements \\[0pt]
Les comportements humains s'expliquent-il par l'instinct naturel ? \\[0pt]
Les concitoyens doivent-ils être des amis ? \\[0pt]
Les conditions d'existence \\[0pt]
Les conflits menacent-ils la société ? \\[0pt]
Les connaissances scientifiques sont-elles vraies ? \\[0pt]
Les considérations morales ont-elles leur place en politique ? \\[0pt]
Les démonstrations ont-elles des propriétés esthétiques ? \\[0pt]
Les devoirs de l'homme varient-ils selon la culture ? \\[0pt]
Les devoirs de l'homme varient-ils selon les cultures ? \\[0pt]
Les devoirs du citoyen \\[0pt]
Les droits de l'homme ont-ils besoin d'être fondés ? \\[0pt]
Les droits de l'individu \\[0pt]
Les échanges \\[0pt]
Les échanges économiques sont-ils facteurs de paix ? \\[0pt]
Les échanges favorisent-ils la paix ? \\[0pt]
Les échanges sont-ils facteurs de paix ? \\[0pt]
Les écrans \\[0pt]
Le secret \\[0pt]
Le sens caché \\[0pt]
Le sens commun \\[0pt]
Le sens de la justice \\[0pt]
Le sens de l'Histoire \\[0pt]
Le sens des mots \\[0pt]
Le sens de tout énoncé dépend-il de son interprétation ? \\[0pt]
Le sens du devoir \\[0pt]
Le sens du problème \\[0pt]
Le sensible \\[0pt]
Le sensible et l'intelligible \\[0pt]
Le sensible peut-il être connu ? \\[0pt]
Le sens interne \\[0pt]
Le sentiment \\[0pt]
Le sentiment de liberté \\[0pt]
Le sentiment d'injustice \\[0pt]
Le sentiment d'injustice est-il naturel ? \\[0pt]
Le sentiment du juste et de l'injuste \\[0pt]
Les êtres vivants sont-ils des machines ? \\[0pt]
Les faits et les valeurs \\[0pt]
Les faits existent-ils indépendamment de leur établissement par l'esprit humain ? \\[0pt]
Les faits parlent-ils d'eux-mêmes ? \\[0pt]
Les faits peuvent-ils faire autorité ? \\[0pt]
Les faits sont-ils des preuves ? \\[0pt]
Les faits sont-ils têtus ? \\[0pt]
« Les faits sont là » \\[0pt]
Les fins de la culture \\[0pt]
Les fins dernières \\[0pt]
Les formes du vivant \\[0pt]
Les frontières de l'art \\[0pt]
Les frontières du vivant \\[0pt]
Les générations \\[0pt]
Les générations futures ont-elles des droits ? \\[0pt]
Les habitudes nous forment-elles ? \\[0pt]
Les hommes naissent-ils libres ? \\[0pt]
Les hommes ont-ils besoin de maîtres ? \\[0pt]
Les hommes savent-ils ce qu'ils désirent ? \\[0pt]
Les hommes sont-ils seulement le produit de leur culture ? \\[0pt]
Les idées et les choses \\[0pt]
Les idées ont-elles une existence éternelle ? \\[0pt]
Le signe \\[0pt]
Le signe et l'indice \\[0pt]
Le silence \\[0pt]
Le silence a-t-il un sens ? \\[0pt]
Les images peuvent-elles mentir ? \\[0pt]
Le simple \\[0pt]
Le simple et le complexe \\[0pt]
Les inégalités menacent-elles la société ? \\[0pt]
Les inégalités sociales sont-elles naturelles ? \\[0pt]
Les intentions et les actes \\[0pt]
Les leçons de l'histoire \\[0pt]
Les limites de la connaissance \\[0pt]
Les limites de la raison \\[0pt]
Les limites de la science \\[0pt]
Les limites de l'expérience \\[0pt]
Les limites de l'interprétation \\[0pt]
Les lois \\[0pt]
Les lois de la nature \\[0pt]
Les lois naturelles \\[0pt]
Les machines nous rendent-elles libres ? \\[0pt]
Les machines permettent-elles de mieux connaître le corps humain ? \\[0pt]
Les maîtres de vérité \\[0pt]
Les mathématiques parlent-elles du réel ? \\[0pt]
Les mathématiques sont-elles une science ? \\[0pt]
Les mathématiques sont-elles un instrument ? \\[0pt]
Les mécanismes de la vie \\[0pt]
Les mœurs \\[0pt]
Les monstres \\[0pt]
Les mots disent-ils les choses ? \\[0pt]
Les mots et les concepts \\[0pt]
Les mots expriment-ils les choses ? \\[0pt]
Les mots nous éloignent-ils des autres ? \\[0pt]
Les mots nous éloignent-ils des choses ? \\[0pt]
Les mots parviennent-ils à tout exprimer ? \\[0pt]
Les mots sont-ils trompeurs ? \\[0pt]
Les moyens et les fins \\[0pt]
Les objets du désir \\[0pt]
Les œuvres d'art sont-elles des réalités comme les autres ? \\[0pt]
Le soi et le je \\[0pt]
Le soleil se lèvera-t-il demain ? \\[0pt]
Le solipsisme \\[0pt]
Le sommeil \\[0pt]
Le souci de soi \\[0pt]
Les outils \\[0pt]
Le souverain bien \\[0pt]
L'espace nous sépare-t-il ? \\[0pt]
Les paroles et les actes \\[0pt]
Le spectacle du monde \\[0pt]
L'espérance \\[0pt]
Les perceptions sont-elles des jugements ? \\[0pt]
Les personnages de fiction peuvent-ils avoir une réalité ? \\[0pt]
Les peuples font-ils l'histoire ? \\[0pt]
Les preuves de la liberté \\[0pt]
Les principes \\[0pt]
Les principes de la morale dépendent-ils de la culture ? \\[0pt]
L'esprit \\[0pt]
L'esprit critique \\[0pt]
L'esprit dépend-il du corps ? \\[0pt]
L'esprit de système \\[0pt]
L'esprit domine-t-il la matière ? \\[0pt]
L'esprit est-il chose ? \\[0pt]
L'esprit est-il irréductible à la matière ? \\[0pt]
L'esprit est-il mieux connu que le corps ? \\[0pt]
L'esprit est-il objet de science ? \\[0pt]
L'esprit est-il plus difficile à connaître que la matière ? \\[0pt]
L'esprit est-il plus facile à connaître que le corps ? \\[0pt]
L'esprit est-il une partie du corps ? \\[0pt]
L'esprit et la matière peuvent-ils agir l'un sur l'autre ? \\[0pt]
L'esprit humain progresse-t-il ? \\[0pt]
Les progrès techniques constituent-ils des progrès de la civilisation ? \\[0pt]
Les querelles de mots sont-elles toujours stériles ? \\[0pt]
Les raisons de croire \\[0pt]
Les règles de l'art \\[0pt]
Les règles de l'interprétation \\[0pt]
Les religions peuvent-elles prétendre libérer les hommes ? \\[0pt]
Les rêves sont-ils des pensées ? \\[0pt]
Les scélérats peuvent-ils être heureux ? \\[0pt]
Les sciences décrivent-elles le réel ? \\[0pt]
Les sciences de l'homme sont-elles des sciences ? \\[0pt]
Les sciences peuvent-elles se passer de fondements métaphysiques ? \\[0pt]
Les sciences peuvent-elles se passer d'images ? \\[0pt]
L'essence et l'existence \\[0pt]
Les sens jugent-ils ? \\[0pt]
Les sens sont-ils source d'illusion ? \\[0pt]
Les sens sont-ils trompeurs ? \\[0pt]
L'estime de soi \\[0pt]
Le style \\[0pt]
Le sublime est-il dans les choses ? \\[0pt]
Le succès peut-il être un critère de vérité ? \\[0pt]
Le sujet \\[0pt]
Le sujet et la personne \\[0pt]
Le sujet et l'individu \\[0pt]
Le sujet moral \\[0pt]
Le sujet n'est-il qu'une fiction ? \\[0pt]
Le sujet peut-il s'aliéner par un libre choix ? \\[0pt]
Les valeurs morales ont-elles leur origine dans la raison ? \\[0pt]
Les valeurs universelles \\[0pt]
Les vérités empiriques \\[0pt]
Les vérités éternelles \\[0pt]
Les vérités sont-elles intemporelles ? \\[0pt]
Les vertus du commerce \\[0pt]
Les vivants peuvent-ils se passer des morts ? \\[0pt]
Le symbole \\[0pt]
Le système \\[0pt]
Le système des arts \\[0pt]
Le tact \\[0pt]
Le talent \\[0pt]
L'État \\[0pt]
L'État a-t-il tous les droits ? \\[0pt]
L'État contribue-t-il à pacifier les relations entre les hommes ? \\[0pt]
L'État de droit \\[0pt]
L'état de nature \\[0pt]
L'État doit-il être fort ? \\[0pt]
L'État doit-il être le plus fort ? \\[0pt]
L'État doit-il reconnaître des limites à sa puissance ? \\[0pt]
L'État doit-il se mêler de religion ? \\[0pt]
L'État doit-il se préoccuper des arts ? \\[0pt]
L'État doit-il se préoccuper du bonheur des citoyens ? \\[0pt]
L'État est-il au service de la société ? \\[0pt]
L'État est-il la seule source du droit ? \\[0pt]
L'État est-il le garant du bien commun ? \\[0pt]
L'État est-il l'ennemi de la liberté ? \\[0pt]
L'État est-il nécessaire ? \\[0pt]
L'État est-il notre ennemi ? \\[0pt]
L'État est-il souverain ? \\[0pt]
L'État est-il un mal nécessaire ? \\[0pt]
L'État est-il un « monstre froid » ? \\[0pt]
L'État est-il un tiers impartial ? \\[0pt]
L'État et la justice \\[0pt]
L'État et la nation \\[0pt]
L'État et la société \\[0pt]
L'État et le droit \\[0pt]
L'État et le peuple \\[0pt]
L'État et les communautés \\[0pt]
L'État et l'individu \\[0pt]
L'État n'a-t-il d'existence que contractuelle ? \\[0pt]
L'État nous rend-il meilleurs ? \\[0pt]
L'État peut-il fonder sa propre légitimité ? \\[0pt]
L'État peut-il limiter son pouvoir ? \\[0pt]
L'État peut-il poursuivre une autre fin que sa propre puissance ? \\[0pt]
L'État peut-il se désintéresser de la religion ? \\[0pt]
L'État sert-il l'intérêt général ? \\[0pt]
Le technicien n'est-il qu'un exécutant ? \\[0pt]
Le témoignage \\[0pt]
Le temporel et le spirituel \\[0pt]
Le temps \\[0pt]
Le temps de la mémoire \\[0pt]
Le temps de la vie et le temps de l'existence \\[0pt]
Le temps des origines \\[0pt]
Le temps du bonheur \\[0pt]
Le temps du monde et le temps de la conscience \\[0pt]
Le temps est-il destructeur ? \\[0pt]
Le temps est-il en nous ou hors de nous ? \\[0pt]
Le temps est-il essentiellement destructeur ? \\[0pt]
Le temps est-il notre allié ? \\[0pt]
Le temps est-il une contrainte ? \\[0pt]
Le temps est-il une expérience de la continuité ou de la discontinuité ? \\[0pt]
Le temps est-il une réalité ? \\[0pt]
Le temps et la liberté \\[0pt]
Le temps et le bonheur \\[0pt]
Le temps et l'espace \\[0pt]
Le temps et l'éternité \\[0pt]
Le temps libre \\[0pt]
Le temps n'est-il pour l'homme que ce qui le limite ? \\[0pt]
Le temps nous appartient-il ? \\[0pt]
Le temps nous est-il compté ? \\[0pt]
Le temps perdu \\[0pt]
Le temps peut-il s'arrêter ? \\[0pt]
Le temps physique est-il comparable au temps psychique ? \\[0pt]
L'éternel retour \\[0pt]
L'éternité \\[0pt]
L'éternité des œuvres d'art \\[0pt]
L'éternité est-elle désirable ? \\[0pt]
L'étonnement \\[0pt]
Le toucher \\[0pt]
Le travail \\[0pt]
Le travail a-t-il une valeur morale ? \\[0pt]
Le travail de la pensée \\[0pt]
Le travail du droit \\[0pt]
Le travail est-il le propre de l'homme ? \\[0pt]
Le travail est-il libérateur ? \\[0pt]
Le travail est-il nécessaire au bonheur ? \\[0pt]
Le travail est-il toujours une activité productrice ? \\[0pt]
Le travail est-il un besoin ? \\[0pt]
Le travail est-il une marchandise ? \\[0pt]
Le travail est-il une valeur ? \\[0pt]
Le travail est-il un rapport naturel de l'homme à la nature ? \\[0pt]
Le travail et la propriété \\[0pt]
Le travail et la technique \\[0pt]
Le travail et le bonheur \\[0pt]
Le travail et le jeu \\[0pt]
Le travail et le loisir \\[0pt]
Le travail et le temps \\[0pt]
Le travail fonde-t-il la propriété ? \\[0pt]
Le travaille libère-t-il ? \\[0pt]
Le travail manuel \\[0pt]
Le travail manuel est-il sans pensée ? \\[0pt]
Le travail peut-il être solitaire ? \\[0pt]
Le travail unit-il ou sépare-t-il les hommes ? \\[0pt]
L'être et le néant \\[0pt]
L'être humain est-il la mesure de toute chose ? \\[0pt]
L'être imaginaire et l'être de raison \\[0pt]
Le tribunal de la raison \\[0pt]
Le tribunal de l'histoire \\[0pt]
Le troc \\[0pt]
L'étude de l'histoire conduit-elle à désespérer l'homme ? \\[0pt]
Le tyran \\[0pt]
L'événement \\[0pt]
Le vertige de la liberté \\[0pt]
Le vice \\[0pt]
Le vice et la vertu \\[0pt]
Le vide et le plein \\[0pt]
L'évidence \\[0pt]
L'évidence est-elle un critère de vérité ? \\[0pt]
L'évidence et la démonstration \\[0pt]
L'évidence se passe-t-elle de démonstration ? \\[0pt]
Le visage n'est-il qu'un masque ? \\[0pt]
Le visible et l'invisible \\[0pt]
Le vivant \\[0pt]
Le vivant définit-il ses propres normes ? \\[0pt]
Le vivant est-il réductible au physico-chimique ? \\[0pt]
Le vivant est-il un objet de science comme un autre ? \\[0pt]
Le vivant et la machine \\[0pt]
Le vivant et la mort \\[0pt]
Le vivant et la sensibilité \\[0pt]
Le vivant et la technique \\[0pt]
Le vivant et le vécu \\[0pt]
Le vivant et l'expérimentation \\[0pt]
Le vivant et l'inerte \\[0pt]
Le vivant et son milieu \\[0pt]
Le vivant se définit-il par sa résistance à la mort ? \\[0pt]
Le voile des mots \\[0pt]
Le volontaire et l'involontaire \\[0pt]
Le voyage \\[0pt]
Le vrai admet-il des degrés ? \\[0pt]
Le vrai a-t-il une histoire ? \\[0pt]
Le vrai et le bien \\[0pt]
Le vrai et le vraisemblable \\[0pt]
Le vraisemblable et le probable \\[0pt]
Le vrai se réduit-il à ce qui est vérifiable ? \\[0pt]
L'exactitude \\[0pt]
L'excellence des sens \\[0pt]
L'exception \\[0pt]
L'excès \\[0pt]
L'excuse \\[0pt]
L'exigence morale \\[0pt]
L'existence \\[0pt]
L'existence a-t-elle un sens ? \\[0pt]
L'existence du mal met-elle en échec la raison ? \\[0pt]
L'existence du passé \\[0pt]
L'existence du possible \\[0pt]
L'existence est-elle pensable ? \\[0pt]
L'existence est-elle vaine ? \\[0pt]
L'existence et le temps \\[0pt]
L'existence se laisse-t-elle penser ? \\[0pt]
L'existence se prouve-t-elle ? \\[0pt]
L'existence suppose-t-elle la conscience du temps ? \\[0pt]
L'expérience \\[0pt]
L'expérience a-t-elle le même sens dans toutes les sciences ? \\[0pt]
L'expérience d'autrui nous est-elle utile ? \\[0pt]
L'expérience de l'injustice \\[0pt]
L'expérience démontre-t-elle quelque chose ? \\[0pt]
L'expérience de pensée \\[0pt]
L'expérience du temps \\[0pt]
L'expérience esthétique relève-t-elle de la contemplation ? \\[0pt]
L'expérience et la sensation \\[0pt]
L'expérience imaginaire \\[0pt]
L'expérience instruit-elle ? \\[0pt]
L'expérience morale \\[0pt]
L'expérience permet-elle de départager des théories concurrentes ? \\[0pt]
L'expérience peut-elle avoir raison des principes ? \\[0pt]
L'expérience peut-elle contredire la théorie ? \\[0pt]
L'expérience religieuse \\[0pt]
L'expérience rend-elle raisonnable ? \\[0pt]
L'expérience rend-elle responsable ? \\[0pt]
L'expérience scientifique \\[0pt]
L'expérience suffit-elle pour établir une vérité ? \\[0pt]
L'expression \\[0pt]
L'extinction du désir \\[0pt]
L'extinction du désir est-elle le commencement du bonheur ? \\[0pt]
L'extrémité \\[0pt]
L'habitude \\[0pt]
L'habitude est-elle notre guide dans la vie ? \\[0pt]
L'harmonie \\[0pt]
L'hérédité \\[0pt]
L'héritage \\[0pt]
L'histoire \\[0pt]
L'Histoire \\[0pt]
L'histoire a-t-elle une fin ? \\[0pt]
L'histoire a-t-elle un sens ? \\[0pt]
L'histoire comme science \\[0pt]
L'histoire des sciences \\[0pt]
L'histoire du droit est-elle celle du progrès de la justice ? \\[0pt]
L'histoire est-elle la science du passé ? \\[0pt]
L'histoire est-elle la vérité de la mémoire ? \\[0pt]
L'histoire est-elle rationnelle ? \\[0pt]
L'histoire est-elle une explication ou une justification du passé ? \\[0pt]
L'histoire est-elle une science ? \\[0pt]
L'histoire est-elle un roman ? \\[0pt]
L'histoire et la raison \\[0pt]
« L'histoire jugera » : quel sens faut-il accorder à cette expression ? \\[0pt]
L'histoire n'est-elle que la connaissance du passé ? \\[0pt]
L'histoire n'est-elle qu'un récit ? \\[0pt]
L'histoire peut-elle ne pas avoir de sens ? \\[0pt]
L'histoire se répète-t-elle ? \\[0pt]
L'histoire sert-elle le présent ? \\[0pt]
L'historicité \\[0pt]
L'historien et le politique \\[0pt]
L'homme aime-t-il la justice pour elle-même ? \\[0pt]
L'homme a-t-il besoin de l'art ? \\[0pt]
L'homme a-t-il une place dans la nature ? \\[0pt]
L'homme des droits de l'homme \\[0pt]
L'homme d'État \\[0pt]
L'homme est-il chez lui dans l'univers ? \\[0pt]
L'homme est-il l'artisan de sa dignité ? \\[0pt]
L'homme est-il le sujet de son histoire ? \\[0pt]
L'homme est-il naturellement artiste ? \\[0pt]
L'homme est-il un animal ? \\[0pt]
L'homme est-il un animal comme un autre ? \\[0pt]
L'homme est-il un animal dénaturé ? \\[0pt]
L'homme est-il un animal politique ? \\[0pt]
L'homme est-il un animal rationnel ? \\[0pt]
L'homme est-il un animal social ? \\[0pt]
L'homme est-il un corps pensant ? \\[0pt]
L'homme est-il un loup pour l'homme ? \\[0pt]
L'homme et la machine \\[0pt]
L'homme et l'animal \\[0pt]
L'homme et le citoyen \\[0pt]
L'homme injuste est-il heureux ? \\[0pt]
L'homme injuste peut-il être heureux ? \\[0pt]
L'homme injuste peut-il être malheureux ? \\[0pt]
L'homme peut-il être inhumain ? \\[0pt]
L'homme se réalise-t-il dans le travail ? \\[0pt]
L'homme trouble-t-il l'ordre de la nature ? \\[0pt]
L'homme vertueux doit-il s'attendre à être malheureux ? \\[0pt]
L'honneur \\[0pt]
L'hospitalité \\[0pt]
L'humanité \\[0pt]
L'humanité a-t-elle une histoire ? \\[0pt]
L'humanité est-elle aimable ? \\[0pt]
L'humilité \\[0pt]
L'hypothèse \\[0pt]
L'hypothèse de l'inconscient \\[0pt]
Liberté d'agir, liberté de penser \\[0pt]
« Liberté, égalité, fraternité » \\[0pt]
Liberté et courage \\[0pt]
Liberté et déterminisme \\[0pt]
Liberté et éducation \\[0pt]
Liberté et égalité \\[0pt]
Liberté et engagement \\[0pt]
Liberté et existence \\[0pt]
Liberté et indépendance \\[0pt]
Liberté et libération \\[0pt]
Liberté et licence \\[0pt]
Liberté et nécessité \\[0pt]
Liberté et pouvoir \\[0pt]
Liberté et responsabilité \\[0pt]
Liberté et savoir \\[0pt]
Liberté et sécurité \\[0pt]
Liberté et solitude \\[0pt]
Liberté et transgression \\[0pt]
Libre arbitre et déterminisme sont-ils compatibles ? \\[0pt]
Libre et heureux \\[0pt]
L'idéal \\[0pt]
L'idéal et le réel \\[0pt]
L'idée de bioéthique \\[0pt]
L'idée de décadence \\[0pt]
L'idée de langue universelle \\[0pt]
L'idée de milieu \\[0pt]
L'idée de « nature » n'est-elle qu'un mythe ? \\[0pt]
L'idée de progrès \\[0pt]
L'idée de progrès historique est-elle devenue désuète ? \\[0pt]
L'idée de république \\[0pt]
L'idée d'inconscient \\[0pt]
L'idée d'organisme \\[0pt]
L'idée d'une pensée inconsciente est-elle contradictoire ? \\[0pt]
L'identité \\[0pt]
L'identité personnelle \\[0pt]
L'idéologie \\[0pt]
L'idiot \\[0pt]
L'ignorance \\[0pt]
L'ignorance est-elle préférable à l'erreur ? \\[0pt]
L'ignorance peut-elle être une excuse ? \\[0pt]
L'illusion \\[0pt]
L'illusion est-elle nécessaire au bonheur des hommes ? \\[0pt]
L'imagination a-t-elle sa place dans la connaissance scientifique ? \\[0pt]
L'imagination dans les sciences \\[0pt]
L'imagination enrichit-elle la connaissance ? \\[0pt]
L'imagination est-elle le refuge de la liberté ? \\[0pt]
L'imagination scientifique \\[0pt]
L'imitation \\[0pt]
Limite, borne, frontière \\[0pt]
L'immatériel \\[0pt]
L'immédiat \\[0pt]
L'immoralisme \\[0pt]
L'immortalité \\[0pt]
L'immortalité de l'âme \\[0pt]
L'impartialité \\[0pt]
L'impensable \\[0pt]
L'imperceptible \\[0pt]
L'impiété \\[0pt]
L'impossible \\[0pt]
L'imprévisible \\[0pt]
L'inaperçu \\[0pt]
L'inattendu \\[0pt]
L'incertitude \\[0pt]
L'incertitude interdit-elle de raisonner ? \\[0pt]
L'inconscience \\[0pt]
L'inconscience peut-elle être coupable ? \\[0pt]
L'inconscient \\[0pt]
L'inconscient est-il dans l'âme ou dans le corps ? \\[0pt]
L'inconscient est-il une excuse ? \\[0pt]
L'inconscient est-il un obstacle à la liberté ? \\[0pt]
L'inconscient et l'involontaire \\[0pt]
L'inconscient n'est-il qu'une hypothèse ? \\[0pt]
L'inconscient peut-il se manifester ? \\[0pt]
L'indécidable \\[0pt]
L'indéfini \\[0pt]
L'indémontrable \\[0pt]
L'indépendance \\[0pt]
L'indescriptible \\[0pt]
L'indésirable \\[0pt]
L'indice et la preuve \\[0pt]
L'indicible et l'impensable \\[0pt]
L'indicible et l'ineffable \\[0pt]
L'indifférence \\[0pt]
L'indifférence peut-elle être une vertu ? \\[0pt]
L'indignation \\[0pt]
L'indignité \\[0pt]
L'indiscutable \\[0pt]
L'individu a-t-il des droits ? \\[0pt]
L'individu et l'espèce \\[0pt]
L'induction \\[0pt]
L'indulgence \\[0pt]
L'ineffable et l'innommable \\[0pt]
L'inestimable \\[0pt]
L'inexistant \\[0pt]
L'infini et l'indéfini \\[0pt]
L'infinité de l'univers a-t-elle de quoi nous effrayer ? \\[0pt]
L'ingénieur et le politique \\[0pt]
L'ingratitude \\[0pt]
L'inhumain \\[0pt]
L'inimaginable \\[0pt]
L'injustifiable \\[0pt]
L'innocence \\[0pt]
L'innovation \\[0pt]
L'innovation technique nous impose-t-elle de nouveaux devoirs ? \\[0pt]
L'inoubliable \\[0pt]
L'inquiétude \\[0pt]
L'inquiétude de l'avenir \\[0pt]
L'inquiétude peut-elle définir l'existence humaine ? \\[0pt]
L'inquiétude peut-elle devenir l'existence humaine ? \\[0pt]
L'insatisfaction \\[0pt]
L'insouciance \\[0pt]
L'instant \\[0pt]
L'instant et la durée \\[0pt]
L'instruction est-elle facteur de moralité ? \\[0pt]
L'instrument \\[0pt]
L'intellect \\[0pt]
L'intelligence \\[0pt]
L'intelligence artificielle \\[0pt]
L'intelligence de la technique \\[0pt]
L'intelligence peut-elle être artificielle ? \\[0pt]
L'intelligence peut-elle être inhumaine ? \\[0pt]
L'intemporel \\[0pt]
L'interdit \\[0pt]
L'interdit est-il au fondement de la culture ? \\[0pt]
L'intérêt constitue-t-il l'unique lien social ? \\[0pt]
L'intérêt de la justice \\[0pt]
L'intérêt de la société l'emporte-t-il sur celui des individus ? \\[0pt]
L'intérêt de l'État \\[0pt]
L'intérêt des machines \\[0pt]
L'intérêt est-il le principe de tout échange ? \\[0pt]
L'intérêt général est-il la somme des intérêts particuliers ? \\[0pt]
L'intériorité \\[0pt]
L'interprétation \\[0pt]
L'interprétation est-elle un art ? \\[0pt]
L'interprétation est-elle un art ou une science ? \\[0pt]
L'interprétation est-elle une activité sans fin ? \\[0pt]
L'interprétation est-elle une forme de connaissance ? \\[0pt]
L'interprétation : mode d'être ou mode de connaissance ? \\[0pt]
L'interprète et le créateur \\[0pt]
L'interprète sait-il ce qu'il cherche ? \\[0pt]
L'intersubjectivité \\[0pt]
L'intimité \\[0pt]
L'intolérable \\[0pt]
L'introspection \\[0pt]
L'intuition \\[0pt]
L'intuition intellectuelle \\[0pt]
L'inutile \\[0pt]
L'inutile est-il sans valeur ? \\[0pt]
L'invention \\[0pt]
L'invention de soi \\[0pt]
L'invention et la découverte \\[0pt]
L'invention technique \\[0pt]
L'invérifiable \\[0pt]
L'invisible \\[0pt]
L'involontaire \\[0pt]
Lire et écrire \\[0pt]
L'ironie \\[0pt]
L'irrationnel \\[0pt]
L'irrationnel est-il pensable ? \\[0pt]
L'irrationnel est-il toujours absurde ? \\[0pt]
L'irrationnel et l'absurde \\[0pt]
L'irrationnel existe-t-il ? \\[0pt]
L'irréel \\[0pt]
L'irréfléchi \\[0pt]
L'irréfutable \\[0pt]
L'irréparable \\[0pt]
L'irrésolution \\[0pt]
L'irrespect \\[0pt]
L'irresponsabilité \\[0pt]
L'irréversibilité \\[0pt]
L'irréversible \\[0pt]
L'obéissance \\[0pt]
L'obéissance est-elle compatible avec la liberté ? \\[0pt]
L'objectivité \\[0pt]
L'objectivité de l'historien \\[0pt]
L'objet d'amour \\[0pt]
L'objet de la physique \\[0pt]
L'objet du désir \\[0pt]
L'objet du désir en est-il la cause ? \\[0pt]
L'objet et la chose \\[0pt]
L'objet technique \\[0pt]
L'obligation \\[0pt]
L'obscur \\[0pt]
L'observation \\[0pt]
L'occasion \\[0pt]
Locke \\[0pt]
L'œuvre \\[0pt]
L'œuvre d'art a-t-elle un sens ? \\[0pt]
L'œuvre d'art donne-t-elle à penser ? \\[0pt]
L'œuvre d'art échappe-t-elle au temps ? \\[0pt]
L'œuvre d'art échappe-t-elle nécessairement au temps ? \\[0pt]
L'œuvre d'art est-elle une marchandise ? \\[0pt]
L'œuvre d'art est-elle un monde ? \\[0pt]
L'œuvre d'art est-elle un objet d'échange ? \\[0pt]
L'œuvre d'art est-elle un symbole ? \\[0pt]
L'œuvre d'art instruit-elle ? \\[0pt]
Loi scientifique et causalité \\[0pt]
Loisir et oisiveté \\[0pt]
L'oisiveté \\[0pt]
L'omniscience \\[0pt]
L'opinion a-t-elle nécessairement tort ? \\[0pt]
L'opinion est-elle un savoir ? \\[0pt]
L'ordre des choses \\[0pt]
L'ordre des échanges est-il un ordre rationnel ? \\[0pt]
L'ordre du monde \\[0pt]
L'ordre et le désordre \\[0pt]
L'ordre social \\[0pt]
L'ordre social peut-il être juste ? \\[0pt]
L'organique \\[0pt]
L'organique et l'inorganique \\[0pt]
L'organisme \\[0pt]
L'originalité \\[0pt]
L'origine des idées \\[0pt]
L'oubli \\[0pt]
L'oubli et le pardon \\[0pt]
L'outil \\[0pt]
L'outil et la machine \\[0pt]
L'outil, l'instrument, la machine \\[0pt]
L'ouverture d'esprit \\[0pt]
L'unanimité est-elle un critère de vérité ? \\[0pt]
L'unité de l'État \\[0pt]
L'universel \\[0pt]
L'universel et le particulier \\[0pt]
L'universel et le singulier \\[0pt]
L'urbanité \\[0pt]
L'urgence \\[0pt]
L'urgence de vivre \\[0pt]
L'usage du doute \\[0pt]
L'utile et l'agréable \\[0pt]
L'utile et le beau \\[0pt]
L'utile et le bon \\[0pt]
L'utile et l'inutile \\[0pt]
L'utilité \\[0pt]
L'utopie et l'idéologie \\[0pt]
Machine et organisme \\[0pt]
Ma conscience est-elle digne de confiance ? \\[0pt]
Maître et disciple \\[0pt]
Maîtrise et puissance \\[0pt]
Maladie de l'âme et maladie du corps \\[0pt]
Mal et liberté \\[0pt]
Ma liberté s'arrête-t-elle où commence celle des autres ? \\[0pt]
Mémoire et souvenir \\[0pt]
Mentir \\[0pt]
Mérite-t-on d'être heureux ? \\[0pt]
Mettre en commun \\[0pt]
Mettre en ordre \\[0pt]
Modèle et copie \\[0pt]
Mon corps \\[0pt]
Mon corps est-il naturel ? \\[0pt]
Mon corps fait-il obstacle à ma liberté ? \\[0pt]
Mon passé m'appartient-il ? \\[0pt]
Mon prochain est-il mon semblable ? \\[0pt]
Mon semblable \\[0pt]
Montrer et démontrer \\[0pt]
Morale et calcul \\[0pt]
Morale et économie \\[0pt]
Morale et liberté \\[0pt]
Moralité et utilité \\[0pt]
Naît-on sujet ou le devient-on ? \\[0pt]
N'apprend-on que par l'expérience ? \\[0pt]
Narration et identité \\[0pt]
Nature et artifice \\[0pt]
Nature et convention \\[0pt]
Nature et histoire \\[0pt]
Nature et loi \\[0pt]
Nature et morale \\[0pt]
Nécessité et fatalité \\[0pt]
N'échange-t-on que ce qui a de la valeur ? \\[0pt]
N'échange-t-on que par intérêt ? \\[0pt]
Ne faire que son devoir \\[0pt]
Ne rien devoir à personne \\[0pt]
N'est-il d'inférence que démonstrative ? \\[0pt]
Ne travaille-t-on que par nécessité ? \\[0pt]
Ne veut-on que ce qui est désirable ? \\[0pt]
Ne vit-on bien qu'avec ses amis ? \\[0pt]
N'exprime-t-on que ce dont on a conscience ? \\[0pt]
Nier le libre arbitre, est-ce nier la liberté ? \\[0pt]
N'interprète-t-on que ce qui est équivoque ? \\[0pt]
Nommer \\[0pt]
Normes et valeurs \\[0pt]
Nos convictions morales sont-elles le simple reflet de notre temps ? \\[0pt]
Nos désirs nous appartiennent-ils ? \\[0pt]
Nos paroles peuvent-elles dépasser notre pensée ? \\[0pt]
Nos pensées dépendent-elles de nous ? \\[0pt]
Nos pensées sont-elles entièrement en notre pouvoir ? \\[0pt]
Nos sens nous font-ils connaître la matière ? \\[0pt]
Notre liberté de pensée a-t-elle des limites ? \\[0pt]
Notre rapport au monde est-il essentiellement technique ? \\[0pt]
Notre rapport au monde peut-il être exclusivement technique ? \\[0pt]
Nous trouvons-nous nous-mêmes dans l'animal ? \\[0pt]
Nouveauté et tradition \\[0pt]
N'y a-t-il de bonheur que dans l'instant ? \\[0pt]
N'y a-t-il de devoirs qu'envers autrui ? \\[0pt]
N'y a-t-il de droit qu'écrit ? \\[0pt]
N'y a-t-il de foi que religieuse ? \\[0pt]
N'y a-t-il de réalité que de l'individuel ? \\[0pt]
N'y a-t-il de savoir que livresque ? \\[0pt]
N'y a-t-il de science que de ce qui est mathématisable ? \\[0pt]
N'y a-t-il de vérité que vérifiable ? \\[0pt]
N'y a-t-il de vérités que scientifiques ? \\[0pt]
N'y a-t-il de vrai que le vérifiable ? \\[0pt]
N'y a-t-il que des individus ? \\[0pt]
Obéissance et liberté \\[0pt]
Obéissance et soumission \\[0pt]
Objectivé et subjectivité \\[0pt]
Objectivité et impartialité \\[0pt]
Observer et expérimenter \\[0pt]
Observer et interpréter \\[0pt]
Ontologie et métaphysique \\[0pt]
Opinion et ignorance \\[0pt]
Ordre et désordre \\[0pt]
Ordre et justice \\[0pt]
Ordre et liberté \\[0pt]
Ordre et progrès \\[0pt]
Organisme et milieu \\[0pt]
Origine et fondement \\[0pt]
Où commence la violence ? \\[0pt]
Où commence ma liberté ? \\[0pt]
Où est l'esprit ? \\[0pt]
Outil et machine \\[0pt]
Outil et organe \\[0pt]
Paraître \\[0pt]
Parier \\[0pt]
Par le langage, peut-on agir sur la réalité ? \\[0pt]
Parler, est-ce agir ? \\[0pt]
Parler, est-ce communiquer ? \\[0pt]
Parler, est-ce donner sa parole ? \\[0pt]
Parler et agir \\[0pt]
Parler, n'est-ce que désigner ? \\[0pt]
Parler vrai \\[0pt]
Parole et pouvoir \\[0pt]
Passions et intérêts \\[0pt]
Penser, est-ce calculer ? \\[0pt]
Penser, est-ce désobéir ? \\[0pt]
Penser, est-ce se parler à soi-même ? \\[0pt]
Penser et connaître \\[0pt]
Penser et imaginer \\[0pt]
Penser et parler \\[0pt]
Penser et savoir \\[0pt]
Penser et sentir \\[0pt]
Penser l'avenir \\[0pt]
Penser le changement \\[0pt]
Penser l'infini \\[0pt]
Penser par soi-même \\[0pt]
Penser par soi-même, est-ce être l'auteur de ses pensées ? \\[0pt]
Penser peut-il nous rendre heureux ? \\[0pt]
Pense-t-on jamais seul ? \\[0pt]
Perception et connaissance \\[0pt]
Perception et imagination \\[0pt]
Perception et interprétation \\[0pt]
Perception et sensation \\[0pt]
Percevoir \\[0pt]
Percevoir, est-ce interpréter ? \\[0pt]
Percevoir, est-ce nécessaire pour penser ? \\[0pt]
Percevoir, est-ce savoir ? \\[0pt]
Percevoir, est-ce s'ouvrir au monde ? \\[0pt]
Percevoir et concevoir \\[0pt]
Percevoir et imaginer \\[0pt]
Percevoir et témoigner \\[0pt]
Perçoit-on le réel tel qu'il est ? \\[0pt]
Perçoit-on les choses comme elles sont ? \\[0pt]
Perdre la raison \\[0pt]
Perdre ses illusions \\[0pt]
Perdre son identité \\[0pt]
Permanence et identité \\[0pt]
Personne et individu \\[0pt]
Peuple et multitude \\[0pt]
Peut-il exister une action désintéressée ? \\[0pt]
Peut-il y avoir conflit entre nos devoirs ? \\[0pt]
Peut-il y avoir de bonnes raisons de croire ? \\[0pt]
Peut-il y avoir de bons tyrans ? \\[0pt]
Peut-il y avoir des conflits de devoirs ? \\[0pt]
Peut-il y avoir des échanges équitables ? \\[0pt]
Peut-il y avoir des lois de l'histoire ? \\[0pt]
Peut-il y avoir des lois injustes ? \\[0pt]
Peut-il y avoir des modèles en morale ? \\[0pt]
Peut-il y avoir des vérités partielles ? \\[0pt]
Peut-il y avoir esprit sans corps ? \\[0pt]
Peut-il y avoir plusieurs vérités religieuses ? \\[0pt]
Peut-il y avoir savoir-faire sans savoir ? \\[0pt]
Peut-il y avoir un bien commun ? \\[0pt]
Peut-il y avoir une science de la morale ? \\[0pt]
Peut-il y avoir une science de l'inconscient ? \\[0pt]
Peut-il y avoir une société sans État ? \\[0pt]
Peut-il y avoir un État mondial ? \\[0pt]
Peut-il y avoir une vérité en art ? \\[0pt]
Peut-il y avoir un langage universel ? \\[0pt]
Peut-on aimer l'autre tel qu'il est ? \\[0pt]
Peut-on aimer sans perdre sa liberté ? \\[0pt]
Peut-on aimer son prochain comme soi-même ? \\[0pt]
Peut-on aimer une œuvre d'art sans la comprendre ? \\[0pt]
Peut-on aller à l'encontre de la nature ? \\[0pt]
Peut-on apprendre à mourir ? \\[0pt]
Peut-on apprendre à vivre ? \\[0pt]
Peut-on assimiler le vivant à une machine ? \\[0pt]
Peut-on atteindre une certitude ? \\[0pt]
Peut-on attendre de la politique qu'elle soit conforme aux exigences de la raison ? \\[0pt]
Peut-on attribuer à chacun son dû ? \\[0pt]
Peut-on avoir de bonnes raisons de ne pas dire la vérité ? \\[0pt]
Peut-on avoir peur de la liberté ? \\[0pt]
Peut-on avoir peur de soi-même ? \\[0pt]
Peut-on avoir raison contre la science ? \\[0pt]
Peut-on avoir raison contre les faits ? \\[0pt]
Peut-on avoir raison contre tout le monde ? \\[0pt]
Peut-on avoir raison tout seul ? \\[0pt]
Peut-on cesser de croire ? \\[0pt]
Peut-on cesser de désirer ? \\[0pt]
Peut-on changer au point de ne plus être le même ? \\[0pt]
Peut-on changer le monde ? \\[0pt]
Peut-on changer ses désirs ? \\[0pt]
Peut-on changer ses désirs à volonté ? \\[0pt]
Peut-on choisir ses désirs ? \\[0pt]
Peut-on combattre une croyance par le raisonnement ? \\[0pt]
Peut-on commander à la nature ? \\[0pt]
Peut-on communiquer ses perceptions à autrui ? \\[0pt]
Peut-on communiquer son expérience ? \\[0pt]
Peut-on comparer les cultures ? \\[0pt]
Peut-on comparer l'organisme à une machine ? \\[0pt]
Peut-on comprendre le présent ? \\[0pt]
Peut-on comprendre un acte que l'on désapprouve ? \\[0pt]
Peut-on comprendre un auteur mieux qu'il ne s'est compris lui-même ? \\[0pt]
Peut-on concevoir une humanité sans art ? \\[0pt]
Peut-on concevoir une science sans expérience ? \\[0pt]
Peut-on concevoir une société juste sans que les hommes ne le soient ? \\[0pt]
Peut-on concevoir une société sans État ? \\[0pt]
Peut-on concilier bonheur et liberté ? \\[0pt]
Peut-on connaître les choses telles qu'elles sont ? \\[0pt]
Peut-on connaître l'esprit ? \\[0pt]
Peut-on connaître le vivant sans le dénaturer ? \\[0pt]
Peut-on connaître le vivant sans recourir à la notion de finalité ? \\[0pt]
Peut-on connaître l'individuel ? \\[0pt]
Peut-on connaître par intuition ? \\[0pt]
Peut-on contredire l'expérience ? \\[0pt]
Peut-on craindre la liberté ? \\[0pt]
Peut-on critiquer la démocratie ? \\[0pt]
Peut-on croire en rien ? \\[0pt]
Peut-on décider d'être heureux ? \\[0pt]
Peut-on définir la matière ? \\[0pt]
Peut-on définir la morale comme l'art d'être heureux ? \\[0pt]
Peut-on définir le bonheur ? \\[0pt]
Peut-on délimiter le réel ? \\[0pt]
Peut-on démontrer une théorie ? \\[0pt]
Peut-on dépasser la subjectivité ? \\[0pt]
Peut-on désirer autre chose qu'être heureux ? \\[0pt]
Peut-on désirer ce qui est ? \\[0pt]
Peut-on désirer ce qu'on sait être impossible ? \\[0pt]
Peut-on désirer l'impossible ? \\[0pt]
Peut-on désobéir aux lois ? \\[0pt]
Peut-on désobéir par devoir ? \\[0pt]
Peut-on dire ce que l'on pense ? \\[0pt]
Peut-on dire d'un homme qu'il est supérieur à un autre homme ? \\[0pt]
Peut-on dire la vérité ? \\[0pt]
Peut-on dire le singulier ? \\[0pt]
Peut-on dire que la science désenchante le monde ? \\[0pt]
Peut-on dire que les machines travaillent pour nous ? \\[0pt]
Peut-on dire que les mots pensent pour nous ? \\[0pt]
Peut-on dire que l'humanité progresse ? \\[0pt]
Peut-on dire que toutes les croyances se valent ? \\[0pt]
Peut-on discuter des goûts et des couleurs ? \\[0pt]
Peut-on distinguer les faits de leurs interprétations ? \\[0pt]
Peut-on donner un sens à son existence ? \\[0pt]
Peut-on douter de sa propre existence ? \\[0pt]
Peut-on douter de soi ? \\[0pt]
Peut-on douter de tout ? \\[0pt]
Peut-on douter de toute vérité ? \\[0pt]
Peut-on échanger des idées ? \\[0pt]
Peut-on échapper à son temps ? \\[0pt]
Peut-on échapper aux relations de pouvoir ? \\[0pt]
Peut-on éduquer la conscience ? \\[0pt]
Peut-on éduquer la sensibilité ? \\[0pt]
Peut-on en appeler à la conscience contre l'État ? \\[0pt]
Peut-on encore soutenir que l'homme est un animal rationnel ? \\[0pt]
Peut-on espérer être libéré du travail ? \\[0pt]
Peut-on être à la fois lucide et heureux ? \\[0pt]
Peut-on être apolitique ? \\[0pt]
Peut-on être assuré d'avoir raison ? \\[0pt]
Peut-on être dans le présent ? \\[0pt]
Peut-on être en conflit avec soi-même ? \\[0pt]
Peut-on être esclave de soi-même ? \\[0pt]
Peut-on être heureux dans la solitude ? \\[0pt]
Peut-on être heureux sans être sage ? \\[0pt]
Peut-on être heureux sans s'en rendre compte ? \\[0pt]
Peut-on être ignorant ? \\[0pt]
Peut-on être injuste envers soi-même ? \\[0pt]
Peut-on être juste sans être impartial ? \\[0pt]
Peut-on être méchant volontairement ? \\[0pt]
Peut-on être obligé d'aimer ? \\[0pt]
Peut-on être plus ou moins libre ? \\[0pt]
Peut-on être sage inconsciemment ? \\[0pt]
Peut-on être sûr d'avoir raison ? \\[0pt]
Peut-on être sûr de bien agir ? \\[0pt]
Peut-on être sûr de ne pas se tromper ? \\[0pt]
Peut-on être trop sensible ? \\[0pt]
Peut-on étudier le passé de façon objective ? \\[0pt]
Peut-on expliquer une œuvre d'art ? \\[0pt]
Peut-on faire de la politique sans supposer les hommes méchants ? \\[0pt]
Peut-on faire de l'esprit un objet de science ? \\[0pt]
Peut-on faire la philosophie de l'histoire ? \\[0pt]
Peut-on faire le bien d'autrui malgré lui ? \\[0pt]
Peut-on faire l'éloge de l'illusion ? \\[0pt]
Peut-on faire le mal innocemment ? \\[0pt]
Peut-on faire l'expérience de la nécessité ? \\[0pt]
Peut-on faire son devoir par habitude ? \\[0pt]
Peut-on faire table rase du passé ? \\[0pt]
Peut-on faire un mauvais usage de la raison ? \\[0pt]
Peut-on feindre la vertu ? \\[0pt]
Peut-on fonder la morale ? \\[0pt]
Peut-on fonder le droit sur la morale ? \\[0pt]
Peut-on fonder un droit de désobéir ? \\[0pt]
Peut-on fonder une éthique sur la biologie ? \\[0pt]
Peut-on fonder une morale sur le plaisir ? \\[0pt]
Peut-on fuir la société ? \\[0pt]
Peut-on haïr la raison ? \\[0pt]
Peut-on haïr la vie ? \\[0pt]
Peut-on haïr les images ? \\[0pt]
Peut-on ignorer sa propre liberté ? \\[0pt]
Peut-on ignorer volontairement la vérité ? \\[0pt]
Peut-on imaginer l'avenir ? \\[0pt]
Peut-on imaginer un langage universel ? \\[0pt]
Peut-on juger autrui ? \\[0pt]
Peut-on justifier la violence ? \\[0pt]
Peut-on justifier le mal ? \\[0pt]
Peut-on maîtriser la nature ? \\[0pt]
Peut-on maîtriser l'évolution de la technique ? \\[0pt]
Peut-on manquer de volonté ? \\[0pt]
Peut-on mentir par humanité ? \\[0pt]
Peut-on mesurer le temps ? \\[0pt]
Peut-on moraliser la guerre ? \\[0pt]
Peut-on ne croire en rien ? \\[0pt]
Peut-on ne pas croire ? \\[0pt]
Peut-on ne pas croire à la science ? \\[0pt]
Peut-on ne pas croire au progrès ? \\[0pt]
Peut-on ne pas être égoïste ? \\[0pt]
Peut-on ne pas être soi-même ? \\[0pt]
Peut-on ne pas rechercher le bonheur ? \\[0pt]
Peut-on ne pas savoir ce que l'on dit ? \\[0pt]
Peut-on ne pas savoir ce que l'on fait ? \\[0pt]
Peut-on ne penser à rien ? \\[0pt]
Peut-on ne rien devoir à personne ? \\[0pt]
Peut-on nier le réel ? \\[0pt]
Peut-on nier l'évidence ? \\[0pt]
Peut-on nier l'existence de la matière ? \\[0pt]
Peut-on opposer connaissance scientifique et création artistique ? \\[0pt]
Peut-on opposer le loisir au travail ? \\[0pt]
Peut-on ôter à l'homme sa liberté ? \\[0pt]
Peut-on parler de conscience collective ? \\[0pt]
Peut-on parler de dialogue des cultures ? \\[0pt]
Peut-on parler de mondes imaginaires ? \\[0pt]
Peut-on parler de « nature humaine » ? \\[0pt]
Peut-on parler de nourriture spirituelle ? \\[0pt]
Peut-on parler de problèmes techniques ? \\[0pt]
Peut-on parler des miracles de la technique ? \\[0pt]
Peut-on parler de travail intellectuel ? \\[0pt]
Peut-on parler de vérité en art ? \\[0pt]
Peut-on parler de vérité subjective ? \\[0pt]
Peut-on parler de violence d'État ? \\[0pt]
Peut-on parler d'une morale collective ? \\[0pt]
Peut-on parler d'une religion de l'humanité ? \\[0pt]
Peut-on parler d'univers symbolique ? \\[0pt]
Peut-on parler d'un progrès de la liberté ? \\[0pt]
Peut-on parler pour ne rien dire ? \\[0pt]
Peut-on parler sans incohérence d'une libre obéissance ? \\[0pt]
Peut-on penser ce qu'on ne peut dire ? \\[0pt]
Peut-on penser contre l'expérience ? \\[0pt]
Peut-on penser la justice comme une compétence ? \\[0pt]
Peut-on penser la matière ? \\[0pt]
Peut-on penser la vie ? \\[0pt]
Peut-on penser la vie sans penser la mort ? \\[0pt]
Peut-on penser l'homme à partir de la nature ? \\[0pt]
Peut-on penser l'infini ? \\[0pt]
Peut-on penser sans image ? \\[0pt]
Peut-on penser sans images ? \\[0pt]
Peut-on penser sans les mots ? \\[0pt]
Peut-on penser sans méthode ? \\[0pt]
Peut-on penser sans préjugés ? \\[0pt]
Peut-on penser une fin de l'histoire ? \\[0pt]
Peut-on percevoir sans juger ? \\[0pt]
Peut-on perdre la raison ? \\[0pt]
Peut-on perdre sa liberté ? \\[0pt]
Peut-on perdre son temps ? \\[0pt]
Peut-on prédire les événements ? \\[0pt]
Peut-on préférer le bonheur à la vérité ? \\[0pt]
Peut-on préférer l'injustice au désordre ? \\[0pt]
Peut-on protéger les libertés sans les réduire ? \\[0pt]
Peut-on prouver la réalité de l'esprit ? \\[0pt]
Peut-on prouver l'existence ? \\[0pt]
Peut-on prouver une existence ? \\[0pt]
Peut-on raconter sa vie ? \\[0pt]
Peut-on recommencer sa vie ? \\[0pt]
Peut-on réduire le raisonnement au calcul ? \\[0pt]
Peut-on refuser la violence ? \\[0pt]
Peut-on refuser le bonheur ? \\[0pt]
Peut-on refuser l'évidence ? \\[0pt]
Peut-on refuser le vrai ? \\[0pt]
Peut-on réfuter le scepticisme ? \\[0pt]
Peut-on rejeter le principe de non-contradiction ? \\[0pt]
Peut-on rendre raison des émotions ? \\[0pt]
Peut-on rendre raison de tout ? \\[0pt]
Peut-on renoncer à être libre ? \\[0pt]
Peut-on renoncer à la liberté ? \\[0pt]
Peut-on renoncer à la vérité ? \\[0pt]
Peut-on renoncer au bonheur ? \\[0pt]
Peut-on répondre d'autrui ? \\[0pt]
Peut-on reprocher au langage d'être équivoque ? \\[0pt]
Peut-on reprocher au langage d'être parfait ? \\[0pt]
Peut-on résister au vrai ? \\[0pt]
Peut-on rompre avec la société ? \\[0pt]
Peut-on rompre avec le passé ? \\[0pt]
Peut-on s'affranchir de la subjectivité ? \\[0pt]
Peut-on s'affranchir des lois ? \\[0pt]
Peut-on s'attendre à tout ? \\[0pt]
Peut-on savoir sans croire ? \\[0pt]
Peut-on se choisir un destin ? \\[0pt]
Peut-on se connaître soi-même ? \\[0pt]
Peut-on se gouverner soi-même ? \\[0pt]
Peut-on se mentir à soi-même ? \\[0pt]
Peut-on se mettre à la place d'autrui ? \\[0pt]
Peut-on se mettre à la place de l'autre ? \\[0pt]
Peut-on sentir le temps qui passe ? \\[0pt]
Peut-on séparer la politique de l'exigence de vérité ? \\[0pt]
Peut-on se passer de croyances ? \\[0pt]
Peut-on se passer de l'État ? \\[0pt]
Peut-on se passer de l'idée de droit naturel ? \\[0pt]
Peut-on se passer de maître ? \\[0pt]
Peut-on se passer de métaphysique ? \\[0pt]
Peut-on se passer de religion ? \\[0pt]
Peut-on se passer d'État ? \\[0pt]
Peut-on se passer de technique ? \\[0pt]
Peut-on se passer de toute religion ? \\[0pt]
Peut-on se passer d'idéal ? \\[0pt]
Peut-on se passer d'un maître ? \\[0pt]
Peut-on se prescrire une loi ? \\[0pt]
Peut-on s'obliger soi-même ? \\[0pt]
Peut-on sympathiser avec l'ennemi ? \\[0pt]
Peut-on tirer des leçons de l'histoire ? \\[0pt]
Peut-on tolérer l'injustice ? \\[0pt]
Peut-on toujours faire ce qu'on doit ? \\[0pt]
Peut-on toujours raison garder ? \\[0pt]
Peut-on tout analyser ? \\[0pt]
Peut-on tout attendre de l'État ? \\[0pt]
Peut-on tout démontrer ? \\[0pt]
Peut-on tout dire ? \\[0pt]
Peut-on tout donner ? \\[0pt]
Peut-on tout échanger ? \\[0pt]
Peut-on tout interpréter ? \\[0pt]
Peut-on tout ordonner ? \\[0pt]
Peut-on trancher les conflits entre interprétations ? \\[0pt]
Peut-on vivre en sceptique ? \\[0pt]
Peut-on vivre hors du temps ? \\[0pt]
Peut-on vivre pour la vérité ? \\[0pt]
Peut-on vivre sans échange ? \\[0pt]
Peut-on vivre sans le plaisir de vivre ? \\[0pt]
Peut-on vivre sans lois ? \\[0pt]
Peut-on vivre sans peur ? \\[0pt]
Peut-on vivre sans réfléchir ? \\[0pt]
Peut-on vivre sans sacré ? \\[0pt]
Peut-on vouloir ce qu'on ne désire pas ? \\[0pt]
Peut-on vouloir le mal ? \\[0pt]
Peut-on vraiment créer ? \\[0pt]
Philosophe-t-on pour être heureux ? \\[0pt]
Physique et mathématiques \\[0pt]
Physique et métaphysique \\[0pt]
Pitié et compassion \\[0pt]
Pitié et cruauté \\[0pt]
Pitié et mépris \\[0pt]
Plaisir et bonheur \\[0pt]
Poésie et philosophie \\[0pt]
Poésie et vérité \\[0pt]
Politique et vérité \\[0pt]
Politique et vertu \\[0pt]
Possession et propriété \\[0pt]
Pour agir moralement, faut-il ne pas se soucier de soi ? \\[0pt]
Pour bien agir, faut-il vouloir le bien d'autrui ? \\[0pt]
Pour connaître, suffit-il de démontrer ? \\[0pt]
Pour être homme, faut-il être citoyen ? \\[0pt]
Pour être libre, faut-il renoncer à être heureux ? \\[0pt]
Pour être un bon observateur faut-il être un bon théoricien ? \\[0pt]
Pour juger, faut-il seulement apprendre à raisonner ? \\[0pt]
Pour penser rigoureusement, faut-il renoncer au langage ordinaire ? \\[0pt]
Pourquoi aimer la liberté ? \\[0pt]
Pourquoi aller contre son désir ? \\[0pt]
Pourquoi avons-nous besoin des autres pour être heureux ? \\[0pt]
Pourquoi avons-nous du mal à reconnaître la vérité ? \\[0pt]
Pourquoi avons-nous tant de mal à renoncer à nos illusions ? \\[0pt]
Pourquoi chercher à se distinguer ? \\[0pt]
Pourquoi chercher à vivre libre ? \\[0pt]
Pourquoi chercher la vérité ? \\[0pt]
Pourquoi cherche-t-on à connaître ? \\[0pt]
Pourquoi défendre le faible ? \\[0pt]
Pourquoi délibérer ? \\[0pt]
Pourquoi des artistes ? \\[0pt]
Pourquoi des cérémonies ? \\[0pt]
Pourquoi des devoirs ? \\[0pt]
Pourquoi des hommes politiques ? \\[0pt]
Pourquoi des idoles ? \\[0pt]
Pourquoi désirons-nous ? \\[0pt]
Pourquoi des lois ? \\[0pt]
Pourquoi des maîtres ? \\[0pt]
Pourquoi des poètes ? \\[0pt]
Pourquoi des symboles ? \\[0pt]
Pourquoi des utopies ? \\[0pt]
Pourquoi dialogue-t-on ? \\[0pt]
Pourquoi distinguer nature et culture ? \\[0pt]
Pourquoi donner ? \\[0pt]
Pourquoi donner des leçons de morale ? \\[0pt]
Pourquoi échanger des idées ? \\[0pt]
Pourquoi écrit-on ? \\[0pt]
Pourquoi écrit-on les lois ? \\[0pt]
Pourquoi être moral ? \\[0pt]
Pourquoi faire confiance ? \\[0pt]
Pourquoi faire son devoir ? \\[0pt]
Pourquoi faut-il diviser le travail ? \\[0pt]
Pourquoi faut-il être juste ? \\[0pt]
Pourquoi faut-il travailler ? \\[0pt]
Pourquoi interprète-t-on ? \\[0pt]
Pourquoi joue-t-on ? \\[0pt]
Pourquoi la justice a-t-elle besoin d'institutions ? \\[0pt]
Pourquoi les États se font-ils la guerre ? \\[0pt]
Pourquoi les hommes se soumettent-ils à l'autorité ? \\[0pt]
Pourquoi les sciences ont-elles une histoire ? \\[0pt]
Pourquoi les sociétés ont-elles besoin de lois ? \\[0pt]
Pourquoi l'Homme se pose-t-il des questions métaphysiques ? \\[0pt]
Pourquoi l'homme travaille-t-il ? \\[0pt]
Pourquoi lire les poètes ? \\[0pt]
Pourquoi ne peut-on concevoir la science comme achevée ? \\[0pt]
Pourquoi nous trompons-nous ? \\[0pt]
Pourquoi n'y aurait-il pas de sots métiers ? \\[0pt]
Pourquoi parle-t-on ? \\[0pt]
Pourquoi punir ? \\[0pt]
Pourquoi rechercher la vérité ? \\[0pt]
Pourquoi respecter autrui ? \\[0pt]
Pourquoi respecter le droit ? \\[0pt]
Pourquoi s'ennuie-t-on ? \\[0pt]
Pourquoi s'interroger sur l'origine du langage ? \\[0pt]
Pourquoi sommes-nous des êtres moraux ? \\[0pt]
Pourquoi sommes-nous moraux ? \\[0pt]
Pourquoi suspendre son jugement ? \\[0pt]
Pourquoi théoriser ? \\[0pt]
Pourquoi transmettre ? \\[0pt]
Pourquoi travailler ? \\[0pt]
Pourquoi un fait devrait-il être établi ? \\[0pt]
Pourquoi vivre ensemble ? \\[0pt]
Pourquoi vouloir devenir « comme maîtres et possesseurs de la nature » ? \\[0pt]
Pourquoi vouloir se connaître ? \\[0pt]
Pourquoi y a-t-il des institutions ? \\[0pt]
Pourquoi y a-t-il plusieurs sciences ? \\[0pt]
Pourrait-on se passer de l'argent ? \\[0pt]
Pouvoir et autorité \\[0pt]
Pouvoir et devoir \\[0pt]
Pouvoir et domination \\[0pt]
Pouvoir et puissance \\[0pt]
Pouvoir et savoir \\[0pt]
Pouvons-nous connaître sans interpréter ? \\[0pt]
Pouvons-nous dissocier le réel de nos interprétations ? \\[0pt]
Pouvons-nous être certains que nous ne rêvons pas ? \\[0pt]
Pouvons-nous faire l'expérience de la liberté ? \\[0pt]
Prédire et expliquer \\[0pt]
Prendre conscience \\[0pt]
Prendre la parole \\[0pt]
Prendre la parole, est-ce prendre le pouvoir ? \\[0pt]
Prendre ses responsabilités \\[0pt]
Prendre soin \\[0pt]
Prendre son temps \\[0pt]
Prêter attention \\[0pt]
Preuve et démonstration \\[0pt]
Principe et axiome \\[0pt]
Privation et négation \\[0pt]
Production et action \\[0pt]
Production et création \\[0pt]
Produire et créer \\[0pt]
Prose et poésie \\[0pt]
Prouver \\[0pt]
Prouver et démontrer \\[0pt]
Prouver et éprouver \\[0pt]
Prouver et réfuter \\[0pt]
Prudence et liberté \\[0pt]
Puis-je être dans le vrai sans le savoir ? \\[0pt]
Puis-je être libre sans être responsable ? \\[0pt]
Puis-je faire confiance à mes sens ? \\[0pt]
Puis-je invoquer l'inconscient sans ruiner la morale ? \\[0pt]
Puis-je me passer d'imiter autrui ? \\[0pt]
Puis-je ne pas vouloir ce que je désire ? \\[0pt]
Puis-je répondre des autres ? \\[0pt]
Puis-je savoir ce qui m'est propre ? \\[0pt]
Punir \\[0pt]
Punition et vengeance \\[0pt]
Qu'ai-je le droit d'exiger d'autrui ? \\[0pt]
Qu'ai-je le droit d'exiger des autres ? \\[0pt]
Qu'aime-t-on dans l'amour ? \\[0pt]
Qualité et quantité \\[0pt]
Quand une autorité est-elle légitime ? \\[0pt]
Qu'apprend-on des romans ? \\[0pt]
Qu'apprend-on en commettant une faute ? \\[0pt]
Qu'apprend-on quand on apprend à parler ? \\[0pt]
Qu'a-t-on le droit d'interdire ? \\[0pt]
Qu'attendons-nous de la science ? \\[0pt]
Qu'attendons-nous de la technique ? \\[0pt]
Qu'attendons-nous pour être heureux ? \\[0pt]
Que célèbre l'art ? \\[0pt]
Que démontrent nos actions ? \\[0pt]
Que devons-nous à autrui ? \\[0pt]
Que devons-nous à l'État ? \\[0pt]
Que dois-je faire ? \\[0pt]
Que dois-je respecter en autrui ? \\[0pt]
Que doit la pensée à l'écriture ? \\[0pt]
Que doit-on à l'État ? \\[0pt]
Que doit-on croire ? \\[0pt]
Que doit-on désirer pour ne pas être déçu ? \\[0pt]
Que faire de nos passions ? \\[0pt]
Que faire des adversaires ? \\[0pt]
Que faire quand la loi est injuste ? \\[0pt]
Que faut-il absolument savoir ? \\[0pt]
Que faut-il respecter ? \\[0pt]
Que faut-il savoir pour bien agir ? \\[0pt]
Que faut-il savoir pour pouvoir gouverner ? \\[0pt]
Que gagne-t-on à travailler ? \\[0pt]
Quel besoin avons-nous de chercher la vérité ? \\[0pt]
Quel est le poids du passé ? \\[0pt]
Quel est le sens du progrès technique ? \\[0pt]
Quel est l'objet de la biologie ? \\[0pt]
Quel est l'objet de la conscience de soi ? \\[0pt]
Quel est l'objet de la métaphysique ? \\[0pt]
Quel est l'objet des « sciences de l'esprit » ? \\[0pt]
Quel est l'objet du désir ? \\[0pt]
Quel genre de conscience peut-on accorder à l'animal ? \\[0pt]
Quelle causalité pour le vivant ? \\[0pt]
Quelle est la cause du désir ? \\[0pt]
Quelle est la fin de la science ? \\[0pt]
Quelle est la fonction première de l'État ? \\[0pt]
Quelle est la force de la loi ? \\[0pt]
Quelle est la place de l'imagination dans la vie de l'esprit ? \\[0pt]
Quelle est la réalité de l'avenir ? \\[0pt]
Quelle est la réalité d'une idée ? \\[0pt]
Quelle est la réalité du passé ? \\[0pt]
Quelle est la valeur culturelle de la science ? \\[0pt]
Quelle est la valeur d'une expérimentation ? \\[0pt]
Quelle est la valeur du rêve ? \\[0pt]
Quelle est l'unité du « je » ? \\[0pt]
Quelle réalité attribuer à la matière ? \\[0pt]
Quelle réalité l'art nous fait-il connaître ? \\[0pt]
Quelle sorte d'histoire ont les sciences ? \\[0pt]
Quelles sont les caractéristiques d'un être vivant ? \\[0pt]
Quelles sont les limites de mon monde ? \\[0pt]
Quelle valeur devons-nous accorder à l'expérience ? \\[0pt]
Quel sens donner à l'expression « gagner sa vie » ? \\[0pt]
Quels enseignements peut-on tirer de l'histoire des sciences ? \\[0pt]
Quel usage faut-il faire des exemples ? \\[0pt]
Que manque-t-il à une machine pour être vivante ? \\[0pt]
Que mesure-t-on du temps ? \\[0pt]
Que montre une démonstration ? \\[0pt]
Que m'ordonne ma conscience ? \\[0pt]
Que nous apprend la diversité des langues ? \\[0pt]
Que nous apprend la fiction sur la réalité ? \\[0pt]
Que nous apprend la maladie sur la santé ? \\[0pt]
Que nous apprend la musique ? \\[0pt]
Que nous apprend la vie ? \\[0pt]
Que nous apprend l'expérience ? \\[0pt]
Que nous apprennent les animaux ? \\[0pt]
Que nous apprennent les animaux sur nous-mêmes ? \\[0pt]
Que nous apprennent les machines ? \\[0pt]
Que nous apprennent les métaphores ? \\[0pt]
Que nous apprennent nos erreurs ? \\[0pt]
Que nous enseigne l'expérience ? \\[0pt]
Que nous enseignent les sens ? \\[0pt]
Que nous réserve l'avenir ? \\[0pt]
Que peint le peintre ? \\[0pt]
Que penser de l'adage : « Que la justice s'accomplisse, le monde dût-il périr » (Fiat justitia pereat mundus) ? \\[0pt]
Que percevons-nous d'autrui ? \\[0pt]
Que perd-on quand on perd son temps ? \\[0pt]
Que perdrait la pensée en perdant l'écriture ? \\[0pt]
Que peut la musique ? \\[0pt]
Que peut la volonté ? \\[0pt]
Que peut le corps ? \\[0pt]
Que peut l'esprit sur la matière ? \\[0pt]
Que peut l'État ? \\[0pt]
Que peut-on contre un préjugé ? \\[0pt]
Que peut-on savoir de l'inconscient ? \\[0pt]
Que peut-on savoir de soi ? \\[0pt]
Que peut-on savoir par expérience ? \\[0pt]
Que pouvons-nous faire de notre passé ? \\[0pt]
Que produit l'inconscient ? \\[0pt]
Que prouvent les preuves de l'existence de Dieu ? \\[0pt]
Que reste-t-il d'une existence ? \\[0pt]
Que sait-on du réel ? \\[0pt]
Que serait une pure sensation ? \\[0pt]
Que signifie : « échanger des arguments » ? \\[0pt]
Que signifie être en guerre ? \\[0pt]
Que signifie l'expression : « l'histoire jugera » ? \\[0pt]
Que signifie l'idée de technoscience ? \\[0pt]
Que signifie refuser l'injustice ? \\[0pt]
Que signifier « juger en son âme et conscience » ? \\[0pt]
Que sont les apparences ? \\[0pt]
Que sont les objets de la perception ? \\[0pt]
Qu'est-ce qu'apprendre ? \\[0pt]
Qu'est-ce qu'argumenter ? \\[0pt]
Qu'est-ce qu'avoir du jugement ? \\[0pt]
Qu'est-ce que bien vivre ? \\[0pt]
Qu'est-ce que commencer ? \\[0pt]
Qu'est-ce que composer une œuvre ? \\[0pt]
Qu'est-ce que comprendre une œuvre d'art ? \\[0pt]
Qu'est-ce que contempler ? \\[0pt]
Qu'est-ce que créer ? \\[0pt]
Qu'est-ce que croire ? \\[0pt]
Qu'est-ce que définir ? \\[0pt]
Qu'est-ce que donner sa parole ? \\[0pt]
Qu'est-ce que faire une expérience ? \\[0pt]
Qu'est-ce que « faire usage de sa raison » ? \\[0pt]
Qu'est-ce que gouverner ? \\[0pt]
Qu'est-ce que juger ? \\[0pt]
Qu'est-ce que la causalité ? \\[0pt]
Qu'est-ce que la religion nous donne à savoir ? \\[0pt]
Qu'est-ce que la science doit à l'expérience ? \\[0pt]
Qu'est-ce que la science saisit du vivant ? \\[0pt]
Qu'est-ce que la technique doit à la nature ? \\[0pt]
Qu'est-ce que le langage ordinaire ? \\[0pt]
Qu'est-ce que le malheur ? \\[0pt]
Qu'est-ce que le moi ? \\[0pt]
Qu'est-ce que le présent ? \\[0pt]
Qu'est-ce que le progrès technique ? \\[0pt]
Qu'est-ce que le réel ? \\[0pt]
Qu'est-ce que le sacré ? \\[0pt]
Qu'est-ce que l'inconscient ? \\[0pt]
Qu'est-ce que l'intérêt général ? \\[0pt]
Qu'est-ce que l'objectivité scientifique ? \\[0pt]
Qu'est-ce que manquer de culture ? \\[0pt]
Qu'est-ce que nous apprennent les animaux sur nous-même ? \\[0pt]
Qu'est-ce que parler ? \\[0pt]
Qu'est-ce que parler à-propos ? \\[0pt]
Qu'est-ce que « parler le même langage » ? \\[0pt]
Qu'est-ce que parler le même langage ? \\[0pt]
Qu'est-ce que prendre conscience ? \\[0pt]
Qu'est-ce que produire ? \\[0pt]
Qu'est-ce que prouver ? \\[0pt]
Qu'est-ce que « rester soi-même » ? \\[0pt]
Qu'est-ce que se cultiver ? \\[0pt]
Qu'est-ce que traduire ? \\[0pt]
Qu'est-ce qu'être artiste ? \\[0pt]
Qu'est-ce qu'être en vie ? \\[0pt]
Qu'est-ce qu'être esclave ? \\[0pt]
Qu'est-ce qu'être inhumain ? \\[0pt]
Qu'est-ce qu'être l'auteur de son acte ? \\[0pt]
Qu'est-ce qu'être maître de soi ? \\[0pt]
Qu'est-ce qu'être malade ? \\[0pt]
Qu'est-ce qu'être matérialiste ? \\[0pt]
Qu'est-ce qu'être normal ? \\[0pt]
Qu'est-ce qu'être rationaliste ? \\[0pt]
Qu'est-ce qu'être réaliste ? \\[0pt]
Qu'est-ce qu'être spirituel ? \\[0pt]
Qu'est-ce que vérifier une théorie ? \\[0pt]
Qu'est-ce que vivre bien ? \\[0pt]
Qu'est-ce que vivre vertueusement ? \\[0pt]
Qu'est-ce qu'exister ? \\[0pt]
Qu'est-ce qu'exister pour un individu ? \\[0pt]
Qu'est-ce qui distingue un argument d'une démonstration ? \\[0pt]
Qu'est-ce qui distingue un vivant d'une machine ? \\[0pt]
Qu'est-ce qui est absurde ? \\[0pt]
Qu'est-ce qui est intolérable ? \\[0pt]
Qu'est-ce qui est irrationnel ? \\[0pt]
Qu'est-ce qui est irréversible ? \\[0pt]
Qu'est-ce qui est naturel ? \\[0pt]
Qu'est-ce qui est nécessaire ? \\[0pt]
Qu'est-ce qui est possible ? \\[0pt]
Qu'est-ce qui est réel ? \\[0pt]
Qu'est-ce qui est respectable ? \\[0pt]
Qu'est-ce qui est scientifique ? \\[0pt]
Qu'est-ce qui est sublime ? \\[0pt]
Qu'est-ce qui est vital ? \\[0pt]
Qu'est-ce qui est vivant ? \\[0pt]
Qu'est-ce qui exige d'être interprété ? \\[0pt]
Qu'est-ce qui fait autorité ? \\[0pt]
Qu'est-ce qui fait changer les sociétés ? \\[0pt]
Qu'est-ce qui fait d'une activité un travail ? \\[0pt]
Qu'est-ce qui fait du scepticisme une philosophie ? \\[0pt]
Qu'est-ce qui fait la valeur de la technique ? \\[0pt]
Qu'est-ce qui fait la valeur d'une existence ? \\[0pt]
Qu'est-ce qui fait le pouvoir des mots ? \\[0pt]
Qu'est-ce qui fait l'unité d'une science ? \\[0pt]
Qu'est-ce qui fait l'unité d'un organisme ? \\[0pt]
Qu'est-ce qui fait qu'un corps est humain ? \\[0pt]
Qu'est-ce qui fait un peuple ? \\[0pt]
Qu'est-ce qui fonde le respect d'autrui ? \\[0pt]
Qu'est-ce qui importe ? \\[0pt]
Qu'est-ce qui m'appartient ? \\[0pt]
Qu'est-ce qui menace la liberté ? \\[0pt]
Qu'est-ce qui mesure la valeur d'un travail ? \\[0pt]
Qu'est-ce qui n'a pas d'histoire ? \\[0pt]
Qu'est-ce qui n'est pas technique dans l'activité humaine ? \\[0pt]
Qu'est-ce qu'interpréter ? \\[0pt]
Qu'est-ce qu'interpréter une œuvre d'art ? \\[0pt]
Qu'est-ce qu'interpréter un texte ? \\[0pt]
Qu'est-ce qui oppose les hommes ? \\[0pt]
Qu'est-ce qui peut se transformer ? \\[0pt]
Qu'est-ce qui rend une parole poétique ? \\[0pt]
Qu'est-ce qu'on ne peut comprendre ? \\[0pt]
Qu'est-ce qu'un acte libre ? \\[0pt]
Qu'est-ce qu'un alter ego ? \\[0pt]
Qu'est-ce qu'un animal ? \\[0pt]
Qu'est-ce qu'un art abstrait ? \\[0pt]
Qu'est-ce qu'un art populaire ? \\[0pt]
Qu'est-ce qu'un bon argument ? \\[0pt]
Qu'est-ce qu'un bon citoyen ? \\[0pt]
Qu'est-ce qu'un cas de conscience ? \\[0pt]
Qu'est-ce qu'un chef-d'œuvre ? \\[0pt]
Qu'est-ce qu'un choix éclairé ? \\[0pt]
Qu'est-ce qu'un choix rationnel ? \\[0pt]
Qu'est-ce qu'un citoyen libre ? \\[0pt]
Qu'est-ce qu'un classique ? \\[0pt]
Qu'est-ce qu'un concept ? \\[0pt]
Qu'est-ce qu'un consommateur ? \\[0pt]
Qu'est-ce qu'un corps politique ? \\[0pt]
Qu'est-ce qu'un dieu ? \\[0pt]
Qu'est-ce qu'un droit naturel ? \\[0pt]
Qu'est-ce qu'une action juste ? \\[0pt]
Qu'est-ce qu'une action politique ? \\[0pt]
Qu'est-ce qu'une autorité légitime ? \\[0pt]
Qu'est-ce qu'une belle action ? \\[0pt]
Qu'est-ce qu'une belle forme ? \\[0pt]
Qu'est-ce qu'une bonne délibération ? \\[0pt]
Qu'est-ce qu'une bonne traduction ? \\[0pt]
Qu'est-ce qu'un échange juste ? \\[0pt]
Qu'est-ce qu'un échange réussi ? \\[0pt]
Qu'est-ce qu'une chose matérielle ? \\[0pt]
Qu'est-ce qu'une communauté ? \\[0pt]
Qu'est-ce qu'une connaissance fiable ? \\[0pt]
Qu'est-ce qu'une constitution ? \\[0pt]
Qu'est-ce qu'une crise ? \\[0pt]
Qu'est-ce qu'une décision rationnelle ? \\[0pt]
Qu'est-ce qu'une erreur ? \\[0pt]
Qu'est-ce qu'une expérience scientifique ? \\[0pt]
Qu'est-ce qu'une fausse science ? \\[0pt]
Qu'est-ce qu'une faute de goût ? \\[0pt]
Qu'est-ce qu'une fiction ? \\[0pt]
Qu'est-ce qu'une hypothèse scientifique ? \\[0pt]
Qu'est-ce qu'une image ? \\[0pt]
Qu'est-ce qu'une injustice ? \\[0pt]
Qu'est-ce qu'une langue artificielle ? \\[0pt]
Qu'est-ce qu'une libre interprétation ? \\[0pt]
Qu'est-ce qu'une mauvaise idée ? \\[0pt]
Qu'est-ce qu'une méthode ? \\[0pt]
Qu'est-ce qu'une nation ? \\[0pt]
Qu'est-ce qu'une occasion ? \\[0pt]
Qu'est-ce qu'une œuvre d'art réaliste ? \\[0pt]
Qu'est-ce qu'une œuvre ratée ? \\[0pt]
Qu'est-ce qu'une parole vraie ? \\[0pt]
Qu'est-ce qu'une perception sensible ? \\[0pt]
Qu'est-ce qu'une philosophie de la vie ? \\[0pt]
Qu'est-ce qu'une preuve ? \\[0pt]
Qu'est-ce qu'une pseudoscience ? \\[0pt]
Qu'est-ce qu'une question métaphysique ? \\[0pt]
Qu'est-ce qu'une république ? \\[0pt]
Qu'est-ce qu'une révolution ? \\[0pt]
Qu'est-ce qu'une révolution scientifique ? \\[0pt]
Qu'est-ce qu'une science expérimentale ? \\[0pt]
Qu'est-ce qu'une société juste ? \\[0pt]
Qu'est-ce qu'une solution ? \\[0pt]
Qu'est-ce qu'un esprit juste ? \\[0pt]
Qu'est-ce qu'un esprit libre ? \\[0pt]
Qu'est-ce qu'un État de droit ? \\[0pt]
Qu'est-ce qu'un État libre ? \\[0pt]
Qu'est-ce qu'une théorie ? \\[0pt]
Qu'est-ce qu'une théorie scientifique ? \\[0pt]
Qu'est-ce qu'une tradition ? \\[0pt]
Qu'est-ce qu'un événement ? \\[0pt]
Qu'est-ce qu'un événement fondateur ? \\[0pt]
Qu'est-ce qu'un événement historique ? \\[0pt]
Qu'est-ce qu'une vérité contingente ? \\[0pt]
Qu'est-ce qu'une vérité historique ? \\[0pt]
Qu'est-ce qu'une vérité subjective ? \\[0pt]
Qu'est-ce qu'une vie heureuse ? \\[0pt]
Qu'est-ce qu'une vie humaine ? \\[0pt]
Qu'est-ce qu'un exemple ? \\[0pt]
Qu'est-ce qu'un expérimentateur ? \\[0pt]
Qu'est-ce qu'un fait ? \\[0pt]
Qu'est-ce qu'un fait culturel ? \\[0pt]
Qu'est-ce qu'un fait de culture ? \\[0pt]
Qu'est-ce qu'un fait historique ? \\[0pt]
Qu'est-ce qu'un faux ? \\[0pt]
Qu'est-ce qu'un faux problème ? \\[0pt]
Qu'est-ce qu'un gouvernement démocratique ? \\[0pt]
Qu'est-ce qu'un homme d'action ? \\[0pt]
Qu'est-ce qu'un homme d'État ? \\[0pt]
Qu'est-ce qu'un homme d'expérience ? \\[0pt]
Qu'est-ce qu'un homme juste ? \\[0pt]
Qu'est-ce qu'un homme méchant ? \\[0pt]
Qu'est-ce qu'un homme politique ? \\[0pt]
Qu'est-ce qu'un impératif technique ? \\[0pt]
Qu'est-ce qu'un juste ? \\[0pt]
Qu'est-ce qu'un juste salaire ? \\[0pt]
Qu'est-ce qu'un justicier ? \\[0pt]
Qu'est-ce qu'un maître ? \\[0pt]
Qu'est-ce qu'un modèle ? \\[0pt]
Qu'est-ce qu'un monstre ? \\[0pt]
Qu'est-ce qu'un musée ? \\[0pt]
Qu'est-ce qu'un mythe ? \\[0pt]
Qu'est-ce qu'un objet de science ? \\[0pt]
Qu'est-ce qu'un objet technique ? \\[0pt]
Qu'est-ce qu'un organisme ? \\[0pt]
Qu'est-ce qu'un outil ? \\[0pt]
Qu'est-ce qu'un paradoxe ? \\[0pt]
Qu'est-ce qu'un pauvre ? \\[0pt]
Qu'est-ce qu'un peuple ? \\[0pt]
Qu'est-ce qu'un problème ? \\[0pt]
Qu'est-ce qu'un problème scientifique ? \\[0pt]
Qu'est-ce qu'un problème technique ? \\[0pt]
Qu'est-ce qu'un progrès technique ? \\[0pt]
Qu'est-ce qu'un public ? \\[0pt]
Qu'est-ce qu'un pur esprit ? \\[0pt]
Qu'est-ce qu'un récit véridique ? \\[0pt]
Qu'est-ce qu'un style ? \\[0pt]
Qu'est-ce qu'un tabou ? \\[0pt]
Qu'est-ce qu'un technicien ? \\[0pt]
Qu'est-ce qu'un témoin ? \\[0pt]
Qu'est-ce qu'un tyran ? \\[0pt]
Que valent les mots ? \\[0pt]
Que valent les théories ? \\[0pt]
Que vaut la définition de l'homme comme animal doué de raison ? \\[0pt]
Que vaut le conseil : « vivez avec votre temps » ? \\[0pt]
Que vaut l'excuse : c'est plus fort que moi ? \\[0pt]
Que vaut une parole ? \\[0pt]
Que vaut une preuve contre un préjugé ? \\[0pt]
Que veut dire : être athée ? \\[0pt]
Que veut dire : « le temps passe » ? \\[0pt]
Que voulons-nous vraiment savoir ? \\[0pt]
Qui accroît son savoir accroît sa douleur \\[0pt]
Qui a le droit de punir ? \\[0pt]
Qui commande ? \\[0pt]
Qui croire ? \\[0pt]
Qui est digne du bonheur ? \\[0pt]
Qui est le peuple ? \\[0pt]
Qui est libre ? \\[0pt]
Qui est mon prochain ? \\[0pt]
Qui est mon semblable ? \\[0pt]
Qui est riche ? \\[0pt]
Qui est sage ? \\[0pt]
Qui fait la loi ? \\[0pt]
Qui gouverne ? \\[0pt]
Qui me dit ce que je dois faire ? \\[0pt]
Qui nous dicte nos devoirs ? \\[0pt]
Qui parle ? \\[0pt]
Qui parle quand je dis « je » ? \\[0pt]
Qui peut avoir des droits ? \\[0pt]
Qui peut me dire « tu ne dois pas » ? \\[0pt]
Qui travaille ? \\[0pt]
Qu'y a-t-il à craindre de la technique ? \\[0pt]
Qu'y a-t-il que la nature fait en vain ? \\[0pt]
Qu'y a-t-il sinon le réel ? \\[0pt]
Raison et calcul \\[0pt]
Raison et démonstration \\[0pt]
Raison et dialogue \\[0pt]
Raison et folie \\[0pt]
Raison et fondement \\[0pt]
Raison et langage \\[0pt]
Raison et religion sont-elles incompatibles ? \\[0pt]
Raison et tradition \\[0pt]
Raisonnable et rationnel \\[0pt]
Raisonner \\[0pt]
Rationalité et irrationalité du travail \\[0pt]
Réalité et apparence \\[0pt]
Réalité et perception \\[0pt]
Réalité et représentation \\[0pt]
Récit et histoire \\[0pt]
Réfuter \\[0pt]
Regarder et voir \\[0pt]
Règles sociales et loi morale \\[0pt]
Regrets et remords \\[0pt]
Religion et démocratie \\[0pt]
Religion et moralité \\[0pt]
Religion et politique \\[0pt]
Religion et raison \\[0pt]
Religion et superstition \\[0pt]
Religion et vérité \\[0pt]
Religions et démocratie \\[0pt]
Rendre justice \\[0pt]
Répondre de soi \\[0pt]
Représentation et expression \\[0pt]
Représenter \\[0pt]
République et démocratie \\[0pt]
Résistance et obéissance \\[0pt]
Résister peut-il être un droit ? \\[0pt]
Respect et tolérance \\[0pt]
Rester soi-même \\[0pt]
Réussir sa vie \\[0pt]
Rêver \\[0pt]
Revient-il à l'État d'assurer le bonheur des citoyens ? \\[0pt]
Révolte et révolution \\[0pt]
Rhétorique et vérité \\[0pt]
Richesse et pauvreté \\[0pt]
« Rien de ce qui est humain ne m'est étranger » \\[0pt]
« Rien de nouveau sous le soleil » \\[0pt]
Rire \\[0pt]
Sait-on ce que l'on veut ? \\[0pt]
Sait-on ce qu'on fait ? \\[0pt]
Sait-on nécessairement ce que l'on désire ? \\[0pt]
Sait-on toujours ce qu'on veut ? \\[0pt]
Sait-on vivre au présent ? \\[0pt]
S'amuser \\[0pt]
Sans justice, pas de liberté ? \\[0pt]
Savoir, est-ce cesser de croire ? \\[0pt]
Savoir, est-ce pouvoir ? \\[0pt]
Savoir et croire \\[0pt]
Savoir et démontrer \\[0pt]
Savoir et pouvoir \\[0pt]
Savoir et savoir faire \\[0pt]
Savoir par cœur \\[0pt]
Science du vivant et finalisme \\[0pt]
Science du vivant, science de l'inerte \\[0pt]
Science et croyance \\[0pt]
Science et métaphysique \\[0pt]
Science et méthode \\[0pt]
Science et mythe \\[0pt]
Science et religion \\[0pt]
Science et sagesse \\[0pt]
Se cultiver \\[0pt]
Se cultiver, est-ce s'affranchir de son appartenance culturelle ? \\[0pt]
Sécurité et liberté \\[0pt]
Se décider \\[0pt]
Se faire comprendre \\[0pt]
S'émanciper \\[0pt]
Se mentir à soi-même : est-ce possible ? \\[0pt]
Se nourrir \\[0pt]
Sensation et perception \\[0pt]
Sens et existence \\[0pt]
Sens et non-sens \\[0pt]
Sens et signification \\[0pt]
Sens propre et sens figuré \\[0pt]
Sentiment et justice sont-ils compatibles ? \\[0pt]
Sentir et juger \\[0pt]
Sentir et penser \\[0pt]
Serions-nous heureux dans un ordre politique parfait ? \\[0pt]
Servir, est-ce nécessairement renoncer à sa liberté ? \\[0pt]
Se suffire à soi-même \\[0pt]
S'exprimer \\[0pt]
Signe et symbole \\[0pt]
Sincérité et vérité \\[0pt]
Si nous étions moraux, le droit serait-il inutile ? \\[0pt]
Si tout est historique, tout est-il relatif ? \\[0pt]
Société et communauté \\[0pt]
Société et contrat \\[0pt]
Société et organisme \\[0pt]
Société humaines, sociétés animales \\[0pt]
« Sois naturel » : est-ce un bon conseil ? \\[0pt]
« Sois toi-même ! » : un impératif absurde ? \\[0pt]
Solitude et liberté \\[0pt]
Sommes-nous adaptés au monde de la technique ? \\[0pt]
Sommes-nous dans le temps comme dans l'espace ? \\[0pt]
Sommes-nous des sujets ? \\[0pt]
Sommes-nous déterminés par notre culture ? \\[0pt]
Sommes-nous égaux devant le bonheur ? \\[0pt]
Sommes-nous faits pour être heureux ? \\[0pt]
Sommes-nous jamais certains d'avoir choisi librement ? \\[0pt]
Sommes-nous libres face à l'évidence ? \\[0pt]
Sommes-nous libres lorsque les actes émanent de notre personne ? \\[0pt]
Sommes-nous maîtres de nos paroles ? \\[0pt]
Sommes-nous maîtres de nos pensées ? \\[0pt]
Sommes-nous portés au bien ? \\[0pt]
Sommes-nous prisonniers du temps ? \\[0pt]
Sommes-nous responsables de nos désirs ? \\[0pt]
Sommes-nous responsables de nos opinions ? \\[0pt]
Sommes-nous séparés du réel par les représentations que nous nous en faisons ? \\[0pt]
Sommes-nous sujets de nos désirs ? \\[0pt]
S'opposer \\[0pt]
Soumission et servitude \\[0pt]
Substance et accident \\[0pt]
Suffit-il d'avoir raison ? \\[0pt]
Suffit-il de bien juger pour bien faire ? \\[0pt]
Suffit-il de communiquer pour dialoguer ? \\[0pt]
Suffit-il de démontrer pour convaincre ? \\[0pt]
Suffit-il de faire son devoir ? \\[0pt]
Suffit-il de faire son devoir pour être vertueux ? \\[0pt]
Suffit-il de n'avoir rien fait pour être innocent ? \\[0pt]
Suffit-il de vivre pour exister ? \\[0pt]
Suffit-il de voir pour savoir ? \\[0pt]
Suffit-il, pour être juste, d'obéir aux lois et aux coutumes de son pays ? \\[0pt]
Suffit-il que nos intentions soient bonnes pour que nos actions le soient aussi ? \\[0pt]
Suis-ce que j'ai conscience d'être ? \\[0pt]
Suis-je ce que mon passé a fait de moi ? \\[0pt]
Suis-je dans le temps comme je suis dans l'espace ? \\[0pt]
Suis-je étranger à moi-même ? \\[0pt]
Suis-je l'auteur de ce que je dis ? \\[0pt]
Suis-je le mieux placé pour me connaître ? \\[0pt]
Suis-je le mieux placé pour savoir qui je suis ? \\[0pt]
Suis-je libre ? \\[0pt]
Suis-je mon corps ? \\[0pt]
Suis-je mon passé ? \\[0pt]
Suis-je propriétaire de mon corps ? \\[0pt]
Suis-je responsable de ce dont je n'ai pas conscience ? \\[0pt]
Suis-je responsable de ce que je suis ? \\[0pt]
Suis-je toujours autre que moi-même ? \\[0pt]
Suivre son intuition \\[0pt]
Surface et profondeur \\[0pt]
Sur quoi fonder la justice ? \\[0pt]
Sur quoi fonder l'autorité ? \\[0pt]
Sur quoi fonder l'autorité politique ? \\[0pt]
Sur quoi fonder le devoir ? \\[0pt]
Sur quoi fonder le droit de punir ? \\[0pt]
Sur quoi le langage doit-il se régler ? \\[0pt]
Sur quoi peut-on fonder la certitude d'avoir raison ? \\[0pt]
Sur quoi repose la croyance au progrès ? \\[0pt]
Sur quoi sont fondées les mathématiques ? \\[0pt]
Survivre \\[0pt]
Suspendre son jugement \\[0pt]
Sympathie et respect \\[0pt]
Talent et génie \\[0pt]
Technique et idéologie \\[0pt]
Technique et liberté \\[0pt]
Technique et nature \\[0pt]
Technique et savoir-faire \\[0pt]
Technique et violence \\[0pt]
Témoignage et vérité \\[0pt]
Temps et commencement \\[0pt]
Temps et création \\[0pt]
Temps et histoire \\[0pt]
Temps et irréversibilité \\[0pt]
Temps et liberté \\[0pt]
Temps et mémoire \\[0pt]
Temps et vérité \\[0pt]
Temps physique et temps biologique \\[0pt]
Théorie et expérience \\[0pt]
Tolérance et laïcité \\[0pt]
Toucher, sentir, goûter \\[0pt]
Tous les conflits peuvent-ils être résolus par le dialogue ? \\[0pt]
Tous les hommes désirent-ils naturellement être heureux ? \\[0pt]
Tous les hommes désirent-ils naturellement savoir ? \\[0pt]
Tous les paradis sont-ils perdus ? \\[0pt]
Tous les rapports humains sont-ils des échanges ? \\[0pt]
Tout a-t-il une raison d'être ? \\[0pt]
Tout ce qui est naturel est-il normal ? \\[0pt]
Tout ce qui est rationnel est-il raisonnable ? \\[0pt]
Tout ce qui est vrai doit-il être prouvé ? \\[0pt]
Tout démontrer \\[0pt]
Tout désir est-il désir de posséder ? \\[0pt]
Tout désir est-il désir de quelque chose ? \\[0pt]
Tout désir est-il égoïste ? \\[0pt]
Tout désir est-il une souffrance ? \\[0pt]
Tout dire \\[0pt]
Tout droit est-il un pouvoir ? \\[0pt]
Toute compréhension implique-t-elle une interprétation ? \\[0pt]
Toute connaissance est-elle hypothétique ? \\[0pt]
Toute connaissance s'enracine-t-elle dans la perception ? \\[0pt]
Toute conscience est-elle conscience de quelque chose ? \\[0pt]
Toute conscience est-elle subjective ? \\[0pt]
Toute description est-elle une interprétation ? \\[0pt]
Toute faute est-elle une erreur ? \\[0pt]
Toute inégalité est-elle injuste ? \\[0pt]
Toute interprétation est-elle contestable ? \\[0pt]
Toute interprétation est-elle subjective ? \\[0pt]
Toute morale s'oppose-t-elle aux désirs ? \\[0pt]
Toute œuvre d'art exige-t-elle une interprétation ? \\[0pt]
Toute pensée revêt-elle une forme linguistique ? \\[0pt]
Toute polémique est-elle stérile ? \\[0pt]
Toute relation humaine est-elle un échange ? \\[0pt]
Toutes les croyances se valent-elles ? \\[0pt]
Toutes les fautes se valent-elles ? \\[0pt]
Toutes les inégalités sont-elles des injustices ? \\[0pt]
Toutes les interprétations se valent-elles ? \\[0pt]
Toute société a-t-elle besoin d'une religion ? \\[0pt]
Tout est-il matière ? \\[0pt]
« Tout est relatif » \\[0pt]
Toute vérité est-elle démontrable ? \\[0pt]
Toute vérité est-elle une erreur rectifiée ? \\[0pt]
Toute vérité se laisse-t-elle démontrer ? \\[0pt]
Toute vie est-elle intrinsèquement respectable ? \\[0pt]
Tout futur est-il contingent ? \\[0pt]
Tout ordre est-il une violence déguisée ? \\[0pt]
Tout peut-il avoir une valeur marchande ? \\[0pt]
Tout peut-il avoir valeur marchande ? \\[0pt]
Tout peut-il être objet d'échange ? \\[0pt]
Tout peut-il être objet de science ? \\[0pt]
Tout peut-il s'acheter ? \\[0pt]
Tout peut-il se démontrer ? \\[0pt]
Tout s'en va-t-il avec le temps ? \\[0pt]
Tout se prête-il à la mesure ? \\[0pt]
Tout texte est-il à interpréter ? \\[0pt]
Tout travail est-il forcé ? \\[0pt]
Tout travail est-il social ? \\[0pt]
Tout vouloir \\[0pt]
Tradition et liberté \\[0pt]
Tradition et nouveauté \\[0pt]
Tradition et transmission \\[0pt]
« Tradition n'est pas raison » \\[0pt]
Traduire, est-ce trahir ? \\[0pt]
Transcendance et immanence \\[0pt]
Transmettre \\[0pt]
Travail, besoin, désir \\[0pt]
Travail et aliénation \\[0pt]
Travail et besoin \\[0pt]
Travail et bonheur \\[0pt]
Travail et capital \\[0pt]
Travail et liberté \\[0pt]
Travail et loisir \\[0pt]
Travail et nécessité \\[0pt]
Travail et œuvre \\[0pt]
Travail et propriété \\[0pt]
Travailler, est-ce faire œuvre ? \\[0pt]
Travailler, est-ce lutter contre soi-même ? \\[0pt]
Travailler, est-ce seulement produire ? \\[0pt]
Travailler et œuvrer \\[0pt]
Travaille-t-on pour soi-même ? \\[0pt]
Travail manuel, travail intellectuel \\[0pt]
Travail servile et travail salarié \\[0pt]
Trouver et prouver \\[0pt]
« Tu dois, donc tu peux » \\[0pt]
Tyrannie et dictature \\[0pt]
Un acte gratuit est-il possible ? \\[0pt]
Un acte libre est-il un acte imprévisible ? \\[0pt]
Un acte peut-il être inhumain ? \\[0pt]
Un artiste doit-il être original ? \\[0pt]
Un bien peut-il être commun ? \\[0pt]
Un chef-d'œuvre est-il immortel ? \\[0pt]
Un choix peut-il être rationnel ? \\[0pt]
Un choix rationnel est-il un choix libre ? \\[0pt]
Un désir peut-il être coupable ? \\[0pt]
Un désir peut-il être inconscient ? \\[0pt]
Une activité inutile est-elle sans valeur ? \\[0pt]
Une communauté politique n'est-elle qu'une communauté d'intérêt ? \\[0pt]
Une communauté politique n'est-elle qu'une communauté d'intérêts ? \\[0pt]
Une connaissance peut-elle ne pas être relative ? \\[0pt]
Une connaissance scientifique du vivant est-elle possible ? \\[0pt]
Une croyance peut-elle être libre ? \\[0pt]
Une croyance peut-elle être rationnelle ? \\[0pt]
Une culture peut-elle être porteuse de valeurs universelles ? \\[0pt]
Une destruction peut-elle être créatrice ? \\[0pt]
Une fausse science est-elle une science qui commet des erreurs ? \\[0pt]
Une fiction peut-elle être vraie ? \\[0pt]
Une idée peut-elle être générale ? \\[0pt]
Une imitation peut-elle être parfaite ? \\[0pt]
Une interprétation est-elle nécessairement subjective ? \\[0pt]
Une interprétation peut-elle échapper à l'arbitraire ? \\[0pt]
Une interprétation peut-elle être définitive ? \\[0pt]
Une interprétation peut-elle être fausse ? \\[0pt]
Une interprétation peut-elle être objective ? \\[0pt]
Une interprétation peut-elle prétendre à la vérité ? \\[0pt]
Une langue n'est-elle faite que de mots ? \\[0pt]
Une loi peut-elle être injuste ? \\[0pt]
Une machine n'est-elle qu'un outil perfectionné ? \\[0pt]
Une machine peut-elle penser ? \\[0pt]
Une matière sans forme est-elle concevable ? \\[0pt]
Une morale peut-elle être permissive ? \\[0pt]
Une morale sans devoirs est-elle possible ? \\[0pt]
Une morale sans obligation est-elle possible ? \\[0pt]
Une morale sceptique est-elle possible ? \\[0pt]
Une œuvre d'art a-t-elle toujours un sens ? \\[0pt]
Une œuvre d'art doit-elle nécessairement être belle ? \\[0pt]
Une œuvre d'art doit-elle plaire ? \\[0pt]
Une œuvre d'art peut-elle être immorale ? \\[0pt]
Une pensée contradictoire est-elle dénuée de valeur ? \\[0pt]
Une perception peut-elle être illusoire ? \\[0pt]
Une psychologie peut-elle être matérialiste ? \\[0pt]
Une religion peut-elle être rationnelle ? \\[0pt]
Une science de l'esprit est-elle possible ? \\[0pt]
Une science peut-elle se passer d'expérimentation ? \\[0pt]
Une sensation peut-elle être fausse ? \\[0pt]
Une société basée sur le don pourrait-elle être prospère ? \\[0pt]
Une société juste, est-ce une société sans conflit ? \\[0pt]
Une société peut-elle être juste ? \\[0pt]
Une société peut-elle se passer de religion ? \\[0pt]
Une société sans conflit est-elle pensable ? \\[0pt]
Une société sans conflits est-elle souhaitable ? \\[0pt]
Une société sans État est-elle possible ? \\[0pt]
Une société sans religion est-elle possible ? \\[0pt]
Une société sans travail est-elle souhaitable ? \\[0pt]
Une théorie peut-elle être vérifiée ? \\[0pt]
Une vérité approchée est-elle pour autant approximative ? \\[0pt]
Une vérité peut-elle être approximative ? \\[0pt]
Une vérité peut-elle être indicible ? \\[0pt]
Une vérité peut-elle être provisoire ? \\[0pt]
Une vie libre exclut-elle le travail ? \\[0pt]
Un fait existe-t-il sans interprétation ? \\[0pt]
Un gouvernement de savants est-il souhaitable ? \\[0pt]
Un grand bonheur \\[0pt]
« Un instant d'éternité » \\[0pt]
Un mensonge peut-il avoir une valeur morale ? \\[0pt]
Un monde meilleur \\[0pt]
Un monde sans travail est-il souhaitable ? \\[0pt]
Un peuple est-il responsable de son histoire ? \\[0pt]
Un peuple est-il un rassemblement d'individus ? \\[0pt]
Un plaisir peut-il être désintéressé ? \\[0pt]
Un problème moral peut-il recevoir une solution certaine ? \\[0pt]
Un savoir peut-il être inconscient ? \\[0pt]
Un savoir scientifique sur l'homme est-il compatible avec l'idée de liberté ? \\[0pt]
User de violence peut-il être moral ? \\[0pt]
Utilité et beauté \\[0pt]
Vaincre la mort \\[0pt]
Vaut-il mieux subir ou commettre l'injustice ? \\[0pt]
Vérité et apparence \\[0pt]
Vérité et certitude \\[0pt]
Vérité et efficacité \\[0pt]
Vérité et exactitude \\[0pt]
Vérité et illusion \\[0pt]
Vérité et liberté \\[0pt]
Vérité et réalité \\[0pt]
Vérité et religion \\[0pt]
Vérité et sincérité \\[0pt]
Vérité et validité \\[0pt]
Vérité et vérification \\[0pt]
Vérité et vraisemblance \\[0pt]
Vérité théorique, vérité pratique \\[0pt]
Veut-on toujours savoir ? \\[0pt]
Vice et délice \\[0pt]
Vie et finalité \\[0pt]
Vie politique et vie contemplative \\[0pt]
Vie privée et vie publique \\[0pt]
Vie publique et vie privée \\[0pt]
Violence et force \\[0pt]
Violence et pouvoir \\[0pt]
« Vis caché » \\[0pt]
Vivons-nous au présent ? \\[0pt]
Vivrait-on mieux sans désirs ? \\[0pt]
Vivre au présent \\[0pt]
Vivre en société, est-ce seulement vivre ensemble ? \\[0pt]
Vivre, est-ce interpréter ? \\[0pt]
Vivre et exister \\[0pt]
Vivre et exister, est-ce la même chose ? \\[0pt]
Vivre libre \\[0pt]
Vivre sa vie \\[0pt]
Vivre ses désirs \\[0pt]
Voir et entendre \\[0pt]
Voir et observer \\[0pt]
Voir et toucher \\[0pt]
Voir le meilleur et faire le pire \\[0pt]
Voit-on ce qu'on croit ? \\[0pt]
Volonté et désir \\[0pt]
Vouloir dire \\[0pt]
Vouloir et pouvoir \\[0pt]
Vouloir être heureux \\[0pt]
Vouloir la paix sociale peut-il aller jusqu'à accepter l'injustice ? \\[0pt]
Vouloir la solitude \\[0pt]
Vouloir oublier \\[0pt]
Vouloir vivre \\[0pt]
Y a-t-il à la question « qu'est-ce que l'homme ? » une réponse scientifique ? \\[0pt]
Y a-t-il d'autres moyens que la démonstration pour établir la vérité ? \\[0pt]
Y a-t-il de bons préjugés ? \\[0pt]
Y a-t-il de justes inégalités ? \\[0pt]
Y a-t-il de la fatalité dans la vie de l'homme ? \\[0pt]
Y a-t-il de l'inconnaissable ? \\[0pt]
Y a-t-il de l'indémontrable ? \\[0pt]
Y a-t-il de mauvais désirs ? \\[0pt]
Y a-t-il des arts mineurs ? \\[0pt]
Y a-t-il des biens inestimables ? \\[0pt]
Y a-t-il des certitudes morales ? \\[0pt]
Y a-t-il des choses dont on ne peut parler ? \\[0pt]
Y a-t-il des choses qu'on n'échange pas ? \\[0pt]
Y a-t-il des connaissances dangereuses ? \\[0pt]
Y a-t-il des contraintes légitimes ? \\[0pt]
Y a-t-il des convictions philosophiques ? \\[0pt]
Y a-t-il des correspondances entre les arts ? \\[0pt]
Y a-t-il des croyances nécessaires ? \\[0pt]
Y a-t-il des degrés de conscience ? \\[0pt]
Y a-t-il des degrés de vérité ? \\[0pt]
Y a-t-il des démonstrations en philosophie ? \\[0pt]
Y a-t-il des désirs moraux ? \\[0pt]
Y a-t-il des devoirs envers soi ? \\[0pt]
Y a-t-il des erreurs de la nature ? \\[0pt]
Y a-t-il des évidences morales ? \\[0pt]
Y a-t-il des expériences sans théorie ? \\[0pt]
Y a-t-il des faits religieux ? \\[0pt]
Y a-t-il des faits scientifiques ? \\[0pt]
Y a-t-il des fins de la nature ? \\[0pt]
Y a-t-il des guerres injustes ? \\[0pt]
Y a-t-il des guerres justes ? \\[0pt]
Y a-t-il des illusions de la liberté ? \\[0pt]
Y a-t-il des inégalités justes ? \\[0pt]
Y a-t-il des injustices naturelles ? \\[0pt]
Y a-t-il des liens qui libèrent ? \\[0pt]
Y a-t-il des limites à la connaissance ? \\[0pt]
Y a-t-il des limites à la tolérance ? \\[0pt]
Y a-t-il des mondes imaginaires ? \\[0pt]
Y a-t-il des mots vides de sens ? \\[0pt]
Y a-t-il des normes naturelles ? \\[0pt]
Y a-t-il des objets qui n'existent pas ? \\[0pt]
Y a-t-il des perceptions insensibles ? \\[0pt]
Y a-t-il des peuples sans histoire ? \\[0pt]
Y a-t-il des plaisirs meilleurs que d'autres ? \\[0pt]
Y a-t-il des principes de justice universels ? \\[0pt]
Y a-t-il des progrès dans l'art ? \\[0pt]
Y a-t-il des questions sans réponse ? \\[0pt]
Y a-t-il des raisons de douter de la raison ? \\[0pt]
Y a-t-il des révolutions dans l'art ? \\[0pt]
Y a-t-il des révolutions scientifiques ? \\[0pt]
Y a-t-il des secrets de la nature ? \\[0pt]
Y a-t-il des solutions en politique ? \\[0pt]
Y a-t-il des sots métiers ? \\[0pt]
Y a-t-il des techniques de pensée ? \\[0pt]
Y a-t-il des techniques pour être heureux ? \\[0pt]
Y a-t-il des valeurs absolues ? \\[0pt]
Y a-t-il des valeurs propres à la science ? \\[0pt]
Y a-t-il des vérités de fait ? \\[0pt]
Y a-t-il des vérités définitives ? \\[0pt]
Y a-t-il des vérités en art ? \\[0pt]
Y a-t-il des vérités en politique ? \\[0pt]
Y a-t-il des vérités éternelles ? \\[0pt]
Y a-t-il des vérités indémontrables ? \\[0pt]
Y a-t-il des vérités indiscutables ? \\[0pt]
Y a-t-il des vérités morales ? \\[0pt]
Y a-t-il des vérités qui échappent à la raison ? \\[0pt]
Y a-t-il des vérités religieuses ? \\[0pt]
Y a-t-il différentes façons d'exister ? \\[0pt]
Y a-t-il du nouveau dans l'histoire ? \\[0pt]
Y a-t-il incompatibilité entre science et religion ? \\[0pt]
Y a-t-il nécessairement du religieux dans l'art ? \\[0pt]
Y a-t-il plusieurs sens du mot vie ? \\[0pt]
Y a-t-il plusieurs sortes de matières ? \\[0pt]
Y a-t-il plusieurs vérités ? \\[0pt]
Y a-t-il progrès en art ? \\[0pt]
Y a-t-il un art d'être heureux ? \\[0pt]
Y a-t-il un art de vivre ? \\[0pt]
Y a-t-il un art d'interpréter ? \\[0pt]
Y a-t-il un art du bonheur ? \\[0pt]
Y a-t-il un Bien suprême ? \\[0pt]
Y a-t-il un bonheur sans illusion ? \\[0pt]
Y a-t-il un devoir d'émancipation ? \\[0pt]
Y a-t-il un devoir de mémoire ? \\[0pt]
Y a-t-il un devoir d'être heureux ? \\[0pt]
Y a-t-il un droit à la propriété ? \\[0pt]
Y a-t-il un droit au bonheur ? \\[0pt]
Y a-t-il un droit au travail ? \\[0pt]
Y a-t-il un droit de désobéissance ? \\[0pt]
Y a-t-il un droit de mentir ? \\[0pt]
Y a-t-il un droit de révolte ? \\[0pt]
Y a-t-il un droit des peuples ? \\[0pt]
Y a-t-il un droit naturel ? \\[0pt]
Y a-t-il une beauté propre à l'objet technique ? \\[0pt]
Y a-t-il une causalité en histoire ? \\[0pt]
Y a-t-il une compétence politique ? \\[0pt]
Y a-t-il une condition humaine ? \\[0pt]
Y a-t-il une conscience collective ? \\[0pt]
Y a-t-il une définition du bonheur ? \\[0pt]
Y a-t-il une dimension métaphysique du désir ? \\[0pt]
Y a-t-il une enfance de l'humanité ? \\[0pt]
Y a-t-il une expérience de la liberté ? \\[0pt]
Y a-t-il une finalité dans la nature ? \\[0pt]
Y a-t-il une fonction propre à l'œuvre d'art ? \\[0pt]
Y a-t-il une force des faibles ? \\[0pt]
Y a-t-il une histoire de la raison ? \\[0pt]
Y a-t-il une histoire de la vérité ? \\[0pt]
Y a-t-il une histoire universelle ? \\[0pt]
Y a-t-il une justice naturelle ? \\[0pt]
Y a-t-il une logique du désir ? \\[0pt]
Y a-t-il une méthode de l'interprétation ? \\[0pt]
Y a-t-il une morale universelle ? \\[0pt]
Y a-t-il une nature humaine ? \\[0pt]
Y a-t-il une nécessité de l'erreur ? \\[0pt]
Y a-t-il une nécessité morale ? \\[0pt]
Y a-t-il une œuvre du temps ? \\[0pt]
Y a-t-il une ou des morales ? \\[0pt]
Y a-t-il une pensée technique ? \\[0pt]
Y a-t-il une primauté du devoir sur le droit ? \\[0pt]
Y a-t-il une rationalité de la contingence ? \\[0pt]
Y a-t-il une rationalité du hasard ? \\[0pt]
Y a-t-il une responsabilité de l'artiste ? \\[0pt]
Y a-t-il une sagesse de l'inconscient ? \\[0pt]
Y a-t-il une science du juste ? \\[0pt]
Y a-t-il une science ou des sciences ? \\[0pt]
Y a-t-il une science politique ? \\[0pt]
Y a-t-il une servitude volontaire ? \\[0pt]
Y a-t-il une spécificité du vivant ? \\[0pt]
Y a-t-il une technique pour tout ? \\[0pt]
Y a-t-il une unité des devoirs ? \\[0pt]
Y a-t-il une unité des sciences ? \\[0pt]
Y a-t-il une valeur de l'inutile ? \\[0pt]
Y a-t-il une vérité de l'œuvre d'art ? \\[0pt]
Y a-t-il une vérité des apparences ? \\[0pt]
Y a-t-il une vérité des représentations ? \\[0pt]
Y a-t-il une vérité en art ? \\[0pt]
Y a-t-il une vérité sur le bonheur ? \\[0pt]
Y a-t-il une vertu de l'imitation ? \\[0pt]
Y a-t-il une violence du droit ? \\[0pt]
Y a-t-il une volonté du mal ? \\[0pt]
Y a-t-il un fondement de la croyance ? \\[0pt]
Y a-t-il un jugement de l'histoire ? \\[0pt]
Y a-t-il un langage des animaux ? \\[0pt]
Y a-t-il un langage du corps ? \\[0pt]
Y a-t-il un moteur de l'histoire ? \\[0pt]
Y a-t-il un objet du désir ? \\[0pt]
Y a-t-il un ordre dans la nature ? \\[0pt]
Y a-t-il un ordre des choses ? \\[0pt]
Y a-t-il un ordre du monde ? \\[0pt]
Y a-t-il un pouvoir de la technique ? \\[0pt]
Y a-t-il un primat de la nature sur la culture ? \\[0pt]
Y a-t-il un progrès du droit ? \\[0pt]
Y a-t-il un progrès en art ? \\[0pt]
Y a-t-il un propre de l'homme ? \\[0pt]
Y a-t-il un rapport moral à soi-même ? \\[0pt]
Y a-t-il un savoir de la justice ? \\[0pt]
Y a-t-il un savoir du juste ? \\[0pt]
Y a-t-il un sens à ne plus rien désirer ? \\[0pt]
Y a-t-il un sens moral ? \\[0pt]
Y a-t-il un temps pour tout ? \\[0pt]


\subsection{CAPES interne}
\label{sec:org8c48b88}

\noindent
À chacun sa vérité ? \\[0pt]
Agir \\[0pt]
Agir par devoir, est-ce agir contre son intérêt ? \\[0pt]
Ai-je des devoirs envers moi-même ? \\[0pt]
Ai-je intérêt à la liberté d'autrui ? \\[0pt]
Ai-je le pouvoir de me changer ? \\[0pt]
Aimer la vérité \\[0pt]
Appartenons-nous à une culture ? \\[0pt]
À quelles conditions une croyance devient-elle religieuse ? \\[0pt]
À qui doit-on la vérité ? \\[0pt]
À quoi bon culpabiliser ? \\[0pt]
À quoi bon des héros ? \\[0pt]
À quoi bon parler ? \\[0pt]
À quoi bon se parler ? \\[0pt]
À quoi faut-il renoncer pour dialoguer ? \\[0pt]
À quoi la religion sert-elle ? \\[0pt]
À quoi l'art nous rend-il sensibles ? \\[0pt]
À quoi reconnaît-on la vérité ? \\[0pt]
À quoi reconnaît-on qu'une expérience est scientifique ? \\[0pt]
À quoi reconnaît-on qu'un événement est historique ? \\[0pt]
À quoi reconnaît-on un acte vraiment libre ? \\[0pt]
À quoi reconnaît-on une œuvre d'art ? \\[0pt]
À quoi reconnaît-on une religion ? \\[0pt]
À quoi reconnaît-on un être vivant ? \\[0pt]
À quoi renonçons-nous quand nous obéissons ? \\[0pt]
À quoi sert la technique ? \\[0pt]
À quoi sert la vérité ? \\[0pt]
À quoi sert l'État ? \\[0pt]
À quoi sert l'histoire ? \\[0pt]
À quoi servent les religions ? \\[0pt]
À quoi tient la force de l'État ? \\[0pt]
À quoi tient la force des religions ? \\[0pt]
Art et illusion \\[0pt]
Art et vérité \\[0pt]
A-t-on des devoirs envers les animaux ? \\[0pt]
A-t-on des devoirs envers qui n'a aucun droit ? \\[0pt]
A-t-on des devoirs envers soi-même ? \\[0pt]
A-t-on raison de mettre la technique en accusation ? \\[0pt]
Autrui, est-ce n'importe quel autre ? \\[0pt]
Avoir de l'expérience \\[0pt]
Avoir des remords \\[0pt]
Avoir raison \\[0pt]
Avons-nous besoin de Dieu ? \\[0pt]
Avons-nous des devoirs envers le passé ? \\[0pt]
Avons-nous des devoirs envers les autres êtres vivants ? \\[0pt]
Avons-nous des devoirs envers les générations futures ? \\[0pt]
Avons-nous des devoirs envers tous les vivants ? \\[0pt]
Avons-nous le devoir de dire la vérité ? \\[0pt]
Avons-nous le temps d'apprendre à vivre ? \\[0pt]
Beauté et moralité \\[0pt]
Beauté et vérité \\[0pt]
Bien agir, est-ce nécessairement faire son devoir ? \\[0pt]
Bien parler exige-t-il des qualités morales ? \\[0pt]
Bonheur et société \\[0pt]
Ce que la technique rend possible, peut-on jamais en empêcher la réalisation ? \\[0pt]
Ce que nous avons le devoir de faire peut-il toujours s'exprimer sous forme de loi ? \\[0pt]
Ce qui dépasse la raison est-il nécessairement irréel ? \\[0pt]
Ce qui n'est pas matériel peut-il être réel ? \\[0pt]
Ceux qui oppriment sont-ils libres ? \\[0pt]
Chacun a-t-il le droit d'invoquer sa vérité ? \\[0pt]
Chacun a-t-il sa propre vérité ? \\[0pt]
Chance et bonheur \\[0pt]
Changer, est-ce se trahir ? \\[0pt]
Comment connaître le passé ? \\[0pt]
Comment distingue-t-on le vrai du réel ? \\[0pt]
Comment fonder nos devoirs ? \\[0pt]
Comment le devoir peut-il déterminer l'action ? \\[0pt]
Comment l'homme peut-il se représenter le temps ? \\[0pt]
Comment penser le futur ? \\[0pt]
Comment peut-on définir un être vivant ? \\[0pt]
Comment peut-on être heureux ? \\[0pt]
Comment prend-on connaissance de ses devoirs ? \\[0pt]
Comment prouver la liberté ? \\[0pt]
Comment savoir quels sont nos devoirs ? \\[0pt]
Connaissons-nous mieux le présent que le passé ? \\[0pt]
Connaît-on jamais pour le plaisir ? \\[0pt]
Connaît-on la vie ou bien connaît-on le vivant ? \\[0pt]
Connaît-on pour le plaisir ? \\[0pt]
Conscience de soi et connaissance de soi \\[0pt]
Conscience et liberté \\[0pt]
Contrainte et obligation \\[0pt]
Convaincre \\[0pt]
Convient-il d'opposer liberté et nécessité ? \\[0pt]
Dans quelle mesure le temps nous appartient-il ? \\[0pt]
Dépend-il de soi d'être heureux ? \\[0pt]
De quelle réalité parlons-nous lorsque nous nous exprimons ? \\[0pt]
De quoi dépend le bonheur ? \\[0pt]
De quoi la technique ne peut-elle pas nous libérer ? \\[0pt]
De quoi le devoir libère-t-il ? \\[0pt]
De quoi l'État ne doit-il pas se mêler ? \\[0pt]
De quoi le tyran est-il libre ? \\[0pt]
De quoi sommes-nous responsables ? \\[0pt]
Déraisonner, est-ce perdre de vue le réel ? \\[0pt]
Désirons-nous les choses parce qu'elles sont bonnes ? \\[0pt]
Devoir, est-ce avoir une dette envers quelqu'un ? \\[0pt]
Devoir, est-ce vouloir ? \\[0pt]
Devoir et plaisir \\[0pt]
Devoirs envers les autres et devoirs envers soi-même \\[0pt]
Devons-nous apprendre à vivre ? \\[0pt]
Devons-nous être obéissants ? \\[0pt]
Devons-nous toujours dire la vérité ? \\[0pt]
Dire, est-ce trahir ? \\[0pt]
Dire la vérité \\[0pt]
Dois-je admettre tout ce que je ne peux réfuter ? \\[0pt]
Doit-on attendre de la technique qu'elle mette fin au travail ? \\[0pt]
Doit-on chercher à être heureux ? \\[0pt]
Doit-on distinguer devoir moral et obligation sociale ? \\[0pt]
Doit-on respecter les êtres vivants ? \\[0pt]
Doit-on tout accepter de l'État ? \\[0pt]
Doit-on tout attendre de l'État ? \\[0pt]
Doit-on tout pardonner ? \\[0pt]
Doit-on vraiment tout pardonner ? \\[0pt]
D'où vient la force d'une religion ? \\[0pt]
D'où vient l'amour de Dieu ? \\[0pt]
Droit et coutume \\[0pt]
Échappe-t-on à la nécessité ? \\[0pt]
Écrire l'histoire, est-ce donner un sens à la violence ? \\[0pt]
En histoire, tout est-il affaire d'interprétation ? \\[0pt]
En morale, est-ce seulement l'intention qui compte ? \\[0pt]
En politique, faut-il refuser l'utopie ? \\[0pt]
En quel sens la perception engage-t-elle le corps ? \\[0pt]
En quel sens le vivant a-t-il une histoire ? \\[0pt]
En quel sens peut-on dire que le langage est la condition de la pensée ? \\[0pt]
En quel sens peut-on dire que nos paroles nous trahissent ? \\[0pt]
En quel sens peut-on parler d'une « culture technique » ? \\[0pt]
En quoi la connaissance du vivant contribue-t-elle à la connaissance de l'homme ? \\[0pt]
En quoi la liberté n'est-elle pas une illusion ? \\[0pt]
En quoi la nature peut-elle être source de création ? \\[0pt]
En quoi le bonheur est-il l'affaire de l'État ? \\[0pt]
Entre croire et savoir, y a-t-il une différence de nature ? \\[0pt]
Envers qui avons-nous des devoirs ? \\[0pt]
Essence et existence \\[0pt]
Est-ce de la force que l'État tient son autorité ? \\[0pt]
Est-ce l'utilité qui définit un objet technique ? \\[0pt]
Est-il bien vrai qu'« on n'arrête pas le progrès » ? \\[0pt]
Est-il dans mon intérêt d'accomplir mes devoirs ? \\[0pt]
Est-il facile d'être libre ? \\[0pt]
Est-il légitime d'affirmer que seul le présent existe ? \\[0pt]
Est-il méritoire de ne faire que son devoir ? \\[0pt]
Est-il pertinent d'opposer les actes et les paroles ? \\[0pt]
Est-il pertinent d'opposer l'individuel et le collectif ? \\[0pt]
Est-il possible d'ignorer toute vérité ? \\[0pt]
Est-il raisonnable de se quereller pour des mots ? \\[0pt]
Est-il toujours moral de faire son devoir ? \\[0pt]
Estimer \\[0pt]
Est-on libre par hasard ? \\[0pt]
Est-on propriétaire de son corps ? \\[0pt]
Est-on vraiment libre d'exprimer ce que l'on veut ? \\[0pt]
Être à l'écoute de son désir, est-ce nier le désir de l'autre ? \\[0pt]
Être conscient, est-ce être maître de soi ? \\[0pt]
Être dans le vrai \\[0pt]
Être éduqué, est-ce devenir soi-même ? \\[0pt]
Être heureux, est-ce devoir ? \\[0pt]
Être libre, cela s'apprend-il ? \\[0pt]
Être libre, est-ce avoir le choix ? \\[0pt]
Être libre, est-ce faire ce que l'on veut ? \\[0pt]
Être libre, est-ce n'avoir aucun maître ? \\[0pt]
Être libre, est-ce n'obéir qu'à soi-même ? \\[0pt]
Être libre, est-ce une question de volonté ? \\[0pt]
Être libre, est-ce vivre comme on l'entend ? \\[0pt]
Être raisonnable, est-ce renoncer à ses désirs ? \\[0pt]
Être spontané, est-ce être libre ? \\[0pt]
Exister \\[0pt]
Existe-t-il au moins un devoir universel ? \\[0pt]
Existe-t-il des devoirs envers soi-même ? \\[0pt]
Expérience immédiate et expérimentation scientifique \\[0pt]
Expliquer, est-ce interpréter ? \\[0pt]
Exploiter la nature, est-ce lui vouloir du mal ? \\[0pt]
Fabriquer et créer \\[0pt]
Faire son devoir, est-ce faire son propre malheur ? \\[0pt]
Faire son devoir, est-ce là toute la morale ? \\[0pt]
Faire son devoir, est-ce obéir ? \\[0pt]
Faire son devoir, est-ce payer une dette ? \\[0pt]
Faire son devoir, est-ce perdre toute sensibilité ? \\[0pt]
Faire son devoir, est-ce se soumettre ? \\[0pt]
Faisons-nous l'histoire ? \\[0pt]
Famille et société \\[0pt]
Faut-il accorder de l'importance aux mots ? \\[0pt]
Faut-il aimer la nature plus que soi-même ? \\[0pt]
Faut-il apprendre à vivre en renonçant au bonheur ? \\[0pt]
Faut-il assimiler la pensée au langage ? \\[0pt]
Faut-il avoir foi en la technique ? \\[0pt]
Faut-il avoir peur de la technique ? \\[0pt]
Faut-il changer ses désirs plutôt que l'ordre du monde ? \\[0pt]
Faut-il chasser le naturel ? \\[0pt]
Faut-il chercher à satisfaire tous nos désirs ? \\[0pt]
Faut-il chercher à se connaître ? \\[0pt]
Faut-il comprendre pour croire ? \\[0pt]
Faut-il connaître l'Histoire pour gouverner ? \\[0pt]
Faut-il craindre de perdre son temps ? \\[0pt]
Faut-il craindre le pouvoir des machines ? \\[0pt]
Faut-il craindre l'État ? \\[0pt]
Faut-il croire les historiens ? \\[0pt]
Faut-il croire que l'histoire a un sens ? \\[0pt]
Faut-il distinguer culture et civilisation ? \\[0pt]
Faut-il distinguer désir et besoin ? \\[0pt]
Faut-il distinguer devoir moral et obligation sociale ? \\[0pt]
Faut-il être courageux pour être libre ? \\[0pt]
Faut-il être libre pour être heureux ? \\[0pt]
Faut-il faire des efforts pour être soi-même ? \\[0pt]
Faut-il faire table rase du passé ? \\[0pt]
Faut-il faire une place à la croyance ? \\[0pt]
Faut-il hiérarchiser les désirs ? \\[0pt]
Faut-il laisser faire la nature ? \\[0pt]
Faut-il limiter le pouvoir de l'État ? \\[0pt]
Faut-il obéir à la voix de sa conscience ? \\[0pt]
Faut-il opposer culture et nature ? \\[0pt]
Faut-il opposer droits et devoirs ? \\[0pt]
Faut-il opposer histoire et mémoire ? \\[0pt]
Faut-il opposer technique et nature ? \\[0pt]
Faut-il opposer théorie et expérience ? \\[0pt]
Faut-il oublier le passé pour se donner un avenir ? \\[0pt]
Faut-il préférer le bonheur à la vérité ? \\[0pt]
Faut-il privilégier le naturel ? \\[0pt]
Faut-il rechercher la perfection dans le langage ? \\[0pt]
Faut-il rechercher le bonheur ? \\[0pt]
Faut-il renoncer à la vérité ? \\[0pt]
Faut-il renoncer à rechercher la vérité ? \\[0pt]
Faut-il respecter la nature ? \\[0pt]
Faut-il respecter le vivant ? \\[0pt]
Faut-il rester soi-même ? \\[0pt]
Faut-il sauver les apparences ? \\[0pt]
Faut-il se connaître pour être maître de soi ? \\[0pt]
Faut-il se fier à sa propre raison ? \\[0pt]
Faut-il se libérer de soi-même ? \\[0pt]
Faut-il se libérer du travail ? \\[0pt]
Faut-il se libérer pour être libre ? \\[0pt]
Faut-il se méfier de ses désirs ? \\[0pt]
Faut-il se méfier de soi-même ? \\[0pt]
Faut-il se méfier du progrès technique ? \\[0pt]
Faut-il toujours dire la vérité ? \\[0pt]
Faut-il une méthode pour découvrir la vérité ? \\[0pt]
Faut-il vivre avec son temps ? \\[0pt]
Gouverner \\[0pt]
Histoire et mémoire \\[0pt]
Histoire et progrès \\[0pt]
Identité et appartenance \\[0pt]
Interpréter, est-ce renoncer à toute objectivité ? \\[0pt]
Justice et pardon \\[0pt]
La barrière de la langue \\[0pt]
La beauté s'explique-t-elle ? \\[0pt]
La causalité \\[0pt]
La certitude est-elle une marque de vérité ? \\[0pt]
La cohérence suffit-elle à la vérité ? \\[0pt]
La connaissance de soi est-elle évidente ? \\[0pt]
La connaissance du passé est-elle nécessaire à la compréhension du présent ? \\[0pt]
La connaissance du vivant est-elle désintéressée ? \\[0pt]
La connaissance du vivant peut-elle être désintéressée ? \\[0pt]
La connaissance historique est-elle une interprétation des faits ? \\[0pt]
La connaissance historique est-elle utile à l'homme ? \\[0pt]
La connaissance scientifique \\[0pt]
La connaissance scientifique est-elle désintéressée ? \\[0pt]
La connaissance sensible \\[0pt]
La conscience collective \\[0pt]
La conscience de soi \\[0pt]
La conscience de soi est-elle une méconnaissance de soi ? \\[0pt]
La conscience de soi et l'identité personnelle \\[0pt]
La conscience est-elle source d'illusions ? \\[0pt]
La conscience est-elle toujours morale ? \\[0pt]
La conscience est-elle une activité ? \\[0pt]
La conscience morale \\[0pt]
La conscience morale est-elle une illusion ? \\[0pt]
La conscience morale n'est-elle que le fruit de l'éducation ? \\[0pt]
La conscience peut-elle nous tromper ? \\[0pt]
La contingence est-elle une condition de la liberté ? \\[0pt]
La contrainte des lois est-elle une violence ? \\[0pt]
La contrainte s'oppose-t-elle à la liberté ? \\[0pt]
L'acte et la parole \\[0pt]
L'action morale dépend-elle des circonstances ? \\[0pt]
L'activité technique dévalorise-t-elle l'homme ? \\[0pt]
La culture est-elle ce qui unit ou ce qui sépare les hommes ? \\[0pt]
La culture historique forge-t-elle le jugement moral ? \\[0pt]
La décision \\[0pt]
La domination technique de la nature doit-elle susciter la crainte ou l'espoir ? \\[0pt]
La finalité est-elle nécessaire pour penser le vivant ? \\[0pt]
La fin de la technique se résume-t-elle à son utilité ? \\[0pt]
La fin du travail \\[0pt]
La fraternité a-t-elle un sens politique ? \\[0pt]
La fuite du temps est-elle nécessairement un malheur ? \\[0pt]
La joie \\[0pt]
La liberté comporte-t-elle des degrés ? \\[0pt]
La liberté de croire \\[0pt]
La liberté doit-elle être limitée ? \\[0pt]
La liberté est-elle contraire au principe de causalité ? \\[0pt]
La liberté est-elle innée ? \\[0pt]
La liberté est-elle le fondement de la responsabilité ? \\[0pt]
La liberté est-elle une illusion nécessaire ? \\[0pt]
La liberté est-elle un fait ? \\[0pt]
La liberté impose-t-elle des devoirs ? \\[0pt]
La liberté n'est-elle qu'un droit ? \\[0pt]
La liberté peut-elle être prouvée ? \\[0pt]
La liberté peut-elle se constater ? \\[0pt]
La liberté peut-elle se prouver ? \\[0pt]
La liberté peut-elle se refuser ? \\[0pt]
La liberté se mérite-t-elle ? \\[0pt]
La liberté se réduit-elle au libre arbitre ? \\[0pt]
La liberté suppose-t-elle l'absence de déterminisme ? \\[0pt]
L'aliénation \\[0pt]
La limite \\[0pt]
La machine fournit-elle un modèle pour penser le vivant ? \\[0pt]
La maladie est-elle à l'organisme vivant ce que la panne est à la machine ? \\[0pt]
La maladie est-elle indispensable à la connaissance du vivant ? \\[0pt]
La matière est-elle plus aisée à connaître que l'esprit ? \\[0pt]
La mauvaise conscience \\[0pt]
L'ambiguïté des mots peut-elle être heureuse ? \\[0pt]
La méthode \\[0pt]
La méthode expérimentale est-elle appropriée à l'étude du vivant ? \\[0pt]
L'amitié peut-elle obliger ? \\[0pt]
La morale a-t-elle besoin d'un fondement ? \\[0pt]
La morale dépend-elle de la culture ? \\[0pt]
La morale n'est-elle qu'un dressage ? \\[0pt]
La moralité se juge-t-elle aux actes ? \\[0pt]
L'amour de soi \\[0pt]
La nature donne-t-elle un sens à notre existence ? \\[0pt]
La nature est-elle écrite en langage mathématique ? \\[0pt]
La nature est-elle une invention de l'homme ? \\[0pt]
La nature est-elle un guide ? \\[0pt]
La nature fait-elle bien les choses ? \\[0pt]
La nature nous domine-t-elle ? \\[0pt]
Langage et communication \\[0pt]
L'animal \\[0pt]
L'animal a-t-il des droits ? \\[0pt]
L'animal nous apprend-il quelque chose sur l'homme ? \\[0pt]
L'anomalie \\[0pt]
La non-violence \\[0pt]
La nouveauté \\[0pt]
La paix met-elle fin à l'histoire ? \\[0pt]
La parole \\[0pt]
La parole et l'écriture \\[0pt]
La parole et le geste \\[0pt]
La parole nous libère-t-elle ? \\[0pt]
La passion n'est-elle que souffrance ? \\[0pt]
La patience \\[0pt]
La pénibilité du travail \\[0pt]
La pensée peut-elle devenir une technique ? \\[0pt]
La perception est-elle l'interprétation du réel ? \\[0pt]
La pluralité des vérités condamne-t-elle l'idée de vérité ? \\[0pt]
La politique : affaire de compétence ? \\[0pt]
La politique doit-elle protéger la liberté des citoyens ? \\[0pt]
La politique doit-elle refuser l'utopie ? \\[0pt]
La politique est-elle l'affaire de tous ? \\[0pt]
La politique est-elle une science ? \\[0pt]
La politique est-elle une technique ? \\[0pt]
La proportion \\[0pt]
La puissance technique est- elle sans limites ? \\[0pt]
La raison a-t-elle des limites ? \\[0pt]
La raison doit-elle être cultivée ? \\[0pt]
La raison doit-elle se soumettre au réel ? \\[0pt]
La raison engendre-t-elle des illusions ? \\[0pt]
La raison épuise-t-elle le réel ? \\[0pt]
La raison est-elle l'esclave du désir ? \\[0pt]
La raison est-elle plus fiable que l'expérience ? \\[0pt]
La raison est-elle seulement affaire de logique ? \\[0pt]
La raison ne connaît-elle du réel que ce qu'elle y met d'elle-même ? \\[0pt]
La raison peut-elle entrer en conflit avec elle-même ? \\[0pt]
La raison peut-elle errer ? \\[0pt]
La raison peut-elle nous commander de croire ? \\[0pt]
La raison peut-elle rendre compte du réel entièrement ? \\[0pt]
La raison peut-elle s'opposer à elle-même ? \\[0pt]
La raison transforme-t-elle le réel ? \\[0pt]
La rationalité technique \\[0pt]
La réalisation du devoir exclut-elle toute forme de plaisir ? \\[0pt]
La réalité du temps se réduit-elle à la conscience que nous en avons ? \\[0pt]
La recherche de la vérité a-t- elle une fin ? \\[0pt]
La recherche de la vérité peut-elle être une passion ? \\[0pt]
La religion a-t-elle des vertus ? \\[0pt]
La religion a-t-elle les mêmes fins que la morale ? \\[0pt]
La religion divise-t-elle les hommes ? \\[0pt]
La religion est-elle contraire à la raison ? \\[0pt]
La religion est-elle relation à l'absolu ? \\[0pt]
La religion est-elle une consolation pour les hommes ? \\[0pt]
La religion est-elle une production culturelle comme les autres ? \\[0pt]
La religion est-elle un facteur de lien social ? \\[0pt]
La religion implique-t-elle la croyance en un être divin ? \\[0pt]
La religion impose t-elle un joug salutaire à l'intelligence ? \\[0pt]
La religion n'est-elle qu'une affaire privée ? \\[0pt]
La religion n'est-elle qu'un fait de culture ? \\[0pt]
La religion relève-t-elle de l'irrationnel ? \\[0pt]
La religion relie-t-elle les hommes ? \\[0pt]
La religion rend-elle l'homme heureux ? \\[0pt]
La religion rend-elle meilleur ? \\[0pt]
La religion repose-t-elle sur une illusion ? \\[0pt]
La religion se distingue-t-elle de la superstition ? \\[0pt]
La religion se réduit-elle à la foi ? \\[0pt]
La responsabilité \\[0pt]
L'art a-t-il pour fin la vérité ? \\[0pt]
L'art a-t-il pour fin le plaisir ? \\[0pt]
L'art est-il le règne des apparences ? \\[0pt]
L'art est-il moins nécessaire que la science ? \\[0pt]
L'art est-il une affaire sérieuse ? \\[0pt]
L'art est-il une forme de langage ? \\[0pt]
L'art est-il un jeu ? \\[0pt]
L'art est-il un luxe ? \\[0pt]
L'art et le beau \\[0pt]
L'art et le réel \\[0pt]
L'art exprime-t-il ce que nous ne saurions dire ? \\[0pt]
L'artiste doit-il être original ? \\[0pt]
L'artiste doit-il se donner des modèles ? \\[0pt]
L'artiste et l'artisan \\[0pt]
L'art n'est qu'une affaire de goût ? \\[0pt]
L'art nous fait-il mieux percevoir le réel ? \\[0pt]
L'art nous réconcilie-t-il avec le monde ? \\[0pt]
L'art peut-il mourir ? \\[0pt]
L'art peut-il s'enseigner ? \\[0pt]
L'art peut-il se passer de règles ? \\[0pt]
L'art rend-il les hommes meilleurs ? \\[0pt]
La science commence-t-elle avec la perception ? \\[0pt]
La science est-elle indépendante de toute métaphysique ? \\[0pt]
La science nous éloigne-t-elle de la religion ? \\[0pt]
La sensibilité se cultive-t-elle ? \\[0pt]
La servitude peut-elle être volontaire ? \\[0pt]
La sincérité \\[0pt]
La société peut-elle se passer de l'État ? \\[0pt]
La technique a-t-elle une finalité ? \\[0pt]
La technique augmente-t-elle notre puissance d'agir ? \\[0pt]
La technique change-t-elle l'homme ? \\[0pt]
La technique change-t-elle notre rapport au monde ? \\[0pt]
La technique déshumanise-t-elle le monde ? \\[0pt]
La technique détermine-t-elle les rapports sociaux ? \\[0pt]
La technique doit-elle nous libérer du travail ? \\[0pt]
La technique doit-elle permettre de dépasser les limites de l'humain ? \\[0pt]
La technique donne-t-elle une illusion de pouvoir ? \\[0pt]
La technique est-elle le propre de l'homme ? \\[0pt]
La technique est-elle libératrice ? \\[0pt]
La technique est-elle neutre ? \\[0pt]
La technique est-elle une forme de savoir ? \\[0pt]
La technique facilite-t-elle la vie ? \\[0pt]
La technique fait-elle des miracles ? \\[0pt]
La technique imite-t-elle la nature ? \\[0pt]
La technique met-elle le monde à notre disposition ? \\[0pt]
La technique ne pose-t-elle que des problèmes techniques ? \\[0pt]
La technique n'est-elle qu'une application de la science ? \\[0pt]
La technique n'est-elle qu'une ruse ? \\[0pt]
La technique n'est-elle qu'un moyen ? \\[0pt]
La technique n'est-elle qu'un moyen de domination ? \\[0pt]
La technique nous délivre-t-elle d'un rapport irrationnel au monde ? \\[0pt]
La technique nous éloigne-t-elle de la réalité ? \\[0pt]
La technique pense-t-elle ? \\[0pt]
La technique permet-elle de réaliser tous les désirs ? \\[0pt]
La technique peut-elle être tenue pour la forme moderne de la culture ? \\[0pt]
La technique peut-elle respecter la nature ? \\[0pt]
La technique pose-t-elle plus de problèmes qu'elle n'en résout ? \\[0pt]
La technique produit-elle autre chose que condition de la pensée ? \\[0pt]
La technique produit-elle son propre savoir ? \\[0pt]
La technique provoque-t-elle inévitablement des catastrophes ? \\[0pt]
La technique s'oppose-t-elle à la nature ? \\[0pt]
La terre est-elle un bien commun ? \\[0pt]
L'athéisme condamne-t-il l'existence à l'absurdité ? \\[0pt]
L'athéisme est-il une croyance ? \\[0pt]
La traduction \\[0pt]
L'autonomie \\[0pt]
La valeur \\[0pt]
La valeur d'une action se mesure-t-elle à sa réussite ? \\[0pt]
La vanité \\[0pt]
L'avenir peut-il être objet de connaissance ? \\[0pt]
La vérification expérimentale \\[0pt]
La vérité a-t-elle une histoire ? \\[0pt]
La vérité doit-elle toujours être démontrée ? \\[0pt]
La vérité est-elle préférable à l'illusion ? \\[0pt]
La vérité est-elle une idole ? \\[0pt]
La vérité est-elle une valeur ? \\[0pt]
La vérité n'est-elle qu'un idéal inaccessible ? \\[0pt]
La vérité nous appartient-elle ? \\[0pt]
La vérité nous contraint-elle ? \\[0pt]
La vérité peut-elle faire peur ? \\[0pt]
La vérité peut-elle se discuter ? \\[0pt]
La vérité requiert-elle du courage ? \\[0pt]
La vérité se discute-t-elle ? \\[0pt]
La vertu peut-elle s'enseigner ? \\[0pt]
La vie est-elle le bien le plus précieux ? \\[0pt]
La vie intérieure \\[0pt]
La violence a-t-elle un sens dans l'histoire ? \\[0pt]
La violence de l'État \\[0pt]
La violence engendre-t-elle toujours la violence ? \\[0pt]
La violence est-elle la condition du progrès ? \\[0pt]
La violence est-elle le prix du progrès ? \\[0pt]
La violence est-elle toujours injustifiable ? \\[0pt]
La volonté et le désir \\[0pt]
Le bavardage \\[0pt]
Le beau et le bien \\[0pt]
Le beau et l'utile \\[0pt]
Le bonheur est-il l'absence de maux ? \\[0pt]
Le bonheur est-il la fin de la vie ? \\[0pt]
Le bonheur est-il le bien suprême ? \\[0pt]
Le bonheur est-il le prix de la vertu ? \\[0pt]
Le bonheur est-il un droit ? \\[0pt]
Le bonheur n'est-il qu'un idéal ? \\[0pt]
Le bonheur peut-il être le but de la politique ? \\[0pt]
Le bonheur peut-il être un droit ? \\[0pt]
Le bonheur se mérite-t-il ? \\[0pt]
Le bon sens \\[0pt]
Le bricolage \\[0pt]
Le chef-d'œuvre \\[0pt]
Le commencement \\[0pt]
Le concret \\[0pt]
Le consensus peut-il faire le vrai ? \\[0pt]
Le corps est-il un tout ? \\[0pt]
L'écriture ne sert-elle qu'à consigner la pensée ? \\[0pt]
Le désir a-t-il un objet ? \\[0pt]
Le désir est-il aveugle ? \\[0pt]
Le désir est-il ce qui nous fait vivre ? \\[0pt]
Le désir est-il désir de l'autre ? \\[0pt]
Le désir est-il l'essence de l'homme ? \\[0pt]
Le désir est-il par nature illimité ? \\[0pt]
Le désir n'est-il que l'épreuve d'un manque ? \\[0pt]
Le désir peut-il être désintéressé ? \\[0pt]
Le désordre \\[0pt]
Le despotisme \\[0pt]
Le développement de la technique est-il toujours facteur de progrès ? \\[0pt]
Le devenir \\[0pt]
Le devoir nous apparaît-il d'autant mieux qu'on ne l'a pas accompli ? \\[0pt]
Le dialogue \\[0pt]
Le droit et le devoir \\[0pt]
Le droit peut-il être naturel ? \\[0pt]
L'efficacité technique est-elle la norme de toute réussite ? \\[0pt]
Le futur est-il contingent ? \\[0pt]
Le futur nous appartient-il ? \\[0pt]
L'égalité \\[0pt]
L'égalité est-elle toujours juste ? \\[0pt]
L'égalité peut-elle être une menace pour la liberté ? \\[0pt]
Le geste et la parole \\[0pt]
Le geste technique exprime t-il une liberté sans fin ? \\[0pt]
Le hasard n'est-il que la mesure de notre ignorance ? \\[0pt]
Le jugement moral \\[0pt]
Le langage a-t-il une prétention à la vérité ? \\[0pt]
Le langage est-il le moyen d'expression le plus efficace ? \\[0pt]
Le langage et la pensée \\[0pt]
Le langage ne sert-il qu'à communiquer ? \\[0pt]
Le langage n'est-il qu'un instrument de communication ? \\[0pt]
Le langage nous donne-t-il une connaissance des choses ? \\[0pt]
Le langage nous éloigne-t-il de la réalité ? \\[0pt]
Le langage peut-il être un obstacle à la recherche de la vérité ? \\[0pt]
Le langage reste-t-il toujours à la surface des choses ? \\[0pt]
Le langage voile-t-il les choses ? \\[0pt]
Le libre cours de l'imagination est-il libérateur ? \\[0pt]
Le mal \\[0pt]
Le malheur est-il injuste ? \\[0pt]
Le mensonge est-il une forme d'indifférence à la vérité ? \\[0pt]
Le métier de politique \\[0pt]
Le moi \\[0pt]
Le moi est-il insaisissable ? \\[0pt]
Le moi et la conscience \\[0pt]
Le moi n'est-il qu'une fiction ? \\[0pt]
Le moi reste-t-il identique à lui-même au cours du temps ? \\[0pt]
Le monde n'est-il que l'ensemble de ce qui existe ? \\[0pt]
Le monde n'est-il qu'une représentation ? \\[0pt]
Le naturel a-t-il toutes les vertus ? \\[0pt]
L'enseignement de la vérité \\[0pt]
Le passé a-t-il plus de réalité que l'avenir ? \\[0pt]
Le passé a-t-il une réalité ? \\[0pt]
Le passé détermine-t-il notre présent ? \\[0pt]
Le passé, est-ce du passé ? \\[0pt]
Le passé peut-il être un objet de connaissance ? \\[0pt]
Le pessimisme \\[0pt]
Le plaisir esthétique est-il affaire de jugement ? \\[0pt]
Le plaisir et la joie \\[0pt]
Le pouvoir de l'État est-il arbitraire ? \\[0pt]
Le pouvoir des mots \\[0pt]
Le progrès \\[0pt]
Le progrès de l'humanité se réduit-il au progrès technique ? \\[0pt]
Le progrès technique a-t-il une fin ? \\[0pt]
Le rationnel et l'irrationnel \\[0pt]
Le récit historique \\[0pt]
Le réel est-il ce que l'on croit ? \\[0pt]
Le réel est-il inaccessible ? \\[0pt]
Le réel est-il rationnel ? \\[0pt]
Le réel obéit-il à la raison ? \\[0pt]
Le réel se limite-t-il à ce que nous percevons ? \\[0pt]
Le réel se réduit-il à ce que l'on perçoit ? \\[0pt]
Le réel se réduit-il à l'objectivité ? \\[0pt]
Le remords est-il stérile ? \\[0pt]
Le rôle de l'État est-il de faire régner la justice ? \\[0pt]
Le rôle de l'État est-il de préserver la liberté de l'individu ? \\[0pt]
Le rôle de l'historien est-il de juger ? \\[0pt]
Le salaire \\[0pt]
Les choses \\[0pt]
Les concepts sont-ils des fictions ? \\[0pt]
Les conflits de devoirs \\[0pt]
Les croyances religieuses sont-elles indiscutables ? \\[0pt]
Les désirs ont-ils nécessairement un objet ? \\[0pt]
Les divers sens du mot culture peuvent-ils être ramenés à l'unité ? \\[0pt]
Le sens du devoir \\[0pt]
Le sens du devoir nous fait-il connaître le bien de façon indiscutable ? \\[0pt]
Le sentiment du devoir accompli \\[0pt]
Le serment \\[0pt]
Les États ont-ils pour fin d'assurer la paix ? \\[0pt]
Les événements historiques sont-ils de nature imprévisible ? \\[0pt]
Les hommes peuvent-ils s'associer sans renoncer à leur liberté ? \\[0pt]
Les hommes sont-ils faits pour s'entendre ? \\[0pt]
Le silence est-il signifiant ? \\[0pt]
Les langues sont-elles condamnées à l'approximation ? \\[0pt]
Les limites de la vérité \\[0pt]
Les limites du pouvoir \\[0pt]
Les maladies de l'âme \\[0pt]
Les mots disent-ils les choses ? \\[0pt]
Les mots rendent-ils compte de la nature des choses ? \\[0pt]
Le soi est-il inaccessible ? \\[0pt]
Le souverain bien \\[0pt]
L'esprit est-il libre ? \\[0pt]
L'esprit est-il plus aisé à connaître que le corps ? \\[0pt]
L'esprit est-il réductible à la matière ? \\[0pt]
L'esprit n'a-t-il jamais affaire qu'à lui-même ? \\[0pt]
Les progrès de la technique sont-ils nécessairement des progrès de la raison ? \\[0pt]
Les règles du langage limitent-elles notre liberté d'expression ? \\[0pt]
Les règles du langage sont- elles un obstacle à la communication ? \\[0pt]
Les religions naissent-elles du besoin de justice ? \\[0pt]
Les religions peuvent-elles être objets de science ? \\[0pt]
Les religions peuvent-elles prétendre libérer les hommes ? \\[0pt]
Les religions sont-elles affaire de foi ? \\[0pt]
Les sciences ont-elles le monopole de la vérité ? \\[0pt]
Les sciences pensent-elles leurs fondements ? \\[0pt]
Les sciences permettent-elles de connaître la réalité-même ? \\[0pt]
Les théories scientifiques décrivent-elles la réalité ? \\[0pt]
Le sublime \\[0pt]
Le sujet du droit \\[0pt]
L'État a-t-il des intérêts propres ? \\[0pt]
L'État a-t-il pour but de maintenir l'ordre ? \\[0pt]
L'État a-t-il tous les droits ? \\[0pt]
L'État doit-il éduquer le peuple ? \\[0pt]
L'État doit-il être sans pitié ? \\[0pt]
L'État doit-il préférer l'injustice au désordre ? \\[0pt]
L'État doit-il se soucier de la morale ? \\[0pt]
L'État doit-il veiller au bonheur des individus ? \\[0pt]
L'État est-il l'ennemi de la liberté ? \\[0pt]
L'État est-il l'ennemi de l'individu ? \\[0pt]
L'État est-il libérateur ? \\[0pt]
L'État est-il oppresseur ? \\[0pt]
L'État est-il toujours juste ? \\[0pt]
L'État est-il un mal nécessaire ? \\[0pt]
L'État est-il un moindre mal ? \\[0pt]
L'État n'est-il qu'un instrument de domination ? \\[0pt]
L'État nous rend-il meilleurs ? \\[0pt]
L'État peut-il demeurer indifférent à la religion ? \\[0pt]
L'État peut-il être impartial ? \\[0pt]
L'État peut-il renoncer à la violence ? \\[0pt]
Le témoignage des sens \\[0pt]
Le temps dépend-il de la mémoire ? \\[0pt]
Le temps détruit-il tout ? \\[0pt]
Le temps est-il essentiellement destructeur ? \\[0pt]
Le temps est-il la marque de notre impuissance ? \\[0pt]
Le temps est-il notre ennemi ? \\[0pt]
Le temps libre \\[0pt]
Le temps n'existe-t-il que subjectivement ? \\[0pt]
Le temps nous appartient-il ? \\[0pt]
Le temps passe-t-il ? \\[0pt]
Le temps se mesure-t-il ? \\[0pt]
Le tout est-il la somme de ses parties ? \\[0pt]
Le travail et le temps \\[0pt]
Le travail fait-il de l'homme un être moral ? \\[0pt]
L'être humain est-il par nature un être religieux \\[0pt]
L'événement \\[0pt]
L'événement historique a-t-il un sens par lui-même ? \\[0pt]
L'évidence a-t-elle une valeur absolue ? \\[0pt]
L'évidence est-elle le signe de la vérité ? \\[0pt]
L'évidence est-elle un obstacle ou un instrument de la recherche de la vérité ? \\[0pt]
Le vivant définit-il ses propres normes ? \\[0pt]
Le vivant est-il entièrement connaissable ? \\[0pt]
Le vivant est-il entièrement explicable ? \\[0pt]
Le vivant n'est-il que matière ? \\[0pt]
Le vivant n'est-il qu'une machine ingénieuse ? \\[0pt]
Le vivant obéit-il à des lois ? \\[0pt]
Le vivant obéit-il à une nécessité ? \\[0pt]
Le vrai et le vraisemblable \\[0pt]
Le vrai n'est-il que ce sur quoi l'on s'accorde ? \\[0pt]
L'expérience de la vie produit-elle un savoir ? \\[0pt]
L'expérience de pensée \\[0pt]
L'expérience de soi-même \\[0pt]
L'expérience est-elle un guide suffisant ? \\[0pt]
L'expression « perdre son temps » a-t-elle un sens ? \\[0pt]
L'harmonie \\[0pt]
L'histoire a-t-elle un commencement et une fin ? \\[0pt]
L'histoire a-t-elle une fin ? \\[0pt]
L'histoire a-t-elle un sens ? \\[0pt]
L'histoire est-elle la connaissance du passé humain ? \\[0pt]
L'histoire est-elle la mémoire de l'humanité ? \\[0pt]
L'histoire est-elle le récit objectif des faits passés ? \\[0pt]
L'histoire est-elle le règne du hasard ? \\[0pt]
L'histoire est-elle le théâtre des passions ? \\[0pt]
L'histoire est-elle une science ? \\[0pt]
L'histoire est-elle une science comme les autres ? \\[0pt]
L'histoire jugera-t-elle ? \\[0pt]
L'histoire n'a-t-elle pour objet que le passé ? \\[0pt]
L'histoire n'est-elle que le produit de rapports de force ? \\[0pt]
L'histoire n'est-elle qu'un déchaînement de violence ? \\[0pt]
L'histoire nous appartient-elle ? \\[0pt]
L'histoire obéit-elle à des lois ? \\[0pt]
L'histoire pèse-t-elle comme un destin sur les individus ? \\[0pt]
L'histoire peut-elle être contemporaine ? \\[0pt]
L'histoire se répète-t-elle ? \\[0pt]
L'historien \\[0pt]
L'historien peut-il être impartial ? \\[0pt]
L'homme est-il face à la nature ? \\[0pt]
L'homme est-il la mesure de toute chose ? \\[0pt]
L'homme est-il le seul être à avoir une histoire ? \\[0pt]
L'homme est-il libre par nature ? \\[0pt]
L'homme est-il par nature un être religieux ? \\[0pt]
L'homme est-il un animal religieux ? \\[0pt]
L'homme est-il un animal technicien ? \\[0pt]
L'homme est-il un être social par nature ? \\[0pt]
L'homme et le citoyen \\[0pt]
L'homme injuste peut-il être heureux ? \\[0pt]
L'homme peut-il faire l'objet d'une science ? \\[0pt]
L'hypothèse de la liberté est-elle compatible avec les exigences de la raison ? \\[0pt]
Liberté et responsabilité \\[0pt]
Liberté et vérité \\[0pt]
L'idée de bonheur collectif a-t-elle un sens ? \\[0pt]
L'idée de devoir requiert-elle l'idée de liberté ? \\[0pt]
L'idée de république \\[0pt]
L'idée d'une liberté totale a-t- elle un sens ? \\[0pt]
L'idée d'une religion personnelle a-t-elle un sens ? \\[0pt]
L'illusion \\[0pt]
L'image \\[0pt]
L'imagination \\[0pt]
L'imitation \\[0pt]
L'inconscient et l'oubli \\[0pt]
L'inconscient n'est-il qu'un défaut de conscience ? \\[0pt]
L'indécision \\[0pt]
L'indicible \\[0pt]
L'individu \\[0pt]
L'individu est-il ineffable ? \\[0pt]
L'inexpérience \\[0pt]
L'inquiétude \\[0pt]
L'instrument et la machine \\[0pt]
L'interprétation est-elle sans fin ? \\[0pt]
L'irrationnel est-il nécessairement absurde ? \\[0pt]
L'objectivité historique est-elle synonyme de neutralité ? \\[0pt]
L'objet du désir en est-il la cause ? \\[0pt]
L'œuvre d'art doit-elle être belle ? \\[0pt]
L'œuvre d'art nous apprend-elle quelque chose ? \\[0pt]
L'ordre du vivant est-il façonné par le hasard ? \\[0pt]
L'organisme \\[0pt]
L'outil et la machine \\[0pt]
L'unanimité est-elle un critère de vérité ? \\[0pt]
L'unité de la nature \\[0pt]
L'unité des sciences \\[0pt]
Machine et organisme \\[0pt]
Mesurer \\[0pt]
Mon devoir dépend-il de moi ? \\[0pt]
N'a-t-on des devoirs qu'envers autrui ? \\[0pt]
N'avons-nous affaire qu'au réel ? \\[0pt]
N'avons-nous de devoirs qu'envers autrui ? \\[0pt]
Nécessité et contingence \\[0pt]
Ne désire-t-on que ce dont on manque ? \\[0pt]
Ne désirons-nous que ce qui est bon pour nous ? \\[0pt]
Ne désirons-nous que les choses que nous estimons bonnes ? \\[0pt]
Ne doit-on tenir pour vrai que ce qui est démontré ? \\[0pt]
Ne faut-il pas craindre la liberté ? \\[0pt]
Ne faut-il vivre que dans le présent ? \\[0pt]
Ne faut-il vivre que pour faire son devoir ? \\[0pt]
Ne sait-on rien que par expérience ? \\[0pt]
Ne sommes-nous justes que par intérêt ? \\[0pt]
Ne sommes-nous véritablement maîtres que de nos pensées ? \\[0pt]
Nier la liberté de l'homme, est-ce ôter tout sens à la morale ? \\[0pt]
Nos désirs nous appartiennent-ils ? \\[0pt]
Nos désirs nous opposent-ils ? \\[0pt]
Nos devoirs se ramènent-ils aux conventions sociales ? \\[0pt]
Nos habitudes compromettent-elles notre moralité ? \\[0pt]
Notre liberté est-elle toujours relative ? \\[0pt]
Notre pensée est-elle prisonnière de la langue que nous parlons ? \\[0pt]
Notre rapport au monde peut-il n'être que technique ? \\[0pt]
N'y a-t-il de beauté qu'artistique ? \\[0pt]
N'y a-t-il de bonheur qu'éphémère ? \\[0pt]
N'y a-t-il de liberté qu'individuelle ? \\[0pt]
N'y a-t-il de rationalité que scientifique ? \\[0pt]
N'y a-t-il de vérité que scientifique ? \\[0pt]
N'y a-t-il que l'intention qui compte ? \\[0pt]
Obéir aux lois, est-ce renoncer à la liberté ? \\[0pt]
Observation et expérience \\[0pt]
Observer et comprendre \\[0pt]
Observer et expérimenter \\[0pt]
Où commence la liberté ? \\[0pt]
Par le langage, peut-on agir sur la réalité ? \\[0pt]
Parler, est-ce ne pas agir ? \\[0pt]
Parler pour ne rien dire \\[0pt]
Penser, est-ce dire non ? \\[0pt]
Percevoir, est-ce interpréter ? \\[0pt]
Percevoir, est-ce juger ? \\[0pt]
Percevoir s'apprend-il ? \\[0pt]
Perçoit-on le réel ? \\[0pt]
Peut-il exister une morale sceptique ? \\[0pt]
Peut-il y avoir des choix collectifs ? \\[0pt]
Peut-il y avoir des lois de l'histoire ? \\[0pt]
Peut-il y avoir une pensée sans langage ? \\[0pt]
Peut-il y avoir une science du désordre ? \\[0pt]
Peut-il y avoir une science du moi ? \\[0pt]
Peut-il y avoir une vérité religieuse ? \\[0pt]
Peut-on abolir la religion ? \\[0pt]
Peut-on affirmer que le monde a un ordre ? \\[0pt]
Peut-on agir de manière désintéressée ? \\[0pt]
Peut-on apprendre à être libre ? \\[0pt]
Peut-on avoir des droits sans avoir de devoirs ? \\[0pt]
Peut-on avoir le droit de se révolter ? \\[0pt]
Peut-on avoir raisons contre les faits ? \\[0pt]
Peut-on avoir raison tout seul ? \\[0pt]
Peut-on cesser de désirer ? \\[0pt]
Peut-on changer le cours de l'histoire ? \\[0pt]
Peut-on choisir de renoncer à sa liberté ? \\[0pt]
Peut-on concevoir une religion dans les limites de la simple raison ? \\[0pt]
Peut-on concilier le droit à la liberté et l'égalité des droits ? \\[0pt]
Peut-on connaître le vivant sans le dénaturer ? \\[0pt]
Peut-on connaître sans concept ? \\[0pt]
Peut-on considérer la main comme un outil ? \\[0pt]
Peut-on contredire l'expérience ? \\[0pt]
Peut-on convaincre quelqu'un de la beauté d'une œuvre d'art ? \\[0pt]
Peut-on croire librement ? \\[0pt]
Peut-on définir la matière ? \\[0pt]
Peut-on définir la vie ? \\[0pt]
Peut-on démontrer l'existence de la matière ? \\[0pt]
Peut-on démontrer que l'on est libre ? \\[0pt]
Peut-on désirer l'absolu ? \\[0pt]
Peut-on désirer l'impossible ? \\[0pt]
Peut-on désirer sans souffrir ? \\[0pt]
Peut-on désobéir à l'État ? \\[0pt]
Peut-on dire que les hommes font l'histoire ? \\[0pt]
Peut-on dire que rien n'échappe à la technique ? \\[0pt]
Peut-on distinguer entre de bons et de mauvais désirs ? \\[0pt]
Peut-on distinguer entre les bons et les mauvais désirs ? \\[0pt]
Peut-on distinguer la vie et l'existence ? \\[0pt]
Peut-on distinguer le réel de l'imaginaire ? \\[0pt]
Peut-on douter de tout ? \\[0pt]
Peut-on échapper à ses désirs ? \\[0pt]
Peut-on échapper au cours de l'histoire ? \\[0pt]
Peut-on échapper au temps ? \\[0pt]
Peut-on en finir avec la violence ? \\[0pt]
Peut-on être athée ? \\[0pt]
Peut-on être citoyen du monde ? \\[0pt]
Peut-on être en conflit avec soi-même ? \\[0pt]
Peut-on être étranger à soi-même ? \\[0pt]
Peut-on être heureux sans philosophie ? \\[0pt]
Peut-on être indifférent à l'histoire ? \\[0pt]
Peut-on être injuste avec soi-même ? \\[0pt]
Peut-on être insensible au vrai ? \\[0pt]
Peut-on être moral sans religion ? \\[0pt]
Peut-on expérimenter sur le vivant ? \\[0pt]
Peut-on expliquer le vivant ? \\[0pt]
Peut-on expliquer un acte libre ? \\[0pt]
Peut-on expliquer une œuvre d'art ? \\[0pt]
Peut-on faire son devoir par habitude ? \\[0pt]
Peut-on fonder la liberté ? \\[0pt]
Peut-on fonder la morale sur la pitié ? \\[0pt]
Peut-on gouverner le développement des innovations techniques ? \\[0pt]
Peut-on hiérarchiser les arts ? \\[0pt]
Peut-on hiérarchiser les devoirs ? \\[0pt]
Peut-on identifier le désir au besoin ? \\[0pt]
Peut-on invoquer la nature comme une excuse ? \\[0pt]
Peut-on jamais avoir la conscience tranquille ? \\[0pt]
Peut-on maîtriser le temps ? \\[0pt]
Peut-on maîtriser ses désirs ? \\[0pt]
Peut-on me contraindre à faire mon devoir ? \\[0pt]
Peut-on ne pas connaître son bonheur ? \\[0pt]
Peut-on ne pas manquer de temps ? \\[0pt]
Peut-on ne pas perdre son temps ? \\[0pt]
Peut-on ne pas savoir ce que l'on veut ? \\[0pt]
Peut-on nier l'existence de la matière ? \\[0pt]
Peut-on nier l'existence du temps ? \\[0pt]
Peut-on opposer nature et culture ? \\[0pt]
Peut-on opposer religion et raison ? \\[0pt]
Peut-on parler d'un progrès dans l'histoire ? \\[0pt]
Peut-on penser l'art sans référence au beau ? \\[0pt]
Peut-on penser le temps ? \\[0pt]
Peut-on penser l'œuvre d'art sans référence à l'idée de beauté ? \\[0pt]
Peut-on penser sans image ? \\[0pt]
Peut-on penser un État sans violence ? \\[0pt]
Peut-on percevoir le temps ? \\[0pt]
Peut-on prédire l'histoire ? \\[0pt]
Peut-on préférer l'ordre à la justice ? \\[0pt]
Peut-on prendre les moyens pour la fin ? \\[0pt]
Peut-on promettre le bonheur ? \\[0pt]
Peut-on ralentir la course du temps ? \\[0pt]
Peut-on reconnaître un sens à l'histoire sans lui assigner une fin ? \\[0pt]
Peut-on réduire la vie à la matière ? \\[0pt]
Peut-on réduire l'esprit à la matière ? \\[0pt]
Peut-on refuser la vérité ? \\[0pt]
Peut-on rendre raison de tout ? \\[0pt]
Peut-on rendre raison du réel ? \\[0pt]
Peut-on renoncer à la liberté ? \\[0pt]
Peut-on renoncer au bonheur ? \\[0pt]
Peut-on réparer le vivant ? \\[0pt]
Peut-on répondre au sceptique ? \\[0pt]
Peut-on retenir le temps ? \\[0pt]
Peut-on revendiquer le droit de désobéir ? \\[0pt]
Peut-on saisir le temps ? \\[0pt]
Peut-on se connaître soi-même ? \\[0pt]
Peut-on se fier à la technique ? \\[0pt]
Peut-on se passer de croire ? \\[0pt]
Peut-on se passer de la religion ? \\[0pt]
Peut-on se passer de la technique ? \\[0pt]
Peut-on se passer de l'État ? \\[0pt]
Peut-on se passer de l'idée de finalité ? \\[0pt]
Peut-on se passer d'un maître ? \\[0pt]
Peut-on se rendre maître de la technique ? \\[0pt]
Peut-on servir deux maîtres à la fois ? \\[0pt]
Peut-on se soustraire à son devoir ? \\[0pt]
Peut-on se tromper en se croyant heureux ? \\[0pt]
Peut-on se tromper sur son devoir ? \\[0pt]
Peut-on s'exprimer par le silence ? \\[0pt]
Peut-on s'oublier ? \\[0pt]
Peut-on souhaiter ne pas être libre ? \\[0pt]
Peut-on toujours savoir entièrement ce que l'on dit ? \\[0pt]
Peut-on tout démontrer ? \\[0pt]
Peut-on tout dire ? \\[0pt]
Peut-on tout échanger ? \\[0pt]
Peut-on tout enseigner ? \\[0pt]
Peut-on traiter un être vivant comme une machine ? \\[0pt]
Peut-on vivre en dehors de l'histoire ? \\[0pt]
Peut-on vivre sans désir ? \\[0pt]
Peut-on vivre sans foi ni loi ? \\[0pt]
Peut-on vouloir le bien sans le faire ? \\[0pt]
Peut-on vouloir le bonheur d'autrui ? \\[0pt]
Peut-on vouloir le mal ? \\[0pt]
Peut-on vouloir ne pas être libre ? \\[0pt]
Peut-on vraiment dire ce qu'on pense ? \\[0pt]
Peut-on vraiment parler de soi ? \\[0pt]
Plaisir et bonheur \\[0pt]
Plusieurs religions valent-elles mieux qu'une seule ? \\[0pt]
Pour bien penser, faut-il renoncer au langage courant ? \\[0pt]
Pour être heureux, faut-il ne rien désirer ? \\[0pt]
Pourquoi accomplir son devoir ? \\[0pt]
Pourquoi chercher à connaître le passé ? \\[0pt]
Pourquoi chercher à se connaître soi-même ? \\[0pt]
Pourquoi désire-t-on ce dont on n'a nul besoin ? \\[0pt]
Pourquoi désirons-nous ? \\[0pt]
Pourquoi écrit-on les lois ? \\[0pt]
Pourquoi écrit-on l'Histoire ? \\[0pt]
Pourquoi étudier le vivant ? \\[0pt]
Pourquoi étudier l'Histoire ? \\[0pt]
Pourquoi faire confiance au dialogue ? \\[0pt]
Pourquoi faire son devoir ? \\[0pt]
Pourquoi faudrait-il avoir peur de la technique ? \\[0pt]
Pourquoi parle-t-on ? \\[0pt]
Pourquoi préserver la nature ? \\[0pt]
Pourquoi punir ? \\[0pt]
Pourquoi refuser de faire son devoir ? \\[0pt]
Pourquoi refuse-t-on la conscience à l'animal ? \\[0pt]
Pourquoi se taire ? \\[0pt]
Pourquoi s'intéresser à l'histoire ? \\[0pt]
Pourquoi travailler ? \\[0pt]
Pourquoi vouloir être libre ? \\[0pt]
Pourquoi vouloir la vérité ? \\[0pt]
Pour s'aimer, faut-il se connaître ? \\[0pt]
Pour se trouver, faut-il savoir se perdre ? \\[0pt]
Pouvons-nous désirer ce qui nous nuit ? \\[0pt]
Prendre son temps, est-ce le perdre ? \\[0pt]
Production et création \\[0pt]
Produire et créer \\[0pt]
Promettre, est-ce renoncer à sa liberté ? \\[0pt]
Puis-je avoir la certitude que mes choix sont libres ? \\[0pt]
Puis-je devenir un autre ? \\[0pt]
Puis-je invoquer l'inconscient sans ruiner la morale ? \\[0pt]
Puis-je me connaître sans l'aide des autres ? \\[0pt]
Puis-je me connaître si je change à travers le temps ? \\[0pt]
Puis-je me trouver moi-même sans l'aide d'autrui ? \\[0pt]
Puis-je ne pas vouloir ce que je désire ? \\[0pt]
Puis-je ne rien devoir à personne ? \\[0pt]
Puis-je savoir avec assurance en quoi consiste mon devoir ? \\[0pt]
Quand manquons-nous à nos devoirs ? \\[0pt]
Qu'apprend-on en apprenant une technique ? \\[0pt]
Qu'apprenons-nous en éduquant ? \\[0pt]
Qu'attendons-nous pour être heureux ? \\[0pt]
Que désirons-nous ? \\[0pt]
Que désirons-nous quand nous désirons savoir ? \\[0pt]
Que devons-nous à l'État ? \\[0pt]
Que doit la science à la technique ? \\[0pt]
Quel est le contraire du travail ? \\[0pt]
Quel est le propre de la pensée technique ? \\[0pt]
Quel est l'objet de la physique ? \\[0pt]
Quel est l'objet des mathématiques ? \\[0pt]
Quel intérêt y a-t-il à se connaître ? \\[0pt]
Quelle confiance accorder au langage ? \\[0pt]
Quelle différence y a-t-il entre un corps vivant et un corps mort ? \\[0pt]
Quelle est la fonction première de l'État ? \\[0pt]
Quelle est la place du hasard dans l'histoire ? \\[0pt]
Quelle est la réalité du temps ? \\[0pt]
Quelle est la source de nos devoirs ? \\[0pt]
Quelle est la valeur d'une promesse ? \\[0pt]
Quelle est la valeur du temps ? \\[0pt]
Quelle est la valeur du vivant ? \\[0pt]
Quelle maîtrise avons-nous du temps ? \\[0pt]
Quelles limites assigner au pouvoir de la parole ? \\[0pt]
Quelles limites poser à l'autorité de l'État ? \\[0pt]
Quels devoirs les religions peuvent-elles énoncer ? \\[0pt]
Que nous append l'histoire ? \\[0pt]
Que nous apporte la technique ? \\[0pt]
Que nous apporte la vérité ? \\[0pt]
Que nous apporte le travail ? \\[0pt]
Que nous apprend la définition de la vérité ? \\[0pt]
Que nous apprend la fiction sur la réalité ? \\[0pt]
Que nous apprend la maladie ? \\[0pt]
Que nous apprend la technique ? \\[0pt]
Que nous impose le temps ? \\[0pt]
Que peint le peintre ? \\[0pt]
Que peut l'État ? \\[0pt]
Que peut-on savoir du réel ? \\[0pt]
Que peut prétendre imposer une religion ? \\[0pt]
Que peut signifier : « gérer son temps » ? \\[0pt]
Que peut-signifier « tuer le temps » ? \\[0pt]
Que pouvons-nous attendre de la technique ? \\[0pt]
Que pouvons-nous espérer de la connaissance du vivant ? \\[0pt]
Que prouvent les preuves de l'existence de Dieu ? \\[0pt]
Que sait la conscience ? \\[0pt]
Que savons-nous de notre vie ? \\[0pt]
Que serions-nous sans l'État ? \\[0pt]
Que signifie : « se rendre à l'évidence » ? \\[0pt]
Qu'est-ce que Dieu pour un athée ? \\[0pt]
Qu'est-ce que mesurer le temps ? \\[0pt]
Qu'est-ce que parler pour ne rien dire ? \\[0pt]
Qu'est-ce qu'être matérialiste ? \\[0pt]
Qu'est-ce que voir ? \\[0pt]
Qu'est-ce qui fait la valeur d'une œuvre d'art ? \\[0pt]
Qu'est-ce qui fait l'unité du vivant ? \\[0pt]
Qu'est-ce qui nous échappe dans le temps ? \\[0pt]
Qu'est-ce qui rapproche le vivant de la machine ? \\[0pt]
Qu'est-ce qu'un abus de langage ? \\[0pt]
Qu'est-ce qu'un ami ? \\[0pt]
Qu'est-ce qu'un chef-d'œuvre ? \\[0pt]
Qu'est-ce qu'un conflit de devoirs ? \\[0pt]
Qu'est-ce qu'un désir satisfait ? \\[0pt]
Qu'est-ce qu'un doute raisonnable ? \\[0pt]
Qu'est-ce qu'une crise ? \\[0pt]
Qu'est-ce qu'une encyclopédie ? \\[0pt]
Qu'est-ce qu'une idée fausse ? \\[0pt]
Qu'est-ce qu'une injustice ? \\[0pt]
Qu'est-ce qu'une institution ? \\[0pt]
Qu'est-ce qu'une liberté fondamentale ? \\[0pt]
Qu'est-ce qu'un ennemi politique ? \\[0pt]
Qu'est-ce qu'une parole juste ? \\[0pt]
Qu'est-ce qu'une pensée libre ? \\[0pt]
Qu'est-ce qu'une preuve ? \\[0pt]
Qu'est-ce qu'une société juste ? \\[0pt]
Qu'est-ce qu'un État libre ? \\[0pt]
Qu'est-ce qu'un événement historique ? \\[0pt]
Qu'est-ce qu'un fait historique ? \\[0pt]
Qu'est-ce qu'un fait religieux ? \\[0pt]
Qu'est-ce qu'un fait scientifique ? \\[0pt]
Qu'est-ce qu'un moment ? \\[0pt]
Qu'est-ce qu'un objet technique ? \\[0pt]
Qu'est-ce qu'un principe ? \\[0pt]
Qu'est-ce qu'un problème technique ? \\[0pt]
Qu'est-ce qu'un savoir-faire ? \\[0pt]
Qu'est-ce qu'un tableau ? \\[0pt]
Que veut dire : « être cultivé » ? \\[0pt]
Qui doit faire les lois ? \\[0pt]
Qui peut prétendre énoncer des devoirs ? \\[0pt]
Qui peut prétendre imposer des bornes à la technique ? \\[0pt]
Qui peut se passer de religion ? \\[0pt]
Qu'y a-t-il de sacré ? \\[0pt]
Rechercher la vérité, est-ce renoncer à toute opinion ? \\[0pt]
Reconnaître la vérité \\[0pt]
Recourir au langage, est-ce renoncer à la violence ? \\[0pt]
Religion et superstition \\[0pt]
Respecter autrui, est-ce l'aimer ? \\[0pt]
Respecter la nature, est-ce renoncer à l'exploiter ? \\[0pt]
Respecter la nature, est-ce renoncer à s'en emparer ? \\[0pt]
Retenons-nous le temps par le souvenir ? \\[0pt]
Sans référence à la beauté, peut-on rendre compte de l'art ? \\[0pt]
Savons-nous ce que nous disons ? \\[0pt]
Science et imagination \\[0pt]
Sciences et vérité \\[0pt]
Se chercher soi-même, est-ce devenir un autre ? \\[0pt]
Se force-t-on à faire son devoir ? \\[0pt]
S'engager, est-ce perdre sa liberté ? \\[0pt]
Sentir et percevoir \\[0pt]
Serions-nous plus libres sans État ? \\[0pt]
« Se trouver » a-t-il un sens ? \\[0pt]
Seuls les humains sont-ils libres ? \\[0pt]
Sommes-nous faits pour le bonheur ? \\[0pt]
Sommes-nous les jouets de l'histoire ? \\[0pt]
Sommes-nous maîtres de nos désirs ? \\[0pt]
Sommes-nous menacés par les progrès techniques ? \\[0pt]
Sommes-nous prisonniers de nos désirs ? \\[0pt]
Sommes-nous prisonniers de notre histoire ? \\[0pt]
Sommes-nous prisonniers du temps ? \\[0pt]
Sommes-nous responsables de ce dont nous n'avons pas conscience ? \\[0pt]
Sommes-nous responsables de nos désirs ? \\[0pt]
Sommes-nous toujours conscients des causes de nos désirs ? \\[0pt]
Sommes-nous vulnérables toute notre vie ? \\[0pt]
S'opposer à l'autorité, est-ce toujours une marque de liberté ? \\[0pt]
Subissons-nous la langue que nous parlons ? \\[0pt]
Suffit-il de bien penser pour bien agir ? \\[0pt]
Suffit-il de faire son devoir ? \\[0pt]
Suffit-il de se parler pour éviter la discorde ? \\[0pt]
Suffit-il de tenir ses promesses pour faire son devoir ? \\[0pt]
Suffit-il d'être soi-même pour être libre ? \\[0pt]
Suffit-il d'être vertueux pour être heureux ? \\[0pt]
Suffit-il de voir le meilleur pour le suivre ? \\[0pt]
Suis-je ce que j'ai conscience d'être ? \\[0pt]
Suis-je ce que je fais ? \\[0pt]
Suis-je ce que les autres font de moi ? \\[0pt]
Suis-je toujours le même ? \\[0pt]
Superstition et fanatisme sont-ils inhérents à la religion ? \\[0pt]
Sur quoi fonder le droit de punir ? \\[0pt]
Technique et progrès \\[0pt]
Tous les hommes désirent-ils être heureux ? \\[0pt]
Tous les problèmes peuvent-ils recevoir une solution technique ? \\[0pt]
Tout ce qui est rationnel est-il raisonnable ? \\[0pt]
Tout change-t-il avec le temps ? \\[0pt]
Tout désir est-il manque ? \\[0pt]
Toute inégalité est-elle injuste ? \\[0pt]
Toute morale est-elle rationnelle ? \\[0pt]
Tout énoncé admet-il une interprétation ? \\[0pt]
Toute passion fait-elle souffrir ? \\[0pt]
Toute religion a-t-elle sa vérité ? \\[0pt]
Tout est-il historique ? \\[0pt]
Tout est-il vraiment permis, si Dieu n'existe pas ? \\[0pt]
Tout événement a-t-il une cause ? \\[0pt]
Toute vérité doit-elle être dite ? \\[0pt]
Toute vérité est-elle nécessaire ? \\[0pt]
Tout exercice du pouvoir politique implique-t-il l'usage de la violence ? \\[0pt]
Tout ordre est-il rationnel ? \\[0pt]
Tout passe-t-il avec le temps ? \\[0pt]
Tout savoir est-il pouvoir ? \\[0pt]
Tout savoir est-il transmissible ? \\[0pt]
Tout travail a-t-il un sens ? \\[0pt]
Traiter des faits humains comme des choses, est-ce considérer l'homme comme une chose ? \\[0pt]
Traiter les faits humains comme des choses, est-ce réduire les hommes à des choses ? \\[0pt]
Travail manuel et travail intellectuel \\[0pt]
Un acte libre est-il un acte imprévisible ? \\[0pt]
Un désir peut-il être coupable ? \\[0pt]
Un devoir admet-il des exceptions ? \\[0pt]
Un devoir peut-il être absolu ? \\[0pt]
Une durée peut-elle être éternelle ? \\[0pt]
Une éducation morale est-elle possible ? \\[0pt]
Une expérience peut-elle être fictive ? \\[0pt]
Une imitation peut-elle être parfaite ? \\[0pt]
Une religion peut-elle être universelle ? \\[0pt]
Une religion peut-elle prétendre à la vérité ? \\[0pt]
Une religion peut-elle se passer de pratiques ? \\[0pt]
Une société n'est-elle qu'un ensemble d'individus ? \\[0pt]
Une société sans religion est-elle possible ? \\[0pt]
Une technique ne se réduit-elle pas toujours à une forme de bricolage ? \\[0pt]
Un être vivant peut-il être comparé à une œuvre d'art ? \\[0pt]
Un événement historique est-il toujours imprévisible ? \\[0pt]
Une vie heureuse est-elle une vie de plaisirs ? \\[0pt]
Une vie juste est-elle une bonne vie ? \\[0pt]
Un monde sans guerre serait-il un monde sans histoire ? \\[0pt]
Un peuple en paix n'a-t-il plus d'histoire ? \\[0pt]
Un peuple se définit-il par son histoire ? \\[0pt]
Vérité et certitude \\[0pt]
Vérité et démonstration \\[0pt]
Vérité et réalité \\[0pt]
Vivre, est-ce lutter pour survivre ? \\[0pt]
Vivre, est-ce résister à la mort ? \\[0pt]
Voir et savoir \\[0pt]
Y a-t-il de bons et de mauvais désirs ? \\[0pt]
Y a-t-il de la raison dans la perception ? \\[0pt]
Y a-t-il de l'impensable ? \\[0pt]
Y a-t-il de l'indésirable ? \\[0pt]
Y a-t-il des actions libres ? \\[0pt]
Y a-t-il des critères de l'humanité ? \\[0pt]
Y a-t-il des croyances rationnelles ? \\[0pt]
Y a-t-il des désordres naturels ? \\[0pt]
Y a-t-il des guerres justes ? \\[0pt]
Y a-t-il des illusions de la conscience ? \\[0pt]
Y a-t-il des inégalités justes ? \\[0pt]
Y a-t-il des leçons de l'histoire ? \\[0pt]
Y a-t-il des limites au pouvoir de la technique ? \\[0pt]
Y a-t-il des obstacles à la connaissance du vivant ? \\[0pt]
Y a-t-il des preuves de la liberté ? \\[0pt]
Y a-t-il des progrès en art ? \\[0pt]
Y a-t-il des révolutions dans les sciences ? \\[0pt]
Y a-t-il des techniques de pensée ? \\[0pt]
Y a-t-il des tyrannies douces ? \\[0pt]
Y a-t-il des vérités éternelles ? \\[0pt]
Y a-t-il des vérités métaphysiques ? \\[0pt]
Y a-t-il des vérités morales ? \\[0pt]
Y a-t-il des vérités plus importantes que d'autres ? \\[0pt]
Y a-t-il des vérités sans preuve ? \\[0pt]
Y a-t-il du non-être ? \\[0pt]
Y a-t-il du nouveau dans l'histoire ? \\[0pt]
Y a-t-il plusieurs temps ? \\[0pt]
Y a-t-il quoi que ce soit de nouveau dans l'histoire ? \\[0pt]
Y a-t-il sens à vouloir justifier le mal ? \\[0pt]
Y a-t-il un auteur de l'histoire ? \\[0pt]
Y a-t-il un bon usage du temps ? \\[0pt]
Y a-t-il un devoir de désobéissance ? \\[0pt]
Y a-t-il un devoir de mémoire ? \\[0pt]
Y a-t-il un devoir de vérité ? \\[0pt]
Y a-t-il un droit du plus fort ? \\[0pt]
Y a-t-il une bonne imitation ? \\[0pt]
Y a-t-il une connaissance du singulier ? \\[0pt]
Y a-t-il une éthique de la technique ? \\[0pt]
Y a-t-il une expérience de l'éternité ? \\[0pt]
Y a-t-il une expérience du temps ? \\[0pt]
Y a-t-il une force du droit ? \\[0pt]
Y a-t-il une hiérarchie des devoirs ? \\[0pt]
Y a-t-il une hiérarchie du vivant ? \\[0pt]
Y a-t-il une histoire de la vérité ? \\[0pt]
Y a-t-il une irréversibilité du temps ? \\[0pt]
Y a-t-il une limite à la connaissance du vivant ? \\[0pt]
Y a-t-il une limite au désir ? \\[0pt]
Y a-t-il une logique dans l'histoire ? \\[0pt]
Y a-t-il une logique des événements historiques ? \\[0pt]
Y a-t-il une réalité du hasard ? \\[0pt]
Y a-t-il une réalité du mal ? \\[0pt]
Y a-t-il une science politique ? \\[0pt]
Y a-t-il un État idéal ? \\[0pt]
Y a-t-il une technique pour tout ? \\[0pt]
Y a-t-il une tyrannie du vrai ? \\[0pt]
Y a-t-il une vérité des sentiments ? \\[0pt]
Y a-t-il une vérité en art ? \\[0pt]
Y a-t-il une vérité en histoire ? \\[0pt]
Y a-t-il une vision du monde propre à la pensée technique ? \\[0pt]
Y a-t-il un intérêt à faire son devoir ? \\[0pt]
Y a-t-il un sens à instituer une langue universelle ? \\[0pt]
Y a-t-il un sens à s'opposer à la technique ? \\[0pt]
Y a-t-il un temps pour tout ? \\[0pt]
Y a-t-il vérité sans interprétation ? \\[0pt]
Y aura-t-il toujours des religions ? \\[0pt]


\subsection{ENS A​/​L}
\label{sec:orgd06311d}

\noindent
1, 2, 3 \\[0pt]
2+2 = 4 \\[0pt]
2+2 pourrait-il ne pas être égal à 4 ? \\[0pt]
Abstraire, est-ce se couper du réel ? \\[0pt]
Abuser du pouvoir \\[0pt]
À chacun des goûts \\[0pt]
« À chacun sa vérité » \\[0pt]
À chacun son dû \\[0pt]
Additionner et soustraire \\[0pt]
Agir \\[0pt]
Agir contre ses intérêts \\[0pt]
Aide-toi, le ciel t'aidera \\[0pt]
Ai-je un corps ? \\[0pt]
Aimer ce qui est beau \\[0pt]
Aimer et être aimé \\[0pt]
Aimer la vie \\[0pt]
Aimer son prochain \\[0pt]
Aimer son prochain comme soi-même \\[0pt]
Aime ton prochain comme toi-même \\[0pt]
« Aimez vos ennemis » \\[0pt]
« À l'impossible nul n'est tenu » \\[0pt]
À l'impossible nul n'est tenu \\[0pt]
Ami et ennemi \\[0pt]
Amitié et société \\[0pt]
Amour et amitié \\[0pt]
Analyse et intuition \\[0pt]
Analyse et synthèse \\[0pt]
Appartient-il aux lois d'éduquer les hommes ? \\[0pt]
Apprend-on à aimer ? \\[0pt]
Apprend-on à être artiste ? \\[0pt]
Apprend-on à penser ? \\[0pt]
Apprend-on à voir ? \\[0pt]
Apprendre \\[0pt]
Apprendre à vivre \\[0pt]
Apprendre à voir \\[0pt]
Apprendre des autres \\[0pt]
Après moi le déluge \\[0pt]
À quelle condition une démarche est-elle scientifique ? \\[0pt]
À quelles conditions peut-on dire qu'une action est historique ? \\[0pt]
À quelles conditions un choix peut-il être rationnel ? \\[0pt]
À quelles conditions une théorie est-elle scientifique ? \\[0pt]
« À quelque chose malheur est bon » \\[0pt]
À quelque chose malheur est bon \\[0pt]
À qui devons-nous obéir ? \\[0pt]
À qui se fier ? \\[0pt]
À quoi bon ? \\[0pt]
À quoi bon les regrets ? \\[0pt]
À quoi bon les romans ? \\[0pt]
À quoi bon raconter des histoires ? \\[0pt]
À quoi bon voyager ? \\[0pt]
À quoi la valeur d'un homme se mesure-t-elle ? \\[0pt]
À quoi reconnaît-on la vérité ? \\[0pt]
À quoi reconnaît-on l'injustice ? \\[0pt]
À quoi reconnaît-on une œuvre d'art ? \\[0pt]
À quoi sert la connaissance du passé ? \\[0pt]
À quoi sert la mémoire ? \\[0pt]
À quoi sert la technique ? \\[0pt]
À quoi sert le contrat social ? \\[0pt]
À quoi servent les émotions ? \\[0pt]
À quoi servent les expériences ? \\[0pt]
À quoi servent les fictions ? \\[0pt]
À quoi servent les mythes ? \\[0pt]
À quoi servent les symboles ? \\[0pt]
À quoi servent les utopies ? \\[0pt]
À quoi servent les voyages ? \\[0pt]
À quoi tenons-nous ? \\[0pt]
À quoi tient l'autorité ? \\[0pt]
Arbitrer \\[0pt]
Art et apparences \\[0pt]
Art et connaissance \\[0pt]
Art et illusion \\[0pt]
Art et morale \\[0pt]
Art et politique \\[0pt]
Art et représentation \\[0pt]
Art et vérité \\[0pt]
Artiste et artisan \\[0pt]
Art populaire et art savant \\[0pt]
Arts de l'espace et arts du temps \\[0pt]
A-t-on besoin de fonder la connaissance ? \\[0pt]
A-t-on besoin de spécialistes en politique ? \\[0pt]
A-t-on des devoirs envers soi-même ? \\[0pt]
A-t-on intérêt à tout savoir ? \\[0pt]
A-t-on le droit de mentir ? \\[0pt]
A-t-on le droit de résister ? \\[0pt]
A-t-on le droit de s'évader ? \\[0pt]
A-t-on l'obligation de pardonner ? \\[0pt]
Autrui \\[0pt]
Autrui est-il aimable ? \\[0pt]
Autrui est-il mon semblable ? \\[0pt]
Autrui est-il un autre moi ? \\[0pt]
Autrui est-il un autre moi-même ? \\[0pt]
Autrui me connaît-il mieux que moi-même ? \\[0pt]
« Aux armes citoyens ! » \\[0pt]
Aux armes, citoyens ! \\[0pt]
Avez-vous une âme ? \\[0pt]
Avoir bonne conscience \\[0pt]
Avoir confiance \\[0pt]
Avoir de la volonté \\[0pt]
Avoir des principes \\[0pt]
Avoir du goût \\[0pt]
Avoir du jugement \\[0pt]
Avoir la foi \\[0pt]
Avoir la santé \\[0pt]
Avoir le choix \\[0pt]
Avoir le sens de la situation \\[0pt]
Avoir peur des mots \\[0pt]
Avoir un corps \\[0pt]
Avons-nous besoin d'amis ? \\[0pt]
Avons-nous besoin de rêver ? \\[0pt]
Avons-nous des devoirs envers les générations futures ? \\[0pt]
Avons-nous des devoirs envers nous-mêmes ? \\[0pt]
Avons-nous le devoir de vivre ? \\[0pt]
Avons-nous le droit de juger autrui ? \\[0pt]
Avons-nous le droit d'être heureux ? \\[0pt]
Avons-nous raison d'exiger toujours des raisons ? \\[0pt]
Avons-nous un corps ? \\[0pt]
Avons-nous une âme ? \\[0pt]
Avons-nous une identité ? \\[0pt]
Avons-nous une obligation envers les générations à venir ? \\[0pt]
Axiomatiser, est-ce fonder ? \\[0pt]
Beauté naturelle et beauté artistique \\[0pt]
Besoin et désir \\[0pt]
Besoins et désirs \\[0pt]
Bêtise et méchanceté \\[0pt]
Bien parler, est-ce bien penser ? \\[0pt]
Bonheur de chacun bonheur de tous \\[0pt]
Catégories de langue, catégories de pensée \\[0pt]
Cause et loi \\[0pt]
Cause et raison \\[0pt]
Causes et raisons \\[0pt]
« Ceci » \\[0pt]
« Ce ne sont que des mots » \\[0pt]
Ce qui est subjectif est-il arbitraire ? \\[0pt]
Ce qui n'est pas démontré peut-il être vrai ? \\[0pt]
Ce qui vaut en théorie vaut-il toujours en pratique ? \\[0pt]
Certitude et probabilité \\[0pt]
Certitude et vérité \\[0pt]
« C'est en forgeant que l'on devient forgeron » \\[0pt]
« C'est la vie » \\[0pt]
« C'est plus fort que moi » \\[0pt]
« C'est pour ton bien » \\[0pt]
Ceux qui savent doivent-ils gouverner ? \\[0pt]
« Chacun ses goûts » \\[0pt]
Changer d'opinion \\[0pt]
Changer, est-ce devenir un autre ? \\[0pt]
« Changer le monde » \\[0pt]
Chanter \\[0pt]
Choisir \\[0pt]
Choisit-on son corps ? \\[0pt]
Chose et personne \\[0pt]
Choses et personnes \\[0pt]
Citoyen du monde ? \\[0pt]
Civilisé, barbare, sauvage \\[0pt]
Classer \\[0pt]
Classicisme et romantisme \\[0pt]
Colère et indignation \\[0pt]
Commémorer \\[0pt]
Comment conduire ses pensées ? \\[0pt]
Comment croire au progrès ? \\[0pt]
Comment dire la vérité ? \\[0pt]
Comment dire l'individuel ? \\[0pt]
Comment distinguer désirs et besoins ? \\[0pt]
Comment être naturel ? \\[0pt]
Comment évaluer la qualité de la vie ? \\[0pt]
Comment expliquer les phénomènes mentaux ? \\[0pt]
Comment fonder la propriété ? \\[0pt]
Comment mesurer ? \\[0pt]
Comment mesurer une sensation ? \\[0pt]
Comment penser le mouvement ? \\[0pt]
Comment percevons-nous l'espace ? \\[0pt]
Comment peut-on définir la politique ? \\[0pt]
Comment retrouver la nature ? \\[0pt]
Comment sait-on qu'une chose existe ? \\[0pt]
Comment savoir que l'on est dans l'erreur ? \\[0pt]
Comment se mettre à la place d'autrui ? \\[0pt]
Comment s'entendre ? \\[0pt]
Comment s'orienter dans la pensée ? \\[0pt]
Comment voyager dans le temps ? \\[0pt]
Communiquer \\[0pt]
Comparer les cultures \\[0pt]
Comprendre \\[0pt]
Comprendre, est-ce excuser ? \\[0pt]
Comprendre le sens d'un texte \\[0pt]
Compter \\[0pt]
Concept et image \\[0pt]
Concept et intuition \\[0pt]
Concept et métaphore \\[0pt]
Concevoir et expérimenter \\[0pt]
Conduite et comportement \\[0pt]
Conflit et démocratie \\[0pt]
Conflit et liberté \\[0pt]
Connaissance et croyance \\[0pt]
Connaissons-nous la réalité telle qu'elle est ? \\[0pt]
« Connais-toi toi-même » \\[0pt]
Connais-toi toi-même \\[0pt]
Connaît-on la vie ou connaît-on le vivant ? \\[0pt]
Connaître autrui \\[0pt]
Connaître l'infini \\[0pt]
Connaître ses origines \\[0pt]
Conscience de soi et connaissance de soi \\[0pt]
Conscience et existence \\[0pt]
Consensus et conflit \\[0pt]
Construire l'espace \\[0pt]
Contempler \\[0pt]
Contingence et nécessité \\[0pt]
Continuité et discontinuité \\[0pt]
Corps et esprit \\[0pt]
Corps et identité \\[0pt]
Corriger ses erreurs \\[0pt]
Création et fabrication \\[0pt]
Créer \\[0pt]
Crime et châtiment \\[0pt]
Crise et critique \\[0pt]
Critiquer \\[0pt]
Croire et savoir \\[0pt]
Croire savoir \\[0pt]
Croit-on ce que l'on veut ? \\[0pt]
Croit-on comme on veut ? \\[0pt]
Croyance et connaissance \\[0pt]
Culpabilité et responsabilité \\[0pt]
Danser \\[0pt]
Dans quelle mesure l'art est-il un fait social ? \\[0pt]
Décider \\[0pt]
Découverte et justification \\[0pt]
Définir \\[0pt]
Démontrer \\[0pt]
De quoi avons-nous besoin ? \\[0pt]
De quoi doute un sceptique ? \\[0pt]
De quoi faut-il avoir peur ? \\[0pt]
De quoi la musique est-elle l'art ? \\[0pt]
De quoi les métaphysiciens parlent-ils ? \\[0pt]
De quoi peut-on faire l'expérience ? \\[0pt]
De quoi rit-on ? \\[0pt]
De quoi sommes-nous coupables ? \\[0pt]
De quoi sommes-nous responsables ? \\[0pt]
De quoi suis-je responsable ? \\[0pt]
Description et explication \\[0pt]
« Des goûts et des couleurs, on ne dispute pas » \\[0pt]
Des inégalités peuvent-elles être justes ? \\[0pt]
Désire-t-on la reconnaissance ? \\[0pt]
Désir et volonté \\[0pt]
Désobéir \\[0pt]
Détruire et construire \\[0pt]
Deux personnes peuvent-elles partager la même pensée ? \\[0pt]
Devenir ce que l'on est \\[0pt]
Devenir et évolution \\[0pt]
Devons-nous tenir certaines connaissances pour acquises ? \\[0pt]
Dieu est-il mort ? \\[0pt]
Dieu est-il mortel ? \\[0pt]
Dieu est-il une invention humaine ? \\[0pt]
Dire et faire \\[0pt]
Dire et montrer \\[0pt]
Dire « je » \\[0pt]
Dire l'individuel \\[0pt]
Disposer de son corps \\[0pt]
Distinguer \\[0pt]
Dois-je obéir à la loi ? \\[0pt]
Doit-on bien juger pour bien faire ? \\[0pt]
Doit-on chasser les artistes de la cité ? \\[0pt]
Doit-on croire au progrès ? \\[0pt]
Doit-on cultiver l'ironie ? \\[0pt]
Doit-on respecter la nature ? \\[0pt]
Doit-on se mettre à la place d'autrui ? \\[0pt]
Doit-on toujours rechercher la vérité ? \\[0pt]
Doit-on tout contrôler ? \\[0pt]
Doit-on travailler ? \\[0pt]
Don Juan \\[0pt]
Donner \\[0pt]
Donner des preuves \\[0pt]
Donner sa parole \\[0pt]
Donner son assentiment \\[0pt]
Douter \\[0pt]
D'où viennent nos idées ? \\[0pt]
D'où vient le mal ? \\[0pt]
Dressage et éducation \\[0pt]
Droit et coutume \\[0pt]
Droit et moralité \\[0pt]
Droits et devoirs \\[0pt]
Du passé pouvons-nous faire table rase ? \\[0pt]
Durer \\[0pt]
Échange et don \\[0pt]
Échanger \\[0pt]
Économie et société \\[0pt]
Écouter et entendre \\[0pt]
Écouter son corps \\[0pt]
Éducation et instruction \\[0pt]
Éduquer et instruire \\[0pt]
Égalité et solidarité \\[0pt]
Élire \\[0pt]
En finir avec les préjugés \\[0pt]
En histoire, tout est-il affaire d'interprétation ? \\[0pt]
En morale, est-ce seulement l'intention qui compte ? \\[0pt]
Enquêter \\[0pt]
En quoi la métaphysique est-elle une science ? \\[0pt]
En quoi le bien d'autrui m'importe-t-il ? \\[0pt]
En quoi l'œuvre d'art donne-t-elle à penser ? \\[0pt]
En quoi une œuvre d'art est-elle moderne ? \\[0pt]
Enseigner et éduquer \\[0pt]
Enseigner l'art \\[0pt]
Entendement et raison \\[0pt]
Entendre \\[0pt]
Entendre raison \\[0pt]
Entre l'art et la nature, qui imite l'autre ? \\[0pt]
Espace mathématique et espace physique \\[0pt]
Espérer \\[0pt]
Essence et existence \\[0pt]
Est-ce à la fin que le sens apparaît ? \\[0pt]
Est-ce la mémoire qui constitue mon identité ? \\[0pt]
Est-ce seulement l'intention qui compte ? \\[0pt]
Est-il difficile de savoir ce que l'on veut ? \\[0pt]
Est-il difficile de savoir ce qu'on veut ? \\[0pt]
Est-il difficile de vivre en société ? \\[0pt]
Est-il juste de payer l'impôt ? \\[0pt]
Est-il naturel de s'aimer soi-même ? \\[0pt]
Est-il nécessaire d'espérer pour entreprendre ? \\[0pt]
Est-il parfois bon de mentir ? \\[0pt]
Est-il raisonnable d'aimer ? \\[0pt]
Est-il toujours avantageux de promouvoir son propre intérêt ? \\[0pt]
Est-il toujours meilleur d'avoir le choix ? \\[0pt]
Est-il utile d'avoir mal ? \\[0pt]
Est-il vrai que nous ne nous tenons jamais au temps présent ? \\[0pt]
Est-on libre de ne pas vouloir ce que l'on veut ? \\[0pt]
Est-on responsable de son passé ? \\[0pt]
État et société \\[0pt]
Éthique et esthétique \\[0pt]
Ethnologie et ethnocentrisme \\[0pt]
« Et pourtant elle tourne » \\[0pt]
Être adulte \\[0pt]
Être à sa place \\[0pt]
Être attentif \\[0pt]
Être bien élevé \\[0pt]
Être citoyen du monde \\[0pt]
Être conséquent \\[0pt]
Être cultivé, est-ce tout connaître ? \\[0pt]
Être cynique \\[0pt]
Être de son temps \\[0pt]
Être enraciné \\[0pt]
Être équitable \\[0pt]
Être et devenir \\[0pt]
Être exemplaire \\[0pt]
Être hors de soi \\[0pt]
Être hors la loi \\[0pt]
Être impossible \\[0pt]
Être logique avec soi-même \\[0pt]
Être majeur \\[0pt]
Être malade \\[0pt]
Être méthodique \\[0pt]
Être moderne \\[0pt]
« Être négatif » \\[0pt]
Être ou ne pas être, est-ce la question ? \\[0pt]
Être précurseur \\[0pt]
Être raisonnable, est-ce accepter la réalité telle qu'elle est ? \\[0pt]
Être relativiste \\[0pt]
Être sceptique \\[0pt]
Être soi \\[0pt]
Être soi-même \\[0pt]
Être systématique \\[0pt]
Être vulgaire \\[0pt]
Évolution et progrès \\[0pt]
Évolution et révolution \\[0pt]
Exister \\[0pt]
Exister hors du temps \\[0pt]
Existe-t-il de faux besoins ? \\[0pt]
Existe-t-il des choses sans prix ? \\[0pt]
Existe-t-il des croyances collectives ? \\[0pt]
Existe-t-il des démonstrations métaphysiques ? \\[0pt]
Existe-t-il des devoirs envers soi-même ? \\[0pt]
Existe-t-il des signes naturels ? \\[0pt]
Existe-t-il un art de penser ? \\[0pt]
Existe-t-il un droit de mentir ? \\[0pt]
Existe-t-il une vision scientifique du monde ? \\[0pt]
Expérience et expérimentation \\[0pt]
Expérience et vérité \\[0pt]
Expérience, expérimentation \\[0pt]
Expliquer \\[0pt]
Expliquer, est-ce excuser ? \\[0pt]
Expliquer et comprendre \\[0pt]
Expliquer et interpréter \\[0pt]
Expliquer, justifier comprendre \\[0pt]
Faire apprendre \\[0pt]
Faire comme si \\[0pt]
Faire de sa vie une œuvre d'art \\[0pt]
Faire douter \\[0pt]
Faire justice \\[0pt]
Faire la loi \\[0pt]
« Faire la paix » \\[0pt]
Faire la paix \\[0pt]
Faire table rase \\[0pt]
Faire voir \\[0pt]
Fait et théorie \\[0pt]
Familles, je vous hais \\[0pt]
Faudrait-il ne rien oublier ? \\[0pt]
Faut-il aimer le destin ? \\[0pt]
Faut-il aimer son prochain ? \\[0pt]
Faut-il aimer son prochain comme soi-même ? \\[0pt]
Faut-il aller toujours plus vite ? \\[0pt]
Faut-il apprendre à voir ? \\[0pt]
Faut-il avoir des principes ? \\[0pt]
Faut-il avoir peur de la nature ? \\[0pt]
Faut-il avoir peur des habitudes ? \\[0pt]
Faut-il avoir peur des machines ? \\[0pt]
Faut-il avoir peur d'être libre ? \\[0pt]
Faut-il avoir peur du désordre ? \\[0pt]
Faut-il changer ses désirs plutôt que l'ordre du monde ? \\[0pt]
Faut-il chasser les poètes ? \\[0pt]
Faut-il condamner la fiction ? \\[0pt]
Faut-il condamner la rhétorique ? \\[0pt]
Faut-il considérer les faits sociaux comme des choses ? \\[0pt]
Faut-il contrôler les mœurs ? \\[0pt]
Faut-il craindre la mort ? \\[0pt]
Faut-il craindre le regard d'autrui ? \\[0pt]
Faut-il croire en la science ? \\[0pt]
Faut-il défendre l'ordre à tout prix ? \\[0pt]
Faut-il défendre ses convictions \\[0pt]
Faut-il dépasser les apparences ? \\[0pt]
Faut-il désespérer de l'humanité ? \\[0pt]
Faut-il des frontières ? \\[0pt]
Faut-il des héros ? \\[0pt]
Faut-il des outils pour penser ? \\[0pt]
Faut-il douter de tout ? \\[0pt]
Faut-il être à l'écoute du corps ? \\[0pt]
Faut-il être connaisseur pour apprécier une œuvre d'art ? \\[0pt]
Faut-il être idéaliste ? \\[0pt]
Faut-il être logique avec soi-même ? \\[0pt]
Faut-il être mesuré en toutes choses ? \\[0pt]
Faut-il être naturel ? \\[0pt]
Faut-il être original ? \\[0pt]
Faut-il être positif ? \\[0pt]
Faut-il être relativiste ? \\[0pt]
Faut-il faire de nécessité vertu ? \\[0pt]
Faut-il forcer les gens à participer à la vie politique ? \\[0pt]
Faut-il garder ses illusions ? \\[0pt]
Faut-il imaginer que nous sommes heureux ? \\[0pt]
Faut-il joindre l'utile à l'agréable ? \\[0pt]
Faut-il laisser parler la nature ? \\[0pt]
Faut-il lire des romans ? \\[0pt]
Faut-il opposer nature et culture ? \\[0pt]
Faut-il partager la souveraineté ? \\[0pt]
Faut-il perdre ses illusions ? \\[0pt]
Faut-il perdre son temps ? \\[0pt]
Faut-il protéger la dignité humaine ? \\[0pt]
Faut-il protéger la nature ? \\[0pt]
Faut-il rechercher la simplicité ? \\[0pt]
Faut-il rechercher l'harmonie ? \\[0pt]
Faut-il regretter l'équivocité du langage ? \\[0pt]
Faut-il renoncer à la certitude ? \\[0pt]
Faut-il renoncer à l'impossible ? \\[0pt]
Faut-il respecter la nature ? \\[0pt]
Faut-il respecter la tradition ? \\[0pt]
Faut-il respecter les convenances ? \\[0pt]
Faut-il rester impartial ? \\[0pt]
Faut-il rester naturel ? \\[0pt]
Faut-il sauver des vies à tout prix ? \\[0pt]
Faut-il se contenter de peu ? \\[0pt]
Faut-il se délivrer de la peur ? \\[0pt]
Faut-il s'efforcer d'être moins personnel ? \\[0pt]
Faut-il se fier à ce que l'on ressent ? \\[0pt]
Faut-il se fier au témoignage des sens ? \\[0pt]
Faut-il se fier aux apparences ? \\[0pt]
Faut-il se méfier de l'écriture ? \\[0pt]
Faut-il se méfier de l'imagination ? \\[0pt]
Faut-il se méfier de l'inspiration ? \\[0pt]
Faut-il se poser des questions métaphysiques ? \\[0pt]
Faut-il se réjouir d'exister ? \\[0pt]
Faut-il suivre ses intuitions ? \\[0pt]
Faut-il surmonter son enfance ? \\[0pt]
Faut-il toujours être en accord avec soi-même ? \\[0pt]
Faut-il toujours garder espoir ? \\[0pt]
Faut-il un commencement à tout ? \\[0pt]
Faut-il une guerre pour mettre fin à toutes les guerres ? \\[0pt]
Faut-il une théorie de la connaissance ? \\[0pt]
Faut-il vaincre ses désirs plutôt que l'ordre du monde ? \\[0pt]
Faut-il vivre dangereusement ? \\[0pt]
Faut-il voir pour croire ? \\[0pt]
Faut-il vouloir la paix de l'âme ? \\[0pt]
Faut-il vouloir la transparence ? \\[0pt]
Faut-il vouloir savoir ? \\[0pt]
Foi et raison \\[0pt]
Foi et savoir \\[0pt]
Force et droit \\[0pt]
Forme et contenu \\[0pt]
Forme et fonction \\[0pt]
Fuir la civilisation \\[0pt]
Gagner sa vie \\[0pt]
Génie et technique \\[0pt]
Genre et espèce \\[0pt]
Gouvernement des hommes et administration des choses \\[0pt]
Gouverner \\[0pt]
Guerre et paix \\[0pt]
Habiter \\[0pt]
Habiter le monde \\[0pt]
Hériter \\[0pt]
Hésiter \\[0pt]
Histoire et fiction \\[0pt]
Histoire et géographie \\[0pt]
Honte, pudeur, embarras \\[0pt]
Humour et ironie \\[0pt]
Ici et maintenant \\[0pt]
Identité et changement \\[0pt]
Identité et égalité \\[0pt]
Identité et ressemblance \\[0pt]
Ignorer \\[0pt]
« Il ne lui manque que la parole » \\[0pt]
Image et concept \\[0pt]
Image, signe, symbole \\[0pt]
Imagination et conception \\[0pt]
Imagination et raison \\[0pt]
Imaginer \\[0pt]
Imiter \\[0pt]
Individualisme et égoïsme \\[0pt]
Information et opinion \\[0pt]
Instruire et éduquer \\[0pt]
Interpréter \\[0pt]
Interpréter une œuvre d'art \\[0pt]
Interroger et répondre \\[0pt]
Intuition et déduction \\[0pt]
Intuition et intellection \\[0pt]
« J'ai le droit » \\[0pt]
Je \\[0pt]
Je est un autre \\[0pt]
« Je m'en veux » \\[0pt]
« Je ne crois que ce que je vois » \\[0pt]
Je sens donc je suis \\[0pt]
Je, tu, il \\[0pt]
Jouer \\[0pt]
Jouer son rôle \\[0pt]
Juger \\[0pt]
Juger et sentir \\[0pt]
Jusqu'à quel point sommes-nous responsables de nos passions ? \\[0pt]
Jusqu'à quel point suis-je mon propre maître ? \\[0pt]
Jusqu'où interpréter ? \\[0pt]
Jusqu'où peut-on dialoguer ? \\[0pt]
Justice et égalité \\[0pt]
Justice et force \\[0pt]
Justice et utilité \\[0pt]
Justice et vengeance \\[0pt]
Justifier \\[0pt]
La banalité \\[0pt]
La barbarie \\[0pt]
La beauté \\[0pt]
La beauté a-t-elle une histoire ? \\[0pt]
La beauté de la nature \\[0pt]
La beauté du geste \\[0pt]
« La beauté est dans l'œil de celui qui regarde » \\[0pt]
La beauté est-elle affaire de goût ? \\[0pt]
La belle âme \\[0pt]
La bêtise \\[0pt]
La bêtise et la méchanceté sont-elles liées intrinsèquement ? \\[0pt]
La bibliothèque \\[0pt]
La bienséance \\[0pt]
La bienveillance \\[0pt]
La biographie \\[0pt]
L'abondance \\[0pt]
La bonne conscience \\[0pt]
La bonne éducation \\[0pt]
L'absence \\[0pt]
L'absolu \\[0pt]
L'absolu et le relatif \\[0pt]
L'abstraction \\[0pt]
L'abstrait et le concret \\[0pt]
L'absurde \\[0pt]
La bureaucratie \\[0pt]
L'abus de pouvoir \\[0pt]
L'académisme \\[0pt]
La catastrophe \\[0pt]
La causalité \\[0pt]
La causalité historique \\[0pt]
La cause \\[0pt]
La cause première \\[0pt]
L'accès à la vérité \\[0pt]
L'accident \\[0pt]
La censure \\[0pt]
La certitude \\[0pt]
La chance \\[0pt]
L'achèvement de l'œuvre \\[0pt]
La citation \\[0pt]
La citoyenneté \\[0pt]
La civilisation \\[0pt]
La clarté \\[0pt]
La classification \\[0pt]
La classification des arts \\[0pt]
La cohérence \\[0pt]
La cohérence est-elle une vertu ? \\[0pt]
La colère \\[0pt]
La comédie \\[0pt]
La comédie du pouvoir \\[0pt]
La communauté \\[0pt]
La communauté des savants \\[0pt]
La communauté scientifique \\[0pt]
La communication \\[0pt]
La comparaison \\[0pt]
La composition \\[0pt]
La compréhension \\[0pt]
La confiance \\[0pt]
La confusion \\[0pt]
La connaissance de Dieu \\[0pt]
La connaissance de la vie \\[0pt]
La connaissance des faits \\[0pt]
La connaissance de soi \\[0pt]
La connaissance des passions \\[0pt]
La connaissance du futur \\[0pt]
La connaissance du monde \\[0pt]
La connaissance du passé \\[0pt]
La connaissance du singulier \\[0pt]
La connaissance et la morale \\[0pt]
La connaissance peut-elle être pratique ? \\[0pt]
La connaissance peut-elle se passer de l'imagination ? \\[0pt]
La connaissance scientifique est-elle fondée sur l'expérience ? \\[0pt]
La conquête de l'espace \\[0pt]
La conscience a-t-elle des moments ? \\[0pt]
La conscience de soi \\[0pt]
La conscience morale \\[0pt]
La conscience peut-elle être collective ? \\[0pt]
La conscience universelle \\[0pt]
La considération de l'utilité doit-elle déterminer toutes nos actions ? \\[0pt]
La contemplation \\[0pt]
La contingence \\[0pt]
La contingence du monde \\[0pt]
La contradiction \\[0pt]
La contrainte \\[0pt]
La contrainte en art \\[0pt]
La conversation \\[0pt]
La conversion \\[0pt]
La corruption des mœurs \\[0pt]
La couleur \\[0pt]
La coutume \\[0pt]
La création \\[0pt]
La crédibilité \\[0pt]
La crise \\[0pt]
La critique \\[0pt]
La critique d'art \\[0pt]
La critique de l'État \\[0pt]
La critique fait-elle partie de l'art ? \\[0pt]
La croissance \\[0pt]
La croyance \\[0pt]
La croyance religieuse se distingue-t-elle des autres formes de croyance ? \\[0pt]
La cruauté \\[0pt]
L'acte et la puissance \\[0pt]
L'acte et l'œuvre \\[0pt]
L'acte gratuit \\[0pt]
L'acteur \\[0pt]
L'action \\[0pt]
L'action collective \\[0pt]
L'action et son contexte \\[0pt]
L'action humaine nécessite-t-elle la contingence du monde ? \\[0pt]
L'actualité \\[0pt]
La culpabilité \\[0pt]
La culture \\[0pt]
La culture artistique \\[0pt]
La culture est-elle une question politique ? \\[0pt]
La culture est-elle une seconde nature ? \\[0pt]
La culture et les cultures \\[0pt]
La culture générale \\[0pt]
La curiosité \\[0pt]
La danse \\[0pt]
La décadence \\[0pt]
La déception \\[0pt]
La décision \\[0pt]
La décision a-t-elle besoin de raisons ? \\[0pt]
La déduction \\[0pt]
La défense de la liberté \\[0pt]
La définition \\[0pt]
La délibération \\[0pt]
La démesure \\[0pt]
La démocratie \\[0pt]
La démocratie a-t-elle des limites ? \\[0pt]
La démocratie a-t-elle une histoire ? \\[0pt]
La démocratie peut-elle se passer de représentation ? \\[0pt]
La démonstration \\[0pt]
La dérision \\[0pt]
La désobéissance \\[0pt]
La désobéissance civile \\[0pt]
La désobéissance peut-elle être légitime ? \\[0pt]
La deuxième chance \\[0pt]
La déviance \\[0pt]
La dialectique \\[0pt]
La différence \\[0pt]
La différence des sexes \\[0pt]
La différence des sexes est-elle une question philosophique ? \\[0pt]
La différence homme-femme \\[0pt]
La difformité \\[0pt]
La dignité \\[0pt]
La direction de l'esprit \\[0pt]
La discipline \\[0pt]
La disgrâce \\[0pt]
La disposition \\[0pt]
La dissimulation \\[0pt]
La distinction \\[0pt]
La distraction \\[0pt]
La diversité des cultures \\[0pt]
La diversité des langues \\[0pt]
La diversité des langues est-elle une diversité des pensées ? \\[0pt]
La diversité des religions \\[0pt]
La diversité des sciences \\[0pt]
La division du travail \\[0pt]
L'admiration \\[0pt]
La douleur \\[0pt]
La douleur est-elle utile ? \\[0pt]
La douleur et le plaisir \\[0pt]
La durée \\[0pt]
La faiblesse \\[0pt]
La faiblesse de croire \\[0pt]
La faiblesse de la volonté \\[0pt]
La faiblesse de volonté \\[0pt]
La famille \\[0pt]
La famille est-elle une institution politique ? \\[0pt]
La fatalité \\[0pt]
La fatigue \\[0pt]
La fausseté \\[0pt]
La faute \\[0pt]
La faute et le péché \\[0pt]
La faute et l'erreur \\[0pt]
La fête \\[0pt]
L'affirmation \\[0pt]
La fiction \\[0pt]
La fidélité \\[0pt]
La fierté \\[0pt]
La fièvre \\[0pt]
La finalité \\[0pt]
La fin de la guerre \\[0pt]
La fin de la politique \\[0pt]
La fin de l'art \\[0pt]
La fin de l'histoire \\[0pt]
La fin de l'homme \\[0pt]
La fin du monde \\[0pt]
La fin et les moyens \\[0pt]
La finitude \\[0pt]
La fin justifie-t-elle les moyens ? \\[0pt]
La foi est-elle irrationnelle ? \\[0pt]
La folie \\[0pt]
La fonction des exemples \\[0pt]
La force de conviction \\[0pt]
La force de la croyance \\[0pt]
La force de la loi \\[0pt]
La force de l'habitude \\[0pt]
La force des choses \\[0pt]
La force des faibles \\[0pt]
La force des lois \\[0pt]
La force des récits \\[0pt]
La force et le droit \\[0pt]
La force publique \\[0pt]
La forme \\[0pt]
La forme et le fond \\[0pt]
La fortune \\[0pt]
La foule \\[0pt]
La fragilité \\[0pt]
La franchise \\[0pt]
La franchise est-elle une vertu ? \\[0pt]
La frivolité \\[0pt]
La frontière \\[0pt]
La gauche et la droite \\[0pt]
L'âge de raison \\[0pt]
L'âge d'or \\[0pt]
La généalogie \\[0pt]
La généralisation \\[0pt]
La générosité \\[0pt]
La genèse de l'œuvre \\[0pt]
La gloire \\[0pt]
La gloire est-elle un bien ? \\[0pt]
La grâce \\[0pt]
La grammaire \\[0pt]
La grammaire contraint-elle notre pensée ? \\[0pt]
La grandeur \\[0pt]
La gratuité \\[0pt]
L'agressivité \\[0pt]
La grossièreté \\[0pt]
La guerre \\[0pt]
La guerre civile \\[0pt]
La guerre est-elle la politique continuée par d'autres moyens ? \\[0pt]
La guerre et la paix \\[0pt]
La guerre juste \\[0pt]
La guerre met-elle fin au droit ? \\[0pt]
La haine \\[0pt]
La haine de la raison \\[0pt]
La haine des images \\[0pt]
La hiérarchie \\[0pt]
La honte \\[0pt]
Laisser faire la nature \\[0pt]
Laisser mourir, est-ce tuer ? \\[0pt]
La jalousie \\[0pt]
La jeunesse \\[0pt]
La jeunesse est mécontente \\[0pt]
La joie \\[0pt]
La joie de vivre \\[0pt]
La justice et l'égalité \\[0pt]
La justice internationale \\[0pt]
La justice sociale \\[0pt]
La justification \\[0pt]
La laideur \\[0pt]
La légitimité démocratique \\[0pt]
La lettre et l'esprit \\[0pt]
La liberté admet-elle des degrés ? \\[0pt]
La liberté d'autrui \\[0pt]
La liberté de la volonté \\[0pt]
La liberté de penser \\[0pt]
La liberté des uns s'arrête-elle où commence celle des autres ? \\[0pt]
La liberté d'expression \\[0pt]
La liberté et le hasard \\[0pt]
La liberté peut-elle être une illusion ? \\[0pt]
La liberté peut-elle s'aliéner ? \\[0pt]
La liberté s'apprend-elle ? \\[0pt]
La liberté se prouve-t-elle ou s'éprouve-t-elle ? \\[0pt]
L'aliénation \\[0pt]
La limite \\[0pt]
La logique est-elle un art de penser ? \\[0pt]
La logique est-elle un art de raisonner ? \\[0pt]
La logique est-elle utile à la métaphysique ? \\[0pt]
La loi \\[0pt]
La loi du plus fort \\[0pt]
La loi et la coutume \\[0pt]
La loi et la règle \\[0pt]
La loi et les mœurs \\[0pt]
La loi et l'ordre \\[0pt]
La loi peut-elle être injuste ? \\[0pt]
L'alter ego \\[0pt]
L'altruisme \\[0pt]
La machine \\[0pt]
La magie \\[0pt]
La magie des mots \\[0pt]
La magie peut-elle être efficace ? \\[0pt]
La main \\[0pt]
La maîtrise de soi \\[0pt]
La majorité \\[0pt]
La maladie \\[0pt]
La malveillance \\[0pt]
La manifestation de la vérité \\[0pt]
La marchandise \\[0pt]
La marge \\[0pt]
La marginalité \\[0pt]
La mathématisation du réel \\[0pt]
La matière \\[0pt]
La mauvaise conscience \\[0pt]
La mauvaise foi \\[0pt]
La mauvaise volonté \\[0pt]
L'ambition \\[0pt]
L'âme \\[0pt]
La méchanceté \\[0pt]
La médecine est-elle une science ? \\[0pt]
La méditation \\[0pt]
L'âme et le corps \\[0pt]
La méfiance \\[0pt]
La mélancolie \\[0pt]
L'amélioration des hommes peut-elle être considérée comme un objectif politique ? \\[0pt]
La mémoire \\[0pt]
La mémoire collective \\[0pt]
La mémoire et l'histoire \\[0pt]
La mémoire et l'oubli \\[0pt]
La mesure \\[0pt]
La métamorphose \\[0pt]
La métaphore \\[0pt]
La métaphysique est-elle une science ? \\[0pt]
La métaphysique répond-elle à un besoin ? \\[0pt]
La méthode \\[0pt]
L'ami \\[0pt]
L'ami du prince \\[0pt]
La minorité \\[0pt]
La misère \\[0pt]
L'amitié \\[0pt]
L'amitié relève-t-elle d'une décision ? \\[0pt]
La mode \\[0pt]
La modernité \\[0pt]
La morale est-elle objet de science ? \\[0pt]
La morale est-elle un art de vivre ? \\[0pt]
La morale peut-elle être naturelle ? \\[0pt]
La morale s'apprend-elle ? \\[0pt]
La moralité est-elle affaire de principes ou de conséquences ? \\[0pt]
La mort \\[0pt]
La mort a-t-elle un sens ? \\[0pt]
La mort de Dieu \\[0pt]
La mort de l'art \\[0pt]
La mort de l'homme \\[0pt]
L'amour de la patrie \\[0pt]
L'amour de l'argent \\[0pt]
L'amour de la vérité \\[0pt]
L'amour de l'humanité \\[0pt]
L'amour de soi \\[0pt]
L'amour du destin \\[0pt]
L'amour est-il aveugle ? \\[0pt]
L'amour et la haine \\[0pt]
L'amour et l'amitié \\[0pt]
L'amour implique-t-il le respect ? \\[0pt]
L'amour peut-il être un devoir ? \\[0pt]
L'amour-propre \\[0pt]
La multitude \\[0pt]
La musique a-t-elle un sens ? \\[0pt]
La musique est-elle un langage ? \\[0pt]
L'anachronisme \\[0pt]
La naissance \\[0pt]
La naïveté \\[0pt]
La naïveté est-elle une vertu ? \\[0pt]
L'analogie \\[0pt]
L'analyse \\[0pt]
L'analyse du vécu \\[0pt]
L'analyse et la synthèse \\[0pt]
L'anarchie \\[0pt]
La nature artiste \\[0pt]
La nature a-t-elle des droits ? \\[0pt]
La nature a-t-elle une histoire ? \\[0pt]
La nature est-elle belle ? \\[0pt]
La nature est-elle bien faite ? \\[0pt]
La nature est-elle écrite en langage mathématique ? \\[0pt]
La nature est-elle sacrée ? \\[0pt]
La nature est-elle sans histoire ? \\[0pt]
La nature est-elle un modèle ? \\[0pt]
La nature est-elle un système ? \\[0pt]
La nature existe-t-elle ? \\[0pt]
La nature humaine \\[0pt]
La nature morte \\[0pt]
La nature ne fait pas de saut \\[0pt]
La nature obéit-elle à des fins ? \\[0pt]
La nature peut-elle être un modèle ? \\[0pt]
La nécessité \\[0pt]
La nécessité de l'oubli \\[0pt]
La nécessité fait-elle loi ? \\[0pt]
La négation \\[0pt]
La négligence \\[0pt]
La négligence est-elle une faute ? \\[0pt]
La négociation \\[0pt]
La neige est-elle blanche ? \\[0pt]
Langage et pensée \\[0pt]
L'angoisse \\[0pt]
Langue et parole \\[0pt]
L'animal \\[0pt]
L'animal et la bête \\[0pt]
L'animalité \\[0pt]
L'animalité de l'animal, l'animalité de l'homme \\[0pt]
L'animal politique \\[0pt]
La norme \\[0pt]
La nostalgie \\[0pt]
La notion de comportement \\[0pt]
La notion de monde \\[0pt]
La notion de nature humaine \\[0pt]
La notion de nature humaine a-t-elle un sens ? \\[0pt]
La notion de point de vue \\[0pt]
La notion de système \\[0pt]
La nouveauté \\[0pt]
L'anthropocentrisme \\[0pt]
L'anticipation \\[0pt]
La nudité \\[0pt]
La nuit \\[0pt]
La ou les vertus ? \\[0pt]
La paix \\[0pt]
La paix est-elle l'absence de guerre ? \\[0pt]
La paresse \\[0pt]
La parole \\[0pt]
La parole et l'écriture \\[0pt]
La parole peut-elle être une arme ? \\[0pt]
La participation \\[0pt]
La partie et le tout \\[0pt]
La passion \\[0pt]
La patience \\[0pt]
La pauvreté \\[0pt]
La peine de mort \\[0pt]
La peine de mort est-elle juste, injuste, et pourquoi ? \\[0pt]
La peinture apprend-elle à voir ? \\[0pt]
La peinture des mœurs \\[0pt]
La pensée a-t-elle des lois ? \\[0pt]
La pensée échappe-t-elle à la grammaire ? \\[0pt]
La pensée est-elle une activité assimilable à un travail ? \\[0pt]
La pensée formelle \\[0pt]
La pensée obéit-elle à des lois ? \\[0pt]
La perception \\[0pt]
La perception est-elle source de connaissance ? \\[0pt]
La perfection \\[0pt]
La performance \\[0pt]
La personne \\[0pt]
La perspective \\[0pt]
La pertinence \\[0pt]
La perversion \\[0pt]
La pesanteur \\[0pt]
La pétition de principe \\[0pt]
La peur \\[0pt]
La peur des machines \\[0pt]
La peur est-elle mauvaise conseillère ? \\[0pt]
La philosophie est-elle abstraite ? \\[0pt]
La philosophie et le sens commun \\[0pt]
La philosophie peut-elle être populaire ? \\[0pt]
La photographie est-elle un art ? \\[0pt]
La pitié \\[0pt]
La place d'autrui \\[0pt]
La pluralité \\[0pt]
La pluralité des arts \\[0pt]
La pluralité des interprétations \\[0pt]
La pluralité des langues \\[0pt]
La pluralité des mondes \\[0pt]
La pluralité des opinions \\[0pt]
La pluralité des religions \\[0pt]
La poésie \\[0pt]
La poésie pense-t-elle ? \\[0pt]
La polémique \\[0pt]
La politesse \\[0pt]
La politique consiste-t-elle à gérer l'urgence ? \\[0pt]
La politique est-elle l'affaire de tous ? \\[0pt]
La politique est-elle une science ? \\[0pt]
La politique peut-elle se passer de croyance ? \\[0pt]
La pornographie \\[0pt]
L'apparence \\[0pt]
L'appartenance sociale \\[0pt]
La précaution \\[0pt]
La précaution peut-elle être un principe ? \\[0pt]
La préhistoire \\[0pt]
La présence \\[0pt]
La présence de l'œuvre d'art \\[0pt]
La présomption \\[0pt]
La pression du groupe \\[0pt]
La preuve \\[0pt]
La preuve expérimentale \\[0pt]
La prévision \\[0pt]
La prière \\[0pt]
L'\emph{a priori} \\[0pt]
La prise du pouvoir \\[0pt]
La probabilité \\[0pt]
La promesse \\[0pt]
La propagation des idées \\[0pt]
La propriété \\[0pt]
La propriété, est-ce un vol ? \\[0pt]
La propriété est-elle un droit ? \\[0pt]
La prose du monde \\[0pt]
La protection sociale \\[0pt]
La providence \\[0pt]
La prudence \\[0pt]
La psychanalyse est-elle une science ? \\[0pt]
La publicité \\[0pt]
La pudeur \\[0pt]
La puissance \\[0pt]
La puissance de l'image \\[0pt]
La puissance de l'imagination \\[0pt]
La puissance du langage \\[0pt]
La pulsion \\[0pt]
La punition \\[0pt]
La qualité \\[0pt]
La quantité \\[0pt]
La quantité et la qualité \\[0pt]
La question de l'origine \\[0pt]
La question : « qui ? » \\[0pt]
La raison a-t-elle des limites ? \\[0pt]
La raison a-t-elle une histoire ? \\[0pt]
La raison des mythes \\[0pt]
La raison d'État \\[0pt]
La raison d'être \\[0pt]
La raison est-elle l'esclave des passions ? \\[0pt]
La raison peut-elle nous égarer ? \\[0pt]
La raison peut-elle se contredire ? \\[0pt]
La raison peut-elle servir le mal ? \\[0pt]
La raison s'oppose-t-elle aux passions ? \\[0pt]
La rareté \\[0pt]
La rationalité des émotions \\[0pt]
L'arbitraire \\[0pt]
L'architecte et la cité \\[0pt]
L'architecture est-elle un art ? \\[0pt]
La réalité du futur \\[0pt]
La réalité du mouvement \\[0pt]
La réalité du passé \\[0pt]
La réalité du possible \\[0pt]
La réalité du rêve \\[0pt]
La réalité virtuelle \\[0pt]
La recherche de l'authenticité \\[0pt]
La recherche de la vérité \\[0pt]
La recherche des causes \\[0pt]
La réciprocité \\[0pt]
La reconnaissance \\[0pt]
La réflexion \\[0pt]
La réforme \\[0pt]
La règle \\[0pt]
La régularité \\[0pt]
La relation \\[0pt]
La religion \\[0pt]
La religion est-elle l'opium du peuple ? \\[0pt]
La religion est-elle une affaire privée ? \\[0pt]
La religion peut-elle être civile ? \\[0pt]
La renaissance \\[0pt]
La rencontre \\[0pt]
La répétition \\[0pt]
La représentation \\[0pt]
La reproduction \\[0pt]
La résistance \\[0pt]
La responsabilité \\[0pt]
La responsabilité peut-elle être collective ? \\[0pt]
La ressemblance \\[0pt]
La rêverie \\[0pt]
La révolte \\[0pt]
La révolte peut-elle être un droit ? \\[0pt]
La révolution \\[0pt]
L'argent \\[0pt]
L'argument d'autorité \\[0pt]
La richesse \\[0pt]
L'art abstrait \\[0pt]
L'art a-t-il une histoire ? \\[0pt]
L'art de la discussion \\[0pt]
L'art de persuader \\[0pt]
L'art de vivre \\[0pt]
L'art d'inventer \\[0pt]
L'art doit-il divertir ? \\[0pt]
L'art donne-t-il à penser ? \\[0pt]
L'art du corps \\[0pt]
L'art du mensonge \\[0pt]
L'art est-il au service du beau ? \\[0pt]
L'art est-il imitatif ? \\[0pt]
L'art est-il inutile ? \\[0pt]
L'art est-il mensonger ? \\[0pt]
L'art est-il une connaissance ? \\[0pt]
L'art est-il un luxe ? \\[0pt]
L'art est-il un mode de connaissance ? \\[0pt]
L'art et la morale \\[0pt]
L'art et la nature \\[0pt]
L'art et la vie \\[0pt]
L'art et le temps \\[0pt]
L'artificiel \\[0pt]
L'artiste de soi-même \\[0pt]
L'artiste est-il un créateur ? \\[0pt]
L'artiste et l'artisan \\[0pt]
L'artiste et le savant \\[0pt]
L'artiste recherche-t-il le beau ? \\[0pt]
L'art permet-il un accès au divin ? \\[0pt]
L'art peut-il être brut ? \\[0pt]
L'art peut-il être populaire ? \\[0pt]
L'art peut-il sauver le monde ? \\[0pt]
L'art peut-il se passer de la beauté ? \\[0pt]
L'art peut-il se passer d'œuvres ? \\[0pt]
L'art pour l'art \\[0pt]
L'art vise-t-il le beau ? \\[0pt]
La ruse \\[0pt]
La sagesse \\[0pt]
La santé \\[0pt]
La santé est-elle un droit ou un devoir ? \\[0pt]
La satisfaction \\[0pt]
L'ascétisme \\[0pt]
La science est-elle austère ? \\[0pt]
La science et le faux \\[0pt]
La science exclut-elle l'imagination ? \\[0pt]
La science pense-t-elle ? \\[0pt]
La science peut-elle se passer de métaphysique ? \\[0pt]
La seconde nature \\[0pt]
La sécurité \\[0pt]
La séduction \\[0pt]
La sensation \\[0pt]
La sensibilité \\[0pt]
La servitude \\[0pt]
La sexualité \\[0pt]
La signification \\[0pt]
La signification des mots \\[0pt]
L'asile de l'ignorance \\[0pt]
La simplicité \\[0pt]
La simulation \\[0pt]
La sincérité \\[0pt]
La singularité \\[0pt]
La sociabilité \\[0pt]
La société existe-t-elle ? \\[0pt]
La société précède-t-elle l'individu ? \\[0pt]
La solidarité \\[0pt]
La solitude \\[0pt]
La souffrance \\[0pt]
La souffrance a-t-elle une valeur morale ? \\[0pt]
La souffrance a-t-elle un sens ? \\[0pt]
La souffrance d'autrui \\[0pt]
La souveraineté \\[0pt]
La souveraineté peut-elle être déléguée \\[0pt]
La souveraineté peut-elle être limitée ? \\[0pt]
La souveraineté peut-elle se partager ? \\[0pt]
La spéculation \\[0pt]
La spontanéité \\[0pt]
La structure \\[0pt]
La subjectivité \\[0pt]
La substance \\[0pt]
La subtilité \\[0pt]
La superstition \\[0pt]
La surface et la profondeur \\[0pt]
La sympathie \\[0pt]
La tautologie \\[0pt]
La technique \\[0pt]
La technique est-elle dangereuse ? \\[0pt]
La technique est-elle libératrice ? \\[0pt]
La technique est-elle moralement neutre ? \\[0pt]
La technique est-elle neutre ? \\[0pt]
La technique fait-elle violence à la nature ? \\[0pt]
La technique n'est-elle qu'une application de la science ? \\[0pt]
La technique peut-elle améliorer l'homme ? \\[0pt]
La technique repose-t-elle sur le génie du technicien ? \\[0pt]
La télévision \\[0pt]
La tempérance \\[0pt]
La temporalité de l'œuvre d'art \\[0pt]
La tentation \\[0pt]
La terreur \\[0pt]
L'athéisme \\[0pt]
La théodicée \\[0pt]
La théorie nous éloigne-t-elle de la réalité ? \\[0pt]
La théorie peut-elle nous égarer ? \\[0pt]
La tolérance \\[0pt]
La totalité \\[0pt]
La tradition \\[0pt]
La traduction \\[0pt]
La trahison \\[0pt]
La tranquillité \\[0pt]
La transcendance \\[0pt]
La transmission de pensée \\[0pt]
La transparence \\[0pt]
La tristesse \\[0pt]
L'attente \\[0pt]
L'attention \\[0pt]
L'attraction \\[0pt]
L'au-delà \\[0pt]
L'authenticité \\[0pt]
L'autobiographie \\[0pt]
L'autocritique \\[0pt]
L'automate \\[0pt]
L'autonomie \\[0pt]
L'autorité \\[0pt]
L'autorité de la loi \\[0pt]
L'autorité de la science \\[0pt]
L'autorité des lois \\[0pt]
La valeur \\[0pt]
La valeur de la pitié \\[0pt]
La valeur de l'art \\[0pt]
La valeur de la vie \\[0pt]
La valeur de l'exemple \\[0pt]
La valeur de l'hypothèse \\[0pt]
La valeur des images \\[0pt]
La valeur d'une œuvre \\[0pt]
La valeur du travail \\[0pt]
La valeur et le prix \\[0pt]
L'avant-garde \\[0pt]
La veille et le sommeil \\[0pt]
La vengeance \\[0pt]
L'avenir \\[0pt]
L'aventure \\[0pt]
La vérification \\[0pt]
La vérité \\[0pt]
La vérité a-t-elle une histoire ? \\[0pt]
La vérité de la religion \\[0pt]
La vérité en art \\[0pt]
La vérité est-elle intemporelle ? \\[0pt]
La vérité est-elle triste ? \\[0pt]
La vérité mathématique \\[0pt]
La vérité nous rend-elle libres ? \\[0pt]
La vérité peut-elle être tolérante ? \\[0pt]
La vertu \\[0pt]
La vertu de l'oubli \\[0pt]
La vertu peut-elle s'enseigner ? \\[0pt]
La vertu s'enseigne-t-elle ? \\[0pt]
L'aveu \\[0pt]
La vie \\[0pt]
La vie après la mort \\[0pt]
La vie a-t-elle un sens ? \\[0pt]
La vie de l'esprit \\[0pt]
La vie en société est-elle naturelle à l'homme ? \\[0pt]
La vie est-elle l'objet des sciences de la vie ? \\[0pt]
La vie est-elle une valeur ? \\[0pt]
La vie est-elle un roman ? \\[0pt]
La vie est-elle un songe ? \\[0pt]
« La vie est une scène » \\[0pt]
La vieillesse \\[0pt]
La vie sexuelle est-elle volontaire ? \\[0pt]
La vie sociale est-elle une comédie ? \\[0pt]
La ville \\[0pt]
La ville et la campagne \\[0pt]
La violence \\[0pt]
La vision et le toucher \\[0pt]
L'avocat du diable \\[0pt]
La voix \\[0pt]
La voix de la conscience \\[0pt]
La volonté de savoir \\[0pt]
La volonté divine \\[0pt]
La volonté du peuple \\[0pt]
La volonté générale est-elle la volonté de tous ? \\[0pt]
La volonté peut-elle être générale ? \\[0pt]
La vulgarisation \\[0pt]
La vulgarité \\[0pt]
Le bavardage \\[0pt]
Le beau a-t-il une histoire ? \\[0pt]
Le beau est-il universel ? \\[0pt]
Le beau et le bien \\[0pt]
Le beau et le bien sont-ils, au fond, identiques ? \\[0pt]
Le beau et le joli \\[0pt]
Le beau peut-il être bizarre ? \\[0pt]
Le besoin de philosophie \\[0pt]
Le besoin de sens \\[0pt]
Le bien commun \\[0pt]
Le bien d'autrui \\[0pt]
Le bien et le mal \\[0pt]
Le bien et l'utile \\[0pt]
Le bien-être \\[0pt]
Le bon goût \\[0pt]
Le bonheur \\[0pt]
Le bonheur dans le mal \\[0pt]
Le bonheur des méchants \\[0pt]
Le bonheur du juste \\[0pt]
Le bonheur est-il affaire de volonté ? \\[0pt]
Le bonheur est-il affaire privée ? \\[0pt]
Le bon sens \\[0pt]
Le bouc émissaire \\[0pt]
Le bruit et la musique \\[0pt]
Le but de l'association politique \\[0pt]
Le calcul \\[0pt]
Le capitalisme \\[0pt]
Le cas de conscience \\[0pt]
L'échange \\[0pt]
Le changement \\[0pt]
Le chaos \\[0pt]
Le charme \\[0pt]
Le châtiment \\[0pt]
L'échec \\[0pt]
Le choix \\[0pt]
Le choix des moyens \\[0pt]
Le citoyen \\[0pt]
Le clair et l'obscur \\[0pt]
Le classicisme \\[0pt]
Le cliché \\[0pt]
Le cœur et la raison \\[0pt]
Le comédien \\[0pt]
Le comique \\[0pt]
Le commencement \\[0pt]
Le commencement du monde \\[0pt]
Le comment et le pourquoi \\[0pt]
Le commerce adoucit-il les mœurs ? \\[0pt]
Le commerce des idées \\[0pt]
Le commun \\[0pt]
Le compromis \\[0pt]
Le concept \\[0pt]
Le concret et l'abstrait \\[0pt]
Le conditionnel \\[0pt]
Le conflit \\[0pt]
Le conformisme \\[0pt]
Le confort intellectuel \\[0pt]
L'économie est-elle une science ? \\[0pt]
Le conscient et l'inconscient \\[0pt]
Le conseil \\[0pt]
Le consensus \\[0pt]
Le consentement \\[0pt]
Le contemporain \\[0pt]
Le corps de l'autre \\[0pt]
Le corps est-il le reflet de l'âme ? \\[0pt]
Le corps et l'esprit \\[0pt]
Le corps peut-il être objet d'art ? \\[0pt]
Le cosmopolitisme \\[0pt]
Le courage \\[0pt]
Le courage de penser \\[0pt]
Le crime contre l'humanité \\[0pt]
L'écrit et l'oral \\[0pt]
L'écriture \\[0pt]
L'écriture de l'histoire \\[0pt]
L'écriture est-elle une technique parmi d'autres ? \\[0pt]
L'écriture et la pensée \\[0pt]
Le cynisme \\[0pt]
Le dandysme \\[0pt]
Le dégoût \\[0pt]
Le dérèglement \\[0pt]
Le désenchantement \\[0pt]
Le désespoir \\[0pt]
Le désintéressement \\[0pt]
Le désir \\[0pt]
Le désir de gloire \\[0pt]
Le désir de reconnaissance \\[0pt]
Le désir de savoir \\[0pt]
Le désir de savoir est-il naturel ? \\[0pt]
Le désir d'éternité \\[0pt]
Le désir de vérité \\[0pt]
Le désir est-il sans limite ? \\[0pt]
Le désir et la loi \\[0pt]
Le désir n'est-il pas qu'inquiétude ? \\[0pt]
Le désœuvrement \\[0pt]
Le désordre \\[0pt]
Le despotisme \\[0pt]
Le dessin et la couleur \\[0pt]
Le destin \\[0pt]
Le détachement \\[0pt]
Le détail \\[0pt]
Le déterminisme \\[0pt]
Le devenir \\[0pt]
Le devoir de mémoire \\[0pt]
Le dialogue \\[0pt]
Le dieu des philosophes \\[0pt]
Le Dieu des philosophes \\[0pt]
Le divertissement \\[0pt]
Le dogmatisme \\[0pt]
Le don \\[0pt]
Le don est-il toujours généreux ? \\[0pt]
Le donné \\[0pt]
Le double \\[0pt]
Le doute \\[0pt]
Le doute peut-il être méthodique ? \\[0pt]
Le droit à la révolte \\[0pt]
Le droit de propriété \\[0pt]
Le droit de punir \\[0pt]
Le droit de résistance \\[0pt]
Le droit des peuples à disposer d'eux-mêmes \\[0pt]
Le droit du plus fort \\[0pt]
Le droit et la violence \\[0pt]
Le droit naturel \\[0pt]
Le droit peut-il être flexible ? \\[0pt]
Le dualisme \\[0pt]
L'éducation \\[0pt]
Le factice \\[0pt]
Le fait \\[0pt]
Le fait de vivre constitue-t-il un bien en soi ? \\[0pt]
Le fait divers \\[0pt]
Le fait et le droit \\[0pt]
Le fanatisme \\[0pt]
Le fantastique \\[0pt]
Le fatalisme \\[0pt]
Le faux \\[0pt]
Le féminin \\[0pt]
Le féminin et le masculin \\[0pt]
Le féminisme \\[0pt]
L'effet et la cause \\[0pt]
L'effort \\[0pt]
Le fil conducteur \\[0pt]
Le fini \\[0pt]
Le fini et l'infini \\[0pt]
Le fond et la forme \\[0pt]
Le formalisme \\[0pt]
Le futur est-il contingent ? \\[0pt]
L'égalité \\[0pt]
L'égalité des chances \\[0pt]
L'égalité des sexes \\[0pt]
L'égalité est-elle souhaitable ? \\[0pt]
L'égalité est-elle une condition de la liberté ? \\[0pt]
Légalité et causalité \\[0pt]
Légalité et légitimité \\[0pt]
Le général et le particulier \\[0pt]
Le génie \\[0pt]
Le genre \\[0pt]
Le genre et l'espèce \\[0pt]
Le geste \\[0pt]
L'égoïsme \\[0pt]
Le goût \\[0pt]
Le goût de la beauté \\[0pt]
Le goût du risque \\[0pt]
Le goût est-il affaire d'éducation ? \\[0pt]
Le groupe \\[0pt]
Le handicap \\[0pt]
Le hasard \\[0pt]
Le hasard est-il injuste ? \\[0pt]
Le hasard et la nécessité \\[0pt]
Le hasard existe-t-il ? \\[0pt]
Le hasard fait-il bien les choses ? \\[0pt]
Le haut \\[0pt]
Le haut et le bas \\[0pt]
Le héros \\[0pt]
Le « je ne sais quoi » \\[0pt]
Le jeu \\[0pt]
Le jeu et le sérieux \\[0pt]
Le jugement de goût est-il désintéressé ? \\[0pt]
Le jugement de valeur \\[0pt]
Le juste et le bien \\[0pt]
Le langage de la peinture \\[0pt]
Le langage est-il assimilable à un outil ? \\[0pt]
Le langage est-il l'auxiliaire de la pensée ? \\[0pt]
Le langage est-il un instrument ? \\[0pt]
Le langage et le réel \\[0pt]
L'élection \\[0pt]
Le légal et le légitime \\[0pt]
L'élégance \\[0pt]
Le légitime et le légal \\[0pt]
Le libre arbitre \\[0pt]
Le lien social \\[0pt]
Le lieu commun \\[0pt]
Le lieu et l'espace \\[0pt]
Le livre \\[0pt]
Le logique \\[0pt]
Le logique est-elle un art de penser ? \\[0pt]
Le loisir \\[0pt]
Le luxe \\[0pt]
Le lyrisme \\[0pt]
Le maître et l'esclave \\[0pt]
Le malentendu \\[0pt]
Le mal est-il une erreur ? \\[0pt]
Le mal est-il une objection à l'existence de Dieu ? \\[0pt]
Le masculin \\[0pt]
Le masculin et le féminin \\[0pt]
Le masque \\[0pt]
Le matérialisme \\[0pt]
Le mauvais goût \\[0pt]
L'embarras du choix \\[0pt]
Le mécanisme \\[0pt]
Le méchant \\[0pt]
Le méchant peut-il être heureux ? \\[0pt]
Le meilleur \\[0pt]
Le meilleur des mondes \\[0pt]
Le meilleur régime \\[0pt]
Le même et l'autre \\[0pt]
Le mensonge \\[0pt]
Le mépris \\[0pt]
Le mérite \\[0pt]
Le mérite et les talents \\[0pt]
Le merveilleux \\[0pt]
Le mien et le tien \\[0pt]
Le mieux est-il l'ennemi du bien ? \\[0pt]
Le milieu \\[0pt]
Le miracle \\[0pt]
Le miroir \\[0pt]
Le misanthrope \\[0pt]
Le modèle \\[0pt]
Le moi \\[0pt]
Le moi est-il haïssable ? \\[0pt]
Le moindre mal \\[0pt]
Le moi n'est-il qu'une idée ? \\[0pt]
Le moment propice \\[0pt]
Le monde \\[0pt]
Le monde des idées \\[0pt]
Le monde est-il écrit en langage mathématique ? \\[0pt]
Le monde est-il ma représentation ? \\[0pt]
Le monde est-il un théâtre ? \\[0pt]
Le monde extérieur existe-t-il ? \\[0pt]
Le monde intelligible \\[0pt]
Le monstre \\[0pt]
Le moralisme \\[0pt]
Le mot d'esprit \\[0pt]
Le mot et la chose \\[0pt]
L'émotion \\[0pt]
L'émotion esthétique \\[0pt]
Le mot juste \\[0pt]
Le mot vie a-t-il plusieurs sens ? \\[0pt]
Le mouvement \\[0pt]
Le mouvement de la pensée \\[0pt]
L'empirisme \\[0pt]
L'emploi du temps \\[0pt]
Le multiple \\[0pt]
Le musée \\[0pt]
Le Musée \\[0pt]
Le mythe \\[0pt]
Le naturel et l'artificiel \\[0pt]
L'encyclopédie \\[0pt]
Le néant \\[0pt]
Le néant est-il ? \\[0pt]
Le nécessaire et le contingent \\[0pt]
Le nécessaire et le superflu \\[0pt]
Le néologisme \\[0pt]
L'énergie du désespoir \\[0pt]
L'enfance \\[0pt]
L'enfance de l'art \\[0pt]
L'enfant \\[0pt]
L'enfant et l'adulte \\[0pt]
« L'enfer est pavé de bonnes intentions » \\[0pt]
L'enfer est pavé de bonnes intentions \\[0pt]
L'engagement politique \\[0pt]
L'énigme \\[0pt]
Le nihilisme \\[0pt]
L'ennemi \\[0pt]
L'ennemi intérieur \\[0pt]
L'ennui \\[0pt]
Le nombre \\[0pt]
Le nom et le verbe \\[0pt]
Le nominalisme \\[0pt]
Le nom propre \\[0pt]
Le non-sens \\[0pt]
L'enquête \\[0pt]
L'enseignement peut-il se passer d'exemples ? \\[0pt]
L'enthousiasme \\[0pt]
Le nu et la nudité \\[0pt]
L'envie \\[0pt]
L'environnement \\[0pt]
Le pacifisme \\[0pt]
Le paradigme \\[0pt]
Le paradoxe \\[0pt]
Le pardon \\[0pt]
Le pardon et l'oubli \\[0pt]
Le pardon peut-il être une obligation ? \\[0pt]
Le pari \\[0pt]
Le passage à l'acte \\[0pt]
Le passé \\[0pt]
Le passé a-t-il plus de réalité que l'avenir ? \\[0pt]
Le passé a-t-il un intérêt ? \\[0pt]
Le passé est-il réel ? \\[0pt]
Le paternalisme \\[0pt]
Le patriotisme \\[0pt]
Le paysage \\[0pt]
Le pessimisme \\[0pt]
Le peuple \\[0pt]
Le peuple est-il bête ? \\[0pt]
Le philosophe s'écarte-t-il du réel ? \\[0pt]
Le plagiat \\[0pt]
Le plaisir \\[0pt]
Le plaisir d'avoir mal \\[0pt]
Le plaisir d'être libre \\[0pt]
Le plaisir esthétique \\[0pt]
Le plaisir est-il immoral ? \\[0pt]
Le plaisir et la douleur \\[0pt]
Le plaisir peut-il être immoral ? \\[0pt]
Le plaisir peut-il être partagé ? \\[0pt]
Le pluralisme \\[0pt]
Le plus et le moins \\[0pt]
Le plus grand bonheur pour le plus grand nombre \\[0pt]
Le poids de la culture \\[0pt]
Le poids du passé \\[0pt]
Le point de vue \\[0pt]
Le politique doit-il s'appuyer sur la science ? \\[0pt]
Le populaire \\[0pt]
Le portrait \\[0pt]
Le possible \\[0pt]
Le possible et le réel \\[0pt]
Le possible et le virtuel \\[0pt]
Le possible et l'impossible \\[0pt]
Le pour et le contre \\[0pt]
Le pourquoi et le comment \\[0pt]
Le pouvoir corrompt-il ? \\[0pt]
Le pouvoir des images \\[0pt]
Le pouvoir des mots \\[0pt]
Le pouvoir peut-il être limité ? \\[0pt]
Le pragmatisme \\[0pt]
Le préjugé \\[0pt]
Le premier \\[0pt]
Le présent \\[0pt]
L'épreuve \\[0pt]
Le primitif \\[0pt]
Le principe \\[0pt]
Le privé et le public \\[0pt]
Le prix des choses \\[0pt]
Le probable \\[0pt]
Le problème \\[0pt]
Le proche et le lointain \\[0pt]
Le progrès \\[0pt]
Le progrès des sciences \\[0pt]
Le provisoire \\[0pt]
Le public et le privé \\[0pt]
Lequel, de l'art ou du réel, est-il une imitation de l'autre ? \\[0pt]
L'équité \\[0pt]
Le quotidien \\[0pt]
Le raisonnable et le rationnel \\[0pt]
Le rapport de force \\[0pt]
Le rationalisme peut-il être une passion ? \\[0pt]
Le rationnel \\[0pt]
Le rationnel et le raisonnable \\[0pt]
Le réalisme \\[0pt]
Le récit \\[0pt]
Le recours à l'Histoire \\[0pt]
Le réel et le nécessaire \\[0pt]
Le réel et l'idéal \\[0pt]
Le réel et l'irréel \\[0pt]
Le réel peut-il échapper à la logique ? \\[0pt]
Le refus de la vérité \\[0pt]
Le regard \\[0pt]
Le règne des passions \\[0pt]
Le regret \\[0pt]
Le remords \\[0pt]
Le renoncement \\[0pt]
Le repos \\[0pt]
Le respect \\[0pt]
Le respect de la nature \\[0pt]
Le respect de soi \\[0pt]
Le respect de soi-même \\[0pt]
Le ressentiment \\[0pt]
Le retour à la nature \\[0pt]
Le rêve \\[0pt]
Le rêve et la réalité \\[0pt]
Le riche et le pauvre \\[0pt]
Le ridicule \\[0pt]
Le rire \\[0pt]
Le risque \\[0pt]
Le rite \\[0pt]
Le rituel \\[0pt]
Le rôle des exemples \\[0pt]
Le roman \\[0pt]
L'érotisme \\[0pt]
L'erreur \\[0pt]
L'erreur est-elle humaine ? \\[0pt]
L'erreur est humaine \\[0pt]
L'erreur et la faute \\[0pt]
L'erreur et l'illusion \\[0pt]
L'érudition \\[0pt]
Le rythme \\[0pt]
Le sacré \\[0pt]
Le sacrifice \\[0pt]
Les âges de la vie \\[0pt]
Les âges de l'humanité \\[0pt]
Les animaux ont-ils des droits ? \\[0pt]
Les animaux pensent-ils ? \\[0pt]
Les animaux peuvent-ils avoir des droits ? \\[0pt]
Les apparences sont-elles toujours trompeuses ? \\[0pt]
Les arts admettent-ils une hiérarchie ? \\[0pt]
Les arts appliqués \\[0pt]
Les arts ont-ils besoin de théorie ? \\[0pt]
Les arts populaires \\[0pt]
Le sauvage \\[0pt]
Les avant-gardes \\[0pt]
Le savoir absolu \\[0pt]
Le savoir a-t-il des degrés ? \\[0pt]
Le savoir est-il libérateur ? \\[0pt]
Le savoir est-il une forme de pouvoir ? \\[0pt]
Les besoins et les désirs \\[0pt]
Les bonnes manières \\[0pt]
Les bonnes résolutions \\[0pt]
Les bons comptes font-ils les bons amis ? \\[0pt]
Les bons sentiments \\[0pt]
Le scandale \\[0pt]
Les caractères \\[0pt]
Les catégories \\[0pt]
Les causes et les raisons \\[0pt]
Les causes finales \\[0pt]
Le scepticisme \\[0pt]
Les choses en soi \\[0pt]
Les choses et les événements \\[0pt]
Les choses ont-elles un sens ? \\[0pt]
Les cinq sens \\[0pt]
L'esclavage \\[0pt]
Les coïncidences ont-elles des causes ? \\[0pt]
Les convictions d'autrui sont-elles un argument ? \\[0pt]
Les désirs et les valeurs \\[0pt]
Les droits de l'enfant \\[0pt]
Les droits de l'homme \\[0pt]
Les droits de l'homme sont-ils les droits de la femme ? \\[0pt]
Les écrans \\[0pt]
Le secret \\[0pt]
Les élites \\[0pt]
Le semblable \\[0pt]
Le sensationnel \\[0pt]
Le sens commun \\[0pt]
Le sens de la justice \\[0pt]
Le sens de la vie \\[0pt]
Le sens de l'histoire \\[0pt]
Le sens de l'humour \\[0pt]
Le sens du destin \\[0pt]
Le sens du devoir \\[0pt]
Le sens du travail \\[0pt]
Le sens moral \\[0pt]
Le sens moral est-il naturel ? \\[0pt]
Le sentiment de liberté \\[0pt]
Le sentiment d'injustice \\[0pt]
Le sentiment moral \\[0pt]
Le sérieux \\[0pt]
Le serment \\[0pt]
Les exceptions ont-elles une place en morale ? \\[0pt]
Les excuses \\[0pt]
Le sexe \\[0pt]
Les faits et les valeurs \\[0pt]
Les faits parlent-ils d'eux-mêmes ? \\[0pt]
« Les faits, rien que les faits » \\[0pt]
Les fins de la technique sont-elles techniques ? \\[0pt]
Les fins et les moyens \\[0pt]
Les « forces de l'ordre » \\[0pt]
Les foules \\[0pt]
Les frontières \\[0pt]
Les générations futures \\[0pt]
Les genres naturels \\[0pt]
Les hommes et les femmes \\[0pt]
Les hommes sont-ils des animaux ? \\[0pt]
Les hors-la-loi \\[0pt]
Les hypothèses scientifiques ont-elles pour nature d'être confirmées ou infirmées ? \\[0pt]
Les idées ont-elles une histoire ? \\[0pt]
Le signe et le symbole \\[0pt]
Le silence \\[0pt]
Le silence signifie-t-il toujours l'échec du langage ? \\[0pt]
Les images empêchent-elles de penser ? \\[0pt]
Les images ont-elles un sens ? \\[0pt]
Le simple \\[0pt]
Le simple et le complexe \\[0pt]
Le simulacre \\[0pt]
Les leçons de l'expérience \\[0pt]
Les leçons de l'histoire \\[0pt]
Les limites de la discussion \\[0pt]
Les limites de la raison \\[0pt]
Les limites de la science \\[0pt]
Les limites de l'imaginaire \\[0pt]
Les limites du langage \\[0pt]
Les livres \\[0pt]
Les lois de la nature \\[0pt]
Les lois de la nature sont-elles contingentes ? \\[0pt]
Les lois de la pensée \\[0pt]
Les lois et les mœurs \\[0pt]
Les machines pensent-elles ? \\[0pt]
Les mathématiques consistent-elles seulement en des opérations de l'esprit ? \\[0pt]
Les mathématiques ont-elles besoin d'un fondement ? \\[0pt]
Les mathématiques se réduisent-elles à une pensée cohérente ? \\[0pt]
Les mathématiques sont-elles une découverte ou une invention ? \\[0pt]
Les mathématiques sont-elles un langage ? \\[0pt]
Les mathématiques sont-elles utiles au philosophe ? \\[0pt]
Les méchants peuvent-ils être amis ? \\[0pt]
« Les miracles de la technique » \\[0pt]
Les mondes de l'art \\[0pt]
Les mondes possibles \\[0pt]
Les monstres \\[0pt]
Les mots disent-ils les choses ? \\[0pt]
Les mots et la signification \\[0pt]
Les mots et les choses \\[0pt]
Les mots peuvent-ils nous manquer ? \\[0pt]
Les moyens et les fins \\[0pt]
Les noms propres ont-ils une signification ? \\[0pt]
Les objets de la pensée \\[0pt]
Les objets sont-ils colorés ? \\[0pt]
Les œuvres d'art sont-elles des choses ? \\[0pt]
Les œuvres d'art sont-elles éternelles ? \\[0pt]
Le soleil se lèvera-t-il demain ? \\[0pt]
Le sommeil \\[0pt]
L'ésotérisme \\[0pt]
Le souci de soi \\[0pt]
Le souvenir \\[0pt]
Le souverain bien \\[0pt]
L'espace \\[0pt]
L'espace et le lieu \\[0pt]
Les passions ont-elles une place en politique ? \\[0pt]
Les passions peuvent-elles être raisonnables ? \\[0pt]
Les passions sont-elles toujours mauvaises ? \\[0pt]
Les passions s'opposent-elles à la raison ? \\[0pt]
L'espèce humaine \\[0pt]
Le spectacle \\[0pt]
Le spectacle de la nature \\[0pt]
Les philosophes doivent-ils être rois ? \\[0pt]
Les philosophies se classent-elles ? \\[0pt]
L'espoir \\[0pt]
L'espoir et la crainte \\[0pt]
Le sport \\[0pt]
Les possibles \\[0pt]
L'esprit critique \\[0pt]
L'esprit des lois \\[0pt]
L'esprit de système \\[0pt]
L'esprit et la lettre \\[0pt]
L'esprit et le cerveau \\[0pt]
L'esprit peut-il être objet de science ? \\[0pt]
Les qualités sensibles sont-elles dans les choses ou dans l'esprit ? \\[0pt]
Les raisons d'aimer \\[0pt]
Les raisons de croire \\[0pt]
Les raisons d'espérer \\[0pt]
Les règles de l'art \\[0pt]
Les ressources naturelles \\[0pt]
Les sciences décrivent-elles ou construisent-elles leurs objets ? \\[0pt]
Les sciences de la nature \\[0pt]
Les sciences humaines sont-elles des sciences ? \\[0pt]
Les sciences ont-elles besoin de principes fondamentaux ? \\[0pt]
Les sciences peuvent-elles exclure toute notion de finalité ? \\[0pt]
Les sciences sont-elles une description du monde ? \\[0pt]
Les sens nous trompent-ils ? \\[0pt]
L'essentiel \\[0pt]
Les sentiments ont-ils une histoire ? \\[0pt]
Les signes de l'intelligence \\[0pt]
L'esthétique \\[0pt]
L'estime de soi \\[0pt]
Le style \\[0pt]
Le sublime \\[0pt]
Le sujet \\[0pt]
Le sujet de l'histoire \\[0pt]
Le surnaturel \\[0pt]
Les valeurs \\[0pt]
Les vertus cardinales \\[0pt]
Les vertus ne sont-elles que des vices déguisés ? \\[0pt]
Les vices privés peuvent-ils faire le bien public ? \\[0pt]
Les vivants et les morts \\[0pt]
Le système \\[0pt]
Le tableau \\[0pt]
Le tableau ? \\[0pt]
Le tacite \\[0pt]
Le tas et le tout \\[0pt]
L'État a-t-il le droit de contrôler notre habillement ? \\[0pt]
« L'État, c'est moi » \\[0pt]
L'État de droit \\[0pt]
L'État doit-il être neutre ? \\[0pt]
L'État et la protection \\[0pt]
L'État-Providence \\[0pt]
Le témoignage \\[0pt]
Le témoignage des sens \\[0pt]
Le témoin \\[0pt]
Le temps \\[0pt]
Le temps de la réflexion \\[0pt]
Le temps est-il notre ennemi ? \\[0pt]
Le temps est-il une prison ? \\[0pt]
Le temps existe-t-il ? \\[0pt]
Le temps libre \\[0pt]
Le temps s'écoule-t-il ? \\[0pt]
L'éternité \\[0pt]
Le terrorisme est-il un acte de guerre ? \\[0pt]
Le théâtre et l'existence \\[0pt]
L'ethnocentrisme \\[0pt]
L'étonnement \\[0pt]
Le toucher \\[0pt]
Le tout et la partie \\[0pt]
Le tout et les parties \\[0pt]
Le tragique \\[0pt]
Le tragique et le comique \\[0pt]
Le trait d'esprit \\[0pt]
L'étranger \\[0pt]
Le travail \\[0pt]
Le travail du négatif \\[0pt]
Le travail et le labeur \\[0pt]
Le travail et l'œuvre \\[0pt]
Le travail fonde-t-il la propriété ? \\[0pt]
L'être et la relation \\[0pt]
L'être et l'essence \\[0pt]
L'être humain désire-t-il naturellement connaître ? \\[0pt]
Le tribunal de l'histoire \\[0pt]
Le trompe-l'œil \\[0pt]
L'euthanasie \\[0pt]
L'évaluation \\[0pt]
Le vécu \\[0pt]
Le vécu et la vérité \\[0pt]
L'événement \\[0pt]
Le vertige \\[0pt]
Le vêtement \\[0pt]
Le vide \\[0pt]
L'évidence \\[0pt]
Le virtuel \\[0pt]
Le visible et l'invisible \\[0pt]
Le vivant \\[0pt]
Le vivant échappe-t-il à la connaissance ? \\[0pt]
Le vivant et la technique \\[0pt]
Le volontaire et l'involontaire \\[0pt]
Le volontarisme \\[0pt]
Le voyage \\[0pt]
Le voyage dans le temps \\[0pt]
Le vrai a-t-il une histoire ? \\[0pt]
Le vrai et le bien sont-ils analogues ? \\[0pt]
Le vrai et le faux \\[0pt]
Le vrai et le vraisemblable \\[0pt]
Le vraisemblable \\[0pt]
L'exception \\[0pt]
L'exemple \\[0pt]
L'exercice \\[0pt]
L'exigence de vérité a-t-elle un sens moral ? \\[0pt]
L'exil \\[0pt]
L'existence a-t-elle un sens ? \\[0pt]
L'existence d'autrui \\[0pt]
L'existence de Dieu \\[0pt]
L'existence des idées \\[0pt]
L'existence est-elle une propriété ? \\[0pt]
L'expérience \\[0pt]
L'expérience du désir \\[0pt]
L'expérience du temps \\[0pt]
L'expérience esthétique \\[0pt]
L'expérience instruit-elle ? \\[0pt]
L'expérience métaphysique \\[0pt]
L'expérience morale \\[0pt]
L'expérience peut-elle contredire la théorie ? \\[0pt]
L'expérimentation \\[0pt]
L'explication \\[0pt]
L'expression \\[0pt]
L'extraordinaire \\[0pt]
L'extrémisme \\[0pt]
L'habileté \\[0pt]
L'habitude \\[0pt]
L'habitude a-t-elle des vertus ? \\[0pt]
L'habitude est-elle une seconde nature ? \\[0pt]
L'harmonie \\[0pt]
L'harmonie du monde \\[0pt]
L'héroïsme \\[0pt]
L'hésitation \\[0pt]
L'histoire a-t-elle des lois ? \\[0pt]
L'Histoire a-t-elle un commencement ? \\[0pt]
L'histoire a-t-elle une fin ? \\[0pt]
L'histoire a-t-elle un sens ? \\[0pt]
L'histoire de l'art \\[0pt]
L'histoire des sciences \\[0pt]
L'Histoire des sciences \\[0pt]
L'histoire est-elle la science de ce qui ne se répète jamais ? \\[0pt]
L'histoire est-elle rationnelle ? \\[0pt]
L'histoire est-elle une science ? \\[0pt]
« L'histoire jugera » \\[0pt]
L'histoire jugera \\[0pt]
L'histoire peut-elle se répéter ? \\[0pt]
L'histoire se répète-t-elle ? \\[0pt]
L'homme a-t-il besoin de religion ? \\[0pt]
L'homme est-il fait pour le travail ? \\[0pt]
L'homme est-il la mesure de toute chose ? \\[0pt]
L'homme est-il la mesure de toutes choses ? \\[0pt]
L'homme est-il le seul sujet de droit ? \\[0pt]
L'homme est-il objet de science ? \\[0pt]
L'homme est-il raisonnable par nature ? \\[0pt]
L'homme est-il un animal ? \\[0pt]
L'homme est-il un animal comme les autres ? \\[0pt]
L'homme est-il un loup pour l'homme ? \\[0pt]
« L'homme est la mesure de toute chose » \\[0pt]
« L'homme est la mesure de toutes choses » \\[0pt]
L'homme libre est-il un homme seul ? \\[0pt]
L'homme-machine \\[0pt]
L'homme n'est-il qu'un animal comme les autres ? \\[0pt]
L'honneur \\[0pt]
L'horrible \\[0pt]
L'hospitalité \\[0pt]
L'humanité \\[0pt]
L'humilité \\[0pt]
L'humour \\[0pt]
L'humour et l'ironie \\[0pt]
L'hypocrisie \\[0pt]
L'hypothèse \\[0pt]
Liberté et démocratie \\[0pt]
Liberté et déterminisme \\[0pt]
Liberté et égalité \\[0pt]
Liberté et licence \\[0pt]
Liberté et société \\[0pt]
L'idéal \\[0pt]
L'idée de bonheur \\[0pt]
L'idée de civilisation \\[0pt]
L'idée de création \\[0pt]
L'idée de déterminisme \\[0pt]
L'idée de Dieu \\[0pt]
L'idée de langue universelle \\[0pt]
L'idée de modernité \\[0pt]
L'idée de monde \\[0pt]
L'idée d'encyclopédie \\[0pt]
L'idée de paix \\[0pt]
L'idée de progrès \\[0pt]
L'idée de révolution \\[0pt]
L'idée de système \\[0pt]
L'idée d'évolution \\[0pt]
L'idée d'histoire universelle \\[0pt]
L'idée d'ordre \\[0pt]
L'idée d'origine \\[0pt]
L'identité \\[0pt]
L'identité collective \\[0pt]
L'identité des hommes \\[0pt]
L'identité personnelle \\[0pt]
L'idéologie \\[0pt]
L'idiot \\[0pt]
L'idolâtrie \\[0pt]
L'idole \\[0pt]
Lieu et milieu \\[0pt]
L'ignorance \\[0pt]
L'illusion \\[0pt]
L'illusion de la liberté \\[0pt]
L'image \\[0pt]
L'image du monde \\[0pt]
L'image et le modèle \\[0pt]
L'imaginaire \\[0pt]
L'imagination \\[0pt]
L'imagination a-t-elle des limites ? \\[0pt]
L'imagination en science \\[0pt]
L'imagination est-elle créatrice ? \\[0pt]
L'imagination est-elle libre ? \\[0pt]
L'imagination est-elle maîtresse d'erreur et de fausseté ? \\[0pt]
L'imitation \\[0pt]
L'immortalité \\[0pt]
L'impardonnable \\[0pt]
L'impartialité \\[0pt]
L'impartialité est-elle toujours désirable ? \\[0pt]
L'impassibilité \\[0pt]
L'imperceptible \\[0pt]
L'impersonnel \\[0pt]
L'implicite \\[0pt]
L'impossible \\[0pt]
L'imprévisible \\[0pt]
L'improvisation \\[0pt]
L'imprudence \\[0pt]
L'inaliénable \\[0pt]
L'inattendu \\[0pt]
L'incertain \\[0pt]
L'incertitude \\[0pt]
L'incertitude du passé \\[0pt]
L'incompréhensible \\[0pt]
L'inconcevable \\[0pt]
L'inconcevable et l'impossible \\[0pt]
L'inconnu \\[0pt]
L'inconscience \\[0pt]
L'inconscient \\[0pt]
L'inconscient est-il un destin ? \\[0pt]
L'inconstance \\[0pt]
L'indécence \\[0pt]
L'indécision \\[0pt]
L'indétermination \\[0pt]
L'indéterminé \\[0pt]
L'indice \\[0pt]
L'indicible \\[0pt]
L'indifférence \\[0pt]
L'indignation \\[0pt]
L'individu \\[0pt]
L'individualisme \\[0pt]
L'individualisme est-il un égoïsme ? \\[0pt]
L'individualité \\[0pt]
L'individuation \\[0pt]
L'induction \\[0pt]
L'indulgence \\[0pt]
L'inerte \\[0pt]
L'inespéré \\[0pt]
L'inférence \\[0pt]
L'infidélité \\[0pt]
L'infini \\[0pt]
L'infini se réduit-il à l'indéfini ? \\[0pt]
L'ingénieur \\[0pt]
L'inhumain \\[0pt]
L'inhumanité \\[0pt]
L'injustice est-elle préférable au désordre ? \\[0pt]
L'inné et l'acquis \\[0pt]
L'innocence \\[0pt]
L'innommable \\[0pt]
L'inquiétude \\[0pt]
L'insensé \\[0pt]
L'insensibilité \\[0pt]
L'inspiration \\[0pt]
L'instant \\[0pt]
L'instant de la décision est-il une folie ? \\[0pt]
L'instinct \\[0pt]
L'institution \\[0pt]
L'instruction \\[0pt]
L'insurrection est-elle un droit ? \\[0pt]
L'intelligence \\[0pt]
L'intelligence artificielle \\[0pt]
L'intelligence des bêtes \\[0pt]
L'intention \\[0pt]
L'interdit \\[0pt]
L'intéressant \\[0pt]
L'intérêt \\[0pt]
L'intérêt bien compris \\[0pt]
L'intérêt commun \\[0pt]
L'intérêt de l'art \\[0pt]
L'intérêt gouverne-t-il le monde ? \\[0pt]
L'intérêt peut-il être général ? \\[0pt]
L'intérieur et l'extérieur \\[0pt]
L'intériorité \\[0pt]
L'interprétation \\[0pt]
L'interprétation est-elle un art ? \\[0pt]
L'intime conviction \\[0pt]
L'intolérable \\[0pt]
L'intolérance \\[0pt]
L'intraduisible \\[0pt]
L'introspection est-elle une connaissance ? \\[0pt]
L'intuition \\[0pt]
L'inutile a-t-il de la valeur ? \\[0pt]
L'invention \\[0pt]
L'invention technique \\[0pt]
L'invérifiable \\[0pt]
L'invisible \\[0pt]
L'invivable \\[0pt]
L'involontaire \\[0pt]
L'ironie \\[0pt]
L'irrationalité \\[0pt]
L'irrationnel \\[0pt]
L'irréductible \\[0pt]
L'irréel \\[0pt]
L'irrégularité \\[0pt]
L'irrésolution \\[0pt]
L'irréversible \\[0pt]
Littérature et philosophie \\[0pt]
L'ivresse \\[0pt]
L'objectivité \\[0pt]
L'objectivité scientifique \\[0pt]
L'objet d'amour \\[0pt]
L'objet de l'amour \\[0pt]
L'objet de l'intention \\[0pt]
L'objet du désir \\[0pt]
L'objet technique \\[0pt]
L'obligation \\[0pt]
L'obscurantisme \\[0pt]
L'obscurité \\[0pt]
L'observation \\[0pt]
L'occasion \\[0pt]
L'œil du photographe \\[0pt]
L'œuvre d'art doit-elle être belle ? \\[0pt]
L'œuvre d'art doit-elle nous émouvoir ? \\[0pt]
L'œuvre d'art est-elle une marchandise ? \\[0pt]
Logique et grammaire \\[0pt]
Lois et coutumes \\[0pt]
L'oligarchie \\[0pt]
L'opinion \\[0pt]
L'opinion publique \\[0pt]
L'opportunité \\[0pt]
L'optimisme \\[0pt]
L'oral et l'écrit \\[0pt]
L'ordre \\[0pt]
L'ordre des choses \\[0pt]
L'ordre du monde \\[0pt]
L'ordre et le désordre \\[0pt]
L'ordre international \\[0pt]
L'organisation \\[0pt]
L'orgueil \\[0pt]
L'originalité \\[0pt]
L'origine \\[0pt]
L'origine des langues \\[0pt]
L'origine du langage \\[0pt]
L'oubli \\[0pt]
L'outil \\[0pt]
L'ouvrier et l'ingénieur \\[0pt]
L'un et le multiple \\[0pt]
L'uniformité \\[0pt]
L'unité \\[0pt]
L'unité des sciences \\[0pt]
L'univers \\[0pt]
L'universel \\[0pt]
L'universel et le particulier \\[0pt]
L'urgence \\[0pt]
L'usage des fictions \\[0pt]
L'usure \\[0pt]
L'utile et le bien \\[0pt]
L'utile et l'honnête \\[0pt]
L'utilité de croire \\[0pt]
L'utilité est-elle une valeur morale ? \\[0pt]
L'utilité peut-elle être le principe de la moralité ? \\[0pt]
L'utilité peut-elle être un critère pour juger de la valeur de nos actions ? \\[0pt]
L'utopie \\[0pt]
« Machinalement » \\[0pt]
Ma douleur \\[0pt]
Maître et serviteur \\[0pt]
Maîtriser la technique \\[0pt]
Majorité et minorité \\[0pt]
Malaise dans la civilisation \\[0pt]
Malheur aux vaincus \\[0pt]
Manifester \\[0pt]
Masculin et Féminin \\[0pt]
Masculin, féminin \\[0pt]
Mémoire et identité \\[0pt]
Mémoire et souvenir \\[0pt]
Mentir \\[0pt]
Mesure et démesure \\[0pt]
Mesurer \\[0pt]
Mieux vaut subir que commettre l'injustice \\[0pt]
Mœurs et moralité \\[0pt]
Mon corps \\[0pt]
Mon corps m'appartient-il ? \\[0pt]
Monologue et dialogue \\[0pt]
Mon semblable \\[0pt]
Montrer, est-ce démontrer ? \\[0pt]
Montrer et démontrer \\[0pt]
Montrer l'exemple \\[0pt]
Mourir pour la patrie \\[0pt]
Mythe et connaissance \\[0pt]
Mythe et histoire \\[0pt]
Mythe et vérité \\[0pt]
Nature et culture \\[0pt]
Nature et histoire \\[0pt]
Nature et institution \\[0pt]
N'avons-nous de devoir qu'envers autrui ? \\[0pt]
Nécessité et contingence \\[0pt]
Nécessité et liberté \\[0pt]
Ne pas tuer \\[0pt]
N'existe-t-il que des individus ? \\[0pt]
N'existe-t-il que le présent ? \\[0pt]
N'exprime-t-on que ce dont on a conscience ? \\[0pt]
« Ni Dieu ni maître » \\[0pt]
Nier l'évidence \\[0pt]
Nom propre et nom commun \\[0pt]
« Nos amis les animaux » \\[0pt]
Nos sens nous trompent-ils ? \\[0pt]
Notre existence a-t-elle un sens si l'histoire n'en a pas ? \\[0pt]
Notre ignorance nous excuse-t-elle ? \\[0pt]
Nous et les autres \\[0pt]
« Nul n'est censé ignorer la loi » \\[0pt]
N'y a-t-il d'amitié qu'entre égaux ? \\[0pt]
N'y a-t-il de connaissance que de l'universel ? \\[0pt]
N'y a-t-il de démocratie que représentative ? \\[0pt]
N'y a-t-il de droits que de l'homme ? \\[0pt]
N'y a-t-il de science que du général ? \\[0pt]
N'y a-t-il de science que du mesurable ? \\[0pt]
N'y a-t-il de science qu'exacte ? \\[0pt]
N'y a-t-il des droits que de l'homme ? \\[0pt]
N'y a-t-il de vérité que scientifique ? \\[0pt]
Obéir \\[0pt]
Objet et œuvre \\[0pt]
Observer \\[0pt]
« Œil pour œil, dent pour dent » \\[0pt]
On dit \\[0pt]
« On n'arrête pas le progrès » \\[0pt]
Optimisme et pessimisme \\[0pt]
Ordre et désordre \\[0pt]
Origine et commencement \\[0pt]
Oublier \\[0pt]
Où est mon esprit ? \\[0pt]
Où s'arrête l'espace public ? \\[0pt]
Où suis-je ? \\[0pt]
« Par hasard » \\[0pt]
Parler et agir \\[0pt]
Parler pour ne rien dire \\[0pt]
Parler pour quelqu'un \\[0pt]
Parler tout seul \\[0pt]
Paroles et actes \\[0pt]
Partager \\[0pt]
« Pas de liberté pour les ennemis de la liberté » \\[0pt]
Passer le temps \\[0pt]
Passions, intérêt, raison \\[0pt]
Peindre, est-ce nécessairement feindre ? \\[0pt]
Pensée et calcul \\[0pt]
Penser à rien \\[0pt]
« Penser, c'est dire non » \\[0pt]
Penser est-il assimilable à un travail ? \\[0pt]
Penser et calculer \\[0pt]
Penser le changement \\[0pt]
Penser le mouvement \\[0pt]
Penser le rien, est-ce ne rien penser ? \\[0pt]
Penser par soi-même \\[0pt]
Pensez-vous que vous avez une âme ? \\[0pt]
Perception et connaissance \\[0pt]
Percevoir, est-ce savoir ? \\[0pt]
Percevoir et concevoir \\[0pt]
Percevons-nous les choses telles qu'elles sont ? \\[0pt]
Perdre la mémoire \\[0pt]
Perdre la raison \\[0pt]
Perdre le contrôle \\[0pt]
Perdre ses illusions \\[0pt]
Perdre son temps \\[0pt]
Personne et individu \\[0pt]
Persuader et convaincre \\[0pt]
Peut-il être préférable de ne pas savoir ? \\[0pt]
Peut-il exister une action désintéressée ? \\[0pt]
Peut-il y avoir conflit entre nos devoirs ? \\[0pt]
Peut-il y avoir des expériences métaphysiques ? \\[0pt]
Peut-il y avoir un art conceptuel ? \\[0pt]
Peut-il y avoir un droit de la guerre ? \\[0pt]
Peut-il y avoir un intérêt collectif ? \\[0pt]
Peut-on agir machinalement ? \\[0pt]
Peut-on agir sans raison ? \\[0pt]
Peut-on aimer la vie plus que tout ? \\[0pt]
Peut-on aimer son prochain comme soi-même ? \\[0pt]
Peut-on apprendre à vivre ? \\[0pt]
Peut-on argumenter en morale ? \\[0pt]
Peut-on avoir peur de soi-même ? \\[0pt]
Peut-on avoir peur d'être libre ? \\[0pt]
Peut-on avoir raison contre tous ? \\[0pt]
Peut-on avoir raison tout seul ? \\[0pt]
Peut-on choisir le mal ? \\[0pt]
Peut-on choisir sa vie ? \\[0pt]
Peut-on classer les arts ? \\[0pt]
Peut-on comprendre ce qui est illogique ? \\[0pt]
Peut-on concevoir une société qui n'aurait plus besoin du droit ? \\[0pt]
Peut-on connaître autrui ? \\[0pt]
Peut-on connaître le singulier ? \\[0pt]
Peut-on créer un homme nouveau ? \\[0pt]
Peut-on critiquer la religion ? \\[0pt]
Peut-on croire ce qu'on veut ? \\[0pt]
Peut-on décider de croire ? \\[0pt]
Peut-on décider de tout ? \\[0pt]
Peut-on délimiter l'humain ? \\[0pt]
Peut-on dialoguer avec un ordinateur ? \\[0pt]
Peut-on dire ce qui n'est pas ? \\[0pt]
Peut-on dire d'une œuvre d'art qu'elle est ratée ? \\[0pt]
Peut-on dire que tout est relatif ? \\[0pt]
Peut-on donner un sens à l'existence ? \\[0pt]
Peut-on douter de soi ? \\[0pt]
Peut-on douter de tout ? \\[0pt]
Peut-on échapper au temps ? \\[0pt]
Peut-on écrire comme on parle ? \\[0pt]
Peut-on éduquer le goût ? \\[0pt]
Peut-on éduquer quelqu'un à être libre ? \\[0pt]
Peut-on en savoir trop ? \\[0pt]
Peut-on être citoyen du monde ? \\[0pt]
Peut-on être complètement athée ? \\[0pt]
Peut-on être en conflit avec soi-même ? \\[0pt]
Peut-on être étranger au monde ? \\[0pt]
Peut-on être heureux dans la solitude ? \\[0pt]
Peut-on être heureux tout seul ? \\[0pt]
Peut-on être homme sans être citoyen ? \\[0pt]
Peut-on être impartial ? \\[0pt]
Peut-on être juste dans une situation injuste ? \\[0pt]
Peut-on être juste dans une société injuste ? \\[0pt]
Peut-on être maître de soi ? \\[0pt]
Peut-on être sceptique ? \\[0pt]
Peut-on être sceptique de bonne foi ? \\[0pt]
Peut-on être seul avec soi-même ? \\[0pt]
Peut-on faire le bien de quelqu'un malgré lui ? \\[0pt]
Peut-on faire le bonheur d'autrui ? \\[0pt]
Peut-on forcer quelqu'un à être libre ? \\[0pt]
Peut-on fuir hors du monde ? \\[0pt]
Peut-on haïr la raison ? \\[0pt]
Peut-on imposer la liberté ? \\[0pt]
Peut-on justifier la guerre ? \\[0pt]
Peut-on justifier ses choix ? \\[0pt]
Peut-on maîtriser la technique ? \\[0pt]
Peut-on maîtriser le risque ? \\[0pt]
Peut-on maîtriser le temps ? \\[0pt]
Peut-on montrer en cachant ? \\[0pt]
Peut-on ne croire en rien ? \\[0pt]
Peut-on ne pas être de son temps ? \\[0pt]
Peut-on ne pas être soi-même ? \\[0pt]
Peut-on ne pas savoir ce que l'on fait ? \\[0pt]
Peut-on ne pas savoir ce que l'on veut ? \\[0pt]
Peut-on ne pas savoir ce qu'on veut ? \\[0pt]
Peut-on ne pas vouloir être heureux ? \\[0pt]
Peut-on ne penser à rien ? \\[0pt]
Peut-on ne vivre qu'au présent ? \\[0pt]
Peut-on parler de ce qui n'existe pas ? \\[0pt]
Peut-on parler de progrès en art ? \\[0pt]
Peut-on parler d'un droit de la guerre ? \\[0pt]
Peut-on parler d'une santé de l'âme ? \\[0pt]
Peut-on parler d'un règne de la technique ? \\[0pt]
Peut-on parler d'un savoir poétique ? \\[0pt]
Peut-on parler pour ne rien dire ? \\[0pt]
Peut-on penser l'histoire comme progrès ? \\[0pt]
Peut-on penser l'impossible ? \\[0pt]
Peut-on penser sans concept ? \\[0pt]
Peut-on penser sans ordre ? \\[0pt]
Peut-on penser sans préjugé ? \\[0pt]
Peut-on penser sans savoir ce que l'on pense ? \\[0pt]
Peut-on penser sans savoir que l'on pense ? \\[0pt]
Peut-on penser sans signes ? \\[0pt]
Peut-on penser un droit international ? \\[0pt]
Peut-on perdre son identité ? \\[0pt]
Peut-on perdre son temps ? \\[0pt]
Peut-on prévoir l'avenir ? \\[0pt]
Peut-on prévoir le futur ? \\[0pt]
Peut-on prouver l'existence ? \\[0pt]
Peut-on prouver l'existence de Dieu ? \\[0pt]
Peut-on prouver l'existence de l'inconscient ? \\[0pt]
Peut-on renoncer à comprendre ? \\[0pt]
Peut-on renoncer à sa liberté ? \\[0pt]
Peut-on renoncer à soi ? \\[0pt]
Peut-on renoncer au bonheur ? \\[0pt]
Peut-on représenter l'invisible ? \\[0pt]
Peut-on reprocher au langage d'être équivoque ? \\[0pt]
Peut-on rester sceptique ? \\[0pt]
Peut-on rire de tout ? \\[0pt]
Peut-on savoir ce que l'on fait ? \\[0pt]
Peut-on se choisir un destin ? \\[0pt]
Peut-on se duper soi-même ? \\[0pt]
Peut-on se fier à l'expérience vécue ? \\[0pt]
Peut-on se fier à son intuition ? \\[0pt]
Peut-on se fier aux apparences ? \\[0pt]
Peut-on se mentir à soi-même ? \\[0pt]
Peut-on se mettre à la place d'autrui ? \\[0pt]
Peut-on se mettre à la place des autres ? \\[0pt]
Peut-on s'en tenir au présent ? \\[0pt]
Peut-on séparer l'homme et l'œuvre ? \\[0pt]
Peut-on se passer de croyance ? \\[0pt]
Peut-on se passer de croyances ? \\[0pt]
Peut-on se passer de frontières ? \\[0pt]
Peut-on se passer de méthode ? \\[0pt]
Peut-on se passer de principes ? \\[0pt]
Peut-on se passer de techniques de raisonnement ? \\[0pt]
Peut-on se promettre quelque chose à soi-même ? \\[0pt]
Peut-on se retirer du monde ? \\[0pt]
Peut-on sortir de la subjectivité ? \\[0pt]
Peut-on sortir de sa conscience ? \\[0pt]
Peut-on sortir de sa culture ? \\[0pt]
Peut-on suivre une règle ? \\[0pt]
Peut-on tout définir ? \\[0pt]
Peut-on tout démontrer ? \\[0pt]
Peut-on tout désirer ? \\[0pt]
Peut-on tout dire ? \\[0pt]
Peut-on tout expliquer ? \\[0pt]
Peut-on tout exprimer ? \\[0pt]
Peut-on tout interpréter ? \\[0pt]
Peut-on tout mathématiser ? \\[0pt]
Peut-on tout mesurer ? \\[0pt]
Peut-on tout pardonner ? \\[0pt]
Peut-on tout partager ? \\[0pt]
Peut-on tout prévoir ? \\[0pt]
Peut-on tout prouver ? \\[0pt]
Peut-on tout soumettre à la discussion ? \\[0pt]
Peut-on traiter autrui comme un moyen ? \\[0pt]
Peut-on vivre dans le doute ? \\[0pt]
Peut-on vivre sans aimer ? \\[0pt]
Peut-on vivre sans croyances ? \\[0pt]
Peut-on vivre sans l'art ? \\[0pt]
Peut-on vivre sans principes ? \\[0pt]
Peut-on voir sans croire ? \\[0pt]
Peut-on vouloir le mal ? \\[0pt]
Peut-on vouloir le mal pour le mal ? \\[0pt]
Peut-on vouloir l'impossible ? \\[0pt]
Philosophie et système \\[0pt]
Plaisir et douleur \\[0pt]
Plaisirs, honneurs, richesses \\[0pt]
Pluralité et unité \\[0pt]
« Plus vite, plus haut, plus fort » \\[0pt]
Poésie et philosophie \\[0pt]
Pour qui se prend-on ? \\[0pt]
Pourquoi ? \\[0pt]
Pourquoi communiquer ? \\[0pt]
Pourquoi critiquer le conformisme ? \\[0pt]
Pourquoi définir ? \\[0pt]
Pourquoi délibérer ? \\[0pt]
Pourquoi des cérémonies ? \\[0pt]
Pourquoi des châtiments ? \\[0pt]
Pourquoi des fictions ? \\[0pt]
Pourquoi des interdits ? \\[0pt]
Pourquoi désirer la sagesse ? \\[0pt]
Pourquoi des métaphores ? \\[0pt]
Pourquoi des musées ? \\[0pt]
Pourquoi des poètes ? \\[0pt]
Pourquoi des religions ? \\[0pt]
Pourquoi des rites ? \\[0pt]
Pourquoi des traditions ? \\[0pt]
Pourquoi domestiquer ? \\[0pt]
Pourquoi écrire ? \\[0pt]
Pourquoi être raisonnable ? \\[0pt]
Pourquoi faut-il être poli ? \\[0pt]
Pourquoi la critique ? \\[0pt]
Pourquoi la curiosité est-elle un vilain défaut ? \\[0pt]
Pourquoi la guerre ? \\[0pt]
Pourquoi la prison ? \\[0pt]
Pourquoi la prohibition de l'inceste ? \\[0pt]
Pourquoi les droits de l'homme sont-ils universels ? \\[0pt]
Pourquoi le sport ? \\[0pt]
Pourquoi l'État ? \\[0pt]
Pourquoi l'homme a-t-il des droits ? \\[0pt]
Pourquoi lit-on des romans ? \\[0pt]
Pourquoi mentir ? \\[0pt]
Pourquoi ne s'entend-on pas sur la nature de ce qui est réel ? \\[0pt]
Pourquoi nous racontons-nous des histoires ? \\[0pt]
Pourquoi nous soucier du sort des générations futures ? \\[0pt]
Pourquoi nous trompons-nous ? \\[0pt]
Pourquoi obéit-on ? \\[0pt]
Pourquoi parler de fautes de goût ? \\[0pt]
Pourquoi parle-t-on d'une « société civile » ? \\[0pt]
Pourquoi pas ? \\[0pt]
Pourquoi pas plusieurs dieux ? \\[0pt]
Pourquoi philosopher ? \\[0pt]
Pourquoi pleure-t-on ? \\[0pt]
Pourquoi pleure-t-on au cinéma ? \\[0pt]
Pourquoi préférer l'original ? \\[0pt]
Pourquoi préférer l'original à la copie ? \\[0pt]
Pourquoi préserver l'environnement ? \\[0pt]
Pourquoi prier ? \\[0pt]
Pourquoi prouver l'existence de Dieu ? \\[0pt]
Pourquoi punir ? \\[0pt]
Pourquoi raconter des histoires ? \\[0pt]
Pourquoi respecter la nature ? \\[0pt]
Pourquoi respecter les anciens ? \\[0pt]
Pourquoi sauver les apparences ? \\[0pt]
Pourquoi se confesser ? \\[0pt]
Pourquoi se divertir ? \\[0pt]
Pourquoi se fier à autrui ? \\[0pt]
Pourquoi se révolter ? \\[0pt]
Pourquoi se soucier du futur ? \\[0pt]
Pourquoi s'étonner ? \\[0pt]
Pourquoi s'intéresser à l'origine ? \\[0pt]
Pourquoi soigner son apparence ? \\[0pt]
Pourquoi suivre l'actualité ? \\[0pt]
Pourquoi tenir ses promesses ? \\[0pt]
Pourquoi travailler ? \\[0pt]
Pourquoi travaille-t-on ? \\[0pt]
Pourquoi veut-on changer le monde ? \\[0pt]
Pourquoi veut-on la vérité ? \\[0pt]
Pourquoi vivons-nous ? \\[0pt]
Pourquoi vouloir avoir raison ? \\[0pt]
Pourquoi voulons-nous savoir ? \\[0pt]
Pourquoi voyager ? \\[0pt]
Pourquoi y a-t-il des conflits insolubles ? \\[0pt]
Pourquoi y a-t-il des lois ? \\[0pt]
Pourquoi y a-t-il du mal dans le monde ? \\[0pt]
Pourquoi y a-t-il plusieurs façons de démontrer ? \\[0pt]
Pourquoi y a-t-il plusieurs langues ? \\[0pt]
Pourquoi y a-t-il plusieurs philosophies ? \\[0pt]
Pourquoi y a-t-il quelque chose plutôt que rien ? \\[0pt]
Pourrions-nous comprendre une pensée non humaine ? \\[0pt]
Pouvoir et savoir \\[0pt]
Pouvons-nous être objectifs ? \\[0pt]
Pouvons-nous justifier nos croyances ? \\[0pt]
Prendre connaissance \\[0pt]
Prendre la parole \\[0pt]
Prendre le pouvoir \\[0pt]
Prendre ses désirs pour des réalités \\[0pt]
Prendre son temps \\[0pt]
Prévoir \\[0pt]
Principe et fondement \\[0pt]
Privation et négation \\[0pt]
Produire et créer \\[0pt]
Promettre \\[0pt]
Prouver \\[0pt]
Prouver Dieu \\[0pt]
Prouver la force d'âme \\[0pt]
Prouver l'existence de Dieu \\[0pt]
Prouver l'existence du monde extérieur \\[0pt]
Puis-je décider de croire ? \\[0pt]
Puis-je dire « ceci est mon corps » ? \\[0pt]
Puis-je être heureux dans un monde chaotique ? \\[0pt]
Puis-je faire ce que je veux de mon corps ? \\[0pt]
Puis-je ne croire que ce que je vois ? \\[0pt]
Punir ou soigner ? \\[0pt]
Qu'aime-t-on ? \\[0pt]
Qu'aime-t-on quand on aime ? \\[0pt]
Qualité et quantité \\[0pt]
Quand faut-il désobéir ? \\[0pt]
Quand y a-t-il de l'art ? \\[0pt]
Quand y a-t-il peuple ? \\[0pt]
Qu'anticipent les romans d'anticipation ? \\[0pt]
Quantité et qualité \\[0pt]
Qu'apprend-on dans les livres ? \\[0pt]
Qu'avons-nous en commun ? \\[0pt]
Que choisir ? \\[0pt]
Que doit-on désirer pour ne pas être déçu ? \\[0pt]
Que faire ? \\[0pt]
Que faire de la violence ? \\[0pt]
Que faire de nos émotions ? \\[0pt]
Que faut-il craindre ? \\[0pt]
Que la nature soit explicable, est-ce explicable ? \\[0pt]
Quel contrôle a-t-on sur son corps ? \\[0pt]
Quel est le bon nombre d'amis ? \\[0pt]
Quel est le but de la politique ? \\[0pt]
Quel est le fondement de la propriété ? \\[0pt]
Quel est le fondement de l'autorité ? \\[0pt]
Quel est le sujet de la pensée ? \\[0pt]
Quel est l'objet de la métaphysique ? \\[0pt]
Quel est l'objet de l'amour ? \\[0pt]
Quel est l'objet de l'histoire ? \\[0pt]
Quel est l'objet du désir ? \\[0pt]
Quelle est la fin de l'État ? \\[0pt]
Quelle est la portée d'un exemple ? \\[0pt]
Quelle est la valeur de l'expérience ? \\[0pt]
Quelle est la valeur des hypothèses ? \\[0pt]
Quelle est la valeur d'une œuvre d'art ? \\[0pt]
Quelle est la valeur du témoignage ? \\[0pt]
Quelle peut être la force de nos idées ? \\[0pt]
Quelle réalité peut-on accorder au temps ? \\[0pt]
Quelles sont les limites de la souveraineté ? \\[0pt]
Quelle valeur accorder à l'expérience ? \\[0pt]
Quelle valeur devons accorder à l'expérience ? \\[0pt]
Quelle valeur devons-nous accorder à l'expérience ? \\[0pt]
Quelle valeur devons-nous accorder à l'intuition ? \\[0pt]
Quelle valeur peut-on accorder à l'expérience ? \\[0pt]
« Quelle vanité que la peinture » \\[0pt]
Quels désirs dois-je m'interdire ? \\[0pt]
Quels sont les fondements de l'autorité ? \\[0pt]
Quel usage peut-on faire des fictions ? \\[0pt]
Que manque-t-il aux machines pour être des organismes ? \\[0pt]
Que nous apprend la diversité des langues ? \\[0pt]
Que nous apprend le cinéma ? \\[0pt]
Que nous apprend le faux ? \\[0pt]
Que nous apprend l'expérience ? \\[0pt]
Que nous apprennent les controverses scientifiques ? \\[0pt]
Que nous apprennent les illusions d'optique ? \\[0pt]
Que nous apprennent les mythes ? \\[0pt]
Que nous enseignent les œuvres d'art ? \\[0pt]
Que nous montre l'œuvre d'art ? \\[0pt]
« Que nul n'entre ici s'il n'est géomètre » \\[0pt]
Que nul n'entre ici s'il n'est géomètre \\[0pt]
Que partage-t-on avec les animaux ? \\[0pt]
Que peint le peintre ? \\[0pt]
Que perdrait la pensée en perdant l'écriture ? \\[0pt]
Que peut la pensée ? \\[0pt]
Que peut la philosophie ? \\[0pt]
Que peut la science ? \\[0pt]
Que peut la théorie ? \\[0pt]
Que peut l'esprit ? \\[0pt]
Que peut-on comprendre immédiatement ? \\[0pt]
Que peut-on échanger ? \\[0pt]
Que peut-on interdire ? \\[0pt]
Que peut-on sur autrui ? \\[0pt]
Que peut-on voir ? \\[0pt]
Que peut un corps ? \\[0pt]
Que pouvons-nous connaître ? \\[0pt]
Que prouvent les preuves de l'existence de Dieu ? \\[0pt]
Que recherche l'artiste ? \\[0pt]
Que répondre au sceptique ? \\[0pt]
Que reste-t-il du passé ? \\[0pt]
Que sais-je d'autrui ? \\[0pt]
Que serait la vie sans l'art ? \\[0pt]
Que signifie connaître ? \\[0pt]
Que signifie être mortel ? \\[0pt]
Que signifie la mort ? \\[0pt]
Que signifient les mots ? \\[0pt]
Que signifie pour l'homme être mortel ? \\[0pt]
Qu'est-ce qu'agir ensemble ? \\[0pt]
Qu'est-ce qu'aimer une œuvre d'art ? \\[0pt]
Qu'est-ce qu'apprendre ? \\[0pt]
Qu'est-ce qu'argumenter ? \\[0pt]
Qu'est-ce qu'avoir conscience de soi \\[0pt]
Qu'est-ce qu'avoir de l'expérience ? \\[0pt]
Qu'est-ce qu'avoir un droit ? \\[0pt]
Qu'est-ce que catégoriser ? \\[0pt]
Qu'est-ce que comprendre ? \\[0pt]
Qu'est-ce que créer ? \\[0pt]
Qu'est-ce que croire ? \\[0pt]
Qu'est-ce que décider ? \\[0pt]
Qu'est-ce que définir ? \\[0pt]
Qu'est-ce que démontrer ? \\[0pt]
Qu'est-ce que déraisonner ? \\[0pt]
Qu'est-ce qu'éduquer ? \\[0pt]
Qu'est-ce que faire autorité ? \\[0pt]
Qu'est-ce que faire preuve d'humanité ? \\[0pt]
Qu'est-ce que guérir ? \\[0pt]
Qu'est-ce que jouer ? \\[0pt]
Qu'est-ce que juger ? \\[0pt]
Qu'est-ce que la barbarie ? \\[0pt]
Qu'est-ce que la critique ? \\[0pt]
Qu'est-ce que la démocratie ? \\[0pt]
Qu'est-ce que la folie ? \\[0pt]
Qu'est-ce que la normalité ? \\[0pt]
Qu'est-ce que la perception ? \\[0pt]
Qu'est-ce que la raison d'État ? \\[0pt]
Qu'est-ce que la réalité ? \\[0pt]
Qu'est-ce que la souveraineté ? \\[0pt]
Qu'est-ce que la tragédie ? \\[0pt]
Qu'est-ce que la valeur marchande ? \\[0pt]
Qu'est-ce que la vérité ? \\[0pt]
Qu'est-ce que la vie bonne ? \\[0pt]
Qu'est-ce que le bonheur ? \\[0pt]
Qu'est-ce que le cinéma donne à voir ? \\[0pt]
Qu'est-ce que le courage ? \\[0pt]
Qu'est-ce que le hasard ? \\[0pt]
Qu'est-ce que le mauvais goût ? \\[0pt]
Qu'est-ce que le moi ? \\[0pt]
Qu'est-ce que l'enfance ? \\[0pt]
Qu'est-ce que le sens pratique ? \\[0pt]
Qu'est-ce que le sublime ? \\[0pt]
Qu'est-ce que le travail ? \\[0pt]
Qu'est-ce que l'indifférence ? \\[0pt]
Qu'est-ce que l'intuition ? \\[0pt]
Qu'est-ce que lire ? \\[0pt]
Qu'est-ce que l'ordinaire ? \\[0pt]
Qu'est-ce que maîtriser une technique ? \\[0pt]
Qu'est-ce que mourir ? \\[0pt]
Qu'est-ce qu'enseigner ? \\[0pt]
Qu'est-ce que parler veut dire ? \\[0pt]
Qu'est-ce que percevoir ? \\[0pt]
Qu'est-ce que perdre la raison ? \\[0pt]
Qu'est-ce que perdre son temps ? \\[0pt]
Qu'est-ce que prendre conscience ? \\[0pt]
Qu'est-ce que promettre ? \\[0pt]
Qu'est-ce que réfuter une philosophie ? \\[0pt]
Qu'est-ce que « se rendre maître et possesseur de la nature » ? \\[0pt]
Qu'est-ce que s'orienter ? \\[0pt]
Qu'est-ce que témoigner ? \\[0pt]
Qu'est-ce que traduire ? \\[0pt]
Qu'est-ce que travailler ? \\[0pt]
Qu'est-ce qu'être ? \\[0pt]
Qu'est-ce qu'être adulte ? \\[0pt]
Qu'est-ce qu'être barbare ? \\[0pt]
Qu'est-ce qu'être cohérent ? \\[0pt]
Qu'est-ce qu'être cultivé ? \\[0pt]
Qu'est-ce qu'être de son temps ? \\[0pt]
Qu'est-ce qu'être efficace en politique ? \\[0pt]
Qu'est-ce qu'être fidèle à soi-même ? \\[0pt]
Qu'est-ce qu'être généreux ? \\[0pt]
Qu'est-ce qu'être hors de soi ? \\[0pt]
Qu'est-ce qu'être idéaliste ? \\[0pt]
Qu'est-ce qu'être libre ? \\[0pt]
Qu'est-ce qu'être maître de soi-même ? \\[0pt]
Qu'est-ce qu'être malade ? \\[0pt]
Qu'est-ce qu'être moderne ? \\[0pt]
Qu'est-ce qu'être nihiliste ? \\[0pt]
Qu'est-ce qu'être normal ? \\[0pt]
Qu'est-ce qu'être psychologue ? \\[0pt]
Qu'est-ce qu'être rationnel ? \\[0pt]
Qu'est-ce qu'être réaliste ? \\[0pt]
Qu'est-ce qu'être sceptique ? \\[0pt]
Qu'est-ce qu'être simple ? \\[0pt]
Qu'est-ce qu'être soi-même ? \\[0pt]
Qu'est-ce qu'être témoin ? \\[0pt]
Qu'est-ce qu'être un bon citoyen ? \\[0pt]
Qu'est-ce qu'expliquer ? \\[0pt]
Qu'est-ce qui anime l'animal ? \\[0pt]
Qu'est-ce qui a un sens ? \\[0pt]
Qu'est-ce qui dépend de nous ? \\[0pt]
Qu'est-ce qui est absurde ? \\[0pt]
Qu'est-ce qui est actuel ? \\[0pt]
Qu'est-ce qui est culturel ? \\[0pt]
Qu'est-ce qui est donné ? \\[0pt]
Qu'est-ce qui est étonnant ? \\[0pt]
Qu'est-ce qui est extérieur à ma conscience, ? \\[0pt]
Qu'est-ce qui est hors la loi ? \\[0pt]
Qu'est-ce qui est immoral ? \\[0pt]
Qu'est-ce qui est mauvais dans l'égoïsme ? \\[0pt]
Qu'est-ce qui est prométhéen ? \\[0pt]
Qu'est-ce qui est réel ? \\[0pt]
Qu'est-ce qui est sacré ? \\[0pt]
Qu'est-ce qui est sauvage ? \\[0pt]
Qu'est-ce qui est scientifique ? \\[0pt]
Qu'est-ce qui est tragique ? \\[0pt]
Qu'est-ce qui est vital ? \\[0pt]
Qu'est-ce qui fait la force de la loi ? \\[0pt]
Qu'est-ce qui fait la valeur d'une croyance ? \\[0pt]
Qu'est-ce qui fait la valeur d'une œuvre ? \\[0pt]
Qu'est-ce qui fait mon identité ? \\[0pt]
Qu'est-ce qui fait un peuple ? \\[0pt]
Qu'est-ce qui innocente le bourreau ? \\[0pt]
Qu'est-ce qui justifie une croyance ? \\[0pt]
Qu'est-ce qu'imaginer ? \\[0pt]
Qu'est-ce qui n'est pas politique ? \\[0pt]
Qu'est-ce qui n'existe pas ? \\[0pt]
Qu'est-ce qui nous fait danser ? \\[0pt]
Qu'est-ce qu'interpréter ? \\[0pt]
Qu'est-ce qui plaît dans la musique ? \\[0pt]
Qu'est-ce qu'on attend ? \\[0pt]
Qu'est-ce qu'un abus de langage ? \\[0pt]
Qu'est-ce qu'un accident ? \\[0pt]
Qu'est-ce qu'un acte ? \\[0pt]
Qu'est-ce qu'un acteur ? \\[0pt]
Qu'est-ce qu'un ami ? \\[0pt]
Qu'est-ce qu'un animal ? \\[0pt]
Qu'est-ce qu'un art de vivre ? \\[0pt]
Qu'est-ce qu'un artiste ? \\[0pt]
Qu'est-ce qu'un art populaire ? \\[0pt]
Qu'est-ce qu'un auteur ? \\[0pt]
Qu'est-ce qu'un bon gouvernement ? \\[0pt]
Qu'est-ce qu'un bon jugement ? \\[0pt]
Qu'est-ce qu'un caractère ? \\[0pt]
Qu'est-ce qu'un cas de conscience ? \\[0pt]
Qu'est-ce qu'un châtiment ? \\[0pt]
Qu'est-ce qu'un chef ? \\[0pt]
Qu'est-ce qu'un chef-d'œuvre ? \\[0pt]
Qu'est-ce qu'un citoyen ? \\[0pt]
Qu'est-ce qu'un classique ? \\[0pt]
Qu'est-ce qu'un code ? \\[0pt]
Qu'est-ce qu'un concept ? \\[0pt]
Qu'est-ce qu'un conflit de générations ? \\[0pt]
Qu'est-ce qu'un contrat ? \\[0pt]
Qu'est-ce qu'un corps ? \\[0pt]
Qu'est-ce qu'un créateur ? \\[0pt]
Qu'est-ce qu'un crime ? \\[0pt]
Qu'est-ce qu'un crime contre l'humanité ? \\[0pt]
Qu'est-ce qu'un critère de vérité ? \\[0pt]
Qu'est-ce qu'un détail ? \\[0pt]
Qu'est-ce qu'un dialogue ? \\[0pt]
Qu'est-ce qu'un dilemme ? \\[0pt]
Qu'est-ce qu'une action réussie ? \\[0pt]
Qu'est-ce qu'une analyse ? \\[0pt]
Qu'est-ce qu'une anomalie ? \\[0pt]
Qu'est-ce qu'une aporie ? \\[0pt]
Qu'est-ce qu'une avant-garde ? \\[0pt]
Qu'est-ce qu'une belle mort ? \\[0pt]
Qu'est-ce qu'une bête ? \\[0pt]
Qu'est-ce qu'une bonne définition ? \\[0pt]
Qu'est-ce qu'une bonne délibération ? \\[0pt]
Qu'est-ce qu'une bonne éducation ? \\[0pt]
Qu'est-ce qu'une bonne traduction ? \\[0pt]
Qu'est-ce qu'une catastrophe ? \\[0pt]
Qu'est-ce qu'une cause ? \\[0pt]
Qu'est-ce qu'une chose ? \\[0pt]
Qu'est-ce qu'une civilisation ? \\[0pt]
Qu'est-ce qu'une comédie ? \\[0pt]
Qu'est-ce qu'une communauté scientifique ? \\[0pt]
Qu'est-ce qu'une condition suffisante ? \\[0pt]
Qu'est-ce qu'une constitution ? \\[0pt]
Qu'est-ce qu'une contradiction ? \\[0pt]
Qu'est-ce qu'une contrainte ? \\[0pt]
Qu'est-ce qu'une convention ? \\[0pt]
Qu'est-ce qu'une conviction ? \\[0pt]
Qu'est-ce qu'une crise ? \\[0pt]
Qu'est-ce qu'une croyance ? \\[0pt]
Qu'est-ce qu'une croyance rationnelle ? \\[0pt]
Qu'est-ce qu'une décision politique ? \\[0pt]
Qu'est-ce qu'une décision rationnelle ? \\[0pt]
Qu'est-ce qu'une définition ? \\[0pt]
Qu'est-ce qu'une démocratie ? \\[0pt]
Qu'est-ce qu'une démonstration ? \\[0pt]
Qu'est-ce qu'une exception ? \\[0pt]
Qu'est-ce qu'une expérience ? \\[0pt]
Qu'est-ce qu'une expérience cruciale ? \\[0pt]
Qu'est-ce qu'une « expérience de pensée » ? \\[0pt]
Qu'est-ce qu'une expérience de pensée ? \\[0pt]
Qu'est-ce qu'une expérience scientifique ? \\[0pt]
Qu'est-ce qu'une famille ? \\[0pt]
Qu'est-ce qu'une fausse science ? \\[0pt]
Qu'est-ce qu'une faute de goût ? \\[0pt]
Qu'est-ce qu'une fiction ? \\[0pt]
Qu'est-ce qu'une fonction ? \\[0pt]
Qu'est-ce qu'une hypothèse ? \\[0pt]
Qu'est-ce qu'une hypothèse scientifique ? \\[0pt]
Qu'est-ce qu'une idée ? \\[0pt]
Qu'est-ce qu'une idéologie ? \\[0pt]
Qu'est-ce qu'une illusion ? \\[0pt]
Qu'est-ce qu'une image ? \\[0pt]
Qu'est-ce qu'une inégalité ? \\[0pt]
Qu'est-ce qu'une injustice ? \\[0pt]
Qu'est-ce qu'une institution ? \\[0pt]
Qu'est-ce qu'une interprétation ? \\[0pt]
Qu'est-ce qu'une invention technique ? \\[0pt]
Qu'est-ce qu'une langue ? \\[0pt]
Qu'est-ce qu'une loi ? \\[0pt]
Qu'est-ce qu'une loi de la nature ? \\[0pt]
Qu'est-ce qu'une loi scientifique ? \\[0pt]
Qu'est-ce qu'une machine ? \\[0pt]
Qu'est-ce qu'une maladie ? \\[0pt]
Qu'est-ce qu'une métaphore ? \\[0pt]
Qu'est-ce qu'une morale de la communication ? \\[0pt]
Qu'est-ce qu'une nation ? \\[0pt]
Qu'est-ce qu'un enfant ? \\[0pt]
Qu'est-ce qu'un ennemi ? \\[0pt]
Qu'est-ce qu'une norme ? \\[0pt]
Qu'est-ce qu'une nouveauté ? \\[0pt]
Qu'est-ce qu'une œuvre ? \\[0pt]
Qu'est-ce qu'une œuvre d'art ? \\[0pt]
Qu'est-ce qu'une parole libre ? \\[0pt]
Qu'est-ce qu'une passion ? \\[0pt]
Qu'est-ce qu'une pensée libre ? \\[0pt]
Qu'est-ce qu'une personne ? \\[0pt]
Qu'est-ce qu'une pétition de principe ? \\[0pt]
Qu'est-ce qu'une preuve ? \\[0pt]
Qu'est-ce qu'une promesse ? \\[0pt]
Qu'est-ce qu'une propriété ? \\[0pt]
Qu'est-ce qu'une question ? \\[0pt]
Qu'est-ce qu'une raison d'agir ? \\[0pt]
Qu'est-ce qu'une règle ? \\[0pt]
Qu'est-ce qu'une relation ? \\[0pt]
Qu'est-ce qu'une représentation ? \\[0pt]
Qu'est-ce qu'une représentation du monde ? \\[0pt]
Qu'est-ce qu'une révolution ? \\[0pt]
Qu'est-ce qu'une révolution scientifique ? \\[0pt]
Qu'est-ce qu'une science humaine ? \\[0pt]
Qu'est-ce qu'un esclave ? \\[0pt]
Qu'est-ce qu'une société juste ? \\[0pt]
Qu'est-ce qu'un esprit libre ? \\[0pt]
Qu'est-ce qu'un esprit profond ? \\[0pt]
Qu'est-ce qu'une structure ? \\[0pt]
Qu'est-ce qu'une théorie ? \\[0pt]
Qu'est-ce qu'une théorie scientifique ? \\[0pt]
Qu'est-ce qu'une tradition ? \\[0pt]
Qu'est-ce qu'une tragédie ? \\[0pt]
Qu'est-ce qu'un « être dégénéré » ? \\[0pt]
Qu'est-ce qu'un être vivant ? \\[0pt]
Qu'est-ce qu'un événement ? \\[0pt]
Qu'est-ce qu'une vertu ? \\[0pt]
Qu'est-ce qu'une ville ? \\[0pt]
Qu'est-ce qu'une vision du monde ? \\[0pt]
Qu'est-ce qu'un exemple ? \\[0pt]
Qu'est-ce qu'un expert ? \\[0pt]
Qu'est-ce qu'un fait ? \\[0pt]
Qu'est-ce qu'un fait divers ? \\[0pt]
Qu'est-ce qu'un fait historique ? \\[0pt]
Qu'est-ce qu'un fait social ? \\[0pt]
Qu'est-ce qu'un faux ? \\[0pt]
Qu'est-ce qu'un faux problème ? \\[0pt]
Qu'est-ce qu'un génie ? \\[0pt]
Qu'est-ce qu'un grand homme ? \\[0pt]
Qu'est-ce qu'un grand homme ou une grande femme ? \\[0pt]
Qu'est-ce qu'un grand philosophe ? \\[0pt]
Qu'est-ce qu'un héros ? \\[0pt]
Qu'est-ce qu'un homme libre ? \\[0pt]
Qu'est-ce qu'un homme normal ? \\[0pt]
Qu'est-ce qu'un idéal ? \\[0pt]
Qu'est-ce qu'un idéaliste ? \\[0pt]
Qu'est-ce qu'un individu ? \\[0pt]
Qu'est-ce qu'un intellectuel ? \\[0pt]
Qu'est-ce qu'un jugement analytique ? \\[0pt]
Qu'est-ce qu'un jugement de goût ? \\[0pt]
Qu'est-ce qu'un lieu ? \\[0pt]
Qu'est-ce qu'un lieu commun ? \\[0pt]
Qu'est-ce qu'un livre ? \\[0pt]
Qu'est-ce qu'un maître ? \\[0pt]
Qu'est-ce qu'un miracle ? \\[0pt]
Qu'est-ce qu'un modèle ? \\[0pt]
Qu'est-ce qu'un monde \\[0pt]
Qu'est-ce qu'un monde ? \\[0pt]
Qu'est-ce qu'un monstre ? \\[0pt]
Qu'est-ce qu'un mythe ? \\[0pt]
Qu'est-ce qu'un nombre ? \\[0pt]
Qu'est-ce qu'un nom propre ? \\[0pt]
Qu'est-ce qu'un objet ? \\[0pt]
Qu'est-ce qu'un œuvre d'art ? \\[0pt]
Qu'est-ce qu'un organisme ? \\[0pt]
Qu'est-ce qu'un outil ? \\[0pt]
Qu'est-ce qu'un paradoxe ? \\[0pt]
Qu'est-ce qu'un passe-temps ? \\[0pt]
Qu'est-ce qu'un paysage ? \\[0pt]
Qu'est-ce qu'un peuple ? \\[0pt]
Qu'est-ce qu'un peuple libre ? \\[0pt]
Qu'est-ce qu'un philosophe ? \\[0pt]
Qu'est-ce qu'un plaisir pur ? \\[0pt]
Qu'est-ce qu'un portrait ? \\[0pt]
Qu'est-ce qu'un post-moderne ? \\[0pt]
Qu'est-ce qu'un précurseur ? \\[0pt]
Qu'est-ce qu'un préjugé ? \\[0pt]
Qu'est-ce qu'un principe ? \\[0pt]
Qu'est-ce qu'un problème ? \\[0pt]
Qu'est-ce qu'un problème insoluble ? \\[0pt]
Qu'est-ce qu'un problème politique ? \\[0pt]
Qu'est-ce qu'un programme ? \\[0pt]
Qu'est-ce qu'un progrès scientifique ? \\[0pt]
Qu'est-ce qu'un prophète ? \\[0pt]
Qu'est-ce qu'un récit ? \\[0pt]
Qu'est-ce qu'un réfutation ? \\[0pt]
Qu'est-ce qu'un régime politique ? \\[0pt]
Qu'est-ce qu'un savoir-faire ? \\[0pt]
Qu'est-ce qu'un sentiment moral ? \\[0pt]
Qu'est-ce qu'un signe ? \\[0pt]
Qu'est-ce qu'un sophiste ? \\[0pt]
Qu'est-ce qu'un style ? \\[0pt]
Qu'est-ce qu'un symbole ? \\[0pt]
Qu'est-ce qu'un système ? \\[0pt]
Qu'est-ce qu'un tableau ? \\[0pt]
Qu'est-ce qu'un témoignage ? \\[0pt]
Qu'est-ce qu'un terroriste ? \\[0pt]
Qu'est-ce qu'un traître ? \\[0pt]
Qu'est-ce qu'un travail bien fait ? \\[0pt]
Qu'est-ce qu'un tyran ? \\[0pt]
Qu'est-ce qu'un vice ? \\[0pt]
Qu'est que l'autorité ? \\[0pt]
Qu'est qu'une image ? \\[0pt]
Que suis-je ? \\[0pt]
Que valent les préjugés ? \\[0pt]
« Que va-t-il se passer ? » \\[0pt]
Que vaut la décision de la majorité ? \\[0pt]
Que vaut l'excuse : « C'est plus fort que moi » ? \\[0pt]
Que vaut une preuve contre un préjugé ? \\[0pt]
Que veut dire « essentiel » ? \\[0pt]
Que veut dire « je t'aime » ? \\[0pt]
Que veut dire : « je t'aime » ? \\[0pt]
Que veut dire « réel » ? \\[0pt]
Que veut dire « respecter la nature » ? \\[0pt]
Que veut dire : « respecter la nature » ? \\[0pt]
Que voit-on dans un miroir ? \\[0pt]
Que voit-on dans un tableau ? \\[0pt]
Que voyons-nous ? \\[0pt]
Qui croire ? \\[0pt]
Qui est citoyen ? \\[0pt]
Qui est comédien ? \\[0pt]
Qui est immoral ? \\[0pt]
Qui fait la loi ? \\[0pt]
Qui fait l'histoire ? \\[0pt]
Qui meurt ? \\[0pt]
« Qui ne dit mot consent » \\[0pt]
« Qui suis-je ? » \\[0pt]
Qui suis-je ? \\[0pt]
Qui veut la fin veut les moyens \\[0pt]
Qu'y a-t-il ? \\[0pt]
Qu'y a-t-il de sérieux dans le jeu ? \\[0pt]
Raconter sa vie \\[0pt]
Raison et technique \\[0pt]
Raisonner et calculer \\[0pt]
Raisonner par l'absurde \\[0pt]
Rationnel et raisonnable \\[0pt]
Refaire sa vie \\[0pt]
Réfuter \\[0pt]
Regarder un tableau \\[0pt]
Religion et politique \\[0pt]
Rendre raison \\[0pt]
Représenter \\[0pt]
Reproduire \\[0pt]
Résister \\[0pt]
« Rester soi-même » \\[0pt]
Rester soi-même \\[0pt]
Rêver \\[0pt]
Rêver à un autre monde \\[0pt]
Rêvons-nous ? \\[0pt]
Rien \\[0pt]
« Rien de nouveau sous le soleil » \\[0pt]
Rien n'est sans raison \\[0pt]
Roman et vérité \\[0pt]
S'adapter \\[0pt]
Sait-on ce qu'on veut ? \\[0pt]
Sait-on toujours ce que l'on veut ? \\[0pt]
Sans l'art, parlerait-on de beauté ? \\[0pt]
« Sans l'ombre d'un doute » \\[0pt]
Santé et politique \\[0pt]
Sauver les apparences \\[0pt]
Sauver les phénomènes \\[0pt]
Savoir, est-ce se libérer ? \\[0pt]
Savoir et savoir-faire \\[0pt]
Savoir faire \\[0pt]
Savoir renoncer \\[0pt]
Savoir se décider \\[0pt]
Savoir tout \\[0pt]
Science et expérience \\[0pt]
Science et hypothèse \\[0pt]
Science et idéologie \\[0pt]
Science et métaphysique \\[0pt]
Science et méthode \\[0pt]
Science et objectivité \\[0pt]
Science et technique \\[0pt]
Se connaître soi-même \\[0pt]
Se convertir \\[0pt]
Se cultiver \\[0pt]
Sécurité et liberté \\[0pt]
Se décider \\[0pt]
Se distinguer \\[0pt]
Se faire justice \\[0pt]
S'engager \\[0pt]
S'ennuyer \\[0pt]
Sensation et perception \\[0pt]
Se raconter des histoires \\[0pt]
Serait-il immoral d'autoriser le commerce des organes humains ? \\[0pt]
Se réfugier dans la croyance \\[0pt]
Se sentir libre \\[0pt]
Seul \\[0pt]
Seul le présent existe-t-il ? \\[0pt]
Sexe et genre \\[0pt]
Sexualité et féminité \\[0pt]
Si\ldots{} alors \\[0pt]
Si Dieu n'existe pas, tout est-il permis ? \\[0pt]
Si Dieu n'existe pas, tout est-il possible ? \\[0pt]
Signe et symbole \\[0pt]
Société et communauté \\[0pt]
Société et religion \\[0pt]
Sociologie et anthropologie \\[0pt]
Soi \\[0pt]
Soigner \\[0pt]
Sommes-nous condamnés à être libres ? \\[0pt]
Sommes-nous dominés par la technique ? \\[0pt]
Sommes-nous gouvernés par nos passions ? \\[0pt]
Sommes-nous les jouets de nos pulsions ? \\[0pt]
Sommes-nous libres ? \\[0pt]
Sommes-nous perfectibles ? \\[0pt]
Sommes-nous responsables d'autrui ? \\[0pt]
Sommes-nous responsables de nos erreurs ? \\[0pt]
Sommes-nous responsables de nos opinions ? \\[0pt]
Sommes-nous responsables de nos passions ? \\[0pt]
Sommes-nous soumis au temps ? \\[0pt]
Songe et réalité \\[0pt]
S'orienter \\[0pt]
Soyez naturel ! \\[0pt]
Sport et politique \\[0pt]
Substance et sujet \\[0pt]
Suffit-il d'être informé pour comprendre ? \\[0pt]
Suis-je le même en des temps différents ? \\[0pt]
Suis-je maître de mes pensées ? \\[0pt]
Suis-je mon cerveau ? \\[0pt]
Suis-je mon corps ? \\[0pt]
Suis-je propriétaire de mon corps ? \\[0pt]
Suis-je seul au monde ? \\[0pt]
Suivre une règle \\[0pt]
Superstition et religion \\[0pt]
Sur quoi l'historien travaille-t-il ? \\[0pt]
Sur quoi repose l'accord des esprits ? \\[0pt]
Survivre \\[0pt]
Technique et intérêt \\[0pt]
Technique et responsabilité \\[0pt]
Temps et négation \\[0pt]
Tendances et besoins \\[0pt]
Tenir sa parole \\[0pt]
Terroriser \\[0pt]
Tolérer \\[0pt]
Toujours plus vite ? \\[0pt]
Tous les conflits peuvent-ils être résolus par le dialogue ? \\[0pt]
Tous les hommes sont-ils égaux ? \\[0pt]
Tous les plaisirs se valent-ils ? \\[0pt]
Tout a-t-il une cause ? \\[0pt]
Tout ce qui est excessif est-il insignifiant ? \\[0pt]
Tout ce qui existe a-t-il un prix ? \\[0pt]
Tout désir est-il désir de posséder ? \\[0pt]
Toute conception de l'humain est-elle particulière ? \\[0pt]
Toute connaissance commence-t-elle avec l'expérience ? \\[0pt]
Toute connaissance consiste-t-elle en un savoir-faire ? \\[0pt]
Toute connaissance est-elle hypothétique ? \\[0pt]
Toute conscience est-elle conscience de soi ? \\[0pt]
Toute notre connaissance dérive-t-elle de l'expérience ? \\[0pt]
Toute pensée revêt-elle nécessairement une forme linguistique ? \\[0pt]
Toute peur est-elle irrationnelle ? \\[0pt]
Toute philosophie constitue-t-elle une doctrine ? \\[0pt]
Toutes les choses sont-elles singulières ? \\[0pt]
Toutes les interprétations se valent-elles ? \\[0pt]
Toutes les opinions se valent-elles ? \\[0pt]
Toutes les opinions sont-elles bonnes à dire ? \\[0pt]
Tout est-il à vendre ? \\[0pt]
Tout est-il démontrable ? \\[0pt]
Tout est-il faux dans la fiction ? \\[0pt]
Tout est-il mesurable ? \\[0pt]
Tout est-il politique ? \\[0pt]
Tout est-il quantifiable ? \\[0pt]
Tout est-il relatif ? \\[0pt]
« Tout est possible » \\[0pt]
Tout est relatif \\[0pt]
Toute vérité est-elle bonne à dire ? \\[0pt]
Toute vérité est-elle démontrable ? \\[0pt]
Toute vérité est-elle vérifiable ? \\[0pt]
Tout malheur est-il une injustice ? \\[0pt]
Tout ou rien \\[0pt]
Tout peut-il être dit ? \\[0pt]
Tout peut-il s'expliquer ? \\[0pt]
Tout pouvoir corrompt-il ? \\[0pt]
Tout pouvoir est-il oppresseur ? \\[0pt]
Tout principe est-il un fondement ? \\[0pt]
Tout savoir a-t-il une justification ? \\[0pt]
Tout savoir peut-il se transmettre ? \\[0pt]
Tradition et vérité \\[0pt]
Traduction, Trahison \\[0pt]
Traduire \\[0pt]
Traduire, est-ce trahir ? \\[0pt]
Tragédie et comédie \\[0pt]
Tromper \\[0pt]
Tu aimeras ton prochain comme toi-même \\[0pt]
Tuer et laisser mourir \\[0pt]
Tuer le temps \\[0pt]
« Tu ne tueras point » \\[0pt]
Un autre monde est-il possible ? \\[0pt]
Un contrat peut-il être injuste ? \\[0pt]
Un contrat peut-il être social ? \\[0pt]
Une action peut-elle être désintéressée ? \\[0pt]
Une action peut-elle être machinale ? \\[0pt]
Une expérience se partage-t-elle ? \\[0pt]
Une idée peut-elle être fausse ? \\[0pt]
Une intention peut-elle être coupable ? \\[0pt]
Une machine peut-elle avoir une mémoire ? \\[0pt]
Une machine peut-elle penser ? \\[0pt]
Une œuvre d'art produit-elle du beau ? \\[0pt]
Une œuvre doit-elle nécessairement être belle ? \\[0pt]
Une œuvre est-elle nécessairement singulière ? \\[0pt]
Une science de l'homme est-elle possible ? \\[0pt]
Une science parfaite est-elle possible ? \\[0pt]
Une société d'athées est-elle possible ? \\[0pt]
Une théorie scientifique peut-elle devenir fausse ? \\[0pt]
Une théorie scientifique peut-elle être vraie ? \\[0pt]
Un jeu peut-il être sérieux ? \\[0pt]
Un langage universel est-il concevable ? \\[0pt]
Un objet technique peut-il être beau ? \\[0pt]
Un peuple peut-il être souverain ? \\[0pt]
Un seul peut-il avoir raison contre tous ? \\[0pt]
Vaut-il mieux subir l'injustice que la commettre ? \\[0pt]
Vendre son corps \\[0pt]
Vérifier \\[0pt]
Vérité et certitude \\[0pt]
Vérité et cohérence \\[0pt]
Vérité et fiction \\[0pt]
Vérité et poésie \\[0pt]
Vérité et subjectivité \\[0pt]
Vertu et perfection \\[0pt]
Vivre \\[0pt]
Vivre au présent \\[0pt]
Vivre caché \\[0pt]
Vivre en immortel \\[0pt]
Vivre, est-ce un droit ? \\[0pt]
Vivre et bien vivre \\[0pt]
Vivre et exister \\[0pt]
Vivre le moment présent \\[0pt]
« Vivre libre ou mourir » \\[0pt]
Vivre sa vie \\[0pt]
Vivre selon la nature \\[0pt]
Voir \\[0pt]
Voir et savoir \\[0pt]
Voir et toucher \\[0pt]
Voir le meilleur et faire le pire \\[0pt]
Voir un tableau \\[0pt]
Vouloir dire \\[0pt]
Vouloir, est-ce encore désirer ? \\[0pt]
Vouloir être immortel \\[0pt]
Vouloir le mal \\[0pt]
Vouloir l'impossible \\[0pt]
Vouloir oublier \\[0pt]
Voyager dans le temps \\[0pt]
Y a-t-il de bons préjugés ? \\[0pt]
Y a-t-il de fausses sciences ? \\[0pt]
Y a-t-il de faux problèmes ? \\[0pt]
Y a-t-il de l'impensable ? \\[0pt]
Y a-t-il de l'incommunicable ? \\[0pt]
Y a-t-il de l'inconnaissable ? \\[0pt]
Y a-t-il de l'indicible ? \\[0pt]
Y a-t-il de l'irréductible ? \\[0pt]
Y a-t-il de l'irréfutable ? \\[0pt]
Y a-t-il de mauvais spectateurs ? \\[0pt]
Y a-t-il des acquis définitifs en science ? \\[0pt]
Y a-t-il des actes de pensée ? \\[0pt]
Y a-t-il des arts mineurs ? \\[0pt]
Y a-t-il des colères justes ? \\[0pt]
Y a-t-il des connaissances désintéressées ? \\[0pt]
Y a-t-il des critères de l'humanité ? \\[0pt]
Y a-t-il des critères du goût ? \\[0pt]
Y a-t-il des croyances rationnelles ? \\[0pt]
Y a-t-il des déterminismes sociaux ? \\[0pt]
Y a-t-il des devoirs envers soi-même ? \\[0pt]
Y a-t-il des êtres mathématiques ? \\[0pt]
Y a-t-il des expériences de la liberté ? \\[0pt]
Y a-t-il des expériences métaphysiques ? \\[0pt]
Y a-t-il des faux problèmes ? \\[0pt]
Y a-t-il des fins dans la nature ? \\[0pt]
Y a-t-il des guerres justes ? \\[0pt]
Y a-t-il des illusions nécessaires ? \\[0pt]
Y a-t-il des instincts propres à l'Homme ? \\[0pt]
Y a-t-il des interprétations fausses ? \\[0pt]
Y a-t-il des leçons de l'histoire ? \\[0pt]
Y a-t-il des limites à la connaissance ? \\[0pt]
Y a-t-il des limites à la conscience ? \\[0pt]
Y a-t-il des limites à la pensée ? \\[0pt]
Y a-t-il des limites à l'exprimable ? \\[0pt]
Y a-t-il des lois de la pensée ? \\[0pt]
Y a-t-il des lois de l'histoire ? \\[0pt]
Y a-t-il des lois de l'Histoire ? \\[0pt]
Y a-t-il des normes naturelles ? \\[0pt]
Y a-t-il des passions intraitables ? \\[0pt]
Y a-t-il des passions raisonnables ? \\[0pt]
Y a-t-il des plaisirs purs ? \\[0pt]
Y a-t-il des régressions historiques ? \\[0pt]
Y a-t-il des révolutions scientifiques ? \\[0pt]
Y a-t-il des sciences de l'homme ? \\[0pt]
Y a-t-il des sciences exactes ? \\[0pt]
Y a-t-il des sentiments moraux ? \\[0pt]
Y a-t-il des signes naturels ? \\[0pt]
Y a-t-il des sociétés sans histoire ? \\[0pt]
Y a-t-il des techniques de pensée ? \\[0pt]
Y a-t-il des techniques du corps ? \\[0pt]
Y a-t-il des techniques pour être heureux ? \\[0pt]
Y a-t-il des valeurs naturelles ? \\[0pt]
Y a-t-il des valeurs universelles ? \\[0pt]
Y a-t-il des vérités éternelles ? \\[0pt]
Y a-t-il des vérités philosophiques ? \\[0pt]
Y a-t-il des violences légitimes ? \\[0pt]
Y a-t-il lieu de distinguer le don et l'échange ? \\[0pt]
Y a-t-il plusieurs libertés ? \\[0pt]
Y a-t-il plusieurs morales ? \\[0pt]
Y a-t-il plusieurs sortes de vérité ? \\[0pt]
Y a-t-il un art de gouverner ? \\[0pt]
Y a-t-il un art de penser ? \\[0pt]
Y a-t-il un art d'être heureux ? \\[0pt]
Y a-t-il un art de vivre ? \\[0pt]
Y a-t-il un au-delà du langage ? \\[0pt]
Y a-t-il un beau naturel ? \\[0pt]
Y a-t-il un bien commun ? \\[0pt]
Y a-t-il un bon usage des passions ? \\[0pt]
Y a-t-il un critère de vérité ? \\[0pt]
Y a-t-il un critère du vrai ? \\[0pt]
Y a-t-il un devoir de mémoire ? \\[0pt]
Y a-t-il un devoir d'être heureux ? \\[0pt]
Y a-t-il un droit de la guerre ? \\[0pt]
Y a-t-il un droit de mentir ? \\[0pt]
Y a-t-il un droit de résistance ? \\[0pt]
Y a-t-il un droit d'ingérence ? \\[0pt]
Y a-t-il un droit du plus faible ? \\[0pt]
Y a-t-il un droit du plus fort ? \\[0pt]
Y a-t-il un droit international ? \\[0pt]
Y a-t-il un droit naturel ? \\[0pt]
Y a-t-il un droit universel au mariage ? \\[0pt]
Y a-t-il une cause première ? \\[0pt]
Y a-t-il une connaissance du probable ? \\[0pt]
Y a-t-il une connaissance du singulier ? \\[0pt]
Y a-t-il une connaissance historique ? \\[0pt]
Y a-t-il une connaissance sensible ? \\[0pt]
Y a-t-il une éducation du goût ? \\[0pt]
Y a-t-il une esthétique de la laideur ? \\[0pt]
Y a-t-il une éthique des moyens ? \\[0pt]
Y a-t-il une expérience de la liberté ? \\[0pt]
Y a-t-il une fin de l'histoire ? \\[0pt]
Y a-t-il une histoire de la nature ? \\[0pt]
Y a-t-il une histoire de la raison ? \\[0pt]
Y a-t-il une histoire de la vérité ? \\[0pt]
Y a-t-il une intelligence du corps ? \\[0pt]
Y a-t-il une langue de la philosophie ? \\[0pt]
Y a-t-il une limite au développement scientifique ? \\[0pt]
Y a-t-il une nature humaine ? \\[0pt]
Y a-t-il une pensée sans signes ? \\[0pt]
Y a-t-il une positivité de l'erreur ? \\[0pt]
Y a-t-il une présence du passé ? \\[0pt]
Y a-t-il une rationalité des sentiments ? \\[0pt]
Y a-t-il une sagesse populaire ? \\[0pt]
Y a-t-il une servitude volontaire ? \\[0pt]
Y a-t-il une spécificité des sciences humaines ? \\[0pt]
Y a-t-il une universalité du beau ? \\[0pt]
Y a-t-il une vérité du sensible ? \\[0pt]
Y a-t-il une vérité en histoire ? \\[0pt]
Y a-t-il un langage animal ? \\[0pt]
Y a-t-il un langage de la musique ? \\[0pt]
Y a-t-il un langage de l'art ? \\[0pt]
Y a-t-il un langage universel ? \\[0pt]
Y a-t-il un monde de l'art ? \\[0pt]
Y a-t-il un progrès en art ? \\[0pt]
Y a-t-il un progrès moral ? \\[0pt]
Y a-t-il un savoir du contingent ? \\[0pt]
Y a-t-il un savoir immédiat ? \\[0pt]
Y a-t-il un sens à se dire citoyen du monde ? \\[0pt]
Y a-t-il un sens moral ? \\[0pt]
Y a-t-il un souverain bien ? \\[0pt]
Y a-t-il un travail de la pensée ? \\[0pt]
Y a-t-il un tribunal de l'histoire ? \\[0pt]


\subsection{ENS B​/​L}
\label{sec:org962fdfa}

\noindent
Agir sans raison \\[0pt]
Aimer, est-ce vraiment connaître ? \\[0pt]
Analyse et synthèse \\[0pt]
Apprendre à parler \\[0pt]
Apprendre à philosopher \\[0pt]
Apprendre à voir \\[0pt]
À quelle expérience l'art nous convie-t-il ? \\[0pt]
À quelles conditions une théorie peut-elle être scientifique ? \\[0pt]
À quelles conditions un jugement est-il moral ? \\[0pt]
« À quelque chose malheur est bon » \\[0pt]
À quels signes reconnaît-on la vérité ? \\[0pt]
À qui doit-on le respect ? \\[0pt]
À qui est mon corps ? \\[0pt]
À qui profite le crime ? \\[0pt]
À quoi bon les regrets ? \\[0pt]
À quoi bon penser la fin du monde ? \\[0pt]
À quoi bon voyager ? \\[0pt]
À quoi est-il impossible de s'habituer ? \\[0pt]
À quoi faut-il être fidèle ? \\[0pt]
À quoi reconnaît-on l'injustice ? \\[0pt]
À quoi reconnaît-on une œuvre d'art ? \\[0pt]
À quoi sert la connaissance du passé ? \\[0pt]
À quoi sert la notion d'état de nature ? \\[0pt]
À quoi sert l'État ? \\[0pt]
À quoi sert un exemple ? \\[0pt]
À quoi servent les mythes ? \\[0pt]
À quoi servent les utopies ? \\[0pt]
A-t-on besoin de maîtres ? \\[0pt]
A-t-on besoin des poètes ? \\[0pt]
A-t-on besoin d'experts ? \\[0pt]
A-t-on intérêt à tout savoir ? \\[0pt]
A-t-on le droit de se révolter ? \\[0pt]
Au nom de quoi le plaisir serait-il condamnable ? \\[0pt]
Autrui est-il inconnaissable ? \\[0pt]
Autrui est-il mon semblable ? \\[0pt]
Avoir \\[0pt]
Avoir de la suite dans les idées \\[0pt]
Avoir de l'expérience \\[0pt]
Avoir des principes \\[0pt]
Avoir des scrupules \\[0pt]
Avoir une idée \\[0pt]
Avons-nous besoin d'amis ? \\[0pt]
Avons-nous besoin d'utopies ? \\[0pt]
Avons-nous des devoirs envers les animaux ? \\[0pt]
Avons-nous des raisons d'espérer ? \\[0pt]
Avons-nous un droit au droit ? \\[0pt]
Avons-nous une responsabilité envers le passé ? \\[0pt]
Avons-nous un libre arbitre ? \\[0pt]
Bon sens et philosophie \\[0pt]
Calculer et raisonner \\[0pt]
Cause et loi \\[0pt]
Certitude et vérité \\[0pt]
Changer \\[0pt]
Changer la vie \\[0pt]
Choisir \\[0pt]
Choisit-on son corps ? \\[0pt]
Comment connaissons-nous nos devoirs ? \\[0pt]
Comment deux personnes peuvent-elles partager la même pensée ? \\[0pt]
Comment distinguer entre l'amour et l'amitié ? \\[0pt]
Comment distinguer l'amour de l'amitié ? \\[0pt]
Comment distinguer le vrai du faux ? \\[0pt]
Comment établir des critères d'équité ? \\[0pt]
Comment exprimer l'identité ? \\[0pt]
Comment la science progresse-t-elle ? \\[0pt]
Comment ne pas être humaniste ? \\[0pt]
Comment ne pas être libéral ? \\[0pt]
Comment peut-on être sceptique ? \\[0pt]
Comment peut-on se trahir soi-même ? \\[0pt]
Comment sait-on qu'une chose existe ? \\[0pt]
Communauté, collectivité, société \\[0pt]
Communauté et société \\[0pt]
Comparaison n'est pas raison \\[0pt]
Conclure \\[0pt]
Confiance et crédulité \\[0pt]
Conscience et mauvaise conscience \\[0pt]
Conscience et mémoire \\[0pt]
Contrainte et obligation \\[0pt]
Contrôle et vigilance \\[0pt]
Conventions sociales et moralité \\[0pt]
Conviction et certitude \\[0pt]
Création et production \\[0pt]
Crime et châtiment \\[0pt]
Crise et progrès \\[0pt]
Critiquer \\[0pt]
Croire \\[0pt]
Croire et savoir \\[0pt]
Croit-on ce que l'on veut ? \\[0pt]
Croyance et choix \\[0pt]
Culture et civilisation \\[0pt]
Culture et communauté \\[0pt]
Décider \\[0pt]
Découvrir \\[0pt]
Décrire, est-ce déjà expliquer ? \\[0pt]
De quoi a-t-on conscience lorsqu'on a conscience de soi ? \\[0pt]
De quoi avons-nous besoin ? \\[0pt]
De quoi la philosophie est-elle le désir ? \\[0pt]
De quoi parlent les mathématiques ? \\[0pt]
De quoi rit-on ? \\[0pt]
De quoi y a-t-il histoire ? \\[0pt]
Désirer et vouloir \\[0pt]
Désir et besoin \\[0pt]
Désir et volonté \\[0pt]
Déterminisme et responsabilité \\[0pt]
Devant qui est-on responsable ? \\[0pt]
Devons-nous nous libérer de nos désirs ? \\[0pt]
Devons-nous quelque chose à la nature ? \\[0pt]
Devons-nous vivre comme si nous ne devions jamais mourir ? \\[0pt]
Dialogue et délibération en démocratie \\[0pt]
Dieu aurait-il pu mieux faire ? \\[0pt]
« Dieu est mort » \\[0pt]
Dieu est mort \\[0pt]
Dire, est-ce faire ? \\[0pt]
Discrimination et revendication \\[0pt]
Discussion et conversation \\[0pt]
Dois-je mériter mon bonheur ? \\[0pt]
Doit-on le respect au vivant ? \\[0pt]
Doit-on rechercher l'harmonie ? \\[0pt]
Doit-on se passer des utopies ? \\[0pt]
Doit-on souffrir de n'être pas compris ? \\[0pt]
Dominer la nature \\[0pt]
Donner \\[0pt]
Donner sa parole \\[0pt]
D'où viennent les concepts ? \\[0pt]
D'où viennent les idées générales ? \\[0pt]
D'où viennent les préjugés ? \\[0pt]
Droits, garanties, protection \\[0pt]
Éducation et dressage \\[0pt]
Efficacité et justice \\[0pt]
En politique, ne faut-il croire qu'aux rapports de force ? \\[0pt]
En quel sens les sciences ont-elles une histoire ? \\[0pt]
En quel sens peut-on dire que le mal n'existe pas ? \\[0pt]
En quoi la justice met-elle fin à la violence ? \\[0pt]
En quoi la nature constitue-t-elle un modèle ? \\[0pt]
En quoi la patience est-elle une vertu ? \\[0pt]
En quoi la physique a-t-elle besoin des mathématiques ? \\[0pt]
En quoi la sociologie est-elle fondamentale ? \\[0pt]
En quoi le langage est-il constitutif de l'homme ? \\[0pt]
En quoi les hommes restent-ils des enfants ? \\[0pt]
En quoi une œuvre d'art est-elle moderne ? \\[0pt]
Enseigner \\[0pt]
Enseigner, est-ce transmettre un savoir ? \\[0pt]
Entendre raison \\[0pt]
Espace et représentation \\[0pt]
Espérer \\[0pt]
Est-ce la mémoire qui constitue mon identité ? \\[0pt]
Est-ce par désir de la vérité que l'homme cherche à savoir ? \\[0pt]
Est-il difficile de savoir ce que l'on veut ? \\[0pt]
Est-il naturel de s'aimer soi-même ? \\[0pt]
Est-il possible d'améliorer l'homme ? \\[0pt]
Est-il possible de ne croire à rien ? \\[0pt]
Est-il possible de préparer l'avenir ? \\[0pt]
Est-il raisonnable d'aimer ? \\[0pt]
Est-on responsable de son passé ? \\[0pt]
État et nation \\[0pt]
Étonnement et sidération \\[0pt]
Être chez soi \\[0pt]
Être compris \\[0pt]
Être conséquent avec soi-même \\[0pt]
Être dans son droit \\[0pt]
Être et avoir \\[0pt]
Être et devoir-être \\[0pt]
Être et paraître \\[0pt]
Être et penser, est-ce la même chose ? \\[0pt]
Être hors de soi \\[0pt]
Être objectif \\[0pt]
Être ou avoir \\[0pt]
Être ou ne pas être \\[0pt]
Être soi-même \\[0pt]
Évidence et certitude \\[0pt]
Existe-t-il de faux besoins ? \\[0pt]
Existe-t-il des croyances collectives ? \\[0pt]
Existe-t-il un vocabulaire neutre des droits fondamentaux ? \\[0pt]
Expliquer et comprendre \\[0pt]
Faire ce que l'on dit \\[0pt]
Faire comme si \\[0pt]
Faire de nécessité vertu \\[0pt]
Faire la part des choses \\[0pt]
Faire le mal \\[0pt]
Faire le nécessaire \\[0pt]
Faisons-nous l'histoire ? \\[0pt]
Fait-on de la politique pour changer les choses ? \\[0pt]
Faits et valeurs \\[0pt]
Faudrait-il ne rien oublier ? \\[0pt]
Faut-il avoir peur de la technique ? \\[0pt]
Faut-il changer le monde ? \\[0pt]
Faut-il concilier les contraires ? \\[0pt]
Faut-il condamner le luxe ? \\[0pt]
Faut-il craindre la mort ? \\[0pt]
Faut-il craindre le pire ? \\[0pt]
Faut-il craindre le regard d'autrui ? \\[0pt]
Faut-il défendre la démocratie ? \\[0pt]
Faut-il désirer la vérité ? \\[0pt]
Faut-il détruire pour créer ? \\[0pt]
Faut-il donner un sens à la souffrance ? \\[0pt]
Faut-il douter de l'évidence \\[0pt]
Faut-il être bon ? \\[0pt]
Faut-il être modéré ? \\[0pt]
Faut-il être réaliste ? \\[0pt]
Faut-il imposer la vérité ? \\[0pt]
Faut-il opposer l'État et la société ? \\[0pt]
Faut-il opposer l'histoire et la fiction ? \\[0pt]
Faut-il opposer nature et culture ? \\[0pt]
Faut-il poser des limites à l'activité rationnelle ? \\[0pt]
Faut-il protéger la nature ? \\[0pt]
Faut-il protéger les faibles contre les forts ? \\[0pt]
Faut-il reconnaître pour connaître ? \\[0pt]
Faut-il résister à la peur de mourir ? \\[0pt]
Faut-il respecter la nature ? \\[0pt]
Faut-il rire ou pleurer ? \\[0pt]
Faut-il s'adapter ? \\[0pt]
Faut-il savoir obéir pour gouverner ? \\[0pt]
Faut-il savoir prendre des risques ? \\[0pt]
Faut-il se méfier de l'écriture ? \\[0pt]
Faut-il se méfier des apparences ? \\[0pt]
Faut-il s'en remettre à l'État pour limiter le pouvoir de l'État ? \\[0pt]
Faut-il séparer la science et la technique ? \\[0pt]
Faut-il séparer morale et politique ? \\[0pt]
Faut-il toujours avoir raison ? \\[0pt]
Faut-il toujours dire la vérité ? \\[0pt]
Faut-il un commencement à tout ? \\[0pt]
Faut-il vivre comme si l'on ne devait jamais mourir ? \\[0pt]
Faut-il vouloir changer le monde ? \\[0pt]
Fiction et virtualité \\[0pt]
Foi et raison \\[0pt]
Folie et raison \\[0pt]
Folie et société \\[0pt]
Force et violence \\[0pt]
Fuir la civilisation \\[0pt]
Gagner \\[0pt]
Gouvernement et société \\[0pt]
Guerres justes et injustes \\[0pt]
Habiter \\[0pt]
Habiter sur la terre \\[0pt]
Histoire et devenir \\[0pt]
Histoire et politique \\[0pt]
Histoire et structure \\[0pt]
Illusion et apparence \\[0pt]
Imiter \\[0pt]
Incertitude et action \\[0pt]
Individualisme et égoïsme \\[0pt]
Intentions, plans et stratégies \\[0pt]
Interpréter \\[0pt]
Invention et imitation \\[0pt]
« Je ne crois que ce que je vois » \\[0pt]
« Je préfère une injustice à un désordre » \\[0pt]
Jouer \\[0pt]
Jouer son rôle \\[0pt]
Jugement de goût et jugement esthétique \\[0pt]
Jugement moral et jugement empirique \\[0pt]
Juger \\[0pt]
Juger et décider \\[0pt]
Jusqu'où interpréter ? \\[0pt]
Justice et vengeance \\[0pt]
Justification et politique \\[0pt]
La banalité \\[0pt]
La barbarie \\[0pt]
La béatitude \\[0pt]
La beauté de la nature \\[0pt]
La beauté du diable \\[0pt]
La beauté est-elle dans les choses ? \\[0pt]
La beauté est-elle une promesse de bonheur ? \\[0pt]
La beauté morale \\[0pt]
La beauté peut-elle délivrer une vérité ? \\[0pt]
La bestialité \\[0pt]
La bêtise \\[0pt]
La bêtise et la méchanceté sont-elles liées nécessairement ? \\[0pt]
La bienveillance \\[0pt]
L'absence \\[0pt]
L'absolu \\[0pt]
L'abstraction \\[0pt]
L'abstraction est-elle toujours utile à la science empirique ? \\[0pt]
L'abstrait et le concret \\[0pt]
L'abus de pouvoir \\[0pt]
L'académisme \\[0pt]
L'académisme et les fins de l'art \\[0pt]
La calomnie \\[0pt]
La causalité \\[0pt]
L'accident \\[0pt]
La censure \\[0pt]
La certitude \\[0pt]
La charité \\[0pt]
La charité est-elle une vertu ? \\[0pt]
L'achèvement de l'œuvre \\[0pt]
La chose et l'objet \\[0pt]
La cité sans dieux \\[0pt]
La civilisation \\[0pt]
La classe moyenne \\[0pt]
La classification des sciences \\[0pt]
La cohérence \\[0pt]
La cohérence est-elle un critère de vérité ? \\[0pt]
La colère peut-elle être justifiée ? \\[0pt]
La collection \\[0pt]
La comédie \\[0pt]
La comédie humaine \\[0pt]
La comparaison \\[0pt]
La compassion \\[0pt]
La compétence \\[0pt]
La condition de mortel \\[0pt]
La confiance \\[0pt]
La confiance en la raison \\[0pt]
La connaissance de soi \\[0pt]
La connaissance du bien \\[0pt]
La connaissance du singulier \\[0pt]
La connaissance et la croyance \\[0pt]
La connaissance et le vivant \\[0pt]
La connaissance historique \\[0pt]
La connaissance intuitive \\[0pt]
La connaissance s'interdit-elle tout recours à l'imagination ? \\[0pt]
La conquête de l'espace \\[0pt]
La conscience \\[0pt]
La conscience peut-elle être collective ? \\[0pt]
La conséquence \\[0pt]
La conservation \\[0pt]
La consolation \\[0pt]
La constitution \\[0pt]
La contemplation \\[0pt]
La contingence \\[0pt]
La contingence est-elle la condition de la liberté ? \\[0pt]
La contradiction \\[0pt]
La contrainte déontologique \\[0pt]
La convalescence \\[0pt]
La conversion \\[0pt]
La conviction \\[0pt]
La copie \\[0pt]
La corruption \\[0pt]
La couleur \\[0pt]
La coutume \\[0pt]
« La crainte est le commencement de la sagesse » \\[0pt]
La création \\[0pt]
La crise \\[0pt]
La critique \\[0pt]
La critique des bons sentiments \\[0pt]
La critique des théories \\[0pt]
« La critique est aisée » \\[0pt]
La croissance du savoir \\[0pt]
La croyance peut-elle être rationnelle ? \\[0pt]
La cruauté \\[0pt]
L'acteur \\[0pt]
L'action du temps \\[0pt]
L'action intentionnelle \\[0pt]
L'activité \\[0pt]
L'activité se laisse-t-elle programmer ? \\[0pt]
L'actualité \\[0pt]
La culture de masse \\[0pt]
La culture : pour quoi faire ? \\[0pt]
La culture savante et la culture populaire \\[0pt]
La curiosité \\[0pt]
La danse \\[0pt]
La décadence \\[0pt]
La décision \\[0pt]
La découverte de la vérité peut-elle être le fait du hasard ? \\[0pt]
La démesure \\[0pt]
La démocratie \\[0pt]
La démocratie, est-ce la fin du despotisme ? \\[0pt]
La démocratie et les institutions de la justice \\[0pt]
La démocratie et le statut de la loi \\[0pt]
La démonstration \\[0pt]
La dépense \\[0pt]
La description \\[0pt]
La désillusion \\[0pt]
La dialectique \\[0pt]
La différence \\[0pt]
La différence culturelle \\[0pt]
La différence sexuelle \\[0pt]
La dignité \\[0pt]
La discipline \\[0pt]
La discrimination \\[0pt]
La disponibilité \\[0pt]
La disposition \\[0pt]
La dispute \\[0pt]
La dissidence \\[0pt]
La dissimulation \\[0pt]
La distinction \\[0pt]
La diversité humaine \\[0pt]
La division de la volonté \\[0pt]
La division du travail \\[0pt]
La domestication \\[0pt]
La domination \\[0pt]
La douleur \\[0pt]
L'adversité \\[0pt]
La faiblesse \\[0pt]
La famille est-elle naturelle ? \\[0pt]
La famille est-elle une communauté naturelle ? \\[0pt]
La famille et la cité \\[0pt]
La famille et le droit \\[0pt]
La fatigue \\[0pt]
La faute et l'erreur \\[0pt]
La femme est-elle l'avenir de l'homme ? \\[0pt]
La fête \\[0pt]
La fiction \\[0pt]
La fiction est-elle fausse ? \\[0pt]
La fidélité \\[0pt]
La figure humaine \\[0pt]
La fin de la discussion \\[0pt]
La fin de la guerre \\[0pt]
La fin de la métaphysique \\[0pt]
La fin du monde \\[0pt]
La fin du mythe \\[0pt]
La finitude \\[0pt]
La fin justifie-t-elle les moyens ? \\[0pt]
La folie \\[0pt]
La force \\[0pt]
La force d'âme \\[0pt]
La force de l'habitude \\[0pt]
La force des choses \\[0pt]
La force des faibles \\[0pt]
La force des idées \\[0pt]
La force des lois \\[0pt]
La force du social \\[0pt]
La force et le droit \\[0pt]
La force publique \\[0pt]
La formalisation \\[0pt]
La fortune \\[0pt]
La fragilité \\[0pt]
La frontière \\[0pt]
L'âge atomique \\[0pt]
L'âge de la réflexion \\[0pt]
L'âge d'or \\[0pt]
La grâce \\[0pt]
L'agression \\[0pt]
La guérison \\[0pt]
La guerre \\[0pt]
La guerre est-elle l'essentiel de toute politique ? \\[0pt]
La guerre mondiale \\[0pt]
La haine de la raison \\[0pt]
La haine des images \\[0pt]
La haine et le mépris \\[0pt]
La honte \\[0pt]
La jalousie \\[0pt]
La jeunesse \\[0pt]
La joie \\[0pt]
La justice peut-elle se fonder sur le compromis ? \\[0pt]
La justification \\[0pt]
La lâcheté \\[0pt]
La laïcité \\[0pt]
La laideur \\[0pt]
La langue de la raison \\[0pt]
La langue maternelle \\[0pt]
La lassitude \\[0pt]
L'aléatoire \\[0pt]
La liberté comporte-t-elle des degrés ? \\[0pt]
La liberté peut-elle s'affirmer sans violence ? \\[0pt]
L'aliénation \\[0pt]
La limite \\[0pt]
La logique a-t-elle une histoire ? \\[0pt]
La logique du pire \\[0pt]
La logique est-elle une science ? \\[0pt]
La loyauté \\[0pt]
L'altruisme \\[0pt]
La lumière et les ténèbres \\[0pt]
La lutte des classes \\[0pt]
La machine \\[0pt]
La main \\[0pt]
La maîtrise de la nature \\[0pt]
La maîtrise de soi \\[0pt]
L'amateur \\[0pt]
La matière \\[0pt]
La matière et la forme \\[0pt]
La maturité \\[0pt]
La mauvaise volonté \\[0pt]
L'ambiguïté \\[0pt]
La méchanceté \\[0pt]
La médiation \\[0pt]
La médiocrité \\[0pt]
La méditation \\[0pt]
L'âme et l'esprit \\[0pt]
La mélancolie \\[0pt]
La mémoire collective \\[0pt]
La menace \\[0pt]
La mesure \\[0pt]
La mesure du temps \\[0pt]
La métaphore \\[0pt]
La méthode est-elle nécessaire pour la recherche de la vérité ? \\[0pt]
La minorité \\[0pt]
L'amitié \\[0pt]
L'amitié est-elle une vertu ? \\[0pt]
La modération \\[0pt]
La modernité \\[0pt]
La morale doit-elle en appeler à la nature ? \\[0pt]
La morale et le droit \\[0pt]
La morale et les mœurs \\[0pt]
La moralité consiste-t-elle à se contraindre soi-même ? \\[0pt]
La moralité et le traitement des animaux \\[0pt]
La moralité se réduit-elle aux sentiments ? \\[0pt]
La mort \\[0pt]
La mort d'autrui \\[0pt]
La mort de l'art \\[0pt]
La mort de l'homme \\[0pt]
L'amour a-t-il des raisons ? \\[0pt]
L'amour de l'art \\[0pt]
L'amour de la vie \\[0pt]
L'amour de soi \\[0pt]
L'amour du prochain \\[0pt]
L'amour et la mort \\[0pt]
La multitude \\[0pt]
La naissance \\[0pt]
La naïveté \\[0pt]
L'analogie \\[0pt]
L'analyse \\[0pt]
L'anarchie \\[0pt]
La nation est-elle dépassée ? \\[0pt]
La nature \\[0pt]
La nature a-t-elle des droits ? \\[0pt]
La nature a-t-elle un langage ? \\[0pt]
La nature du fait moral \\[0pt]
La nature est-elle artiste ? \\[0pt]
La nature est-elle politique ? \\[0pt]
La nature est-elle une norme ? \\[0pt]
La nature et le beau \\[0pt]
La nature peut-elle être belle ? \\[0pt]
La nature se donne-t-elle à penser ? \\[0pt]
L'anéantissement \\[0pt]
La négation \\[0pt]
La neutralité \\[0pt]
L'angoisse \\[0pt]
L'animal et la bête \\[0pt]
L'animalité \\[0pt]
La noblesse \\[0pt]
L'anomalie \\[0pt]
L'anormal \\[0pt]
La nostalgie \\[0pt]
La notion de finalité a-t-elle de l'intérêt pour le savant ? \\[0pt]
La notion de genre littéraire \\[0pt]
La nouveauté \\[0pt]
L'anticipation \\[0pt]
L'antinomie \\[0pt]
La nudité \\[0pt]
La nuit et le jour \\[0pt]
La paix \\[0pt]
La paresse \\[0pt]
La parole et l'écriture \\[0pt]
La part de l'ombre \\[0pt]
La participation \\[0pt]
La passion amoureuse \\[0pt]
La passion de la justice \\[0pt]
La passion de la vérité peut-elle être source d'erreur ? \\[0pt]
La passion de l'égalité \\[0pt]
La passivité \\[0pt]
La patrie \\[0pt]
La pauvreté \\[0pt]
La peine \\[0pt]
La peine de mort \\[0pt]
La pensée a-t-elle une histoire ? \\[0pt]
La pensée de l'espace \\[0pt]
La pensée formelle \\[0pt]
La pensée formelle est-elle privée d'objet ? \\[0pt]
La pensée formelle est-elle une pensée vide ? \\[0pt]
La pensée obéit-elle à des lois ? \\[0pt]
La perception \\[0pt]
La perfection morale \\[0pt]
La personnalité \\[0pt]
La perspective \\[0pt]
La peur de la vérité \\[0pt]
La philosophie a-t-elle une histoire ? \\[0pt]
La philosophie doit-elle être une science ? \\[0pt]
La philosophie est-elle une science ? \\[0pt]
La philosophie et le sens commun \\[0pt]
La philosophie et les sciences \\[0pt]
La philosophie peut-elle être une science ? \\[0pt]
La pitié \\[0pt]
La pitié a-t-elle une valeur ? \\[0pt]
La place de la philosophie dans la culture \\[0pt]
La place publique \\[0pt]
La pluralité \\[0pt]
La pluralité des langues \\[0pt]
La politesse \\[0pt]
La politique doit-elle se mêler du bonheur ? \\[0pt]
La politique est-elle affaire de compétence ? \\[0pt]
La politique est-elle une science ? \\[0pt]
La politique et les passions \\[0pt]
La politique suppose-t-elle la morale ? \\[0pt]
La possession \\[0pt]
L'apparence \\[0pt]
L'apparence du pouvoir \\[0pt]
La précaution \\[0pt]
La présence \\[0pt]
La présence d'esprit \\[0pt]
La preuve \\[0pt]
La prévision \\[0pt]
La prévoyance \\[0pt]
La prison \\[0pt]
La privation \\[0pt]
La privation de liberté \\[0pt]
La promesse \\[0pt]
La propriété \\[0pt]
La protection \\[0pt]
La prudence \\[0pt]
La publicité \\[0pt]
La pudeur \\[0pt]
La puissance \\[0pt]
La pureté \\[0pt]
La question de l'essence \\[0pt]
La question des origines \\[0pt]
La question sociale \\[0pt]
La quête du sens ultime \\[0pt]
La radicalité \\[0pt]
La radicalité est-elle une exigence philosophique ? \\[0pt]
La raison \\[0pt]
La raison a-t-elle des limites ? \\[0pt]
La raison a-t-elle une histoire ? \\[0pt]
La raison d'État \\[0pt]
La raison est-elle suffisante ? \\[0pt]
La raison est-elle toujours raisonnable ? \\[0pt]
La raison et le réel \\[0pt]
La raison peut-elle errer ? \\[0pt]
La raison peut-elle servir le mal ? \\[0pt]
L'archive \\[0pt]
La réaction \\[0pt]
La réalité \\[0pt]
La réalité de l'espace \\[0pt]
La réalité du monde extérieur \\[0pt]
La recherche \\[0pt]
La recherche de la perfection \\[0pt]
La recherche d'identité \\[0pt]
La réciprocité \\[0pt]
La réconciliation \\[0pt]
La reconnaissance \\[0pt]
La réflexion \\[0pt]
La réfutation \\[0pt]
La règle du jeu \\[0pt]
La régression à l'infini \\[0pt]
La relation \\[0pt]
La relativité \\[0pt]
La religion est-elle l'opium du peuple ? \\[0pt]
La religion peut-elle être naturelle ? \\[0pt]
La Renaissance \\[0pt]
La rencontre \\[0pt]
La répétition \\[0pt]
La représentation \\[0pt]
La reproduction \\[0pt]
La reproduction des œuvres d'art \\[0pt]
La république \\[0pt]
La résistance \\[0pt]
La résolution \\[0pt]
La ressemblance \\[0pt]
La révélation \\[0pt]
La révolution \\[0pt]
L'argent \\[0pt]
L'argent est-il un mal nécessaire ? \\[0pt]
La rhétorique \\[0pt]
La rhétorique a-t-elle une valeur ? \\[0pt]
La rigueur \\[0pt]
La rivalité \\[0pt]
L'art a-t-il une histoire ? \\[0pt]
L'art cinématographique \\[0pt]
L'art décoratif \\[0pt]
L'art du portrait \\[0pt]
L'art est-il méthodique ? \\[0pt]
L'art est-il subversif ? \\[0pt]
L'art est-il une promesse de bonheur ? \\[0pt]
L'art et la manière \\[0pt]
L'art et la nouveauté \\[0pt]
L'art et l'espace \\[0pt]
L'artificiel \\[0pt]
L'artiste \\[0pt]
L'artiste et la sensation \\[0pt]
L'artiste travaille-t-il ? \\[0pt]
L'art populaire \\[0pt]
L'art pour l'art \\[0pt]
La sagesse \\[0pt]
La sanction \\[0pt]
La santé \\[0pt]
La santé est-elle un devoir ? \\[0pt]
La santé mentale \\[0pt]
La scène du monde \\[0pt]
La science a-t-elle besoin d'un critère de démarcation entre science et non science ? \\[0pt]
La science a-t-elle une histoire ? \\[0pt]
La science est-elle un jeu ? \\[0pt]
La science et la foi \\[0pt]
« La science ne pense pas » \\[0pt]
La science peut-elle tout expliquer ? \\[0pt]
La séduction \\[0pt]
La sensibilité \\[0pt]
La séparation \\[0pt]
La sérénité \\[0pt]
La servitude \\[0pt]
La servitude volontaire \\[0pt]
La simplicité \\[0pt]
La sociabilité \\[0pt]
La société existe-t-elle ? \\[0pt]
La société peut-elle se passer de l'État ? \\[0pt]
La solitude \\[0pt]
La sollicitude \\[0pt]
La somme et le tout \\[0pt]
La souffrance a-t-elle un sens ? \\[0pt]
La souffrance des animaux \\[0pt]
La souffrance peut-elle avoir un sens ? \\[0pt]
La souveraineté \\[0pt]
La spontanéité \\[0pt]
L'assentiment \\[0pt]
L'association des idées \\[0pt]
La standardisation \\[0pt]
La structure \\[0pt]
La subjectivité \\[0pt]
La superstition \\[0pt]
La surface et la profondeur \\[0pt]
La surprise \\[0pt]
La survie \\[0pt]
La sympathie \\[0pt]
La tâche d'exister \\[0pt]
La technique crée-t-elle son propre monde ? \\[0pt]
La technique est-elle l'application de la science ? \\[0pt]
La technocratie \\[0pt]
La tentation \\[0pt]
La terre \\[0pt]
La terreur \\[0pt]
L'athéisme \\[0pt]
La théorie \\[0pt]
La totalité \\[0pt]
La trace \\[0pt]
La tradition \\[0pt]
La traduction \\[0pt]
La tragédie \\[0pt]
La transcendance \\[0pt]
La transgression \\[0pt]
La transgression des règles \\[0pt]
La transmission \\[0pt]
La transparence \\[0pt]
La transparence des consciences \\[0pt]
La tristesse \\[0pt]
L'attention \\[0pt]
L'attraction \\[0pt]
La tyrannie \\[0pt]
L'au-delà \\[0pt]
L'authenticité \\[0pt]
L'automate \\[0pt]
L'automatisation du raisonnement \\[0pt]
L'autonomie \\[0pt]
L'autorité \\[0pt]
L'autorité de la science \\[0pt]
La valeur d'échange \\[0pt]
La valeur de la nature \\[0pt]
La valeur de l'argent \\[0pt]
La valeur de la vie \\[0pt]
La valeur de l'opinion \\[0pt]
La valeur du consensus \\[0pt]
La valeur du témoignage \\[0pt]
La vanité \\[0pt]
La vengeance \\[0pt]
L'aventure \\[0pt]
La vérification \\[0pt]
La vérité de la perception \\[0pt]
La vérité des arts \\[0pt]
La vérité est-elle affaire de croyance ou de savoir ? \\[0pt]
La vérité est-elle objective ? \\[0pt]
La vérité se communique-t-elle ? \\[0pt]
La vertu \\[0pt]
La vertu du citoyen \\[0pt]
L'aveu \\[0pt]
La vie de la cité \\[0pt]
La vie de l'esprit \\[0pt]
La vie des rêves \\[0pt]
La vie est-elle objet de science ? \\[0pt]
La vie est-elle un songe ? \\[0pt]
La vie moderne \\[0pt]
La vie psychique \\[0pt]
La ville \\[0pt]
La violence peut-elle être gratuite ? \\[0pt]
La vitesse \\[0pt]
La volonté peut-elle être générale ? \\[0pt]
La vraie vie \\[0pt]
La vue et l'ouïe \\[0pt]
La vulgarité \\[0pt]
Le bavardage \\[0pt]
Le beau naturel \\[0pt]
Le besoin \\[0pt]
Le besoin de philosophie \\[0pt]
Le besoin de signes \\[0pt]
Le bien commun \\[0pt]
Le bien et le mal \\[0pt]
Le bien-être \\[0pt]
Le bien public \\[0pt]
L'éblouissement \\[0pt]
Le bon Dieu \\[0pt]
Le bon goût \\[0pt]
Le bon gouvernement \\[0pt]
Le bonheur est-il affaire de vertu ? \\[0pt]
Le bonheur peut-il être un objectif politique ? \\[0pt]
Le bon régime \\[0pt]
Le bon sens \\[0pt]
Le calcul des plaisirs \\[0pt]
Le cannibalisme \\[0pt]
Le cas de conscience \\[0pt]
Le cas particulier \\[0pt]
Le cerveau pense-t-il ? \\[0pt]
L'échange \\[0pt]
Le chaos \\[0pt]
Le choc esthétique \\[0pt]
Le choix de philosopher \\[0pt]
Le choix d'un métier \\[0pt]
L'éclat \\[0pt]
Le cœur \\[0pt]
L'école de la vie \\[0pt]
Le combat \\[0pt]
Le commencement \\[0pt]
Le commerce \\[0pt]
Le commun \\[0pt]
Le compromis \\[0pt]
Le concept \\[0pt]
Le concept de matière \\[0pt]
Le concept de structure \\[0pt]
Le concret \\[0pt]
L'économie \\[0pt]
L'économie a-t-elle des lois ? \\[0pt]
L'économie des moyens \\[0pt]
L'économique et le politique \\[0pt]
Le conservatisme \\[0pt]
Le contrat \\[0pt]
Le corps et l'esprit \\[0pt]
Le cosmopolitisme \\[0pt]
Le courage \\[0pt]
Le cours des choses \\[0pt]
Le cours du temps \\[0pt]
Le crime contre l'humanité \\[0pt]
Le crime inexpiable \\[0pt]
Le critère \\[0pt]
L'écriture peut-elle porter secours à la pensée ? \\[0pt]
Le cynisme \\[0pt]
Le déchet \\[0pt]
Le délire \\[0pt]
Le démoniaque \\[0pt]
Le dépaysement \\[0pt]
Le désaccord \\[0pt]
Le désespoir \\[0pt]
Le désintéressement \\[0pt]
Le désir de connaître \\[0pt]
Le désir d'égalité \\[0pt]
Le désir de savoir \\[0pt]
Le désir d'être autre \\[0pt]
Le désir de vivre \\[0pt]
Le désir est-il le signe d'un manque ? \\[0pt]
Le désordre \\[0pt]
Le deuil \\[0pt]
Le devoir de mémoire \\[0pt]
Le dialogue entre nations \\[0pt]
Le dictionnaire \\[0pt]
Le Dieu des philosophes \\[0pt]
Le divertissement \\[0pt]
Le divin \\[0pt]
Le don \\[0pt]
Le donné \\[0pt]
Le doute est-il le principe de la méthode scientifique ? \\[0pt]
Le droit à l'erreur \\[0pt]
Le droit au bonheur \\[0pt]
Le droit d'auteur \\[0pt]
Le droit est-il une science ? \\[0pt]
Le Droit et l'État \\[0pt]
Le droit peut-il échapper à l'histoire ? \\[0pt]
L'éducation du goût \\[0pt]
Le fait et le droit \\[0pt]
Le familier \\[0pt]
Le fantastique \\[0pt]
Le fatalisme l'incarnation \\[0pt]
Le faux \\[0pt]
Le faux en art \\[0pt]
L'effectivité \\[0pt]
L'efficacité des discours \\[0pt]
Le fin mot de l'histoire \\[0pt]
Le fondement \\[0pt]
Le fondement de l'autorité \\[0pt]
Le formalisme \\[0pt]
L'égalité \\[0pt]
L'égalité des citoyens \\[0pt]
Légalité et légitimité \\[0pt]
L'égarement \\[0pt]
Le genre humain \\[0pt]
Le goût du risque \\[0pt]
Le gouvernement des meilleurs \\[0pt]
Le grand art est-il de plaire ? \\[0pt]
Le hasard \\[0pt]
Le jeu \\[0pt]
Le juste et le bien \\[0pt]
Le langage de la morale \\[0pt]
Le langage de la science \\[0pt]
Le langage du corps \\[0pt]
Le langage mathématique \\[0pt]
Le langage rapproche-t-il ou sépare-t-il les hommes ? \\[0pt]
L'élégance \\[0pt]
L'élémentaire \\[0pt]
Le lien causal \\[0pt]
Le lien social \\[0pt]
Le loisir \\[0pt]
Le luxe \\[0pt]
Le mal a-t-il des raisons ? \\[0pt]
Le malentendu \\[0pt]
Le marché \\[0pt]
Le mariage est-il un contrat ? \\[0pt]
Le matérialisme \\[0pt]
Le matériel et le virtuel \\[0pt]
Le mauvais goût \\[0pt]
Le méchant est-il malheureux ? \\[0pt]
Le même et l'autre \\[0pt]
Le mensonge est-il la plus grande transgression ? \\[0pt]
Le mépris \\[0pt]
Le mépris des idées \\[0pt]
Le mérite \\[0pt]
Le métier de philosophe \\[0pt]
Le métier de savant \\[0pt]
Le métier d'historien \\[0pt]
Le mieux est-il l'ennemi du bien ? \\[0pt]
Le milieu \\[0pt]
Le moi \\[0pt]
Le moi est-il objet de connaissance ? \\[0pt]
Le moindre mal \\[0pt]
Le moi n'est-il qu'une fiction ? \\[0pt]
Le monde a-t-il besoin de moi ? \\[0pt]
Le monde des images \\[0pt]
Le monde est-il une marchandise ? \\[0pt]
Le monde extérieur \\[0pt]
Le monde intelligible \\[0pt]
Le monde sensible \\[0pt]
L'empathie est-elle possible ? \\[0pt]
L'empirisme \\[0pt]
L'empirisme exclut-il l'abstraction ? \\[0pt]
Le musée \\[0pt]
Le mystère \\[0pt]
Le mysticisme \\[0pt]
Le naturel \\[0pt]
L'Encyclopédie \\[0pt]
Le néant \\[0pt]
Le négatif \\[0pt]
L'énergie du désespoir \\[0pt]
L'enfance de l'art \\[0pt]
L'enfance est-elle ce qui doit être surmonté ? \\[0pt]
L'enfant \\[0pt]
« L'enfer est pavé de bonnes intentions » \\[0pt]
L'énigme \\[0pt]
L'ennui \\[0pt]
Le nombre \\[0pt]
Le normal et le pathologique \\[0pt]
L'enquête \\[0pt]
L'enthousiasme \\[0pt]
L'environnement \\[0pt]
L'environnement est-il un problème politique ? \\[0pt]
Le paradoxe \\[0pt]
Le passage à l'acte \\[0pt]
Le passé \\[0pt]
Le passé est-il perdu ? \\[0pt]
Le pathologique \\[0pt]
Le patrimoine de l'humanité \\[0pt]
Le patriotisme est-il une vertu ? \\[0pt]
Le paysage \\[0pt]
Le philosophe et le sophiste \\[0pt]
Le plaisir \\[0pt]
Le plaisir esthétique \\[0pt]
Le plaisir et la douleur \\[0pt]
Le plaisir et la jouissance \\[0pt]
Le pluralisme \\[0pt]
Le poétique \\[0pt]
Le poids du passé \\[0pt]
Le poids du souvenir \\[0pt]
Le point de vue \\[0pt]
Le point de vue d'autrui \\[0pt]
Le portrait \\[0pt]
Le possible \\[0pt]
Le possible et le réel \\[0pt]
Le possible et le virtuel \\[0pt]
Le possible existe-t-il ? \\[0pt]
Le pouvoir de l'habitude \\[0pt]
Le pouvoir de l'imagination \\[0pt]
Le pouvoir des images \\[0pt]
Le pouvoir des mots \\[0pt]
Le pouvoir des paroles \\[0pt]
Le pouvoir du concept \\[0pt]
Le pouvoir du peuple \\[0pt]
Le pouvoir et la violence \\[0pt]
Le pouvoir magique \\[0pt]
Le pragmatisme \\[0pt]
Le présent \\[0pt]
Le principe \\[0pt]
Le principe de non-contradiction \\[0pt]
Le principe de réalité \\[0pt]
Le prix des choses \\[0pt]
Le probable \\[0pt]
Le procès d'intention \\[0pt]
Le progrès \\[0pt]
Le progrès est-il réversible ? \\[0pt]
Le propre de l'homme \\[0pt]
Le provisoire \\[0pt]
Le public \\[0pt]
Le public et le privé \\[0pt]
Le quelconque \\[0pt]
L'équité \\[0pt]
Le quotidien \\[0pt]
Le racisme \\[0pt]
Le raffinement \\[0pt]
Le rationalisme \\[0pt]
Le réel est-il rationnel ? \\[0pt]
Le regard \\[0pt]
Le relativisme \\[0pt]
Le religieux est-il inutile ? \\[0pt]
Le respect \\[0pt]
Le respect de la tradition \\[0pt]
Le ridicule \\[0pt]
Le rite \\[0pt]
Le roman \\[0pt]
Le roman peut-il être philosophique ? \\[0pt]
Le romantisme \\[0pt]
L'érotisme \\[0pt]
Le rythme \\[0pt]
Le sacré et le profane \\[0pt]
Le sacrifice \\[0pt]
Les âges de la vie \\[0pt]
Les Anciens et les Modernes \\[0pt]
Les animaux pensent-ils ? \\[0pt]
Le savoir-faire \\[0pt]
Les beaux-arts \\[0pt]
Les belles choses \\[0pt]
Les bêtes travaillent-elles ? \\[0pt]
Les bienfaits de la coopération \\[0pt]
Les biotechnologies \\[0pt]
Les bonnes intentions \\[0pt]
Le scandale \\[0pt]
Les catastrophes \\[0pt]
Les catégories \\[0pt]
Le scepticisme a-t-il des limites ? \\[0pt]
Les choses \\[0pt]
Les cinq sens \\[0pt]
Les conditions du dialogue \\[0pt]
Les conséquences \\[0pt]
Les disciplines scientifiques et leurs interfaces \\[0pt]
Les droits de l'homme \\[0pt]
Les droits de l'homme sont-ils une abstraction ? \\[0pt]
Les droits des animaux \\[0pt]
Le secret \\[0pt]
Les enfants \\[0pt]
Le sens commun \\[0pt]
Le sens de la situation \\[0pt]
Le sens du travail \\[0pt]
Le sensible et la science \\[0pt]
Le sens interne \\[0pt]
Le sens moral \\[0pt]
Le sentiment de culpabilité \\[0pt]
Le sentiment d'injustice \\[0pt]
Le service de l'État \\[0pt]
« Le seul problème philosophique vraiment sérieux, c'est le suicide » \\[0pt]
Les faits parlent-ils d'eux-mêmes ? \\[0pt]
Les fausses sciences \\[0pt]
Les fins de l'art \\[0pt]
Les fins de la science \\[0pt]
Les frontières \\[0pt]
Les fruits du travail \\[0pt]
Les genres de vie \\[0pt]
Les héros \\[0pt]
Les hommes sont-ils frères ? \\[0pt]
Les idées fixes \\[0pt]
Les idées ont-elles une histoire ? \\[0pt]
Le silence a-t-il un sens ? \\[0pt]
Les images empêchent-elles de penser ? \\[0pt]
Les inégalités de la nature doivent-elles être compensées ? \\[0pt]
Les langues que nous parlons sont-elles imparfaites ? \\[0pt]
Les lettres et les sciences \\[0pt]
Les limites de la démocratie \\[0pt]
Les limites de la tolérance \\[0pt]
Les limites de l'obéissance \\[0pt]
Les limites du langage \\[0pt]
Les lois de la guerre \\[0pt]
Les lois de la pensée \\[0pt]
Les lois et les armes \\[0pt]
Les machines \\[0pt]
Les mathématiques et la quantité \\[0pt]
Les mathématiques et l'expérience \\[0pt]
Les mathématiques ont-elles affaire au réel ? \\[0pt]
Les mathématiques sont-elles un jeu de l'esprit ? \\[0pt]
Les méchants peuvent-ils faire société ? \\[0pt]
Les mondes possibles \\[0pt]
Les morts \\[0pt]
Les mots nous éloignent-ils des choses ? \\[0pt]
Le soin \\[0pt]
Le soldat \\[0pt]
Les opérations de l'esprit \\[0pt]
Le sophiste et le philosophe \\[0pt]
Le souci \\[0pt]
Le souci de soi \\[0pt]
Le soupçon \\[0pt]
Le souvenir \\[0pt]
L'espace public \\[0pt]
Les passions sont-elles toutes bonnes ? \\[0pt]
L'espèce et l'individu \\[0pt]
L'espèce humaine \\[0pt]
Le spectacle de la nature \\[0pt]
Le spectacle du monde \\[0pt]
L'espérance est-elle une mauvaise passion ? \\[0pt]
Les phénomènes \\[0pt]
Le spirituel et le temporel \\[0pt]
L'espoir peut-il être raisonnable ? \\[0pt]
Le sport : s'accomplir ou se dépasser ? \\[0pt]
Les preuves de l'existence de Dieu \\[0pt]
L'esprit critique \\[0pt]
L'esprit de corps \\[0pt]
L'esprit du christianisme \\[0pt]
L'esprit est-il une chose ? \\[0pt]
L'esprit et la lettre \\[0pt]
L'esprit peut-il être divisé ? \\[0pt]
L'esprit scientifique \\[0pt]
Les raisons du choix \\[0pt]
Les rapports entre les hommes sont-ils des rapports de force ? \\[0pt]
Les robots \\[0pt]
Les sauvages \\[0pt]
Les sciences appliquées \\[0pt]
Les sciences humaines traitent-elles de l'homme ? \\[0pt]
Les sciences nous donnent-elles des normes ? \\[0pt]
Les sciences sociales ont-elles un objet ? \\[0pt]
L'essence de la technique \\[0pt]
Les sentiments \\[0pt]
Les techniques du corps \\[0pt]
Les témoignages et la preuve \\[0pt]
L'estime de soi \\[0pt]
Le style \\[0pt]
Le style et le beau \\[0pt]
Le sublime \\[0pt]
Le substitut \\[0pt]
Le suffrage universel \\[0pt]
Le sujet \\[0pt]
Les vivants et les morts \\[0pt]
Le symbole \\[0pt]
Le tableau \\[0pt]
L'état de guerre \\[0pt]
L'État doit-il nous rendre meilleurs ? \\[0pt]
L'État est-il un arbitre ? \\[0pt]
L'État mondial \\[0pt]
Le témoignage des sens \\[0pt]
Le temps de l'histoire \\[0pt]
L'éternité \\[0pt]
Le territoire \\[0pt]
Le théâtral \\[0pt]
Le théâtre de l'histoire \\[0pt]
L'étonnement \\[0pt]
Le toucher \\[0pt]
Le tragique \\[0pt]
Le trait d'esprit \\[0pt]
L'étranger \\[0pt]
L'étrangeté \\[0pt]
Le travail de la raison \\[0pt]
« Le travail rend libre » \\[0pt]
L'étude \\[0pt]
L'Europe \\[0pt]
L'évaluation \\[0pt]
L'évasion \\[0pt]
L'événement \\[0pt]
Le vestige \\[0pt]
L'évidence \\[0pt]
L'évidence est-elle critère de vérité ? \\[0pt]
Le virtuel \\[0pt]
L'évolution \\[0pt]
Le voyage \\[0pt]
Le vraisemblable \\[0pt]
Le vraisemblable et le romanesque \\[0pt]
L'exactitude \\[0pt]
L'exception \\[0pt]
L'exception est-elle instructive ? \\[0pt]
L'excès \\[0pt]
L'exemple en morale \\[0pt]
L'exercice de la vertu \\[0pt]
L'exercice de la volonté \\[0pt]
L'existence est-elle pensable ? \\[0pt]
L'expérience de la beauté \\[0pt]
L'expérience de la liberté \\[0pt]
L'expérience de la maladie \\[0pt]
L'expérience de la vie \\[0pt]
L'expérience de pensée \\[0pt]
L'expérience du danger \\[0pt]
L'expérience du mal \\[0pt]
L'expérience et l'expérimentation \\[0pt]
L'expérience nous apprend-elle quelque chose ? \\[0pt]
L'expertise \\[0pt]
L'expression \\[0pt]
L'habitude \\[0pt]
L'harmonie \\[0pt]
L'hérésie \\[0pt]
L'héritage \\[0pt]
L'héroïsme \\[0pt]
L'histoire a-t-elle des lois ? \\[0pt]
L'histoire de l'art peut-elle arriver à son terme ? \\[0pt]
L'histoire est-elle écrite par les vainqueurs ? \\[0pt]
« L'histoire jugera » \\[0pt]
L'histoire naturelle \\[0pt]
L'histoire n'est-elle qu'un récit ? \\[0pt]
L'histoire se répète-t-elle ? \\[0pt]
L'historicité des sciences \\[0pt]
L'homme de l'art \\[0pt]
L'homme est-il un animal politique ? \\[0pt]
L'homme est-il un animal social ? \\[0pt]
L'homme et la machine \\[0pt]
L'homme et le citoyen \\[0pt]
L'homme intérieur \\[0pt]
L'homme pense-t-il toujours ? \\[0pt]
L'homme peut-il être libéré du besoin ? \\[0pt]
L'homme peut-il se représenter un monde sans l'homme ? \\[0pt]
L'homme se reconnaît-il mieux dans le travail ou dans le loisir ? \\[0pt]
L'honneur \\[0pt]
L'horizon \\[0pt]
L'hospitalité est-elle un devoir ? \\[0pt]
L'humain \\[0pt]
L'humanité a-t-elle eu une enfance ? \\[0pt]
L'humour \\[0pt]
L'humour et l'ironie \\[0pt]
Libéral et libertaire \\[0pt]
Libéralité et libéralisme \\[0pt]
Liberté et libération \\[0pt]
L'idéal \\[0pt]
L'idéal systématique \\[0pt]
L'idée de communisme \\[0pt]
L'idée de crise \\[0pt]
L'idée de culpabilité collective \\[0pt]
L'idée de destin a-t-elle un sens ? \\[0pt]
L'idée de destin est-elle une idée périmée ? \\[0pt]
L'idée de justice \\[0pt]
L'idée de loi de la nature \\[0pt]
L'idée de mal nécessaire \\[0pt]
L'idée de métier \\[0pt]
L'idée de monde \\[0pt]
L'idée de science \\[0pt]
L'idée de « sciences exactes » \\[0pt]
L'idée d'Europe \\[0pt]
L'idée d'univers \\[0pt]
L'idée d'université \\[0pt]
L'identification \\[0pt]
L'identité \\[0pt]
L'identité personnelle \\[0pt]
L'idéologie \\[0pt]
L'illimité \\[0pt]
L'illusion \\[0pt]
L'image et le réel \\[0pt]
L'imagination a-t-elle des limites ? \\[0pt]
L'imitation \\[0pt]
L'immédiat \\[0pt]
L'immensité \\[0pt]
L'impardonnable \\[0pt]
L'impassibilité \\[0pt]
L'impatience \\[0pt]
L'impératif d'impartialité \\[0pt]
L'impératif hypothétique \\[0pt]
L'impersonnel \\[0pt]
L'implicite \\[0pt]
L'imprévisible \\[0pt]
L'impunité \\[0pt]
L'inaccessible \\[0pt]
L'incertitude \\[0pt]
L'inconnaissable \\[0pt]
L'inconnu \\[0pt]
L'inconscient \\[0pt]
L'inconscient collectif \\[0pt]
L'inconscient est-il un concept scientifique ? \\[0pt]
L'incroyable \\[0pt]
L'indécence \\[0pt]
L'indécision \\[0pt]
L'indémontrable \\[0pt]
L'indépendance \\[0pt]
L'indéterminé \\[0pt]
L'indicible \\[0pt]
L'indifférence \\[0pt]
L'individualisme \\[0pt]
L'induction \\[0pt]
L'inéluctable \\[0pt]
L'infini \\[0pt]
L'information \\[0pt]
L'ingénuité \\[0pt]
L'inhumain \\[0pt]
L'inimaginable \\[0pt]
L'injustifiable \\[0pt]
L'innocence \\[0pt]
L'innommable \\[0pt]
L'inquiétude \\[0pt]
L'insignifiant \\[0pt]
L'insolite \\[0pt]
L'instant \\[0pt]
L'instinct \\[0pt]
L'institution \\[0pt]
L'instrument \\[0pt]
L'intelligence \\[0pt]
L'interaction \\[0pt]
L'interdisciplinarité \\[0pt]
L'interdit \\[0pt]
L'intérêt \\[0pt]
L'intériorité \\[0pt]
L'interprétation \\[0pt]
L'interprétation est-elle une science ? \\[0pt]
L'intimité \\[0pt]
L'intolérable \\[0pt]
L'introspection \\[0pt]
L'intuition \\[0pt]
L'inutile \\[0pt]
L'invisible \\[0pt]
L'ironie \\[0pt]
L'irrationnel \\[0pt]
L'irréfutable \\[0pt]
L'irréparable \\[0pt]
L'irrésolution \\[0pt]
L'irréversibilité \\[0pt]
L'irréversible \\[0pt]
L'ivresse \\[0pt]
L'objection de conscience \\[0pt]
L'objectivation \\[0pt]
L'objectivité \\[0pt]
L'objet de l'amour \\[0pt]
L'objet des mathématiques \\[0pt]
L'obligation \\[0pt]
L'observation \\[0pt]
L'occasion \\[0pt]
L'omniscience \\[0pt]
L'opinion publique \\[0pt]
L'opposition \\[0pt]
L'optimisme \\[0pt]
L'ordinaire \\[0pt]
L'ordre \\[0pt]
L'ordre du monde \\[0pt]
L'ordre est-il dans les choses ? \\[0pt]
L'ordre public \\[0pt]
L'organisation du travail \\[0pt]
L'orgueil \\[0pt]
L'orientation \\[0pt]
L'Orient et l'Occident \\[0pt]
L'origine de la violence \\[0pt]
L'origine du droit \\[0pt]
L'oubli \\[0pt]
L'outil \\[0pt]
L'un \\[0pt]
L'un et le multiple \\[0pt]
L'uniformité \\[0pt]
L'unité de la science \\[0pt]
L'unité des arts \\[0pt]
L'unité du genre humain \\[0pt]
L'universalisme \\[0pt]
L'universel \\[0pt]
L'usage \\[0pt]
L'utile \\[0pt]
Maître et disciple \\[0pt]
Maladie et convalescence \\[0pt]
Maladies du corps, maladies de l'âme \\[0pt]
Ma parole m'engage-t-elle ? \\[0pt]
Masculin, féminin \\[0pt]
Médecine et philosophie \\[0pt]
Méditer \\[0pt]
Mesurer le temps \\[0pt]
Métaphysique et politique \\[0pt]
Mon corps \\[0pt]
Mon corps m'appartient-il ? \\[0pt]
Morale et calcul \\[0pt]
Morale et intérêt \\[0pt]
Morale et technique \\[0pt]
Moralité et connaissance \\[0pt]
Mythe et pensée \\[0pt]
Naître \\[0pt]
Nature, monde, univers \\[0pt]
Nécessité et contingence \\[0pt]
Ne pas rire, ne pas pleurer, mais comprendre \\[0pt]
Nos pensées dépendent-elles de nous ? \\[0pt]
« Nul n'est censé ignorer la loi » \\[0pt]
Nul n'est méchant volontairement \\[0pt]
N'y a-t-il d'amitié qu'entre égaux ? \\[0pt]
N'y a-t-il de certitude que mathématique ? \\[0pt]
N'y a-t-il de propriété que privée ? \\[0pt]
N'y a-t-il de science que de ce qui est mathématisable ? \\[0pt]
N'y a-t-il de science que du mesurable ? \\[0pt]
N'y a-t-il que des individus ? \\[0pt]
Objectiver \\[0pt]
Observation et expérimentation \\[0pt]
Observer \\[0pt]
Œil pour œil, dent pour dent \\[0pt]
Origine et fondement \\[0pt]
Où commence la violence ? \\[0pt]
Parler, est-ce agir ? \\[0pt]
Par où commencer ? \\[0pt]
« Pas de liberté pour les ennemis de la liberté » ? \\[0pt]
« Pauvre bête » \\[0pt]
Peinture et histoire \\[0pt]
Penser, est-ce calculer ? \\[0pt]
Penser et raisonner \\[0pt]
Penser l'impossible \\[0pt]
Penser sans sujet \\[0pt]
Percevoir et juger \\[0pt]
Perdre son âme \\[0pt]
Perdre son temps \\[0pt]
Personne et individu \\[0pt]
Personne n'est innocent \\[0pt]
Peut-il y avoir une histoire universelle ? \\[0pt]
Peut-il y avoir une philosophie appliquée ? \\[0pt]
Peut-il y avoir une science de l'éducation ? \\[0pt]
Peut-il y avoir un État mondial ? \\[0pt]
Peut-on aimer ce qu'on ne connaît pas ? \\[0pt]
Peut-on aimer son travail ? \\[0pt]
Peut-on apprendre à être juste ? \\[0pt]
Peut-on apprendre à penser ? \\[0pt]
Peut-on avoir conscience de soi sans avoir conscience d'autrui ? \\[0pt]
Peut-on avoir peur de soi-même ? \\[0pt]
Peut-on avoir raison contre tous ? \\[0pt]
Peut-on avoir raison tout.e seul.e ? \\[0pt]
Peut-on changer le monde ? \\[0pt]
Peut-on contester les droits de l'homme ? \\[0pt]
Peut-on critiquer la démocratie ? \\[0pt]
Peut-on décider de croire ? \\[0pt]
Peut-on définir la vie ? \\[0pt]
Peut-on désirer ce qu'on ne veut pas ? \\[0pt]
Peut-on désirer l'impossible ? \\[0pt]
Peut-on douter de sa propre existence ? \\[0pt]
Peut-on douter de tout ? \\[0pt]
Peut-on écrire comme on parle ? \\[0pt]
Peut-on être en avance sur son temps ? \\[0pt]
Peut-on être heureux dans la solitude ? \\[0pt]
Peut-on être indifférent à son bonheur ? \\[0pt]
Peut-on être responsable de ce que l'on n'a pas fait ? \\[0pt]
Peut-on exercer son esprit ? \\[0pt]
Peut-on expliquer une œuvre d'art ? \\[0pt]
Peut-on faire la paix ? \\[0pt]
Peut-on forcer un homme à être libre ? \\[0pt]
Peut-on gâcher son talent ? \\[0pt]
Peut-on haïr la raison ? \\[0pt]
Peut-on inventer en morale ? \\[0pt]
Peut-on légitimer la violence ? \\[0pt]
Peut-on limiter l'expression de la volonté du peuple ? \\[0pt]
Peut-on manipuler les esprits ? \\[0pt]
Peut-on manquer de volonté ? \\[0pt]
Peut-on ne pas croire ? \\[0pt]
Peut-on ne pas être égoïste ? \\[0pt]
Peut-on ne pas savoir ce que l'on dit ? \\[0pt]
Peut-on ne pas savoir ce que l'on fait ? \\[0pt]
Peut-on ne penser à rien ? \\[0pt]
Peut-on nier l'évidence ? \\[0pt]
Peut-on oublier de vivre ? \\[0pt]
Peut-on parler de corruption des mœurs ? \\[0pt]
Peut-on parler de droits des animaux ? \\[0pt]
Peut-on parler de « travail intellectuel » ? \\[0pt]
Peut-on parler de travail intellectuel ? \\[0pt]
Peut-on parler d'une expérience religieuse ? \\[0pt]
Peut-on parler pour ne rien dire ? \\[0pt]
Peut-on penser la nouveauté ? \\[0pt]
Peut-on penser le changement ? \\[0pt]
Peut-on penser l'impossible ? \\[0pt]
Peut-on penser sans images ? \\[0pt]
Peut-on penser sans les signes ? \\[0pt]
Peut-on penser un art sans œuvres ? \\[0pt]
Peut-on penser une société sans État ? \\[0pt]
Peut-on perdre la raison ? \\[0pt]
Peut-on perdre son identité ? \\[0pt]
Peut-on prouver l'existence du monde ? \\[0pt]
Peut-on raconter sa vie ? \\[0pt]
Peut-on raisonner sans règles ? \\[0pt]
Peut-on refuser de voir la vérité ? \\[0pt]
Peut-on refuser la loi ? \\[0pt]
Peut-on représenter le peuple ? \\[0pt]
Peut-on rester dans le doute ? \\[0pt]
Peut-on revenir sur ses erreurs ? \\[0pt]
Peut-on s'attendre à tout ? \\[0pt]
Peut-on savoir quelque chose de l'avenir ? \\[0pt]
Peut-on se désintéresser de la politique ? \\[0pt]
Peut-on se désintéresser de son bonheur ? \\[0pt]
Peut-on se fier à l'intuition ? \\[0pt]
Peut-on se fier à son intuition ? \\[0pt]
Peut-on se méfier de soi-même ? \\[0pt]
Peut-on se mentir à soi-même ? \\[0pt]
Peut-on se mettre à la place de l'autre ? \\[0pt]
Peut-on séparer politique et économie ? \\[0pt]
Peut-on se passer de croyances ? \\[0pt]
Peut-on se passer de mythes ? \\[0pt]
Peut-on se passer de religion ? \\[0pt]
Peut-on se vouloir parfait ? \\[0pt]
Peut-on suspendre son jugement ? \\[0pt]
Peut-on tout définir ? \\[0pt]
Peut-on tout démontrer ? \\[0pt]
Peut-on tout dire ? \\[0pt]
Peut-on tout enseigner ? \\[0pt]
Peut-on tout expliquer ? \\[0pt]
Peut-on tout imaginer ? \\[0pt]
Peut-on tout imiter ? \\[0pt]
Peut-on tout interpréter ? \\[0pt]
Peut-on tout mesurer ? \\[0pt]
Peut-on tout partager ? \\[0pt]
Peut-on tout tolérer ? \\[0pt]
Peut-on vivre sans illusions ? \\[0pt]
Peut-on vivre sans passion ? \\[0pt]
Peut-on vouloir le mal ? \\[0pt]
Philosophie et poésie \\[0pt]
Physique et métaphysique \\[0pt]
Pluralisme et politique \\[0pt]
Poésie et vérité \\[0pt]
Poétique et prosaïque \\[0pt]
Police et politique \\[0pt]
Politique et coopération \\[0pt]
Politique et unité \\[0pt]
« Pourquoi » \\[0pt]
Pourquoi aimons-nous la musique ? \\[0pt]
Pourquoi conserver les œuvres d'art ? \\[0pt]
Pourquoi construire des monuments ? \\[0pt]
Pourquoi critiquer la raison ? \\[0pt]
Pourquoi défendre le faible ? \\[0pt]
Pourquoi des artifices ? \\[0pt]
Pourquoi des fêtes ? \\[0pt]
Pourquoi des guerres ? \\[0pt]
Pourquoi des institutions ? \\[0pt]
Pourquoi désirer l'immortalité ? \\[0pt]
Pourquoi des maîtres ? \\[0pt]
Pourquoi des musées ? \\[0pt]
Pourquoi des philosophes ? \\[0pt]
Pourquoi des poètes ? \\[0pt]
Pourquoi des psychologues ? \\[0pt]
Pourquoi des sociologues ? \\[0pt]
Pourquoi écrit-on les lois ? \\[0pt]
Pourquoi exposer les œuvres d'art ? \\[0pt]
Pourquoi imiter ? \\[0pt]
Pourquoi le théâtre ? \\[0pt]
Pourquoi l'homme est-il l'objet de plusieurs sciences ? \\[0pt]
Pourquoi mentir ? \\[0pt]
Pourquoi parler de « sciences exactes » ? \\[0pt]
Pourquoi parler du travail comme d'un droit ? \\[0pt]
Pourquoi parle-t-on d'économie politique ? \\[0pt]
Pourquoi préférer l'original à la reproduction ? \\[0pt]
Pourquoi prouver l'existence de Dieu ? \\[0pt]
Pourquoi punir ? \\[0pt]
Pourquoi punit-on ? \\[0pt]
Pourquoi rit-on ? \\[0pt]
Pourquoi s'étonner ? \\[0pt]
Pourquoi travaille-t-on ? \\[0pt]
Pourquoi voulons-nous savoir ? \\[0pt]
Pourquoi voyager ? \\[0pt]
Pourquoi y a-t-il quelque chose plutôt que rien ? \\[0pt]
Pourquoi y a-t-il une philosophie de la vie ? \\[0pt]
Pour vivre heureux, vivons cachés \\[0pt]
Pouvoir, magie, secret \\[0pt]
Pouvons-nous être certains que nous ne rêvons pas ? \\[0pt]
Pouvons-nous savoir ce que nous ignorons ? \\[0pt]
Prédiction et prévision \\[0pt]
Prédiction et probabilité \\[0pt]
Prendre soin \\[0pt]
Prévoir \\[0pt]
Principes et stratégie \\[0pt]
Production et création \\[0pt]
Produire et créer \\[0pt]
Puis-je aimer tous les hommes ? \\[0pt]
Puis-je être libre tout seul ? \\[0pt]
Puis-je être sûr de ne pas me tromper ? \\[0pt]
Puis-je être sûr que je ne rêve pas ? \\[0pt]
Quand est-on stupide ? \\[0pt]
Quand faut-il désobéir ? \\[0pt]
Quand faut-il désobéir aux lois ? \\[0pt]
Quand le temps passe, que reste-t-il ? \\[0pt]
Qu'appelle-t-on destin ? \\[0pt]
Qu'appelle-t-on penser ? \\[0pt]
Qu'apprend-on dans les livres ? \\[0pt]
Qu'apprend-on de ses erreurs ? \\[0pt]
Qu'apprend-on quand on apprend à parler ? \\[0pt]
Qu'a-t-on le droit d'exiger ? \\[0pt]
Que coûte une victoire ? \\[0pt]
Que désire-t-on ? \\[0pt]
Que disent les légendes ? \\[0pt]
Que dit la musique ? \\[0pt]
Que doit-on savoir avant d'agir ? \\[0pt]
Que faire de la diversité des arts ? \\[0pt]
Que faire de notre cerveau ? \\[0pt]
Que fait la police ? \\[0pt]
Que faut-il pour faire un monde ? \\[0pt]
Que faut-il savoir pour agir ? \\[0pt]
Quel est le fondement de la propriété ? \\[0pt]
Quel est le pouvoir des mots ? \\[0pt]
Quel est le rôle du concept en art ? \\[0pt]
Quel est l'homme des Droits de l'homme ? \\[0pt]
Quel est l'objet des mathématiques ? \\[0pt]
Quel est l'objet des sciences humaines ? \\[0pt]
Quelle est la spécificité de la communauté politique ? \\[0pt]
Quelle est la valeur de l'expérience ? \\[0pt]
Quelle idée le fanatique se fait-il de la vérité ? \\[0pt]
Quelles sont les limites de la démonstration ? \\[0pt]
Quels sont les droits de la conscience ? \\[0pt]
Quel usage faut-il faire des exemples ? \\[0pt]
Que nous apprend la grammaire ? \\[0pt]
Que nous apprend l'étude du cerveau ? \\[0pt]
Que nous apprennent les expériences de pensée ? \\[0pt]
Que nous apprennent les jeux ? \\[0pt]
Que nous apprennent les langues étrangères ? \\[0pt]
Que nous apprennent les mythes ? \\[0pt]
Que nous enseignent nos peurs ? \\[0pt]
Que nous montre le cinéma ? \\[0pt]
Que nous montrent les natures mortes ? \\[0pt]
Que penser de la formule : « il faut suivre la nature » ? \\[0pt]
Que penser de l'opposition travail manuel, travail intellectuel ? \\[0pt]
Que peut la philosophie ? \\[0pt]
Que prouvent les faits ? \\[0pt]
Que signifie « donner le change » ? \\[0pt]
Qu'est-ce qu'avoir de l'expérience ? \\[0pt]
Qu'est-ce que comprendre ? \\[0pt]
Qu'est-ce que créer ? \\[0pt]
Qu'est-ce que Dieu pour athée ? \\[0pt]
Qu'est-ce qu'éduquer ? \\[0pt]
Qu'est-ce que guérir ? \\[0pt]
Qu'est-ce que juger ? \\[0pt]
Qu'est-ce que la critique ? \\[0pt]
Qu'est-ce que la politique ? \\[0pt]
Qu'est-ce que la science, si elle inclut la psychanalyse ? \\[0pt]
Qu'est-ce que la scientificité ? \\[0pt]
Qu'est-ce que la technique ? \\[0pt]
Qu'est-ce que la vie ? \\[0pt]
Qu'est-ce que le cinéma donne à voir ? \\[0pt]
Qu'est-ce que le dogmatisme ? \\[0pt]
Qu'est-ce que le moi ? \\[0pt]
Qu'est-ce que le nihilisme ? \\[0pt]
Qu'est-ce que le pathologique nous apprend sur le normal ? \\[0pt]
Qu'est-ce que le réel ? \\[0pt]
Qu'est-ce que parler ? \\[0pt]
Qu'est-ce que penser ? \\[0pt]
Qu'est-ce que perdre sa liberté ? \\[0pt]
Qu'est-ce que réfuter ? \\[0pt]
Qu'est-ce que résister ? \\[0pt]
Qu'est-ce que rester soi-même ? \\[0pt]
Qu'est-ce qu'être dans le vrai ? \\[0pt]
Qu'est-ce qu'être moderne ? \\[0pt]
Qu'est-ce qu'être normal ? \\[0pt]
Qu'est-ce qu'être réaliste ? \\[0pt]
Qu'est-ce qu'être souverain ? \\[0pt]
Qu'est-ce qu'être un sujet ? \\[0pt]
Qu'est-ce qu'être vivant ? \\[0pt]
Qu'est-ce que vérifier ? \\[0pt]
Qu'est-ce que vivre ? \\[0pt]
Qu'est-ce qui est concret ? \\[0pt]
Qu'est-ce qui est essentiel ? \\[0pt]
Qu'est-ce qui est indiscutable ? \\[0pt]
Qu'est-ce qui est injuste ? \\[0pt]
Qu'est-ce qui est insignifiant ? \\[0pt]
Qu'est-ce qui est le plus à craindre, l'ordre ou le désordre ? \\[0pt]
Qu'est-ce qui est naturel ? \\[0pt]
Qu'est-ce qui est tragique ? \\[0pt]
Qu'est-ce qui fait autorité ? \\[0pt]
Qu'est-ce qui ne disparaît jamais ? \\[0pt]
Qu'est-ce qu'on attend ? \\[0pt]
Qu'est-ce qu'un acte libre ? \\[0pt]
Qu'est-ce qu'un ami ? \\[0pt]
Qu'est-ce qu'un animal ? \\[0pt]
Qu'est-ce qu'un artiste ? \\[0pt]
Qu'est-ce qu'un axiome ? \\[0pt]
Qu'est-ce qu'un chef ? \\[0pt]
Qu'est-ce qu'un chef-d'œuvre ? \\[0pt]
Qu'est-ce qu'un choix éclairé ? \\[0pt]
Qu'est-ce qu'un classique ? \\[0pt]
Qu'est-ce qu'un concept ? \\[0pt]
Qu'est-ce qu'un concept philosophique ? \\[0pt]
Qu'est-ce qu'un contrat ? \\[0pt]
Qu'est-ce qu'un contre-pouvoir ? \\[0pt]
Qu'est-ce qu'un coup d'État ? \\[0pt]
Qu'est-ce qu'un dieu ? \\[0pt]
Qu'est-ce qu'un dilemme ? \\[0pt]
Qu'est-ce qu'un dogme ? \\[0pt]
Qu'est-ce qu'une action intentionnelle ? \\[0pt]
Qu'est-ce qu'une belle mort ? \\[0pt]
Qu'est-ce qu'une bonne traduction ? \\[0pt]
Qu'est-ce qu'une connaissance par les faits ? \\[0pt]
Qu'est-ce qu'une crise ? \\[0pt]
Qu'est-ce qu'une croyance ? \\[0pt]
Qu'est-ce qu'une découverte ? \\[0pt]
Qu'est-ce qu'une éducation réussie ? \\[0pt]
Qu'est-ce qu'une époque ? \\[0pt]
Qu'est-ce qu'une expérience religieuse ? \\[0pt]
Qu'est-ce qu'une famille ? \\[0pt]
Qu'est-ce qu'une fiction ? \\[0pt]
Qu'est-ce qu'une grande cause ? \\[0pt]
Qu'est-ce qu'une hypothèse ? \\[0pt]
Qu'est-ce qu'une idée ? \\[0pt]
Qu'est-ce qu'une illusion ? \\[0pt]
Qu'est-ce qu'une institution ? \\[0pt]
Qu'est-ce qu'une langue morte ? \\[0pt]
Qu'est-ce qu'une libération ? \\[0pt]
Qu'est-ce qu'une loi ? \\[0pt]
Qu'est-ce qu'une loi de la pensée ? \\[0pt]
Qu'est-ce qu'une machine ? \\[0pt]
Qu'est-ce qu'une marchandise ? \\[0pt]
Qu'est-ce qu'un empire ? \\[0pt]
Qu'est-ce qu'une nation ? \\[0pt]
Qu'est-ce qu'une norme ? \\[0pt]
Qu'est-ce qu'une œuvre ? \\[0pt]
Qu'est-ce qu'une œuvre d'art ? \\[0pt]
Qu'est-ce qu'une œuvre ratée ? \\[0pt]
Qu'est-ce qu'une personne ? \\[0pt]
Qu'est-ce qu'une philosophie ? \\[0pt]
Qu'est-ce qu'une preuve ? \\[0pt]
Qu'est-ce qu'une règle ? \\[0pt]
Qu'est-ce qu'une règle de vie ? \\[0pt]
Qu'est-ce qu'une rencontre ? \\[0pt]
Qu'est-ce qu'une révélation ? \\[0pt]
Qu'est-ce qu'une révolution ? \\[0pt]
Qu'est-ce qu'une révolution scientifique ? \\[0pt]
Qu'est-ce qu'une science exacte ? \\[0pt]
Qu'est-ce qu'une société libre ? \\[0pt]
Qu'est-ce qu'un état mental ? \\[0pt]
Qu'est-ce qu'un événement fondateur ? \\[0pt]
Qu'est-ce qu'une vie humaine ? \\[0pt]
Qu'est-ce qu'une ville ? \\[0pt]
Qu'est-ce qu'une volonté raisonnable ? \\[0pt]
Qu'est-ce qu'un exemple ? \\[0pt]
Qu'est-ce qu'un expert ? \\[0pt]
Qu'est-ce qu'un fait ? \\[0pt]
Qu'est-ce qu'un faux problème ? \\[0pt]
Qu'est-ce qu'un génie ? \\[0pt]
Qu'est-ce qu'un grand philosophe ? \\[0pt]
Qu'est-ce qu'un héros ? \\[0pt]
Qu'est-ce qu'un homme sans éducation ? \\[0pt]
Qu'est-ce qu'un homme seul ? \\[0pt]
Qu'est-ce qu'un jeu ? \\[0pt]
Qu'est-ce qu'un lieu commun ? \\[0pt]
Qu'est-ce qu'un livre ? \\[0pt]
Qu'est-ce qu'un maître ? \\[0pt]
Qu'est-ce qu'un monstre ? \\[0pt]
Qu'est-ce qu'un monument ? \\[0pt]
Qu'est-ce qu'un musée ? \\[0pt]
Qu'est-ce qu'un mythe ? \\[0pt]
Qu'est-ce qu'un nombre ? \\[0pt]
Qu'est-ce qu'un ordre ? \\[0pt]
Qu'est-ce qu'un paradoxe ? \\[0pt]
Qu'est-ce qu'un peuple ? \\[0pt]
Qu'est-ce qu'un plaisir pur ? \\[0pt]
Qu'est-ce qu'un principe ? \\[0pt]
Qu'est-ce qu'un problème ? \\[0pt]
Qu'est-ce qu'un problème politique ? \\[0pt]
Qu'est-ce qu'un programmer ? \\[0pt]
Qu'est-ce qu'un réseau ? \\[0pt]
Qu'est-ce qu'un rhéteur ? \\[0pt]
Qu'est-ce qu'un rite ? \\[0pt]
Qu'est-ce qu'un sceptique ? \\[0pt]
Qu'est-ce qu'un signe ? \\[0pt]
Qu'est-ce qu'un sophisme ? \\[0pt]
Qu'est-ce qu'un sophiste ? \\[0pt]
Qu'est-ce qu'un symbole ? \\[0pt]
Qu'est-ce qu'un système ? \\[0pt]
Qu'est-ce qu'un temple ? \\[0pt]
Qu'est-ce qu'un texte ? \\[0pt]
Qu'est-ce qu'un traître ? \\[0pt]
Que valent les préjugés ? \\[0pt]
Que vaut l'excuse : « Je ne l'ai pas fait exprès » ? \\[0pt]
Que vaut une preuve contre un préjugé ? \\[0pt]
Que veut dire avoir raison ? \\[0pt]
Que voit-on dans un miroir \\[0pt]
Qu'expriment les mythes ? \\[0pt]
Qui est compétent en matière politique ? \\[0pt]
Qui est l'autre ? \\[0pt]
Qui est l'homme des sciences humaines ? \\[0pt]
Qui pense ? \\[0pt]
Qui suis-je et qui es-tu ? \\[0pt]
Raisonnement et expérimentation \\[0pt]
Raisonner et calculer \\[0pt]
Reconnaissance et inégalité \\[0pt]
Réforme et révolution \\[0pt]
Refuser et réfuter \\[0pt]
Règle morale et norme juridique \\[0pt]
Religion et violence \\[0pt]
Répondre \\[0pt]
Répondre de soi \\[0pt]
Représenter \\[0pt]
Résister peut-il être un droit ? \\[0pt]
Respecter la nature, est-ce renoncer à l'exploiter ? \\[0pt]
Rêver \\[0pt]
Révolte et révolution \\[0pt]
Rien n'est sans raison \\[0pt]
Rituels et cérémonies \\[0pt]
Sait-on toujours ce que l'on fait ? \\[0pt]
Sait-on toujours ce que l'on veut ? \\[0pt]
« Sauver les apparences » \\[0pt]
Sauver les apparences \\[0pt]
Savoir, est-ce pouvoir ? \\[0pt]
Savoir et pouvoir \\[0pt]
Science et abstraction \\[0pt]
Science et certitude \\[0pt]
Science et domination sociale \\[0pt]
Science et invention \\[0pt]
Science et libération \\[0pt]
Science et magie \\[0pt]
Sciences de la nature et sciences de l'esprit \\[0pt]
Sciences empiriques et critères du vrai \\[0pt]
Sentir et penser \\[0pt]
Si l'État n'existait pas, faudrait-il l'inventer ? \\[0pt]
Société et biologie \\[0pt]
Socrate \\[0pt]
Solitude et isolement \\[0pt]
Sommes-nous maîtres de nos paroles ? \\[0pt]
Substance et sujet \\[0pt]
Suffit-il de bien juger pour bien faire ? \\[0pt]
Suffit-il de voir le meilleur pour le suivre ? \\[0pt]
Suis-je ma mémoire ? \\[0pt]
Suivre la coutume \\[0pt]
Technique et apprentissage \\[0pt]
Témoigner \\[0pt]
Tenir pour vrai \\[0pt]
Théorie et modèle \\[0pt]
Théorie et pratique \\[0pt]
Toucher \\[0pt]
Tous les désirs sont-ils naturels ? \\[0pt]
Tous les paradis sont-ils perdus ? \\[0pt]
Tout a-t-il un prix ? \\[0pt]
Tout comprendre, est-ce tout pardonner ? \\[0pt]
Toute morale implique-t-elle l'effort ? \\[0pt]
Toute origine est-elle mythique ? \\[0pt]
« Toute peine mérite salaire » \\[0pt]
Toute science est-elle naturelle ? \\[0pt]
Toutes les convictions sont-elles respectables ? \\[0pt]
Toutes les fautes se valent-elles ? \\[0pt]
Toutes les inégalités ont-elles une importance politique ? \\[0pt]
Tout est-il affaire de point de vue ? \\[0pt]
Tout est-il politique ? \\[0pt]
« Tout est possible » \\[0pt]
Tout est vanité \\[0pt]
Toute vérité est-elle vérifiable ? \\[0pt]
Tout pouvoir est-il politique ? \\[0pt]
Tout savoir est-il un pouvoir ? \\[0pt]
Traduire \\[0pt]
Traduire et interpréter \\[0pt]
Travail manuel et travail intellectuel \\[0pt]
Travail manuel, travail intellectuel \\[0pt]
Trouver sa voie \\[0pt]
Un art peut-il être populaire ? \\[0pt]
Une existence se démontre-t-elle ? \\[0pt]
Une loi peut-elle être injuste ? \\[0pt]
Une machine peut-elle penser ? \\[0pt]
Une philosophie peut-elle être réactionnaire ? \\[0pt]
Une religion rationnelle est-elle possible ? \\[0pt]
Une science de la conscience est-elle possible ? \\[0pt]
Une science de l'éducation est-elle possible ? \\[0pt]
Une société sans religion est-elle possible ? \\[0pt]
Un objet technique peut-il être beau ? \\[0pt]
Un philosophe a-t-il des devoirs envers la société ? \\[0pt]
Vaincre la mort \\[0pt]
Vaut-il mieux oublier ou pardonner ? \\[0pt]
Vices privés, vertus publiques \\[0pt]
Vie active, vie contemplative \\[0pt]
Vieillir \\[0pt]
Vivre et bien vivre \\[0pt]
Vivre sans loi \\[0pt]
Vivre sa vie \\[0pt]
Vivre selon la nature \\[0pt]
Voir, observer, penser \\[0pt]
Vouloir croire, est-ce possible ? \\[0pt]
Y a-t-il de faux besoins ? \\[0pt]
Y a-t-il de l'impensable ? \\[0pt]
Y a-t-il de l'incommunicable ? \\[0pt]
Y a-t-il de l'universel ? \\[0pt]
Y a-t-il de mauvais désirs ? \\[0pt]
Y a-t-il des actes gratuits ? \\[0pt]
Y a-t-il des actions désintéressées ? \\[0pt]
Y a-t-il des arts mineurs ? \\[0pt]
Y a-t-il des barbares ? \\[0pt]
Y a-t-il des croyances rationnelles ? \\[0pt]
Y a-t-il des despotes éclairés ? \\[0pt]
Y a-t-il des dilemmes moraux ? \\[0pt]
Y a-t-il des droits sans devoirs ? \\[0pt]
Y a-t-il des erreurs de la nature ? \\[0pt]
Y a-t-il des facultés dans l'esprit ? \\[0pt]
Y a-t-il des faits moraux ? \\[0pt]
Y a-t-il des fondements naturels à l'ordre social ? \\[0pt]
Y a-t-il des genres de plaisir ? \\[0pt]
Y a-t-il des genres du plaisir ? \\[0pt]
Y a-t-il des intuitions morales ? \\[0pt]
Y a-t-il des limites à l'exprimable ? \\[0pt]
Y a-t-il des lois du social ? \\[0pt]
Y a-t-il des pathologies sociales ? \\[0pt]
Y a-t-il des pensées folles ? \\[0pt]
Y a-t-il des peuples sans histoire ? \\[0pt]
Y a-t-il des progrès en philosophie ? \\[0pt]
Y a-t-il des règles de la guerre ? \\[0pt]
Y a-t-il des sociétés sans histoire ? \\[0pt]
Y a-t-il des valeurs objectives ? \\[0pt]
Y a-t-il des vérités indiscutables ? \\[0pt]
Y a-t-il des violences légitimes ? \\[0pt]
Y a-t-il trop d'images ? \\[0pt]
Y a-t-il un art de penser ? \\[0pt]
Y a-t-il un devoir d'indignation ? \\[0pt]
Y a-t-il un droit à la différence ? \\[0pt]
Y a-t-il un droit au travail ? \\[0pt]
Y a-t-il un droit naturel ? \\[0pt]
Y a-t-il une causalité empirique ? \\[0pt]
Y a-t-il une compétence politique ? \\[0pt]
Y a-t-il une expérience de l'éternité ? \\[0pt]
Y a-t-il une expérience du néant ? \\[0pt]
Y a-t-il une fin de l'histoire ? \\[0pt]
Y a-t-il une hiérarchie des sciences ? \\[0pt]
Y a-t-il une histoire de la vérité ? \\[0pt]
Y a-t-il une mécanique des passions ? \\[0pt]
Y a-t-il une médecine de l'âme ? \\[0pt]
Y a-t-il une méthode propre aux sciences humaines ? \\[0pt]
Y a-t-il un empire de la technique ? \\[0pt]
Y a-t-il une nature humaine ? \\[0pt]
Y a-t-il une nécessité de l'Histoire ? \\[0pt]
Y a-t-il une philosophie de la nature ? \\[0pt]
Y a-t-il une science de l'esprit ? \\[0pt]
Y a-t-il une science du moi ? \\[0pt]
Y a-t-il une unité de la science ? \\[0pt]
Y a-t-il une unité des langages humains ? \\[0pt]
Y a-t-il une vérité des apparences ? \\[0pt]
Y a-t-il une vérité des sentiments ? \\[0pt]
Y a-t-il une vertu de l'oubli ? \\[0pt]
Y a-t-il un langage unifié de la science ? \\[0pt]
Y a-t-il un monde technique ? \\[0pt]
Y a-t-il un ordre des choses ? \\[0pt]
Y a-t-il un ordre du monde ? \\[0pt]
Y a-t-il un progrès en philosophie ? \\[0pt]
Y a-t-il un progrès moral ? \\[0pt]
Y a-t-il un rythme de l'histoire ? \\[0pt]
Y a-t-il un savoir du corps ? \\[0pt]
Y a-t-il un savoir politique ? \\[0pt]
Y a-t-il un savoir pratique \\[0pt]
Y a-t-il un usage purement instrumental de la raison ? \\[0pt]


\section{Tri par thème des sujets d'agrégation externe}
\label{sec:org3668218}
\subsection{Philosophie générale}
\label{sec:org8a8cb50}

\noindent
Abolir la propriété \\[0pt]
Abstraire \\[0pt]
À chacun ses goûts \\[0pt]
À chacun son dû \\[0pt]
Action et événement \\[0pt]
Action et fabrication \\[0pt]
Admettre le hasard, est-ce nier l'ordre de la nature ? \\[0pt]
Affirmation et négation \\[0pt]
Affirmer et nier \\[0pt]
Agir \\[0pt]
Agir librement \\[0pt]
Ai-je une âme ? \\[0pt]
Aimer la nature \\[0pt]
Aimer la vérité \\[0pt]
Aimer la vie \\[0pt]
Aimer les lois \\[0pt]
Aimer une œuvre d'art \\[0pt]
« Aimez vos ennemis » \\[0pt]
Aliénation et servitude \\[0pt]
« À l'impossible, nul n'est tenu » \\[0pt]
À l'impossible nul n'est tenu \\[0pt]
Aller au fond des choses \\[0pt]
Aller au vrai \\[0pt]
Amour et respect \\[0pt]
Analogie et ressemblance \\[0pt]
Analyse et synthèse \\[0pt]
Analyser les mœurs \\[0pt]
Apparence et réalité \\[0pt]
Apparition et manifestation \\[0pt]
Apprend-on à penser ? \\[0pt]
Apprend-on à percevoir la beauté ? \\[0pt]
Apprend-on à voir ? \\[0pt]
Apprendre \\[0pt]
Apprendre à aimer \\[0pt]
Apprendre à parler \\[0pt]
Apprendre à penser \\[0pt]
Apprendre à vivre \\[0pt]
Apprendre à voir \\[0pt]
Apprendre et devenir \\[0pt]
Apprendre s'apprend-il ? \\[0pt]
Approcher du vrai \\[0pt]
Après-coup \\[0pt]
« Après moi, le déluge » \\[0pt]
\emph{A priori} et \emph{a posteriori} \\[0pt]
À quelles conditions est-il acceptable de travailler ? \\[0pt]
À quelles conditions une pensée est-elle libre ? \\[0pt]
À quelque chose le malheur peut-il être bon ? \\[0pt]
À quelque chose, le malheur peut-il être bon ? \\[0pt]
À qui appartient-il de décider du juste et de l'injuste ? \\[0pt]
À qui appartient la Terre ? \\[0pt]
À qui la faute ? \\[0pt]
À qui peut-on faire confiance ? \\[0pt]
À qui se fier ? \\[0pt]
À quoi bon ? \\[0pt]
À quoi bon chercher à prouver l'existence de Dieu ? \\[0pt]
À quoi bon des philosophes ? \\[0pt]
À quoi bon discuter ? \\[0pt]
À quoi bon la morale ? \\[0pt]
À quoi bon rêver ? \\[0pt]
À quoi faut-il être fidèle ? \\[0pt]
À quoi faut-il renoncer ? \\[0pt]
À quoi la conscience nous donne-t-elle accès ? \\[0pt]
À quoi les sciences nous forment-elles ? \\[0pt]
À quoi reconnaît-on l'injustice ? \\[0pt]
À quoi reconnaît-on une attitude religieuse ? \\[0pt]
À quoi reconnaît-on une œuvre d'art ? \\[0pt]
À quoi reconnaît-on une science ? \\[0pt]
À quoi rêve-t-on ? \\[0pt]
À quoi sert la bioéthique ? \\[0pt]
À quoi sert la dialectique ? \\[0pt]
À quoi sert la logique ? \\[0pt]
À quoi sert la négation ? \\[0pt]
À quoi sert la technique ? \\[0pt]
À quoi sert l'écriture ? \\[0pt]
À quoi sert l'État ? \\[0pt]
À quoi sert l'intuition ? \\[0pt]
À quoi sert une métaphore ? \\[0pt]
À quoi sert un exemple ? \\[0pt]
À quoi servent les croyances ? \\[0pt]
À quoi servent les encyclopédies ? \\[0pt]
À quoi servent les hypothèses ? \\[0pt]
À quoi servent les preuves ? \\[0pt]
À quoi servent les règles ? \\[0pt]
À quoi servent les religions ? \\[0pt]
À quoi servent les statistiques ? \\[0pt]
À quoi servent les utopies ? \\[0pt]
À quoi suis-je obligé ? \\[0pt]
À quoi tient la fermeté du vouloir ? \\[0pt]
À quoi tient notre humanité ? \\[0pt]
Architecture et utopie \\[0pt]
Argumenter \\[0pt]
Argumenter et démontrer \\[0pt]
Arriver à ses fins \\[0pt]
Art et apparence \\[0pt]
Art et artifice \\[0pt]
Art et critique \\[0pt]
Art et expression \\[0pt]
Art et finitude \\[0pt]
Art et illusion \\[0pt]
Art et industrie \\[0pt]
Art et morale \\[0pt]
Art et politique \\[0pt]
Art et représentation \\[0pt]
Art et utilité \\[0pt]
Art et vérité \\[0pt]
À science nouvelle, nouvelle philosophie ? \\[0pt]
A-t-on besoin de spécialistes ? \\[0pt]
A-t-on des raisons de croire ? \\[0pt]
A-t-on des raisons de croire ce qu'on croit ? \\[0pt]
A-t-on expliqué un phénomène quand on l'a rapporté à des lois ? \\[0pt]
Attendre \\[0pt]
Attente et espérance \\[0pt]
Au nom de quoi le plaisir serait-il condamnable ? \\[0pt]
Au nom du peuple \\[0pt]
Aussitôt dit, aussitôt fait \\[0pt]
Autrui \\[0pt]
Avoir \\[0pt]
Avoir de la chance \\[0pt]
Avoir de l'autorité \\[0pt]
Avoir de l'esprit \\[0pt]
Avoir de l'expérience \\[0pt]
Avoir des ennemis \\[0pt]
Avoir des principes \\[0pt]
Avoir du goût \\[0pt]
Avoir du mérite \\[0pt]
Avoir du style \\[0pt]
Avoir la conscience tranquille \\[0pt]
Avoir mauvaise conscience \\[0pt]
Avoir peur \\[0pt]
Avoir raison \\[0pt]
Avoir ses raisons \\[0pt]
Avoir un corps \\[0pt]
Avoir une bonne mémoire \\[0pt]
Avoir une idée \\[0pt]
Avoir un sens \\[0pt]
Avons-nous à apprendre des images ? \\[0pt]
Avons-nous besoin de l'État ? \\[0pt]
Avons-nous besoin de lois ? \\[0pt]
Avons-nous besoin de métaphysique ? \\[0pt]
Avons-nous besoin de méthodes ? \\[0pt]
Avons-nous besoin de spectacles ? \\[0pt]
Avons-nous besoin d'idéal \\[0pt]
Avons-nous besoin d'un libre arbitre ? \\[0pt]
Avons-nous des devoirs envers les morts ? \\[0pt]
Avons-nous des devoirs envers nous-mêmes ? \\[0pt]
Avons-nous encore besoin de la nature ? \\[0pt]
Avons-nous le devoir d'être heureux ? \\[0pt]
Avons-nous le droit de juger autrui ? \\[0pt]
Avons-nous un accès direct aux phénomènes ? \\[0pt]
Avons-nous une responsabilité envers le passé ? \\[0pt]
Avons-nous un monde commun ? \\[0pt]
Axiomatique et vérité \\[0pt]
Bâtir un monde \\[0pt]
Bien commun et droits individuels \\[0pt]
Bien faire \\[0pt]
Bien jouer son rôle \\[0pt]
Bien juger \\[0pt]
Bien parler \\[0pt]
Bien parler, est-ce bien penser ? \\[0pt]
Bon sens et sens commun \\[0pt]
Calculer \\[0pt]
Calculer et penser \\[0pt]
Cartographier \\[0pt]
Cause et motivation \\[0pt]
Cause et raison \\[0pt]
Causes et lois \\[0pt]
Ceci \\[0pt]
Cela a-t-il un sens de penser par soi-même ? \\[0pt]
Ce que sait le poète \\[0pt]
Ce qui est à moi \\[0pt]
Ce qui est subjectif est-il arbitraire ? \\[0pt]
Ce qui importe \\[0pt]
Ce qu'il y a \\[0pt]
Ce qui n'a pas de prix \\[0pt]
Ce qui n'existe pas \\[0pt]
Ce qu'on ne peut pas vendre \\[0pt]
Certitude et pari \\[0pt]
Certitude et vérité \\[0pt]
Cesser d'espérer \\[0pt]
« C'est humain » \\[0pt]
« C'est la vie » \\[0pt]
C'est trop beau pour être vrai ! \\[0pt]
Ceux qui savent doivent-ils gouverner ? \\[0pt]
Changement et mouvement \\[0pt]
Changer \\[0pt]
Changer le monde \\[0pt]
Chercher ses mots \\[0pt]
Choisir \\[0pt]
Choisir ses souvenirs ? \\[0pt]
Choisit-on ses souvenirs ? \\[0pt]
Choisit-on son corps ? \\[0pt]
Chose et objet \\[0pt]
Choses et personnes \\[0pt]
Civilisé, barbare, sauvage \\[0pt]
Classer \\[0pt]
Classer, est-ce penser ? \\[0pt]
Classer et ordonner \\[0pt]
Classes et essences \\[0pt]
Cohérence et vérité \\[0pt]
Collectionner \\[0pt]
Commander \\[0pt]
Comme d'habitude \\[0pt]
Commémorer \\[0pt]
Commencer \\[0pt]
Commencer en philosophie \\[0pt]
Comment assumer les conséquences de ses actes ? \\[0pt]
Comment bien appliquer une règle ? \\[0pt]
Comment bien vivre ? \\[0pt]
Comment caractériser une idée confuse ? \\[0pt]
Comment comprendre une croyance qu'on ne partage pas ? \\[0pt]
Comment considérer le devenir ? \\[0pt]
Comment définir la démocratie ? \\[0pt]
Comment définir la raison ? \\[0pt]
Comment définir la responsabilité politique \\[0pt]
Comment définir la signification \\[0pt]
Comment définir le bien commun ? \\[0pt]
Comment définir le peuple ? \\[0pt]
Comment définir le style ? \\[0pt]
Comment définir l'État de droit ? \\[0pt]
Comment définir l'excès ? \\[0pt]
Comment définir l'infini ? \\[0pt]
Comment devient-on raisonnable ? \\[0pt]
Comment dire la vérité ? \\[0pt]
Comment entrer dans l'histoire ? \\[0pt]
Comment expliquer l'erreur ? \\[0pt]
Comment juger son éducation ? \\[0pt]
Comment justifier une méthode ? \\[0pt]
Comment la réalité peut-elle dépasser la fiction ? \\[0pt]
Comment la volonté peut-elle être indéterminée ? \\[0pt]
Comment penser la diversité des langues ? \\[0pt]
Comment peut-on être sceptique ? \\[0pt]
Comment sait-on qu'on se comprend ? \\[0pt]
Comment savoir ce qui existe ? \\[0pt]
Comment trancher une controverse ? \\[0pt]
Comment vérifier les hypothèses ? \\[0pt]
Comment vivre ensemble ? \\[0pt]
Comme on dit \\[0pt]
Communauté, collectivité, société \\[0pt]
Communauté et société \\[0pt]
Communiquer \\[0pt]
Communiquer et enseigner \\[0pt]
Comparer les cultures \\[0pt]
Compétence et autorité \\[0pt]
Comportement et conduite \\[0pt]
Comprendre, est-ce excuser ? \\[0pt]
Comprendre, est-ce expliquer par les causes ? \\[0pt]
Compter et mesurer \\[0pt]
Compter sur soi \\[0pt]
Concept et image \\[0pt]
Concept et métaphore \\[0pt]
Conception et perception \\[0pt]
Concevoir et percevoir \\[0pt]
Conduire sa vie \\[0pt]
Conduire ses pensées \\[0pt]
Confondre \\[0pt]
Connaissance et expérience \\[0pt]
Connaissance et liberté \\[0pt]
Connaissance et progrès \\[0pt]
Connaît-on la vie ou le vivant ? \\[0pt]
Connaître, est-ce connaître par les causes ? \\[0pt]
Connaître et agir \\[0pt]
Connaître et penser \\[0pt]
Connaître, expliquer, comprendre \\[0pt]
Connaître les causes \\[0pt]
Connaître les choses, en quoi est-ce déterminer leurs différences ? \\[0pt]
Connaître par les causes \\[0pt]
Connaître ses limites \\[0pt]
Conscience et mémoire \\[0pt]
Consensus et conflit \\[0pt]
Consentir \\[0pt]
Considère-t-on jamais le temps en lui-même ? \\[0pt]
Contemplation et action \\[0pt]
Contempler \\[0pt]
Continuité et discontinuité \\[0pt]
Contrainte et obligation \\[0pt]
Corps et identité \\[0pt]
Correspondre \\[0pt]
Création et production \\[0pt]
Créer \\[0pt]
Crimes et châtiments \\[0pt]
Crise et progrès \\[0pt]
Critiquer \\[0pt]
Critiquer la démocratie \\[0pt]
Croire au bonheur \\[0pt]
Croire aux fictions \\[0pt]
Croire en Dieu \\[0pt]
Croire en soi \\[0pt]
Croire, est-ce être faible ? \\[0pt]
Croire et savoir \\[0pt]
Croire savoir \\[0pt]
Croyance et certitude \\[0pt]
Croyance et passivité \\[0pt]
Croyance et vérité \\[0pt]
Cultivons notre jardin \\[0pt]
Culture et civilisation \\[0pt]
Dans quelle mesure la traduction est-elle une activité cognitive ? \\[0pt]
Décider \\[0pt]
Décomposer les choses \\[0pt]
Décorer \\[0pt]
Découverte et invention \\[0pt]
Découvrir et inventer \\[0pt]
Décrire, est-ce déjà expliquer ? \\[0pt]
Définir \\[0pt]
Définir et démontrer : à quoi bon ? \\[0pt]
Définir l'art : à quoi bon ? \\[0pt]
Déjouer \\[0pt]
« De la musique avant toute chose » \\[0pt]
Délibérer \\[0pt]
Délibérer, est-ce être dans l'incertitude ? \\[0pt]
De l'utilité des voyages \\[0pt]
Démêler le vrai du faux \\[0pt]
Démocratie et consensus \\[0pt]
Démocratie et représentation \\[0pt]
Démocratie et vérité \\[0pt]
Démocratie représentative \\[0pt]
Démontrer et argumenter \\[0pt]
Dénaturer \\[0pt]
Dépasser les apparences ? \\[0pt]
Dépasser les bornes \\[0pt]
Dépasser l'humain \\[0pt]
Déplaire \\[0pt]
De quel bonheur sommes-nous capables ? \\[0pt]
De quel droit ? \\[0pt]
De quelle réalité témoignent nos perceptions ? \\[0pt]
De quelle vérité l'opinion est-elle capable ? \\[0pt]
De qui sommes-nous les obligés ? \\[0pt]
De quoi a-t-on besoin ? \\[0pt]
De quoi doute un sceptique ? \\[0pt]
De quoi est-on conscient ? \\[0pt]
De quoi est-on malheureux ? \\[0pt]
De quoi faisons-nous l'expérience ? \\[0pt]
De quoi « humain » est-il le nom ? \\[0pt]
De quoi la forme est-elle la forme ? \\[0pt]
De quoi la logique est-elle la science ? \\[0pt]
De quoi l'art nous console-t-il ? \\[0pt]
De quoi l'art nous délivre-t-il ? \\[0pt]
De quoi le phénomène est-il l'apparaître ? \\[0pt]
De quoi les logiciens parlent-ils ? \\[0pt]
De quoi les métaphysiciens parlent-ils ? \\[0pt]
De quoi n'avons-nous pas conscience ? \\[0pt]
De quoi ne peut-on pas répondre ? \\[0pt]
De quoi parlent les mathématiques ? \\[0pt]
De quoi parlent les théories physiques ? \\[0pt]
De quoi pâtit-on ? \\[0pt]
De quoi peut-on réclamer la propriété ? \\[0pt]
De quoi pouvons-nous être certains ? \\[0pt]
De quoi sommes-nous prisonniers ? \\[0pt]
De quoi sommes-nous responsables ? \\[0pt]
De quoi suis-je responsable ? \\[0pt]
De quoi y a-t-il expérience ? \\[0pt]
De quoi y a-t-il histoire ? \\[0pt]
Déraisonner \\[0pt]
Désacraliser \\[0pt]
Désespérer de l'homme \\[0pt]
Désirer \\[0pt]
Désirer, est-ce refuser de se satisfaire de la réalité \\[0pt]
Désire-t-on la reconnaissance ? \\[0pt]
Désire-t-on un autre que soi ? \\[0pt]
Désir et temps \\[0pt]
Désir et volonté \\[0pt]
Des mythes peuvent-ils encore apparaître ? \\[0pt]
Désobéir \\[0pt]
Désordre et chaos \\[0pt]
Détermination, déterminisme, liberté \\[0pt]
Déterminer \\[0pt]
Détruire pour reconstruire \\[0pt]
Deux et deux font quatre \\[0pt]
Devenir ce que l'on est \\[0pt]
Devenir citoyen \\[0pt]
Devenir et évolution \\[0pt]
« Deviens ce que tu es » \\[0pt]
« Deviens qui tu es » \\[0pt]
Devient-on raisonnable ? \\[0pt]
Devoir et utilité \\[0pt]
Devons-nous aimer notre destin ? \\[0pt]
Devons-nous nous faire confiance ? \\[0pt]
Dialectique et Philosophie \\[0pt]
Dialectique et vérité \\[0pt]
Dialoguer \\[0pt]
Dieu aurait-il pu mieux faire ? \\[0pt]
Dieu est-il une limite de la pensée ? \\[0pt]
« Dieu est mort » \\[0pt]
Dieu est mort \\[0pt]
Dieu et César \\[0pt]
Dieu, prouvé ou éprouvé ? \\[0pt]
Différer \\[0pt]
Dire ce qui est \\[0pt]
Dire, est-ce faire ? \\[0pt]
Dire et faire \\[0pt]
Dire et montrer \\[0pt]
Dire « je » \\[0pt]
Dire la vérité \\[0pt]
Dire la vérité, est-ce un devoir ? \\[0pt]
Dire oui \\[0pt]
Dire que l'être humain est un animal, est-ce méconnaître sa dignité ? \\[0pt]
Diriger son esprit \\[0pt]
Discerner et juger \\[0pt]
Discret et continu \\[0pt]
Discussion et dialogue \\[0pt]
Disposer de son corps \\[0pt]
Distinguer \\[0pt]
Doit-on bien juger pour bien faire ? \\[0pt]
Doit-on déplorer le désaccord ? \\[0pt]
Doit-on respecter la nature ? \\[0pt]
Doit-on se faire l'avocat du diable ? \\[0pt]
Doit-on tout calculer ? \\[0pt]
Donner \\[0pt]
Donner des exemples \\[0pt]
Donner des raisons \\[0pt]
Donner du sens \\[0pt]
Donner raison \\[0pt]
Donner raison, rendre raison \\[0pt]
Donner sa parole \\[0pt]
Donner un exemple \\[0pt]
Douter \\[0pt]
D'où viennent les idées générales ? \\[0pt]
D'où vient aux objets techniques leur beauté ? \\[0pt]
D'où vient la signification des mots ? \\[0pt]
D'où vient le mal ? \\[0pt]
D'où vient le plaisir de lire ? \\[0pt]
D'où vient que l'histoire soit autre chose qu'un chaos ? \\[0pt]
Droit et coutume \\[0pt]
Droit et nature \\[0pt]
Échange et don \\[0pt]
Éclairer \\[0pt]
Économie et prédiction \\[0pt]
Écrire \\[0pt]
Écrire l'histoire \\[0pt]
Écrire l'histoire, est-ce la faire ? \\[0pt]
Éducation de l'homme, éducation du citoyen \\[0pt]
Éduquer \\[0pt]
Éduquer à la liberté \\[0pt]
Éduquer et instruire \\[0pt]
Éduquer le désir \\[0pt]
Éduquer le goût \\[0pt]
Égalité et identité \\[0pt]
Égoïsme et méchanceté \\[0pt]
Élite et démocratie \\[0pt]
Empirique et expérimental \\[0pt]
En apparence \\[0pt]
En mathématiques, la notion d'existence a-t-elle un sens ? \\[0pt]
En morale, y a-t-il quelque chose à savoir ? \\[0pt]
En quel sens peut-on parler de transcendance ? \\[0pt]
En quels sens les corps peuvent-ils être objets de la métaphysique ? \\[0pt]
Enquêter \\[0pt]
En quoi la matière s'oppose-t-elle à l'esprit ? \\[0pt]
En quoi la technique fait-elle question ? \\[0pt]
En quoi une insulte est-elle blessante ? \\[0pt]
Enseigner \\[0pt]
Entendement et raison \\[0pt]
Entendre \\[0pt]
Entendre raison \\[0pt]
Entrer en scène \\[0pt]
Énumérer \\[0pt]
Épistémologie et psychologie \\[0pt]
Éprouver \\[0pt]
Erreur, illusion, hallucination \\[0pt]
Espace et réalité \\[0pt]
Espace vécu, espace géométrique \\[0pt]
Essayer \\[0pt]
Essence et existence \\[0pt]
Est-il difficile de savoir ce que l'on veut ? \\[0pt]
Est-il difficile d'être heureux ? \\[0pt]
Est-il impossible de moraliser la politique ? \\[0pt]
Est-il judicieux de revenir sur ses décisions ? \\[0pt]
Est-il possible de croire en la vie éternelle ? \\[0pt]
Est-il possible de douter de tout ? \\[0pt]
Est-il possible de ne croire à rien ? \\[0pt]
Est-il toujours illicite d'inférer ce qui doit être à partir de ce qui est ? \\[0pt]
Est-il vrai qu'on apprenne de ses erreurs ? \\[0pt]
Estimer \\[0pt]
Est-on fondé à distinguer la justice et le droit ? \\[0pt]
Est-on fondé à parler d'une imperfection du langage ? \\[0pt]
Est-on le produit d'une culture ? \\[0pt]
Est-on responsable de ses rêves ? \\[0pt]
Est-on responsable de son passé ? \\[0pt]
Établir les faits \\[0pt]
État et société \\[0pt]
État, peuple, nation \\[0pt]
Éternité et immortalité \\[0pt]
Ethnologie et sociologie \\[0pt]
Être absolument libre \\[0pt]
Être adulte \\[0pt]
Être affairé \\[0pt]
Être bien élevé \\[0pt]
Être bon juge \\[0pt]
Être cause de soi \\[0pt]
Être certain \\[0pt]
Être, c'est agir \\[0pt]
Être chez soi \\[0pt]
Être citoyen du monde \\[0pt]
Être civilisé \\[0pt]
Être compris \\[0pt]
« Être contre » \\[0pt]
Être dans son bon droit \\[0pt]
Être dans son droit \\[0pt]
Être de mauvaise humeur \\[0pt]
Être de son temps \\[0pt]
Être égal à soi-même \\[0pt]
Être en bonne santé \\[0pt]
Être en désaccord \\[0pt]
Être en dette \\[0pt]
Être ensemble \\[0pt]
Être, est-ce agir ? \\[0pt]
Être et avoir \\[0pt]
Être et devenir \\[0pt]
Être et devoir être \\[0pt]
Être et exister \\[0pt]
Être exemplaire \\[0pt]
Être fidèle \\[0pt]
Être hors de soi \\[0pt]
Être hors la loi \\[0pt]
Être impossible \\[0pt]
Être logique \\[0pt]
Être maître de soi \\[0pt]
Être malade \\[0pt]
Être matérialiste \\[0pt]
Être méchant volontairement \\[0pt]
Être mortel \\[0pt]
Être naturel \\[0pt]
Être né \\[0pt]
Être normal \\[0pt]
Être passionné \\[0pt]
Être patient \\[0pt]
Être pauvre \\[0pt]
Être poli \\[0pt]
Être politique \\[0pt]
Être pragmatique \\[0pt]
Être présent \\[0pt]
Être réaliste \\[0pt]
Être riche \\[0pt]
Être sans scrupule \\[0pt]
Être sceptique \\[0pt]
Être sensible \\[0pt]
Être seul avec sa conscience \\[0pt]
Être seul avec soi-même \\[0pt]
Être soi-même \\[0pt]
Être solidaire \\[0pt]
Être spirituel \\[0pt]
Être systématique \\[0pt]
Être un artiste \\[0pt]
Être un corps \\[0pt]
Étudier \\[0pt]
Évaluer l'art \\[0pt]
Évolution et progrès \\[0pt]
Exactitude et objectivité \\[0pt]
Existence et essence \\[0pt]
Existence et transcendance \\[0pt]
Exister \\[0pt]
Existe-t-il des autorités en matière d'art ? \\[0pt]
Existe-t-il des connaissances a priori ? \\[0pt]
Existe-t-il des degrés de vérité ? \\[0pt]
Existe-t-il des dilemmes moraux ? \\[0pt]
Existe-t-il des émotions proprement esthétiques ? \\[0pt]
Existe-t-il des langages naturels ? \\[0pt]
Existe-t-il des paroles vraies ? \\[0pt]
Existe-t-il des preuves irréfutables ? \\[0pt]
Existe-t-il des problèmes insolubles ? \\[0pt]
Existe-t-il des questions sans réponse ? \\[0pt]
Existe-t-il des sciences de différentes natures ? \\[0pt]
Existe-t-il différentes sortes de sciences ? \\[0pt]
Existe-t-il plusieurs mondes ? \\[0pt]
Existe-t-il un bien souverain ? \\[0pt]
Existe-t-il une autorité naturelle ? \\[0pt]
Existe-t-il une hiérarchie des êtres ? \\[0pt]
Existe-t-il une opinion publique ? \\[0pt]
Existe-t-il une sensibilité philosophique ? \\[0pt]
Existe-t-il une violence symbolique ? \\[0pt]
Expérience et expérimentation \\[0pt]
Expérimenter \\[0pt]
Explication et compréhension \\[0pt]
Expliquer \\[0pt]
Expliquer et comprendre \\[0pt]
Expression et création \\[0pt]
Expression et signification \\[0pt]
Faire ce qu'on dit \\[0pt]
Faire comme si \\[0pt]
Faire corps \\[0pt]
Faire croire \\[0pt]
Faire de nécessité vertu \\[0pt]
Faire des choix \\[0pt]
Faire école \\[0pt]
Faire et agir \\[0pt]
Faire et laisser faire \\[0pt]
Faire justice \\[0pt]
Faire la loi \\[0pt]
Faire la paix \\[0pt]
Faire la révolution \\[0pt]
Faire la vérité \\[0pt]
Faire le bien \\[0pt]
Faire le mal \\[0pt]
Faire l'histoire \\[0pt]
Faire sa vie \\[0pt]
Faire ses preuves \\[0pt]
Faire société \\[0pt]
Faire son possible \\[0pt]
Faire table rase \\[0pt]
Faire une expérience \\[0pt]
Faits et valeurs \\[0pt]
Falsifier \\[0pt]
Faut-il aimer la vie ? \\[0pt]
Faut-il aimer son prochain comme soi-même ? \\[0pt]
Faut-il aller au-delà des apparences ? \\[0pt]
Faut-il avoir des ennemis ? \\[0pt]
Faut-il avoir des principes ? \\[0pt]
Faut-il avoir peur de la liberté ? \\[0pt]
Faut-il avoir peur de l'avenir ? \\[0pt]
Faut-il avoir peur de la vérité ? \\[0pt]
Faut-il avoir peur des contradictions ? \\[0pt]
Faut-il avoir peur des habitudes ? \\[0pt]
Faut-il chasser les poètes ? \\[0pt]
Faut-il combattre ses illusions ? \\[0pt]
Faut-il concilier les contraires ? \\[0pt]
Faut-il condamner la rhétorique ? \\[0pt]
Faut-il condamner les illusions ? \\[0pt]
Faut-il connaître l'histoire des sciences ? \\[0pt]
Faut-il craindre le pire ? \\[0pt]
Faut-il craindre les masses ? \\[0pt]
Faut-il croire au progrès ? \\[0pt]
Faut-il croire en quelque chose ? \\[0pt]
Faut-il des techniques du raisonnement ? \\[0pt]
Faut-il distinguer ce qui est de ce qui doit être ? \\[0pt]
Faut-il être cosmopolite ? \\[0pt]
Faut-il être discipliné ? \\[0pt]
Faut-il être heureux ? \\[0pt]
Faut-il être mesuré ? \\[0pt]
Faut-il être mesuré en toutes choses ? \\[0pt]
Faut-il être naturel ? \\[0pt]
Faut-il être objectif ? \\[0pt]
Faut-il faire de la philosophie ? \\[0pt]
Faut-il laisser parler la nature ? \\[0pt]
Faut-il maîtriser ses émotions ? \\[0pt]
Faut-il mathématiser le réel pour le connaître ? \\[0pt]
Faut-il ménager les apparences ? \\[0pt]
Faut-il mépriser le luxe ? \\[0pt]
Faut-il n'être jamais méchant ? \\[0pt]
Faut-il opposer éduquer et instruire ? \\[0pt]
Faut-il opposer l'esprit et la matière ? \\[0pt]
Faut-il opposer l'État et la société ? \\[0pt]
Faut-il opposer l'histoire et la fiction ? \\[0pt]
Faut-il opposer rhétorique et philosophie ? \\[0pt]
Faut-il opposer sens et vérité ? \\[0pt]
Faut-il pardonner ? \\[0pt]
Faut-il parler pour avoir des idées générales ? \\[0pt]
Faut-il penser l'État comme un individu ? \\[0pt]
Faut-il perdre ses illusions ? \\[0pt]
Faut-il préférer le bonheur à la vérité ? \\[0pt]
Faut-il rechercher la certitude ? \\[0pt]
Faut-il renoncer à croire ? \\[0pt]
Faut-il renoncer à la distinction entre nature et culture ? \\[0pt]
Faut-il renoncer à l'idée de fondement ? \\[0pt]
Faut-il renoncer à son désir ? \\[0pt]
Faut-il respecter la nature ? \\[0pt]
Faut-il respecter les convenances ? \\[0pt]
Faut-il rompre avec le passé ? \\[0pt]
Faut-il s'adapter aux événements ? \\[0pt]
Faut-il s'aimer pour vivre ensemble ? \\[0pt]
Faut-il s'attendre à l'inattendu ? \\[0pt]
Faut-il savoir ce que l'on veut ? \\[0pt]
Faut-il savoir désobéir ? \\[0pt]
Faut-il se fier à la majorité ? \\[0pt]
Faut-il se méfier de l'imagination ? \\[0pt]
Faut-il se méfier des images ? \\[0pt]
Faut-il séparer morale et politique ? \\[0pt]
Faut-il se résigner aux inégalités ? \\[0pt]
Faut-il suivre ses intuitions ? \\[0pt]
Faut-il supposer un énoncé vrai pour en comprendre le sens ? \\[0pt]
Faut-il toujours respecter ses engagements ? \\[0pt]
Faut-il vivre comme si l'on ne devait jamais mourir ? \\[0pt]
Faut-il vivre conformément à la nature ? \\[0pt]
Faut-il voir pour croire ? \\[0pt]
Fiction et mensonge \\[0pt]
Fiction et réalité \\[0pt]
Foi et raison \\[0pt]
Fonction et limites des modèles en sciences sociales \\[0pt]
Fonder \\[0pt]
Fonder la justice \\[0pt]
Fonder une cité \\[0pt]
Force et violence \\[0pt]
Forme et matière \\[0pt]
Gagner sa vie \\[0pt]
Garder en mémoire \\[0pt]
Généralité de la règle, contingence des faits \\[0pt]
Généralité et universalité \\[0pt]
Géographie et géométrie \\[0pt]
Gérer et gouverner \\[0pt]
Gestion et politique \\[0pt]
Gouvernement des hommes et administration des choses \\[0pt]
Gouverner \\[0pt]
Gouverner, est-ce dominer ? \\[0pt]
Gouverner la nature \\[0pt]
Gouverner les passions \\[0pt]
Grammaire et philosophie \\[0pt]
Grandeur et décadence \\[0pt]
Grandir \\[0pt]
Guerre et paix \\[0pt]
Habiter \\[0pt]
Habiter le monde \\[0pt]
Habiter l'espace \\[0pt]
Haïr la raison \\[0pt]
Hériter \\[0pt]
Histoire et fiction \\[0pt]
Histoire et mémoire \\[0pt]
Histoire et narration \\[0pt]
Histoire et préhistoire \\[0pt]
Humour et ironie \\[0pt]
Hypothèse et vérité \\[0pt]
Ici et maintenant \\[0pt]
Identité et communauté \\[0pt]
Identité et égalité \\[0pt]
Identité et représentation politique \\[0pt]
« Il faut de tout pour faire un monde » \\[0pt]
Illégalité et injustice \\[0pt]
« Il n'y a pas de fait, il n'y a que des interprétations » \\[0pt]
Image, concept et signe \\[0pt]
Image et idée \\[0pt]
Imagination et connaissance \\[0pt]
Imagination et perception \\[0pt]
Imaginer \\[0pt]
Imaginer, est-ce créer ? \\[0pt]
Immortalité et éternité \\[0pt]
Indépendance et autonomie \\[0pt]
Indépendance et liberté \\[0pt]
Individualisme et égoïsme \\[0pt]
Individuation et identité \\[0pt]
Inégalités et différences \\[0pt]
Information et communication \\[0pt]
Innocenter le devenir \\[0pt]
Instinct et morale \\[0pt]
Instruire et éduquer \\[0pt]
Interpréter \\[0pt]
Intuition et concept \\[0pt]
Intuition et déduction \\[0pt]
« J'ai le droit » \\[0pt]
J'ai un corps \\[0pt]
Je mens \\[0pt]
« Je n'ai pas voulu cela » \\[0pt]
Je sens, donc je suis \\[0pt]
Je, tu, il \\[0pt]
Jouer \\[0pt]
Jouer son rôle \\[0pt]
Jouer un rôle \\[0pt]
Jugement et proposition \\[0pt]
Juger \\[0pt]
Juge-t-on un acte ou son auteur ? \\[0pt]
Jusqu'à quel point pouvons-nous juger autrui ? \\[0pt]
Justice et bonheur \\[0pt]
Justice et égalité \\[0pt]
Justice et vengeance \\[0pt]
Justice et violence \\[0pt]
Justice sociale et charité publique \\[0pt]
Justifier \\[0pt]
La banalité \\[0pt]
L'abandon \\[0pt]
La barbarie \\[0pt]
La bassesse \\[0pt]
La béatitude \\[0pt]
La beauté \\[0pt]
La beauté a-t-elle une histoire ? \\[0pt]
La beauté de la nature \\[0pt]
La beauté demande-t-elle des preuves ? \\[0pt]
La beauté des ruines \\[0pt]
La beauté du diable \\[0pt]
La beauté du geste \\[0pt]
La beauté du monde \\[0pt]
La beauté du vrai \\[0pt]
La beauté est-elle l'objet d'une connaissance ? \\[0pt]
La beauté n'est-elle qu'un effet artistique ? \\[0pt]
La beauté peut-elle délivrer une vérité ? \\[0pt]
La belle nature \\[0pt]
La bestialité \\[0pt]
La bêtise \\[0pt]
La bêtise n'est-elle pas proprement humaine ? \\[0pt]
La bibliothèque \\[0pt]
La bienfaisance \\[0pt]
La bienveillance \\[0pt]
La biologie peut-elle se passer de causes finales ? \\[0pt]
La bonne conscience \\[0pt]
L'absence \\[0pt]
L'absolu \\[0pt]
L'Absolu est-il un objet de savoir ? \\[0pt]
L'abstraction \\[0pt]
L'abstrait et le concret \\[0pt]
L'absurde \\[0pt]
L'absurde peut-il avoir un sens pour une pensée 155 \\[0pt]
L'abus de langage \\[0pt]
L'abus de pouvoir \\[0pt]
L'académisme \\[0pt]
L'académisme dans l'art \\[0pt]
La caricature \\[0pt]
La casuistique \\[0pt]
La catastrophe \\[0pt]
La causalité \\[0pt]
La cause \\[0pt]
La cause et la raison \\[0pt]
La cause première \\[0pt]
L'accident \\[0pt]
L'accidentel \\[0pt]
L'accomplissement \\[0pt]
L'accord \\[0pt]
L'accord des esprits \\[0pt]
La censure \\[0pt]
La certitude \\[0pt]
La chair \\[0pt]
La chance \\[0pt]
La charité \\[0pt]
La charité est-elle une vertu ? \\[0pt]
L'achèvement de l'œuvre \\[0pt]
La chose \\[0pt]
La chose et l'objet \\[0pt]
La chose publique \\[0pt]
La chronologie \\[0pt]
La circonspection \\[0pt]
La circonstance \\[0pt]
La citation \\[0pt]
La civilisation \\[0pt]
La civilité \\[0pt]
La clarté \\[0pt]
La clarté suffit-elle au savoir ? \\[0pt]
La classification \\[0pt]
La clémence \\[0pt]
La cohésion sociale \\[0pt]
La colère \\[0pt]
La comédie humaine \\[0pt]
La communauté des savants \\[0pt]
La communauté scientifique \\[0pt]
La communication \\[0pt]
La comparaison \\[0pt]
La compassion \\[0pt]
La compétence \\[0pt]
La compétence fonde-t-elle la compétence juridique ? \\[0pt]
La compétence fonde-t-elle l'autorité politique ? \\[0pt]
La compréhension \\[0pt]
La concorde \\[0pt]
La condition \\[0pt]
La condition humaine \\[0pt]
La confiance \\[0pt]
La confusion \\[0pt]
La connaissance animale \\[0pt]
La connaissance a-t-elle des limites ? \\[0pt]
La connaissance de l'amour \\[0pt]
La connaissance de l'animal \\[0pt]
La connaissance de l'individuel \\[0pt]
La connaissance de soi \\[0pt]
La connaissance du singulier \\[0pt]
La connaissance mathématique \\[0pt]
La conquête \\[0pt]
La conquête de l'espace \\[0pt]
La conscience de soi \\[0pt]
La conscience est-elle le produit de la vie réelle ? \\[0pt]
La conscience est-elle temporelle ? \\[0pt]
La conscience historique \\[0pt]
La conscience morale \\[0pt]
La conscience peut-elle être collective ? \\[0pt]
La conséquence \\[0pt]
La conservation de soi \\[0pt]
La considération \\[0pt]
La consolation \\[0pt]
La consommation \\[0pt]
La constance \\[0pt]
La constitution \\[0pt]
La contemplation \\[0pt]
La contingence \\[0pt]
La contingence du futur \\[0pt]
La contingence du mal \\[0pt]
La continuité \\[0pt]
La contradiction \\[0pt]
La contradiction réside-t-elle dans les choses ? \\[0pt]
La contrainte \\[0pt]
La controverse \\[0pt]
La convention \\[0pt]
La conversation \\[0pt]
La conversion \\[0pt]
La conviction \\[0pt]
La coopération \\[0pt]
La corruption \\[0pt]
La couleur \\[0pt]
La couleur en peinture \\[0pt]
La couleur et le dessin \\[0pt]
La coutume \\[0pt]
L'acquis \\[0pt]
La crainte des Dieux \\[0pt]
La crainte et l'ignorance \\[0pt]
La création de l'humanité \\[0pt]
La créativité \\[0pt]
La crise du sens \\[0pt]
La critique \\[0pt]
La critique et la crise \\[0pt]
La croissance \\[0pt]
La croyance est-elle l'asile de l'ignorance ? \\[0pt]
La cruauté \\[0pt]
L'acteur \\[0pt]
L'action du temps \\[0pt]
L'action requiert-elle décision d'un sujet ? \\[0pt]
L'actualité \\[0pt]
L'actuel \\[0pt]
La culpabilité \\[0pt]
La culture générale \\[0pt]
La curiosité \\[0pt]
La curiosité est-elle à l'origine du savoir ? \\[0pt]
La danse \\[0pt]
L'adaptation \\[0pt]
La décadence \\[0pt]
La décence \\[0pt]
La déception \\[0pt]
La décision \\[0pt]
La déduction \\[0pt]
La déficience \\[0pt]
La définition \\[0pt]
La délibération \\[0pt]
La démagogie \\[0pt]
La démence \\[0pt]
La démesure \\[0pt]
La démocratie est-elle un mythe ? \\[0pt]
La démocratie et les experts \\[0pt]
La démonstration par l'absurde \\[0pt]
La démonstration suffit-elle à établir la vérité ? \\[0pt]
La dépendance \\[0pt]
La déraison \\[0pt]
La description \\[0pt]
La désillusion \\[0pt]
La désinvolture \\[0pt]
La désobéissance \\[0pt]
La désobéissance civile \\[0pt]
La détermination \\[0pt]
La dialectique \\[0pt]
La dialectique est-elle une science ? \\[0pt]
La différence \\[0pt]
La différence des arts \\[0pt]
La différence des sexes \\[0pt]
La différence sexuelle \\[0pt]
La difformité \\[0pt]
La dignité \\[0pt]
La dignité de l'homme \\[0pt]
La dignité humaine \\[0pt]
La digression \\[0pt]
La discipline \\[0pt]
La discrétion \\[0pt]
La discussion \\[0pt]
La distance \\[0pt]
La distinction \\[0pt]
La distinction entre vraie et fausse sciences a-t-elle un sens ? \\[0pt]
La distraction \\[0pt]
La diversion \\[0pt]
La diversité des cultures \\[0pt]
La diversité des langues \\[0pt]
La diversité des perceptions \\[0pt]
La diversité des sciences \\[0pt]
La division \\[0pt]
La division du travail \\[0pt]
L'admiration \\[0pt]
La domination \\[0pt]
La domination du corps \\[0pt]
La domination sociale \\[0pt]
La douceur \\[0pt]
La douleur \\[0pt]
La droiture \\[0pt]
La faiblesse de la démocratie \\[0pt]
La familiarité \\[0pt]
La famille \\[0pt]
La fatalité \\[0pt]
La fatigue \\[0pt]
La faute \\[0pt]
La faute de goût \\[0pt]
La faute et l'erreur \\[0pt]
La fécondité du raisonnement mathématique \\[0pt]
La fête \\[0pt]
L'affection \\[0pt]
La fiction \\[0pt]
La fidélité \\[0pt]
La finalité \\[0pt]
La fin de la politique est-elle d'éliminer la violence ? \\[0pt]
La fin de la vie \\[0pt]
La fin de l'histoire \\[0pt]
La fin du monde \\[0pt]
La fin et les moyens \\[0pt]
La finitude \\[0pt]
La folie \\[0pt]
La folie des grandeurs \\[0pt]
La fonction de penser peut-elle se déléguer ? \\[0pt]
La fonction des exemples \\[0pt]
La fonction sociale de l'écrivain \\[0pt]
La force \\[0pt]
La force de la loi \\[0pt]
La force de l'habitude \\[0pt]
La force de l'idée \\[0pt]
La force de l'oubli \\[0pt]
La force des choses \\[0pt]
La force des idées \\[0pt]
La force des objections \\[0pt]
La force du droit \\[0pt]
La force du génie \\[0pt]
La force et le droit \\[0pt]
La force publique \\[0pt]
La formation du goût \\[0pt]
La formation d'une conscience \\[0pt]
La fortune \\[0pt]
La foule \\[0pt]
La fragilité \\[0pt]
La fraternité \\[0pt]
La fraternité a-t-elle un sens politique ? \\[0pt]
La fraude \\[0pt]
La frivolité \\[0pt]
La frontière \\[0pt]
La fuite du temps \\[0pt]
La futilité \\[0pt]
L'âge d'or \\[0pt]
La généalogie peut-elle être une méthode historique ? \\[0pt]
La générosité \\[0pt]
La genèse \\[0pt]
La géométrie \\[0pt]
La grâce \\[0pt]
La grammaire \\[0pt]
La grammaire contraint-elle notre pensée ? \\[0pt]
La grandeur \\[0pt]
La grandeur d'âme \\[0pt]
La gratuité \\[0pt]
La gravité \\[0pt]
L'agressivité \\[0pt]
La guérison \\[0pt]
La guerre est-elle la plus grande violence ? \\[0pt]
La guerre et la paix \\[0pt]
La guerre met-elle fin au droit ? \\[0pt]
La guerre peut-elle être juste ? \\[0pt]
La haine \\[0pt]
La haine de la pensée \\[0pt]
La haine de la raison \\[0pt]
La haine de la vérité \\[0pt]
La haine de soi \\[0pt]
La hiérarchie \\[0pt]
La honte \\[0pt]
Laisser faire \\[0pt]
La jalousie \\[0pt]
La jeunesse \\[0pt]
La joie \\[0pt]
La jouissance \\[0pt]
La jurisprudence \\[0pt]
La juste mesure \\[0pt]
La juste peine \\[0pt]
La justice a-t-elle besoin des institutions ? \\[0pt]
La justice a-t-elle pour fonction de compenser les inégalités naturelles ? \\[0pt]
La justice comme avantage mutuel \\[0pt]
La justice de l'État \\[0pt]
La justice divine \\[0pt]
La justice est-elle affaire de raison ? \\[0pt]
La justice est-elle le ciment de la paix sociale ? \\[0pt]
La justice est-elle une vertu ? \\[0pt]
La justice peut-elle se passer d'institutions ? \\[0pt]
La justice requiert-elle toujours l'impartialité ? \\[0pt]
La justification \\[0pt]
La lâcheté \\[0pt]
La laideur \\[0pt]
La langue maternelle \\[0pt]
La leçon des choses \\[0pt]
La lecture \\[0pt]
La légende \\[0pt]
La légèreté \\[0pt]
La légitimation \\[0pt]
La lettre et l'esprit \\[0pt]
La liaison des idées \\[0pt]
La liberté a-t-elle une histoire ? \\[0pt]
La liberté de parole \\[0pt]
La liberté de penser \\[0pt]
La liberté d'expression a-t-elle des limites ? \\[0pt]
La liberté d'imaginer \\[0pt]
La liberté du citoyen \\[0pt]
La liberté et la grâce \\[0pt]
La liberté se prouve-t-elle ? \\[0pt]
La liberté s'éprouve-t-elle ? \\[0pt]
L'aliénation \\[0pt]
La limite \\[0pt]
La logique : calcul ou langage ? \\[0pt]
La logique est-elle un art de raisonner ? \\[0pt]
La logique pourrait-elle nous surprendre ? \\[0pt]
La loi \\[0pt]
La loi du genre \\[0pt]
La loi du marché \\[0pt]
La loi peut-elle changer les mœurs ? \\[0pt]
L'alter ego \\[0pt]
L'alternative \\[0pt]
L'altruisme est-il une attitude morale ? \\[0pt]
La lumière \\[0pt]
La lumière de la vérité \\[0pt]
La lumière naturelle \\[0pt]
La lutte des classes \\[0pt]
La machine \\[0pt]
La magie \\[0pt]
La main \\[0pt]
La main et l'outil \\[0pt]
La maîtrise \\[0pt]
La maîtrise de la langue \\[0pt]
La maîtrise de soi \\[0pt]
La maîtrise du geste \\[0pt]
La maîtrise du temps \\[0pt]
La majesté \\[0pt]
La majorité \\[0pt]
La maladie \\[0pt]
La malchance \\[0pt]
La manière \\[0pt]
L'amateurisme \\[0pt]
La matière, est-ce l'informe ? \\[0pt]
La matière est-elle une substance ? \\[0pt]
La matière n'est-elle qu'une idée ? \\[0pt]
La matière peut-elle penser ? \\[0pt]
La matière sensible \\[0pt]
La maturité \\[0pt]
La mauvaise conscience \\[0pt]
La mauvaise éducation \\[0pt]
La mauvaise foi \\[0pt]
La mauvaise volonté \\[0pt]
L'ambiguïté \\[0pt]
L'âme \\[0pt]
La méchanceté \\[0pt]
La méconnaissance \\[0pt]
La médecine est-elle une science ? \\[0pt]
L'âme des bêtes \\[0pt]
La médiation \\[0pt]
La méditation \\[0pt]
L'âme est-elle immortelle ? \\[0pt]
La mélancolie \\[0pt]
La mémoire \\[0pt]
La mémoire sélective \\[0pt]
La menace \\[0pt]
La mesure \\[0pt]
La mesure des choses \\[0pt]
La métaphore \\[0pt]
La méthode \\[0pt]
La méthode hypothético-déductive \\[0pt]
La minorité \\[0pt]
La misère \\[0pt]
La misologie \\[0pt]
L'amitié \\[0pt]
L'amitié est-elle une vertu ? \\[0pt]
La mode \\[0pt]
La modération \\[0pt]
La modernité \\[0pt]
La modestie \\[0pt]
La monnaie \\[0pt]
La morale de l'athée \\[0pt]
La morale doit-elle en appeler à la nature ? \\[0pt]
La morale est-elle affaire de sentiments ? \\[0pt]
La morale est-elle ennemie du bonheur ? \\[0pt]
La morale est-elle fondée sur la liberté ? \\[0pt]
La morale est-elle un art de vivre ? \\[0pt]
La morale est-elle une preuve de la liberté ? \\[0pt]
La morale est-elle une science ? \\[0pt]
La morale et le droit \\[0pt]
La morale peut-elle être personnelle ? \\[0pt]
La morale requiert-elle un fondement ? \\[0pt]
La moralité des lois \\[0pt]
La mort dans l'âme \\[0pt]
La mort de l'art \\[0pt]
La mort fait-elle partie de la vie ? \\[0pt]
La motivation et le jugement moral \\[0pt]
L'amour \\[0pt]
L'amour de la liberté \\[0pt]
L'amour de la nature \\[0pt]
L'amour de l'art \\[0pt]
L'amour de soi \\[0pt]
L'amour est-il aveugle ? \\[0pt]
L'amour est-il sans raison ? \\[0pt]
L'amour et la haine \\[0pt]
L'amour et l'amitié \\[0pt]
L'amour et la mort \\[0pt]
L'amour et la philosophie \\[0pt]
L'amour et l'humanité \\[0pt]
L'amour peut-il être absolu ? \\[0pt]
L'amour-propre \\[0pt]
L'amour vrai \\[0pt]
La multitude \\[0pt]
La musique est-elle un langage ? \\[0pt]
La musique exprime-t-elle quelque chose ? \\[0pt]
L'anachronisme \\[0pt]
La naissance \\[0pt]
La naïveté \\[0pt]
L'analogie \\[0pt]
L'analyse \\[0pt]
L'analyse du vécu \\[0pt]
L'anarchie \\[0pt]
La nation est-elle dépassée ? \\[0pt]
La nature artiste \\[0pt]
La nature a-t-elle des droits ? \\[0pt]
La nature a-t-elle une histoire ? \\[0pt]
La nature a-t-elle une valeur ? \\[0pt]
La nature des choses \\[0pt]
La nature des objets mathématiques \\[0pt]
La nature, est-ce le donné ? \\[0pt]
La nature est-elle belle ? \\[0pt]
La nature est-elle muette ? \\[0pt]
La nature est-elle sacrée ? \\[0pt]
La nature est-elle sauvage ? \\[0pt]
La nature et la grâce \\[0pt]
« La nature fait bien les choses » \\[0pt]
La nature humaine \\[0pt]
La nature morte \\[0pt]
La nature nous dicte-t-elle des fins ? \\[0pt]
La nature peut-elle être un modèle ? \\[0pt]
L'anecdotique \\[0pt]
La nécessité \\[0pt]
La nécessité des contradictions \\[0pt]
La nécessité des signes \\[0pt]
La négation \\[0pt]
La négligence \\[0pt]
La négligence est-elle une faute ? \\[0pt]
La négociation \\[0pt]
La neutralité \\[0pt]
Langage et communication \\[0pt]
Langage et réalité \\[0pt]
Langage, langue, parole \\[0pt]
L'angoisse \\[0pt]
Langue et parole \\[0pt]
L'animal \\[0pt]
L'animal apprend-il à l'homme quelque chose sur l'homme ? \\[0pt]
L'animal domestique \\[0pt]
L'animalité \\[0pt]
L'animal nous apprend-il quelque chose sur l'homme ? \\[0pt]
La noblesse \\[0pt]
L'anomalie \\[0pt]
La non-contradiction : principe logique ou ontologique ? \\[0pt]
L'anonymat \\[0pt]
L'anormal \\[0pt]
La normalité \\[0pt]
La normativité \\[0pt]
La norme et le fait \\[0pt]
La norme et son application \\[0pt]
La nostalgie \\[0pt]
La notion de barbarie a-t-elle un sens ? \\[0pt]
La notion de caractère relève-t- elle de la morale ou de la psychologie ? \\[0pt]
La notion de communauté morale \\[0pt]
La notion de loi a-t-elle une unité ? \\[0pt]
La notion de point de vue \\[0pt]
La notion d'époque \\[0pt]
La notion de responsabilité est-elle suspecte ? \\[0pt]
La notion de souveraineté partagée \\[0pt]
La nouveauté \\[0pt]
L'anthropocentrisme \\[0pt]
L'anticipation \\[0pt]
La nuance \\[0pt]
La nudité \\[0pt]
La paix \\[0pt]
La paix est-elle l'absence de guerre ? \\[0pt]
La paix est-elle moins naturelle que la guerre ? \\[0pt]
La paix est-elle toujours préférable ? \\[0pt]
La paix n'est-elle que l'absence de conflit ? \\[0pt]
La parenté \\[0pt]
La paresse \\[0pt]
La parole \\[0pt]
La participation \\[0pt]
La passion de la vérité \\[0pt]
La passion de l'égalité \\[0pt]
La passion du juste \\[0pt]
La paternité \\[0pt]
L'apathie \\[0pt]
La patience \\[0pt]
La pauvreté \\[0pt]
La peine capitale \\[0pt]
La peinture apprend-elle à voir ? \\[0pt]
La pensée a-t-elle une histoire ? \\[0pt]
La pensée des machines \\[0pt]
La pensée est-elle en lutte avec le langage ? \\[0pt]
La pensée formelle \\[0pt]
La pensée peut-elle s'écrire ? \\[0pt]
La pensée scientifique \\[0pt]
La pensée technique \\[0pt]
La perception musicale \\[0pt]
La perfectibilité \\[0pt]
La perfection \\[0pt]
La perfection morale \\[0pt]
La persécution \\[0pt]
La persévérance \\[0pt]
La personnalité \\[0pt]
La personne et la mémoire \\[0pt]
La perspective \\[0pt]
La pertinence \\[0pt]
La perversité \\[0pt]
La peur \\[0pt]
La peur de la mort \\[0pt]
La peur de la nature \\[0pt]
La peur de l'autre \\[0pt]
La peur de l'avenir \\[0pt]
La peur de la vérité \\[0pt]
La peur de l'inconnu \\[0pt]
La peur de manquer \\[0pt]
La philanthropie \\[0pt]
La philosophie doit-elle être systématique ? \\[0pt]
La philosophie doit-elle se préoccuper du salut ? \\[0pt]
La philosophie est-elle la servante de la science ? \\[0pt]
La philosophie et le sens commun \\[0pt]
La philosophie peut-elle disparaître ? \\[0pt]
La philosophie peut-elle se passer de l'idée de vérité ? \\[0pt]
La philosophie peut-elle se passer de théologie ? \\[0pt]
La philosophie première \\[0pt]
La physique est-elle le modèle de toutes les sciences ? \\[0pt]
La physique et la chimie \\[0pt]
La pitié \\[0pt]
La pitié est-elle dangereuse ? \\[0pt]
La pitié est-elle un sentiment moral ? \\[0pt]
La place d'autrui \\[0pt]
La plénitude \\[0pt]
La pluralité \\[0pt]
La pluralité des arts \\[0pt]
La pluralité des biens \\[0pt]
La pluralité des cultures \\[0pt]
La pluralité des langues \\[0pt]
La pluralité des méthodes en sociologie \\[0pt]
La pluralité des mondes \\[0pt]
La pluralité des sciences \\[0pt]
La pluralité des valeurs \\[0pt]
La poésie et l'idée \\[0pt]
La poésie pense-t-elle ? \\[0pt]
La polémique \\[0pt]
La politesse \\[0pt]
La politique échappe-t-elle à l'exigence de vérité ? \\[0pt]
La politique est-elle affaire de science ? \\[0pt]
La politique est-elle l'affaire de tous ? \\[0pt]
La politique et l'utopie \\[0pt]
La politique peut-elle échapper au mythe ? \\[0pt]
L'aporie \\[0pt]
La possibilité \\[0pt]
La potentialité \\[0pt]
L'apparence \\[0pt]
L'appel \\[0pt]
L'appropriation \\[0pt]
L'approximation \\[0pt]
La pratique de l'espace \\[0pt]
La précaution peut-elle être un principe ? \\[0pt]
La précision \\[0pt]
La première fois \\[0pt]
La première vérité \\[0pt]
La présence \\[0pt]
La présence à soi \\[0pt]
La présence d'autrui nous évite-t-elle la solitude ? \\[0pt]
La présence d'esprit \\[0pt]
La présence du passé \\[0pt]
La présomption \\[0pt]
La preuve \\[0pt]
La prévision \\[0pt]
L'\emph{a priori} \\[0pt]
L'a priori peut-il être historique ? \\[0pt]
La prison \\[0pt]
La prison est-elle utile ? \\[0pt]
La privation \\[0pt]
La probabilité \\[0pt]
La profondeur \\[0pt]
La promenade \\[0pt]
La promesse \\[0pt]
La promesse et le contrat \\[0pt]
La promesse n'est-elle qu'un acte de langage ? \\[0pt]
La propagation des idées \\[0pt]
La proportion \\[0pt]
La proposition \\[0pt]
La propriété \\[0pt]
La protection \\[0pt]
La providence \\[0pt]
La prudence \\[0pt]
La psychologie dit-elle ce qu'est le moi ? \\[0pt]
La publicité \\[0pt]
La pudeur \\[0pt]
La puissance \\[0pt]
La puissance de la technique \\[0pt]
La puissance de l'État \\[0pt]
La puissance du faux \\[0pt]
La puissance du langage \\[0pt]
La pulsion \\[0pt]
La punition \\[0pt]
La pureté \\[0pt]
La qualité \\[0pt]
La question de l'origine \\[0pt]
La question : « qui ? » \\[0pt]
La radicalité \\[0pt]
La raison a-t-elle des limites ? \\[0pt]
La raison a-t-elle le droit d'expliquer ce que morale condamne ? \\[0pt]
La raison a-t-elle une histoire ? \\[0pt]
La raison d'État \\[0pt]
La raison du plus fort \\[0pt]
La raison est-elle suffisante ? \\[0pt]
La raison est-elle un instrument ? \\[0pt]
La raison et le calcul \\[0pt]
La raison peut-elle errer ? \\[0pt]
La raison peut-elle s'opposer à elle-même ? \\[0pt]
La raison s'identifie-t-elle au calcul ? \\[0pt]
La rationalité du langage \\[0pt]
La rationalité économique \\[0pt]
L'arbitraire \\[0pt]
L'architecte et le législateur \\[0pt]
L'architecture du monde \\[0pt]
La réalité \\[0pt]
La réalité de l'idéal \\[0pt]
La réalité de l'idée \\[0pt]
La réalité du futur \\[0pt]
La réalité du possible \\[0pt]
La réalité du progrès \\[0pt]
La réalité du temps \\[0pt]
La réalité peut-elle être virtuelle ? \\[0pt]
La réalité physique \\[0pt]
La recherche de l'absolu \\[0pt]
La recherche de la vérité \\[0pt]
La recherche des origines \\[0pt]
La recherche du bonheur \\[0pt]
La réciprocité \\[0pt]
La reconnaissance \\[0pt]
La rectitude \\[0pt]
La référence \\[0pt]
La référence à la mémoire en histoire \\[0pt]
La réflexion \\[0pt]
La réflexion sur l'expérience participe-t-elle de l'expérience ? \\[0pt]
La réforme \\[0pt]
La réfutation \\[0pt]
La règle et l'exception \\[0pt]
La régression à l'infini \\[0pt]
La régulation \\[0pt]
La relation \\[0pt]
La relation à autrui est-elle symétrique ? \\[0pt]
La religion civile \\[0pt]
La religion peut-elle se passer du recours au sacré ? \\[0pt]
La religion peut-elle suppléer la raison ? \\[0pt]
La réminiscence \\[0pt]
La renaissance \\[0pt]
La rencontre \\[0pt]
La réparation \\[0pt]
La répétition \\[0pt]
La représentation \\[0pt]
La reproduction \\[0pt]
La reproduction des œuvres d'art nuit-elle à l'art ? \\[0pt]
La réputation \\[0pt]
La résilience \\[0pt]
La résistance du réel \\[0pt]
La résolution \\[0pt]
La responsabilité \\[0pt]
La responsabilité collective \\[0pt]
La ressemblance \\[0pt]
La révélation \\[0pt]
La rêverie \\[0pt]
La révolte \\[0pt]
La révolte peut-elle être un droit ? \\[0pt]
La révolution \\[0pt]
L'argent \\[0pt]
L'argumentation \\[0pt]
L'argument d'autorité \\[0pt]
La rhétorique \\[0pt]
La rhétorique a-t-elle une valeur ? \\[0pt]
La richesse \\[0pt]
La richesse du sensible \\[0pt]
La richesse intérieure \\[0pt]
La rigueur \\[0pt]
La rigueur de la loi \\[0pt]
L'art abstrait \\[0pt]
L'art a-t-il des règles ? \\[0pt]
L'art a-t-il une histoire ? \\[0pt]
L'art contre la beauté ? \\[0pt]
L'art de gouverner \\[0pt]
L'art du mensonge \\[0pt]
L'art est-il affaire de goût ? \\[0pt]
L'art est-il interprétation ? \\[0pt]
L'art est-il un divertissement ? \\[0pt]
L'art est-il une connaissance ? \\[0pt]
L'art est-il une critique de la culture ? \\[0pt]
L'art et la manière \\[0pt]
L'art et la mort \\[0pt]
L'art et le divin \\[0pt]
L'art et le temps \\[0pt]
L'artifice \\[0pt]
L'artificiel \\[0pt]
L'artiste est-il le technicien du beau ? \\[0pt]
L'artiste sait-il ce qu'il fait ? \\[0pt]
L'artiste travaille-t-il ? \\[0pt]
L'art n'est-il qu'une question de sentiment ? \\[0pt]
L'art nous enseigne-t-il quelque chose ? \\[0pt]
L'art nous ramène-t-il à la réalité ? \\[0pt]
L'art peut-il changer le monde ? \\[0pt]
L'art peut-il être enseigné ? \\[0pt]
L'art peut-il n'être pas conceptuel ? \\[0pt]
L'art peut-il se passer de règles ? \\[0pt]
L'art peut-il se passer des critères du beau et du laid ? \\[0pt]
L'art peut-il se passer d'œuvres ? \\[0pt]
L'art populaire \\[0pt]
L'art pour l'art \\[0pt]
L'art sait-il montrer ce que le langage ne peut pas dire ? \\[0pt]
La ruine \\[0pt]
La rumeur \\[0pt]
La rupture \\[0pt]
La ruse \\[0pt]
La ruse technique \\[0pt]
La sagesse du corps \\[0pt]
La sagesse et l'expérience \\[0pt]
La sagesse rend-elle heureux ? \\[0pt]
La santé \\[0pt]
L'ascèse \\[0pt]
L'ascétisme \\[0pt]
La science a-t-elle des limites ? \\[0pt]
La science a-t-elle toujours raison ? \\[0pt]
La science commence-t-elle avec la perception ? \\[0pt]
La science connaît-elle ses limites ? \\[0pt]
La science désenchante-t-elle le monde ? \\[0pt]
La science doit-elle nous fournir des représentations du monde ? \\[0pt]
La science et le mythe \\[0pt]
La science explique-t-elle le réel ? \\[0pt]
La science nous éloigne-t-elle des choses ? \\[0pt]
La science pense-t-elle ? \\[0pt]
La science peut-elle errer ? \\[0pt]
La science peut-elle se passer de fondement ? \\[0pt]
La science peut-elle se passer de méthode ? \\[0pt]
La science peut-elle tout expliquer ? \\[0pt]
La sculpture \\[0pt]
La sécurité \\[0pt]
La séduction \\[0pt]
La sensation est-elle une connaissance ? \\[0pt]
La sensibilité \\[0pt]
La sensibilité peut-elle délivrer une vérité ? \\[0pt]
La séparation \\[0pt]
La séparation des pouvoirs \\[0pt]
La sérénité \\[0pt]
La sévérité \\[0pt]
La sexualité \\[0pt]
La signification \\[0pt]
La signification des mots \\[0pt]
La signification et l'usage \\[0pt]
La simplicité \\[0pt]
La sincérité \\[0pt]
La singularité \\[0pt]
La singularité du réel \\[0pt]
La singularité en morale \\[0pt]
La situation \\[0pt]
La sobriété \\[0pt]
La société des nations \\[0pt]
La société existe-t-elle ? \\[0pt]
La sociologie de l'art \\[0pt]
La solitude \\[0pt]
La sollicitude \\[0pt]
La souffrance a-t-elle un sens ? \\[0pt]
La souffrance d'autrui \\[0pt]
La souffrance morale \\[0pt]
La souffrance peut-elle être partagée ? \\[0pt]
La soumission \\[0pt]
La souveraineté \\[0pt]
La souveraineté du peuple \\[0pt]
La souveraineté est-elle indivisible ? \\[0pt]
La souveraineté peut-elle se partager ? \\[0pt]
La spéculation \\[0pt]
La spontanéité \\[0pt]
L'assentiment \\[0pt]
L'asservissement de la pensée \\[0pt]
L'association \\[0pt]
L'association des idées \\[0pt]
La superstition \\[0pt]
La surface et la profondeur \\[0pt]
La survie \\[0pt]
La sympathie \\[0pt]
La sympathie peut-elle tenir lieu de moralité ? \\[0pt]
La systématicité \\[0pt]
La table rase \\[0pt]
La technique fait-elle des miracles ? \\[0pt]
La technique n'est-elle qu'un prolongement de nos organes ? \\[0pt]
La technique peut-elle améliorer l'homme ? \\[0pt]
La technocratie \\[0pt]
La téléologie \\[0pt]
La témérité \\[0pt]
La tempérance \\[0pt]
La tendance \\[0pt]
La tentation \\[0pt]
La terre \\[0pt]
La terre est-elle un bien commun ? \\[0pt]
La terre et le ciel \\[0pt]
La Terre et le Ciel \\[0pt]
La terreur \\[0pt]
L'athéisme \\[0pt]
La théologie est-elle une science ? \\[0pt]
La théologie rationnelle \\[0pt]
La tolérance \\[0pt]
La tolérance a-t-elle des limites ? \\[0pt]
La tolérance est-elle une vertu ? \\[0pt]
L'atome \\[0pt]
La totalité \\[0pt]
La toute-puissance \\[0pt]
La toute-puissance de la pensée \\[0pt]
La trace \\[0pt]
La trace et l'indice \\[0pt]
La tradition \\[0pt]
La traductibilité des langues \\[0pt]
La traduction \\[0pt]
La tragédie \\[0pt]
La trahison \\[0pt]
La tranquillité \\[0pt]
La transcendance \\[0pt]
La transgression \\[0pt]
La transmission \\[0pt]
La transparence \\[0pt]
La tristesse \\[0pt]
La tristesse du fini \\[0pt]
La tristesse est-elle mauvaise ? \\[0pt]
L'attachement \\[0pt]
L'attente \\[0pt]
L'attention \\[0pt]
La tyrannie \\[0pt]
La tyrannie du bonheur \\[0pt]
L'audace \\[0pt]
L'au-delà \\[0pt]
L'auteur et le créateur \\[0pt]
L'authenticité \\[0pt]
L'autobiographie \\[0pt]
L'automate \\[0pt]
L'autonomie \\[0pt]
L'autonomie du vivant \\[0pt]
L'autoportrait \\[0pt]
L'autorité \\[0pt]
L'autorité de la parole \\[0pt]
L'autorité de la science \\[0pt]
L'autorité de l'écrit \\[0pt]
L'autorité des savants \\[0pt]
L'autorité morale \\[0pt]
L'autre monde \\[0pt]
La valeur artistique \\[0pt]
La valeur d'échange \\[0pt]
La valeur de la loi \\[0pt]
La valeur de la vérité \\[0pt]
La valeur de l'exemple \\[0pt]
La valeur des choses \\[0pt]
La valeur des conventions \\[0pt]
La valeur des exceptions \\[0pt]
La valeur des hypothèses \\[0pt]
La valeur d'une action se mesure-t-elle à ses conséquences ? \\[0pt]
La valeur du témoignage \\[0pt]
La valeur du temps \\[0pt]
La valeur du travail \\[0pt]
La valeur et le prix \\[0pt]
« La valeur travail » \\[0pt]
La vanité \\[0pt]
La vanité est-elle toujours sans objet ? \\[0pt]
L'avant-garde \\[0pt]
L'avarice \\[0pt]
La variété \\[0pt]
La vengeance \\[0pt]
L'avenir \\[0pt]
L'avenir de l'humanité \\[0pt]
L'avenir est-il imaginable ? \\[0pt]
L'avenir est-il sans image ? \\[0pt]
L'avenir existe-t-il ? \\[0pt]
L'aventure \\[0pt]
La vérification \\[0pt]
La vérité a-t-elle une histoire ? \\[0pt]
La vérité de la religion \\[0pt]
La vérité demande-t-elle du courage ? \\[0pt]
La vérité des images \\[0pt]
La vérité des sentiments \\[0pt]
La vérité en politique \\[0pt]
La vérité est-elle fille de son temps ? \\[0pt]
La vérité est-elle relative ? \\[0pt]
La vérité scientifique est-elle relative ? \\[0pt]
La vertu \\[0pt]
La vertu de l'abstraction \\[0pt]
La vertu du citoyen \\[0pt]
La vertu est-elle une forme de santé ? \\[0pt]
La vertu peut-elle être excessive ? \\[0pt]
L'aveu \\[0pt]
L'aveuglement \\[0pt]
La vie active \\[0pt]
La vie de la langue \\[0pt]
La vie de l'esprit \\[0pt]
La vie du droit \\[0pt]
La vie est-elle une valeur ? \\[0pt]
La vie est-elle un songe ? \\[0pt]
« La vie est un songe » \\[0pt]
La vie heureuse \\[0pt]
La vieillesse \\[0pt]
La vie intérieure \\[0pt]
La vie ordinaire \\[0pt]
La vie peut-elle être éternelle ? \\[0pt]
La vie peut-elle être sans histoire ? \\[0pt]
La vie privée \\[0pt]
La vie psychique \\[0pt]
La vie quotidienne \\[0pt]
La vie sociale est-elle une comédie ? \\[0pt]
La ville \\[0pt]
La ville et la campagne \\[0pt]
La violence \\[0pt]
La violence a-t-elle des degrés ? \\[0pt]
La violence de l'État \\[0pt]
La violence verbale \\[0pt]
La virtualité \\[0pt]
La vision scientifique du monde \\[0pt]
La vocation \\[0pt]
La voix \\[0pt]
La voix de la conscience \\[0pt]
La voix du peuple \\[0pt]
La volonté de croire \\[0pt]
La volonté du peuple \\[0pt]
La volonté générale \\[0pt]
La volonté peut-elle être indéterminée ? \\[0pt]
La volonté peut-elle être libre ? \\[0pt]
La volupté \\[0pt]
La vraisemblance \\[0pt]
La vulgarisation \\[0pt]
La vulgarité \\[0pt]
La vulnérabilité \\[0pt]
L'axiome \\[0pt]
Le barbare \\[0pt]
Le baroque \\[0pt]
Le bavardage \\[0pt]
Le beau a-t-il une histoire ? \\[0pt]
Le beau est-il aimable ? \\[0pt]
Le beau et l'agréable \\[0pt]
Le beau et le bien \\[0pt]
Le beau et le sublime \\[0pt]
Le beau naturel \\[0pt]
Le beau n'est-il qu'une idée ? \\[0pt]
Le besoin d'absolu \\[0pt]
Le besoin de beauté \\[0pt]
Le besoin de philosophie \\[0pt]
Le besoin de rêve \\[0pt]
Le besoin de vérité \\[0pt]
Le bien commun \\[0pt]
Le bien et le beau \\[0pt]
Le biologisme \\[0pt]
Le bon et l'utile \\[0pt]
Le bon goût \\[0pt]
Le bonheur comporte-t-il des degrés ? \\[0pt]
Le bonheur dans le mal \\[0pt]
Le bonheur de la passion est-il sans lendemain ? \\[0pt]
Le bonheur du citoyen \\[0pt]
Le bonheur est-il affaire de calcul ? \\[0pt]
Le bonheur est-il une valeur morale ? \\[0pt]
Le bonheur n'est-il qu'une idée ? \\[0pt]
Le bon plaisir \\[0pt]
Le bon sens \\[0pt]
Le bon usage des passions \\[0pt]
Le bruit \\[0pt]
Le cadavre \\[0pt]
Le caractère \\[0pt]
L'écart \\[0pt]
Le cas de conscience \\[0pt]
Le certain et le probable \\[0pt]
L'échange \\[0pt]
L'échange inégal \\[0pt]
Le changement \\[0pt]
L'échange symbolique \\[0pt]
Le chant \\[0pt]
Le chaos \\[0pt]
Le charme et la grâce \\[0pt]
Le chemin \\[0pt]
Le choix \\[0pt]
Le choix d'un destin \\[0pt]
Le choix peut-il être éclairé ? \\[0pt]
Le ciel et la terre \\[0pt]
Le cinéma est-il un art comme les autres ? \\[0pt]
Le clair et le distinct \\[0pt]
Le classicisme \\[0pt]
Le code \\[0pt]
Le cœur \\[0pt]
Le cœur et la raison \\[0pt]
L'école de la vie \\[0pt]
Le comédien \\[0pt]
Le comique et le tragique \\[0pt]
Le commencement \\[0pt]
Le commerce \\[0pt]
Le commerce des corps \\[0pt]
Le commerce des hommes \\[0pt]
Le commun \\[0pt]
Le compromis \\[0pt]
Le concept \\[0pt]
Le concept d'évolution \\[0pt]
Le concept et la vie \\[0pt]
Le concept et l'image \\[0pt]
Le concret \\[0pt]
Le concret et l'abstrait \\[0pt]
Le conflit est-il au fondement de la vie sociale ? \\[0pt]
Le conformisme \\[0pt]
L'économie est-elle politique ? \\[0pt]
L'économie fait-elle partie des sciences humaines ? \\[0pt]
Le conseil \\[0pt]
Le consensus \\[0pt]
Le consentement \\[0pt]
Le contemporain \\[0pt]
Le contingent \\[0pt]
Le continu \\[0pt]
Le contrat \\[0pt]
Le corps dit-il quelque chose ? \\[0pt]
Le corps du prince \\[0pt]
Le corps est-il porteur de valeurs ? \\[0pt]
Le corps et la machine \\[0pt]
Le corps et l'âme \\[0pt]
Le corps humain \\[0pt]
Le corps politique \\[0pt]
Le corps propre \\[0pt]
Le cosmopolitisme \\[0pt]
Le cosmopolitisme peut-il être réaliste ? \\[0pt]
Le coupable \\[0pt]
Le couple \\[0pt]
Le courage \\[0pt]
Le courage est-il une vertu ? \\[0pt]
Le cours du temps \\[0pt]
Le cri \\[0pt]
Le critère \\[0pt]
L'écrit et l'oral \\[0pt]
L'écriture des lois \\[0pt]
L'écriture et la parole \\[0pt]
Le cynisme \\[0pt]
Le danger \\[0pt]
Le débat \\[0pt]
Le défaut \\[0pt]
Le déguisement \\[0pt]
Le dernier mot \\[0pt]
Le désaccord \\[0pt]
Le désespoir \\[0pt]
Le déshonneur \\[0pt]
Le désintéressement \\[0pt]
Le désir de connaissance \\[0pt]
Le désir de domination \\[0pt]
Le désir de dominer \\[0pt]
Le désir de gloire \\[0pt]
Le désir de l'autre \\[0pt]
Le désir de pouvoir \\[0pt]
Le désir de savoir \\[0pt]
Le désir d'éternité \\[0pt]
Le désir de vérité \\[0pt]
Le désir d'infini \\[0pt]
Le désir est-il sans limite ? \\[0pt]
Le désir et la loi \\[0pt]
Le désir et le manque \\[0pt]
Le désir n'est-il qu'inquiétude ? \\[0pt]
Le désir peut-il atteindre son objet ? \\[0pt]
Le désœuvrement \\[0pt]
Le désordre \\[0pt]
Le désordre des choses \\[0pt]
Le désordre est-il étonnant ? \\[0pt]
Le désordre est-il un mal ? \\[0pt]
Le despotisme \\[0pt]
Le dessin \\[0pt]
Le dessin et la couleur \\[0pt]
Le destin \\[0pt]
Le désuet \\[0pt]
Le détachement \\[0pt]
Le détail \\[0pt]
Le déterminisme \\[0pt]
Le deuil \\[0pt]
Le développement moral \\[0pt]
Le devenir \\[0pt]
Le devoir s'apprend-il ? \\[0pt]
Le dialogue \\[0pt]
Le dialogue des philosophes \\[0pt]
Le dictionnaire \\[0pt]
Le dieu des philosophes \\[0pt]
Le Dieu des philosophes \\[0pt]
Le dilemme \\[0pt]
Le dire et le faire \\[0pt]
Le discernement \\[0pt]
Le discontinu \\[0pt]
Le divers \\[0pt]
Le divertissement \\[0pt]
Le divertissement est-il vain ? \\[0pt]
Le divin \\[0pt]
Le dogmatisme \\[0pt]
Le domestique \\[0pt]
Le don \\[0pt]
Le don de soi \\[0pt]
Le don et l'échange \\[0pt]
Le donné \\[0pt]
Le donné et le construit \\[0pt]
Le double \\[0pt]
Le doute est-il une faiblesse de la pensée ? \\[0pt]
Le drame \\[0pt]
Le droit à l'erreur \\[0pt]
Le droit au bonheur \\[0pt]
Le droit au travail \\[0pt]
Le droit de disposer de son corps \\[0pt]
Le droit de la guerre \\[0pt]
Le droit de la majorité \\[0pt]
Le droit de punir \\[0pt]
Le droit de résistance \\[0pt]
Le droit des animaux \\[0pt]
Le droit des peuples \\[0pt]
Le droit de veto \\[0pt]
Le droit de vie et de mort \\[0pt]
Le droit de vivre \\[0pt]
Le droit d'ingérence \\[0pt]
Le droit d'intervention \\[0pt]
Le droit divin \\[0pt]
Le droit du plus faible \\[0pt]
Le droit international \\[0pt]
Le droit peut-il être naturel ? \\[0pt]
Le dualisme \\[0pt]
L'éducation des esprits \\[0pt]
L'éducation est-elle par nature conservatrice ? \\[0pt]
L'éducation peut-elle être sentimentale ? \\[0pt]
L'éducation physique \\[0pt]
L'éducation relève-t-elle du politique ? \\[0pt]
Le fait de vivre est-il un bien en soi ? \\[0pt]
Le fait divers \\[0pt]
Le fait et le droit \\[0pt]
Le fanatisme \\[0pt]
Le faux en art \\[0pt]
Le faux et le fictif \\[0pt]
Le fédéralisme \\[0pt]
Le féminin et le masculin \\[0pt]
Le fétichisme \\[0pt]
L'effet de la musique \\[0pt]
L'efficacité est-elle une vertu ? \\[0pt]
L'effort \\[0pt]
Le finalisme \\[0pt]
Le fini et l'infini \\[0pt]
Le flegme \\[0pt]
Le fonctionnel et le beau \\[0pt]
Le fondement \\[0pt]
Le fond et la forme \\[0pt]
Le for intérieur \\[0pt]
Le formalisme \\[0pt]
Le fragment \\[0pt]
Le futur est-il contingent ? \\[0pt]
Le futur est-il nécessairement contingent ? \\[0pt]
L'égalité \\[0pt]
L'égalité des chances \\[0pt]
L'égalité des citoyens \\[0pt]
L'égalité des hommes est-elle un fait ou une idée ? \\[0pt]
L'égalité des sexes \\[0pt]
L'égalité est-elle souhaitable ? \\[0pt]
Légalité et légitimité \\[0pt]
Le génie \\[0pt]
Le génie du lieu \\[0pt]
Le génie du mal \\[0pt]
Le genre humain \\[0pt]
Le genre humain : unité ou pluralité ? \\[0pt]
Le geste \\[0pt]
Le geste et la parole \\[0pt]
L'égoïsme \\[0pt]
Le goût \\[0pt]
Le goût de la polémique \\[0pt]
Le goût des autres \\[0pt]
Le goût du pouvoir \\[0pt]
Le goût est-il nécessairement subjectif ? \\[0pt]
Le goût et la distinction \\[0pt]
Le gouvernement par le peuple est-il nécessairement pour le peuple ? \\[0pt]
Le grandiose \\[0pt]
Le hasard \\[0pt]
Le hasard fait-il bien les choses ? \\[0pt]
Le haut et le bas \\[0pt]
Le je-ne-sais-quoi \\[0pt]
Le jeu \\[0pt]
Le jeu de mots \\[0pt]
Le jeu des apparences \\[0pt]
Le jeu et la règle \\[0pt]
Le joli, le beau \\[0pt]
Le jugement de l'histoire \\[0pt]
Le jugement dernier \\[0pt]
Le jugement de valeur est-il indifférent à la vérité ? \\[0pt]
Le jugement politique \\[0pt]
Le jugement pratique \\[0pt]
Le juste et le bien \\[0pt]
Le juste et le légal \\[0pt]
Le laboratoire \\[0pt]
Le laid \\[0pt]
Le langage de la pensée \\[0pt]
Le langage du corps \\[0pt]
Le langage est-il assimilable à un outil ? \\[0pt]
Le langage est-il d'essence poétique ? \\[0pt]
Le langage est-il l'instrument de la pensée ? \\[0pt]
Le langage fait-il obstacle à la connaissance ? \\[0pt]
Le langage scientifique \\[0pt]
L'élégance \\[0pt]
Le législateur et le juge \\[0pt]
L'élémentaire \\[0pt]
Le libertin \\[0pt]
Le libre arbitre \\[0pt]
Le lien social \\[0pt]
Le lieu \\[0pt]
Le lieu commun \\[0pt]
L'éloge de la démesure \\[0pt]
Le loisir \\[0pt]
Le loisir caractérise-t-il l'homme libre ? \\[0pt]
Le luxe \\[0pt]
Le magistrat \\[0pt]
Le maître et le disciple \\[0pt]
Le mal \\[0pt]
Le mal apparaît-il toujours ? \\[0pt]
Le malentendu \\[0pt]
Le mal est-il une réalité objective ? \\[0pt]
Le malheur \\[0pt]
Le malin plaisir \\[0pt]
Le mal peut-il être absolu ? \\[0pt]
L'émancipation \\[0pt]
Le manque de culture \\[0pt]
Le marché \\[0pt]
Le marché de l'art \\[0pt]
Le masque \\[0pt]
Le mauvais goût \\[0pt]
L'embarras \\[0pt]
L'embarras du choix \\[0pt]
Le mécanisme \\[0pt]
Le meilleur \\[0pt]
Le meilleur régime \\[0pt]
Le meilleur régime politique \\[0pt]
Le mélange \\[0pt]
Le même et l'autre \\[0pt]
Le mensonge \\[0pt]
Le mensonge de l'art ? \\[0pt]
Le mensonge en politique \\[0pt]
Le mépris \\[0pt]
Le mépris peut-il être justifié ? \\[0pt]
Le mérite \\[0pt]
Le mesurable \\[0pt]
Le métier \\[0pt]
Le métier d'homme \\[0pt]
Le mien et le tien \\[0pt]
Le mieux est-il l'ennemi du bien ? \\[0pt]
Le milieu \\[0pt]
Le miracle \\[0pt]
Le miroir \\[0pt]
Le modèle \\[0pt]
Le moi \\[0pt]
Le moi est-il une collection d'idées ? \\[0pt]
Le moi est-il une illusion ? \\[0pt]
Le moi et ses images \\[0pt]
Le moindre mal \\[0pt]
Le monde à l'envers \\[0pt]
Le monde de l'animal \\[0pt]
Le monde de l'animal nous est-il étranger ? \\[0pt]
Le monde de l'art \\[0pt]
Le monde de la vie \\[0pt]
Le monde des machines \\[0pt]
Le monde des œuvres \\[0pt]
Le monde des physiciens \\[0pt]
Le monde des rêves \\[0pt]
Le monde des sens \\[0pt]
Le monde du rêve \\[0pt]
Le monde est-il éternel ? \\[0pt]
Le monde est-il un théâtre ? \\[0pt]
Le monstre \\[0pt]
Le monstrueux \\[0pt]
Le moralisme \\[0pt]
Le mot d'esprit \\[0pt]
Le mot et la chose \\[0pt]
L'émotion \\[0pt]
L'émotion esthétique \\[0pt]
Le mot juste \\[0pt]
Le mot vie a-t-il plusieurs sens ? \\[0pt]
Le mouvement \\[0pt]
Le mouvement de la pensée \\[0pt]
L'emploi du temps \\[0pt]
Le multiple \\[0pt]
Le musée \\[0pt]
Le mystère \\[0pt]
Le mysticisme \\[0pt]
Le mythe \\[0pt]
Le naïf \\[0pt]
Le narcissisme \\[0pt]
Le naturalisme \\[0pt]
Le naturel \\[0pt]
Le naturel et l'artificiel \\[0pt]
Le naturel et le normal \\[0pt]
L'encyclopédie \\[0pt]
Le néant \\[0pt]
Le négatif \\[0pt]
Le néologisme \\[0pt]
L'énergie \\[0pt]
L'enfance \\[0pt]
L'enfance de l'art \\[0pt]
L'enfance est-elle ce qui doit être surmonté ? \\[0pt]
L'enfance peut-elle servir de modèle ? \\[0pt]
L'enfer est-il véritablement pavé de bonnes intentions ? \\[0pt]
L'engagement \\[0pt]
Le nihilisme \\[0pt]
L'ennemi \\[0pt]
L'ennui \\[0pt]
Le noble et le vil \\[0pt]
Le nomade \\[0pt]
Le nombre \\[0pt]
Le nom propre \\[0pt]
Le nom propre et le nom commun \\[0pt]
Le non-sens \\[0pt]
L'enquête \\[0pt]
L'enthousiasme \\[0pt]
Le nu \\[0pt]
L'envie \\[0pt]
Le oui-dire \\[0pt]
Le pacifisme \\[0pt]
Le paradoxal \\[0pt]
Le paradoxe \\[0pt]
Le pardon \\[0pt]
Le pardon et l'oubli \\[0pt]
Le pari \\[0pt]
Le partage \\[0pt]
Le partage des biens \\[0pt]
Le partage des savoirs \\[0pt]
Le particulier \\[0pt]
Le passage à l'acte \\[0pt]
Le passé \\[0pt]
Le passé est-il objet de science ? \\[0pt]
Le passionné mérite-t-il d'être plaint ? \\[0pt]
Le pathologique \\[0pt]
Le paysage \\[0pt]
Le pays natal \\[0pt]
Le pédagogue \\[0pt]
Le peintre et son modèle \\[0pt]
Le pessimisme \\[0pt]
Le peuple a-t-il toujours raison ? \\[0pt]
Le peuple est-il souverain ? \\[0pt]
Le peuple possède-t-il une volonté ? \\[0pt]
Le phantasme \\[0pt]
L'éphémère \\[0pt]
Le phénomène \\[0pt]
Le philanthrope \\[0pt]
Le philosophe et l'écrivain \\[0pt]
Le philosophe et le médecin \\[0pt]
Le philosophe et l'enfant \\[0pt]
Le philosophe se détourne-t-il du réel ? \\[0pt]
Le plagiat \\[0pt]
Le plaisir \\[0pt]
Le plaisir de l'art \\[0pt]
Le plaisir esthétique est-il un plaisir ? \\[0pt]
Le plaisir esthétique suppose-t-il une culture esthétique ? \\[0pt]
Le plaisir est-il immoral ? \\[0pt]
Le plaisir et le bien \\[0pt]
Le plaisir se mérite-t-il ? \\[0pt]
Le pluralisme \\[0pt]
Le plus et le moins \\[0pt]
Le poids des circonstances \\[0pt]
Le poids du passé \\[0pt]
Le point de vue \\[0pt]
Le point de vue du spectateur \\[0pt]
Le populaire \\[0pt]
Le portrait \\[0pt]
Le possible \\[0pt]
Le possible et le réel \\[0pt]
Le possible et l'existence \\[0pt]
Le possible et l'impossible \\[0pt]
Le postulat et l'axiome \\[0pt]
Le pour et le contre \\[0pt]
Le pouvoir de la beauté \\[0pt]
Le pouvoir de l'argent \\[0pt]
Le pouvoir de la science \\[0pt]
Le pouvoir de l'image \\[0pt]
Le pouvoir des idées \\[0pt]
Le pouvoir des images \\[0pt]
Le pouvoir des meilleurs \\[0pt]
Le pouvoir des mots \\[0pt]
Le pouvoir des savants \\[0pt]
Le pouvoir peut-il se transmettre ? \\[0pt]
Le pouvoir politique est-il nécessairement coercitif ? \\[0pt]
Le pouvoir se partage-t-il ? \\[0pt]
Le pouvoir souverain \\[0pt]
Le pouvoir spirituel \\[0pt]
Le préjugé \\[0pt]
Le présent \\[0pt]
L'épreuve de la réalité \\[0pt]
Le principe de raison \\[0pt]
Le principe de réalité \\[0pt]
Le principe de simplicité \\[0pt]
Le privilège de l'original \\[0pt]
Le probable \\[0pt]
Le processus \\[0pt]
Le prochain \\[0pt]
Le proche et le lointain \\[0pt]
Le profane \\[0pt]
Le progrès \\[0pt]
Le progrès technique \\[0pt]
Le projet \\[0pt]
Le projet d'une paix perpétuelle est-il insensé ? \\[0pt]
Le propre \\[0pt]
Le provisoire \\[0pt]
Le psychologisme \\[0pt]
Le public et le privé \\[0pt]
Le pur et l'impur \\[0pt]
L'équilibre \\[0pt]
L'équilibre des pouvoirs \\[0pt]
L'équité \\[0pt]
L'équivalence \\[0pt]
L'équivalence des hypothèses \\[0pt]
L'équivocité \\[0pt]
L'équivoque \\[0pt]
Le quotidien \\[0pt]
Le raffinement \\[0pt]
Le raisonnement par l'absurde \\[0pt]
Le rapport de force \\[0pt]
Le rationnel et le raisonnable \\[0pt]
Le réalisme \\[0pt]
Le récit \\[0pt]
Le réel \\[0pt]
Le réel est-il ce qui résiste ? \\[0pt]
Le réel et l'idéal \\[0pt]
Le réel et l'impossible \\[0pt]
Le refus \\[0pt]
Le refus de l'autorité \\[0pt]
Le refus de la vérité \\[0pt]
Le regard \\[0pt]
Le regard de l'autre \\[0pt]
Le regard du photographe \\[0pt]
Le regard éloigné \\[0pt]
Le règne de l'homme \\[0pt]
Le règne des experts \\[0pt]
Le relativisme \\[0pt]
Le relativisme dans les sciences humaines \\[0pt]
Le repos \\[0pt]
Le respect \\[0pt]
Le ressentiment \\[0pt]
Le rêve \\[0pt]
Le rêve et la veille \\[0pt]
Le ridicule \\[0pt]
Le rien \\[0pt]
Le rigorisme \\[0pt]
Le rire \\[0pt]
Le risque \\[0pt]
Le risque calculé \\[0pt]
Le rôle des institutions \\[0pt]
L'érotisme \\[0pt]
L'erreur et la faute \\[0pt]
L'erreur et l'ignorance \\[0pt]
L'érudition \\[0pt]
Le rythme \\[0pt]
Le sacré \\[0pt]
Le sacré et le profane \\[0pt]
Le sacrifice \\[0pt]
Les âges de la vie \\[0pt]
Le salut \\[0pt]
Les amis \\[0pt]
Les anciens et les modernes \\[0pt]
Les animaux ont-ils des droits ? \\[0pt]
Les animaux pensent-ils ? \\[0pt]
Les animaux révèlent-ils ce que nous sommes ? \\[0pt]
Les animaux sont-ils nos amis ? \\[0pt]
Les apparences font-elles partie du monde ? \\[0pt]
Les apparences sont-elles toujours trompeuses ? \\[0pt]
Les archives \\[0pt]
Les arts appliqués \\[0pt]
Les arts décoratifs \\[0pt]
Les arts de la mémoire \\[0pt]
Le sauvage \\[0pt]
Le sauvage et le barbare \\[0pt]
Le sauvage et le cultivé \\[0pt]
Le savoir a-t-il besoin d'être fondé ? \\[0pt]
Le savoir du peintre \\[0pt]
Le savoir émancipe-t-il ? \\[0pt]
Le savoir est-il libérateur ? \\[0pt]
Le savoir-faire \\[0pt]
Le savoir peut-il être gratuit ? \\[0pt]
Le savoir se vulgarise-t-il ? \\[0pt]
Les beautés de la nature \\[0pt]
Les bénéfices du doute \\[0pt]
Les bêtes \\[0pt]
Les blessures de l'esprit \\[0pt]
Les bonnes intentions \\[0pt]
Les bonnes lois font-elles les bonnes mœurs ? \\[0pt]
Les bons sentiments \\[0pt]
Les canons de la beauté \\[0pt]
Les cas de conscience \\[0pt]
Les catégories \\[0pt]
Les causes et les conditions \\[0pt]
Les causes et les effets \\[0pt]
Les causes naturelles expliquent-elles la nature ? \\[0pt]
Les chemins de traverse \\[0pt]
Les choses \\[0pt]
Les choses ont-elles quelque chose en commun ? \\[0pt]
Les choses ont-elles une essence ? \\[0pt]
Les choses peuvent-elles avoir un sens ? \\[0pt]
Les choses sont-elles dans l'espace ? \\[0pt]
Les cinq sens \\[0pt]
Les circonstances \\[0pt]
L'esclavage \\[0pt]
L'esclave \\[0pt]
L'esclave et son maître \\[0pt]
Les conséquences de l'action \\[0pt]
Les contre-pouvoirs \\[0pt]
Les convenances \\[0pt]
Les conventions \\[0pt]
Les couleurs existent-elles en dehors de notre esprit ? \\[0pt]
Les croyances sont-elles utiles ? \\[0pt]
Les degrés de conscience \\[0pt]
Les devoirs envers nos proches \\[0pt]
Les dictionnaires \\[0pt]
Les difficultés de l'observation \\[0pt]
Les divisions entre les hommes \\[0pt]
Les dogmes \\[0pt]
Les droits de la nature \\[0pt]
Les droits de l'enfant \\[0pt]
Les droits de l'homme \\[0pt]
Le secret \\[0pt]
Les effets de l'esclavage \\[0pt]
Les éléments \\[0pt]
Le semblable \\[0pt]
Le sens commun \\[0pt]
Le sens commun existe-t-il ? \\[0pt]
Le sens de la mesure \\[0pt]
Le sens de la situation \\[0pt]
Le sens de la vie \\[0pt]
Le sens de l'État \\[0pt]
Le sens de l'histoire \\[0pt]
Le sens de l'Histoire \\[0pt]
Le sens de l'humour \\[0pt]
Le sens des mots \\[0pt]
Le sens des mots dépend-il de notre connaissance des choses ? \\[0pt]
Le sens d'un texte \\[0pt]
Le sens du sacrifice \\[0pt]
Le sens du travail \\[0pt]
Le sens et le non-sens \\[0pt]
Le sensible \\[0pt]
Le sensible peut-il fonder des certitudes ? \\[0pt]
Le sens interne \\[0pt]
Le sens musical \\[0pt]
Le sens pratique \\[0pt]
Le sens propre \\[0pt]
Le sentiment de l'existence \\[0pt]
Le sentiment de l'injustice \\[0pt]
Le sentiment d'injustice \\[0pt]
Le sentiment du libre arbitre \\[0pt]
Le sentiment esthétique \\[0pt]
Le sentiment moral \\[0pt]
Les entités mathématiques sont-elles des fictions ? \\[0pt]
Les envieux \\[0pt]
Les épreuves ont-elles un sens ? \\[0pt]
Le sérieux \\[0pt]
Le serment \\[0pt]
Les États ont-ils pour fin d'assurer la paix ? \\[0pt]
Les événements \\[0pt]
Les exceptions ont-elles une place en morale ? \\[0pt]
Les experts \\[0pt]
Les faits \\[0pt]
Les faits parlent-ils d'eux-mêmes ? \\[0pt]
Les faits sont-ils têtus ? \\[0pt]
Les fictions \\[0pt]
Les fins de l'éducation \\[0pt]
Les fins dernières \\[0pt]
Les formes de vie \\[0pt]
Les forts et les faibles \\[0pt]
Les frontières \\[0pt]
Les frontières de la connaissance \\[0pt]
Les fruits du travail \\[0pt]
Les grands hommes \\[0pt]
Les habitudes de pensée \\[0pt]
Les hasards de la vie \\[0pt]
Les hommes et les dieux \\[0pt]
Les hors-la-loi \\[0pt]
Les idées mènent-elles le monde ? \\[0pt]
Les idées ont-elles une histoire ? \\[0pt]
Les idées ont-elles une réalité ? \\[0pt]
Les idoles \\[0pt]
Le silence \\[0pt]
Les images empêchent-elles de penser ? \\[0pt]
Les images nous égarent-elles ? \\[0pt]
Les images peuvent-elles instruire ? \\[0pt]
Le simple et le complexe \\[0pt]
Le simulacre \\[0pt]
Les individus se réduisent-ils à leurs qualités ? \\[0pt]
Les inégalités sociales sont-elles inévitables ? \\[0pt]
Le singulier \\[0pt]
Le singulier est-il objet de connaissance ? \\[0pt]
Le singulier et le pluriel \\[0pt]
Les instruments de la pensée \\[0pt]
Les intentions et les conséquences \\[0pt]
Les interdits \\[0pt]
Les jeux de hasard \\[0pt]
Les langages formels \\[0pt]
Les leçons de l'expérience \\[0pt]
Les libertés publiques \\[0pt]
Les liens \\[0pt]
Les limites de la démocratie \\[0pt]
Les limites de la description \\[0pt]
Les limites de la raison \\[0pt]
Les limites de la tolérance \\[0pt]
Les limites de la vertu \\[0pt]
Les limites de l'expérience \\[0pt]
Les limites de l'humain \\[0pt]
Les limites de l'imagination \\[0pt]
Les limites de l'interprétation \\[0pt]
Les limites de ma langue sont-elles les limites de mon monde ? \\[0pt]
Les limites du corps \\[0pt]
Les limites du déterminisme \\[0pt]
Les limites du langage \\[0pt]
Les limites du possible \\[0pt]
Les limites du pouvoir \\[0pt]
Les limites du pouvoir politique \\[0pt]
Les lois causales \\[0pt]
Les lois de la guerre \\[0pt]
Les lois de l'hospitalité \\[0pt]
Les lois sont-elles sacrées ? \\[0pt]
Les machines \\[0pt]
Les maladies de l'âme \\[0pt]
Les maladies de l'esprit \\[0pt]
Les mathématiques et le monde sensible \\[0pt]
Les mathématiques peuvent-elles se passer de l'intuition ? \\[0pt]
Les mathématiques sont-elles le modèle de toute science ? \\[0pt]
Les mathématiques sont-elles une science de l'ordre ? \\[0pt]
Les mathématiques sont-elles un langage ? \\[0pt]
Les mathématiques sont-elles utiles au philosophe ? \\[0pt]
Les miracles \\[0pt]
Les miracles de la technique \\[0pt]
Les modèles \\[0pt]
Les mœurs \\[0pt]
Les mœurs et la morale \\[0pt]
Les mots et les choses \\[0pt]
Les mots justes \\[0pt]
Les mots ont-ils un pouvoir ? \\[0pt]
Les moyens de l'art \\[0pt]
Les moyens de l'autorité \\[0pt]
Les moyens et la fin \\[0pt]
Les moyens et les fins \\[0pt]
Les nombres gouvernent-ils le monde ? \\[0pt]
Les noms \\[0pt]
Les noms propres \\[0pt]
Les normes \\[0pt]
Les normes du savoir sont-elles universelles ? \\[0pt]
Les normes du vivant \\[0pt]
Les normes et les valeurs \\[0pt]
Les objets existent-ils ? \\[0pt]
Le sociologisme \\[0pt]
Le sociologue et l'historien \\[0pt]
Les œuvres d'art doivent-elles faire l'objet d'une évaluation morale ? \\[0pt]
Le soleil se lèvera demain \\[0pt]
Le sommeil et la veille \\[0pt]
Les opérations de la pensée \\[0pt]
Les origines de l'humanité \\[0pt]
L'ésotérisme \\[0pt]
Le souci \\[0pt]
Le souci de soi \\[0pt]
Le souverain bien \\[0pt]
L'espace dans la peinture \\[0pt]
L'espace et le lieu \\[0pt]
L'espace et le territoire \\[0pt]
L'espace intérieur \\[0pt]
L'espace public \\[0pt]
Les paroles et les actes \\[0pt]
« Les paroles s'envolent, les écrits restent » \\[0pt]
Les parties de l'âme \\[0pt]
L'espèce et l'individu \\[0pt]
Le spectacle \\[0pt]
Le spectacle de la nature \\[0pt]
Le spectacle de la pensée \\[0pt]
L'espérance \\[0pt]
Les personnes et les choses \\[0pt]
Les plaisirs \\[0pt]
Les pouvoirs de la religion \\[0pt]
Les preuves de la liberté \\[0pt]
L'esprit critique \\[0pt]
L'esprit de finesse \\[0pt]
L'esprit de sérieux \\[0pt]
L'esprit de système \\[0pt]
L'esprit est-il matériel ? \\[0pt]
L'esprit et la machine \\[0pt]
L'esprit humain progresse-t-il ? \\[0pt]
L'esprit peut-il être malade ? \\[0pt]
L'esprit peut-il être mesuré ? \\[0pt]
L'esprit scientifique \\[0pt]
Les proverbes \\[0pt]
Les proverbes enseignent-ils quelque chose ? \\[0pt]
L'esquisse \\[0pt]
Les raisons d'aimer \\[0pt]
Les raisons de vivre \\[0pt]
Les règles de l'art \\[0pt]
Les règles du jeu \\[0pt]
Les relations ont-elles une existence objective ? \\[0pt]
Les remords \\[0pt]
Les représentants du peuple \\[0pt]
Les ruines \\[0pt]
Les sciences de l'homme sont-elles utiles à la société ? \\[0pt]
Les sciences humaines sont-elles des sciences ? \\[0pt]
Les sciences humaines traitent-elles de l'homme ? \\[0pt]
Les sciences ont-elles à penser leurs fondements ? \\[0pt]
Les sciences peuvent-elles penser leurs fondements ? \\[0pt]
Les sciences produisent-elles des normes ? \\[0pt]
Les secrets \\[0pt]
L'essence \\[0pt]
L'essence et l'existence \\[0pt]
Les sensations peuvent-elles se partager ? \\[0pt]
Les sens nous trompent-ils ? \\[0pt]
Les sens peuvent-ils nous tromper ? \\[0pt]
Les sentiments \\[0pt]
Les sentiments peuvent-ils s'apprendre ? \\[0pt]
Les signes de l'intelligence \\[0pt]
Les sociétés ont-elles besoin de théâtre ? \\[0pt]
Les systèmes \\[0pt]
L'estime de soi \\[0pt]
Le style \\[0pt]
Le sublime \\[0pt]
Le succès \\[0pt]
Le sujet \\[0pt]
Le sujet de l'action \\[0pt]
Le sujet et l'objet \\[0pt]
Le superflu \\[0pt]
Les vérités sont-elles toujours démontrables ? \\[0pt]
Les vertus \\[0pt]
Les vertus de la formalisation \\[0pt]
Les vertus sont-elles au principe de la morale ? \\[0pt]
Les vices de l'esprit \\[0pt]
Les visages du mal \\[0pt]
Les vivants et les morts \\[0pt]
Le symbole \\[0pt]
Le tact \\[0pt]
Le talent \\[0pt]
L'État de droit \\[0pt]
L'État de droit ? \\[0pt]
L'état de la nature \\[0pt]
L'état d'exception \\[0pt]
L'État doit-il éduquer les citoyens ? \\[0pt]
L'État est-il l'ennemi de la liberté ? \\[0pt]
L'État et la violence \\[0pt]
L'État et les Églises \\[0pt]
L'État n'est-il qu'un moyen ? \\[0pt]
L'État peut-il être fondé sur un contrat ? \\[0pt]
Le témoignage \\[0pt]
Le témoignage des sens \\[0pt]
Le temps est-il une dimension de la nature ? \\[0pt]
Le temps ne fait-il que passer ? \\[0pt]
Le temps perdu \\[0pt]
L'éternité \\[0pt]
L'éternité n'est-elle qu'une illusion ? \\[0pt]
Le territoire \\[0pt]
Le théâtre du monde \\[0pt]
L'éthique à l'épreuve du tragique \\[0pt]
L'ethnocentrisme \\[0pt]
L'étonnement \\[0pt]
Le toucher \\[0pt]
Le tout est-il la somme de ses parties ? \\[0pt]
Le tout et la partie \\[0pt]
Le tout et les parties \\[0pt]
Le tragique \\[0pt]
Le trait d'esprit \\[0pt]
L'étranger \\[0pt]
L'étrangeté \\[0pt]
Le transcendantal \\[0pt]
Le travail bien fait \\[0pt]
Le travail rapproche-t-il les hommes ? \\[0pt]
L'être de la conscience \\[0pt]
L'être du possible \\[0pt]
L'être et le bien \\[0pt]
L'être et le devenir \\[0pt]
L'être se confond-il avec l'être perçu ? \\[0pt]
Le tribunal de l'histoire \\[0pt]
Le tyran \\[0pt]
L'évaluation de l'action politique \\[0pt]
Le vécu \\[0pt]
L'événement \\[0pt]
Le verbalisme \\[0pt]
Le verbe \\[0pt]
Le vertige \\[0pt]
Le vide \\[0pt]
L'évidence \\[0pt]
L'évidence est-elle un critère de vérité ? \\[0pt]
Le virtuel \\[0pt]
Le visage \\[0pt]
Le visible et l'invisible \\[0pt]
Le vivant \\[0pt]
Le vivant a-t-il une temporalité propre ? \\[0pt]
Le vivant échappe-t-il au déterminisme ? \\[0pt]
Le vivant et la technique \\[0pt]
Le vivant et ses lois \\[0pt]
L'évolution \\[0pt]
Le voyage \\[0pt]
Le vrai a-t-il une histoire ? \\[0pt]
Le vrai et le vraisemblable \\[0pt]
Le vrai et l'imaginaire \\[0pt]
Le vrai peut-il rester invérifiable ? \\[0pt]
Le vraisemblable \\[0pt]
Le vraisemblable présente-t-il un intérêt pour la pensée ? \\[0pt]
Le vulgaire \\[0pt]
L'exactitude \\[0pt]
L'excellence \\[0pt]
L'exception \\[0pt]
L'exception peut-elle confirmer la règle ? \\[0pt]
L'excès \\[0pt]
L'excès et le défaut \\[0pt]
L'exclusion \\[0pt]
L'excuse \\[0pt]
L'exemplaire \\[0pt]
L'exemplarité \\[0pt]
L'exemple \\[0pt]
L'exercice de la raison \\[0pt]
L'exercice de la vertu \\[0pt]
L'exigence de vérité a-t-elle un sens moral ? \\[0pt]
L'exil \\[0pt]
L'existence \\[0pt]
L'existence a-t-elle un sens ? \\[0pt]
L'expérience \\[0pt]
L'expérience cruciale \\[0pt]
L'expérience de la beauté est-elle naturelle ? \\[0pt]
L'expérience de l'injustice \\[0pt]
L'expérience de l'irrationnel \\[0pt]
L'expérience directe est-elle une connaissance ? \\[0pt]
L'expérience du beau peut-elle être une expérience métaphysique ? \\[0pt]
L'expérience du mal \\[0pt]
L'expérience enseigne-elle quelque chose ? \\[0pt]
L'expérience est-elle la seule source de nos connaissances ? \\[0pt]
L'expérience esthétique \\[0pt]
L'expérience esthétique peut-elle se communiquer ? \\[0pt]
L'expérience et l'expérimentation \\[0pt]
L'expérience instruit-elle ? \\[0pt]
L'expérimentation \\[0pt]
L'expérimentation en art \\[0pt]
L'expérimentation en psychologie \\[0pt]
L'expérimentation est-elle facteur de vérité ? \\[0pt]
L'expérimentation médicale \\[0pt]
L'expérimentation sur le vivant \\[0pt]
L'exposition \\[0pt]
L'expression \\[0pt]
L'expression des citoyens dans le vote suffit-elle à faire la démocratie ? \\[0pt]
L'expression des sentiments \\[0pt]
L'expression peut-elle être libre ? \\[0pt]
L'extériorité \\[0pt]
L'extrémisme \\[0pt]
L'habileté \\[0pt]
L'habitation \\[0pt]
L'habitude \\[0pt]
L'habitude est-elle une force ? \\[0pt]
L'habitude est-elle une seconde nature ? \\[0pt]
L'harmonie \\[0pt]
L'hérésie \\[0pt]
L'héritage \\[0pt]
L'héroïsme \\[0pt]
L'hésitation \\[0pt]
L'histoire a-t-elle un sens ? \\[0pt]
L'histoire de l'art \\[0pt]
L'histoire des sciences \\[0pt]
L'histoire est-elle cyclique ? \\[0pt]
L'histoire est-elle sans fin ? \\[0pt]
L'histoire est-elle un genre littéraire ? \\[0pt]
L'histoire est-elle un récit ? \\[0pt]
« L'histoire jugera » \\[0pt]
L'histoire n'est-elle qu'une suite d'événements ? \\[0pt]
L'histoire peut-elle s'écrire au présent ? \\[0pt]
L'histoire peut-elle se répéter ? \\[0pt]
L'histoire universelle est-elle l'histoire des guerres ? \\[0pt]
L'historicité des sciences \\[0pt]
L'homme a-t-il une nature ? \\[0pt]
L'homme cultivé \\[0pt]
L'homme des sciences humaines \\[0pt]
L'homme d'État \\[0pt]
L'homme est-il la mesure de toutes choses ? \\[0pt]
L'homme est-il meilleur seul ou en société ? \\[0pt]
L'homme est-il prisonnier du temps ? \\[0pt]
L'homme est-il un animal dénaturé ? \\[0pt]
L'homme est-il un être inachevé ? \\[0pt]
« L'homme est la mesure de toute chose » \\[0pt]
L'homme et la bête \\[0pt]
L'homme et la machine \\[0pt]
L'homme et la nature sont-ils commensurables ? \\[0pt]
L'homme et le citoyen \\[0pt]
L'homme, le citoyen, le soldat \\[0pt]
L'homme libre est-il un homme seul ? \\[0pt]
L'homme peut-il changer ? \\[0pt]
L'honneur \\[0pt]
L'horizon \\[0pt]
L'horreur \\[0pt]
L'horreur du vide \\[0pt]
L'horrible \\[0pt]
L'hospitalité \\[0pt]
L'hospitalité est-elle un devoir ? \\[0pt]
L'humiliation \\[0pt]
L'humilité \\[0pt]
L'humilité est-elle une vertu ? \\[0pt]
L'humour \\[0pt]
L'humour et l'ironie \\[0pt]
L'hypocrisie \\[0pt]
L'hypothèse \\[0pt]
L'hypothèse de l'inconscient \\[0pt]
Libéralisme et démocratie \\[0pt]
Liberté et égalité \\[0pt]
Liberté et libération \\[0pt]
Liberté et nécessité \\[0pt]
Liberté et négativité \\[0pt]
Liberté et propriété \\[0pt]
Liberté et vérité \\[0pt]
Libre-arbitre, impulsion, contrainte \\[0pt]
L'idéal \\[0pt]
L'idéal de vérité \\[0pt]
L'idéalisme \\[0pt]
L'idéaliste \\[0pt]
L'idéalité \\[0pt]
L'idée d'anthropologie \\[0pt]
L'idée de beaux arts \\[0pt]
L'idée de contrat \\[0pt]
L'idée de création \\[0pt]
L'idée de crise \\[0pt]
L'idée de critique \\[0pt]
L'idée de destin \\[0pt]
L'idée de destin a-t-elle un sens ? \\[0pt]
L'idée de Dieu \\[0pt]
L'idée d'égalité \\[0pt]
L'idée de langue parfaite \\[0pt]
L'idée de langue universelle \\[0pt]
L'idée de logique \\[0pt]
L'idée de nature humaine \\[0pt]
L'idée d'encyclopédie \\[0pt]
L'idée de perfection \\[0pt]
L'idée de philosophie première \\[0pt]
L'idée de réduction \\[0pt]
L'idée de République \\[0pt]
L'idée de substance \\[0pt]
L'idée d'éternité \\[0pt]
L'idée d'exactitude \\[0pt]
L'idée d'humanité \\[0pt]
L'idée d'une langue universelle \\[0pt]
L'idée d'une science bien faite \\[0pt]
L'idée et l'idéal \\[0pt]
L'identité \\[0pt]
L'identité et la différence \\[0pt]
L'identité nationale \\[0pt]
L'identité personnelle \\[0pt]
L'idéologie \\[0pt]
L'idolâtrie \\[0pt]
L'idole \\[0pt]
L'ignoble \\[0pt]
L'ignorance \\[0pt]
L'ignorance des savants \\[0pt]
L'ignorance nous excuse-t-elle ? \\[0pt]
L'illimité \\[0pt]
L'illusion \\[0pt]
L'illusoire \\[0pt]
L'image \\[0pt]
L'imaginaire \\[0pt]
L'imagination a-t-elle des limites ? \\[0pt]
L'imagination est-elle dangereuse ? \\[0pt]
L'imagination nous éloigne-t-elle du réel ? \\[0pt]
L'imagination nous instruit-elle ? \\[0pt]
L'imbécile heureux \\[0pt]
L'imitation \\[0pt]
L'immatériel \\[0pt]
L'immédiat \\[0pt]
L'immensité \\[0pt]
L'immoraliste \\[0pt]
L'immoralité \\[0pt]
L'immutabilité \\[0pt]
L'impardonnable \\[0pt]
L'impartialité \\[0pt]
L'impensable \\[0pt]
L'impératif \\[0pt]
L'imperceptible \\[0pt]
L'impersonnel \\[0pt]
L'implicite \\[0pt]
L'importance des détails \\[0pt]
L'impossible \\[0pt]
L'imposteur \\[0pt]
L'imprescriptible \\[0pt]
L'impression \\[0pt]
L'imprévisible \\[0pt]
L'imprévu \\[0pt]
L'improbable \\[0pt]
L'improvisation \\[0pt]
L'imprudence \\[0pt]
L'impuissance \\[0pt]
L'impuissance de la raison \\[0pt]
L'impunité \\[0pt]
L'inacceptable \\[0pt]
L'inachevé \\[0pt]
L'inachèvement \\[0pt]
L'inaction \\[0pt]
L'inapparent \\[0pt]
L'inattendu \\[0pt]
L'incarnation \\[0pt]
L'incertitude \\[0pt]
L'incertitude est-elle dans les choses ou dans les idées ? \\[0pt]
L'incommensurabilité \\[0pt]
L'incommensurable \\[0pt]
L'incompréhensible \\[0pt]
L'inconcevable \\[0pt]
L'inconnu \\[0pt]
L'inconscience \\[0pt]
L'inconscient \\[0pt]
L'inconséquence \\[0pt]
L'inconstance \\[0pt]
L'incrédulité \\[0pt]
L'inculture \\[0pt]
L'indécidable \\[0pt]
L'indécision \\[0pt]
L'indéfini \\[0pt]
L'indémontrable \\[0pt]
L'indépassable \\[0pt]
L'indétermination \\[0pt]
L'indéterminé \\[0pt]
L'indicible \\[0pt]
L'indifférence \\[0pt]
L'indifférence à la politique \\[0pt]
L'indiscernable \\[0pt]
L'indiscutable \\[0pt]
L'indistinct \\[0pt]
L'individu \\[0pt]
L'individualisme \\[0pt]
L'individu a-t-il des droits ? \\[0pt]
L'individu est-il définissable ? \\[0pt]
L'indubitable \\[0pt]
L'induction \\[0pt]
L'ineffable \\[0pt]
L'inégalité a-t-elle des vertus ? \\[0pt]
L'inégalité des chances \\[0pt]
L'inégalité entre les hommes \\[0pt]
L'inégalité naturelle \\[0pt]
L'inégalité sociale \\[0pt]
L'inertie \\[0pt]
L'inexact peut-il avoir une valeur ? \\[0pt]
L'infâme \\[0pt]
L'infamie \\[0pt]
L'inférence \\[0pt]
L'infériorité \\[0pt]
L'infini \\[0pt]
L'infini est-il la négation du fini ? \\[0pt]
L'infini et l'indéfini \\[0pt]
L'infinité du monde \\[0pt]
L'influence \\[0pt]
L'information \\[0pt]
L'informe \\[0pt]
L'ingénieur et l'artiste \\[0pt]
L'ingéniosité \\[0pt]
L'ingérence \\[0pt]
L'ingratitude \\[0pt]
L'inhumain \\[0pt]
L'inimaginable \\[0pt]
L'inimitié \\[0pt]
L'inintelligible \\[0pt]
L'initiation \\[0pt]
L'injustice \\[0pt]
L'injustice est-elle préférable au désordre ? \\[0pt]
L'innocence \\[0pt]
L'innommable \\[0pt]
L'inobservable \\[0pt]
L'inquiétant \\[0pt]
L'inquiétude \\[0pt]
L'insensé \\[0pt]
L'insensibilité \\[0pt]
L'insignifiant \\[0pt]
L'insouciance \\[0pt]
L'insoumission \\[0pt]
L'insoutenable \\[0pt]
L'inspiration \\[0pt]
L'instant \\[0pt]
L'instant a-t-il une réalité ? \\[0pt]
L'instinct \\[0pt]
L'institution \\[0pt]
L'institution du mariage \\[0pt]
L'instrument, l'outil, le jouet \\[0pt]
L'insulte \\[0pt]
L'intellectuel \\[0pt]
L'intelligence \\[0pt]
L'intelligence de la machine \\[0pt]
L'intelligence de la main \\[0pt]
L'intelligence de la matière \\[0pt]
L'intelligence des bêtes \\[0pt]
L'intelligence du sensible \\[0pt]
L'intelligence du vivant \\[0pt]
L'intelligible \\[0pt]
L'intemporel \\[0pt]
L'intention \\[0pt]
L'intentionnalité \\[0pt]
L'interdit \\[0pt]
L'interdit et le sacré \\[0pt]
L'intérêt \\[0pt]
L'intérêt bien compris \\[0pt]
L'intérêt commun \\[0pt]
L'intérêt des machines \\[0pt]
L'intérêt général n'est-il qu'un mythe ? \\[0pt]
L'intérêt peut-il avoir une valeur morale ? \\[0pt]
L'intérêt peut-il être général ? \\[0pt]
L'intérieur et l'extérieur \\[0pt]
L'intériorité \\[0pt]
L'interprétation \\[0pt]
L'interprétation est-elle un art ? \\[0pt]
L'interprète est-il un créateur ? \\[0pt]
L'intime \\[0pt]
L'intime conviction \\[0pt]
L'intimité \\[0pt]
L'intolérable \\[0pt]
L'intolérance \\[0pt]
L'intraduisible \\[0pt]
L'intransigeance \\[0pt]
L'intransmissible \\[0pt]
L'introspection \\[0pt]
L'intuition \\[0pt]
L'inutile \\[0pt]
L'invention \\[0pt]
L'invention de soi \\[0pt]
L'invention des valeurs \\[0pt]
L'invention en politique \\[0pt]
L'invérifiable \\[0pt]
L'invisibilité \\[0pt]
L'invisible \\[0pt]
L'involontaire \\[0pt]
L'invraisemblable \\[0pt]
Lire \\[0pt]
Lire et écrire \\[0pt]
L'ironie \\[0pt]
L'irrationalité \\[0pt]
L'irrationnel \\[0pt]
L'irréel \\[0pt]
L'irréfutable \\[0pt]
L'irrégularité \\[0pt]
L'irréparable \\[0pt]
L'irrésolution \\[0pt]
L'irresponsabilité \\[0pt]
L'irréversible \\[0pt]
L'irrévocable \\[0pt]
Littérature et philosophie \\[0pt]
L'ivresse \\[0pt]
L'obéissance \\[0pt]
L'objectivité \\[0pt]
L'objectivité des valeurs \\[0pt]
L'objectivité en histoire \\[0pt]
L'objectivité existe-t-elle ? \\[0pt]
L'objet \\[0pt]
L'objet d'amour \\[0pt]
L'objet de l'amour \\[0pt]
L'objet de la réflexion \\[0pt]
L'objet de l'art \\[0pt]
L'obligation \\[0pt]
L'obscène \\[0pt]
L'obscénité \\[0pt]
L'obscur \\[0pt]
L'obscurité \\[0pt]
L'observable et l'inobservable \\[0pt]
L'observation \\[0pt]
L'obsession \\[0pt]
L'obstacle \\[0pt]
L'occasion \\[0pt]
L'œil et l'oreille \\[0pt]
L'œuvre d'art échappe-t-elle au monde ? \\[0pt]
L'œuvre d'art est-elle une marchandise ? \\[0pt]
L'œuvre d'art représente-t-elle quelque chose ? \\[0pt]
L'œuvre et le produit \\[0pt]
Logique et dialectique \\[0pt]
Logique et écriture \\[0pt]
Logique et vérité \\[0pt]
Lois et normes \\[0pt]
Lois et valeurs \\[0pt]
L'oisiveté \\[0pt]
L'oligarchie \\[0pt]
L'ombre \\[0pt]
L'ombre et la lumière \\[0pt]
L'opinion \\[0pt]
L'opinion droite \\[0pt]
L'opinion publique \\[0pt]
L'opinion vraie \\[0pt]
L'opposition \\[0pt]
L'optimisme \\[0pt]
L'ordinaire est-il ennuyeux ? \\[0pt]
L'ordinateur \\[0pt]
L'ordre \\[0pt]
L'ordre des choses \\[0pt]
L'ordre des raisons \\[0pt]
L'ordre du monde \\[0pt]
L'ordre du temps \\[0pt]
L'ordre établi \\[0pt]
L'ordre et la mesure \\[0pt]
L'ordre international \\[0pt]
L'ordre, le nombre, la mesure \\[0pt]
L'ordre naturel \\[0pt]
L'ordre social \\[0pt]
L'organique et le mécanique \\[0pt]
L'organisation \\[0pt]
L'orgueil \\[0pt]
L'orientation \\[0pt]
L'originalité \\[0pt]
L'origine \\[0pt]
L'origine de la culpabilité \\[0pt]
L'origine de la négation \\[0pt]
L'origine des langues \\[0pt]
L'origine des langues est-elle un faux problème ? \\[0pt]
L'origine et le fondement \\[0pt]
L'ornement \\[0pt]
L'orthodoxie \\[0pt]
L'oubli \\[0pt]
L'oubli est-il un échec de la mémoire ? \\[0pt]
L'oubli n'est-il qu'une perte de mémoire ? \\[0pt]
L'outil \\[0pt]
L'Un \\[0pt]
L'unanimité \\[0pt]
L'un et le multiple \\[0pt]
L'uniformité \\[0pt]
L'unité \\[0pt]
L'unité dans le beau \\[0pt]
L'unité de l'art \\[0pt]
L'unité de la science \\[0pt]
L'unité des sciences \\[0pt]
L'unité des sciences humaines \\[0pt]
L'unité des sciences humaines ? \\[0pt]
L'universel \\[0pt]
L'universel est-il une illusion ? \\[0pt]
L'universel et le particulier \\[0pt]
L'universel et le singulier \\[0pt]
L'universel implique-t-il l'indifférence à la singularité ? \\[0pt]
L'univers infini \\[0pt]
L'urbanité \\[0pt]
L'urgence \\[0pt]
L'usage \\[0pt]
L'usage de l'analogie \\[0pt]
L'usage des fictions \\[0pt]
L'usage des généalogies \\[0pt]
L'usage des mots \\[0pt]
L'usage des passions \\[0pt]
L'usage des principes \\[0pt]
L'usage du monde \\[0pt]
L'utile et l'agréable \\[0pt]
L'utilité \\[0pt]
L'utilité de la poésie \\[0pt]
L'utilité des préjugés \\[0pt]
L'utopie \\[0pt]
L'utopie a-t-elle un lien avec la pratique ? \\[0pt]
Machine et organisme \\[0pt]
Maître et disciple \\[0pt]
Maître et serviteur \\[0pt]
Maîtriser l'absence \\[0pt]
Maîtriser la nature \\[0pt]
Mal faire \\[0pt]
« Malheur aux vaincus » \\[0pt]
Manger \\[0pt]
Manquer de jugement \\[0pt]
Ma parole m'engage-t-elle ? \\[0pt]
Masculin, féminin \\[0pt]
Matérialisme et mécanisme \\[0pt]
Mathématique et imagination \\[0pt]
Mathématiques et réalité \\[0pt]
Mathématiques et universel \\[0pt]
Matière et corps \\[0pt]
Matière et forme \\[0pt]
Matière et matériaux \\[0pt]
Ma vraie nature \\[0pt]
Méditer \\[0pt]
Mémoire et anticipation \\[0pt]
Mémoire et fiction \\[0pt]
Mémoire et identité \\[0pt]
Mémoire et imagination \\[0pt]
Ménager les apparences \\[0pt]
Mensonge et politique \\[0pt]
Mensonge, vérité, véracité \\[0pt]
Mentir \\[0pt]
Mérite et démocratie \\[0pt]
Mesurer \\[0pt]
Mesurer le risque \\[0pt]
Métaphysique et poésie \\[0pt]
Méthode et métaphysique \\[0pt]
Métier et vocation \\[0pt]
Mettre en ordre \\[0pt]
Microscope et télescope \\[0pt]
Milieux naturels et milieux sociaux \\[0pt]
Misère et pauvreté \\[0pt]
Modèle, type et paradigme \\[0pt]
Moi d'abord \\[0pt]
Mon corps \\[0pt]
Mon corps est-il ma propriété ? \\[0pt]
Mon corps m'appartient-il ? \\[0pt]
Monde et nature \\[0pt]
Monde perçu et monde conçu \\[0pt]
Monologue et dialogue \\[0pt]
Montrer et démontrer \\[0pt]
Montrer et dire \\[0pt]
Morale et intérêt \\[0pt]
Morale et prudence \\[0pt]
Morale et sentiments \\[0pt]
Mourir \\[0pt]
Mythe et histoire \\[0pt]
Mythe et philosophie \\[0pt]
N'aimer que soi \\[0pt]
Naître \\[0pt]
Nature et histoire \\[0pt]
Nature et institution \\[0pt]
Nature et institutions \\[0pt]
Nature et mesure du temps \\[0pt]
Nature et nature humaine \\[0pt]
Nature, monde, univers \\[0pt]
Naviguer \\[0pt]
Nécessité et contingence \\[0pt]
Ne pas choisir \\[0pt]
Ne pas comprendre \\[0pt]
Ne pas raconter d'histoires \\[0pt]
Ne pas savoir ce que l'on fait \\[0pt]
Ne pas tuer \\[0pt]
Ne penser à rien \\[0pt]
Ne prêche-t-on que les convertis ? \\[0pt]
Ne rien devoir à personne \\[0pt]
Ne travaille-t- on que pour un salaire ? \\[0pt]
N'exprime-t-on que ce dont on a conscience ? \\[0pt]
Nier le monde \\[0pt]
Nier l'évidence \\[0pt]
Ni regrets, ni remords \\[0pt]
Nomade et sédentaire \\[0pt]
Nommer \\[0pt]
Non-sens et absurdité \\[0pt]
Nos devoirs nous motivent-ils ? \\[0pt]
Nos goûts sont-ils justifiables ? \\[0pt]
Nos sens peuvent-ils s'éduquer ? \\[0pt]
Notre besoin de fictions \\[0pt]
N'y a-t-il d'amitié qu'entre égaux ? \\[0pt]
N'y a-t-il de bonheur que dans l'instant ? \\[0pt]
N'y a-t-il de connaissance que scientifique ? \\[0pt]
N'y a-t-il de droit que de l'homme ? \\[0pt]
N'y a-t-il de propriété que privée ? \\[0pt]
N'y a-t-il de science que du général ? \\[0pt]
N'y a-t-il de science que du nécessaire ? \\[0pt]
N'y a-t-il de science que provisoire ? \\[0pt]
N'y a-t-il de science que systématique ? \\[0pt]
N'y a-t-il de sens que par le langage ? \\[0pt]
N'y a-t-il de vérité que scientifique ? \\[0pt]
N'y a-t-il que du mesurable ? \\[0pt]
N'y a-t-il qu'une substance ? \\[0pt]
N'y a-t-il qu'un seul monde ? \\[0pt]
N'y a-t-il rien au monde de certain ? \\[0pt]
Obéir \\[0pt]
Obéir, est-ce se soumettre ? \\[0pt]
Obéir et désobéir \\[0pt]
Objecter et répondre \\[0pt]
Objet d'art, objet d'échange \\[0pt]
Observer \\[0pt]
Observer et expérimenter \\[0pt]
Ontologie et théologie \\[0pt]
Opinion, croyance, jugement \\[0pt]
Opinion publique et conscience universelle \\[0pt]
Ordre et désordre \\[0pt]
Ordre et liberté \\[0pt]
Organe et fonction \\[0pt]
Organiser \\[0pt]
Origine et commencement \\[0pt]
Origine et fondement \\[0pt]
Où commence la servitude ? \\[0pt]
Où commence la vie privée ? \\[0pt]
Où est le danger ? \\[0pt]
Où est le passé ? \\[0pt]
Où est le pouvoir ? \\[0pt]
Où est-on quand on pense ? \\[0pt]
Où s'arrête la responsabilité ? \\[0pt]
Où s'arrête l'espace public ? \\[0pt]
Où suis-je ? \\[0pt]
Où suis-je quand je pense ? \\[0pt]
Pardonner \\[0pt]
Pardonner et oublier \\[0pt]
Parfaire \\[0pt]
Parier \\[0pt]
Parler de Dieu, est-ce nécessairement tenir un discours religieux ? \\[0pt]
Parler de soi \\[0pt]
Parler de soi est-il intéressant ? \\[0pt]
Parler, écrire, penser \\[0pt]
Parler, est-ce communiquer ? \\[0pt]
Parler et agir \\[0pt]
Parler et penser \\[0pt]
Parler pour ne rien dire \\[0pt]
Parler pour quelqu'un \\[0pt]
Par où commencer ? \\[0pt]
Par quoi un individu se distingue-t-il d'un autre ? \\[0pt]
Partager \\[0pt]
Partager les richesses \\[0pt]
Passer du fait au droit \\[0pt]
Peindre d'après nature \\[0pt]
Peindre un paysage, est-ce représenter la nature ? \\[0pt]
Peinture et histoire \\[0pt]
Peinture et photographie \\[0pt]
Pensée et calcul \\[0pt]
Pensée et réalité \\[0pt]
Penser, est-ce aller contre l'évidence ? \\[0pt]
Penser, est-ce calculer ? \\[0pt]
Penser, est-ce comparer ? \\[0pt]
Penser et calculer \\[0pt]
Penser et parler \\[0pt]
Penser la technique \\[0pt]
Penser l'avenir \\[0pt]
Penser le devenir \\[0pt]
Penser le réel \\[0pt]
Penser le rien, est-ce ne rien penser ? \\[0pt]
Penser par soi-même \\[0pt]
Penser requiert-il un corps ? \\[0pt]
Perception et aperception \\[0pt]
Perception et jugement \\[0pt]
Perception et mémoire \\[0pt]
Perception et mouvement \\[0pt]
Perception et observation \\[0pt]
Percer les apparences \\[0pt]
Percevoir \\[0pt]
Percevoir, est-ce connaître ? \\[0pt]
Percevoir, est-ce savoir ? \\[0pt]
Percevoir et imaginer \\[0pt]
Percevoir et sentir \\[0pt]
Perdre la mémoire \\[0pt]
Perdre la raison \\[0pt]
Perdre ses habitudes \\[0pt]
Perdre ses illusions \\[0pt]
Perdre son âme \\[0pt]
Permettre \\[0pt]
Persister \\[0pt]
Persuader \\[0pt]
Persuader et convaincre \\[0pt]
« Petites causes, grands effets » \\[0pt]
Peut-il y avoir de bons tyrans ? \\[0pt]
Peut-il y avoir des degrés dans la vérité ? \\[0pt]
Peut-il y avoir des nations sans État ? \\[0pt]
Peut-il y avoir une méthode commune à toutes les sciences ? \\[0pt]
Peut-il y avoir une philosophie applicable ? \\[0pt]
Peut-il y avoir une pratique sans théorie ? \\[0pt]
Peut-il y avoir une science de l'homme ? \\[0pt]
Peut-il y avoir un intérêt collectif ? \\[0pt]
Peut-on admettre ce qui n'est pas vérifié ? \\[0pt]
Peut-on affirmer qu'être, c'est être perçu ou percevoir ? \\[0pt]
Peut-on agir sans prendre de risques ? \\[0pt]
Peut-on aimer ce qu'on ne connaît pas ? \\[0pt]
Peut-on aimer les animaux ? \\[0pt]
Peut-on aimer l'humanité ? \\[0pt]
Peut-on aimer son prochain comme soi-même ? \\[0pt]
Peut-on à la fois préserver et dominer la nature ? \\[0pt]
Peut-on apprendre à être juste ? \\[0pt]
Peut-on apprendre à mourir ? \\[0pt]
Peut-on apprendre à penser ? \\[0pt]
Peut-on apprendre à vivre ? \\[0pt]
Peut-on avoir raison contre les faits ? \\[0pt]
Peut-on avoir raison contre tous ? \\[0pt]
Peut-on avoir raison tout seul ? \\[0pt]
Peut-on avoir trop d'imagination ? \\[0pt]
Peut-on bien agir sans le savoir ? \\[0pt]
Peut-on changer de logique ? \\[0pt]
Peut-on changer le monde ? \\[0pt]
Peut-on changer le passé ? \\[0pt]
Peut-on comparer deux philosophies ? \\[0pt]
Peut-on comprendre ce qui est illogique ? \\[0pt]
Peut-on concevoir la science achevée ? \\[0pt]
Peut-on concilier nécessité et liberté ? \\[0pt]
Peut-on connaître autrui ? \\[0pt]
Peut-on considérer l'art comme un langage ? \\[0pt]
Peut-on considérer les hommes de science comme des autorités morales ? \\[0pt]
Peut-on décider de croire ? \\[0pt]
Peut-on définir ce qu'est un progrès technique ? \\[0pt]
Peut-on définir la santé ? \\[0pt]
Peut-on définir la vérité par la correspondance ? \\[0pt]
Peut-on dériver des concepts de l'expérience ? \\[0pt]
Peut-on dire ce qui n'est pas ? \\[0pt]
Peut-on dire d'une image qu'elle parle ? \\[0pt]
Peut-on dire toute la vérité ? \\[0pt]
Peut-on distinguer le vrai du faux ? \\[0pt]
Peut-on donner un sens à la souffrance ? \\[0pt]
Peut-on douter de l'humanité d'un homme ? \\[0pt]
Peut-on douter de tout ? \\[0pt]
Peut-on douter du sens des mots ? \\[0pt]
Peut-on éclairer la liberté ? \\[0pt]
Peut-on encore être cosmopolite ? \\[0pt]
Peut-on en finir avec les préjugés ? \\[0pt]
Peut-on en savoir trop ? \\[0pt]
Peut-on être citoyen du monde ? \\[0pt]
Peut-on être content de soi ? \\[0pt]
Peut-on être en conflit avec soi-même ? \\[0pt]
Peut-on être fidèle à soi-même ? \\[0pt]
Peut-on être heureux tout seul ? \\[0pt]
Peut-on être homme sans être citoyen ? \\[0pt]
Peut-on être hors de soi ? \\[0pt]
Peut-on être juste dans une société injuste ? \\[0pt]
Peut-on être objectif en matière d'art ? \\[0pt]
Peut-on être plus ou moins libre ? \\[0pt]
Peut-on être prisonnier de soi-même ? \\[0pt]
Peut-on être sage tout seul ? \\[0pt]
Peut-on être sans opinion ? \\[0pt]
Peut-on être sûr d'avoir raison ? \\[0pt]
Peut-on être trop sage ? \\[0pt]
Peut-on être trop sensible ? \\[0pt]
Peut-on expliquer une œuvre d'art ? \\[0pt]
Peut-on faire comme si le passé n'existait pas ? \\[0pt]
Peut-on faire de sa vie une œuvre d'art ? \\[0pt]
Peut-on faire le bien de quelqu'un malgré lui ? \\[0pt]
Peut-on faire l'économie de la notion de forme ? \\[0pt]
Peut-on faire l'inventaire du monde ? \\[0pt]
Peut-on fixer des limites à la science ? \\[0pt]
Peut-on fonder les mathématiques ? \\[0pt]
Peut-on fonder une morale sur la nature ? \\[0pt]
Peut-on jamais aimer son prochain ? \\[0pt]
Peut-on juger de la valeur d'une vie humaine ? \\[0pt]
Peut-on justifier le mensonge ? \\[0pt]
Peut-on justifier le recours à l'idée de finalité ? \\[0pt]
Peut-on montrer en cachant ? \\[0pt]
Peut-on ne croire en rien \\[0pt]
Peut-on ne croire en rien ? \\[0pt]
Peut-on ne pas être de son temps ? \\[0pt]
Peut-on ne pas être matérialiste ? \\[0pt]
Peut-on ne pas être soi-même ? \\[0pt]
Peut-on ne pas savoir ce que l'on fait ? \\[0pt]
Peut-on ne penser à rien ? \\[0pt]
Peut-on ne rien aimer ? \\[0pt]
Peut-on ne rien vouloir ? \\[0pt]
Peut-on oublier ? \\[0pt]
Peut-on parler de corruption des mœurs ? \\[0pt]
Peut-on parler de droits des animaux ? \\[0pt]
Peut-on parler d'un droit de la guerre ? \\[0pt]
Peut-on parler d'un travail intellectuel ? \\[0pt]
Peut-on penser illogiquement ? \\[0pt]
Peut-on penser la douleur ? \\[0pt]
Peut-on penser la mort ? \\[0pt]
Peut-on penser l'avenir ? \\[0pt]
Peut-on penser le divin ? \\[0pt]
Peut-on penser le temps ? \\[0pt]
Peut-on penser l'irrationnel ? \\[0pt]
Peut-on penser sans concept ? \\[0pt]
Peut-on penser sans concepts ? \\[0pt]
Peut-on penser sans les mots ? \\[0pt]
Peut-on penser sans méthode ? \\[0pt]
Peut-on penser sans règles ? \\[0pt]
Peut-on penser sans signes ? \\[0pt]
Peut-on percevoir sans s'en apercevoir ? \\[0pt]
Peut-on perdre la raison ? \\[0pt]
Peut-on perdre sa liberté ? \\[0pt]
Peut-on perdre son identité ? \\[0pt]
Peut-on permettre le mensonge ? \\[0pt]
Peut-on prouver l'existence ? \\[0pt]
Peut-on recommencer sa vie ? \\[0pt]
Peut-on refuser l'évidence ? \\[0pt]
Peut-on réparer une injustice ? \\[0pt]
Peut-on représenter l'espace ? \\[0pt]
Peut-on rester insensible à la beauté ? \\[0pt]
Peut-on résumer une philosophie ? \\[0pt]
Peut-on rire de tout ? \\[0pt]
Peut-on savoir ce qui est bien ? \\[0pt]
Peut-on se connaître soi-même ? \\[0pt]
Peut-on se faire une idée de tout ? \\[0pt]
Peut-on se fier à son intuition ? \\[0pt]
Peut-on se mettre à la place d'autrui ? \\[0pt]
Peut-on se passer de l'idée de nature humaine ? \\[0pt]
Peut-on se passer d'un maître ? \\[0pt]
Peut-on se retirer du monde ? \\[0pt]
Peut-on tout attendre de l'État ? \\[0pt]
Peut-on tout démontrer ? \\[0pt]
Peut-on tout dire ? \\[0pt]
Peut-on tout justifier ? \\[0pt]
Peut-on tout mesurer ? \\[0pt]
Peut-on tout prouver ? \\[0pt]
Peut-on tout soumettre à la discussion ? \\[0pt]
Peut-on trop penser ? \\[0pt]
Peut-on trouver du plaisir à l'ennui ? \\[0pt]
Peut-on vivre avec les autres ? \\[0pt]
Peut-on vivre dans le doute ? \\[0pt]
Peut-on vivre sans aucune certitude ? \\[0pt]
Peut-on vivre sans croyance ? \\[0pt]
Peut-on vivre sans foi ni loi ? \\[0pt]
Peut-on vivre sans opinions ? \\[0pt]
Peut-on vivre sans ressentiment ? \\[0pt]
Peut-on vivre sans travailler ? \\[0pt]
Peut-on vouloir le mal ? \\[0pt]
Philosopher, est-ce apprendre à vivre ? \\[0pt]
Philosophie et mathématiques \\[0pt]
Philosophie et métaphysique \\[0pt]
Philosophie et religion \\[0pt]
Philosophie et théologie \\[0pt]
Physique et cosmologie \\[0pt]
Physique et mathématique \\[0pt]
Plaider \\[0pt]
Plaisir et préférences \\[0pt]
Poésie et philosophie \\[0pt]
Poésie et vérité \\[0pt]
Police et politique \\[0pt]
Possibilité et intelligibilité \\[0pt]
Pour dialoguer faut-il parler le même langage ? \\[0pt]
Pourquoi a-t-on peur de la folie ? \\[0pt]
Pourquoi avoir recours à la notion d'inconscient ? \\[0pt]
Pourquoi chercher un sens à l'histoire ? \\[0pt]
Pourquoi commémorer ? \\[0pt]
Pourquoi croyons-nous ? \\[0pt]
Pourquoi des artifices ? \\[0pt]
Pourquoi des artistes ? \\[0pt]
Pourquoi des cérémonies ? \\[0pt]
Pourquoi des châtiments ? \\[0pt]
Pourquoi des classifications ? \\[0pt]
Pourquoi des conflits ? \\[0pt]
Pourquoi des définitions ? \\[0pt]
Pourquoi des exemples ? \\[0pt]
Pourquoi des historiens ? \\[0pt]
Pourquoi des hypothèses ? \\[0pt]
Pourquoi désirer ? \\[0pt]
Pourquoi des métaphores ? \\[0pt]
Pourquoi des modèles ? \\[0pt]
Pourquoi des musées ? \\[0pt]
Pourquoi des poètes ? \\[0pt]
Pourquoi des rites ? \\[0pt]
Pourquoi des syllogismes ? \\[0pt]
Pourquoi des symboles ? \\[0pt]
Pourquoi dire la vérité ? \\[0pt]
Pourquoi donner ? \\[0pt]
Pourquoi douter ? \\[0pt]
Pourquoi écrire ? \\[0pt]
Pourquoi est-il difficile de rectifier une erreur ? \\[0pt]
Pourquoi être exigeant ? \\[0pt]
Pourquoi être moral ? \\[0pt]
Pourquoi exiger la cohérence \\[0pt]
Pourquoi faire de la métaphysique ? \\[0pt]
Pourquoi faire de l'histoire ? \\[0pt]
Pourquoi fait-on le mal ? \\[0pt]
Pourquoi faudrait-il être cohérent ? \\[0pt]
Pourquoi faudrait-il sauver les apparences ? \\[0pt]
Pourquoi faut-il agir ? \\[0pt]
Pourquoi les hommes jouent-ils ? \\[0pt]
Pourquoi lire des romans ? \\[0pt]
Pourquoi mentir ? \\[0pt]
Pourquoi nous souvenons-nous ? \\[0pt]
Pourquoi nous trompons-nous ? \\[0pt]
Pourquoi obéir ? \\[0pt]
Pourquoi obéir aux lois ? \\[0pt]
Pourquoi oublie-t-on ? \\[0pt]
Pourquoi pardonner ? \\[0pt]
Pourquoi parler de jugement de goût ? \\[0pt]
Pourquoi parlons-nous ? \\[0pt]
Pourquoi peindre la nature ? \\[0pt]
Pourquoi pensons-nous ? \\[0pt]
Pourquoi préférer l'original à sa reproduction ? \\[0pt]
Pourquoi promettre ? \\[0pt]
Pourquoi punir ? \\[0pt]
Pourquoi punit-on ? \\[0pt]
Pourquoi sauver les phénomènes ? \\[0pt]
Pourquoi se mettre à la place d'autrui ? \\[0pt]
Pourquoi se référer au destin ? \\[0pt]
Pourquoi s'étonner ? \\[0pt]
Pourquoi sommes-nous déçus par les œuvres d'un faussaire ? \\[0pt]
Pourquoi sommes-nous moraux ? \\[0pt]
Pourquoi travailler ? \\[0pt]
Pourquoi travaille-t-on ? \\[0pt]
Pourquoi un droit du travail ? \\[0pt]
Pourquoi vouloir gagner ? \\[0pt]
Pourquoi vouloir la vérité ? \\[0pt]
Pourquoi vouloir s'abstraire du quotidien ? \\[0pt]
Pourquoi voyager ? \\[0pt]
Pourquoi y a-t-il des conflits insolubles ? \\[0pt]
Pourquoi y a-t-il plusieurs philosophies ? \\[0pt]
Pourrait-on vivre sans idéal ? \\[0pt]
Pourrions-nous vivre sans religion ? \\[0pt]
Pouvoir et devoir \\[0pt]
Pouvoir et puissance \\[0pt]
Pouvoirs et libertés \\[0pt]
Pouvons-nous vivre sans préjugés ? \\[0pt]
Prendre des risques \\[0pt]
« Prendre ses désirs pour des réalités » \\[0pt]
Prendre son temps \\[0pt]
Prendre sur soi \\[0pt]
Prendre une décision \\[0pt]
Prescrire et décrire \\[0pt]
Prescrire et interdire \\[0pt]
Présence et représentation \\[0pt]
Preuve et démonstration \\[0pt]
Prévoir \\[0pt]
Principe et commencement \\[0pt]
Principe et conséquence \\[0pt]
Principe et fondement \\[0pt]
Problème, énigme, mystère \\[0pt]
Produire et consommer \\[0pt]
Promettre \\[0pt]
Prophétie, utopie, pronostic \\[0pt]
Prose et poésie \\[0pt]
Prouver \\[0pt]
Prouver et éprouver \\[0pt]
Providence et destin \\[0pt]
Publier \\[0pt]
Puis-je être sûr de bien agir ? \\[0pt]
Puis-je être universel ? \\[0pt]
Puis-je me mettre à la place d'un autre ? \\[0pt]
Puis-je ne rien croire ? \\[0pt]
Puissance et pouvoir \\[0pt]
Pulsion et instinct \\[0pt]
Pulsions et passions \\[0pt]
Qu'admire-t-on dans une œuvre d'art ? \\[0pt]
Qu'aime-t-on dans l'amour ? \\[0pt]
Qu'aimons-nous dans l'amour ? \\[0pt]
Qualités premières et qualités secondes \\[0pt]
Qualités premières, qualités secondes \\[0pt]
Quand agit-on ? \\[0pt]
Quand faut-il désobéir ? \\[0pt]
Quand faut-il se taire ? \\[0pt]
Quand la guerre finira-t-elle ? \\[0pt]
Quand pense-t-on ? \\[0pt]
Quand suis-je en faute ? \\[0pt]
Quand suis-je moi-même ? \\[0pt]
Quand y a-t-il art ? \\[0pt]
Quand y a-t-il métaphore ? \\[0pt]
Quand y a-t-il œuvre ? \\[0pt]
Qu'a perdu le fou ? \\[0pt]
Qu'apporte au mathématicien l'histoire des mathématiques ? \\[0pt]
Qu'apprend-on dans les livres ? \\[0pt]
Qu'apprenons-nous de nos affects ? \\[0pt]
Qu'a-t-on le droit d'interpréter ? \\[0pt]
Qu'attendons-nous des œuvres d'art ? \\[0pt]
Qu'avons-nous à apprendre de nos illusions ? \\[0pt]
Qu'avons-nous à apprendre des historiens ? \\[0pt]
Que changent les révolutions ? \\[0pt]
Que cherchons-nous dans le regard des autres ? \\[0pt]
Que comprend-on d'une œuvre d'art ? \\[0pt]
Que connaissons-nous du vivant ? \\[0pt]
Que connaît-on de ses passions ? \\[0pt]
Que désirons-nous ? \\[0pt]
Que devons-nous à l'animal ? \\[0pt]
Que dit la loi ? \\[0pt]
Que dit la musique ? \\[0pt]
Que doit l'art à la nature ? \\[0pt]
Que doit-on à autrui ? \\[0pt]
Que doit-on faire de ses rêves ? \\[0pt]
Que doit-on protéger ? \\[0pt]
Que faire de nos passions ? \\[0pt]
Que faire du vraisemblable ? \\[0pt]
Que fait la police ? \\[0pt]
Que fait l'artiste ? \\[0pt]
Que faut-il craindre ? \\[0pt]
Que faut-il pour faire un monde ? \\[0pt]
Que faut-il savoir pour agir ? \\[0pt]
Que gagne-t-on à se mettre à la place d'autrui ? \\[0pt]
Que garantit la séparation des pouvoirs ? \\[0pt]
Quel est le but de la politique ? \\[0pt]
Quel est le fondement de la propriété ? \\[0pt]
Quel est le fondement de l'autorité ? \\[0pt]
Quel est le problème posé par la pluralité des langues ? \\[0pt]
Quel est le rôle de la créativité dans les sciences ? \\[0pt]
Quel est le rôle du médecin ? \\[0pt]
Quel est l'être de l'illusion ? \\[0pt]
Quel est l'homme des Droits de l'homme ? \\[0pt]
Quel est l'objet de la géométrie ? \\[0pt]
Quel est l'objet de la logique ? \\[0pt]
Quel est l'objet de la métaphysique ? \\[0pt]
Quel est l'objet de la psychologie collective ? \\[0pt]
Quel est l'objet de la science ? \\[0pt]
Quel est l'objet de l'échange ? \\[0pt]
Quel est l'objet de l'ontologie ? \\[0pt]
Quelle espèce de vivants sommes-nous ? \\[0pt]
Quelle est la nature du droit naturel ? \\[0pt]
Quelle est la réalité de la matière ? \\[0pt]
Quelle est la réalité des objets mathématiques ? \\[0pt]
Quelle est la valeur de la connaissance ? \\[0pt]
Quelle est la valeur du témoignage ? \\[0pt]
Quelle idée le fanatique se fait-il de la vérité ? \\[0pt]
Quelle place donner au vraisemblable ? \\[0pt]
Quelle place la raison peut-elle faire à la croyance ? \\[0pt]
Quelle réalité conférer aux fictions ? \\[0pt]
Quelles actions permettent d'être heureux ? \\[0pt]
Quel rôle l'imagination joue-t-elle en mathématiques ? \\[0pt]
Quel statut philosophique pour l'opinion ? \\[0pt]
Quel type de réalité faut-il attribuer au temps ? \\[0pt]
Quel usage faire de la finalité ? \\[0pt]
Quel vivant sommes-nous ? \\[0pt]
Que montre l'image ? \\[0pt]
Que ne peut-on pas expliquer ? \\[0pt]
Que nous apprend la biologie sur l'homme ? \\[0pt]
Que nous apprend la géométrie ? \\[0pt]
Que nous apprend la littérature ? \\[0pt]
Que nous apprend l'animal ? \\[0pt]
Que nous apprend l'anthropologie ? \\[0pt]
Que nous apprend l'approche sociologique de l'art ? \\[0pt]
Que nous apprend le plaisir ? \\[0pt]
Que nous apprend l'erreur ? \\[0pt]
Que nous apprend le témoignage ? \\[0pt]
Que nous apprend le théâtre ? \\[0pt]
Que nous apprend le toucher ? \\[0pt]
Que nous apprend l'histoire des sciences ? \\[0pt]
Que nous apprennent les Anciens ? \\[0pt]
Que nous apprennent les controverses scientifiques ? \\[0pt]
Que nous apprennent les jeux ? \\[0pt]
Que nous apprennent les langues étrangères ? \\[0pt]
Que nous apprennent les objets techniques ? \\[0pt]
Que nous apprennent les statistiques ? \\[0pt]
Que nous apprennent nos sentiments ? \\[0pt]
Que nous devons-nous ? \\[0pt]
Que nous enseigne la lecture des romans ? \\[0pt]
Que nous enseigne la monstruosité ? \\[0pt]
Que nous montrent les natures mortes ? \\[0pt]
« Que nul n'entre ici s'il n'est géomètre » \\[0pt]
Que peindre ? \\[0pt]
Que peint le peintre ? \\[0pt]
Que perd la pensée en perdant l'écriture ? \\[0pt]
Que peut la force ? \\[0pt]
Que peut l'art ? \\[0pt]
Que peut la volonté contre les passions ? \\[0pt]
Que peut le droit ? \\[0pt]
Que peut l'intelligence artificielle ? \\[0pt]
Que peut-on attendre d'une philosophie de l'art ? \\[0pt]
Que peut-on calculer ? \\[0pt]
Que peut-on comprendre qu'on ne puisse expliquer ? \\[0pt]
Que peut-on démontrer ? \\[0pt]
Que peut-on enseigner ? \\[0pt]
Que peut-on espérer ? \\[0pt]
Que peut-on exprimer ? \\[0pt]
Que peut-on partager ? \\[0pt]
Que peut-on perdre ? \\[0pt]
Que peut signifier « avoir le sens de la réalité » ? \\[0pt]
Que peut un corps ? \\[0pt]
Que peuvent les idées ? \\[0pt]
Que peuvent nous apprendre les expériences de pensée ? \\[0pt]
Que pouvons-nous aujourd'hui apprendre des sciences d'autrefois ? \\[0pt]
Que pouvons-nous espérer ? \\[0pt]
Que pouvons-nous partager ? \\[0pt]
Que puis-je dire être mien ? \\[0pt]
Que sais-je de ma souffrance ? \\[0pt]
Que sait la science ? \\[0pt]
Que savons-nous de l'avenir ? \\[0pt]
Que savons-nous de nous-mêmes ? \\[0pt]
Que sent le corps ? \\[0pt]
Que serait le meilleur des mondes ? \\[0pt]
Que serait un art total ? \\[0pt]
Que serions-nous sans l'État ? \\[0pt]
Que signifie apprendre ? \\[0pt]
Que signifie : « avoir de l'esprit » ? \\[0pt]
Que signifie être dépendant ? \\[0pt]
Que signifie l'angoisse ? \\[0pt]
Que sont les institutions sans les individus ? \\[0pt]
Qu'est-ce la vérité ? \\[0pt]
Qu'est-ce qu'abstraire ? \\[0pt]
Qu'est-ce qu'agir avec discernement ? \\[0pt]
Qu'est-ce qu'apparaître ? \\[0pt]
Qu'est-ce qu'appliquer une règle de droit ? \\[0pt]
Qu'est-ce qu'apprendre ? \\[0pt]
Qu'est-ce qu'avoir conscience de soi ? \\[0pt]
Qu'est-ce qu'avoir de l'expérience ? \\[0pt]
Qu'est-ce qu'avoir du style ? \\[0pt]
Qu'est-ce qu'avoir raison ? \\[0pt]
Qu'est-ce qu'avoir un droit ? \\[0pt]
Qu'est-ce qu'avoir une idée ? \\[0pt]
Qu'est-ce que bien vivre ? \\[0pt]
Qu'est-ce que classer ? \\[0pt]
Qu'est-ce que connaître ? \\[0pt]
Qu'est-ce que démontrer ? \\[0pt]
Qu'est-ce que déraisonner ? \\[0pt]
Qu'est-ce que Dieu pour un athée ? \\[0pt]
Qu'est-ce que discuter ? \\[0pt]
Qu'est-ce que faire autorité ? \\[0pt]
Qu'est-ce que faire la paix ? \\[0pt]
Qu'est-ce que faire preuve d'humanité ? \\[0pt]
Qu'est-ce que gouverner ? \\[0pt]
Qu'est-ce que guérir ? \\[0pt]
Qu'est-ce que juger ? \\[0pt]
Qu'est-ce que la culture générale \\[0pt]
Qu'est-ce que la psychologie ? \\[0pt]
Qu'est-ce que le désordre ? \\[0pt]
Qu'est-ce que le moi ? \\[0pt]
Qu'est-ce que le naturalisme ? \\[0pt]
Qu'est-ce que l'enfance ? \\[0pt]
Qu'est-ce que l'expérience pour un empiriste ? \\[0pt]
Qu'est-ce que l'harmonie ? \\[0pt]
Qu'est-ce que lire ? \\[0pt]
Qu'est-ce que méditer ? \\[0pt]
Qu'est-ce qu'enquêter ? \\[0pt]
Qu'est-ce que parler ? \\[0pt]
Qu'est-ce que perdre son temps ? \\[0pt]
Qu'est-ce que prendre conscience ? \\[0pt]
Qu'est-ce que raisonner ? \\[0pt]
Qu'est-ce que rendre raison d'un effet ? \\[0pt]
Qu'est-ce que résoudre une contradiction ? \\[0pt]
Qu'est-ce que réussir sa vie ? \\[0pt]
Qu'est-ce que se tromper ? \\[0pt]
Qu'est-ce que s'orienter ? \\[0pt]
Qu'est-ce que traduire ? \\[0pt]
Qu'est-ce qu'être chez soi ? \\[0pt]
Qu'est-ce qu'être de bonne volonté ? \\[0pt]
Qu'est-ce qu'être démocrate ? \\[0pt]
Qu'est-ce qu'être fou ? \\[0pt]
Qu'est-ce qu'être malade ? \\[0pt]
Qu'est-ce qu'être réaliste ? \\[0pt]
Qu'est-ce qu'être sceptique ? \\[0pt]
Qu'est-ce qu'être seul ? \\[0pt]
Qu'est-ce qu'être visionnaire ? \\[0pt]
Qu'est-ce qu'être vivant ? \\[0pt]
Qu'est-ce que un individu \\[0pt]
Qu'est-ce que vieillir ? \\[0pt]
Qu'est-ce que vivre ? \\[0pt]
Qu'est-ce que vivre au présent ? \\[0pt]
Qu'est-ce qu'exercer un pouvoir ? \\[0pt]
Qu'est-ce qu'habiter ? \\[0pt]
Qu'est-ce qui adoucit les mœurs ? \\[0pt]
Qu'est-ce qui a du sens ? \\[0pt]
Qu'est-ce qui agit ? \\[0pt]
Qu'est-ce qui apparaît ? \\[0pt]
Qu'est-ce qu'identifier ? \\[0pt]
Qu'est-ce qui dépend de nous ? \\[0pt]
Qu'est-ce qui est anormal ? \\[0pt]
Qu'est-ce qui est concret ? \\[0pt]
Qu'est-ce qui est contre nature ? \\[0pt]
Qu'est-ce qui est démonstratif ? \\[0pt]
Qu'est-ce qui est donné ? \\[0pt]
Qu'est-ce qui est historique ? \\[0pt]
Qu'est-ce qui est impossible ? \\[0pt]
Qu'est-ce qui est irréfutable ? \\[0pt]
Qu'est-ce qui est mien ? \\[0pt]
Qu'est-ce qui est moderne ? \\[0pt]
Qu'est-ce qui est nécessaire ? \\[0pt]
Qu'est-ce qui est noble ? \\[0pt]
Qu'est-ce qui est réel ? \\[0pt]
Qu'est-ce qui est respectable ? \\[0pt]
Qu'est-ce qui est sacré ? \\[0pt]
Qu'est-ce qui est sublime ? \\[0pt]
Qu'est-ce qui est vital pour le vivant ? \\[0pt]
Qu'est-ce qui existe ? \\[0pt]
Qu'est-ce qui fait la force de la loi ? \\[0pt]
Qu'est-ce qui fait la valeur d'une croyance ? \\[0pt]
Qu'est-ce qui fait le propre d'un corps propre ? \\[0pt]
Qu'est-ce qui fait l'humanité d'un corps ? \\[0pt]
Qu'est-ce qui fait l'unité d'une science ? \\[0pt]
Qu'est-ce qui fait l'unité d'un peuple ? \\[0pt]
Qu'est-ce qui fait obstacle à la science ? \\[0pt]
Qu'est-ce qui fait qu'un peuple est un peuple ? \\[0pt]
Qu'est-ce qui fait un peuple ? \\[0pt]
Qu'est-ce qui importe dans l'éducation ? \\[0pt]
Qu'est-ce qui me rend plus fort ? \\[0pt]
Qu'est-ce qui mérite l'admiration ? \\[0pt]
Qu'est-ce qui m'est étranger ? \\[0pt]
Qu'est-ce qui n'est pas maîtrisable ? \\[0pt]
Qu'est-ce qui n'est pas mathématisable ? \\[0pt]
Qu'est-ce qui n'est pas normal ? \\[0pt]
Qu'est-ce qu'interpréter ? \\[0pt]
Qu'est-ce qu'interpréter une œuvre d'art ? \\[0pt]
Qu'est-ce qui peut être hors du temps ? \\[0pt]
Qu'est-ce qu'obéir ? \\[0pt]
Qu'est-ce qu'on attend ? \\[0pt]
Qu'est-ce qu'on ne peut pas partager ? \\[0pt]
Qu'est-ce qu'un abus de langage ? \\[0pt]
Qu'est-ce qu'un abus de pouvoir ? \\[0pt]
Qu'est-ce qu'un acteur ? \\[0pt]
Qu'est-ce qu'un alter ego \\[0pt]
Qu'est-ce qu'un ami ? \\[0pt]
Qu'est-ce qu'un animal ? \\[0pt]
Qu'est-ce qu'un animal domestique ? \\[0pt]
Qu'est-ce qu'un argument ? \\[0pt]
Qu'est-ce qu'un auteur ? \\[0pt]
Qu'est-ce qu'un axiome ? \\[0pt]
Qu'est-ce qu'un beau paysage ? \\[0pt]
Qu'est-ce qu'un bon commentaire ? \\[0pt]
Qu'est-ce qu'un bon conseil ? \\[0pt]
Qu'est-ce qu'un cas de conscience ? \\[0pt]
Qu'est-ce qu'un chef ? \\[0pt]
Qu'est-ce qu'un chef-d'œuvre ? \\[0pt]
Qu'est-ce qu'un choix éclairé ? \\[0pt]
Qu'est-ce qu'un choix rationnel ? \\[0pt]
Qu'est-ce qu'un citoyen ? \\[0pt]
Qu'est-ce qu'un concept ? \\[0pt]
Qu'est-ce qu'un conflit de valeurs ? \\[0pt]
Qu'est-ce qu'un contenu de conscience ? \\[0pt]
Qu'est-ce qu'un contrat ? \\[0pt]
Qu'est-ce qu'un coup d'État ? \\[0pt]
Qu'est-ce qu'un crime contre l'humanité ? \\[0pt]
Qu'est-ce qu'un déni ? \\[0pt]
Qu'est-ce qu'un dieu ? \\[0pt]
Qu'est-ce qu'un Dieu ? \\[0pt]
Qu'est-ce qu'un dogme ? \\[0pt]
Qu'est-ce qu'une alternative ? \\[0pt]
Qu'est-ce qu'une âme ? \\[0pt]
Qu'est-ce qu'une aporie ? \\[0pt]
Qu'est-ce qu'une bonne loi ? \\[0pt]
Qu'est-ce qu'une bonne méthode ? \\[0pt]
Qu'est-ce qu'une bonne théorie ? \\[0pt]
Qu'est-ce qu'une catégorie ? \\[0pt]
Qu'est-ce qu'une cause ? \\[0pt]
Qu'est-ce qu'une chimère ? \\[0pt]
Qu'est-ce qu'une chose ? \\[0pt]
Qu'est-ce qu'une civilisation ? \\[0pt]
Qu'est-ce qu'une classe sociale ? \\[0pt]
Qu'est-ce qu'une classification scientifique ? \\[0pt]
Qu'est-ce qu'une collectivité ? \\[0pt]
Qu'est-ce qu'une communauté ? \\[0pt]
Qu'est-ce qu'une communauté humaine ? \\[0pt]
Qu'est-ce qu'une communauté scientifique ? \\[0pt]
Qu'est-ce qu'une condition de possibilité ? \\[0pt]
Qu'est-ce qu'une condition suffisante ? \\[0pt]
Qu'est-ce qu'une conduite irrationnelle ? \\[0pt]
Qu'est-ce qu'une connaissance a priori ? \\[0pt]
Qu'est-ce qu'une constitution ? \\[0pt]
Qu'est-ce qu'une crise ? \\[0pt]
Qu'est-ce qu'une croyance rationnelle ? \\[0pt]
Qu'est-ce qu'une découverte ? \\[0pt]
Qu'est-ce qu'une découverte scientifique ? \\[0pt]
Qu'est-ce qu'une disposition ? \\[0pt]
Qu'est-ce qu'une école philosophique ? \\[0pt]
Qu'est-ce qu'une éducation réussie ? \\[0pt]
Qu'est-ce qu'une erreur ? \\[0pt]
Qu'est-ce qu'une espèce naturelle ? \\[0pt]
Qu'est-ce qu'une expérience cruciale ? \\[0pt]
Qu'est-ce qu'une expérience de pensée ? \\[0pt]
Qu'est-ce qu'une expérience esthétique ? \\[0pt]
Qu'est-ce qu'une expérience politique ? \\[0pt]
Qu'est-ce qu'une famille ? \\[0pt]
Qu'est-ce qu'une forme ? \\[0pt]
Qu'est-ce qu'une hypothèse ? \\[0pt]
Qu'est-ce qu'une hypothèse scientifique ? \\[0pt]
Qu'est-ce qu'une idée ? \\[0pt]
Qu'est-ce qu'une idée incertaine ? \\[0pt]
Qu'est-ce qu'une idéologie scientifique ? \\[0pt]
Qu'est-ce qu'une image ? \\[0pt]
Qu'est-ce qu'une impression ? \\[0pt]
Qu'est-ce qu'une inférence ? \\[0pt]
Qu'est-ce qu'une injustice ? \\[0pt]
Qu'est-ce qu'une innovation technologique ? \\[0pt]
Qu'est-ce qu'une institution ? \\[0pt]
Qu'est-ce qu'une langue morte ? \\[0pt]
Qu'est-ce qu'une limite ? \\[0pt]
Qu'est-ce qu'une loi ? \\[0pt]
Qu'est-ce qu'une loi scientifique ? \\[0pt]
Qu'est-ce qu'une machine ? \\[0pt]
Qu'est-ce qu'une machine ne peut pas faire ? \\[0pt]
Qu'est-ce qu'une marchandise ? \\[0pt]
Qu'est-ce qu'une mauvaise interprétation ? \\[0pt]
Qu'est-ce qu'une méditation ? \\[0pt]
Qu'est-ce qu'une méthode ? \\[0pt]
Qu'est-ce qu'un empire ? \\[0pt]
Qu'est-ce qu'une nation ? \\[0pt]
Qu'est-ce qu'un enfant ? \\[0pt]
Qu'est-ce qu'une norme ? \\[0pt]
Qu'est-ce qu'une œuvre ? \\[0pt]
Qu'est-ce qu'une œuvre d'art ? \\[0pt]
Qu'est-ce qu'une œuvre d'art réussie ? \\[0pt]
Qu'est-ce qu'une parole vivante ? \\[0pt]
Qu'est-ce qu'une phrase ? \\[0pt]
Qu'est-ce qu'une preuve ? \\[0pt]
Qu'est-ce qu'une promesse ? \\[0pt]
Qu'est-ce qu'une propriété ? \\[0pt]
Qu'est-ce qu'une question dénuée de sens ? \\[0pt]
Qu'est-ce qu'une question métaphysique ? \\[0pt]
Qu'est-ce qu'une réfutation ? \\[0pt]
Qu'est-ce qu'une règle de grammaire ? \\[0pt]
Qu'est-ce qu'une religion ? \\[0pt]
Qu'est-ce qu'une représentation réussie ? \\[0pt]
Qu'est-ce qu'une révolution ? \\[0pt]
Qu'est-ce qu'une révolution scientifique ? \\[0pt]
Qu'est-ce qu'une science exacte ? \\[0pt]
Qu'est-ce qu'une situation ? \\[0pt]
Qu'est-ce qu'une situation tragique ? \\[0pt]
Qu'est-ce qu'une société ? \\[0pt]
Qu'est-ce qu'un esprit faux ? \\[0pt]
Qu'est-ce qu'une structure ? \\[0pt]
Qu'est-ce qu'une substance ? \\[0pt]
Qu'est-ce qu'un état mental ? \\[0pt]
Qu'est-ce qu'un État souverain ? \\[0pt]
Qu'est-ce qu'une théorie scientifique ? \\[0pt]
Qu'est-ce qu'une tradition ? \\[0pt]
Qu'est-ce qu'une tragédie historique ? \\[0pt]
Qu'est-ce qu'un être cultivé ? \\[0pt]
Qu'est-ce qu'un être imaginaire ? \\[0pt]
Qu'est-ce qu'une valeur ? \\[0pt]
Qu'est-ce qu'un événement ? \\[0pt]
Qu'est-ce qu'un événement historique ? \\[0pt]
Qu'est-ce qu'une vie humaine ? \\[0pt]
Qu'est-ce qu'une ville ? \\[0pt]
Qu'est-ce qu'une vision du monde ? \\[0pt]
Qu'est-ce qu'un exemple ? \\[0pt]
Qu'est-ce qu'un expert ? \\[0pt]
Qu'est-ce qu'un fait ? \\[0pt]
Qu'est-ce qu'un fait divers ? \\[0pt]
Qu'est-ce qu'un fait historique ? \\[0pt]
Qu'est-ce qu'un fait scientifique ? \\[0pt]
Qu'est-ce qu'un fait social ? \\[0pt]
Qu'est-ce qu'un faux problème ? \\[0pt]
Qu'est-ce qu'un faux sentiment ? \\[0pt]
Qu'est-ce qu'un geste technique ? \\[0pt]
Qu'est-ce qu'un grand philosophe ? \\[0pt]
Qu'est-ce qu'un héros ? \\[0pt]
Qu'est-ce qu'un homme bon ? \\[0pt]
Qu'est-ce qu'un homme politique ? \\[0pt]
Qu'est-ce qu'un homme seul ? \\[0pt]
Qu'est-ce qu'un individu ? \\[0pt]
Qu'est-ce qu'un jeu ? \\[0pt]
Qu'est-ce qu'un juge ? \\[0pt]
Qu'est-ce qu'un jugement analytique ? \\[0pt]
Qu'est-ce qu'un laboratoire ? \\[0pt]
Qu'est-ce qu'un législateur ? \\[0pt]
Qu'est-ce qu'un lien logique ? \\[0pt]
Qu'est-ce qu'un lieu commun ? \\[0pt]
Qu'est-ce qu'un livre ? \\[0pt]
Qu'est-ce qu'un maître ? \\[0pt]
Qu'est-ce qu'un modèle ? \\[0pt]
Qu'est-ce qu'un moderne ? \\[0pt]
Qu'est-ce qu'un monde ? \\[0pt]
Qu'est-ce qu'un monde commun ? \\[0pt]
Qu'est-ce qu'un monstre ? \\[0pt]
Qu'est-ce qu'un monument ? \\[0pt]
Qu'est-ce qu'un mythe ? \\[0pt]
Qu'est-ce qu'un nombre ? \\[0pt]
Qu'est-ce qu'un nom propre ? \\[0pt]
Qu'est-ce qu'un objet ? \\[0pt]
Qu'est-ce qu'un objet d'art ? \\[0pt]
Qu'est-ce qu'un objet de connaissance ? \\[0pt]
Qu'est-ce qu'un objet esthétique ? \\[0pt]
Qu'est-ce qu'un organisme ? \\[0pt]
Qu'est-ce qu'un original ? \\[0pt]
Qu'est-ce qu'un outil ? \\[0pt]
Qu'est-ce qu'un paradoxe ? \\[0pt]
Qu'est-ce qu'un paysage ? \\[0pt]
Qu'est-ce qu'un pédant ? \\[0pt]
Qu'est-ce qu'un peuple \\[0pt]
Qu'est-ce qu'un peuple ? \\[0pt]
Qu'est-ce qu'un peuple libre ? \\[0pt]
Qu'est-ce qu'un phénomène ? \\[0pt]
Qu'est-ce qu'un plaisir pur ? \\[0pt]
Qu'est-ce qu'un point de vue ? \\[0pt]
Qu'est-ce qu'un postulat ? \\[0pt]
Qu'est-ce qu'un principe ? \\[0pt]
Qu'est-ce qu'un problème ? \\[0pt]
Qu'est-ce qu'un problème éthique ? \\[0pt]
Qu'est-ce qu'un problème philosophique ? \\[0pt]
Qu'est-ce qu'un problème politique ? \\[0pt]
Qu'est-ce qu'un projet ? \\[0pt]
Qu'est-ce qu'un rapport de force ? \\[0pt]
Qu'est-ce qu'un rite ? \\[0pt]
Qu'est-ce qu'un sage ? \\[0pt]
Qu'est-ce qu'un signe ? \\[0pt]
Qu'est-ce qu'un sophisme ? \\[0pt]
Qu'est-ce qu'un sophiste ? \\[0pt]
Qu'est-ce qu'un souvenir ? \\[0pt]
Qu'est-ce qu'un spécialiste ? \\[0pt]
Qu'est-ce qu'un style ? \\[0pt]
Qu'est-ce qu'un sujet de droit ? \\[0pt]
Qu'est-ce qu'un symbole ? \\[0pt]
Qu'est-ce qu'un symptôme ? \\[0pt]
Qu'est-ce qu'un système ? \\[0pt]
Qu'est-ce qu'un système philosophique ? \\[0pt]
Qu'est-ce qu'un tableau \\[0pt]
Qu'est-ce qu'un tableau ? \\[0pt]
Qu'est-ce qu'un témoin ? \\[0pt]
Qu'est-ce qu'un territoire ? \\[0pt]
Qu'est-ce qu'un texte ? \\[0pt]
Qu'est-ce qu'un tout ? \\[0pt]
Que suppose le mouvement ? \\[0pt]
Que transmettons-nous ? \\[0pt]
Que valent les décisions de la majorité ? \\[0pt]
Que valent les excuses ? \\[0pt]
Que valent les idées générales ? \\[0pt]
Que valent les preuves ? \\[0pt]
Que vaut la distinction entre nature et culture ? \\[0pt]
Que vaut la fidélité ? \\[0pt]
Que vaut l'excuse : « C'est plus fort que moi » ? \\[0pt]
Que vaut l'incertain ? \\[0pt]
Que vaut un argument d'autorité ? \\[0pt]
Que vaut un consensus ? \\[0pt]
Que vaut une preuve contre un préjugé ? \\[0pt]
Que veut dire avoir raison ? \\[0pt]
Que veut dire l'expression « aller au fond des choses » ? \\[0pt]
Que veut-on dire quand on dit que « rien n'est sans raison » ? \\[0pt]
Que voit-on dans une image ? \\[0pt]
Que voit-on dans un miroir ? \\[0pt]
Que voulons-nous ? \\[0pt]
Qui a des droits ? \\[0pt]
Qui agit ? \\[0pt]
Qui a le droit de juger ? \\[0pt]
Qui a une histoire ? \\[0pt]
Qui connaît le mieux mon corps ? \\[0pt]
Qui croire ? \\[0pt]
Qui doit dire le droit ? \\[0pt]
Qui doit éduquer ? \\[0pt]
Qui doit faire les lois ? \\[0pt]
Qui est crédible ? \\[0pt]
Qui est cultivé ? \\[0pt]
Qui est le maître ? \\[0pt]
Qui est libre ? \\[0pt]
Qui est méchant ? \\[0pt]
Qui est métaphysicien ? \\[0pt]
Qui est responsable ? \\[0pt]
Qui est sage ? \\[0pt]
Qui est souverain ? \\[0pt]
Qui est une personne ? \\[0pt]
Qui fait la loi ? \\[0pt]
Qui faut-il protéger ? \\[0pt]
Qui mérite d'être aimé ? \\[0pt]
Qui meurt ? \\[0pt]
« Qui ne dit mot consent » \\[0pt]
Qui parle ? \\[0pt]
Qui pense ? \\[0pt]
Qui peut obliger ? \\[0pt]
Qui peut parler ? \\[0pt]
Qui sont mes amis ? \\[0pt]
Qui sont mes semblables ? \\[0pt]
Qui sont nos ennemis ? \\[0pt]
Qui suis-je ? \\[0pt]
Qui suis-je pour me juger ? \\[0pt]
Qu'y a-t-il à comprendre en histoire ? \\[0pt]
Qu'y a-t-il à l'origine de toutes choses ? \\[0pt]
Qu'y a-t-il au fondement de l'objectivité ? \\[0pt]
Qu'y a-t-il de technique dans l'art ? \\[0pt]
Raconter sa vie \\[0pt]
Raconter son histoire \\[0pt]
Raison et déraison \\[0pt]
Raison et révélation \\[0pt]
Raisonner et argumenter \\[0pt]
Raisonner et décider \\[0pt]
Rapports de la science et de la politique \\[0pt]
Réalisme et idéalisme \\[0pt]
Réalité et idéal \\[0pt]
Recevoir \\[0pt]
Récit et mémoire \\[0pt]
Reconnaissons-nous le bien comme nous reconnaissons le vrai ? \\[0pt]
Réécrire l'histoire \\[0pt]
Refaire le monde \\[0pt]
Réforme et révolution \\[0pt]
Réfuter \\[0pt]
Regarder \\[0pt]
Regarder un tableau \\[0pt]
Règles logiques et lois psychologiques \\[0pt]
Régularité et singularité \\[0pt]
Religion et superstition \\[0pt]
Religion naturelle et religion révélée \\[0pt]
Rendre justice \\[0pt]
Rendre la justice \\[0pt]
Renoncer à ses préjugés \\[0pt]
Renoncer au passé \\[0pt]
Rentrer en soi-même \\[0pt]
Réparer \\[0pt]
Répondre \\[0pt]
Représentation et illusion \\[0pt]
République et démocratie \\[0pt]
Résistance et soumission \\[0pt]
Résister \\[0pt]
Respecter la nature, est-ce renoncer à l'exploiter ? \\[0pt]
Ressemblance et imitation \\[0pt]
Ressentir \\[0pt]
Rêve et réalité \\[0pt]
Révéler \\[0pt]
Rêver \\[0pt]
Rhétorique et vérité \\[0pt]
Rien de nouveau sous le soleil \\[0pt]
« Rien n'est sans raison » \\[0pt]
« Rien n'est simple » \\[0pt]
Rire \\[0pt]
Rire et pleurer \\[0pt]
S'adapter \\[0pt]
Sagesse et renoncement \\[0pt]
Sait-on toujours ce que l'on fait ? \\[0pt]
Sait-on toujours ce qu'on veut ? \\[0pt]
S'aliéner \\[0pt]
Sans foi ni loi \\[0pt]
S'approprier une œuvre d'art \\[0pt]
« Sauver les phénomènes » \\[0pt]
Sauver les phénomènes \\[0pt]
Savoir ce qu'on dit \\[0pt]
Savoir ce qu'on fait \\[0pt]
Savoir ce qu'on veut \\[0pt]
Savoir de quoi on parle \\[0pt]
Savoir, est-ce déjouer le réel ? \\[0pt]
Savoir, est-ce pouvoir ? \\[0pt]
Savoir et liberté \\[0pt]
Savoir et pouvoir \\[0pt]
Savoir être heureux \\[0pt]
Savoir, pouvoir \\[0pt]
Savoir s'arrêter \\[0pt]
Savoir se décider \\[0pt]
Savoir vivre \\[0pt]
Savons-nous ce que nous disons ? \\[0pt]
Savons-nous ce que peut un corps ? \\[0pt]
Savons-nous de ce que nous disons ? \\[0pt]
Savons-nous nous souvenir ? \\[0pt]
Scepticisme et relativisme \\[0pt]
Science et conscience \\[0pt]
Science et hypothèse \\[0pt]
Science et imagination \\[0pt]
Science et mythe \\[0pt]
Science et philosophie \\[0pt]
Science et sagesse \\[0pt]
Science et société \\[0pt]
Science et technique \\[0pt]
Sciences de la nature et sciences de l'esprit \\[0pt]
Sciences et philosophie \\[0pt]
Se conserver \\[0pt]
Sécularisation et laïcisation \\[0pt]
Se cultiver \\[0pt]
Se défendre \\[0pt]
Se divertir \\[0pt]
Se donner corps et âme \\[0pt]
Séduire \\[0pt]
Se faire justice \\[0pt]
Se méfier des idées \\[0pt]
S'ennuyer \\[0pt]
Sensation et perception \\[0pt]
Sens et fait \\[0pt]
Sens et sensible \\[0pt]
Sentiment et émotion \\[0pt]
Sentir \\[0pt]
Se parler et s'entendre \\[0pt]
Se passer de philosophie \\[0pt]
Se prendre au sérieux \\[0pt]
Se raisonner \\[0pt]
Se réfugier dans la croyance ? \\[0pt]
Se retirer dans la pensée ? \\[0pt]
Se retirer du monde \\[0pt]
Servir \\[0pt]
Se souvenir \\[0pt]
Se souvenir, est-ce préparer l'avenir ? \\[0pt]
Se taire \\[0pt]
S'étonner \\[0pt]
Se voiler la face \\[0pt]
S'exercer \\[0pt]
S'exprimer \\[0pt]
Si Dieu n'existe pas, tout est-il permis ? \\[0pt]
Signes naturels, signes artificiels \\[0pt]
Signification et expression \\[0pt]
Signification et vérité \\[0pt]
Sincérité et authenticité \\[0pt]
S'indigner, est-ce un devoir ? \\[0pt]
S'intéresser à l'art \\[0pt]
Société et communauté \\[0pt]
Société et réciprocité \\[0pt]
Sociologie et ethnologie \\[0pt]
Soi \\[0pt]
Solitude et liberté \\[0pt]
Sommes-nous capables d'agir de manière désintéressée ? \\[0pt]
Sommes-nous libres de nos croyances ? \\[0pt]
Sommes-nous libres de nos pensées ? \\[0pt]
Sommes-nous libres par nature ? \\[0pt]
Sommes-nous maîtres de la nature ? \\[0pt]
Sommes-nous responsables de ce que nous sommes ? \\[0pt]
Sommes-nous responsables de nos erreurs ? \\[0pt]
Sommes-nous responsables du sens que prennent nos paroles ? \\[0pt]
Sophisme et paralogisme \\[0pt]
Sophismes et paradoxes \\[0pt]
S'orienter \\[0pt]
S'orienter dans la pensée \\[0pt]
S'orienter dans le monde \\[0pt]
Sortir de soi \\[0pt]
Soupçonner \\[0pt]
Soutenir une thèse \\[0pt]
Souveraineté et liberté \\[0pt]
Sphère privée et sphère publique \\[0pt]
Spontanéité et liberté \\[0pt]
Structure et histoire \\[0pt]
Subir \\[0pt]
Subjectivité et vérité \\[0pt]
Substance et sujet \\[0pt]
Suffit-il d'être dans le vrai pour savoir ? \\[0pt]
Suffit-il d'être juste ? \\[0pt]
Suis-je au centre de l'espace ? \\[0pt]
Suis-je aussi ce que j'aurais pu être ? \\[0pt]
Suis-je le sujet de mes pensées ? \\[0pt]
Suis-je maître de mes pensées ? \\[0pt]
Suis-je ma mémoire ? \\[0pt]
Suis-je mon corps ? \\[0pt]
Suivre la nature \\[0pt]
Sujet et citoyen \\[0pt]
Sur quels fondements distinguer ce qui est public et ce qui est privé ? \\[0pt]
Sur quoi fonder la propriété ? \\[0pt]
Sur quoi reposent nos certitudes ? \\[0pt]
Sur quoi se fonde la vérité des énoncés ? \\[0pt]
Surveiller son comportement \\[0pt]
Survivre \\[0pt]
Suspendre son assentiment \\[0pt]
Suspendre son jugement \\[0pt]
Tantôt je pense, tantôt je suis \\[0pt]
Technique et démocratie \\[0pt]
Technique et technologie \\[0pt]
Témoigner \\[0pt]
Temps et durée \\[0pt]
Temps et réalité \\[0pt]
Tenir parole \\[0pt]
Tester une hypothèse permet-il de la prouver ? \\[0pt]
Texte, son, image \\[0pt]
Théorie et pratique \\[0pt]
Théorie et pratique en médecine \\[0pt]
Tolérer \\[0pt]
Tomber amoureux \\[0pt]
Toucher \\[0pt]
Tous les arts se valent-ils ? \\[0pt]
Tous les droits sont-ils formels ? \\[0pt]
Tous les hommes désirent-ils connaître ? \\[0pt]
Tous les hommes désirent-ils être heureux ? \\[0pt]
Tout art est-il abstrait ? \\[0pt]
Tout a-t-il une fin ? \\[0pt]
Tout a-t-il un sens ? \\[0pt]
Tout définir, tout démontrer \\[0pt]
Tout désir est-il nostalgique ? \\[0pt]
Toute chose a-t-elle une essence ? \\[0pt]
Toute connaissance est-elle relative ? \\[0pt]
Toute conversation est-elle futile ? \\[0pt]
Toute définition est-elle une convention ? \\[0pt]
Toute description est-elle une interprétation ? \\[0pt]
Toute expression est-elle métaphorique ? \\[0pt]
Toute loi est-elle arbitraire ? \\[0pt]
Toute origine est-elle mythique ? \\[0pt]
« Toute peine mérite salaire » \\[0pt]
Toute peur est-elle irrationnelle ? \\[0pt]
Toute philosophie est-elle systématique ? \\[0pt]
Tout est corps \\[0pt]
Tout est-il à vendre ? \\[0pt]
Tout est-il connaissable ? \\[0pt]
Tout est-il contingent ? \\[0pt]
Tout est-il digne de mémoire ? \\[0pt]
Tout est-il mesurable ? \\[0pt]
Tout est-il nécessaire ? \\[0pt]
« Tout est politique » \\[0pt]
Tout événement a-t-il une cause ? \\[0pt]
Toute vérité est-elle démontrable ? \\[0pt]
Toute vérité est-elle un rapport ? \\[0pt]
Toute violence est-elle contre nature ? \\[0pt]
Tout malheur est-il une injustice ? \\[0pt]
Tout ou rien \\[0pt]
Tout peut-il être expliqué historiquement ? \\[0pt]
Tout peut-il n'être qu'apparence ? \\[0pt]
Tout peut-il se quantifier ? \\[0pt]
Tout pouvoir est-il politique ? \\[0pt]
Tout pouvoir relève-t-il de la domination ? \\[0pt]
Tout savoir \\[0pt]
Tout savoir est-il transmissible ? \\[0pt]
Tradition et innovation \\[0pt]
Tradition et raison \\[0pt]
Traduire \\[0pt]
Tragédie et comédie \\[0pt]
Trahir \\[0pt]
Transcendance et altérité \\[0pt]
Transmettre \\[0pt]
Travail et subjectivité \\[0pt]
Travail manuel, travail intellectuel \\[0pt]
« Trop beau pour être vrai » \\[0pt]
Tuer le temps \\[0pt]
« Tu ne tueras point » \\[0pt]
Un acte désintéressé est-il possible ? \\[0pt]
Un artiste doit-il être génial ? \\[0pt]
Un art peut-il se passer de règles ? \\[0pt]
Un corps n'est-il qu'un objet ? \\[0pt]
Une éducation des sentiments est-elle possible ? \\[0pt]
Une éducation esthétique est-elle possible ? \\[0pt]
Une explication peut-elle être réductrice ? \\[0pt]
Une fiction peut-elle être vraie ? \\[0pt]
Une idée peut-elle être fausse ? \\[0pt]
Une imitation peut-elle être parfaite ? \\[0pt]
Une inégalité peut-elle être juste ? \\[0pt]
Une justice sans égalité est-elle possible ? \\[0pt]
Une loi n'est-elle qu'une règle ? \\[0pt]
Une machine peut-elle être naturelle ? \\[0pt]
Une machine peut-elle penser ? \\[0pt]
Une machine pourrait-elle penser ? \\[0pt]
Une machine se réduit-elle à un mécanisme ? \\[0pt]
Une morale du plaisir est-elle concevable ? \\[0pt]
Une morale personnelle est-elle une contradiction dans les termes ? \\[0pt]
Une morale peut-elle se passer de la notion de vertu ? \\[0pt]
Une morale qui ne vise pas l'utilité est-elle rationnelle ? \\[0pt]
Une œuvre d'art a-t-elle un prix ? \\[0pt]
Une œuvre d'art est-elle toujours originale ? \\[0pt]
Une passion peut-elle résister au temps ? \\[0pt]
Une perception peut-elle être illusoire ? \\[0pt]
Une philosophie de l'amour est-elle possible ? \\[0pt]
Une religion civile est-elle possible ? \\[0pt]
Une religion peut-elle être rationnelle ? \\[0pt]
Une science bien faite est-elle une langue bien faite ? \\[0pt]
Une science de la conscience est-elle possible ? \\[0pt]
Une science des symboles est-elle possible ? \\[0pt]
Une théorie sans expérience nous apprend-elle quelque chose ? \\[0pt]
Une théorie scientifique peut-elle devenir fausse ? \\[0pt]
Une vérité est-elle discutable ? \\[0pt]
Une volonté peut-elle être générale ? \\[0pt]
Un homme n'est-il que la somme de ses actes ? \\[0pt]
Un jugement moral peut-il relever de la vérité ? \\[0pt]
Un langage peut-il être privé ? \\[0pt]
Un moment d'éternité \\[0pt]
Un monde sans beauté \\[0pt]
Un monde sans nature est-il pensable ? \\[0pt]
Un organisme n'est-il qu'un objet ? \\[0pt]
Un peuple est-il responsable de son histoire ? \\[0pt]
Un pouvoir a-t-il besoin d'être légitime ? \\[0pt]
Un préjugé n'a-t-il aucun fondement ? \\[0pt]
Un problème philosophique peut-il être périmé ? \\[0pt]
Urbanisme et politique \\[0pt]
Vainqueurs et vaincus \\[0pt]
Valeur et évaluation \\[0pt]
Vanité des vanités \\[0pt]
Vérité et cohérence \\[0pt]
Vérité et cohérence ? \\[0pt]
Vérité et convention \\[0pt]
Vérité et fiction \\[0pt]
Vérité et référence \\[0pt]
Vérité et sensibilité \\[0pt]
Vérité et signification \\[0pt]
Vérité et totalité \\[0pt]
Vérités mathématiques, vérités philosophiques \\[0pt]
Vertus morales, vertus intellectuelles \\[0pt]
Vie et destin \\[0pt]
Vie et existence \\[0pt]
Vie et matière \\[0pt]
Vie et matière : une distinction périmée ? \\[0pt]
Vie et mécanisme \\[0pt]
Vieillir \\[0pt]
Violence et discours \\[0pt]
Violence et politique \\[0pt]
Vit-on au présent ? \\[0pt]
Vivons-nous tous dans le même monde ? \\[0pt]
Vivre au présent \\[0pt]
Vivre caché \\[0pt]
Vivre comme si nous ne devions pas mourir \\[0pt]
Vivre et bien vivre \\[0pt]
Vivre et exister \\[0pt]
Vivre intensément \\[0pt]
Vivre pour les autres \\[0pt]
Vivre sa vie \\[0pt]
Vivre selon la nature \\[0pt]
Vivre sous la conduite de la raison \\[0pt]
Voir \\[0pt]
Voir et concevoir \\[0pt]
Voir et entendre \\[0pt]
Voir et savoir \\[0pt]
Voir et toucher \\[0pt]
Voir la réalité en face \\[0pt]
Voir les choses comme elles sont \\[0pt]
Vouloir ce que l'on peut \\[0pt]
Vouloir la paix \\[0pt]
Vouloir le mal \\[0pt]
Voyager \\[0pt]
Vulgariser la science ? \\[0pt]
Y a-t-il continuité entre l'expérience et la science ? \\[0pt]
Y a-t-il de bons préjugés ? \\[0pt]
Y a-t-il de fausses religions ? \\[0pt]
Y a-t-il de faux besoins ? \\[0pt]
Y a-t-il de la grandeur à être libre ? \\[0pt]
Y a-t-il de l'inconcevable ? \\[0pt]
Y a-t-il de l'inconnaissable ? \\[0pt]
Y a-t-il de l'indémontrable ? \\[0pt]
Y a-t-il de l'indicible ? \\[0pt]
Y a-t-il de l'universel ? \\[0pt]
Y a-t-il de mauvaises raisons ? \\[0pt]
Y a-t-il des actes intrinsèquement mauvais ? \\[0pt]
Y a-t-il des actions désintéressées ? \\[0pt]
Y a-t-il des aliénations heureuses ? \\[0pt]
Y a-t-il des arts majeurs ? \\[0pt]
Y a-t-il des arts mineurs ? \\[0pt]
Y a-t-il des canons de la beauté ? \\[0pt]
Y a-t-il des certitudes historiques ? \\[0pt]
Y a-t-il des choses à savoir ? \\[0pt]
Y a-t-il des choses qui échappent au droit ? \\[0pt]
Y a-t-il des conflits de devoirs ? \\[0pt]
Y a-t-il des critères de la scientificité ? \\[0pt]
Y a-t-il des critères de l'humanité ? \\[0pt]
Y a-t-il des croyances démocratiques ? \\[0pt]
Y a-t-il des croyances raisonnables ? \\[0pt]
Y a-t-il des degrés dans la certitude ? \\[0pt]
Y a-t-il des degrés de liberté ? \\[0pt]
Y a-t-il des degrés de réalité ? \\[0pt]
Y a-t-il des degrés de vérité ? \\[0pt]
Y a-t-il des démonstrations en philosophie ? \\[0pt]
Y a-t-il des devoirs envers soi-même ? \\[0pt]
Y a-t-il des droits naturels ? \\[0pt]
Y a-t-il des êtres de raison ? \\[0pt]
Y a-t-il des expériences métaphysiques ? \\[0pt]
Y a-t-il des faits moraux ? \\[0pt]
Y a-t-il des faits sans essence ? \\[0pt]
Y a-t-il des fins de la nature ? \\[0pt]
Y a-t-il des guerres justes ? \\[0pt]
Y a-t-il des héritages philosophiques ? \\[0pt]
Y a-t-il des idées générales ? \\[0pt]
Y a-t-il des idées innées ? \\[0pt]
Y a-t-il des leçons de l'histoire ? \\[0pt]
Y a-t-il des limites à la critique ? \\[0pt]
Y a-t-il des limites à la description ? \\[0pt]
Y a-t-il des limites à l'exprimable ? \\[0pt]
Y a-t-il des limites au droit ? \\[0pt]
Y a-t-il des lois non écrites ? \\[0pt]
Y a-t-il des normes de la vie ? \\[0pt]
Y a-t-il des notions communes ? \\[0pt]
Y a-t-il des pensées folles ? \\[0pt]
Y a-t-il des pensées inconscientes ? \\[0pt]
Y a-t-il des petites vertus ? \\[0pt]
Y a-t-il des plaisirs mauvais ? \\[0pt]
Y a-t-il des plaisirs purs ? \\[0pt]
Y a-t-il des plaisirs simples ? \\[0pt]
Y a-t-il des preuves d'amour ? \\[0pt]
Y a-t-il des preuves en métaphysique ? \\[0pt]
Y a-t-il des preuves en philosophie ? \\[0pt]
Y a-t-il des progrès en philosophie ? \\[0pt]
Y a-t-il des questions sans réponse ? \\[0pt]
Y a-t-il des questions sans réponses ? \\[0pt]
Y a-t-il des raisons d'aimer ? \\[0pt]
Y a-t-il des savoirs immédiats ? \\[0pt]
Y a-t-il des secrets de la nature ? \\[0pt]
Y a-t-il des sociétés sans histoire ? \\[0pt]
Y a-t-il des substances incorporelles ? \\[0pt]
Y a-t-il des vérités morales ? \\[0pt]
Y a-t-il des violences justifiées ? \\[0pt]
Y a-t-il des violences légitimes ? \\[0pt]
Y a-t-il du hasard objectif ? \\[0pt]
Y a-t-il du sacré dans la nature ? \\[0pt]
Y a-t-il du vrai dans le beau ? \\[0pt]
Y a-t-il lieu de distinguer entre éthique et morale ? \\[0pt]
Y a-t-il plusieurs manières de définir ? \\[0pt]
Y a-t-il plusieurs morales ? \\[0pt]
Y a-t-il quelque chose de commun aux œuvres d'art ? \\[0pt]
Y a-t-il un art de penser ? \\[0pt]
Y a-t-il un art de persuader ? \\[0pt]
Y a-t-il un autre monde ? \\[0pt]
Y a-t-il un besoin métaphysique ? \\[0pt]
Y a-t-il un bien commun ? \\[0pt]
Y a-t-il un bon usage de la rhétorique ? \\[0pt]
Y a-t-il un canon de la beauté ? \\[0pt]
Y a-t-il un désir de connaître ? \\[0pt]
Y a-t-il un devoir d'indignation ? \\[0pt]
Y a-t-il un devoir d'oubli ? \\[0pt]
Y a-t-il un droit de la guerre ? \\[0pt]
Y a-t-il un droit de résistance ? \\[0pt]
Y a-t-il un droit international ? \\[0pt]
Y a-t-il un droit naturel ? \\[0pt]
Y a-t-il une connaissance du singulier ? \\[0pt]
Y a-t-il une connaissance logique ? \\[0pt]
Y a-t-il une connaissance sensible ? \\[0pt]
Y a-t-il une esthétique du laid ? \\[0pt]
Y a-t-il une éthique de l'authenticité ? \\[0pt]
Y a-t-il une expérience de la liberté ? \\[0pt]
Y a-t-il une expérience de l'éternité ? \\[0pt]
Y a-t-il une expérience du néant ? \\[0pt]
Y a-t-il une fin de l'histoire ? \\[0pt]
Y a-t-il une fin dernière ? \\[0pt]
Y a-t-il une histoire universelle ? \\[0pt]
Y a-t-il une logique de la déraison ? \\[0pt]
Y a-t-il une logique du mythe ? \\[0pt]
Y a-t-il une loi suprême ? \\[0pt]
Y a-t-il une mathématique universelle ? \\[0pt]
Y a-t-il une nécessité de l'erreur ? \\[0pt]
Y a-t-il une ou plusieurs philosophies ? \\[0pt]
Y a-t-il une pensée sans signes ? \\[0pt]
Y a-t-il une pensée technique ? \\[0pt]
Y a-t-il une philosophie de la philosophie ? \\[0pt]
Y a-t-il une philosophie première ? \\[0pt]
Y a-t-il une psychologie animale ? \\[0pt]
Y a-t-il une raison des choses ? \\[0pt]
Y a-t-il une rationalité propre de l'intérêt ? \\[0pt]
Y a-t-il une réalité des idées ? \\[0pt]
Y a-t-il une religion civile ? \\[0pt]
Y a-t-il une science de l'être ? \\[0pt]
Y a-t-il une unité des sciences ? \\[0pt]
Y a-t-il une utilité des lieux communs ? \\[0pt]
Y a-t-il une vérité de l'œuvre d'art ? \\[0pt]
Y a-t-il une vérité des symboles ? \\[0pt]
Y a-t-il une vérité du sentiment ? \\[0pt]
Y a-t-il une vérité philosophique ? \\[0pt]
Y a-t-il une vie de l'esprit ? \\[0pt]
Y a-t-il un langage commun ? \\[0pt]
Y a-t-il un langage du corps ? \\[0pt]
Y a-t-il un mal absolu ? \\[0pt]
Y a-t-il un monde extérieur ? \\[0pt]
Y a-t-il un ordre des choses ? \\[0pt]
Y a-t-il un progrès dans l'art ? \\[0pt]
Y a-t-il un progrès moral ? \\[0pt]
Y a-t-il un savoir du bien ? \\[0pt]
Y a-t-il un savoir du corps ? \\[0pt]
Y a-t-il un savoir immédiat ? \\[0pt]
Y a-t-il un sens à penser un droit des générations futures ? \\[0pt]
Y a-t-il un sens à se dire « citoyen du monde » ? \\[0pt]
Y a-t-il un temps des choses ? \\[0pt]
Y a-t-il un temps pour l'action ? \\[0pt]
Y a-t-il un temps pour tout ? \\[0pt]
Y a-t-une science du comportement ? \\[0pt]


\subsection{Esthétique}
\label{sec:orgb28f4e7}

\noindent
À chacun ses goûts \\[0pt]
Apprend-on à percevoir la beauté ? \\[0pt]
Apprendre à voir \\[0pt]
À quoi l'art est-il bon ? \\[0pt]
À quoi reconnaît-on une œuvre d'art ? \\[0pt]
À quoi sert la critique ? \\[0pt]
Architecture et religion \\[0pt]
Architecture et utopie \\[0pt]
Art et abstraction \\[0pt]
Art et artifice \\[0pt]
Art et authenticité \\[0pt]
Art et décadence \\[0pt]
Art et divertissement \\[0pt]
Art et émotion \\[0pt]
Art et folie \\[0pt]
Art et forme \\[0pt]
Art et illusion \\[0pt]
Art et image \\[0pt]
Art et industrie \\[0pt]
Art et interdit \\[0pt]
Art et jeu \\[0pt]
Art et langage \\[0pt]
Art et marchandise \\[0pt]
Art et mélancolie \\[0pt]
Art et mémoire \\[0pt]
Art et métaphysique \\[0pt]
Art et politique \\[0pt]
Art et présence \\[0pt]
Art et propagande \\[0pt]
Art et religieux \\[0pt]
Art et religion \\[0pt]
Art et représentation \\[0pt]
Art et technique \\[0pt]
Art et tradition \\[0pt]
Art et transgression \\[0pt]
Art et vérité \\[0pt]
Artiste et artisan \\[0pt]
Artistes et ingénieurs \\[0pt]
Art populaire et art savant \\[0pt]
Arts de l'espace et arts du temps \\[0pt]
Arts du temps et arts de l'espace \\[0pt]
Avoir du goût \\[0pt]
Avons-nous besoin d'experts en matière d'art ? \\[0pt]
Avons-nous besoin d'une définition de l'art ? \\[0pt]
Beauté et vérité \\[0pt]
Beauté naturelle et beauté artistique \\[0pt]
Beauté réelle, beauté idéale \\[0pt]
Certaines œuvres d'art ont-elles plus de valeur que d'autres ? \\[0pt]
« C'est tout un art » \\[0pt]
Chanter et parler \\[0pt]
Cinéma et réalité \\[0pt]
Cinéma et vérité \\[0pt]
Comment définir le laid ? \\[0pt]
Comment devient-on artiste ? \\[0pt]
Comment évaluer l'art ? \\[0pt]
Comment juger d'une œuvre d'art ? \\[0pt]
Comment reconnaît-on une œuvre d'art ? \\[0pt]
Comment représenter la douleur ? \\[0pt]
Composition et construction \\[0pt]
Contemplation et création \\[0pt]
Contemplation et distraction \\[0pt]
Contempler \\[0pt]
Contempler une œuvre d'art \\[0pt]
Couleur et forme \\[0pt]
Création et critique \\[0pt]
Création et production \\[0pt]
Création et réception \\[0pt]
Créativité et contrainte \\[0pt]
Crise et création \\[0pt]
D'après nature \\[0pt]
« De la musique avant toute chose » \\[0pt]
De quelle transgression l'art est-il susceptible ? \\[0pt]
De quelle vérité l'art est-il capable ? \\[0pt]
De quoi fait-on l'expérience face à une œuvre ? \\[0pt]
De quoi l'art peut-il nous libérer ? \\[0pt]
De quoi l'expérience esthétique est-elle expérience ? \\[0pt]
De quoi l'expérience esthétique est-elle l'expérience ? \\[0pt]
Des goûts et des couleurs \\[0pt]
Dessiner \\[0pt]
Doit-on cesser de chercher à définir l'œuvre d'art ? \\[0pt]
Donner une représentation \\[0pt]
Écouter \\[0pt]
En quel sens une œuvre d'art est-elle un document ? \\[0pt]
En quoi l'œuvre d'art donne-t-elle à penser ? \\[0pt]
Enseigner l'art \\[0pt]
Est beau ce qui ne sert à rien \\[0pt]
Esthétique et poétique \\[0pt]
Esthétique et politique \\[0pt]
Être acteur \\[0pt]
Être inspiré \\[0pt]
Existe-t-il des émotions proprement esthétiques ? \\[0pt]
Existe-t-il une unité des arts ? \\[0pt]
Expérience esthétique et sens commun \\[0pt]
Expression et création \\[0pt]
Faire est-il nécessairement savoir faire ? \\[0pt]
Faut-il distinguer esthétique et philosophie de l'art ? \\[0pt]
Faut-il enfermer les œuvres dans les musées ? \\[0pt]
Faut-il en finir avec l'esthétique ? \\[0pt]
Faut-il faire de sa vie une œuvre d'art ? \\[0pt]
Faut-il opposer l'art à la connaissance ? \\[0pt]
Faut-il opposer produire et créer ? \\[0pt]
Faut-il restaurer les œuvres d'art ? \\[0pt]
Faut-il s'intéresser aux œuvres mineures ? \\[0pt]
Fiction et vérité \\[0pt]
Forme et rythme \\[0pt]
Hiérarchiser les arts \\[0pt]
Histoire des techniques et histoire de l'art \\[0pt]
« Il faudrait rester des années entières pour contempler une telle œuvre » \\[0pt]
Imitation et création \\[0pt]
Imiter, est-ce copier ? \\[0pt]
Improviser \\[0pt]
Interpréter une œuvre d'art \\[0pt]
Invention et création \\[0pt]
Jugement esthétique et jugement de valeur \\[0pt]
Jugement esthétique et objectivité \\[0pt]
La beauté a-t-elle une histoire ? \\[0pt]
La beauté comporte-t-elle des degrés ? \\[0pt]
La beauté de la nature \\[0pt]
La beauté des corps \\[0pt]
La beauté des ruines \\[0pt]
La beauté des villes \\[0pt]
La beauté du geste \\[0pt]
La beauté du monde \\[0pt]
La beauté est-elle dans le regard ou dans la chose vue ? \\[0pt]
La beauté est-elle partout ? \\[0pt]
La beauté est-elle sensible ? \\[0pt]
La beauté est-elle une promesse de bonheur ? \\[0pt]
La beauté et la grâce \\[0pt]
La beauté et l'ennui \\[0pt]
La beauté idéale \\[0pt]
La beauté naturelle \\[0pt]
La beauté naturelle est-elle une catégorie esthétique périmée ? \\[0pt]
La beauté n'est-elle qu'une idée ? \\[0pt]
La beauté peut-elle être cachée ? \\[0pt]
L'absence d'œuvre \\[0pt]
L'abstraction \\[0pt]
L'abstraction en art \\[0pt]
L'académisme \\[0pt]
La catharsis \\[0pt]
La censure \\[0pt]
L'achèvement de l'œuvre \\[0pt]
La classification des arts \\[0pt]
La composition \\[0pt]
La conscience de soi de l'art \\[0pt]
La contemplation \\[0pt]
La correspondance des arts \\[0pt]
La couleur \\[0pt]
La création artistique \\[0pt]
La création dans l'art \\[0pt]
La critique d'art \\[0pt]
La critique de l'art \\[0pt]
L'acteur et son rôle \\[0pt]
La culture artistique \\[0pt]
La culture est-elle nécessaire à l'appréciation d'une œuvre d'art ? \\[0pt]
La danse est-elle l'œuvre du corps ? \\[0pt]
La danse et l'espace \\[0pt]
La fiction \\[0pt]
La figuration \\[0pt]
La fin de l'art \\[0pt]
La fonction de l'art \\[0pt]
La force de l'art \\[0pt]
La forme \\[0pt]
La forme et la couleur \\[0pt]
La fragilité du beau \\[0pt]
La genèse de l'œuvre \\[0pt]
La grâce \\[0pt]
La haine des images \\[0pt]
La hiérarchie des arts \\[0pt]
La laideur \\[0pt]
La laideur est-elle une valeur esthétique ? \\[0pt]
La liberté créatrice \\[0pt]
La liberté de l'artiste \\[0pt]
La littérature engagée \\[0pt]
La littérature est-elle la mémoire de l'humanité ? \\[0pt]
La littérature et le mythe \\[0pt]
L'allégorie \\[0pt]
L'amateur \\[0pt]
L'amateur d'art \\[0pt]
La médiocrité artistique \\[0pt]
La métaphore \\[0pt]
La mise en scène \\[0pt]
La mode \\[0pt]
La modernité dans les arts \\[0pt]
La monumentalité \\[0pt]
La mort de l'art \\[0pt]
L'amour de l'art \\[0pt]
La musique a-t-elle une essence ? \\[0pt]
La musique de film \\[0pt]
La musique donne-t-elle à penser ? \\[0pt]
La musique est-elle un discours ? \\[0pt]
La musique et la danse \\[0pt]
La musique et le bruit \\[0pt]
La musique et le temps \\[0pt]
La musique peut-elle être narrative ? \\[0pt]
La naissance de l'art \\[0pt]
La nature est-elle artiste ? \\[0pt]
La nature est-elle belle ? \\[0pt]
La nature et l'artifice \\[0pt]
La nature imite-t-elle l'art ? \\[0pt]
La nature morte \\[0pt]
La nature peut-elle être belle ? \\[0pt]
La norme du beau \\[0pt]
La norme du goût \\[0pt]
La notion d'art contemporain \\[0pt]
La notion d'art poétique \\[0pt]
La notion de genre littéraire \\[0pt]
La nouveauté en art \\[0pt]
La parure \\[0pt]
La peinture apprend-elle à voir ? \\[0pt]
La peinture est-elle une poésie muette ? \\[0pt]
La peinture peut-elle être un art du temps ? \\[0pt]
La perfection artistique \\[0pt]
La perfection en art \\[0pt]
La perspective \\[0pt]
La photographie est-elle un art ? \\[0pt]
La place de l'art est-elle sur le marché de l'art ? \\[0pt]
La pluralité des arts \\[0pt]
La poésie \\[0pt]
La poésie dit-elle le réel ? \\[0pt]
La poésie est-elle comme une peinture ? \\[0pt]
L'appréciation de la nature \\[0pt]
La productivité de l'art \\[0pt]
La profondeur \\[0pt]
La puissance des images \\[0pt]
La question de l'œuvre d'art \\[0pt]
L'architecte : artiste ou ingénieur ? \\[0pt]
L'architecte : artiste ou l'ingénieur \\[0pt]
L'architecte dans la cité \\[0pt]
L'architecte et l'ingénieur \\[0pt]
L'architecture est-elle un art ? \\[0pt]
L'architecture et l'espace \\[0pt]
La réalité du beau \\[0pt]
La réception de l'œuvre d'art \\[0pt]
La religion de l'art \\[0pt]
La répétition \\[0pt]
La reproductibilité de l'œuvre d'art \\[0pt]
La reproduction des œuvres d'art \\[0pt]
La responsabilité de l'artiste \\[0pt]
La restauration des œuvres d'art \\[0pt]
La rhétorique est-elle un art ? \\[0pt]
La rime et la raison \\[0pt]
L'art à l'épreuve du goût \\[0pt]
L'art apprend-il à percevoir ? \\[0pt]
L'art a-t-il besoin d'un discours sur l'art ? \\[0pt]
L'art a-t-il des vertus thérapeutiques ? \\[0pt]
L'art a-t-il plus de valeur que la vérité ? \\[0pt]
L'art a-t-il une fin morale ? \\[0pt]
L'art a-t-il une histoire ? \\[0pt]
L'art a-t-il une responsabilité morale ? \\[0pt]
L'art a-t-il une valeur sociale ? \\[0pt]
L'art d'écrire \\[0pt]
L'art décrit-il ? \\[0pt]
L'art de masse \\[0pt]
L'art de persuader \\[0pt]
L'art des images \\[0pt]
L'art de vivre est-il un art ? \\[0pt]
L'art doit-il être critique ? \\[0pt]
L'art doit-il nous étonner ? \\[0pt]
L'art donne-t-il à voir l'invisible ? \\[0pt]
L'art dramatique \\[0pt]
L'art du comédien \\[0pt]
L'art échappe-t-il à la raison ? \\[0pt]
L'art engagé \\[0pt]
L'art, est-ce ce qui résiste à la certitude ? \\[0pt]
L'art est-il affaire d'imagination ? \\[0pt]
L'art est-il à lui-même son propre but ? \\[0pt]
L'art est-il ce qui permet de partager ses émotions ? \\[0pt]
L'art est-il destiné à embellir ? \\[0pt]
L'art est-il le miroir du monde ? \\[0pt]
L'art est-il objet de compréhension ? \\[0pt]
L'art est-il politique ? \\[0pt]
L'art est-il révolutionnaire ? \\[0pt]
L'art est-il subversif ? \\[0pt]
L'art est-il une expérience de la liberté ? \\[0pt]
L'art est-il une valeur ? \\[0pt]
L'art est-il un langage ? \\[0pt]
L'art est-il un langage universel ? \\[0pt]
L'art est-il un luxe ? \\[0pt]
L'art est-il un mode de connaissance ? \\[0pt]
L'art est-il un modèle pour la philosophie ? \\[0pt]
L'art est-il un monde ? \\[0pt]
L'art est par-delà beauté et laideur ? \\[0pt]
L'art et la manière \\[0pt]
L'art et la mort \\[0pt]
L'art et la nature \\[0pt]
L'art et l'artifice \\[0pt]
L'art et la tradition \\[0pt]
L'art et la vérité \\[0pt]
L'art et la vie \\[0pt]
L'art et le mouvement \\[0pt]
L'art et le mythe \\[0pt]
L'art et l'éphémère \\[0pt]
L'art et le rêve \\[0pt]
L'art et le sacré \\[0pt]
L'art et les arts \\[0pt]
L'art et le temps \\[0pt]
L'art et le vivant \\[0pt]
L'art et l'immoralité \\[0pt]
L'art et l'interdit \\[0pt]
L'art et morale \\[0pt]
L'art et ses institutions \\[0pt]
L'art : expérience, exercice ou habitude ? \\[0pt]
L'art fait-il penser ? \\[0pt]
L'art imite-t-il la nature ? \\[0pt]
L'art industriel \\[0pt]
L'artiste a-t-il besoin d'une idée de l'art ? \\[0pt]
L'artiste a-t-il besoin d'un public ? \\[0pt]
L'artiste a-t-il toujours raison ? \\[0pt]
L'artiste a-t-il une méthode ? \\[0pt]
L'artiste dans la cité \\[0pt]
L'artiste est-il le mieux placé pour comprendre son œuvre ? \\[0pt]
L'artiste est-il maître de son œuvre ? \\[0pt]
L'artiste et l'artisan \\[0pt]
L'artiste et son public \\[0pt]
L'artiste exprime-t-il quelque chose ? \\[0pt]
L'artiste peut-il se passer d'un maître ? \\[0pt]
L'artiste sait-il ce qu'il fait ? \\[0pt]
L'art modifie-t-il notre rapport au réel ? \\[0pt]
L'art n'est-il pas toujours politique ? \\[0pt]
L'art n'est-il pas toujours religieux ? \\[0pt]
L'art n'est-il qu'apparence ? \\[0pt]
L'art n'est-il qu'un artifice ? \\[0pt]
L'art n'est-il qu'une affaire d'esthétique ? \\[0pt]
L'art nous donne-t-il des raisons d'espérer ? \\[0pt]
L'art nous libère-t-il de l'insignifiance ? \\[0pt]
L'art nous permet-il de lutter contre l'irréversibilité ? \\[0pt]
L'art officiel \\[0pt]
L'art ou les arts \\[0pt]
L'art peut-il changer le monde ? \\[0pt]
L'art peut-il encore imiter la nature ? \\[0pt]
L'art peut-il être brut ? \\[0pt]
L'art peut-il être révolutionnaire ? \\[0pt]
L'art peut-il être sans œuvre ? \\[0pt]
L'art peut-il être utile ? \\[0pt]
L'art peut-il finir ? \\[0pt]
L'art peut-il mourir ? \\[0pt]
L'art peut-il nous rendre meilleurs ? \\[0pt]
L'art peut-il prétendre à la vérité ? \\[0pt]
L'art peut-il quelque chose contre la morale ? \\[0pt]
L'art peut-il quelque chose pour la morale ? \\[0pt]
L'art peut-il rendre le mouvement ? \\[0pt]
L'art peut-il s'affranchir des lois ? \\[0pt]
L'art peut-il s'enseigner ? \\[0pt]
L'art peut-il se passer de règles ? \\[0pt]
L'art peut-il se passer d'idéal ? \\[0pt]
L'art peut-il se passer d'œuvres ? \\[0pt]
L'art peut-il tenir lieu de métaphysique ? \\[0pt]
L'art politique \\[0pt]
L'art populaire \\[0pt]
L'art pour l'art \\[0pt]
L'art produit-il nécessairement des œuvres ? \\[0pt]
L'art s'adresse-t-il à la sensibilité ? \\[0pt]
L'art s'apparente-t-il à la philosophie ? \\[0pt]
L'art : une arithmétique sensible ? \\[0pt]
L'art vise-t-il nécessairement le beau ? \\[0pt]
La sacralisation de l'œuvre \\[0pt]
La scène \\[0pt]
La scène théâtrale \\[0pt]
La sculpture et la peinture \\[0pt]
La sculpture et le mouvement \\[0pt]
La sculpture et l'espace \\[0pt]
La signification dans l'œuvre \\[0pt]
La signification en musique \\[0pt]
La solitude de l'artiste \\[0pt]
L'aspiration esthétique \\[0pt]
La temporalité de l'œuvre d'art \\[0pt]
La tragédie \\[0pt]
La transgression des règles \\[0pt]
L'attrait du beau \\[0pt]
L'authenticité artistique \\[0pt]
L'authenticité de l'œuvre d'art \\[0pt]
L'autonomie de l'art \\[0pt]
L'autonomie de l'œuvre d'art \\[0pt]
L'autoportrait \\[0pt]
La valeur de l'art \\[0pt]
La valeur des arts \\[0pt]
La valeur du beau \\[0pt]
L'avant-garde \\[0pt]
La vérité de la fiction \\[0pt]
La vérité du roman \\[0pt]
« La vie des formes » \\[0pt]
La violence de l'art \\[0pt]
La virtuosité \\[0pt]
La vocation utopique de l'art \\[0pt]
Le baroque \\[0pt]
Le beau a-t-il une histoire ? \\[0pt]
Le beau est-il dicible ? \\[0pt]
Le beau est-il l'objet de l'esthétique ? \\[0pt]
Le beau est-il une valeur commune ? \\[0pt]
Le beau et le bien \\[0pt]
Le beau et le bon \\[0pt]
Le beau et le sublime \\[0pt]
Le beau existe-t-il indépendamment du bien ? \\[0pt]
Le beau peut-il être effrayant ? \\[0pt]
Le bon goût \\[0pt]
Le cadre \\[0pt]
Le canon \\[0pt]
Le chant et le cri \\[0pt]
Le chef-d'œuvre \\[0pt]
Le cinéma, art de la représentation ? \\[0pt]
Le cinéma est-il un art ? \\[0pt]
Le cinéma est-il un art ou une industrie ? \\[0pt]
Le cinéma est-il un art populaire ? \\[0pt]
Le clair-obscur \\[0pt]
Le comédien \\[0pt]
Le conflit esthétique \\[0pt]
Le corps dansant \\[0pt]
Le corps et la danse \\[0pt]
Le critique d'art \\[0pt]
Le culte des images \\[0pt]
Le design \\[0pt]
Le désintéressement esthétique \\[0pt]
Le désir d'originalité \\[0pt]
Le dessin \\[0pt]
Le détail \\[0pt]
Le dieu artiste \\[0pt]
L'éducation artistique \\[0pt]
L'éducation du goût \\[0pt]
L'éducation esthétique \\[0pt]
Le fantastique \\[0pt]
Le faux en art \\[0pt]
Le fond et la forme \\[0pt]
Le formalisme \\[0pt]
Le frivole \\[0pt]
Le geste \\[0pt]
Le geste créateur \\[0pt]
Le goût : certitude ou conviction ? \\[0pt]
Le goût de l'artiste \\[0pt]
Le goût du beau \\[0pt]
Le goût est-il une faculté ? \\[0pt]
Le goût est-il une question de classe ? \\[0pt]
Le goût est-il une vertu sociale ? \\[0pt]
Le goût se forme-t-il ? \\[0pt]
Le grotesque \\[0pt]
Le jugement artistique se fait-il sans concept ? \\[0pt]
Le jugement de goût \\[0pt]
Le jugement de goût est-il universel ? \\[0pt]
Le laid \\[0pt]
Le langage de l'art \\[0pt]
Le langage poétique \\[0pt]
Le libre jeu des formes \\[0pt]
Le lyrisme \\[0pt]
Le maniérisme \\[0pt]
Le marché de l'art \\[0pt]
Le matériau musical \\[0pt]
Le mauvais goût \\[0pt]
Le mécénat \\[0pt]
Le mode d'existence de l'œuvre d'art \\[0pt]
Le modèle et la copie \\[0pt]
Le monde de l'art \\[0pt]
Le monde de l'art et l'ordinateur \\[0pt]
L'émotion esthétique \\[0pt]
L'émotion esthétique peut-elle se communiquer ? \\[0pt]
L'émotion esthétique peut-elle se partager ? \\[0pt]
Le musée \\[0pt]
L'enfance de l'art \\[0pt]
L'engagement dans l'art \\[0pt]
Le noir et blanc \\[0pt]
Le nu \\[0pt]
Le patrimoine artistique \\[0pt]
Le paysage \\[0pt]
Le paysage urbain \\[0pt]
Le peintre et le sculpteur \\[0pt]
Le plaisir artistique est-il affaire de jugement ? \\[0pt]
Le plaisir d'imiter \\[0pt]
Le plaisir esthétique \\[0pt]
Le plaisir esthétique est-il un plaisir ? \\[0pt]
Le plaisir esthétique peut-il se communiquer ? \\[0pt]
Le plaisir esthétique suppose-t-il une culture ? \\[0pt]
Le poète réinvente-t-il la langue ? \\[0pt]
Le point de vue de l'auteur \\[0pt]
Le portrait \\[0pt]
Le pouvoir des images \\[0pt]
Le primitivisme en art \\[0pt]
Le propre de la musique \\[0pt]
Le public \\[0pt]
Lequel, de l'art ou du réel, est-il une imitation de l'autre ? \\[0pt]
Le réalisme \\[0pt]
Le réel peut-il être beau ? \\[0pt]
Le rythme \\[0pt]
Le rythme en peinture \\[0pt]
Les artistes sont-ils sérieux ? \\[0pt]
Les arts appliqués \\[0pt]
Les arts communiquent-ils entre eux ? \\[0pt]
Les arts industriels \\[0pt]
Les arts mineurs \\[0pt]
Les arts nobles \\[0pt]
Les arts ont-ils besoin de théorie ? \\[0pt]
Les arts populaires \\[0pt]
Les arts sont-ils des jeux ? \\[0pt]
Les arts vivants \\[0pt]
Les beaux-arts sont-ils compatibles entre eux ? \\[0pt]
Les biens culturels \\[0pt]
Les couleurs \\[0pt]
Les degrés de la beauté \\[0pt]
Les degrés du beau \\[0pt]
Le sentiment esthétique \\[0pt]
Les fins de l'art \\[0pt]
Les fonctions de l'image \\[0pt]
Les frontières de l'art \\[0pt]
Les genres esthétiques \\[0pt]
Les institutions artistiques \\[0pt]
Les institutions de l'art \\[0pt]
Les intentions de l'artiste \\[0pt]
Les jugements de goût sont-ils susceptibles de vérité ? \\[0pt]
Les langages de l'art \\[0pt]
Les lois de l'art \\[0pt]
Les métamorphoses du goût \\[0pt]
Les moyens et les fins en art \\[0pt]
Les muses \\[0pt]
Les normes esthétiques \\[0pt]
Les nouvelles technologies transforment-elles l'idée de l'art ? \\[0pt]
Les œuvres d'art ont-elles besoin d'un commentaire ? \\[0pt]
L'espace du tableau \\[0pt]
Le spectateur \\[0pt]
Les poètes et la cité \\[0pt]
Les qualités esthétiques \\[0pt]
L'esquisse \\[0pt]
Les règles de l'art \\[0pt]
Les reproductions \\[0pt]
Les révolutions techniques suscitent-elles des révolutions dans l'art ? \\[0pt]
Les styles \\[0pt]
Les techniques artistiques \\[0pt]
L'esthète \\[0pt]
L'esthète et l'artiste \\[0pt]
L'esthétique des ruines \\[0pt]
L'esthétique est-elle une métaphysique de l'art ? \\[0pt]
L'esthétisme \\[0pt]
Le style \\[0pt]
Le sublime \\[0pt]
Les usages de l'art \\[0pt]
Le symbole \\[0pt]
Le symbolisme \\[0pt]
Le système des arts \\[0pt]
Le système des beaux-arts \\[0pt]
Le tableau \\[0pt]
Le tableau vivant \\[0pt]
Le talent et le génie \\[0pt]
Le talent s'enseigne-t-il ? \\[0pt]
Le temps de l'art \\[0pt]
Le théâtre et les mœurs \\[0pt]
L'éthique du spectateur \\[0pt]
Le tragique \\[0pt]
Le travail artistique \\[0pt]
Le travail artistique doit-il demeurer caché ? \\[0pt]
Le trompe-l'œil \\[0pt]
L'évaluation des œuvres d'art \\[0pt]
L'exécution d'une œuvre d'art est-elle toujours une œuvre d'art ? \\[0pt]
L'expérience artistique \\[0pt]
L'expérience esthétique \\[0pt]
L'expert et l'amateur \\[0pt]
L'explication de l'œuvre d'art \\[0pt]
L'exposition de l'œuvre d'art \\[0pt]
L'expression \\[0pt]
L'expression artistique \\[0pt]
L'expression du mouvement \\[0pt]
L'expressivité musicale \\[0pt]
L'harmonie \\[0pt]
L'hétéronomie de l'art \\[0pt]
L'histoire de l'art \\[0pt]
L'histoire de l'art est-elle celle des styles ? \\[0pt]
L'histoire de l'art est-elle finie ? \\[0pt]
L'histoire des arts est-elle liée à l'histoire des techniques ? \\[0pt]
L'hybridation des arts \\[0pt]
L'idéal \\[0pt]
L'idéal dans l'art \\[0pt]
L'idéal de l'art \\[0pt]
L'idée esthétique \\[0pt]
L'illustration \\[0pt]
L'image \\[0pt]
L'imaginaire \\[0pt]
L'imaginaire et le réel \\[0pt]
L'imagination \\[0pt]
L'imagination dans l'art \\[0pt]
L'imagination esthétique \\[0pt]
L'imitation \\[0pt]
L'immortalité des œuvres d'art \\[0pt]
L'improvisation \\[0pt]
L'improvisation dans l'art \\[0pt]
L'impuissance de l'art \\[0pt]
L'inachevé \\[0pt]
L'inconscient de l'art \\[0pt]
L'industrie culturelle \\[0pt]
L'industrie du beau \\[0pt]
L'inesthétique \\[0pt]
L'informe \\[0pt]
L'informe et le difforme \\[0pt]
L'inspiration \\[0pt]
L'intériorité de l'œuvre \\[0pt]
L'interprétation \\[0pt]
L'interprétation des œuvres \\[0pt]
L'irreprésentable \\[0pt]
Littérature et réalité \\[0pt]
L'objectivité de l'art \\[0pt]
L'objectivité de l'œuvre d'art \\[0pt]
L'objet de la littérature \\[0pt]
L'objet de l'art \\[0pt]
L'objet du jugement esthétique \\[0pt]
L'obscène \\[0pt]
L'œil du photographe \\[0pt]
L'œuvre anonyme \\[0pt]
L'œuvre d'art est-elle anhistorique ? \\[0pt]
L'œuvre d'art est-elle l'expression d'une idée ? \\[0pt]
L'œuvre d'art est-elle toujours destinée à un public ? \\[0pt]
L'œuvre d'art est-elle une belle apparence ? \\[0pt]
L'œuvre d'art et sa reproduction \\[0pt]
L'œuvre d'art et son auteur \\[0pt]
L'œuvre d'art nous apprend-elle quelque chose ? \\[0pt]
L'œuvre d'art représente-t-elle quelque chose ? \\[0pt]
L'œuvre d'art totale \\[0pt]
L'œuvre d'art traduit-elle une vision du monde ? \\[0pt]
L'œuvre de fiction \\[0pt]
L'œuvre et le produit \\[0pt]
L'œuvre inachevée \\[0pt]
L'œuvre : l'universel et le particulier \\[0pt]
L'œuvre totale \\[0pt]
L'opéra est-il un art complet ? \\[0pt]
L'ordre et la beauté \\[0pt]
L'original et la copie \\[0pt]
L'originalité \\[0pt]
L'originalité en art \\[0pt]
L'origine de l'art \\[0pt]
L'ornement \\[0pt]
Lumière et couleur \\[0pt]
L'unité de l'œuvre d'art \\[0pt]
L'utilité de l'art \\[0pt]
Mass-media et création artistique \\[0pt]
Musique et architecture \\[0pt]
Musique et bruit \\[0pt]
Musique et émotion \\[0pt]
Musique et poésie \\[0pt]
Musique et rhétorique \\[0pt]
Œuvre et événement \\[0pt]
Œuvrer \\[0pt]
Par-delà beauté et laideur \\[0pt]
Peindre \\[0pt]
Peindre, est-ce traduire ? \\[0pt]
Peindre la présence \\[0pt]
Peindre un paysage, est-ce représenter la nature ? \\[0pt]
Peinture et abstraction \\[0pt]
Peinture et photographie \\[0pt]
Peinture et réalité \\[0pt]
Peinture et vérité \\[0pt]
Peut-on dire de l'art qu'il donne un monde en partage ? \\[0pt]
Peut-on établir une hiérarchie des arts ? \\[0pt]
Peut-on être insensible à l'art ? \\[0pt]
Peut-on expliquer une œuvre d'art ? \\[0pt]
Peut-on faire de l'art avec tout ? \\[0pt]
Peut-on hiérarchiser les œuvres d'art ? \\[0pt]
Peut-on juger des œuvres d'art sans recourir à l'idée de beauté ? \\[0pt]
Peut-on parler d'art primitif ? \\[0pt]
Peut-on parler de progrès en art ? \\[0pt]
Peut-on parler des œuvres d'art ? \\[0pt]
Peut-on parler de vérité théâtrale ? \\[0pt]
Peut-on parler d'une science de l'art ? \\[0pt]
Peut-on parler d'un savoir poétique ? \\[0pt]
Peut-on partager ses goûts ? \\[0pt]
Peut-on penser l'art comme un langage ? \\[0pt]
Peut-on penser un art sans œuvres ? \\[0pt]
Peut-on représenter l'invisible ? \\[0pt]
Peut-on réunir des arts différents dans une même œuvre ? \\[0pt]
Peut-on se peindre soi-même ? \\[0pt]
Peut-on vivre sans art ? \\[0pt]
Poésie et vérité \\[0pt]
Point de vue du créateur et point de vue du spectateur \\[0pt]
Pour apprécier une œuvre, faut-il être cultivé ? \\[0pt]
Pourquoi conserver des œuvres d'art ? \\[0pt]
Pourquoi conserver les œuvres d'art ? \\[0pt]
Pourquoi des fictions ? \\[0pt]
Pourquoi des musées ? \\[0pt]
Pourquoi des œuvres d'art ? \\[0pt]
Pourquoi il n'y a pas de société sans art ? \\[0pt]
Pourquoi la musique intéresse-t-elle le philosophe ? \\[0pt]
Pourquoi l'art intéresse-t-il les philosophes ? \\[0pt]
Pourquoi les œuvres d'art résistent-elles au temps ? \\[0pt]
Pourquoi s'inspirer de l'art antique ? \\[0pt]
Pourrait-on vivre sans art ? \\[0pt]
Pourrions-nous nous passer des musées ? \\[0pt]
Production et création \\[0pt]
Production et réception de l'œuvre d'art \\[0pt]
Propriétés artistiques, propriétés esthétiques \\[0pt]
Qu'aime-t-on quand on aime une œuvre d'art ? \\[0pt]
Quand l'art est-il abstrait ? \\[0pt]
Quand la technique devient-elle art ? \\[0pt]
Quand y a-t-il paysage ? \\[0pt]
Qu'appelle-t-on chef-d'œuvre ? \\[0pt]
Qu'apporte la photographie aux arts ? \\[0pt]
Que crée l'artiste ? \\[0pt]
Que fait aux œuvres d'art leur reproductibilité ? \\[0pt]
Que fait l'art à nos vies ? \\[0pt]
Que fait le spectateur ? \\[0pt]
Que gagne l'art à devenir abstrait ? \\[0pt]
Que gagne l'art à se réfléchir ? \\[0pt]
Quel est le pouvoir de la beauté ? \\[0pt]
Quel est le pouvoir de l'art ? \\[0pt]
Quel est le pouvoir des métaphores ? \\[0pt]
Quel est l'objet de l'esthétique ? \\[0pt]
Quelle est la matière de l'œuvre d'art ? \\[0pt]
Quelles règles la technique dicte-t-elle à l'art ? \\[0pt]
Quel réel pour l'art ? \\[0pt]
Que montre un tableau ? \\[0pt]
Que nous apporte l'art ? \\[0pt]
Que nous apprend le théâtre ? \\[0pt]
Que nous apprend l'expérience esthétique ? \\[0pt]
Que nous apprend l'histoire de l'art ? \\[0pt]
Que peint le peintre ? \\[0pt]
Que peut-on apprendre des émotions esthétiques ? \\[0pt]
Que peuvent les images ? \\[0pt]
Que rend visible l'art ? \\[0pt]
Qu'est-ce qu'apprécier une œuvre d'art ? \\[0pt]
Qu'est-ce qu'avoir du goût ? \\[0pt]
Qu'est-ce que comprendre une œuvre d'art ? \\[0pt]
Qu'est-ce qu'éduquer le sens esthétique ? \\[0pt]
Qu'est-ce que l'art contemporain ? \\[0pt]
Qu'est-ce que le génie ? \\[0pt]
Qu'est-ce que le style ? \\[0pt]
Qu'est-ce qui est artificiel ? \\[0pt]
Qu'est-ce qui est beau ? \\[0pt]
Qu'est-ce qui est spectaculaire ? \\[0pt]
Qu'est-ce qui fait la valeur de l'œuvre d'art ? \\[0pt]
Qu'est-ce qui fait la valeur des objets d'art ? \\[0pt]
Qu'est-ce qu'interpréter une œuvre d'art ? \\[0pt]
Qu'est-ce qu'un acteur ? \\[0pt]
Qu'est-ce qu'un artefact ? \\[0pt]
Qu'est-ce qu'un artiste ? \\[0pt]
Qu'est-ce qu'un art moral ? \\[0pt]
Qu'est-ce qu'un beau paysage ? \\[0pt]
Qu'est-ce qu'un beau travail ? \\[0pt]
Qu'est-ce qu'un « champ artistique » ? \\[0pt]
Qu'est-ce qu'un chef-d'œuvre ? \\[0pt]
Qu'est-ce qu'un classique ? \\[0pt]
Qu'est-ce qu'une expérience esthétique ? \\[0pt]
Qu'est-ce qu'une exposition ? \\[0pt]
Qu'est-ce qu'une idée esthétique ? \\[0pt]
Qu'est-ce qu'une œuvre classique ? \\[0pt]
Qu'est-ce qu'une œuvre d'art ? \\[0pt]
Qu'est-ce qu'une œuvre d'art authentique ? \\[0pt]
Qu'est-ce qu'une œuvre d'art réussie ? \\[0pt]
Qu'est-ce qu'une œuvre « géniale » ? \\[0pt]
Qu'est-ce qu'une œuvre ratée ? \\[0pt]
Qu'est-ce qu'une peinture abstraite ? \\[0pt]
Qu'est-ce qu'une « performance » ? \\[0pt]
Qu'est-ce qu'une révolution artistique ? \\[0pt]
Qu'est-ce qu'une valeur esthétique ? \\[0pt]
Qu'est-ce qu'un faux ? \\[0pt]
Qu'est-ce qu'un film ? \\[0pt]
Qu'est-ce qu'un geste artistique ? \\[0pt]
Qu'est-ce qu'un modèle ? \\[0pt]
Qu'est-ce qu'un objet d'art ? \\[0pt]
Qu'est-ce qu'un paysage ? \\[0pt]
Qu'est-ce qu'un produit culturel ? \\[0pt]
Qu'est-ce qu'un spectacle ? \\[0pt]
Qu'est-ce qu'un spectateur ? \\[0pt]
Qu'est-ce qu'un style ? \\[0pt]
Qu'est-ce qu'un tableau ? \\[0pt]
Que trouve-t-on dans ce que l'on trouve beau ? \\[0pt]
Que voyons-nous sur un tableau ? \\[0pt]
Qu'expriment les œuvres d'art ? \\[0pt]
Qu'exprime une œuvre d'art ? \\[0pt]
Qui donne la norme du goût ? \\[0pt]
Qu'y a-t-il à comprendre dans une œuvre d'art ? \\[0pt]
Rebuts et objets quelconques : une matière pour l'art ? \\[0pt]
Regarder \\[0pt]
Regarder une œuvre d'art \\[0pt]
Rendre visible l'invisible \\[0pt]
Reproduire, copier, imiter \\[0pt]
Ressent-on ou apprécie-t-on l'art ? \\[0pt]
Roman et vérité \\[0pt]
« Sans titre » \\[0pt]
Savoir faire \\[0pt]
Sculpter \\[0pt]
Sens et sensibilité \\[0pt]
Se peindre \\[0pt]
Sommes-nous tous artistes ? \\[0pt]
Technique et esthétique \\[0pt]
Temps et musique \\[0pt]
Thème et variations \\[0pt]
Tout art est-il abstrait ? \\[0pt]
Tout art est-il décoratif ? \\[0pt]
Tout art est-il poésie ? \\[0pt]
Tout art est-il symbolique ? \\[0pt]
Tout peut-il être objet de jugement esthétique ? \\[0pt]
Un art sans sublimation est-il possible ? \\[0pt]
Une bonne cité peut-elle se passer du beau ? \\[0pt]
Une d'œuvre peut-elle être achevée ? \\[0pt]
Une œuvre d'art doit-elle avoir un sens ? \\[0pt]
Une œuvre d'art est-elle immortelle ? \\[0pt]
Une œuvre d'art est-elle une marchandise ? \\[0pt]
Une œuvre d'art peut-elle être exemplaire ? \\[0pt]
Une œuvre d'art peut-elle être laide ? \\[0pt]
Une œuvre d'art s'explique-t-elle à partir de ses influences ? \\[0pt]
Une œuvre est-elle toujours de son temps ? \\[0pt]
Un jugement de goût est-il culturel ? \\[0pt]
Un objet technique peut-il être beau ? \\[0pt]
Un tableau peut-il être une dénonciation ? \\[0pt]
Y a-t-il de l'intelligible dans l'art ? \\[0pt]
Y a-t-il des arts du corps ? \\[0pt]
Y a-t-il des arts majeurs ? \\[0pt]
Y a-t-il des arts mineurs ? \\[0pt]
Y a-t-il des critères du beau ? \\[0pt]
Y a-t-il des excès en art ? \\[0pt]
Y a-t-il des progrès en art ? \\[0pt]
Y a-t-il des révolutions en art ? \\[0pt]
Y a-t-il un art populaire ? \\[0pt]
Y a-t-il un beau idéal ? \\[0pt]
Y a-t-il un différend entre poésie et philosophie ? \\[0pt]
Y a-t-il une beauté naturelle ? \\[0pt]
Y a-t-il une beauté propre à la photographie ? \\[0pt]
Y a-t-il une correspondance des arts ? \\[0pt]
Y a-t-il une esthétique industrielle ? \\[0pt]
Y a-t-il une histoire de l'art ? \\[0pt]
Y a-t-il une logique de l'art ? \\[0pt]
Y a-t-il une perception esthétique ? \\[0pt]
Y a-t-il une sensibilité esthétique ? \\[0pt]
Y a-t-il une singularité de l'histoire de l'art ? \\[0pt]
Y a-t-il une vérité dans les arts ? \\[0pt]
Y a-t-il une vertu de l'imitation ? \\[0pt]
Y a-t-il un progrès dans l'art ? \\[0pt]
Y a-t-il un progrès en art ? \\[0pt]
Y a-t-il un sentiment naturel du beau ? \\[0pt]


\subsection{Logique et épistémologie}
\label{sec:org364e7ad}

\noindent
Analyse et synthèse \\[0pt]
À quelles conditions une démarche est-elle scientifique ? \\[0pt]
À quelles conditions une explication est-elle scientifique ? \\[0pt]
À quelles conditions une hypothèse est-elle scientifique ? \\[0pt]
À quelles conditions un énoncé est-il doué de sens ? \\[0pt]
À quoi la logique peut-elle servir dans les sciences ? \\[0pt]
À quoi reconnaît-on la vérité ? \\[0pt]
À quoi reconnaît-on qu'une théorie est scientifique ? \\[0pt]
À quoi sert la logique ? \\[0pt]
À quoi servent les sciences ? \\[0pt]
À quoi tient la vérité d'une interprétation ? \\[0pt]
Calculer et penser \\[0pt]
Catégories logiques et catégories linguistiques \\[0pt]
Ce qui est faux est-il dénué de sens ? \\[0pt]
Classer \\[0pt]
Comment justifier l'autonomie des sciences de la vie ? \\[0pt]
Comment peut-on choisir entre différentes hypothèses ? \\[0pt]
Concevoir et juger \\[0pt]
Connaissance commune et connaissance scientifique \\[0pt]
Connaissance, croyance, conjecture \\[0pt]
Connaissance du futur et connaissance du passé \\[0pt]
Connaissance et croyance \\[0pt]
Connaître, est-ce connaître par les causes ? \\[0pt]
Connaître et comprendre \\[0pt]
Contradiction et opposition \\[0pt]
Convention et observation \\[0pt]
Croire et savoir \\[0pt]
Croyance et probabilité \\[0pt]
Découverte et invention \\[0pt]
Découverte et invention dans les sciences \\[0pt]
Découvrir \\[0pt]
Décrire \\[0pt]
Déduction et expérience \\[0pt]
Définir la vérité, est-ce la connaître ? \\[0pt]
Définition et démonstration \\[0pt]
Définition nominale et définition réelle \\[0pt]
Définitions, axiomes, postulats \\[0pt]
Démonstration et argumentation \\[0pt]
Démonstration et déduction \\[0pt]
De quelle certitude la science est-elle capable ? \\[0pt]
De quoi la logique est-elle la science ? \\[0pt]
Des événements aléatoires peuvent-ils obéir à des lois ? \\[0pt]
D'où vient la certitude dans les sciences ? \\[0pt]
Épistémologie générale et épistémologie des sciences particulières \\[0pt]
Erreur et illusion \\[0pt]
Espace mathématique et espace physique \\[0pt]
Est-ce par son objet ou par ses méthodes qu'une science peut se définir ? \\[0pt]
Est-il vrai qu'en science, « rien n'est donné, tout est construit » ? \\[0pt]
Évidence et certitude \\[0pt]
Expérience et approximation \\[0pt]
Expérience et expérimentation \\[0pt]
Explication et prévision \\[0pt]
Expliquer \\[0pt]
Expliquer et comprendre \\[0pt]
Expliquer et interpréter \\[0pt]
Extension et compréhension \\[0pt]
Fonction et prédicat \\[0pt]
Forger des hypothèses \\[0pt]
Formaliser et axiomatiser \\[0pt]
Identité et indiscernabilité \\[0pt]
Interpréter et expliquer \\[0pt]
Intuition et concept \\[0pt]
Intuition et déduction \\[0pt]
Jugement analytique et jugement synthétique \\[0pt]
Juger et raisonner \\[0pt]
Justifier et prouver \\[0pt]
L'abstraction \\[0pt]
L'abstrait est-il en dehors de l'espace et du temps ? \\[0pt]
L'abstrait et le concret \\[0pt]
La causalité \\[0pt]
La causalité en histoire \\[0pt]
La causalité suppose-t-elle des lois ? \\[0pt]
La certitude \\[0pt]
La classification des sciences \\[0pt]
La cohérence est-elle un critère de la vérité ? \\[0pt]
La communauté scientifique \\[0pt]
La connaissance adéquate \\[0pt]
La connaissance commune est-elle le point de départ de la science ? \\[0pt]
La connaissance de la vie \\[0pt]
La connaissance des causes \\[0pt]
La connaissance des principes \\[0pt]
La connaissance du futur \\[0pt]
La connaissance du singulier \\[0pt]
La connaissance du vivant \\[0pt]
La connaissance est-elle une croyance justifiée ? \\[0pt]
La connaissance objective \\[0pt]
La connaissance scientifique abolit-elle toute croyance ? \\[0pt]
La connaissance scientifique n'est-elle qu'une croyance argumentée ? \\[0pt]
La contingence des lois de la nature \\[0pt]
La contradiction \\[0pt]
La culture scientifique \\[0pt]
La déduction \\[0pt]
La définition \\[0pt]
La démonstration \\[0pt]
La dialectique \\[0pt]
La grammaire et la logique \\[0pt]
La hiérarchie des énoncés scientifiques \\[0pt]
L'aléatoire \\[0pt]
La liberté de la science \\[0pt]
La limite \\[0pt]
La logique a-t-elle une histoire ? \\[0pt]
La logique a-t-elle un intérêt philosophique ? \\[0pt]
La logique : découverte ou invention ? \\[0pt]
La logique décrit-elle le monde ? \\[0pt]
La logique est-elle indépendante de la psychologie ? \\[0pt]
La logique est-elle un art de penser ? \\[0pt]
La logique est-elle une discipline normative ? \\[0pt]
La logique est-elle une forme de calcul ? \\[0pt]
La logique est-elle une science de la vérité ? \\[0pt]
La logique est-elle utile à la métaphysique ? \\[0pt]
La logique et le réel \\[0pt]
La logique nous apprend-elle quelque chose sur le langage ordinaire ? \\[0pt]
« La logique » ou bien « les logiques » ? \\[0pt]
La logique peut-elle se passer de la métaphysique ? \\[0pt]
La maîtrise de la nature \\[0pt]
La mathématique est-elle une ontologie ? \\[0pt]
La matière \\[0pt]
La mesure \\[0pt]
La mesure du temps \\[0pt]
La méthode \\[0pt]
La méthode de la science \\[0pt]
La modalité \\[0pt]
La naissance de la science \\[0pt]
L'analogie \\[0pt]
La nature est-elle écrite en langage mathématique ? \\[0pt]
La nature et le monde \\[0pt]
La nature parle-t-elle le langage des mathématiques ? \\[0pt]
La nécessité historique \\[0pt]
La négation \\[0pt]
Langage ordinaire et langage de la science \\[0pt]
La notion de possible \\[0pt]
La notion d'évolution \\[0pt]
La notion physique de force \\[0pt]
La pensée formelle peut-elle avoir un contenu ? \\[0pt]
La pertinence \\[0pt]
La place du hasard dans la science \\[0pt]
La place du sujet dans la science \\[0pt]
La pluralité des sciences de la nature \\[0pt]
La politique scientifique \\[0pt]
La possibilité logique \\[0pt]
L'approximation \\[0pt]
La pratique des sciences met-elle à l'abri des préjugés ? \\[0pt]
La preuve \\[0pt]
L'\emph{a priori} \\[0pt]
La probabilité \\[0pt]
La proposition \\[0pt]
La psychologie est-elle une science ? \\[0pt]
La réalité a-t-elle une forme logique ? \\[0pt]
La réalité décrite par la science s'oppose-t-elle à la démonstration ? \\[0pt]
La recherche de la vérité \\[0pt]
La recherche scientifique est-elle désintéressée ? \\[0pt]
La réfutation \\[0pt]
La relation de cause à effet \\[0pt]
La relation de nécessité \\[0pt]
La science admet-elle des degrés de croyance ? \\[0pt]
La science a-t-elle besoin du principe de causalité ? \\[0pt]
La science a-t-elle le monopole de la vérité ? \\[0pt]
La science a-t-elle une histoire ? \\[0pt]
La science commence-t-elle avec la perception ? \\[0pt]
La science découvre-t-elle ou construit-elle son objet ? \\[0pt]
La science de l'individuel \\[0pt]
La science dévoile-t-elle le réel ? \\[0pt]
La science doit-elle se fonder sur une idée de la nature ? \\[0pt]
La science doit-elle se passer de l'idée de finalité ? \\[0pt]
La science est-elle indépendante de toute métaphysique ? \\[0pt]
La science est-elle une langue bien faite ? \\[0pt]
La science et les sciences \\[0pt]
La science et l'irrationnel \\[0pt]
La science nous indique-t-elle ce que nous devons faire ? \\[0pt]
La science peut-elle lutter contre les préjugés ? \\[0pt]
La science peut-elle se passer de métaphysique ? \\[0pt]
La science peut-elle se passer d'hypothèses ? \\[0pt]
La science peut-elle se passer d'institutions ? \\[0pt]
La science porte-elle au scepticisme ? \\[0pt]
La science procède-t-elle par rectification ? \\[0pt]
La somme et le tout \\[0pt]
La technique n'est-elle qu'une application de la science ? \\[0pt]
La théorie et l'expérience \\[0pt]
L'autonomie du théorique \\[0pt]
L'autorité de la science \\[0pt]
La valeur de la science \\[0pt]
La valeur d'une théorie scientifique se mesure-t-elle à son efficacité ? \\[0pt]
La validité \\[0pt]
La vérité admet-elle des degrés ? \\[0pt]
La vérité du déterminisme \\[0pt]
La vérité d'une théorie dépend-elle de sa correspondance avec les faits ? \\[0pt]
Le calcul \\[0pt]
Le cerveau et la pensée \\[0pt]
Le concept \\[0pt]
Le concept de nature est-il un concept scientifique ? \\[0pt]
Le contingent \\[0pt]
Le continu \\[0pt]
Le déterminisme \\[0pt]
Le doute dans les sciences \\[0pt]
Le fait scientifique \\[0pt]
Le faux et l'absurde \\[0pt]
Le fondement de l'induction \\[0pt]
Le formalisme \\[0pt]
Le genre et l'espèce \\[0pt]
Le hasard existe-t-il ? \\[0pt]
Le hasard n'est-il que la mesure de notre ignorance ? \\[0pt]
Le jugement \\[0pt]
Le langage des sciences \\[0pt]
Le mécanisme et la mécanique \\[0pt]
Le mouvement \\[0pt]
Le nécessaire et le contingent \\[0pt]
Le nombre et la mesure \\[0pt]
Le non-sens \\[0pt]
Le normal et le pathologique \\[0pt]
Le paradigme \\[0pt]
Le partage des connaissances \\[0pt]
L'épistémologie est-elle une logique de la science ? \\[0pt]
Le possible et le probable \\[0pt]
Le pouvoir de la science \\[0pt]
Le principe de contradiction \\[0pt]
Le principe d'identité \\[0pt]
Le progrès des sciences \\[0pt]
Le progrès des sciences infirme-t-il les résultats anciens ? \\[0pt]
Le progrès en logique \\[0pt]
Le progrès scientifique fait-il disparaître la superstition ? \\[0pt]
Le raisonnement par l'absurde \\[0pt]
Le raisonnement scientifique \\[0pt]
Le raisonnement suit-il des règles ? \\[0pt]
Le réalisme de la science \\[0pt]
Le rôle de la théorie dans l'expérience scientifique \\[0pt]
L'erreur peut-elle jouer un rôle dans la connaissance scientifique ? \\[0pt]
L'erreur scientifique \\[0pt]
Les causes et les lois \\[0pt]
Les changements scientifiques et la réalité \\[0pt]
Les connaissances scientifiques peuvent-elles être à la fois vraies et provisoires ? \\[0pt]
Les connaissances scientifiques peuvent-elles être vulgarisées ? \\[0pt]
Les conquêtes de la science \\[0pt]
Les ensembles \\[0pt]
Les fausses sciences \\[0pt]
Les genres naturels \\[0pt]
Les jugements analytiques \\[0pt]
Les limites de la connaissance scientifique \\[0pt]
Les lois de la nature sont-elles de simples régularités ? \\[0pt]
Les lois de la nature sont elles nécessaires ? \\[0pt]
Les lois de l'histoire \\[0pt]
Les lois scientifiques sont-elles des lois de la nature ? \\[0pt]
Les mathématiques du mouvement \\[0pt]
Les mathématiques et la pensée de l'infini \\[0pt]
Les mathématiques sont-elles réductibles à la logique ? \\[0pt]
Les mathématiques sont-elles un langage ? \\[0pt]
Les modalités \\[0pt]
Les modèles \\[0pt]
Les mondes possibles \\[0pt]
Les objets scientifiques \\[0pt]
Les principes de la démonstration \\[0pt]
Les principes d'une science sont-ils des conventions ? \\[0pt]
L'esprit est-il objet de science ? \\[0pt]
L'esprit scientifique \\[0pt]
Les relations \\[0pt]
Les révolutions scientifiques \\[0pt]
Les sciences décrivent-elles le réel ? \\[0pt]
Les sciences de la vie et de la Terre \\[0pt]
Les sciences de la vie visent-elles un objet irréductible à la matière ? \\[0pt]
Les sciences de l'esprit \\[0pt]
Les sciences doivent-elle prétendre à l'unification ? \\[0pt]
Les sciences et le vivant \\[0pt]
Les sciences exactes \\[0pt]
Les sciences forment-elle un système ? \\[0pt]
Les sciences historiques \\[0pt]
Les sciences humaines peuvent-elles adopter les méthodes des sciences de la nature ? \\[0pt]
Les sciences naturelles \\[0pt]
Les sciences sociales \\[0pt]
Les sciences sociales sont-elles nécessairement inexactes ? \\[0pt]
Le statut de l'axiome \\[0pt]
Le statut des hypothèses dans la démarche scientifique \\[0pt]
Les théories scientifiques sont-elles vraies ? \\[0pt]
Les vérités scientifiques sont-elles relatives ? \\[0pt]
Le syllogisme \\[0pt]
Le symbolisme mathématique \\[0pt]
Le temps se laisse-t-il décrire logiquement ? \\[0pt]
Le tiers exclu \\[0pt]
L'étonnement \\[0pt]
L'évidence \\[0pt]
Le vivant comme problème pour la philosophie des sciences \\[0pt]
Le vrai est-il à lui-même sa propre marque ? \\[0pt]
Le vrai se réduit-il à l'utile ? \\[0pt]
L'expérience \\[0pt]
L'expérience cruciale \\[0pt]
L'expérience sensible est-elle la seule source légitime de connaissance ? \\[0pt]
L'expérimentation \\[0pt]
L'expérimentation en psychologie \\[0pt]
L'explication scientifique \\[0pt]
L'histoire des sciences est-elle une histoire ? \\[0pt]
L'homme est-il objet de science ? \\[0pt]
L'hypothèse \\[0pt]
L'idéal de vérité \\[0pt]
L'idée de connaissance approchée \\[0pt]
L'idée de continuité \\[0pt]
L'idée de logique transcendantale \\[0pt]
L'idée de logique universelle \\[0pt]
L'idée de loi logique \\[0pt]
L'idée de loi naturelle \\[0pt]
L'idée de mathesis universalis \\[0pt]
L'idée de norme \\[0pt]
L'idée de science expérimentale \\[0pt]
L'idée de « sciences exactes » \\[0pt]
L'identité \\[0pt]
L'imagination dans les sciences \\[0pt]
L'indifférence \\[0pt]
L'induction \\[0pt]
L'induction et la déduction \\[0pt]
L'inexactitude et le savoir scientifique \\[0pt]
L'inférence \\[0pt]
L'institution scientifique \\[0pt]
L'instrument scientifique \\[0pt]
L'intuition \\[0pt]
L'intuition a-t-elle une place en logique ? \\[0pt]
L'intuition en mathématiques \\[0pt]
L'objectivité \\[0pt]
L'objectivité historique \\[0pt]
L'obstacle épistémologique \\[0pt]
Logique et dialectique \\[0pt]
Logique et existence \\[0pt]
Logique et logiques \\[0pt]
Logique et mathématique \\[0pt]
Logique et mathématiques \\[0pt]
Logique et métaphysique \\[0pt]
Logique et méthode \\[0pt]
Logique et ontologie \\[0pt]
Logique et psychologie \\[0pt]
Logique générale et logique transcendantale \\[0pt]
Lois et règles en logique \\[0pt]
L'ordre du monde \\[0pt]
L'ordre et la mesure \\[0pt]
L'unité de la science \\[0pt]
L'universel et le particulier \\[0pt]
L'universel et le singulier \\[0pt]
Machine et organisme \\[0pt]
Mathématiques et réalité \\[0pt]
Mathématiques pures et mathématiques appliquées \\[0pt]
Mécanisme et finalité \\[0pt]
Mesurer \\[0pt]
Montrer et démontrer \\[0pt]
Notre connaissance du réel se limite-t-elle au savoir scientifique ? \\[0pt]
N'y a-t-il de rationalité que scientifique ? \\[0pt]
N'y a-t-il de science qu'autant qu'il s'y trouve de mathématique ? \\[0pt]
Observation et expérimentation \\[0pt]
Observer \\[0pt]
Organisme et milieu \\[0pt]
Où sont les relations ? \\[0pt]
Penser, est-ce calculer ? \\[0pt]
Peut-il y avoir science sans intuition du vrai ? \\[0pt]
Peut-on changer de logique ? \\[0pt]
Peut-on définir la vérité ? \\[0pt]
Peut-on définir la vie ? \\[0pt]
Peut-on dire de la connaissance scientifique qu'elle procède par approximation ? \\[0pt]
Peut-on dire d'une théorie scientifique qu'elle n'est jamais plus vraie qu'une autre mais seulement plus commode ? \\[0pt]
Peut-on dire que la science ne nous fait pas connaître les choses mais les rapports entre les choses ? \\[0pt]
Peut-on dire qu'est vrai ce qui correspond aux faits ? \\[0pt]
Peut-on dire qu'une théorie physique en contredit une autre ? \\[0pt]
Peut-on distinguer différents types de causes ? \\[0pt]
Peut-on penser illogiquement ? \\[0pt]
Peut-on préconiser, dans les sciences humaines et sociales, l'imitation des sciences de la nature ? \\[0pt]
Peut-on réduire la pensée à une espèce de comportement ? \\[0pt]
Peut-on restreindre la logique à la pensée formelle ? \\[0pt]
Peut-on se passer des relations ? \\[0pt]
Peut-on tout définir ? \\[0pt]
Peut-on tout démontrer ? \\[0pt]
Physique et mathématiques \\[0pt]
Pourquoi définir ? \\[0pt]
Pourquoi des géométries ? \\[0pt]
Pourquoi des logiciens ? \\[0pt]
Pourquoi est-il difficile de rectifier une erreur ? \\[0pt]
Pourquoi faut-il être cohérent ? \\[0pt]
Pourquoi formaliser des arguments ? \\[0pt]
Pourquoi la raison recourt-elle à l'hypothèse ? \\[0pt]
Pourquoi les mathématiques s'appliquent-elles à la réalité ? \\[0pt]
Pourquoi plusieurs sciences ? \\[0pt]
Prédicats et relations \\[0pt]
Prémisses et conclusions \\[0pt]
Prévoir \\[0pt]
Probabilité et explication scientifique \\[0pt]
Proposition et jugement \\[0pt]
Quantification et pensée scientifique \\[0pt]
Quantité et qualité \\[0pt]
Que déduire d'une contradiction ? \\[0pt]
Que disent les tables de vérité ? \\[0pt]
Quel est le but d'une théorie physique ? \\[0pt]
Quel est le but du travail scientifique ? \\[0pt]
Quelle réalité la science décrit-elle ? \\[0pt]
Quel rôle attribuer à l'intuition \emph{a priori} dans une philosophie des mathématiques ? \\[0pt]
Quel rôle la logique joue-t-elle en mathématiques ? \\[0pt]
Quel sens y a-t-il à se demander si les sciences humaines sont vraiment des sciences ? \\[0pt]
Que nous apprend l'histoire des sciences ? \\[0pt]
Que peut-on calculer ? \\[0pt]
Qu'est-ce que calculer ? \\[0pt]
Qu'est-ce que la psychologie ? \\[0pt]
Qu'est-ce qui est indiscutable ? \\[0pt]
Qu'est-ce qui est invérifiable ? \\[0pt]
Qu'est-ce qu'ignore la science ? \\[0pt]
Qu'est-ce qui rend vrai un énoncé ? \\[0pt]
Qu'est-ce qu'un argument ? \\[0pt]
Qu'est-ce qu'un concept scientifique ? \\[0pt]
Qu'est-ce qu'une belle démonstration ? \\[0pt]
Qu'est-ce qu'une catégorie ? \\[0pt]
Qu'est-ce qu'une conception scientifique du monde ? \\[0pt]
Qu'est-ce qu'une connaissance non scientifique ? \\[0pt]
Qu'est-ce qu'une croyance vraie ? \\[0pt]
Qu'est-ce qu'une découverte scientifique ? \\[0pt]
Qu'est-ce qu'une discipline savante ? \\[0pt]
Qu'est-ce qu'une éducation scientifique ? \\[0pt]
Qu'est-ce qu'une fonction ? \\[0pt]
Qu'est-ce qu'une hypothèse scientifique ? \\[0pt]
Qu'est-ce qu'une idée vraie ? \\[0pt]
Qu'est-ce qu'une loi scientifique ? \\[0pt]
Qu'est-ce qu'une preuve ? \\[0pt]
Qu'est-ce qu'une psychologie scientifique ? \\[0pt]
Qu'est-ce qu'une révolution scientifique ? \\[0pt]
Qu'est-ce qu'une science rigoureuse ? \\[0pt]
Qu'est-ce qu'une vérité scientifique ? \\[0pt]
Qu'est-ce qu'une vision scientifique du monde ? \\[0pt]
Qu'est-ce qu'un fait scientifique ? \\[0pt]
Qu'est-ce qu'un modèle ? \\[0pt]
Qu'est-ce qu'un nombre ? \\[0pt]
Qu'est-ce qu'un paradoxe ? \\[0pt]
Qu'est-ce qu'un principe ? \\[0pt]
Qu'est-ce qu'un problème scientifique ? \\[0pt]
Question et problème \\[0pt]
Que vaut une preuve contre un préjugé ? \\[0pt]
Raisonner et calculer \\[0pt]
Réfutation et confirmation \\[0pt]
Sauver les phénomènes \\[0pt]
Savoir et objectivité dans les sciences \\[0pt]
Savoir et pouvoir \\[0pt]
Savoir et rectification \\[0pt]
Savoir et vérifier \\[0pt]
Savoir pour prévoir \\[0pt]
Science et complexité \\[0pt]
Science et histoire \\[0pt]
Science et idéologie \\[0pt]
Science et magie \\[0pt]
Science et opinion \\[0pt]
Science et persuasion \\[0pt]
Science et réalité \\[0pt]
Science et religion \\[0pt]
Science et sagesse \\[0pt]
Science et technologie \\[0pt]
Science pure et science appliquée \\[0pt]
Sciences de la nature et sciences de l'esprit \\[0pt]
Sciences de la nature et sciences humaines \\[0pt]
Signification et vérité \\[0pt]
Sujet et prédicat \\[0pt]
Sur quoi se fonde la connaissance scientifique ? \\[0pt]
Syllogisme et démonstration \\[0pt]
Tautologie et contradiction \\[0pt]
Technique et pratiques scientifiques \\[0pt]
Théorie et modélisation \\[0pt]
Toute connaissance autre que scientifique doit-elle être considérée comme une illusion ? \\[0pt]
Toute connaissance est-elle historique ? \\[0pt]
Tout énoncé est-il nécessairement vrai ou faux ? \\[0pt]
Toutes les vérités scientifiques sont-elles révisables ? \\[0pt]
Tout savoir est-il fondé sur un savoir premier ? \\[0pt]
Une logique non-formelle est-elle possible ? \\[0pt]
Une théorie scientifique peut-elle être ramenée à des propositions empiriques élémentaires ? \\[0pt]
Universalité et nécessité dans les sciences \\[0pt]
Un problème scientifique peut-il être insoluble ? \\[0pt]
Un sceptique peut-il être logicien ? \\[0pt]
Vérité et histoire \\[0pt]
Vitalisme et mécanique \\[0pt]
Y a-t-il de l'indémontrable ? \\[0pt]
Y a-t-il des expériences cruciales ? \\[0pt]
Y a-t-il des lois du hasard ? \\[0pt]
Y a-t-il des propriétés singulières ? \\[0pt]
Y a-t-il des révolutions scientifiques ? \\[0pt]
Y a-t-il différentes manières de connaître ? \\[0pt]
Y a-t-il du synthétique \emph{a priori} ? \\[0pt]
Y a-t-il plusieurs nécessités ? \\[0pt]
Y a-t-il un art d'inventer ? \\[0pt]
Y a-t-il un critère du vrai ? \\[0pt]
Y a-t-il une hiérarchie des sciences ? \\[0pt]
Y a-t-il une histoire de la vérité ? \\[0pt]
Y a-t-il une logique de la découverte ? \\[0pt]
Y a-t-il une logique de la découverte scientifique ? \\[0pt]
Y a-t-il une science de l'individuel ? \\[0pt]
Y a-t-il une science du qualitatif ? \\[0pt]
Y a-t-il une unité de la science ? \\[0pt]
Y a-t-il une universalité des mathématiques ? \\[0pt]


\subsection{Métaphysique}
\label{sec:orgd7abb8a}

\noindent
Absolu / relatif \\[0pt]
Admettre une cause première, est-ce faire une pétition de principe ? \\[0pt]
Agir \\[0pt]
Aller au-delà des apparences \\[0pt]
Altérité, diversité, pluralité \\[0pt]
Âme, conscience, esprit \\[0pt]
Antérieur / postérieur \\[0pt]
Apparaître \\[0pt]
Apparence et réalité \\[0pt]
À quoi sert la métaphysique ? \\[0pt]
À quoi sert la théodicée ? \\[0pt]
À quoi sert l'ontologie ? \\[0pt]
Arrive-t-il que l'impossible se produise ? \\[0pt]
Au-delà \\[0pt]
Au-delà de la nature ? \\[0pt]
Avons-nous accès aux choses-mêmes ? \\[0pt]
Avons-nous besoin d'une conception métaphysique du monde ? \\[0pt]
Catégories de l'être, catégories de langue \\[0pt]
Causalité et finalité \\[0pt]
Cause et loi \\[0pt]
Cause et raison \\[0pt]
Causes premières et causes secondes \\[0pt]
Ce qui doit-être, est-ce autre chose que ce qui est ? \\[0pt]
Ce qui est contradictoire peut-il exister ? \\[0pt]
Ce qui fut et ce qui sera \\[0pt]
Ce qui n'a pas lieu d'être \\[0pt]
Ce qui n'est pas \\[0pt]
Ce qui n'est pas réel est-il impossible ? \\[0pt]
Ce qui passe et ce qui demeure \\[0pt]
Ce qui subsiste et ce qui change \\[0pt]
Certitude et vérité \\[0pt]
Chaos et cosmos \\[0pt]
Chaque science porte-t-elle une métaphysique qui lui est propre ? \\[0pt]
Chose et objet \\[0pt]
Commencer \\[0pt]
Comment penser les universaux ? \\[0pt]
Comment réfuter une thèse métaphysique ? \\[0pt]
Comment s'assurer de ce qui est réel ? \\[0pt]
Comment s'assurer qu'on est libre ? \\[0pt]
Complet / incomplet \\[0pt]
Concept et existence \\[0pt]
Concevoir le possible \\[0pt]
Connaître et penser \\[0pt]
Consistance et précarité \\[0pt]
Contingence et nécessité \\[0pt]
Contingence et rationalité \\[0pt]
Création et production \\[0pt]
Définir, est-ce déterminer l'essence ? \\[0pt]
De quoi le réel est-il constitué ? \\[0pt]
Des hommes et des dieux \\[0pt]
Devenir \\[0pt]
Devenir autre \\[0pt]
Devoir mourir \\[0pt]
Devons-nous douter de l'existence des choses ? \\[0pt]
Dieu des philosophes et Dieu des croyants \\[0pt]
Dieu est-il l'être ? \\[0pt]
Dieu est-il un concept ? \\[0pt]
Dieu et l'âme \\[0pt]
Dieu et le monde \\[0pt]
Dieu, l'âme, le monde \\[0pt]
Dieu pense-t-il ? \\[0pt]
Dieu peut-il tout faire ? \\[0pt]
Dire ce qui est \\[0pt]
Dire le monde \\[0pt]
Discuter de la beauté d'une chose, est-ce discuter sur une réalité ? \\[0pt]
Douter de tout \\[0pt]
Du penser à l'être, la conséquence est-elle bonne ? \\[0pt]
Durer \\[0pt]
Empirique et transcendantal \\[0pt]
En quel sens la métaphysique a-t-elle une histoire ? \\[0pt]
En quel sens la métaphysique est-elle une science ? \\[0pt]
En quel sens parler de structure métaphysique ? \\[0pt]
En quoi la connaissance de la matière peut-elle relever de la métaphysique ? \\[0pt]
Entité et identité \\[0pt]
Essence et existence \\[0pt]
Essence et nature \\[0pt]
Éternité et perpétuité \\[0pt]
Être cause de soi \\[0pt]
Être dans l'esprit \\[0pt]
Être dans le temps \\[0pt]
Être dans le vrai \\[0pt]
Être de son temps \\[0pt]
Être déterminé \\[0pt]
Être en puissance \\[0pt]
Être et avoir \\[0pt]
Être et devenir \\[0pt]
Être et devoir être \\[0pt]
Être et être pensé \\[0pt]
Être et être perçu \\[0pt]
Être et ne plus être \\[0pt]
Être et paraître \\[0pt]
Être et représentation \\[0pt]
Être et sens \\[0pt]
Être identique \\[0pt]
Être par soi \\[0pt]
Être par soi / être par autre chose \\[0pt]
Être possible \\[0pt]
Être sans cause \\[0pt]
« Être » se dit-il en plusieurs sens ? \\[0pt]
Être sujet \\[0pt]
Être une chose qui pense \\[0pt]
Être, vie et pensée \\[0pt]
Existence et contingence \\[0pt]
Exister \\[0pt]
Existe-t-il dans le monde des réalités identiques ? \\[0pt]
Existe-t-il des expériences métaphysiques ? \\[0pt]
Existe-t-il des intuitions métaphysiques ? \\[0pt]
Existe-t-il des principes premiers ? \\[0pt]
Existe-t-il plusieurs déterminismes ? \\[0pt]
Existe-t-il plusieurs mondes ? \\[0pt]
Existe-t-il une réalité subjective ? \\[0pt]
Existe-t-il une réalité symbolique ? \\[0pt]
Existe-t-il une science de l'être ? \\[0pt]
Faire de la métaphysique, est-ce se détourner du monde ? \\[0pt]
Fait et essence \\[0pt]
Faut-il opposer le réel et l'imaginaire ? \\[0pt]
Faut-il opposer science et métaphysique ? \\[0pt]
Finalité et causalité \\[0pt]
Grammaire et logique \\[0pt]
Grammaire et métaphysique \\[0pt]
Hasard et destin \\[0pt]
Ici et maintenant \\[0pt]
Identité et différence \\[0pt]
Il faut de tout pour faire un monde \\[0pt]
Il y a \\[0pt]
Incréé / créé \\[0pt]
Indépendant / dépendant \\[0pt]
Infini et indéfini \\[0pt]
La béatitude \\[0pt]
L'absence \\[0pt]
L'absence de fondement \\[0pt]
L'absolu \\[0pt]
L'absolu est-il connaissable ? \\[0pt]
L'abstrait et le concret \\[0pt]
L'abstrait et l'immatériel \\[0pt]
La cause de soi \\[0pt]
La cause efficiente \\[0pt]
La cause finale \\[0pt]
La cause formelle \\[0pt]
La cause motrice \\[0pt]
La cause première \\[0pt]
L'accident \\[0pt]
La chair et l'esprit \\[0pt]
La chose \\[0pt]
La chose en soi \\[0pt]
La chose et l'objet \\[0pt]
La cohérence des attributs divins \\[0pt]
La connaissance de la nécessité a priori peut-elle évoluer ? \\[0pt]
La connaissance de l'infini \\[0pt]
La connexion des choses et la connexion des idées \\[0pt]
La contemplation \\[0pt]
La contingence \\[0pt]
La contingence du futur \\[0pt]
La contradiction \\[0pt]
La création \\[0pt]
La croyance en Dieu est-elle irrationnelle ? \\[0pt]
L'acte \\[0pt]
L'action et la passion \\[0pt]
L'activité philosophique conduit-elle à la métaphysique ? \\[0pt]
La déficience \\[0pt]
La destruction \\[0pt]
La différence \\[0pt]
La différence de l'être et de l'étant \\[0pt]
La disposition \\[0pt]
La disposition. Le possible et le réel \\[0pt]
La division \\[0pt]
La dualité \\[0pt]
La fin \\[0pt]
La finalité \\[0pt]
La fin de la métaphysique \\[0pt]
La finitude \\[0pt]
La forme \\[0pt]
La grammaire véhicule-t-elle une métaphysique ? \\[0pt]
La hiérarchie des êtres \\[0pt]
La limitation des créatures \\[0pt]
La limite \\[0pt]
L'altérité \\[0pt]
La lumière naturelle \\[0pt]
La manifestation \\[0pt]
La matière \\[0pt]
La matière et la forme \\[0pt]
La matière peut-elle penser ? \\[0pt]
La matière première \\[0pt]
L'âme et le corps \\[0pt]
L'âme, le monde et Dieu \\[0pt]
La métaphysique a-t-elle ses fictions ? \\[0pt]
La métaphysique est-elle affaire de raisonnement ? \\[0pt]
La métaphysique est-elle le fondement de la morale ? \\[0pt]
La métaphysique est-elle nécessairement une réflexion sur Dieu ? \\[0pt]
La métaphysique est-elle une discipline théorique ? \\[0pt]
La métaphysique est-elle une science ? \\[0pt]
La métaphysique peut-elle être autre chose qu'une science recherchée ? \\[0pt]
La métaphysique peut-elle faire appel à l'expérience ? \\[0pt]
La métaphysique procure-t-elle un savoir ? \\[0pt]
La métaphysique relève-t-elle de la philosophie ou de la poésie ? \\[0pt]
La métaphysique répond-elle à un besoin ? \\[0pt]
La métaphysique repose-t-elle sur des croyances ? \\[0pt]
La métaphysique se définit-elle par son objet ou sa démarche ? \\[0pt]
La modalité \\[0pt]
La multiplicité \\[0pt]
La multiplicité des acceptions de l'étant \\[0pt]
L'anachronisme \\[0pt]
La naissance \\[0pt]
L'analogie \\[0pt]
La nature \\[0pt]
La nature et la grâce \\[0pt]
La nature s'oppose-t-elle à l'esprit ? \\[0pt]
La négation \\[0pt]
L'angoisse \\[0pt]
La notion de paradis a-t-elle un sens exclusivement religieux ? \\[0pt]
La notion d'ordre \\[0pt]
La nouveauté \\[0pt]
L'antériorité \\[0pt]
La participation \\[0pt]
La pensée de la mort a-t-elle un objet ? \\[0pt]
La perfection \\[0pt]
La personne \\[0pt]
La pluralité des mondes \\[0pt]
La pluralité des sens de l'être \\[0pt]
La plus grande perfection \\[0pt]
La position \\[0pt]
La possibilité métaphysique \\[0pt]
La possibilité réelle \\[0pt]
L'apparence \\[0pt]
La présence \\[0pt]
La preuve de l'existence de Dieu \\[0pt]
L'\emph{a priori} \\[0pt]
La privation \\[0pt]
La puissance \\[0pt]
La puissance des contraires \\[0pt]
La puissance et l'acte \\[0pt]
La puissance et le possible \\[0pt]
La quête des origines \\[0pt]
La raison suffisante \\[0pt]
La réalité \\[0pt]
La réalité de la contradiction \\[0pt]
La réalité du bien \\[0pt]
La réalité du corps \\[0pt]
La réalité du mal \\[0pt]
La réalité du passé \\[0pt]
La réalité du sensible \\[0pt]
La réalité du temps \\[0pt]
La réalité mentale \\[0pt]
La recherche de l'absolu \\[0pt]
La relation \\[0pt]
La relation de causalité est-elle temporelle ? \\[0pt]
La relation d'identité \\[0pt]
L'artiste est-il un métaphysicien ? \\[0pt]
La science de l'être \\[0pt]
La séparation \\[0pt]
La singularité \\[0pt]
La spontanéité \\[0pt]
La subsistance \\[0pt]
La substance \\[0pt]
La substance et l'accident \\[0pt]
La substance et le substrat \\[0pt]
La surface et la profondeur \\[0pt]
La téléologie \\[0pt]
La théologie peut-elle être rationnelle ? \\[0pt]
La totalité \\[0pt]
La toute-puissance \\[0pt]
La transcendance \\[0pt]
L'au-delà \\[0pt]
L'au-delà de l'être \\[0pt]
La variété \\[0pt]
L'avenir est-il prévisible ? \\[0pt]
L'avenir est-il sans image ? \\[0pt]
La vérité est-elle hors de notre portée ? \\[0pt]
La vie brève \\[0pt]
La vie de l'esprit \\[0pt]
La vie est-elle une notion métaphysique ? \\[0pt]
La vie éternelle \\[0pt]
Le besoin de métaphysique est-il un besoin de connaissance ? \\[0pt]
Le besoin métaphysique \\[0pt]
Le changement \\[0pt]
Le chaos \\[0pt]
Le chaos du monde \\[0pt]
Le commencement \\[0pt]
Le concret \\[0pt]
Le contenu empirique \\[0pt]
Le contradictoire peut-il exister ? \\[0pt]
Le corps et l'âme \\[0pt]
Le corps et l'esprit \\[0pt]
Le créé et l'incréé \\[0pt]
Le désir métaphysique \\[0pt]
Le désordre \\[0pt]
Le destin \\[0pt]
Le devenir \\[0pt]
Le devenir est-il pensable ? \\[0pt]
Le dieu des philosophes \\[0pt]
Le Dieu des philosophes \\[0pt]
Le divin \\[0pt]
Le don \\[0pt]
Le doute métaphysique \\[0pt]
L'efficience \\[0pt]
Le fini et l'infini \\[0pt]
Le fond \\[0pt]
Le fondement \\[0pt]
Le genre et l'espèce \\[0pt]
Le grand livre de la nature \\[0pt]
Le hasard \\[0pt]
Le hasard gouverne-t-il le monde ? \\[0pt]
Le hasard n'est-il que le nom de notre ignorance ? \\[0pt]
Le jeu des possibles \\[0pt]
Le jeu du monde \\[0pt]
L'élément \\[0pt]
Le lieu \\[0pt]
Le lieu de la pensée \\[0pt]
Le mal constitue-t-il une objection à l'existence de Dieu ? \\[0pt]
Le mal métaphysique \\[0pt]
Le meilleur des mondes possible \\[0pt]
Le même et l'autre \\[0pt]
Le métaphysicien est-il un physicien à sa façon ? \\[0pt]
Le miracle \\[0pt]
Le mode \\[0pt]
Le moi \\[0pt]
Le monde a-t-il une histoire ? \\[0pt]
Le monde aurait-il un sens si l'homme n'existait pas ? \\[0pt]
Le monde des idées \\[0pt]
Le monde est-il en progrès ? \\[0pt]
Le monde est-il notre représentation ? \\[0pt]
Le monde et la nature \\[0pt]
Le monde intérieur \\[0pt]
Le monde se suffit-il à lui-même ? \\[0pt]
Le monde vrai \\[0pt]
L'enchevêtrement des causes \\[0pt]
Le néant \\[0pt]
L'ennui \\[0pt]
Le nombre \\[0pt]
Le non-être \\[0pt]
Le parfait \\[0pt]
Le particulier \\[0pt]
L'éphémère \\[0pt]
Le phénomène \\[0pt]
Le possible \\[0pt]
Le possible et le contingent \\[0pt]
Le possible et le réel \\[0pt]
Le premier principe \\[0pt]
Le présent \\[0pt]
Le principe de causalité \\[0pt]
Le principe de contradiction \\[0pt]
Le principe d'identité \\[0pt]
Le problème de l'être \\[0pt]
Le propre et l'impropre \\[0pt]
Le quelque chose \\[0pt]
Le réel est-il rationnel ? \\[0pt]
Le réel et le virtuel \\[0pt]
Le réel excède-t-il l'intelligible ? \\[0pt]
Le réel peut-il être contradictoire ? \\[0pt]
Le réel se donne-t-il à voir ? \\[0pt]
Le règne des fins \\[0pt]
Le royaume du possible \\[0pt]
Les affections \\[0pt]
Les atomes \\[0pt]
Les catégories sont-elles définitives ? \\[0pt]
Les catégories sont-elles des effets de langue ? \\[0pt]
Les causes \\[0pt]
Les causes finales \\[0pt]
Les choses \\[0pt]
Les contradictions de la raison \\[0pt]
Les contradictions sont-elles un échec de la raison ? \\[0pt]
Les différentes preuves de l'existence de Dieu prouvent-elles le même Dieu ? \\[0pt]
Le sens de la réalité \\[0pt]
Le sens de l'être \\[0pt]
Les faits et les valeurs \\[0pt]
Les fins dernières \\[0pt]
Les fins sont-elles toujours intentionnelles ? \\[0pt]
Les genres de Dieu \\[0pt]
Les genres de l'être \\[0pt]
Les idées et les choses \\[0pt]
Les idées existent-elles ? \\[0pt]
Les idées sont-elles vivantes ? \\[0pt]
Le simple \\[0pt]
Les individus \\[0pt]
Le singulier \\[0pt]
Les limites de la raison \\[0pt]
Les limites de l'expérience \\[0pt]
Les lois de la nature \\[0pt]
Les lois de la nature sont-elles contingentes ? \\[0pt]
Les mots et les choses \\[0pt]
Les nombres gouvernent-ils le monde ? \\[0pt]
Les noms divins \\[0pt]
Les objets de pensée \\[0pt]
Les objets impossibles \\[0pt]
Le solipsisme \\[0pt]
Le souverain bien \\[0pt]
Les phénomènes \\[0pt]
L'esprit appartient-il à la nature ? \\[0pt]
Les propositions métaphysiques sont-elles des illusions ? \\[0pt]
Les questions métaphysiques ont-elles un sens ? \\[0pt]
Les ressources de l'analogie \\[0pt]
Les sciences ont-elles besoin d'une fondation ? \\[0pt]
Les sciences ont-elles besoin d'une fondation métaphysique ? \\[0pt]
L'essence et l'existence \\[0pt]
Les substances \\[0pt]
Les transcendantaux \\[0pt]
Le substrat \\[0pt]
Le sujet \\[0pt]
Le sujet de la pensée \\[0pt]
Les universaux \\[0pt]
Les universaux existent-ils ? \\[0pt]
Les vérités éternelles \\[0pt]
L'étant est-il le premier pensable ? \\[0pt]
Le temps du monde \\[0pt]
Le temps est-il une réalité ? \\[0pt]
Le temps rend-il tout vain ? \\[0pt]
L'éternel présent \\[0pt]
L'éternité \\[0pt]
Le tout et les parties \\[0pt]
L'être de la vérité \\[0pt]
L'être de l'image \\[0pt]
L'être en tant qu'être \\[0pt]
L'être en tant qu'être est-il connaissable ? \\[0pt]
L'être est-il un genre ? \\[0pt]
L'être et l'apparence \\[0pt]
L'être et la volonté \\[0pt]
L'être et le bien \\[0pt]
L'être et le néant \\[0pt]
L'être et les êtres \\[0pt]
L'être et l'esprit \\[0pt]
L'être et l'essence \\[0pt]
L'être et l'étant \\[0pt]
L'être et le temps \\[0pt]
L'Être suprême \\[0pt]
L'événement \\[0pt]
L'événement manque-t-il d'être ? \\[0pt]
Le vide \\[0pt]
Le virtuel existe-t-il ? \\[0pt]
L'existence du mal \\[0pt]
L'existence se démontre-t-elle ? \\[0pt]
L'existence se prouve-t-elle ? \\[0pt]
L'expérience métaphysique \\[0pt]
L'homme est-il un animal métaphysique ? \\[0pt]
L'homme fait-il partie de la nature ? \\[0pt]
Liberté humaine et liberté divine \\[0pt]
L'idéal et le réel \\[0pt]
L'idée de Dieu \\[0pt]
L'idée d'un commencement absolu \\[0pt]
L'immanence \\[0pt]
L'immatériel \\[0pt]
L'immédiat \\[0pt]
L'immortalité \\[0pt]
L'immortalité de l'âme \\[0pt]
L'immuable \\[0pt]
L'imparfait \\[0pt]
L'impossible \\[0pt]
L'impossible est-il concevable ? \\[0pt]
L'imprévisible \\[0pt]
L'impuissance de la raison \\[0pt]
L'inapparent \\[0pt]
L'incompréhensible \\[0pt]
L'incorporel \\[0pt]
L'indéterminé \\[0pt]
L'individu \\[0pt]
L'individuel \\[0pt]
L'indivisible \\[0pt]
L'infini et l'indéfini \\[0pt]
L'infinité de l'espace \\[0pt]
L'inquiétude \\[0pt]
L'intangible \\[0pt]
L'intériorité \\[0pt]
L'interrogation humaine \\[0pt]
L'invisible \\[0pt]
L'irréel \\[0pt]
L'irréversibilité \\[0pt]
Logique et métaphysique \\[0pt]
L'omniscience \\[0pt]
L'ontologie peut-elle être relative ? \\[0pt]
L'ordre des choses \\[0pt]
L'ordre est-il dans les choses ? \\[0pt]
L'origine du mal \\[0pt]
L'un \\[0pt]
L'un et le multiple \\[0pt]
L'un et le tout \\[0pt]
L'un et l'être \\[0pt]
L'unité des contraires \\[0pt]
L'univers \\[0pt]
L'universel \\[0pt]
L'univocité de l'étant \\[0pt]
L'un, le vrai, le bien \\[0pt]
Matériel / immatériel \\[0pt]
Métaphysique et histoire \\[0pt]
Métaphysique et mystique \\[0pt]
Métaphysique et ontologie \\[0pt]
Métaphysique et psychologie \\[0pt]
Métaphysique et religion \\[0pt]
Métaphysique et théologie \\[0pt]
Métaphysique spéciale, métaphysique générale \\[0pt]
Monde et nature \\[0pt]
Mon existence est-elle contingente ? \\[0pt]
Mouvement et changement \\[0pt]
Naturaliser l'esprit \\[0pt]
Nature et liberté \\[0pt]
Nature et monde \\[0pt]
Nécessaire / contingent \\[0pt]
Négation et privation \\[0pt]
Ne pas multiplier en vain les entités \\[0pt]
N'existe-t-il que des individus ? \\[0pt]
Notre corps pense-t-il ? \\[0pt]
N'y a-t-il de connaissance que par les causes ? \\[0pt]
N'y a-t-il de réel que le présent ? \\[0pt]
N'y a-t-il d'être que sensible ? \\[0pt]
N'y a-t-il d'origine que mythique ? \\[0pt]
N'y a-t-il qu'un seul monde ? \\[0pt]
Ordre et chaos \\[0pt]
Ordre et désordre \\[0pt]
Origine et commencement \\[0pt]
Origine et fondement \\[0pt]
Par quoi un individu diffère-t-il réellement d'un autre ? \\[0pt]
Participation et imitation \\[0pt]
Pâtir \\[0pt]
Penser, est-ce calculer ? \\[0pt]
Penser et être \\[0pt]
Penser requiert-il d'avoir un corps ? \\[0pt]
Penser sans corps \\[0pt]
Percevons-nous le monde extérieur ? \\[0pt]
Permanent / successif \\[0pt]
Persévérer dans son être \\[0pt]
Peut-on appréhender les choses telles qu'elles sont ? \\[0pt]
Peut-on connaître les causes ? \\[0pt]
Peut-on dire ce qui n'est pas ? \\[0pt]
Peut-on dire l'être ? \\[0pt]
Peut-on douter de la raison ? \\[0pt]
Peut-on douter de sa propre existence ? \\[0pt]
Peut-on entreprendre d'éliminer la métaphysique ? \\[0pt]
Peut-on ne pas être soi-même ? \\[0pt]
Peut-on parler de vérités métaphysiques ? \\[0pt]
Peut-on penser la création ? \\[0pt]
Peut-on penser la fin de toute chose ? \\[0pt]
Peut-on penser le réel comme un tout ? \\[0pt]
Peut-on penser l'extériorité ? \\[0pt]
Peut-on penser une métaphysique sans Dieu ? \\[0pt]
Peut-on prévoir l'avenir ? \\[0pt]
Peut-on prouver l'existence du monde ? \\[0pt]
Peut-on réduire une métaphysique à une conception du monde ? \\[0pt]
Peut-on se passer de Dieu ? \\[0pt]
Peut-on se passer de l'idée de cause finale ? \\[0pt]
Peut-on se passer de l'idée de Dieu ? \\[0pt]
Peut-on se passer de métaphysique ? \\[0pt]
Peut-on se passer des causes finales ? \\[0pt]
Peut-on se passer d'ontologie ? \\[0pt]
Peut-on tout définir ? \\[0pt]
Physique et métaphysique \\[0pt]
Pourquoi Dieu se soucierait-il des affaires humaines ? \\[0pt]
Pourquoi penser l'impossible ? \\[0pt]
Pourquoi y a-t-il quelque chose plutôt que rien ? \\[0pt]
Présence et absence \\[0pt]
Principe et cause \\[0pt]
Principe et fondement \\[0pt]
Privation et négation \\[0pt]
Propriétés et dispositions \\[0pt]
Prouver en métaphysique \\[0pt]
Psychologie et métaphysique \\[0pt]
Quantité et qualité \\[0pt]
Quel est l'objet de la métaphysique ? \\[0pt]
Quelle réalité l'imagination nous fait-elle connaître ? \\[0pt]
Que nous apprend la métaphysique ? \\[0pt]
Que peut-on dire de l'être ? \\[0pt]
Que pouvons-nous comprendre du monde ? \\[0pt]
Que prouvent les preuves de l'existence de Dieu ? \\[0pt]
Que savons-nous des principes ? \\[0pt]
Qu'est-ce qu'avoir une idée ? \\[0pt]
Qu'est-ce que le temps ? \\[0pt]
Qu'est-ce que peut un corps ? \\[0pt]
Qu'est-ce qui apparaît ? \\[0pt]
Qu'est-ce qui est essentiel ? \\[0pt]
Qu'est-ce qui est nécessaire ? \\[0pt]
Qu'est-ce qui est réel ? \\[0pt]
Qu'est-ce qui est sans raison ? \\[0pt]
Qu'est-ce qui est transcendant ? \\[0pt]
Qu'est-ce qui n'appartient pas au monde ? \\[0pt]
Qu'est-ce qui ne change pas ? \\[0pt]
Qu'est-ce qui n'est pas en mouvement ? \\[0pt]
Qu'est-ce qui n'existe pas ? \\[0pt]
Qu'est-ce qu'un accident ? \\[0pt]
Qu'est-ce qu'une âme ? \\[0pt]
Qu'est-ce qu'une catégorie ? \\[0pt]
Qu'est-ce qu'une catégorie de l'être ? \\[0pt]
Qu'est-ce qu'une connaissance métaphysique \\[0pt]
Qu'est-ce qu'un élément ? \\[0pt]
Qu'est-ce qu'une limite ? \\[0pt]
Qu'est-ce qu'une méditation métaphysique ? \\[0pt]
Qu'est-ce qu'une philosophie première ? \\[0pt]
Qu'est-ce qu'une propriété essentielle ? \\[0pt]
Qu'est-ce qu'une question métaphysique ? \\[0pt]
Qu'est-ce qu'une substance ? \\[0pt]
Qu'est-ce qu'un événement ? \\[0pt]
Qu'est-ce qu'un métaphysicien ? \\[0pt]
Qu'est-ce qu'un monde ? \\[0pt]
Qu'est-ce qu'un objet métaphysique ? \\[0pt]
Qu'est-ce qu'un phénomène ? \\[0pt]
Qu'est-ce qu'un principe ? \\[0pt]
Qu'est-ce qu'un problème métaphysique ? \\[0pt]
Qu'est-ce qu'un système ? \\[0pt]
Que veut dire introduire à la métaphysique ? \\[0pt]
Que veut-on dire quand on dit « rien n'est sans raison » ? \\[0pt]
Qui est libre ? \\[0pt]
Qui pense ? \\[0pt]
Qu'y a-t-il au-delà de l'être ? \\[0pt]
Religion et métaphysique \\[0pt]
Rendre raison \\[0pt]
Ressemblance et identité \\[0pt]
Ressembler \\[0pt]
Rêver éloigne-t-il de la réalité ? \\[0pt]
Rien \\[0pt]
Rien n'est sans raison \\[0pt]
Sans les mots, que seraient les choses ? \\[0pt]
Sauver les apparences \\[0pt]
Se détacher des sens \\[0pt]
Sensible et intelligible \\[0pt]
Se savoir mortel \\[0pt]
Seul le présent existe-t-il ? \\[0pt]
Si l'esprit n'est pas une table rase, qu'est-il ? \\[0pt]
Simple / composé \\[0pt]
Sommes-nous des êtres métaphysiques ? \\[0pt]
Suffit-il à Dieu d'être possible pour exister ? \\[0pt]
Système et liberté \\[0pt]
Temps et éternité \\[0pt]
Tout a-t-il une raison d'être ? \\[0pt]
Tout ce qui existe est-il matériel ? \\[0pt]
Toute chose a-t-elle une fin ? \\[0pt]
Toute métaphysique implique-t-elle une transcendance ? \\[0pt]
Toute physique exige-t-elle une métaphysique ? \\[0pt]
Toutes les choses sont-elles singulières ? \\[0pt]
Tout est-il relatif ? \\[0pt]
Tout être est-il dans l'espace ? \\[0pt]
Tout fondement de la connaissance est-il métaphysique ? \\[0pt]
Tout savoir \\[0pt]
Un Dieu unique ? \\[0pt]
Une cause peut-elle être libre ? \\[0pt]
Une métaphysique athée est-elle possible ? \\[0pt]
Une métaphysique n'est-elle qu'une ontologie ? \\[0pt]
Une métaphysique peut-elle être sceptique ? \\[0pt]
Unité et unicité \\[0pt]
Univocité et équivocité \\[0pt]
Vérités de fait et vérités de raison \\[0pt]
Vie et volonté \\[0pt]
Vivons-nous tous dans le même monde ? \\[0pt]
Y a-t-il de l'incompréhensible ? \\[0pt]
Y a-t-il de l'inconnaissable ? \\[0pt]
Y a-t-il des degrés de réalité ? \\[0pt]
Y a-t-il des êtres mathématiques ? \\[0pt]
Y a-t-il des fins dernières ? \\[0pt]
Y a-t-il des preuves de la non-existence de Dieu ? \\[0pt]
Y a-t-il des preuves de l'existence de Dieu ? \\[0pt]
Y a-t-il des raisons de vivre ? \\[0pt]
Y a-t-il plusieurs métaphysiques ? \\[0pt]
Y a-t-il un commencement à tout ? \\[0pt]
Y a-t-il une argumentation métaphysique ? \\[0pt]
Y a-t-il une connaissance métaphysique ? \\[0pt]
Y a-t-il une expérience de l'éternité ? \\[0pt]
Y a-t-il une expérience du néant ? \\[0pt]
Y a-t-il une expérience métaphysique ? \\[0pt]
Y a-t-il une hiérarchie des êtres ? \\[0pt]
Y a-t-il une métaphysique de l'amour ? \\[0pt]
Y a-t-il une science des principes ? \\[0pt]
Y a-t-il une vérité absolue ? \\[0pt]
Y a-t-il un principe du mal ? \\[0pt]
Y a-t-il un savoir du contingent ? \\[0pt]
Y a-t-il un sens à dire que Dieu comprend, veut, fait ? \\[0pt]
Y a-t-il un sentiment métaphysique ? \\[0pt]


\subsection{Morale}
\label{sec:org0054c8e}

\noindent
Abjection et sainteté \\[0pt]
À chacun son dû \\[0pt]
Agir en conscience \\[0pt]
Agir moralement, est-ce lutter contre ses idées ? \\[0pt]
« Aime, et fais ce que tu veux » \\[0pt]
Aimer ses proches \\[0pt]
Aimer son prochain comme soi-même \\[0pt]
« À l'impossible, nul n'est tenu » \\[0pt]
À quoi servent les doctrines morales ? \\[0pt]
Arriver à ses fins \\[0pt]
Autrui est-il mon prochain ? \\[0pt]
Avoir des principes \\[0pt]
Avoir du caractère \\[0pt]
Avons-nous des devoirs envers les animaux ? \\[0pt]
Avons-nous des devoirs envers le vivant ? \\[0pt]
Avons-nous des devoirs envers nous-mêmes ? \\[0pt]
Bien agir \\[0pt]
« Bienheureuse faute » \\[0pt]
Bien vivre \\[0pt]
Ce que la morale autorise, l'État peut-il légitimement l'interdire ? \\[0pt]
Ce qui dépend de moi \\[0pt]
Ce qui ne dépend pas de nous \\[0pt]
C'est pour ton bien \\[0pt]
Changer ses désirs plutôt que l'ordre du monde \\[0pt]
Chercher son intérêt, est-ce être immoral ? \\[0pt]
Comment traiter les animaux ? \\[0pt]
Compatir \\[0pt]
Composer avec les circonstances \\[0pt]
Conduire sa vie \\[0pt]
Conviction et responsabilité \\[0pt]
Culture et moralité \\[0pt]
Défendre son honneur \\[0pt]
Démériter \\[0pt]
Désintérêt et désintéressement \\[0pt]
Désobéir \\[0pt]
Devant qui sommes-nous responsables ? \\[0pt]
Doit-on bien juger pour bien faire ? \\[0pt]
Doit-on répondre de ce qu'on est devenu ? \\[0pt]
Doit-on toujours dire la vérité ? \\[0pt]
Donner sa parole \\[0pt]
Droit et morale \\[0pt]
Droits et devoirs \\[0pt]
Égoïsme et altruisme \\[0pt]
Égoïsme et individualisme \\[0pt]
Enfance et moralité \\[0pt]
En morale, y a-t-il quelque chose à savoir ? \\[0pt]
Entre l'utile et l'honnête, peut-on choisir ? \\[0pt]
Éprouver sa valeur \\[0pt]
Est-ce pour des raisons morales qu'il faut protéger l'environnement ? \\[0pt]
Esthétisme et moralité \\[0pt]
Est-il mauvais de suivre son désir ? \\[0pt]
Est-il parfois bon de mentir ? \\[0pt]
Est-on propriétaire de son corps ? \\[0pt]
Est-on responsable de ce qu'on n'a pas voulu ? \\[0pt]
Est-on responsable de l'avenir de l'humanité \\[0pt]
Éthique et authenticité \\[0pt]
Être content de soi \\[0pt]
Être en règle avec soi-même \\[0pt]
Être exemplaire \\[0pt]
Être juge et partie \\[0pt]
Être maître de soi \\[0pt]
Être méchant \\[0pt]
Être sans cœur \\[0pt]
Être sans foi ni loi \\[0pt]
Être une personne \\[0pt]
Évaluer \\[0pt]
Expliquer et justifier \\[0pt]
Faire de nécessité vertu \\[0pt]
Faire de sa vie une œuvre d'art \\[0pt]
Faire la morale \\[0pt]
Fait et valeur \\[0pt]
Faut-il être bon ? \\[0pt]
Faut-il expliquer la morale par son utilité ? \\[0pt]
Faut-il mieux vivre comme si nous ne devions jamais mourir ? \\[0pt]
Faut-il prendre soin de soi ? \\[0pt]
Faut-il se délivrer des passions ? \\[0pt]
Fins et moyens \\[0pt]
Fonder la morale \\[0pt]
Fuir le monde \\[0pt]
Garder la mesure \\[0pt]
Haïr \\[0pt]
Identité et responsabilité \\[0pt]
Immoralité et amoralité \\[0pt]
Je ne l'ai pas fait exprès \\[0pt]
Jouir sans entraves \\[0pt]
Juger en conscience \\[0pt]
Jusqu'où peut-on soigner ? \\[0pt]
La beauté morale \\[0pt]
La belle âme \\[0pt]
La bestialité \\[0pt]
La bienfaisance \\[0pt]
La bienveillance \\[0pt]
La bioéthique \\[0pt]
L'abnégation \\[0pt]
La bonne conscience \\[0pt]
La bonne volonté \\[0pt]
La calomnie \\[0pt]
La casuistique \\[0pt]
L'accusation \\[0pt]
La censure \\[0pt]
La charité \\[0pt]
La clémence \\[0pt]
La colère \\[0pt]
La communauté morale \\[0pt]
La confiance \\[0pt]
La connaissance suppose-t-elle une éthique ? \\[0pt]
La conscience est-elle intrinsèquement morale ? \\[0pt]
La conscience morale \\[0pt]
La conscience morale est-elle innée ? \\[0pt]
La contingence \\[0pt]
La contrainte morale \\[0pt]
La corruption \\[0pt]
La crainte \\[0pt]
La cruauté \\[0pt]
La culpabilité \\[0pt]
La culture morale \\[0pt]
La curiosité \\[0pt]
La décision morale \\[0pt]
La délibération \\[0pt]
La délibération en morale \\[0pt]
La déontologie \\[0pt]
La désobéissance \\[0pt]
La détermination \\[0pt]
La dette \\[0pt]
La dignité \\[0pt]
La discipline \\[0pt]
La discipline de vie \\[0pt]
La disposition morale \\[0pt]
L'admiration \\[0pt]
La docilité est-elle un vice ou une vertu ? \\[0pt]
L'adoucissement des mœurs \\[0pt]
La droiture \\[0pt]
La duplicité \\[0pt]
La faiblesse \\[0pt]
La faiblesse de la volonté \\[0pt]
La famille est-elle le lieu de la formation morale ? \\[0pt]
La faute \\[0pt]
La fidélité \\[0pt]
La fidélité à soi \\[0pt]
La force d'âme \\[0pt]
La force de la morale \\[0pt]
La force de l'obligation \\[0pt]
La force est-elle une vertu ? \\[0pt]
La franchise \\[0pt]
La franchise est-elle une vertu ? \\[0pt]
La fraternité est-elle un idéal moral ? \\[0pt]
La générosité \\[0pt]
La gentillesse \\[0pt]
La gloire \\[0pt]
La gratitude \\[0pt]
L'agressivité \\[0pt]
La haine \\[0pt]
La honte \\[0pt]
La jalousie \\[0pt]
La jurisprudence \\[0pt]
La juste colère \\[0pt]
La justice \\[0pt]
La justice est-elle une notion morale ? \\[0pt]
La lâcheté \\[0pt]
La légitimité d'un clonage humain ? \\[0pt]
La libération des mœurs \\[0pt]
La liberté comporte-t-elle des degrés ? \\[0pt]
La liberté, concept moral ? \\[0pt]
La liberté, est-ce l'indépendance à l'égard des passions ? \\[0pt]
La liberté morale \\[0pt]
La loi \\[0pt]
La louange et le blâme \\[0pt]
La loyauté \\[0pt]
L'altruisme \\[0pt]
La magnanimité \\[0pt]
La maîtrise de soi \\[0pt]
La mauvaise conscience \\[0pt]
La mauvaise foi \\[0pt]
La mauvaise volonté \\[0pt]
L'ambition \\[0pt]
La médisance \\[0pt]
La mélancolie \\[0pt]
L'ami et l'ennemi \\[0pt]
La misanthropie \\[0pt]
La miséricorde \\[0pt]
L'amitié \\[0pt]
La modération \\[0pt]
La modération est-elle l'essence de la vertu ? \\[0pt]
La morale a-t-elle besoin d'être fondée ? \\[0pt]
La morale a-t-elle besoin d'un au-delà ? \\[0pt]
La morale commune \\[0pt]
La morale consiste-t-elle à suivre la nature ? \\[0pt]
La morale de l'intérêt \\[0pt]
La morale des fables \\[0pt]
La morale doit-elle en appeler à la nature ? \\[0pt]
La morale doit-elle fournir des préceptes ? \\[0pt]
La morale du citoyen \\[0pt]
La morale du plus fort \\[0pt]
La morale est-elle affaire de jugement ? \\[0pt]
La morale est-elle affaire de sentiment ? \\[0pt]
La morale est-elle affaire de sentiments ? \\[0pt]
La morale est-elle affaire individuelle ? \\[0pt]
La morale est-elle émotion ? \\[0pt]
La morale est-elle ennemie du bonheur ? \\[0pt]
La morale est-elle incompatible avec le déterminisme ? \\[0pt]
La morale est-elle l'ennemie de la vie ? \\[0pt]
La morale est-elle nécessairement répressive ? \\[0pt]
La morale est-elle un art de vivre ? \\[0pt]
La morale est-elle un besoin ? \\[0pt]
La morale est-elle une affaire d'habitude ? \\[0pt]
La morale est-elle une médecine de l'âme ? \\[0pt]
La morale est-elle un fait social ? \\[0pt]
La morale et le droit \\[0pt]
La morale peut-elle être fondée sur la science ? \\[0pt]
La morale peut-elle être naturelle ? \\[0pt]
La morale peut-elle être un calcul ? \\[0pt]
La morale peut-elle être une science ? \\[0pt]
La morale peut-elle se passer d'un fondement religieux ? \\[0pt]
La morale requiert-elle une casuistique ? \\[0pt]
La morale se déduit-elle de la métaphysique ? \\[0pt]
La morale s'enracine-t-elle dans le sens commun ? \\[0pt]
La morale suppose-t-elle le libre arbitre ? \\[0pt]
La moralité n'est-elle que dressage ? \\[0pt]
La moralité réside-t-elle dans l'intention ? \\[0pt]
L'amour \\[0pt]
L'amour de l'humanité \\[0pt]
L'amour de soi \\[0pt]
L'amour et la justice \\[0pt]
L'amour-propre \\[0pt]
La nature du bien \\[0pt]
La nature est-elle digne de respect ? \\[0pt]
La neutralité \\[0pt]
L'angélisme \\[0pt]
L'angoisse \\[0pt]
L'animal peut-il être un sujet moral ? \\[0pt]
La norme et l'écart \\[0pt]
La parole donnée \\[0pt]
L'apathie \\[0pt]
La patience \\[0pt]
La patience est-elle une vertu ? \\[0pt]
La personne \\[0pt]
La perversion \\[0pt]
La perversion morale \\[0pt]
La perversité \\[0pt]
La peur du châtiment \\[0pt]
La pitié \\[0pt]
La pitié est-elle morale ? \\[0pt]
La pitié peut-elle fonder la morale ? \\[0pt]
La politesse \\[0pt]
La politique doit-elle être morale ? \\[0pt]
La probité \\[0pt]
La promesse \\[0pt]
La prudence \\[0pt]
La pudeur \\[0pt]
La racine du mal \\[0pt]
La raison, est-ce la modération ? \\[0pt]
La raison est-elle morale par elle-même ? \\[0pt]
La raison peut-elle être immédiatement pratique ? \\[0pt]
La raison peut-elle être pratique ? \\[0pt]
La recherche du bonheur suffit-elle à déterminer une morale ? \\[0pt]
La réciprocité \\[0pt]
La reconnaissance \\[0pt]
La règle et l'exception \\[0pt]
La réparation \\[0pt]
La résignation \\[0pt]
La résolution \\[0pt]
La responsabilité \\[0pt]
La responsabilité implique-t-elle autrui ? \\[0pt]
L'argumentation morale \\[0pt]
La rigueur morale \\[0pt]
La ruse \\[0pt]
La sagesse \\[0pt]
La sainteté \\[0pt]
La sanction \\[0pt]
La satisfaction des penchants \\[0pt]
L'ascèse \\[0pt]
L'ascétisme \\[0pt]
La science peut-elle guider notre conduite ? \\[0pt]
La séduction \\[0pt]
La servitude \\[0pt]
La servitude humaine \\[0pt]
La sincérité \\[0pt]
La singularité en morale \\[0pt]
La sollicitude \\[0pt]
La souffrance \\[0pt]
La souffrance a-t-elle un sens moral ? \\[0pt]
La sympathie \\[0pt]
La technique est-elle moralement neutre ? \\[0pt]
La tentation \\[0pt]
La terreur morale \\[0pt]
L'athée peut-il être vertueux ? \\[0pt]
La tolérance \\[0pt]
La trahison \\[0pt]
La transgression \\[0pt]
La tristesse \\[0pt]
La tristesse est-elle mauvaise ? \\[0pt]
L'autarcie \\[0pt]
L'authenticité \\[0pt]
L'autonomie \\[0pt]
L'autorité morale \\[0pt]
L'autre est-il le fondement de la conscience morale ? \\[0pt]
La valeur de l'exemple \\[0pt]
La valeur des normes \\[0pt]
La valeur du consensus \\[0pt]
La valeur morale \\[0pt]
La vanité \\[0pt]
La vénalité \\[0pt]
La vénération \\[0pt]
La vengeance \\[0pt]
La véracité \\[0pt]
La vérité est-elle morale ? \\[0pt]
La vertu est-elle utile ? \\[0pt]
La vertu, les vertus \\[0pt]
La vertu peut-elle être purement morale ? \\[0pt]
La vertu peut-elle s'enseigner ? \\[0pt]
L'aveu \\[0pt]
L'aveu diminue-t-il la faute ? \\[0pt]
La vie est-elle la valeur suprême ? \\[0pt]
La vie est-elle le bien suprême ? \\[0pt]
La vie suffit-elle à créer des valeurs ? \\[0pt]
La vigilance \\[0pt]
La voix de la conscience \\[0pt]
La vraie morale se moque-t-elle de la morale ? \\[0pt]
Le bien détermine-t-il la volonté ? \\[0pt]
Le bien est-il conformité à la nature ? \\[0pt]
Le bien est-il l'utile ? \\[0pt]
Le bien et le mal \\[0pt]
Le bien et les biens \\[0pt]
Le bien suppose-t-il la transcendance ? \\[0pt]
Le bon et l'utile \\[0pt]
Le bonheur comporte-t-il des degrés ? \\[0pt]
Le bonheur des uns, le malheur des autres \\[0pt]
Le bonheur est-il une affaire de circonstances ? \\[0pt]
Le bonheur est-il une fin morale ? \\[0pt]
Le bonheur et la vertu \\[0pt]
Le calcul des plaisirs \\[0pt]
Le cas de conscience \\[0pt]
Le catéchisme moral \\[0pt]
Le châtiment \\[0pt]
Le châtiment rend-il meilleur ? \\[0pt]
Le choix \\[0pt]
L'école des vertus \\[0pt]
Le combat contre l'injustice a-t-il une source morale ? \\[0pt]
Le conflit de devoirs \\[0pt]
Le conflit des devoirs \\[0pt]
Le conformisme \\[0pt]
Le conformisme moral \\[0pt]
Le conseil \\[0pt]
Le convenable \\[0pt]
Le corps est-il respectable ? \\[0pt]
Le courage \\[0pt]
Le crime suppose-t-il la loi ? \\[0pt]
Le cynique \\[0pt]
Le cynisme \\[0pt]
Le dandysme \\[0pt]
Le dérèglement \\[0pt]
Le désespoir \\[0pt]
Le désespoir est-il une faute morale ? \\[0pt]
Le désintéressement \\[0pt]
Le devoir d'assistance \\[0pt]
Le devoir de vérité \\[0pt]
Le devoir-être \\[0pt]
Le dévouement \\[0pt]
L'édification morale \\[0pt]
Le don de soi \\[0pt]
Le droit de punir \\[0pt]
Le for intérieur \\[0pt]
Le formalisme moral \\[0pt]
Légalité et moralité \\[0pt]
L'égoïsme \\[0pt]
Le héros moral \\[0pt]
Le jugement moral \\[0pt]
Le juste milieu \\[0pt]
Le libertin, le libertaire \\[0pt]
Le luxe \\[0pt]
Le mal \\[0pt]
Le malheur \\[0pt]
L'embryon humain est-il une personne ? \\[0pt]
L'embryon humain peut-il être l'objet d'expérimentation ? \\[0pt]
Le méchant peut-il être heureux ? \\[0pt]
Le mensonge \\[0pt]
Le mépris \\[0pt]
Le mérite \\[0pt]
Le mérite est-il le critère de la vertu ? \\[0pt]
Le modèle en morale \\[0pt]
Le moindre mal \\[0pt]
Le moralisme \\[0pt]
Le moraliste \\[0pt]
L'émotion morale \\[0pt]
L'empire sur soi \\[0pt]
« L'enfer est pavé de bonnes intentions » \\[0pt]
Le nihilisme \\[0pt]
L'enthousiasme est-il moral ? \\[0pt]
L'entraide \\[0pt]
L'envie \\[0pt]
Le pardon \\[0pt]
Le partage est-il une obligation morale ? \\[0pt]
Le péché \\[0pt]
Le plaisir a-t-il un rôle à jouer dans la morale ? \\[0pt]
Le plaisir est-il immoral ? \\[0pt]
Le plaisir est-il un bien ? \\[0pt]
Le plus grand des maux \\[0pt]
Le pouvoir peut-il ignorer la morale ? \\[0pt]
Le préférable \\[0pt]
Le principe de réciprocité \\[0pt]
Le probabilisme \\[0pt]
Le progrès moral \\[0pt]
L'équité \\[0pt]
Le rapport de l'homme à son milieu a-t-il une dimension morale ? \\[0pt]
Le regret \\[0pt]
Le relativisme moral \\[0pt]
Le remords \\[0pt]
Le renoncement \\[0pt]
Le repentir \\[0pt]
Le respect \\[0pt]
Le ressentiment \\[0pt]
Le rigorisme \\[0pt]
L'erreur et la faute \\[0pt]
Le sacrifice \\[0pt]
Les animaux échappent-ils à la moralité ? \\[0pt]
Les animaux ont-ils des droits ? \\[0pt]
Les autorités \\[0pt]
Les belles actions \\[0pt]
Les bénéfices moraux \\[0pt]
Les bonnes intentions \\[0pt]
Les bonnes mœurs \\[0pt]
Le scandale \\[0pt]
Les caractères moraux \\[0pt]
Les conventions \\[0pt]
Le scrupule \\[0pt]
Les devoirs à l'égard de la nature \\[0pt]
Les devoirs envers soi-même \\[0pt]
Les dilemmes moraux \\[0pt]
Les droits de l'homme ont-ils un fondement moral ? \\[0pt]
Le sens de l'opportunité est-il moral ? \\[0pt]
Le sens moral \\[0pt]
Le sentiment de la faute \\[0pt]
Le sentiment moral \\[0pt]
Les éthiques professionnelles \\[0pt]
Les fins naturelles et les fins morales \\[0pt]
Les hommes n'agissent-ils que par intérêt ? \\[0pt]
Les leçons de morale \\[0pt]
Les limites du plaisir \\[0pt]
Les mauvaises habitudes \\[0pt]
Les mauvaises intentions \\[0pt]
Les mœurs \\[0pt]
Les mœurs sont-ils l'objet d'une science ? \\[0pt]
Le souci d'autrui résume-t-il la morale ? \\[0pt]
Le souci de soi \\[0pt]
Le souverain bien \\[0pt]
Les paroles et les actes \\[0pt]
Les passions peuvent-elles être raisonnables ? \\[0pt]
Les passions sont-elles mauvaises ? \\[0pt]
L'espérance est-elle une vertu ? \\[0pt]
Les plaisirs de l'amitié \\[0pt]
L'espoir \\[0pt]
Les préjugés moraux \\[0pt]
Les principes moraux \\[0pt]
Les proverbes nous instruisent-ils moralement ? \\[0pt]
L'estime \\[0pt]
L'estime de soi \\[0pt]
Le sujet moral \\[0pt]
Les valeurs ont-elles une histoire ? \\[0pt]
Les valeurs ont-elles une origine ? \\[0pt]
L'éthique des plaisirs \\[0pt]
L'éthique est-elle affaire de choix ? \\[0pt]
L'éthique suppose-t-elle la liberté ? \\[0pt]
Le tourment moral \\[0pt]
Le travail est-il une valeur morale ? \\[0pt]
Le vol \\[0pt]
Le volontaire et l'involontaire \\[0pt]
L'excuse \\[0pt]
L'exemple en morale \\[0pt]
L'exigence morale \\[0pt]
L'expérience du mal \\[0pt]
L'expérience morale \\[0pt]
L'habileté \\[0pt]
L'habitude morale \\[0pt]
L'hétéronomie \\[0pt]
L'homicide \\[0pt]
L'homme injuste peut-il être heureux ? \\[0pt]
L'honnêteté \\[0pt]
L'honneur \\[0pt]
L'hospitalité \\[0pt]
L'humiliation \\[0pt]
L'humilité \\[0pt]
L'hypocrisie \\[0pt]
L'hypothèse de l'inconscient \\[0pt]
L'idéal moral est-il vain ? \\[0pt]
L'idée de morale appliquée \\[0pt]
L'idée de rétribution est-elle nécessaire à la morale ? \\[0pt]
L'ignoble \\[0pt]
L'ignorance est-elle la raison du mal ? \\[0pt]
L'ignorance est-elle une excuse ? \\[0pt]
L'imitation a-t-elle une fonction morale ? \\[0pt]
L'impardonnable \\[0pt]
L'impartialité \\[0pt]
L'impératif \\[0pt]
L'indécision \\[0pt]
L'indépendance \\[0pt]
L'indifférence \\[0pt]
L'indignation \\[0pt]
L'indulgence \\[0pt]
L'ingratitude \\[0pt]
L'inhibition \\[0pt]
L'inhumain \\[0pt]
L'iniquité \\[0pt]
L'injonction \\[0pt]
L'injure \\[0pt]
L'injustice \\[0pt]
L'injustifiable \\[0pt]
L'innocence \\[0pt]
L'instruction est-elle facteur de moralité ? \\[0pt]
L'insulte \\[0pt]
L'intempérance \\[0pt]
L'intention morale \\[0pt]
L'intention morale suffit-elle à constituer la valeur morale de l'action ? \\[0pt]
L'interdiction \\[0pt]
L'interdit \\[0pt]
L'intérêt \\[0pt]
L'intérêt peut-il être une valeur morale ? \\[0pt]
L'intolérable \\[0pt]
L'intolérance \\[0pt]
L'intuition morale \\[0pt]
L'involontaire \\[0pt]
L'irrésolution \\[0pt]
L'obéissance \\[0pt]
L'obligation \\[0pt]
L'obligation morale \\[0pt]
L'obligation morale peut-elle se réduire à une obligation sociale ? \\[0pt]
L'obscène \\[0pt]
L'offense \\[0pt]
Loi et commandement \\[0pt]
Loi morale et loi politique \\[0pt]
Loi naturelle et loi morale \\[0pt]
L'opportunisme \\[0pt]
L'ordre moral \\[0pt]
L'orgueil \\[0pt]
L'origine des vertus \\[0pt]
L'oubli des fautes \\[0pt]
L'unité des vices \\[0pt]
L'utilité de la peine \\[0pt]
L'utilité est-elle étrangère à la morale ? \\[0pt]
Médecine et morale \\[0pt]
Mémoire et responsabilité \\[0pt]
Mentir \\[0pt]
Morale et calcul \\[0pt]
Morale et convention \\[0pt]
Morale et dispositions \\[0pt]
Morale et éducation \\[0pt]
Morale et histoire \\[0pt]
Morale et liberté \\[0pt]
Morale et pratique \\[0pt]
Morale et religion \\[0pt]
Morale et science des mœurs \\[0pt]
Morale et sexualité \\[0pt]
Morale et société \\[0pt]
Morale et technique \\[0pt]
Morale et violence \\[0pt]
Mourir dans la dignité \\[0pt]
Mourir pour des principes \\[0pt]
Nature et morale \\[0pt]
Nécessité fait loi \\[0pt]
« Ne fais pas à autrui ce que tu ne voudrais pas qu'on te fasse » \\[0pt]
Ne lèse personne \\[0pt]
Ne penser qu'à soi \\[0pt]
Ne rien devoir à personne \\[0pt]
N'est-on juste que par crainte du châtiment ? \\[0pt]
Normes morales et normes vitales \\[0pt]
Notre ignorance nous excuse-t-elle ? \\[0pt]
Obéir \\[0pt]
« Œil pour œil, dent pour dent » \\[0pt]
Perdre son âme \\[0pt]
Peut-il être moral de tuer ? \\[0pt]
Peut-il y avoir une morale dans les échanges ? \\[0pt]
Peut-on apprendre à vivre ? \\[0pt]
Peut-on concevoir une morale sans sanction ni obligation ? \\[0pt]
Peut-on conclure de l'être au devoir-être ? \\[0pt]
Peut-on définir le bien ? \\[0pt]
Peut-on disposer de son corps ? \\[0pt]
Peut-on être amoral ? \\[0pt]
Peut-on être désintéressé ? \\[0pt]
Peut-on faire ce qu'on ne veut pas ? \\[0pt]
Peut-on faire du dialogue un modèle de relation morale ? \\[0pt]
Peut-on faire le mal en vue du bien ? \\[0pt]
Peut-on inventer en morale ? \\[0pt]
Peut-on ne pas savoir ce que l'on fait ? \\[0pt]
Peut-on opposer morale et technique ? \\[0pt]
Peut-on parler d'un progrès moral ? \\[0pt]
Peut-on parler d'un sens moral ? \\[0pt]
Peut-on parler d'un style moral ? \\[0pt]
Peut-on penser une volonté diabolique ? \\[0pt]
Peut-on prendre le risque de donner la mort en voulant soulager la souffrance ? \\[0pt]
Peut-on reprocher à la morale d'être abstraite ? \\[0pt]
Peut-on s'accorder sur des vérités morales ? \\[0pt]
Peut-on se punir soi-même ? \\[0pt]
Peut-on transiger avec les principes ? \\[0pt]
Peut-on vivre sans valeurs ? \\[0pt]
Peut-on vouloir le bien sans le faire ? \\[0pt]
Peut-on vouloir le mal ? \\[0pt]
Pourquoi châtier ? \\[0pt]
Pourquoi des comités d'éthique ? \\[0pt]
Pourquoi être moral ? \\[0pt]
Pouvons-nous devenir meilleurs ? \\[0pt]
Prendre soin \\[0pt]
Protester \\[0pt]
Punir \\[0pt]
Qu'a-t-on le droit de pardonner ? \\[0pt]
Que doit-on aux morts ? \\[0pt]
Quelles sont les caractéristiques d'une proposition morale ? \\[0pt]
Qu'est-ce que le mal radical ? \\[0pt]
Qu'est-ce qu'être vertueux ? \\[0pt]
Qu'est-ce qui est respectable ? \\[0pt]
Qu'est-ce qui nous oblige ? \\[0pt]
Qu'est-ce qu'un acte moral ? \\[0pt]
Qu'est-ce qu'un ami ? \\[0pt]
Qu'est-ce qu'un cas en morale ? \\[0pt]
Qu'est-ce qu'une idée morale ? \\[0pt]
Qu'est-ce qu'une règle de vie ? \\[0pt]
Qu'est-ce qu'une vie réussie ? \\[0pt]
Qu'est-ce qu'une volonté libre ? \\[0pt]
Qu'est-ce qu'un fait moral ? \\[0pt]
Qu'est-ce qu'un homme juste ? \\[0pt]
Qu'est-ce qu'un idéal moral ? \\[0pt]
Qu'est-ce qu'un sentiment moral ? \\[0pt]
Que vaut en morale la justification par l'utilité ? \\[0pt]
Qui est mon prochain ? \\[0pt]
Qui est une personne ? \\[0pt]
Qui pose les fins ? \\[0pt]
Règle et commandement \\[0pt]
Réprouver \\[0pt]
Respect de la volonté, respect de la vie \\[0pt]
Sans foi ni loi \\[0pt]
Se mentir à soi-même \\[0pt]
Se mettre à la place d'autrui \\[0pt]
S'indigner \\[0pt]
Si tu veux, tu peux \\[0pt]
Sommes-nous libres de nos préférences morales ? \\[0pt]
Sommes-nous toujours dépendants d'autrui ? \\[0pt]
Suffit-il de bien juger pour bien faire ? \\[0pt]
Suffit-il d'être conscient de ses actes pour en être responsable ? \\[0pt]
Suffit-il d'être raisonnable pour être moral ? \\[0pt]
Suffit-il de vouloir pour bien faire ? \\[0pt]
Suicide et homicide \\[0pt]
Suis-je l'auteur de mes actions ? \\[0pt]
Suivre le plus sûr \\[0pt]
Suivre sa nature \\[0pt]
Témoigner \\[0pt]
Théologie et morale \\[0pt]
Tout bien n'est-il qu'un moindre mal ? \\[0pt]
Tout devoir est-il l'envers d'un droit ? \\[0pt]
Toute inégalité est-elle injuste ? \\[0pt]
Toute morale est-elle rationnelle ? \\[0pt]
Toute morale implique-t-elle l'effort ? \\[0pt]
Toutes les fautes se valent-elles ? \\[0pt]
Tout est permis \\[0pt]
Trahir \\[0pt]
Traiter autrui comme une chose \\[0pt]
Tricher \\[0pt]
« Tu ne tueras point » \\[0pt]
Un bien peut-il sortir d'un mal ? \\[0pt]
Une action se juge-t-elle à ses conséquences ? \\[0pt]
Une action vertueuse se reconnaît-elle à sa difficulté ? \\[0pt]
Une éthique sceptique est-elle possible ? \\[0pt]
Une ligne de conduite peut-elle tenir lieu de morale ? \\[0pt]
Une morale du plaisir est-elle concevable ? \\[0pt]
Une morale peut-elle être dépassée ? \\[0pt]
Une morale peut-elle être provisoire ? \\[0pt]
Une morale peut-elle prétendre à l'universalité ? \\[0pt]
Une morale sans Dieu \\[0pt]
Une morale sans obligation est-elle possible ? \\[0pt]
Une politique morale est-elle possible ? \\[0pt]
Une science de la morale est-elle possible ? \\[0pt]
Un mensonge peut-il être légitime ? \\[0pt]
Un témoin de moralité est-il possible ? \\[0pt]
Un vice, est-ce un manque ? \\[0pt]
Utiliser les embryons humains ? \\[0pt]
Vendre son âme \\[0pt]
Vertu et habitude \\[0pt]
Vices privés, vertus publiques \\[0pt]
Vivrait-on mieux sans morale ? \\[0pt]
Vivre sans morale \\[0pt]
Vivre vertueusement \\[0pt]
Voir le meilleur et faire le pire \\[0pt]
Vouloir être heureux \\[0pt]
Vouloir le bien \\[0pt]
Vouloir le mal \\[0pt]
Y a-t-il de l'irréparable ? \\[0pt]
Y a-t-il des actes moralement indifférents ? \\[0pt]
Y a-t-il des démonstrations en morale ? \\[0pt]
Y a-t-il des devoirs envers soi-même ? \\[0pt]
Y a-t-il des faits moraux ? \\[0pt]
Y a-t-il des fins dernières ? \\[0pt]
Y a-t-il des limites à la responsabilité ? \\[0pt]
Y a-t-il des limites proprement morales à la discussion ? \\[0pt]
Y a-t-il des lois morales ? \\[0pt]
Y a-t-il des normes naturelles ? \\[0pt]
Y a-t-il des plaisirs innocents ? \\[0pt]
Y a-t-il des points de vue en morale ? \\[0pt]
Y a-t-il des vérités morales ? \\[0pt]
Y a-t-il des vertus mineures ? \\[0pt]
Y a-t-il place pour l'idée de vérité en morale ? \\[0pt]
Y a-t-il un devoir d'être heureux ? \\[0pt]
Y a-t-il un droit de mourir ? \\[0pt]
Y a-t-il une beauté morale ? \\[0pt]
Y a-t-il une forme morale de fanatisme ? \\[0pt]
Y a-t-il une justice sans morale ? \\[0pt]
Y a-t-il une place pour la morale dans l'économie ? \\[0pt]
Y a-t-il une temporalité proprement morale ? \\[0pt]
Y a-t-il un fondement moral aux circonstances atténuantes ? \\[0pt]
Y a t-il un instinct moral ? \\[0pt]
Y a-t-il un usage moral des passions ? \\[0pt]


\subsection{Politique}
\label{sec:orgdb25f08}

\noindent
Accepter l'injustice \\[0pt]
Amitié et société \\[0pt]
Appliquer la loi \\[0pt]
Apprendre à gouverner \\[0pt]
À quoi bon imaginer la cité idéale ? \\[0pt]
À quoi est-il nécessaire d'obéir ? \\[0pt]
À quoi juger l'action d'un gouvernement ? \\[0pt]
À quoi reconnaît-on qu'une politique est juste ? \\[0pt]
À quoi reconnaît-on un bon gouvernement ? \\[0pt]
À quoi sert la notion de contrat social ? \\[0pt]
À quoi sert la notion d'état de nature ? \\[0pt]
À quoi sert l'État ? \\[0pt]
À quoi servent les constitutions ? \\[0pt]
À quoi servent les élections ? \\[0pt]
Art et politique \\[0pt]
A-t-on des droits contre l'État ? \\[0pt]
A-t-on raison de tenir à la démocratie ? \\[0pt]
Autorité et autoritarisme \\[0pt]
Autorité et pouvoir \\[0pt]
Avoir des ennemis, est-ce le principe de la politique ? \\[0pt]
Avoir le droit \\[0pt]
Avons-nous besoin de partis politiques ? \\[0pt]
Avons-nous besoin de traditions ? \\[0pt]
Avons-nous besoin d'être gouvernés ? \\[0pt]
Avons-nous besoin d'un chef ? \\[0pt]
Avons-nous besoin d'une philosophie politique ? \\[0pt]
Avons-nous besoin d'utopies ? \\[0pt]
Ceux qui savent doivent-ils gouverner ? \\[0pt]
Cité juste ou citoyen juste ? \\[0pt]
Citoyen et soldat \\[0pt]
Citoyenneté et solidarité \\[0pt]
Citoyens du monde \\[0pt]
Commémorer \\[0pt]
Comment décider, sinon à la majorité ? \\[0pt]
Comment devient-on citoyen ? \\[0pt]
Comment juger de la politique ? \\[0pt]
Comment ne pas être libéral ? \\[0pt]
Comment penser un pouvoir qui ne corrompe pas ? \\[0pt]
Communauté et conflit \\[0pt]
Communauté et société \\[0pt]
Connaissance historique et action politique \\[0pt]
Conseiller le prince \\[0pt]
Conservatisme et tradition \\[0pt]
Constitution et lois \\[0pt]
Consumérisme et démocratie \\[0pt]
Contrainte et désobéissance \\[0pt]
Crime et châtiment \\[0pt]
Critiquer la démocratie \\[0pt]
Débattre \\[0pt]
Défendre ses droits \\[0pt]
Démocrates et démagogues \\[0pt]
Démocratie ancienne et démocratie moderne \\[0pt]
Démocratie et anarchie \\[0pt]
Démocratie et démagogie \\[0pt]
Démocratie et impérialisme \\[0pt]
Démocratie et représentation \\[0pt]
Démocratie et république \\[0pt]
Démocratie et transparence \\[0pt]
De quoi la politique s'occupe-t-elle ? \\[0pt]
De quoi l'État doit-il être propriétaire ? \\[0pt]
Désir et politique \\[0pt]
Des nations peuvent-elles former une société ? \\[0pt]
Désobéir aux lois \\[0pt]
Désobéissance et résistance \\[0pt]
Des sociétés sans État sont-elles des sociétés politiques ? \\[0pt]
Diviser pour mieux régner \\[0pt]
Division du travail et cohésion sociale \\[0pt]
Doit-on préférer l'injustice au désordre ? \\[0pt]
D'où la politique tire-t-elle sa légitimité ? \\[0pt]
Droit naturel et loi naturelle \\[0pt]
Droits de l'homme, droits du citoyen \\[0pt]
Droits de l'homme et droits du citoyen \\[0pt]
Droits et devoirs sont-ils réciproques ? \\[0pt]
Droits et obligations \\[0pt]
Économie et politique \\[0pt]
Éducation et citoyenneté \\[0pt]
Éduquer le citoyen \\[0pt]
Égalité des droits, égalité des conditions \\[0pt]
Égalité et liberté \\[0pt]
En politique, ne fait-on que défendre ses intérêts ? \\[0pt]
En politique n'y a-t-il que des rapports de force ? \\[0pt]
En politique, peut-on faire table rase du passé ? \\[0pt]
En politique, y a-t-il des modèles ? \\[0pt]
En quel sens peut-on dire de l'écologie qu'elle est politique ? \\[0pt]
En quoi une discussion est-elle politique ? \\[0pt]
Espace public et vie privée \\[0pt]
Est-il bon qu'un seul commande ? \\[0pt]
Est-il possible d'être neutre politiquement ? \\[0pt]
État et nation \\[0pt]
État et vie privée \\[0pt]
Être apolitique \\[0pt]
Être citoyen du monde \\[0pt]
Être dans l'opposition \\[0pt]
Être dans son droit \\[0pt]
Être démocrate \\[0pt]
Être égaux \\[0pt]
Être hors-la-loi \\[0pt]
Être sans foi ni loi \\[0pt]
Existe-t-il des violences légitimes ? \\[0pt]
Existe-t-il un bien commun qui soit la norme de la vie politique ? \\[0pt]
Faire de la politique \\[0pt]
Faire la loi \\[0pt]
Faire la paix \\[0pt]
Faire société \\[0pt]
Faut-il considérer le droit pénal comme instituant une violence légitime ? \\[0pt]
Faut-il craindre la démocratie ? \\[0pt]
Faut-il craindre la révolution ? \\[0pt]
Faut-il craindre le pouvoir souverain ? \\[0pt]
Faut-il craindre les foules ? \\[0pt]
Faut-il détruire l'État ? \\[0pt]
Faut-il diriger l'économie ? \\[0pt]
Faut-il être réaliste en politique ? \\[0pt]
Faut-il fuir la politique ? \\[0pt]
Faut-il limiter la souveraineté ? \\[0pt]
Faut-il limiter l'exercice de la puissance publique ? \\[0pt]
Faut-il opposer à la politique la souveraineté du droit ? \\[0pt]
Faut-il penser l'État comme un corps ? \\[0pt]
Faut-il préférer l'injustice au désordre ? \\[0pt]
Faut-il préférer une injustice au désordre ? \\[0pt]
Faut-il se méfier du volontarisme politique ? \\[0pt]
Faut-il tolérer les intolérants ? \\[0pt]
Faut-il vouloir changer le monde ? \\[0pt]
Faut-il vouloir la paix ? \\[0pt]
Féminité et citoyenneté \\[0pt]
Fonder une cité \\[0pt]
Gouverner \\[0pt]
Gouverner, administrer, gérer \\[0pt]
Gouverner, est-ce prévoir ? \\[0pt]
Gouverner et se gouverner \\[0pt]
Grève et politique \\[0pt]
Groupe, classe, société \\[0pt]
Guerre et paix \\[0pt]
Guerre et politique \\[0pt]
Idéologie et utopie \\[0pt]
Imaginaire et politique \\[0pt]
Justice et équité \\[0pt]
Justice et fiscalité \\[0pt]
L'abstention \\[0pt]
L'abus de pouvoir \\[0pt]
La chose publique \\[0pt]
La cité idéale \\[0pt]
La citoyenneté \\[0pt]
La civilité \\[0pt]
La clause de conscience \\[0pt]
La cohésion nationale \\[0pt]
La comédie du pouvoir \\[0pt]
La communauté internationale \\[0pt]
La communication est-elle nécessaire à la démocratie ? \\[0pt]
La compassion risque-t-elle d'abolir l'exigence politique ? \\[0pt]
La compétence politique \\[0pt]
La compétence technique peut-elle fonder l'autorité publique ? \\[0pt]
La conflictualité politique est-elle indépassable ? \\[0pt]
La conscience politique \\[0pt]
La constitution \\[0pt]
La contestation \\[0pt]
La contrôle social \\[0pt]
La corruption \\[0pt]
La corruption politique \\[0pt]
La cruauté politique \\[0pt]
L'action politique \\[0pt]
L'action politique a-t-elle un fondement rationnel ? \\[0pt]
L'action politique peut-elle se passer de mots ? \\[0pt]
La culture démocratique \\[0pt]
La culture est-elle affaire de politique ? \\[0pt]
La décision politique \\[0pt]
La défense nationale \\[0pt]
La délibération politique \\[0pt]
La démagogie \\[0pt]
La démocratie conduit-elle au règne de l'opinion ? \\[0pt]
La démocratie est-elle le pire des régimes politiques ? \\[0pt]
La démocratie est-elle moyen ou fin ? \\[0pt]
La démocratie est-elle nécessairement libérale ? \\[0pt]
La démocratie est-elle possible ? \\[0pt]
La démocratie n'est-elle que la force des faibles ? \\[0pt]
La démocratie participative \\[0pt]
La désobéissance civile \\[0pt]
La dictature \\[0pt]
La discrimination \\[0pt]
La division des pouvoirs \\[0pt]
La domination \\[0pt]
La droit de conquête \\[0pt]
La fête nationale \\[0pt]
La fin de la politique \\[0pt]
La fin de la politique est-elle l'établissement de la justice ? \\[0pt]
La fin de l'État \\[0pt]
La fin justifie-t-elle les moyens ? \\[0pt]
La fonction première de l'État est-elle de durer ? \\[0pt]
La force de la loi \\[0pt]
La force des institutions \\[0pt]
La force des lois \\[0pt]
La force du pouvoir \\[0pt]
La force et le droit \\[0pt]
La force fait-elle le droit ? \\[0pt]
La formation des citoyens \\[0pt]
La fraternité \\[0pt]
La fraternité a-t-elle un sens politique ? \\[0pt]
La géopolitique \\[0pt]
La grève \\[0pt]
La guerre civile \\[0pt]
La guerre est-elle la continuation de la politique ? \\[0pt]
La guerre est-elle la continuation de la politique par d'autres moyens ? \\[0pt]
La guerre et la paix \\[0pt]
La guerre juste \\[0pt]
La guerre totale \\[0pt]
La hiérarchie \\[0pt]
La justice consiste-t-elle à traiter tout le monde de la même manière ? \\[0pt]
La justice entre les générations \\[0pt]
La justice : moyen ou fin de la politique ? \\[0pt]
La justice sociale \\[0pt]
La laïcité \\[0pt]
La légitimité \\[0pt]
La liberté civile \\[0pt]
La liberté de culte \\[0pt]
La liberté des citoyens \\[0pt]
La liberté d'expression \\[0pt]
La liberté d'opinion \\[0pt]
La liberté est-elle menacée par l'égalité ? \\[0pt]
La liberté et la loi \\[0pt]
La liberté politique \\[0pt]
La limite du politique \\[0pt]
La loi \\[0pt]
La loi et le contrat \\[0pt]
La loi et le règlement \\[0pt]
La loyauté \\[0pt]
La lutte des classes \\[0pt]
La majorité a-t-elle toujours raison ? \\[0pt]
La majorité peut-elle être tyrannique ? \\[0pt]
L'ambition politique \\[0pt]
La meilleure constitution \\[0pt]
L'ami et l'ennemi \\[0pt]
L'amitié est-elle un principe politique ? \\[0pt]
La modération est-elle une vertu politique ? \\[0pt]
La morale politique \\[0pt]
L'amour des lois \\[0pt]
L'anarchie \\[0pt]
La nation \\[0pt]
La nation et l'État \\[0pt]
La neutralité de l'État \\[0pt]
L'animal politique \\[0pt]
La notion de progrès a-t-elle un sens en politique ? \\[0pt]
La notion de sujet en politique \\[0pt]
La paix \\[0pt]
La paix civile \\[0pt]
La paix est-elle la visée de la politique ? \\[0pt]
La paix est-elle possible ? \\[0pt]
La paix n'est-elle que l'absence de guerre ? \\[0pt]
La paix n'est-elle qu'un idéal ? \\[0pt]
La paix perpétuelle \\[0pt]
La paix sociale est-elle une fin en soi ? \\[0pt]
La parole politique \\[0pt]
La parole publique \\[0pt]
La participation des citoyens \\[0pt]
La participation politique \\[0pt]
L'apatride \\[0pt]
La patrie \\[0pt]
La pauvreté \\[0pt]
La peur du désordre \\[0pt]
La pluralité des opinions \\[0pt]
La pluralité des pouvoirs \\[0pt]
La police et l'armée \\[0pt]
La politique a-t-elle besoin de héros ? \\[0pt]
La politique a-t-elle besoin de la fiction ? \\[0pt]
La politique a-t-elle besoin de la religion ? \\[0pt]
La politique a-t-elle besoin de modèles ? \\[0pt]
La politique a-t-elle besoin d'experts ? \\[0pt]
La politique a-t-elle pour fin d'éliminer la violence ? \\[0pt]
La politique consiste-t-elle à faire des compromis ? \\[0pt]
La politique de la santé \\[0pt]
La politique doit-elle être rationnelle ? \\[0pt]
La politique doit-elle se mêler de l'art ? \\[0pt]
La politique doit-elle viser la concorde ? \\[0pt]
La politique doit-elle viser le consensus ? \\[0pt]
La politique du pire \\[0pt]
La politique échappe-t-elle à l'exigence de vérité ? \\[0pt]
La politique est-elle affaire de décision ? \\[0pt]
La politique est-elle affaire de jugement ? \\[0pt]
La politique est-elle architectonique ? \\[0pt]
La politique est-elle continuation de la guerre, par d'autres moyens ? \\[0pt]
La politique est-elle la continuation de la guerre ? \\[0pt]
La politique est-elle la continuation de la guerre par d'autres moyens ? \\[0pt]
La politique est-elle l'affaire de tous ? \\[0pt]
La politique est-elle l'art de la délibération ? \\[0pt]
La politique est-elle l'art des possibles ? \\[0pt]
La politique est-elle l'art du possible ? \\[0pt]
La politique est-elle le refus de la violence ? \\[0pt]
La politique est-elle le règne des passions ? \\[0pt]
La politique est-elle naturelle ? \\[0pt]
La politique est-elle par nature sujette à dispute ? \\[0pt]
La politique est-elle plus importante que tout ? \\[0pt]
La politique est-elle un art ? \\[0pt]
La politique est-elle une science ? \\[0pt]
La politique est-elle une technique ? \\[0pt]
La politique est-elle un métier ? \\[0pt]
La politique est-elle un moyen ou une fin ? \\[0pt]
La politique et la gloire \\[0pt]
La politique et la ville \\[0pt]
La politique et le mal \\[0pt]
La politique et le politique \\[0pt]
La politique et l'opinion \\[0pt]
La politique exclut-elle le désordre ? \\[0pt]
La politique implique-t-elle la hiérarchie ? \\[0pt]
La politique peut-elle changer la société ? \\[0pt]
La politique peut-elle changer le monde ? \\[0pt]
La politique peut-elle disparaître ? \\[0pt]
La politique peut-elle être indépendante de la morale ? \\[0pt]
La politique peut-elle être morale ? \\[0pt]
La politique peut-elle être objet de science ? \\[0pt]
La politique peut-elle être un objet de science ? \\[0pt]
La politique peut-elle n'être qu'une pratique ? \\[0pt]
La politique peut-elle se passer de croyances ? \\[0pt]
La politique peut-elle se passer des idéologies ? \\[0pt]
La politique peut-elle unir les hommes ? \\[0pt]
La politique peut-elle viser à autre chose qu'à l'efficacité ? \\[0pt]
La politique repose-t-elle sur un contrat ? \\[0pt]
La politique requière-t-elle le compromis \\[0pt]
La politique se réduit-elle à des rapports de force ? \\[0pt]
La politique suppose-t-elle la morale ? \\[0pt]
La politique suppose-t-elle une idée de l'homme ? \\[0pt]
La politique vise-t-elle à dépasser les injustices ? \\[0pt]
L'apolitisme \\[0pt]
La populace \\[0pt]
La prise de parti est-elle essentielle en politique ? \\[0pt]
La propagande \\[0pt]
La propriété \\[0pt]
La propriété est-elle une garantie de liberté ? \\[0pt]
La protection sociale \\[0pt]
La prudence \\[0pt]
La prudence est-elle une vertu politique ? \\[0pt]
La raison d'État \\[0pt]
La rationalité des choix politiques \\[0pt]
La rationalité politique \\[0pt]
L'arbitraire en politique \\[0pt]
La réaction en politique \\[0pt]
La réciprocité est-elle indispensable à la communauté politique ? \\[0pt]
La réforme des institutions \\[0pt]
La religion \\[0pt]
La religion civile \\[0pt]
La religion peut-elle faire lien social ? \\[0pt]
La représentation en politique \\[0pt]
La représentation politique \\[0pt]
La république \\[0pt]
La résistance à l'oppression \\[0pt]
La responsabilité politique \\[0pt]
La révolte \\[0pt]
L'aristocratie \\[0pt]
L'art de gouverner \\[0pt]
L'art de la guerre \\[0pt]
L'art politique \\[0pt]
La science est-elle une affaire politique ? \\[0pt]
La science politique \\[0pt]
La sécurité \\[0pt]
La sécurité nationale \\[0pt]
La sécurité publique \\[0pt]
La séduction en politique \\[0pt]
La ségrégation \\[0pt]
La séparation des pouvoirs \\[0pt]
La servitude volontaire \\[0pt]
La société civile \\[0pt]
La société civile et l'État \\[0pt]
La société des nations \\[0pt]
La société et l'État \\[0pt]
La société peut-elle se passer de l'État ? \\[0pt]
La solidarité \\[0pt]
La solitude constitue-t-elle un obstacle à la citoyenneté ? \\[0pt]
La souveraineté \\[0pt]
La souveraineté de l'État \\[0pt]
La souveraineté du peuple \\[0pt]
La souveraineté nationale \\[0pt]
La souveraineté peut-elle se partager ? \\[0pt]
La souveraineté populaire \\[0pt]
La sphère privée échappe-t-elle au politique ? \\[0pt]
L'assassinat politique \\[0pt]
La sûreté \\[0pt]
La surveillance de la société \\[0pt]
La technocratie \\[0pt]
La terreur \\[0pt]
La tolérance \\[0pt]
La tolérance envers les intolérants \\[0pt]
La tolérance est-elle un concept politique ? \\[0pt]
La tolérance peut-elle constituer un problème pour la démocratie ? \\[0pt]
La totalitarisme \\[0pt]
La transparence est-elle un idéal démocratique ? \\[0pt]
La tyrannie \\[0pt]
La tyrannie de la majorité \\[0pt]
L'audace politique \\[0pt]
L'autorité \\[0pt]
L'autorité de la loi \\[0pt]
L'autorité de l'État \\[0pt]
L'autorité politique \\[0pt]
La valeur du consentement \\[0pt]
La vertu de l'homme politique \\[0pt]
La vertu politique \\[0pt]
La vie collective est-elle nécessairement frustrante ? \\[0pt]
La vie politique \\[0pt]
La vie politique est-elle aliénante ? \\[0pt]
La vie privée \\[0pt]
La violence de l'État \\[0pt]
La violence politique \\[0pt]
La violence révolutionnaire \\[0pt]
La volonté constitue-t-elle le principe de la politique ? \\[0pt]
La volonté générale \\[0pt]
La volonté peut-elle être collective ? \\[0pt]
La volonté politique peut-elle être commune ? \\[0pt]
Le bien commun \\[0pt]
Le bien public \\[0pt]
Le bonheur des citoyens est-il un idéal politique ? \\[0pt]
Le bonheur est-il une fin politique ? \\[0pt]
Le bonheur est-il un principe politique ? \\[0pt]
Le bourgeois et le citoyen \\[0pt]
Le charisme en politique \\[0pt]
Le charisme politique \\[0pt]
Le citoyen \\[0pt]
Le citoyen peut-il être à la fois libre et soumis à l'État ? \\[0pt]
Le civisme \\[0pt]
L'écologie est-elle un problème politique ? \\[0pt]
L'écologie politique \\[0pt]
Le commerce est-il pacificateur ? \\[0pt]
Le commun \\[0pt]
Le compromis \\[0pt]
Le conflit est-il constitutif de la politique ? \\[0pt]
Le conflit est-il la raison d'être de la politique ? \\[0pt]
L'économie est-elle politique ? \\[0pt]
L'économie politique \\[0pt]
Le conseiller du prince \\[0pt]
Le consensus \\[0pt]
Le consentement des gouvernés \\[0pt]
Le consentement du peuple \\[0pt]
Le corps politique \\[0pt]
Le cosmopolitisme \\[0pt]
Le cosmopolitisme peut-il devenir réalité ? \\[0pt]
Le coup d'État \\[0pt]
Le courage politique \\[0pt]
Le débat politique \\[0pt]
Le despote peut-il être éclairé ? \\[0pt]
Le despotisme \\[0pt]
Le despotisme peut-il être éclairé ? \\[0pt]
Le devoir d'obéissance \\[0pt]
Le discours politique \\[0pt]
Le droit au bonheur \\[0pt]
Le droit au Bonheur \\[0pt]
Le droit d'asile \\[0pt]
Le droit de guerre \\[0pt]
Le droit de la guerre \\[0pt]
Le droit de propriété \\[0pt]
Le droit de punir \\[0pt]
Le droit de résistance à l'oppression \\[0pt]
Le droit de révolte \\[0pt]
Le droit des gens \\[0pt]
Le droit des minorités \\[0pt]
Le droit des peuples à disposer d'eux-mêmes \\[0pt]
Le droit de vie et de mort \\[0pt]
Le droit d'ingérence \\[0pt]
Le droit doit-il être le seul régulateur de la vie sociale ? \\[0pt]
Le droit du plus fort \\[0pt]
Le droit du premier occupant \\[0pt]
Le droit et la violence \\[0pt]
Le droit humanitaire \\[0pt]
L'éducation civique \\[0pt]
L'éducation est-elle affaire de politique ? \\[0pt]
L'éducation politique \\[0pt]
Le fait du prince \\[0pt]
Le fanatisme \\[0pt]
Le fétichisme \\[0pt]
Le fonctionnaire \\[0pt]
L'égalité civile \\[0pt]
L'égalité des chances \\[0pt]
L'égalité des citoyens \\[0pt]
L'égalité des conditions \\[0pt]
L'égalité des hommes et des femmes est-elle une question politique ? \\[0pt]
Légalité et légitimité \\[0pt]
Légitimité et légalité \\[0pt]
Le gouvernement des experts \\[0pt]
Le gouvernement des hommes et l'administration des choses \\[0pt]
Le gouvernement des hommes libres \\[0pt]
Le gouvernement des juges \\[0pt]
Le gouvernement des meilleurs \\[0pt]
Le hors-la-loi \\[0pt]
Le jugement politique \\[0pt]
Le juste et le bien \\[0pt]
Le législateur \\[0pt]
Le lien politique \\[0pt]
Le lien social \\[0pt]
Le lien social peut-il être compassionnel ? \\[0pt]
Le lien social relève-t-il de la politique ? \\[0pt]
Le maintien de l'ordre \\[0pt]
L'émancipation des femmes \\[0pt]
Le manifeste politique \\[0pt]
Le meilleur régime \\[0pt]
Le meilleur régime politique \\[0pt]
Le mensonge politique \\[0pt]
Le métier de politique \\[0pt]
Le métier du politique \\[0pt]
Le monde politique \\[0pt]
Le monopole de la violence légitime \\[0pt]
L'empire \\[0pt]
Le multiculturalisme \\[0pt]
L'engagement politique \\[0pt]
L'ennemi intérieur \\[0pt]
Le pacifisme \\[0pt]
Le patriotisme \\[0pt]
Le peuple a-t-il toujours raison ? \\[0pt]
Le peuple et la nation \\[0pt]
Le peuple et les élites \\[0pt]
Le peuple possède-t-il une volonté ? \\[0pt]
Le philosophe a-t-il des leçons à donner au politique ? \\[0pt]
Le philosophe est-il le vrai politique ? \\[0pt]
Le philosophe-roi \\[0pt]
Le pluralisme politique \\[0pt]
Le poids du préjugé en politique \\[0pt]
Le politique a-t-il à régler les passions ? \\[0pt]
Le politique a-t-il à régler les passions humaines ? \\[0pt]
Le politique doit-il être un technicien ? \\[0pt]
Le politique doit-il se soucier des émotions ? \\[0pt]
Le politique et le religieux \\[0pt]
Le politique et le social \\[0pt]
Le politique peut-il faire abstraction de la morale ? \\[0pt]
Le populisme \\[0pt]
Le pouvoir absolu \\[0pt]
Le pouvoir corrompt-il ? \\[0pt]
Le pouvoir corrompt-il nécessairement ? \\[0pt]
Le pouvoir de la minorité \\[0pt]
Le pouvoir de l'opinion \\[0pt]
Le pouvoir du peuple \\[0pt]
Le pouvoir législatif \\[0pt]
Le pouvoir peut-il arrêter le pouvoir ? \\[0pt]
Le pouvoir peut-il limiter le pouvoir ? \\[0pt]
Le pouvoir peut-il se déléguer ? \\[0pt]
Le pouvoir peut-il se passer de sa mise en scène ? \\[0pt]
Le pouvoir politique est-il nécessairement coercitif ? \\[0pt]
Le pouvoir politique repose-t-il sur la contrainte ? \\[0pt]
Le pouvoir politique repose-t-il sur un savoir ? \\[0pt]
Le pouvoir souverain \\[0pt]
Le pouvoir tend-il toujours à l'excès ? \\[0pt]
Le premier devoir de l'État est-il de se défendre ? \\[0pt]
L'épreuve du pouvoir \\[0pt]
Le prince \\[0pt]
Le principe d'égalité \\[0pt]
Le principe de précaution \\[0pt]
Le projet politique \\[0pt]
Le protectionnisme \\[0pt]
Le public et le privé \\[0pt]
Le réalisme politique \\[0pt]
Le récit national \\[0pt]
Le règlement politique des conflits ? \\[0pt]
Le respect des institutions \\[0pt]
L'erreur politique, la faute politique \\[0pt]
Les affaires publiques \\[0pt]
Le salut du peuple \\[0pt]
Le savant et le politique \\[0pt]
Le savant, l'artiste et le politique \\[0pt]
Le savoir utile au citoyen \\[0pt]
Les biens communs \\[0pt]
L'esclavage \\[0pt]
Les conditions de la démocratie \\[0pt]
Les conflits politiques \\[0pt]
Les conflits politiques ne sont-ils que des conflits sociaux ? \\[0pt]
Les conflits sociaux sont-ils des conflits politiques ? \\[0pt]
Les croyances politiques \\[0pt]
Les décisions politiques peuvent-elles être collectives ? \\[0pt]
Les désaccords politiques peuvent-ils être surmontés ? \\[0pt]
Les devoirs de l'État \\[0pt]
Les droits de l'homme \\[0pt]
Les droits de l'homme et ceux du citoyen \\[0pt]
Les droits de l'homme sont-ils ceux du citoyen ? \\[0pt]
Les droits de l'homme sont-ils une abstraction ? \\[0pt]
Les droits et les devoirs \\[0pt]
Les droits naturels imposent-ils une limite à la politique ? \\[0pt]
Les droits sociaux \\[0pt]
Le secret d'État \\[0pt]
Les émotions politiques \\[0pt]
Le sens de l'État \\[0pt]
Le service public \\[0pt]
Les factions politiques \\[0pt]
Les fondements de l'État \\[0pt]
Les forces de l'ordre \\[0pt]
Les forts et les faibles \\[0pt]
Les frontières \\[0pt]
Les grands hommes \\[0pt]
Les hommes de pouvoir \\[0pt]
Les hommes sont-ils naturellement sociables ? \\[0pt]
Les idées politiques \\[0pt]
Le silence des lois \\[0pt]
Les inégalités sociales \\[0pt]
Les institutions politiques : qu'ont-elles en propre ? \\[0pt]
Les intérêts particuliers peuvent-ils tempérer l'autorité politique ? \\[0pt]
Les jeux du pouvoir \\[0pt]
Les libertés civiles \\[0pt]
Les libertés fondamentales \\[0pt]
Les libertés politiques \\[0pt]
Les lieux du pouvoir \\[0pt]
Les limites de la démocratie \\[0pt]
Les limites de l'État \\[0pt]
Les limites du pouvoir \\[0pt]
Les limites du pouvoir politique \\[0pt]
Les lois et les mœurs \\[0pt]
Les lois sont-elles seulement utiles ? \\[0pt]
Les luttes politiques \\[0pt]
Les moyens de l'autorité \\[0pt]
Le social et le politique \\[0pt]
Les opinions politiques \\[0pt]
Le souci de l'ordre \\[0pt]
Le souci du bien-être est-il politique ? \\[0pt]
L'espace public \\[0pt]
Les partis politiques \\[0pt]
Les passions politiques \\[0pt]
Les pauvres \\[0pt]
Les peuples ont-ils les gouvernements qu'ils méritent ? \\[0pt]
Les politiques publiques \\[0pt]
Les problèmes politiques peuvent-ils se ramener à des problèmes techniques ? \\[0pt]
Les problèmes politiques sont-ils des problèmes techniques ? \\[0pt]
Les règles d'un bon gouvernement \\[0pt]
Les riches et les pauvres \\[0pt]
Les sans-papiers \\[0pt]
Les services publics \\[0pt]
Les symboles du pouvoir \\[0pt]
Les valeurs de la République \\[0pt]
Les vertus politiques \\[0pt]
L'État a-t-il pour finalité de maintenir l'ordre ? \\[0pt]
L'État a-t-il un fondement rationnel ? \\[0pt]
« L'État, c'est moi » \\[0pt]
L'État crée-t-il la liberté ? \\[0pt]
L'État de droit \\[0pt]
L'état de nature \\[0pt]
L'état d'exception \\[0pt]
L'État doit-il disparaître ? \\[0pt]
L'État doit-il éduquer le citoyen ? \\[0pt]
L'État doit-il éduquer les citoyens ? \\[0pt]
L'État doit-il faire le bonheur des citoyens ? \\[0pt]
L'État est-il ennemi de la liberté ? \\[0pt]
L'État est-il fin ou moyen ? \\[0pt]
L'État est-il le garant de la propriété privée ? \\[0pt]
L'État et la culture \\[0pt]
L'État et la Nation \\[0pt]
L'État et la violence \\[0pt]
L'État et le marché \\[0pt]
L'État libéral \\[0pt]
L'État peut-il avoir tous les droits ? \\[0pt]
L'État peut-il créer la liberté ? \\[0pt]
L'État peut-il disparaître ? \\[0pt]
L'État peut-il être indifférent à la religion ? \\[0pt]
L'État-Providence \\[0pt]
L'État universel \\[0pt]
Le temps politique \\[0pt]
Le territoire \\[0pt]
Le totalitarisme \\[0pt]
L'étranger \\[0pt]
Le travail \\[0pt]
Le vainqueur a-t-il tous les droits ? \\[0pt]
Le vivant est-il l'affaire du politique ? \\[0pt]
L'exclusion \\[0pt]
L'exercice du pouvoir \\[0pt]
L'exercice du pouvoir est-il nécessairement solitaire ? \\[0pt]
L'exercice solitaire du pouvoir \\[0pt]
L'existence de l'État dépend-elle d'un contrat ? \\[0pt]
L'expertise politique \\[0pt]
L'exploitation de l'homme par l'homme \\[0pt]
L'expression des citoyens dans le vote suffit-elle à faire vivre la démocratie ? \\[0pt]
L'hégémonie politique \\[0pt]
L'histoire est-elle utile à la politique ? \\[0pt]
L'homme des droits de l'homme n'est-il qu'une fiction ? \\[0pt]
L'homme d'État \\[0pt]
L'homme est-il un animal politique ? \\[0pt]
L'homme et le citoyen \\[0pt]
L'homme, le citoyen, le soldat \\[0pt]
L'honneur \\[0pt]
L'hospitalité a-t-elle un sens politique ? \\[0pt]
L'hospitalité est-elle une vertu politique ? \\[0pt]
Liberté, égalité, fraternité \\[0pt]
Liberté, égalité, solidarité \\[0pt]
Liberté réelle, liberté formelle \\[0pt]
Libertés publiques et culture politique \\[0pt]
L'idée de communauté \\[0pt]
L'idée de contrat social \\[0pt]
L'idée de domination \\[0pt]
L'idée de guerre juste \\[0pt]
L'idée de nation \\[0pt]
L'idée de république \\[0pt]
L'idée de révolution \\[0pt]
L'imaginaire politique \\[0pt]
L'imagination politique \\[0pt]
L'impartialité \\[0pt]
L'impuissance de l'État \\[0pt]
L'incarnation du pouvoir \\[0pt]
L'individualisme a-t-il sa place en politique ? \\[0pt]
L'individualisme rend-il la politique impossible ? \\[0pt]
L'individu a-t-il des droits ? \\[0pt]
L'injustice de la loi \\[0pt]
L'insociable sociabilité \\[0pt]
L'insoumission \\[0pt]
L'insurrection \\[0pt]
L'intégrité \\[0pt]
L'intelligence politique \\[0pt]
L'intérêt commun \\[0pt]
L'intérêt général \\[0pt]
L'intérêt général est-il le bien commun ? \\[0pt]
L'intérêt général est-il un mythe ? \\[0pt]
L'intérêt public est-il une illusion ? \\[0pt]
L'interprétation de la loi \\[0pt]
L'intime est-il politique ? \\[0pt]
L'intolérance \\[0pt]
L'irrationnel et le politique \\[0pt]
L'obéissance \\[0pt]
L'objet de la politique \\[0pt]
Loi naturelle et loi politique \\[0pt]
L'oligarchie \\[0pt]
L'opinion du citoyen \\[0pt]
L'opinion publique \\[0pt]
L'opposant \\[0pt]
L'ordre politique \\[0pt]
L'ordre politique peut-il exclure la violence ? \\[0pt]
L'ordre public \\[0pt]
L'unanimité est-elle un critère de légitimité ? \\[0pt]
L'unité du corps politique \\[0pt]
L'usure du pouvoir \\[0pt]
L'utilité publique \\[0pt]
L'utopie a-t-elle une signification politique ? \\[0pt]
L'utopie en politique \\[0pt]
Mensonge et politique \\[0pt]
Mœurs, coutumes, lois \\[0pt]
Morale et politique sont-elles indépendantes ? \\[0pt]
Mourir pour la patrie \\[0pt]
Murs et frontières \\[0pt]
Nation et richesse \\[0pt]
« Ni Dieu, ni maître ! » \\[0pt]
Ni Dieu ni maître \\[0pt]
Ni Dieu, ni maître \\[0pt]
Nul n'est censé ignorer la loi \\[0pt]
Obéir, est-ce se soumettre ? \\[0pt]
Obéissance et esclavage \\[0pt]
Peuple et société \\[0pt]
Peuples et masses \\[0pt]
Peut-il y avoir de la politique sans conflit ? \\[0pt]
Peut-il y avoir un droit à désobéir ? \\[0pt]
Peut-il y avoir une philosophie politique sans Dieu ? \\[0pt]
Peut-il y avoir une politique de la langue ? \\[0pt]
Peut-il y avoir une science politique ? \\[0pt]
Peut-il y avoir une servitude volontaire ? \\[0pt]
Peut-il y avoir une société des nations ? \\[0pt]
Peut-il y avoir une société sans État ? \\[0pt]
Peut-il y avoir une vérité en politique ? \\[0pt]
Peut-on admettre un droit à la révolte ? \\[0pt]
Peut-on concevoir une société qui n'aurait plus besoin du droit ? \\[0pt]
Peut-on concevoir un État mondial ? \\[0pt]
Peut-on critiquer la démocratie ? \\[0pt]
Peut-on en appeler à la conscience contre la loi ? \\[0pt]
Peut-on être apolitique ? \\[0pt]
Peut-on être citoyen du monde ? \\[0pt]
Peut-on être citoyen sans être soldat ? \\[0pt]
Peut-on fonder le droit de propriété ? \\[0pt]
Peut-on fonder les droits de l'homme ? \\[0pt]
Peut-on gouverner sans lois ? \\[0pt]
Peut-on innover en politique ? \\[0pt]
Peut-on justifier la discrimination ? \\[0pt]
Peut-on justifier la guerre ? \\[0pt]
Peut-on justifier la raison d'État ? \\[0pt]
Peut-on opposer justice et liberté ? \\[0pt]
Peut-on pactiser avec un brigand ? \\[0pt]
Peut-on parler de despotisme démocratique ? \\[0pt]
Peut-on parler de vertu politique ? \\[0pt]
Peut-on penser le social sans la politique ? \\[0pt]
Peut-on penser un pouvoir absolu ? \\[0pt]
Peut-on refuser la loi ? \\[0pt]
Peut-on régner innocemment ? \\[0pt]
Peut-on revendiquer la paix comme un droit ? \\[0pt]
Peut-on s'abstenir de penser politiquement ? \\[0pt]
Peut-on se désintéresser de la politique ? \\[0pt]
Peut-on se gouverner soi-même ? \\[0pt]
Peut-on séparer politique et économie ? \\[0pt]
Peut-on se passer de chef ? \\[0pt]
Peut-on se passer de l'État ? \\[0pt]
Peut-on se passer de représentants ? \\[0pt]
Peut-on se passer d'un maître ? \\[0pt]
Peut-on se régler sur des exemples en politique ? \\[0pt]
Peut-on souhaiter le gouvernement des meilleurs ? \\[0pt]
Politique et esthétique \\[0pt]
Politique et médias \\[0pt]
Politique et mémoire \\[0pt]
Politique et parole \\[0pt]
Politique et participation \\[0pt]
Politique et passions \\[0pt]
Politique et propagande \\[0pt]
Politique et religion \\[0pt]
Politique et secret \\[0pt]
Politique et technique \\[0pt]
Politique et technologie \\[0pt]
Politique et territoire \\[0pt]
Politique et trahison \\[0pt]
Politique et vertu \\[0pt]
Pourquoi des constitutions ? \\[0pt]
Pourquoi des corps intermédiaires ? \\[0pt]
Pourquoi des élections ? \\[0pt]
Pourquoi des institutions ? \\[0pt]
Pourquoi des lois ? \\[0pt]
Pourquoi des utopies ? \\[0pt]
Pourquoi écrit-on des lois ? \\[0pt]
Pourquoi faire de la politique ? \\[0pt]
Pourquoi faire la guerre ? \\[0pt]
Pourquoi le droit international est-il imparfait ? \\[0pt]
Pourquoi les États se font-ils la guerre ? \\[0pt]
Pourquoi obéir aux lois ? \\[0pt]
Pourquoi punir ? \\[0pt]
Pourquoi séparer les pouvoirs ? \\[0pt]
Pourquoi une instruction publique ? \\[0pt]
Pouvoir et autorité \\[0pt]
Pouvoir et contre-pouvoir \\[0pt]
Pouvoir et politique \\[0pt]
Pouvoir et puissance \\[0pt]
Pouvoir et savoir \\[0pt]
Pouvoir temporel et pouvoir spirituel \\[0pt]
Prendre le pouvoir \\[0pt]
Prendre les armes \\[0pt]
Prendre une décision politique \\[0pt]
Prince : nature ou fonction ? \\[0pt]
Propriété privée, propriété collective \\[0pt]
Prospérité et sécurité \\[0pt]
Quand y a-t-il peuple ? \\[0pt]
Qu'apporte l'histoire à la vie politique ? \\[0pt]
 Qu'avons-nous à attendre de la vie politique ? \\[0pt]
Que construit le politique ? \\[0pt]
Que dois-je à l'État ? \\[0pt]
Que faire des adversaires ? \\[0pt]
Que fait la police ? \\[0pt]
Que faut-il savoir pour gouverner ? \\[0pt]
Quel est le meilleur régime politique ? \\[0pt]
Quel est l'objet de la philosophie politique ? \\[0pt]
Quel est l'objet des sciences politiques ? \\[0pt]
Quelle est la spécificité de la communauté politique ? \\[0pt]
Quelle expérience la politique requiert-elle ? \\[0pt]
Quelle valeur donner à la notion de « corps social » ? \\[0pt]
Quels sont les moyens légitimes de la politique ? \\[0pt]
Que nous apprend, sur la politique, l'utopie ? \\[0pt]
Que nous doit l'État ? \\[0pt]
Que peut la politique ? \\[0pt]
Que peut le politique ? \\[0pt]
Que peut l'État ? \\[0pt]
Que peut-on attendre de l'État ? \\[0pt]
Que peut-on attendre du droit international ? \\[0pt]
Que peut-on interdire ? \\[0pt]
Que peuvent les lois ? \\[0pt]
Que protège la loi ? \\[0pt]
Que serait une démocratie parfaite ? \\[0pt]
Que signifie « politiquement correct » ? \\[0pt]
Qu'est-ce que « faire de la politique » ? \\[0pt]
Qu'est-ce que gouverner ? \\[0pt]
Qu'est-ce que prendre le pouvoir ? \\[0pt]
Qu'est-ce qu'être libéral ? \\[0pt]
Qu'est-ce qu'être républicain ? \\[0pt]
Qu'est-ce qu'être souverain ? \\[0pt]
Qu'est-ce qu'être un esclave ? \\[0pt]
Qu'est-ce qui est politique ? \\[0pt]
Qu'est-ce qui fait la force des lois ? \\[0pt]
Qu'est-ce qui fait la justice des lois ? \\[0pt]
Qu'est-ce qui fait la légitimité d'une autorité politique ? \\[0pt]
Qu'est-ce qui fait l'unité d'un peuple ? \\[0pt]
Qu'est-ce qui fait un peuple ? \\[0pt]
Qu'est-ce qui n'est pas politique ? \\[0pt]
Qu'est-ce qu'un adversaire en politique ? \\[0pt]
Qu'est-ce qu'un bon citoyen ? \\[0pt]
Qu'est-ce qu'un chef ? \\[0pt]
Qu'est-ce qu'un conflit politique ? \\[0pt]
Qu'est-ce qu'un contre-pouvoir ? \\[0pt]
Qu'est-ce qu'un coup d'État ? \\[0pt]
Qu'est-ce qu'un crime politique ? \\[0pt]
Qu'est-ce qu'une communauté politique ? \\[0pt]
Qu'est-ce qu'une constitution ? \\[0pt]
Qu'est-ce qu'une crise politique ? \\[0pt]
Qu'est-ce qu'une guerre juste ? \\[0pt]
Qu'est-ce qu'une idéologie ? \\[0pt]
Qu'est-ce qu'une loi d'exception ? \\[0pt]
Qu'est-ce qu'une loi injuste ? \\[0pt]
Qu'est-ce qu'une minorité ? \\[0pt]
Qu'est-ce qu'une nation ? \\[0pt]
Qu'est-ce qu'une politique sociale ? \\[0pt]
Qu'est-ce qu'une révolution ? \\[0pt]
Qu'est-ce qu'une société juste ? \\[0pt]
Qu'est-ce qu'une violence symbolique ? \\[0pt]
Qu'est-ce qu'un gouvernement ? \\[0pt]
Qu'est-ce qu'un mouvement politique \\[0pt]
Qu'est-ce qu'un mouvement politique ? \\[0pt]
Qu'est-ce qu'un peuple ? \\[0pt]
Qu'est-ce qu'un pouvoir despotique ? \\[0pt]
Qu'est-ce qu'un pouvoir légitime ? \\[0pt]
Qu'est-ce qu'un prince juste ? \\[0pt]
Qu'est-ce qu'un problème politique ? \\[0pt]
Qu'est-ce qu'un programme politique ? \\[0pt]
Qu'est-ce qu'un réfugié politique ? \\[0pt]
Qu'est-il légitime d'interdire ? \\[0pt]
Qu'est-il utile d'interdire ? \\[0pt]
Qu'est qu'un régime politique ? \\[0pt]
Qui a une parole politique ? \\[0pt]
Qui commande ? \\[0pt]
Qui détient le pouvoir ? \\[0pt]
Qui doit gouverner ? \\[0pt]
Qui est compétent en matière politique ? \\[0pt]
Qui est souverain ? \\[0pt]
Qui gouverne ? \\[0pt]
Qu'y a-t-il au-dessus des lois ? \\[0pt]
Race, classe, nation \\[0pt]
Raison et politique \\[0pt]
Rapports de force, rapport de pouvoir \\[0pt]
Rassembler les hommes, est-ce les unir ? \\[0pt]
Réforme et révolution \\[0pt]
République et démocratie \\[0pt]
Résister à l'oppression \\[0pt]
Résister peut-il être un droit ? \\[0pt]
Revient-il à l'État d'assurer votre bonheur ? \\[0pt]
Révolte et révolution \\[0pt]
Science et démocratie \\[0pt]
Science et pouvoir \\[0pt]
Sécurité et liberté \\[0pt]
Se révolter \\[0pt]
Servir l'État \\[0pt]
« Si tu veux la paix, prépare la guerre » \\[0pt]
Société et organisme \\[0pt]
Suffit-il, pour être juste, d'obéir aux lois et aux coutumes de son pays ? \\[0pt]
Sujet et citoyen \\[0pt]
Sur quoi fonder l'autorité ? \\[0pt]
Surveillance et discipline \\[0pt]
Toute action politique est-elle collective ? \\[0pt]
Toute communauté est-elle politique ? \\[0pt]
Toute hiérarchie est-elle inégalitaire ? \\[0pt]
Toute philosophie implique-t-elle une politique ? \\[0pt]
Tout est-il politique ? \\[0pt]
Tout pouvoir est-il oppresseur ? \\[0pt]
Tout pouvoir est-il politique ? \\[0pt]
Tout pouvoir est-il usurpé ? \\[0pt]
Tout pouvoir n'est-il pas abusif ? \\[0pt]
Une décision politique peut-elle être juste ? \\[0pt]
Une femme politique est-elle un homme politique comme les autres ? \\[0pt]
Une guerre peut-elle être juste ? \\[0pt]
« Une main de fer dans un gant de velours » \\[0pt]
Un enfant est-il un citoyen ? \\[0pt]
Une politique de la nature est-elle concevable ? \\[0pt]
Une politique peut-elle se réclamer de la vie ? \\[0pt]
Une société juste est-elle une société sans conflits ? \\[0pt]
Une société sans conflit est-elle possible ? \\[0pt]
Une société sans État est-elle une société sans politique ? \\[0pt]
Un État peut-il être trop étendu ? \\[0pt]
Un État peut-il être « voyou » ? \\[0pt]
Un souverain élu peut-il être illégitime ? \\[0pt]
Utopie et tradition \\[0pt]
Vices privés, vertus publiques \\[0pt]
Vie familiale, vie politique \\[0pt]
Vouloir la paix \\[0pt]
Vouloir l'égalité \\[0pt]
Y a-t-il des biens communs ? \\[0pt]
Y a-t-il des compétences politiques ? \\[0pt]
Y a-t-il des erreurs en politique ? \\[0pt]
Y a-t-il des fondements naturels à l'ordre social ? \\[0pt]
Y a-t-il des guerres justes ? \\[0pt]
Y a-t-il des lois injustes ? \\[0pt]
Y a-t-il du progrès en politique ? \\[0pt]
Y a-t-il un art de gouverner ? \\[0pt]
Y a-t-il un bien plus précieux que la paix ? \\[0pt]
Y a-t-il un droit international ? \\[0pt]
Y a-t-il une compétence en politique ? \\[0pt]
Y a-t-il une opinion publique mondiale ? \\[0pt]
Y a-t-il une spécificité de la délibération politique ? \\[0pt]
Y a-t-il un savoir du politique ? \\[0pt]


\subsection{Sciences humaines}
\label{sec:org48ad0af}

\noindent
Acteurs sociaux et usages sociaux \\[0pt]
Actions et pratiques \\[0pt]
Animal politique ou social ? \\[0pt]
Anomalie et anomie \\[0pt]
Anthropologie et humanisme \\[0pt]
Anthropologie et ontologie \\[0pt]
Anthropologie et politique \\[0pt]
Apprentissage et conditionnement \\[0pt]
À quelle généralité peuvent prétendre les sciences humaines ? \\[0pt]
À quelle objectivité peuvent prétendre les sciences sociales ? \\[0pt]
À quelles conditions penser une seconde nature ? \\[0pt]
À quoi bon les sciences humaines et sociales ? \\[0pt]
À quoi bon les systèmes ? \\[0pt]
À quoi reconnaît-on un comportement humain ? \\[0pt]
À quoi servent les cérémonies ? \\[0pt]
À quoi servent les sciences humaines ? \\[0pt]
Avons-nous une identité ? \\[0pt]
Castes et classes \\[0pt]
Causes et motivations \\[0pt]
Classes et histoire \\[0pt]
Comment connaître l'autre ? \\[0pt]
Comment déterminer le sens d'une conduite ? \\[0pt]
Comment faire de l'homme un objet d'étude ? \\[0pt]
Comment fonder une psychologie sociale ? \\[0pt]
Comment la psychanalyse use-t-elle de l'interprétation ? \\[0pt]
Comment les sciences sociales expliquent-elles l'irrationnel ? \\[0pt]
Comment les sociétés changent-elles ? \\[0pt]
« Comment peut-on être persan ? » \\[0pt]
Comment s'entendre ? \\[0pt]
Communauté et société \\[0pt]
Comparer les cultures ? \\[0pt]
Comparer les langues \\[0pt]
Crise et révolution \\[0pt]
Croyance et société \\[0pt]
Cultes et rituels \\[0pt]
Culture et civilisation \\[0pt]
Culture et conscience \\[0pt]
Culture et sociétés \\[0pt]
Dans les sciences humaines et sociales, peut-on préconiser l'imitation des sciences de la nature ? \\[0pt]
Décrire une langue \\[0pt]
Dégager la structure, est-ce rendre compte de ce qu'est une chose ? \\[0pt]
De quelle expertise les sciences humaines peuvent-elles se prévaloir ? \\[0pt]
De quelle science humaine la folie peut-elle être l'objet ? \\[0pt]
De quoi les sciences humaines nous instruisent-elles ? \\[0pt]
De quoi le tissu social est-il fait ? \\[0pt]
De quoi l'historien fait-il l'histoire ? \\[0pt]
Des comportements économiques peuvent-ils être rationnels ? \\[0pt]
Désir et société \\[0pt]
Des motivations peuvent-elles être sociales ? \\[0pt]
Des peuples sans histoire \\[0pt]
Des sciences molles ? \\[0pt]
Des sociétés sans histoire \\[0pt]
Des statistiques et des hommes \\[0pt]
Des structures et des hommes \\[0pt]
Déterminisme naturel et déterminisme social \\[0pt]
Déterminisme psychique et déterminisme physique \\[0pt]
Diachronie et synchronie \\[0pt]
Diversité des sciences humaines et unité de l'homme \\[0pt]
Diversité ou universalité \\[0pt]
Document, archive, témoignage \\[0pt]
Documents et monuments \\[0pt]
Don, échange, potlatch \\[0pt]
Donner sens \\[0pt]
Économie et société \\[0pt]
Économie politique et politique économique \\[0pt]
Écrire l'histoire \\[0pt]
Empathie et sympathie \\[0pt]
En quel sens l'anthropologie peut-elle être historique ? \\[0pt]
En quel sens le vécu est-il l'objet des sciences humaines ? \\[0pt]
En quel sens peut-on parler de la vie sociale comme d'un jeu ? \\[0pt]
Enquêter \\[0pt]
En quoi la linguistique est-elle une science ? \\[0pt]
En quoi l'animal intéresse-t-il les sciences humaines ? \\[0pt]
En quoi la sexualité peut-elle être objet de science ? \\[0pt]
En quoi le langage est-il objet de science ? \\[0pt]
En quoi les sciences humaines nous éclairent-elles sur la barbarie ? \\[0pt]
En quoi les sciences humaines sont-elles normatives ? \\[0pt]
En quoi l'histoire est-elle une science du réel ? \\[0pt]
Enseigner, instruire, éduquer \\[0pt]
Espace et structure sociale \\[0pt]
Espace privé, espace public \\[0pt]
Espèce et genre \\[0pt]
Ethnologie et cinéma \\[0pt]
Ethnologie et ethnocentrisme \\[0pt]
Être asocial \\[0pt]
Être civilisé \\[0pt]
Être conformiste \\[0pt]
Être l'entrepreneur de soi-même \\[0pt]
Être mère \\[0pt]
Être père \\[0pt]
Événement et structure \\[0pt]
Existe-il des mentalités primitives ? \\[0pt]
Existe-t-il des états mentaux collectifs ? \\[0pt]
Existe-t-il des identités collectives ? \\[0pt]
Existe-t-il des mondes sociaux ? \\[0pt]
Existe-t-il des sujets collectifs ? \\[0pt]
Existe-t-il une histoire du présent ? \\[0pt]
Explication causale, explication génétique dans les sciences humaines \\[0pt]
Explication et compréhension \\[0pt]
Expliquer, est-ce justifier ? \\[0pt]
Expliquer et comprendre \\[0pt]
« Expliquer les faits sociaux par des faits sociaux » \\[0pt]
Faire comme les autres \\[0pt]
Faire société \\[0pt]
Fait et valeur \\[0pt]
Famille et tribu \\[0pt]
Faut-il considérer les faits sociaux comme des choses ? \\[0pt]
Faut-il enfermer ? \\[0pt]
Faut-il opposer sciences humaines et sciences de la nature ? \\[0pt]
Folie et société \\[0pt]
Fonder une famille \\[0pt]
Genèse et structure \\[0pt]
Guérir \\[0pt]
Herméneutique et sciences humaines \\[0pt]
Histoire et anthropologie \\[0pt]
Histoire et ethnologie \\[0pt]
Histoire et géographie \\[0pt]
Histoire et géographie font-elles couple ? \\[0pt]
Histoire et mémoire \\[0pt]
Historicité et vérité \\[0pt]
Homo religiosus \\[0pt]
Humanisation et déshumanisation \\[0pt]
Imagination et fantasme \\[0pt]
Imitation et identification \\[0pt]
Inconscient et langage \\[0pt]
Individu et société \\[0pt]
Inégalités et discriminations \\[0pt]
Information et communication \\[0pt]
Intégration et assimilation \\[0pt]
Intégration et exclusion \\[0pt]
Interdire et prohiber \\[0pt]
Intériorité et extériorité \\[0pt]
Interprétation et scientificité \\[0pt]
Interpréter \\[0pt]
Interpréter, est-ce un art ? \\[0pt]
Interpréter et formaliser dans les sciences humaines \\[0pt]
Invariance et pluralité des langues \\[0pt]
« Je ne voulais pas cela » : en quoi les sciences humaines permettent-elles de comprendre cette excuse ? \\[0pt]
L'abondance \\[0pt]
La carte et le territoire \\[0pt]
La causalité dans les sciences humaines \\[0pt]
La causalité en sciences humaines \\[0pt]
La causalité historique \\[0pt]
La chasse et la guerre \\[0pt]
La chronologie \\[0pt]
La comédie sociale \\[0pt]
La communauté des individus \\[0pt]
La communication \\[0pt]
La comparaison en sociologie \\[0pt]
La concurrence \\[0pt]
La condition humaine \\[0pt]
La condition sociale \\[0pt]
La connaissance de l'histoire nous rend-elle plus libres ? \\[0pt]
La connaissance du singulier dans les sciences humaines \\[0pt]
La cosmogonie \\[0pt]
La criminalité \\[0pt]
La crise sociale \\[0pt]
L'acteur et l'observateur dans les sciences humaines \\[0pt]
L'action collective \\[0pt]
La cuisine \\[0pt]
La culture de masse \\[0pt]
La culture d'entreprise \\[0pt]
La culture est-elle ce qui nous relie au passé ? \\[0pt]
La culture est-elle l'objet naturel des sciences humaines ? \\[0pt]
La culture est-elle un objet ? \\[0pt]
La culture et les cultures \\[0pt]
La culture populaire \\[0pt]
La descendance \\[0pt]
La déviance \\[0pt]
La différence des sexes \\[0pt]
La distinction de genre \\[0pt]
La distinction de la nature et de la culture est-elle un fait de culture ? \\[0pt]
La distinction du sacré et du profane \\[0pt]
La diversité des cultures \\[0pt]
La diversité des religions \\[0pt]
La diversité humaine \\[0pt]
La division des tâches \\[0pt]
La division du travail \\[0pt]
La domestication \\[0pt]
La domination \\[0pt]
La domination sociale \\[0pt]
La famille \\[0pt]
La famille est-elle une invention culturelle ? \\[0pt]
La fête \\[0pt]
La finalité des sciences humaines \\[0pt]
La folie \\[0pt]
La folie et l'étude du psychisme \\[0pt]
La fonction des rêves \\[0pt]
La fonction symbolique \\[0pt]
La force du social \\[0pt]
La foule \\[0pt]
La généalogie \\[0pt]
L'agent \\[0pt]
La géographie \\[0pt]
La géographie est-elle une science humaine ? \\[0pt]
La grammaire générative \\[0pt]
L'agriculture \\[0pt]
La guérison en psychanalyse \\[0pt]
La hiérarchie \\[0pt]
La hiérarchie des cultures \\[0pt]
La honte \\[0pt]
La jeunesse \\[0pt]
La langue nous parle-t-elle ? \\[0pt]
La langue officielle \\[0pt]
La liberté intéresse-t-elle les sciences humaines ? \\[0pt]
La linguistique est-elle une science de l'information ? \\[0pt]
La linguistique fait-elle partie des sciences humaines ? \\[0pt]
La linguistique peut-elle se passer de la psychologie ? \\[0pt]
La littérature peut-elle suppléer les sciences de l'homme ? \\[0pt]
L'altérité culturelle \\[0pt]
La magie \\[0pt]
La maîtrise du feu \\[0pt]
La maladie mentale a-t-elle une histoire ? \\[0pt]
La marchandisation \\[0pt]
La marginalité \\[0pt]
La masse \\[0pt]
L'âme concerne-t-elle les sciences humaines ? \\[0pt]
La mémoire collective \\[0pt]
La mémoire et l'individu \\[0pt]
La mesure dans les sciences humaines \\[0pt]
La mesure de l'homme \\[0pt]
La mesure de l'intelligence \\[0pt]
La méthode en sciences sociales \\[0pt]
La méthode statistique \\[0pt]
La mise à l'épreuve des hypothèses en sciences humaines et sociales \\[0pt]
La misère \\[0pt]
La mode \\[0pt]
La modélisation en sciences sociales \\[0pt]
La modernité \\[0pt]
La mondialisation \\[0pt]
La naissance \\[0pt]
La naissance de l'homme \\[0pt]
L'analyse des comportements humains permet-elle de réduire le hasard ? \\[0pt]
L'analyse des mythes doit-elle être uniquement linguistique \\[0pt]
L'analyse économique rend-elle compte des comportements humains ? \\[0pt]
La nécessité sociale des cérémonies \\[0pt]
La neutralité axiologique \\[0pt]
La névrose \\[0pt]
Langage et réification \\[0pt]
Langage et société \\[0pt]
Langage, langue et parole \\[0pt]
Langage, langue, parole \\[0pt]
Langue et langage \\[0pt]
Langue et parole \\[0pt]
L'animisme \\[0pt]
L'anomie \\[0pt]
L'anonyme \\[0pt]
La notion d'administration \\[0pt]
La notion de civilisation \\[0pt]
La notion de classe dominante \\[0pt]
La notion de classe sociale \\[0pt]
La notion de corps social \\[0pt]
La notion de fait social \\[0pt]
La notion de loi dans les sciences de la nature et dans les sciences de l'homme \\[0pt]
La notion de milieu \\[0pt]
La notion de peuple \\[0pt]
La notion de race a-t-elle un statut scientifique ? \\[0pt]
La notion de signification est-elle indispensable aux sciences humaines ? \\[0pt]
La notion d'intérêt \\[0pt]
La notion économique d'abondance \\[0pt]
L'anthropologie est-elle une ontologie ? \\[0pt]
La nudité \\[0pt]
La parenté \\[0pt]
La parenté et la famille \\[0pt]
La parole \\[0pt]
La parole comme acte social \\[0pt]
La parole peut-elle être étudiée comme un acte social ? \\[0pt]
La pauvreté \\[0pt]
La pédagogie, philosophie pratique ou psychologie appliquée ? \\[0pt]
La pensée collective \\[0pt]
La pensée magique \\[0pt]
La pensée mythique \\[0pt]
La perception de la folie change-t-elle de nature avec les sciences humaines ? \\[0pt]
La place du sujet dans les sciences humaines \\[0pt]
La pluralité des cultures \\[0pt]
La pluralité des langues \\[0pt]
La pluralité des représentations \\[0pt]
La politesse \\[0pt]
La population \\[0pt]
L'appartenance sociale \\[0pt]
L'apprentissage de la langue \\[0pt]
L'approche psychologique de l'homme met-elle en question la notion d'âme ? \\[0pt]
La pratique \\[0pt]
La prévision des conduites rationnelles \\[0pt]
La prise en charge de l'individuel par la psychanalyse \\[0pt]
La production \\[0pt]
La prohibition de l'inceste \\[0pt]
La propreté \\[0pt]
La psychanalyse donne-t-elle un sens au rêve ? \\[0pt]
La psychanalyse est-elle une science humaine ? \\[0pt]
La psychanalyse permet-elle la construction du sujet ? \\[0pt]
La psychanalyse s'applique-t-elle aux phénomènes sociaux ? \\[0pt]
La psychologie est-elle l'étude de ce qui n'est pas social ? \\[0pt]
La psychologie est-elle l'étude des comportements individuels ? \\[0pt]
La psychologie est-elle une science ? \\[0pt]
La psychologie est-elle une science de la nature ? \\[0pt]
La psychologie est-elle une science humaine ? \\[0pt]
La pudeur \\[0pt]
La question de la nature humaine relève-t-elle des sciences humaines ? \\[0pt]
La question « qu'est-ce que l'homme ? » est-elle scientifique ? \\[0pt]
La question sociale \\[0pt]
La rationalité de la psychanalyse \\[0pt]
La rationalité de l'homo œconomicus \\[0pt]
La rationalité des comportements économiques \\[0pt]
La rationalité du marché \\[0pt]
La rationalité en sciences sociales \\[0pt]
L'arbitraire de la règle \\[0pt]
L'arbitraire du signe \\[0pt]
L'archéologie \\[0pt]
La réalité de l'inconscient \\[0pt]
La réalité sociale \\[0pt]
La réalité symbolique \\[0pt]
La recherche-action en sciences sociales \\[0pt]
La recherche de la vérité dans les sciences humaines \\[0pt]
La recherche des invariants \\[0pt]
La reconnaissance \\[0pt]
La reproduction \\[0pt]
La reproduction sociale \\[0pt]
La réputation \\[0pt]
La responsabilité envers les générations futures \\[0pt]
L'argent \\[0pt]
L'argent et la valeur \\[0pt]
La rumeur \\[0pt]
La science des mœurs \\[0pt]
La science historique peut-elle être prédictive ? \\[0pt]
La scientificité des sciences humaines \\[0pt]
La sécularisation \\[0pt]
La signification des rites \\[0pt]
La socialisation \\[0pt]
La socialisation des comportements \\[0pt]
La société de consommation \\[0pt]
La société des individus \\[0pt]
La société des loisirs \\[0pt]
La société des savants \\[0pt]
La société existe-t-elle ? \\[0pt]
La société peut-elle être objet de connaissance ? \\[0pt]
La sociologie de la religion \\[0pt]
La sociologie de l'art nous permet-elle de comprendre l'art ? \\[0pt]
La sociologie est-elle déterministe ? \\[0pt]
La sociologie est-elle une physique sociale ? \\[0pt]
La sociologie relativise-t-elle la valeur des œuvres d'art ? \\[0pt]
La solidarité \\[0pt]
La solitude \\[0pt]
La souffrance au travail \\[0pt]
La spécificité des sciences humaines \\[0pt]
La structure \\[0pt]
La structure et les éléments \\[0pt]
La structure et le sujet \\[0pt]
La technologie modifie-t-elle les rapports sociaux ? \\[0pt]
La tentation réductionniste \\[0pt]
La théogonie \\[0pt]
La tradition \\[0pt]
La traduction \\[0pt]
La transe \\[0pt]
La transgression \\[0pt]
La transmission \\[0pt]
L'autarcie \\[0pt]
La valeur d'échange \\[0pt]
La valeur de l'argent \\[0pt]
La valeur de l'échange \\[0pt]
La valeur d'usage \\[0pt]
La valeur du témoignage \\[0pt]
La vie des signes \\[0pt]
La vie sauvage \\[0pt]
La vie sociale \\[0pt]
La vie sociale relève-t-elle des habitudes ? \\[0pt]
La ville \\[0pt]
La violence sociale \\[0pt]
Le besoin \\[0pt]
Le biologisme en sciences sociales \\[0pt]
Le cannibalisme \\[0pt]
Le capitalisme \\[0pt]
Le capital social \\[0pt]
L'échange \\[0pt]
L'échange des marchandises et les rapports humains \\[0pt]
L'échange des signes \\[0pt]
L'échange symbolique \\[0pt]
Le choix en économie \\[0pt]
L'écologie est-elle aussi une science humaine ? \\[0pt]
L'écologie est-elle une science de la nature ou une science sociale ? \\[0pt]
L'écologie, une science humaine ? \\[0pt]
Le comparatisme dans les sciences humaines \\[0pt]
Le comportement \\[0pt]
Le concept de civilisation \\[0pt]
Le concept de comportement rationnel dans les sciences humaines \\[0pt]
Le concept de pulsion \\[0pt]
Le concept de race est-il anthropologique ? \\[0pt]
Le concept de société sans écriture est-il pertinent ? \\[0pt]
Le concept de structure sociale \\[0pt]
Le concept d'inconscient est-il nécessaire en sciences humaines ? \\[0pt]
Le conflit des interprétations \\[0pt]
Le conflit social \\[0pt]
Le conformisme social \\[0pt]
L'économie a-t-elle des lois ? \\[0pt]
L'économie a-t-elle ses propres lois ? \\[0pt]
L'économie de la langue \\[0pt]
L'économie du don \\[0pt]
L'économie est-elle une science humaine ? \\[0pt]
L'économie et la valeur humaine \\[0pt]
L'économie politique \\[0pt]
L'économie primitive \\[0pt]
L'économie psychique \\[0pt]
L'économique et le politique \\[0pt]
Le consensus social \\[0pt]
Le contrat social \\[0pt]
Le contrôle social \\[0pt]
Le corps est-il un enjeu des sciences humaines ? \\[0pt]
Le corps humain est-il naturel ? \\[0pt]
Le cru et le cuit \\[0pt]
Le culte des ancêtres \\[0pt]
Le décentrement ethnologique \\[0pt]
Le déclassement \\[0pt]
Le déterminisme psychique \\[0pt]
Le déterminisme social \\[0pt]
Le dialogue entre les cultures \\[0pt]
Le don \\[0pt]
Le droit est-il une science humaine ? \\[0pt]
L'éducation physique \\[0pt]
Le fait historique \\[0pt]
Le fait religieux \\[0pt]
Le féminisme \\[0pt]
Le fétichisme de la marchandise \\[0pt]
L'efficacité thérapeutique de la psychanalyse \\[0pt]
Le folklore \\[0pt]
Le formalisme dans les sciences humaines \\[0pt]
Le fou \\[0pt]
L'égalité des chances \\[0pt]
L'égalité des sexes \\[0pt]
Le geste et la parole \\[0pt]
Le goût de la fête \\[0pt]
Le groupe \\[0pt]
Le jeu \\[0pt]
Le jeu social \\[0pt]
Le langage et les langues \\[0pt]
Le latent et le manifeste \\[0pt]
Le lien social \\[0pt]
Le mariage \\[0pt]
Le masque \\[0pt]
Le ménage \\[0pt]
Le mépris de classe \\[0pt]
Le mérite \\[0pt]
Le milieu social \\[0pt]
Le modèle du jeu dans les sciences humaines ? \\[0pt]
Le modèle évolutionniste en histoire \\[0pt]
Le modèle organiciste \\[0pt]
Le modèle organiciste en sciences humaines \\[0pt]
Le monde de la culture \\[0pt]
Le monde de l'entreprise \\[0pt]
Le monde de l'être humain \\[0pt]
L'empathie \\[0pt]
L'empathie est-elle nécessaire aux sciences sociales ? \\[0pt]
L'empirisme des sciences humaines \\[0pt]
Le multiculturalisme \\[0pt]
Le mythe \\[0pt]
Le mythe est-il objet de science ? \\[0pt]
Le naturalisme des sciences humaines et sociales \\[0pt]
L'enfant sauvage \\[0pt]
Le nomadisme \\[0pt]
L'enquête de terrain \\[0pt]
L'enquête sociale \\[0pt]
L'environnement est-il un nouvel objet pour les sciences humaines ? \\[0pt]
Le partage des savoirs \\[0pt]
Le patriarcat \\[0pt]
Le patrimoine \\[0pt]
Le père, la mère, l'enfant \\[0pt]
Le phénomène religieux \\[0pt]
L'épistémologie des sciences humaines \\[0pt]
Le pouvoir causal de l'inconscient \\[0pt]
Le pouvoir des mots \\[0pt]
Le pouvoir des sciences humaines et sociales \\[0pt]
Le pouvoir traditionnel \\[0pt]
Le premier et le primitif \\[0pt]
Le prix \\[0pt]
Le processus de civilisation \\[0pt]
Le profit est-il le moteur de la production économique ? \\[0pt]
Le propre de l'homme \\[0pt]
Le propriétaire \\[0pt]
Le psychisme est-il objet de connaissance ? \\[0pt]
Le public et le privé \\[0pt]
L'équilibre économique \\[0pt]
Le récit en histoire \\[0pt]
Le refoulement \\[0pt]
Le relativisme culturel \\[0pt]
Le relativisme culturel est-il indépassable ? \\[0pt]
Le relativisme peut-il être un principe de méthode dans les sciences humaines ? \\[0pt]
Le repas \\[0pt]
Le respect des convenances \\[0pt]
Le rêve \\[0pt]
Le révisionnisme \\[0pt]
Le rituel \\[0pt]
Le roman national \\[0pt]
Le sacré et le profane \\[0pt]
Le sacrifice \\[0pt]
Les affects sont-ils des objets sociologiques ? \\[0pt]
Les agents économiques ont-ils un comportement rationnel ? \\[0pt]
Les agents sociaux poursuivent-ils l'utilité ? \\[0pt]
Les agents sociaux sont-ils rationnels ? \\[0pt]
Les analogies dans les sciences humaines \\[0pt]
Les antagonismes sociaux \\[0pt]
Les archives \\[0pt]
Les autres \\[0pt]
Le sauvage \\[0pt]
Le sauvage et le barbare \\[0pt]
Le sauvage et le civilisé \\[0pt]
Les choses \\[0pt]
Les choses humaines \\[0pt]
Les classes sociales \\[0pt]
Les conditions d'une science historique \\[0pt]
Les conflits sociaux \\[0pt]
Les conflits sociaux sont-ils des conflits de classe ? \\[0pt]
Les conventions \\[0pt]
Les coutumes \\[0pt]
Les critères de vérité dans les sciences humaines \\[0pt]
Les cultures sont-elles incommensurables ? \\[0pt]
Les différences entre les sciences humaines sont-elles des différences d'objet ? \\[0pt]
Les dispositions sociales \\[0pt]
Les distinctions sociales \\[0pt]
Les divisions sociales \\[0pt]
Les échanges \\[0pt]
Le sens de la famille \\[0pt]
Le sens de la fête \\[0pt]
Le sens de l'humour \\[0pt]
Les études comparatives \\[0pt]
Les études de genre \\[0pt]
Les faits sociaux se répètent-ils ? \\[0pt]
Les foules \\[0pt]
Les fous \\[0pt]
Les frontières \\[0pt]
Les frontières du moi \\[0pt]
Les hommes croient-ils dans leurs mythes ? \\[0pt]
Les hommes et les femmes \\[0pt]
Le signe et l'indice \\[0pt]
Les industries culturelles \\[0pt]
Les inégalités sociales \\[0pt]
Les interdits \\[0pt]
Les invariants culturels \\[0pt]
Les langues meurent-elles ? \\[0pt]
Les liens sociaux \\[0pt]
Les lieux de mémoire \\[0pt]
Les limites de l'interprétation \\[0pt]
Les lois du sang \\[0pt]
Les maladies mentales sont-elles des maladies sociales ? \\[0pt]
Les marginaux \\[0pt]
Les masses \\[0pt]
Les mécanismes cérébraux \\[0pt]
Les mécanismes d'identification \\[0pt]
Les mœurs \\[0pt]
Les mœurs sont-elles objet de science ? \\[0pt]
Les mythes collectifs \\[0pt]
Les mythes relèvent-ils de l'anthropologie ? \\[0pt]
Les normes \\[0pt]
Le soin \\[0pt]
L'espace social \\[0pt]
L'espèce humaine \\[0pt]
Les phénomènes de mode \\[0pt]
Les pouvoirs de la parole \\[0pt]
Les prêtres \\[0pt]
Les procédures de validation dans les sciences humaines ? \\[0pt]
Les rapports sociaux sont-ils des rapports entre individus ? \\[0pt]
Les règles sociales \\[0pt]
Les règles sociales sont-elles répressives ? \\[0pt]
Les relations sociales \\[0pt]
Les représentations collectives \\[0pt]
Les réseaux sociaux \\[0pt]
Les ressources humaines \\[0pt]
Les riches et les pauvres \\[0pt]
Les rites \\[0pt]
Les rites funéraires \\[0pt]
Les rituels \\[0pt]
Les rôles sociaux \\[0pt]
Les sacrifices \\[0pt]
Les sauvages \\[0pt]
Les sciences de l'éducation \\[0pt]
Les sciences de l'homme et l'évolution \\[0pt]
Les sciences de l'homme ont-elles inventé leur objet ? \\[0pt]
Les sciences de l'homme permettent-elles d'affiner la notion de responsabilité ? \\[0pt]
Les sciences de l'homme peuvent-elles expliquer l'impuissance de la liberté ? \\[0pt]
Les sciences de l'homme rendent-elles l'homme prévisible ? \\[0pt]
Les sciences du comportement \\[0pt]
Les sciences humaines doivent-elles être transdisciplinaires ? \\[0pt]
Les sciences humaines échappent-elles à l'abstraction ? \\[0pt]
Les sciences humaines éliminent-elles la contingence du futur ? \\[0pt]
Les sciences humaines, est-ce la mort de l'homme ? \\[0pt]
Les sciences humaines et le droit \\[0pt]
Les sciences humaines et l'expérience \\[0pt]
Les sciences humaines et sociales n'ont-elles de sciences que le nom ? \\[0pt]
Les sciences humaines finissent-elles de ruiner l'idée de liberté ? \\[0pt]
Les sciences humaines jouent elles un rôle dans l'exploitation de l'homme par l'homme ? \\[0pt]
Les sciences humaines nient-elles la liberté de l'homme ? \\[0pt]
Les sciences humaines nient-elles la liberté du sujet humain ? \\[0pt]
Les sciences humaines nous font-elles connaître l'homme ? \\[0pt]
Les sciences humaines nous obligent-elles à renoncer à l'idée de libre arbitre ? \\[0pt]
Les sciences humaines nous protègent-elles de l'essentialisme ? \\[0pt]
Les sciences humaines ont-elles une utilité ? \\[0pt]
Les sciences humaines ont-elles un objet commun ? \\[0pt]
Les sciences humaines opposent-elles histoire et système ? \\[0pt]
Les sciences humaines permettent-elles de comprendre la vie d'un homme ? \\[0pt]
Les sciences humaines peuvent-elles éclairer la réflexion éthique ? \\[0pt]
Les sciences humaines peuvent-elles être objectives ? \\[0pt]
Les sciences humaines peuvent-elles faire abstraction de la nature ? \\[0pt]
Les sciences humaines peuvent-elles s'affranchir de l'ethnocentrisme ? \\[0pt]
Les sciences humaines peuvent-elles se passer de la notion d'inconscient ? \\[0pt]
Les sciences humaines présupposent-elles une définition de l'homme ? \\[0pt]
Les sciences humaines recherchent-elles des invariants ? \\[0pt]
Les sciences humaines rendent-elle l'homme prévisible ? \\[0pt]
Les sciences humaines reposent-elles sur des faits ? \\[0pt]
Les sciences humaines sont-elles des sciences ? \\[0pt]
Les sciences humaines sont-elles des sciences de la nature humaine ? \\[0pt]
Les sciences humaines sont-elles des sciences de la vie humaine ? \\[0pt]
Les sciences humaines sont-elles des sciences de l'esprit ? \\[0pt]
Les sciences humaines sont-elles des sciences de l'interprétation ? \\[0pt]
Les sciences humaines sont-elles des sciences d'interprétation ? \\[0pt]
Les sciences humaines sont-elles explicatives ou compréhensives ? \\[0pt]
Les sciences humaines sont-elles les sciences de l'esprit ? \\[0pt]
Les sciences humaines sont-elles nécessairement empiriques ? \\[0pt]
Les sciences humaines sont-elles normatives ? \\[0pt]
Les sciences humaines sont-elles relativistes ? \\[0pt]
Les sciences humaines sont-elles subversives ? \\[0pt]
Les sciences humaines suffisent-elles à connaître l'homme ? \\[0pt]
Les sciences humaines supposent-elles une définition de l'homme ? \\[0pt]
Les sciences humaines supposent-elles une rupture avec l'ordre naturel ? \\[0pt]
Les sciences humaines traitent-elles de l'individu ? \\[0pt]
Les sciences humaines transforment-elles la notion de causalité ? \\[0pt]
Les sciences peuvent-elles penser l'individu ? \\[0pt]
Les sciences sociales peuvent-elles être expérimentales ? \\[0pt]
Les sciences sociales peuvent-elles servir à gouverner ? \\[0pt]
Les sens humains \\[0pt]
Les signes \\[0pt]
Les signes de l'humanité \\[0pt]
Les sociétés animales \\[0pt]
Les sociétés évoluent-elles ? \\[0pt]
Les sociétés humaines se construisent-elles contre la nature ? \\[0pt]
Les sociétés ont-elles un inconscient ? \\[0pt]
Les sociétés sans histoire \\[0pt]
Les sociétés savantes \\[0pt]
Les sociétés sont-elles hiérarchisables ? \\[0pt]
Les sociétés sont-elles imprévisibles ? \\[0pt]
Les statistiques comme outil dans les sciences humaines \\[0pt]
Les structures expliquent-elles tout ? \\[0pt]
Les structures que dégagent les sciences humaines agissent-elles ? \\[0pt]
Les structures sociales sont-elles économiques ? \\[0pt]
Les tabous \\[0pt]
Les techniques de manipulation des hommes peuvent-elles s'appuyer sur les sciences humaines ? \\[0pt]
Les traditions \\[0pt]
Le suicide \\[0pt]
Les usages \\[0pt]
Les vivants et les morts \\[0pt]
Le symbole, comme concept, dans les sciences humaines \\[0pt]
Le symbolique \\[0pt]
Le système des besoins \\[0pt]
Le « terrain » \\[0pt]
Le terrain \\[0pt]
L'ethnocentrisme \\[0pt]
L'ethnocentrisme comme problème pour l'ethnologie \\[0pt]
L'ethnologie est-elle une science mélancolique ? \\[0pt]
Le totémisme \\[0pt]
L'étranger \\[0pt]
Le travail sur le terrain \\[0pt]
L'être intelligent \\[0pt]
L'événement et le fait divers \\[0pt]
Le village global \\[0pt]
L'évolution des langues \\[0pt]
Le voyage \\[0pt]
L'exclusion \\[0pt]
L'expérience dans les sciences humaines \\[0pt]
L'expérience du sociologue \\[0pt]
L'expérience en psychologie \\[0pt]
L'expérience en sciences humaines \\[0pt]
L'expérimentation en sciences sociales \\[0pt]
L'expertise \\[0pt]
L'explication causale dans les sciences sociales \\[0pt]
L'expression de l'inconscient \\[0pt]
L'habitation \\[0pt]
L'habitus \\[0pt]
L'héritage \\[0pt]
L'hétérogénéité sociale \\[0pt]
L'histoire des civilisations \\[0pt]
L'histoire : enquête ou science ? \\[0pt]
L'histoire est-elle déterministe ? \\[0pt]
L'histoire est-elle la science de l'événementiel ? \\[0pt]
L'histoire est-elle une science de l'individuel ? \\[0pt]
L'histoire est-elle un roman vrai ? \\[0pt]
L'histoire et la géographie \\[0pt]
« L'histoire jugera » \\[0pt]
L'histoire n'est-elle qu'un récit ? \\[0pt]
L'histoire peut-elle être objective ? \\[0pt]
L'histoire : science ou récit ? \\[0pt]
L'histoire se réduit-elle à l'étude du passé ? \\[0pt]
L'historicisme \\[0pt]
L'historien est-il un homme qui se souvient du passé ? \\[0pt]
L'hominisation \\[0pt]
L'homme de la rue \\[0pt]
L'homme des foules \\[0pt]
L'homme des sciences de l'homme \\[0pt]
L'homme des sciences humaines \\[0pt]
L'homme est-il objet de science ? \\[0pt]
L'homme est-il un animal symbolique ? \\[0pt]
L'homme et le sacré \\[0pt]
L'homme moyen \\[0pt]
L'homme peut-il être objet d'expérience ? \\[0pt]
L'homo œconomicus est-il rationnel ? \\[0pt]
L'humain est-il un animal irrationnel ? \\[0pt]
L'humanité : un concept normatif ? \\[0pt]
L'idéal-type \\[0pt]
L'idée de conscience collective \\[0pt]
L'idée de forme sociale \\[0pt]
L'idée de préhistoire \\[0pt]
L'idée de progrès \\[0pt]
L'idée de société primitive \\[0pt]
L'idée d'une science générale des signes \\[0pt]
L'identité collective \\[0pt]
L'idéologie \\[0pt]
L'imaginaire social \\[0pt]
L'immémorial \\[0pt]
L'implicite dans la communication \\[0pt]
L'inconscient \\[0pt]
L'inconscient a-t-il une histoire ? \\[0pt]
L'inconscient collectif \\[0pt]
L'inconscient psychanalytique réfute-t-il la notion philosophique de conscience ? \\[0pt]
L'indigène et l'étranger \\[0pt]
L'individualisme \\[0pt]
L'individualisme méthodologique \\[0pt]
L'individu échappe-t-il aux sciences humaines ? \\[0pt]
L'individuel et le collectif \\[0pt]
L'individu et la multitude \\[0pt]
L'individu et la société \\[0pt]
L'individu et le groupe \\[0pt]
L'individu et les masses \\[0pt]
L'individu particulier peut-il être l'objet d'une connaissance sociologique ? \\[0pt]
Linguistique et sens \\[0pt]
L'inhumain \\[0pt]
L'initiation \\[0pt]
« L'insociable sociabilité » \\[0pt]
L'institution de la société \\[0pt]
L'institutionnalisation des conduites \\[0pt]
L'institution scolaire \\[0pt]
L'instrument mathématique en sciences humaines \\[0pt]
L'intégration sociale \\[0pt]
L'intelligence des foules \\[0pt]
L'intelligence du lointain \\[0pt]
L'interdépendance \\[0pt]
L'interdépendance des faits humains \\[0pt]
L'interdit \\[0pt]
L'interdit a-t-il une origine sociale ? \\[0pt]
L'intérêt \\[0pt]
L'intériorisation des normes \\[0pt]
L'intersectionnalité \\[0pt]
L'introspection \\[0pt]
L'obéissance à l'autorité \\[0pt]
L'objectivité \\[0pt]
L'objectivité des sciences humaines \\[0pt]
L'objectivité des sciences sociales \\[0pt]
L'objet de culte \\[0pt]
L'obligation d'échanger \\[0pt]
L'observation de l'homme \\[0pt]
L'observation participante \\[0pt]
L'œuvre de l'historien \\[0pt]
L'opinion publique \\[0pt]
L'ordre social \\[0pt]
L'ordre symbolique \\[0pt]
L'origine de la famille \\[0pt]
L'origine des croyances \\[0pt]
L'origine du symbolisme \\[0pt]
L'unité des langues \\[0pt]
L'unité des sciences humaines \\[0pt]
L'unité des sciences humaines ? \\[0pt]
L'unité du moi \\[0pt]
L'universel en histoire \\[0pt]
L'usage \\[0pt]
L'usage de la langue \\[0pt]
L'usage de l'analogie dans les sciences humaines \\[0pt]
L'usage des modèles dans les sciences humaines \\[0pt]
L'usage du concept de race \\[0pt]
L'utilité des sciences humaines \\[0pt]
Machines et liberté \\[0pt]
Machines et mémoire \\[0pt]
Magie et religion \\[0pt]
Masculin et féminin \\[0pt]
Masculin, féminin \\[0pt]
Mathématiques et sciences humaines \\[0pt]
Mythe et symbole \\[0pt]
Mythes et idéologies \\[0pt]
Nature et fonction du sacrifice \\[0pt]
N'échange-t-on que ce qui a de la valeur ? \\[0pt]
N'échange-t-on que des biens ? \\[0pt]
N'échange-t-on que des symboles ? \\[0pt]
Névroses et psychoses \\[0pt]
Normalité et normativité \\[0pt]
Norme et moyenne \\[0pt]
Objectivité et subjectivité \\[0pt]
Observer, décrire et expliquer \\[0pt]
Offre et demande \\[0pt]
Origine des hommes et principe de l'humanité \\[0pt]
Où commence le territoire de la psychologie ? \\[0pt]
Outil et machine \\[0pt]
Parler, est-ce communiquer ? \\[0pt]
Penser les sociétés comme des organismes \\[0pt]
Perdre la face \\[0pt]
Peuple et culture \\[0pt]
Peuple et masse \\[0pt]
Peut-il y avoir une anthropologie non anthropocentriste ? \\[0pt]
Peut-il y avoir une ethnologie non ethnocentriste ? \\[0pt]
Peut-il y avoir une science de l'éducation ? \\[0pt]
Peut-il y avoir une science des signes ? \\[0pt]
Peut-on changer de culture ? \\[0pt]
Peut-on comprendre les actions humaines ? \\[0pt]
Peut-on connaître une vie humaine ? \\[0pt]
Peut-on douter de l'humanité d'un homme ? \\[0pt]
Peut-on encore parler d'une nature humaine ? \\[0pt]
Peut-on étudier l'homme de l'extérieur ? \\[0pt]
Peut-on hiérarchiser les sociétés ? \\[0pt]
Peut-on mesurer les phénomènes sociaux ? \\[0pt]
Peut-on objectiver le psychisme ? \\[0pt]
Peut-on parler de capital culturel ? \\[0pt]
Peut-on prévoir les hommes ? \\[0pt]
Phénomène ou fait social ? \\[0pt]
Philosophie, psychologie, sociologie \\[0pt]
Position, rôle, fonction \\[0pt]
Pourquoi des cérémonies ? \\[0pt]
Pourquoi des historiens ? \\[0pt]
Pourquoi des rites ? \\[0pt]
Pourquoi des sociologues ? \\[0pt]
Pourquoi les sciences humaines s'intéressent-elles à la parenté ? \\[0pt]
Pourquoi l'ethnologue s'intéresse-t-il à la vie urbaine ? \\[0pt]
Pourquoi l'homme est-il l'objet de plusieurs sciences ? \\[0pt]
Pourquoi voyager ? \\[0pt]
Preuve et résultat dans les sciences humaines \\[0pt]
Prévoir les comportements humains \\[0pt]
Primitif ou premier ? \\[0pt]
Psychanalyse et critique littéraire \\[0pt]
Psychanalyse et esthétique \\[0pt]
Psychanalyse et interprétation \\[0pt]
Psychologie et contrôle des comportements \\[0pt]
Psychologie et ethnologie \\[0pt]
Psychologie et neurosciences \\[0pt]
Psychologie et philosophie \\[0pt]
Puis-je juger une culture à laquelle je n'appartiens pas ? \\[0pt]
Quantification et intelligibilité \\[0pt]
Quantifier \\[0pt]
Qu'échange-t-on ? \\[0pt]
Que dit l'interdit ? \\[0pt]
Quel est le statut de la preuve en sciences humaines et sociales ? \\[0pt]
Quel est le sujet de l'histoire ? \\[0pt]
Quel est le sujet des sciences sociales ? \\[0pt]
Quel est l'homme de l'ethnologie ? \\[0pt]
Quel est l'humain des sciences humaines ? \\[0pt]
Quel est l'objectif des sciences humaines ? \\[0pt]
Quel est l'objet de la psychologie ? \\[0pt]
Quel est l'objet de la psychologie sociale ? \\[0pt]
« Quelle chimère est-ce donc que l'homme » ? \\[0pt]
Quelle est la place du sujet dans les sciences humaines ? \\[0pt]
Quelle place les sciences humaines doivent-elles accorder à la subjectivité ? \\[0pt]
Quelle place pour la mesure dans les sciences de l'homme ? \\[0pt]
Quelle politique fait-on avec les sciences humaines ? \\[0pt]
Quelle responsabilité publique pour le chercheur en sciences sociales ? \\[0pt]
Quelles lois pour les sciences humaines ? \\[0pt]
Quel rôle les statistiques jouent-elles dans les sciences humaines ? \\[0pt]
Quels critères de démarcation pour les sciences sociales ? \\[0pt]
Quel sens donner au projet de forger une langue parfaite ? \\[0pt]
Quels matériaux pour les sciences humaines et sociales ? \\[0pt]
Quel type de réalité faut-il attribuer à l'inconscient en psychanalyse ? \\[0pt]
Que nous apprend la psychanalyse de l'homme ? \\[0pt]
Que nous apprend la sociologie des sciences ? \\[0pt]
Que nous apprennent les algorithmes sur nos sociétés ? \\[0pt]
Que nous apprennent les faits divers ? \\[0pt]
Que signifie prouver en sciences sociales ? \\[0pt]
Que sondent les sondages d'opinion ? \\[0pt]
Qu'est-ce que l'anthropologie politique ? \\[0pt]
Qu'est-ce que la psychanalyse nous apprend sur les sociétés humaines ? \\[0pt]
Qu'est-ce que la sociologie nous apprend sur la culture ? \\[0pt]
Qu'est-ce que les sciences humaines apportent à la critique littéraire ? \\[0pt]
Qu'est-ce que les sciences religieuses nous apprennent de la religion ? \\[0pt]
Qu'est-ce que le symbolique ? \\[0pt]
Qu'est-ce que l'identité sociale des individus ? \\[0pt]
Qu'est-ce que l'individualisme méthodologique ? \\[0pt]
Qu'est-ce que l'interprétation des Écritures nous apprend sur les religions ? \\[0pt]
Qu'est-ce que lire ? \\[0pt]
Qu'est-ce qu'être comportementaliste ? \\[0pt]
Qu'est-ce qu'expliquer un comportement ? \\[0pt]
Qu'est-ce qui échappe aux sciences humaines ? \\[0pt]
Qu'est-ce qui est commun à toutes les langues ? \\[0pt]
Qu'est-ce qui est humain ? \\[0pt]
Qu'est-ce qui est normal ? \\[0pt]
Qu'est-ce qui fait des sciences humaines des sciences ? \\[0pt]
Qu'est-ce qui n'est pas normal ? \\[0pt]
Qu'est-ce qui rend l'objectivité difficile dans les sciences humaines ? \\[0pt]
Qu'est-ce qu'un acte symbolique ? \\[0pt]
Qu'est-ce qu'un capital culturel ? \\[0pt]
Qu'est-ce qu'un civilisé ? \\[0pt]
Qu'est-ce qu'un collectif ? \\[0pt]
Qu'est-ce qu'un corps social ? \\[0pt]
Qu'est-ce qu'un document ? \\[0pt]
Qu'est-ce qu'une culture ? \\[0pt]
Qu'est-ce qu'une époque ? \\[0pt]
Qu'est-ce qu'une institution ? \\[0pt]
Qu'est-ce qu'une logique sociale ? \\[0pt]
Qu'est-ce qu'une mentalité collective ? \\[0pt]
Qu'est-ce qu'un énoncé scientifique sur l'homme ? \\[0pt]
Qu'est-ce qu'une norme sociale ? \\[0pt]
Qu'est-ce qu'une période en histoire ? \\[0pt]
Qu'est-ce qu'une règle sociale ? \\[0pt]
Qu'est-ce qu'une secte ? \\[0pt]
Qu'est-ce qu'une société mondialisée ? \\[0pt]
Qu'est-ce qu'une société primitive ? \\[0pt]
Qu'est-ce qu'une structure ? \\[0pt]
Qu'est-ce qu'un fait de société ? \\[0pt]
Qu'est-ce qu'un fait social ? \\[0pt]
Qu'est-ce qu'un imaginaire social ? \\[0pt]
Qu'est-ce qu'un individu ? \\[0pt]
Qu'est-ce qu'un marginal ? \\[0pt]
Qu'est-ce qu'un mécanisme social ? \\[0pt]
Qu'est-ce qu'un monument ? \\[0pt]
Qu'est-ce qu'un mythe ? \\[0pt]
Qu'est-ce qu'un patrimoine ? \\[0pt]
Qu'est-ce qu'un peuple ? \\[0pt]
Qu'est-ce qu'un primitif ? \\[0pt]
Qu'est-ce qu'un symbole ? \\[0pt]
Qu'est-ce qu'un symptôme ? \\[0pt]
Qu'est-ce qu'un trouble social ? \\[0pt]
Que valent les classifications en sciences humaines ? \\[0pt]
Que vaut une connaissance empirique de l'homme ? \\[0pt]
Qui a une histoire ? \\[0pt]
Qui domine ? \\[0pt]
Qui est l'homme des sciences humaines ? \\[0pt]
Qui parle ? \\[0pt]
Recherche et militantisme \\[0pt]
Récit et explication en histoire \\[0pt]
Récit et fiction \\[0pt]
Relativisme et sciences humaines \\[0pt]
Reproduire \\[0pt]
Richesse et pauvreté \\[0pt]
Rites et cérémonies \\[0pt]
Rythmes sociaux, rythmes naturels \\[0pt]
S'associer \\[0pt]
Savoir et pouvoir \\[0pt]
Science et subjectivité \\[0pt]
Sciences humaines et déterminisme \\[0pt]
Sciences humaines et herméneutique \\[0pt]
Sciences humaines et idéologie \\[0pt]
Sciences humaines et liberté sont-elles compatibles ? \\[0pt]
Sciences humaines et littérature \\[0pt]
Sciences humaines et naturalisme \\[0pt]
Sciences humaines et nature humaine \\[0pt]
Sciences humaines et objectivité \\[0pt]
Sciences humaines et philosophie \\[0pt]
Sciences humaines et vérité \\[0pt]
Sciences humaines ou sciences sociales ? \\[0pt]
Sciences humaines, sciences de l'homme \\[0pt]
Sciences sociales et humanités \\[0pt]
Sciences sociales et questions environnementales \\[0pt]
Se distinguer \\[0pt]
Sens et limites de la notion de capital culturel \\[0pt]
Sens et structure \\[0pt]
S'entendre \\[0pt]
Sexe et genre \\[0pt]
Sexualité et nature \\[0pt]
Signes, traces et indices \\[0pt]
Société et liberté \\[0pt]
Société et mouvement \\[0pt]
Sociologie et économie \\[0pt]
Sommes-nous responsables de notre être social ? \\[0pt]
Sommes-nous tous contemporains ? \\[0pt]
Structure et événement \\[0pt]
Suivre la règle \\[0pt]
Terres et territoires \\[0pt]
Territoire et peuplement \\[0pt]
Tout ce qui est humain est-il signifiant ? \\[0pt]
Tout ce qui est humain nous est-il familier ? \\[0pt]
Toute psychologie est-elle normalisatrice ? \\[0pt]
Tout est-il relatif ? \\[0pt]
Tout peut-il se quantifier ? \\[0pt]
Transmettre \\[0pt]
Une culture constitue-t-elle un paradigme ? \\[0pt]
Une culture de masse est-elle une culture ? \\[0pt]
Une interprétation peut-elle être vraie ? \\[0pt]
Une science de la culture est-elle possible ? \\[0pt]
Une science de l'imaginaire est-elle possible ? \\[0pt]
Une seconde nature \\[0pt]
Une théorie de la culture peut-elle être scientifique ? \\[0pt]
Un savoir sur l'homme peut-il être séparé d'un pouvoir sur les hommes ? \\[0pt]
Voyager \\[0pt]
Voyager entre les mondes \\[0pt]
Y a-t-il continuité ou discontinuité entre la pensée mythique et la science ? \\[0pt]
Y a-t-il des actions collectives ? \\[0pt]
Y a-t-il des formes élémentaires de la société ? \\[0pt]
Y a-t-il des limites à la connaissance de l'homme par les sciences ? \\[0pt]
Y a-t-il des lois de l'histoire ? \\[0pt]
Y a-t-il des lois du comportement humain ? \\[0pt]
Y a-t-il des lois du social ? \\[0pt]
Y a-t-il des lois en histoire ? \\[0pt]
Y a-t-il des lois sociales ? \\[0pt]
Y a-t-il des mécanismes humains ? \\[0pt]
Y a-t-il des mentalités collectives ? \\[0pt]
Y a-t-il des passions collectives ? \\[0pt]
Y a-t-il des pathologies du social ? \\[0pt]
Y a-t-il des pathologies sociales ? \\[0pt]
Y a-t-il des phénomènes psychiques ? \\[0pt]
Y a-t-il des sciences de l'éducation ? \\[0pt]
Y a-t-il des sociétés archaïques ? \\[0pt]
Y a-t-il des sociétés sans État ? \\[0pt]
Y a-t-il des sociétés sans histoire ? \\[0pt]
Y a-t-il des universaux anthropologiques ? \\[0pt]
Y a-t-il du déterminisme dans les sciences humaines ? \\[0pt]
Y a-t-il encore des mythologies ? \\[0pt]
Y a-t-il encore une sphère privée ? \\[0pt]
Y a-t-il lieu de distinguer instinct et pulsion ? \\[0pt]
Y a-t-il lieu d'opposer la philosophie et les sciences humaines ? \\[0pt]
Y a-t-il quelque chose de subversif dans les sciences humaines ? \\[0pt]
Y a-t-il un déterminisme dans les sciences humaines ? \\[0pt]
Y a-t-il une bonne éducation ? \\[0pt]
Y a-t-il une causalité historique ? \\[0pt]
Y a-t-il une formation de la pensée logique ? \\[0pt]
Y a-t-il une hiérarchie des sciences sociales ? \\[0pt]
Y a-t-il une histoire de la folie ? \\[0pt]
Y a-t-il une histoire de la vie quotidienne ? \\[0pt]
Y a-t-il une intentionnalité collective ? \\[0pt]
Y a-t-il une méthode propre aux sciences humaines ? \\[0pt]
Y a-t-il une nature humaine ? \\[0pt]
Y a-t-il une place dans les sciences humaines pour la catégorie de personne ? \\[0pt]
Y a-t-il une rationalité du marché ? \\[0pt]
Y a-t-il une science de la vie mentale ? \\[0pt]
Y a-t-il une spécificité des sciences humaines ? \\[0pt]
Y a-t-il une unité en psychologie ? \\[0pt]
Y a-t-il un inconscient collectif ? \\[0pt]
Y a-t-il un progrès des/en sciences humaines ? \\[0pt]
Y a-t-il un traitement scientifique du langage ? \\[0pt]


\section{Tri par type}
\label{sec:org8cb259d}

\subsection{Question}
\label{sec:org2ed97d8}

\noindent
2+2 pourrait-il ne pas être égal à 4 ? \\[0pt]
Abstraire, est-ce se couper du réel ? \\[0pt]
À chacun sa vérité ? \\[0pt]
Admettre le hasard, est-ce nier l'ordre de la nature ? \\[0pt]
Admettre une cause première, est-ce faire une pétition de principe ? \\[0pt]
Agir en politique, est-ce agir dans l'incertain ? \\[0pt]
Agir justement fait-il de moi un homme juste ? \\[0pt]
Agir moralement, est-ce lutter contre ses idées ? \\[0pt]
Agir moralement, est-ce lutter contre soi-même ? \\[0pt]
Agir par devoir, est-ce agir contre son intérêt ? \\[0pt]
Agir par devoir, est-ce renoncer à ses intérêts ? \\[0pt]
Agir volontairement, est-ce nécessairement savoir ce que l'on fait ? \\[0pt]
Ai-je des devoirs envers moi-même ? \\[0pt]
Ai-je intérêt à la liberté d'autrui ? \\[0pt]
Ai-je le pouvoir de me changer ? \\[0pt]
Ai-je un corps ? \\[0pt]
Ai-je un corps ou suis-je mon corps ? \\[0pt]
Ai-je une âme ? \\[0pt]
Aimer, est-ce vraiment connaître ? \\[0pt]
Aimer peut-il être un devoir ? \\[0pt]
« Aimer » se dit-il en un seul sens ? \\[0pt]
Animal politique ou social ? \\[0pt]
Appartenons-nous à une culture ? \\[0pt]
Appartient-il aux lois d'éduquer les hommes ? \\[0pt]
Apprend-on à aimer ? \\[0pt]
Apprend-on à être artiste ? \\[0pt]
Apprend-on à être libre ? \\[0pt]
Apprend-on à penser ? \\[0pt]
Apprend-on à percevoir ? \\[0pt]
Apprend-on à percevoir la beauté ? \\[0pt]
Apprend-on à voir ? \\[0pt]
Apprendre à voir ? \\[0pt]
Apprendre s'apprend-il ? \\[0pt]
À quelle condition une démarche est-elle scientifique ? \\[0pt]
À quelle condition un travail est-il humain ? \\[0pt]
À quelle expérience l'art nous convie-t-il ? \\[0pt]
À quelle généralité peuvent prétendre les sciences humaines ? \\[0pt]
À quelle objectivité peuvent prétendre les sciences sociales ? \\[0pt]
À quelles conditions est-il acceptable de travailler ? \\[0pt]
À quelles conditions le vivant peut-il être objet de science ? \\[0pt]
À quelles conditions penser une seconde nature ? \\[0pt]
À quelles conditions peut-on dire qu'une action est historique ? \\[0pt]
À quelles conditions un choix peut-il être rationnel ? \\[0pt]
À quelles conditions une activité est-elle un travail ? \\[0pt]
À quelles conditions une croyance devient-elle religieuse ? \\[0pt]
À quelles conditions une démarche est-elle scientifique ? \\[0pt]
À quelles conditions une expérience est-elle possible ? \\[0pt]
À quelles conditions une explication est-elle scientifique ? \\[0pt]
À quelles conditions une hypothèse est-elle scientifique ? \\[0pt]
À quelles conditions un énoncé est-il doué de sens ? \\[0pt]
À quelles conditions une pensée est-elle libre ? \\[0pt]
À quelles conditions un État peut-il être juste ? \\[0pt]
À quelles conditions une théorie est-elle scientifique ? \\[0pt]
À quelles conditions une théorie peut-elle être scientifique ? \\[0pt]
À quelles conditions une vérité s'impose-t-elle à nous ? \\[0pt]
À quelles conditions un jugement est-il moral ? \\[0pt]
À quelles conditions y a-t-il progrès technique ? \\[0pt]
À quelque chose le malheur peut-il être bon ? \\[0pt]
À quelque chose, le malheur peut-il être bon ? \\[0pt]
À quels signes reconnaît-on la vérité ? \\[0pt]
À qui appartient-il de décider du juste et de l'injuste ? \\[0pt]
À qui appartient la Terre ? \\[0pt]
À qui devons-nous obéir ? \\[0pt]
À qui dois-je la vérité ? \\[0pt]
À qui doit-on la vérité ? \\[0pt]
À qui doit-on le respect ? \\[0pt]
À qui doit-on obéir ? \\[0pt]
À qui est mon corps ? \\[0pt]
À qui faut-il obéir ? \\[0pt]
À qui la faute ? \\[0pt]
À qui le pouvoir appartient-il ? \\[0pt]
À qui peut-on faire confiance ? \\[0pt]
À qui profite le crime ? \\[0pt]
À qui profite le travail ? \\[0pt]
À qui se fier ? \\[0pt]
À quoi bon ? \\[0pt]
À quoi bon avoir mauvaise conscience ? \\[0pt]
À quoi bon chercher à prouver l'existence de Dieu ? \\[0pt]
À quoi bon critiquer les autres ? \\[0pt]
À quoi bon culpabiliser ? \\[0pt]
À quoi bon démontrer ? \\[0pt]
À quoi bon des héros ? \\[0pt]
À quoi bon des philosophes ? \\[0pt]
À quoi bon discuter ? \\[0pt]
À quoi bon imaginer la cité idéale ? \\[0pt]
À quoi bon imiter la nature ? \\[0pt]
À quoi bon la morale ? \\[0pt]
À quoi bon la science ? \\[0pt]
À quoi bon les regrets ? \\[0pt]
À quoi bon les romans ? \\[0pt]
À quoi bon les sciences humaines et sociales ? \\[0pt]
À quoi bon les systèmes ? \\[0pt]
À quoi bon parler ? \\[0pt]
À quoi bon penser la fin du monde ? \\[0pt]
À quoi bon raconter des histoires ? \\[0pt]
À quoi bon rêver ? \\[0pt]
À quoi bon se parler ? \\[0pt]
À quoi bon voyager ? \\[0pt]
À quoi est-il impossible de s'habituer ? \\[0pt]
À quoi est-il nécessaire d'obéir ? \\[0pt]
À quoi faut-il être fidèle ? \\[0pt]
À quoi faut-il renoncer ? \\[0pt]
À quoi faut-il renoncer pour dialoguer ? \\[0pt]
À quoi juger l'action d'un gouvernement ? \\[0pt]
À quoi la conscience nous donne-t-elle accès ? \\[0pt]
À quoi la logique peut-elle servir dans les sciences ? \\[0pt]
À quoi la perception donne-t-elle accès ? \\[0pt]
À quoi la religion sert-elle ? \\[0pt]
À quoi l'art est-il bon ? \\[0pt]
À quoi l'art nous rend-il sensibles ? \\[0pt]
À quoi la valeur d'un homme se mesure-t-elle ? \\[0pt]
À quoi les sciences nous forment-elles ? \\[0pt]
À quoi nos illusions tiennent-elles ? \\[0pt]
À quoi peut-on reconnaître la vérité ? \\[0pt]
À quoi peut-on reconnaître une œuvre d'art ? \\[0pt]
À quoi reconnaît-on la rationalité ? \\[0pt]
À quoi reconnaît-on la vérité ? \\[0pt]
À quoi reconnaît-on le faux ? \\[0pt]
À quoi reconnaît-on le réel ? \\[0pt]
À quoi reconnaît-on l'humanité en chaque homme ? \\[0pt]
À quoi reconnaît-on l'injustice ? \\[0pt]
À quoi reconnaît-on qu'une activité est un travail ? \\[0pt]
À quoi reconnaît-on qu'une expérience est scientifique ? \\[0pt]
À quoi reconnaît-on qu'une pensée est vraie ? \\[0pt]
À quoi reconnaît-on qu'une politique est juste ? \\[0pt]
À quoi reconnaît-on qu'une science est une science ? \\[0pt]
À quoi reconnaît-on qu'une théorie est scientifique ? \\[0pt]
À quoi reconnaît-on qu'un événement est historique ? \\[0pt]
À quoi reconnaît-on un acte libre ? \\[0pt]
À quoi reconnaît-on un acte vraiment libre ? \\[0pt]
À quoi reconnaît-on un bon artisan ? \\[0pt]
À quoi reconnaît-on un bon gouvernement ? \\[0pt]
À quoi reconnaît-on un comportement humain ? \\[0pt]
À quoi reconnaît-on un devoir ? \\[0pt]
À quoi reconnaît-on une action morale ? \\[0pt]
À quoi reconnaît-on une attitude religieuse ? \\[0pt]
À quoi reconnaît-on une bonne interprétation ? \\[0pt]
À quoi reconnaît-on une idéologie ? \\[0pt]
À quoi reconnaît-on une œuvre d'art ? \\[0pt]
À quoi reconnaît-on une religion ? \\[0pt]
À quoi reconnaît-on une science ? \\[0pt]
À quoi reconnaît-on une théorie scientifique ? \\[0pt]
À quoi reconnaît-on un être vivant ? \\[0pt]
À quoi renonçons-nous quand nous obéissons ? \\[0pt]
À quoi rêve-t-on ? \\[0pt]
À quoi sert la bioéthique ? \\[0pt]
À quoi sert la connaissance du passé ? \\[0pt]
À quoi sert la critique ? \\[0pt]
À quoi sert la dialectique ? \\[0pt]
À quoi sert la logique ? \\[0pt]
À quoi sert la mémoire ? \\[0pt]
À quoi sert la métaphysique ? \\[0pt]
À quoi sert la négation ? \\[0pt]
À quoi sert la notion de contrat social ? \\[0pt]
À quoi sert la notion d'état de nature ? \\[0pt]
À quoi sert la technique ? \\[0pt]
À quoi sert la théodicée ? \\[0pt]
À quoi sert la vérité ? \\[0pt]
À quoi sert le contrat social ? \\[0pt]
À quoi sert l'écriture ? \\[0pt]
À quoi sert l'État ? \\[0pt]
À quoi sert l'histoire ? \\[0pt]
À quoi sert l'intuition ? \\[0pt]
À quoi sert l'ontologie ? \\[0pt]
À quoi sert un critique d'art ? \\[0pt]
À quoi sert une métaphore ? \\[0pt]
À quoi sert un exemple ? \\[0pt]
À quoi servent les cérémonies ? \\[0pt]
À quoi servent les constitutions ? \\[0pt]
À quoi servent les croyances ? \\[0pt]
À quoi servent les doctrines morales ? \\[0pt]
À quoi servent les élections ? \\[0pt]
À quoi servent les émotions ? \\[0pt]
À quoi servent les encyclopédies ? \\[0pt]
À quoi servent les expériences ? \\[0pt]
À quoi servent les fictions ? \\[0pt]
À quoi servent les hypothèses ? \\[0pt]
À quoi servent les images ? \\[0pt]
À quoi servent les lois ? \\[0pt]
À quoi servent les machines ? \\[0pt]
À quoi servent les mythes ? \\[0pt]
À quoi servent les œuvres d'art ? \\[0pt]
À quoi servent les preuves ? \\[0pt]
À quoi servent les preuves de l'existence de Dieu ? \\[0pt]
À quoi servent les règles ? \\[0pt]
À quoi servent les religions ? \\[0pt]
À quoi servent les sciences ? \\[0pt]
À quoi servent les sciences humaines ? \\[0pt]
À quoi servent les statistiques ? \\[0pt]
À quoi servent les symboles ? \\[0pt]
À quoi servent les théories ? \\[0pt]
À quoi servent les utopies ? \\[0pt]
À quoi servent les voyages ? \\[0pt]
À quoi suis-je obligé ? \\[0pt]
À quoi tenons-nous ? \\[0pt]
À quoi tient la fermeté du vouloir ? \\[0pt]
À quoi tient la force de l'État ? \\[0pt]
À quoi tient la force des religions ? \\[0pt]
À quoi tient l'autorité ? \\[0pt]
À quoi tient la valeur d'une pensée ? \\[0pt]
À quoi tient la vérité d'une interprétation ? \\[0pt]
À quoi tient l'efficacité d'une technique ? \\[0pt]
À quoi tient le pouvoir des mots ? \\[0pt]
À quoi tient le pouvoir des sciences ? \\[0pt]
À quoi tient notre humanité ? \\[0pt]
Arrive-t-il que l'impossible se produise ? \\[0pt]
À science nouvelle, nouvelle philosophie ? \\[0pt]
A-t-on besoin de certitudes ? \\[0pt]
A-t-on besoin de fonder la connaissance ? \\[0pt]
A-t-on besoin de maîtres ? \\[0pt]
A-t-on besoin de spécialistes ? \\[0pt]
A-t-on besoin de spécialistes en politique ? \\[0pt]
A-t-on besoin des poètes ? \\[0pt]
A-t-on besoin d'experts ? \\[0pt]
A-t-on besoin d'un chef ? \\[0pt]
A-t-on des devoirs envers les animaux ? \\[0pt]
A-t-on des devoirs envers l'humanité ? \\[0pt]
A-t-on des devoirs envers qui n'a aucun droit ? \\[0pt]
A-t-on des devoirs envers soi-même ? \\[0pt]
A-t-on des droits contre l'État ? \\[0pt]
A-t-on des raisons de croire ? \\[0pt]
A-t-on des raisons de croire ce qu'on croit ? \\[0pt]
A-t-on expliqué un phénomène quand on l'a rapporté à des lois ? \\[0pt]
A-t-on intérêt à tout savoir ? \\[0pt]
A-t-on le droit de faire tout ce qui est permis par la loi ? \\[0pt]
A-t-on le droit de mentir ? \\[0pt]
A-t-on le droit de résister ? \\[0pt]
A-t-on le droit de se désintéresser de la politique ? \\[0pt]
A-t-on le droit de se révolter ? \\[0pt]
A-t-on le droit de s'évader ? \\[0pt]
A-t-on l'obligation de pardonner ? \\[0pt]
A-t-on raison de mettre la technique en accusation ? \\[0pt]
A-t-on raison de tenir à la démocratie ? \\[0pt]
Au-delà de la nature ? \\[0pt]
Au nom de qui rend-on justice ? \\[0pt]
Au nom de quoi le plaisir serait-il condamnable ? \\[0pt]
Au nom de quoi rend-on justice ? \\[0pt]
Autrui, est-ce n'importe quel autre ? \\[0pt]
Autrui est-il aimable ? \\[0pt]
Autrui est-il inconnaissable ? \\[0pt]
Autrui est-il mon prochain ? \\[0pt]
Autrui est-il mon semblable ? \\[0pt]
Autrui est-il pour moi un mystère ? \\[0pt]
Autrui est-il un autre moi ? \\[0pt]
Autrui est-il un autre moi-même ? \\[0pt]
Autrui me connaît-il mieux que moi-même ? \\[0pt]
Autrui m'est-il étranger ? \\[0pt]
Avez-vous une âme ? \\[0pt]
Avoir des ennemis, est-ce le principe de la politique ? \\[0pt]
Avoir la parole, est-ce avoir le pouvoir ? \\[0pt]
Avoir raison, est-ce nécessairement être raisonnable ? \\[0pt]
Avons-nous à apprendre des images ? \\[0pt]
Avons-nous accès aux choses-mêmes ? \\[0pt]
Avons-nous besoin d'amis ? \\[0pt]
Avons-nous besoin de cérémonies ? \\[0pt]
Avons-nous besoin de Dieu ? \\[0pt]
Avons-nous besoin de héros ? \\[0pt]
Avons-nous besoin de l'État ? \\[0pt]
Avons-nous besoin de lois ? \\[0pt]
Avons-nous besoin de maîtres ? \\[0pt]
Avons-nous besoin de métaphysique ? \\[0pt]
Avons-nous besoin de méthodes ? \\[0pt]
Avons-nous besoin de nous tromper nous-même ? \\[0pt]
Avons-nous besoin de partis politiques ? \\[0pt]
Avons-nous besoin de rêver ? \\[0pt]
Avons-nous besoin de spectacles ? \\[0pt]
Avons-nous besoin de traditions ? \\[0pt]
Avons-nous besoin d'être gouvernés ? \\[0pt]
Avons-nous besoin d'experts en matière d'art ? \\[0pt]
Avons-nous besoin d'un chef ? \\[0pt]
Avons-nous besoin d'une conception métaphysique du monde ? \\[0pt]
Avons-nous besoin d'une définition de l'art ? \\[0pt]
Avons-nous besoin d'une philosophie politique ? \\[0pt]
Avons-nous besoin d'un libre arbitre ? \\[0pt]
Avons-nous besoin d'utopies ? \\[0pt]
Avons-nous des devoirs à l'égard de la vérité ? \\[0pt]
Avons-nous des devoirs envers la nature ? \\[0pt]
Avons-nous des devoirs envers le passé ? \\[0pt]
Avons-nous des devoirs envers les animaux ? \\[0pt]
Avons-nous des devoirs envers les autres êtres vivants ? \\[0pt]
Avons-nous des devoirs envers les générations futures ? \\[0pt]
Avons-nous des devoirs envers les morts ? \\[0pt]
Avons-nous des devoirs envers le vivant ? \\[0pt]
Avons-nous des devoirs envers nous-mêmes ? \\[0pt]
Avons-nous des devoirs envers tous les vivants ? \\[0pt]
Avons-nous des droits sur la nature ? \\[0pt]
Avons-nous des raisons d'espérer ? \\[0pt]
Avons-nous encore besoin de la nature ? \\[0pt]
Avons-nous intérêt à la liberté d'autrui ? \\[0pt]
Avons-nous le devoir de dire la vérité ? \\[0pt]
Avons-nous le devoir d'être heureux ? \\[0pt]
Avons-nous le devoir de vivre ? \\[0pt]
Avons-nous le droit de juger autrui ? \\[0pt]
Avons-nous le droit d'être heureux ? \\[0pt]
Avons-nous le temps d'apprendre à vivre ? \\[0pt]
Avons-nous peur de la liberté ? \\[0pt]
Avons-nous raison de croire ? \\[0pt]
Avons-nous raison d'exiger toujours des raisons ? \\[0pt]
Avons-nous un accès direct aux phénomènes ? \\[0pt]
Avons-nous un corps ? \\[0pt]
Avons-nous un devoir de vérité ? \\[0pt]
Avons-nous un droit au droit ? \\[0pt]
Avons-nous une âme ? \\[0pt]
Avons-nous une identité ? \\[0pt]
Avons-nous une intuition du temps ? \\[0pt]
Avons-nous une obligation envers les générations à venir ? \\[0pt]
Avons-nous une responsabilité envers le passé ? \\[0pt]
Avons-nous un libre arbitre ? \\[0pt]
Avons-nous un monde commun ? \\[0pt]
Axiomatiser, est-ce fonder ? \\[0pt]
Bien agir, est-ce nécessairement faire son devoir ? \\[0pt]
Bien agir, est-ce toujours être moral ? \\[0pt]
Bien parler, est-ce bien penser ? \\[0pt]
Bien parler exige-t-il des qualités morales ? \\[0pt]
Cela a-t-il un sens de penser par soi-même ? \\[0pt]
Ce que je pense est-il nécessairement vrai ? \\[0pt]
Ce que la morale autorise, l'État peut-il légitimement l'interdire ? \\[0pt]
Ce que la technique rend possible, peut-on jamais en empêcher la réalisation ? \\[0pt]
Ce que nous avons le devoir de faire peut-il toujours s'exprimer sous forme de loi ? \\[0pt]
Ce qui dépasse la raison est-il nécessairement irréel ? \\[0pt]
Ce qui doit-être, est-ce autre chose que ce qui est ? \\[0pt]
Ce qui est contingent peut-il être objet de science ? \\[0pt]
Ce qui est contradictoire peut-il exister ? \\[0pt]
Ce qui est démontré est-il nécessairement vrai ? \\[0pt]
Ce qui est faux est-il dénué de sens ? \\[0pt]
Ce qui est ordinaire est-il normal ? \\[0pt]
Ce qui est passé est-il dépassé ? \\[0pt]
Ce qui est subjectif est-il arbitraire ? \\[0pt]
Ce qui est vrai est-il toujours vérifiable ? \\[0pt]
Ce qui ne peut s'acheter est-il dépourvu de valeur ? \\[0pt]
Ce qui n'est pas démontré peut-il être vrai ? \\[0pt]
Ce qui n'est pas matériel peut-il être réel ? \\[0pt]
Ce qui n'est pas réel est-il impossible ? \\[0pt]
Ce qui vaut en théorie vaut-il toujours en pratique ? \\[0pt]
Certaines œuvres d'art ont-elles plus de valeur que d'autres ? \\[0pt]
Ceux qui oppriment sont-ils libres ? \\[0pt]
Ceux qui savent doivent-ils gouverner ? \\[0pt]
Chacun a-t-il le droit d'invoquer sa vérité ? \\[0pt]
Chacun a-t-il sa propre vérité ? \\[0pt]
Changer, est-ce devenir un autre ? \\[0pt]
Changer, est-ce se trahir ? \\[0pt]
Change-t-on avec le temps ? \\[0pt]
Chaque science porte-t-elle une métaphysique qui lui est propre ? \\[0pt]
Châtier, est ce faire honneur au criminel ? \\[0pt]
Chercher son intérêt, est-ce être immoral ? \\[0pt]
Choisir, est-ce renoncer ? \\[0pt]
Choisir ses souvenirs ? \\[0pt]
Choisissons-nous qui nous sommes ? \\[0pt]
Choisit-on ses souvenirs ? \\[0pt]
Choisit-on son corps ? \\[0pt]
Cité juste ou citoyen juste ? \\[0pt]
Citoyen du monde ? \\[0pt]
Classer, est-ce penser ? \\[0pt]
Comment assumer les conséquences de ses actes ? \\[0pt]
Comment autrui peut-il m'aider à rechercher le bonheur ? \\[0pt]
Comment bien appliquer une règle ? \\[0pt]
Comment bien vivre ? \\[0pt]
Comment caractériser une idée confuse ? \\[0pt]
Comment chercher ce qu'on ignore ? \\[0pt]
Comment comprendre les faits sociaux ? \\[0pt]
Comment comprendre une croyance qu'on ne partage pas ? \\[0pt]
Comment conduire ses pensées ? \\[0pt]
Comment connaissons-nous nos devoirs ? \\[0pt]
Comment connaître l'autre ? \\[0pt]
Comment connaître le passé ? \\[0pt]
Comment connaître l'inconscient ? \\[0pt]
Comment connaître nos devoirs ? \\[0pt]
Comment considérer le devenir ? \\[0pt]
Comment croire au progrès ? \\[0pt]
Comment décider, sinon à la majorité ? \\[0pt]
Comment définir la démocratie ? \\[0pt]
Comment définir la raison ? \\[0pt]
Comment définir le bien commun ? \\[0pt]
Comment définir le laid ? \\[0pt]
Comment définir le peuple ? \\[0pt]
Comment définir le style ? \\[0pt]
Comment définir l'État de droit ? \\[0pt]
Comment définir l'excès ? \\[0pt]
Comment définir l'infini ? \\[0pt]
Comment déterminer le sens d'une conduite ? \\[0pt]
Comment deux personnes peuvent-elles partager la même pensée ? \\[0pt]
Comment devient-on artiste ? \\[0pt]
Comment devient-on citoyen ? \\[0pt]
Comment devient-on raisonnable ? \\[0pt]
Comment dire la vérité ? \\[0pt]
Comment dire l'individuel ? \\[0pt]
Comment distinguer culture et civilisation ? \\[0pt]
Comment distinguer désirs et besoins ? \\[0pt]
Comment distinguer entre l'amour et l'amitié ? \\[0pt]
Comment distinguer l'amour de l'amitié ? \\[0pt]
Comment distinguer le rêvé du perçu ? \\[0pt]
Comment distinguer le vrai du faux ? \\[0pt]
Comment distingue-t-on le vrai du réel ? \\[0pt]
Comment entrer dans l'histoire ? \\[0pt]
Comment établir des critères d'équité ? \\[0pt]
Comment être naturel ? \\[0pt]
Comment évaluer la qualité de la vie ? \\[0pt]
Comment évaluer l'art ? \\[0pt]
Comment expliquer l'erreur ? \\[0pt]
Comment expliquer les phénomènes mentaux ? \\[0pt]
Comment exprimer l'identité ? \\[0pt]
Comment faire de l'homme un objet d'étude ? \\[0pt]
Comment fonder la propriété ? \\[0pt]
Comment fonder nos devoirs ? \\[0pt]
Comment fonder une psychologie sociale ? \\[0pt]
Comment juger de la justesse d'une interprétation ? \\[0pt]
Comment juger de la politique ? \\[0pt]
Comment juger d'une œuvre d'art ? \\[0pt]
Comment juger son éducation ? \\[0pt]
Comment juger une œuvre d'art ? \\[0pt]
Comment justifier l'autonomie des sciences de la vie ? \\[0pt]
Comment justifier une méthode ? \\[0pt]
Comment la psychanalyse use-t-elle de l'interprétation ? \\[0pt]
Comment la réalité peut-elle dépasser la fiction ? \\[0pt]
Comment la science progresse-t-elle ? \\[0pt]
Comment la volonté peut-elle être faible ? \\[0pt]
Comment la volonté peut-elle être indéterminée ? \\[0pt]
Comment le devoir peut-il déterminer l'action ? \\[0pt]
Comment le passé nous est-il présent ? \\[0pt]
Comment le passé peut-il demeurer présent ? \\[0pt]
Comment l'erreur est-elle possible ? \\[0pt]
Comment les mathématiques peuvent-elles s'appliquer au réel ? \\[0pt]
Comment les sciences sociales expliquent-elles l'irrationnel ? \\[0pt]
Comment les sociétés changent-elles ? \\[0pt]
Comment l'homme peut-il se représenter le temps ? \\[0pt]
Comment l'inconscient peut-il se manifester ? \\[0pt]
Comment mesurer ? \\[0pt]
Comment mesurer une sensation ? \\[0pt]
Comment ne pas être humaniste ? \\[0pt]
Comment ne pas être libéral ? \\[0pt]
Comment penser la diversité des langues ? \\[0pt]
Comment penser l'écoulement du temps ? \\[0pt]
Comment penser le futur ? \\[0pt]
Comment penser le hasard ? \\[0pt]
Comment penser le mouvement ? \\[0pt]
Comment penser les universaux ? \\[0pt]
Comment penser l'éternel ? \\[0pt]
Comment penser un pouvoir qui ne corrompe pas ? \\[0pt]
Comment percevons-nous l'espace ? \\[0pt]
Comment peut-on choisir entre différentes hypothèses ? \\[0pt]
Comment peut-on définir la politique ? \\[0pt]
Comment peut-on définir un être vivant ? \\[0pt]
Comment peut-on être heureux ? \\[0pt]
« Comment peut-on être persan ? » \\[0pt]
Comment peut-on être sceptique ? \\[0pt]
Comment peut-on savoir que l'on a raison ? \\[0pt]
Comment peut-on se trahir soi-même ? \\[0pt]
Comment prend-on connaissance de ses devoirs ? \\[0pt]
Comment prouver la liberté ? \\[0pt]
Comment puis-je devenir ce que je suis ? \\[0pt]
Comment reconnaît-on une œuvre d'art ? \\[0pt]
Comment reconnaît-on un vivant ? \\[0pt]
Comment réfuter une thèse métaphysique ? \\[0pt]
Comment rendre la justice ? \\[0pt]
Comment représenter la douleur ? \\[0pt]
Comment retrouver la nature ? \\[0pt]
Comment sait-on qu'on se comprend ? \\[0pt]
Comment sait-on qu'une chose existe ? \\[0pt]
Comment s'assurer de ce qui est réel ? \\[0pt]
Comment s'assurer qu'on est libre ? \\[0pt]
Comment savoir ce qui existe ? \\[0pt]
Comment savoir quand nous sommes libres ? \\[0pt]
Comment savoir que l'on est dans l'erreur ? \\[0pt]
Comment savoir quels sont nos devoirs ? \\[0pt]
Comment se libérer du temps ? \\[0pt]
Comment se mettre à la place d'autrui ? \\[0pt]
Comment s'entendre ? \\[0pt]
Comment s'orienter dans la pensée ? \\[0pt]
Comment traiter les animaux ? \\[0pt]
Comment trancher une controverse ? \\[0pt]
Comment vérifier les hypothèses ? \\[0pt]
Comment vivre ensemble ? \\[0pt]
Comment voyager dans le temps ? \\[0pt]
Comparer les cultures ? \\[0pt]
Comprendre, est-ce excuser ? \\[0pt]
Comprendre, est-ce expliquer par les causes ? \\[0pt]
Comprendre, est-ce interpréter ? \\[0pt]
Comprendre le réel, est-ce le dominer ? \\[0pt]
Connaissons-nous la réalité des choses ? \\[0pt]
Connaissons-nous la réalité telle qu'elle est ? \\[0pt]
Connaissons-nous mieux le présent que le passé ? \\[0pt]
Connaît-on jamais pour le plaisir ? \\[0pt]
Connaît-on la vie ou bien connaît-on le vivant ? \\[0pt]
Connaît-on la vie ou connaît-on le vivant ? \\[0pt]
Connaît-on la vie ou le vivant ? \\[0pt]
Connaît-on les choses telles qu'elles sont ? \\[0pt]
Connaît-on pour le plaisir ? \\[0pt]
Connaître, est-ce connaître par les causes ? \\[0pt]
Connaître, est-ce découvrir le réel ? \\[0pt]
Connaître, est-ce dépasser les apparences ? \\[0pt]
Connaître la vie ou le vivant ? \\[0pt]
Connaître les choses, en quoi est-ce déterminer leurs différences ? \\[0pt]
Connaître une chose, est-ce en connaître la cause ? \\[0pt]
Considère-t-on jamais le temps en lui-même ? \\[0pt]
Convient-il d'opposer explication et interprétation ? \\[0pt]
Convient-il d'opposer liberté et nécessité ? \\[0pt]
Croire, est-ce être faible ? \\[0pt]
Croire, est-ce obéir ? \\[0pt]
Croire, est-ce renoncer à l'usage de la raison ? \\[0pt]
Croire, est-ce renoncer à savoir ? \\[0pt]
Croire, est-ce renoncer au savoir ? \\[0pt]
Croire que Dieu existe, est-ce croire en lui ? \\[0pt]
Croit-on ce que l'on veut ? \\[0pt]
Croit-on comme on veut ? \\[0pt]
Dans l'action, est-ce l'intention qui compte ? \\[0pt]
Dans les sciences humaines et sociales, peut-on préconiser l'imitation des sciences de la nature ? \\[0pt]
Dans quel but les hommes se donnent-ils des lois ? \\[0pt]
Dans quelle mesure est-on l'auteur de sa propre vie ? \\[0pt]
Dans quelle mesure l'art est-il un fait social ? \\[0pt]
Dans quelle mesure la technique nous libère-t-elle de la nature ? \\[0pt]
Dans quelle mesure la traduction est-elle une activité cognitive ? \\[0pt]
Dans quelle mesure les sciences ne sont-elles pas à l'abri de l'erreur ? \\[0pt]
Dans quelle mesure le temps nous appartient-il ? \\[0pt]
Dans quelle mesure pouvons-nous échapper à notre passé ? \\[0pt]
Dans quelle mesure suis-je responsable de mon inconscient ? \\[0pt]
Dans quelle mesure toute philosophie est-elle critique du langage ? \\[0pt]
Décrire, est-ce déjà expliquer ? \\[0pt]
Définir, est-ce déterminer l'essence ? \\[0pt]
Définir et démontrer : à quoi bon ? \\[0pt]
Définir l'art : à quoi bon ? \\[0pt]
Définir la vérité, est-ce la connaître ? \\[0pt]
Dégager la structure, est-ce rendre compte de ce qu'est une chose ? \\[0pt]
Délibérer, est-ce être dans l'incertitude ? \\[0pt]
Démontrer est-il le privilège du mathématicien ? \\[0pt]
Démontre-t-on ailleurs qu'en mathématiques ? \\[0pt]
Dépasser les apparences ? \\[0pt]
Dépend-il de nous d'être heureux ? \\[0pt]
Dépend-il de soi d'être heureux ? \\[0pt]
De quel bonheur sommes-nous capables ? \\[0pt]
De quel droit ? \\[0pt]
De quel droit l'État exerce-t-il un pouvoir ? \\[0pt]
De quel droit punit-on ? \\[0pt]
De quel homme parlent les sciences humaines ? \\[0pt]
De quelle certitude la science est-elle capable ? \\[0pt]
De quelle expertise les sciences humaines peuvent-elles se prévaloir ? \\[0pt]
De quelle liberté témoigne l'œuvre d'art ? \\[0pt]
De quelle réalité nos perceptions témoignent-elles ? \\[0pt]
De quelle réalité parlons-nous lorsque nous nous exprimons ? \\[0pt]
De quelle réalité témoignent nos perceptions ? \\[0pt]
De quelle science humaine la folie peut-elle être l'objet ? \\[0pt]
De quelle transgression l'art est-il susceptible ? \\[0pt]
De quelle vérité l'art est-il capable ? \\[0pt]
De quelle vérité l'inconscient est-il porteur ? \\[0pt]
De quelle vérité l'opinion est-elle capable ? \\[0pt]
De qui sommes-nous les obligés ? \\[0pt]
De quoi a-t-on besoin ? \\[0pt]
De quoi a-t-on conscience lorsqu'on a conscience de soi ? \\[0pt]
De quoi avons-nous besoin ? \\[0pt]
De quoi avons-nous vraiment besoin ? \\[0pt]
De quoi dépend le bonheur ? \\[0pt]
De quoi dépend notre bonheur ? \\[0pt]
De quoi doute un sceptique ? \\[0pt]
De quoi est fait mon présent ? \\[0pt]
De quoi est fait notre présent ? \\[0pt]
De quoi est-on conscient ? \\[0pt]
De quoi est-on malheureux ? \\[0pt]
De quoi faisons-nous l'expérience ? \\[0pt]
De quoi fait-on l'expérience face à une œuvre ? \\[0pt]
De quoi faut-il attendre son bonheur ? \\[0pt]
De quoi faut-il avoir peur ? \\[0pt]
De quoi « humain » est-il le nom ? \\[0pt]
De quoi la forme est-elle la forme ? \\[0pt]
De quoi la logique est-elle la science ? \\[0pt]
De quoi la musique est-elle l'art ? \\[0pt]
De quoi la philosophie est-elle le désir ? \\[0pt]
De quoi la politique s'occupe-t-elle ? \\[0pt]
De quoi la religion sauve-t-elle ? \\[0pt]
De quoi l'art nous console-t-il ? \\[0pt]
De quoi l'art nous délivre-t-il ? \\[0pt]
De quoi l'art peut-il être contemporain ? \\[0pt]
De quoi l'art peut-il nous libérer ? \\[0pt]
De quoi la technique ne peut-elle pas nous libérer ? \\[0pt]
De quoi la vérité libère-t-elle ? \\[0pt]
De quoi le devoir libère-t-il ? \\[0pt]
De quoi le phénomène est-il l'apparaître ? \\[0pt]
De quoi le réel est-il constitué ? \\[0pt]
De quoi les logiciens parlent-ils ? \\[0pt]
De quoi les métaphysiciens parlent-ils ? \\[0pt]
De quoi les sciences humaines nous instruisent-elles ? \\[0pt]
De quoi l'État doit-il être propriétaire ? \\[0pt]
De quoi l'État ne doit-il pas se mêler ? \\[0pt]
De quoi le tissu social est-il fait ? \\[0pt]
De quoi le tyran est-il libre ? \\[0pt]
De quoi l'expérience esthétique est-elle expérience ? \\[0pt]
De quoi l'expérience esthétique est-elle l'expérience ? \\[0pt]
De quoi l'historien fait-il l'histoire ? \\[0pt]
De quoi l'inconscient peut-il être la cause ? \\[0pt]
De quoi l'outil nous libère-t-il ? \\[0pt]
De quoi n'avons-nous pas conscience ? \\[0pt]
De quoi ne peut-on pas répondre ? \\[0pt]
De quoi parlent les mathématiques ? \\[0pt]
De quoi parlent les théories physiques ? \\[0pt]
De quoi pâtit-on ? \\[0pt]
De quoi peut-il y avoir science ? \\[0pt]
De quoi peut-on être certain ? \\[0pt]
De quoi peut-on être inconscient ? \\[0pt]
De quoi peut-on faire l'expérience ? \\[0pt]
De quoi peut-on réclamer la propriété ? \\[0pt]
De quoi pouvons-nous être certains ? \\[0pt]
De quoi pouvons-nous être sûrs ? \\[0pt]
De quoi puis-je répondre ? \\[0pt]
De quoi rit-on ? \\[0pt]
De quoi sommes-nous coupables ? \\[0pt]
De quoi sommes-nous prisonniers ? \\[0pt]
De quoi sommes-nous responsables ? \\[0pt]
De quoi suis-je inconscient ? \\[0pt]
De quoi suis-je responsable ? \\[0pt]
De quoi une œuvre d'art nous instruit-elle ? \\[0pt]
De quoi y a-t-il expérience ? \\[0pt]
De quoi y a-t-il histoire ? \\[0pt]
Déraisonner, est-ce perdre de vue le réel ? \\[0pt]
Des comportements économiques peuvent-ils être rationnels ? \\[0pt]
Des événements aléatoires peuvent-ils obéir à des lois ? \\[0pt]
Des inégalités peuvent-elles être justes ? \\[0pt]
Désirer, est-ce être aliéné ? \\[0pt]
Désire-t-on la reconnaissance ? \\[0pt]
Désire-t-on un autre que soi ? \\[0pt]
Désirons-nous les choses parce qu'elles sont bonnes ? \\[0pt]
Des lois justes suffisent-elles à assurer la justice ? \\[0pt]
Des motivations peuvent-elles être sociales ? \\[0pt]
Des mythes peuvent-ils encore apparaître ? \\[0pt]
Des nations peuvent-elles former une société ? \\[0pt]
Des sciences molles ? \\[0pt]
Des sociétés sans État sont-elles des sociétés politiques ? \\[0pt]
Deux personnes peuvent-elles partager la même pensée ? \\[0pt]
Devant qui est-on responsable ? \\[0pt]
Devant qui sommes-nous responsables ? \\[0pt]
Devient-on raisonnable ? \\[0pt]
Devoir, est-ce avoir une dette envers quelqu'un ? \\[0pt]
Devoir, est-ce pouvoir ? \\[0pt]
Devoir, est-ce vouloir ? \\[0pt]
Devons-nous aimer notre destin ? \\[0pt]
Devons-nous apprendre à vivre ? \\[0pt]
Devons-nous dire la vérité ? \\[0pt]
Devons-nous douter de l'existence des choses ? \\[0pt]
Devons-nous espérer vivre sans travailler ? \\[0pt]
Devons-nous être heureux ? \\[0pt]
Devons-nous être obéissants ? \\[0pt]
Devons-nous nous faire confiance ? \\[0pt]
Devons-nous nous libérer de nos désirs ? \\[0pt]
Devons-nous quelque chose à la nature ? \\[0pt]
Devons-nous tenir certaines connaissances pour acquises ? \\[0pt]
Devons-nous toujours dire la vérité ? \\[0pt]
Devons-nous vivre comme si nous ne devions jamais mourir ? \\[0pt]
Dieu aurait-il pu mieux faire ? \\[0pt]
Dieu est-il l'être ? \\[0pt]
Dieu est-il mort ? \\[0pt]
Dieu est-il mortel ? \\[0pt]
Dieu est-il un concept ? \\[0pt]
Dieu est-il une hypothèse dont le savant et le philosophe peuvent se passer ? \\[0pt]
Dieu est-il une invention humaine ? \\[0pt]
Dieu est-il une limite de la pensée ? \\[0pt]
Dieu pense-t-il ? \\[0pt]
Dieu peut-il tout faire ? \\[0pt]
Dieu, prouvé ou éprouvé ? \\[0pt]
Dire, est-ce autre chose que vouloir dire ? \\[0pt]
Dire, est-ce faire ? \\[0pt]
Dire, est-ce trahir ? \\[0pt]
Dire la vérité, est-ce un devoir ? \\[0pt]
Dire que l'être humain est un animal, est-ce méconnaître sa dignité ? \\[0pt]
Discuter de la beauté d'une chose, est-ce discuter sur une réalité ? \\[0pt]
Dois-je admettre tout ce que je ne peux réfuter ? \\[0pt]
Dois-je mériter mon bonheur ? \\[0pt]
Dois-je obéir à la loi ? \\[0pt]
Dois-je tenir compte de ce que font les autres pour orienter ma conduite ? \\[0pt]
Doit-on apprendre à devenir soi-même ? \\[0pt]
Doit-on apprendre à percevoir ? \\[0pt]
Doit-on apprendre à vivre ? \\[0pt]
Doit-on attendre de la technique qu'elle mette fin au travail ? \\[0pt]
Doit-on bien juger pour bien faire ? \\[0pt]
Doit-on cesser de chercher à définir l'œuvre d'art ? \\[0pt]
Doit-on changer ses désirs, plutôt que l'ordre du monde ? \\[0pt]
Doit-on chasser les artistes de la cité ? \\[0pt]
Doit-on chercher à être heureux ? \\[0pt]
Doit-on corriger les inégalités sociales ? \\[0pt]
Doit-on croire au progrès ? \\[0pt]
Doit-on croire en l'humanité ? \\[0pt]
Doit-on cultiver l'ironie ? \\[0pt]
Doit-on déplorer le désaccord ? \\[0pt]
Doit-on distinguer devoir moral et obligation sociale ? \\[0pt]
Doit-on identifier l'âme à la conscience ? \\[0pt]
Doit-on interpréter les rêves ? \\[0pt]
Doit-on justifier les inégalités ? \\[0pt]
Doit-on le respect au vivant ? \\[0pt]
Doit-on mûrir pour la liberté ? \\[0pt]
Doit-on préférer l'injustice au désordre ? \\[0pt]
Doit-on rechercher le bonheur ? \\[0pt]
Doit-on rechercher l'harmonie ? \\[0pt]
Doit-on refuser d'interpréter ? \\[0pt]
Doit-on répondre de ce qu'on est devenu ? \\[0pt]
Doit-on respecter la nature ? \\[0pt]
Doit-on respecter les êtres vivants ? \\[0pt]
Doit-on se faire l'avocat du diable ? \\[0pt]
Doit-on se justifier d'exister ? \\[0pt]
Doit-on se mettre à la place d'autrui ? \\[0pt]
Doit-on se passer des utopies ? \\[0pt]
Doit-on souffrir de n'être pas compris ? \\[0pt]
Doit-on tenir le plaisir pour une fin ? \\[0pt]
Doit-on toujours dire la vérité ? \\[0pt]
Doit-on toujours rechercher la vérité ? \\[0pt]
Doit-on tout accepter de l'État ? \\[0pt]
Doit-on tout attendre de l'État ? \\[0pt]
Doit-on tout calculer ? \\[0pt]
Doit-on tout contrôler ? \\[0pt]
Doit-on tout pardonner ? \\[0pt]
Doit-on travailler ? \\[0pt]
Doit-on vraiment tout pardonner ? \\[0pt]
Donner, à quoi bon ? \\[0pt]
Donner l'exemple ? \\[0pt]
D'où la politique tire-t-elle sa légitimité ? \\[0pt]
D'où viennent les concepts ? \\[0pt]
D'où viennent les idées générales ? \\[0pt]
D'où viennent les préjugés ? \\[0pt]
D'où viennent nos idées ? \\[0pt]
D'où vient aux objets techniques leur beauté ? \\[0pt]
D'où vient la certitude ? \\[0pt]
D'où vient la certitude dans les sciences ? \\[0pt]
D'où vient la force d'une religion ? \\[0pt]
D'où vient l'amour de Dieu ? \\[0pt]
D'où vient la servitude ? \\[0pt]
D'où vient la signification des mots ? \\[0pt]
D'où vient le mal ? \\[0pt]
D'où vient le plaisir de lire ? \\[0pt]
D'où vient le sens du devoir ? \\[0pt]
D'où vient que l'histoire soit autre chose qu'un chaos ? \\[0pt]
Droit et devoir sont-ils liés ? \\[0pt]
Droits de l'homme ou droits du citoyen ? \\[0pt]
Droits et devoirs sont-ils réciproques ? \\[0pt]
Du passé pouvons-nous faire table rase ? \\[0pt]
Du penser à l'être, la conséquence est-elle bonne ? \\[0pt]
Échanger, est-ce créer de la valeur ? \\[0pt]
Échanger, est-ce partager ? \\[0pt]
Échanger, est-ce risquer ? \\[0pt]
Échappe-t-on à la nécessité ? \\[0pt]
Écrire l'histoire, est-ce donner un sens à la violence ? \\[0pt]
Écrire l'histoire, est-ce la faire ? \\[0pt]
Éduquer, est-ce dénaturer ? \\[0pt]
En histoire, tout est-il affaire d'interprétation ? \\[0pt]
En mathématiques, la notion d'existence a-t-elle un sens ? \\[0pt]
En morale, est-ce seulement l'intention qui compte ? \\[0pt]
En morale, peut-on dire : « C'est l'intention qui compte » ? \\[0pt]
En morale, y a-t-il quelque chose à savoir ? \\[0pt]
En politique, faut-il refuser l'utopie ? \\[0pt]
En politique, ne fait-on que défendre ses intérêts ? \\[0pt]
En politique, ne faut-il croire qu'aux rapports de force ? \\[0pt]
En politique n'y a-t-il que des rapports de force ? \\[0pt]
En politique, peut-on faire table rase du passé ? \\[0pt]
En politique, y a-t-il des modèles ? \\[0pt]
En quel sens la maladie peut-elle transformer notre vie ? \\[0pt]
En quel sens la métaphysique a-t-elle une histoire ? \\[0pt]
En quel sens la métaphysique est-elle une science ? \\[0pt]
En quel sens l'anthropologie peut-elle être historique ? \\[0pt]
En quel sens la perception engage-t-elle le corps ? \\[0pt]
En quel sens les sciences ont-elles une histoire ? \\[0pt]
En quel sens l'État est-il rationnel ? \\[0pt]
En quel sens le vécu est-il l'objet des sciences humaines ? \\[0pt]
En quel sens le vivant a-t-il une histoire ? \\[0pt]
En quel sens parler de lois de la pensée ? \\[0pt]
En quel sens parler de structure métaphysique ? \\[0pt]
En quel sens parler d'identité culturelle ? \\[0pt]
En quel sens parler d'illusion religieuse ? \\[0pt]
En quel sens peut-on appeler les sciences humaines « sciences de l'esprit » ? \\[0pt]
En quel sens peut-on dire de l'écologie qu'elle est politique ? \\[0pt]
En quel sens peut-on dire que la vérité s'impose ? \\[0pt]
En quel sens peut-on dire que le langage est la condition de la pensée ? \\[0pt]
En quel sens peut-on dire que le mal n'existe pas ? \\[0pt]
En quel sens peut-on dire que l'homme est un animal politique ? \\[0pt]
En quel sens peut-on dire que nos paroles nous trahissent ? \\[0pt]
En quel sens peut-on dire qu'« on expérimente avec sa raison » ? \\[0pt]
En quel sens peut-on parler de la mort de l'art ? \\[0pt]
En quel sens peut-on parler de la vie sociale comme d'un jeu ? \\[0pt]
En quel sens peut-on parler de responsabilité collective ? \\[0pt]
En quel sens peut-on parler de transcendance ? \\[0pt]
En quel sens peut-on parler d'expérience possible ? \\[0pt]
En quel sens peut-on parler d'une « culture technique » ? \\[0pt]
En quel sens peut-on parler d'une culture technique ? \\[0pt]
En quel sens peut-on parler d'une interprétation de la nature ? \\[0pt]
En quel sens une œuvre d'art est-elle un document ? \\[0pt]
En quels sens les corps peuvent-ils être objets de la métaphysique ? \\[0pt]
En quoi la connaissance de la matière peut-elle relever de la métaphysique ? \\[0pt]
En quoi la connaissance du vivant contribue-t-elle à la connaissance de l'homme ? \\[0pt]
En quoi la croyance religieuse se distingue-t-elle des autres croyances ? \\[0pt]
En quoi la justice met-elle fin à la violence ? \\[0pt]
En quoi la liberté n'est-elle pas une illusion ? \\[0pt]
En quoi la linguistique est-elle une science ? \\[0pt]
En quoi la matière s'oppose-t-elle à l'esprit ? \\[0pt]
En quoi la métaphysique est-elle une science ? \\[0pt]
En quoi la méthode est-elle un art de penser ? \\[0pt]
En quoi la nature constitue-t-elle un modèle ? \\[0pt]
En quoi la nature peut-elle être source de création ? \\[0pt]
En quoi l'animal intéresse-t-il les sciences humaines ? \\[0pt]
En quoi la patience est-elle une vertu ? \\[0pt]
En quoi la physique a-t-elle besoin des mathématiques ? \\[0pt]
En quoi la question de Dieu est-elle philosophique ? \\[0pt]
En quoi l'art peut-il intéresser le philosophe ? \\[0pt]
En quoi la sexualité peut-elle être objet de science ? \\[0pt]
En quoi la sociologie est-elle fondamentale ? \\[0pt]
En quoi la technique fait-elle question ? \\[0pt]
En quoi le bien d'autrui m'importe-t-il ? \\[0pt]
En quoi le bonheur est-il l'affaire de l'État ? \\[0pt]
En quoi le langage est-il constitutif de l'homme ? \\[0pt]
En quoi le langage est-il objet de science ? \\[0pt]
En quoi les hommes restent-ils des enfants ? \\[0pt]
En quoi les sciences humaines nous éclairent-elles sur la barbarie ? \\[0pt]
En quoi les sciences humaines sont-elles normatives ? \\[0pt]
En quoi les vivants témoignent-ils d'une histoire ? \\[0pt]
En quoi l'histoire est-elle une science du réel ? \\[0pt]
En quoi l'œuvre d'art donne-t-elle à penser ? \\[0pt]
En quoi une culture peut-elle être la mienne ? \\[0pt]
En quoi une discussion est-elle politique ? \\[0pt]
En quoi une insulte est-elle blessante ? \\[0pt]
En quoi une œuvre d'art est-elle moderne ? \\[0pt]
Enseigner, est-ce transmettre un savoir ? \\[0pt]
Entre croire et savoir, y a-t-il une différence de nature ? \\[0pt]
Entre l'art et la nature, qui imite l'autre ? \\[0pt]
Entre la sécurité de tous et la liberté de chacun, le politique doit-il choisir ? \\[0pt]
Entre le vrai et le faux y a-t-il une place pour le probable ? \\[0pt]
Entre l'opinion et la science, n'y a-t-il qu'une différence de degré ? \\[0pt]
Entre l'utile et l'honnête, peut-on choisir ? \\[0pt]
Envers qui avons-nous des devoirs ? \\[0pt]
Est-ce à la fin que le sens apparaît ? \\[0pt]
Est-ce à la raison de déterminer ce qui est réel ? \\[0pt]
Est-ce à l'État de faire le bonheur du peuple ? \\[0pt]
Est-ce de la force que l'État tient son autorité ? \\[0pt]
Est-ce la certitude qui fait la science ? \\[0pt]
Est-ce la démonstration qui fait la science ? \\[0pt]
Est-ce la majorité qui doit décider ? \\[0pt]
Est-ce la mémoire qui constitue mon identité ? \\[0pt]
Est-ce l'autorité qui fait la loi ? \\[0pt]
Est-ce le cerveau qui pense ? \\[0pt]
Est-ce l'échange utilitaire qui fait le lien social ? \\[0pt]
Est-ce le corps qui perçoit ? \\[0pt]
Est-ce l'ignorance qui nous fait croire ? \\[0pt]
Est-ce l'ignorance qui rend les hommes croyants ? \\[0pt]
Est-ce l'intérêt qui fonde le lien social ? \\[0pt]
Est-ce l'utilité qui définit un objet technique ? \\[0pt]
Est-ce par désir de la vérité que l'homme cherche à savoir ? \\[0pt]
Est-ce par son objet ou par ses méthodes qu'une science peut se définir ? \\[0pt]
Est-ce pour des raisons morales qu'il faut protéger l'environnement ? \\[0pt]
Est-ce seulement l'intention qui compte ? \\[0pt]
Est-ce un devoir d'aimer son prochain ? \\[0pt]
Est-ce un devoir de rechercher la vérité ? \\[0pt]
Est-ce un devoir d'être sincère ? \\[0pt]
Est-ce une chance de naître humain ? \\[0pt]
Est-il bien vrai qu'« on n'arrête pas le progrès » ? \\[0pt]
Est-il bon qu'un seul commande ? \\[0pt]
Est-il dans mon intérêt d'accomplir mes devoirs ? \\[0pt]
Est-il difficile de savoir ce que l'on veut ? \\[0pt]
Est-il difficile de savoir ce qu'on veut ? \\[0pt]
Est-il difficile d'être heureux ? \\[0pt]
Est-il difficile de vivre en société ? \\[0pt]
Est-il facile d'être libre ? \\[0pt]
Est-il immoral de se rendre heureux ? \\[0pt]
Est-il impossible de moraliser la politique ? \\[0pt]
Est-il judicieux de revenir sur ses décisions ? \\[0pt]
Est-il juste de payer l'impôt ? \\[0pt]
Est-il juste d'interpréter la loi ? \\[0pt]
Est-il justifié de parler de « corps social » ? \\[0pt]
Est-il légitime d'affirmer que seul le présent existe ? \\[0pt]
Est-il légitime d'opposer liberté et nécessité ? \\[0pt]
Est-il mauvais de suivre son désir ? \\[0pt]
Est-il méritoire de ne faire que son devoir ? \\[0pt]
Est-il naturel à l'homme de parler ? \\[0pt]
Est-il naturel de s'aimer soi-même ? \\[0pt]
Est-il nécessaire d'espérer pour entreprendre ? \\[0pt]
Est-il parfois bon de mentir ? \\[0pt]
Est-il pertinent d'opposer les actes et les paroles ? \\[0pt]
Est-il pertinent d'opposer l'individuel et le collectif ? \\[0pt]
Est-il possible d'améliorer l'homme ? \\[0pt]
Est-il possible de croire en la vie éternelle ? \\[0pt]
Est-il possible de douter de tout ? \\[0pt]
Est-il possible de ne croire à rien ? \\[0pt]
Est-il possible de ne pas savoir ce que l'on pense ? \\[0pt]
Est-il possible de préparer l'avenir ? \\[0pt]
Est-il possible de tout avoir pour être heureux ? \\[0pt]
Est-il possible d'être immoral sans le savoir ? \\[0pt]
Est-il possible d'être neutre politiquement ? \\[0pt]
Est-il possible d'ignorer toute vérité ? \\[0pt]
Est-il raisonnable d'aimer ? \\[0pt]
Est-il raisonnable d'avoir des certitudes ? \\[0pt]
Est-il raisonnable de chercher à interpréter les rêves ? \\[0pt]
Est-il raisonnable de lutter contre le temps ? \\[0pt]
Est-il raisonnable de se quereller pour des mots ? \\[0pt]
Est-il raisonnable d'être rationnel ? \\[0pt]
Est-il raisonnable de vouloir maîtriser la nature ? \\[0pt]
Est-il si difficile d'accéder à la vérité ? \\[0pt]
Est-il toujours avantageux de promouvoir son propre intérêt ? \\[0pt]
Est-il toujours illicite d'inférer ce qui doit être à partir de ce qui est ? \\[0pt]
Est-il toujours meilleur d'avoir le choix ? \\[0pt]
Est-il toujours moral de faire son devoir ? \\[0pt]
Est-il toujours possible de faire ce que l'on dit ? \\[0pt]
Est-il toujours possible de savoir ce que l'on doit faire ? \\[0pt]
Est-il utile d'avoir mal ? \\[0pt]
Est-il vrai que les animaux ne pensent pas ? \\[0pt]
Est-il vrai que l'ignorant n'est pas libre ? \\[0pt]
Est-il vrai que ma liberté s'arrête là où commence celle des autres ? \\[0pt]
Est-il vrai que nous ne nous tenons jamais au temps présent ? \\[0pt]
Est-il vrai qu'en science, « rien n'est donné, tout est construit » ? \\[0pt]
Est-il vrai que plus on échange, moins on se bat ? \\[0pt]
Est-il vrai qu'on apprenne de ses erreurs ? \\[0pt]
Est-on fondé à distinguer la justice et le droit ? \\[0pt]
Est-on fondé à parler d'une imperfection du langage ? \\[0pt]
Est-on l'auteur de sa propre vie ? \\[0pt]
Est-on le produit d'une culture ? \\[0pt]
Est-on libre de ne pas vouloir ce que l'on veut ? \\[0pt]
Est-on libre de travailler ? \\[0pt]
Est-on libre face à la vérité ? \\[0pt]
Est-on libre par hasard ? \\[0pt]
Est-on propriétaire de son corps ? \\[0pt]
Est-on responsable de ce qu'on n'a pas voulu ? \\[0pt]
Est-on responsable de ses croyances ? \\[0pt]
Est-on responsable de ses rêves ? \\[0pt]
Est-on responsable de son passé ? \\[0pt]
Est-on sociable par nature ? \\[0pt]
Est-on vraiment libre d'exprimer ce que l'on veut ? \\[0pt]
Établir la vérité, est-ce nécessairement démontrer ? \\[0pt]
Être à l'écoute de son désir, est-ce nier le désir de l'autre ? \\[0pt]
Être conscient de soi, est-ce être maître de soi ? \\[0pt]
Être conscient, est-ce être maître de soi ? \\[0pt]
Être cultivé, est-ce tout connaître ? \\[0pt]
Être cultivé rend-il meilleur ? \\[0pt]
Être éduqué, est-ce devenir soi-même ? \\[0pt]
Être, est-ce agir ? \\[0pt]
Être et penser, est-ce la même chose ? \\[0pt]
Être heureux, est-ce devoir ? \\[0pt]
Être libre, cela s'apprend-il ? \\[0pt]
Être libre, est-ce avoir le choix ? \\[0pt]
Être libre, est-ce dire non ? \\[0pt]
Être libre, est-ce échapper aux prévisions ? \\[0pt]
Être libre, est-ce faire ce que l'on veut ? \\[0pt]
Être libre, est-ce n'avoir aucun maître ? \\[0pt]
Être libre, est-ce n'obéir qu'à soi-même ? \\[0pt]
Être libre, est-ce pouvoir choisir ? \\[0pt]
Être libre, est-ce se suffire à soi-même ? \\[0pt]
Être libre, est-ce une question de volonté ? \\[0pt]
Être libre, est-ce vivre comme on l'entend ? \\[0pt]
Être ou ne pas être ? \\[0pt]
Être ou ne pas être, est-ce la question ? \\[0pt]
Être raisonnable, est-ce accepter la réalité telle qu'elle est ? \\[0pt]
Être raisonnable, est-ce renoncer à ses désirs ? \\[0pt]
Être religieux, est-ce nécessairement être dogmatique ? \\[0pt]
« Être » se dit-il en plusieurs sens ? \\[0pt]
« Être soi-même », cela a-t-il un sens ? \\[0pt]
Être spontané, est-ce être libre ? \\[0pt]
Être un sujet, est-ce être maître de soi ? \\[0pt]
Existe-il des mentalités primitives ? \\[0pt]
Exister, est-ce simplement vivre ? \\[0pt]
Existe-t-il au moins un devoir universel ? \\[0pt]
Existe-t-il dans le monde des réalités identiques ? \\[0pt]
Existe-t-il de faux besoins ? \\[0pt]
Existe-t-il des autorités en matière d'art ? \\[0pt]
Existe-t-il des choses en soi ? \\[0pt]
Existe-t-il des choses réellement sublimes ? \\[0pt]
Existe-t-il des choses sans prix ? \\[0pt]
Existe-t-il des comportements contraires à la nature ? \\[0pt]
Existe-t-il des connaissances a priori ? \\[0pt]
Existe-t-il des croyances collectives ? \\[0pt]
Existe-t-il des degrés de vérité ? \\[0pt]
Existe-t-il des démonstrations métaphysiques ? \\[0pt]
Existe-t-il des désirs coupables ? \\[0pt]
Existe-t-il des devoirs envers soi-même ? \\[0pt]
Existe-t-il des dilemmes moraux ? \\[0pt]
Existe-t-il des émotions proprement esthétiques ? \\[0pt]
Existe-t-il des états mentaux collectifs ? \\[0pt]
Existe-t-il des expériences métaphysiques ? \\[0pt]
Existe-t-il des identités collectives ? \\[0pt]
Existe-t-il des intuitions métaphysiques ? \\[0pt]
Existe-t-il des langages naturels ? \\[0pt]
Existe-t-il des mondes sociaux ? \\[0pt]
Existe-t-il des paroles vraies ? \\[0pt]
Existe-t-il des plaisirs purs ? \\[0pt]
Existe-t-il des preuves irréfutables ? \\[0pt]
Existe-t-il des principes premiers ? \\[0pt]
Existe-t-il des problèmes insolubles ? \\[0pt]
Existe-t-il des questions sans réponse ? \\[0pt]
Existe-t-il des sciences de différentes natures ? \\[0pt]
Existe-t-il des signes naturels ? \\[0pt]
Existe-t-il des sujets collectifs ? \\[0pt]
Existe-t-il des violences légitimes ? \\[0pt]
Existe-t-il différentes sortes de sciences ? \\[0pt]
Existe-t-il plusieurs déterminismes ? \\[0pt]
Existe-t-il plusieurs mondes ? \\[0pt]
Existe-t-il un art de la parole ? \\[0pt]
Existe-t-il un art de penser ? \\[0pt]
Existe-t-il un bien commun qui soit la norme de la vie politique ? \\[0pt]
Existe-t-il un bien souverain ? \\[0pt]
Existe-t-il un déterminisme social ? \\[0pt]
Existe-t-il un droit de mentir ? \\[0pt]
Existe-t-il une autorité naturelle ? \\[0pt]
Existe-t-il une hiérarchie des êtres ? \\[0pt]
Existe-t-il une histoire du présent ? \\[0pt]
Existe-t-il une méthode pour rechercher la vérité ? \\[0pt]
Existe-t-il une méthode pour trouver la vérité ? \\[0pt]
Existe-t-il une opinion publique ? \\[0pt]
Existe-t-il une réalité subjective ? \\[0pt]
Existe-t-il une réalité symbolique ? \\[0pt]
Existe-t-il une science de la morale ? \\[0pt]
Existe-t-il une science de l'être ? \\[0pt]
Existe-t-il une sensibilité philosophique ? \\[0pt]
Existe-t-il une unité des arts ? \\[0pt]
Existe-t-il une violence symbolique ? \\[0pt]
Existe-t-il une vision scientifique du monde ? \\[0pt]
Existe-t-il un vocabulaire neutre des droits fondamentaux ? \\[0pt]
Expliquer, est-ce excuser ? \\[0pt]
Expliquer, est-ce interpréter ? \\[0pt]
Expliquer, est-ce justifier ? \\[0pt]
Exploiter la nature, est-ce lui vouloir du mal ? \\[0pt]
Faire de la métaphysique, est-ce se détourner du monde ? \\[0pt]
Faire est-il nécessairement savoir faire ? \\[0pt]
Faire son devoir, est-ce faire son propre malheur ? \\[0pt]
Faire son devoir, est-ce là toute la morale ? \\[0pt]
Faire son devoir, est-ce obéir ? \\[0pt]
Faire son devoir, est-ce payer une dette ? \\[0pt]
Faire son devoir, est-ce perdre toute sensibilité ? \\[0pt]
Faire son devoir, est-ce se soumettre ? \\[0pt]
Faisons-nous l'histoire ? \\[0pt]
Faisons-nous l'Histoire ? \\[0pt]
Faisons-nous notre propre malheur ? \\[0pt]
Fait-on de la politique pour changer les choses ? \\[0pt]
Faudrait-il bannir la polysémie du langage ? \\[0pt]
Faudrait-il ne rien oublier ? \\[0pt]
Faudrait-il vivre sans passion ? \\[0pt]
Faut-avoir peur de la technique ? \\[0pt]
Faut-il accepter sa condition ? \\[0pt]
Faut-il accorder de l'importance aux mots ? \\[0pt]
Faut-il accorder l'esprit aux bêtes ? \\[0pt]
Faut-il affirmer son identité ? \\[0pt]
Faut-il aimer autrui pour le respecter ? \\[0pt]
Faut-il aimer la nature plus que soi-même ? \\[0pt]
Faut-il aimer la vie ? \\[0pt]
Faut-il aimer le destin ? \\[0pt]
Faut-il aimer son prochain ? \\[0pt]
Faut-il aimer son prochain comme soi-même ? \\[0pt]
Faut-il aller au-delà des apparences ? \\[0pt]
Faut-il aller toujours plus vite ? \\[0pt]
Faut-il apprendre à être libre ? \\[0pt]
Faut-il apprendre à obéir ? \\[0pt]
Faut-il apprendre à vivre en renonçant au bonheur ? \\[0pt]
Faut-il apprendre à voir ? \\[0pt]
Faut-il assimiler la pensée au langage ? \\[0pt]
Faut-il avoir des ennemis ? \\[0pt]
Faut-il avoir des principes ? \\[0pt]
Faut-il avoir foi en la raison ? \\[0pt]
Faut-il avoir foi en la technique ? \\[0pt]
Faut-il avoir peur de la liberté ? \\[0pt]
Faut-il avoir peur de la nature ? \\[0pt]
Faut-il avoir peur de la technique ? \\[0pt]
Faut-il avoir peur de l'avenir ? \\[0pt]
Faut-il avoir peur de la vérité ? \\[0pt]
Faut-il avoir peur des contradictions ? \\[0pt]
Faut-il avoir peur des habitudes ? \\[0pt]
Faut-il avoir peur des machines ? \\[0pt]
Faut-il avoir peur d'être libre ? \\[0pt]
Faut-il avoir peur du désordre ? \\[0pt]
Faut-il changer le monde ? \\[0pt]
Faut-il changer ses désirs plutôt que l'ordre du monde ? \\[0pt]
Faut-il chasser le naturel ? \\[0pt]
Faut-il chasser les poètes ? \\[0pt]
Faut-il chercher à être heureux ? \\[0pt]
Faut-il chercher à satisfaire tous nos désirs ? \\[0pt]
Faut-il chercher à se connaître ? \\[0pt]
Faut-il chercher à tout démontrer ? \\[0pt]
Faut-il chercher la paix à tout prix ? \\[0pt]
Faut-il chercher le bonheur à tout prix ? \\[0pt]
Faut-il chercher un sens à l'histoire ? \\[0pt]
Faut-il choisir entre être heureux et être libre ? \\[0pt]
Faut-il combattre ses illusions ? \\[0pt]
Faut-il comprendre pour croire ? \\[0pt]
Faut-il concilier les contraires ? \\[0pt]
Faut-il condamner la fiction ? \\[0pt]
Faut-il condamner la rhétorique ? \\[0pt]
Faut-il condamner le luxe ? \\[0pt]
Faut-il condamner les illusions ? \\[0pt]
Faut-il connaître la vérité pour pouvoir mentir ? \\[0pt]
Faut-il connaître l'histoire des sciences ? \\[0pt]
Faut-il connaître l'Histoire pour gouverner ? \\[0pt]
Faut-il considérer le droit pénal comme instituant une violence légitime ? \\[0pt]
Faut-il considérer les faits sociaux comme des choses ? \\[0pt]
Faut-il contrôler les mœurs ? \\[0pt]
Faut-il craindre de perdre son temps ? \\[0pt]
Faut-il craindre la démocratie ? \\[0pt]
Faut-il craindre la mort ? \\[0pt]
Faut-il craindre la révolution ? \\[0pt]
Faut-il craindre la tyrannie du bonheur ? \\[0pt]
Faut-il craindre le développement des techniques ? \\[0pt]
Faut-il craindre le pire ? \\[0pt]
Faut-il craindre le pouvoir des machines ? \\[0pt]
Faut-il craindre le pouvoir souverain ? \\[0pt]
Faut-il craindre le regard d'autrui ? \\[0pt]
Faut-il craindre les foules ? \\[0pt]
Faut-il craindre les machines ? \\[0pt]
Faut-il craindre les masses ? \\[0pt]
Faut-il craindre l'État ? \\[0pt]
Faut-il craindre l'ordre ? \\[0pt]
Faut-il croire au progrès ? \\[0pt]
Faut-il croire en la science ? \\[0pt]
Faut-il croire en quelque chose ? \\[0pt]
Faut-il croire les historiens ? \\[0pt]
Faut-il croire que l'histoire a un sens ? \\[0pt]
Faut-il cultiver le naturel ? \\[0pt]
Faut-il défendre la démocratie ? \\[0pt]
Faut-il défendre les faibles ? \\[0pt]
Faut-il défendre l'ordre à tout prix ? \\[0pt]
Faut-il dépasser les apparences ? \\[0pt]
Faut-il désespérer de l'humanité ? \\[0pt]
Faut-il des frontières ? \\[0pt]
Faut-il des héros ? \\[0pt]
Faut-il désirer la vérité ? \\[0pt]
Faut-il des outils pour penser ? \\[0pt]
Faut-il des techniques du raisonnement ? \\[0pt]
Faut-il détruire l'État ? \\[0pt]
Faut-il détruire pour créer ? \\[0pt]
Faut-il dire de la justice qu'elle n'existe pas ? \\[0pt]
Faut-il dire tout haut ce que les autres pensent tout bas ? \\[0pt]
Faut-il diriger l'économie ? \\[0pt]
Faut-il distinguer ce qui est de ce qui doit être ? \\[0pt]
Faut-il distinguer culture et civilisation ? \\[0pt]
Faut-il distinguer désir et besoin ? \\[0pt]
Faut-il distinguer devoir moral et obligation sociale ? \\[0pt]
Faut-il distinguer esthétique et philosophie de l'art ? \\[0pt]
Faut-il donner un sens à la souffrance ? \\[0pt]
Faut-il douter de ce qu'on ne peut pas démontrer ? \\[0pt]
Faut-il douter de tout ? \\[0pt]
Faut-il du passé faire table rase ? \\[0pt]
Faut-il écouter sa conscience ? \\[0pt]
Faut-il enfermer ? \\[0pt]
Faut-il enfermer les œuvres dans les musées ? \\[0pt]
Faut-il en finir avec l'esthétique ? \\[0pt]
Faut-il espérer pour agir ? \\[0pt]
Faut-il être à l'écoute du corps ? \\[0pt]
Faut-il être bon ? \\[0pt]
Faut-il être cohérent ? \\[0pt]
Faut-il être connaisseur pour apprécier une œuvre d'art ? \\[0pt]
Faut-il être cosmopolite ? \\[0pt]
Faut-il être courageux pour être libre ? \\[0pt]
Faut-il être discipliné ? \\[0pt]
Faut-il être fidèle à soi-même ? \\[0pt]
Faut-il être heureux ? \\[0pt]
Faut-il être idéaliste ? \\[0pt]
Faut-il être libre pour être heureux ? \\[0pt]
Faut-il être logique avec soi-même ? \\[0pt]
Faut-il être mesuré ? \\[0pt]
Faut-il être mesuré en toutes choses ? \\[0pt]
Faut-il être modéré ? \\[0pt]
Faut-il être naturel ? \\[0pt]
Faut-il être objectif ? \\[0pt]
Faut-il être original ? \\[0pt]
Faut-il être positif ? \\[0pt]
Faut-il être pragmatique ? \\[0pt]
Faut-il être réaliste ? \\[0pt]
Faut-il être réaliste en politique ? \\[0pt]
Faut-il être relativiste ? \\[0pt]
Faut-il expliquer la morale par son utilité ? \\[0pt]
Faut-il faire confiance au progrès technique ? \\[0pt]
Faut-il faire de la philosophie ? \\[0pt]
Faut-il faire de nécessité vertu ? \\[0pt]
Faut-il faire de sa vie une œuvre d'art ? \\[0pt]
Faut-il faire des efforts pour être soi-même ? \\[0pt]
Faut-il faire table rase du passé ? \\[0pt]
Faut-il faire une place à la croyance ? \\[0pt]
Faut-il forcer les gens à participer à la vie politique ? \\[0pt]
Faut-il fuir la politique ? \\[0pt]
Faut-il garder ses illusions ? \\[0pt]
Faut-il hiérarchiser les désirs ? \\[0pt]
Faut-il hiérarchiser les formes de vie ? \\[0pt]
Faut-il hiérarchiser ses désirs ? \\[0pt]
Faut-il imaginer que nous sommes heureux ? \\[0pt]
Faut-il imposer la vérité ? \\[0pt]
Faut-il interpréter la loi ? \\[0pt]
Faut-il joindre l'utile à l'agréable ? \\[0pt]
Faut-il laisser faire la nature ? \\[0pt]
Faut-il laisser parler la nature ? \\[0pt]
Faut-il libérer la parole ? \\[0pt]
Faut-il libérer l'humanité du travail ? \\[0pt]
Faut-il limiter la souveraineté ? \\[0pt]
Faut-il limiter la souveraineté de l'État ? \\[0pt]
Faut-il limiter le pouvoir de l'État ? \\[0pt]
Faut-il limiter les prétentions de la science ? \\[0pt]
Faut-il limiter l'exercice de la puissance publique ? \\[0pt]
Faut-il lire des romans ? \\[0pt]
Faut-il maîtriser ses émotions ? \\[0pt]
Faut-il mathématiser le réel pour le connaître ? \\[0pt]
Faut-il ménager les apparences ? \\[0pt]
Faut-il mépriser le luxe ? \\[0pt]
Faut-il mériter son bonheur ? \\[0pt]
Faut-il mieux vivre comme si nous ne devions jamais mourir ? \\[0pt]
Faut-il ne manquer de rien pour être heureux ? \\[0pt]
Faut-il n'être jamais méchant ? \\[0pt]
Faut-il obéir à la voix de sa conscience ? \\[0pt]
Faut-il opposer à la politique la souveraineté du droit ? \\[0pt]
Faut-il opposer culture et nature ? \\[0pt]
Faut-il opposer droits et devoirs ? \\[0pt]
Faut-il opposer éduquer et instruire ? \\[0pt]
Faut-il opposer histoire et mémoire ? \\[0pt]
Faut-il opposer la foi et la raison ? \\[0pt]
Faut-il opposer la matière et l'esprit ? \\[0pt]
Faut-il opposer l'art à la connaissance ? \\[0pt]
Faut-il opposer la société et l'État ? \\[0pt]
Faut-il opposer la théorie et la pratique ? \\[0pt]
Faut-il opposer la vie à la mort ? \\[0pt]
Faut-il opposer le devoir-être à l'être ? \\[0pt]
Faut-il opposer le don et l'échange ? \\[0pt]
Faut-il opposer le réel et l'imaginaire ? \\[0pt]
Faut-il opposer l'esprit et la matière ? \\[0pt]
Faut-il opposer l'État et la société ? \\[0pt]
Faut-il opposer le temps vécu et le temps des choses ? \\[0pt]
Faut-il opposer l'histoire et la fiction ? \\[0pt]
Faut-il opposer nature et culture ? \\[0pt]
Faut-il opposer produire et créer ? \\[0pt]
Faut-il opposer raison et sensation ? \\[0pt]
Faut-il opposer rhétorique et philosophie ? \\[0pt]
Faut-il opposer science et croyance ? \\[0pt]
Faut-il opposer science et métaphysique ? \\[0pt]
Faut-il opposer sciences humaines et sciences de la nature ? \\[0pt]
Faut-il opposer sens et vérité ? \\[0pt]
Faut-il opposer subjectivité et objectivité ? \\[0pt]
Faut-il opposer technique et nature ? \\[0pt]
Faut-il opposer théorie et expérience ? \\[0pt]
Faut-il oublier le passé ? \\[0pt]
Faut-il oublier le passé pour se donner un avenir ? \\[0pt]
Faut-il pardonner ? \\[0pt]
Faut-il parfois désobéir aux lois ? \\[0pt]
Faut-il parfois sacrifier la vérité ? \\[0pt]
Faut-il parler pour avoir des idées générales ? \\[0pt]
Faut-il partager la souveraineté ? \\[0pt]
Faut-il penser l'État comme un corps ? \\[0pt]
Faut-il penser l'État comme un individu ? \\[0pt]
Faut-il perdre ses illusions ? \\[0pt]
Faut-il perdre son temps ? \\[0pt]
Faut-il poser des limites à l'activité rationnelle ? \\[0pt]
Faut-il pour le connaître faire du vivant un objet ? \\[0pt]
Faut-il préférer la liberté à l'égalité ? \\[0pt]
Faut-il préférer l'art à la nature ? \\[0pt]
Faut-il préférer le bonheur à la vérité ? \\[0pt]
Faut-il préférer l'injustice au désordre ? \\[0pt]
Faut-il préférer une injustice au désordre ? \\[0pt]
Faut-il prendre soin de soi ? \\[0pt]
Faut-il privilégier le naturel ? \\[0pt]
Faut-il protéger la dignité humaine ? \\[0pt]
Faut-il protéger la nature ? \\[0pt]
Faut-il protéger les faibles contre les forts ? \\[0pt]
Faut-il que le réel ait un sens ? \\[0pt]
Faut-il que les meilleurs gouvernent ? \\[0pt]
Faut-il rechercher la certitude ? \\[0pt]
Faut-il rechercher la perfection dans le langage ? \\[0pt]
Faut-il rechercher la simplicité ? \\[0pt]
Faut-il rechercher le bonheur ? \\[0pt]
Faut-il rechercher l'harmonie ? \\[0pt]
Faut-il reconnaître pour connaître ? \\[0pt]
Faut-il regretter l'équivocité du langage ? \\[0pt]
Faut-il réguler la technique ? \\[0pt]
Faut-il rejeter tous les préjugés ? \\[0pt]
Faut-il rejeter toute norme ? \\[0pt]
Faut-il renoncer à connaître la nature des choses ? \\[0pt]
Faut-il renoncer à croire ? \\[0pt]
Faut-il renoncer à faire du travail une valeur ? \\[0pt]
Faut-il renoncer à la certitude ? \\[0pt]
Faut-il renoncer à la distinction entre nature et culture ? \\[0pt]
Faut-il renoncer à la vérité ? \\[0pt]
Faut-il renoncer à l'idée d'âme ? \\[0pt]
Faut-il renoncer à l'idée de fondement ? \\[0pt]
Faut-il renoncer à l'impossible ? \\[0pt]
Faut-il renoncer à rechercher la vérité ? \\[0pt]
Faut-il renoncer à son désir ? \\[0pt]
Faut-il résister à la peur de mourir ? \\[0pt]
Faut-il respecter la nature ? \\[0pt]
Faut-il respecter la tradition ? \\[0pt]
Faut-il respecter la vie ? \\[0pt]
Faut-il respecter les convenances ? \\[0pt]
Faut-il respecter le vivant ? \\[0pt]
Faut-il restaurer les œuvres d'art ? \\[0pt]
Faut-il rester impartial ? \\[0pt]
Faut-il rester naturel ? \\[0pt]
Faut-il rester soi-même ? \\[0pt]
Faut-il rire ou pleurer ? \\[0pt]
Faut-il rompre avec le passé ? \\[0pt]
Faut-il s'adapter ? \\[0pt]
Faut-il s'adapter aux événements ? \\[0pt]
Faut-il s'affranchir des désirs ? \\[0pt]
Faut-il s'aimer pour vivre ensemble ? \\[0pt]
Faut-il s'aimer soi-même ? \\[0pt]
Faut-il s'attendre à l'inattendu ? \\[0pt]
Faut-il sauver des vies à tout prix ? \\[0pt]
Faut-il sauver les apparences ? \\[0pt]
Faut-il savoir ce que l'on veut ? \\[0pt]
Faut-il savoir désobéir ? \\[0pt]
Faut-il savoir mentir ? \\[0pt]
Faut-il savoir obéir pour gouverner ? \\[0pt]
Faut-il savoir pour agir ? \\[0pt]
Faut-il savoir prendre des risques ? \\[0pt]
Faut-il se chercher pour se trouver ? \\[0pt]
Faut-il se connaître pour être maître de soi ? \\[0pt]
Faut-il se contenter de peu ? \\[0pt]
Faut-il se cultiver ? \\[0pt]
Faut-il se délivrer de la peur ? \\[0pt]
Faut-il se délivrer des passions ? \\[0pt]
Faut-il se détacher du monde ? \\[0pt]
Faut-il s'efforcer d'être moins personnel ? \\[0pt]
Faut-il se fier à ce que l'on ressent ? \\[0pt]
Faut-il se fier à la majorité ? \\[0pt]
Faut-il se fier à sa propre raison ? \\[0pt]
Faut-il se fier au témoignage des sens ? \\[0pt]
Faut-il se fier aux apparences ? \\[0pt]
Faut-il se libérer de soi-même ? \\[0pt]
Faut-il se libérer du travail ? \\[0pt]
Faut-il se libérer pour être libre ? \\[0pt]
Faut-il se méfier de la technique ? \\[0pt]
Faut-il se méfier de l'écriture ? \\[0pt]
Faut-il se méfier de l'imagination ? \\[0pt]
Faut-il se méfier de l'inspiration ? \\[0pt]
Faut-il se méfier de l'intuition ? \\[0pt]
Faut-il se méfier des apparences ? \\[0pt]
Faut-il se méfier de ses désirs ? \\[0pt]
Faut-il se méfier des images ? \\[0pt]
Faut-il se méfier de soi-même ? \\[0pt]
Faut-il se méfier du progrès technique ? \\[0pt]
Faut-il se méfier du volontarisme politique ? \\[0pt]
Faut-il s'en remettre à l'État pour limiter le pouvoir de l'État ? \\[0pt]
Faut-il s'en tenir aux faits ? \\[0pt]
Faut-il séparer la science et la technique ? \\[0pt]
Faut-il séparer morale et politique ? \\[0pt]
Faut-il se poser des questions métaphysiques ? \\[0pt]
Faut-il se réjouir d'exister ? \\[0pt]
Faut-il se rendre à l'évidence ? \\[0pt]
Faut-il se résigner aux inégalités ? \\[0pt]
Faut-il se ressembler pour former une société ? \\[0pt]
Faut-il s'intéresser aux œuvres mineures ? \\[0pt]
Faut-il souhaiter la fin du travail ? \\[0pt]
Faut-il souhaiter une langue universelle ? \\[0pt]
Faut-il suivre la nature ? \\[0pt]
Faut-il suivre ses intuitions ? \\[0pt]
Faut-il supposer un énoncé vrai pour en comprendre le sens ? \\[0pt]
Faut-il surmonter son enfance ? \\[0pt]
Faut-il tolérer les intolérants ? \\[0pt]
Faut-il toujours avoir raison ? \\[0pt]
Faut-il toujours chercher à savoir ? \\[0pt]
Faut-il toujours dire la vérité ? \\[0pt]
Faut-il toujours être en accord avec soi-même ? \\[0pt]
Faut-il toujours éviter de se contredire ? \\[0pt]
Faut-il toujours faire son devoir ? \\[0pt]
Faut-il toujours garder espoir ? \\[0pt]
Faut-il toujours respecter ses engagements ? \\[0pt]
Faut-il tout critiquer ? \\[0pt]
Faut-il tout démontrer ? \\[0pt]
Faut-il tout interpréter ? \\[0pt]
Faut-il un commencement à tout ? \\[0pt]
Faut-il un corps pour penser ? \\[0pt]
Faut-il une guerre pour mettre fin à toutes les guerres ? \\[0pt]
Faut-il une méthode pour découvrir la vérité ? \\[0pt]
Faut-il une théorie de la connaissance ? \\[0pt]
Faut-il vaincre ses désirs plutôt que l'ordre du monde ? \\[0pt]
Faut-il vivre avec son temps ? \\[0pt]
Faut-il vivre comme si l'on ne devait jamais mourir ? \\[0pt]
Faut-il vivre comme si nous étions immortels ? \\[0pt]
Faut-il vivre comme si nous ne devions jamais mourir ? \\[0pt]
Faut-il vivre comme si on ne devait jamais mourir ? \\[0pt]
Faut-il vivre conformément à la nature ? \\[0pt]
Faut-il vivre dangereusement ? \\[0pt]
Faut-il vivre hors de la société pour être heureux ? \\[0pt]
Faut-il voir pour croire ? \\[0pt]
Faut-il vouloir changer le monde ? \\[0pt]
Faut-il vouloir être heureux ? \\[0pt]
Faut-il vouloir la paix ? \\[0pt]
Faut-il vouloir la paix de l'âme ? \\[0pt]
Faut-il vouloir la transparence ? \\[0pt]
Faut-il vouloir savoir ? \\[0pt]
Forme-t-on son esprit en transformant la matière ? \\[0pt]
Gouverner, est-ce dominer ? \\[0pt]
Gouverner, est-ce prévoir ? \\[0pt]
Gouverner, est-ce régner ? \\[0pt]
Hier a-t-il plus de réalité que demain ? \\[0pt]
Histoire et géographie font-elles couple ? \\[0pt]
Imaginer, est-ce créer ? \\[0pt]
Imiter, est-ce copier ? \\[0pt]
Imiter, est-ce manquer d'imagination ? \\[0pt]
Interpréter, est-ce connaître ? \\[0pt]
Interpréter, est-ce renoncer à prouver ? \\[0pt]
Interpréter, est-ce renoncer à toute objectivité ? \\[0pt]
Interpréter, est-ce savoir ? \\[0pt]
Interpréter, est-ce un art ? \\[0pt]
Interpréter est-il subjectif ? \\[0pt]
Interprète-t-on à défaut de connaître ? \\[0pt]
« Je ne voulais pas cela » : en quoi les sciences humaines permettent-elles de comprendre cette excuse ? \\[0pt]
Joue-t-on toujours un rôle ? \\[0pt]
Juge-t-on un acte ou son auteur ? \\[0pt]
Jusqu'à quel point la nature est-elle objet de science ? \\[0pt]
Jusqu'à quel point pouvons-nous juger autrui ? \\[0pt]
Jusqu'à quel point sommes-nous responsables de nos passions ? \\[0pt]
Jusqu'à quel point suis-je mon propre maître ? \\[0pt]
Jusqu'où interpréter ? \\[0pt]
Jusqu'où peut-on dialoguer ? \\[0pt]
Jusqu'où peut-on soigner ? \\[0pt]
Jusqu'où s'étend le domaine de la science ? \\[0pt]
La barbarie peut-elle avoir un visage humain ? \\[0pt]
La beauté a-t-elle une histoire ? \\[0pt]
La beauté comporte-t-elle des degrés ? \\[0pt]
La beauté demande-t-elle des preuves ? \\[0pt]
La beauté est-elle affaire de goût ? \\[0pt]
La beauté est-elle dans le regard ou dans la chose vue ? \\[0pt]
La beauté est-elle dans les choses ? \\[0pt]
La beauté est-elle intemporelle ? \\[0pt]
La beauté est-elle l'objet d'une connaissance ? \\[0pt]
La beauté est-elle partout ? \\[0pt]
La beauté est-elle sensible ? \\[0pt]
La beauté est-elle une promesse de bonheur ? \\[0pt]
La beauté naturelle est-elle une catégorie esthétique périmée ? \\[0pt]
La beauté n'est-elle qu'un effet artistique ? \\[0pt]
La beauté n'est-elle qu'une idée ? \\[0pt]
La beauté nous rend-elle meilleurs ? \\[0pt]
La beauté peut-elle délivrer une vérité ? \\[0pt]
La beauté peut-elle être cachée ? \\[0pt]
La beauté peut-elle laisser indifférent ? \\[0pt]
La beauté s'explique-t-elle ? \\[0pt]
La bêtise et la méchanceté sont-elles liées intrinsèquement ? \\[0pt]
La bêtise et la méchanceté sont-elles liées nécessairement ? \\[0pt]
La bêtise n'est-elle pas proprement humaine ? \\[0pt]
La biologie peut-elle se passer de causes finales ? \\[0pt]
L'absolu est-il connaissable ? \\[0pt]
L'Absolu est-il un objet de savoir ? \\[0pt]
L'abstraction est-elle toujours utile à la science empirique ? \\[0pt]
L'abstrait est-il en dehors de l'espace et du temps ? \\[0pt]
La causalité suppose-t-elle des lois ? \\[0pt]
L'accord des esprits est-il un critère de vérité ? \\[0pt]
L'accord est-il le terme ou le principe du dialogue ? \\[0pt]
La certitude est-elle une marque de vérité ? \\[0pt]
La certitude ne peut-elle naître que de l'évidence ? \\[0pt]
La charité est-elle la négation de la justice ? \\[0pt]
La charité est-elle une vertu ? \\[0pt]
La clarté suffit-elle au savoir ? \\[0pt]
La cohérence est-elle la norme du vrai ? \\[0pt]
La cohérence est-elle un critère de la vérité ? \\[0pt]
La cohérence est-elle un critère de vérité ? \\[0pt]
La cohérence est-elle une vertu ? \\[0pt]
La cohérence logique est-elle une condition suffisante de la démonstration ? \\[0pt]
La cohérence suffit-elle à la vérité ? \\[0pt]
La colère peut-elle être justifiée ? \\[0pt]
La communication est-elle nécessaire à la démocratie ? \\[0pt]
La compassion risque-t-elle d'abolir l'exigence politique ? \\[0pt]
La compétence fonde-t-elle la compétence juridique ? \\[0pt]
La compétence fonde-t-elle l'autorité politique ? \\[0pt]
La compétence technique peut-elle fonder l'autorité publique ? \\[0pt]
La confiance est-elle une vertu ? \\[0pt]
La conflictualité politique est-elle indépassable ? \\[0pt]
La connaissance a-t-elle des limites ? \\[0pt]
La connaissance commune est-elle le point de départ de la science ? \\[0pt]
La connaissance commune fait-elle obstacle à la vérité ? \\[0pt]
La connaissance de la nécessité a priori peut-elle évoluer ? \\[0pt]
La connaissance de la vie se confond-elle avec celle du vivant ? \\[0pt]
La connaissance de l'histoire est-elle utile à l'action ? \\[0pt]
La connaissance de l'histoire nous rend-elle plus libres ? \\[0pt]
La connaissance de soi est-elle évidente ? \\[0pt]
La connaissance du passé est-elle nécessaire à la compréhension du présent ? \\[0pt]
La connaissance du vivant est-elle désintéressée ? \\[0pt]
La connaissance du vivant peut-elle être désintéressée ? \\[0pt]
La connaissance est-elle une contemplation ? \\[0pt]
La connaissance est-elle une croyance justifiée ? \\[0pt]
La connaissance historique est-elle une interprétation des faits ? \\[0pt]
La connaissance historique est-elle utile à l'homme ? \\[0pt]
La connaissance objective doit-elle s'interdire toute interprétation ? \\[0pt]
La connaissance objective exclut-elle toute forme de subjectivité ? \\[0pt]
La connaissance peut-elle être pratique ? \\[0pt]
La connaissance peut-elle se passer de l'imagination ? \\[0pt]
La connaissance scientifique abolit-elle toute croyance ? \\[0pt]
La connaissance scientifique est-elle désintéressée ? \\[0pt]
La connaissance scientifique est-elle fondée sur l'expérience ? \\[0pt]
La connaissance scientifique n'est-elle qu'une croyance argumentée ? \\[0pt]
La connaissance s'interdit-elle tout recours à l'imagination ? \\[0pt]
La connaissance suppose-t-elle une éthique ? \\[0pt]
La conscience a-t-elle des degrés ? \\[0pt]
La conscience a-t-elle des moments ? \\[0pt]
La conscience d'agir suffit-elle à garantir notre liberté ? \\[0pt]
La conscience d'autrui est-elle impénétrable ? \\[0pt]
La conscience définit-elle l'homme en propre ? \\[0pt]
La conscience de la mort est-elle une condition de la sagesse ? \\[0pt]
La conscience de soi est-elle une donnée immédiate ? \\[0pt]
La conscience de soi est-elle une méconnaissance de soi ? \\[0pt]
La conscience de soi suppose-t-elle autrui ? \\[0pt]
La conscience du temps rend-elle l'existence tragique ? \\[0pt]
La conscience entrave-t-elle l'action ? \\[0pt]
La conscience est-elle ce qui fait le sujet ? \\[0pt]
La conscience est-elle intrinsèquement morale ? \\[0pt]
La conscience est-elle le produit de la vie réelle ? \\[0pt]
La conscience est-elle nécessairement malheureuse ? \\[0pt]
La conscience est-elle ou n'est-elle pas ? \\[0pt]
La conscience est-elle source d'illusions ? \\[0pt]
La conscience est-elle temporelle ? \\[0pt]
La conscience est-elle toujours morale ? \\[0pt]
La conscience est-elle une activité ? \\[0pt]
La conscience est-elle une connaissance ? \\[0pt]
La conscience est-elle une illusion ? \\[0pt]
La conscience morale est-elle innée ? \\[0pt]
La conscience morale est-elle naturelle ? \\[0pt]
La conscience morale est-elle une illusion ? \\[0pt]
La conscience morale n'est-elle que le fruit de l'éducation ? \\[0pt]
La conscience morale n'est-elle que le produit de l'éducation ? \\[0pt]
La conscience nous leurre-t-elle ? \\[0pt]
La conscience peut-elle être collective ? \\[0pt]
La conscience peut-elle être objet de science ? \\[0pt]
La conscience peut-elle nous tromper ? \\[0pt]
La considération de l'utilité doit-elle déterminer toutes nos actions ? \\[0pt]
La contingence est-elle la condition de la liberté ? \\[0pt]
La contingence est-elle une condition de la liberté ? \\[0pt]
La contradiction réside-t-elle dans les choses ? \\[0pt]
La contrainte des lois est-elle une violence ? \\[0pt]
La contrainte peut-elle être légitime ? \\[0pt]
La contrainte s'oppose-t-elle à la liberté ? \\[0pt]
La contrainte supprime-t-elle la responsabilité ? \\[0pt]
La critique du pouvoir peut-elle conduire à la désobéissance ? \\[0pt]
La critique fait-elle partie de l'art ? \\[0pt]
La croyance en Dieu est-elle irrationnelle ? \\[0pt]
La croyance est-elle l'asile de l'ignorance ? \\[0pt]
La croyance est-elle signe de faiblesse ? \\[0pt]
La croyance est-elle une opinion ? \\[0pt]
La croyance est-elle une opinion comme les autres ? \\[0pt]
La croyance peut-elle être rationnelle ? \\[0pt]
La croyance peut-elle s'accompagner de certitude ? \\[0pt]
La croyance peut-elle tenir lieu de savoir ? \\[0pt]
La croyance religieuse échappe-t-elle à toute logique ? \\[0pt]
La croyance religieuse se distingue-t-elle des autres formes de croyance ? \\[0pt]
L'action humaine nécessite-t-elle la contingence du monde ? \\[0pt]
L'action morale dépend-elle des circonstances ? \\[0pt]
L'action politique a-t-elle un fondement rationnel ? \\[0pt]
L'action politique peut-elle se passer de mots ? \\[0pt]
L'action requiert-elle décision d'un sujet ? \\[0pt]
L'activité philosophique conduit-elle à la métaphysique ? \\[0pt]
L'activité se laisse-t-elle programmer ? \\[0pt]
L'activité technique dévalorise-t-elle l'homme ? \\[0pt]
La culture commence-t-elle avec le refus de la nature ? \\[0pt]
La culture engendre-t-elle le progrès ? \\[0pt]
La culture est-elle affaire de politique ? \\[0pt]
La culture est-elle ce qui nous relie au passé ? \\[0pt]
La culture est-elle ce qui unit ou ce qui sépare les hommes ? \\[0pt]
La culture est-elle la négation de la nature ? \\[0pt]
La culture est-elle l'objet naturel des sciences humaines ? \\[0pt]
La culture est-elle nécessaire à l'appréciation d'une œuvre d'art ? \\[0pt]
La culture est-elle une question politique ? \\[0pt]
La culture est-elle une seconde nature ? \\[0pt]
La culture est-elle un luxe ? \\[0pt]
La culture est-elle un objet ? \\[0pt]
La culture garantit-elle l'excellence humaine ? \\[0pt]
La culture historique forge-t-elle le jugement moral ? \\[0pt]
La culture libère-t-elle des préjugés ? \\[0pt]
La culture nous rend-elle meilleurs ? \\[0pt]
La culture nous rend-elle plus humains ? \\[0pt]
La culture nous unit-elle ? \\[0pt]
La culture peut-elle être instituée ? \\[0pt]
La culture peut-elle être objet de science ? \\[0pt]
La culture : pour quoi faire ? \\[0pt]
La culture rend-elle plus humain ? \\[0pt]
La curiosité est-elle à l'origine du savoir ? \\[0pt]
La danse est-elle l'œuvre du corps ? \\[0pt]
La décision a-t-elle besoin de raisons ? \\[0pt]
La découverte de la vérité peut-elle être le fait du hasard ? \\[0pt]
La découverte scientifique a-t-elle une logique ? \\[0pt]
La défense de l'intérêt général est-il la fin dernière de la politique ? \\[0pt]
La démarche scientifique exclut-elle tout recours à l'imagination ? \\[0pt]
La démocratie a-t-elle des limites ? \\[0pt]
La démocratie a-t-elle une histoire ? \\[0pt]
La démocratie conduit-elle au règne de l'opinion ? \\[0pt]
La démocratie, est-ce la fin du despotisme ? \\[0pt]
La démocratie, est-ce le pouvoir du plus grand nombre ? \\[0pt]
La démocratie est-elle la loi du plus fort ? \\[0pt]
La démocratie est-elle le pire des régimes politiques ? \\[0pt]
La démocratie est-elle le règne de l'opinion ? \\[0pt]
La démocratie est-elle moyen ou fin ? \\[0pt]
La démocratie est-elle nécessairement libérale ? \\[0pt]
La démocratie est-elle possible ? \\[0pt]
La démocratie est-elle un mythe ? \\[0pt]
La démocratie n'est-elle que la force des faibles ? \\[0pt]
La démocratie peut-elle échapper à la démagogie ? \\[0pt]
La démocratie peut-elle être représentative ? \\[0pt]
La démocratie peut-elle se passer de représentation ? \\[0pt]
La démonstration mathématique peut-elle servir de modèle à toute démonstration ? \\[0pt]
La démonstration n'est-elle qu'une méthode d'exposition des découvertes ? \\[0pt]
La démonstration nous garantit-elle l'accès à la vérité ? \\[0pt]
La démonstration obéit-elle à des lois ? \\[0pt]
La démonstration suffit-elle à établir la vérité ? \\[0pt]
La démonstration supprime-t-elle le doute ? \\[0pt]
L'adéquation aux choses suffit-elle à définir la vérité ? \\[0pt]
La désobéissance peut-elle être légitime ? \\[0pt]
La dialectique est-elle une science ? \\[0pt]
La dialectique remet-elle en question le principe de contradiction ? \\[0pt]
La différence des sexes est-elle une question philosophique ? \\[0pt]
La différence des sexes est-elle un problème philosophique ? \\[0pt]
La distinction de la nature et de la culture est-elle un fait de culture ? \\[0pt]
La distinction entre vraie et fausse sciences a-t-elle un sens ? \\[0pt]
La diversité des cultures fait-elle obstacle à l'unité du genre humain ? \\[0pt]
La diversité des langues est-elle une diversité des pensées ? \\[0pt]
La diversité des langues est-elle un obstacle à l'entente entre les hommes ? \\[0pt]
La diversité des opinions conduit-elle à douter de tout ? \\[0pt]
La docilité est-elle un vice ou une vertu ? \\[0pt]
La domination technique de la nature doit-elle susciter la crainte ou l'espoir ? \\[0pt]
La douleur est-elle utile ? \\[0pt]
La douleur nous apprend-elle quelque chose ? \\[0pt]
La famille est-elle le lieu de la formation morale ? \\[0pt]
La famille est-elle naturelle ? \\[0pt]
La famille est-elle une communauté naturelle ? \\[0pt]
La famille est-elle une institution politique ? \\[0pt]
La famille est-elle une invention culturelle ? \\[0pt]
La famille est-elle un modèle de société ? \\[0pt]
La femme est-elle l'avenir de l'homme ? \\[0pt]
La fiction est-elle fausse ? \\[0pt]
La finalité est-elle nécessaire pour penser le vivant ? \\[0pt]
La fin de la politique est-elle d'éliminer la violence ? \\[0pt]
La fin de la politique est-elle l'établissement de la justice ? \\[0pt]
La fin de la technique se résume-t-elle à son utilité ? \\[0pt]
La fin justifie-t-elle les moyens ? \\[0pt]
La foi est-elle aveugle ? \\[0pt]
La foi est-elle irrationnelle ? \\[0pt]
La foi est-elle rationnelle ? \\[0pt]
La fonction de penser peut-elle se déléguer ? \\[0pt]
La fonction du philosophe est-elle de diriger l'État ? \\[0pt]
La fonction première de l'État est-elle de durer ? \\[0pt]
La force de l'État est-elle nécessaire à la liberté des citoyens ? \\[0pt]
La force donne-t-elle des droits ? \\[0pt]
La force est-elle une vertu ? \\[0pt]
La force et le droit s'opposent-ils nécessairement ? \\[0pt]
La force fait-elle le droit ? \\[0pt]
La franchise est-elle une vertu ? \\[0pt]
La fraternité a-t-elle un sens politique ? \\[0pt]
La fraternité est-elle un idéal moral ? \\[0pt]
La fraternité peut-elle se passer d'un fondement religieux ? \\[0pt]
La fuite du temps est-elle nécessairement un malheur ? \\[0pt]
La généalogie peut-elle être une méthode historique ? \\[0pt]
La géographie est-elle une science humaine ? \\[0pt]
La gloire est-elle un bien ? \\[0pt]
La grammaire contraint-elle la pensée ? \\[0pt]
La grammaire contraint-elle notre pensée ? \\[0pt]
La grammaire véhicule-t-elle une métaphysique ? \\[0pt]
La guerre est-elle la continuation de la politique ? \\[0pt]
La guerre est-elle la continuation de la politique par d'autres moyens ? \\[0pt]
La guerre est-elle la plus grande violence ? \\[0pt]
La guerre est-elle la politique continuée par d'autres moyens ? \\[0pt]
La guerre est-elle l'essentiel de toute politique ? \\[0pt]
La guerre met-elle fin au droit ? \\[0pt]
La guerre peut-elle être juste ? \\[0pt]
La guerre peut-elle être justifiée ? \\[0pt]
La guerre peut-elle être un droit ? \\[0pt]
Laisser mourir, est-ce tuer ? \\[0pt]
La justice a-t-elle besoin des institutions ? \\[0pt]
La justice a-t-elle pour fonction de compenser les inégalités naturelles ? \\[0pt]
La justice a-t-elle un fondement rationnel ? \\[0pt]
La justice consiste-t-elle à traiter tout le monde de la même manière ? \\[0pt]
La justice consiste-t-elle dans l'application de la loi ? \\[0pt]
La justice est-elle affaire de raison ? \\[0pt]
La justice est-elle de ce monde ? \\[0pt]
La justice est-elle de l'ordre du sentiment ? \\[0pt]
La justice est-elle l'affaire de l'État ? \\[0pt]
La justice est-elle le ciment de la paix sociale ? \\[0pt]
La justice est-elle une idée ? \\[0pt]
La justice est-elle une notion morale ? \\[0pt]
La justice est-elle une vertu ? \\[0pt]
La justice est-elle un idéal rationnel ? \\[0pt]
La justice exclut-elle l'arbitraire ? \\[0pt]
La justice : moyen ou fin de la politique ? \\[0pt]
La justice n'est-elle qu'une institution ? \\[0pt]
La justice n'est-elle qu'un idéal ? \\[0pt]
La justice peut-elle être fondée en nature ? \\[0pt]
La justice peut-elle se fonder sur le compromis ? \\[0pt]
La justice peut-elle se passer de la force ? \\[0pt]
La justice peut-elle se passer d'institutions ? \\[0pt]
La justice requiert-elle l'infaillibilité des juges ? \\[0pt]
La justice requiert-elle toujours l'impartialité ? \\[0pt]
La justice suppose-t-elle l'égalité ? \\[0pt]
La laideur est-elle une valeur esthétique ? \\[0pt]
La langue nous parle-t-elle ? \\[0pt]
La légitimité d'un clonage humain ? \\[0pt]
La liberté admet-elle des degrés ? \\[0pt]
La liberté a-t-elle une histoire ? \\[0pt]
La liberté a-t-elle un prix ? \\[0pt]
La liberté comporte-t-elle des degrés ? \\[0pt]
La liberté, concept moral ? \\[0pt]
La liberté connaît-elle des excès ? \\[0pt]
La liberté des uns s'arrête-elle où commence celle des autres ? \\[0pt]
La liberté d'expression a-t-elle des limites ? \\[0pt]
La liberté d'expression est-elle nécessaire à la liberté de pensée ? \\[0pt]
La liberté d'expression est-elle nécessaire à la liberté de penser ? \\[0pt]
La liberté doit-elle être limitée ? \\[0pt]
La liberté doit-elle se conquérir ? \\[0pt]
La liberté, est-ce l'indépendance à l'égard des passions ? \\[0pt]
La liberté est-elle ce qui définit l'homme ? \\[0pt]
La liberté est-elle contraire au principe de causalité ? \\[0pt]
La liberté est-elle innée ? \\[0pt]
La liberté est-elle la condition du bonheur ? \\[0pt]
La liberté est-elle le fondement de la responsabilité ? \\[0pt]
La liberté est-elle le pouvoir de refuser ? \\[0pt]
La liberté est-elle menacée par l'égalité ? \\[0pt]
La liberté est-elle une illusion ? \\[0pt]
La liberté est-elle une illusion nécessaire ? \\[0pt]
La liberté est-elle un fait ? \\[0pt]
La liberté et l'égalité sont-elles compatibles ? \\[0pt]
La liberté fait-elle de nous des êtres meilleurs ? \\[0pt]
La liberté implique-t-elle l'indifférence ? \\[0pt]
La liberté impose-t-elle des devoirs ? \\[0pt]
La liberté intéresse-t-elle les sciences humaines ? \\[0pt]
La liberté ne s'éprouve-t-elle que dans la solitude ? \\[0pt]
La liberté n'est-elle qu'un droit ? \\[0pt]
La liberté n'est-elle qu'une illusion ? \\[0pt]
La liberté nous rend-elle inexcusables ? \\[0pt]
La liberté peut-elle être prouvée ? \\[0pt]
La liberté peut-elle être une illusion ? \\[0pt]
La liberté peut-elle faire peur ? \\[0pt]
La liberté peut-elle s'affirmer sans violence ? \\[0pt]
La liberté peut-elle s'aliéner ? \\[0pt]
La liberté peut-elle se constater ? \\[0pt]
La liberté peut-elle se prouver ? \\[0pt]
La liberté peut-elle se refuser ? \\[0pt]
La liberté requiert-elle le libre échange ? \\[0pt]
La liberté s'achète-t-elle ? \\[0pt]
La liberté s'apprend-elle ? \\[0pt]
La liberté se mérite-t-elle ? \\[0pt]
La liberté se prouve-t-elle ? \\[0pt]
La liberté s'éprouve-t-elle ? \\[0pt]
La liberté se prouve-t-elle ou s'éprouve-t-elle ? \\[0pt]
La liberté se réduit-elle au libre arbitre ? \\[0pt]
La liberté suppose-t-elle l'absence de déterminisme ? \\[0pt]
La linguistique est-elle une science de l'information ? \\[0pt]
La linguistique fait-elle partie des sciences humaines ? \\[0pt]
La linguistique peut-elle se passer de la psychologie ? \\[0pt]
La littérature est-elle la mémoire de l'humanité ? \\[0pt]
La littérature peut-elle suppléer les sciences de l'homme ? \\[0pt]
La logique a-t-elle une histoire ? \\[0pt]
La logique a-t-elle un intérêt philosophique ? \\[0pt]
La logique : calcul ou langage ? \\[0pt]
La logique : découverte ou invention ? \\[0pt]
La logique décrit-elle le monde ? \\[0pt]
La logique est-elle indépendante de la psychologie ? \\[0pt]
La logique est-elle la norme du vrai ? \\[0pt]
La logique est-elle l'art de penser ? \\[0pt]
La logique est-elle un art de penser ? \\[0pt]
La logique est-elle un art de raisonner ? \\[0pt]
La logique est-elle une discipline normative ? \\[0pt]
La logique est-elle une forme de calcul ? \\[0pt]
La logique est-elle une science ? \\[0pt]
La logique est-elle une science de la vérité ? \\[0pt]
La logique est-elle utile à la métaphysique ? \\[0pt]
La logique nous apprend-elle quelque chose sur le langage ordinaire ? \\[0pt]
« La logique » ou bien « les logiques » ? \\[0pt]
La logique peut-elle se passer de la métaphysique ? \\[0pt]
La logique pourrait-elle nous surprendre ? \\[0pt]
La loi dit-elle ce qui est juste ? \\[0pt]
La loi doit-elle être la même pour tous ? \\[0pt]
La loi éduque-t-elle ? \\[0pt]
La loi est-elle une garantie contre l'injustice ? \\[0pt]
La loi peut-elle changer les mœurs ? \\[0pt]
La loi peut-elle être injuste ? \\[0pt]
L'altruisme est-il une attitude morale ? \\[0pt]
L'altruisme n'est-il qu'un égoïsme bien compris ? \\[0pt]
La machine fournit-elle un modèle pour penser le vivant ? \\[0pt]
La magie peut-elle être efficace ? \\[0pt]
La majorité a-t-elle toujours raison ? \\[0pt]
La majorité doit-elle toujours l'emporter ? \\[0pt]
La majorité, force ou droit ? \\[0pt]
La majorité peut-elle être tyrannique ? \\[0pt]
La maladie est-elle à l'organisme vivant ce que la panne est à la machine ? \\[0pt]
La maladie est-elle indispensable à la connaissance du vivant ? \\[0pt]
La maladie mentale a-t-elle une histoire ? \\[0pt]
La mathématique est-elle une ontologie ? \\[0pt]
La matière, est-ce le mal ? \\[0pt]
La matière, est-ce l'informe ? \\[0pt]
La matière est-elle amorphe ? \\[0pt]
La matière est-elle connaissable ? \\[0pt]
La matière est-elle plus aisée à connaître que l'esprit ? \\[0pt]
La matière est-elle plus facile à connaître que l'esprit ? \\[0pt]
La matière est-elle une substance ? \\[0pt]
La matière est-elle une vue de l'esprit ? \\[0pt]
La matière et l'esprit peuvent-ils constituer une seule chose ? \\[0pt]
La matière n'est-elle que ce que l'on perçoit ? \\[0pt]
La matière n'est-elle qu'une idée ? \\[0pt]
La matière n'est-elle qu'un obstacle ? \\[0pt]
La matière pense-t-elle ? \\[0pt]
La matière peut-elle agir ? \\[0pt]
La matière peut-elle être objet de connaissance ? \\[0pt]
La matière peut-elle penser ? \\[0pt]
L'ambiguïté des mots peut-elle être heureuse ? \\[0pt]
L'âme concerne-t-elle les sciences humaines ? \\[0pt]
La médecine est-elle une science ? \\[0pt]
L'âme est-elle immortelle ? \\[0pt]
L'âme et le corps sont-ils une seule et même chose ? \\[0pt]
L'âme jouit-elle d'une vie propre ? \\[0pt]
L'amélioration des hommes peut-elle être considérée comme un objectif politique ? \\[0pt]
La mémoire est-elle une fonction vitale ? \\[0pt]
La métaphore n'est-elle qu'un ornement du discours ? \\[0pt]
La métaphysique a-t-elle ses fictions ? \\[0pt]
La métaphysique est-elle affaire de raisonnement ? \\[0pt]
La métaphysique est-elle le fondement de la morale ? \\[0pt]
La métaphysique est-elle nécessairement une réflexion sur Dieu ? \\[0pt]
La métaphysique est-elle un discours sans objet ? \\[0pt]
La métaphysique est-elle une discipline théorique ? \\[0pt]
La métaphysique est-elle une science ? \\[0pt]
La métaphysique peut-elle être autre chose qu'une science recherchée ? \\[0pt]
La métaphysique peut-elle faire appel à l'expérience ? \\[0pt]
La métaphysique procure-t-elle un savoir ? \\[0pt]
La métaphysique relève-t-elle de la philosophie ou de la poésie ? \\[0pt]
La métaphysique répond-elle à un besoin ? \\[0pt]
La métaphysique repose-t-elle sur des croyances ? \\[0pt]
La métaphysique se définit-elle par son objet ou sa démarche ? \\[0pt]
La méthode est-elle nécessaire pour la recherche de la vérité ? \\[0pt]
La méthode expérimentale est-elle appropriée à l'étude du vivant ? \\[0pt]
L'amitié est-elle une vertu ? \\[0pt]
L'amitié est-elle un principe politique ? \\[0pt]
L'amitié peut-elle obliger ? \\[0pt]
L'amitié relève-t-elle d'une décision ? \\[0pt]
La modération est-elle l'essence de la vertu ? \\[0pt]
La modération est-elle une vertu politique ? \\[0pt]
La monnaie n'est-elle qu'un moyen d'échange ? \\[0pt]
La morale a-t-elle à décider de la sexualité ? \\[0pt]
La morale a-t-elle besoin de la notion de sainteté ? \\[0pt]
La morale a-t-elle besoin d'être fondée ? \\[0pt]
La morale a-t-elle besoin d'un au-delà ? \\[0pt]
La morale a-t-elle besoin d'un fondement ? \\[0pt]
La morale a-t-elle sa place dans l'économie ? \\[0pt]
La morale consiste-t-elle à respecter le droit ? \\[0pt]
La morale consiste-t-elle à suivre la nature ? \\[0pt]
La morale dépend-elle de la culture ? \\[0pt]
La morale doit-elle en appeler à la nature ? \\[0pt]
La morale doit-elle être rationnelle ? \\[0pt]
La morale doit-elle fournir des préceptes ? \\[0pt]
La morale est-elle affaire de convention ? \\[0pt]
La morale est-elle affaire de jugement ? \\[0pt]
La morale est-elle affaire de sentiment ? \\[0pt]
La morale est-elle affaire de sentiments ? \\[0pt]
La morale est-elle affaire individuelle ? \\[0pt]
La morale est-elle condamnée à n'être qu'un champ de bataille ? \\[0pt]
La morale est-elle désintéressée ? \\[0pt]
La morale est-elle émotion ? \\[0pt]
La morale est-elle en conflit avec le désir ? \\[0pt]
La morale est-elle ennemie du bonheur ? \\[0pt]
La morale est-elle fondée sur la liberté ? \\[0pt]
La morale est-elle incompatible avec le déterminisme ? \\[0pt]
La morale est-elle l'ennemie de la vie ? \\[0pt]
La morale est-elle nécessairement répressive ? \\[0pt]
La morale est-elle objet de science ? \\[0pt]
La morale est-elle un art de vivre ? \\[0pt]
La morale est-elle un besoin ? \\[0pt]
La morale est-elle une affaire de raison ? \\[0pt]
La morale est-elle une affaire d'habitude ? \\[0pt]
La morale est-elle une affaire solitaire ? \\[0pt]
La morale est-elle une médecine de l'âme ? \\[0pt]
La morale est-elle une preuve de la liberté ? \\[0pt]
La morale est-elle une science ? \\[0pt]
La morale est-elle un fait de culture ? \\[0pt]
La morale est-elle un fait social ? \\[0pt]
La morale et la religion visent-elles les mêmes fins ? \\[0pt]
La morale n'est-elle qu'un art de vivre ? \\[0pt]
La morale n'est-elle qu'un dressage ? \\[0pt]
La morale n'est-elle qu'un ensemble de conventions ? \\[0pt]
La morale peut-elle être fondée sur la science ? \\[0pt]
La morale peut-elle être naturelle ? \\[0pt]
La morale peut-elle être personnelle ? \\[0pt]
La morale peut-elle être un calcul ? \\[0pt]
La morale peut-elle être une science ? \\[0pt]
La morale peut-elle se définir comme l'art d'être heureux ? \\[0pt]
La morale peut-elle se fonder sur les sentiments ? \\[0pt]
La morale peut-elle s'enseigner ? \\[0pt]
La morale peut-elle se passer d'un fondement religieux ? \\[0pt]
La morale requiert-elle une casuistique ? \\[0pt]
La morale requiert-elle un fondement ? \\[0pt]
La morale s'apprend-elle ? \\[0pt]
La morale se déduit-elle de la métaphysique ? \\[0pt]
La morale s'enracine-t-elle dans le sens commun ? \\[0pt]
La morale s'enseigne-t-elle ? \\[0pt]
La morale s'oppose-t-elle à la politique ? \\[0pt]
La morale suppose-t-elle le libre arbitre ? \\[0pt]
La moralité consiste-t-elle à se contraindre soi-même ? \\[0pt]
La moralité est-elle affaire de principes ou de conséquences ? \\[0pt]
La moralité est-elle utile à la vie sociale ? \\[0pt]
La moralité n'est-elle que dressage ? \\[0pt]
La moralité réside-t-elle dans l'intention ? \\[0pt]
La moralité se juge-t-elle aux actes ? \\[0pt]
La moralité se réduit-elle aux sentiments ? \\[0pt]
La mort a-t-elle un sens ? \\[0pt]
La mort fait-elle partie de la vie ? \\[0pt]
L'amour a-t-il des raisons ? \\[0pt]
L'amour de soi est-il immoral ? \\[0pt]
L'amour du pouvoir est-il compatible avec le dévouement à la chose publique ? \\[0pt]
L'amour est-il aveugle ? \\[0pt]
L'amour est-il désir ? \\[0pt]
L'amour est-il sans raison ? \\[0pt]
L'amour est-il toujours une passion ? \\[0pt]
L'amour est-il une vertu ? \\[0pt]
L'amour implique-t-il le respect ? \\[0pt]
L'amour peut-il être absolu ? \\[0pt]
L'amour peut-il être raisonnable ? \\[0pt]
L'amour peut-il être un devoir ? \\[0pt]
La musique a-t-elle une essence ? \\[0pt]
La musique a-t-elle un sens ? \\[0pt]
La musique donne-t-elle à penser ? \\[0pt]
La musique est-elle un discours ? \\[0pt]
La musique est-elle un langage ? \\[0pt]
La musique exprime-t-elle quelque chose ? \\[0pt]
La musique peut-elle être narrative ? \\[0pt]
La naïveté est-elle une vertu ? \\[0pt]
L'analyse des comportements humains permet-elle de réduire le hasard ? \\[0pt]
L'analyse du langage ordinaire peut-elle avoir un intérêt philosophique ? \\[0pt]
L'analyse économique rend-elle compte des comportements humains ? \\[0pt]
La nation est-elle dépassée ? \\[0pt]
La nature a-t-elle des droits ? \\[0pt]
La nature a-t-elle une histoire ? \\[0pt]
La nature a-t-elle une valeur ? \\[0pt]
La nature a-t-elle un langage ? \\[0pt]
La nature a-t-elle un ordre ? \\[0pt]
La nature donne-t-elle un sens à notre existence ? \\[0pt]
La nature, est-ce le donné ? \\[0pt]
La nature est-elle artiste ? \\[0pt]
La nature est-elle belle ? \\[0pt]
La nature est-elle bien faite ? \\[0pt]
La nature est-elle digne de respect ? \\[0pt]
La nature est-elle écrite en langage mathématique ? \\[0pt]
La nature est-elle muette ? \\[0pt]
La nature est-elle politique ? \\[0pt]
La nature est-elle prévisible ? \\[0pt]
La nature est-elle sacrée ? \\[0pt]
La nature est-elle sans histoire ? \\[0pt]
La nature est-elle sauvage ? \\[0pt]
La nature est-elle une grande artiste ? \\[0pt]
La nature est-elle une idée ? \\[0pt]
La nature est-elle une invention de l'homme ? \\[0pt]
La nature est-elle une norme ? \\[0pt]
La nature est-elle une ressource ? \\[0pt]
La nature est-elle un guide ? \\[0pt]
La nature est-elle un modèle ? \\[0pt]
La nature est-elle un système ? \\[0pt]
La nature existe-t-elle ? \\[0pt]
La nature fait-elle bien les choses ? \\[0pt]
La nature imite-t-elle l'art ? \\[0pt]
La nature ne fait-elle rien en vain ? \\[0pt]
La nature nous dicte-t-elle des fins ? \\[0pt]
La nature nous domine-t-elle ? \\[0pt]
La nature nous donne-t-elle des commandements ? \\[0pt]
La nature nous indique-t-elle ce qui est bon ? \\[0pt]
La nature obéit-elle à des fins ? \\[0pt]
La nature parle-t-elle le langage des mathématiques ? \\[0pt]
La nature peut-elle avoir des droits ? \\[0pt]
La nature peut-elle constituer une norme ? \\[0pt]
La nature peut-elle être belle ? \\[0pt]
La nature peut-elle être détruite ? \\[0pt]
La nature peut-elle être un modèle ? \\[0pt]
La nature peut-elle nous indiquer ce que nous devons faire ? \\[0pt]
La nature reprend-elle toujours ses droits ? \\[0pt]
La nature se donne-t-elle à penser ? \\[0pt]
La nature s'oppose-t-elle à l'esprit ? \\[0pt]
La nécessité fait-elle loi ? \\[0pt]
La négligence est-elle une faute ? \\[0pt]
La neige est-elle blanche ? \\[0pt]
L'animal apprend-il à l'homme quelque chose sur l'homme ? \\[0pt]
L'animal a-t-il des droits ? \\[0pt]
L'animal est-il une personne ? \\[0pt]
L'animal nous apprend-il quelque chose sur l'homme ? \\[0pt]
L'animal peut-il être un sujet moral ? \\[0pt]
La non-contradiction : principe logique ou ontologique ? \\[0pt]
La notion de barbarie a-t-elle un sens ? \\[0pt]
La notion de caractère relève-t- elle de la morale ou de la psychologie ? \\[0pt]
La notion de démocratie représentative est-elle contradictoire ? \\[0pt]
La notion de droit international est-elle contradictoire ? \\[0pt]
La notion de finalité a-t-elle de l'intérêt pour le savant ? \\[0pt]
La notion de loi a-t-elle une unité ? \\[0pt]
La notion de nature humaine a-t-elle un sens ? \\[0pt]
La notion de paradis a-t-elle un sens exclusivement religieux ? \\[0pt]
La notion de progrès a-t-elle un sens en politique ? \\[0pt]
La notion de progrès moral a-t-elle encore un sens ? \\[0pt]
La notion de race a-t-elle un statut scientifique ? \\[0pt]
La notion de responsabilité est-elle suspecte ? \\[0pt]
La notion de signification est-elle indispensable aux sciences humaines ? \\[0pt]
La notion d'histoire commune a-t-elle un sens ? \\[0pt]
L'anthropologie est-elle une ontologie ? \\[0pt]
La ou les vertus ? \\[0pt]
La paix est-elle l'absence de guerre ? \\[0pt]
La paix est-elle l'absence de guerres ? \\[0pt]
La paix est-elle la visée de la politique ? \\[0pt]
La paix est-elle le plus grand des biens ? \\[0pt]
La paix est-elle moins naturelle que la guerre ? \\[0pt]
La paix est-elle possible ? \\[0pt]
La paix est-elle toujours préférable ? \\[0pt]
La paix met-elle fin à l'histoire ? \\[0pt]
La paix n'est-elle que l'absence de conflit ? \\[0pt]
La paix n'est-elle que l'absence de guerre ? \\[0pt]
La paix n'est-elle qu'un idéal ? \\[0pt]
La paix sociale est-elle la finalité de la politique ? \\[0pt]
La paix sociale est-elle le but de la politique ? \\[0pt]
La paix sociale est-elle une fin en soi ? \\[0pt]
La parole nous libère-t-elle ? \\[0pt]
La parole peut-elle agir ? \\[0pt]
La parole peut-elle être étudiée comme un acte social ? \\[0pt]
La parole peut-elle être une arme ? \\[0pt]
La passion de la vérité peut-elle être source d'erreur ? \\[0pt]
La passion est-elle immorale ? \\[0pt]
La passion est-elle l'ennemi de la raison ? \\[0pt]
La passion exclut-elle la lucidité ? \\[0pt]
La passion n'est-elle que souffrance ? \\[0pt]
La patience est-elle une vertu ? \\[0pt]
La pauvreté est-elle une injustice ? \\[0pt]
La pédagogie, philosophie pratique ou psychologie appliquée ? \\[0pt]
La peine de mort est-elle juste, injuste, et pourquoi ? \\[0pt]
La peinture apprend-elle à voir ? \\[0pt]
La peinture est-elle une poésie muette ? \\[0pt]
La peinture peut-elle être un art du temps ? \\[0pt]
La pensée a-t-elle besoin d'images ? \\[0pt]
La pensée a-t-elle des lois ? \\[0pt]
La pensée a-t-elle une histoire ? \\[0pt]
La pensée de la mort a-t-elle un objet ? \\[0pt]
La pensée doit-elle se soumettre aux règles de la logique ? \\[0pt]
La pensée échappe-t-elle à la grammaire ? \\[0pt]
La pensée est-elle en lutte avec le langage ? \\[0pt]
La pensée est-elle une activité assimilable à un travail ? \\[0pt]
La pensée et la conscience sont-elles une seule et même chose ? \\[0pt]
La pensée formelle est-elle privée d'objet ? \\[0pt]
La pensée formelle est-elle une pensée vide ? \\[0pt]
La pensée formelle peut-elle avoir un contenu ? \\[0pt]
La pensée obéit-elle à des lois ? \\[0pt]
La pensée peut-elle devenir une technique ? \\[0pt]
La pensée peut-elle s'écrire ? \\[0pt]
La pensée peut-elle se passer de mots ? \\[0pt]
La pensée philosophique doit-elle être une pensée systématique ? \\[0pt]
La perception construit-elle son objet ? \\[0pt]
La perception de la folie change-t-elle de nature avec les sciences humaines ? \\[0pt]
La perception de l'espace est-elle innée ou acquise ? \\[0pt]
La perception est-elle le premier degré de la connaissance ? \\[0pt]
La perception est-elle l'interprétation du réel ? \\[0pt]
La perception est-elle source de connaissance ? \\[0pt]
La perception est-elle une interprétation ? \\[0pt]
La perception est-elle un mode de connaissance ? \\[0pt]
La perception me donne-t-elle le réel ? \\[0pt]
La perception peut-elle être désintéressée ? \\[0pt]
La perception peut-elle s'éduquer ? \\[0pt]
La perfection est-elle désirable ? \\[0pt]
La peur est-elle mauvaise conseillère ? \\[0pt]
La philosophie a-t-elle une histoire ? \\[0pt]
La philosophie a-t-elle un objet qui lui soit propre ? \\[0pt]
La philosophie doit-elle être systématique ? \\[0pt]
La philosophie doit-elle être une science ? \\[0pt]
La philosophie doit-elle se préoccuper du salut ? \\[0pt]
La philosophie est-elle abstraite ? \\[0pt]
La philosophie est-elle la servante de la science ? \\[0pt]
La philosophie est-elle une méditation de la mort ? \\[0pt]
La philosophie est-elle une méditation de la vie ou de la mort ? \\[0pt]
La philosophie est-elle une science ? \\[0pt]
La philosophie peut-elle disparaître ? \\[0pt]
La philosophie peut-elle être expérimentale ? \\[0pt]
La philosophie peut-elle être populaire ? \\[0pt]
La philosophie peut-elle être une science ? \\[0pt]
La philosophie peut-elle ne pas parler de Dieu ? \\[0pt]
La philosophie peut-elle se passer de l'idée de vérité ? \\[0pt]
La philosophie peut-elle se passer de théologie ? \\[0pt]
La philosophie rend-elle inefficace la propagande ? \\[0pt]
La photographie est-elle un art ? \\[0pt]
La physique est-elle le modèle de toutes les sciences ? \\[0pt]
La pitié a-t-elle une valeur ? \\[0pt]
La pitié est-elle dangereuse ? \\[0pt]
La pitié est-elle morale ? \\[0pt]
La pitié est-elle un sentiment moral ? \\[0pt]
La pitié fonde-t-elle la morale ? \\[0pt]
La pitié peut-elle fonder la morale ? \\[0pt]
La place de l'art est-elle sur le marché de l'art ? \\[0pt]
La pluralité des vérités condamne-t-elle l'idée de vérité ? \\[0pt]
La poésie dit-elle le réel ? \\[0pt]
La poésie est-elle comme une peinture ? \\[0pt]
La poésie pense-t-elle ? \\[0pt]
La politesse est-elle une vertu ? \\[0pt]
La politique : affaire de compétence ? \\[0pt]
La politique a-t-elle besoin de héros ? \\[0pt]
La politique a-t-elle besoin de la fiction ? \\[0pt]
La politique a-t-elle besoin de la religion ? \\[0pt]
La politique a-t-elle besoin de modèles ? \\[0pt]
La politique a-t-elle besoin d'experts ? \\[0pt]
La politique a-t-elle pour but de nous faire vivre dans un monde meilleur ? \\[0pt]
La politique a-t-elle pour fin d'éliminer la violence ? \\[0pt]
La politique consiste-t-elle à faire cause commune ? \\[0pt]
La politique consiste-t-elle à faire des compromis ? \\[0pt]
La politique consiste-t-elle à gérer l'urgence ? \\[0pt]
La politique doit-elle avoir pour visée le bonheur ? \\[0pt]
La politique doit-elle être morale ? \\[0pt]
La politique doit-elle être rationnelle ? \\[0pt]
La politique doit-elle protéger la liberté des citoyens ? \\[0pt]
La politique doit-elle refuser l'utopie ? \\[0pt]
La politique doit-elle se mêler de l'art ? \\[0pt]
La politique doit-elle se mêler du bonheur ? \\[0pt]
La politique doit-elle viser la concorde ? \\[0pt]
La politique doit-elle viser le consensus ? \\[0pt]
La politique échappe-t-elle à l'exigence de vérité ? \\[0pt]
La politique est-elle affaire de compétence ? \\[0pt]
La politique est-elle affaire de décision ? \\[0pt]
La politique est-elle affaire de jugement ? \\[0pt]
La politique est-elle affaire de science ? \\[0pt]
La politique est-elle affaire d'expérience ou de théorie ? \\[0pt]
La politique est-elle architectonique ? \\[0pt]
La politique est-elle continuation de la guerre, par d'autres moyens ? \\[0pt]
La politique est-elle extérieure au droit ? \\[0pt]
La politique est-elle la continuation de la guerre ? \\[0pt]
La politique est-elle la continuation de la guerre par d'autres moyens ? \\[0pt]
La politique est-elle l'affaire des spécialistes ? \\[0pt]
La politique est-elle l'affaire de tous ? \\[0pt]
La politique est-elle l'art de convaincre le peuple ? \\[0pt]
La politique est-elle l'art de la délibération ? \\[0pt]
La politique est-elle l'art des possibles ? \\[0pt]
La politique est-elle l'art du possible ? \\[0pt]
La politique est-elle le refus de la violence ? \\[0pt]
La politique est-elle le règne des passions ? \\[0pt]
La politique est-elle naturelle ? \\[0pt]
La politique est-elle par nature sujette à dispute ? \\[0pt]
La politique est-elle plus importante que tout ? \\[0pt]
La politique est-elle un art ? \\[0pt]
La politique est-elle une affaire d'experts ? \\[0pt]
La politique est-elle une science ? \\[0pt]
La politique est-elle une technique ? \\[0pt]
La politique est-elle un métier ? \\[0pt]
La politique est-elle un moyen ou une fin ? \\[0pt]
La politique exclut-elle le désordre ? \\[0pt]
La politique implique-t-elle la hiérarchie ? \\[0pt]
La politique n'est-elle que l'art de conquérir et de conserver le pouvoir ? \\[0pt]
La politique peut-elle changer la société ? \\[0pt]
La politique peut-elle changer le monde ? \\[0pt]
La politique peut-elle disparaître ? \\[0pt]
La politique peut-elle échapper au mythe ? \\[0pt]
La politique peut-elle être indépendante de la morale ? \\[0pt]
La politique peut-elle être morale ? \\[0pt]
La politique peut-elle être objet de science ? \\[0pt]
La politique peut-elle être un objet de science ? \\[0pt]
La politique peut-elle n'être qu'une pratique ? \\[0pt]
La politique peut-elle se passer de croyance ? \\[0pt]
La politique peut-elle se passer de croyances ? \\[0pt]
La politique peut-elle se passer des idéologies ? \\[0pt]
La politique peut-elle unir les hommes ? \\[0pt]
La politique peut-elle viser à autre chose qu'à l'efficacité ? \\[0pt]
La politique repose-t-elle sur un contrat ? \\[0pt]
La politique se réduit-elle à des rapports de force ? \\[0pt]
La politique suppose-t-elle la morale ? \\[0pt]
La politique suppose-t-elle une idée de l'homme ? \\[0pt]
La politique vise-t-elle à dépasser les injustices ? \\[0pt]
La poursuite de mon intérêt m'oppose-t-elle aux autres ? \\[0pt]
L'apparence est-elle toujours trompeuse ? \\[0pt]
L'apparence fait-elle obstacle à la vérité ? \\[0pt]
L'apparence fait-elle partie de la réalité ? \\[0pt]
L'approche psychologique de l'homme met-elle en question la notion d'âme ? \\[0pt]
La pratique des sciences met-elle à l'abri des préjugés ? \\[0pt]
La précaution peut-elle être un principe ? \\[0pt]
La présence d'autrui nous évite-t-elle la solitude ? \\[0pt]
L'a priori peut-il être historique ? \\[0pt]
La prise de parti est-elle essentielle en politique ? \\[0pt]
La prison est-elle utile ? \\[0pt]
La promesse n'est-elle qu'un acte de langage ? \\[0pt]
La propriété, est-ce un vol ? \\[0pt]
La propriété est-elle un droit ? \\[0pt]
La propriété est-elle une garantie de liberté ? \\[0pt]
La prudence est-elle une vertu politique ? \\[0pt]
La psychanalyse donne-t-elle un sens au rêve ? \\[0pt]
La psychanalyse est-elle une science ? \\[0pt]
La psychanalyse est-elle une science humaine ? \\[0pt]
La psychanalyse permet-elle la construction du sujet ? \\[0pt]
La psychanalyse s'applique-t-elle aux phénomènes sociaux ? \\[0pt]
La psychologie dit-elle ce qu'est le moi ? \\[0pt]
La psychologie est-elle l'étude de ce qui n'est pas social ? \\[0pt]
La psychologie est-elle l'étude des comportements individuels ? \\[0pt]
La psychologie est-elle une science ? \\[0pt]
La psychologie est-elle une science de la nature ? \\[0pt]
La psychologie est-elle une science humaine ? \\[0pt]
La puissance technique est- elle sans limites ? \\[0pt]
La question de la nature humaine relève-t-elle des sciences humaines ? \\[0pt]
La question du sens de la vie peut-elle être sans réponses ? \\[0pt]
La question « qu'est-ce que l'homme ? » est-elle scientifique ? \\[0pt]
La question : « qui ? » \\[0pt]
La question « qui suis-je » admet-elle une réponse exacte ? \\[0pt]
La radicalité est-elle une exigence philosophique ? \\[0pt]
La raison a-t-elle des limites ? \\[0pt]
La raison a-t-elle le droit d'expliquer ce que morale condamne ? \\[0pt]
La raison a-t-elle pour fin la connaissance ? \\[0pt]
La raison a-t-elle toujours raison ? \\[0pt]
La raison a-t-elle une histoire ? \\[0pt]
La raison d'État peut-elle être justifiée ? \\[0pt]
La raison doit-elle critiquer la croyance ? \\[0pt]
La raison doit-elle être cultivée ? \\[0pt]
La raison doit-elle être notre guide ? \\[0pt]
La raison doit-elle faire une place à la croyance ? \\[0pt]
La raison doit-elle se soumettre au réel ? \\[0pt]
La raison engendre-t-elle des illusions ? \\[0pt]
La raison épuise-t-elle le réel ? \\[0pt]
La raison, est-ce la modération ? \\[0pt]
La raison est-elle impersonnelle ? \\[0pt]
La raison est-elle le pouvoir de distinguer le vrai du faux ? \\[0pt]
La raison est-elle l'esclave des passions ? \\[0pt]
La raison est-elle l'esclave du désir ? \\[0pt]
La raison est-elle morale par elle-même ? \\[0pt]
La raison est-elle plus fiable que l'expérience ? \\[0pt]
La raison est-elle seulement affaire de logique ? \\[0pt]
La raison est-elle suffisante ? \\[0pt]
La raison est-elle toujours raisonnable ? \\[0pt]
La raison est-elle une valeur ? \\[0pt]
La raison est-elle un instrument ? \\[0pt]
La raison est-elle un obstacle au bonheur ? \\[0pt]
La raison gouverne-t-elle le monde ? \\[0pt]
La raison ne connaît-elle du réel que ce qu'elle y met d'elle-même ? \\[0pt]
La raison ne veut-elle que connaître ? \\[0pt]
La raison peut-elle entrer en conflit avec elle-même ? \\[0pt]
La raison peut-elle entrer en contradiction avec elle-même ? \\[0pt]
La raison peut-elle errer ? \\[0pt]
La raison peut-elle être immédiatement pratique ? \\[0pt]
La raison peut-elle être pratique ? \\[0pt]
La raison peut-elle nous commander de croire ? \\[0pt]
La raison peut-elle nous égarer ? \\[0pt]
La raison peut-elle nous induire en erreur ? \\[0pt]
La raison peut-elle rendre compte du réel entièrement ? \\[0pt]
La raison peut-elle rendre raison de tout ? \\[0pt]
La raison peut-elle s'aveugler elle-même ? \\[0pt]
La raison peut-elle se contredire ? \\[0pt]
La raison peut-elle servir le mal ? \\[0pt]
La raison peut-elle s'opposer à elle-même ? \\[0pt]
La raison s'identifie-t-elle au calcul ? \\[0pt]
La raison s'oppose-t-elle aux passions ? \\[0pt]
La raison transforme-t-elle le réel ? \\[0pt]
La rationalité se confond-elle avec la systématicité ? \\[0pt]
L'architecte : artiste ou ingénieur ? \\[0pt]
L'architecture est-elle un art ? \\[0pt]
La réalisation du devoir exclut-elle toute forme de plaisir ? \\[0pt]
La réalité a-t-elle une forme logique ? \\[0pt]
La réalité décrite par la science s'oppose-t-elle à la démonstration ? \\[0pt]
La réalité de la vie s'épuise-t-elle dans celle des vivants ? \\[0pt]
La réalité du temps se réduit-elle à la conscience que nous en avons ? \\[0pt]
La réalité est-elle une idée ? \\[0pt]
La réalité n'est-elle qu'une construction ? \\[0pt]
La réalité nourrit-elle la fiction ? \\[0pt]
La réalité peut-elle être virtuelle ? \\[0pt]
La recherche de la vérité a-t- elle une fin ? \\[0pt]
La recherche de la vérité peut-elle être désintéressée ? \\[0pt]
La recherche de la vérité peut-elle être une passion ? \\[0pt]
La recherche du bonheur est-elle égoïste ? \\[0pt]
La recherche du bonheur est-elle un idéal égoïste ? \\[0pt]
La recherche du bonheur est-il un guide suffisant dans la vie ? \\[0pt]
La recherche du bonheur peut-elle être un devoir ? \\[0pt]
La recherche du bonheur suffit-elle à déterminer une morale ? \\[0pt]
La recherche scientifique est-elle désintéressée ? \\[0pt]
La réciprocité est-elle indispensable à la communauté politique ? \\[0pt]
La reconnaissance de la diversité des cultures condamne-t-elle au relativisme ? \\[0pt]
La référence aux faits suffit-elle à garantir l'objectivité de la connaissance ? \\[0pt]
La réflexion sur l'expérience participe-t-elle de l'expérience ? \\[0pt]
La relation à autrui est-elle symétrique ? \\[0pt]
La relation de causalité est-elle temporelle ? \\[0pt]
La religion a-t-elle besoin d'un dieu ? \\[0pt]
La religion a-t-elle des vertus ? \\[0pt]
La religion a-t-elle les mêmes fins que la morale ? \\[0pt]
La religion a-t-elle une fonction de cohésion sociale ? \\[0pt]
La religion a-t-elle une fonction sociale ? \\[0pt]
La religion conduit-elle l'homme au-delà de lui-même ? \\[0pt]
La religion divise-t-elle les hommes ? \\[0pt]
La religion est-elle à craindre ? \\[0pt]
La religion est-elle au fondement de la morale ? \\[0pt]
La religion est-elle contraire à la raison ? \\[0pt]
La religion est-elle fondée sur la peur de la mort ? \\[0pt]
La religion est-elle la sagesse des pauvres ? \\[0pt]
La religion est-elle l'asile de l'ignorance ? \\[0pt]
La religion est-elle l'opium du peuple ? \\[0pt]
La religion est-elle relation à l'absolu ? \\[0pt]
La religion est-elle simple affaire de croyance ? \\[0pt]
La religion est-elle source de conflit ? \\[0pt]
La religion est-elle une affaire privée ? \\[0pt]
La religion est-elle une consolation pour les hommes ? \\[0pt]
La religion est-elle une production culturelle comme les autres ? \\[0pt]
La religion est-elle un facteur de lien social ? \\[0pt]
La religion est-elle un instrument de pouvoir ? \\[0pt]
La religion est-elle un obstacle à la liberté ? \\[0pt]
La religion implique-t-elle la croyance en un être divin ? \\[0pt]
La religion impose t-elle un joug salutaire à l'intelligence ? \\[0pt]
La religion n'est-elle que l'affaire des croyants ? \\[0pt]
La religion n'est-elle qu'une affaire privée ? \\[0pt]
La religion n'est-elle qu'un fait de culture ? \\[0pt]
La religion outrepasse-t-elle les limites de la raison ? \\[0pt]
La religion peut-elle être civile ? \\[0pt]
La religion peut-elle être naturelle ? \\[0pt]
La religion peut-elle faire lien social ? \\[0pt]
La religion peut-elle n'être qu'une affaire privée ? \\[0pt]
La religion peut-elle se passer de transcendant ? \\[0pt]
La religion peut-elle se passer du recours au sacré ? \\[0pt]
La religion peut-elle suppléer la raison ? \\[0pt]
La religion relève-t-elle de l'irrationnel ? \\[0pt]
La religion relève-t-elle de l'opinion ? \\[0pt]
La religion relie-t-elle les hommes ? \\[0pt]
La religion rend-elle l'homme heureux ? \\[0pt]
La religion rend-elle meilleur ? \\[0pt]
La religion repose-t-elle sur une illusion ? \\[0pt]
La religion se distingue-t-elle de la superstition ? \\[0pt]
La religion se réduit-elle à la foi ? \\[0pt]
La reproduction des œuvres d'art nuit-elle à l'art ? \\[0pt]
La responsabilité est-elle une fiction juridique ? \\[0pt]
La responsabilité implique-t-elle autrui ? \\[0pt]
La responsabilité peut-elle être collective ? \\[0pt]
La responsabilité politique n'est-elle le fait que de ceux qui gouvernent ? \\[0pt]
La révolte peut-elle être un droit ? \\[0pt]
L'argent est-il la mesure de tout échange ? \\[0pt]
L'argent est-il un mal nécessaire ? \\[0pt]
La rhétorique a-t-elle une valeur ? \\[0pt]
La rhétorique est-elle un art ? \\[0pt]
La rigueur des lois ? \\[0pt]
L'art ajoute-t-il quelque chose à la vie ? \\[0pt]
L'art apprend-il à percevoir ? \\[0pt]
L'art a-t-il à être populaire ? \\[0pt]
L'art a-t-il besoin de théorie ? \\[0pt]
L'art a-t-il besoin d'un discours sur l'art ? \\[0pt]
L'art a-t-il des règles ? \\[0pt]
L'art a-t-il des vertus thérapeutiques ? \\[0pt]
L'art a-t-il plus de valeur que la vérité ? \\[0pt]
L'art a-t-il pour fin la vérité ? \\[0pt]
L'art a-t-il pour fin le plaisir ? \\[0pt]
L'art a-t-il pour fonction de sublimer le réel ? \\[0pt]
L'art a-t-il une fin morale ? \\[0pt]
L'art a-t-il une histoire ? \\[0pt]
L'art a-t-il une responsabilité morale ? \\[0pt]
L'art a-t-il une valeur sociale ? \\[0pt]
L'art a-t-il un rôle à jouer dans l'éducation ? \\[0pt]
L'art change-t-il la vie ? \\[0pt]
L'art contre la beauté ? \\[0pt]
L'art décrit-il ? \\[0pt]
L'art de vivre est-il un art ? \\[0pt]
L'art doit-il divertir ? \\[0pt]
L'art doit-il être critique ? \\[0pt]
L'art doit-il nécessairement représenter la réalité ? \\[0pt]
L'art doit-il nous étonner ? \\[0pt]
L'art doit-il refaire le monde ? \\[0pt]
L'art donne-t-il à penser ? \\[0pt]
L'art donne-t-il à rêver ou à penser ? \\[0pt]
L'art donne-t-il à voir l'invisible ? \\[0pt]
L'art donne-t-il nécessairement lieu à la production d'une œuvre ? \\[0pt]
L'art échappe-t-il à la raison ? \\[0pt]
L'art éduque-t-il la perception ? \\[0pt]
L'art éduque-t-il l'homme ? \\[0pt]
L'art, est-ce ce qui résiste à la certitude ? \\[0pt]
L'art est-il affaire d'apparence ? \\[0pt]
L'art est-il affaire de goût ? \\[0pt]
L'art est-il affaire d'imagination ? \\[0pt]
L'art est-il à lui-même son propre but ? \\[0pt]
L'art est-il au service du beau ? \\[0pt]
L'art est-il ce qui permet de partager ses émotions ? \\[0pt]
L'art est-il désintéressé ? \\[0pt]
L'art est-il destiné à embellir ? \\[0pt]
L'art est-il hors du temps ? \\[0pt]
L'art est-il imitatif ? \\[0pt]
L'art est-il interprétation ? \\[0pt]
L'art est-il inutile ? \\[0pt]
L'art est-il le miroir du monde ? \\[0pt]
L'art est-il le produit de l'inconscient ? \\[0pt]
L'art est-il le propre de l'homme ? \\[0pt]
L'art est-il le règne des apparences ? \\[0pt]
L'art est-il mensonger ? \\[0pt]
L'art est-il méthodique ? \\[0pt]
L'art est-il moins nécessaire que la science ? \\[0pt]
L'art est-il objet de compréhension ? \\[0pt]
L'art est-il politique ? \\[0pt]
L'art est-il révolutionnaire ? \\[0pt]
L'art est-il subversif ? \\[0pt]
L'art est-il un divertissement ? \\[0pt]
L'art est-il une affaire sérieuse ? \\[0pt]
L'art est-il une connaissance ? \\[0pt]
L'art est-il une critique de la culture ? \\[0pt]
L'art est-il une expérience de la liberté ? \\[0pt]
L'art est-il une forme de langage ? \\[0pt]
L'art est-il une histoire ? \\[0pt]
L'art est-il une promesse de bonheur ? \\[0pt]
L'art est-il une valeur ? \\[0pt]
L'art est-il universel ? \\[0pt]
L'art est-il un jeu ? \\[0pt]
L'art est-il un langage ? \\[0pt]
L'art est-il un langage universel ? \\[0pt]
L'art est-il un luxe ? \\[0pt]
L'art est-il un mode de connaissance ? \\[0pt]
L'art est-il un modèle pour la philosophie ? \\[0pt]
L'art est-il un monde ? \\[0pt]
L'art est-il un moyen de connaître ? \\[0pt]
L'art est-il un refuge ? \\[0pt]
L'art est par-delà beauté et laideur ? \\[0pt]
L'art éveille-t-il des émotions spécifiques ? \\[0pt]
L'art : expérience, exercice ou habitude ? \\[0pt]
L'art exprime-t-il ce que nous ne saurions dire ? \\[0pt]
L'art fait-il penser ? \\[0pt]
L'art imite-t-il la nature ? \\[0pt]
L'artiste a-t-il besoin de modèle ? \\[0pt]
L'artiste a-t-il besoin d'une idée de l'art ? \\[0pt]
L'artiste a-t-il besoin d'un public ? \\[0pt]
L'artiste a-t-il toujours raison ? \\[0pt]
L'artiste a-t-il une méthode ? \\[0pt]
L'artiste doit-il être de son temps ? \\[0pt]
L'artiste doit-il être original ? \\[0pt]
L'artiste doit-il se donner des modèles ? \\[0pt]
L'artiste doit-il se soucier du goût du public ? \\[0pt]
L'artiste est-il le mieux placé pour comprendre son œuvre ? \\[0pt]
L'artiste est-il le seul travailleur libre ? \\[0pt]
L'artiste est-il le technicien du beau ? \\[0pt]
L'artiste est-il maître de son œuvre ? \\[0pt]
L'artiste est-il souverain ? \\[0pt]
L'artiste est-il un créateur ? \\[0pt]
L'artiste est-il un métaphysicien ? \\[0pt]
L'artiste est-il un travailleur ? \\[0pt]
L'artiste exprime-t-il quelque chose ? \\[0pt]
L'artiste peut-il se passer d'un maître ? \\[0pt]
L'artiste recherche-t-il le beau ? \\[0pt]
L'artiste sait-il ce qu'il fait ? \\[0pt]
L'artiste travaille-t-il ? \\[0pt]
L'art modifie-t-il notre rapport à la réalité ? \\[0pt]
L'art modifie-t-il notre rapport au réel ? \\[0pt]
L'art n'est-il pas toujours politique ? \\[0pt]
L'art n'est-il pas toujours religieux ? \\[0pt]
L'art n'est-il qu'apparence ? \\[0pt]
L'art n'est-il qu'un artifice ? \\[0pt]
L'art n'est-il qu'une affaire d'esthétique ? \\[0pt]
L'art n'est-il qu'une question de sentiment ? \\[0pt]
L'art n'est-il qu'un mode d'expression subjectif ? \\[0pt]
L'art n'est qu'une affaire de goût ? \\[0pt]
L'art nous détourne-t-il de la réalité ? \\[0pt]
L'art nous donne-t-il des raisons d'espérer ? \\[0pt]
L'art nous enseigne-t-il quelque chose ? \\[0pt]
L'art nous fait-il mieux percevoir le réel ? \\[0pt]
L'art nous libère-t-il de l'insignifiance ? \\[0pt]
L'art nous mène-t-il au vrai ? \\[0pt]
L'art nous permet-il de lutter contre l'irréversibilité ? \\[0pt]
L'art nous ramène-t-il à la réalité ? \\[0pt]
L'art nous réconcilie-t-il avec le monde ? \\[0pt]
L'art parachève-t-il la nature ? \\[0pt]
L'art participe-t-il à la vie politique ? \\[0pt]
L'art pense-t-il ? \\[0pt]
L'art permet-il un accès au divin ? \\[0pt]
L'art peut-il changer le monde ? \\[0pt]
L'art peut-il contribuer à éduquer les hommes ? \\[0pt]
L'art peut-il encore imiter la nature ? \\[0pt]
L'art peut-il être abstrait ? \\[0pt]
L'art peut-il être brut ? \\[0pt]
L'art peut-il être conceptuel ? \\[0pt]
L'art peut-il être enseigné ? \\[0pt]
L'art peut-il être populaire ? \\[0pt]
L'art peut-il être révolutionnaire ? \\[0pt]
L'art peut-il être sans œuvre ? \\[0pt]
L'art peut-il être utile ? \\[0pt]
L'art peut-il finir ? \\[0pt]
L'art peut-il mourir ? \\[0pt]
L'art peut-il ne pas être sacré ? \\[0pt]
L'art peut-il n'être aucunement mimétique ? \\[0pt]
L'art peut-il n'être pas conceptuel ? \\[0pt]
L'art peut-il nous rendre meilleurs ? \\[0pt]
L'art peut-il prétendre à la vérité ? \\[0pt]
L'art peut-il quelque chose contre la morale ? \\[0pt]
L'art peut-il quelque chose pour la morale ? \\[0pt]
L'art peut-il rendre le mouvement ? \\[0pt]
L'art peut-il s'affranchir des lois ? \\[0pt]
L'art peut-il sauver le monde ? \\[0pt]
L'art peut-il s'enseigner ? \\[0pt]
L'art peut-il se passer de la beauté ? \\[0pt]
L'art peut-il se passer de règles ? \\[0pt]
L'art peut-il se passer des critères du beau et du laid ? \\[0pt]
L'art peut-il se passer d'idéal ? \\[0pt]
L'art peut-il se passer d'œuvres ? \\[0pt]
L'art peut-il tenir lieu de métaphysique ? \\[0pt]
L'art produit-il nécessairement des œuvres ? \\[0pt]
L'art progresse-t-il ? \\[0pt]
L'art prolonge-t-il la nature ? \\[0pt]
L'art rend-il heureux ? \\[0pt]
L'art rend-il les hommes meilleurs ? \\[0pt]
L'art s'adresse-t-il à la sensibilité ? \\[0pt]
L'art s'adresse-t-il à tous ? \\[0pt]
L'art sait-il montrer ce que le langage ne peut pas dire ? \\[0pt]
L'art s'apparente-t-il à la philosophie ? \\[0pt]
L'art s'apprend-il ? \\[0pt]
L'art : une arithmétique sensible ? \\[0pt]
L'art vise-t-il le beau ? \\[0pt]
L'art vise-t-il nécessairement le beau ? \\[0pt]
La sagesse rend-elle heureux ? \\[0pt]
La santé est-elle un devoir ? \\[0pt]
La santé est-elle un droit ou un devoir ? \\[0pt]
La santé n'est-elle que l'absence de maladie ? \\[0pt]
L'ascétisme est-il une vertu ? \\[0pt]
La science admet-elle des degrés de croyance ? \\[0pt]
La science a-t-elle besoin d'imagination ? \\[0pt]
La science a-t-elle besoin d'un critère de démarcation entre science et non science ? \\[0pt]
La science a-t-elle besoin d'une méthode ? \\[0pt]
La science a-t-elle besoin du principe de causalité ? \\[0pt]
La science a-t-elle des limites ? \\[0pt]
La science a-t-elle le monopole de la raison ? \\[0pt]
La science a-t-elle le monopole de la vérité ? \\[0pt]
La science a-t-elle pour fin de prévoir ? \\[0pt]
La science a-t-elle réponse à tout ? \\[0pt]
La science a-t-elle toujours raison ? \\[0pt]
La science a-t-elle une histoire ? \\[0pt]
La science commence-t-elle avec la perception ? \\[0pt]
La science connaît-elle ses limites ? \\[0pt]
La science découvre-t-elle ou construit-elle son objet ? \\[0pt]
La science dépend-elle nécessairement de l'expérience ? \\[0pt]
La science désenchante-t-elle le monde ? \\[0pt]
La science dévoile-t-elle le réel ? \\[0pt]
La science doit-elle nous fournir des représentations du monde ? \\[0pt]
La science doit-elle se fonder sur une idée de la nature ? \\[0pt]
La science doit-elle se passer de l'idée de finalité ? \\[0pt]
La science du vivant peut-elle se passer de l'idée de finalité ? \\[0pt]
La science est-elle austère ? \\[0pt]
La science est-elle indépendante de toute métaphysique ? \\[0pt]
La science est-elle inhumaine ? \\[0pt]
La science est-elle le lieu de la vérité ? \\[0pt]
La science est-elle une affaire politique ? \\[0pt]
La science est-elle une connaissance du réel ? \\[0pt]
La science est-elle une langue bien faite ? \\[0pt]
La science est-elle un jeu ? \\[0pt]
La science exclut-elle l'imagination ? \\[0pt]
La science explique-t-elle le réel ? \\[0pt]
La science historique peut-elle être prédictive ? \\[0pt]
La science n'est-elle qu'une activité théorique ? \\[0pt]
La science n'est-elle qu'une fiction ? \\[0pt]
La science nous éloigne-t-elle de la religion ? \\[0pt]
La science nous éloigne-t-elle des choses ? \\[0pt]
La science nous indique-t-elle ce que nous devons faire ? \\[0pt]
La science pense-t-elle ? \\[0pt]
La science permet-elle de comprendre le monde ? \\[0pt]
La science permet-elle de mieux comprendre la religion ? \\[0pt]
La science permet-elle d'expliquer toute la réalité ? \\[0pt]
La science peut-elle errer ? \\[0pt]
La science peut-elle être une métaphysique ? \\[0pt]
La science peut-elle guider notre conduite ? \\[0pt]
La science peut-elle lutter contre les préjugés ? \\[0pt]
La science peut-elle produire des croyances ? \\[0pt]
La science peut-elle se passer de fondement ? \\[0pt]
La science peut-elle se passer de l'idée de finalité ? \\[0pt]
La science peut-elle se passer de métaphysique ? \\[0pt]
La science peut-elle se passer de méthode ? \\[0pt]
La science peut-elle se passer d'hypothèses ? \\[0pt]
La science peut-elle se passer d'institutions ? \\[0pt]
La science peut-elle tout expliquer ? \\[0pt]
La science porte-elle au scepticisme ? \\[0pt]
La science procède-t-elle par rectification ? \\[0pt]
La science rend-elle la religion caduque ? \\[0pt]
La science se limite-t-elle à constater les faits ? \\[0pt]
La science s'oppose-t-elle à la religion ? \\[0pt]
La sensation est-elle une connaissance ? \\[0pt]
La sensibilité peut-elle délivrer une vérité ? \\[0pt]
La sensibilité se cultive-t-elle ? \\[0pt]
La servitude peut-elle être volontaire ? \\[0pt]
La société doit-elle reconnaître les désirs individuels ? \\[0pt]
La société est-elle concevable sans le travail ? \\[0pt]
La société est-elle un grand marché ? \\[0pt]
La société est-elle un organisme ? \\[0pt]
La société existe-t-elle ? \\[0pt]
La société fait-elle l'homme ? \\[0pt]
La société peut-elle être l'objet d'une science ? \\[0pt]
La société peut-elle être objet de connaissance ? \\[0pt]
La société peut-elle se passer de l'État ? \\[0pt]
La société précède-t-elle l'individu ? \\[0pt]
La société repose-t-elle sur l'altruisme ? \\[0pt]
La sociologie de l'art nous permet-elle de comprendre l'art ? \\[0pt]
La sociologie est-elle déterministe ? \\[0pt]
La sociologie est-elle une physique sociale ? \\[0pt]
La sociologie relativise-t-elle la valeur des œuvres d'art ? \\[0pt]
La solidarité est-elle naturelle ? \\[0pt]
La solitude constitue-t-elle un obstacle à la citoyenneté ? \\[0pt]
La souffrance a-t-elle une valeur morale ? \\[0pt]
La souffrance a-t-elle un sens ? \\[0pt]
La souffrance a-t-elle un sens moral ? \\[0pt]
La souffrance d'autrui m'importe-t-elle ? \\[0pt]
La souffrance peut-elle avoir un sens ? \\[0pt]
La souffrance peut-elle être partagée ? \\[0pt]
La souffrance peut-elle être un mode de connaissance ? \\[0pt]
La souveraineté est-elle indivisible ? \\[0pt]
La souveraineté peut-elle être limitée ? \\[0pt]
La souveraineté peut-elle se partager ? \\[0pt]
La sphère privée échappe-t-elle au politique ? \\[0pt]
La sympathie peut-elle tenir lieu de morale ? \\[0pt]
La sympathie peut-elle tenir lieu de moralité ? \\[0pt]
La technique accroît-elle notre liberté ? \\[0pt]
La technique a-t-elle sa place en politique ? \\[0pt]
La technique a-t-elle une finalité ? \\[0pt]
La technique a-t-elle une histoire ? \\[0pt]
La technique augmente-t-elle notre puissance d'agir ? \\[0pt]
La technique change-t-elle l'homme ? \\[0pt]
La technique change-t-elle notre rapport au monde ? \\[0pt]
La technique crée-t-elle son propre monde ? \\[0pt]
La technique déshumanise-t-elle le monde ? \\[0pt]
La technique détermine-t-elle les rapports sociaux ? \\[0pt]
La technique doit-elle nous libérer du travail ? \\[0pt]
La technique doit-elle permettre de dépasser les limites de l'humain ? \\[0pt]
La technique donne-t-elle une illusion de pouvoir ? \\[0pt]
La technique est-elle civilisatrice ? \\[0pt]
La technique est-elle contre nature ? \\[0pt]
La technique est-elle dangereuse ? \\[0pt]
La technique est-elle l'application de la science ? \\[0pt]
La technique est-elle le propre de l'homme ? \\[0pt]
La technique est-elle libératrice ? \\[0pt]
La technique est-elle moralement neutre ? \\[0pt]
La technique est-elle neutre ? \\[0pt]
La technique est-elle une forme de savoir ? \\[0pt]
La technique est-elle un savoir ? \\[0pt]
La technique facilite-t-elle la vie ? \\[0pt]
La technique fait-elle des miracles ? \\[0pt]
La technique fait-elle violence à la nature ? \\[0pt]
La technique imite-t-elle la nature ? \\[0pt]
La technique invente-t-elle des formes ? \\[0pt]
La technique libère-t-elle les hommes ? \\[0pt]
La technique met-elle le monde à notre disposition ? \\[0pt]
La technique ne fait-elle qu'appliquer la science ? \\[0pt]
La technique ne pose-t-elle que des problèmes techniques ? \\[0pt]
La technique n'est-elle pour l'homme qu'un moyen ? \\[0pt]
La technique n'est-elle qu'une application de la science ? \\[0pt]
La technique n'est-elle qu'une ruse ? \\[0pt]
La technique n'est-elle qu'un moyen ? \\[0pt]
La technique n'est-elle qu'un moyen de domination ? \\[0pt]
La technique n'est-elle qu'un outil au service de l'homme ? \\[0pt]
La technique n'est-elle qu'un prolongement de nos organes ? \\[0pt]
La technique n'est-elle qu'un savoir-faire ? \\[0pt]
La technique n'existe-elle que pour satisfaire des besoins ? \\[0pt]
La technique nous délivre-t-elle d'un rapport irrationnel au monde ? \\[0pt]
La technique nous éloigne-t-elle de la nature ? \\[0pt]
La technique nous éloigne-t-elle de la réalité ? \\[0pt]
La technique nous libère-t-elle ? \\[0pt]
La technique nous libère-t-elle du travail ? \\[0pt]
La technique nous oppose-t-elle à la nature ? \\[0pt]
La technique nous permet-elle de comprendre la nature ? \\[0pt]
La technique pense-t-elle ? \\[0pt]
La technique permet-elle de réaliser tous les désirs ? \\[0pt]
La technique peut-elle améliorer l'homme ? \\[0pt]
La technique peut-elle être tenue pour la forme moderne de la culture ? \\[0pt]
La technique peut-elle respecter la nature ? \\[0pt]
La technique peut-elle se déduire de la science ? \\[0pt]
La technique peut-elle se passer de la science ? \\[0pt]
La technique pose-t-elle plus de problèmes qu'elle n'en résout ? \\[0pt]
La technique produit-elle autre chose que condition de la pensée ? \\[0pt]
La technique produit-elle son propre savoir ? \\[0pt]
La technique provoque-t-elle inévitablement des catastrophes ? \\[0pt]
La technique repose-t-elle sur le génie du technicien ? \\[0pt]
La technique sert-elle nos désirs ? \\[0pt]
La technique s'oppose-t-elle à la nature ? \\[0pt]
La technologie modifie-t-elle les rapports sociaux ? \\[0pt]
La terre est-elle un bien commun ? \\[0pt]
L'athée peut-il être vertueux ? \\[0pt]
L'athéisme condamne-t-il l'existence à l'absurdité ? \\[0pt]
L'athéisme est-il une croyance ? \\[0pt]
La théologie est-elle une science ? \\[0pt]
La théologie peut-elle être rationnelle ? \\[0pt]
La théorie nous éloigne-t-elle de la réalité ? \\[0pt]
La théorie peut-elle nous égarer ? \\[0pt]
La tolérance a-t-elle des limites ? \\[0pt]
La tolérance est-elle un concept politique ? \\[0pt]
La tolérance est-elle une vertu ? \\[0pt]
La tolérance peut-elle constituer un problème pour la démocratie ? \\[0pt]
La transparence est-elle un idéal démocratique ? \\[0pt]
La tristesse est-elle mauvaise ? \\[0pt]
L'attention caractérise-t-elle la conscience ? \\[0pt]
L'autre est-il le fondement de la conscience morale ? \\[0pt]
La valeur d'une action se mesure-t-elle à sa réussite ? \\[0pt]
La valeur d'une action se mesure-t-elle à ses conséquences ? \\[0pt]
La valeur d'une théorie scientifique se mesure-t-elle à son efficacité ? \\[0pt]
La valeur morale d'une action se juge-t-elle à ses conséquences ? \\[0pt]
La vanité est-elle toujours sans objet ? \\[0pt]
L'avenir a-t-il une réalité ? \\[0pt]
L'avenir est-il imaginable ? \\[0pt]
L'avenir est-il incertain ? \\[0pt]
L'avenir est-il l'affaire de la pensée ? \\[0pt]
L'avenir est-il prévisible ? \\[0pt]
L'avenir est-il sans image ? \\[0pt]
L'avenir est-il une page blanche ? \\[0pt]
L'avenir existe-t-il ? \\[0pt]
L'avenir peut-il être objet de connaissance ? \\[0pt]
La vérification fait-elle la vérité ? \\[0pt]
La vérité admet-elle des degrés ? \\[0pt]
La vérité a-t-elle une histoire ? \\[0pt]
La vérité demande-t-elle du courage ? \\[0pt]
La vérité doit-elle toujours être démontrée ? \\[0pt]
La vérité donne-t-elle le droit d'être injuste ? \\[0pt]
La vérité d'une théorie dépend-elle de sa correspondance avec les faits ? \\[0pt]
La vérité échappe-t-elle au temps ? \\[0pt]
La vérité est-elle affaire de cohérence ? \\[0pt]
La vérité est-elle affaire de croyance ou de savoir ? \\[0pt]
La vérité est-elle affaire d'unanimité ? \\[0pt]
La vérité est-elle contraignante ? \\[0pt]
La vérité est-elle éternelle ? \\[0pt]
La vérité est-elle fille de son temps ? \\[0pt]
La vérité est-elle hors de notre portée ? \\[0pt]
La vérité est-elle intemporelle ? \\[0pt]
La vérité est-elle la valeur suprême ? \\[0pt]
La vérité est-elle libératrice ? \\[0pt]
La vérité est-elle morale ? \\[0pt]
La vérité est-elle objective ? \\[0pt]
La vérité est-elle préférable à l'illusion ? \\[0pt]
La vérité est-elle relative ? \\[0pt]
La vérité est-elle triste ? \\[0pt]
La vérité est-elle une ? \\[0pt]
La vérité est-elle une construction ? \\[0pt]
La vérité est-elle une idole ? \\[0pt]
La vérité est-elle une valeur ? \\[0pt]
La vérité n'est-elle qu'une erreur rectifiée ? \\[0pt]
La vérité n'est-elle qu'un idéal inaccessible ? \\[0pt]
La vérité nous appartient-elle ? \\[0pt]
La vérité nous contraint-elle ? \\[0pt]
La vérité nous rend-elle libres ? \\[0pt]
La vérité peut-elle changer avec le temps ? \\[0pt]
La vérité peut-elle être équivoque ? \\[0pt]
La vérité peut-elle être indicible ? \\[0pt]
La vérité peut-elle être relative ? \\[0pt]
La vérité peut-elle être tolérante ? \\[0pt]
La vérité peut-elle faire peur ? \\[0pt]
La vérité peut-elle laisser indifférent ? \\[0pt]
La vérité peut-elle se définir par le consensus ? \\[0pt]
La vérité peut-elle se discuter ? \\[0pt]
La vérité rend-elle heureux ? \\[0pt]
La vérité rend-elle libre ? \\[0pt]
La vérité requiert-elle du courage ? \\[0pt]
La vérité résiste-elle à sa révélation ? \\[0pt]
La vérité scientifique est-elle relative ? \\[0pt]
La vérité se communique-t-elle ? \\[0pt]
La vérité se discute-t-elle ? \\[0pt]
La vertu est-elle une forme de santé ? \\[0pt]
La vertu est-elle utile ? \\[0pt]
La vertu peut-elle être excessive ? \\[0pt]
La vertu peut-elle être purement morale ? \\[0pt]
La vertu peut-elle s'enseigner ? \\[0pt]
La vertu requiert-elle l'idée de bien ? \\[0pt]
La vertu s'enseigne-t-elle ? \\[0pt]
L'aveu diminue-t-il la faute ? \\[0pt]
La vie a-t-elle un sens ? \\[0pt]
La vie collective est-elle nécessairement frustrante ? \\[0pt]
La vie en société est-elle naturelle à l'homme ? \\[0pt]
La vie en société impose-t-elle de n'être pas soi-même ? \\[0pt]
La vie en société menace-t-elle la liberté ? \\[0pt]
La vie est-elle la valeur suprême ? \\[0pt]
La vie est-elle le bien le plus précieux ? \\[0pt]
La vie est-elle le bien suprême ? \\[0pt]
La vie est-elle l'objet des sciences de la vie ? \\[0pt]
La vie est-elle objet de science ? \\[0pt]
La vie est-elle sacrée ? \\[0pt]
La vie est-elle trop courte ? \\[0pt]
La vie est-elle une notion métaphysique ? \\[0pt]
La vie est-elle une valeur ? \\[0pt]
La vie est-elle un roman ? \\[0pt]
La vie est-elle un songe ? \\[0pt]
La vie intérieure est-elle une illusion ? \\[0pt]
La vie peut-elle être éternelle ? \\[0pt]
La vie peut-elle être objet de science ? \\[0pt]
La vie peut-elle être sans histoire ? \\[0pt]
La vie politique est-elle aliénante ? \\[0pt]
La vie sexuelle est-elle volontaire ? \\[0pt]
La vie sociale est-elle toujours conflictuelle ? \\[0pt]
La vie sociale est-elle une comédie ? \\[0pt]
La vie sociale relève-t-elle des habitudes ? \\[0pt]
La vie suffit-elle à créer des valeurs ? \\[0pt]
La violence a-t-elle des degrés ? \\[0pt]
La violence a-t-elle un sens dans l'histoire ? \\[0pt]
La violence engendre-t-elle toujours la violence ? \\[0pt]
La violence est-elle la condition du progrès ? \\[0pt]
La violence est-elle le fondement du droit ? \\[0pt]
La violence est-elle le prix du progrès ? \\[0pt]
La violence est-elle toujours destructrice ? \\[0pt]
La violence est-elle toujours injustifiable ? \\[0pt]
La violence peut-elle avoir raison ? \\[0pt]
La violence peut-elle être gratuite ? \\[0pt]
La violence peut-elle être morale ? \\[0pt]
La vision peut-elle être le modèle de toute connaissance ? \\[0pt]
La volonté constitue-t-elle le principe de la politique ? \\[0pt]
La volonté est-elle la somme des inclinations ? \\[0pt]
La volonté est-elle libre ? \\[0pt]
La volonté générale est-elle la volonté de tous ? \\[0pt]
La volonté peut-elle être collective ? \\[0pt]
La volonté peut-elle être générale ? \\[0pt]
La volonté peut-elle être indéterminée ? \\[0pt]
La volonté peut-elle être libre ? \\[0pt]
La volonté peut-elle nous manquer ? \\[0pt]
La volonté politique peut-elle être commune ? \\[0pt]
La vraie morale se moque-t-elle de la morale ? \\[0pt]
Le beau a-t-il une histoire ? \\[0pt]
Le beau est-il aimable ? \\[0pt]
Le beau est-il dicible ? \\[0pt]
Le beau est-il l'objet de l'esthétique ? \\[0pt]
Le beau est-il toujours moral ? \\[0pt]
Le beau est-il une valeur commune ? \\[0pt]
Le beau est-il universel ? \\[0pt]
Le beau et le bien sont-ils, au fond, identiques ? \\[0pt]
Le beau existe-t-il indépendamment du bien ? \\[0pt]
Le beau n'est-il qu'une idée ? \\[0pt]
Le beau peut-il être bizarre ? \\[0pt]
Le beau peut-il être effrayant ? \\[0pt]
Le besoin de métaphysique est-il un besoin de connaissance ? \\[0pt]
Le bien commun est-il une illusion ? \\[0pt]
Le bien détermine-t-il la volonté ? \\[0pt]
Le bien, est-ce l'utile ? \\[0pt]
Le bien est-il conformité à la nature ? \\[0pt]
Le bien est-il l'utile ? \\[0pt]
Le bien est-il relatif ? \\[0pt]
Le bien n'est-il réalisable que comme moindre mal ? \\[0pt]
Le bien suppose-t-il la transcendance ? \\[0pt]
Le bonheur a-t-il nécessairement un objet ? \\[0pt]
Le bonheur comporte-t-il des degrés ? \\[0pt]
Le bonheur de la passion est-il sans lendemain ? \\[0pt]
Le bonheur des citoyens est-il un idéal politique ? \\[0pt]
Le bonheur est-il affaire de calcul ? \\[0pt]
Le bonheur est-il affaire de hasard ou de nécessité ? \\[0pt]
Le bonheur est-il affaire de vertu ? \\[0pt]
Le bonheur est-il affaire de volonté ? \\[0pt]
Le bonheur est-il affaire privée ? \\[0pt]
Le bonheur est-il au nombre de nos devoirs ? \\[0pt]
Le bonheur est-il dans l'inconscience ? \\[0pt]
Le bonheur est-il l'absence de maux ? \\[0pt]
Le bonheur est-il l'affaire du politique ? \\[0pt]
Le bonheur est-il la fin de la vie ? \\[0pt]
Le bonheur est-il le bien suprême ? \\[0pt]
Le bonheur est-il le but de la politique ? \\[0pt]
Le bonheur est-il le prix de la vertu ? \\[0pt]
Le bonheur est-il nécessairement lié au plaisir ? \\[0pt]
Le bonheur est-il un accident ? \\[0pt]
Le bonheur est-il un but politique ? \\[0pt]
Le bonheur est-il un droit ? \\[0pt]
Le bonheur est-il une affaire de circonstances ? \\[0pt]
Le bonheur est-il une affaire privée ? \\[0pt]
Le bonheur est-il une fin morale ? \\[0pt]
Le bonheur est-il une fin politique ? \\[0pt]
Le bonheur est-il une récompense ? \\[0pt]
Le bonheur est-il une valeur morale ? \\[0pt]
Le bonheur est-il un idéal ? \\[0pt]
Le bonheur est-il un principe politique ? \\[0pt]
Le bonheur n'est-il qu'une idée ? \\[0pt]
Le bonheur n'est-il qu'un idéal ? \\[0pt]
Le bonheur peut-il être collectif ? \\[0pt]
Le bonheur peut-il être le but de la politique ? \\[0pt]
Le bonheur peut-il être un droit ? \\[0pt]
Le bonheur peut-il être un objectif politique ? \\[0pt]
Le bonheur s'apprend-il ? \\[0pt]
Le bonheur se calcule-t-il ? \\[0pt]
Le bonheur se fonde-t-il sur un savoir ? \\[0pt]
Le bonheur se mérite-t-il ? \\[0pt]
Le cerveau pense-t-il ? \\[0pt]
L'échange constitue-t-il un lien social ? \\[0pt]
L'échange est-il un facteur de paix ? \\[0pt]
L'échange n'a-t-il de fondement qu'économique ? \\[0pt]
L'échange ne porte-t-il que sur les choses ? \\[0pt]
L'échange peut-il être désintéressé ? \\[0pt]
Le châtiment peut-il ne rien devoir au désir de vengeance ? \\[0pt]
Le châtiment rend-il meilleur ? \\[0pt]
Le choix peut-il être éclairé ? \\[0pt]
Le cinéma, art de la représentation ? \\[0pt]
Le cinéma est-il un art ? \\[0pt]
Le cinéma est-il un art comme les autres ? \\[0pt]
Le cinéma est-il un art narratif ? \\[0pt]
Le cinéma est-il un art ou une industrie ? \\[0pt]
Le cinéma est-il un art populaire ? \\[0pt]
Le citoyen a-t-il perdu toute naturalité ? \\[0pt]
Le citoyen peut-il être à la fois libre et soumis à l'État ? \\[0pt]
L'écologie est-elle aussi une science humaine ? \\[0pt]
L'écologie est-elle une science de la nature ou une science sociale ? \\[0pt]
L'écologie est-elle un problème politique ? \\[0pt]
L'écologie, une science humaine ? \\[0pt]
Le combat contre l'injustice a-t-il une source morale ? \\[0pt]
Le commerce adoucit-il les mœurs ? \\[0pt]
Le commerce est-il pacificateur ? \\[0pt]
Le commerce est-il un facteur de paix ? \\[0pt]
Le commerce peut-il être équitable ? \\[0pt]
Le commerce unit-il les hommes ? \\[0pt]
Le concept de nature est-il un concept scientifique ? \\[0pt]
Le concept de race est-il anthropologique ? \\[0pt]
Le concept de société sans écriture est-il pertinent ? \\[0pt]
Le concept d'inconscient est-il nécessaire en sciences humaines ? \\[0pt]
Le conflit entre la science et la religion est-il inévitable ? \\[0pt]
Le conflit entre l'individu et la société est-il irréductible ? \\[0pt]
Le conflit est-il au fondement de la vie sociale ? \\[0pt]
Le conflit est-il constitutif de la politique ? \\[0pt]
Le conflit est-il la raison d'être de la politique ? \\[0pt]
Le conflit est-il une maladie sociale ? \\[0pt]
L'économie a-t-elle des lois ? \\[0pt]
L'économie a-t-elle ses propres lois ? \\[0pt]
L'économie est-elle politique ? \\[0pt]
L'économie est-elle une science ? \\[0pt]
L'économie est-elle une science humaine ? \\[0pt]
L'économie et le politique obéissent-ils à une même rationalité ? \\[0pt]
L'économie fait-elle partie des sciences humaines ? \\[0pt]
Le consensus peut-il être critère de vérité ? \\[0pt]
Le consensus peut-il faire le vrai ? \\[0pt]
Le contradictoire peut-il exister ? \\[0pt]
Le contrat est-il au fondement de la politique ? \\[0pt]
Le contrat peut-il remplacer la loi ? \\[0pt]
Le corps dit-il quelque chose ? \\[0pt]
Le corps est-il le reflet de l'âme ? \\[0pt]
Le corps est-il négociable ? \\[0pt]
Le corps est-il porteur de valeurs ? \\[0pt]
Le corps est-il respectable ? \\[0pt]
Le corps est-il une entrave pour l'esprit ? \\[0pt]
Le corps est-il un enjeu des sciences humaines ? \\[0pt]
Le corps est-il un tout ? \\[0pt]
Le corps humain est-il naturel ? \\[0pt]
Le corps impose-t-il des perspectives ? \\[0pt]
Le corps n'est-il que matière ? \\[0pt]
Le corps n'est-il qu'un mécanisme ? \\[0pt]
Le corps obéit-il à l'esprit ? \\[0pt]
Le corps pense-t-il ? \\[0pt]
Le corps peut-il être objet d'art ? \\[0pt]
Le cosmopolitisme peut-il devenir réalité ? \\[0pt]
Le cosmopolitisme peut-il être réaliste ? \\[0pt]
Le courage est-il une vertu ? \\[0pt]
Le crime suppose-t-il la loi ? \\[0pt]
L'écriture est-elle une technique parmi d'autres ? \\[0pt]
L'écriture ne sert-elle qu'à consigner la pensée ? \\[0pt]
L'écriture peut-elle porter secours à la pensée ? \\[0pt]
Le désespoir est-il une faute morale ? \\[0pt]
Le désespoir peut-il passer pour une sagesse ? \\[0pt]
Le désir a-t-il un objet ? \\[0pt]
Le désir de savoir est-il comblé par la science ? \\[0pt]
Le désir de savoir est-il naturel ? \\[0pt]
Le désir de vérité peut-il être interprété comme un désir de pouvoir ? \\[0pt]
Le désir du bonheur est-il universel ? \\[0pt]
Le désir est-il aveugle ? \\[0pt]
Le désir est-il ce qui nous fait vivre ? \\[0pt]
Le désir est-il désir de l'autre ? \\[0pt]
Le désir est-il le signe d'un manque ? \\[0pt]
Le désir est-il l'essence de l'homme ? \\[0pt]
Le désir est-il nécessairement l'expression d'un manque ? \\[0pt]
Le désir est-il par nature illimité ? \\[0pt]
Le désir est-il sans limite ? \\[0pt]
Le désir n'est-il pas qu'inquiétude ? \\[0pt]
Le désir n'est-il que l'épreuve d'un manque ? \\[0pt]
Le désir n'est-il que manque ? \\[0pt]
Le désir n'est-il qu'inquiétude ? \\[0pt]
Le désir peut-il atteindre son objet ? \\[0pt]
Le désir peut-il être désintéressé ? \\[0pt]
Le désir peut-il ne pas avoir d'objet ? \\[0pt]
Le désir peut-il nous rendre libre ? \\[0pt]
Le désir peut-il se satisfaire de la réalité ? \\[0pt]
Le désordre est-il étonnant ? \\[0pt]
Le désordre est-il un mal ? \\[0pt]
Le despote peut-il être éclairé ? \\[0pt]
Le despotisme peut-il être éclairé ? \\[0pt]
Le développement de la technique est-il toujours facteur de progrès ? \\[0pt]
Le développement des techniques fait-il reculer la croyance ? \\[0pt]
Le devenir est-il pensable ? \\[0pt]
Le devoir est-il l'expression de la contrainte sociale ? \\[0pt]
Le devoir nous apparaît-il d'autant mieux qu'on ne l'a pas accompli ? \\[0pt]
Le devoir rend-il libre ? \\[0pt]
Le devoir s'apprend-il ? \\[0pt]
Le devoir se présente-t-il avec la force de l'évidence ? \\[0pt]
Le devoir supprime-t-il la liberté ? \\[0pt]
Le dialogue conduit-il à la vérité ? \\[0pt]
Le dialogue suffit-il à rompre la solitude ? \\[0pt]
Le divertissement est-il vain ? \\[0pt]
Le don est-il toujours généreux ? \\[0pt]
Le don est-il une modalité de l'échange ? \\[0pt]
Le don exclut-il le calcul ? \\[0pt]
Le doute est-il le principe de la méthode scientifique ? \\[0pt]
Le doute est-il une faiblesse de la pensée ? \\[0pt]
Le doute peut-il être méthodique ? \\[0pt]
Le droit à la différence met-il en péril l'égalité des droits ? \\[0pt]
Le droit doit-il être indépendant de la morale ? \\[0pt]
Le droit doit-il être le seul régulateur de la vie sociale ? \\[0pt]
Le droit doit-il être moral ? \\[0pt]
Le droit est-il facteur de paix ? \\[0pt]
Le droit est-il le fondement de l'État ? \\[0pt]
Le droit est-il raison du plus fort ? \\[0pt]
Le droit est-il une science ? \\[0pt]
Le droit est-il une science humaine ? \\[0pt]
Le droit exclut-il la force ? \\[0pt]
Le droit ne peut-il se fonder sur des faits ? \\[0pt]
Le droit n'est-il qu'une justice par défaut ? \\[0pt]
Le droit n'est-il qu'un ensemble de conventions ? \\[0pt]
Le droit peut-il échapper à l'histoire ? \\[0pt]
Le droit peut-il être flexible ? \\[0pt]
Le droit peut-il être naturel ? \\[0pt]
Le droit peut-il se fonder sur la force ? \\[0pt]
Le droit peut-il se passer de la morale ? \\[0pt]
Le droit sert-il à établir l'ordre ou la justice ? \\[0pt]
L'éducation du goût est-elle la condition de l'expérience esthétique ? \\[0pt]
L'éducation est-elle affaire de politique ? \\[0pt]
L'éducation est-elle par nature conservatrice ? \\[0pt]
L'éducation peut-elle être sentimentale ? \\[0pt]
L'éducation relève-t-elle du politique ? \\[0pt]
Le fait de vivre constitue-t-il un bien en soi ? \\[0pt]
Le fait de vivre est-il un bien en soi ? \\[0pt]
Le fait social est-il une chose ? \\[0pt]
L'efficacité est-elle une vertu ? \\[0pt]
L'efficacité technique est-elle la norme de toute réussite ? \\[0pt]
Le futur est-il contingent ? \\[0pt]
Le futur est-il nécessairement contingent ? \\[0pt]
Le futur existe-t-il ? \\[0pt]
Le futur nous appartient-il ? \\[0pt]
L'égalité des hommes est-elle un fait ou une idée ? \\[0pt]
L'égalité des hommes et des femmes est-elle une question politique ? \\[0pt]
L'égalité est-elle contre nature ? \\[0pt]
L'égalité est-elle l'ennemie de la liberté ? \\[0pt]
L'égalité est-elle souhaitable ? \\[0pt]
L'égalité est-elle toujours juste ? \\[0pt]
L'égalité est-elle une condition de la liberté ? \\[0pt]
L'égalité peut-elle être une menace pour la liberté ? \\[0pt]
Le génie est-il la marque de l'excellence artistique ? \\[0pt]
Le génie n'a-t-il sa place que dans les arts ? \\[0pt]
Le genre humain : unité ou pluralité ? \\[0pt]
Le geste technique exprime t-il une liberté sans fin ? \\[0pt]
Le goût : certitude ou conviction ? \\[0pt]
Le goût est-il affaire d'éducation ? \\[0pt]
Le goût est-il nécessairement subjectif ? \\[0pt]
Le goût est-il une faculté ? \\[0pt]
Le goût est-il une question de classe ? \\[0pt]
Le goût est-il une vertu sociale ? \\[0pt]
Le goût s'éduque-t-il ? \\[0pt]
Le goût se forme-t-il ? \\[0pt]
Le gouvernement par le peuple est-il nécessairement pour le peuple ? \\[0pt]
Le grand art est-il de plaire ? \\[0pt]
Le hasard est-il injuste ? \\[0pt]
Le hasard existe-t-il ? \\[0pt]
Le hasard fait-il bien les choses ? \\[0pt]
Le hasard gouverne-t-il le monde ? \\[0pt]
Le hasard n'est-il que la mesure de notre ignorance ? \\[0pt]
Le hasard n'est-il que le nom de notre ignorance ? \\[0pt]
Le hasard peut-il être un concept explicatif ?La morale doit-elle s'adapter à la réalité ? \\[0pt]
Le jugement artistique se fait-il sans concept ? \\[0pt]
Le jugement critique peut-il s'exercer sans culture ? \\[0pt]
Le jugement de goût est-il désintéressé ? \\[0pt]
Le jugement de goût est-il universel ? \\[0pt]
Le jugement de valeur est-il indifférent à la vérité ? \\[0pt]
Le langage a-t-il une prétention à la vérité ? \\[0pt]
Le langage est-il assimilable à un outil ? \\[0pt]
Le langage est-il d'essence poétique ? \\[0pt]
Le langage est-il l'auxiliaire de la pensée ? \\[0pt]
Le langage est-il le lieu de la vérité ? \\[0pt]
Le langage est-il le miroir du monde ? \\[0pt]
Le langage est-il le moyen d'expression le plus efficace ? \\[0pt]
Le langage est-il le propre de l'homme ? \\[0pt]
Le langage est-il l'instrument de la pensée ? \\[0pt]
Le langage est-il logique ? \\[0pt]
Le langage est-il naturel ? \\[0pt]
Le langage est-il une prise de possession des choses ? \\[0pt]
Le langage est-il un instrument ? \\[0pt]
Le langage est-il un instrument de connaissance ? \\[0pt]
Le langage est-il un obstacle pour la pensée ? \\[0pt]
Le langage fait-il obstacle à la connaissance ? \\[0pt]
Le langage masque-t-il la pensée ? \\[0pt]
Le langage ne sert-il qu'à communiquer ? \\[0pt]
Le langage n'est-il qu'un instrument de communication ? \\[0pt]
Le langage n'est-il qu'un outil ? \\[0pt]
Le langage nous donne-t-il une connaissance des choses ? \\[0pt]
Le langage nous éloigne-t-il de la réalité ? \\[0pt]
Le langage permet-il seulement de communiquer ? \\[0pt]
Le langage peut-il être un obstacle à la recherche de la vérité ? \\[0pt]
Le langage rapproche-t-il ou sépare-t-il les hommes ? \\[0pt]
Le langage rend-il l'homme plus puissant ? \\[0pt]
Le langage reste-t-il toujours à la surface des choses ? \\[0pt]
Le langage traduit-il la pensée ? \\[0pt]
Le langage trahit-il la pensée ? \\[0pt]
Le langage usuel est-il un obstacle à la connaissance ? \\[0pt]
Le langage voile-t-il les choses ? \\[0pt]
Le libre cours de l'imagination est-il libérateur ? \\[0pt]
Le lien social peut-il être compassionnel ? \\[0pt]
Le lien social relève-t-il de la politique ? \\[0pt]
Le logique est-elle un art de penser ? \\[0pt]
Le loisir caractérise-t-il l'homme libre ? \\[0pt]
Le mal apparaît-il toujours ? \\[0pt]
Le mal a-t-il des raisons ? \\[0pt]
Le mal constitue-t-il une objection à l'existence de Dieu ? \\[0pt]
Le mal est-il une erreur ? \\[0pt]
Le mal est-il une objection à l'existence de Dieu ? \\[0pt]
Le mal est-il une réalité objective ? \\[0pt]
Le mal existe-t-il ? \\[0pt]
Le malheur donne-t-il le droit d'être injuste ? \\[0pt]
Le malheur est-il injuste ? \\[0pt]
Le mal peut-il être absolu ? \\[0pt]
Le mal peut-il être involontaire ? \\[0pt]
Le mariage est-il un contrat ? \\[0pt]
L'embryon humain est-il une personne ? \\[0pt]
L'embryon humain peut-il être l'objet d'expérimentation ? \\[0pt]
Le méchant est-il malheureux ? \\[0pt]
Le méchant peut-il être heureux ? \\[0pt]
Le meilleur est-il l'ennemi du bien ? \\[0pt]
Le meilleur gouvernement est-il le gouvernement des meilleurs ? \\[0pt]
Le mensonge de l'art ? \\[0pt]
Le mensonge est-il la plus grande transgression ? \\[0pt]
Le mensonge est-il une forme d'indifférence à la vérité ? \\[0pt]
Le mensonge peut-il être au service de la vérité ? \\[0pt]
Le mépris peut-il être justifié ? \\[0pt]
Le mérite est-il le critère de la vertu ? \\[0pt]
Le métaphysicien est-il un physicien à sa façon ? \\[0pt]
Le mieux est-il l'ennemi du bien ? \\[0pt]
Le modèle du jeu dans les sciences humaines ? \\[0pt]
Le moi est-il haïssable ? \\[0pt]
Le moi est-il insaisissable ? \\[0pt]
Le moi est-il objet de connaissance ? \\[0pt]
Le moi est-il une collection d'idées ? \\[0pt]
Le moi est-il une fiction ? \\[0pt]
Le moi est-il une illusion ? \\[0pt]
Le moi n'est-il qu'une fiction ? \\[0pt]
Le moi n'est-il qu'une idée ? \\[0pt]
Le moi peut-il être l'objet d'un savoir ? \\[0pt]
Le moi reste-t-il identique à lui-même au cours du temps ? \\[0pt]
Le monde a-t-il besoin de moi ? \\[0pt]
Le monde a-t-il une histoire ? \\[0pt]
Le monde aurait-il un sens si l'homme n'existait pas ? \\[0pt]
Le monde de l'animal nous est-il étranger ? \\[0pt]
Le monde est-il autre chose que ce que j'en dis ? \\[0pt]
Le monde est-il écrit en langage mathématique ? \\[0pt]
Le monde est-il en progrès ? \\[0pt]
Le monde est-il éternel ? \\[0pt]
Le monde est-il ma représentation ? \\[0pt]
Le monde est-il notre représentation ? \\[0pt]
Le monde est-il une marchandise ? \\[0pt]
Le monde est-il un théâtre ? \\[0pt]
Le monde extérieur existe-t-il ? \\[0pt]
Le monde n'est-il que l'ensemble de ce qui existe ? \\[0pt]
Le monde n'est-il que ma représentation ? \\[0pt]
Le monde n'est-il qu'une représentation ? \\[0pt]
Le monde peut-il être nouveau ? \\[0pt]
Le monde se réduit-il à ce que nous en voyons ? \\[0pt]
Le monde se suffit-il à lui-même ? \\[0pt]
L'émotion esthétique peut-elle se communiquer ? \\[0pt]
L'émotion esthétique peut-elle se partager ? \\[0pt]
Le mot vie a-t-il plusieurs sens ? \\[0pt]
L'empathie est-elle nécessaire aux sciences sociales ? \\[0pt]
L'empathie est-elle possible ? \\[0pt]
L'empirisme exclut-il l'abstraction ? \\[0pt]
Le mythe est-il objet de science ? \\[0pt]
Le naturel a-t-il toutes les vertus ? \\[0pt]
Le néant est-il ? \\[0pt]
L'enfance est-elle ce qui doit être surmonté ? \\[0pt]
L'enfance est-elle en nous ce qui doit être abandonné ? \\[0pt]
L'enfance peut-elle servir de modèle ? \\[0pt]
L'enfer est-il véritablement pavé de bonnes intentions ? \\[0pt]
L'enquête empirique rend-elle la métaphysique inutile ? \\[0pt]
L'enseignement peut-il se passer d'exemples ? \\[0pt]
L'enthousiasme est-il moral ? \\[0pt]
L'environnement est-il un nouvel objet pour les sciences humaines ? \\[0pt]
L'environnement est-il un problème politique ? \\[0pt]
Le pardon peut-il être une obligation ? \\[0pt]
Le partage est-il une obligation morale ? \\[0pt]
Le passé a-t-il plus de réalité que l'avenir ? \\[0pt]
Le passé a-t-il une réalité ? \\[0pt]
Le passé a-t-il un intérêt ? \\[0pt]
Le passé détermine-t-il notre présent ? \\[0pt]
Le passé, est-ce du passé ? \\[0pt]
Le passé est-il ce qui a disparu ? \\[0pt]
Le passé est-il indépassable ? \\[0pt]
Le passé est-il objet de science ? \\[0pt]
Le passé est-il perdu ? \\[0pt]
Le passé est-il réel ? \\[0pt]
Le passé existe-t-il ? \\[0pt]
Le passé peut-il être un objet de connaissance ? \\[0pt]
Le passionné mérite-t-il d'être plaint ? \\[0pt]
Le patriotisme est-il une vertu ? \\[0pt]
Le peuple a-t-il toujours raison ? \\[0pt]
Le peuple est-il bête ? \\[0pt]
Le peuple est-il souverain ? \\[0pt]
Le peuple est-il un acteur politique ? \\[0pt]
Le peuple peut-il se tromper ? \\[0pt]
Le peuple possède-t-il une volonté ? \\[0pt]
L'éphémère a-t-il une valeur ? \\[0pt]
Le philosophe a-t-il besoin de l'histoire ? \\[0pt]
Le philosophe a-t-il des leçons à donner au politique ? \\[0pt]
Le philosophe est-il le vrai politique ? \\[0pt]
Le philosophe s'écarte-t-il du réel ? \\[0pt]
Le philosophe se détourne-t-il du réel ? \\[0pt]
L'épistémologie est-elle une logique de la science ? \\[0pt]
Le plaisir artistique est-il affaire de jugement ? \\[0pt]
Le plaisir a-t-il un rôle à jouer dans la morale ? \\[0pt]
Le plaisir esthétique est-il affaire de jugement ? \\[0pt]
Le plaisir esthétique est-il un plaisir ? \\[0pt]
Le plaisir esthétique peut-il se communiquer ? \\[0pt]
Le plaisir esthétique peut-il se partager ? \\[0pt]
Le plaisir esthétique suppose-t-il une culture ? \\[0pt]
Le plaisir esthétique suppose-t-il une culture esthétique ? \\[0pt]
Le plaisir est-il immoral ? \\[0pt]
Le plaisir est-il la fin du désir ? \\[0pt]
Le plaisir est-il tout le bonheur ? \\[0pt]
Le plaisir est-il un bien ? \\[0pt]
Le plaisir peut-il être immoral ? \\[0pt]
Le plaisir peut-il être partagé ? \\[0pt]
Le plaisir se mérite-t-il ? \\[0pt]
Le plaisir suffit-il au bonheur ? \\[0pt]
Le poète réinvente-t-il la langue ? \\[0pt]
Le politique a-t-il à régler les passions ? \\[0pt]
Le politique a-t-il à régler les passions humaines ? \\[0pt]
Le politique doit-il être un technicien ? \\[0pt]
Le politique doit-il s'appuyer sur la science ? \\[0pt]
Le politique doit-il se soucier des émotions ? \\[0pt]
Le politique peut-il faire abstraction de la morale ? \\[0pt]
Le possible existe-t-il ? \\[0pt]
Le pouvoir corrompt-il ? \\[0pt]
Le pouvoir corrompt-il nécessairement ? \\[0pt]
Le pouvoir corrompt-il toujours ? \\[0pt]
Le pouvoir de l'État est-il arbitraire ? \\[0pt]
Le pouvoir est-il une forme de domination ? \\[0pt]
Le pouvoir peut-il arrêter le pouvoir ? \\[0pt]
Le pouvoir peut-il être limité ? \\[0pt]
Le pouvoir peut-il ignorer la morale ? \\[0pt]
Le pouvoir peut-il limiter le pouvoir ? \\[0pt]
Le pouvoir peut-il se déléguer ? \\[0pt]
Le pouvoir peut-il se passer de sa mise en scène ? \\[0pt]
Le pouvoir peut-il se transmettre ? \\[0pt]
Le pouvoir politique est-il nécessairement coercitif ? \\[0pt]
Le pouvoir politique peut-il échapper à l'arbitraire ? \\[0pt]
Le pouvoir politique repose-t-il sur la contrainte ? \\[0pt]
Le pouvoir politique repose-t-il sur un savoir ? \\[0pt]
Le pouvoir se partage-t-il ? \\[0pt]
Le pouvoir tend-il toujours à l'excès ? \\[0pt]
Le premier devoir de l'État est-il de se défendre ? \\[0pt]
Le profit est-il la fin de l'échange ? \\[0pt]
Le profit est-il le moteur de la production économique ? \\[0pt]
Le progrès de l'humanité se réduit-il au progrès technique ? \\[0pt]
Le progrès des sciences infirme-t-il les résultats anciens ? \\[0pt]
Le progrès est-il réversible ? \\[0pt]
Le progrès est-il un mythe ? \\[0pt]
Le progrès scientifique fait-il disparaître la superstition ? \\[0pt]
Le progrès technique a-t-il une fin ? \\[0pt]
Le progrès technique est-il source de bonheur ? \\[0pt]
Le progrès technique peut-il être aliénant ? \\[0pt]
Le projet d'une paix perpétuelle est-il insensé ? \\[0pt]
Le propre du vivant est-il de tomber malade ? \\[0pt]
Le psychisme est-il objet de connaissance ? \\[0pt]
Lequel, de l'art ou du réel, est-il une imitation de l'autre ? \\[0pt]
Le raisonnement suit-il des règles ? \\[0pt]
Le rapport de l'homme à son milieu a-t-il une dimension morale ? \\[0pt]
Le rationalisme peut-il être une passion ? \\[0pt]
Le recours à la force signifie-t-il l'échec de la justice ? \\[0pt]
Le recours au symbole est-il la marque de la finitude de la pensée ? \\[0pt]
Le réel est-il ce que l'on croit ? \\[0pt]
Le réel est-il ce que nous expérimentons ? \\[0pt]
Le réel est-il ce que nous percevons ? \\[0pt]
Le réel est-il ce qui apparaît ? \\[0pt]
Le réel est-il ce qui est perçu ? \\[0pt]
Le réel est-il ce qui résiste ? \\[0pt]
Le réel est-il inaccessible ? \\[0pt]
Le réel est-il l'objet de la science ? \\[0pt]
Le réel est-il objet d'interprétation ? \\[0pt]
Le réel est-il rationnel ? \\[0pt]
Le réel excède-t-il l'intelligible ? \\[0pt]
Le réel n'est-il qu'un ensemble de contraintes ? \\[0pt]
Le réel obéit-il à la raison ? \\[0pt]
Le réel peut-il échapper à la logique ? \\[0pt]
Le réel peut-il être beau ? \\[0pt]
Le réel peut-il être contradictoire ? \\[0pt]
Le réel résiste-t-il à la connaissance ? \\[0pt]
Le réel se donne-t-il à voir ? \\[0pt]
Le réel se limite-t-il à ce que font connaître les théories scientifiques ? \\[0pt]
Le réel se limite-t-il à ce que nous percevons ? \\[0pt]
Le réel se réduit-il à ce que l'on perçoit ? \\[0pt]
Le réel se réduit-il à l'objectivité ? \\[0pt]
Le règlement politique des conflits ? \\[0pt]
Le relativisme culturel est-il indépassable ? \\[0pt]
Le relativisme peut-il être un principe de méthode dans les sciences humaines ? \\[0pt]
Le religieux est-il inutile ? \\[0pt]
Le remords est-il stérile ? \\[0pt]
Le respect n'est-il dû qu'aux personnes ? \\[0pt]
Le retour à la nature est-il souhaitable ? \\[0pt]
Le rôle de l'État est-il de faire régner la justice ? \\[0pt]
Le rôle de l'État est-il de préserver la liberté de l'individu ? \\[0pt]
Le rôle de l'historien est-il de juger ? \\[0pt]
Le rôle des théories est-il d'expliquer ou de décrire ? \\[0pt]
Le roman peut-il être philosophique ? \\[0pt]
L'erreur est-elle humaine ? \\[0pt]
L'erreur peut-elle donner un accès à la vérité ? \\[0pt]
L'erreur peut-elle jouer un rôle dans la connaissance scientifique ? \\[0pt]
Les acteurs de l'histoire en sont-ils les auteurs ? \\[0pt]
Les affects sont-ils déraisonnables ? \\[0pt]
Les affects sont-ils des objets sociologiques ? \\[0pt]
Le sage a-t-il besoin d'autrui ? \\[0pt]
Le sage est-il insensible ? \\[0pt]
Les agents économiques ont-ils un comportement rationnel ? \\[0pt]
Les agents sociaux poursuivent-ils l'utilité ? \\[0pt]
Les agents sociaux sont-ils rationnels ? \\[0pt]
Le salut vient-il de la raison ? \\[0pt]
Les animaux échappent-ils à la moralité ? \\[0pt]
Les animaux ont-ils des droits ? \\[0pt]
Les animaux pensent-ils ? \\[0pt]
Les animaux peuvent-ils avoir des droits ? \\[0pt]
Les animaux révèlent-ils ce que nous sommes ? \\[0pt]
Les animaux sont-ils nos amis ? \\[0pt]
Les apparences font-elles partie du monde ? \\[0pt]
Les apparences sont-elles toujours trompeuses ? \\[0pt]
Les artistes sont-ils sérieux ? \\[0pt]
Les arts admettent-ils une hiérarchie ? \\[0pt]
Les arts communiquent-ils entre eux ? \\[0pt]
Les arts ont-ils besoin de théorie ? \\[0pt]
Les arts ont-ils pour fonction de divertir ? \\[0pt]
Les arts sont-ils des jeux ? \\[0pt]
Le savoir a-t-il besoin d'être fondé ? \\[0pt]
Le savoir a-t-il des degrés ? \\[0pt]
Le savoir émancipe-t-il ? \\[0pt]
Le savoir est-il libérateur ? \\[0pt]
Le savoir est-il une condition du bonheur ? \\[0pt]
Le savoir est-il une forme de pouvoir ? \\[0pt]
Le savoir exclut-il toute forme de croyance ? \\[0pt]
Le savoir peut-il être gratuit ? \\[0pt]
Le savoir rend-il libre ? \\[0pt]
Le savoir se vulgarise-t-il ? \\[0pt]
Les beaux-arts sont-ils compatibles entre eux ? \\[0pt]
Les bêtes travaillent-elles ? \\[0pt]
Les bonnes lois font-elles les bonnes mœurs ? \\[0pt]
Les bons comptes font-ils les bons amis ? \\[0pt]
Les catégories sont-elles définitives ? \\[0pt]
Les catégories sont-elles des effets de langue ? \\[0pt]
Les causes naturelles expliquent-elles la nature ? \\[0pt]
Le scepticisme a-t-il des limites ? \\[0pt]
Les changements de théorie s'expliquent-ils seulement par le résultat des expériences ? \\[0pt]
Les choses ont-elles l'air de ce qu'elles sont ? \\[0pt]
Les choses ont-elles quelque chose en commun ? \\[0pt]
Les choses ont-elles une essence ? \\[0pt]
Les choses ont-elles un sens ? \\[0pt]
Les choses peuvent-elles avoir un sens ? \\[0pt]
Les choses sont-elles dans l'espace ? \\[0pt]
Les coïncidences ont-elles des causes ? \\[0pt]
Les comportements humains s'expliquent-il par l'instinct naturel ? \\[0pt]
Les concepts sont-ils des fictions ? \\[0pt]
Les concitoyens doivent-ils être des amis ? \\[0pt]
Les conflits menacent-ils la société ? \\[0pt]
Les conflits politiques ne sont-ils que des conflits sociaux ? \\[0pt]
Les conflits sociaux sont-ils des conflits de classe ? \\[0pt]
Les conflits sociaux sont-ils des conflits politiques ? \\[0pt]
Les connaissances scientifiques peuvent-elles être à la fois vraies et provisoires ? \\[0pt]
Les connaissances scientifiques peuvent-elles être vulgarisées ? \\[0pt]
Les connaissances scientifiques sont-elles vraies ? \\[0pt]
Les considérations morales ont-elles leur place en politique ? \\[0pt]
Les contradictions sont-elles un échec de la raison ? \\[0pt]
Les convictions d'autrui sont-elles un argument ? \\[0pt]
Les couleurs existent-elles en dehors de notre esprit ? \\[0pt]
Les croyances religieuses sont-elles indiscutables ? \\[0pt]
Les croyances sont-elles utiles ? \\[0pt]
Les cultures sont-elles incommensurables ? \\[0pt]
Les décisions politiques peuvent-elles être collectives ? \\[0pt]
Les démonstrations ont-elles des propriétés esthétiques ? \\[0pt]
Les désaccords politiques peuvent-ils être surmontés ? \\[0pt]
Les désirs ont-ils nécessairement un objet ? \\[0pt]
Les devoirs de l'homme varient-ils selon la culture ? \\[0pt]
Les devoirs de l'homme varient-ils selon les cultures ? \\[0pt]
Les différences entre les sciences humaines sont-elles des différences d'objet ? \\[0pt]
Les différentes preuves de l'existence de Dieu prouvent-elles le même Dieu ? \\[0pt]
Les divers sens du mot culture peuvent-ils être ramenés à l'unité ? \\[0pt]
Les droits de l'homme ont-ils besoin d'être fondés ? \\[0pt]
Les droits de l'homme ont-ils un fondement moral ? \\[0pt]
Les droits de l'homme sont-ils ceux du citoyen ? \\[0pt]
Les droits de l'homme sont-ils les droits de la femme ? \\[0pt]
Les droits de l'homme sont-ils une abstraction ? \\[0pt]
Les droits naturels imposent-ils une limite à la politique ? \\[0pt]
Les échanges économiques sont-ils facteurs de paix ? \\[0pt]
Les échanges, facteurs de paix ? \\[0pt]
Les échanges favorisent-ils la paix ? \\[0pt]
Les échanges sont-ils facteurs de paix ? \\[0pt]
Le sens commun existe-t-il ? \\[0pt]
Le sens de l'opportunité est-il moral ? \\[0pt]
Le sens des mots dépend-il de notre connaissance des choses ? \\[0pt]
Le sens de tout énoncé dépend-il de son interprétation ? \\[0pt]
Le sens du devoir nous fait-il connaître le bien de façon indiscutable ? \\[0pt]
Le sensible est-il communicable ? \\[0pt]
Le sensible est-il irréductible à l'intelligible ? \\[0pt]
Le sensible peut-il être connu ? \\[0pt]
Le sensible peut-il fonder des certitudes ? \\[0pt]
Le sens moral est-il naturel ? \\[0pt]
Le sentiment d'injustice est-il naturel ? \\[0pt]
Les entités mathématiques sont-elles des fictions ? \\[0pt]
Les épreuves ont-elles un sens ? \\[0pt]
Les États ont-ils pour fin d'assurer la paix ? \\[0pt]
Les êtres vivants sont-ils des machines ? \\[0pt]
Les événements historiques sont-ils de nature imprévisible ? \\[0pt]
Les exceptions ont-elles une place en morale ? \\[0pt]
Les faits existent-ils indépendamment de leur établissement par l'esprit humain ? \\[0pt]
Les faits parlent-ils d'eux-mêmes ? \\[0pt]
Les faits peuvent-ils faire autorité ? \\[0pt]
Les faits sociaux se répètent-ils ? \\[0pt]
Les faits sont-ils des preuves ? \\[0pt]
Les faits sont-ils têtus ? \\[0pt]
Les fins de la technique sont-elles techniques ? \\[0pt]
Les fins sont-elles toujours intentionnelles ? \\[0pt]
Les générations futures ont-elles des droits ? \\[0pt]
Les habitudes nous forment-elles ? \\[0pt]
Les hommes croient-ils dans leurs mythes ? \\[0pt]
Les hommes n'agissent-ils que par intérêt ? \\[0pt]
Les hommes naissent-ils libres ? \\[0pt]
Les hommes ont-ils besoin de maîtres ? \\[0pt]
Les hommes peuvent-ils s'associer sans renoncer à leur liberté ? \\[0pt]
Les hommes savent-ils ce qu'ils désirent ? \\[0pt]
Les hommes sont-ils des animaux ? \\[0pt]
Les hommes sont-ils faits pour s'entendre ? \\[0pt]
Les hommes sont-ils frères ? \\[0pt]
Les hommes sont-ils naturellement sociables ? \\[0pt]
Les hommes sont-ils seulement le produit de leur culture ? \\[0pt]
Les hypothèses scientifiques ont-elles pour nature d'être confirmées ou infirmées ? \\[0pt]
Les idées existent-elles ? \\[0pt]
Les idées mènent-elles le monde ? \\[0pt]
Les idées ont-elles une existence éternelle ? \\[0pt]
Les idées ont-elles une histoire ? \\[0pt]
Les idées ont-elles une réalité ? \\[0pt]
Les idées sont-elles vivantes ? \\[0pt]
Le silence a-t-il un sens ? \\[0pt]
Le silence est-il signifiant ? \\[0pt]
Le silence signifie-t-il toujours l'échec du langage ? \\[0pt]
Les images empêchent-elles de penser ? \\[0pt]
Les images nous égarent-elles ? \\[0pt]
Les images ont-elles un sens ? \\[0pt]
Les images peuvent-elles instruire ? \\[0pt]
Les images peuvent-elles mentir ? \\[0pt]
Les individus se réduisent-ils à leurs qualités ? \\[0pt]
Les inégalités de la nature doivent-elles être compensées ? \\[0pt]
Les inégalités menacent-elles la société ? \\[0pt]
Les inégalités sociales sont-elles inévitables ? \\[0pt]
Les inégalités sociales sont-elles naturelles ? \\[0pt]
Le singulier est-il objet de connaissance ? \\[0pt]
Les institutions politiques : qu'ont-elles en propre ? \\[0pt]
Les intérêts particuliers peuvent-ils tempérer l'autorité politique ? \\[0pt]
Les jugements de goût sont-ils susceptibles de vérité ? \\[0pt]
Les langues meurent-elles ? \\[0pt]
Les langues que nous parlons sont-elles imparfaites ? \\[0pt]
Les langues sont-elles condamnées à l'approximation ? \\[0pt]
Les limites de ma langue sont-elles les limites de mon monde ? \\[0pt]
Les lois de la nature sont-elles contingentes ? \\[0pt]
Les lois de la nature sont-elles de simples régularités ? \\[0pt]
Les lois de la nature sont elles nécessaires ? \\[0pt]
Les lois nous rendent-elles meilleurs ? \\[0pt]
Les lois scientifiques sont-elles des lois de la nature ? \\[0pt]
Les lois sont-elles sacrées ? \\[0pt]
Les lois sont-elles seulement utiles ? \\[0pt]
Les lois suffisent-elles à garantir nos libertés ? \\[0pt]
Les machines nous rendent-elles libres ? \\[0pt]
Les machines pensent-elles ? \\[0pt]
Les machines permettent-elles de mieux connaître le corps humain ? \\[0pt]
Les maladies mentales sont-elles des maladies sociales ? \\[0pt]
Les mathématiques consistent-elles seulement en des opérations de l'esprit ? \\[0pt]
Les mathématiques ont-elles affaire au réel ? \\[0pt]
Les mathématiques ont-elles besoin d'un fondement ? \\[0pt]
Les mathématiques parlent-elles du réel ? \\[0pt]
Les mathématiques peuvent-elles se passer de l'intuition ? \\[0pt]
Les mathématiques se réduisent-elles à une pensée cohérente ? \\[0pt]
Les mathématiques sont-elles le modèle de toute science ? \\[0pt]
Les mathématiques sont-elles réductibles à la logique ? \\[0pt]
Les mathématiques sont-elles une découverte ou une invention ? \\[0pt]
Les mathématiques sont-elles une science ? \\[0pt]
Les mathématiques sont-elles une science de l'ordre ? \\[0pt]
Les mathématiques sont-elles un instrument ? \\[0pt]
Les mathématiques sont-elles un jeu de l'esprit ? \\[0pt]
Les mathématiques sont-elles un langage ? \\[0pt]
Les mathématiques sont-elles utiles au philosophe ? \\[0pt]
Les méchants peuvent-ils être amis ? \\[0pt]
Les méchants peuvent-ils faire société ? \\[0pt]
Les mœurs sont-elles objet de science ? \\[0pt]
Les mœurs sont-ils l'objet d'une science ? \\[0pt]
Les mots disent-ils les choses ? \\[0pt]
Les mots expriment-ils les choses ? \\[0pt]
Les mots n'ont-ils que le pouvoir qu'on leur donne ? \\[0pt]
Les mots nous éloignent-ils des autres ? \\[0pt]
Les mots nous éloignent-ils des choses ? \\[0pt]
Les mots ont-ils un pouvoir ? \\[0pt]
Les mots parviennent-ils à tout exprimer ? \\[0pt]
Les mots peuvent-ils nous manquer ? \\[0pt]
Les mots rendent-ils compte de la nature des choses ? \\[0pt]
Les mots sont-ils trompeurs ? \\[0pt]
Les mythes relèvent-ils de l'anthropologie ? \\[0pt]
Les nombres gouvernent-ils le monde ? \\[0pt]
Les noms propres ont-ils une signification ? \\[0pt]
Les normes du savoir sont-elles universelles ? \\[0pt]
Les nouvelles technologies transforment-elles l'idée de l'art ? \\[0pt]
Les objets existent-ils ? \\[0pt]
Les objets sont-ils colorés ? \\[0pt]
Les objets techniques nous imposent-ils une manière de vivre ? \\[0pt]
Les œuvres d'art doivent-elles faire l'objet d'une évaluation morale ? \\[0pt]
Les œuvres d'art ont-elles besoin d'un commentaire ? \\[0pt]
Les œuvres d'art sont-elles des choses ? \\[0pt]
Les œuvres d'art sont-elles des réalités comme les autres ? \\[0pt]
Les œuvres d'art sont-elles éternelles ? \\[0pt]
Le soi est-il inaccessible ? \\[0pt]
Le soleil se lèvera-t-il demain ? \\[0pt]
Le souci d'autrui résume-t-il la morale ? \\[0pt]
Le souci de soi est-il une attitude morale ? \\[0pt]
Le souci du bien-être est-il politique ? \\[0pt]
L'espace nous sépare-t-il ? \\[0pt]
Les passions ont-elles une place en politique ? \\[0pt]
Les passions peuvent-elles être raisonnables ? \\[0pt]
Les passions sont-elles mauvaises ? \\[0pt]
Les passions sont-elles toujours mauvaises ? \\[0pt]
Les passions sont-elles toutes bonnes ? \\[0pt]
Les passions sont-elles un obstacle à la vie sociale ? \\[0pt]
Les passions s'opposent-elles à la raison ? \\[0pt]
L'espérance est-elle une mauvaise passion ? \\[0pt]
L'espérance est-elle une vertu ? \\[0pt]
Les perceptions sont-elles des jugements ? \\[0pt]
Les personnages de fiction peuvent-ils avoir une réalité ? \\[0pt]
Les peuples font-ils l'histoire ? \\[0pt]
Les peuples ont-ils les gouvernements qu'ils méritent ? \\[0pt]
Les phénomènes inconscients sont-ils réductibles à une mécanique cérébrale ? \\[0pt]
Les philosophes doivent-ils être rois ? \\[0pt]
Les philosophies se classent-elles ? \\[0pt]
L'espoir peut-il être raisonnable ? \\[0pt]
Le sport : s'accomplir ou se dépasser ? \\[0pt]
Les principes de la morale dépendent-ils de la culture ? \\[0pt]
Les principes d'une science sont-ils des conventions ? \\[0pt]
Les principes sont-ils indémontrables ? \\[0pt]
L'esprit appartient-il à la nature ? \\[0pt]
L'esprit dépend-il du corps ? \\[0pt]
L'esprit domine-t-il la matière ? \\[0pt]
L'esprit est-il chose ? \\[0pt]
L'esprit est-il irréductible à la matière ? \\[0pt]
L'esprit est-il libre ? \\[0pt]
L'esprit est-il matériel ? \\[0pt]
L'esprit est-il mieux connu que le corps ? \\[0pt]
L'esprit est-il objet de science ? \\[0pt]
L'esprit est-il plus aisé à connaître que le corps ? \\[0pt]
L'esprit est-il plus difficile à connaître que la matière ? \\[0pt]
L'esprit est-il plus facile à connaître que le corps ? \\[0pt]
L'esprit est-il réductible à la matière ? \\[0pt]
L'esprit est-il une chose ? \\[0pt]
L'esprit est-il une machine ? \\[0pt]
L'esprit est-il un ensemble de facultés ? \\[0pt]
L'esprit est-il une partie du corps ? \\[0pt]
L'esprit et la matière peuvent-ils agir l'un sur l'autre ? \\[0pt]
L'esprit humain progresse-t-il ? \\[0pt]
L'esprit n'a-t-il jamais affaire qu'à lui-même ? \\[0pt]
L'esprit peut-il être divisé ? \\[0pt]
L'esprit peut-il être malade ? \\[0pt]
L'esprit peut-il être mesuré ? \\[0pt]
L'esprit peut-il être objet de science ? \\[0pt]
L'esprit s'explique-t-il par une activité cérébrale ? \\[0pt]
Les problèmes politiques peuvent-ils se ramener à des problèmes techniques ? \\[0pt]
Les problèmes politiques sont-ils des problèmes techniques ? \\[0pt]
Les procédures de validation dans les sciences humaines ? \\[0pt]
Les progrès de la technique sont-ils nécessairement des progrès de la raison ? \\[0pt]
Les progrès techniques constituent-ils des progrès de la civilisation ? \\[0pt]
Les propositions métaphysiques sont-elles des illusions ? \\[0pt]
Les proverbes enseignent-ils quelque chose ? \\[0pt]
Les proverbes nous instruisent-ils moralement ? \\[0pt]
Les qualités sensibles sont-elles dans les choses ou dans l'esprit ? \\[0pt]
Les querelles de mots sont-elles toujours stériles ? \\[0pt]
Les questions métaphysiques ont-elles un sens ? \\[0pt]
Les rapports entre les hommes sont-ils des rapports de force ? \\[0pt]
Les rapports sociaux sont-ils des rapports entre individus ? \\[0pt]
Les règles du langage limitent-elles notre liberté d'expression ? \\[0pt]
Les règles du langage sont- elles un obstacle à la communication ? \\[0pt]
Les règles sociales sont-elles répressives ? \\[0pt]
Les relations ont-elles une existence objective ? \\[0pt]
Les religions naissent-elles du besoin de justice ? \\[0pt]
Les religions peuvent-elles être objets de science ? \\[0pt]
Les religions peuvent-elles prétendre libérer les hommes ? \\[0pt]
Les religions sont-elles affaire de foi ? \\[0pt]
Les religions sont-elles des illusions ? \\[0pt]
Les rêves ont-ils un sens ? \\[0pt]
Les rêves sont-ils des pensées ? \\[0pt]
Les révolutions techniques suscitent-elles des révolutions dans l'art ? \\[0pt]
Les scélérats peuvent-ils être heureux ? \\[0pt]
Les sciences décrivent-elles le réel ? \\[0pt]
Les sciences décrivent-elles ou construisent-elles leurs objets ? \\[0pt]
Les sciences de la vie visent-elles un objet irréductible à la matière ? \\[0pt]
Les sciences de l'homme ont-elles inventé leur objet ? \\[0pt]
Les sciences de l'homme permettent-elles d'affiner la notion de responsabilité ? \\[0pt]
Les sciences de l'homme peuvent-elles expliquer l'impuissance de la liberté ? \\[0pt]
Les sciences de l'homme rendent-elles l'homme prévisible ? \\[0pt]
Les sciences de l'homme sont-elles des sciences ? \\[0pt]
Les sciences de l'homme sont-elles utiles à la société ? \\[0pt]
Les sciences doivent-elle prétendre à l'unification ? \\[0pt]
Les sciences forment-elle un système ? \\[0pt]
Les sciences humaines doivent-elles être transdisciplinaires ? \\[0pt]
Les sciences humaines échappent-elles à l'abstraction ? \\[0pt]
Les sciences humaines éliminent-elles la contingence du futur ? \\[0pt]
Les sciences humaines, est-ce la mort de l'homme ? \\[0pt]
Les sciences humaines et sociales n'ont-elles de sciences que le nom ? \\[0pt]
Les sciences humaines finissent-elles de ruiner l'idée de liberté ? \\[0pt]
Les sciences humaines jouent elles un rôle dans l'exploitation de l'homme par l'homme ? \\[0pt]
Les sciences humaines nient-elles la liberté de l'homme ? \\[0pt]
Les sciences humaines nient-elles la liberté du sujet humain ? \\[0pt]
Les sciences humaines nous font-elles connaître l'homme ? \\[0pt]
Les sciences humaines nous obligent-elles à renoncer à l'idée de libre arbitre ? \\[0pt]
Les sciences humaines nous protègent-elles de l'essentialisme ? \\[0pt]
Les sciences humaines ont-elles une utilité ? \\[0pt]
Les sciences humaines ont-elles un objet commun ? \\[0pt]
Les sciences humaines opposent-elles histoire et système ? \\[0pt]
Les sciences humaines permettent-elles de comprendre la vie d'un homme ? \\[0pt]
Les sciences humaines peuvent-elles adopter les méthodes des sciences de la nature ? \\[0pt]
Les sciences humaines peuvent-elles éclairer la réflexion éthique ? \\[0pt]
Les sciences humaines peuvent-elles être objectives ? \\[0pt]
Les sciences humaines peuvent-elles faire abstraction de la nature ? \\[0pt]
Les sciences humaines peuvent-elles s'affranchir de l'ethnocentrisme ? \\[0pt]
Les sciences humaines peuvent-elles se passer de la notion d'inconscient ? \\[0pt]
Les sciences humaines présupposent-elles une définition de l'homme ? \\[0pt]
Les sciences humaines recherchent-elles des invariants ? \\[0pt]
Les sciences humaines rendent-elle l'homme prévisible ? \\[0pt]
Les sciences humaines reposent-elles sur des faits ? \\[0pt]
Les sciences humaines sont-elles des sciences ? \\[0pt]
Les sciences humaines sont-elles des sciences de la nature humaine ? \\[0pt]
Les sciences humaines sont-elles des sciences de la vie humaine ? \\[0pt]
Les sciences humaines sont-elles des sciences de l'esprit ? \\[0pt]
Les sciences humaines sont-elles des sciences de l'interprétation ? \\[0pt]
Les sciences humaines sont-elles des sciences d'interprétation ? \\[0pt]
Les sciences humaines sont-elles explicatives ou compréhensives ? \\[0pt]
Les sciences humaines sont-elles les sciences de l'esprit ? \\[0pt]
Les sciences humaines sont-elles nécessairement empiriques ? \\[0pt]
Les sciences humaines sont-elles normatives ? \\[0pt]
Les sciences humaines sont-elles relativistes ? \\[0pt]
Les sciences humaines sont-elles subversives ? \\[0pt]
Les sciences humaines suffisent-elles à connaître l'homme ? \\[0pt]
Les sciences humaines supposent-elles une définition de l'homme ? \\[0pt]
Les sciences humaines supposent-elles une rupture avec l'ordre naturel ? \\[0pt]
Les sciences humaines traitent-elles de l'homme ? \\[0pt]
Les sciences humaines traitent-elles de l'individu ? \\[0pt]
Les sciences humaines transforment-elles la notion de causalité ? \\[0pt]
Les sciences ne sont-elles qu'une description du monde ? \\[0pt]
Les sciences nous donnent-elles des normes ? \\[0pt]
Les sciences ont-elles à penser leurs fondements ? \\[0pt]
Les sciences ont-elles besoin de principes fondamentaux ? \\[0pt]
Les sciences ont-elles besoin d'une fondation ? \\[0pt]
Les sciences ont-elles besoin d'une fondation métaphysique ? \\[0pt]
Les sciences ont-elles le monopole de la vérité ? \\[0pt]
Les sciences pensent-elles leurs fondements ? \\[0pt]
Les sciences permettent-elles de connaître la réalité-même ? \\[0pt]
Les sciences peuvent-elles exclure toute notion de finalité ? \\[0pt]
Les sciences peuvent-elles penser leurs fondements ? \\[0pt]
Les sciences peuvent-elles penser l'individu ? \\[0pt]
Les sciences peuvent-elles se passer de fondements métaphysiques ? \\[0pt]
Les sciences peuvent-elles se passer d'images ? \\[0pt]
Les sciences produisent-elles des normes ? \\[0pt]
Les sciences sociales ont-elles un objet ? \\[0pt]
Les sciences sociales peuvent-elles être expérimentales ? \\[0pt]
Les sciences sociales peuvent-elles servir à gouverner ? \\[0pt]
Les sciences sociales sont-elles nécessairement inexactes ? \\[0pt]
Les sciences sont-elles une description du monde ? \\[0pt]
Les sensations peuvent-elles se partager ? \\[0pt]
Les sens jugent-ils ? \\[0pt]
Les sens nous trompent-ils ? \\[0pt]
Les sens peuvent-ils nous tromper ? \\[0pt]
Les sens sont-ils source d'illusion ? \\[0pt]
Les sens sont-ils trompeurs ? \\[0pt]
Les sentiments ont-ils une histoire ? \\[0pt]
Les sentiments peuvent-ils s'apprendre ? \\[0pt]
Les sociétés évoluent-elles ? \\[0pt]
Les sociétés humaines se construisent-elles contre la nature ? \\[0pt]
Les sociétés ont-elles besoin de théâtre ? \\[0pt]
Les sociétés ont-elles un inconscient ? \\[0pt]
Les sociétés sont-elles hiérarchisables ? \\[0pt]
Les sociétés sont-elles imprévisibles ? \\[0pt]
Les structures expliquent-elles tout ? \\[0pt]
Les structures que dégagent les sciences humaines agissent-elles ? \\[0pt]
Les structures sociales sont-elles économiques ? \\[0pt]
Les techniques de manipulation des hommes peuvent-elles s'appuyer sur les sciences humaines ? \\[0pt]
Les théories scientifiques décrivent-elles la réalité ? \\[0pt]
Les théories scientifiques sont-elles vraies ? \\[0pt]
L'esthétique est-elle une métaphysique de l'art ? \\[0pt]
Le sublime est-il dans les choses ? \\[0pt]
Le succès peut-il être un critère de vérité ? \\[0pt]
Le sujet n'est-il qu'une fiction ? \\[0pt]
Le sujet peut-il s'aliéner par un libre choix ? \\[0pt]
Les universaux existent-ils ? \\[0pt]
Les valeurs morales ont-elles leur origine dans la raison ? \\[0pt]
Les valeurs ont-elles une histoire ? \\[0pt]
Les valeurs ont-elles une origine ? \\[0pt]
Les vérités scientifiques sont-elles relatives ? \\[0pt]
Les vérités sont-elles intemporelles ? \\[0pt]
Les vérités sont-elles toujours démontrables ? \\[0pt]
Les vertus ne sont-elles que des vices déguisés ? \\[0pt]
Les vertus sont-elles au principe de la morale ? \\[0pt]
Les vices privés peuvent-ils faire le bien public ? \\[0pt]
Les vivants peuvent-ils se passer des morts ? \\[0pt]
Le tableau ? \\[0pt]
Le talent s'enseigne-t-il ? \\[0pt]
L'étant est-il le premier pensable ? \\[0pt]
L'État a-t-il des intérêts propres ? \\[0pt]
L'État a-t-il le droit de contrôler notre habillement ? \\[0pt]
L'État a-t-il pour but de maintenir l'ordre ? \\[0pt]
L'État a-t-il pour finalité de maintenir l'ordre ? \\[0pt]
L'État a-t-il tous les droits ? \\[0pt]
L'État a-t-il un fondement rationnel ? \\[0pt]
L'État contribue-t-il à pacifier les relations entre les hommes ? \\[0pt]
L'État crée-t-il la liberté ? \\[0pt]
L'État de droit ? \\[0pt]
L'État doit-il disparaître ? \\[0pt]
L'État doit-il éduquer le citoyen ? \\[0pt]
L'État doit-il éduquer le peuple ? \\[0pt]
L'État doit-il éduquer les citoyens ? \\[0pt]
L'État doit-il être fort ? \\[0pt]
L'État doit-il être le plus fort ? \\[0pt]
L'État doit-il être neutre ? \\[0pt]
L'État doit-il être sans pitié ? \\[0pt]
L'État doit-il faire le bonheur des citoyens ? \\[0pt]
L'État doit-il nous rendre meilleurs ? \\[0pt]
L'État doit-il préférer l'injustice au désordre ? \\[0pt]
L'État doit-il reconnaître des limites à sa puissance ? \\[0pt]
L'État doit-il se mêler de religion ? \\[0pt]
L'État doit-il se préoccuper des arts ? \\[0pt]
L'État doit-il se préoccuper du bonheur des citoyens ? \\[0pt]
L'État doit-il se soucier de la morale ? \\[0pt]
L'État doit-il veiller au bonheur des individus ? \\[0pt]
L'État est-il appelé à disparaître ? \\[0pt]
L'État est-il au service de la société ? \\[0pt]
L'État est-il ennemi de la liberté ? \\[0pt]
L'État est-il fin ou moyen ? \\[0pt]
L'État est-il la seule source du droit ? \\[0pt]
L'État est-il le garant de la propriété privée ? \\[0pt]
L'État est-il le garant du bien commun ? \\[0pt]
L'État est-il l'ennemi de la liberté ? \\[0pt]
L'État est-il l'ennemi de l'individu ? \\[0pt]
L'État est-il libérateur ? \\[0pt]
L'État est-il nécessaire ? \\[0pt]
L'État est-il notre ennemi ? \\[0pt]
L'État est-il oppresseur ? \\[0pt]
L'État est-il souverain ? \\[0pt]
L'État est-il toujours juste ? \\[0pt]
L'État est-il un arbitre ? \\[0pt]
L'État est-il un mal nécessaire ? \\[0pt]
L'État est-il un moindre mal ? \\[0pt]
L'État est-il un « monstre froid » ? \\[0pt]
L'État est-il un tiers impartial ? \\[0pt]
L'État n'a-t-il d'existence que contractuelle ? \\[0pt]
L'État n'est-il qu'un instrument de domination ? \\[0pt]
L'État n'est-il qu'un moyen ? \\[0pt]
L'État nous rend-il meilleurs ? \\[0pt]
L'État peut-il avoir tous les droits ? \\[0pt]
L'État peut-il créer la liberté ? \\[0pt]
L'État peut-il demeurer indifférent à la religion ? \\[0pt]
L'État peut-il disparaître ? \\[0pt]
L'État peut-il être fondé sur un contrat ? \\[0pt]
L'État peut-il être impartial ? \\[0pt]
L'État peut-il être indifférent à la religion ? \\[0pt]
L'État peut-il être libéral ? \\[0pt]
L'État peut-il fonder sa propre légitimité ? \\[0pt]
L'État peut-il limiter son pouvoir ? \\[0pt]
L'État peut-il poursuivre une autre fin que sa propre puissance ? \\[0pt]
L'État peut-il renoncer à la violence ? \\[0pt]
L'État peut-il se désintéresser de la religion ? \\[0pt]
L'État sert-il l'intérêt général ? \\[0pt]
Le technicien n'est-il qu'un exécutant ? \\[0pt]
Le temps dépend-il de la mémoire ? \\[0pt]
Le temps détruit-il tout ? \\[0pt]
Le temps est-il destructeur ? \\[0pt]
Le temps est-il en nous ou hors de nous ? \\[0pt]
Le temps est-il essentiellement destructeur ? \\[0pt]
Le temps est-il la marque de notre impuissance ? \\[0pt]
Le temps est-il notre allié ? \\[0pt]
Le temps est-il notre ennemi ? \\[0pt]
Le temps est-il une contrainte ? \\[0pt]
Le temps est-il une dimension de la nature ? \\[0pt]
Le temps est-il une expérience de la continuité ou de la discontinuité ? \\[0pt]
Le temps est-il une prison ? \\[0pt]
Le temps est-il une réalité ? \\[0pt]
Le temps existe-t-il ? \\[0pt]
Le temps ne fait-il que passer ? \\[0pt]
Le temps n'est-il pour l'homme que ce qui le limite ? \\[0pt]
Le temps n'existe-t-il que subjectivement ? \\[0pt]
Le temps nous appartient-il ? \\[0pt]
Le temps nous est-il compté ? \\[0pt]
Le temps passe-t-il ? \\[0pt]
Le temps peut-il s'arrêter ? \\[0pt]
Le temps physique est-il comparable au temps psychique ? \\[0pt]
Le temps rend-il tout vain ? \\[0pt]
Le temps s'écoule-t-il ? \\[0pt]
Le temps se laisse-t-il décrire logiquement ? \\[0pt]
Le temps se mesure-t-il ? \\[0pt]
L'éternité est-elle désirable ? \\[0pt]
L'éternité n'est-elle qu'une illusion ? \\[0pt]
Le terrorisme est-il un acte de guerre ? \\[0pt]
L'éthique est-elle affaire de choix ? \\[0pt]
L'éthique suppose-t-elle la liberté ? \\[0pt]
L'ethnologie est-elle une science mélancolique ? \\[0pt]
Le tout est-il la somme de ses parties ? \\[0pt]
Le travail artistique doit-il demeurer caché ? \\[0pt]
Le travail a-t-il une valeur morale ? \\[0pt]
Le travail est-il le propre de l'homme ? \\[0pt]
Le travail est-il libérateur ? \\[0pt]
Le travail est-il nécessaire au bonheur ? \\[0pt]
Le travail est-il toujours une activité productrice ? \\[0pt]
Le travail est-il un besoin ? \\[0pt]
Le travail est-il une fin ? \\[0pt]
Le travail est-il une marchandise ? \\[0pt]
Le travail est-il une valeur ? \\[0pt]
Le travail est-il une valeur morale ? \\[0pt]
Le travail est-il un rapport naturel de l'homme à la nature ? \\[0pt]
Le travail fait-il de l'homme un être moral ? \\[0pt]
Le travail fonde-t-il la propriété ? \\[0pt]
Le travaille libère-t-il ? \\[0pt]
Le travail manuel est-il sans pensée ? \\[0pt]
Le travail nous libère-t-il ? \\[0pt]
Le travail nous rend-il heureux ? \\[0pt]
Le travail nous rend-il solidaires ? \\[0pt]
Le travail peut-il être solitaire ? \\[0pt]
Le travail rapproche-t-il les hommes ? \\[0pt]
Le travail unit-il ou sépare-t-il les hommes ? \\[0pt]
L'être en tant qu'être est-il connaissable ? \\[0pt]
L'être est-il un genre ? \\[0pt]
L'être humain désire-t-il naturellement connaître ? \\[0pt]
L'être humain est-il la mesure de toute chose ? \\[0pt]
L'être se confond-il avec l'être perçu ? \\[0pt]
L'étude de l'histoire conduit-elle à désespérer l'homme ? \\[0pt]
Le vainqueur a-t-il tous les droits ? \\[0pt]
L'événement historique a-t-il un sens par lui-même ? \\[0pt]
L'événement manque-t-il d'être ? \\[0pt]
L'évidence a-t-elle une valeur absolue ? \\[0pt]
L'évidence est-elle critère de vérité ? \\[0pt]
L'évidence est-elle le signe de la vérité ? \\[0pt]
L'évidence est-elle toujours un critère de vérité ? \\[0pt]
L'évidence est-elle un critère de vérité ? \\[0pt]
L'évidence est-elle un obstacle ou un instrument de la recherche de la vérité ? \\[0pt]
L'évidence se passe-t-elle de démonstration ? \\[0pt]
Le virtuel existe-t-il ? \\[0pt]
Le visage n'est-il qu'un masque ? \\[0pt]
Le vivant a-t-il des droits ? \\[0pt]
Le vivant a-t-il une temporalité propre ? \\[0pt]
Le vivant définit-il ses propres normes ? \\[0pt]
Le vivant échappe-t-il à la connaissance ? \\[0pt]
Le vivant échappe-t-il au déterminisme ? \\[0pt]
Le vivant est-il entièrement connaissable ? \\[0pt]
Le vivant est-il entièrement explicable ? \\[0pt]
Le vivant est-il l'affaire du politique ? \\[0pt]
Le vivant est-il réductible au physico-chimique ? \\[0pt]
Le vivant est-il un objet de science comme un autre ? \\[0pt]
Le vivant n'est-il que matière ? \\[0pt]
Le vivant n'est-il qu'une machine ingénieuse ? \\[0pt]
Le vivant obéit-il à des lois ? \\[0pt]
Le vivant obéit-il à une nécessité ? \\[0pt]
Le vivant se définit-il par sa résistance à la mort ? \\[0pt]
Le vrai admet-il des degrés ? \\[0pt]
Le vrai a-t-il une histoire ? \\[0pt]
Le vrai doit-il être démontré ? \\[0pt]
Le vrai est-il à lui-même sa propre marque ? \\[0pt]
Le vrai et le bien sont-ils analogues ? \\[0pt]
Le vrai n'est-il que ce sur quoi l'on s'accorde ? \\[0pt]
Le vrai peut-il rester invérifiable ? \\[0pt]
Le vraisemblable présente-t-il un intérêt pour la pensée ? \\[0pt]
Le vrai se perçoit-il ? \\[0pt]
Le vrai se réduit-il à ce qui est vérifiable ? \\[0pt]
Le vrai se réduit-il à l'utile ? \\[0pt]
L'exception est-elle instructive ? \\[0pt]
L'exception peut-elle confirmer la règle ? \\[0pt]
L'exécution d'une œuvre d'art est-elle toujours une œuvre d'art ? \\[0pt]
L'exercice du pouvoir est-il nécessairement solitaire ? \\[0pt]
L'exigence de vérité a-t-elle un sens moral ? \\[0pt]
L'existence a-t-elle un sens ? \\[0pt]
L'existence de l'État dépend-elle d'un contrat ? \\[0pt]
L'existence du mal met-elle en échec la raison ? \\[0pt]
L'existence est-elle pensable ? \\[0pt]
L'existence est-elle une propriété ? \\[0pt]
L'existence est-elle un jeu ? \\[0pt]
L'existence est-elle vaine ? \\[0pt]
L'existence se démontre-t-elle ? \\[0pt]
L'existence se laisse-t-elle penser ? \\[0pt]
L'existence se prouve-t-elle ? \\[0pt]
L'existence suppose-t-elle la conscience du temps ? \\[0pt]
L'expérience a-t-elle le même sens dans toutes les sciences ? \\[0pt]
L'expérience d'autrui nous est-elle utile ? \\[0pt]
L'expérience de la beauté est-elle naturelle ? \\[0pt]
L'expérience de la vie produit-elle un savoir ? \\[0pt]
L'expérience démontre-t-elle quelque chose ? \\[0pt]
L'expérience directe est-elle une connaissance ? \\[0pt]
L'expérience du beau peut-elle être une expérience métaphysique ? \\[0pt]
L'expérience enseigne-elle quelque chose ? \\[0pt]
L'expérience, est-ce l'observation ? \\[0pt]
L'expérience est-elle la seule source de nos connaissances ? \\[0pt]
L'expérience est-elle un guide suffisant ? \\[0pt]
L'expérience esthétique peut-elle se communiquer ? \\[0pt]
L'expérience esthétique relève-t-elle de la contemplation ? \\[0pt]
L'expérience instruit-elle ? \\[0pt]
L'expérience nous apprend-elle quelque chose ? \\[0pt]
L'expérience permet-elle de départager des théories concurrentes ? \\[0pt]
L'expérience peut-elle avoir raison des principes ? \\[0pt]
L'expérience peut-elle contredire la théorie ? \\[0pt]
L'expérience rend-elle raisonnable ? \\[0pt]
L'expérience rend-elle responsable ? \\[0pt]
L'expérience sensible est-elle la seule source légitime de connaissance ? \\[0pt]
L'expérience suffit-elle pour établir une vérité ? \\[0pt]
L'expérimentation est-elle facteur de vérité ? \\[0pt]
L'expression des citoyens dans le vote suffit-elle à faire la démocratie ? \\[0pt]
L'expression des citoyens dans le vote suffit-elle à faire vivre la démocratie ? \\[0pt]
L'expression « perdre son temps » a-t-elle un sens ? \\[0pt]
L'expression peut-elle être libre ? \\[0pt]
L'extinction du désir est-elle le commencement du bonheur ? \\[0pt]
L'habitude a-t-elle des vertus ? \\[0pt]
L'habitude est-elle notre guide dans la vie ? \\[0pt]
L'habitude est-elle une force ? \\[0pt]
L'habitude est-elle une seconde nature ? \\[0pt]
L'histoire a-t-elle des lois ? \\[0pt]
L'Histoire a-t-elle un commencement ? \\[0pt]
L'histoire a-t-elle un commencement et une fin ? \\[0pt]
L'histoire a-t-elle une fin ? \\[0pt]
L'histoire a-t-elle un sens ? \\[0pt]
L'histoire de l'art est-elle celle des styles ? \\[0pt]
L'histoire de l'art est-elle finie ? \\[0pt]
L'histoire de l'art peut-elle arriver à son terme ? \\[0pt]
L'histoire des arts est-elle liée à l'histoire des techniques ? \\[0pt]
L'histoire des sciences est-elle une histoire ? \\[0pt]
L'histoire du droit est-elle celle du progrès de la justice ? \\[0pt]
L'histoire : enquête ou science ? \\[0pt]
L'histoire est-elle avant tout mémoire ? \\[0pt]
L'histoire est-elle cyclique ? \\[0pt]
L'histoire est-elle déterministe ? \\[0pt]
L'histoire est-elle écrite par les vainqueurs ? \\[0pt]
L'histoire est-elle la connaissance du passé humain ? \\[0pt]
L'histoire est-elle la mémoire de l'humanité ? \\[0pt]
L'histoire est-elle la science de ce qui ne se répète jamais ? \\[0pt]
L'histoire est-elle la science de l'événementiel ? \\[0pt]
L'histoire est-elle la science du passé ? \\[0pt]
L'histoire est-elle la vérité de la mémoire ? \\[0pt]
L'histoire est-elle le récit objectif des faits passés ? \\[0pt]
L'histoire est-elle le règne du hasard ? \\[0pt]
L'histoire est-elle le théâtre des passions ? \\[0pt]
L'histoire est-elle rationnelle ? \\[0pt]
L'histoire est-elle sans fin ? \\[0pt]
L'histoire est-elle tragique ? \\[0pt]
L'histoire est-elle une explication ou une justification du passé ? \\[0pt]
L'histoire est-elle une science ? \\[0pt]
L'histoire est-elle une science comme les autres ? \\[0pt]
L'histoire est-elle une science de l'individuel ? \\[0pt]
L'histoire est-elle un genre littéraire ? \\[0pt]
L'histoire est-elle un récit ? \\[0pt]
L'histoire est-elle un roman ? \\[0pt]
L'histoire est-elle un roman vrai ? \\[0pt]
L'histoire est-elle utile ? \\[0pt]
L'histoire est-elle utile à la politique ? \\[0pt]
« L'histoire jugera » : quel sens faut-il accorder à cette expression ? \\[0pt]
L'histoire jugera-t-elle ? \\[0pt]
L'histoire n'a-t-elle pour objet que le passé ? \\[0pt]
L'histoire n'est-elle que bruit et fureur ? \\[0pt]
L'histoire n'est-elle que la connaissance du passé ? \\[0pt]
L'histoire n'est-elle que le produit de rapports de force ? \\[0pt]
L'histoire n'est-elle qu'un déchaînement de violence ? \\[0pt]
L'histoire n'est-elle qu'une suite d'événements ? \\[0pt]
L'histoire n'est-elle qu'un récit ? \\[0pt]
L'histoire nous appartient-elle ? \\[0pt]
L'histoire obéit-elle à des lois ? \\[0pt]
L'histoire pèse-t-elle comme un destin sur les individus ? \\[0pt]
L'histoire peut-elle être contemporaine ? \\[0pt]
L'histoire peut-elle être objective ? \\[0pt]
L'histoire peut-elle être universelle ? \\[0pt]
L'histoire peut-elle ne pas avoir de sens ? \\[0pt]
L'histoire peut-elle s'écrire au présent ? \\[0pt]
L'histoire peut-elle se répéter ? \\[0pt]
L'histoire : science ou récit ? \\[0pt]
L'histoire se réduit-elle à l'étude du passé ? \\[0pt]
L'histoire se répète-t-elle ? \\[0pt]
L'histoire sert-elle le présent ? \\[0pt]
L'histoire universelle est-elle l'histoire des guerres ? \\[0pt]
L'historien est-il un homme qui se souvient du passé ? \\[0pt]
L'historien peut-il être impartial ? \\[0pt]
L'historien peut-il se passer du concept de causalité ? \\[0pt]
L'homme aime-t-il la justice pour elle-même ? \\[0pt]
L'homme a-t-il besoin de l'art ? \\[0pt]
L'homme a-t-il besoin de religion ? \\[0pt]
L'homme a-t-il une nature ? \\[0pt]
L'homme a-t-il une place dans la nature ? \\[0pt]
L'homme des droits de l'homme n'est-il qu'une fiction ? \\[0pt]
L'homme est-il chez lui dans l'univers ? \\[0pt]
L'homme est-il face à la nature ? \\[0pt]
L'homme est-il fait pour le travail ? \\[0pt]
L'homme est-il la mesure de toute chose ? \\[0pt]
L'homme est-il la mesure de toutes choses ? \\[0pt]
L'homme est-il l'artisan de sa dignité ? \\[0pt]
L'homme est-il le seul être à avoir une histoire ? \\[0pt]
L'homme est-il le seul sujet de droit ? \\[0pt]
L'homme est-il le sujet de son histoire ? \\[0pt]
L'homme est-il libre par nature ? \\[0pt]
L'homme est-il meilleur seul ou en société ? \\[0pt]
L'homme est-il naturellement artiste ? \\[0pt]
L'homme est-il objet de science ? \\[0pt]
L'homme est-il par nature un être religieux ? \\[0pt]
L'homme est-il prisonnier du temps ? \\[0pt]
L'homme est-il raisonnable par nature ? \\[0pt]
L'homme est-il religieux par nature ? \\[0pt]
L'homme est-il un animal ? \\[0pt]
L'homme est-il un animal comme les autres ? \\[0pt]
L'homme est-il un animal comme un autre ? \\[0pt]
L'homme est-il un animal dénaturé ? \\[0pt]
L'homme est-il un animal métaphysique ? \\[0pt]
L'homme est-il un animal politique ? \\[0pt]
L'homme est-il un animal rationnel ? \\[0pt]
L'homme est-il un animal religieux ? \\[0pt]
L'homme est-il un animal social ? \\[0pt]
L'homme est-il un animal symbolique ? \\[0pt]
L'homme est-il un animal technicien ? \\[0pt]
L'homme est-il un corps pensant ? \\[0pt]
L'homme est-il un être de devoir ? \\[0pt]
L'homme est-il un être inachevé ? \\[0pt]
L'homme est-il un être social par nature ? \\[0pt]
L'homme est-il un loup pour l'homme ? \\[0pt]
L'homme est-il vraiment un animal politique ? \\[0pt]
L'homme et la nature sont-ils commensurables ? \\[0pt]
L'homme fait-il partie de la nature ? \\[0pt]
L'homme injuste est-il heureux ? \\[0pt]
L'homme injuste peut-il être heureux ? \\[0pt]
L'homme injuste peut-il être malheureux ? \\[0pt]
L'homme libre est-il un homme seul ? \\[0pt]
L'homme n'est-il qu'un animal comme les autres ? \\[0pt]
L'homme pense-t-il toujours ? \\[0pt]
L'homme peut-il changer ? \\[0pt]
L'homme peut-il être inhumain ? \\[0pt]
L'homme peut-il être libéré du besoin ? \\[0pt]
L'homme peut-il être objet d'expérience ? \\[0pt]
L'homme peut-il faire l'objet d'une science ? \\[0pt]
L'homme peut-il se représenter un monde sans l'homme ? \\[0pt]
L'homme se réalise-t-il dans le travail ? \\[0pt]
L'homme se reconnaît-il mieux dans le travail ou dans le loisir ? \\[0pt]
L'homme trouble-t-il l'ordre de la nature ? \\[0pt]
L'homme vertueux doit-il s'attendre à être malheureux ? \\[0pt]
L'homo œconomicus est-il rationnel ? \\[0pt]
L'hospitalité a-t-elle un sens politique ? \\[0pt]
L'hospitalité est-elle un devoir ? \\[0pt]
L'hospitalité est-elle une vertu politique ? \\[0pt]
L'humain est-il un animal irrationnel ? \\[0pt]
L'humanité a-t-elle eu une enfance ? \\[0pt]
L'humanité a-t-elle une histoire ? \\[0pt]
L'humanité est-elle aimable ? \\[0pt]
L'humanité : un concept normatif ? \\[0pt]
L'humilité est-elle une vertu ? \\[0pt]
L'hypothèse de la liberté est-elle compatible avec les exigences de la raison ? \\[0pt]
Libre arbitre et déterminisme sont-ils compatibles ? \\[0pt]
L'idéal moral est-il vain ? \\[0pt]
L'idée de bonheur collectif a-t-elle un sens ? \\[0pt]
L'idée de destin a-t-elle un sens ? \\[0pt]
L'idée de destin est-elle une idée périmée ? \\[0pt]
L'idée de devoir requiert-elle l'idée de liberté ? \\[0pt]
L'idée de « nature » n'est-elle qu'un mythe ? \\[0pt]
L'idée de progrès historique est-elle devenue désuète ? \\[0pt]
L'idée de rétribution est-elle nécessaire à la morale ? \\[0pt]
L'idée d'une liberté totale a-t- elle un sens ? \\[0pt]
L'idée d'une pensée inconsciente est-elle contradictoire ? \\[0pt]
L'idée d'une religion personnelle a-t-elle un sens ? \\[0pt]
L'identité personnelle est-elle donnée ou construite ? \\[0pt]
L'identité relève-t-elle du champ politique ? \\[0pt]
L'ignorance est-elle la raison du mal ? \\[0pt]
L'ignorance est-elle préférable à l'erreur ? \\[0pt]
L'ignorance est-elle une excuse ? \\[0pt]
L'ignorance nous excuse-t-elle ? \\[0pt]
L'ignorance peut-elle être une excuse ? \\[0pt]
L'illusion est-elle nécessaire au bonheur des hommes ? \\[0pt]
L'imagination a-t-elle des limites ? \\[0pt]
L'imagination a-t-elle sa place dans la connaissance scientifique ? \\[0pt]
L'imagination enrichit-elle la connaissance ? \\[0pt]
L'imagination est-elle créatrice ? \\[0pt]
L'imagination est-elle dangereuse ? \\[0pt]
L'imagination est-elle le refuge de la liberté ? \\[0pt]
L'imagination est-elle libre ? \\[0pt]
L'imagination est-elle maîtresse d'erreur et de fausseté ? \\[0pt]
L'imagination nous éloigne-t-elle du réel ? \\[0pt]
L'imagination nous instruit-elle ? \\[0pt]
L'imitation a-t-elle une fonction morale ? \\[0pt]
L'impartialité est-elle toujours désirable ? \\[0pt]
L'impossible est-il concevable ? \\[0pt]
L'incertitude est-elle dans les choses ou dans les idées ? \\[0pt]
L'incertitude interdit-elle de raisonner ? \\[0pt]
L'inconscience peut-elle être coupable ? \\[0pt]
L'inconscient a-t-il son propre langage ? \\[0pt]
L'inconscient a-t-il une histoire ? \\[0pt]
L'inconscient est-il dans l'âme ou dans le corps ? \\[0pt]
L'inconscient est-il l'animal en nous ? \\[0pt]
L'inconscient est-il pure négation de la conscience ? \\[0pt]
L'inconscient est-il un concept scientifique ? \\[0pt]
L'inconscient est-il un destin ? \\[0pt]
L'inconscient est-il une découverte ou une invention ? \\[0pt]
L'inconscient est-il une dimension de la conscience ? \\[0pt]
L'inconscient est-il une excuse ? \\[0pt]
L'inconscient est-il un obstacle à la liberté ? \\[0pt]
L'inconscient n'est-il qu'un défaut de conscience ? \\[0pt]
L'inconscient n'est-il qu'une hypothèse ? \\[0pt]
L'inconscient nous révèle-t-il à nous-même ? \\[0pt]
L'inconscient peut-il se manifester ? \\[0pt]
L'inconscient psychanalytique réfute-t-il la notion philosophique de conscience ? \\[0pt]
L'indifférence peut-elle être une vertu ? \\[0pt]
L'individualisme a-t-il sa place en politique ? \\[0pt]
L'individualisme est-il un égoïsme ? \\[0pt]
L'individualisme rend-il la politique impossible ? \\[0pt]
L'individu a-t-il des droits ? \\[0pt]
L'individu échappe-t-il aux sciences humaines ? \\[0pt]
L'individu est-il définissable ? \\[0pt]
L'individu est-il ineffable ? \\[0pt]
L'individu particulier peut-il être l'objet d'une connaissance sociologique ? \\[0pt]
L'inégalité a-t-elle des vertus ? \\[0pt]
L'inégalité est-elle nécessairement injuste ? \\[0pt]
L'inexact peut-il avoir une valeur ? \\[0pt]
L'infini est-il la négation du fini ? \\[0pt]
L'infini se réduit-il à l'indéfini ? \\[0pt]
L'infinité de l'univers a-t-elle de quoi nous effrayer ? \\[0pt]
L'injustice est-elle préférable au désordre ? \\[0pt]
L'innovation technique nous impose-t-elle de nouveaux devoirs ? \\[0pt]
L'inquiétude peut-elle définir l'existence humaine ? \\[0pt]
L'inquiétude peut-elle devenir l'existence humaine ? \\[0pt]
L'instant a-t-il une réalité ? \\[0pt]
L'instant de la décision est-il une folie ? \\[0pt]
L'instruction est-elle facteur de moralité ? \\[0pt]
L'insurrection est-elle un droit ? \\[0pt]
L'intelligence peut-elle être artificielle ? \\[0pt]
L'intelligence peut-elle être inhumaine ? \\[0pt]
L'intention morale suffit-elle à constituer la valeur morale de l'action ? \\[0pt]
L'interdit a-t-il une origine sociale ? \\[0pt]
L'interdit est-il au fondement de la culture ? \\[0pt]
L'intérêt constitue-t-il l'unique lien social ? \\[0pt]
L'intérêt de la société l'emporte-t-il sur celui des individus ? \\[0pt]
L'intérêt est-il le principe de tout échange ? \\[0pt]
L'intérêt général est-il la somme des intérêts particuliers ? \\[0pt]
L'intérêt général est-il le bien commun ? \\[0pt]
L'intérêt général est-il un mythe ? \\[0pt]
L'intérêt général n'est-il qu'un mythe ? \\[0pt]
L'intérêt gouverne-t-il le monde ? \\[0pt]
L'intérêt peut-il avoir une valeur morale ? \\[0pt]
L'intérêt peut-il être général ? \\[0pt]
L'intérêt peut-il être une valeur morale ? \\[0pt]
L'intérêt public est-il une illusion ? \\[0pt]
L'intériorité est-elle un mythe ? \\[0pt]
L'interprétation est-elle sans fin ? \\[0pt]
L'interprétation est-elle un art ? \\[0pt]
L'interprétation est-elle un art ou une science ? \\[0pt]
L'interprétation est-elle une activité sans fin ? \\[0pt]
L'interprétation est-elle une forme de connaissance ? \\[0pt]
L'interprétation est-elle une science ? \\[0pt]
L'interprétation : mode d'être ou mode de connaissance ? \\[0pt]
L'interprète est-il un créateur ? \\[0pt]
L'interprète sait-il ce qu'il cherche ? \\[0pt]
L'intime est-il politique ? \\[0pt]
L'introspection est-elle une connaissance ? \\[0pt]
L'intuition a-t-elle une place en logique ? \\[0pt]
L'inutile a-t-il de la valeur ? \\[0pt]
L'inutile est-il sans valeur ? \\[0pt]
L'irrationnel est-il nécessairement absurde ? \\[0pt]
L'irrationnel est-il pensable ? \\[0pt]
L'irrationnel est-il toujours absurde ? \\[0pt]
L'irrationnel existe-t-il ? \\[0pt]
L'obéissance est-elle compatible avec la liberté ? \\[0pt]
L'obéissance peut-elle être un acte de liberté ? \\[0pt]
L'objectivité existe-t-elle ? \\[0pt]
L'objectivité historique est-elle synonyme de neutralité ? \\[0pt]
L'objet du désir en est-il la cause ? \\[0pt]
L'obligation morale peut-elle se réduire à une obligation sociale ? \\[0pt]
L'œuvre d'art a-t-elle un sens ? \\[0pt]
L'œuvre d'art doit-elle être belle ? \\[0pt]
L'œuvre d'art doit-elle nous émouvoir ? \\[0pt]
L'œuvre d'art donne-t-elle à penser ? \\[0pt]
L'œuvre d'art échappe-t-elle au monde ? \\[0pt]
L'œuvre d'art échappe-t-elle au temps ? \\[0pt]
L'œuvre d'art échappe-t-elle nécessairement au temps ? \\[0pt]
L'œuvre d'art est-elle anhistorique ? \\[0pt]
L'œuvre d'art est-elle intemporelle ? \\[0pt]
L'œuvre d'art est-elle l'expression d'une idée ? \\[0pt]
L'œuvre d'art est-elle toujours destinée à un public ? \\[0pt]
L'œuvre d'art est-elle une belle apparence ? \\[0pt]
L'œuvre d'art est-elle une marchandise ? \\[0pt]
L'œuvre d'art est-elle un monde ? \\[0pt]
L'œuvre d'art est-elle un objet d'échange ? \\[0pt]
L'œuvre d'art est-elle un symbole ? \\[0pt]
L'œuvre d'art instruit-elle ? \\[0pt]
L'œuvre d'art nous apprend-elle quelque chose ? \\[0pt]
L'œuvre d'art représente-t-elle quelque chose ? \\[0pt]
L'œuvre d'art traduit-elle une vision du monde ? \\[0pt]
L'ontologie peut-elle être relative ? \\[0pt]
L'opéra est-il un art complet ? \\[0pt]
L'opinion a-t-elle nécessairement tort ? \\[0pt]
L'opinion est-elle un savoir ? \\[0pt]
L'ordinaire est-il ennuyeux ? \\[0pt]
L'ordre des échanges est-il un ordre rationnel ? \\[0pt]
L'ordre du vivant est-il façonné par le hasard ? \\[0pt]
L'ordre est-il dans les choses ? \\[0pt]
L'ordre politique exclut-il la violence ? \\[0pt]
L'ordre politique peut-il exclure la violence ? \\[0pt]
L'ordre social peut-il être juste ? \\[0pt]
L'origine des langues est-elle un faux problème ? \\[0pt]
L'oubli est-il nécessaire à la vie ? \\[0pt]
L'oubli est-il un échec de la mémoire ? \\[0pt]
L'oubli n'est-il qu'une perte de mémoire ? \\[0pt]
L'unanimité est-elle un critère de légitimité ? \\[0pt]
L'unanimité est-elle un critère de vérité ? \\[0pt]
L'unité des sciences humaines ? \\[0pt]
L'universel est-il une illusion ? \\[0pt]
L'universel implique-t-il l'indifférence à la singularité ? \\[0pt]
L'utilité est-elle étrangère à la morale ? \\[0pt]
L'utilité est-elle une valeur morale ? \\[0pt]
L'utilité peut-elle être le principe de la moralité ? \\[0pt]
L'utilité peut-elle être un critère pour juger de la valeur de nos actions ? \\[0pt]
L'utopie a-t-elle une signification politique ? \\[0pt]
L'utopie a-t-elle un lien avec la pratique ? \\[0pt]
Ma conscience est-elle digne de confiance ? \\[0pt]
Ma liberté s'arrête-t-elle où commence celle des autres ? \\[0pt]
Ma parole m'engage-t-elle ? \\[0pt]
Mérite-t-on d'être heureux ? \\[0pt]
Mon corps est-il ma propriété ? \\[0pt]
Mon corps est-il naturel ? \\[0pt]
Mon corps fait-il obstacle à ma liberté ? \\[0pt]
Mon corps m'appartient-il ? \\[0pt]
Mon devoir dépend-il de moi ? \\[0pt]
Mon existence est-elle contingente ? \\[0pt]
Mon passé m'appartient-il ? \\[0pt]
Mon prochain est-il mon semblable ? \\[0pt]
Montrer, est-ce démontrer ? \\[0pt]
Morale et politique sont-elles indépendantes ? \\[0pt]
Naît-on sujet ou le devient-on ? \\[0pt]
N'apprend-on que par l'expérience ? \\[0pt]
N'a-t-on des devoirs qu'envers autrui ? \\[0pt]
N'avons-nous affaire qu'au réel ? \\[0pt]
N'avons-nous de devoir qu'envers autrui ? \\[0pt]
N'avons-nous de devoirs qu'envers autrui ? \\[0pt]
N'échange-t-on que ce qui a de la valeur ? \\[0pt]
N'échange-t-on que des biens ? \\[0pt]
N'échange-t-on que des symboles ? \\[0pt]
N'échange-t-on que par intérêt ? \\[0pt]
Ne désire-t-on que ce dont on manque ? \\[0pt]
Ne désirons-nous que ce qui est bon pour nous ? \\[0pt]
Ne désirons-nous que les choses que nous estimons bonnes ? \\[0pt]
Ne doit-on tenir pour vrai que ce qui est démontré ? \\[0pt]
Ne faut-il pas craindre la liberté ? \\[0pt]
Ne faut-il vivre que dans le présent ? \\[0pt]
Ne faut-il vivre que pour faire son devoir ? \\[0pt]
Ne prêche-t-on que les convertis ? \\[0pt]
Ne sait-on rien que par expérience ? \\[0pt]
Ne sommes-nous justes que par intérêt ? \\[0pt]
Ne sommes-nous véritablement maîtres que de nos pensées ? \\[0pt]
N'est-il d'inférence que démonstrative ? \\[0pt]
N'est-on juste que par crainte du châtiment ? \\[0pt]
Ne travaille-t-on que par nécessité ? \\[0pt]
Ne travaille-t- on que pour un salaire ? \\[0pt]
Ne veut-on que ce qui est désirable ? \\[0pt]
Ne vit-on bien qu'avec ses amis ? \\[0pt]
N'existe-t-il que des individus ? \\[0pt]
N'existe-t-il que le présent ? \\[0pt]
N'existe-t-il qu'un seul temps ? \\[0pt]
N'exprime-t-on que ce dont on a conscience ? \\[0pt]
Nier la liberté de l'homme, est-ce ôter tout sens à la morale ? \\[0pt]
Nier le libre arbitre, est-ce nier la liberté ? \\[0pt]
N'interprète-t-on que ce qui est équivoque ? \\[0pt]
Nos convictions morales sont-elles le simple reflet de notre temps ? \\[0pt]
Nos désirs nous appartiennent-ils ? \\[0pt]
Nos désirs nous opposent-ils ? \\[0pt]
Nos devoirs nous motivent-ils ? \\[0pt]
Nos devoirs se ramènent-ils aux conventions sociales ? \\[0pt]
Nos goûts sont-ils justifiables ? \\[0pt]
Nos habitudes compromettent-elles notre moralité ? \\[0pt]
Nos paroles peuvent-elles dépasser notre pensée ? \\[0pt]
Nos pensées dépendent-elles de nous ? \\[0pt]
Nos pensées sont-elles entièrement en notre pouvoir ? \\[0pt]
Nos sens nous font-ils connaître la matière ? \\[0pt]
Nos sens nous trompent-ils ? \\[0pt]
Nos sens peuvent-ils s'éduquer ? \\[0pt]
Notre connaissance du réel se limite-t-elle au savoir scientifique ? \\[0pt]
Notre corps pense-t-il ? \\[0pt]
Notre existence a-t-elle un sens si l'histoire n'en a pas ? \\[0pt]
Notre ignorance nous excuse-t-elle ? \\[0pt]
Notre liberté de pensée a-t-elle des limites ? \\[0pt]
Notre liberté est-elle toujours relative ? \\[0pt]
Notre pensée est-elle prisonnière de la langue que nous parlons ? \\[0pt]
Notre rapport au monde est-il essentiellement technique ? \\[0pt]
Notre rapport au monde peut-il être exclusivement technique ? \\[0pt]
Notre rapport au monde peut-il n'être que technique ? \\[0pt]
Nous trouvons-nous nous-mêmes dans l'animal ? \\[0pt]
N'y a-t-il d'amitié qu'entre égaux ? \\[0pt]
N'y a-t-il de beauté qu'artistique ? \\[0pt]
N'y a-t-il de bonheur que dans l'instant ? \\[0pt]
N'y a-t-il de bonheur qu'éphémère ? \\[0pt]
N'y a-t-il de certitude que mathématique ? \\[0pt]
N'y a-t-il de connaissance que de l'universel ? \\[0pt]
N'y a-t-il de connaissance que par les causes ? \\[0pt]
N'y a-t-il de connaissance que scientifique ? \\[0pt]
N'y a-t-il de démocratie que représentative ? \\[0pt]
N'y a-t-il de devoirs qu'envers autrui ? \\[0pt]
N'y a-t-il de droit qu'écrit ? \\[0pt]
N'y a-t-il de droit que de l'homme ? \\[0pt]
N'y a-t-il de droits que de l'homme ? \\[0pt]
N'y a-t-il de foi que religieuse ? \\[0pt]
N'y a-t-il de liberté qu'individuelle ? \\[0pt]
N'y a-t-il de propriété que privée ? \\[0pt]
N'y a-t-il de rationalité que scientifique ? \\[0pt]
N'y a-t-il de réalité que de l'individuel ? \\[0pt]
N'y a-t-il de réel que le présent ? \\[0pt]
N'y a-t-il de savoir que livresque ? \\[0pt]
N'y a-t-il de science qu'autant qu'il s'y trouve de mathématique ? \\[0pt]
N'y a-t-il de science que de ce qui est mathématisable ? \\[0pt]
N'y a-t-il de science que du général ? \\[0pt]
N'y a-t-il de science que du mesurable ? \\[0pt]
N'y a-t-il de science que du nécessaire ? \\[0pt]
N'y a-t-il de science que provisoire ? \\[0pt]
N'y a-t-il de science que systématique ? \\[0pt]
N'y a-t-il de science qu'exacte ? \\[0pt]
N'y a-t-il des droits que de l'homme ? \\[0pt]
N'y a-t-il de sens que par le langage ? \\[0pt]
N'y a-t-il d'être que sensible ? \\[0pt]
N'y a-t-il de vérité que scientifique ? \\[0pt]
N'y a-t-il de vérité que vérifiable ? \\[0pt]
N'y a-t-il de vérités que scientifiques ? \\[0pt]
N'y a-t-il de vrai que le vérifiable ? \\[0pt]
N'y a-t-il d'origine que mythique ? \\[0pt]
N'y a-t-il que des individus ? \\[0pt]
N'y a-t-il que du mesurable ? \\[0pt]
N'y a-t-il que l'intention qui compte ? \\[0pt]
N'y a-t-il qu'une substance ? \\[0pt]
N'y a-t-il qu'un seul monde ? \\[0pt]
N'y a-t-il rien au monde de certain ? \\[0pt]
Obéir aux lois, est-ce renoncer à la liberté ? \\[0pt]
Obéir, est-ce se soumettre ? \\[0pt]
Où commence la liberté ? \\[0pt]
Où commence la servitude ? \\[0pt]
Où commence la vie privée ? \\[0pt]
Où commence la violence ? \\[0pt]
Où commence le territoire de la psychologie ? \\[0pt]
Où commence l'interprétation ? \\[0pt]
Où commence ma liberté ? \\[0pt]
Où est le danger ? \\[0pt]
Où est le passé ? \\[0pt]
Où est le pouvoir ? \\[0pt]
Où est l'esprit ? \\[0pt]
Où est mon esprit ? \\[0pt]
Où est-on quand on pense ? \\[0pt]
Où s'arrête la responsabilité ? \\[0pt]
Où s'arrête l'espace public ? \\[0pt]
Où sont les relations ? \\[0pt]
Où suis-je ? \\[0pt]
Où suis-je quand je pense ? \\[0pt]
Par le langage, peut-on agir sur la réalité ? \\[0pt]
Parler de Dieu, est-ce nécessairement tenir un discours religieux ? \\[0pt]
Parler de soi est-il intéressant ? \\[0pt]
Parler, est-ce agir ? \\[0pt]
Parler, est-ce communiquer ? \\[0pt]
Parler, est-ce donner sa parole ? \\[0pt]
Parler, est-ce ne pas agir ? \\[0pt]
Parler, n'est-ce que désigner ? \\[0pt]
Par où commencer ? \\[0pt]
Par quoi un individu diffère-t-il réellement d'un autre ? \\[0pt]
Par quoi un individu se distingue-t-il d'un autre ? \\[0pt]
« Pas de liberté pour les ennemis de la liberté » ? \\[0pt]
Peindre, est-ce nécessairement feindre ? \\[0pt]
Peindre, est-ce traduire ? \\[0pt]
Peindre un paysage, est-ce représenter la nature ? \\[0pt]
Penser, est-ce aller contre l'évidence ? \\[0pt]
Penser, est-ce calculer ? \\[0pt]
Penser, est-ce comparer ? \\[0pt]
Penser, est-ce désobéir ? \\[0pt]
Penser, est-ce dire non ? \\[0pt]
Penser, est-ce se parler à soi-même ? \\[0pt]
Penser est-il assimilable à un travail ? \\[0pt]
Penser le rien, est-ce ne rien penser ? \\[0pt]
Penser par soi-même, est-ce être l'auteur de ses pensées ? \\[0pt]
Penser peut-il nous rendre heureux ? \\[0pt]
Penser requiert-il d'avoir un corps ? \\[0pt]
Penser requiert-il un corps ? \\[0pt]
Pense-t-on jamais seul ? \\[0pt]
Pensez-vous que vous avez une âme ? \\[0pt]
Percevoir, est-ce connaître ? \\[0pt]
Percevoir, est-ce interpréter ? \\[0pt]
Percevoir, est-ce juger ? \\[0pt]
Percevoir, est-ce nécessaire pour penser ? \\[0pt]
Percevoir, est-ce reconnaître ? \\[0pt]
Percevoir, est-ce savoir ? \\[0pt]
Percevoir, est-ce s'ouvrir au monde ? \\[0pt]
Percevoir s'apprend-il ? \\[0pt]
Percevons-nous le monde extérieur ? \\[0pt]
Percevons-nous les choses telles qu'elles sont ? \\[0pt]
Perçoit-on le réel ? \\[0pt]
Perçoit-on le réel tel qu'il est ? \\[0pt]
Perçoit-on les choses comme elles sont ? \\[0pt]
Peut-il être moral de tuer ? \\[0pt]
Peut-il être préférable de ne pas savoir ? \\[0pt]
Peut-il exister une action désintéressée ? \\[0pt]
Peut-il exister une morale sceptique ? \\[0pt]
Peut-il y avoir conflit entre nos devoirs ? \\[0pt]
Peut-il y avoir de bonnes raisons de croire ? \\[0pt]
Peut-il y avoir de bons tyrans ? \\[0pt]
Peut-il y avoir de la politique sans conflit ? \\[0pt]
Peut-il y avoir des choix collectifs ? \\[0pt]
Peut-il y avoir des conflits de devoirs ? \\[0pt]
Peut-il y avoir des degrés dans la vérité ? \\[0pt]
Peut-il y avoir des échanges équitables ? \\[0pt]
Peut-il y avoir des expériences métaphysiques ? \\[0pt]
Peut-il y avoir des lois de l'histoire ? \\[0pt]
Peut-il y avoir des lois injustes ? \\[0pt]
Peut-il y avoir des modèles en morale ? \\[0pt]
Peut-il y avoir des nations sans État ? \\[0pt]
Peut-il y avoir des vérités partielles ? \\[0pt]
Peut-il y avoir esprit sans corps ? \\[0pt]
Peut-il y avoir plusieurs vérités religieuses ? \\[0pt]
Peut-il y avoir savoir-faire sans savoir ? \\[0pt]
Peut-il y avoir science sans intuition du vrai ? \\[0pt]
Peut-il y avoir un art conceptuel ? \\[0pt]
Peut-il y avoir un bien commun ? \\[0pt]
Peut-il y avoir un droit à désobéir ? \\[0pt]
Peut-il y avoir un droit de la guerre ? \\[0pt]
Peut-il y avoir une anthropologie non anthropocentriste ? \\[0pt]
Peut-il y avoir une ethnologie non ethnocentriste ? \\[0pt]
Peut-il y avoir une histoire universelle ? \\[0pt]
Peut-il y avoir une méthode commune à toutes les sciences ? \\[0pt]
Peut-il y avoir une morale dans les échanges ? \\[0pt]
Peut-il y avoir une pensée sans langage ? \\[0pt]
Peut-il y avoir une philosophie applicable ? \\[0pt]
Peut-il y avoir une philosophie appliquée ? \\[0pt]
Peut-il y avoir une philosophie politique sans Dieu ? \\[0pt]
Peut-il y avoir une politique de la langue ? \\[0pt]
Peut-il y avoir une pratique sans théorie ? \\[0pt]
Peut-il y avoir une science de la morale ? \\[0pt]
Peut-il y avoir une science de l'éducation ? \\[0pt]
Peut-il y avoir une science de l'homme ? \\[0pt]
Peut-il y avoir une science de l'inconscient ? \\[0pt]
Peut-il y avoir une science des signes ? \\[0pt]
Peut-il y avoir une science du désordre ? \\[0pt]
Peut-il y avoir une science du moi ? \\[0pt]
Peut-il y avoir une science politique ? \\[0pt]
Peut-il y avoir une servitude volontaire ? \\[0pt]
Peut-il y avoir une société des nations ? \\[0pt]
Peut-il y avoir une société sans État ? \\[0pt]
Peut-il y avoir un État mondial ? \\[0pt]
Peut-il y avoir une vérité en art ? \\[0pt]
Peut-il y avoir une vérité en politique ? \\[0pt]
Peut-il y avoir une vérité religieuse ? \\[0pt]
Peut-il y avoir un intérêt collectif ? \\[0pt]
Peut-il y avoir un langage universel ? \\[0pt]
Peut-on abolir la religion ? \\[0pt]
Peut-on admettre ce qui n'est pas vérifié ? \\[0pt]
Peut-on admettre un droit à la révolte ? \\[0pt]
Peut-on affirmer que le monde a un ordre ? \\[0pt]
Peut-on affirmer qu'être, c'est être perçu ou percevoir ? \\[0pt]
Peut-on agir de manière désintéressée ? \\[0pt]
Peut-on agir machinalement ? \\[0pt]
Peut-on agir sans prendre de risques ? \\[0pt]
Peut-on agir sans raison ? \\[0pt]
Peut-on aimer ce qu'on ne connaît pas ? \\[0pt]
Peut-on aimer l'autre tel qu'il est ? \\[0pt]
Peut-on aimer la vie plus que tout ? \\[0pt]
Peut-on aimer les animaux ? \\[0pt]
Peut-on aimer l'humanité ? \\[0pt]
Peut-on aimer sans perdre sa liberté ? \\[0pt]
Peut-on aimer son prochain comme soi-même ? \\[0pt]
Peut-on aimer son travail ? \\[0pt]
Peut-on aimer une œuvre d'art sans la comprendre ? \\[0pt]
Peut-on à la fois préserver et dominer la nature ? \\[0pt]
Peut-on aller à l'encontre de la nature ? \\[0pt]
Peut-on appréhender les choses telles qu'elles sont ? \\[0pt]
Peut-on apprendre à être heureux ? \\[0pt]
Peut-on apprendre à être juste ? \\[0pt]
Peut-on apprendre à être libre ? \\[0pt]
Peut-on apprendre à mourir ? \\[0pt]
Peut-on apprendre à penser ? \\[0pt]
Peut-on apprendre à vivre ? \\[0pt]
Peut-on argumenter en morale ? \\[0pt]
Peut-on arrêter le progrès ? \\[0pt]
Peut-on assimiler le vivant à une machine ? \\[0pt]
Peut-on atteindre une certitude ? \\[0pt]
Peut-on attendre de la politique qu'elle soit conforme aux exigences de la raison ? \\[0pt]
Peut-on attribuer à chacun son dû ? \\[0pt]
Peut-on avoir conscience de soi sans avoir conscience d'autrui ? \\[0pt]
Peut-on avoir de bonnes raisons de ne pas dire la vérité ? \\[0pt]
Peut-on avoir des droits sans avoir de devoirs ? \\[0pt]
Peut-on avoir le droit de se révolter ? \\[0pt]
Peut-on avoir peur de la liberté ? \\[0pt]
Peut-on avoir peur de soi-même ? \\[0pt]
Peut-on avoir peur d'être libre ? \\[0pt]
Peut-on avoir raison contre la science ? \\[0pt]
Peut-on avoir raison contre les faits ? \\[0pt]
Peut-on avoir raison contre tous ? \\[0pt]
Peut-on avoir raison contre tout le monde ? \\[0pt]
Peut-on avoir raisons contre les faits ? \\[0pt]
Peut-on avoir raison tout.e seul.e ? \\[0pt]
Peut-on avoir raison tout seul ? \\[0pt]
Peut-on avoir trop d'imagination ? \\[0pt]
Peut-on bien agir sans le savoir ? \\[0pt]
Peut-on cesser de croire ? \\[0pt]
Peut-on cesser de désirer ? \\[0pt]
Peut-on changer au point de ne plus être le même ? \\[0pt]
Peut-on changer de culture ? \\[0pt]
Peut-on changer de logique ? \\[0pt]
Peut-on changer le cours de l'histoire ? \\[0pt]
Peut-on changer le monde ? \\[0pt]
Peut-on changer le passé ? \\[0pt]
Peut-on changer ses désirs ? \\[0pt]
Peut-on changer ses désirs à volonté ? \\[0pt]
Peut-on choisir de renoncer à sa liberté ? \\[0pt]
Peut-on choisir le mal ? \\[0pt]
Peut-on choisir sa vie ? \\[0pt]
Peut-on choisir ses désirs ? \\[0pt]
Peut-on classer les arts ? \\[0pt]
Peut-on combattre une croyance par le raisonnement ? \\[0pt]
Peut-on commander à la nature ? \\[0pt]
Peut-on communiquer ses perceptions à autrui ? \\[0pt]
Peut-on communiquer son expérience ? \\[0pt]
Peut-on comparer deux philosophies ? \\[0pt]
Peut-on comparer les cultures ? \\[0pt]
Peut-on comparer l'organisme à une machine ? \\[0pt]
Peut-on comprendre ce qui est illogique ? \\[0pt]
Peut-on comprendre le présent ? \\[0pt]
Peut-on comprendre les actions humaines ? \\[0pt]
Peut-on comprendre un acte que l'on désapprouve ? \\[0pt]
Peut-on comprendre un auteur mieux qu'il ne s'est compris lui-même ? \\[0pt]
Peut-on concevoir la science achevée ? \\[0pt]
Peut-on concevoir une humanité sans art ? \\[0pt]
Peut-on concevoir une morale sans sanction ni obligation ? \\[0pt]
Peut-on concevoir une religion dans les limites de la simple raison ? \\[0pt]
Peut-on concevoir une science qui ne soit pas démonstrative ? \\[0pt]
Peut-on concevoir une science sans expérience ? \\[0pt]
Peut-on concevoir une société juste sans que les hommes ne le soient ? \\[0pt]
Peut-on concevoir une société qui n'aurait plus besoin du droit ? \\[0pt]
Peut-on concevoir une société sans État ? \\[0pt]
Peut-on concevoir un État mondial ? \\[0pt]
Peut-on concilier bonheur et liberté ? \\[0pt]
Peut-on concilier le droit à la liberté et l'égalité des droits ? \\[0pt]
Peut-on concilier nécessité et liberté ? \\[0pt]
Peut-on conclure de l'être au devoir-être ? \\[0pt]
Peut-on connaître autrui ? \\[0pt]
Peut-on connaître les causes ? \\[0pt]
Peut-on connaître les choses telles qu'elles sont ? \\[0pt]
Peut-on connaître le singulier ? \\[0pt]
Peut-on connaître l'esprit ? \\[0pt]
Peut-on connaître le vivant sans le dénaturer ? \\[0pt]
Peut-on connaître le vivant sans recourir à la notion de finalité ? \\[0pt]
Peut-on connaître l'individuel ? \\[0pt]
Peut-on connaître par intuition ? \\[0pt]
Peut-on connaître sans concept ? \\[0pt]
Peut-on connaître une vie humaine ? \\[0pt]
Peut-on considérer la conscience comme une chose ? \\[0pt]
Peut-on considérer la main comme un outil ? \\[0pt]
Peut-on considérer l'art comme un langage ? \\[0pt]
Peut-on considérer les hommes de science comme des autorités morales ? \\[0pt]
Peut-on contester les droits de l'homme ? \\[0pt]
Peut-on contredire l'expérience ? \\[0pt]
Peut-on convaincre quelqu'un de la beauté d'une œuvre d'art ? \\[0pt]
Peut-on craindre la liberté ? \\[0pt]
Peut-on créer un homme nouveau ? \\[0pt]
Peut-on critiquer la démocratie ? \\[0pt]
Peut-on critiquer la religion ? \\[0pt]
Peut-on croire ce qu'on veut ? \\[0pt]
Peut-on croire en rien ? \\[0pt]
Peut-on croire librement ? \\[0pt]
Peut-on croire sans être crédule ? \\[0pt]
Peut-on croire sans savoir pourquoi ? \\[0pt]
Peut-on décider de croire ? \\[0pt]
Peut-on décider de tout ? \\[0pt]
Peut-on décider d'être heureux ? \\[0pt]
Peut-on définir ce qu'est un progrès technique ? \\[0pt]
Peut-on définir la matière ? \\[0pt]
Peut-on définir la morale comme l'art d'être heureux ? \\[0pt]
Peut-on définir la santé ? \\[0pt]
Peut-on définir la vérité ? \\[0pt]
Peut-on définir la vérité par la correspondance ? \\[0pt]
Peut-on définir la vie ? \\[0pt]
Peut-on définir le bien ? \\[0pt]
Peut-on définir le bonheur ? \\[0pt]
Peut-on délimiter le réel ? \\[0pt]
Peut-on délimiter l'humain ? \\[0pt]
Peut-on démontrer l'existence de la matière ? \\[0pt]
Peut-on démontrer que l'on est libre ? \\[0pt]
Peut-on démontrer qu'on ne rêve pas ? \\[0pt]
Peut-on démontrer une théorie ? \\[0pt]
Peut-on dépasser la subjectivité ? \\[0pt]
Peut-on dériver des concepts de l'expérience ? \\[0pt]
Peut-on désirer autre chose qu'être heureux ? \\[0pt]
Peut-on désirer ce qui est ? \\[0pt]
Peut-on désirer ce qu'on ne veut pas ? \\[0pt]
Peut-on désirer ce qu'on possède ? \\[0pt]
Peut-on désirer ce qu'on sait être impossible ? \\[0pt]
Peut-on désirer l'absolu ? \\[0pt]
Peut-on désirer l'impossible ? \\[0pt]
Peut-on désirer sans souffrir ? \\[0pt]
Peut-on désobéir à l'État ? \\[0pt]
Peut-on désobéir aux lois ? \\[0pt]
Peut-on désobéir par devoir ? \\[0pt]
Peut-on dialoguer avec soi-même ? \\[0pt]
Peut-on dialoguer avec un ordinateur ? \\[0pt]
Peut-on dire ce que l'on pense ? \\[0pt]
Peut-on dire ce qui n'est pas ? \\[0pt]
Peut-on dire de la connaissance scientifique qu'elle procède par approximation ? \\[0pt]
Peut-on dire de l'art qu'il donne un monde en partage ? \\[0pt]
Peut-on dire d'une image qu'elle parle ? \\[0pt]
Peut-on dire d'une œuvre d'art qu'elle est ratée ? \\[0pt]
Peut-on dire d'une théorie scientifique qu'elle n'est jamais plus vraie qu'une autre mais seulement plus commode ? \\[0pt]
Peut-on dire d'un homme qu'il est supérieur à un autre homme ? \\[0pt]
Peut-on dire la vérité ? \\[0pt]
Peut-on dire le singulier ? \\[0pt]
Peut-on dire l'être ? \\[0pt]
Peut-on dire que la science désenchante le monde ? \\[0pt]
Peut-on dire que la science ne nous fait pas connaître les choses mais les rapports entre les choses ? \\[0pt]
Peut-on dire que les hommes font l'histoire ? \\[0pt]
Peut-on dire que les machines travaillent pour nous ? \\[0pt]
Peut-on dire que les mots pensent pour nous ? \\[0pt]
Peut-on dire que l'humanité progresse ? \\[0pt]
Peut-on dire que rien n'échappe à la technique ? \\[0pt]
Peut-on dire qu'est vrai ce qui correspond aux faits ? \\[0pt]
Peut-on dire que toutes les croyances se valent ? \\[0pt]
Peut-on dire que tout est politique ? \\[0pt]
Peut-on dire que tout est relatif ? \\[0pt]
Peut-on dire qu'une théorie physique en contredit une autre ? \\[0pt]
Peut-on dire toute la vérité ? \\[0pt]
Peut-on discuter des goûts et des couleurs ? \\[0pt]
Peut-on disposer de son corps ? \\[0pt]
Peut-on distinguer différents types de causes ? \\[0pt]
Peut-on distinguer entre de bons et de mauvais désirs ? \\[0pt]
Peut-on distinguer entre les bons et les mauvais désirs ? \\[0pt]
Peut-on distinguer la vie et l'existence ? \\[0pt]
Peut-on distinguer le réel de l'imaginaire ? \\[0pt]
Peut-on distinguer les faits de leurs interprétations ? \\[0pt]
Peut-on distinguer le vrai du faux ? \\[0pt]
Peut-on donner un sens à la souffrance ? \\[0pt]
Peut-on donner un sens à l'existence ? \\[0pt]
Peut-on donner un sens à son existence ? \\[0pt]
Peut-on douter de la raison ? \\[0pt]
Peut-on douter de l'humanité d'un homme ? \\[0pt]
Peut-on douter de sa propre existence ? \\[0pt]
Peut-on douter de soi ? \\[0pt]
Peut-on douter de tout ? \\[0pt]
Peut-on douter de toute vérité ? \\[0pt]
Peut-on douter du sens des mots ? \\[0pt]
Peut-on échanger des idées ? \\[0pt]
Peut-on échapper à ses désirs ? \\[0pt]
Peut-on échapper à son temps ? \\[0pt]
Peut-on échapper au cours de l'histoire ? \\[0pt]
Peut-on échapper au temps ? \\[0pt]
Peut-on échapper aux relations de pouvoir ? \\[0pt]
Peut-on éclairer la liberté ? \\[0pt]
Peut-on écrire comme on parle ? \\[0pt]
Peut-on éduquer la conscience ? \\[0pt]
Peut-on éduquer la sensibilité ? \\[0pt]
Peut-on éduquer le goût ? \\[0pt]
Peut-on éduquer quelqu'un à être libre ? \\[0pt]
Peut-on en appeler à la conscience contre la loi ? \\[0pt]
Peut-on en appeler à la conscience contre l'État ? \\[0pt]
Peut-on encore être cosmopolite ? \\[0pt]
Peut-on encore parler d'une nature humaine ? \\[0pt]
Peut-on encore soutenir que l'homme est un animal rationnel ? \\[0pt]
Peut-on en finir avec la violence ? \\[0pt]
Peut-on en finir avec les préjugés ? \\[0pt]
Peut-on en savoir trop ? \\[0pt]
Peut-on entreprendre d'éliminer la métaphysique ? \\[0pt]
Peut-on espérer être libéré du travail ? \\[0pt]
Peut-on établir une hiérarchie des arts ? \\[0pt]
Peut-on être à la fois lucide et heureux ? \\[0pt]
Peut-on être aliéné sans le savoir ? \\[0pt]
Peut-on être amoral ? \\[0pt]
Peut-on être apolitique ? \\[0pt]
Peut-on être assuré d'avoir raison ? \\[0pt]
Peut-on être athée ? \\[0pt]
Peut-on être citoyen du monde ? \\[0pt]
Peut-on être citoyen sans être soldat ? \\[0pt]
Peut-on être complètement athée ? \\[0pt]
Peut-on être content de soi ? \\[0pt]
Peut-on être dans le présent ? \\[0pt]
Peut-on être désintéressé ? \\[0pt]
Peut-on être en avance sur son temps ? \\[0pt]
Peut-on être en conflit avec soi-même ? \\[0pt]
Peut-on être esclave de soi-même ? \\[0pt]
Peut-on être étranger à soi-même ? \\[0pt]
Peut-on être étranger au monde ? \\[0pt]
Peut-on être fidèle à soi-même ? \\[0pt]
Peut-on être heureux dans la solitude ? \\[0pt]
Peut-on être heureux sans être sage ? \\[0pt]
Peut-on être heureux sans le savoir ? \\[0pt]
Peut-on être heureux sans philosophie ? \\[0pt]
Peut-on être heureux sans s'en rendre compte ? \\[0pt]
Peut-on être heureux tout seul ? \\[0pt]
Peut-on être homme sans être citoyen ? \\[0pt]
Peut-on être hors de soi ? \\[0pt]
Peut-on être ignorant ? \\[0pt]
Peut-on être impartial ? \\[0pt]
Peut-on être indifférent à l'histoire ? \\[0pt]
Peut-on être indifférent à son bonheur ? \\[0pt]
Peut-on être injuste avec soi-même ? \\[0pt]
Peut-on être injuste envers soi-même ? \\[0pt]
Peut-on être injuste et heureux ? \\[0pt]
Peut-on être insensible à l'art ? \\[0pt]
Peut-on être insensible au vrai ? \\[0pt]
Peut-on être juste dans une situation injuste ? \\[0pt]
Peut-on être juste dans une société injuste ? \\[0pt]
Peut-on être juste sans être impartial ? \\[0pt]
Peut-on être libre sans le savoir ? \\[0pt]
Peut-on être maître de soi ? \\[0pt]
Peut-on être méchant volontairement ? \\[0pt]
Peut-on être moral sans religion ? \\[0pt]
Peut-on être objectif en matière d'art ? \\[0pt]
Peut-on être obligé d'aimer ? \\[0pt]
Peut-on être plus ou moins libre ? \\[0pt]
Peut-on être prisonnier de soi-même ? \\[0pt]
Peut-on être responsable de ce que l'on n'a pas fait ? \\[0pt]
Peut-on être sage inconsciemment ? \\[0pt]
Peut-on être sage tout seul ? \\[0pt]
Peut-on être sans opinion ? \\[0pt]
Peut-on être sceptique ? \\[0pt]
Peut-on être sceptique de bonne foi ? \\[0pt]
Peut-on être seul ? \\[0pt]
Peut-on être seul avec soi-même ? \\[0pt]
Peut-on être soi-même en société ? \\[0pt]
Peut-on être sûr d'avoir raison ? \\[0pt]
Peut-on être sûr de bien agir ? \\[0pt]
Peut-on être sûr de ne pas se tromper ? \\[0pt]
Peut-on être trop religieux ? \\[0pt]
Peut-on être trop sage ? \\[0pt]
Peut-on être trop sensible ? \\[0pt]
Peut-on étudier le passé de façon objective ? \\[0pt]
Peut-on étudier l'homme de l'extérieur ? \\[0pt]
Peut-on exercer son esprit ? \\[0pt]
Peut-on expérimenter sur le vivant ? \\[0pt]
Peut-on expliquer le mal ? \\[0pt]
Peut-on expliquer le monde par la matière ? \\[0pt]
Peut-on expliquer le vivant ? \\[0pt]
Peut-on expliquer un acte libre ? \\[0pt]
Peut-on expliquer une œuvre d'art ? \\[0pt]
Peut-on faire ce qu'on ne veut pas ? \\[0pt]
Peut-on faire comme si le passé n'existait pas ? \\[0pt]
Peut-on faire de la politique sans supposer les hommes méchants ? \\[0pt]
Peut-on faire de l'art avec tout ? \\[0pt]
Peut-on faire de l'esprit un objet de science ? \\[0pt]
Peut-on faire de sa vie une œuvre d'art ? \\[0pt]
Peut-on faire du dialogue un modèle de relation morale ? \\[0pt]
Peut-on faire la paix ? \\[0pt]
Peut-on faire la philosophie de l'histoire ? \\[0pt]
Peut-on faire le bien d'autrui malgré lui ? \\[0pt]
Peut-on faire le bien de quelqu'un malgré lui ? \\[0pt]
Peut-on faire le bonheur d'autrui ? \\[0pt]
Peut-on faire l'économie de la notion de forme ? \\[0pt]
Peut-on faire l'éloge de l'illusion ? \\[0pt]
Peut-on faire le mal en vue du bien ? \\[0pt]
Peut-on faire le mal innocemment ? \\[0pt]
Peut-on faire l'expérience de la nécessité ? \\[0pt]
Peut-on faire l'inventaire du monde ? \\[0pt]
Peut-on faire son devoir par habitude ? \\[0pt]
Peut-on faire table rase du passé ? \\[0pt]
Peut-on faire un mauvais usage de la raison ? \\[0pt]
Peut-on feindre la vertu ? \\[0pt]
Peut-on fixer des limites à la science ? \\[0pt]
Peut-on fonder la liberté ? \\[0pt]
Peut-on fonder la morale ? \\[0pt]
Peut-on fonder la morale sur la pitié ? \\[0pt]
Peut-on fonder le droit de propriété ? \\[0pt]
Peut-on fonder le droit sur la morale ? \\[0pt]
Peut-on fonder les droits de l'homme ? \\[0pt]
Peut-on fonder les mathématiques ? \\[0pt]
Peut-on fonder un droit de désobéir ? \\[0pt]
Peut-on fonder une éthique sur la biologie ? \\[0pt]
Peut-on fonder une morale sur la nature ? \\[0pt]
Peut-on fonder une morale sur le plaisir ? \\[0pt]
Peut-on forcer quelqu'un à être libre ? \\[0pt]
Peut-on forcer un homme à être libre ? \\[0pt]
Peut-on fuir hors du monde ? \\[0pt]
Peut-on fuir la société ? \\[0pt]
Peut-on gâcher son talent ? \\[0pt]
Peut-on gouverner le développement des innovations techniques ? \\[0pt]
Peut-on gouverner sans lois ? \\[0pt]
Peut-on guérir par la parole ? \\[0pt]
Peut-on haïr la raison ? \\[0pt]
Peut-on haïr la vie ? \\[0pt]
Peut-on haïr les images ? \\[0pt]
Peut-on hiérarchiser les arts ? \\[0pt]
Peut-on hiérarchiser les devoirs ? \\[0pt]
Peut-on hiérarchiser les œuvres d'art ? \\[0pt]
Peut-on hiérarchiser les sociétés ? \\[0pt]
Peut-on identifier le désir au besoin ? \\[0pt]
Peut-on ignorer sa propre liberté ? \\[0pt]
Peut-on ignorer volontairement la vérité ? \\[0pt]
Peut-on imaginer l'avenir ? \\[0pt]
Peut-on imaginer un langage universel ? \\[0pt]
Peut-on imposer la liberté ? \\[0pt]
Peut-on innover en politique ? \\[0pt]
Peut-on interpréter la nature ? \\[0pt]
Peut-on inventer en morale ? \\[0pt]
Peut-on invoquer la nature comme une excuse ? \\[0pt]
Peut-on jamais aimer son prochain ? \\[0pt]
Peut-on jamais avoir la conscience tranquille ? \\[0pt]
Peut-on juger autrui ? \\[0pt]
Peut-on juger de la valeur d'une vie humaine ? \\[0pt]
Peut-on juger des œuvres d'art sans recourir à l'idée de beauté ? \\[0pt]
Peut-on justifier la discrimination ? \\[0pt]
Peut-on justifier la guerre ? \\[0pt]
Peut-on justifier la raison d'État ? \\[0pt]
Peut-on justifier la violence ? \\[0pt]
Peut-on justifier le mal ? \\[0pt]
Peut-on justifier le mensonge ? \\[0pt]
Peut-on justifier le recours à l'idée de finalité ? \\[0pt]
Peut-on justifier ses choix ? \\[0pt]
Peut-on légitimer la violence ? \\[0pt]
Peut-on limiter l'expression de la volonté du peuple ? \\[0pt]
Peut-on lutter contre le destin ? \\[0pt]
Peut-on lutter contre soi-même ? \\[0pt]
Peut-on maîtriser la nature ? \\[0pt]
Peut-on maîtriser la technique ? \\[0pt]
Peut-on maîtriser le risque ? \\[0pt]
Peut-on maîtriser le temps ? \\[0pt]
Peut-on maîtriser l'évolution de la technique ? \\[0pt]
Peut-on maîtriser l'inconscient ? \\[0pt]
Peut-on maîtriser ses désirs ? \\[0pt]
Peut-on manipuler les esprits ? \\[0pt]
Peut-on manquer de culture ? \\[0pt]
Peut-on manquer de volonté ? \\[0pt]
Peut-on me contraindre à faire mon devoir ? \\[0pt]
Peut-on mentir par humanité ? \\[0pt]
Peut-on mesurer les phénomènes sociaux ? \\[0pt]
Peut-on mesurer le temps ? \\[0pt]
Peut-on montrer en cachant ? \\[0pt]
Peut-on moraliser la guerre ? \\[0pt]
Peut-on ne croire en rien ? \\[0pt]
Peut-on ne pas connaître son bonheur ? \\[0pt]
Peut-on ne pas croire ? \\[0pt]
Peut-on ne pas croire à la science ? \\[0pt]
Peut-on ne pas croire au progrès ? \\[0pt]
Peut-on ne pas être de son temps ? \\[0pt]
Peut-on ne pas être égoïste ? \\[0pt]
Peut-on ne pas être matérialiste ? \\[0pt]
Peut-on ne pas être soi-même ? \\[0pt]
Peut-on ne pas interpréter ? \\[0pt]
Peut-on ne pas manquer de temps ? \\[0pt]
Peut-on ne pas perdre son temps ? \\[0pt]
Peut-on ne pas rechercher le bonheur ? \\[0pt]
Peut-on ne pas savoir ce que l'on dit ? \\[0pt]
Peut-on ne pas savoir ce que l'on fait ? \\[0pt]
Peut-on ne pas savoir ce que l'on veut ? \\[0pt]
Peut-on ne pas savoir ce qu'on veut ? \\[0pt]
Peut-on ne pas vouloir être heureux ? \\[0pt]
Peut-on ne penser à rien ? \\[0pt]
Peut-on ne rien aimer ? \\[0pt]
Peut-on ne rien devoir à personne ? \\[0pt]
Peut-on ne rien vouloir ? \\[0pt]
Peut-on ne vivre qu'au présent ? \\[0pt]
Peut-on nier la réalité ? \\[0pt]
Peut-on nier le réel ? \\[0pt]
Peut-on nier l'évidence ? \\[0pt]
Peut-on nier l'existence de la matière ? \\[0pt]
Peut-on nier l'existence du temps ? \\[0pt]
Peut-on objectiver le psychisme ? \\[0pt]
Peut-on opposer connaissance scientifique et création artistique ? \\[0pt]
Peut-on opposer justice et liberté ? \\[0pt]
Peut-on opposer le loisir au travail ? \\[0pt]
Peut-on opposer morale et technique ? \\[0pt]
Peut-on opposer nature et culture ? \\[0pt]
Peut-on opposer religion et raison ? \\[0pt]
Peut-on ôter à l'homme sa liberté ? \\[0pt]
Peut-on oublier ? \\[0pt]
Peut-on oublier de vivre ? \\[0pt]
Peut-on pactiser avec un brigand ? \\[0pt]
Peut-on parler d'art primitif ? \\[0pt]
Peut-on parler de capital culturel ? \\[0pt]
Peut-on parler de ce qui n'existe pas ? \\[0pt]
Peut-on parler de conscience collective ? \\[0pt]
Peut-on parler de corruption des mœurs ? \\[0pt]
Peut-on parler de despotisme démocratique ? \\[0pt]
Peut-on parler de dialogue des cultures ? \\[0pt]
Peut-on parler de droits des animaux ? \\[0pt]
Peut-on parler de mondes imaginaires ? \\[0pt]
Peut-on parler de « nature humaine » ? \\[0pt]
Peut-on parler de nourriture spirituelle ? \\[0pt]
Peut-on parler de problèmes techniques ? \\[0pt]
Peut-on parler de progrès en art ? \\[0pt]
Peut-on parler des miracles de la technique ? \\[0pt]
Peut-on parler des œuvres d'art ? \\[0pt]
Peut-on parler de « travail intellectuel » ? \\[0pt]
Peut-on parler de travail intellectuel ? \\[0pt]
Peut-on parler de vérité en art ? \\[0pt]
Peut-on parler de vérités métaphysiques ? \\[0pt]
Peut-on parler de vérité subjective ? \\[0pt]
Peut-on parler de vérité théâtrale ? \\[0pt]
Peut-on parler de vertu politique ? \\[0pt]
Peut-on parler de violence d'État ? \\[0pt]
Peut-on parler d'un droit de la guerre ? \\[0pt]
Peut-on parler d'un droit de résistance ? \\[0pt]
Peut-on parler d'une expérience religieuse ? \\[0pt]
Peut-on parler d'une morale collective ? \\[0pt]
Peut-on parler d'une religion de l'humanité ? \\[0pt]
Peut-on parler d'une santé de l'âme ? \\[0pt]
Peut-on parler d'une science de l'art ? \\[0pt]
Peut-on parler d'univers symbolique ? \\[0pt]
Peut-on parler d'un progrès dans l'histoire ? \\[0pt]
Peut-on parler d'un progrès de la liberté ? \\[0pt]
Peut-on parler d'un progrès moral ? \\[0pt]
Peut-on parler d'un règne de la technique ? \\[0pt]
Peut-on parler d'un savoir poétique ? \\[0pt]
Peut-on parler d'un sens moral ? \\[0pt]
Peut-on parler d'un style moral ? \\[0pt]
Peut-on parler d'un travail intellectuel ? \\[0pt]
Peut-on parler pour ne rien dire ? \\[0pt]
Peut-on parler sans incohérence d'une libre obéissance ? \\[0pt]
Peut-on partager ses goûts ? \\[0pt]
Peut-on partager une expérience ? \\[0pt]
Peut-on penser ce qu'on ne peut dire ? \\[0pt]
Peut-on penser contre l'expérience ? \\[0pt]
Peut-on penser illogiquement ? \\[0pt]
Peut-on penser la création ? \\[0pt]
Peut-on penser la douleur ? \\[0pt]
Peut-on penser la fin de toute chose ? \\[0pt]
Peut-on penser la justice comme une compétence ? \\[0pt]
Peut-on penser la matière ? \\[0pt]
Peut-on penser la mort ? \\[0pt]
Peut-on penser la nouveauté ? \\[0pt]
Peut-on penser l'art comme un langage ? \\[0pt]
Peut-on penser l'art sans référence au beau ? \\[0pt]
Peut-on penser l'avenir ? \\[0pt]
Peut-on penser la vie ? \\[0pt]
Peut-on penser la vie sans penser la mort ? \\[0pt]
Peut-on penser le changement ? \\[0pt]
Peut-on penser le divin ? \\[0pt]
Peut-on penser le monde sans la technique ? \\[0pt]
Peut-on penser le réel comme un tout ? \\[0pt]
Peut-on penser le social sans la politique ? \\[0pt]
Peut-on penser le temps ? \\[0pt]
Peut-on penser le temps sans l'espace ? \\[0pt]
Peut-on penser l'extériorité ? \\[0pt]
Peut-on penser l'histoire comme progrès ? \\[0pt]
Peut-on penser l'homme à partir de la nature ? \\[0pt]
Peut-on penser l'impossible ? \\[0pt]
Peut-on penser l'infini ? \\[0pt]
Peut-on penser l'irrationnel ? \\[0pt]
Peut-on penser l'œuvre d'art sans référence à l'idée de beauté ? \\[0pt]
Peut-on penser sans concept ? \\[0pt]
Peut-on penser sans concepts ? \\[0pt]
Peut-on penser sans image ? \\[0pt]
Peut-on penser sans images ? \\[0pt]
Peut-on penser sans les mots ? \\[0pt]
Peut-on penser sans les signes ? \\[0pt]
Peut-on penser sans méthode ? \\[0pt]
Peut-on penser sans ordre ? \\[0pt]
Peut-on penser sans préjugé ? \\[0pt]
Peut-on penser sans préjugés ? \\[0pt]
Peut-on penser sans règles ? \\[0pt]
Peut-on penser sans savoir ce que l'on pense ? \\[0pt]
Peut-on penser sans savoir que l'on pense ? \\[0pt]
Peut-on penser sans signes ? \\[0pt]
Peut-on penser sans son corps ? \\[0pt]
Peut-on penser un art sans œuvres ? \\[0pt]
Peut-on penser un droit international ? \\[0pt]
Peut-on penser une conscience sans mémoire ? \\[0pt]
Peut-on penser une conscience sans objet ? \\[0pt]
Peut-on penser une fin de l'histoire ? \\[0pt]
Peut-on penser une métaphysique sans Dieu ? \\[0pt]
Peut-on penser une religion sans le recours au divin ? \\[0pt]
Peut-on penser une société sans État ? \\[0pt]
Peut-on penser un État sans violence ? \\[0pt]
Peut-on penser une volonté diabolique ? \\[0pt]
Peut-on penser un pouvoir absolu ? \\[0pt]
Peut-on percevoir le temps ? \\[0pt]
Peut-on percevoir sans juger ? \\[0pt]
Peut-on percevoir sans s'en apercevoir ? \\[0pt]
Peut-on perdre la raison ? \\[0pt]
Peut-on perdre sa dignité ? \\[0pt]
Peut-on perdre sa liberté ? \\[0pt]
Peut-on perdre son identité ? \\[0pt]
Peut-on perdre son temps ? \\[0pt]
Peut-on permettre le mensonge ? \\[0pt]
Peut-on préconiser, dans les sciences humaines et sociales, l'imitation des sciences de la nature ? \\[0pt]
Peut-on prédire les événements ? \\[0pt]
Peut-on prédire l'histoire ? \\[0pt]
Peut-on préférer le bonheur à la vérité ? \\[0pt]
Peut-on préférer l'injustice au désordre ? \\[0pt]
Peut-on préférer l'ordre à la justice ? \\[0pt]
Peut-on prendre le risque de donner la mort en voulant soulager la souffrance ? \\[0pt]
Peut-on prendre les moyens pour la fin ? \\[0pt]
Peut-on prévoir l'avenir ? \\[0pt]
Peut-on prévoir le futur ? \\[0pt]
Peut-on prévoir les hommes ? \\[0pt]
Peut-on promettre le bonheur ? \\[0pt]
Peut-on protéger les libertés sans les réduire ? \\[0pt]
Peut-on prouver la réalité de l'esprit ? \\[0pt]
Peut-on prouver l'existence ? \\[0pt]
Peut-on prouver l'existence de Dieu ? \\[0pt]
Peut-on prouver l'existence de l'inconscient ? \\[0pt]
Peut-on prouver l'existence du monde ? \\[0pt]
Peut-on prouver une existence ? \\[0pt]
Peut-on raconter sa vie ? \\[0pt]
Peut-on raisonner sans règles ? \\[0pt]
Peut-on ralentir la course du temps ? \\[0pt]
Peut-on recommencer sa vie ? \\[0pt]
Peut-on reconnaître un sens à l'histoire sans lui assigner une fin ? \\[0pt]
Peut-on réduire la pensée à une espèce de comportement ? \\[0pt]
Peut-on réduire la vie à la matière ? \\[0pt]
Peut-on réduire le raisonnement au calcul ? \\[0pt]
Peut-on réduire l'esprit à la matière ? \\[0pt]
Peut-on réduire une métaphysique à une conception du monde ? \\[0pt]
Peut-on réduire un homme à la somme de ses actes ? \\[0pt]
Peut-on refaire le monde ? \\[0pt]
Peut-on refuser de voir la vérité ? \\[0pt]
Peut-on refuser la loi ? \\[0pt]
Peut-on refuser la vérité ? \\[0pt]
Peut-on refuser la violence ? \\[0pt]
Peut-on refuser le bonheur ? \\[0pt]
Peut-on refuser l'évidence ? \\[0pt]
Peut-on refuser le vrai ? \\[0pt]
Peut-on réfuter le scepticisme ? \\[0pt]
Peut-on régner innocemment ? \\[0pt]
Peut-on rejeter le principe de non-contradiction ? \\[0pt]
Peut-on rendre raison des émotions ? \\[0pt]
Peut-on rendre raison de tout ? \\[0pt]
Peut-on rendre raison du réel ? \\[0pt]
Peut-on renoncer à comprendre ? \\[0pt]
Peut-on renoncer à être libre ? \\[0pt]
Peut-on renoncer à la liberté ? \\[0pt]
Peut-on renoncer à la vérité ? \\[0pt]
Peut-on renoncer à sa liberté ? \\[0pt]
Peut-on renoncer à ses droits ? \\[0pt]
Peut-on renoncer à soi ? \\[0pt]
Peut-on renoncer au bonheur ? \\[0pt]
Peut-on réparer le vivant ? \\[0pt]
Peut-on réparer une injustice ? \\[0pt]
Peut-on répondre au sceptique ? \\[0pt]
Peut-on répondre d'autrui ? \\[0pt]
Peut-on représenter le peuple ? \\[0pt]
Peut-on représenter l'espace ? \\[0pt]
Peut-on représenter l'invisible ? \\[0pt]
Peut-on reprocher à la morale d'être abstraite ? \\[0pt]
Peut-on reprocher au langage d'être équivoque ? \\[0pt]
Peut-on reprocher au langage d'être parfait ? \\[0pt]
Peut-on résister à la vérité ? \\[0pt]
Peut-on résister au vrai ? \\[0pt]
Peut-on rester dans le doute ? \\[0pt]
Peut-on rester insensible à la beauté ? \\[0pt]
Peut-on rester sceptique ? \\[0pt]
Peut-on restreindre la logique à la pensée formelle ? \\[0pt]
Peut-on résumer une philosophie ? \\[0pt]
Peut-on retenir le temps ? \\[0pt]
Peut-on réunir des arts différents dans une même œuvre ? \\[0pt]
Peut-on revendiquer la paix comme un droit ? \\[0pt]
Peut-on revendiquer le droit de désobéir ? \\[0pt]
Peut-on revenir sur ses erreurs ? \\[0pt]
Peut-on rire de tout ? \\[0pt]
Peut-on rompre avec la société ? \\[0pt]
Peut-on rompre avec le passé ? \\[0pt]
Peut-on s'abstenir de penser politiquement ? \\[0pt]
Peut-on s'accorder sur des vérités morales ? \\[0pt]
Peut-on s'affranchir de la subjectivité ? \\[0pt]
Peut-on s'affranchir de sa propre nature ? \\[0pt]
Peut-on s'affranchir des lois ? \\[0pt]
Peut-on saisir le temps ? \\[0pt]
Peut-on s'attendre à tout ? \\[0pt]
Peut-on savoir ce que l'on fait ? \\[0pt]
Peut-on savoir ce qui est bien ? \\[0pt]
Peut-on savoir quelque chose de l'avenir ? \\[0pt]
Peut-on savoir sans croire ? \\[0pt]
Peut-on se choisir un destin ? \\[0pt]
Peut-on se connaître soi-même ? \\[0pt]
Peut-on se désintéresser de la politique ? \\[0pt]
Peut-on se désintéresser de son bonheur ? \\[0pt]
Peut-on se duper soi-même ? \\[0pt]
Peut-on se faire une idée de tout ? \\[0pt]
Peut-on se fier à la technique ? \\[0pt]
Peut-on se fier à l'expérience vécue ? \\[0pt]
Peut-on se fier à l'intuition ? \\[0pt]
Peut-on se fier à sa propre raison ? \\[0pt]
Peut-on se fier à son intuition ? \\[0pt]
Peut-on se fier aux apparences ? \\[0pt]
Peut-on se gouverner soi-même ? \\[0pt]
Peut-on se méfier de soi-même ? \\[0pt]
Peut-on se mentir à soi-même ? \\[0pt]
Peut-on se mettre à la place d'autrui ? \\[0pt]
Peut-on se mettre à la place de l'autre ? \\[0pt]
Peut-on se mettre à la place des autres ? \\[0pt]
Peut-on s'en tenir au présent ? \\[0pt]
Peut-on sentir le temps qui passe ? \\[0pt]
Peut-on séparer la politique de l'exigence de vérité ? \\[0pt]
Peut-on séparer l'homme et l'œuvre ? \\[0pt]
Peut-on séparer politique et économie ? \\[0pt]
Peut-on se passer de chef ? \\[0pt]
Peut-on se passer de croire ? \\[0pt]
Peut-on se passer de croyance ? \\[0pt]
Peut-on se passer de croyances ? \\[0pt]
Peut-on se passer de Dieu ? \\[0pt]
Peut-on se passer de frontières ? \\[0pt]
Peut-on se passer de la religion ? \\[0pt]
Peut-on se passer de la technique ? \\[0pt]
Peut-on se passer de l'État ? \\[0pt]
Peut-on se passer de l'idée de cause finale ? \\[0pt]
Peut-on se passer de l'idée de Dieu ? \\[0pt]
Peut-on se passer de l'idée de droit naturel ? \\[0pt]
Peut-on se passer de l'idée de finalité ? \\[0pt]
Peut-on se passer de l'idée de nature humaine ? \\[0pt]
Peut-on se passer de maître ? \\[0pt]
Peut-on se passer de métaphysique ? \\[0pt]
Peut-on se passer de méthode ? \\[0pt]
Peut-on se passer de mythes ? \\[0pt]
Peut-on se passer de principes ? \\[0pt]
Peut-on se passer de religion ? \\[0pt]
Peut-on se passer de représentants ? \\[0pt]
Peut-on se passer des causes finales ? \\[0pt]
Peut-on se passer de spiritualité ? \\[0pt]
Peut-on se passer des relations ? \\[0pt]
Peut-on se passer d'État ? \\[0pt]
Peut-on se passer de technique ? \\[0pt]
Peut-on se passer de techniques de raisonnement ? \\[0pt]
Peut-on se passer de toute religion ? \\[0pt]
Peut-on se passer d'idéal ? \\[0pt]
Peut-on se passer d'ontologie ? \\[0pt]
Peut-on se passer d'un maître ? \\[0pt]
Peut-on se peindre soi-même ? \\[0pt]
Peut-on se prescrire une loi ? \\[0pt]
Peut-on se promettre quelque chose à soi-même ? \\[0pt]
Peut-on se punir soi-même ? \\[0pt]
Peut-on se régler sur des exemples en politique ? \\[0pt]
Peut-on se rendre maître de la technique ? \\[0pt]
Peut-on se retirer du monde ? \\[0pt]
Peut-on servir deux maîtres à la fois ? \\[0pt]
Peut-on se soustraire à son devoir ? \\[0pt]
Peut-on se tromper en se croyant heureux ? \\[0pt]
Peut-on se tromper sur son devoir ? \\[0pt]
Peut-on se vouloir parfait ? \\[0pt]
Peut-on s'exprimer par le silence ? \\[0pt]
Peut-on s'obliger soi-même ? \\[0pt]
Peut-on sortir de la subjectivité ? \\[0pt]
Peut-on sortir de sa conscience ? \\[0pt]
Peut-on sortir de sa culture ? \\[0pt]
Peut-on s'oublier ? \\[0pt]
Peut-on souhaiter le gouvernement des meilleurs ? \\[0pt]
Peut-on souhaiter ne pas être libre ? \\[0pt]
Peut-on suivre une règle ? \\[0pt]
Peut-on suspendre le temps ? \\[0pt]
Peut-on suspendre son jugement ? \\[0pt]
Peut-on sympathiser avec l'ennemi ? \\[0pt]
Peut-on tirer des leçons de l'histoire ? \\[0pt]
Peut-on tolérer l'injustice ? \\[0pt]
Peut-on toujours faire ce qu'on doit ? \\[0pt]
Peut-on toujours raison garder ? \\[0pt]
Peut-on toujours savoir entièrement ce que l'on dit ? \\[0pt]
Peut-on tout analyser ? \\[0pt]
Peut-on tout attendre de l'État ? \\[0pt]
Peut-on tout définir ? \\[0pt]
Peut-on tout démontrer ? \\[0pt]
Peut-on tout désirer ? \\[0pt]
Peut-on tout dire ? \\[0pt]
Peut-on tout donner ? \\[0pt]
Peut-on tout échanger ? \\[0pt]
Peut-on tout enseigner ? \\[0pt]
Peut-on tout expliquer ? \\[0pt]
Peut-on tout exprimer ? \\[0pt]
Peut-on tout imaginer ? \\[0pt]
Peut-on tout imiter ? \\[0pt]
Peut-on tout interpréter ? \\[0pt]
Peut-on tout justifier ? \\[0pt]
Peut-on tout mathématiser ? \\[0pt]
Peut-on tout mesurer ? \\[0pt]
Peut-on tout ordonner ? \\[0pt]
Peut-on tout pardonner ? \\[0pt]
Peut-on tout partager ? \\[0pt]
Peut-on tout prévoir ? \\[0pt]
Peut-on tout prouver ? \\[0pt]
Peut-on tout soumettre à la discussion ? \\[0pt]
Peut-on tout tolérer ? \\[0pt]
Peut-on traiter autrui comme un moyen ? \\[0pt]
Peut-on traiter un être vivant comme une machine ? \\[0pt]
Peut-on trancher les conflits entre interprétations ? \\[0pt]
Peut-on transformer le réel ? \\[0pt]
Peut-on transiger avec les principes ? \\[0pt]
Peut-on trop penser ? \\[0pt]
Peut-on trouver du plaisir à l'ennui ? \\[0pt]
Peut-on vivre avec les autres ? \\[0pt]
Peut-on vivre dans le doute ? \\[0pt]
Peut-on vivre en dehors de l'histoire ? \\[0pt]
Peut-on vivre en marge de la société ? \\[0pt]
Peut-on vivre en paix avec son inconscient ? \\[0pt]
Peut-on vivre en sceptique ? \\[0pt]
Peut-on vivre hors du temps ? \\[0pt]
Peut-on vivre pour la vérité ? \\[0pt]
Peut-on vivre sans aimer ? \\[0pt]
Peut-on vivre sans art ? \\[0pt]
Peut-on vivre sans aucune certitude ? \\[0pt]
Peut-on vivre sans croyance ? \\[0pt]
Peut-on vivre sans croyances ? \\[0pt]
Peut-on vivre sans désir ? \\[0pt]
Peut-on vivre sans échange ? \\[0pt]
Peut-on vivre sans foi ni loi ? \\[0pt]
Peut-on vivre sans illusions ? \\[0pt]
Peut-on vivre sans l'art ? \\[0pt]
Peut-on vivre sans le plaisir de vivre ? \\[0pt]
Peut-on vivre sans lois ? \\[0pt]
Peut-on vivre sans opinions ? \\[0pt]
Peut-on vivre sans passion ? \\[0pt]
Peut-on vivre sans peur ? \\[0pt]
Peut-on vivre sans principes ? \\[0pt]
Peut-on vivre sans réfléchir ? \\[0pt]
Peut-on vivre sans ressentiment ? \\[0pt]
Peut-on vivre sans rien espérer ? \\[0pt]
Peut-on vivre sans sacré ? \\[0pt]
Peut-on vivre sans travailler ? \\[0pt]
Peut-on vivre sans valeurs ? \\[0pt]
Peut-on voir sans croire ? \\[0pt]
Peut-on vouloir ce qu'on ne désire pas ? \\[0pt]
Peut-on vouloir le bien sans le faire ? \\[0pt]
Peut-on vouloir le bonheur d'autrui ? \\[0pt]
Peut-on vouloir le mal ? \\[0pt]
Peut-on vouloir le mal pour le mal ? \\[0pt]
Peut-on vouloir le mal sachant que c'est le mal ? \\[0pt]
Peut-on vouloir l'impossible ? \\[0pt]
Peut-on vouloir ne pas être libre ? \\[0pt]
Peut-on vouloir sans désirer ? \\[0pt]
Peut-on vraiment créer ? \\[0pt]
Peut-on vraiment dire ce qu'on pense ? \\[0pt]
Peut-on vraiment parler de soi ? \\[0pt]
Peut-on vraiment tirer des leçons du passé ? \\[0pt]
Phénomène ou fait social ? \\[0pt]
Philosopher, est-ce apprendre à vivre ? \\[0pt]
Philosophe-t-on pour être heureux ? \\[0pt]
Plusieurs religions valent-elles mieux qu'une seule ? \\[0pt]
Pour agir moralement, faut-il ne pas se soucier de soi ? \\[0pt]
Pour apprécier une œuvre, faut-il être cultivé ? \\[0pt]
Pour bien agir, faut-il vouloir le bien d'autrui ? \\[0pt]
Pour bien penser, faut-il renoncer au langage courant ? \\[0pt]
Pour connaître, suffit-il de démontrer ? \\[0pt]
Pour dialoguer faut-il parler le même langage ? \\[0pt]
Pour être heureux, faut-il ne rien désirer ? \\[0pt]
Pour être heureux, faut-il renoncer à la perfection ? \\[0pt]
Pour être homme, faut-il être citoyen ? \\[0pt]
Pour être libre, faut-il renoncer à être heureux ? \\[0pt]
Pour être un bon observateur faut-il être un bon théoricien ? \\[0pt]
Pour juger, faut-il seulement apprendre à raisonner ? \\[0pt]
Pour penser rigoureusement, faut-il renoncer au langage ordinaire ? \\[0pt]
Pour qui se prend-on ? \\[0pt]
Pourquoi ? \\[0pt]
Pourquoi accomplir son devoir ? \\[0pt]
Pourquoi aimer la liberté ? \\[0pt]
Pourquoi aimons-nous la musique ? \\[0pt]
Pourquoi aller contre son désir ? \\[0pt]
Pourquoi a-t-on peur de la folie ? \\[0pt]
Pourquoi avoir recours à la notion d'inconscient ? \\[0pt]
Pourquoi avons-nous besoin des autres pour être heureux ? \\[0pt]
Pourquoi avons-nous du mal à reconnaître la vérité ? \\[0pt]
Pourquoi avons-nous tant de mal à renoncer à nos illusions ? \\[0pt]
Pourquoi châtier ? \\[0pt]
Pourquoi chercher à connaître le passé ? \\[0pt]
Pourquoi chercher à se connaître soi-même ? \\[0pt]
Pourquoi chercher à se distinguer ? \\[0pt]
Pourquoi chercher à vivre libre ? \\[0pt]
Pourquoi chercher la vérité ? \\[0pt]
Pourquoi chercher un sens à l'histoire ? \\[0pt]
Pourquoi cherche-t-on à connaître ? \\[0pt]
Pourquoi commémorer ? \\[0pt]
Pourquoi communiquer ? \\[0pt]
Pourquoi conserver des œuvres d'art ? \\[0pt]
Pourquoi conserver les œuvres d'art ? \\[0pt]
Pourquoi construire des monuments ? \\[0pt]
Pourquoi critiquer la raison ? \\[0pt]
Pourquoi critiquer le conformisme ? \\[0pt]
Pourquoi croyons-nous ? \\[0pt]
Pourquoi défendre le faible ? \\[0pt]
Pourquoi définir ? \\[0pt]
Pourquoi délibérer ? \\[0pt]
Pourquoi démontrer ? \\[0pt]
Pourquoi démontrer ce que l'on sait être vrai ? \\[0pt]
Pourquoi des artifices ? \\[0pt]
Pourquoi des artistes ? \\[0pt]
Pourquoi des cérémonies ? \\[0pt]
Pourquoi des châtiments ? \\[0pt]
Pourquoi des classifications ? \\[0pt]
Pourquoi des comités d'éthique ? \\[0pt]
Pourquoi des conflits ? \\[0pt]
Pourquoi des constitutions ? \\[0pt]
Pourquoi des corps intermédiaires ? \\[0pt]
Pourquoi des définitions ? \\[0pt]
Pourquoi des devoirs ? \\[0pt]
Pourquoi des élections ? \\[0pt]
Pourquoi des exemples ? \\[0pt]
Pourquoi des fêtes ? \\[0pt]
Pourquoi des fictions ? \\[0pt]
Pourquoi des géométries ? \\[0pt]
Pourquoi des guerres ? \\[0pt]
Pourquoi des historiens ? \\[0pt]
Pourquoi des hommes politiques ? \\[0pt]
Pourquoi des hypothèses ? \\[0pt]
Pourquoi des idoles ? \\[0pt]
Pourquoi des institutions ? \\[0pt]
Pourquoi des interdits ? \\[0pt]
Pourquoi désirer ? \\[0pt]
Pourquoi désirer la sagesse ? \\[0pt]
Pourquoi désirer l'immortalité ? \\[0pt]
Pourquoi désire-t-on ce dont on n'a nul besoin ? \\[0pt]
Pourquoi désirons-nous ? \\[0pt]
Pourquoi des logiciens ? \\[0pt]
Pourquoi des lois ? \\[0pt]
Pourquoi des maîtres ? \\[0pt]
Pourquoi des métaphores ? \\[0pt]
Pourquoi des modèles ? \\[0pt]
Pourquoi des musées ? \\[0pt]
Pourquoi des œuvres d'art ? \\[0pt]
Pourquoi des philosophes ? \\[0pt]
Pourquoi des poètes ? \\[0pt]
Pourquoi des psychologues ? \\[0pt]
Pourquoi des religions ? \\[0pt]
Pourquoi des rites ? \\[0pt]
Pourquoi des sociologues ? \\[0pt]
Pourquoi des syllogismes ? \\[0pt]
Pourquoi des symboles ? \\[0pt]
Pourquoi des traditions ? \\[0pt]
Pourquoi des utopies ? \\[0pt]
Pourquoi dialogue-t-on ? \\[0pt]
Pourquoi Dieu se soucierait-il des affaires humaines ? \\[0pt]
Pourquoi dire la vérité ? \\[0pt]
Pourquoi distinguer nature et culture ? \\[0pt]
Pourquoi domestiquer ? \\[0pt]
Pourquoi donner ? \\[0pt]
Pourquoi donner des leçons de morale ? \\[0pt]
Pourquoi douter ? \\[0pt]
Pourquoi échanger des idées ? \\[0pt]
Pourquoi écrire ? \\[0pt]
Pourquoi écrit-on ? \\[0pt]
Pourquoi écrit-on des lois ? \\[0pt]
Pourquoi écrit-on les lois ? \\[0pt]
Pourquoi écrit-on l'Histoire ? \\[0pt]
Pourquoi est-il difficile de rectifier une erreur ? \\[0pt]
Pourquoi être exigeant ? \\[0pt]
Pourquoi être moral ? \\[0pt]
Pourquoi être raisonnable ? \\[0pt]
Pourquoi étudier le vivant ? \\[0pt]
Pourquoi étudier l'Histoire ? \\[0pt]
Pourquoi exposer les œuvres d'art ? \\[0pt]
Pourquoi faire confiance ? \\[0pt]
Pourquoi faire confiance au dialogue ? \\[0pt]
Pourquoi faire de la métaphysique ? \\[0pt]
Pourquoi faire de la politique ? \\[0pt]
Pourquoi faire de l'histoire ? \\[0pt]
Pourquoi faire la guerre ? \\[0pt]
Pourquoi faire l'hypothèse de l'inconscient ? \\[0pt]
Pourquoi faire son devoir ? \\[0pt]
Pourquoi fait-on le mal ? \\[0pt]
Pourquoi faudrait-il avoir peur de la technique ? \\[0pt]
Pourquoi faudrait-il être cohérent ? \\[0pt]
Pourquoi faudrait-il sauver les apparences ? \\[0pt]
Pourquoi faut-il agir ? \\[0pt]
Pourquoi faut-il diviser le travail ? \\[0pt]
Pourquoi faut-il être cohérent ? \\[0pt]
Pourquoi faut-il être juste ? \\[0pt]
Pourquoi faut-il être poli ? \\[0pt]
Pourquoi faut-il travailler ? \\[0pt]
Pourquoi formaliser des arguments ? \\[0pt]
Pourquoi il n'y a pas de société sans art ? \\[0pt]
Pourquoi imiter ? \\[0pt]
Pourquoi interprète-t-on ? \\[0pt]
Pourquoi joue-t-on ? \\[0pt]
Pourquoi la critique ? \\[0pt]
Pourquoi la curiosité est-elle un vilain défaut ? \\[0pt]
Pourquoi la guerre ? \\[0pt]
Pourquoi la justice a-t-elle besoin d'institutions ? \\[0pt]
Pourquoi la musique intéresse-t-elle le philosophe ? \\[0pt]
Pourquoi la prison ? \\[0pt]
Pourquoi la prohibition de l'inceste ? \\[0pt]
Pourquoi la raison recourt-elle à l'hypothèse ? \\[0pt]
Pourquoi la réalité peut-elle dépasser la fiction ? \\[0pt]
Pourquoi l'art intéresse-t-il les philosophes ? \\[0pt]
Pourquoi l'économie est-elle politique ? \\[0pt]
Pourquoi le droit international est-il imparfait ? \\[0pt]
Pourquoi les droits de l'homme sont-ils universels ? \\[0pt]
Pourquoi les États se font-ils la guerre ? \\[0pt]
Pourquoi les hommes jouent-ils ? \\[0pt]
Pourquoi les hommes mentent-ils ? \\[0pt]
Pourquoi les hommes se soumettent-ils à l'autorité ? \\[0pt]
Pourquoi les mathématiques s'appliquent-elles à la réalité ? \\[0pt]
Pourquoi les œuvres d'art résistent-elles au temps ? \\[0pt]
Pourquoi le sport ? \\[0pt]
Pourquoi les sciences humaines s'intéressent-elles à la parenté ? \\[0pt]
Pourquoi les sciences ont-elles une histoire ? \\[0pt]
Pourquoi les sociétés ont-elles besoin de lois ? \\[0pt]
Pourquoi l'État ? \\[0pt]
Pourquoi le théâtre ? \\[0pt]
Pourquoi l'ethnologue s'intéresse-t-il à la vie urbaine ? \\[0pt]
Pourquoi l'homme a-t-il des droits ? \\[0pt]
Pourquoi l'homme est-il l'objet de plusieurs sciences ? \\[0pt]
Pourquoi l'Homme se pose-t-il des questions métaphysiques ? \\[0pt]
Pourquoi l'homme travaille-t-il ? \\[0pt]
Pourquoi lire des romans ? \\[0pt]
Pourquoi lire les poètes ? \\[0pt]
Pourquoi lit-on des romans ? \\[0pt]
Pourquoi mentir ? \\[0pt]
Pourquoi ne peut-on concevoir la science comme achevée ? \\[0pt]
Pourquoi ne s'entend-on pas sur la nature de ce qui est réel ? \\[0pt]
Pourquoi nous racontons-nous des histoires ? \\[0pt]
Pourquoi nous soucier du sort des générations futures ? \\[0pt]
Pourquoi nous souvenons-nous ? \\[0pt]
Pourquoi nous trompons-nous ? \\[0pt]
Pourquoi n'y aurait-il pas de sots métiers ? \\[0pt]
Pourquoi obéir ? \\[0pt]
Pourquoi obéir aux lois ? \\[0pt]
Pourquoi obéit-on ? \\[0pt]
Pourquoi obéit-on aux lois ? \\[0pt]
Pourquoi oublie-t-on ? \\[0pt]
Pourquoi pardonner ? \\[0pt]
Pourquoi parler de fautes de goût ? \\[0pt]
Pourquoi parler de jugement de goût ? \\[0pt]
Pourquoi parler de « sciences exactes » ? \\[0pt]
Pourquoi parler du travail comme d'un droit ? \\[0pt]
Pourquoi parle-t-on ? \\[0pt]
Pourquoi parle-t-on d'économie politique ? \\[0pt]
Pourquoi parle-t-on d'une « société civile » ? \\[0pt]
Pourquoi parlons-nous ? \\[0pt]
Pourquoi pas ? \\[0pt]
Pourquoi pas plusieurs dieux ? \\[0pt]
Pourquoi peindre la nature ? \\[0pt]
Pourquoi penser à la mort ? \\[0pt]
Pourquoi penser l'impossible ? \\[0pt]
Pourquoi pensons-nous ? \\[0pt]
Pourquoi philosopher ? \\[0pt]
Pourquoi pleure-t-on ? \\[0pt]
Pourquoi pleure-t-on au cinéma ? \\[0pt]
Pourquoi plusieurs sciences ? \\[0pt]
Pourquoi préférer l'original ? \\[0pt]
Pourquoi préférer l'original à la copie ? \\[0pt]
Pourquoi préférer l'original à la reproduction ? \\[0pt]
Pourquoi préférer l'original à sa reproduction ? \\[0pt]
Pourquoi préserver la nature ? \\[0pt]
Pourquoi préserver l'environnement ? \\[0pt]
Pourquoi prier ? \\[0pt]
Pourquoi promettre ? \\[0pt]
Pourquoi prouver l'existence de Dieu ? \\[0pt]
Pourquoi punir ? \\[0pt]
Pourquoi punit-on ? \\[0pt]
Pourquoi raconter des histoires ? \\[0pt]
Pourquoi rechercher la vérité ? \\[0pt]
Pourquoi rechercher le bonheur ? \\[0pt]
Pourquoi refuser de faire son devoir ? \\[0pt]
Pourquoi refuse-t-on la conscience à l'animal ? \\[0pt]
Pourquoi respecter autrui ? \\[0pt]
Pourquoi respecter la nature ? \\[0pt]
Pourquoi respecter le droit ? \\[0pt]
Pourquoi respecter les anciens ? \\[0pt]
Pourquoi rit-on ? \\[0pt]
Pourquoi sauver les apparences ? \\[0pt]
Pourquoi sauver les phénomènes ? \\[0pt]
Pourquoi se confesser ? \\[0pt]
Pourquoi se cultiver ? \\[0pt]
Pourquoi se divertir ? \\[0pt]
Pourquoi se fier à autrui ? \\[0pt]
Pourquoi se mettre à la place d'autrui ? \\[0pt]
Pourquoi s'ennuie-t-on ? \\[0pt]
Pourquoi séparer les pouvoirs ? \\[0pt]
Pourquoi se référer au destin ? \\[0pt]
Pourquoi se révolter ? \\[0pt]
Pourquoi se soucier du futur ? \\[0pt]
Pourquoi se taire ? \\[0pt]
Pourquoi s'étonner ? \\[0pt]
Pourquoi s'exprimer ? \\[0pt]
Pourquoi s'inspirer de l'art antique ? \\[0pt]
Pourquoi s'intéresser à l'histoire ? \\[0pt]
Pourquoi s'intéresser à l'origine ? \\[0pt]
Pourquoi s'interroger sur l'origine du langage ? \\[0pt]
Pourquoi soigner son apparence ? \\[0pt]
Pourquoi sommes-nous déçus par les œuvres d'un faussaire ? \\[0pt]
Pourquoi sommes-nous des êtres moraux ? \\[0pt]
Pourquoi sommes-nous moraux ? \\[0pt]
Pourquoi suivre l'actualité ? \\[0pt]
Pourquoi suspendre son jugement ? \\[0pt]
Pourquoi tenir ses promesses ? \\[0pt]
Pourquoi théoriser ? \\[0pt]
Pourquoi transformer le monde ? \\[0pt]
Pourquoi transmettre ? \\[0pt]
Pourquoi travailler ? \\[0pt]
Pourquoi travaille-t-on ? \\[0pt]
Pourquoi un droit du travail ? \\[0pt]
Pourquoi une instruction publique ? \\[0pt]
Pourquoi un fait devrait-il être établi ? \\[0pt]
Pourquoi veut-on changer le monde ? \\[0pt]
Pourquoi veut-on la vérité ? \\[0pt]
Pourquoi vivons-nous ? \\[0pt]
Pourquoi vivre ensemble ? \\[0pt]
Pourquoi vouloir avoir raison ? \\[0pt]
Pourquoi vouloir devenir « comme maîtres et possesseurs de la nature » ? \\[0pt]
Pourquoi vouloir être libre ? \\[0pt]
Pourquoi vouloir gagner ? \\[0pt]
Pourquoi vouloir la vérité ? \\[0pt]
Pourquoi vouloir s'abstraire du quotidien ? \\[0pt]
Pourquoi vouloir se connaître ? \\[0pt]
Pourquoi voulons-nous savoir ? \\[0pt]
Pourquoi voyager ? \\[0pt]
Pourquoi y a-t-il des conflits insolubles ? \\[0pt]
Pourquoi y a-t-il des institutions ? \\[0pt]
Pourquoi y a-t-il des lois ? \\[0pt]
Pourquoi y a-t-il des religions ? \\[0pt]
Pourquoi y a-t-il du mal dans le monde ? \\[0pt]
Pourquoi y a-t-il plusieurs façons de démontrer ? \\[0pt]
Pourquoi y a-t-il plusieurs langues ? \\[0pt]
Pourquoi y a-t-il plusieurs philosophies ? \\[0pt]
Pourquoi y a-t-il plusieurs sciences ? \\[0pt]
Pourquoi y a-t-il quelque chose plutôt que rien ? \\[0pt]
Pourquoi y a-t-il une philosophie de la vie ? \\[0pt]
Pourrait-on se passer de l'argent ? \\[0pt]
Pourrait-on vivre sans art ? \\[0pt]
Pourrait-on vivre sans idéal ? \\[0pt]
Pourrions-nous comprendre une pensée non humaine ? \\[0pt]
Pourrions-nous nous passer des musées ? \\[0pt]
Pourrions-nous vivre sans religion ? \\[0pt]
Pour s'aimer, faut-il se connaître ? \\[0pt]
Pour se trouver, faut-il savoir se perdre ? \\[0pt]
Pouvons-nous communiquer ce que nous sentons ? \\[0pt]
Pouvons-nous connaître sans interpréter ? \\[0pt]
Pouvons-nous désirer ce qui nous nuit ? \\[0pt]
Pouvons-nous devenir meilleurs ? \\[0pt]
Pouvons-nous dissocier le réel de nos interprétations ? \\[0pt]
Pouvons-nous être certains que nous ne rêvons pas ? \\[0pt]
Pouvons-nous être objectifs ? \\[0pt]
Pouvons-nous faire l'expérience de la liberté ? \\[0pt]
Pouvons-nous justifier nos croyances ? \\[0pt]
Pouvons-nous réellement tirer des leçons du passé ? \\[0pt]
Pouvons-nous savoir ce que nous ignorons ? \\[0pt]
Pouvons-nous vivre sans préjugés ? \\[0pt]
Prendre la parole, est-ce prendre le pouvoir ? \\[0pt]
Prendre son temps, est-ce le perdre ? \\[0pt]
Primitif ou premier ? \\[0pt]
Prince : nature ou fonction ? \\[0pt]
Promettre, est-ce renoncer à sa liberté ? \\[0pt]
Puis-je aimer tous les hommes ? \\[0pt]
Puis-je avoir la certitude que mes choix sont libres ? \\[0pt]
Puis-je comprendre autrui ? \\[0pt]
Puis-je décider de croire ? \\[0pt]
Puis-je devenir moi-même ? \\[0pt]
Puis-je devenir un autre ? \\[0pt]
Puis-je dire « ceci est mon corps » ? \\[0pt]
Puis-je douter de ma propre existence ? \\[0pt]
Puis-je être dans le vrai sans le savoir ? \\[0pt]
Puis-je être heureux dans un monde chaotique ? \\[0pt]
Puis-je être libre sans être responsable ? \\[0pt]
Puis-je être libre tout seul ? \\[0pt]
Puis-je être sûr de bien agir ? \\[0pt]
Puis-je être sûr de ne pas me tromper ? \\[0pt]
Puis-je être sûr que je ne rêve pas ? \\[0pt]
Puis-je être universel ? \\[0pt]
Puis-je faire ce que je veux de mon corps ? \\[0pt]
Puis-je faire confiance à mes sens ? \\[0pt]
Puis-je invoquer l'inconscient sans ruiner la morale ? \\[0pt]
Puis-je juger une culture à laquelle je n'appartiens pas ? \\[0pt]
Puis-je me connaître sans l'aide des autres ? \\[0pt]
Puis-je me connaître si je change à travers le temps ? \\[0pt]
Puis-je me mettre à la place d'un autre ? \\[0pt]
Puis-je me passer d'imiter autrui ? \\[0pt]
Puis-je me trouver moi-même sans l'aide d'autrui ? \\[0pt]
Puis-je ne croire que ce que je vois ? \\[0pt]
Puis-je ne pas vouloir ce que je désire ? \\[0pt]
Puis-je ne rien croire ? \\[0pt]
Puis-je ne rien devoir à personne ? \\[0pt]
Puis-je répondre des autres ? \\[0pt]
Puis-je savoir avec assurance en quoi consiste mon devoir ? \\[0pt]
Puis-je savoir ce qui m'est propre ? \\[0pt]
Punir ou soigner ? \\[0pt]
Qu'admire-t-on dans une œuvre d'art ? \\[0pt]
Qu'ai-je le droit d'exiger d'autrui ? \\[0pt]
Qu'ai-je le droit d'exiger des autres ? \\[0pt]
Qu'aime-t-on ? \\[0pt]
Qu'aime-t-on dans l'amour ? \\[0pt]
Qu'aime-t-on quand on aime ? \\[0pt]
Qu'aime-t-on quand on aime une œuvre d'art ? \\[0pt]
Qu'aimons-nous dans l'amour ? \\[0pt]
Quand agit-on ? \\[0pt]
Quand est-on stupide ? \\[0pt]
Quand faut-il désobéir ? \\[0pt]
Quand faut-il désobéir aux lois ? \\[0pt]
Quand faut-il mentir ? \\[0pt]
Quand faut-il se taire ? \\[0pt]
Quand la guerre finira-t-elle ? \\[0pt]
Quand l'art est-il abstrait ? \\[0pt]
Quand la technique devient-elle art ? \\[0pt]
Quand le temps passe, que reste-t-il ? \\[0pt]
Quand manquons-nous à nos devoirs ? \\[0pt]
Quand pense-t-on ? \\[0pt]
Quand peut-on se passer de théories ? \\[0pt]
Quand suis-je en faute ? \\[0pt]
Quand suis-je moi-même ? \\[0pt]
Quand une autorité est-elle légitime ? \\[0pt]
Quand y a-t-il art ? \\[0pt]
Quand y a-t-il de l'art ? \\[0pt]
Quand y a-t-il métaphore ? \\[0pt]
Quand y a-t-il œuvre ? \\[0pt]
Quand y a-t-il paysage ? \\[0pt]
Quand y a-t-il peuple ? \\[0pt]
Qu'anticipent les romans d'anticipation ? \\[0pt]
Qu'a perdu le fou ? \\[0pt]
Qu'appelle-t-on chef-d'œuvre ? \\[0pt]
Qu'appelle-t-on destin ? \\[0pt]
Qu'appelle-t-on penser ? \\[0pt]
Qu'apporte au mathématicien l'histoire des mathématiques ? \\[0pt]
Qu'apporte la photographie aux arts ? \\[0pt]
Qu'apporte l'histoire à la vie politique ? \\[0pt]
Qu'apprend-on dans les livres ? \\[0pt]
Qu'apprend-on de ses erreurs ? \\[0pt]
Qu'apprend-on des romans ? \\[0pt]
Qu'apprend-on en apprenant une technique ? \\[0pt]
Qu'apprend-on en commettant une faute ? \\[0pt]
Qu'apprend-on quand on apprend à parler ? \\[0pt]
Qu'apprenons-nous de nos affects ? \\[0pt]
Qu'apprenons-nous en éduquant ? \\[0pt]
Qu'a-t-on le droit de pardonner ? \\[0pt]
Qu'a-t-on le droit d'exiger ? \\[0pt]
Qu'a-t-on le droit d'interdire ? \\[0pt]
Qu'a-t-on le droit d'interpréter ? \\[0pt]
Qu'attendons-nous de la science ? \\[0pt]
Qu'attendons-nous de la technique ? \\[0pt]
Qu'attendons-nous des œuvres d'art ? \\[0pt]
Qu'attendons-nous pour être heureux ? \\[0pt]
Qu'attendre de l'État ? \\[0pt]
Qu'avons-nous à apprendre de nos illusions ? \\[0pt]
Qu'avons-nous à apprendre des historiens ? \\[0pt]
 Qu'avons-nous à attendre de la vie politique ? \\[0pt]
Qu'avons-nous en commun ? \\[0pt]
Que célèbre l'art ? \\[0pt]
Que changent les révolutions ? \\[0pt]
Qu'échange-t-on ? \\[0pt]
Que cherchons-nous dans le regard des autres ? \\[0pt]
Que choisir ? \\[0pt]
Que comprend-on d'une œuvre d'art ? \\[0pt]
Que connaissons-nous du vivant ? \\[0pt]
Que connaît-on de ses passions ? \\[0pt]
Que construit le politique ? \\[0pt]
Que coûte une victoire ? \\[0pt]
Que crée l'artiste ? \\[0pt]
Que déduire d'une contradiction ? \\[0pt]
Que démontrent nos actions ? \\[0pt]
Que désire-t-on ? \\[0pt]
Que désirons-nous ? \\[0pt]
Que désirons-nous quand nous désirons savoir ? \\[0pt]
Que devons-nous à autrui ? \\[0pt]
Que devons-nous à l'animal ? \\[0pt]
Que devons-nous à l'État ? \\[0pt]
Que disent les légendes ? \\[0pt]
Que disent les tables de vérité ? \\[0pt]
Que dit la loi ? \\[0pt]
Que dit la musique ? \\[0pt]
Que dit l'interdit ? \\[0pt]
Que dois-je à autrui ? \\[0pt]
Que dois-je à l'État ? \\[0pt]
Que dois-je faire ? \\[0pt]
Que dois-je respecter en autrui ? \\[0pt]
Que doit la pensée à l'écriture ? \\[0pt]
Que doit l'art à la nature ? \\[0pt]
Que doit la science à la technique ? \\[0pt]
Que doit-on à autrui ? \\[0pt]
Que doit-on à l'État ? \\[0pt]
Que doit-on aux morts ? \\[0pt]
Que doit-on croire ? \\[0pt]
Que doit-on désirer pour ne pas être déçu ? \\[0pt]
Que doit-on faire de ses rêves ? \\[0pt]
Que doit-on protéger ? \\[0pt]
Que doit-on savoir avant d'agir ? \\[0pt]
Que faire ? \\[0pt]
Que faire de la diversité des arts ? \\[0pt]
Que faire de la violence ? \\[0pt]
Que faire de nos émotions ? \\[0pt]
Que faire de nos passions ? \\[0pt]
Que faire de notre cerveau ? \\[0pt]
Que faire des adversaires ? \\[0pt]
Que faire du vraisemblable ? \\[0pt]
Que faire quand la loi est injuste ? \\[0pt]
Que fait aux œuvres d'art leur reproductibilité ? \\[0pt]
Que fait la machine ? \\[0pt]
Que fait la police ? \\[0pt]
Que fait l'art à nos vies ? \\[0pt]
Que fait l'artiste ? \\[0pt]
Que fait le spectateur ? \\[0pt]
Que faut-il absolument savoir ? \\[0pt]
Que faut-il craindre ? \\[0pt]
Que faut-il pour faire un monde ? \\[0pt]
Que faut-il respecter ? \\[0pt]
Que faut-il savoir pour agir ? \\[0pt]
Que faut-il savoir pour bien agir ? \\[0pt]
Que faut-il savoir pour gouverner ? \\[0pt]
Que faut-il savoir pour pouvoir gouverner ? \\[0pt]
Que faut-il vouloir pour être heureux ? \\[0pt]
Que gagne l'art à devenir abstrait ? \\[0pt]
Que gagne l'art à se réfléchir ? \\[0pt]
Que gagne-t-on à se mettre à la place d'autrui ? \\[0pt]
Que gagne-t-on à travailler ? \\[0pt]
Que garantit la séparation des pouvoirs ? \\[0pt]
Que la nature soit explicable, est-ce explicable ? \\[0pt]
Quel besoin avons-nous de chercher la vérité ? \\[0pt]
Quel contrôle a-t-on sur son corps ? \\[0pt]
Quel est le bon nombre d'amis ? \\[0pt]
Quel est le but de la politique ? \\[0pt]
Quel est le but d'une théorie physique ? \\[0pt]
Quel est le but du travail scientifique ? \\[0pt]
Quel est le contraire du travail ? \\[0pt]
Quel est le fondement de la propriété ? \\[0pt]
Quel est le fondement de l'autorité ? \\[0pt]
Quel est le meilleur régime politique ? \\[0pt]
Quel est le poids du passé ? \\[0pt]
Quel est le pouvoir de la beauté ? \\[0pt]
Quel est le pouvoir de l'art ? \\[0pt]
Quel est le pouvoir des métaphores ? \\[0pt]
Quel est le pouvoir des mots ? \\[0pt]
Quel est le problème posé par la pluralité des langues ? \\[0pt]
Quel est le propre de la pensée technique ? \\[0pt]
Quel est le rôle de la créativité dans les sciences ? \\[0pt]
Quel est le rôle du concept en art ? \\[0pt]
Quel est le rôle du médecin ? \\[0pt]
Quel est le sens du progrès technique ? \\[0pt]
Quel est le statut de la preuve en sciences humaines et sociales ? \\[0pt]
Quel est le sujet de la pensée ? \\[0pt]
Quel est le sujet de l'histoire ? \\[0pt]
Quel est le sujet des sciences sociales ? \\[0pt]
Quel est le sujet du devenir ? \\[0pt]
Quel est l'être de l'illusion ? \\[0pt]
Quel est l'homme de l'ethnologie ? \\[0pt]
Quel est l'homme des Droits de l'homme ? \\[0pt]
Quel est l'humain des sciences humaines ? \\[0pt]
Quel est l'objectif des sciences humaines ? \\[0pt]
Quel est l'objet de la biologie ? \\[0pt]
Quel est l'objet de la conscience de soi ? \\[0pt]
Quel est l'objet de la géométrie ? \\[0pt]
Quel est l'objet de la logique ? \\[0pt]
Quel est l'objet de la métaphysique ? \\[0pt]
Quel est l'objet de l'amour ? \\[0pt]
Quel est l'objet de la perception ? \\[0pt]
Quel est l'objet de la philosophie politique ? \\[0pt]
Quel est l'objet de la physique ? \\[0pt]
Quel est l'objet de la psychologie ? \\[0pt]
Quel est l'objet de la psychologie collective ? \\[0pt]
Quel est l'objet de la psychologie sociale ? \\[0pt]
Quel est l'objet de la science ? \\[0pt]
Quel est l'objet de l'échange ? \\[0pt]
Quel est l'objet de l'esthétique ? \\[0pt]
Quel est l'objet de l'histoire ? \\[0pt]
Quel est l'objet de l'ontologie ? \\[0pt]
Quel est l'objet des mathématiques ? \\[0pt]
Quel est l'objet des « sciences de l'esprit » ? \\[0pt]
Quel est l'objet des sciences humaines ? \\[0pt]
Quel est l'objet des sciences politiques ? \\[0pt]
Quel est l'objet du désir ? \\[0pt]
Quel être peut être un sujet de droits ? \\[0pt]
Quel genre de conscience peut-on accorder à l'animal ? \\[0pt]
Quel intérêt y a-t-il à se connaître ? \\[0pt]
Quelle causalité pour le vivant ? \\[0pt]
« Quelle chimère est-ce donc que l'homme » ? \\[0pt]
Quelle confiance accorder au langage ? \\[0pt]
Quelle différence y a-t-il entre un corps vivant et un corps mort ? \\[0pt]
Quelle espèce de vivants sommes-nous ? \\[0pt]
Quelle est la cause du désir ? \\[0pt]
Quelle est la fin de la science ? \\[0pt]
Quelle est la fin de l'État ? \\[0pt]
Quelle est la fonction première de l'État ? \\[0pt]
Quelle est la force de la loi ? \\[0pt]
Quelle est la limite du pouvoir de l'État ? \\[0pt]
Quelle est la matière de l'œuvre d'art ? \\[0pt]
Quelle est la nature du droit naturel ? \\[0pt]
Quelle est la place de l'imagination dans la vie de l'esprit ? \\[0pt]
Quelle est la place du hasard dans l'histoire ? \\[0pt]
Quelle est la place du sujet dans les sciences humaines ? \\[0pt]
Quelle est la portée d'un exemple ? \\[0pt]
Quelle est la réalité de la matière ? \\[0pt]
Quelle est la réalité de l'avenir ? \\[0pt]
Quelle est la réalité des objets mathématiques ? \\[0pt]
Quelle est la réalité d'une idée ? \\[0pt]
Quelle est la réalité du passé ? \\[0pt]
Quelle est la réalité du temps ? \\[0pt]
Quelle est la source de nos devoirs ? \\[0pt]
Quelle est la spécificité de la communauté politique ? \\[0pt]
Quelle est la valeur culturelle de la science ? \\[0pt]
Quelle est la valeur de la connaissance ? \\[0pt]
Quelle est la valeur de l'expérience ? \\[0pt]
Quelle est la valeur des hypothèses ? \\[0pt]
Quelle est la valeur d'une expérimentation ? \\[0pt]
Quelle est la valeur d'une œuvre d'art ? \\[0pt]
Quelle est la valeur d'une promesse ? \\[0pt]
Quelle est la valeur du rêve ? \\[0pt]
Quelle est la valeur du témoignage ? \\[0pt]
Quelle est la valeur du temps ? \\[0pt]
Quelle est la valeur du vivant ? \\[0pt]
Quelle est l'unité du « je » ? \\[0pt]
Quelle expérience la politique requiert-elle ? \\[0pt]
Quelle idée le fanatique se fait-il de la vérité ? \\[0pt]
Quelle maîtrise avons-nous du temps ? \\[0pt]
Quelle peut être la force de nos idées ? \\[0pt]
Quelle place donner au vraisemblable ? \\[0pt]
Quelle place la raison peut-elle faire à la croyance ? \\[0pt]
Quelle place les sciences humaines doivent-elles accorder à la subjectivité ? \\[0pt]
Quelle place pour la mesure dans les sciences de l'homme ? \\[0pt]
Quelle politique fait-on avec les sciences humaines ? \\[0pt]
Quelle réalité attribuer à la matière ? \\[0pt]
Quelle réalité conférer aux fictions ? \\[0pt]
Quelle réalité l'art nous fait-il connaître ? \\[0pt]
Quelle réalité la science décrit-elle ? \\[0pt]
Quelle réalité l'imagination nous fait-elle connaître ? \\[0pt]
Quelle réalité peut-on accorder au temps ? \\[0pt]
Quelle réalité peut-on attribuer au temps ? \\[0pt]
Quelle responsabilité publique pour le chercheur en sciences sociales ? \\[0pt]
Quelles actions permettent d'être heureux ? \\[0pt]
Quelles limites assigner au pouvoir de la parole ? \\[0pt]
Quelles limites poser à l'autorité de l'État ? \\[0pt]
Quelles lois pour les sciences humaines ? \\[0pt]
Quelle sorte d'histoire ont les sciences ? \\[0pt]
Quelles règles la technique dicte-t-elle à l'art ? \\[0pt]
Quelles sont les caractéristiques d'une proposition morale ? \\[0pt]
Quelles sont les caractéristiques d'un être vivant ? \\[0pt]
Quelles sont les limites de la démonstration ? \\[0pt]
Quelles sont les limites de la souveraineté ? \\[0pt]
Quelles sont les limites de mon monde ? \\[0pt]
Quelle valeur accorder à l'expérience ? \\[0pt]
Quelle valeur devons accorder à l'expérience ? \\[0pt]
Quelle valeur devons-nous accorder à l'expérience ? \\[0pt]
Quelle valeur devons-nous accorder à l'intuition ? \\[0pt]
Quelle valeur donner à la notion de « corps social » ? \\[0pt]
Quelle valeur peut-on accorder à l'expérience ? \\[0pt]
Quelle vérité y a-t-il dans la perception ? \\[0pt]
Quel réel pour l'art ? \\[0pt]
Quel rôle attribuer à l'intuition \emph{a priori} dans une philosophie des mathématiques ? \\[0pt]
Quel rôle la logique joue-t-elle en mathématiques ? \\[0pt]
Quel rôle les statistiques jouent-elles dans les sciences humaines ? \\[0pt]
Quel rôle l'imagination joue-t-elle en mathématiques ? \\[0pt]
Quels critères de démarcation pour les sciences sociales ? \\[0pt]
Quels désirs dois-je m'interdire ? \\[0pt]
Quels devoirs les religions peuvent-elles énoncer ? \\[0pt]
Quel sens donner à l'expression « gagner sa vie » ? \\[0pt]
Quel sens donner au projet de forger une langue parfaite ? \\[0pt]
Quels enseignements peut-on tirer de l'histoire des sciences ? \\[0pt]
Quel sens y a-t-il à se demander si les sciences humaines sont vraiment des sciences ? \\[0pt]
Quels matériaux pour les sciences humaines et sociales ? \\[0pt]
Quels sont les droits de la conscience ? \\[0pt]
Quels sont les fondements de l'autorité ? \\[0pt]
Quels sont les moyens légitimes de la politique ? \\[0pt]
Quel statut philosophique pour l'opinion ? \\[0pt]
Quel type de réalité faut-il attribuer à l'inconscient en psychanalyse ? \\[0pt]
Quel type de réalité faut-il attribuer au temps ? \\[0pt]
Quel usage faire de la finalité ? \\[0pt]
Quel usage faut-il faire des exemples ? \\[0pt]
Quel usage peut-on faire des fictions ? \\[0pt]
Quel vivant sommes-nous ? \\[0pt]
Que manque-t-il à une machine pour être vivante ? \\[0pt]
Que manque-t-il aux machines pour être des organismes ? \\[0pt]
Que mesure-t-on du temps ? \\[0pt]
Que montre l'image ? \\[0pt]
Que montre une démonstration ? \\[0pt]
Que montre un tableau ? \\[0pt]
Que m'ordonne ma conscience ? \\[0pt]
Que ne peut-on pas expliquer ? \\[0pt]
Que nous append l'histoire ? \\[0pt]
Que nous apporte l'art ? \\[0pt]
Que nous apporte la technique ? \\[0pt]
Que nous apporte la vérité ? \\[0pt]
Que nous apporte le travail ? \\[0pt]
Que nous apprend la biologie sur l'homme ? \\[0pt]
Que nous apprend la définition de la vérité ? \\[0pt]
Que nous apprend la diversité des langues ? \\[0pt]
Que nous apprend la fiction sur la réalité ? \\[0pt]
Que nous apprend la géométrie ? \\[0pt]
Que nous apprend la grammaire ? \\[0pt]
Que nous apprend la littérature ? \\[0pt]
Que nous apprend la maladie ? \\[0pt]
Que nous apprend la maladie sur la santé ? \\[0pt]
Que nous apprend la métaphysique ? \\[0pt]
Que nous apprend la musique ? \\[0pt]
Que nous apprend l'animal ? \\[0pt]
Que nous apprend l'anthropologie ? \\[0pt]
Que nous apprend la poésie ? \\[0pt]
Que nous apprend l'approche sociologique de l'art ? \\[0pt]
Que nous apprend la psychanalyse de l'homme ? \\[0pt]
Que nous apprend la religion ? \\[0pt]
Que nous apprend la sociologie des sciences ? \\[0pt]
Que nous apprend la technique ? \\[0pt]
Que nous apprend la vie ? \\[0pt]
Que nous apprend le cinéma ? \\[0pt]
Que nous apprend le faux ? \\[0pt]
Que nous apprend le plaisir ? \\[0pt]
Que nous apprend l'erreur ? \\[0pt]
Que nous apprend le témoignage ? \\[0pt]
Que nous apprend le théâtre ? \\[0pt]
Que nous apprend le toucher ? \\[0pt]
Que nous apprend l'étude du cerveau ? \\[0pt]
Que nous apprend l'expérience ? \\[0pt]
Que nous apprend l'expérience esthétique ? \\[0pt]
Que nous apprend l'histoire de l'art ? \\[0pt]
Que nous apprend l'histoire des sciences ? \\[0pt]
Que nous apprend, sur la politique, l'utopie ? \\[0pt]
Que nous apprennent les algorithmes sur nos sociétés ? \\[0pt]
Que nous apprennent les Anciens ? \\[0pt]
Que nous apprennent les animaux ? \\[0pt]
Que nous apprennent les animaux sur nous-mêmes ? \\[0pt]
Que nous apprennent les controverses scientifiques ? \\[0pt]
Que nous apprennent les expériences de pensée ? \\[0pt]
Que nous apprennent les faits divers ? \\[0pt]
Que nous apprennent les illusions d'optique ? \\[0pt]
Que nous apprennent les jeux ? \\[0pt]
Que nous apprennent les langues étrangères ? \\[0pt]
Que nous apprennent les machines ? \\[0pt]
Que nous apprennent les métaphores ? \\[0pt]
Que nous apprennent les mythes ? \\[0pt]
Que nous apprennent les objets techniques ? \\[0pt]
Que nous apprennent les statistiques ? \\[0pt]
Que nous apprennent nos erreurs ? \\[0pt]
Que nous apprennent nos sentiments ? \\[0pt]
Que nous devons-nous ? \\[0pt]
Que nous doit l'État ? \\[0pt]
Que nous enseigne la lecture des romans ? \\[0pt]
Que nous enseigne la monstruosité ? \\[0pt]
Que nous enseigne l'expérience ? \\[0pt]
Que nous enseignent les œuvres d'art ? \\[0pt]
Que nous enseignent les sens ? \\[0pt]
Que nous enseignent nos peurs ? \\[0pt]
Que nous impose la nature ? \\[0pt]
Que nous impose le temps ? \\[0pt]
Que nous montre le cinéma ? \\[0pt]
Que nous montre l'œuvre d'art ? \\[0pt]
Que nous montrent les natures mortes ? \\[0pt]
Que nous réserve l'avenir ? \\[0pt]
Que partage-t-on avec les animaux ? \\[0pt]
Que peindre ? \\[0pt]
Que peint le peintre ? \\[0pt]
Que penser de l'adage : « Que la justice s'accomplisse, le monde dût-il périr » (Fiat justitia pereat mundus) ? \\[0pt]
Que penser de la formule : « il faut suivre la nature » ? \\[0pt]
Que penser de l'opposition travail manuel, travail intellectuel ? \\[0pt]
Que percevons-nous ? \\[0pt]
Que percevons-nous d'autrui ? \\[0pt]
Que percevons-nous du monde extérieur ? \\[0pt]
Que perçoit-on ? \\[0pt]
Que perd la pensée en perdant l'écriture ? \\[0pt]
Que perd-on quand on perd son temps ? \\[0pt]
Que perdrait la pensée en perdant l'écriture ? \\[0pt]
Que peut expliquer l'histoire ? \\[0pt]
Que peut la force ? \\[0pt]
Que peut la musique ? \\[0pt]
Que peut la pensée ? \\[0pt]
Que peut la philosophie ? \\[0pt]
Que peut la politique ? \\[0pt]
Que peut la raison ? \\[0pt]
Que peut la raison contre une croyance ? \\[0pt]
Que peut l'art ? \\[0pt]
Que peut la science ? \\[0pt]
Que peut la théorie ? \\[0pt]
Que peut la volonté ? \\[0pt]
Que peut la volonté contre les passions ? \\[0pt]
Que peut le corps ? \\[0pt]
Que peut le droit ? \\[0pt]
Que peut le politique ? \\[0pt]
Que peut l'esprit ? \\[0pt]
Que peut l'esprit sur la matière ? \\[0pt]
Que peut l'État ? \\[0pt]
Que peut l'intelligence artificielle ? \\[0pt]
Que peut-on apprendre des émotions esthétiques ? \\[0pt]
Que peut-on attendre de l'État ? \\[0pt]
Que peut-on attendre du droit international ? \\[0pt]
Que peut-on attendre d'une philosophie de l'art ? \\[0pt]
Que peut-on calculer ? \\[0pt]
Que peut-on comprendre immédiatement ? \\[0pt]
Que peut-on comprendre qu'on ne puisse expliquer ? \\[0pt]
Que peut-on contre un préjugé ? \\[0pt]
Que peut-on cultiver ? \\[0pt]
Que peut-on démontrer ? \\[0pt]
Que peut-on dire de l'être ? \\[0pt]
Que peut-on échanger ? \\[0pt]
Que peut-on enseigner ? \\[0pt]
Que peut-on espérer ? \\[0pt]
Que peut-on exprimer ? \\[0pt]
Que peut-on interdire ? \\[0pt]
Que peut-on partager ? \\[0pt]
Que peut-on perdre ? \\[0pt]
Que peut-on savoir de l'inconscient ? \\[0pt]
Que peut-on savoir de soi ? \\[0pt]
Que peut-on savoir du réel ? \\[0pt]
Que peut-on savoir par expérience ? \\[0pt]
Que peut-on sur autrui ? \\[0pt]
Que peut-on voir ? \\[0pt]
Que peut prétendre imposer une religion ? \\[0pt]
Que peut signifier « avoir le sens de la réalité » ? \\[0pt]
Que peut signifier : « gérer son temps » ? \\[0pt]
Que peut-signifier « tuer le temps » ? \\[0pt]
Que peut un corps ? \\[0pt]
Que peuvent les idées ? \\[0pt]
Que peuvent les images ? \\[0pt]
Que peuvent les lois ? \\[0pt]
Que peuvent nous apprendre les expériences de pensée ? \\[0pt]
Que pouvons-nous attendre de la technique ? \\[0pt]
Que pouvons-nous aujourd'hui apprendre des sciences d'autrefois ? \\[0pt]
Que pouvons-nous comprendre du monde ? \\[0pt]
Que pouvons-nous connaître ? \\[0pt]
Que pouvons-nous espérer ? \\[0pt]
Que pouvons-nous espérer de la connaissance du vivant ? \\[0pt]
Que pouvons-nous faire de notre passé ? \\[0pt]
Que pouvons-nous partager ? \\[0pt]
Que produit la poésie ? \\[0pt]
Que produit l'inconscient ? \\[0pt]
Que protège la loi ? \\[0pt]
Que prouvent les faits ? \\[0pt]
Que prouvent les preuves de l'existence de Dieu ? \\[0pt]
Que puis-je dire être mien ? \\[0pt]
Que recherche l'artiste ? \\[0pt]
Que rend visible l'art ? \\[0pt]
Que répondre au sceptique ? \\[0pt]
Que reste-t-il d'une existence ? \\[0pt]
Que reste-t-il du passé ? \\[0pt]
Que sais-je d'autrui ? \\[0pt]
Que sais-je de ma souffrance ? \\[0pt]
Que sait la conscience ? \\[0pt]
Que sait la science ? \\[0pt]
Que sait-on de soi ? \\[0pt]
Que sait-on du réel ? \\[0pt]
Que savons-nous de l'avenir ? \\[0pt]
Que savons-nous de l'inconscient ? \\[0pt]
Que savons-nous de notre vie ? \\[0pt]
Que savons-nous de nous-mêmes ? \\[0pt]
Que savons-nous des principes ? \\[0pt]
Que sent le corps ? \\[0pt]
Que serait la fin de l'histoire ? \\[0pt]
Que serait la vie sans l'art ? \\[0pt]
Que serait le meilleur des mondes ? \\[0pt]
Que serait un art total ? \\[0pt]
Que serait une démocratie parfaite ? \\[0pt]
Que serait une pure sensation ? \\[0pt]
Que serions-nous sans l'État ? \\[0pt]
Que signifie apprendre ? \\[0pt]
Que signifie : « avoir de l'esprit » ? \\[0pt]
Que signifie connaître ? \\[0pt]
Que signifie « donner le change » ? \\[0pt]
Que signifie : « échanger des arguments » ? \\[0pt]
Que signifie être dépendant ? \\[0pt]
Que signifie être en guerre ? \\[0pt]
Que signifie être mortel ? \\[0pt]
Que signifie la mort ? \\[0pt]
Que signifie l'angoisse ? \\[0pt]
Que signifie l'expression : « l'histoire jugera » ? \\[0pt]
Que signifie l'idée de technoscience ? \\[0pt]
Que signifient les mots ? \\[0pt]
Que signifie « politiquement correct » ? \\[0pt]
Que signifie pour l'homme être mortel ? \\[0pt]
Que signifie prouver en sciences sociales ? \\[0pt]
Que signifie refuser l'injustice ? \\[0pt]
Que signifier « juger en son âme et conscience » ? \\[0pt]
Que signifie : « se rendre à l'évidence » ? \\[0pt]
Que sondent les sondages d'opinion ? \\[0pt]
Que sont les apparences ? \\[0pt]
Que sont les institutions sans les individus ? \\[0pt]
Que sont les objets de la perception ? \\[0pt]
Qu'est-ce la vérité ? \\[0pt]
Qu'est-ce qu'abstraire ? \\[0pt]
Qu'est-ce qu'agir avec discernement ? \\[0pt]
Qu'est-ce qu'agir ensemble ? \\[0pt]
Qu'est-ce qu'aimer une œuvre d'art ? \\[0pt]
Qu'est-ce qu'apparaître ? \\[0pt]
Qu'est-ce qu'appliquer une règle de droit ? \\[0pt]
Qu'est-ce qu'apprécier une œuvre d'art ? \\[0pt]
Qu'est-ce qu'apprendre ? \\[0pt]
Qu'est-ce qu'argumenter ? \\[0pt]
Qu'est-ce qu'avoir conscience de soi ? \\[0pt]
Qu'est-ce qu'avoir de l'expérience ? \\[0pt]
Qu'est-ce qu'avoir du goût ? \\[0pt]
Qu'est-ce qu'avoir du jugement ? \\[0pt]
Qu'est-ce qu'avoir du style ? \\[0pt]
Qu'est-ce qu'avoir raison ? \\[0pt]
Qu'est-ce qu'avoir un droit ? \\[0pt]
Qu'est-ce qu'avoir une idée ? \\[0pt]
Qu'est-ce qu'avoir un esprit scientifique ? \\[0pt]
Qu'est-ce que bien vivre ? \\[0pt]
Qu'est-ce que calculer ? \\[0pt]
Qu'est-ce que catégoriser ? \\[0pt]
Qu'est-ce que classer ? \\[0pt]
Qu'est-ce que commencer ? \\[0pt]
Qu'est-ce que composer une œuvre ? \\[0pt]
Qu'est-ce que comprendre ? \\[0pt]
Qu'est-ce que comprendre une œuvre d'art ? \\[0pt]
Qu'est-ce que connaître ? \\[0pt]
Qu'est-ce que contempler ? \\[0pt]
Qu'est-ce que créer ? \\[0pt]
Qu'est-ce que croire ? \\[0pt]
Qu'est-ce que décider ? \\[0pt]
Qu'est-ce que définir ? \\[0pt]
Qu'est-ce que démontrer ? \\[0pt]
Qu'est-ce que déraisonner ? \\[0pt]
Qu'est-ce que Dieu pour athée ? \\[0pt]
Qu'est-ce que Dieu pour un athée ? \\[0pt]
Qu'est-ce que discuter ? \\[0pt]
Qu'est-ce que donner sa parole ? \\[0pt]
Qu'est-ce qu'éduquer ? \\[0pt]
Qu'est-ce qu'éduquer le sens esthétique ? \\[0pt]
Qu'est-ce que faire autorité ? \\[0pt]
Qu'est-ce que « faire de la politique » ? \\[0pt]
Qu'est-ce que faire la paix ? \\[0pt]
Qu'est-ce que faire preuve d'humanité ? \\[0pt]
Qu'est-ce que faire une expérience ? \\[0pt]
Qu'est-ce que « faire usage de sa raison » ? \\[0pt]
Qu'est-ce que gouverner ? \\[0pt]
Qu'est-ce que guérir ? \\[0pt]
Qu'est-ce que jouer ? \\[0pt]
Qu'est-ce que juger ? \\[0pt]
Qu'est-ce que la barbarie ? \\[0pt]
Qu'est-ce que la causalité ? \\[0pt]
Qu'est-ce que la critique ? \\[0pt]
Qu'est-ce que la démocratie ? \\[0pt]
Qu'est-ce que la folie ? \\[0pt]
Qu'est-ce que la normalité ? \\[0pt]
Qu'est-ce que l'anthropologie politique ? \\[0pt]
Qu'est-ce que la perception ? \\[0pt]
Qu'est-ce que la politique ? \\[0pt]
Qu'est-ce que la psychanalyse nous apprend sur les sociétés humaines ? \\[0pt]
Qu'est-ce que la psychologie ? \\[0pt]
Qu'est-ce que la raison d'État ? \\[0pt]
Qu'est-ce que la réalité ? \\[0pt]
Qu'est-ce que la religion nous donne à savoir ? \\[0pt]
Qu'est-ce que l'art contemporain ? \\[0pt]
Qu'est-ce que la science doit à l'expérience ? \\[0pt]
Qu'est-ce que la science saisit du vivant ? \\[0pt]
Qu'est-ce que la science, si elle inclut la psychanalyse ? \\[0pt]
Qu'est-ce que la scientificité ? \\[0pt]
Qu'est-ce que la sociologie nous apprend sur la culture ? \\[0pt]
Qu'est-ce que la souveraineté ? \\[0pt]
Qu'est-ce que la technique ? \\[0pt]
Qu'est-ce que la technique doit à la nature ? \\[0pt]
Qu'est-ce que la tragédie ? \\[0pt]
Qu'est-ce que la valeur marchande ? \\[0pt]
Qu'est-ce que la vérité ? \\[0pt]
Qu'est-ce que la vie ? \\[0pt]
Qu'est-ce que la vie bonne ? \\[0pt]
Qu'est-ce que le bonheur ? \\[0pt]
Qu'est-ce que le cinéma a changé dans l'idée que l'on se fait du temps ? \\[0pt]
Qu'est-ce que le cinéma donne à voir ? \\[0pt]
Qu'est-ce que le courage ? \\[0pt]
Qu'est-ce que le désordre ? \\[0pt]
Qu'est-ce que le dogmatisme ? \\[0pt]
Qu'est-ce que le génie ? \\[0pt]
Qu'est-ce que le hasard ? \\[0pt]
Qu'est-ce que le langage ordinaire ? \\[0pt]
Qu'est-ce que le malheur ? \\[0pt]
Qu'est-ce que le mal radical ? \\[0pt]
Qu'est-ce que le mauvais goût ? \\[0pt]
Qu'est-ce que le moi ? \\[0pt]
Qu'est-ce que le naturalisme ? \\[0pt]
Qu'est-ce que l'enfance ? \\[0pt]
Qu'est-ce que le nihilisme ? \\[0pt]
Qu'est-ce que le pathologique nous apprend sur le normal ? \\[0pt]
Qu'est-ce que le présent ? \\[0pt]
Qu'est-ce que le progrès technique ? \\[0pt]
Qu'est-ce que le réel ? \\[0pt]
Qu'est-ce que le sacré ? \\[0pt]
Qu'est-ce que le sens pratique ? \\[0pt]
Qu'est-ce que les sciences humaines apportent à la critique littéraire ? \\[0pt]
Qu'est-ce que les sciences religieuses nous apprennent de la religion ? \\[0pt]
Qu'est-ce que le style ? \\[0pt]
Qu'est-ce que le sublime ? \\[0pt]
Qu'est-ce que le symbolique ? \\[0pt]
Qu'est-ce que le temps ? \\[0pt]
Qu'est-ce que le travail ? \\[0pt]
Qu'est-ce que l'expérience pour un empiriste ? \\[0pt]
Qu'est-ce que l'harmonie ? \\[0pt]
Qu'est-ce que l'identité sociale des individus ? \\[0pt]
Qu'est-ce que l'inconscient ? \\[0pt]
Qu'est-ce que l'indifférence ? \\[0pt]
Qu'est-ce que l'individualisme méthodologique ? \\[0pt]
Qu'est-ce que l'intérêt général ? \\[0pt]
Qu'est-ce que l'interprétation des Écritures nous apprend sur les religions ? \\[0pt]
Qu'est-ce que l'intuition ? \\[0pt]
Qu'est-ce que lire ? \\[0pt]
Qu'est-ce que l'objectivité scientifique ? \\[0pt]
Qu'est-ce que l'ordinaire ? \\[0pt]
Qu'est-ce que maîtriser une technique ? \\[0pt]
Qu'est-ce que manquer de culture ? \\[0pt]
Qu'est-ce que méditer ? \\[0pt]
Qu'est-ce que mesurer le temps ? \\[0pt]
Qu'est-ce que mourir ? \\[0pt]
Qu'est-ce que nous apprennent les animaux sur nous-même ? \\[0pt]
Qu'est-ce qu'enquêter ? \\[0pt]
Qu'est-ce qu'enseigner ? \\[0pt]
Qu'est-ce que parler ? \\[0pt]
Qu'est-ce que parler à-propos ? \\[0pt]
Qu'est-ce que « parler le même langage » ? \\[0pt]
Qu'est-ce que parler le même langage ? \\[0pt]
Qu'est-ce que parler pour ne rien dire ? \\[0pt]
Qu'est-ce que parler veut dire ? \\[0pt]
Qu'est-ce que penser ? \\[0pt]
Qu'est-ce que percevoir ? \\[0pt]
Qu'est-ce que perdre la raison ? \\[0pt]
Qu'est-ce que perdre sa liberté ? \\[0pt]
Qu'est-ce que perdre son temps ? \\[0pt]
Qu'est-ce que peut un corps ? \\[0pt]
Qu'est-ce que prendre conscience ? \\[0pt]
Qu'est-ce que prendre le pouvoir ? \\[0pt]
Qu'est-ce que produire ? \\[0pt]
Qu'est-ce que promettre ? \\[0pt]
Qu'est-ce que prouver ? \\[0pt]
Qu'est-ce que raisonner ? \\[0pt]
Qu'est-ce que réfuter ? \\[0pt]
Qu'est-ce que réfuter une philosophie ? \\[0pt]
Qu'est-ce que rendre raison d'un effet ? \\[0pt]
Qu'est-ce que résister ? \\[0pt]
Qu'est-ce que résoudre une contradiction ? \\[0pt]
Qu'est-ce que « rester soi-même » ? \\[0pt]
Qu'est-ce que rester soi-même ? \\[0pt]
Qu'est-ce que réussir sa vie ? \\[0pt]
Qu'est-ce que savoir ? \\[0pt]
Qu'est-ce que se cultiver ? \\[0pt]
Qu'est-ce que « se rendre maître et possesseur de la nature » ? \\[0pt]
Qu'est-ce que se tromper ? \\[0pt]
Qu'est-ce que s'orienter ? \\[0pt]
Qu'est-ce que témoigner ? \\[0pt]
Qu'est-ce que traduire ? \\[0pt]
Qu'est-ce que travailler ? \\[0pt]
Qu'est-ce qu'être ? \\[0pt]
Qu'est-ce qu'être adulte ? \\[0pt]
Qu'est-ce qu'être artiste ? \\[0pt]
Qu'est-ce qu'être asocial ? \\[0pt]
Qu'est-ce qu'être barbare ? \\[0pt]
Qu'est-ce qu'être chez soi ? \\[0pt]
Qu'est-ce qu'être cohérent ? \\[0pt]
Qu'est-ce qu'être comportementaliste ? \\[0pt]
Qu'est-ce qu'être cultivé ? \\[0pt]
Qu'est-ce qu'être dans le vrai ? \\[0pt]
Qu'est-ce qu'être de bonne volonté ? \\[0pt]
Qu'est-ce qu'être démocrate ? \\[0pt]
Qu'est-ce qu'être de son temps ? \\[0pt]
Qu'est-ce qu'être efficace en politique ? \\[0pt]
Qu'est-ce qu'être ensemble ? \\[0pt]
Qu'est-ce qu'être en vie ? \\[0pt]
Qu'est-ce qu'être esclave ? \\[0pt]
Qu'est-ce qu'être fidèle à soi-même ? \\[0pt]
Qu'est-ce qu'être fou ? \\[0pt]
Qu'est-ce qu'être généreux ? \\[0pt]
Qu'est-ce qu'être hors de soi ? \\[0pt]
Qu'est-ce qu'être idéaliste ? \\[0pt]
Qu'est-ce qu'être inhumain ? \\[0pt]
Qu'est-ce qu'être l'auteur de son acte ? \\[0pt]
Qu'est-ce qu'être libéral ? \\[0pt]
Qu'est-ce qu'être libre ? \\[0pt]
Qu'est-ce qu'être maître de soi ? \\[0pt]
Qu'est-ce qu'être maître de soi-même ? \\[0pt]
Qu'est-ce qu'être malade ? \\[0pt]
Qu'est-ce qu'être matérialiste ? \\[0pt]
Qu'est-ce qu'être moderne ? \\[0pt]
Qu'est-ce qu'être nihiliste ? \\[0pt]
Qu'est-ce qu'être normal ? \\[0pt]
Qu'est-ce qu'être psychologue ? \\[0pt]
Qu'est-ce qu'être rationaliste ? \\[0pt]
Qu'est-ce qu'être rationnel ? \\[0pt]
Qu'est-ce qu'être réaliste ? \\[0pt]
Qu'est-ce qu'être républicain ? \\[0pt]
Qu'est-ce qu'être sceptique ? \\[0pt]
Qu'est-ce qu'être seul ? \\[0pt]
Qu'est-ce qu'être simple ? \\[0pt]
Qu'est-ce qu'être soi-même ? \\[0pt]
Qu'est-ce qu'être souverain ? \\[0pt]
Qu'est-ce qu'être spirituel ? \\[0pt]
Qu'est-ce qu'être témoin ? \\[0pt]
Qu'est-ce qu'être un bon citoyen ? \\[0pt]
Qu'est-ce qu'être un esclave ? \\[0pt]
Qu'est-ce qu'être un sujet ? \\[0pt]
Qu'est-ce qu'être vertueux ? \\[0pt]
Qu'est-ce qu'être visionnaire ? \\[0pt]
Qu'est-ce qu'être vivant ? \\[0pt]
Qu'est-ce que vérifier ? \\[0pt]
Qu'est-ce que vérifier une théorie ? \\[0pt]
Qu'est-ce que vieillir ? \\[0pt]
Qu'est-ce que vivre ? \\[0pt]
Qu'est-ce que vivre au présent ? \\[0pt]
Qu'est-ce que vivre bien ? \\[0pt]
Qu'est-ce que vivre vertueusement ? \\[0pt]
Qu'est-ce que voir ? \\[0pt]
Qu'est-ce qu'exercer un pouvoir ? \\[0pt]
Qu'est-ce qu'exister ? \\[0pt]
Qu'est-ce qu'exister pour un individu ? \\[0pt]
Qu'est-ce qu'expliquer ? \\[0pt]
Qu'est-ce qu'expliquer un comportement ? \\[0pt]
Qu'est-ce qu'habiter ? \\[0pt]
Qu'est-ce qui adoucit les mœurs ? \\[0pt]
Qu'est-ce qui a du sens ? \\[0pt]
Qu'est-ce qui agit ? \\[0pt]
Qu'est-ce qui anime l'animal ? \\[0pt]
Qu'est-ce qui apparaît ? \\[0pt]
Qu'est-ce qui a un sens ? \\[0pt]
Qu'est-ce qu'identifier ? \\[0pt]
Qu'est-ce qui dépend de nous ? \\[0pt]
Qu'est-ce qui distingue un argument d'une démonstration ? \\[0pt]
Qu'est-ce qui distingue un vivant d'une machine ? \\[0pt]
Qu'est-ce qui échappe aux sciences humaines ? \\[0pt]
Qu'est-ce qui est absurde ? \\[0pt]
Qu'est-ce qui est actuel ? \\[0pt]
Qu'est-ce qui est anormal ? \\[0pt]
Qu'est-ce qui est artificiel ? \\[0pt]
Qu'est-ce qui est beau ? \\[0pt]
Qu'est-ce qui est commun à toutes les langues ? \\[0pt]
Qu'est-ce qui est concret ? \\[0pt]
Qu'est-ce qui est contre nature ? \\[0pt]
Qu'est-ce qui est culturel ? \\[0pt]
Qu'est-ce qui est démonstratif ? \\[0pt]
Qu'est-ce qui est donné ? \\[0pt]
Qu'est-ce qui est essentiel ? \\[0pt]
Qu'est-ce qui est étonnant ? \\[0pt]
Qu'est-ce qui est extérieur à ma conscience, ? \\[0pt]
Qu'est-ce qui est historique ? \\[0pt]
Qu'est-ce qui est hors la loi ? \\[0pt]
Qu'est-ce qui est humain ? \\[0pt]
Qu'est-ce qui est immoral ? \\[0pt]
Qu'est-ce qui est impossible ? \\[0pt]
Qu'est-ce qui est indiscutable ? \\[0pt]
Qu'est-ce qui est injuste ? \\[0pt]
Qu'est-ce qui est insignifiant ? \\[0pt]
Qu'est-ce qui est intolérable ? \\[0pt]
Qu'est-ce qui est invérifiable ? \\[0pt]
Qu'est-ce qui est irrationnel ? \\[0pt]
Qu'est-ce qui est irréfutable ? \\[0pt]
Qu'est-ce qui est irréversible ? \\[0pt]
Qu'est-ce qui est le plus à craindre, l'ordre ou le désordre ? \\[0pt]
Qu'est-ce qui est mauvais dans l'égoïsme ? \\[0pt]
Qu'est-ce qui est mien ? \\[0pt]
Qu'est-ce qui est moderne ? \\[0pt]
Qu'est-ce qui est naturel ? \\[0pt]
Qu'est-ce qui est nécessaire ? \\[0pt]
Qu'est-ce qui est noble ? \\[0pt]
Qu'est-ce qui est normal ? \\[0pt]
Qu'est-ce qui est politique ? \\[0pt]
Qu'est-ce qui est possible ? \\[0pt]
Qu'est-ce qui est prométhéen ? \\[0pt]
Qu'est-ce qui est public ? \\[0pt]
Qu'est-ce qui est réel ? \\[0pt]
Qu'est-ce qui est respectable ? \\[0pt]
Qu'est-ce qui est sacré ? \\[0pt]
Qu'est-ce qui est sans raison ? \\[0pt]
Qu'est-ce qui est sauvage ? \\[0pt]
Qu'est-ce qui est scientifique ? \\[0pt]
Qu'est-ce qui est sensible ? \\[0pt]
Qu'est-ce qui est spectaculaire ? \\[0pt]
Qu'est-ce qui est sublime ? \\[0pt]
Qu'est-ce qui est tragique ? \\[0pt]
Qu'est-ce qui est transcendant ? \\[0pt]
Qu'est-ce qui est vital ? \\[0pt]
Qu'est-ce qui est vital pour le vivant ? \\[0pt]
Qu'est-ce qui est vivant ? \\[0pt]
Qu'est-ce qui exige d'être interprété ? \\[0pt]
Qu'est-ce qui existe ? \\[0pt]
Qu'est-ce qui fait autorité ? \\[0pt]
Qu'est-ce qui fait changer les sociétés ? \\[0pt]
Qu'est-ce qui fait des sciences humaines des sciences ? \\[0pt]
Qu'est-ce qui fait d'une activité un travail ? \\[0pt]
Qu'est-ce qui fait du scepticisme une philosophie ? \\[0pt]
Qu'est-ce qui fait la force de la loi ? \\[0pt]
Qu'est-ce qui fait la force des lois ? \\[0pt]
Qu'est-ce qui fait la justice des lois ? \\[0pt]
Qu'est-ce qui fait la légitimité d'une autorité politique ? \\[0pt]
Qu'est-ce qui fait la valeur de la technique ? \\[0pt]
Qu'est-ce qui fait la valeur de l'œuvre d'art ? \\[0pt]
Qu'est-ce qui fait la valeur des objets d'art ? \\[0pt]
Qu'est-ce qui fait la valeur d'une croyance ? \\[0pt]
Qu'est-ce qui fait la valeur d'une existence ? \\[0pt]
Qu'est-ce qui fait la valeur d'une œuvre ? \\[0pt]
Qu'est-ce qui fait la valeur d'une œuvre d'art ? \\[0pt]
Qu'est-ce qui fait le pouvoir des mots ? \\[0pt]
Qu'est-ce qui fait le propre d'un corps propre ? \\[0pt]
Qu'est-ce qui fait l'humanité d'un corps ? \\[0pt]
Qu'est-ce qui fait l'unité d'une science ? \\[0pt]
Qu'est-ce qui fait l'unité d'un organisme ? \\[0pt]
Qu'est-ce qui fait l'unité d'un peuple ? \\[0pt]
Qu'est-ce qui fait l'unité du vivant ? \\[0pt]
Qu'est-ce qui fait mon identité ? \\[0pt]
Qu'est-ce qui fait obstacle à la science ? \\[0pt]
Qu'est-ce qui fait qu'un corps est humain ? \\[0pt]
Qu'est-ce qui fait qu'une théorie est vraie ? \\[0pt]
Qu'est-ce qui fait qu'un peuple est un peuple ? \\[0pt]
Qu'est-ce qui fait un peuple ? \\[0pt]
Qu'est-ce qui fonde la croyance ? \\[0pt]
Qu'est-ce qui fonde le respect d'autrui ? \\[0pt]
Qu'est-ce qu'ignore la science ? \\[0pt]
Qu'est-ce qui importe ? \\[0pt]
Qu'est-ce qui importe dans l'éducation ? \\[0pt]
Qu'est-ce qui innocente le bourreau ? \\[0pt]
Qu'est-ce qui justifie l'hypothèse d'un inconscient ? \\[0pt]
Qu'est-ce qui justifie une croyance ? \\[0pt]
Qu'est-ce qu'imaginer ? \\[0pt]
Qu'est-ce qui m'appartient ? \\[0pt]
Qu'est-ce qui menace la liberté ? \\[0pt]
Qu'est-ce qui me rend plus fort ? \\[0pt]
Qu'est-ce qui mérite l'admiration ? \\[0pt]
Qu'est-ce qui m'est étranger ? \\[0pt]
Qu'est-ce qui mesure la valeur d'un travail ? \\[0pt]
Qu'est-ce qui n'a pas d'histoire ? \\[0pt]
Qu'est-ce qui n'appartient pas au monde ? \\[0pt]
Qu'est-ce qui ne change pas ? \\[0pt]
Qu'est-ce qui ne disparaît jamais ? \\[0pt]
Qu'est-ce qui ne s'achète pas ? \\[0pt]
Qu'est-ce qui ne s'échange pas ? \\[0pt]
Qu'est-ce qui n'est pas démontrable ? \\[0pt]
Qu'est-ce qui n'est pas en mouvement ? \\[0pt]
Qu'est-ce qui n'est pas maîtrisable ? \\[0pt]
Qu'est-ce qui n'est pas mathématisable ? \\[0pt]
Qu'est-ce qui n'est pas normal ? \\[0pt]
Qu'est-ce qui n'est pas politique ? \\[0pt]
Qu'est-ce qui n'est pas technique dans l'activité humaine ? \\[0pt]
Qu'est-ce qui n'existe pas ? \\[0pt]
Qu'est-ce qui nous échappe dans le temps ? \\[0pt]
Qu'est-ce qui nous fait danser ? \\[0pt]
Qu'est-ce qui nous oblige ? \\[0pt]
Qu'est-ce qu'interpréter ? \\[0pt]
Qu'est-ce qu'interpréter une œuvre d'art ? \\[0pt]
Qu'est-ce qu'interpréter un texte ? \\[0pt]
Qu'est-ce qui oppose les hommes ? \\[0pt]
Qu'est-ce qui peut être hors du temps ? \\[0pt]
Qu'est-ce qui peut se transformer ? \\[0pt]
Qu'est-ce qui plaît dans la musique ? \\[0pt]
Qu'est-ce qui rapproche le vivant de la machine ? \\[0pt]
Qu'est-ce qui rend l'objectivité difficile dans les sciences humaines ? \\[0pt]
Qu'est-ce qui rend une parole poétique ? \\[0pt]
Qu'est-ce qui rend vrai un énoncé ? \\[0pt]
Qu'est-ce qu'obéir ? \\[0pt]
Qu'est-ce qu'on attend ? \\[0pt]
Qu'est-ce qu'on ne peut comprendre ? \\[0pt]
Qu'est-ce qu'on ne peut pas partager ? \\[0pt]
Qu'est-ce qu'un abus de langage ? \\[0pt]
Qu'est-ce qu'un abus de pouvoir ? \\[0pt]
Qu'est-ce qu'un accident ? \\[0pt]
Qu'est-ce qu'un acte ? \\[0pt]
Qu'est-ce qu'un acte libre ? \\[0pt]
Qu'est-ce qu'un acte moral ? \\[0pt]
Qu'est-ce qu'un acte symbolique ? \\[0pt]
Qu'est-ce qu'un acteur ? \\[0pt]
Qu'est-ce qu'un adversaire en politique ? \\[0pt]
Qu'est-ce qu'un alter ego ? \\[0pt]
Qu'est-ce qu'un ami ? \\[0pt]
Qu'est-ce qu'un animal ? \\[0pt]
Qu'est-ce qu'un animal domestique ? \\[0pt]
Qu'est-ce qu'un argument ? \\[0pt]
Qu'est-ce qu'un art abstrait ? \\[0pt]
Qu'est-ce qu'un art de vivre ? \\[0pt]
Qu'est-ce qu'un artefact ? \\[0pt]
Qu'est-ce qu'un artiste ? \\[0pt]
Qu'est-ce qu'un art moral ? \\[0pt]
Qu'est-ce qu'un art populaire ? \\[0pt]
Qu'est-ce qu'un auteur ? \\[0pt]
Qu'est-ce qu'un axiome ? \\[0pt]
Qu'est-ce qu'un beau parleur ? \\[0pt]
Qu'est-ce qu'un beau paysage ? \\[0pt]
Qu'est-ce qu'un beau travail ? \\[0pt]
Qu'est-ce qu'un bon argument ? \\[0pt]
Qu'est-ce qu'un bon citoyen ? \\[0pt]
Qu'est-ce qu'un bon commentaire ? \\[0pt]
Qu'est-ce qu'un bon conseil ? \\[0pt]
Qu'est-ce qu'un bon gouvernement ? \\[0pt]
Qu'est-ce qu'un bon jugement ? \\[0pt]
Qu'est-ce qu'un capital culturel ? \\[0pt]
Qu'est-ce qu'un caractère ? \\[0pt]
Qu'est-ce qu'un cas de conscience ? \\[0pt]
Qu'est-ce qu'un cas en morale ? \\[0pt]
Qu'est-ce qu'un « champ artistique » ? \\[0pt]
Qu'est-ce qu'un châtiment ? \\[0pt]
Qu'est-ce qu'un chef ? \\[0pt]
Qu'est-ce qu'un chef-d'œuvre ? \\[0pt]
Qu'est-ce qu'un choix éclairé ? \\[0pt]
Qu'est-ce qu'un choix rationnel ? \\[0pt]
Qu'est-ce qu'un citoyen ? \\[0pt]
Qu'est-ce qu'un citoyen libre ? \\[0pt]
Qu'est-ce qu'un civilisé ? \\[0pt]
Qu'est-ce qu'un classique ? \\[0pt]
Qu'est-ce qu'un code ? \\[0pt]
Qu'est-ce qu'un collectif ? \\[0pt]
Qu'est-ce qu'un concept ? \\[0pt]
Qu'est-ce qu'un concept philosophique ? \\[0pt]
Qu'est-ce qu'un concept scientifique ? \\[0pt]
Qu'est-ce qu'un conflit de devoirs ? \\[0pt]
Qu'est-ce qu'un conflit de générations ? \\[0pt]
Qu'est-ce qu'un conflit de valeurs ? \\[0pt]
Qu'est-ce qu'un conflit politique ? \\[0pt]
Qu'est-ce qu'un consommateur ? \\[0pt]
Qu'est-ce qu'un contenu de conscience ? \\[0pt]
Qu'est-ce qu'un contrat ? \\[0pt]
Qu'est-ce qu'un contre-pouvoir ? \\[0pt]
Qu'est-ce qu'un corps ? \\[0pt]
Qu'est-ce qu'un corps politique ? \\[0pt]
Qu'est-ce qu'un corps social ? \\[0pt]
Qu'est-ce qu'un coup d'État ? \\[0pt]
Qu'est-ce qu'un créateur ? \\[0pt]
Qu'est-ce qu'un crime ? \\[0pt]
Qu'est-ce qu'un crime contre l'humanité ? \\[0pt]
Qu'est-ce qu'un crime politique ? \\[0pt]
Qu'est-ce qu'un critère de vérité ? \\[0pt]
Qu'est-ce qu'un déni ? \\[0pt]
Qu'est-ce qu'un désir satisfait ? \\[0pt]
Qu'est-ce qu'un détail ? \\[0pt]
Qu'est-ce qu'un dialogue ? \\[0pt]
Qu'est-ce qu'un dieu ? \\[0pt]
Qu'est-ce qu'un Dieu ? \\[0pt]
Qu'est-ce qu'un dilemme ? \\[0pt]
Qu'est-ce qu'un document ? \\[0pt]
Qu'est-ce qu'un dogme ? \\[0pt]
Qu'est-ce qu'un doute raisonnable ? \\[0pt]
Qu'est-ce qu'un droit naturel ? \\[0pt]
Qu'est-ce qu'une action intentionnelle ? \\[0pt]
Qu'est-ce qu'une action juste ? \\[0pt]
Qu'est-ce qu'une action politique ? \\[0pt]
Qu'est-ce qu'une action réussie ? \\[0pt]
Qu'est-ce qu'une alternative ? \\[0pt]
Qu'est-ce qu'une âme ? \\[0pt]
Qu'est-ce qu'une analyse ? \\[0pt]
Qu'est-ce qu'une anomalie ? \\[0pt]
Qu'est-ce qu'une aporie ? \\[0pt]
Qu'est-ce qu'une autorité légitime ? \\[0pt]
Qu'est-ce qu'une avant-garde ? \\[0pt]
Qu'est-ce qu'une belle action ? \\[0pt]
Qu'est-ce qu'une belle démonstration ? \\[0pt]
Qu'est-ce qu'une belle forme ? \\[0pt]
Qu'est-ce qu'une belle mort ? \\[0pt]
Qu'est-ce qu'une bête ? \\[0pt]
Qu'est-ce qu'une bonne définition ? \\[0pt]
Qu'est-ce qu'une bonne délibération ? \\[0pt]
Qu'est-ce qu'une bonne éducation ? \\[0pt]
Qu'est-ce qu'une bonne loi ? \\[0pt]
Qu'est-ce qu'une bonne méthode ? \\[0pt]
Qu'est-ce qu'une bonne théorie ? \\[0pt]
Qu'est-ce qu'une bonne traduction ? \\[0pt]
Qu'est-ce qu'une catastrophe ? \\[0pt]
Qu'est-ce qu'une catégorie ? \\[0pt]
Qu'est-ce qu'une catégorie de l'être ? \\[0pt]
Qu'est-ce qu'une cause ? \\[0pt]
Qu'est-ce qu'un échange juste ? \\[0pt]
Qu'est-ce qu'un échange réussi ? \\[0pt]
Qu'est-ce qu'une chimère ? \\[0pt]
Qu'est-ce qu'une chose ? \\[0pt]
Qu'est-ce qu'une chose matérielle ? \\[0pt]
Qu'est-ce qu'une civilisation ? \\[0pt]
Qu'est-ce qu'une classe sociale ? \\[0pt]
Qu'est-ce qu'une classification scientifique ? \\[0pt]
Qu'est-ce qu'une collectivité ? \\[0pt]
Qu'est-ce qu'une comédie ? \\[0pt]
Qu'est-ce qu'une communauté ? \\[0pt]
Qu'est-ce qu'une communauté humaine ? \\[0pt]
Qu'est-ce qu'une communauté politique ? \\[0pt]
Qu'est-ce qu'une communauté scientifique ? \\[0pt]
Qu'est-ce qu'une conception scientifique du monde ? \\[0pt]
Qu'est-ce qu'une condition de possibilité ? \\[0pt]
Qu'est-ce qu'une condition suffisante ? \\[0pt]
Qu'est-ce qu'une conduite irrationnelle ? \\[0pt]
Qu'est-ce qu'une connaissance a priori ? \\[0pt]
Qu'est-ce qu'une connaissance fiable ? \\[0pt]
Qu'est-ce qu'une connaissance non scientifique ? \\[0pt]
Qu'est-ce qu'une connaissance par les faits ? \\[0pt]
Qu'est-ce qu'une conscience collective ? \\[0pt]
Qu'est-ce qu'une constitution ? \\[0pt]
Qu'est-ce qu'une contradiction ? \\[0pt]
Qu'est-ce qu'une contrainte ? \\[0pt]
Qu'est-ce qu'une convention ? \\[0pt]
Qu'est-ce qu'une conviction ? \\[0pt]
Qu'est-ce qu'une crise ? \\[0pt]
Qu'est-ce qu'une crise politique ? \\[0pt]
Qu'est-ce qu'une croyance ? \\[0pt]
Qu'est-ce qu'une croyance rationnelle ? \\[0pt]
Qu'est-ce qu'une croyance vraie ? \\[0pt]
Qu'est-ce qu'une culture ? \\[0pt]
Qu'est-ce qu'une décision politique ? \\[0pt]
Qu'est-ce qu'une décision rationnelle ? \\[0pt]
Qu'est-ce qu'une découverte ? \\[0pt]
Qu'est-ce qu'une découverte scientifique ? \\[0pt]
Qu'est-ce qu'une définition ? \\[0pt]
Qu'est-ce qu'une démocratie ? \\[0pt]
Qu'est-ce qu'une démonstration ? \\[0pt]
Qu'est-ce qu'une discipline savante ? \\[0pt]
Qu'est-ce qu'une disposition ? \\[0pt]
Qu'est-ce qu'une école philosophique ? \\[0pt]
Qu'est-ce qu'une éducation réussie ? \\[0pt]
Qu'est-ce qu'une éducation scientifique ? \\[0pt]
Qu'est-ce qu'une encyclopédie ? \\[0pt]
Qu'est-ce qu'une époque ? \\[0pt]
Qu'est-ce qu'une erreur ? \\[0pt]
Qu'est-ce qu'une espèce naturelle ? \\[0pt]
Qu'est-ce qu'une exception ? \\[0pt]
Qu'est-ce qu'une existence historique ? \\[0pt]
Qu'est-ce qu'une expérience ? \\[0pt]
Qu'est-ce qu'une expérience cruciale ? \\[0pt]
Qu'est-ce qu'une « expérience de pensée » ? \\[0pt]
Qu'est-ce qu'une expérience de pensée ? \\[0pt]
Qu'est-ce qu'une expérience esthétique ? \\[0pt]
Qu'est-ce qu'une expérience politique ? \\[0pt]
Qu'est-ce qu'une expérience religieuse ? \\[0pt]
Qu'est-ce qu'une expérience scientifique ? \\[0pt]
Qu'est-ce qu'une explication matérialiste ? \\[0pt]
Qu'est-ce qu'une exposition ? \\[0pt]
Qu'est-ce qu'une famille ? \\[0pt]
Qu'est-ce qu'une fausse science ? \\[0pt]
Qu'est-ce qu'une faute de goût ? \\[0pt]
Qu'est-ce qu'une fiction ? \\[0pt]
Qu'est-ce qu'une fonction ? \\[0pt]
Qu'est-ce qu'une forme ? \\[0pt]
Qu'est-ce qu'une grande cause ? \\[0pt]
Qu'est-ce qu'une guerre juste ? \\[0pt]
Qu'est-ce qu'une histoire vraie ? \\[0pt]
Qu'est-ce qu'une hypothèse ? \\[0pt]
Qu'est-ce qu'une hypothèse scientifique ? \\[0pt]
Qu'est-ce qu'une idée ? \\[0pt]
Qu'est-ce qu'une idée esthétique ? \\[0pt]
Qu'est-ce qu'une idée fausse ? \\[0pt]
Qu'est-ce qu'une idée incertaine ? \\[0pt]
Qu'est-ce qu'une idée morale ? \\[0pt]
Qu'est-ce qu'une idée vraie ? \\[0pt]
Qu'est-ce qu'une idéologie ? \\[0pt]
Qu'est-ce qu'une idéologie scientifique ? \\[0pt]
Qu'est-ce qu'une illusion ? \\[0pt]
Qu'est-ce qu'une image ? \\[0pt]
Qu'est-ce qu'une impression ? \\[0pt]
Qu'est-ce qu'une inégalité ? \\[0pt]
Qu'est-ce qu'une inférence ? \\[0pt]
Qu'est-ce qu'une injustice ? \\[0pt]
Qu'est-ce qu'une innovation technologique ? \\[0pt]
Qu'est-ce qu'une institution ? \\[0pt]
Qu'est-ce qu'une interprétation ? \\[0pt]
Qu'est-ce qu'une invention technique ? \\[0pt]
Qu'est-ce qu'une langue ? \\[0pt]
Qu'est-ce qu'une langue artificielle ? \\[0pt]
Qu'est-ce qu'une langue bien faite ? \\[0pt]
Qu'est-ce qu'une langue morte ? \\[0pt]
Qu'est-ce qu'un élément ? \\[0pt]
Qu'est-ce qu'une libération ? \\[0pt]
Qu'est-ce qu'une liberté fondamentale ? \\[0pt]
Qu'est-ce qu'une libre interprétation ? \\[0pt]
Qu'est-ce qu'une limite ? \\[0pt]
Qu'est-ce qu'une logique sociale ? \\[0pt]
Qu'est-ce qu'une loi ? \\[0pt]
Qu'est-ce qu'une loi de la nature ? \\[0pt]
Qu'est-ce qu'une loi de la pensée ? \\[0pt]
Qu'est-ce qu'une loi d'exception ? \\[0pt]
Qu'est-ce qu'une loi injuste ? \\[0pt]
Qu'est-ce qu'une loi scientifique ? \\[0pt]
Qu'est-ce qu'une machine ? \\[0pt]
Qu'est-ce qu'une machine ne peut pas faire ? \\[0pt]
Qu'est-ce qu'une maladie ? \\[0pt]
Qu'est-ce qu'une marchandise ? \\[0pt]
Qu'est-ce qu'une mauvaise idée ? \\[0pt]
Qu'est-ce qu'une mauvaise interprétation ? \\[0pt]
Qu'est-ce qu'une méditation ? \\[0pt]
Qu'est-ce qu'une méditation métaphysique ? \\[0pt]
Qu'est-ce qu'une mentalité collective ? \\[0pt]
Qu'est-ce qu'une métaphore ? \\[0pt]
Qu'est-ce qu'une méthode ? \\[0pt]
Qu'est-ce qu'une minorité ? \\[0pt]
Qu'est-ce qu'une morale de la communication ? \\[0pt]
Qu'est-ce qu'un empire ? \\[0pt]
Qu'est-ce qu'une nation ? \\[0pt]
Qu'est-ce qu'un enfant ? \\[0pt]
Qu'est-ce qu'un ennemi ? \\[0pt]
Qu'est-ce qu'un ennemi politique ? \\[0pt]
Qu'est-ce qu'un énoncé scientifique sur l'homme ? \\[0pt]
Qu'est-ce qu'une norme ? \\[0pt]
Qu'est-ce qu'une norme sociale ? \\[0pt]
Qu'est-ce qu'une nouveauté ? \\[0pt]
Qu'est-ce qu'une occasion ? \\[0pt]
Qu'est-ce qu'une œuvre ? \\[0pt]
Qu'est-ce qu'une œuvre classique ? \\[0pt]
Qu'est-ce qu'une œuvre d'art ? \\[0pt]
Qu'est-ce qu'une œuvre d'art authentique ? \\[0pt]
Qu'est-ce qu'une œuvre d'art réaliste ? \\[0pt]
Qu'est-ce qu'une œuvre d'art réussie ? \\[0pt]
Qu'est-ce qu'une œuvre « géniale » ? \\[0pt]
Qu'est-ce qu'une œuvre ratée ? \\[0pt]
Qu'est-ce qu'une parole en l'air ? \\[0pt]
Qu'est-ce qu'une parole juste ? \\[0pt]
Qu'est-ce qu'une parole libre ? \\[0pt]
Qu'est-ce qu'une parole vivante ? \\[0pt]
Qu'est-ce qu'une parole vraie ? \\[0pt]
Qu'est-ce qu'une passion ? \\[0pt]
Qu'est-ce qu'une patrie ? \\[0pt]
Qu'est-ce qu'une peinture abstraite ? \\[0pt]
Qu'est-ce qu'une pensée libre ? \\[0pt]
Qu'est-ce qu'une perception sensible ? \\[0pt]
Qu'est-ce qu'une « performance » ? \\[0pt]
Qu'est-ce qu'une période en histoire ? \\[0pt]
Qu'est-ce qu'une personne ? \\[0pt]
Qu'est-ce qu'une personne morale ? \\[0pt]
Qu'est-ce qu'une pétition de principe ? \\[0pt]
Qu'est-ce qu'une philosophie ? \\[0pt]
Qu'est-ce qu'une philosophie de la vie ? \\[0pt]
Qu'est-ce qu'une philosophie première ? \\[0pt]
Qu'est-ce qu'une phrase ? \\[0pt]
Qu'est-ce qu'une politique sociale ? \\[0pt]
Qu'est-ce qu'une preuve ? \\[0pt]
Qu'est-ce qu'une preuve scientifique ? \\[0pt]
Qu'est-ce qu'une promesse ? \\[0pt]
Qu'est-ce qu'une propriété ? \\[0pt]
Qu'est-ce qu'une propriété essentielle ? \\[0pt]
Qu'est-ce qu'une pseudoscience ? \\[0pt]
Qu'est-ce qu'une psychologie scientifique ? \\[0pt]
Qu'est-ce qu'une question ? \\[0pt]
Qu'est-ce qu'une question dénuée de sens ? \\[0pt]
Qu'est-ce qu'une question métaphysique ? \\[0pt]
Qu'est-ce qu'une raison d'agir ? \\[0pt]
Qu'est-ce qu'une réfutation ? \\[0pt]
Qu'est-ce qu'une règle ? \\[0pt]
Qu'est-ce qu'une règle de grammaire ? \\[0pt]
Qu'est-ce qu'une règle de vie ? \\[0pt]
Qu'est-ce qu'une règle sociale ? \\[0pt]
Qu'est-ce qu'une relation ? \\[0pt]
Qu'est-ce qu'une religion ? \\[0pt]
Qu'est-ce qu'une rencontre ? \\[0pt]
Qu'est-ce qu'une représentation ? \\[0pt]
Qu'est-ce qu'une représentation du monde ? \\[0pt]
Qu'est-ce qu'une représentation réussie ? \\[0pt]
Qu'est-ce qu'une république ? \\[0pt]
Qu'est-ce qu'une révélation ? \\[0pt]
Qu'est-ce qu'une révolution ? \\[0pt]
Qu'est-ce qu'une révolution artistique ? \\[0pt]
Qu'est-ce qu'une révolution politique ? \\[0pt]
Qu'est-ce qu'une révolution scientifique ? \\[0pt]
Qu'est-ce qu'une science exacte ? \\[0pt]
Qu'est-ce qu'une science expérimentale ? \\[0pt]
Qu'est-ce qu'une science humaine ? \\[0pt]
Qu'est-ce qu'une science rigoureuse ? \\[0pt]
Qu'est-ce qu'un esclave ? \\[0pt]
Qu'est-ce qu'une secte ? \\[0pt]
Qu'est-ce qu'une situation ? \\[0pt]
Qu'est-ce qu'une situation tragique ? \\[0pt]
Qu'est-ce qu'une société ? \\[0pt]
Qu'est-ce qu'une société juste ? \\[0pt]
Qu'est-ce qu'une société libre ? \\[0pt]
Qu'est-ce qu'une société mondialisée ? \\[0pt]
Qu'est-ce qu'une société ouverte ? \\[0pt]
Qu'est-ce qu'une société primitive ? \\[0pt]
Qu'est-ce qu'une solution ? \\[0pt]
Qu'est-ce qu'un esprit faux ? \\[0pt]
Qu'est-ce qu'un esprit juste ? \\[0pt]
Qu'est-ce qu'un esprit libre ? \\[0pt]
Qu'est-ce qu'un esprit profond ? \\[0pt]
Qu'est-ce qu'une structure ? \\[0pt]
Qu'est-ce qu'une substance ? \\[0pt]
Qu'est-ce qu'un État de droit ? \\[0pt]
Qu'est-ce qu'un État libre ? \\[0pt]
Qu'est-ce qu'un état mental ? \\[0pt]
Qu'est-ce qu'un État souverain ? \\[0pt]
Qu'est-ce qu'une théorie ? \\[0pt]
Qu'est-ce qu'une théorie scientifique ? \\[0pt]
Qu'est-ce qu'une tradition ? \\[0pt]
Qu'est-ce qu'une tragédie ? \\[0pt]
Qu'est-ce qu'une tragédie historique ? \\[0pt]
Qu'est-ce qu'un être cultivé ? \\[0pt]
Qu'est-ce qu'un « être dégénéré » ? \\[0pt]
Qu'est-ce qu'un être imaginaire ? \\[0pt]
Qu'est-ce qu'un être vivant ? \\[0pt]
Qu'est-ce qu'une valeur ? \\[0pt]
Qu'est-ce qu'une valeur esthétique ? \\[0pt]
Qu'est-ce qu'un événement ? \\[0pt]
Qu'est-ce qu'un événement fondateur ? \\[0pt]
Qu'est-ce qu'un événement historique ? \\[0pt]
Qu'est-ce qu'une vérité contingente ? \\[0pt]
Qu'est-ce qu'une vérité historique ? \\[0pt]
Qu'est-ce qu'une vérité scientifique ? \\[0pt]
Qu'est-ce qu'une vérité subjective ? \\[0pt]
Qu'est-ce qu'une vertu ? \\[0pt]
Qu'est-ce qu'une vie d'artiste ? \\[0pt]
Qu'est-ce qu'une vie heureuse ? \\[0pt]
Qu'est-ce qu'une vie humaine ? \\[0pt]
Qu'est-ce qu'une vie réussie ? \\[0pt]
Qu'est-ce qu'une ville ? \\[0pt]
Qu'est-ce qu'une violence symbolique ? \\[0pt]
Qu'est-ce qu'une vision du monde ? \\[0pt]
Qu'est-ce qu'une vision scientifique du monde ? \\[0pt]
Qu'est-ce qu'une volonté libre ? \\[0pt]
Qu'est-ce qu'une volonté raisonnable ? \\[0pt]
Qu'est-ce qu'un exemple ? \\[0pt]
Qu'est-ce qu'un expérimentateur ? \\[0pt]
Qu'est-ce qu'un expert ? \\[0pt]
Qu'est-ce qu'un fait ? \\[0pt]
Qu'est-ce qu'un fait culturel ? \\[0pt]
Qu'est-ce qu'un fait de culture ? \\[0pt]
Qu'est-ce qu'un fait de société ? \\[0pt]
Qu'est-ce qu'un fait divers ? \\[0pt]
Qu'est-ce qu'un fait historique ? \\[0pt]
Qu'est-ce qu'un fait moral ? \\[0pt]
Qu'est-ce qu'un fait religieux ? \\[0pt]
Qu'est-ce qu'un fait scientifique ? \\[0pt]
Qu'est-ce qu'un fait social ? \\[0pt]
Qu'est-ce qu'un faux ? \\[0pt]
Qu'est-ce qu'un faux problème ? \\[0pt]
Qu'est-ce qu'un faux sentiment ? \\[0pt]
Qu'est-ce qu'un film ? \\[0pt]
Qu'est-ce qu'un génie ? \\[0pt]
Qu'est-ce qu'un geste artistique ? \\[0pt]
Qu'est-ce qu'un geste technique ? \\[0pt]
Qu'est-ce qu'un gouvernement ? \\[0pt]
Qu'est-ce qu'un gouvernement démocratique ? \\[0pt]
Qu'est-ce qu'un gouvernement juste ? \\[0pt]
Qu'est-ce qu'un gouvernement républicain ? \\[0pt]
Qu'est-ce qu'un grand homme ? \\[0pt]
Qu'est-ce qu'un grand homme ou une grande femme ? \\[0pt]
Qu'est-ce qu'un grand philosophe ? \\[0pt]
Qu'est-ce qu'un héros ? \\[0pt]
Qu'est-ce qu'un homme bon ? \\[0pt]
Qu'est-ce qu'un homme d'action ? \\[0pt]
Qu'est-ce qu'un homme d'État ? \\[0pt]
Qu'est-ce qu'un homme d'expérience ? \\[0pt]
Qu'est-ce qu'un homme juste ? \\[0pt]
Qu'est-ce qu'un homme libre ? \\[0pt]
Qu'est-ce qu'un homme méchant ? \\[0pt]
Qu'est-ce qu'un homme normal ? \\[0pt]
Qu'est-ce qu'un homme politique ? \\[0pt]
Qu'est-ce qu'un homme sans éducation ? \\[0pt]
Qu'est-ce qu'un homme seul ? \\[0pt]
Qu'est-ce qu'un idéal ? \\[0pt]
Qu'est-ce qu'un idéaliste ? \\[0pt]
Qu'est-ce qu'un idéal moral ? \\[0pt]
Qu'est-ce qu'un imaginaire social ? \\[0pt]
Qu'est-ce qu'un impératif technique ? \\[0pt]
Qu'est-ce qu'un individu ? \\[0pt]
Qu'est-ce qu'un intellectuel ? \\[0pt]
Qu'est-ce qu'un jeu ? \\[0pt]
Qu'est-ce qu'un juge ? \\[0pt]
Qu'est-ce qu'un jugement analytique ? \\[0pt]
Qu'est-ce qu'un jugement de goût ? \\[0pt]
Qu'est-ce qu'un juste ? \\[0pt]
Qu'est-ce qu'un juste salaire ? \\[0pt]
Qu'est-ce qu'un justicier ? \\[0pt]
Qu'est-ce qu'un laboratoire ? \\[0pt]
Qu'est-ce qu'un langage technique ? \\[0pt]
Qu'est-ce qu'un législateur ? \\[0pt]
Qu'est-ce qu'un lien logique ? \\[0pt]
Qu'est-ce qu'un lieu ? \\[0pt]
Qu'est-ce qu'un lieu commun ? \\[0pt]
Qu'est-ce qu'un livre ? \\[0pt]
Qu'est-ce qu'un maître ? \\[0pt]
Qu'est-ce qu'un marginal ? \\[0pt]
Qu'est-ce qu'un mécanisme social ? \\[0pt]
Qu'est-ce qu'un métaphysicien ? \\[0pt]
Qu'est-ce qu'un métier ? \\[0pt]
Qu'est-ce qu'un mineur ? \\[0pt]
Qu'est-ce qu'un miracle ? \\[0pt]
Qu'est-ce qu'un modèle ? \\[0pt]
Qu'est-ce qu'un moderne ? \\[0pt]
Qu'est-ce qu'un moment ? \\[0pt]
Qu'est-ce qu'un monde ? \\[0pt]
Qu'est-ce qu'un monde commun ? \\[0pt]
Qu'est-ce qu'un monstre ? \\[0pt]
Qu'est-ce qu'un monument ? \\[0pt]
Qu'est-ce qu'un mouvement politique ? \\[0pt]
Qu'est-ce qu'un musée ? \\[0pt]
Qu'est-ce qu'un mythe ? \\[0pt]
Qu'est-ce qu'un nombre ? \\[0pt]
Qu'est-ce qu'un nom propre ? \\[0pt]
Qu'est-ce qu'un objet ? \\[0pt]
Qu'est-ce qu'un objet d'art ? \\[0pt]
Qu'est-ce qu'un objet de connaissance ? \\[0pt]
Qu'est-ce qu'un objet de science ? \\[0pt]
Qu'est-ce qu'un objet esthétique ? \\[0pt]
Qu'est-ce qu'un objet mathématique ? \\[0pt]
Qu'est-ce qu'un objet métaphysique ? \\[0pt]
Qu'est-ce qu'un objet technique ? \\[0pt]
Qu'est-ce qu'un œuvre d'art ? \\[0pt]
Qu'est-ce qu'un ordre ? \\[0pt]
Qu'est-ce qu'un organisme ? \\[0pt]
Qu'est-ce qu'un original ? \\[0pt]
Qu'est-ce qu'un outil ? \\[0pt]
Qu'est-ce qu'un paradoxe ? \\[0pt]
Qu'est-ce qu'un passe-temps ? \\[0pt]
Qu'est-ce qu'un patrimoine ? \\[0pt]
Qu'est-ce qu'un pauvre ? \\[0pt]
Qu'est-ce qu'un paysage ? \\[0pt]
Qu'est-ce qu'un pédant ? \\[0pt]
Qu'est-ce qu'un peuple ? \\[0pt]
Qu'est-ce qu'un peuple libre ? \\[0pt]
Qu'est-ce qu'un phénomène ? \\[0pt]
Qu'est-ce qu'un philosophe ? \\[0pt]
Qu'est-ce qu'un plaisir pur ? \\[0pt]
Qu'est-ce qu'un point de vue ? \\[0pt]
Qu'est-ce qu'un portrait ? \\[0pt]
Qu'est-ce qu'un post-moderne ? \\[0pt]
Qu'est-ce qu'un postulat ? \\[0pt]
Qu'est-ce qu'un pouvoir despotique ? \\[0pt]
Qu'est-ce qu'un pouvoir légitime ? \\[0pt]
Qu'est-ce qu'un précurseur ? \\[0pt]
Qu'est-ce qu'un préjugé ? \\[0pt]
Qu'est-ce qu'un primitif ? \\[0pt]
Qu'est-ce qu'un prince juste ? \\[0pt]
Qu'est-ce qu'un principe ? \\[0pt]
Qu'est-ce qu'un problème ? \\[0pt]
Qu'est-ce qu'un problème éthique ? \\[0pt]
Qu'est-ce qu'un problème insoluble ? \\[0pt]
Qu'est-ce qu'un problème métaphysique ? \\[0pt]
Qu'est-ce qu'un problème philosophique ? \\[0pt]
Qu'est-ce qu'un problème politique ? \\[0pt]
Qu'est-ce qu'un problème scientifique ? \\[0pt]
Qu'est-ce qu'un problème technique ? \\[0pt]
Qu'est-ce qu'un produit culturel ? \\[0pt]
Qu'est-ce qu'un programme ? \\[0pt]
Qu'est-ce qu'un programme politique ? \\[0pt]
Qu'est-ce qu'un programmer ? \\[0pt]
Qu'est-ce qu'un progrès scientifique ? \\[0pt]
Qu'est-ce qu'un progrès technique ? \\[0pt]
Qu'est-ce qu'un projet ? \\[0pt]
Qu'est-ce qu'un prophète ? \\[0pt]
Qu'est-ce qu'un public ? \\[0pt]
Qu'est-ce qu'un pur esprit ? \\[0pt]
Qu'est-ce qu'un rapport de force ? \\[0pt]
Qu'est-ce qu'un récit ? \\[0pt]
Qu'est-ce qu'un récit véridique ? \\[0pt]
Qu'est-ce qu'un réfugié politique ? \\[0pt]
Qu'est-ce qu'un réfutation ? \\[0pt]
Qu'est-ce qu'un régime politique ? \\[0pt]
Qu'est-ce qu'un réseau ? \\[0pt]
Qu'est-ce qu'un rhéteur ? \\[0pt]
Qu'est-ce qu'un rite ? \\[0pt]
Qu'est-ce qu'un rival ? \\[0pt]
Qu'est-ce qu'un sage ? \\[0pt]
Qu'est-ce qu'un savoir-faire ? \\[0pt]
Qu'est-ce qu'un sceptique ? \\[0pt]
Qu'est-ce qu'un sentiment moral ? \\[0pt]
Qu'est-ce qu'un sentiment vrai ? \\[0pt]
Qu'est-ce qu'un signe ? \\[0pt]
Qu'est-ce qu'un sophisme ? \\[0pt]
Qu'est-ce qu'un sophiste ? \\[0pt]
Qu'est-ce qu'un souvenir ? \\[0pt]
Qu'est-ce qu'un spécialiste ? \\[0pt]
Qu'est-ce qu'un spectacle ? \\[0pt]
Qu'est-ce qu'un spectateur ? \\[0pt]
Qu'est-ce qu'un style ? \\[0pt]
Qu'est-ce qu'un sujet de droit ? \\[0pt]
Qu'est-ce qu'un symbole ? \\[0pt]
Qu'est-ce qu'un symptôme ? \\[0pt]
Qu'est-ce qu'un système ? \\[0pt]
Qu'est-ce qu'un système philosophique ? \\[0pt]
Qu'est-ce qu'un tableau ? \\[0pt]
Qu'est-ce qu'un tabou ? \\[0pt]
Qu'est-ce qu'un technicien ? \\[0pt]
Qu'est-ce qu'un témoignage ? \\[0pt]
Qu'est-ce qu'un témoin ? \\[0pt]
Qu'est-ce qu'un temple ? \\[0pt]
Qu'est-ce qu'un territoire ? \\[0pt]
Qu'est-ce qu'un terroriste ? \\[0pt]
Qu'est-ce qu'un texte ? \\[0pt]
Qu'est-ce qu'un tout ? \\[0pt]
Qu'est-ce qu'un traître ? \\[0pt]
Qu'est-ce qu'un travail bien fait ? \\[0pt]
Qu'est-ce qu'un trouble social ? \\[0pt]
Qu'est-ce qu'un tyran ? \\[0pt]
Qu'est-ce qu'un vice ? \\[0pt]
Qu'est-ce qu'un visage ? \\[0pt]
Qu'est-ce qu'un vrai changement ? \\[0pt]
Qu'est-il légitime d'interdire ? \\[0pt]
Qu'est-il utile d'interdire ? \\[0pt]
Qu'est que l'autorité ? \\[0pt]
Qu'est qu'une image ? \\[0pt]
Qu'est qu'un régime politique ? \\[0pt]
Que suis-je ? \\[0pt]
Que suppose le mouvement ? \\[0pt]
Que transmettons-nous ? \\[0pt]
Que trouve-t-on dans ce que l'on trouve beau ? \\[0pt]
Que valent les classifications en sciences humaines ? \\[0pt]
Que valent les décisions de la majorité ? \\[0pt]
Que valent les excuses ? \\[0pt]
Que valent les idées générales ? \\[0pt]
Que valent les mots ? \\[0pt]
Que valent les préjugés ? \\[0pt]
Que valent les preuves ? \\[0pt]
Que valent les théories ? \\[0pt]
« Que va-t-il se passer ? » \\[0pt]
Que vaut en morale la justification par l'utilité ? \\[0pt]
Que vaut la décision de la majorité ? \\[0pt]
Que vaut la définition de l'homme comme animal doué de raison ? \\[0pt]
Que vaut la distinction entre nature et culture ? \\[0pt]
Que vaut la fidélité ? \\[0pt]
Que vaut le conseil : « vivez avec votre temps » ? \\[0pt]
Que vaut le travail ? \\[0pt]
Que vaut l'excuse : c'est plus fort que moi ? \\[0pt]
Que vaut l'excuse : « C'est plus fort que moi » ? \\[0pt]
Que vaut l'excuse : « Je ne l'ai pas fait exprès » ? \\[0pt]
Que vaut l'incertain ? \\[0pt]
Que vaut un argument d'autorité ? \\[0pt]
Que vaut un consensus ? \\[0pt]
Que vaut une connaissance empirique de l'homme ? \\[0pt]
Que vaut une parole ? \\[0pt]
Que vaut une preuve contre un préjugé ? \\[0pt]
Que veut dire avoir raison ? \\[0pt]
Que veut dire « essentiel » ? \\[0pt]
Que veut dire : être athée ? \\[0pt]
Que veut dire : « être cultivé » ? \\[0pt]
Que veut dire introduire à la métaphysique ? \\[0pt]
Que veut dire « je t'aime » ? \\[0pt]
Que veut dire : « je t'aime » ? \\[0pt]
Que veut dire : « le temps passe » ? \\[0pt]
Que veut dire l'expression « aller au fond des choses » ? \\[0pt]
Que veut dire « réel » ? \\[0pt]
Que veut dire « respecter la nature » ? \\[0pt]
Que veut dire : « respecter la nature » ? \\[0pt]
Que veut-on dire quand on dit que « rien n'est sans raison » ? \\[0pt]
Que veut-on dire quand on dit « rien n'est sans raison » ? \\[0pt]
Que voit-on dans une image ? \\[0pt]
Que voit-on dans un miroir ? \\[0pt]
Que voit-on dans un tableau ? \\[0pt]
Que voulons-nous ? \\[0pt]
Que voulons-nous vraiment savoir ? \\[0pt]
Que voyons-nous ? \\[0pt]
Que voyons-nous sur un tableau ? \\[0pt]
Qu'expriment les mythes ? \\[0pt]
Qu'expriment les œuvres d'art ? \\[0pt]
Qu'exprime une œuvre d'art ? \\[0pt]
Qui a des droits ? \\[0pt]
Qui agit ? \\[0pt]
Qui a le droit de juger ? \\[0pt]
Qui a le droit de punir ? \\[0pt]
Qui a une histoire ? \\[0pt]
Qui a une parole politique ? \\[0pt]
Qui commande ? \\[0pt]
Qui connaît le mieux mon corps ? \\[0pt]
Qui croire ? \\[0pt]
Qui détient le pouvoir ? \\[0pt]
Qui doit dire le droit ? \\[0pt]
Qui doit éduquer ? \\[0pt]
Qui doit faire les lois ? \\[0pt]
Qui doit gouverner ? \\[0pt]
Qui domine ? \\[0pt]
Qui donne la norme du goût ? \\[0pt]
Qui écrit l'histoire ? \\[0pt]
Qui est autorisé à me dire « tu dois » ? \\[0pt]
Qui est citoyen ? \\[0pt]
Qui est comédien ? \\[0pt]
Qui est compétent en matière politique ? \\[0pt]
Qui est crédible ? \\[0pt]
Qui est cultivé ? \\[0pt]
Qui est digne du bonheur ? \\[0pt]
Qui est immoral ? \\[0pt]
Qui est l'autre ? \\[0pt]
Qui est le maître ? \\[0pt]
Qui est le peuple ? \\[0pt]
Qui est l'homme des sciences humaines ? \\[0pt]
Qui est libre ? \\[0pt]
Qui est méchant ? \\[0pt]
Qui est métaphysicien ? \\[0pt]
Qui est mon prochain ? \\[0pt]
Qui est mon semblable ? \\[0pt]
Qui est responsable ? \\[0pt]
Qui est riche ? \\[0pt]
Qui est sage ? \\[0pt]
Qui est souverain ? \\[0pt]
Qui est une personne ? \\[0pt]
Qui fait la loi ? \\[0pt]
Qui fait l'histoire ? \\[0pt]
Qui fait l'unité du peuple ? \\[0pt]
Qui faut-il protéger ? \\[0pt]
Qui gouverne ? \\[0pt]
Qui me dit ce que je dois faire ? \\[0pt]
Qui mérite d'être aimé ? \\[0pt]
Qui meurt ? \\[0pt]
Qui nous dicte nos devoirs ? \\[0pt]
Qui parle ? \\[0pt]
Qui parle quand je dis « je » ? \\[0pt]
Qui pense ? \\[0pt]
Qui peut avoir des droits ? \\[0pt]
Qui peut me dire « tu ne dois pas » ? \\[0pt]
Qui peut obliger ? \\[0pt]
Qui peut parler ? \\[0pt]
Qui peut prétendre énoncer des devoirs ? \\[0pt]
Qui peut prétendre imposer des bornes à la technique ? \\[0pt]
Qui peut se passer de religion ? \\[0pt]
Qui pose les fins ? \\[0pt]
Qui sont mes amis ? \\[0pt]
Qui sont mes semblables ? \\[0pt]
Qui sont nos ennemis ? \\[0pt]
« Qui suis-je ? » \\[0pt]
Qui suis-je ? \\[0pt]
Qui suis-je et qui es-tu ? \\[0pt]
Qui suis-je pour me juger ? \\[0pt]
Qui travaille ? \\[0pt]
Qu'oppose-t-on à la vérité ? \\[0pt]
Qu'y a-t-il ? \\[0pt]
Qu'y a-t-il à comprendre dans une œuvre d'art ? \\[0pt]
Qu'y a-t-il à comprendre en histoire ? \\[0pt]
Qu'y a-t-il à craindre de la technique ? \\[0pt]
Qu'y a-t-il à l'origine de toutes choses ? \\[0pt]
Qu'y a-t-il au-delà de l'être ? \\[0pt]
Qu'y a-t-il au-delà du réel ? \\[0pt]
Qu'y a-t-il au-dessus des lois ? \\[0pt]
Qu'y a-t-il au fondement de l'objectivité ? \\[0pt]
Qu'y a-t-il de sacré ? \\[0pt]
Qu'y a-t-il de sérieux dans le jeu ? \\[0pt]
Qu'y a-t-il de technique dans l'art ? \\[0pt]
Qu'y a-t-il d'universel dans la culture ? \\[0pt]
Qu'y a-t-il que la nature fait en vain ? \\[0pt]
Qu'y a-t-il sinon le réel ? \\[0pt]
Raison et religion sont-elles incompatibles ? \\[0pt]
Rassembler les hommes, est-ce les unir ? \\[0pt]
Rebuts et objets quelconques : une matière pour l'art ? \\[0pt]
Rechercher la vérité, est-ce renoncer à toute opinion ? \\[0pt]
Reconnaissons-nous le bien comme nous reconnaissons le vrai ? \\[0pt]
Recourir au langage, est-ce renoncer à la violence ? \\[0pt]
Résister peut-il être un droit ? \\[0pt]
Respecter autrui, est-ce l'aimer ? \\[0pt]
Respecter la nature, est-ce renoncer à l'exploiter ? \\[0pt]
Respecter la nature, est-ce renoncer à s'en emparer ? \\[0pt]
Ressent-on ou apprécie-t-on l'art ? \\[0pt]
Retenons-nous le temps par le souvenir ? \\[0pt]
Rêver éloigne-t-il de la réalité ? \\[0pt]
Revient-il à l'État d'assurer le bonheur des citoyens ? \\[0pt]
Revient-il à l'État d'assurer votre bonheur ? \\[0pt]
Rêvons-nous ? \\[0pt]
Sait-on ce que l'on veut ? \\[0pt]
Sait-on ce qu'on fait ? \\[0pt]
Sait-on ce qu'on veut ? \\[0pt]
Sait-on nécessairement ce que l'on désire ? \\[0pt]
Sait-on toujours ce que l'on fait ? \\[0pt]
Sait-on toujours ce que l'on veut ? \\[0pt]
Sait-on toujours ce qu'on veut ? \\[0pt]
Sait-on vivre au présent ? \\[0pt]
Sans justice, pas de liberté ? \\[0pt]
Sans l'art, parlerait-on de beauté ? \\[0pt]
Sans les mots, que seraient les choses ? \\[0pt]
Sans référence à la beauté, peut-on rendre compte de l'art ? \\[0pt]
Savoir, est-ce cesser de croire ? \\[0pt]
Savoir, est-ce déjouer le réel ? \\[0pt]
Savoir, est-ce pouvoir ? \\[0pt]
Savoir, est-ce se libérer ? \\[0pt]
Savons-nous ce que nous disons ? \\[0pt]
Savons-nous ce que peut un corps ? \\[0pt]
Savons-nous ce qui nous rend heureux ? \\[0pt]
Savons-nous de ce que nous disons ? \\[0pt]
Savons-nous nous souvenir ? \\[0pt]
Science sans conscience n'est-elle que ruine de l'âme ? \\[0pt]
Sciences humaines et liberté sont-elles compatibles ? \\[0pt]
Sciences humaines ou sciences sociales ? \\[0pt]
Se chercher soi-même, est-ce devenir un autre ? \\[0pt]
Se cultiver, est-ce s'affranchir de son appartenance culturelle ? \\[0pt]
Se force-t-on à faire son devoir ? \\[0pt]
Se mentir à soi-même : est-ce possible ? \\[0pt]
S'engager, est-ce perdre sa liberté ? \\[0pt]
Sentiment et justice sont-ils compatibles ? \\[0pt]
Serait-il immoral d'autoriser le commerce des organes humains ? \\[0pt]
Se réfugier dans la croyance ? \\[0pt]
Se retirer dans la pensée ? \\[0pt]
Serions-nous heureux dans un ordre politique parfait ? \\[0pt]
Serions-nous plus libres sans État ? \\[0pt]
Servir, est-ce nécessairement renoncer à sa liberté ? \\[0pt]
Se sentir libre implique-t-il qu'on le soit ? \\[0pt]
Se souvenir, est-ce préparer l'avenir ? \\[0pt]
« Se trouver » a-t-il un sens ? \\[0pt]
Seul le présent existe-t-il ? \\[0pt]
Seuls les humains sont-ils libres ? \\[0pt]
Si Dieu n'existe pas, tout est-il permis ? \\[0pt]
Si Dieu n'existe pas, tout est-il possible ? \\[0pt]
Si l'esprit n'est pas une table rase, qu'est-il ? \\[0pt]
Si l'État n'existait pas, faudrait-il l'inventer ? \\[0pt]
S'indigner, est-ce un devoir ? \\[0pt]
Si nous étions moraux, le droit serait-il inutile ? \\[0pt]
Si tout est historique, tout est-il relatif ? \\[0pt]
« Sois naturel » : est-ce un bon conseil ? \\[0pt]
« Sois toi-même ! » : un impératif absurde ? \\[0pt]
Sommes-nous adaptés au monde de la technique ? \\[0pt]
Sommes-nous capables d'agir de manière désintéressée ? \\[0pt]
Sommes-nous condamnés à être libres ? \\[0pt]
Sommes-nous conscients de nos mobiles ? \\[0pt]
Sommes-nous dans le temps comme dans l'espace ? \\[0pt]
Sommes-nous des êtres métaphysiques ? \\[0pt]
Sommes-nous des sujets ? \\[0pt]
Sommes-nous déterminés par notre culture ? \\[0pt]
Sommes-nous dominés par la technique ? \\[0pt]
Sommes-nous égaux devant le bonheur ? \\[0pt]
Sommes-nous faits pour être heureux ? \\[0pt]
Sommes-nous faits pour la vérité ? \\[0pt]
Sommes-nous faits pour le bonheur ? \\[0pt]
Sommes-nous faits pour vivre en société ? \\[0pt]
Sommes-nous gouvernés par nos passions ? \\[0pt]
Sommes-nous jamais certains d'avoir choisi librement ? \\[0pt]
Sommes-nous les jouets de l'histoire ? \\[0pt]
Sommes-nous les jouets de nos pulsions ? \\[0pt]
Sommes-nous libres ? \\[0pt]
Sommes-nous libres de nos croyances ? \\[0pt]
Sommes-nous libres de nos pensées ? \\[0pt]
Sommes-nous libres de nos préférences morales ? \\[0pt]
Sommes-nous libres face à l'évidence ? \\[0pt]
Sommes-nous libres lorsque les actes émanent de notre personne ? \\[0pt]
Sommes-nous libres par nature ? \\[0pt]
Sommes-nous maîtres de la nature ? \\[0pt]
Sommes-nous maîtres de nos désirs ? \\[0pt]
Sommes-nous maîtres de nos paroles ? \\[0pt]
Sommes-nous maîtres de nos pensées ? \\[0pt]
Sommes-nous menacés par les progrès techniques ? \\[0pt]
Sommes-nous perfectibles ? \\[0pt]
Sommes-nous portés au bien ? \\[0pt]
Sommes-nous prisonniers de nos désirs ? \\[0pt]
Sommes-nous prisonniers de notre histoire ? \\[0pt]
Sommes-nous prisonniers du temps ? \\[0pt]
Sommes-nous responsables d'autrui ? \\[0pt]
Sommes-nous responsables de ce dont nous n'avons pas conscience ? \\[0pt]
Sommes-nous responsables de ce que nous sommes ? \\[0pt]
Sommes-nous responsables de nos désirs ? \\[0pt]
Sommes-nous responsables de nos erreurs ? \\[0pt]
Sommes-nous responsables de nos opinions ? \\[0pt]
Sommes-nous responsables de nos passions ? \\[0pt]
Sommes-nous responsables de notre être social ? \\[0pt]
Sommes-nous responsables du sens que prennent nos paroles ? \\[0pt]
Sommes-nous séparés du réel par les représentations que nous nous en faisons ? \\[0pt]
Sommes-nous soumis au temps ? \\[0pt]
Sommes-nous sujets de nos désirs ? \\[0pt]
Sommes-nous toujours conscients des causes de nos désirs ? \\[0pt]
Sommes-nous toujours dépendants d'autrui ? \\[0pt]
Sommes-nous tous artistes ? \\[0pt]
Sommes-nous tous contemporains ? \\[0pt]
Sommes-nous vulnérables toute notre vie ? \\[0pt]
S'opposer à l'autorité, est-ce toujours une marque de liberté ? \\[0pt]
Subissons-nous la langue que nous parlons ? \\[0pt]
Suffit-il à Dieu d'être possible pour exister ? \\[0pt]
Suffit-il d'avoir raison ? \\[0pt]
Suffit-il de bien juger pour bien faire ? \\[0pt]
Suffit-il de bien penser pour bien agir ? \\[0pt]
Suffit-il de communiquer pour dialoguer ? \\[0pt]
Suffit-il de démontrer pour convaincre ? \\[0pt]
Suffit-il de faire son devoir ? \\[0pt]
Suffit-il de faire son devoir pour être vertueux ? \\[0pt]
Suffit-il de n'avoir rien fait pour être innocent ? \\[0pt]
Suffit-il de se parler pour éviter la discorde ? \\[0pt]
Suffit-il de suivre ses convictions pour bien agir ? \\[0pt]
Suffit-il de tenir ses promesses pour faire son devoir ? \\[0pt]
Suffit-il d'être conscient de ses actes pour en être responsable ? \\[0pt]
Suffit-il d'être dans le vrai pour savoir ? \\[0pt]
Suffit-il d'être informé pour comprendre ? \\[0pt]
Suffit-il d'être juste ? \\[0pt]
Suffit-il d'être raisonnable pour être moral ? \\[0pt]
Suffit-il d'être soi-même pour être libre ? \\[0pt]
Suffit-il d'être vertueux pour être heureux ? \\[0pt]
Suffit-il de vivre pour exister ? \\[0pt]
Suffit-il de voir le meilleur pour le suivre ? \\[0pt]
Suffit-il de voir pour savoir ? \\[0pt]
Suffit-il de vouloir pour bien faire ? \\[0pt]
Suffit-il, pour croire, de le vouloir ? \\[0pt]
Suffit-il, pour être juste, d'obéir aux lois et aux coutumes de son pays ? \\[0pt]
Suffit-il que nos intentions soient bonnes pour que nos actions le soient aussi ? \\[0pt]
Suis-ce que j'ai conscience d'être ? \\[0pt]
Suis-je au centre de l'espace ? \\[0pt]
Suis-je aussi ce que j'aurais pu être ? \\[0pt]
Suis-je ce que j'ai conscience d'être ? \\[0pt]
Suis-je ce que je fais ? \\[0pt]
Suis-je ce que les autres font de moi ? \\[0pt]
Suis-je ce que mon passé a fait de moi ? \\[0pt]
Suis-je dans le temps comme je suis dans l'espace ? \\[0pt]
Suis-je étranger à moi-même ? \\[0pt]
Suis-je l'auteur de ce que je dis ? \\[0pt]
Suis-je l'auteur de mes actions ? \\[0pt]
Suis-je le même en des temps différents ? \\[0pt]
Suis-je le mieux placé pour me connaître ? \\[0pt]
Suis-je le mieux placé pour savoir qui je suis ? \\[0pt]
Suis-je le sujet de mes pensées ? \\[0pt]
Suis-je libre ? \\[0pt]
Suis-je maître de ma conscience ? \\[0pt]
Suis-je maître de mes pensées ? \\[0pt]
Suis-je ma mémoire ? \\[0pt]
Suis-je mon cerveau ? \\[0pt]
Suis-je mon corps ? \\[0pt]
Suis-je mon passé ? \\[0pt]
Suis-je propriétaire de mon corps ? \\[0pt]
Suis-je responsable de ce dont je n'ai pas conscience ? \\[0pt]
Suis-je responsable de ce que je suis ? \\[0pt]
Suis-je seul au monde ? \\[0pt]
Suis-je toujours autre que moi-même ? \\[0pt]
Suis-je toujours le même ? \\[0pt]
Superstition et fanatisme sont-ils inhérents à la religion ? \\[0pt]
Sur quels fondements distinguer ce qui est public et ce qui est privé ? \\[0pt]
Sur quoi fonder la justice ? \\[0pt]
Sur quoi fonder la légitimité de la loi ? \\[0pt]
Sur quoi fonder la propriété ? \\[0pt]
Sur quoi fonder la société ? \\[0pt]
Sur quoi fonder l'autorité ? \\[0pt]
Sur quoi fonder l'autorité politique ? \\[0pt]
Sur quoi fonder le devoir ? \\[0pt]
Sur quoi fonder le droit de punir ? \\[0pt]
Sur quoi le langage doit-il se régler ? \\[0pt]
Sur quoi l'historien travaille-t-il ? \\[0pt]
Sur quoi peut-on fonder la certitude d'avoir raison ? \\[0pt]
Sur quoi repose l'accord des esprits ? \\[0pt]
Sur quoi repose la croyance au progrès ? \\[0pt]
Sur quoi reposent nos certitudes ? \\[0pt]
Sur quoi se fonde la connaissance scientifique ? \\[0pt]
Sur quoi se fonde la vérité des énoncés ? \\[0pt]
Sur quoi sont fondées les mathématiques ? \\[0pt]
Tester une hypothèse permet-il de la prouver ? \\[0pt]
Toujours plus vite ? \\[0pt]
Tous les arts se valent-ils ? \\[0pt]
Tous les conflits peuvent-ils être résolus par le dialogue ? \\[0pt]
Tous les désirs sont-ils naturels ? \\[0pt]
Tous les droits sont-ils formels ? \\[0pt]
Tous les hommes désirent-ils connaître ? \\[0pt]
Tous les hommes désirent-ils être heureux ? \\[0pt]
Tous les hommes désirent-ils naturellement être heureux ? \\[0pt]
Tous les hommes désirent-ils naturellement savoir ? \\[0pt]
Tous les hommes sont-ils égaux ? \\[0pt]
Tous les paradis sont-ils perdus ? \\[0pt]
Tous les plaisirs se valent-ils ? \\[0pt]
Tous les problèmes peuvent-ils recevoir une solution technique ? \\[0pt]
Tous les rapports humains sont-ils des échanges ? \\[0pt]
Tout art est-il abstrait ? \\[0pt]
Tout art est-il décoratif ? \\[0pt]
Tout art est-il poésie ? \\[0pt]
Tout art est-il symbolique ? \\[0pt]
Tout a-t-il une cause ? \\[0pt]
Tout a-t-il une fin ? \\[0pt]
Tout a-t-il une raison d'être ? \\[0pt]
Tout a-t-il un prix ? \\[0pt]
Tout a-t-il un sens ? \\[0pt]
Tout bien n'est-il qu'un moindre mal ? \\[0pt]
Tout ce qui est excessif est-il insignifiant ? \\[0pt]
Tout ce qui est humain est-il signifiant ? \\[0pt]
Tout ce qui est humain nous est-il familier ? \\[0pt]
Tout ce qui est naturel est-il normal ? \\[0pt]
Tout ce qui est rationnel est-il raisonnable ? \\[0pt]
Tout ce qui est vrai doit-il être prouvé ? \\[0pt]
Tout ce qui existe a-t-il un prix ? \\[0pt]
Tout ce qui existe est-il matériel ? \\[0pt]
Tout change-t-il avec le temps ? \\[0pt]
Tout comprendre, est-ce tout pardonner ? \\[0pt]
Tout désir est-il désir de posséder ? \\[0pt]
Tout désir est-il désir de quelque chose ? \\[0pt]
Tout désir est-il égoïste ? \\[0pt]
Tout désir est-il manque ? \\[0pt]
Tout désir est-il nostalgique ? \\[0pt]
Tout désir est-il une souffrance ? \\[0pt]
Tout devoir est-il l'envers d'un droit ? \\[0pt]
Tout droit est-il un pouvoir ? \\[0pt]
Toute action politique est-elle collective ? \\[0pt]
Toute chose a-t-elle une essence ? \\[0pt]
Toute chose a-t-elle une fin ? \\[0pt]
Toute communauté est-elle politique ? \\[0pt]
Toute compréhension implique-t-elle une interprétation ? \\[0pt]
Toute conception de l'humain est-elle particulière ? \\[0pt]
Toute connaissance autre que scientifique doit-elle être considérée comme une illusion ? \\[0pt]
Toute connaissance commence-t-elle avec l'expérience ? \\[0pt]
Toute connaissance consiste-t-elle en un savoir-faire ? \\[0pt]
Toute connaissance est-elle historique ? \\[0pt]
Toute connaissance est-elle hypothétique ? \\[0pt]
Toute connaissance est-elle relative ? \\[0pt]
Toute connaissance s'enracine-t-elle dans la perception ? \\[0pt]
Toute conscience est-elle conscience de quelque chose ? \\[0pt]
Toute conscience est-elle conscience de soi ? \\[0pt]
Toute conscience est-elle subjective ? \\[0pt]
Toute conscience n'est-elle pas implicitement morale ? \\[0pt]
Toute conversation est-elle futile ? \\[0pt]
Toute définition est-elle une convention ? \\[0pt]
Toute description est-elle une interprétation ? \\[0pt]
Toute existence est-elle indémontrable ? \\[0pt]
Toute expérience appelle-t-elle une interprétation ? \\[0pt]
Toute expression est-elle métaphorique ? \\[0pt]
Toute faute est-elle une erreur ? \\[0pt]
Toute hiérarchie est-elle inégalitaire ? \\[0pt]
Toute inégalité est-elle injuste ? \\[0pt]
Toute interprétation est-elle contestable ? \\[0pt]
Toute interprétation est-elle subjective ? \\[0pt]
Toute loi est-elle arbitraire ? \\[0pt]
Toute métaphysique implique-t-elle une transcendance ? \\[0pt]
Toute morale est-elle rationnelle ? \\[0pt]
Toute morale implique-t-elle l'effort ? \\[0pt]
Toute morale s'oppose-t-elle aux désirs ? \\[0pt]
Tout énoncé admet-il une interprétation ? \\[0pt]
Tout énoncé est-il nécessairement vrai ou faux ? \\[0pt]
Toute notre connaissance dérive-t-elle de l'expérience ? \\[0pt]
Toute œuvre d'art exige-t-elle une interprétation ? \\[0pt]
Toute origine est-elle mythique ? \\[0pt]
Toute passion fait-elle souffrir ? \\[0pt]
Toute pensée revêt-elle nécessairement une forme linguistique ? \\[0pt]
Toute pensée revêt-elle une forme linguistique ? \\[0pt]
Toute peur est-elle irrationnelle ? \\[0pt]
Toute philosophie constitue-t-elle une doctrine ? \\[0pt]
Toute philosophie est-elle systématique ? \\[0pt]
Toute philosophie implique-t-elle une politique ? \\[0pt]
Toute physique exige-t-elle une métaphysique ? \\[0pt]
Toute polémique est-elle stérile ? \\[0pt]
Toute psychologie est-elle normalisatrice ? \\[0pt]
Toute relation humaine est-elle un échange ? \\[0pt]
Toute religion a-t-elle sa vérité ? \\[0pt]
Toute science est-elle naturelle ? \\[0pt]
Toutes les choses sont-elles singulières ? \\[0pt]
Toutes les convictions sont-elles respectables ? \\[0pt]
Toutes les croyances se valent-elles ? \\[0pt]
Toutes les fautes se valent-elles ? \\[0pt]
Toutes les inégalités ont-elles une importance politique ? \\[0pt]
Toutes les inégalités sont-elles des injustices ? \\[0pt]
Toutes les interprétations se valent-elles ? \\[0pt]
Toutes les opinions se valent-elles ? \\[0pt]
Toutes les opinions sont-elles bonnes à dire ? \\[0pt]
Toutes les vérités scientifiques sont-elles révisables ? \\[0pt]
Toute société a-t-elle besoin d'une religion ? \\[0pt]
Tout est-il affaire de point de vue ? \\[0pt]
Tout est-il à vendre ? \\[0pt]
Tout est-il connaissable ? \\[0pt]
Tout est-il contingent ? \\[0pt]
Tout est-il démontrable ? \\[0pt]
Tout est-il digne de mémoire ? \\[0pt]
Tout est-il faux dans la fiction ? \\[0pt]
Tout est-il historique ? \\[0pt]
Tout est-il matière ? \\[0pt]
Tout est-il mesurable ? \\[0pt]
Tout est-il nécessaire ? \\[0pt]
Tout est-il politique ? \\[0pt]
Tout est-il quantifiable ? \\[0pt]
Tout est-il relatif ? \\[0pt]
Tout est-il vraiment permis, si Dieu n'existe pas ? \\[0pt]
Tout être est-il dans l'espace ? \\[0pt]
Tout événement a-t-il une cause ? \\[0pt]
Toute vérité doit-elle être dite ? \\[0pt]
Toute vérité est-elle bonne à dire ? \\[0pt]
Toute vérité est-elle démontrable ? \\[0pt]
Toute vérité est-elle nécessaire ? \\[0pt]
Toute vérité est-elle une erreur rectifiée ? \\[0pt]
Toute vérité est-elle un rapport ? \\[0pt]
Toute vérité est-elle vérifiable ? \\[0pt]
Toute vérité se laisse-t-elle démontrer ? \\[0pt]
Toute vie est-elle intrinsèquement respectable ? \\[0pt]
Toute violence est-elle contre nature ? \\[0pt]
Tout exercice du pouvoir politique implique-t-il l'usage de la violence ? \\[0pt]
Tout fondement de la connaissance est-il métaphysique ? \\[0pt]
Tout futur est-il contingent ? \\[0pt]
Tout malheur est-il une injustice ? \\[0pt]
Tout ordre est-il rationnel ? \\[0pt]
Tout ordre est-il une violence déguisée ? \\[0pt]
Tout passe-t-il avec le temps ? \\[0pt]
Tout peut-il avoir une valeur marchande ? \\[0pt]
Tout peut-il avoir valeur marchande ? \\[0pt]
Tout peut-il être dit ? \\[0pt]
Tout peut-il être expliqué historiquement ? \\[0pt]
Tout peut-il être objet d'échange ? \\[0pt]
Tout peut-il être objet de jugement esthétique ? \\[0pt]
Tout peut-il être objet de science ? \\[0pt]
Tout peut-il n'être qu'apparence ? \\[0pt]
Tout peut-il s'acheter ? \\[0pt]
Tout peut-il se démontrer ? \\[0pt]
Tout peut-il se quantifier ? \\[0pt]
Tout peut-il se vendre ? \\[0pt]
Tout peut-il s'expliquer ? \\[0pt]
Tout pouvoir corrompt-il ? \\[0pt]
Tout pouvoir est-il oppresseur ? \\[0pt]
Tout pouvoir est-il politique ? \\[0pt]
Tout pouvoir est-il usurpé ? \\[0pt]
Tout pouvoir n'est-il pas abusif ? \\[0pt]
Tout pouvoir relève-t-il de la domination ? \\[0pt]
Tout principe est-il un fondement ? \\[0pt]
Tout savoir a-t-il une justification ? \\[0pt]
Tout savoir est-il fondé sur un savoir premier ? \\[0pt]
Tout savoir est-il pouvoir ? \\[0pt]
Tout savoir est-il transmissible ? \\[0pt]
Tout savoir est-il un pouvoir ? \\[0pt]
Tout savoir peut-il se transmettre ? \\[0pt]
Tout s'en va-t-il avec le temps ? \\[0pt]
Tout se prête-il à la mesure ? \\[0pt]
Tout texte est-il à interpréter ? \\[0pt]
Tout travail a-t-il un sens ? \\[0pt]
Tout travail est-il forcé ? \\[0pt]
Tout travail est-il social ? \\[0pt]
Traduire, est-ce trahir ? \\[0pt]
Traiter des faits humains comme des choses, est-ce considérer l'homme comme une chose ? \\[0pt]
Traiter les faits humains comme des choses, est-ce réduire les hommes à des choses ? \\[0pt]
Travailler, est-ce faire œuvre ? \\[0pt]
Travailler, est-ce lutter contre soi-même ? \\[0pt]
Travailler, est-ce seulement produire ? \\[0pt]
Travailler par plaisir, est-ce encore travailler ? \\[0pt]
Travailler sans souffrir peut-il être un idéal ? \\[0pt]
Travaille-t-on pour soi-même ? \\[0pt]
Un acte désintéressé est-il possible ? \\[0pt]
Un acte gratuit est-il possible ? \\[0pt]
Un acte inconscient est-il nécessairement un acte involontaire ? \\[0pt]
Un acte libre est-il un acte imprévisible ? \\[0pt]
Un acte peut-il être inhumain ? \\[0pt]
Un artiste doit-il être génial ? \\[0pt]
Un artiste doit-il être original ? \\[0pt]
Un art peut-il être populaire ? \\[0pt]
Un art peut-il se passer de règles ? \\[0pt]
Un art sans sublimation est-il possible ? \\[0pt]
Un autre monde est-il possible ? \\[0pt]
Un bien peut-il être commun ? \\[0pt]
Un bien peut-il sortir d'un mal ? \\[0pt]
Un chef-d'œuvre est-il immortel ? \\[0pt]
Un choix peut-il être rationnel ? \\[0pt]
Un choix rationnel est-il un choix libre ? \\[0pt]
Un contrat peut-il être injuste ? \\[0pt]
Un contrat peut-il être social ? \\[0pt]
Un corps n'est-il qu'un objet ? \\[0pt]
Un désir peut-il être coupable ? \\[0pt]
Un désir peut-il être inconscient ? \\[0pt]
Un devoir admet-il des exceptions ? \\[0pt]
Un devoir peut-il être absolu ? \\[0pt]
Un Dieu unique ? \\[0pt]
Une action peut-elle être désintéressée ? \\[0pt]
Une action peut-elle être machinale ? \\[0pt]
Une action se juge-t-elle à ses conséquences ? \\[0pt]
Une action vertueuse se reconnaît-elle à sa difficulté ? \\[0pt]
Une activité inutile est-elle sans valeur ? \\[0pt]
Une bonne cité peut-elle se passer du beau ? \\[0pt]
Une cause peut-elle être libre ? \\[0pt]
Une communauté politique n'est-elle qu'une communauté d'intérêt ? \\[0pt]
Une communauté politique n'est-elle qu'une communauté d'intérêts ? \\[0pt]
Une connaissance peut-elle ne pas être relative ? \\[0pt]
Une connaissance scientifique du vivant est-elle possible ? \\[0pt]
Une croyance infondée est-elle illégitime ? \\[0pt]
Une croyance peut-elle être libre ? \\[0pt]
Une croyance peut-elle être rationnelle ? \\[0pt]
Une culture constitue-t-elle un paradigme ? \\[0pt]
Une culture de masse est-elle une culture ? \\[0pt]
Une culture peut-elle être porteuse de valeurs universelles ? \\[0pt]
Une décision politique peut-elle être juste ? \\[0pt]
Une destruction peut-elle être créatrice ? \\[0pt]
Une d'œuvre peut-elle être achevée ? \\[0pt]
Une durée peut-elle être éternelle ? \\[0pt]
Une éducation des sentiments est-elle possible ? \\[0pt]
Une éducation esthétique est-elle possible ? \\[0pt]
Une éducation morale est-elle possible ? \\[0pt]
Une éthique sceptique est-elle possible ? \\[0pt]
Une existence se démontre-t-elle ? \\[0pt]
Une expérience peut-elle être fictive ? \\[0pt]
Une expérience se partage-t-elle ? \\[0pt]
Une explication peut-elle être réductrice ? \\[0pt]
Une fausse science est-elle une science qui commet des erreurs ? \\[0pt]
Une femme politique est-elle un homme politique comme les autres ? \\[0pt]
Une fiction peut-elle être vraie ? \\[0pt]
Une guerre peut-elle être juste ? \\[0pt]
Une idée peut-elle être fausse ? \\[0pt]
Une idée peut-elle être générale ? \\[0pt]
Une imitation peut-elle être parfaite ? \\[0pt]
Une inégalité peut-elle être juste ? \\[0pt]
Une injustice vaut elle mieux qu'un désordre ? \\[0pt]
Une intention peut-elle être coupable ? \\[0pt]
Une interprétation est-elle nécessairement subjective ? \\[0pt]
Une interprétation peut-elle échapper à l'arbitraire ? \\[0pt]
Une interprétation peut-elle être définitive ? \\[0pt]
Une interprétation peut-elle être fausse ? \\[0pt]
Une interprétation peut-elle être objective ? \\[0pt]
Une interprétation peut-elle être vraie ? \\[0pt]
Une interprétation peut-elle prétendre à la vérité ? \\[0pt]
Une justice sans égalité est-elle possible ? \\[0pt]
Une langue n'est-elle faite que de mots ? \\[0pt]
Une ligne de conduite peut-elle tenir lieu de morale ? \\[0pt]
Une logique non-formelle est-elle possible ? \\[0pt]
Une loi n'est-elle qu'une règle ? \\[0pt]
Une loi peut-elle être injuste ? \\[0pt]
Une machine n'est-elle qu'un outil perfectionné ? \\[0pt]
Une machine peut-elle avoir une mémoire ? \\[0pt]
Une machine peut-elle être naturelle ? \\[0pt]
Une machine peut-elle penser ? \\[0pt]
Une machine pourrait-elle penser ? \\[0pt]
Une machine se réduit-elle à un mécanisme ? \\[0pt]
Une matière sans forme est-elle concevable ? \\[0pt]
Une métaphysique athée est-elle possible ? \\[0pt]
Une métaphysique n'est-elle qu'une ontologie ? \\[0pt]
Une métaphysique peut-elle être sceptique ? \\[0pt]
Une morale du plaisir est-elle concevable ? \\[0pt]
Une morale personnelle est-elle une contradiction dans les termes ? \\[0pt]
Une morale peut-elle être dépassée ? \\[0pt]
Une morale peut-elle être permissive ? \\[0pt]
Une morale peut-elle être provisoire ? \\[0pt]
Une morale peut-elle prétendre à l'universalité ? \\[0pt]
Une morale peut-elle se passer de la notion de vertu ? \\[0pt]
Une morale qui ne vise pas l'utilité est-elle rationnelle ? \\[0pt]
Une morale sans devoirs est-elle possible ? \\[0pt]
Une morale sans obligation est-elle possible ? \\[0pt]
Une morale sceptique est-elle possible ? \\[0pt]
Un enfant est-il un citoyen ? \\[0pt]
Une œuvre d'art a-t-elle toujours un sens ? \\[0pt]
Une œuvre d'art a-t-elle un prix ? \\[0pt]
Une œuvre d'art doit-elle avoir un sens ? \\[0pt]
Une œuvre d'art doit-elle nécessairement être belle ? \\[0pt]
Une œuvre d'art doit-elle plaire ? \\[0pt]
Une œuvre d'art est-elle immortelle ? \\[0pt]
Une œuvre d'art est-elle toujours originale ? \\[0pt]
Une œuvre d'art est-elle une marchandise ? \\[0pt]
Une œuvre d'art peut-elle être exemplaire ? \\[0pt]
Une œuvre d'art peut-elle être immorale ? \\[0pt]
Une œuvre d'art peut-elle être laide ? \\[0pt]
Une œuvre d'art produit-elle du beau ? \\[0pt]
Une œuvre d'art s'explique-t-elle à partir de ses influences ? \\[0pt]
Une œuvre doit-elle nécessairement être belle ? \\[0pt]
Une œuvre est-elle nécessairement singulière ? \\[0pt]
Une œuvre est-elle toujours de son temps ? \\[0pt]
Une passion peut-elle résister au temps ? \\[0pt]
Une pensée contradictoire est-elle dénuée de valeur ? \\[0pt]
Une perception peut-elle être illusoire ? \\[0pt]
Une philosophie de l'amour est-elle possible ? \\[0pt]
Une philosophie peut-elle être réactionnaire ? \\[0pt]
Une politique de la nature est-elle concevable ? \\[0pt]
Une politique morale est-elle possible ? \\[0pt]
Une politique peut-elle se réclamer de la vie ? \\[0pt]
Une psychologie peut-elle être matérialiste ? \\[0pt]
Une religion civile est-elle possible ? \\[0pt]
Une religion peut-elle être fausse ? \\[0pt]
Une religion peut-elle être rationnelle ? \\[0pt]
Une religion peut-elle être universelle ? \\[0pt]
Une religion peut-elle prétendre à la vérité ? \\[0pt]
Une religion peut-elle se passer de pratiques ? \\[0pt]
Une religion rationnelle est-elle possible ? \\[0pt]
Une science bien faite est-elle une langue bien faite ? \\[0pt]
Une science de la conscience est-elle possible ? \\[0pt]
Une science de la culture est-elle possible ? \\[0pt]
Une science de la morale est-elle possible ? \\[0pt]
Une science de l'éducation est-elle possible ? \\[0pt]
Une science de l'esprit est-elle possible ? \\[0pt]
Une science de l'homme est-elle possible ? \\[0pt]
Une science de l'imaginaire est-elle possible ? \\[0pt]
Une science des symboles est-elle possible ? \\[0pt]
Une science parfaite est-elle possible ? \\[0pt]
Une science peut-elle se passer d'expérimentation ? \\[0pt]
Une sensation peut-elle être fausse ? \\[0pt]
Une société basée sur le don pourrait-elle être prospère ? \\[0pt]
Une société d'athées est-elle possible ? \\[0pt]
Une société juste, est-ce une société sans conflit ? \\[0pt]
Une société juste est-elle une société sans conflits ? \\[0pt]
Une société n'est-elle qu'un ensemble d'individus ? \\[0pt]
Une société peut-elle être juste ? \\[0pt]
Une société peut-elle se passer de religion ? \\[0pt]
Une société sans conflit est-elle pensable ? \\[0pt]
Une société sans conflit est-elle possible ? \\[0pt]
Une société sans conflits est-elle souhaitable ? \\[0pt]
Une société sans État est-elle possible ? \\[0pt]
Une société sans État est-elle une société sans politique ? \\[0pt]
Une société sans religion est-elle possible ? \\[0pt]
Une société sans travail est-elle souhaitable ? \\[0pt]
Un État mondial ? \\[0pt]
Un État peut-il être trop étendu ? \\[0pt]
Un État peut-il être « voyou » ? \\[0pt]
Une technique ne se réduit-elle pas toujours à une forme de bricolage ? \\[0pt]
Une théorie de la culture peut-elle être scientifique ? \\[0pt]
Une théorie peut-elle être vérifiée ? \\[0pt]
Une théorie sans expérience nous apprend-elle quelque chose ? \\[0pt]
Une théorie scientifique peut-elle devenir fausse ? \\[0pt]
Une théorie scientifique peut-elle être ramenée à des propositions empiriques élémentaires ? \\[0pt]
Une théorie scientifique peut-elle être vraie ? \\[0pt]
Un être vivant peut-il être comparé à une œuvre d'art ? \\[0pt]
Un événement historique est-il toujours imprévisible ? \\[0pt]
Une vérité approchée est-elle pour autant approximative ? \\[0pt]
Une vérité est-elle discutable ? \\[0pt]
Une vérité peut-elle être approximative ? \\[0pt]
Une vérité peut-elle être indicible ? \\[0pt]
Une vérité peut-elle être provisoire ? \\[0pt]
Une vérité sans preuve est-elle une vérité ? \\[0pt]
Une vie heureuse est-elle une vie de plaisirs ? \\[0pt]
Une vie juste est-elle une bonne vie ? \\[0pt]
Une vie libre exclut-elle le travail ? \\[0pt]
Une volonté peut-elle être générale ? \\[0pt]
Un fait existe-t-il sans interprétation ? \\[0pt]
Un fait scientifique doit-il être nécessairement démontré ? \\[0pt]
Un gouvernement de savants est-il souhaitable ? \\[0pt]
Un homme n'est-il que la somme de ses actes ? \\[0pt]
Un jeu peut-il être sérieux ? \\[0pt]
Un jugement de goût est-il culturel ? \\[0pt]
Un jugement moral peut-il relever de la vérité ? \\[0pt]
Un langage peut-il être privé ? \\[0pt]
Un langage universel est-il concevable ? \\[0pt]
Un mensonge peut-il avoir une valeur morale ? \\[0pt]
Un mensonge peut-il être légitime ? \\[0pt]
Un monde sans guerre serait-il un monde sans histoire ? \\[0pt]
Un monde sans nature est-il pensable ? \\[0pt]
Un monde sans travail est-il souhaitable ? \\[0pt]
Un objet technique peut-il être beau ? \\[0pt]
Un organisme n'est-il qu'un objet ? \\[0pt]
Un peuple en paix n'a-t-il plus d'histoire ? \\[0pt]
Un peuple est-il responsable de son histoire ? \\[0pt]
Un peuple est-il un rassemblement d'individus ? \\[0pt]
Un peuple peut-il être sans histoire ? \\[0pt]
Un peuple peut-il être souverain ? \\[0pt]
Un peuple se définit-il par son histoire ? \\[0pt]
Un philosophe a-t-il des devoirs envers la société ? \\[0pt]
Un plaisir peut-il être désintéressé ? \\[0pt]
Un pouvoir a-t-il besoin d'être légitime ? \\[0pt]
Un préjugé n'a-t-il aucun fondement ? \\[0pt]
Un problème moral peut-il recevoir une solution certaine ? \\[0pt]
Un problème philosophique peut-il être périmé ? \\[0pt]
Un problème scientifique peut-il être insoluble ? \\[0pt]
Un savoir peut-il être inconscient ? \\[0pt]
Un savoir scientifique sur l'homme est-il compatible avec l'idée de liberté ? \\[0pt]
Un savoir sur l'homme peut-il être séparé d'un pouvoir sur les hommes ? \\[0pt]
Un sceptique peut-il être logicien ? \\[0pt]
Un seul peut-il avoir raison contre tous ? \\[0pt]
Un souverain élu peut-il être illégitime ? \\[0pt]
Un tableau peut-il être une dénonciation ? \\[0pt]
Un témoin de moralité est-il possible ? \\[0pt]
Un travail bien fait est-il nécessairement un bon travail ? \\[0pt]
Un vice, est-ce un manque ? \\[0pt]
User de violence peut-il être moral ? \\[0pt]
Utiliser les embryons humains ? \\[0pt]
Vaut-il mieux oublier ou pardonner ? \\[0pt]
Vaut-il mieux subir l'injustice que la commettre ? \\[0pt]
Vaut-il mieux subir ou commettre l'injustice ? \\[0pt]
Vérité et cohérence ? \\[0pt]
Veut-on toujours savoir ? \\[0pt]
Vie et matière : une distinction périmée ? \\[0pt]
Vit-on au présent ? \\[0pt]
Vivons-nous au présent ? \\[0pt]
Vivons-nous tous dans le même monde ? \\[0pt]
Vivrait-on mieux sans désirs ? \\[0pt]
Vivrait-on mieux sans morale ? \\[0pt]
Vivre en société, est-ce seulement vivre ensemble ? \\[0pt]
Vivre, est-ce interpréter ? \\[0pt]
Vivre, est-ce lutter contre la mort ? \\[0pt]
Vivre, est-ce lutter pour survivre ? \\[0pt]
Vivre, est-ce résister à la mort ? \\[0pt]
Vivre, est-ce un droit ? \\[0pt]
Vivre et exister, est-ce la même chose ? \\[0pt]
Vivre sans mémoire, est-ce être libre ? \\[0pt]
Vivre sans religion, est-ce vivre sans espoir ? \\[0pt]
Voit-on ce qu'on croit ? \\[0pt]
Vouloir croire, est-ce possible ? \\[0pt]
Vouloir, est-ce encore désirer ? \\[0pt]
Vouloir la paix sociale peut-il aller jusqu'à accepter l'injustice ? \\[0pt]
Voulons-nous la vérité ? \\[0pt]
Vulgariser la science ? \\[0pt]
Y a-t-il à la question « qu'est-ce que l'homme ? » une réponse scientifique ? \\[0pt]
Y a-t-il autant de mondes que de cultures ? \\[0pt]
Y a-t-il continuité entre l'expérience et la science ? \\[0pt]
Y a-t-il continuité ou discontinuité entre la pensée mythique et la science ? \\[0pt]
Y a-t-il d'autres moyens que la démonstration pour établir la vérité ? \\[0pt]
Y a-t-il de bons et de mauvais désirs ? \\[0pt]
Y a-t-il de bons préjugés ? \\[0pt]
Y a-t-il de fausses religions ? \\[0pt]
Y a-t-il de fausses sciences ? \\[0pt]
Y a-t-il de faux besoins ? \\[0pt]
Y a-t-il de faux problèmes ? \\[0pt]
Y a-t-il de justes inégalités ? \\[0pt]
Y a-t-il de la fatalité dans la vie de l'homme ? \\[0pt]
Y a-t-il de la grandeur à être libre ? \\[0pt]
Y a-t-il de la raison dans la perception ? \\[0pt]
Y a-t-il de l'impensable ? \\[0pt]
Y a-t-il de l'incommunicable ? \\[0pt]
Y a-t-il de l'incompréhensible ? \\[0pt]
Y a-t-il de l'inconcevable ? \\[0pt]
Y a-t-il de l'inconnaissable ? \\[0pt]
Y a-t-il de l'indémontrable ? \\[0pt]
Y a-t-il de l'indésirable ? \\[0pt]
Y a-t-il de l'indicible ? \\[0pt]
Y a-t-il de l'inexprimable ? \\[0pt]
Y a-t-il de l'intelligible dans l'art ? \\[0pt]
Y a-t-il de l'irréductible ? \\[0pt]
Y a-t-il de l'irréfutable ? \\[0pt]
Y a-t-il de l'irréparable ? \\[0pt]
Y a-t-il de l'universel ? \\[0pt]
Y a-t-il de mauvais désirs ? \\[0pt]
Y a-t-il de mauvaises raisons ? \\[0pt]
Y a-t-il de mauvais spectateurs ? \\[0pt]
Y a-t-il des acquis définitifs en science ? \\[0pt]
Y a-t-il des actes de pensée ? \\[0pt]
Y a-t-il des actes désintéressés ? \\[0pt]
Y a-t-il des actes gratuits ? \\[0pt]
Y a-t-il des actes intrinsèquement mauvais ? \\[0pt]
Y a-t-il des actes moralement indifférents ? \\[0pt]
Y a-t-il des actions collectives ? \\[0pt]
Y a-t-il des actions désintéressées ? \\[0pt]
Y a-t-il des actions libres ? \\[0pt]
Y a-t-il des aliénations heureuses ? \\[0pt]
Y a-t-il des arts du corps ? \\[0pt]
Y a-t-il des arts majeurs ? \\[0pt]
Y a-t-il des arts mineurs ? \\[0pt]
Y a-t-il des barbares ? \\[0pt]
Y a-t-il des biens communs ? \\[0pt]
Y a-t-il des biens inestimables ? \\[0pt]
Y a-t-il des canons de la beauté ? \\[0pt]
Y a-t-il des certitudes historiques ? \\[0pt]
Y a-t-il des certitudes morales ? \\[0pt]
Y a-t-il des choses à savoir ? \\[0pt]
Y a-t-il des choses dont on ne peut parler ? \\[0pt]
Y a-t-il des choses qui échappent au droit ? \\[0pt]
Y a-t-il des choses qu'on n'échange pas ? \\[0pt]
Y a-t-il des colères justes ? \\[0pt]
Y a-t-il des compétences politiques ? \\[0pt]
Y a-t-il des conflits de devoirs ? \\[0pt]
Y a-t-il des connaissances dangereuses ? \\[0pt]
Y a-t-il des connaissances désintéressées ? \\[0pt]
Y a-t-il des contraintes légitimes ? \\[0pt]
Y a-t-il des convictions philosophiques ? \\[0pt]
Y a-t-il des correspondances entre les arts ? \\[0pt]
Y a-t-il des critères de la scientificité ? \\[0pt]
Y a-t-il des critères de l'humanité ? \\[0pt]
Y a-t-il des critères du beau ? \\[0pt]
Y a-t-il des critères du goût ? \\[0pt]
Y a-t-il des croyances démocratiques ? \\[0pt]
Y a-t-il des croyances nécessaires ? \\[0pt]
Y a-t-il des croyances raisonnables ? \\[0pt]
Y a-t-il des croyances rationnelles ? \\[0pt]
Y a-t-il des degrés dans la certitude ? \\[0pt]
Y a-t-il des degrés de conscience ? \\[0pt]
Y a-t-il des degrés de liberté ? \\[0pt]
Y a-t-il des degrés de réalité ? \\[0pt]
Y a-t-il des degrés de vérité ? \\[0pt]
Y a-t-il des démonstrations en morale ? \\[0pt]
Y a-t-il des démonstrations en philosophie ? \\[0pt]
Y a-t-il des désirs moraux ? \\[0pt]
Y a-t-il des désordres naturels ? \\[0pt]
Y a-t-il des despotes éclairés ? \\[0pt]
Y a-t-il des déterminismes sociaux ? \\[0pt]
Y a-t-il des devoirs envers soi ? \\[0pt]
Y a-t-il des devoirs envers soi-même ? \\[0pt]
Y a-t-il des dilemmes moraux ? \\[0pt]
Y a-t-il des droits naturels ? \\[0pt]
Y a-t-il des droits sans devoirs ? \\[0pt]
Y a-t-il des erreurs de la nature ? \\[0pt]
Y a-t-il des erreurs en politique ? \\[0pt]
Y a-t-il des êtres de raison ? \\[0pt]
Y a-t-il des êtres mathématiques ? \\[0pt]
Y a-t-il des évidences morales ? \\[0pt]
Y a-t-il des excès en art ? \\[0pt]
Y a-t-il des expériences absolument certaines ? \\[0pt]
Y a-t-il des expériences cruciales ? \\[0pt]
Y a-t-il des expériences de la liberté ? \\[0pt]
Y a-t-il des expériences métaphysiques ? \\[0pt]
Y a-t-il des expériences sans théorie ? \\[0pt]
Y a-t-il des facultés dans l'esprit ? \\[0pt]
Y a-t-il des faits moraux ? \\[0pt]
Y a-t-il des faits religieux ? \\[0pt]
Y a-t-il des faits sans essence ? \\[0pt]
Y a-t-il des faits scientifiques ? \\[0pt]
Y a-t-il des faux problèmes ? \\[0pt]
Y a-t-il des fins dans la nature ? \\[0pt]
Y a-t-il des fins de la nature ? \\[0pt]
Y a-t-il des fins dernières ? \\[0pt]
Y a-t-il des fondements naturels à l'ordre social ? \\[0pt]
Y a-t-il des formes élémentaires de la société ? \\[0pt]
Y a-t-il des genres de plaisir ? \\[0pt]
Y a-t-il des genres du plaisir ? \\[0pt]
Y a-t-il des guerres injustes ? \\[0pt]
Y a-t-il des guerres justes ? \\[0pt]
Y a-t-il des héritages philosophiques ? \\[0pt]
Y a-t-il des idées générales ? \\[0pt]
Y a-t-il des idées innées ? \\[0pt]
Y a-t-il des illusions de la conscience ? \\[0pt]
Y a-t-il des illusions de la liberté ? \\[0pt]
Y a-t-il des illusions nécessaires ? \\[0pt]
Y a-t-il des inégalités justes ? \\[0pt]
Y a-t-il des injustices naturelles ? \\[0pt]
Y a-t-il des instincts propres à l'Homme ? \\[0pt]
Y a-t-il des interprétations fausses ? \\[0pt]
Y a-t-il des intuitions morales ? \\[0pt]
Y a-t-il des jours où je n'existe pas ? \\[0pt]
Y a-t-il des leçons de l'histoire ? \\[0pt]
Y a-t-il des liens qui libèrent ? \\[0pt]
Y a-t-il des limites à la connaissance ? \\[0pt]
Y a-t-il des limites à la connaissance de l'homme par les sciences ? \\[0pt]
Y a-t-il des limites à la conscience ? \\[0pt]
Y a-t-il des limites à la critique ? \\[0pt]
Y a-t-il des limites à la description ? \\[0pt]
Y a-t-il des limites à la pensée ? \\[0pt]
Y a-t-il des limites à la responsabilité ? \\[0pt]
Y a-t-il des limites à la tolérance ? \\[0pt]
Y a-t-il des limites à l'exprimable ? \\[0pt]
Y a-t-il des limites au droit ? \\[0pt]
Y a-t-il des limites au pouvoir de la technique ? \\[0pt]
Y a-t-il des limites proprement morales à la discussion ? \\[0pt]
Y a-t-il des lois de la pensée ? \\[0pt]
Y a-t-il des lois de l'histoire ? \\[0pt]
Y a-t-il des lois de l'Histoire ? \\[0pt]
Y a-t-il des lois du comportement humain ? \\[0pt]
Y a-t-il des lois du hasard ? \\[0pt]
Y a-t-il des lois du social ? \\[0pt]
Y a-t-il des lois du vivant ? \\[0pt]
Y a-t-il des lois en histoire ? \\[0pt]
Y a-t-il des lois injustes ? \\[0pt]
Y a-t-il des lois morales ? \\[0pt]
Y a-t-il des lois non écrites ? \\[0pt]
Y a-t-il des lois sociales ? \\[0pt]
Y a-t-il des mécanismes humains ? \\[0pt]
Y a-t-il des mentalités collectives ? \\[0pt]
Y a-t-il des méthodes pour être heureux ? \\[0pt]
Y a-t-il des modèles en morale ? \\[0pt]
Y a-t-il des mondes imaginaires ? \\[0pt]
Y a-t-il des mots vides de sens ? \\[0pt]
Y a-t-il des normes de la vie ? \\[0pt]
Y a-t-il des normes naturelles ? \\[0pt]
Y a-t-il des notions communes ? \\[0pt]
Y a-t-il des objets qui n'existent pas ? \\[0pt]
Y a-t-il des obstacles à la connaissance du vivant ? \\[0pt]
Y a-t-il de sots métiers ? \\[0pt]
Y a-t-il des passions collectives ? \\[0pt]
Y a-t-il des passions intraitables ? \\[0pt]
Y a-t-il des passions raisonnables ? \\[0pt]
Y a-t-il des pathologies du social ? \\[0pt]
Y a-t-il des pathologies sociales ? \\[0pt]
Y a-t-il des pensées folles ? \\[0pt]
Y a-t-il des pensées inconscientes ? \\[0pt]
Y a-t-il des perceptions insensibles ? \\[0pt]
Y a-t-il des petites vertus ? \\[0pt]
Y a-t-il des peuples sans histoire ? \\[0pt]
Y a-t-il des phénomènes psychiques ? \\[0pt]
Y a-t-il des plaisirs innocents ? \\[0pt]
Y a-t-il des plaisirs mauvais ? \\[0pt]
Y a-t-il des plaisirs meilleurs que d'autres ? \\[0pt]
Y a-t-il des plaisirs purs ? \\[0pt]
Y a-t-il des plaisirs simples ? \\[0pt]
Y a-t-il des points de vue en morale ? \\[0pt]
Y a-t-il des preuves d'amour ? \\[0pt]
Y a-t-il des preuves de la liberté ? \\[0pt]
Y a-t-il des preuves de la non-existence de Dieu ? \\[0pt]
Y a-t-il des preuves de l'existence de Dieu ? \\[0pt]
Y a-t-il des preuves en métaphysique ? \\[0pt]
Y a-t-il des preuves en philosophie ? \\[0pt]
Y a-t-il des principes de justice universels ? \\[0pt]
Y a-t-il des progrès dans l'art ? \\[0pt]
Y a-t-il des progrès en art ? \\[0pt]
Y a-t-il des progrès en philosophie ? \\[0pt]
Y a-t-il des propriétés singulières ? \\[0pt]
Y a-t-il des questions sans réponse ? \\[0pt]
Y a-t-il des questions sans réponses ? \\[0pt]
Y a-t-il des raisons d'aimer ? \\[0pt]
Y a-t-il des raisons de douter de la raison ? \\[0pt]
Y a-t-il des raisons de vivre ? \\[0pt]
Y a-t-il des réalités mathématiques ? \\[0pt]
Y a-t-il des règles de la guerre ? \\[0pt]
Y a-t-il des règles de l'art ? \\[0pt]
Y a-t-il des régressions historiques ? \\[0pt]
Y a-t-il des révolutions dans l'art ? \\[0pt]
Y a-t-il des révolutions dans les sciences ? \\[0pt]
Y a-t-il des révolutions en art ? \\[0pt]
Y a-t-il des révolutions scientifiques ? \\[0pt]
Y a-t-il des savoirs immédiats ? \\[0pt]
Y a-t-il des sciences de l'éducation ? \\[0pt]
Y a-t-il des sciences de l'homme ? \\[0pt]
Y a-t-il des sciences exactes ? \\[0pt]
Y a-t-il des secrets de la nature ? \\[0pt]
Y a-t-il des sentiments moraux ? \\[0pt]
Y a-t-il des signes naturels ? \\[0pt]
Y a-t-il des sociétés archaïques ? \\[0pt]
Y a-t-il des sociétés sans État ? \\[0pt]
Y a-t-il des sociétés sans histoire ? \\[0pt]
Y a-t-il des solutions en politique ? \\[0pt]
Y a-t-il des sots métiers ? \\[0pt]
Y a-t-il des substances incorporelles ? \\[0pt]
Y a-t-il des techniques de pensée ? \\[0pt]
Y a-t-il des techniques du corps ? \\[0pt]
Y a-t-il des techniques pour être heureux ? \\[0pt]
Y a-t-il des tyrannies douces ? \\[0pt]
Y a-t-il des universaux anthropologiques ? \\[0pt]
Y a-t-il des valeurs absolues ? \\[0pt]
Y a-t-il des valeurs naturelles ? \\[0pt]
Y a-t-il des valeurs objectives ? \\[0pt]
Y a-t-il des valeurs propres à la science ? \\[0pt]
Y a-t-il des valeurs universelles ? \\[0pt]
Y a-t-il des vérités de fait ? \\[0pt]
Y a-t-il des vérités définitives ? \\[0pt]
Y a-t-il des vérités en art ? \\[0pt]
Y a-t-il des vérités en politique ? \\[0pt]
Y a-t-il des vérités éternelles ? \\[0pt]
Y a-t-il des vérités indémontrables ? \\[0pt]
Y a-t-il des vérités indiscutables ? \\[0pt]
Y a-t-il des vérités métaphysiques ? \\[0pt]
Y a-t-il des vérités morales ? \\[0pt]
Y a-t-il des vérités philosophiques ? \\[0pt]
Y a-t-il des vérités plus importantes que d'autres ? \\[0pt]
Y a-t-il des vérités qui échappent à la raison ? \\[0pt]
Y a-t-il des vérités religieuses ? \\[0pt]
Y a-t-il des vérités sans preuve ? \\[0pt]
Y a-t-il des vertus mineures ? \\[0pt]
Y a-t-il des violences justifiées ? \\[0pt]
Y a-t-il des violences légitimes ? \\[0pt]
Y a-t-il différentes façons d'exister ? \\[0pt]
Y a-t-il différentes manières de connaître ? \\[0pt]
Y a-t-il du déterminisme dans les sciences humaines ? \\[0pt]
Y a-t-il du hasard objectif ? \\[0pt]
Y a-t-il du non-être ? \\[0pt]
Y a-t-il du nouveau dans l'histoire ? \\[0pt]
Y a-t-il du progrès en politique ? \\[0pt]
Y a-t-il du sacré dans la nature ? \\[0pt]
Y a-t-il du synthétique \emph{a priori} ? \\[0pt]
Y a-t-il du vrai dans le beau ? \\[0pt]
Y a-t-il encore des mythologies ? \\[0pt]
Y a-t-il encore une sphère privée ? \\[0pt]
Y a-t-il incompatibilité entre science et religion ? \\[0pt]
Y a-t-il lieu de distinguer entre éthique et morale ? \\[0pt]
Y a-t-il lieu de distinguer instinct et pulsion ? \\[0pt]
Y a-t-il lieu de distinguer le don et l'échange ? \\[0pt]
Y a-t-il lieu d'opposer la philosophie et les sciences humaines ? \\[0pt]
Y a-t-il lieu d'opposer matière et esprit ? \\[0pt]
Y a-t-il nécessairement du religieux dans l'art ? \\[0pt]
Y a-t-il place pour l'idée de vérité en morale ? \\[0pt]
Y a-t-il plusieurs libertés ? \\[0pt]
Y a-t-il plusieurs manières de définir ? \\[0pt]
Y a-t-il plusieurs métaphysiques ? \\[0pt]
Y a-t-il plusieurs morales ? \\[0pt]
Y a-t-il plusieurs nécessités ? \\[0pt]
Y a-t-il plusieurs sens du mot vie ? \\[0pt]
Y a-t-il plusieurs sortes de matières ? \\[0pt]
Y a-t-il plusieurs sortes de vérité ? \\[0pt]
Y a-t-il plusieurs temps ? \\[0pt]
Y a-t-il plusieurs vérités ? \\[0pt]
Y a-t-il progrès en art ? \\[0pt]
Y a-t-il quelque chose de commun aux œuvres d'art ? \\[0pt]
Y a-t-il quelque chose de subversif dans les sciences humaines ? \\[0pt]
Y a-t-il quoi que ce soit de nouveau dans l'histoire ? \\[0pt]
Y a-t-il sens à vouloir justifier le mal ? \\[0pt]
Y a-t-il trop d'images ? \\[0pt]
Y a-t-il un art de gouverner ? \\[0pt]
Y a-t-il un art de penser ? \\[0pt]
Y a-t-il un art de persuader ? \\[0pt]
Y a-t-il un art d'être heureux ? \\[0pt]
Y a-t-il un art de vivre ? \\[0pt]
Y a-t-il un art d'interpréter ? \\[0pt]
Y a-t-il un art d'inventer ? \\[0pt]
Y a-t-il un art du bonheur ? \\[0pt]
Y a-t-il un art populaire ? \\[0pt]
Y a-t-il un art total ? \\[0pt]
Y a-t-il un au-delà de la vérité ? \\[0pt]
Y a-t-il un au-delà du langage ? \\[0pt]
Y a-t-il un auteur de l'histoire ? \\[0pt]
Y a-t-il un autre monde ? \\[0pt]
Y a-t-il un beau idéal ? \\[0pt]
Y a-t-il un beau naturel ? \\[0pt]
Y a-t-il un besoin métaphysique ? \\[0pt]
Y a-t-il un bien commun ? \\[0pt]
Y a-t-il un bien plus précieux que la paix ? \\[0pt]
Y a-t-il un Bien suprême ? \\[0pt]
Y a-t-il un bonheur sans illusion ? \\[0pt]
Y a-t-il un bon usage de la rhétorique ? \\[0pt]
Y a-t-il un bon usage des passions ? \\[0pt]
Y a-t-il un bon usage du temps ? \\[0pt]
Y a-t-il un canon de la beauté ? \\[0pt]
Y a-t-il un commencement à tout ? \\[0pt]
Y a-t-il un critère de vérité ? \\[0pt]
Y a-t-il un critère du vrai ? \\[0pt]
Y a-t-il un désir de connaître ? \\[0pt]
Y a-t-il un déterminisme dans les sciences humaines ? \\[0pt]
Y a-t-il un devoir de désobéissance ? \\[0pt]
Y a-t-il un devoir d'émancipation ? \\[0pt]
Y a-t-il un devoir de mémoire ? \\[0pt]
Y a-t-il un devoir de travailler ? \\[0pt]
Y a-t-il un devoir d'être heureux ? \\[0pt]
Y a-t-il un devoir de vérité ? \\[0pt]
Y a-t-il un devoir d'indignation ? \\[0pt]
Y a-t-il un devoir d'oubli ? \\[0pt]
Y a-t-il un différend entre poésie et philosophie ? \\[0pt]
Y a-t-il un droit à la différence ? \\[0pt]
Y a-t-il un droit à la propriété ? \\[0pt]
Y a-t-il un droit au bonheur ? \\[0pt]
Y a-t-il un droit au travail ? \\[0pt]
Y a-t-il un droit de désobéissance ? \\[0pt]
Y a-t-il un droit de la guerre ? \\[0pt]
Y a-t-il un droit de mentir ? \\[0pt]
Y a-t-il un droit de mourir ? \\[0pt]
Y a-t-il un droit de résistance ? \\[0pt]
Y a-t-il un droit de révolte ? \\[0pt]
Y a-t-il un droit des peuples ? \\[0pt]
Y a-t-il un droit d'ingérence ? \\[0pt]
Y a-t-il un droit du plus faible ? \\[0pt]
Y a-t-il un droit du plus fort ? \\[0pt]
Y a-t-il un droit international ? \\[0pt]
Y a-t-il un droit naturel ? \\[0pt]
Y a-t-il un droit universel au mariage ? \\[0pt]
Y a-t-il une argumentation métaphysique ? \\[0pt]
Y a-t-il une beauté morale ? \\[0pt]
Y a-t-il une beauté naturelle ? \\[0pt]
Y a-t-il une beauté propre à la photographie ? \\[0pt]
Y a-t-il une beauté propre à l'objet technique ? \\[0pt]
Y a-t-il une bonne éducation ? \\[0pt]
Y a-t-il une bonne imitation ? \\[0pt]
Y a-t-il une causalité empirique ? \\[0pt]
Y a-t-il une causalité en histoire ? \\[0pt]
Y a-t-il une causalité historique ? \\[0pt]
Y a-t-il une cause première ? \\[0pt]
Y a-t-il une compétence en politique ? \\[0pt]
Y a-t-il une compétence politique ? \\[0pt]
Y a-t-il une condition humaine ? \\[0pt]
Y a-t-il une connaissance du probable ? \\[0pt]
Y a-t-il une connaissance du singulier ? \\[0pt]
Y a-t-il une connaissance historique ? \\[0pt]
Y a-t-il une connaissance logique ? \\[0pt]
Y a-t-il une connaissance métaphysique ? \\[0pt]
Y a-t-il une connaissance sensible ? \\[0pt]
Y a-t-il une conscience collective ? \\[0pt]
Y a-t-il une correspondance des arts ? \\[0pt]
Y a-t-il une définition du bonheur ? \\[0pt]
Y a-t-il une dimension métaphysique du désir ? \\[0pt]
Y a-t-il une éducation du goût ? \\[0pt]
Y a-t-il une enfance de l'humanité ? \\[0pt]
Y a-t-il une esthétique de la laideur ? \\[0pt]
Y a-t-il une esthétique du laid ? \\[0pt]
Y a-t-il une esthétique industrielle ? \\[0pt]
Y a-t-il une éthique de la technique ? \\[0pt]
Y a-t-il une éthique de l'authenticité ? \\[0pt]
Y a-t-il une éthique des moyens ? \\[0pt]
Y a-t-il une expérience de la liberté ? \\[0pt]
Y a-t-il une expérience de l'éternité ? \\[0pt]
Y a-t-il une expérience du néant ? \\[0pt]
Y a-t-il une expérience du temps ? \\[0pt]
Y a-t-il une expérience métaphysique ? \\[0pt]
Y a-t-il une finalité dans la nature ? \\[0pt]
Y a-t-il une fin de l'histoire ? \\[0pt]
Y a-t-il une fin dernière ? \\[0pt]
Y a-t-il une fonction propre à l'œuvre d'art ? \\[0pt]
Y a-t-il une force des faibles ? \\[0pt]
Y a-t-il une force du droit ? \\[0pt]
Y a-t-il une formation de la pensée logique ? \\[0pt]
Y a-t-il une forme morale de fanatisme ? \\[0pt]
Y a-t-il une hiérarchie des devoirs ? \\[0pt]
Y a-t-il une hiérarchie des êtres ? \\[0pt]
Y a-t-il une hiérarchie des sciences ? \\[0pt]
Y a-t-il une hiérarchie des sciences sociales ? \\[0pt]
Y a-t-il une hiérarchie du vivant ? \\[0pt]
Y a-t-il une histoire de la folie ? \\[0pt]
Y a-t-il une histoire de la nature ? \\[0pt]
Y a-t-il une histoire de la raison ? \\[0pt]
Y a-t-il une histoire de l'art ? \\[0pt]
Y a-t-il une histoire de la vérité ? \\[0pt]
Y a-t-il une histoire de la vie quotidienne ? \\[0pt]
Y a-t-il une histoire universelle ? \\[0pt]
Y a-t-il une intelligence du corps ? \\[0pt]
Y a-t-il une intentionnalité collective ? \\[0pt]
Y a-t-il une irréversibilité du temps ? \\[0pt]
Y a-t-il une justice naturelle ? \\[0pt]
Y a-t-il une justice sans morale ? \\[0pt]
Y a-t-il une langue de la philosophie ? \\[0pt]
Y a-t-il une limite à la connaissance du vivant ? \\[0pt]
Y a-t-il une limite au désir ? \\[0pt]
Y a-t-il une limite au développement scientifique ? \\[0pt]
Y a-t-il une logique dans l'histoire ? \\[0pt]
Y a-t-il une logique de la découverte ? \\[0pt]
Y a-t-il une logique de la découverte scientifique ? \\[0pt]
Y a-t-il une logique de la déraison ? \\[0pt]
Y a-t-il une logique de l'art ? \\[0pt]
Y a-t-il une logique de l'inconscient ? \\[0pt]
Y a-t-il une logique des événements historiques ? \\[0pt]
Y a-t-il une logique du désir ? \\[0pt]
Y a-t-il une logique du mythe ? \\[0pt]
Y a-t-il une loi suprême ? \\[0pt]
Y a-t-il une mathématique universelle ? \\[0pt]
Y a-t-il une mécanique des passions ? \\[0pt]
Y a-t-il une médecine de l'âme ? \\[0pt]
Y a-t-il une métaphysique de l'amour ? \\[0pt]
Y a-t-il une méthode de l'interprétation ? \\[0pt]
Y a-t-il une méthode propre aux sciences humaines ? \\[0pt]
Y a-t-il une morale universelle ? \\[0pt]
Y a-t-il un empire de la technique ? \\[0pt]
Y a-t-il une nature humaine ? \\[0pt]
Y a-t-il une nécessité de l'erreur ? \\[0pt]
Y a-t-il une nécessité de l'Histoire ? \\[0pt]
Y a-t-il une nécessité morale ? \\[0pt]
Y a-t-il une œuvre du temps ? \\[0pt]
Y a-t-il une opinion publique mondiale ? \\[0pt]
Y a-t-il une ou des morales ? \\[0pt]
Y a-t-il une ou plusieurs philosophies ? \\[0pt]
Y a-t-il une pensée sans signes ? \\[0pt]
Y a-t-il une pensée technique ? \\[0pt]
Y a-t-il une perception esthétique ? \\[0pt]
Y a-t-il une philosophie de la nature ? \\[0pt]
Y a-t-il une philosophie de la philosophie ? \\[0pt]
Y a-t-il une philosophie première ? \\[0pt]
Y a-t-il une place dans les sciences humaines pour la catégorie de personne ? \\[0pt]
Y a-t-il une place pour la morale dans l'économie ? \\[0pt]
Y a-t-il une positivité de l'erreur ? \\[0pt]
Y a-t-il une présence du passé ? \\[0pt]
Y a-t-il une primauté du devoir sur le droit ? \\[0pt]
Y a-t-il une psychologie animale ? \\[0pt]
Y a-t-il une raison à tout ? \\[0pt]
Y a-t-il une raison des choses ? \\[0pt]
Y a-t-il une rationalité dans la religion ? \\[0pt]
Y a-t-il une rationalité de la contingence ? \\[0pt]
Y a-t-il une rationalité des sentiments ? \\[0pt]
Y a-t-il une rationalité du hasard ? \\[0pt]
Y a-t-il une rationalité du marché ? \\[0pt]
Y a-t-il une rationalité propre de l'intérêt ? \\[0pt]
Y a-t-il une réalité des idées ? \\[0pt]
Y a-t-il une réalité du hasard ? \\[0pt]
Y a-t-il une réalité du mal ? \\[0pt]
Y a-t-il une religion civile ? \\[0pt]
Y a-t-il une responsabilité de l'artiste ? \\[0pt]
Y a-t-il une sagesse de l'inconscient ? \\[0pt]
Y a-t-il une sagesse populaire ? \\[0pt]
Y a-t-il une science de la vie ? \\[0pt]
Y a-t-il une science de la vie mentale ? \\[0pt]
Y a-t-il une science de l'esprit ? \\[0pt]
Y a-t-il une science de l'être ? \\[0pt]
Y a-t-il une science de l'homme ? \\[0pt]
Y a-t-il une science de l'individuel ? \\[0pt]
Y a-t-il une science des principes ? \\[0pt]
Y a-t-il une science du juste ? \\[0pt]
Y a-t-il une science du moi ? \\[0pt]
Y a-t-il une science du qualitatif ? \\[0pt]
Y a-t-il une science ou des sciences ? \\[0pt]
Y a-t-il une science politique ? \\[0pt]
Y a-t-il une sensibilité esthétique ? \\[0pt]
Y a-t-il une servitude volontaire ? \\[0pt]
Y a-t-il une singularité de l'histoire de l'art ? \\[0pt]
Y a-t-il une spécificité de la délibération politique ? \\[0pt]
Y a-t-il une spécificité des sciences humaines ? \\[0pt]
Y a-t-il une spécificité du vivant ? \\[0pt]
Y a-t-il un esprit scientifique ? \\[0pt]
Y a-t-il un État idéal ? \\[0pt]
Y a-t-il une technique de la nature ? \\[0pt]
Y a-t-il une technique pour tout ? \\[0pt]
Y a-t-il une temporalité proprement morale ? \\[0pt]
Y a-t-il une tyrannie du vrai ? \\[0pt]
Y a-t-il une unité de la science ? \\[0pt]
Y a-t-il une unité des devoirs ? \\[0pt]
Y a-t-il une unité des langages humains ? \\[0pt]
Y a-t-il une unité des sciences ? \\[0pt]
Y a-t-il une unité en psychologie ? \\[0pt]
Y a-t-il une universalité des mathématiques ? \\[0pt]
Y a-t-il une universalité du beau ? \\[0pt]
Y a-t-il une utilité des lieux communs ? \\[0pt]
Y a-t-il une valeur de l'inutile ? \\[0pt]
Y a-t-il une vérité absolue ? \\[0pt]
Y a-t-il une vérité dans les arts ? \\[0pt]
Y a-t-il une vérité de la fiction ? \\[0pt]
Y a-t-il une vérité de l'inconscient ? \\[0pt]
Y a-t-il une vérité de l'œuvre d'art ? \\[0pt]
Y a-t-il une vérité des apparences ? \\[0pt]
Y a-t-il une vérité des représentations ? \\[0pt]
Y a-t-il une vérité des sentiments ? \\[0pt]
Y a-t-il une vérité des symboles ? \\[0pt]
Y a-t-il une vérité du sensible ? \\[0pt]
Y a-t-il une vérité du sentiment ? \\[0pt]
Y a-t-il une vérité en art ? \\[0pt]
Y a-t-il une vérité en histoire ? \\[0pt]
Y a-t-il une vérité philosophique ? \\[0pt]
Y a-t-il une vérité sur le bonheur ? \\[0pt]
Y a-t-il une vertu de l'imitation ? \\[0pt]
Y a-t-il une vertu de l'oubli ? \\[0pt]
Y a-t-il une vie de l'esprit ? \\[0pt]
Y a-t-il une violence du droit ? \\[0pt]
Y a-t-il une vision du monde propre à la pensée technique ? \\[0pt]
Y a-t-il une volonté du mal ? \\[0pt]
Y a-t-il un fondement de la croyance ? \\[0pt]
Y a-t-il un fondement moral aux circonstances atténuantes ? \\[0pt]
Y a-t-il un inconscient collectif ? \\[0pt]
Y a-t-il un inconscient psychique ? \\[0pt]
Y a-t-il un inconscient social ? \\[0pt]
Y a t-il un instinct moral ? \\[0pt]
Y a-t-il un intérêt à faire son devoir ? \\[0pt]
Y a-t-il un jugement de l'histoire ? \\[0pt]
Y a-t-il un langage animal ? \\[0pt]
Y a-t-il un langage commun ? \\[0pt]
Y a-t-il un langage de la musique ? \\[0pt]
Y a-t-il un langage de l'art ? \\[0pt]
Y a-t-il un langage de l'inconscient ? \\[0pt]
Y a-t-il un langage des animaux ? \\[0pt]
Y a-t-il un langage du corps ? \\[0pt]
Y a-t-il un langage unifié de la science ? \\[0pt]
Y a-t-il un langage universel ? \\[0pt]
Y a-t-il un mal absolu ? \\[0pt]
Y a-t-il un monde de l'art ? \\[0pt]
Y a-t-il un monde extérieur ? \\[0pt]
Y a-t-il un monde technique ? \\[0pt]
Y a-t-il un moteur de l'histoire ? \\[0pt]
Y a-t-il un objet du désir ? \\[0pt]
Y a-t-il un ordre dans la nature ? \\[0pt]
Y a-t-il un ordre des choses ? \\[0pt]
Y a-t-il un ordre du monde ? \\[0pt]
Y a-t-il un pouvoir de la technique ? \\[0pt]
Y a-t-il un primat de la nature sur la culture ? \\[0pt]
Y a-t-il un principe du mal ? \\[0pt]
Y a-t-il un progrès dans l'art ? \\[0pt]
Y a-t-il un progrès des/en sciences humaines ? \\[0pt]
Y a-t-il un progrès du droit ? \\[0pt]
Y a-t-il un progrès en art ? \\[0pt]
Y a-t-il un progrès en philosophie ? \\[0pt]
Y a-t-il un progrès moral ? \\[0pt]
Y a-t-il un propre de l'homme ? \\[0pt]
Y a-t-il un rapport moral à soi-même ? \\[0pt]
Y a-t-il un rythme de l'histoire ? \\[0pt]
Y a-t-il un savoir de la justice ? \\[0pt]
Y a-t-il un savoir du bien ? \\[0pt]
Y a-t-il un savoir du contingent ? \\[0pt]
Y a-t-il un savoir du corps ? \\[0pt]
Y a-t-il un savoir du juste ? \\[0pt]
Y a-t-il un savoir du politique ? \\[0pt]
Y a-t-il un savoir immédiat ? \\[0pt]
Y a-t-il un savoir inconscient ? \\[0pt]
Y a-t-il un savoir politique ? \\[0pt]
Y a-t-il un sens à dire que Dieu comprend, veut, fait ? \\[0pt]
Y a-t-il un sens à instituer une langue universelle ? \\[0pt]
Y a-t-il un sens à ne plus rien désirer ? \\[0pt]
Y a-t-il un sens à penser un droit des générations futures ? \\[0pt]
Y a-t-il un sens à se dire « citoyen du monde » ? \\[0pt]
Y a-t-il un sens à se dire citoyen du monde ? \\[0pt]
Y a-t-il un sens à s'opposer à la technique ? \\[0pt]
Y a-t-il un sens du beau ? \\[0pt]
Y a-t-il un sens moral ? \\[0pt]
Y a-t-il un sentiment métaphysique ? \\[0pt]
Y a-t-il un sentiment naturel du beau ? \\[0pt]
Y a-t-il un souverain bien ? \\[0pt]
Y a-t-il un temps des choses ? \\[0pt]
Y a-t-il un temps pour l'action ? \\[0pt]
Y a-t-il un temps pour tout ? \\[0pt]
Y a-t-il un traitement scientifique du langage ? \\[0pt]
Y a-t-il un travail de la pensée ? \\[0pt]
Y a-t-il un tribunal de l'histoire ? \\[0pt]
Y a-t-il un usage moral des passions ? \\[0pt]
Y a-t-il un usage purement instrumental de la raison ? \\[0pt]
Y a-t-il vérité sans interprétation ? \\[0pt]
Y a-t-une science du comportement ? \\[0pt]
Y aura-t-il toujours des religions ? \\[0pt]


\subsection{Question totale (en oui/non)}
\label{sec:org7e7456b}
\noindent
2+2 pourrait-il ne pas être égal à 4 ? \\[0pt]
Abstraire, est-ce se couper du réel ? \\[0pt]
À chacun sa vérité ? \\[0pt]
Admettre le hasard, est-ce nier l'ordre de la nature ? \\[0pt]
Admettre une cause première, est-ce faire une pétition de principe ? \\[0pt]
Agir en politique, est-ce agir dans l'incertain ? \\[0pt]
Agir justement fait-il de moi un homme juste ? \\[0pt]
Agir moralement, est-ce lutter contre ses idées ? \\[0pt]
Agir moralement, est-ce lutter contre soi-même ? \\[0pt]
Agir par devoir, est-ce agir contre son intérêt ? \\[0pt]
Agir par devoir, est-ce renoncer à ses intérêts ? \\[0pt]
Agir volontairement, est-ce nécessairement savoir ce que l'on fait ? \\[0pt]
Ai-je des devoirs envers moi-même ? \\[0pt]
Ai-je intérêt à la liberté d'autrui ? \\[0pt]
Ai-je le pouvoir de me changer ? \\[0pt]
Ai-je un corps ? \\[0pt]
Ai-je un corps ou suis-je mon corps ? \\[0pt]
Ai-je une âme ? \\[0pt]
Aimer, est-ce vraiment connaître ? \\[0pt]
Aimer peut-il être un devoir ? \\[0pt]
« Aimer » se dit-il en un seul sens ? \\[0pt]
Animal politique ou social ? \\[0pt]
Appartenons-nous à une culture ? \\[0pt]
Appartient-il aux lois d'éduquer les hommes ? \\[0pt]
Apprend-on à aimer ? \\[0pt]
Apprend-on à être artiste ? \\[0pt]
Apprend-on à être libre ? \\[0pt]
Apprend-on à penser ? \\[0pt]
Apprend-on à percevoir ? \\[0pt]
Apprend-on à percevoir la beauté ? \\[0pt]
Apprend-on à voir ? \\[0pt]
Apprendre à voir ? \\[0pt]
Apprendre s'apprend-il ? \\[0pt]
Arrive-t-il que l'impossible se produise ? \\[0pt]
À science nouvelle, nouvelle philosophie ? \\[0pt]
A-t-on besoin de certitudes ? \\[0pt]
A-t-on besoin de fonder la connaissance ? \\[0pt]
A-t-on besoin de maîtres ? \\[0pt]
A-t-on besoin de spécialistes ? \\[0pt]
A-t-on besoin de spécialistes en politique ? \\[0pt]
A-t-on besoin des poètes ? \\[0pt]
A-t-on besoin d'experts ? \\[0pt]
A-t-on besoin d'un chef ? \\[0pt]
A-t-on des devoirs envers les animaux ? \\[0pt]
A-t-on des devoirs envers l'humanité ? \\[0pt]
A-t-on des devoirs envers qui n'a aucun droit ? \\[0pt]
A-t-on des devoirs envers soi-même ? \\[0pt]
A-t-on des droits contre l'État ? \\[0pt]
A-t-on des raisons de croire ? \\[0pt]
A-t-on des raisons de croire ce qu'on croit ? \\[0pt]
A-t-on expliqué un phénomène quand on l'a rapporté à des lois ? \\[0pt]
A-t-on intérêt à tout savoir ? \\[0pt]
A-t-on le droit de faire tout ce qui est permis par la loi ? \\[0pt]
A-t-on le droit de mentir ? \\[0pt]
A-t-on le droit de résister ? \\[0pt]
A-t-on le droit de se désintéresser de la politique ? \\[0pt]
A-t-on le droit de se révolter ? \\[0pt]
A-t-on le droit de s'évader ? \\[0pt]
A-t-on l'obligation de pardonner ? \\[0pt]
A-t-on raison de mettre la technique en accusation ? \\[0pt]
A-t-on raison de tenir à la démocratie ? \\[0pt]
Au-delà de la nature ? \\[0pt]
Au nom de qui rend-on justice ? \\[0pt]
Au nom de quoi le plaisir serait-il condamnable ? \\[0pt]
Au nom de quoi rend-on justice ? \\[0pt]
Autrui, est-ce n'importe quel autre ? \\[0pt]
Autrui est-il aimable ? \\[0pt]
Autrui est-il inconnaissable ? \\[0pt]
Autrui est-il mon prochain ? \\[0pt]
Autrui est-il mon semblable ? \\[0pt]
Autrui est-il pour moi un mystère ? \\[0pt]
Autrui est-il un autre moi ? \\[0pt]
Autrui est-il un autre moi-même ? \\[0pt]
Autrui me connaît-il mieux que moi-même ? \\[0pt]
Autrui m'est-il étranger ? \\[0pt]
Avez-vous une âme ? \\[0pt]
Avoir des ennemis, est-ce le principe de la politique ? \\[0pt]
Avoir la parole, est-ce avoir le pouvoir ? \\[0pt]
Avoir raison, est-ce nécessairement être raisonnable ? \\[0pt]
Avons-nous à apprendre des images ? \\[0pt]
Avons-nous accès aux choses-mêmes ? \\[0pt]
Avons-nous besoin d'amis ? \\[0pt]
Avons-nous besoin de cérémonies ? \\[0pt]
Avons-nous besoin de Dieu ? \\[0pt]
Avons-nous besoin de héros ? \\[0pt]
Avons-nous besoin de l'État ? \\[0pt]
Avons-nous besoin de lois ? \\[0pt]
Avons-nous besoin de maîtres ? \\[0pt]
Avons-nous besoin de métaphysique ? \\[0pt]
Avons-nous besoin de méthodes ? \\[0pt]
Avons-nous besoin de nous tromper nous-même ? \\[0pt]
Avons-nous besoin de partis politiques ? \\[0pt]
Avons-nous besoin de rêver ? \\[0pt]
Avons-nous besoin de spectacles ? \\[0pt]
Avons-nous besoin de traditions ? \\[0pt]
Avons-nous besoin d'être gouvernés ? \\[0pt]
Avons-nous besoin d'experts en matière d'art ? \\[0pt]
Avons-nous besoin d'un chef ? \\[0pt]
Avons-nous besoin d'une conception métaphysique du monde ? \\[0pt]
Avons-nous besoin d'une définition de l'art ? \\[0pt]
Avons-nous besoin d'une philosophie politique ? \\[0pt]
Avons-nous besoin d'un libre arbitre ? \\[0pt]
Avons-nous besoin d'utopies ? \\[0pt]
Avons-nous des devoirs à l'égard de la vérité ? \\[0pt]
Avons-nous des devoirs envers la nature ? \\[0pt]
Avons-nous des devoirs envers le passé ? \\[0pt]
Avons-nous des devoirs envers les animaux ? \\[0pt]
Avons-nous des devoirs envers les autres êtres vivants ? \\[0pt]
Avons-nous des devoirs envers les générations futures ? \\[0pt]
Avons-nous des devoirs envers les morts ? \\[0pt]
Avons-nous des devoirs envers le vivant ? \\[0pt]
Avons-nous des devoirs envers nous-mêmes ? \\[0pt]
Avons-nous des devoirs envers tous les vivants ? \\[0pt]
Avons-nous des droits sur la nature ? \\[0pt]
Avons-nous des raisons d'espérer ? \\[0pt]
Avons-nous encore besoin de la nature ? \\[0pt]
Avons-nous intérêt à la liberté d'autrui ? \\[0pt]
Avons-nous le devoir de dire la vérité ? \\[0pt]
Avons-nous le devoir d'être heureux ? \\[0pt]
Avons-nous le devoir de vivre ? \\[0pt]
Avons-nous le droit de juger autrui ? \\[0pt]
Avons-nous le droit d'être heureux ? \\[0pt]
Avons-nous le temps d'apprendre à vivre ? \\[0pt]
Avons-nous peur de la liberté ? \\[0pt]
Avons-nous raison de croire ? \\[0pt]
Avons-nous raison d'exiger toujours des raisons ? \\[0pt]
Avons-nous un accès direct aux phénomènes ? \\[0pt]
Avons-nous un corps ? \\[0pt]
Avons-nous un devoir de vérité ? \\[0pt]
Avons-nous un droit au droit ? \\[0pt]
Avons-nous une âme ? \\[0pt]
Avons-nous une identité ? \\[0pt]
Avons-nous une intuition du temps ? \\[0pt]
Avons-nous une obligation envers les générations à venir ? \\[0pt]
Avons-nous une responsabilité envers le passé ? \\[0pt]
Avons-nous un libre arbitre ? \\[0pt]
Avons-nous un monde commun ? \\[0pt]
Axiomatiser, est-ce fonder ? \\[0pt]
Bien agir, est-ce nécessairement faire son devoir ? \\[0pt]
Bien agir, est-ce toujours être moral ? \\[0pt]
Bien parler, est-ce bien penser ? \\[0pt]
Bien parler exige-t-il des qualités morales ? \\[0pt]
Cela a-t-il un sens de penser par soi-même ? \\[0pt]
Ce que je pense est-il nécessairement vrai ? \\[0pt]
Ce que la morale autorise, l'État peut-il légitimement l'interdire ? \\[0pt]
Ce que la technique rend possible, peut-on jamais en empêcher la réalisation ? \\[0pt]
Ce que nous avons le devoir de faire peut-il toujours s'exprimer sous forme de loi ? \\[0pt]
Ce qui dépasse la raison est-il nécessairement irréel ? \\[0pt]
Ce qui doit-être, est-ce autre chose que ce qui est ? \\[0pt]
Ce qui est contingent peut-il être objet de science ? \\[0pt]
Ce qui est contradictoire peut-il exister ? \\[0pt]
Ce qui est démontré est-il nécessairement vrai ? \\[0pt]
Ce qui est faux est-il dénué de sens ? \\[0pt]
Ce qui est ordinaire est-il normal ? \\[0pt]
Ce qui est passé est-il dépassé ? \\[0pt]
Ce qui est subjectif est-il arbitraire ? \\[0pt]
Ce qui est vrai est-il toujours vérifiable ? \\[0pt]
Ce qui ne peut s'acheter est-il dépourvu de valeur ? \\[0pt]
Ce qui n'est pas démontré peut-il être vrai ? \\[0pt]
Ce qui n'est pas matériel peut-il être réel ? \\[0pt]
Ce qui n'est pas réel est-il impossible ? \\[0pt]
Ce qui vaut en théorie vaut-il toujours en pratique ? \\[0pt]
Certaines œuvres d'art ont-elles plus de valeur que d'autres ? \\[0pt]
Ceux qui oppriment sont-ils libres ? \\[0pt]
Ceux qui savent doivent-ils gouverner ? \\[0pt]
Chacun a-t-il le droit d'invoquer sa vérité ? \\[0pt]
Chacun a-t-il sa propre vérité ? \\[0pt]
Changer, est-ce devenir un autre ? \\[0pt]
Changer, est-ce se trahir ? \\[0pt]
Change-t-on avec le temps ? \\[0pt]
Chaque science porte-t-elle une métaphysique qui lui est propre ? \\[0pt]
Châtier, est ce faire honneur au criminel ? \\[0pt]
Chercher son intérêt, est-ce être immoral ? \\[0pt]
Choisir, est-ce renoncer ? \\[0pt]
Choisir ses souvenirs ? \\[0pt]
Choisissons-nous qui nous sommes ? \\[0pt]
Choisit-on ses souvenirs ? \\[0pt]
Choisit-on son corps ? \\[0pt]
Cité juste ou citoyen juste ? \\[0pt]
Citoyen du monde ? \\[0pt]
Classer, est-ce penser ? \\[0pt]
Comparer les cultures ? \\[0pt]
Comprendre, est-ce excuser ? \\[0pt]
Comprendre, est-ce expliquer par les causes ? \\[0pt]
Comprendre, est-ce interpréter ? \\[0pt]
Comprendre le réel, est-ce le dominer ? \\[0pt]
Connaissons-nous la réalité des choses ? \\[0pt]
Connaissons-nous la réalité telle qu'elle est ? \\[0pt]
Connaissons-nous mieux le présent que le passé ? \\[0pt]
Connaît-on jamais pour le plaisir ? \\[0pt]
Connaît-on la vie ou bien connaît-on le vivant ? \\[0pt]
Connaît-on la vie ou connaît-on le vivant ? \\[0pt]
Connaît-on la vie ou le vivant ? \\[0pt]
Connaît-on les choses telles qu'elles sont ? \\[0pt]
Connaît-on pour le plaisir ? \\[0pt]
Connaître, est-ce connaître par les causes ? \\[0pt]
Connaître, est-ce découvrir le réel ? \\[0pt]
Connaître, est-ce dépasser les apparences ? \\[0pt]
Connaître la vie ou le vivant ? \\[0pt]
Connaître les choses, en quoi est-ce déterminer leurs différences ? \\[0pt]
Connaître une chose, est-ce en connaître la cause ? \\[0pt]
Considère-t-on jamais le temps en lui-même ? \\[0pt]
Convient-il d'opposer explication et interprétation ? \\[0pt]
Convient-il d'opposer liberté et nécessité ? \\[0pt]
Croire, est-ce être faible ? \\[0pt]
Croire, est-ce obéir ? \\[0pt]
Croire, est-ce renoncer à l'usage de la raison ? \\[0pt]
Croire, est-ce renoncer à savoir ? \\[0pt]
Croire, est-ce renoncer au savoir ? \\[0pt]
Croire que Dieu existe, est-ce croire en lui ? \\[0pt]
Croit-on ce que l'on veut ? \\[0pt]
Croit-on comme on veut ? \\[0pt]
Dans l'action, est-ce l'intention qui compte ? \\[0pt]
Dans les sciences humaines et sociales, peut-on préconiser l'imitation des sciences de la nature ? \\[0pt]
Décrire, est-ce déjà expliquer ? \\[0pt]
Définir, est-ce déterminer l'essence ? \\[0pt]
Définir et démontrer : à quoi bon ? \\[0pt]
Définir l'art : à quoi bon ? \\[0pt]
Définir la vérité, est-ce la connaître ? \\[0pt]
Dégager la structure, est-ce rendre compte de ce qu'est une chose ? \\[0pt]
Délibérer, est-ce être dans l'incertitude ? \\[0pt]
Démontrer est-il le privilège du mathématicien ? \\[0pt]
Démontre-t-on ailleurs qu'en mathématiques ? \\[0pt]
Dépasser les apparences ? \\[0pt]
Dépend-il de nous d'être heureux ? \\[0pt]
Dépend-il de soi d'être heureux ? \\[0pt]
Déraisonner, est-ce perdre de vue le réel ? \\[0pt]
Des comportements économiques peuvent-ils être rationnels ? \\[0pt]
Des événements aléatoires peuvent-ils obéir à des lois ? \\[0pt]
Des inégalités peuvent-elles être justes ? \\[0pt]
Désirer, est-ce être aliéné ? \\[0pt]
Désire-t-on la reconnaissance ? \\[0pt]
Désire-t-on un autre que soi ? \\[0pt]
Désirons-nous les choses parce qu'elles sont bonnes ? \\[0pt]
Des lois justes suffisent-elles à assurer la justice ? \\[0pt]
Des motivations peuvent-elles être sociales ? \\[0pt]
Des mythes peuvent-ils encore apparaître ? \\[0pt]
Des nations peuvent-elles former une société ? \\[0pt]
Des sciences molles ? \\[0pt]
Des sociétés sans État sont-elles des sociétés politiques ? \\[0pt]
Deux personnes peuvent-elles partager la même pensée ? \\[0pt]
Devant qui est-on responsable ? \\[0pt]
Devant qui sommes-nous responsables ? \\[0pt]
Devient-on raisonnable ? \\[0pt]
Devoir, est-ce avoir une dette envers quelqu'un ? \\[0pt]
Devoir, est-ce pouvoir ? \\[0pt]
Devoir, est-ce vouloir ? \\[0pt]
Devons-nous aimer notre destin ? \\[0pt]
Devons-nous apprendre à vivre ? \\[0pt]
Devons-nous dire la vérité ? \\[0pt]
Devons-nous douter de l'existence des choses ? \\[0pt]
Devons-nous espérer vivre sans travailler ? \\[0pt]
Devons-nous être heureux ? \\[0pt]
Devons-nous être obéissants ? \\[0pt]
Devons-nous nous faire confiance ? \\[0pt]
Devons-nous nous libérer de nos désirs ? \\[0pt]
Devons-nous quelque chose à la nature ? \\[0pt]
Devons-nous tenir certaines connaissances pour acquises ? \\[0pt]
Devons-nous toujours dire la vérité ? \\[0pt]
Devons-nous vivre comme si nous ne devions jamais mourir ? \\[0pt]
Dieu aurait-il pu mieux faire ? \\[0pt]
Dieu est-il l'être ? \\[0pt]
Dieu est-il mort ? \\[0pt]
Dieu est-il mortel ? \\[0pt]
Dieu est-il un concept ? \\[0pt]
Dieu est-il une hypothèse dont le savant et le philosophe peuvent se passer ? \\[0pt]
Dieu est-il une invention humaine ? \\[0pt]
Dieu est-il une limite de la pensée ? \\[0pt]
Dieu pense-t-il ? \\[0pt]
Dieu peut-il tout faire ? \\[0pt]
Dieu, prouvé ou éprouvé ? \\[0pt]
Dire, est-ce autre chose que vouloir dire ? \\[0pt]
Dire, est-ce faire ? \\[0pt]
Dire, est-ce trahir ? \\[0pt]
Dire la vérité, est-ce un devoir ? \\[0pt]
Dire que l'être humain est un animal, est-ce méconnaître sa dignité ? \\[0pt]
Discuter de la beauté d'une chose, est-ce discuter sur une réalité ? \\[0pt]
Dois-je admettre tout ce que je ne peux réfuter ? \\[0pt]
Dois-je mériter mon bonheur ? \\[0pt]
Dois-je obéir à la loi ? \\[0pt]
Dois-je tenir compte de ce que font les autres pour orienter ma conduite ? \\[0pt]
Doit-on apprendre à devenir soi-même ? \\[0pt]
Doit-on apprendre à percevoir ? \\[0pt]
Doit-on apprendre à vivre ? \\[0pt]
Doit-on attendre de la technique qu'elle mette fin au travail ? \\[0pt]
Doit-on bien juger pour bien faire ? \\[0pt]
Doit-on cesser de chercher à définir l'œuvre d'art ? \\[0pt]
Doit-on changer ses désirs, plutôt que l'ordre du monde ? \\[0pt]
Doit-on chasser les artistes de la cité ? \\[0pt]
Doit-on chercher à être heureux ? \\[0pt]
Doit-on corriger les inégalités sociales ? \\[0pt]
Doit-on croire au progrès ? \\[0pt]
Doit-on croire en l'humanité ? \\[0pt]
Doit-on cultiver l'ironie ? \\[0pt]
Doit-on déplorer le désaccord ? \\[0pt]
Doit-on distinguer devoir moral et obligation sociale ? \\[0pt]
Doit-on identifier l'âme à la conscience ? \\[0pt]
Doit-on interpréter les rêves ? \\[0pt]
Doit-on justifier les inégalités ? \\[0pt]
Doit-on le respect au vivant ? \\[0pt]
Doit-on mûrir pour la liberté ? \\[0pt]
Doit-on préférer l'injustice au désordre ? \\[0pt]
Doit-on rechercher le bonheur ? \\[0pt]
Doit-on rechercher l'harmonie ? \\[0pt]
Doit-on refuser d'interpréter ? \\[0pt]
Doit-on répondre de ce qu'on est devenu ? \\[0pt]
Doit-on respecter la nature ? \\[0pt]
Doit-on respecter les êtres vivants ? \\[0pt]
Doit-on se faire l'avocat du diable ? \\[0pt]
Doit-on se justifier d'exister ? \\[0pt]
Doit-on se mettre à la place d'autrui ? \\[0pt]
Doit-on se passer des utopies ? \\[0pt]
Doit-on souffrir de n'être pas compris ? \\[0pt]
Doit-on tenir le plaisir pour une fin ? \\[0pt]
Doit-on toujours dire la vérité ? \\[0pt]
Doit-on toujours rechercher la vérité ? \\[0pt]
Doit-on tout accepter de l'État ? \\[0pt]
Doit-on tout attendre de l'État ? \\[0pt]
Doit-on tout calculer ? \\[0pt]
Doit-on tout contrôler ? \\[0pt]
Doit-on tout pardonner ? \\[0pt]
Doit-on travailler ? \\[0pt]
Doit-on vraiment tout pardonner ? \\[0pt]
Donner, à quoi bon ? \\[0pt]
Donner l'exemple ? \\[0pt]
D'où la politique tire-t-elle sa légitimité ? \\[0pt]
D'où viennent les concepts ? \\[0pt]
D'où viennent les idées générales ? \\[0pt]
D'où viennent les préjugés ? \\[0pt]
D'où viennent nos idées ? \\[0pt]
D'où vient aux objets techniques leur beauté ? \\[0pt]
D'où vient la certitude ? \\[0pt]
D'où vient la certitude dans les sciences ? \\[0pt]
D'où vient la force d'une religion ? \\[0pt]
D'où vient l'amour de Dieu ? \\[0pt]
D'où vient la servitude ? \\[0pt]
D'où vient la signification des mots ? \\[0pt]
D'où vient le mal ? \\[0pt]
D'où vient le plaisir de lire ? \\[0pt]
D'où vient le sens du devoir ? \\[0pt]
D'où vient que l'histoire soit autre chose qu'un chaos ? \\[0pt]
Droit et devoir sont-ils liés ? \\[0pt]
Droits de l'homme ou droits du citoyen ? \\[0pt]
Droits et devoirs sont-ils réciproques ? \\[0pt]
Du passé pouvons-nous faire table rase ? \\[0pt]
Du penser à l'être, la conséquence est-elle bonne ? \\[0pt]
Échanger, est-ce créer de la valeur ? \\[0pt]
Échanger, est-ce partager ? \\[0pt]
Échanger, est-ce risquer ? \\[0pt]
Échappe-t-on à la nécessité ? \\[0pt]
Écrire l'histoire, est-ce donner un sens à la violence ? \\[0pt]
Écrire l'histoire, est-ce la faire ? \\[0pt]
Éduquer, est-ce dénaturer ? \\[0pt]
En histoire, tout est-il affaire d'interprétation ? \\[0pt]
En mathématiques, la notion d'existence a-t-elle un sens ? \\[0pt]
En morale, est-ce seulement l'intention qui compte ? \\[0pt]
En morale, peut-on dire : « C'est l'intention qui compte » ? \\[0pt]
En morale, y a-t-il quelque chose à savoir ? \\[0pt]
En politique, faut-il refuser l'utopie ? \\[0pt]
En politique, ne fait-on que défendre ses intérêts ? \\[0pt]
En politique, ne faut-il croire qu'aux rapports de force ? \\[0pt]
En politique n'y a-t-il que des rapports de force ? \\[0pt]
En politique, peut-on faire table rase du passé ? \\[0pt]
En politique, y a-t-il des modèles ? \\[0pt]
Enseigner, est-ce transmettre un savoir ? \\[0pt]
Entre croire et savoir, y a-t-il une différence de nature ? \\[0pt]
Entre l'art et la nature, qui imite l'autre ? \\[0pt]
Entre la sécurité de tous et la liberté de chacun, le politique doit-il choisir ? \\[0pt]
Entre le vrai et le faux y a-t-il une place pour le probable ? \\[0pt]
Entre l'opinion et la science, n'y a-t-il qu'une différence de degré ? \\[0pt]
Entre l'utile et l'honnête, peut-on choisir ? \\[0pt]
Envers qui avons-nous des devoirs ? \\[0pt]
Est-ce à la fin que le sens apparaît ? \\[0pt]
Est-ce à la raison de déterminer ce qui est réel ? \\[0pt]
Est-ce à l'État de faire le bonheur du peuple ? \\[0pt]
Est-ce de la force que l'État tient son autorité ? \\[0pt]
Est-ce la certitude qui fait la science ? \\[0pt]
Est-ce la démonstration qui fait la science ? \\[0pt]
Est-ce la majorité qui doit décider ? \\[0pt]
Est-ce la mémoire qui constitue mon identité ? \\[0pt]
Est-ce l'autorité qui fait la loi ? \\[0pt]
Est-ce le cerveau qui pense ? \\[0pt]
Est-ce l'échange utilitaire qui fait le lien social ? \\[0pt]
Est-ce le corps qui perçoit ? \\[0pt]
Est-ce l'ignorance qui nous fait croire ? \\[0pt]
Est-ce l'ignorance qui rend les hommes croyants ? \\[0pt]
Est-ce l'intérêt qui fonde le lien social ? \\[0pt]
Est-ce l'utilité qui définit un objet technique ? \\[0pt]
Est-ce par désir de la vérité que l'homme cherche à savoir ? \\[0pt]
Est-ce par son objet ou par ses méthodes qu'une science peut se définir ? \\[0pt]
Est-ce pour des raisons morales qu'il faut protéger l'environnement ? \\[0pt]
Est-ce seulement l'intention qui compte ? \\[0pt]
Est-ce un devoir d'aimer son prochain ? \\[0pt]
Est-ce un devoir de rechercher la vérité ? \\[0pt]
Est-ce un devoir d'être sincère ? \\[0pt]
Est-ce une chance de naître humain ? \\[0pt]
Est-il bien vrai qu'« on n'arrête pas le progrès » ? \\[0pt]
Est-il bon qu'un seul commande ? \\[0pt]
Est-il dans mon intérêt d'accomplir mes devoirs ? \\[0pt]
Est-il difficile de savoir ce que l'on veut ? \\[0pt]
Est-il difficile de savoir ce qu'on veut ? \\[0pt]
Est-il difficile d'être heureux ? \\[0pt]
Est-il difficile de vivre en société ? \\[0pt]
Est-il facile d'être libre ? \\[0pt]
Est-il immoral de se rendre heureux ? \\[0pt]
Est-il impossible de moraliser la politique ? \\[0pt]
Est-il judicieux de revenir sur ses décisions ? \\[0pt]
Est-il juste de payer l'impôt ? \\[0pt]
Est-il juste d'interpréter la loi ? \\[0pt]
Est-il justifié de parler de « corps social » ? \\[0pt]
Est-il légitime d'affirmer que seul le présent existe ? \\[0pt]
Est-il légitime d'opposer liberté et nécessité ? \\[0pt]
Est-il mauvais de suivre son désir ? \\[0pt]
Est-il méritoire de ne faire que son devoir ? \\[0pt]
Est-il naturel à l'homme de parler ? \\[0pt]
Est-il naturel de s'aimer soi-même ? \\[0pt]
Est-il nécessaire d'espérer pour entreprendre ? \\[0pt]
Est-il parfois bon de mentir ? \\[0pt]
Est-il pertinent d'opposer les actes et les paroles ? \\[0pt]
Est-il pertinent d'opposer l'individuel et le collectif ? \\[0pt]
Est-il possible d'améliorer l'homme ? \\[0pt]
Est-il possible de croire en la vie éternelle ? \\[0pt]
Est-il possible de douter de tout ? \\[0pt]
Est-il possible de ne croire à rien ? \\[0pt]
Est-il possible de ne pas savoir ce que l'on pense ? \\[0pt]
Est-il possible de préparer l'avenir ? \\[0pt]
Est-il possible de tout avoir pour être heureux ? \\[0pt]
Est-il possible d'être immoral sans le savoir ? \\[0pt]
Est-il possible d'être neutre politiquement ? \\[0pt]
Est-il possible d'ignorer toute vérité ? \\[0pt]
Est-il raisonnable d'aimer ? \\[0pt]
Est-il raisonnable d'avoir des certitudes ? \\[0pt]
Est-il raisonnable de chercher à interpréter les rêves ? \\[0pt]
Est-il raisonnable de lutter contre le temps ? \\[0pt]
Est-il raisonnable de se quereller pour des mots ? \\[0pt]
Est-il raisonnable d'être rationnel ? \\[0pt]
Est-il raisonnable de vouloir maîtriser la nature ? \\[0pt]
Est-il si difficile d'accéder à la vérité ? \\[0pt]
Est-il toujours avantageux de promouvoir son propre intérêt ? \\[0pt]
Est-il toujours illicite d'inférer ce qui doit être à partir de ce qui est ? \\[0pt]
Est-il toujours meilleur d'avoir le choix ? \\[0pt]
Est-il toujours moral de faire son devoir ? \\[0pt]
Est-il toujours possible de faire ce que l'on dit ? \\[0pt]
Est-il toujours possible de savoir ce que l'on doit faire ? \\[0pt]
Est-il utile d'avoir mal ? \\[0pt]
Est-il vrai que les animaux ne pensent pas ? \\[0pt]
Est-il vrai que l'ignorant n'est pas libre ? \\[0pt]
Est-il vrai que ma liberté s'arrête là où commence celle des autres ? \\[0pt]
Est-il vrai que nous ne nous tenons jamais au temps présent ? \\[0pt]
Est-il vrai qu'en science, « rien n'est donné, tout est construit » ? \\[0pt]
Est-il vrai que plus on échange, moins on se bat ? \\[0pt]
Est-il vrai qu'on apprenne de ses erreurs ? \\[0pt]
Est-on fondé à distinguer la justice et le droit ? \\[0pt]
Est-on fondé à parler d'une imperfection du langage ? \\[0pt]
Est-on l'auteur de sa propre vie ? \\[0pt]
Est-on le produit d'une culture ? \\[0pt]
Est-on libre de ne pas vouloir ce que l'on veut ? \\[0pt]
Est-on libre de travailler ? \\[0pt]
Est-on libre face à la vérité ? \\[0pt]
Est-on libre par hasard ? \\[0pt]
Est-on propriétaire de son corps ? \\[0pt]
Est-on responsable de ce qu'on n'a pas voulu ? \\[0pt]
Est-on responsable de ses croyances ? \\[0pt]
Est-on responsable de ses rêves ? \\[0pt]
Est-on responsable de son passé ? \\[0pt]
Est-on sociable par nature ? \\[0pt]
Est-on vraiment libre d'exprimer ce que l'on veut ? \\[0pt]
Établir la vérité, est-ce nécessairement démontrer ? \\[0pt]
Être à l'écoute de son désir, est-ce nier le désir de l'autre ? \\[0pt]
Être conscient de soi, est-ce être maître de soi ? \\[0pt]
Être conscient, est-ce être maître de soi ? \\[0pt]
Être cultivé, est-ce tout connaître ? \\[0pt]
Être cultivé rend-il meilleur ? \\[0pt]
Être éduqué, est-ce devenir soi-même ? \\[0pt]
Être, est-ce agir ? \\[0pt]
Être et penser, est-ce la même chose ? \\[0pt]
Être heureux, est-ce devoir ? \\[0pt]
Être libre, cela s'apprend-il ? \\[0pt]
Être libre, est-ce avoir le choix ? \\[0pt]
Être libre, est-ce dire non ? \\[0pt]
Être libre, est-ce échapper aux prévisions ? \\[0pt]
Être libre, est-ce faire ce que l'on veut ? \\[0pt]
Être libre, est-ce n'avoir aucun maître ? \\[0pt]
Être libre, est-ce n'obéir qu'à soi-même ? \\[0pt]
Être libre, est-ce pouvoir choisir ? \\[0pt]
Être libre, est-ce se suffire à soi-même ? \\[0pt]
Être libre, est-ce une question de volonté ? \\[0pt]
Être libre, est-ce vivre comme on l'entend ? \\[0pt]
Être ou ne pas être ? \\[0pt]
Être ou ne pas être, est-ce la question ? \\[0pt]
Être raisonnable, est-ce accepter la réalité telle qu'elle est ? \\[0pt]
Être raisonnable, est-ce renoncer à ses désirs ? \\[0pt]
Être religieux, est-ce nécessairement être dogmatique ? \\[0pt]
« Être » se dit-il en plusieurs sens ? \\[0pt]
« Être soi-même », cela a-t-il un sens ? \\[0pt]
Être spontané, est-ce être libre ? \\[0pt]
Être un sujet, est-ce être maître de soi ? \\[0pt]
Existe-il des mentalités primitives ? \\[0pt]
Exister, est-ce simplement vivre ? \\[0pt]
Existe-t-il au moins un devoir universel ? \\[0pt]
Existe-t-il dans le monde des réalités identiques ? \\[0pt]
Existe-t-il de faux besoins ? \\[0pt]
Existe-t-il des autorités en matière d'art ? \\[0pt]
Existe-t-il des choses en soi ? \\[0pt]
Existe-t-il des choses réellement sublimes ? \\[0pt]
Existe-t-il des choses sans prix ? \\[0pt]
Existe-t-il des comportements contraires à la nature ? \\[0pt]
Existe-t-il des connaissances a priori ? \\[0pt]
Existe-t-il des croyances collectives ? \\[0pt]
Existe-t-il des degrés de vérité ? \\[0pt]
Existe-t-il des démonstrations métaphysiques ? \\[0pt]
Existe-t-il des désirs coupables ? \\[0pt]
Existe-t-il des devoirs envers soi-même ? \\[0pt]
Existe-t-il des dilemmes moraux ? \\[0pt]
Existe-t-il des émotions proprement esthétiques ? \\[0pt]
Existe-t-il des états mentaux collectifs ? \\[0pt]
Existe-t-il des expériences métaphysiques ? \\[0pt]
Existe-t-il des identités collectives ? \\[0pt]
Existe-t-il des intuitions métaphysiques ? \\[0pt]
Existe-t-il des langages naturels ? \\[0pt]
Existe-t-il des mondes sociaux ? \\[0pt]
Existe-t-il des paroles vraies ? \\[0pt]
Existe-t-il des plaisirs purs ? \\[0pt]
Existe-t-il des preuves irréfutables ? \\[0pt]
Existe-t-il des principes premiers ? \\[0pt]
Existe-t-il des problèmes insolubles ? \\[0pt]
Existe-t-il des questions sans réponse ? \\[0pt]
Existe-t-il des sciences de différentes natures ? \\[0pt]
Existe-t-il des signes naturels ? \\[0pt]
Existe-t-il des sujets collectifs ? \\[0pt]
Existe-t-il des violences légitimes ? \\[0pt]
Existe-t-il différentes sortes de sciences ? \\[0pt]
Existe-t-il plusieurs déterminismes ? \\[0pt]
Existe-t-il plusieurs mondes ? \\[0pt]
Existe-t-il un art de la parole ? \\[0pt]
Existe-t-il un art de penser ? \\[0pt]
Existe-t-il un bien commun qui soit la norme de la vie politique ? \\[0pt]
Existe-t-il un bien souverain ? \\[0pt]
Existe-t-il un déterminisme social ? \\[0pt]
Existe-t-il un droit de mentir ? \\[0pt]
Existe-t-il une autorité naturelle ? \\[0pt]
Existe-t-il une hiérarchie des êtres ? \\[0pt]
Existe-t-il une histoire du présent ? \\[0pt]
Existe-t-il une méthode pour rechercher la vérité ? \\[0pt]
Existe-t-il une méthode pour trouver la vérité ? \\[0pt]
Existe-t-il une opinion publique ? \\[0pt]
Existe-t-il une réalité subjective ? \\[0pt]
Existe-t-il une réalité symbolique ? \\[0pt]
Existe-t-il une science de la morale ? \\[0pt]
Existe-t-il une science de l'être ? \\[0pt]
Existe-t-il une sensibilité philosophique ? \\[0pt]
Existe-t-il une unité des arts ? \\[0pt]
Existe-t-il une violence symbolique ? \\[0pt]
Existe-t-il une vision scientifique du monde ? \\[0pt]
Existe-t-il un vocabulaire neutre des droits fondamentaux ? \\[0pt]
Expliquer, est-ce excuser ? \\[0pt]
Expliquer, est-ce interpréter ? \\[0pt]
Expliquer, est-ce justifier ? \\[0pt]
Exploiter la nature, est-ce lui vouloir du mal ? \\[0pt]
Faire de la métaphysique, est-ce se détourner du monde ? \\[0pt]
Faire est-il nécessairement savoir faire ? \\[0pt]
Faire son devoir, est-ce faire son propre malheur ? \\[0pt]
Faire son devoir, est-ce là toute la morale ? \\[0pt]
Faire son devoir, est-ce obéir ? \\[0pt]
Faire son devoir, est-ce payer une dette ? \\[0pt]
Faire son devoir, est-ce perdre toute sensibilité ? \\[0pt]
Faire son devoir, est-ce se soumettre ? \\[0pt]
Faisons-nous l'histoire ? \\[0pt]
Faisons-nous l'Histoire ? \\[0pt]
Faisons-nous notre propre malheur ? \\[0pt]
Fait-on de la politique pour changer les choses ? \\[0pt]
Faudrait-il bannir la polysémie du langage ? \\[0pt]
Faudrait-il ne rien oublier ? \\[0pt]
Faudrait-il vivre sans passion ? \\[0pt]
Faut-avoir peur de la technique ? \\[0pt]
Faut-il accepter sa condition ? \\[0pt]
Faut-il accorder de l'importance aux mots ? \\[0pt]
Faut-il accorder l'esprit aux bêtes ? \\[0pt]
Faut-il affirmer son identité ? \\[0pt]
Faut-il aimer autrui pour le respecter ? \\[0pt]
Faut-il aimer la nature plus que soi-même ? \\[0pt]
Faut-il aimer la vie ? \\[0pt]
Faut-il aimer le destin ? \\[0pt]
Faut-il aimer son prochain ? \\[0pt]
Faut-il aimer son prochain comme soi-même ? \\[0pt]
Faut-il aller au-delà des apparences ? \\[0pt]
Faut-il aller toujours plus vite ? \\[0pt]
Faut-il apprendre à être libre ? \\[0pt]
Faut-il apprendre à obéir ? \\[0pt]
Faut-il apprendre à vivre en renonçant au bonheur ? \\[0pt]
Faut-il apprendre à voir ? \\[0pt]
Faut-il assimiler la pensée au langage ? \\[0pt]
Faut-il avoir des ennemis ? \\[0pt]
Faut-il avoir des principes ? \\[0pt]
Faut-il avoir foi en la raison ? \\[0pt]
Faut-il avoir foi en la technique ? \\[0pt]
Faut-il avoir peur de la liberté ? \\[0pt]
Faut-il avoir peur de la nature ? \\[0pt]
Faut-il avoir peur de la technique ? \\[0pt]
Faut-il avoir peur de l'avenir ? \\[0pt]
Faut-il avoir peur de la vérité ? \\[0pt]
Faut-il avoir peur des contradictions ? \\[0pt]
Faut-il avoir peur des habitudes ? \\[0pt]
Faut-il avoir peur des machines ? \\[0pt]
Faut-il avoir peur d'être libre ? \\[0pt]
Faut-il avoir peur du désordre ? \\[0pt]
Faut-il changer le monde ? \\[0pt]
Faut-il changer ses désirs plutôt que l'ordre du monde ? \\[0pt]
Faut-il chasser le naturel ? \\[0pt]
Faut-il chasser les poètes ? \\[0pt]
Faut-il chercher à être heureux ? \\[0pt]
Faut-il chercher à satisfaire tous nos désirs ? \\[0pt]
Faut-il chercher à se connaître ? \\[0pt]
Faut-il chercher à tout démontrer ? \\[0pt]
Faut-il chercher la paix à tout prix ? \\[0pt]
Faut-il chercher le bonheur à tout prix ? \\[0pt]
Faut-il chercher un sens à l'histoire ? \\[0pt]
Faut-il choisir entre être heureux et être libre ? \\[0pt]
Faut-il combattre ses illusions ? \\[0pt]
Faut-il comprendre pour croire ? \\[0pt]
Faut-il concilier les contraires ? \\[0pt]
Faut-il condamner la fiction ? \\[0pt]
Faut-il condamner la rhétorique ? \\[0pt]
Faut-il condamner le luxe ? \\[0pt]
Faut-il condamner les illusions ? \\[0pt]
Faut-il connaître la vérité pour pouvoir mentir ? \\[0pt]
Faut-il connaître l'histoire des sciences ? \\[0pt]
Faut-il connaître l'Histoire pour gouverner ? \\[0pt]
Faut-il considérer le droit pénal comme instituant une violence légitime ? \\[0pt]
Faut-il considérer les faits sociaux comme des choses ? \\[0pt]
Faut-il contrôler les mœurs ? \\[0pt]
Faut-il craindre de perdre son temps ? \\[0pt]
Faut-il craindre la démocratie ? \\[0pt]
Faut-il craindre la mort ? \\[0pt]
Faut-il craindre la révolution ? \\[0pt]
Faut-il craindre la tyrannie du bonheur ? \\[0pt]
Faut-il craindre le développement des techniques ? \\[0pt]
Faut-il craindre le pire ? \\[0pt]
Faut-il craindre le pouvoir des machines ? \\[0pt]
Faut-il craindre le pouvoir souverain ? \\[0pt]
Faut-il craindre le regard d'autrui ? \\[0pt]
Faut-il craindre les foules ? \\[0pt]
Faut-il craindre les machines ? \\[0pt]
Faut-il craindre les masses ? \\[0pt]
Faut-il craindre l'État ? \\[0pt]
Faut-il craindre l'ordre ? \\[0pt]
Faut-il croire au progrès ? \\[0pt]
Faut-il croire en la science ? \\[0pt]
Faut-il croire en quelque chose ? \\[0pt]
Faut-il croire les historiens ? \\[0pt]
Faut-il croire que l'histoire a un sens ? \\[0pt]
Faut-il cultiver le naturel ? \\[0pt]
Faut-il défendre la démocratie ? \\[0pt]
Faut-il défendre les faibles ? \\[0pt]
Faut-il défendre l'ordre à tout prix ? \\[0pt]
Faut-il dépasser les apparences ? \\[0pt]
Faut-il désespérer de l'humanité ? \\[0pt]
Faut-il des frontières ? \\[0pt]
Faut-il des héros ? \\[0pt]
Faut-il désirer la vérité ? \\[0pt]
Faut-il des outils pour penser ? \\[0pt]
Faut-il des techniques du raisonnement ? \\[0pt]
Faut-il détruire l'État ? \\[0pt]
Faut-il détruire pour créer ? \\[0pt]
Faut-il dire de la justice qu'elle n'existe pas ? \\[0pt]
Faut-il dire tout haut ce que les autres pensent tout bas ? \\[0pt]
Faut-il diriger l'économie ? \\[0pt]
Faut-il distinguer ce qui est de ce qui doit être ? \\[0pt]
Faut-il distinguer culture et civilisation ? \\[0pt]
Faut-il distinguer désir et besoin ? \\[0pt]
Faut-il distinguer devoir moral et obligation sociale ? \\[0pt]
Faut-il distinguer esthétique et philosophie de l'art ? \\[0pt]
Faut-il donner un sens à la souffrance ? \\[0pt]
Faut-il douter de ce qu'on ne peut pas démontrer ? \\[0pt]
Faut-il douter de tout ? \\[0pt]
Faut-il du passé faire table rase ? \\[0pt]
Faut-il écouter sa conscience ? \\[0pt]
Faut-il enfermer ? \\[0pt]
Faut-il enfermer les œuvres dans les musées ? \\[0pt]
Faut-il en finir avec l'esthétique ? \\[0pt]
Faut-il espérer pour agir ? \\[0pt]
Faut-il être à l'écoute du corps ? \\[0pt]
Faut-il être bon ? \\[0pt]
Faut-il être cohérent ? \\[0pt]
Faut-il être connaisseur pour apprécier une œuvre d'art ? \\[0pt]
Faut-il être cosmopolite ? \\[0pt]
Faut-il être courageux pour être libre ? \\[0pt]
Faut-il être discipliné ? \\[0pt]
Faut-il être fidèle à soi-même ? \\[0pt]
Faut-il être heureux ? \\[0pt]
Faut-il être idéaliste ? \\[0pt]
Faut-il être libre pour être heureux ? \\[0pt]
Faut-il être logique avec soi-même ? \\[0pt]
Faut-il être mesuré ? \\[0pt]
Faut-il être mesuré en toutes choses ? \\[0pt]
Faut-il être modéré ? \\[0pt]
Faut-il être naturel ? \\[0pt]
Faut-il être objectif ? \\[0pt]
Faut-il être original ? \\[0pt]
Faut-il être positif ? \\[0pt]
Faut-il être pragmatique ? \\[0pt]
Faut-il être réaliste ? \\[0pt]
Faut-il être réaliste en politique ? \\[0pt]
Faut-il être relativiste ? \\[0pt]
Faut-il expliquer la morale par son utilité ? \\[0pt]
Faut-il faire confiance au progrès technique ? \\[0pt]
Faut-il faire de la philosophie ? \\[0pt]
Faut-il faire de nécessité vertu ? \\[0pt]
Faut-il faire de sa vie une œuvre d'art ? \\[0pt]
Faut-il faire des efforts pour être soi-même ? \\[0pt]
Faut-il faire table rase du passé ? \\[0pt]
Faut-il faire une place à la croyance ? \\[0pt]
Faut-il forcer les gens à participer à la vie politique ? \\[0pt]
Faut-il fuir la politique ? \\[0pt]
Faut-il garder ses illusions ? \\[0pt]
Faut-il hiérarchiser les désirs ? \\[0pt]
Faut-il hiérarchiser les formes de vie ? \\[0pt]
Faut-il hiérarchiser ses désirs ? \\[0pt]
Faut-il imaginer que nous sommes heureux ? \\[0pt]
Faut-il imposer la vérité ? \\[0pt]
Faut-il interpréter la loi ? \\[0pt]
Faut-il joindre l'utile à l'agréable ? \\[0pt]
Faut-il laisser faire la nature ? \\[0pt]
Faut-il laisser parler la nature ? \\[0pt]
Faut-il libérer la parole ? \\[0pt]
Faut-il libérer l'humanité du travail ? \\[0pt]
Faut-il limiter la souveraineté ? \\[0pt]
Faut-il limiter la souveraineté de l'État ? \\[0pt]
Faut-il limiter le pouvoir de l'État ? \\[0pt]
Faut-il limiter les prétentions de la science ? \\[0pt]
Faut-il limiter l'exercice de la puissance publique ? \\[0pt]
Faut-il lire des romans ? \\[0pt]
Faut-il maîtriser ses émotions ? \\[0pt]
Faut-il mathématiser le réel pour le connaître ? \\[0pt]
Faut-il ménager les apparences ? \\[0pt]
Faut-il mépriser le luxe ? \\[0pt]
Faut-il mériter son bonheur ? \\[0pt]
Faut-il mieux vivre comme si nous ne devions jamais mourir ? \\[0pt]
Faut-il ne manquer de rien pour être heureux ? \\[0pt]
Faut-il n'être jamais méchant ? \\[0pt]
Faut-il obéir à la voix de sa conscience ? \\[0pt]
Faut-il opposer à la politique la souveraineté du droit ? \\[0pt]
Faut-il opposer culture et nature ? \\[0pt]
Faut-il opposer droits et devoirs ? \\[0pt]
Faut-il opposer éduquer et instruire ? \\[0pt]
Faut-il opposer histoire et mémoire ? \\[0pt]
Faut-il opposer la foi et la raison ? \\[0pt]
Faut-il opposer la matière et l'esprit ? \\[0pt]
Faut-il opposer l'art à la connaissance ? \\[0pt]
Faut-il opposer la société et l'État ? \\[0pt]
Faut-il opposer la théorie et la pratique ? \\[0pt]
Faut-il opposer la vie à la mort ? \\[0pt]
Faut-il opposer le devoir-être à l'être ? \\[0pt]
Faut-il opposer le don et l'échange ? \\[0pt]
Faut-il opposer le réel et l'imaginaire ? \\[0pt]
Faut-il opposer l'esprit et la matière ? \\[0pt]
Faut-il opposer l'État et la société ? \\[0pt]
Faut-il opposer le temps vécu et le temps des choses ? \\[0pt]
Faut-il opposer l'histoire et la fiction ? \\[0pt]
Faut-il opposer nature et culture ? \\[0pt]
Faut-il opposer produire et créer ? \\[0pt]
Faut-il opposer raison et sensation ? \\[0pt]
Faut-il opposer rhétorique et philosophie ? \\[0pt]
Faut-il opposer science et croyance ? \\[0pt]
Faut-il opposer science et métaphysique ? \\[0pt]
Faut-il opposer sciences humaines et sciences de la nature ? \\[0pt]
Faut-il opposer sens et vérité ? \\[0pt]
Faut-il opposer subjectivité et objectivité ? \\[0pt]
Faut-il opposer technique et nature ? \\[0pt]
Faut-il opposer théorie et expérience ? \\[0pt]
Faut-il oublier le passé ? \\[0pt]
Faut-il oublier le passé pour se donner un avenir ? \\[0pt]
Faut-il pardonner ? \\[0pt]
Faut-il parfois désobéir aux lois ? \\[0pt]
Faut-il parfois sacrifier la vérité ? \\[0pt]
Faut-il parler pour avoir des idées générales ? \\[0pt]
Faut-il partager la souveraineté ? \\[0pt]
Faut-il penser l'État comme un corps ? \\[0pt]
Faut-il penser l'État comme un individu ? \\[0pt]
Faut-il perdre ses illusions ? \\[0pt]
Faut-il perdre son temps ? \\[0pt]
Faut-il poser des limites à l'activité rationnelle ? \\[0pt]
Faut-il pour le connaître faire du vivant un objet ? \\[0pt]
Faut-il préférer la liberté à l'égalité ? \\[0pt]
Faut-il préférer l'art à la nature ? \\[0pt]
Faut-il préférer le bonheur à la vérité ? \\[0pt]
Faut-il préférer l'injustice au désordre ? \\[0pt]
Faut-il préférer une injustice au désordre ? \\[0pt]
Faut-il prendre soin de soi ? \\[0pt]
Faut-il privilégier le naturel ? \\[0pt]
Faut-il protéger la dignité humaine ? \\[0pt]
Faut-il protéger la nature ? \\[0pt]
Faut-il protéger les faibles contre les forts ? \\[0pt]
Faut-il que le réel ait un sens ? \\[0pt]
Faut-il que les meilleurs gouvernent ? \\[0pt]
Faut-il rechercher la certitude ? \\[0pt]
Faut-il rechercher la perfection dans le langage ? \\[0pt]
Faut-il rechercher la simplicité ? \\[0pt]
Faut-il rechercher le bonheur ? \\[0pt]
Faut-il rechercher l'harmonie ? \\[0pt]
Faut-il reconnaître pour connaître ? \\[0pt]
Faut-il regretter l'équivocité du langage ? \\[0pt]
Faut-il réguler la technique ? \\[0pt]
Faut-il rejeter tous les préjugés ? \\[0pt]
Faut-il rejeter toute norme ? \\[0pt]
Faut-il renoncer à connaître la nature des choses ? \\[0pt]
Faut-il renoncer à croire ? \\[0pt]
Faut-il renoncer à faire du travail une valeur ? \\[0pt]
Faut-il renoncer à la certitude ? \\[0pt]
Faut-il renoncer à la distinction entre nature et culture ? \\[0pt]
Faut-il renoncer à la vérité ? \\[0pt]
Faut-il renoncer à l'idée d'âme ? \\[0pt]
Faut-il renoncer à l'idée de fondement ? \\[0pt]
Faut-il renoncer à l'impossible ? \\[0pt]
Faut-il renoncer à rechercher la vérité ? \\[0pt]
Faut-il renoncer à son désir ? \\[0pt]
Faut-il résister à la peur de mourir ? \\[0pt]
Faut-il respecter la nature ? \\[0pt]
Faut-il respecter la tradition ? \\[0pt]
Faut-il respecter la vie ? \\[0pt]
Faut-il respecter les convenances ? \\[0pt]
Faut-il respecter le vivant ? \\[0pt]
Faut-il restaurer les œuvres d'art ? \\[0pt]
Faut-il rester impartial ? \\[0pt]
Faut-il rester naturel ? \\[0pt]
Faut-il rester soi-même ? \\[0pt]
Faut-il rire ou pleurer ? \\[0pt]
Faut-il rompre avec le passé ? \\[0pt]
Faut-il s'adapter ? \\[0pt]
Faut-il s'adapter aux événements ? \\[0pt]
Faut-il s'affranchir des désirs ? \\[0pt]
Faut-il s'aimer pour vivre ensemble ? \\[0pt]
Faut-il s'aimer soi-même ? \\[0pt]
Faut-il s'attendre à l'inattendu ? \\[0pt]
Faut-il sauver des vies à tout prix ? \\[0pt]
Faut-il sauver les apparences ? \\[0pt]
Faut-il savoir ce que l'on veut ? \\[0pt]
Faut-il savoir désobéir ? \\[0pt]
Faut-il savoir mentir ? \\[0pt]
Faut-il savoir obéir pour gouverner ? \\[0pt]
Faut-il savoir pour agir ? \\[0pt]
Faut-il savoir prendre des risques ? \\[0pt]
Faut-il se chercher pour se trouver ? \\[0pt]
Faut-il se connaître pour être maître de soi ? \\[0pt]
Faut-il se contenter de peu ? \\[0pt]
Faut-il se cultiver ? \\[0pt]
Faut-il se délivrer de la peur ? \\[0pt]
Faut-il se délivrer des passions ? \\[0pt]
Faut-il se détacher du monde ? \\[0pt]
Faut-il s'efforcer d'être moins personnel ? \\[0pt]
Faut-il se fier à ce que l'on ressent ? \\[0pt]
Faut-il se fier à la majorité ? \\[0pt]
Faut-il se fier à sa propre raison ? \\[0pt]
Faut-il se fier au témoignage des sens ? \\[0pt]
Faut-il se fier aux apparences ? \\[0pt]
Faut-il se libérer de soi-même ? \\[0pt]
Faut-il se libérer du travail ? \\[0pt]
Faut-il se libérer pour être libre ? \\[0pt]
Faut-il se méfier de la technique ? \\[0pt]
Faut-il se méfier de l'écriture ? \\[0pt]
Faut-il se méfier de l'imagination ? \\[0pt]
Faut-il se méfier de l'inspiration ? \\[0pt]
Faut-il se méfier de l'intuition ? \\[0pt]
Faut-il se méfier des apparences ? \\[0pt]
Faut-il se méfier de ses désirs ? \\[0pt]
Faut-il se méfier des images ? \\[0pt]
Faut-il se méfier de soi-même ? \\[0pt]
Faut-il se méfier du progrès technique ? \\[0pt]
Faut-il se méfier du volontarisme politique ? \\[0pt]
Faut-il s'en remettre à l'État pour limiter le pouvoir de l'État ? \\[0pt]
Faut-il s'en tenir aux faits ? \\[0pt]
Faut-il séparer la science et la technique ? \\[0pt]
Faut-il séparer morale et politique ? \\[0pt]
Faut-il se poser des questions métaphysiques ? \\[0pt]
Faut-il se réjouir d'exister ? \\[0pt]
Faut-il se rendre à l'évidence ? \\[0pt]
Faut-il se résigner aux inégalités ? \\[0pt]
Faut-il se ressembler pour former une société ? \\[0pt]
Faut-il s'intéresser aux œuvres mineures ? \\[0pt]
Faut-il souhaiter la fin du travail ? \\[0pt]
Faut-il souhaiter une langue universelle ? \\[0pt]
Faut-il suivre la nature ? \\[0pt]
Faut-il suivre ses intuitions ? \\[0pt]
Faut-il supposer un énoncé vrai pour en comprendre le sens ? \\[0pt]
Faut-il surmonter son enfance ? \\[0pt]
Faut-il tolérer les intolérants ? \\[0pt]
Faut-il toujours avoir raison ? \\[0pt]
Faut-il toujours chercher à savoir ? \\[0pt]
Faut-il toujours dire la vérité ? \\[0pt]
Faut-il toujours être en accord avec soi-même ? \\[0pt]
Faut-il toujours éviter de se contredire ? \\[0pt]
Faut-il toujours faire son devoir ? \\[0pt]
Faut-il toujours garder espoir ? \\[0pt]
Faut-il toujours respecter ses engagements ? \\[0pt]
Faut-il tout critiquer ? \\[0pt]
Faut-il tout démontrer ? \\[0pt]
Faut-il tout interpréter ? \\[0pt]
Faut-il un commencement à tout ? \\[0pt]
Faut-il un corps pour penser ? \\[0pt]
Faut-il une guerre pour mettre fin à toutes les guerres ? \\[0pt]
Faut-il une méthode pour découvrir la vérité ? \\[0pt]
Faut-il une théorie de la connaissance ? \\[0pt]
Faut-il vaincre ses désirs plutôt que l'ordre du monde ? \\[0pt]
Faut-il vivre avec son temps ? \\[0pt]
Faut-il vivre comme si l'on ne devait jamais mourir ? \\[0pt]
Faut-il vivre comme si nous étions immortels ? \\[0pt]
Faut-il vivre comme si nous ne devions jamais mourir ? \\[0pt]
Faut-il vivre comme si on ne devait jamais mourir ? \\[0pt]
Faut-il vivre conformément à la nature ? \\[0pt]
Faut-il vivre dangereusement ? \\[0pt]
Faut-il vivre hors de la société pour être heureux ? \\[0pt]
Faut-il voir pour croire ? \\[0pt]
Faut-il vouloir changer le monde ? \\[0pt]
Faut-il vouloir être heureux ? \\[0pt]
Faut-il vouloir la paix ? \\[0pt]
Faut-il vouloir la paix de l'âme ? \\[0pt]
Faut-il vouloir la transparence ? \\[0pt]
Faut-il vouloir savoir ? \\[0pt]
Forme-t-on son esprit en transformant la matière ? \\[0pt]
Gouverner, est-ce dominer ? \\[0pt]
Gouverner, est-ce prévoir ? \\[0pt]
Gouverner, est-ce régner ? \\[0pt]
Hier a-t-il plus de réalité que demain ? \\[0pt]
Histoire et géographie font-elles couple ? \\[0pt]
Imaginer, est-ce créer ? \\[0pt]
Imiter, est-ce copier ? \\[0pt]
Imiter, est-ce manquer d'imagination ? \\[0pt]
Interpréter, est-ce connaître ? \\[0pt]
Interpréter, est-ce renoncer à prouver ? \\[0pt]
Interpréter, est-ce renoncer à toute objectivité ? \\[0pt]
Interpréter, est-ce savoir ? \\[0pt]
Interpréter, est-ce un art ? \\[0pt]
Interpréter est-il subjectif ? \\[0pt]
Interprète-t-on à défaut de connaître ? \\[0pt]
« Je ne voulais pas cela » : en quoi les sciences humaines permettent-elles de comprendre cette excuse ? \\[0pt]
Joue-t-on toujours un rôle ? \\[0pt]
Juge-t-on un acte ou son auteur ? \\[0pt]
Jusqu'à quel point la nature est-elle objet de science ? \\[0pt]
Jusqu'à quel point pouvons-nous juger autrui ? \\[0pt]
Jusqu'à quel point sommes-nous responsables de nos passions ? \\[0pt]
Jusqu'à quel point suis-je mon propre maître ? \\[0pt]
Jusqu'où interpréter ? \\[0pt]
Jusqu'où peut-on dialoguer ? \\[0pt]
Jusqu'où peut-on soigner ? \\[0pt]
Jusqu'où s'étend le domaine de la science ? \\[0pt]
La barbarie peut-elle avoir un visage humain ? \\[0pt]
La beauté a-t-elle une histoire ? \\[0pt]
La beauté comporte-t-elle des degrés ? \\[0pt]
La beauté demande-t-elle des preuves ? \\[0pt]
La beauté est-elle affaire de goût ? \\[0pt]
La beauté est-elle dans le regard ou dans la chose vue ? \\[0pt]
La beauté est-elle dans les choses ? \\[0pt]
La beauté est-elle intemporelle ? \\[0pt]
La beauté est-elle l'objet d'une connaissance ? \\[0pt]
La beauté est-elle partout ? \\[0pt]
La beauté est-elle sensible ? \\[0pt]
La beauté est-elle une promesse de bonheur ? \\[0pt]
La beauté naturelle est-elle une catégorie esthétique périmée ? \\[0pt]
La beauté n'est-elle qu'un effet artistique ? \\[0pt]
La beauté n'est-elle qu'une idée ? \\[0pt]
La beauté nous rend-elle meilleurs ? \\[0pt]
La beauté peut-elle délivrer une vérité ? \\[0pt]
La beauté peut-elle être cachée ? \\[0pt]
La beauté peut-elle laisser indifférent ? \\[0pt]
La beauté s'explique-t-elle ? \\[0pt]
La bêtise et la méchanceté sont-elles liées intrinsèquement ? \\[0pt]
La bêtise et la méchanceté sont-elles liées nécessairement ? \\[0pt]
La bêtise n'est-elle pas proprement humaine ? \\[0pt]
La biologie peut-elle se passer de causes finales ? \\[0pt]
L'absolu est-il connaissable ? \\[0pt]
L'Absolu est-il un objet de savoir ? \\[0pt]
L'abstraction est-elle toujours utile à la science empirique ? \\[0pt]
L'abstrait est-il en dehors de l'espace et du temps ? \\[0pt]
La causalité suppose-t-elle des lois ? \\[0pt]
L'accord des esprits est-il un critère de vérité ? \\[0pt]
L'accord est-il le terme ou le principe du dialogue ? \\[0pt]
La certitude est-elle une marque de vérité ? \\[0pt]
La certitude ne peut-elle naître que de l'évidence ? \\[0pt]
La charité est-elle la négation de la justice ? \\[0pt]
La charité est-elle une vertu ? \\[0pt]
La clarté suffit-elle au savoir ? \\[0pt]
La cohérence est-elle la norme du vrai ? \\[0pt]
La cohérence est-elle un critère de la vérité ? \\[0pt]
La cohérence est-elle un critère de vérité ? \\[0pt]
La cohérence est-elle une vertu ? \\[0pt]
La cohérence logique est-elle une condition suffisante de la démonstration ? \\[0pt]
La cohérence suffit-elle à la vérité ? \\[0pt]
La colère peut-elle être justifiée ? \\[0pt]
La communication est-elle nécessaire à la démocratie ? \\[0pt]
La compassion risque-t-elle d'abolir l'exigence politique ? \\[0pt]
La compétence fonde-t-elle la compétence juridique ? \\[0pt]
La compétence fonde-t-elle l'autorité politique ? \\[0pt]
La compétence technique peut-elle fonder l'autorité publique ? \\[0pt]
La confiance est-elle une vertu ? \\[0pt]
La conflictualité politique est-elle indépassable ? \\[0pt]
La connaissance a-t-elle des limites ? \\[0pt]
La connaissance commune est-elle le point de départ de la science ? \\[0pt]
La connaissance commune fait-elle obstacle à la vérité ? \\[0pt]
La connaissance de la nécessité a priori peut-elle évoluer ? \\[0pt]
La connaissance de la vie se confond-elle avec celle du vivant ? \\[0pt]
La connaissance de l'histoire est-elle utile à l'action ? \\[0pt]
La connaissance de l'histoire nous rend-elle plus libres ? \\[0pt]
La connaissance de soi est-elle évidente ? \\[0pt]
La connaissance du passé est-elle nécessaire à la compréhension du présent ? \\[0pt]
La connaissance du vivant est-elle désintéressée ? \\[0pt]
La connaissance du vivant peut-elle être désintéressée ? \\[0pt]
La connaissance est-elle une contemplation ? \\[0pt]
La connaissance est-elle une croyance justifiée ? \\[0pt]
La connaissance historique est-elle une interprétation des faits ? \\[0pt]
La connaissance historique est-elle utile à l'homme ? \\[0pt]
La connaissance objective doit-elle s'interdire toute interprétation ? \\[0pt]
La connaissance objective exclut-elle toute forme de subjectivité ? \\[0pt]
La connaissance peut-elle être pratique ? \\[0pt]
La connaissance peut-elle se passer de l'imagination ? \\[0pt]
La connaissance scientifique abolit-elle toute croyance ? \\[0pt]
La connaissance scientifique est-elle désintéressée ? \\[0pt]
La connaissance scientifique est-elle fondée sur l'expérience ? \\[0pt]
La connaissance scientifique n'est-elle qu'une croyance argumentée ? \\[0pt]
La connaissance s'interdit-elle tout recours à l'imagination ? \\[0pt]
La connaissance suppose-t-elle une éthique ? \\[0pt]
La conscience a-t-elle des degrés ? \\[0pt]
La conscience a-t-elle des moments ? \\[0pt]
La conscience d'agir suffit-elle à garantir notre liberté ? \\[0pt]
La conscience d'autrui est-elle impénétrable ? \\[0pt]
La conscience définit-elle l'homme en propre ? \\[0pt]
La conscience de la mort est-elle une condition de la sagesse ? \\[0pt]
La conscience de soi est-elle une donnée immédiate ? \\[0pt]
La conscience de soi est-elle une méconnaissance de soi ? \\[0pt]
La conscience de soi suppose-t-elle autrui ? \\[0pt]
La conscience du temps rend-elle l'existence tragique ? \\[0pt]
La conscience entrave-t-elle l'action ? \\[0pt]
La conscience est-elle ce qui fait le sujet ? \\[0pt]
La conscience est-elle intrinsèquement morale ? \\[0pt]
La conscience est-elle le produit de la vie réelle ? \\[0pt]
La conscience est-elle nécessairement malheureuse ? \\[0pt]
La conscience est-elle ou n'est-elle pas ? \\[0pt]
La conscience est-elle source d'illusions ? \\[0pt]
La conscience est-elle temporelle ? \\[0pt]
La conscience est-elle toujours morale ? \\[0pt]
La conscience est-elle une activité ? \\[0pt]
La conscience est-elle une connaissance ? \\[0pt]
La conscience est-elle une illusion ? \\[0pt]
La conscience morale est-elle innée ? \\[0pt]
La conscience morale est-elle naturelle ? \\[0pt]
La conscience morale est-elle une illusion ? \\[0pt]
La conscience morale n'est-elle que le fruit de l'éducation ? \\[0pt]
La conscience morale n'est-elle que le produit de l'éducation ? \\[0pt]
La conscience nous leurre-t-elle ? \\[0pt]
La conscience peut-elle être collective ? \\[0pt]
La conscience peut-elle être objet de science ? \\[0pt]
La conscience peut-elle nous tromper ? \\[0pt]
La considération de l'utilité doit-elle déterminer toutes nos actions ? \\[0pt]
La contingence est-elle la condition de la liberté ? \\[0pt]
La contingence est-elle une condition de la liberté ? \\[0pt]
La contradiction réside-t-elle dans les choses ? \\[0pt]
La contrainte des lois est-elle une violence ? \\[0pt]
La contrainte peut-elle être légitime ? \\[0pt]
La contrainte s'oppose-t-elle à la liberté ? \\[0pt]
La contrainte supprime-t-elle la responsabilité ? \\[0pt]
La critique du pouvoir peut-elle conduire à la désobéissance ? \\[0pt]
La critique fait-elle partie de l'art ? \\[0pt]
La croyance en Dieu est-elle irrationnelle ? \\[0pt]
La croyance est-elle l'asile de l'ignorance ? \\[0pt]
La croyance est-elle signe de faiblesse ? \\[0pt]
La croyance est-elle une opinion ? \\[0pt]
La croyance est-elle une opinion comme les autres ? \\[0pt]
La croyance peut-elle être rationnelle ? \\[0pt]
La croyance peut-elle s'accompagner de certitude ? \\[0pt]
La croyance peut-elle tenir lieu de savoir ? \\[0pt]
La croyance religieuse échappe-t-elle à toute logique ? \\[0pt]
La croyance religieuse se distingue-t-elle des autres formes de croyance ? \\[0pt]
L'action humaine nécessite-t-elle la contingence du monde ? \\[0pt]
L'action morale dépend-elle des circonstances ? \\[0pt]
L'action politique a-t-elle un fondement rationnel ? \\[0pt]
L'action politique peut-elle se passer de mots ? \\[0pt]
L'action requiert-elle décision d'un sujet ? \\[0pt]
L'activité philosophique conduit-elle à la métaphysique ? \\[0pt]
L'activité se laisse-t-elle programmer ? \\[0pt]
L'activité technique dévalorise-t-elle l'homme ? \\[0pt]
La culture commence-t-elle avec le refus de la nature ? \\[0pt]
La culture engendre-t-elle le progrès ? \\[0pt]
La culture est-elle affaire de politique ? \\[0pt]
La culture est-elle ce qui nous relie au passé ? \\[0pt]
La culture est-elle ce qui unit ou ce qui sépare les hommes ? \\[0pt]
La culture est-elle la négation de la nature ? \\[0pt]
La culture est-elle l'objet naturel des sciences humaines ? \\[0pt]
La culture est-elle nécessaire à l'appréciation d'une œuvre d'art ? \\[0pt]
La culture est-elle une question politique ? \\[0pt]
La culture est-elle une seconde nature ? \\[0pt]
La culture est-elle un luxe ? \\[0pt]
La culture est-elle un objet ? \\[0pt]
La culture garantit-elle l'excellence humaine ? \\[0pt]
La culture historique forge-t-elle le jugement moral ? \\[0pt]
La culture libère-t-elle des préjugés ? \\[0pt]
La culture nous rend-elle meilleurs ? \\[0pt]
La culture nous rend-elle plus humains ? \\[0pt]
La culture nous unit-elle ? \\[0pt]
La culture peut-elle être instituée ? \\[0pt]
La culture peut-elle être objet de science ? \\[0pt]
La culture : pour quoi faire ? \\[0pt]
La culture rend-elle plus humain ? \\[0pt]
La curiosité est-elle à l'origine du savoir ? \\[0pt]
La danse est-elle l'œuvre du corps ? \\[0pt]
La décision a-t-elle besoin de raisons ? \\[0pt]
La découverte de la vérité peut-elle être le fait du hasard ? \\[0pt]
La découverte scientifique a-t-elle une logique ? \\[0pt]
La défense de l'intérêt général est-il la fin dernière de la politique ? \\[0pt]
La démarche scientifique exclut-elle tout recours à l'imagination ? \\[0pt]
La démocratie a-t-elle des limites ? \\[0pt]
La démocratie a-t-elle une histoire ? \\[0pt]
La démocratie conduit-elle au règne de l'opinion ? \\[0pt]
La démocratie, est-ce la fin du despotisme ? \\[0pt]
La démocratie, est-ce le pouvoir du plus grand nombre ? \\[0pt]
La démocratie est-elle la loi du plus fort ? \\[0pt]
La démocratie est-elle le pire des régimes politiques ? \\[0pt]
La démocratie est-elle le règne de l'opinion ? \\[0pt]
La démocratie est-elle moyen ou fin ? \\[0pt]
La démocratie est-elle nécessairement libérale ? \\[0pt]
La démocratie est-elle possible ? \\[0pt]
La démocratie est-elle un mythe ? \\[0pt]
La démocratie n'est-elle que la force des faibles ? \\[0pt]
La démocratie peut-elle échapper à la démagogie ? \\[0pt]
La démocratie peut-elle être représentative ? \\[0pt]
La démocratie peut-elle se passer de représentation ? \\[0pt]
La démonstration mathématique peut-elle servir de modèle à toute démonstration ? \\[0pt]
La démonstration n'est-elle qu'une méthode d'exposition des découvertes ? \\[0pt]
La démonstration nous garantit-elle l'accès à la vérité ? \\[0pt]
La démonstration obéit-elle à des lois ? \\[0pt]
La démonstration suffit-elle à établir la vérité ? \\[0pt]
La démonstration supprime-t-elle le doute ? \\[0pt]
L'adéquation aux choses suffit-elle à définir la vérité ? \\[0pt]
La désobéissance peut-elle être légitime ? \\[0pt]
La dialectique est-elle une science ? \\[0pt]
La dialectique remet-elle en question le principe de contradiction ? \\[0pt]
La différence des sexes est-elle une question philosophique ? \\[0pt]
La différence des sexes est-elle un problème philosophique ? \\[0pt]
La distinction de la nature et de la culture est-elle un fait de culture ? \\[0pt]
La distinction entre vraie et fausse sciences a-t-elle un sens ? \\[0pt]
La diversité des cultures fait-elle obstacle à l'unité du genre humain ? \\[0pt]
La diversité des langues est-elle une diversité des pensées ? \\[0pt]
La diversité des langues est-elle un obstacle à l'entente entre les hommes ? \\[0pt]
La diversité des opinions conduit-elle à douter de tout ? \\[0pt]
La docilité est-elle un vice ou une vertu ? \\[0pt]
La domination technique de la nature doit-elle susciter la crainte ou l'espoir ? \\[0pt]
La douleur est-elle utile ? \\[0pt]
La douleur nous apprend-elle quelque chose ? \\[0pt]
La famille est-elle le lieu de la formation morale ? \\[0pt]
La famille est-elle naturelle ? \\[0pt]
La famille est-elle une communauté naturelle ? \\[0pt]
La famille est-elle une institution politique ? \\[0pt]
La famille est-elle une invention culturelle ? \\[0pt]
La famille est-elle un modèle de société ? \\[0pt]
La femme est-elle l'avenir de l'homme ? \\[0pt]
La fiction est-elle fausse ? \\[0pt]
La finalité est-elle nécessaire pour penser le vivant ? \\[0pt]
La fin de la politique est-elle d'éliminer la violence ? \\[0pt]
La fin de la politique est-elle l'établissement de la justice ? \\[0pt]
La fin de la technique se résume-t-elle à son utilité ? \\[0pt]
La fin justifie-t-elle les moyens ? \\[0pt]
La foi est-elle aveugle ? \\[0pt]
La foi est-elle irrationnelle ? \\[0pt]
La foi est-elle rationnelle ? \\[0pt]
La fonction de penser peut-elle se déléguer ? \\[0pt]
La fonction du philosophe est-elle de diriger l'État ? \\[0pt]
La fonction première de l'État est-elle de durer ? \\[0pt]
La force de l'État est-elle nécessaire à la liberté des citoyens ? \\[0pt]
La force donne-t-elle des droits ? \\[0pt]
La force est-elle une vertu ? \\[0pt]
La force et le droit s'opposent-ils nécessairement ? \\[0pt]
La force fait-elle le droit ? \\[0pt]
La franchise est-elle une vertu ? \\[0pt]
La fraternité a-t-elle un sens politique ? \\[0pt]
La fraternité est-elle un idéal moral ? \\[0pt]
La fraternité peut-elle se passer d'un fondement religieux ? \\[0pt]
La fuite du temps est-elle nécessairement un malheur ? \\[0pt]
La généalogie peut-elle être une méthode historique ? \\[0pt]
La géographie est-elle une science humaine ? \\[0pt]
La gloire est-elle un bien ? \\[0pt]
La grammaire contraint-elle la pensée ? \\[0pt]
La grammaire contraint-elle notre pensée ? \\[0pt]
La grammaire véhicule-t-elle une métaphysique ? \\[0pt]
La guerre est-elle la continuation de la politique ? \\[0pt]
La guerre est-elle la continuation de la politique par d'autres moyens ? \\[0pt]
La guerre est-elle la plus grande violence ? \\[0pt]
La guerre est-elle la politique continuée par d'autres moyens ? \\[0pt]
La guerre est-elle l'essentiel de toute politique ? \\[0pt]
La guerre met-elle fin au droit ? \\[0pt]
La guerre peut-elle être juste ? \\[0pt]
La guerre peut-elle être justifiée ? \\[0pt]
La guerre peut-elle être un droit ? \\[0pt]
Laisser mourir, est-ce tuer ? \\[0pt]
La justice a-t-elle besoin des institutions ? \\[0pt]
La justice a-t-elle pour fonction de compenser les inégalités naturelles ? \\[0pt]
La justice a-t-elle un fondement rationnel ? \\[0pt]
La justice consiste-t-elle à traiter tout le monde de la même manière ? \\[0pt]
La justice consiste-t-elle dans l'application de la loi ? \\[0pt]
La justice est-elle affaire de raison ? \\[0pt]
La justice est-elle de ce monde ? \\[0pt]
La justice est-elle de l'ordre du sentiment ? \\[0pt]
La justice est-elle l'affaire de l'État ? \\[0pt]
La justice est-elle le ciment de la paix sociale ? \\[0pt]
La justice est-elle une idée ? \\[0pt]
La justice est-elle une notion morale ? \\[0pt]
La justice est-elle une vertu ? \\[0pt]
La justice est-elle un idéal rationnel ? \\[0pt]
La justice exclut-elle l'arbitraire ? \\[0pt]
La justice : moyen ou fin de la politique ? \\[0pt]
La justice n'est-elle qu'une institution ? \\[0pt]
La justice n'est-elle qu'un idéal ? \\[0pt]
La justice peut-elle être fondée en nature ? \\[0pt]
La justice peut-elle se fonder sur le compromis ? \\[0pt]
La justice peut-elle se passer de la force ? \\[0pt]
La justice peut-elle se passer d'institutions ? \\[0pt]
La justice requiert-elle l'infaillibilité des juges ? \\[0pt]
La justice requiert-elle toujours l'impartialité ? \\[0pt]
La justice suppose-t-elle l'égalité ? \\[0pt]
La laideur est-elle une valeur esthétique ? \\[0pt]
La langue nous parle-t-elle ? \\[0pt]
La légitimité d'un clonage humain ? \\[0pt]
La liberté admet-elle des degrés ? \\[0pt]
La liberté a-t-elle une histoire ? \\[0pt]
La liberté a-t-elle un prix ? \\[0pt]
La liberté comporte-t-elle des degrés ? \\[0pt]
La liberté, concept moral ? \\[0pt]
La liberté connaît-elle des excès ? \\[0pt]
La liberté des uns s'arrête-elle où commence celle des autres ? \\[0pt]
La liberté d'expression a-t-elle des limites ? \\[0pt]
La liberté d'expression est-elle nécessaire à la liberté de pensée ? \\[0pt]
La liberté d'expression est-elle nécessaire à la liberté de penser ? \\[0pt]
La liberté doit-elle être limitée ? \\[0pt]
La liberté doit-elle se conquérir ? \\[0pt]
La liberté, est-ce l'indépendance à l'égard des passions ? \\[0pt]
La liberté est-elle ce qui définit l'homme ? \\[0pt]
La liberté est-elle contraire au principe de causalité ? \\[0pt]
La liberté est-elle innée ? \\[0pt]
La liberté est-elle la condition du bonheur ? \\[0pt]
La liberté est-elle le fondement de la responsabilité ? \\[0pt]
La liberté est-elle le pouvoir de refuser ? \\[0pt]
La liberté est-elle menacée par l'égalité ? \\[0pt]
La liberté est-elle une illusion ? \\[0pt]
La liberté est-elle une illusion nécessaire ? \\[0pt]
La liberté est-elle un fait ? \\[0pt]
La liberté et l'égalité sont-elles compatibles ? \\[0pt]
La liberté fait-elle de nous des êtres meilleurs ? \\[0pt]
La liberté implique-t-elle l'indifférence ? \\[0pt]
La liberté impose-t-elle des devoirs ? \\[0pt]
La liberté intéresse-t-elle les sciences humaines ? \\[0pt]
La liberté ne s'éprouve-t-elle que dans la solitude ? \\[0pt]
La liberté n'est-elle qu'un droit ? \\[0pt]
La liberté n'est-elle qu'une illusion ? \\[0pt]
La liberté nous rend-elle inexcusables ? \\[0pt]
La liberté peut-elle être prouvée ? \\[0pt]
La liberté peut-elle être une illusion ? \\[0pt]
La liberté peut-elle faire peur ? \\[0pt]
La liberté peut-elle s'affirmer sans violence ? \\[0pt]
La liberté peut-elle s'aliéner ? \\[0pt]
La liberté peut-elle se constater ? \\[0pt]
La liberté peut-elle se prouver ? \\[0pt]
La liberté peut-elle se refuser ? \\[0pt]
La liberté requiert-elle le libre échange ? \\[0pt]
La liberté s'achète-t-elle ? \\[0pt]
La liberté s'apprend-elle ? \\[0pt]
La liberté se mérite-t-elle ? \\[0pt]
La liberté se prouve-t-elle ? \\[0pt]
La liberté s'éprouve-t-elle ? \\[0pt]
La liberté se prouve-t-elle ou s'éprouve-t-elle ? \\[0pt]
La liberté se réduit-elle au libre arbitre ? \\[0pt]
La liberté suppose-t-elle l'absence de déterminisme ? \\[0pt]
La linguistique est-elle une science de l'information ? \\[0pt]
La linguistique fait-elle partie des sciences humaines ? \\[0pt]
La linguistique peut-elle se passer de la psychologie ? \\[0pt]
La littérature est-elle la mémoire de l'humanité ? \\[0pt]
La littérature peut-elle suppléer les sciences de l'homme ? \\[0pt]
La logique a-t-elle une histoire ? \\[0pt]
La logique a-t-elle un intérêt philosophique ? \\[0pt]
La logique : calcul ou langage ? \\[0pt]
La logique : découverte ou invention ? \\[0pt]
La logique décrit-elle le monde ? \\[0pt]
La logique est-elle indépendante de la psychologie ? \\[0pt]
La logique est-elle la norme du vrai ? \\[0pt]
La logique est-elle l'art de penser ? \\[0pt]
La logique est-elle un art de penser ? \\[0pt]
La logique est-elle un art de raisonner ? \\[0pt]
La logique est-elle une discipline normative ? \\[0pt]
La logique est-elle une forme de calcul ? \\[0pt]
La logique est-elle une science ? \\[0pt]
La logique est-elle une science de la vérité ? \\[0pt]
La logique est-elle utile à la métaphysique ? \\[0pt]
La logique nous apprend-elle quelque chose sur le langage ordinaire ? \\[0pt]
« La logique » ou bien « les logiques » ? \\[0pt]
La logique peut-elle se passer de la métaphysique ? \\[0pt]
La logique pourrait-elle nous surprendre ? \\[0pt]
La loi dit-elle ce qui est juste ? \\[0pt]
La loi doit-elle être la même pour tous ? \\[0pt]
La loi éduque-t-elle ? \\[0pt]
La loi est-elle une garantie contre l'injustice ? \\[0pt]
La loi peut-elle changer les mœurs ? \\[0pt]
La loi peut-elle être injuste ? \\[0pt]
L'altruisme est-il une attitude morale ? \\[0pt]
L'altruisme n'est-il qu'un égoïsme bien compris ? \\[0pt]
La machine fournit-elle un modèle pour penser le vivant ? \\[0pt]
La magie peut-elle être efficace ? \\[0pt]
La majorité a-t-elle toujours raison ? \\[0pt]
La majorité doit-elle toujours l'emporter ? \\[0pt]
La majorité, force ou droit ? \\[0pt]
La majorité peut-elle être tyrannique ? \\[0pt]
La maladie est-elle à l'organisme vivant ce que la panne est à la machine ? \\[0pt]
La maladie est-elle indispensable à la connaissance du vivant ? \\[0pt]
La maladie mentale a-t-elle une histoire ? \\[0pt]
La mathématique est-elle une ontologie ? \\[0pt]
La matière, est-ce le mal ? \\[0pt]
La matière, est-ce l'informe ? \\[0pt]
La matière est-elle amorphe ? \\[0pt]
La matière est-elle connaissable ? \\[0pt]
La matière est-elle plus aisée à connaître que l'esprit ? \\[0pt]
La matière est-elle plus facile à connaître que l'esprit ? \\[0pt]
La matière est-elle une substance ? \\[0pt]
La matière est-elle une vue de l'esprit ? \\[0pt]
La matière et l'esprit peuvent-ils constituer une seule chose ? \\[0pt]
La matière n'est-elle que ce que l'on perçoit ? \\[0pt]
La matière n'est-elle qu'une idée ? \\[0pt]
La matière n'est-elle qu'un obstacle ? \\[0pt]
La matière pense-t-elle ? \\[0pt]
La matière peut-elle agir ? \\[0pt]
La matière peut-elle être objet de connaissance ? \\[0pt]
La matière peut-elle penser ? \\[0pt]
L'ambiguïté des mots peut-elle être heureuse ? \\[0pt]
L'âme concerne-t-elle les sciences humaines ? \\[0pt]
La médecine est-elle une science ? \\[0pt]
L'âme est-elle immortelle ? \\[0pt]
L'âme et le corps sont-ils une seule et même chose ? \\[0pt]
L'âme jouit-elle d'une vie propre ? \\[0pt]
L'amélioration des hommes peut-elle être considérée comme un objectif politique ? \\[0pt]
La mémoire est-elle une fonction vitale ? \\[0pt]
La métaphore n'est-elle qu'un ornement du discours ? \\[0pt]
La métaphysique a-t-elle ses fictions ? \\[0pt]
La métaphysique est-elle affaire de raisonnement ? \\[0pt]
La métaphysique est-elle le fondement de la morale ? \\[0pt]
La métaphysique est-elle nécessairement une réflexion sur Dieu ? \\[0pt]
La métaphysique est-elle un discours sans objet ? \\[0pt]
La métaphysique est-elle une discipline théorique ? \\[0pt]
La métaphysique est-elle une science ? \\[0pt]
La métaphysique peut-elle être autre chose qu'une science recherchée ? \\[0pt]
La métaphysique peut-elle faire appel à l'expérience ? \\[0pt]
La métaphysique procure-t-elle un savoir ? \\[0pt]
La métaphysique relève-t-elle de la philosophie ou de la poésie ? \\[0pt]
La métaphysique répond-elle à un besoin ? \\[0pt]
La métaphysique repose-t-elle sur des croyances ? \\[0pt]
La métaphysique se définit-elle par son objet ou sa démarche ? \\[0pt]
La méthode est-elle nécessaire pour la recherche de la vérité ? \\[0pt]
La méthode expérimentale est-elle appropriée à l'étude du vivant ? \\[0pt]
L'amitié est-elle une vertu ? \\[0pt]
L'amitié est-elle un principe politique ? \\[0pt]
L'amitié peut-elle obliger ? \\[0pt]
L'amitié relève-t-elle d'une décision ? \\[0pt]
La modération est-elle l'essence de la vertu ? \\[0pt]
La modération est-elle une vertu politique ? \\[0pt]
La monnaie n'est-elle qu'un moyen d'échange ? \\[0pt]
La morale a-t-elle à décider de la sexualité ? \\[0pt]
La morale a-t-elle besoin de la notion de sainteté ? \\[0pt]
La morale a-t-elle besoin d'être fondée ? \\[0pt]
La morale a-t-elle besoin d'un au-delà ? \\[0pt]
La morale a-t-elle besoin d'un fondement ? \\[0pt]
La morale a-t-elle sa place dans l'économie ? \\[0pt]
La morale consiste-t-elle à respecter le droit ? \\[0pt]
La morale consiste-t-elle à suivre la nature ? \\[0pt]
La morale dépend-elle de la culture ? \\[0pt]
La morale doit-elle en appeler à la nature ? \\[0pt]
La morale doit-elle être rationnelle ? \\[0pt]
La morale doit-elle fournir des préceptes ? \\[0pt]
La morale est-elle affaire de convention ? \\[0pt]
La morale est-elle affaire de jugement ? \\[0pt]
La morale est-elle affaire de sentiment ? \\[0pt]
La morale est-elle affaire de sentiments ? \\[0pt]
La morale est-elle affaire individuelle ? \\[0pt]
La morale est-elle condamnée à n'être qu'un champ de bataille ? \\[0pt]
La morale est-elle désintéressée ? \\[0pt]
La morale est-elle émotion ? \\[0pt]
La morale est-elle en conflit avec le désir ? \\[0pt]
La morale est-elle ennemie du bonheur ? \\[0pt]
La morale est-elle fondée sur la liberté ? \\[0pt]
La morale est-elle incompatible avec le déterminisme ? \\[0pt]
La morale est-elle l'ennemie de la vie ? \\[0pt]
La morale est-elle nécessairement répressive ? \\[0pt]
La morale est-elle objet de science ? \\[0pt]
La morale est-elle un art de vivre ? \\[0pt]
La morale est-elle un besoin ? \\[0pt]
La morale est-elle une affaire de raison ? \\[0pt]
La morale est-elle une affaire d'habitude ? \\[0pt]
La morale est-elle une affaire solitaire ? \\[0pt]
La morale est-elle une médecine de l'âme ? \\[0pt]
La morale est-elle une preuve de la liberté ? \\[0pt]
La morale est-elle une science ? \\[0pt]
La morale est-elle un fait de culture ? \\[0pt]
La morale est-elle un fait social ? \\[0pt]
La morale et la religion visent-elles les mêmes fins ? \\[0pt]
La morale n'est-elle qu'un art de vivre ? \\[0pt]
La morale n'est-elle qu'un dressage ? \\[0pt]
La morale n'est-elle qu'un ensemble de conventions ? \\[0pt]
La morale peut-elle être fondée sur la science ? \\[0pt]
La morale peut-elle être naturelle ? \\[0pt]
La morale peut-elle être personnelle ? \\[0pt]
La morale peut-elle être un calcul ? \\[0pt]
La morale peut-elle être une science ? \\[0pt]
La morale peut-elle se définir comme l'art d'être heureux ? \\[0pt]
La morale peut-elle se fonder sur les sentiments ? \\[0pt]
La morale peut-elle s'enseigner ? \\[0pt]
La morale peut-elle se passer d'un fondement religieux ? \\[0pt]
La morale requiert-elle une casuistique ? \\[0pt]
La morale requiert-elle un fondement ? \\[0pt]
La morale s'apprend-elle ? \\[0pt]
La morale se déduit-elle de la métaphysique ? \\[0pt]
La morale s'enracine-t-elle dans le sens commun ? \\[0pt]
La morale s'enseigne-t-elle ? \\[0pt]
La morale s'oppose-t-elle à la politique ? \\[0pt]
La morale suppose-t-elle le libre arbitre ? \\[0pt]
La moralité consiste-t-elle à se contraindre soi-même ? \\[0pt]
La moralité est-elle affaire de principes ou de conséquences ? \\[0pt]
La moralité est-elle utile à la vie sociale ? \\[0pt]
La moralité n'est-elle que dressage ? \\[0pt]
La moralité réside-t-elle dans l'intention ? \\[0pt]
La moralité se juge-t-elle aux actes ? \\[0pt]
La moralité se réduit-elle aux sentiments ? \\[0pt]
La mort a-t-elle un sens ? \\[0pt]
La mort fait-elle partie de la vie ? \\[0pt]
L'amour a-t-il des raisons ? \\[0pt]
L'amour de soi est-il immoral ? \\[0pt]
L'amour du pouvoir est-il compatible avec le dévouement à la chose publique ? \\[0pt]
L'amour est-il aveugle ? \\[0pt]
L'amour est-il désir ? \\[0pt]
L'amour est-il sans raison ? \\[0pt]
L'amour est-il toujours une passion ? \\[0pt]
L'amour est-il une vertu ? \\[0pt]
L'amour implique-t-il le respect ? \\[0pt]
L'amour peut-il être absolu ? \\[0pt]
L'amour peut-il être raisonnable ? \\[0pt]
L'amour peut-il être un devoir ? \\[0pt]
La musique a-t-elle une essence ? \\[0pt]
La musique a-t-elle un sens ? \\[0pt]
La musique donne-t-elle à penser ? \\[0pt]
La musique est-elle un discours ? \\[0pt]
La musique est-elle un langage ? \\[0pt]
La musique exprime-t-elle quelque chose ? \\[0pt]
La musique peut-elle être narrative ? \\[0pt]
La naïveté est-elle une vertu ? \\[0pt]
L'analyse des comportements humains permet-elle de réduire le hasard ? \\[0pt]
L'analyse du langage ordinaire peut-elle avoir un intérêt philosophique ? \\[0pt]
L'analyse économique rend-elle compte des comportements humains ? \\[0pt]
La nation est-elle dépassée ? \\[0pt]
La nature a-t-elle des droits ? \\[0pt]
La nature a-t-elle une histoire ? \\[0pt]
La nature a-t-elle une valeur ? \\[0pt]
La nature a-t-elle un langage ? \\[0pt]
La nature a-t-elle un ordre ? \\[0pt]
La nature donne-t-elle un sens à notre existence ? \\[0pt]
La nature, est-ce le donné ? \\[0pt]
La nature est-elle artiste ? \\[0pt]
La nature est-elle belle ? \\[0pt]
La nature est-elle bien faite ? \\[0pt]
La nature est-elle digne de respect ? \\[0pt]
La nature est-elle écrite en langage mathématique ? \\[0pt]
La nature est-elle muette ? \\[0pt]
La nature est-elle politique ? \\[0pt]
La nature est-elle prévisible ? \\[0pt]
La nature est-elle sacrée ? \\[0pt]
La nature est-elle sans histoire ? \\[0pt]
La nature est-elle sauvage ? \\[0pt]
La nature est-elle une grande artiste ? \\[0pt]
La nature est-elle une idée ? \\[0pt]
La nature est-elle une invention de l'homme ? \\[0pt]
La nature est-elle une norme ? \\[0pt]
La nature est-elle une ressource ? \\[0pt]
La nature est-elle un guide ? \\[0pt]
La nature est-elle un modèle ? \\[0pt]
La nature est-elle un système ? \\[0pt]
La nature existe-t-elle ? \\[0pt]
La nature fait-elle bien les choses ? \\[0pt]
La nature imite-t-elle l'art ? \\[0pt]
La nature ne fait-elle rien en vain ? \\[0pt]
La nature nous dicte-t-elle des fins ? \\[0pt]
La nature nous domine-t-elle ? \\[0pt]
La nature nous donne-t-elle des commandements ? \\[0pt]
La nature nous indique-t-elle ce qui est bon ? \\[0pt]
La nature obéit-elle à des fins ? \\[0pt]
La nature parle-t-elle le langage des mathématiques ? \\[0pt]
La nature peut-elle avoir des droits ? \\[0pt]
La nature peut-elle constituer une norme ? \\[0pt]
La nature peut-elle être belle ? \\[0pt]
La nature peut-elle être détruite ? \\[0pt]
La nature peut-elle être un modèle ? \\[0pt]
La nature peut-elle nous indiquer ce que nous devons faire ? \\[0pt]
La nature reprend-elle toujours ses droits ? \\[0pt]
La nature se donne-t-elle à penser ? \\[0pt]
La nature s'oppose-t-elle à l'esprit ? \\[0pt]
La nécessité fait-elle loi ? \\[0pt]
La négligence est-elle une faute ? \\[0pt]
La neige est-elle blanche ? \\[0pt]
L'animal apprend-il à l'homme quelque chose sur l'homme ? \\[0pt]
L'animal a-t-il des droits ? \\[0pt]
L'animal est-il une personne ? \\[0pt]
L'animal nous apprend-il quelque chose sur l'homme ? \\[0pt]
L'animal peut-il être un sujet moral ? \\[0pt]
La non-contradiction : principe logique ou ontologique ? \\[0pt]
La notion de barbarie a-t-elle un sens ? \\[0pt]
La notion de caractère relève-t- elle de la morale ou de la psychologie ? \\[0pt]
La notion de démocratie représentative est-elle contradictoire ? \\[0pt]
La notion de droit international est-elle contradictoire ? \\[0pt]
La notion de finalité a-t-elle de l'intérêt pour le savant ? \\[0pt]
La notion de loi a-t-elle une unité ? \\[0pt]
La notion de nature humaine a-t-elle un sens ? \\[0pt]
La notion de paradis a-t-elle un sens exclusivement religieux ? \\[0pt]
La notion de progrès a-t-elle un sens en politique ? \\[0pt]
La notion de progrès moral a-t-elle encore un sens ? \\[0pt]
La notion de race a-t-elle un statut scientifique ? \\[0pt]
La notion de responsabilité est-elle suspecte ? \\[0pt]
La notion de signification est-elle indispensable aux sciences humaines ? \\[0pt]
La notion d'histoire commune a-t-elle un sens ? \\[0pt]
L'anthropologie est-elle une ontologie ? \\[0pt]
La ou les vertus ? \\[0pt]
La paix est-elle l'absence de guerre ? \\[0pt]
La paix est-elle l'absence de guerres ? \\[0pt]
La paix est-elle la visée de la politique ? \\[0pt]
La paix est-elle le plus grand des biens ? \\[0pt]
La paix est-elle moins naturelle que la guerre ? \\[0pt]
La paix est-elle possible ? \\[0pt]
La paix est-elle toujours préférable ? \\[0pt]
La paix met-elle fin à l'histoire ? \\[0pt]
La paix n'est-elle que l'absence de conflit ? \\[0pt]
La paix n'est-elle que l'absence de guerre ? \\[0pt]
La paix n'est-elle qu'un idéal ? \\[0pt]
La paix sociale est-elle la finalité de la politique ? \\[0pt]
La paix sociale est-elle le but de la politique ? \\[0pt]
La paix sociale est-elle une fin en soi ? \\[0pt]
La parole nous libère-t-elle ? \\[0pt]
La parole peut-elle agir ? \\[0pt]
La parole peut-elle être étudiée comme un acte social ? \\[0pt]
La parole peut-elle être une arme ? \\[0pt]
La passion de la vérité peut-elle être source d'erreur ? \\[0pt]
La passion est-elle immorale ? \\[0pt]
La passion est-elle l'ennemi de la raison ? \\[0pt]
La passion exclut-elle la lucidité ? \\[0pt]
La passion n'est-elle que souffrance ? \\[0pt]
La patience est-elle une vertu ? \\[0pt]
La pauvreté est-elle une injustice ? \\[0pt]
La pédagogie, philosophie pratique ou psychologie appliquée ? \\[0pt]
La peine de mort est-elle juste, injuste, et pourquoi ? \\[0pt]
La peinture apprend-elle à voir ? \\[0pt]
La peinture est-elle une poésie muette ? \\[0pt]
La peinture peut-elle être un art du temps ? \\[0pt]
La pensée a-t-elle besoin d'images ? \\[0pt]
La pensée a-t-elle des lois ? \\[0pt]
La pensée a-t-elle une histoire ? \\[0pt]
La pensée de la mort a-t-elle un objet ? \\[0pt]
La pensée doit-elle se soumettre aux règles de la logique ? \\[0pt]
La pensée échappe-t-elle à la grammaire ? \\[0pt]
La pensée est-elle en lutte avec le langage ? \\[0pt]
La pensée est-elle une activité assimilable à un travail ? \\[0pt]
La pensée et la conscience sont-elles une seule et même chose ? \\[0pt]
La pensée formelle est-elle privée d'objet ? \\[0pt]
La pensée formelle est-elle une pensée vide ? \\[0pt]
La pensée formelle peut-elle avoir un contenu ? \\[0pt]
La pensée obéit-elle à des lois ? \\[0pt]
La pensée peut-elle devenir une technique ? \\[0pt]
La pensée peut-elle s'écrire ? \\[0pt]
La pensée peut-elle se passer de mots ? \\[0pt]
La pensée philosophique doit-elle être une pensée systématique ? \\[0pt]
La perception construit-elle son objet ? \\[0pt]
La perception de la folie change-t-elle de nature avec les sciences humaines ? \\[0pt]
La perception de l'espace est-elle innée ou acquise ? \\[0pt]
La perception est-elle le premier degré de la connaissance ? \\[0pt]
La perception est-elle l'interprétation du réel ? \\[0pt]
La perception est-elle source de connaissance ? \\[0pt]
La perception est-elle une interprétation ? \\[0pt]
La perception est-elle un mode de connaissance ? \\[0pt]
La perception me donne-t-elle le réel ? \\[0pt]
La perception peut-elle être désintéressée ? \\[0pt]
La perception peut-elle s'éduquer ? \\[0pt]
La perfection est-elle désirable ? \\[0pt]
La peur est-elle mauvaise conseillère ? \\[0pt]
La philosophie a-t-elle une histoire ? \\[0pt]
La philosophie a-t-elle un objet qui lui soit propre ? \\[0pt]
La philosophie doit-elle être systématique ? \\[0pt]
La philosophie doit-elle être une science ? \\[0pt]
La philosophie doit-elle se préoccuper du salut ? \\[0pt]
La philosophie est-elle abstraite ? \\[0pt]
La philosophie est-elle la servante de la science ? \\[0pt]
La philosophie est-elle une méditation de la mort ? \\[0pt]
La philosophie est-elle une méditation de la vie ou de la mort ? \\[0pt]
La philosophie est-elle une science ? \\[0pt]
La philosophie peut-elle disparaître ? \\[0pt]
La philosophie peut-elle être expérimentale ? \\[0pt]
La philosophie peut-elle être populaire ? \\[0pt]
La philosophie peut-elle être une science ? \\[0pt]
La philosophie peut-elle ne pas parler de Dieu ? \\[0pt]
La philosophie peut-elle se passer de l'idée de vérité ? \\[0pt]
La philosophie peut-elle se passer de théologie ? \\[0pt]
La philosophie rend-elle inefficace la propagande ? \\[0pt]
La photographie est-elle un art ? \\[0pt]
La physique est-elle le modèle de toutes les sciences ? \\[0pt]
La pitié a-t-elle une valeur ? \\[0pt]
La pitié est-elle dangereuse ? \\[0pt]
La pitié est-elle morale ? \\[0pt]
La pitié est-elle un sentiment moral ? \\[0pt]
La pitié fonde-t-elle la morale ? \\[0pt]
La pitié peut-elle fonder la morale ? \\[0pt]
La place de l'art est-elle sur le marché de l'art ? \\[0pt]
La pluralité des vérités condamne-t-elle l'idée de vérité ? \\[0pt]
La poésie dit-elle le réel ? \\[0pt]
La poésie est-elle comme une peinture ? \\[0pt]
La poésie pense-t-elle ? \\[0pt]
La politesse est-elle une vertu ? \\[0pt]
La politique : affaire de compétence ? \\[0pt]
La politique a-t-elle besoin de héros ? \\[0pt]
La politique a-t-elle besoin de la fiction ? \\[0pt]
La politique a-t-elle besoin de la religion ? \\[0pt]
La politique a-t-elle besoin de modèles ? \\[0pt]
La politique a-t-elle besoin d'experts ? \\[0pt]
La politique a-t-elle pour but de nous faire vivre dans un monde meilleur ? \\[0pt]
La politique a-t-elle pour fin d'éliminer la violence ? \\[0pt]
La politique consiste-t-elle à faire cause commune ? \\[0pt]
La politique consiste-t-elle à faire des compromis ? \\[0pt]
La politique consiste-t-elle à gérer l'urgence ? \\[0pt]
La politique doit-elle avoir pour visée le bonheur ? \\[0pt]
La politique doit-elle être morale ? \\[0pt]
La politique doit-elle être rationnelle ? \\[0pt]
La politique doit-elle protéger la liberté des citoyens ? \\[0pt]
La politique doit-elle refuser l'utopie ? \\[0pt]
La politique doit-elle se mêler de l'art ? \\[0pt]
La politique doit-elle se mêler du bonheur ? \\[0pt]
La politique doit-elle viser la concorde ? \\[0pt]
La politique doit-elle viser le consensus ? \\[0pt]
La politique échappe-t-elle à l'exigence de vérité ? \\[0pt]
La politique est-elle affaire de compétence ? \\[0pt]
La politique est-elle affaire de décision ? \\[0pt]
La politique est-elle affaire de jugement ? \\[0pt]
La politique est-elle affaire de science ? \\[0pt]
La politique est-elle affaire d'expérience ou de théorie ? \\[0pt]
La politique est-elle architectonique ? \\[0pt]
La politique est-elle continuation de la guerre, par d'autres moyens ? \\[0pt]
La politique est-elle extérieure au droit ? \\[0pt]
La politique est-elle la continuation de la guerre ? \\[0pt]
La politique est-elle la continuation de la guerre par d'autres moyens ? \\[0pt]
La politique est-elle l'affaire des spécialistes ? \\[0pt]
La politique est-elle l'affaire de tous ? \\[0pt]
La politique est-elle l'art de convaincre le peuple ? \\[0pt]
La politique est-elle l'art de la délibération ? \\[0pt]
La politique est-elle l'art des possibles ? \\[0pt]
La politique est-elle l'art du possible ? \\[0pt]
La politique est-elle le refus de la violence ? \\[0pt]
La politique est-elle le règne des passions ? \\[0pt]
La politique est-elle naturelle ? \\[0pt]
La politique est-elle par nature sujette à dispute ? \\[0pt]
La politique est-elle plus importante que tout ? \\[0pt]
La politique est-elle un art ? \\[0pt]
La politique est-elle une affaire d'experts ? \\[0pt]
La politique est-elle une science ? \\[0pt]
La politique est-elle une technique ? \\[0pt]
La politique est-elle un métier ? \\[0pt]
La politique est-elle un moyen ou une fin ? \\[0pt]
La politique exclut-elle le désordre ? \\[0pt]
La politique implique-t-elle la hiérarchie ? \\[0pt]
La politique n'est-elle que l'art de conquérir et de conserver le pouvoir ? \\[0pt]
La politique peut-elle changer la société ? \\[0pt]
La politique peut-elle changer le monde ? \\[0pt]
La politique peut-elle disparaître ? \\[0pt]
La politique peut-elle échapper au mythe ? \\[0pt]
La politique peut-elle être indépendante de la morale ? \\[0pt]
La politique peut-elle être morale ? \\[0pt]
La politique peut-elle être objet de science ? \\[0pt]
La politique peut-elle être un objet de science ? \\[0pt]
La politique peut-elle n'être qu'une pratique ? \\[0pt]
La politique peut-elle se passer de croyance ? \\[0pt]
La politique peut-elle se passer de croyances ? \\[0pt]
La politique peut-elle se passer des idéologies ? \\[0pt]
La politique peut-elle unir les hommes ? \\[0pt]
La politique peut-elle viser à autre chose qu'à l'efficacité ? \\[0pt]
La politique repose-t-elle sur un contrat ? \\[0pt]
La politique se réduit-elle à des rapports de force ? \\[0pt]
La politique suppose-t-elle la morale ? \\[0pt]
La politique suppose-t-elle une idée de l'homme ? \\[0pt]
La politique vise-t-elle à dépasser les injustices ? \\[0pt]
La poursuite de mon intérêt m'oppose-t-elle aux autres ? \\[0pt]
L'apparence est-elle toujours trompeuse ? \\[0pt]
L'apparence fait-elle obstacle à la vérité ? \\[0pt]
L'apparence fait-elle partie de la réalité ? \\[0pt]
L'approche psychologique de l'homme met-elle en question la notion d'âme ? \\[0pt]
La pratique des sciences met-elle à l'abri des préjugés ? \\[0pt]
La précaution peut-elle être un principe ? \\[0pt]
La présence d'autrui nous évite-t-elle la solitude ? \\[0pt]
L'a priori peut-il être historique ? \\[0pt]
La prise de parti est-elle essentielle en politique ? \\[0pt]
La prison est-elle utile ? \\[0pt]
La promesse n'est-elle qu'un acte de langage ? \\[0pt]
La propriété, est-ce un vol ? \\[0pt]
La propriété est-elle un droit ? \\[0pt]
La propriété est-elle une garantie de liberté ? \\[0pt]
La prudence est-elle une vertu politique ? \\[0pt]
La psychanalyse donne-t-elle un sens au rêve ? \\[0pt]
La psychanalyse est-elle une science ? \\[0pt]
La psychanalyse est-elle une science humaine ? \\[0pt]
La psychanalyse permet-elle la construction du sujet ? \\[0pt]
La psychanalyse s'applique-t-elle aux phénomènes sociaux ? \\[0pt]
La psychologie dit-elle ce qu'est le moi ? \\[0pt]
La psychologie est-elle l'étude de ce qui n'est pas social ? \\[0pt]
La psychologie est-elle l'étude des comportements individuels ? \\[0pt]
La psychologie est-elle une science ? \\[0pt]
La psychologie est-elle une science de la nature ? \\[0pt]
La psychologie est-elle une science humaine ? \\[0pt]
La puissance technique est- elle sans limites ? \\[0pt]
La question de la nature humaine relève-t-elle des sciences humaines ? \\[0pt]
La question du sens de la vie peut-elle être sans réponses ? \\[0pt]
La question « qu'est-ce que l'homme ? » est-elle scientifique ? \\[0pt]
La question : « qui ? » \\[0pt]
La question « qui suis-je » admet-elle une réponse exacte ? \\[0pt]
La radicalité est-elle une exigence philosophique ? \\[0pt]
La raison a-t-elle des limites ? \\[0pt]
La raison a-t-elle le droit d'expliquer ce que morale condamne ? \\[0pt]
La raison a-t-elle pour fin la connaissance ? \\[0pt]
La raison a-t-elle toujours raison ? \\[0pt]
La raison a-t-elle une histoire ? \\[0pt]
La raison d'État peut-elle être justifiée ? \\[0pt]
La raison doit-elle critiquer la croyance ? \\[0pt]
La raison doit-elle être cultivée ? \\[0pt]
La raison doit-elle être notre guide ? \\[0pt]
La raison doit-elle faire une place à la croyance ? \\[0pt]
La raison doit-elle se soumettre au réel ? \\[0pt]
La raison engendre-t-elle des illusions ? \\[0pt]
La raison épuise-t-elle le réel ? \\[0pt]
La raison, est-ce la modération ? \\[0pt]
La raison est-elle impersonnelle ? \\[0pt]
La raison est-elle le pouvoir de distinguer le vrai du faux ? \\[0pt]
La raison est-elle l'esclave des passions ? \\[0pt]
La raison est-elle l'esclave du désir ? \\[0pt]
La raison est-elle morale par elle-même ? \\[0pt]
La raison est-elle plus fiable que l'expérience ? \\[0pt]
La raison est-elle seulement affaire de logique ? \\[0pt]
La raison est-elle suffisante ? \\[0pt]
La raison est-elle toujours raisonnable ? \\[0pt]
La raison est-elle une valeur ? \\[0pt]
La raison est-elle un instrument ? \\[0pt]
La raison est-elle un obstacle au bonheur ? \\[0pt]
La raison gouverne-t-elle le monde ? \\[0pt]
La raison ne connaît-elle du réel que ce qu'elle y met d'elle-même ? \\[0pt]
La raison ne veut-elle que connaître ? \\[0pt]
La raison peut-elle entrer en conflit avec elle-même ? \\[0pt]
La raison peut-elle entrer en contradiction avec elle-même ? \\[0pt]
La raison peut-elle errer ? \\[0pt]
La raison peut-elle être immédiatement pratique ? \\[0pt]
La raison peut-elle être pratique ? \\[0pt]
La raison peut-elle nous commander de croire ? \\[0pt]
La raison peut-elle nous égarer ? \\[0pt]
La raison peut-elle nous induire en erreur ? \\[0pt]
La raison peut-elle rendre compte du réel entièrement ? \\[0pt]
La raison peut-elle rendre raison de tout ? \\[0pt]
La raison peut-elle s'aveugler elle-même ? \\[0pt]
La raison peut-elle se contredire ? \\[0pt]
La raison peut-elle servir le mal ? \\[0pt]
La raison peut-elle s'opposer à elle-même ? \\[0pt]
La raison s'identifie-t-elle au calcul ? \\[0pt]
La raison s'oppose-t-elle aux passions ? \\[0pt]
La raison transforme-t-elle le réel ? \\[0pt]
La rationalité se confond-elle avec la systématicité ? \\[0pt]
L'architecte : artiste ou ingénieur ? \\[0pt]
L'architecture est-elle un art ? \\[0pt]
La réalisation du devoir exclut-elle toute forme de plaisir ? \\[0pt]
La réalité a-t-elle une forme logique ? \\[0pt]
La réalité décrite par la science s'oppose-t-elle à la démonstration ? \\[0pt]
La réalité de la vie s'épuise-t-elle dans celle des vivants ? \\[0pt]
La réalité du temps se réduit-elle à la conscience que nous en avons ? \\[0pt]
La réalité est-elle une idée ? \\[0pt]
La réalité n'est-elle qu'une construction ? \\[0pt]
La réalité nourrit-elle la fiction ? \\[0pt]
La réalité peut-elle être virtuelle ? \\[0pt]
La recherche de la vérité a-t- elle une fin ? \\[0pt]
La recherche de la vérité peut-elle être désintéressée ? \\[0pt]
La recherche de la vérité peut-elle être une passion ? \\[0pt]
La recherche du bonheur est-elle égoïste ? \\[0pt]
La recherche du bonheur est-elle un idéal égoïste ? \\[0pt]
La recherche du bonheur est-il un guide suffisant dans la vie ? \\[0pt]
La recherche du bonheur peut-elle être un devoir ? \\[0pt]
La recherche du bonheur suffit-elle à déterminer une morale ? \\[0pt]
La recherche scientifique est-elle désintéressée ? \\[0pt]
La réciprocité est-elle indispensable à la communauté politique ? \\[0pt]
La reconnaissance de la diversité des cultures condamne-t-elle au relativisme ? \\[0pt]
La référence aux faits suffit-elle à garantir l'objectivité de la connaissance ? \\[0pt]
La réflexion sur l'expérience participe-t-elle de l'expérience ? \\[0pt]
La relation à autrui est-elle symétrique ? \\[0pt]
La relation de causalité est-elle temporelle ? \\[0pt]
La religion a-t-elle besoin d'un dieu ? \\[0pt]
La religion a-t-elle des vertus ? \\[0pt]
La religion a-t-elle les mêmes fins que la morale ? \\[0pt]
La religion a-t-elle une fonction de cohésion sociale ? \\[0pt]
La religion a-t-elle une fonction sociale ? \\[0pt]
La religion conduit-elle l'homme au-delà de lui-même ? \\[0pt]
La religion divise-t-elle les hommes ? \\[0pt]
La religion est-elle à craindre ? \\[0pt]
La religion est-elle au fondement de la morale ? \\[0pt]
La religion est-elle contraire à la raison ? \\[0pt]
La religion est-elle fondée sur la peur de la mort ? \\[0pt]
La religion est-elle la sagesse des pauvres ? \\[0pt]
La religion est-elle l'asile de l'ignorance ? \\[0pt]
La religion est-elle l'opium du peuple ? \\[0pt]
La religion est-elle relation à l'absolu ? \\[0pt]
La religion est-elle simple affaire de croyance ? \\[0pt]
La religion est-elle source de conflit ? \\[0pt]
La religion est-elle une affaire privée ? \\[0pt]
La religion est-elle une consolation pour les hommes ? \\[0pt]
La religion est-elle une production culturelle comme les autres ? \\[0pt]
La religion est-elle un facteur de lien social ? \\[0pt]
La religion est-elle un instrument de pouvoir ? \\[0pt]
La religion est-elle un obstacle à la liberté ? \\[0pt]
La religion implique-t-elle la croyance en un être divin ? \\[0pt]
La religion impose t-elle un joug salutaire à l'intelligence ? \\[0pt]
La religion n'est-elle que l'affaire des croyants ? \\[0pt]
La religion n'est-elle qu'une affaire privée ? \\[0pt]
La religion n'est-elle qu'un fait de culture ? \\[0pt]
La religion outrepasse-t-elle les limites de la raison ? \\[0pt]
La religion peut-elle être civile ? \\[0pt]
La religion peut-elle être naturelle ? \\[0pt]
La religion peut-elle faire lien social ? \\[0pt]
La religion peut-elle n'être qu'une affaire privée ? \\[0pt]
La religion peut-elle se passer de transcendant ? \\[0pt]
La religion peut-elle se passer du recours au sacré ? \\[0pt]
La religion peut-elle suppléer la raison ? \\[0pt]
La religion relève-t-elle de l'irrationnel ? \\[0pt]
La religion relève-t-elle de l'opinion ? \\[0pt]
La religion relie-t-elle les hommes ? \\[0pt]
La religion rend-elle l'homme heureux ? \\[0pt]
La religion rend-elle meilleur ? \\[0pt]
La religion repose-t-elle sur une illusion ? \\[0pt]
La religion se distingue-t-elle de la superstition ? \\[0pt]
La religion se réduit-elle à la foi ? \\[0pt]
La reproduction des œuvres d'art nuit-elle à l'art ? \\[0pt]
La responsabilité est-elle une fiction juridique ? \\[0pt]
La responsabilité implique-t-elle autrui ? \\[0pt]
La responsabilité peut-elle être collective ? \\[0pt]
La responsabilité politique n'est-elle le fait que de ceux qui gouvernent ? \\[0pt]
La révolte peut-elle être un droit ? \\[0pt]
L'argent est-il la mesure de tout échange ? \\[0pt]
L'argent est-il un mal nécessaire ? \\[0pt]
La rhétorique a-t-elle une valeur ? \\[0pt]
La rhétorique est-elle un art ? \\[0pt]
La rigueur des lois ? \\[0pt]
L'art ajoute-t-il quelque chose à la vie ? \\[0pt]
L'art apprend-il à percevoir ? \\[0pt]
L'art a-t-il à être populaire ? \\[0pt]
L'art a-t-il besoin de théorie ? \\[0pt]
L'art a-t-il besoin d'un discours sur l'art ? \\[0pt]
L'art a-t-il des règles ? \\[0pt]
L'art a-t-il des vertus thérapeutiques ? \\[0pt]
L'art a-t-il plus de valeur que la vérité ? \\[0pt]
L'art a-t-il pour fin la vérité ? \\[0pt]
L'art a-t-il pour fin le plaisir ? \\[0pt]
L'art a-t-il pour fonction de sublimer le réel ? \\[0pt]
L'art a-t-il une fin morale ? \\[0pt]
L'art a-t-il une histoire ? \\[0pt]
L'art a-t-il une responsabilité morale ? \\[0pt]
L'art a-t-il une valeur sociale ? \\[0pt]
L'art a-t-il un rôle à jouer dans l'éducation ? \\[0pt]
L'art change-t-il la vie ? \\[0pt]
L'art contre la beauté ? \\[0pt]
L'art décrit-il ? \\[0pt]
L'art de vivre est-il un art ? \\[0pt]
L'art doit-il divertir ? \\[0pt]
L'art doit-il être critique ? \\[0pt]
L'art doit-il nécessairement représenter la réalité ? \\[0pt]
L'art doit-il nous étonner ? \\[0pt]
L'art doit-il refaire le monde ? \\[0pt]
L'art donne-t-il à penser ? \\[0pt]
L'art donne-t-il à rêver ou à penser ? \\[0pt]
L'art donne-t-il à voir l'invisible ? \\[0pt]
L'art donne-t-il nécessairement lieu à la production d'une œuvre ? \\[0pt]
L'art échappe-t-il à la raison ? \\[0pt]
L'art éduque-t-il la perception ? \\[0pt]
L'art éduque-t-il l'homme ? \\[0pt]
L'art, est-ce ce qui résiste à la certitude ? \\[0pt]
L'art est-il affaire d'apparence ? \\[0pt]
L'art est-il affaire de goût ? \\[0pt]
L'art est-il affaire d'imagination ? \\[0pt]
L'art est-il à lui-même son propre but ? \\[0pt]
L'art est-il au service du beau ? \\[0pt]
L'art est-il ce qui permet de partager ses émotions ? \\[0pt]
L'art est-il désintéressé ? \\[0pt]
L'art est-il destiné à embellir ? \\[0pt]
L'art est-il hors du temps ? \\[0pt]
L'art est-il imitatif ? \\[0pt]
L'art est-il interprétation ? \\[0pt]
L'art est-il inutile ? \\[0pt]
L'art est-il le miroir du monde ? \\[0pt]
L'art est-il le produit de l'inconscient ? \\[0pt]
L'art est-il le propre de l'homme ? \\[0pt]
L'art est-il le règne des apparences ? \\[0pt]
L'art est-il mensonger ? \\[0pt]
L'art est-il méthodique ? \\[0pt]
L'art est-il moins nécessaire que la science ? \\[0pt]
L'art est-il objet de compréhension ? \\[0pt]
L'art est-il politique ? \\[0pt]
L'art est-il révolutionnaire ? \\[0pt]
L'art est-il subversif ? \\[0pt]
L'art est-il un divertissement ? \\[0pt]
L'art est-il une affaire sérieuse ? \\[0pt]
L'art est-il une connaissance ? \\[0pt]
L'art est-il une critique de la culture ? \\[0pt]
L'art est-il une expérience de la liberté ? \\[0pt]
L'art est-il une forme de langage ? \\[0pt]
L'art est-il une histoire ? \\[0pt]
L'art est-il une promesse de bonheur ? \\[0pt]
L'art est-il une valeur ? \\[0pt]
L'art est-il universel ? \\[0pt]
L'art est-il un jeu ? \\[0pt]
L'art est-il un langage ? \\[0pt]
L'art est-il un langage universel ? \\[0pt]
L'art est-il un luxe ? \\[0pt]
L'art est-il un mode de connaissance ? \\[0pt]
L'art est-il un modèle pour la philosophie ? \\[0pt]
L'art est-il un monde ? \\[0pt]
L'art est-il un moyen de connaître ? \\[0pt]
L'art est-il un refuge ? \\[0pt]
L'art est par-delà beauté et laideur ? \\[0pt]
L'art éveille-t-il des émotions spécifiques ? \\[0pt]
L'art : expérience, exercice ou habitude ? \\[0pt]
L'art exprime-t-il ce que nous ne saurions dire ? \\[0pt]
L'art fait-il penser ? \\[0pt]
L'art imite-t-il la nature ? \\[0pt]
L'artiste a-t-il besoin de modèle ? \\[0pt]
L'artiste a-t-il besoin d'une idée de l'art ? \\[0pt]
L'artiste a-t-il besoin d'un public ? \\[0pt]
L'artiste a-t-il toujours raison ? \\[0pt]
L'artiste a-t-il une méthode ? \\[0pt]
L'artiste doit-il être de son temps ? \\[0pt]
L'artiste doit-il être original ? \\[0pt]
L'artiste doit-il se donner des modèles ? \\[0pt]
L'artiste doit-il se soucier du goût du public ? \\[0pt]
L'artiste est-il le mieux placé pour comprendre son œuvre ? \\[0pt]
L'artiste est-il le seul travailleur libre ? \\[0pt]
L'artiste est-il le technicien du beau ? \\[0pt]
L'artiste est-il maître de son œuvre ? \\[0pt]
L'artiste est-il souverain ? \\[0pt]
L'artiste est-il un créateur ? \\[0pt]
L'artiste est-il un métaphysicien ? \\[0pt]
L'artiste est-il un travailleur ? \\[0pt]
L'artiste exprime-t-il quelque chose ? \\[0pt]
L'artiste peut-il se passer d'un maître ? \\[0pt]
L'artiste recherche-t-il le beau ? \\[0pt]
L'artiste sait-il ce qu'il fait ? \\[0pt]
L'artiste travaille-t-il ? \\[0pt]
L'art modifie-t-il notre rapport à la réalité ? \\[0pt]
L'art modifie-t-il notre rapport au réel ? \\[0pt]
L'art n'est-il pas toujours politique ? \\[0pt]
L'art n'est-il pas toujours religieux ? \\[0pt]
L'art n'est-il qu'apparence ? \\[0pt]
L'art n'est-il qu'un artifice ? \\[0pt]
L'art n'est-il qu'une affaire d'esthétique ? \\[0pt]
L'art n'est-il qu'une question de sentiment ? \\[0pt]
L'art n'est-il qu'un mode d'expression subjectif ? \\[0pt]
L'art n'est qu'une affaire de goût ? \\[0pt]
L'art nous détourne-t-il de la réalité ? \\[0pt]
L'art nous donne-t-il des raisons d'espérer ? \\[0pt]
L'art nous enseigne-t-il quelque chose ? \\[0pt]
L'art nous fait-il mieux percevoir le réel ? \\[0pt]
L'art nous libère-t-il de l'insignifiance ? \\[0pt]
L'art nous mène-t-il au vrai ? \\[0pt]
L'art nous permet-il de lutter contre l'irréversibilité ? \\[0pt]
L'art nous ramène-t-il à la réalité ? \\[0pt]
L'art nous réconcilie-t-il avec le monde ? \\[0pt]
L'art parachève-t-il la nature ? \\[0pt]
L'art participe-t-il à la vie politique ? \\[0pt]
L'art pense-t-il ? \\[0pt]
L'art permet-il un accès au divin ? \\[0pt]
L'art peut-il changer le monde ? \\[0pt]
L'art peut-il contribuer à éduquer les hommes ? \\[0pt]
L'art peut-il encore imiter la nature ? \\[0pt]
L'art peut-il être abstrait ? \\[0pt]
L'art peut-il être brut ? \\[0pt]
L'art peut-il être conceptuel ? \\[0pt]
L'art peut-il être enseigné ? \\[0pt]
L'art peut-il être populaire ? \\[0pt]
L'art peut-il être révolutionnaire ? \\[0pt]
L'art peut-il être sans œuvre ? \\[0pt]
L'art peut-il être utile ? \\[0pt]
L'art peut-il finir ? \\[0pt]
L'art peut-il mourir ? \\[0pt]
L'art peut-il ne pas être sacré ? \\[0pt]
L'art peut-il n'être aucunement mimétique ? \\[0pt]
L'art peut-il n'être pas conceptuel ? \\[0pt]
L'art peut-il nous rendre meilleurs ? \\[0pt]
L'art peut-il prétendre à la vérité ? \\[0pt]
L'art peut-il quelque chose contre la morale ? \\[0pt]
L'art peut-il quelque chose pour la morale ? \\[0pt]
L'art peut-il rendre le mouvement ? \\[0pt]
L'art peut-il s'affranchir des lois ? \\[0pt]
L'art peut-il sauver le monde ? \\[0pt]
L'art peut-il s'enseigner ? \\[0pt]
L'art peut-il se passer de la beauté ? \\[0pt]
L'art peut-il se passer de règles ? \\[0pt]
L'art peut-il se passer des critères du beau et du laid ? \\[0pt]
L'art peut-il se passer d'idéal ? \\[0pt]
L'art peut-il se passer d'œuvres ? \\[0pt]
L'art peut-il tenir lieu de métaphysique ? \\[0pt]
L'art produit-il nécessairement des œuvres ? \\[0pt]
L'art progresse-t-il ? \\[0pt]
L'art prolonge-t-il la nature ? \\[0pt]
L'art rend-il heureux ? \\[0pt]
L'art rend-il les hommes meilleurs ? \\[0pt]
L'art s'adresse-t-il à la sensibilité ? \\[0pt]
L'art s'adresse-t-il à tous ? \\[0pt]
L'art sait-il montrer ce que le langage ne peut pas dire ? \\[0pt]
L'art s'apparente-t-il à la philosophie ? \\[0pt]
L'art s'apprend-il ? \\[0pt]
L'art : une arithmétique sensible ? \\[0pt]
L'art vise-t-il le beau ? \\[0pt]
L'art vise-t-il nécessairement le beau ? \\[0pt]
La sagesse rend-elle heureux ? \\[0pt]
La santé est-elle un devoir ? \\[0pt]
La santé est-elle un droit ou un devoir ? \\[0pt]
La santé n'est-elle que l'absence de maladie ? \\[0pt]
L'ascétisme est-il une vertu ? \\[0pt]
La science admet-elle des degrés de croyance ? \\[0pt]
La science a-t-elle besoin d'imagination ? \\[0pt]
La science a-t-elle besoin d'un critère de démarcation entre science et non science ? \\[0pt]
La science a-t-elle besoin d'une méthode ? \\[0pt]
La science a-t-elle besoin du principe de causalité ? \\[0pt]
La science a-t-elle des limites ? \\[0pt]
La science a-t-elle le monopole de la raison ? \\[0pt]
La science a-t-elle le monopole de la vérité ? \\[0pt]
La science a-t-elle pour fin de prévoir ? \\[0pt]
La science a-t-elle réponse à tout ? \\[0pt]
La science a-t-elle toujours raison ? \\[0pt]
La science a-t-elle une histoire ? \\[0pt]
La science commence-t-elle avec la perception ? \\[0pt]
La science connaît-elle ses limites ? \\[0pt]
La science découvre-t-elle ou construit-elle son objet ? \\[0pt]
La science dépend-elle nécessairement de l'expérience ? \\[0pt]
La science désenchante-t-elle le monde ? \\[0pt]
La science dévoile-t-elle le réel ? \\[0pt]
La science doit-elle nous fournir des représentations du monde ? \\[0pt]
La science doit-elle se fonder sur une idée de la nature ? \\[0pt]
La science doit-elle se passer de l'idée de finalité ? \\[0pt]
La science du vivant peut-elle se passer de l'idée de finalité ? \\[0pt]
La science est-elle austère ? \\[0pt]
La science est-elle indépendante de toute métaphysique ? \\[0pt]
La science est-elle inhumaine ? \\[0pt]
La science est-elle le lieu de la vérité ? \\[0pt]
La science est-elle une affaire politique ? \\[0pt]
La science est-elle une connaissance du réel ? \\[0pt]
La science est-elle une langue bien faite ? \\[0pt]
La science est-elle un jeu ? \\[0pt]
La science exclut-elle l'imagination ? \\[0pt]
La science explique-t-elle le réel ? \\[0pt]
La science historique peut-elle être prédictive ? \\[0pt]
La science n'est-elle qu'une activité théorique ? \\[0pt]
La science n'est-elle qu'une fiction ? \\[0pt]
La science nous éloigne-t-elle de la religion ? \\[0pt]
La science nous éloigne-t-elle des choses ? \\[0pt]
La science nous indique-t-elle ce que nous devons faire ? \\[0pt]
La science pense-t-elle ? \\[0pt]
La science permet-elle de comprendre le monde ? \\[0pt]
La science permet-elle de mieux comprendre la religion ? \\[0pt]
La science permet-elle d'expliquer toute la réalité ? \\[0pt]
La science peut-elle errer ? \\[0pt]
La science peut-elle être une métaphysique ? \\[0pt]
La science peut-elle guider notre conduite ? \\[0pt]
La science peut-elle lutter contre les préjugés ? \\[0pt]
La science peut-elle produire des croyances ? \\[0pt]
La science peut-elle se passer de fondement ? \\[0pt]
La science peut-elle se passer de l'idée de finalité ? \\[0pt]
La science peut-elle se passer de métaphysique ? \\[0pt]
La science peut-elle se passer de méthode ? \\[0pt]
La science peut-elle se passer d'hypothèses ? \\[0pt]
La science peut-elle se passer d'institutions ? \\[0pt]
La science peut-elle tout expliquer ? \\[0pt]
La science porte-elle au scepticisme ? \\[0pt]
La science procède-t-elle par rectification ? \\[0pt]
La science rend-elle la religion caduque ? \\[0pt]
La science se limite-t-elle à constater les faits ? \\[0pt]
La science s'oppose-t-elle à la religion ? \\[0pt]
La sensation est-elle une connaissance ? \\[0pt]
La sensibilité peut-elle délivrer une vérité ? \\[0pt]
La sensibilité se cultive-t-elle ? \\[0pt]
La servitude peut-elle être volontaire ? \\[0pt]
La société doit-elle reconnaître les désirs individuels ? \\[0pt]
La société est-elle concevable sans le travail ? \\[0pt]
La société est-elle un grand marché ? \\[0pt]
La société est-elle un organisme ? \\[0pt]
La société existe-t-elle ? \\[0pt]
La société fait-elle l'homme ? \\[0pt]
La société peut-elle être l'objet d'une science ? \\[0pt]
La société peut-elle être objet de connaissance ? \\[0pt]
La société peut-elle se passer de l'État ? \\[0pt]
La société précède-t-elle l'individu ? \\[0pt]
La société repose-t-elle sur l'altruisme ? \\[0pt]
La sociologie de l'art nous permet-elle de comprendre l'art ? \\[0pt]
La sociologie est-elle déterministe ? \\[0pt]
La sociologie est-elle une physique sociale ? \\[0pt]
La sociologie relativise-t-elle la valeur des œuvres d'art ? \\[0pt]
La solidarité est-elle naturelle ? \\[0pt]
La solitude constitue-t-elle un obstacle à la citoyenneté ? \\[0pt]
La souffrance a-t-elle une valeur morale ? \\[0pt]
La souffrance a-t-elle un sens ? \\[0pt]
La souffrance a-t-elle un sens moral ? \\[0pt]
La souffrance d'autrui m'importe-t-elle ? \\[0pt]
La souffrance peut-elle avoir un sens ? \\[0pt]
La souffrance peut-elle être partagée ? \\[0pt]
La souffrance peut-elle être un mode de connaissance ? \\[0pt]
La souveraineté est-elle indivisible ? \\[0pt]
La souveraineté peut-elle être limitée ? \\[0pt]
La souveraineté peut-elle se partager ? \\[0pt]
La sphère privée échappe-t-elle au politique ? \\[0pt]
La sympathie peut-elle tenir lieu de morale ? \\[0pt]
La sympathie peut-elle tenir lieu de moralité ? \\[0pt]
La technique accroît-elle notre liberté ? \\[0pt]
La technique a-t-elle sa place en politique ? \\[0pt]
La technique a-t-elle une finalité ? \\[0pt]
La technique a-t-elle une histoire ? \\[0pt]
La technique augmente-t-elle notre puissance d'agir ? \\[0pt]
La technique change-t-elle l'homme ? \\[0pt]
La technique change-t-elle notre rapport au monde ? \\[0pt]
La technique crée-t-elle son propre monde ? \\[0pt]
La technique déshumanise-t-elle le monde ? \\[0pt]
La technique détermine-t-elle les rapports sociaux ? \\[0pt]
La technique doit-elle nous libérer du travail ? \\[0pt]
La technique doit-elle permettre de dépasser les limites de l'humain ? \\[0pt]
La technique donne-t-elle une illusion de pouvoir ? \\[0pt]
La technique est-elle civilisatrice ? \\[0pt]
La technique est-elle contre nature ? \\[0pt]
La technique est-elle dangereuse ? \\[0pt]
La technique est-elle l'application de la science ? \\[0pt]
La technique est-elle le propre de l'homme ? \\[0pt]
La technique est-elle libératrice ? \\[0pt]
La technique est-elle moralement neutre ? \\[0pt]
La technique est-elle neutre ? \\[0pt]
La technique est-elle une forme de savoir ? \\[0pt]
La technique est-elle un savoir ? \\[0pt]
La technique facilite-t-elle la vie ? \\[0pt]
La technique fait-elle des miracles ? \\[0pt]
La technique fait-elle violence à la nature ? \\[0pt]
La technique imite-t-elle la nature ? \\[0pt]
La technique invente-t-elle des formes ? \\[0pt]
La technique libère-t-elle les hommes ? \\[0pt]
La technique met-elle le monde à notre disposition ? \\[0pt]
La technique ne fait-elle qu'appliquer la science ? \\[0pt]
La technique ne pose-t-elle que des problèmes techniques ? \\[0pt]
La technique n'est-elle pour l'homme qu'un moyen ? \\[0pt]
La technique n'est-elle qu'une application de la science ? \\[0pt]
La technique n'est-elle qu'une ruse ? \\[0pt]
La technique n'est-elle qu'un moyen ? \\[0pt]
La technique n'est-elle qu'un moyen de domination ? \\[0pt]
La technique n'est-elle qu'un outil au service de l'homme ? \\[0pt]
La technique n'est-elle qu'un prolongement de nos organes ? \\[0pt]
La technique n'est-elle qu'un savoir-faire ? \\[0pt]
La technique n'existe-elle que pour satisfaire des besoins ? \\[0pt]
La technique nous délivre-t-elle d'un rapport irrationnel au monde ? \\[0pt]
La technique nous éloigne-t-elle de la nature ? \\[0pt]
La technique nous éloigne-t-elle de la réalité ? \\[0pt]
La technique nous libère-t-elle ? \\[0pt]
La technique nous libère-t-elle du travail ? \\[0pt]
La technique nous oppose-t-elle à la nature ? \\[0pt]
La technique nous permet-elle de comprendre la nature ? \\[0pt]
La technique pense-t-elle ? \\[0pt]
La technique permet-elle de réaliser tous les désirs ? \\[0pt]
La technique peut-elle améliorer l'homme ? \\[0pt]
La technique peut-elle être tenue pour la forme moderne de la culture ? \\[0pt]
La technique peut-elle respecter la nature ? \\[0pt]
La technique peut-elle se déduire de la science ? \\[0pt]
La technique peut-elle se passer de la science ? \\[0pt]
La technique pose-t-elle plus de problèmes qu'elle n'en résout ? \\[0pt]
La technique produit-elle autre chose que condition de la pensée ? \\[0pt]
La technique produit-elle son propre savoir ? \\[0pt]
La technique provoque-t-elle inévitablement des catastrophes ? \\[0pt]
La technique repose-t-elle sur le génie du technicien ? \\[0pt]
La technique sert-elle nos désirs ? \\[0pt]
La technique s'oppose-t-elle à la nature ? \\[0pt]
La technologie modifie-t-elle les rapports sociaux ? \\[0pt]
La terre est-elle un bien commun ? \\[0pt]
L'athée peut-il être vertueux ? \\[0pt]
L'athéisme condamne-t-il l'existence à l'absurdité ? \\[0pt]
L'athéisme est-il une croyance ? \\[0pt]
La théologie est-elle une science ? \\[0pt]
La théologie peut-elle être rationnelle ? \\[0pt]
La théorie nous éloigne-t-elle de la réalité ? \\[0pt]
La théorie peut-elle nous égarer ? \\[0pt]
La tolérance a-t-elle des limites ? \\[0pt]
La tolérance est-elle un concept politique ? \\[0pt]
La tolérance est-elle une vertu ? \\[0pt]
La tolérance peut-elle constituer un problème pour la démocratie ? \\[0pt]
La transparence est-elle un idéal démocratique ? \\[0pt]
La tristesse est-elle mauvaise ? \\[0pt]
L'attention caractérise-t-elle la conscience ? \\[0pt]
L'autre est-il le fondement de la conscience morale ? \\[0pt]
La valeur d'une action se mesure-t-elle à sa réussite ? \\[0pt]
La valeur d'une action se mesure-t-elle à ses conséquences ? \\[0pt]
La valeur d'une théorie scientifique se mesure-t-elle à son efficacité ? \\[0pt]
La valeur morale d'une action se juge-t-elle à ses conséquences ? \\[0pt]
La vanité est-elle toujours sans objet ? \\[0pt]
L'avenir a-t-il une réalité ? \\[0pt]
L'avenir est-il imaginable ? \\[0pt]
L'avenir est-il incertain ? \\[0pt]
L'avenir est-il l'affaire de la pensée ? \\[0pt]
L'avenir est-il prévisible ? \\[0pt]
L'avenir est-il sans image ? \\[0pt]
L'avenir est-il une page blanche ? \\[0pt]
L'avenir existe-t-il ? \\[0pt]
L'avenir peut-il être objet de connaissance ? \\[0pt]
La vérification fait-elle la vérité ? \\[0pt]
La vérité admet-elle des degrés ? \\[0pt]
La vérité a-t-elle une histoire ? \\[0pt]
La vérité demande-t-elle du courage ? \\[0pt]
La vérité doit-elle toujours être démontrée ? \\[0pt]
La vérité donne-t-elle le droit d'être injuste ? \\[0pt]
La vérité d'une théorie dépend-elle de sa correspondance avec les faits ? \\[0pt]
La vérité échappe-t-elle au temps ? \\[0pt]
La vérité est-elle affaire de cohérence ? \\[0pt]
La vérité est-elle affaire de croyance ou de savoir ? \\[0pt]
La vérité est-elle affaire d'unanimité ? \\[0pt]
La vérité est-elle contraignante ? \\[0pt]
La vérité est-elle éternelle ? \\[0pt]
La vérité est-elle fille de son temps ? \\[0pt]
La vérité est-elle hors de notre portée ? \\[0pt]
La vérité est-elle intemporelle ? \\[0pt]
La vérité est-elle la valeur suprême ? \\[0pt]
La vérité est-elle libératrice ? \\[0pt]
La vérité est-elle morale ? \\[0pt]
La vérité est-elle objective ? \\[0pt]
La vérité est-elle préférable à l'illusion ? \\[0pt]
La vérité est-elle relative ? \\[0pt]
La vérité est-elle triste ? \\[0pt]
La vérité est-elle une ? \\[0pt]
La vérité est-elle une construction ? \\[0pt]
La vérité est-elle une idole ? \\[0pt]
La vérité est-elle une valeur ? \\[0pt]
La vérité n'est-elle qu'une erreur rectifiée ? \\[0pt]
La vérité n'est-elle qu'un idéal inaccessible ? \\[0pt]
La vérité nous appartient-elle ? \\[0pt]
La vérité nous contraint-elle ? \\[0pt]
La vérité nous rend-elle libres ? \\[0pt]
La vérité peut-elle changer avec le temps ? \\[0pt]
La vérité peut-elle être équivoque ? \\[0pt]
La vérité peut-elle être indicible ? \\[0pt]
La vérité peut-elle être relative ? \\[0pt]
La vérité peut-elle être tolérante ? \\[0pt]
La vérité peut-elle faire peur ? \\[0pt]
La vérité peut-elle laisser indifférent ? \\[0pt]
La vérité peut-elle se définir par le consensus ? \\[0pt]
La vérité peut-elle se discuter ? \\[0pt]
La vérité rend-elle heureux ? \\[0pt]
La vérité rend-elle libre ? \\[0pt]
La vérité requiert-elle du courage ? \\[0pt]
La vérité résiste-elle à sa révélation ? \\[0pt]
La vérité scientifique est-elle relative ? \\[0pt]
La vérité se communique-t-elle ? \\[0pt]
La vérité se discute-t-elle ? \\[0pt]
La vertu est-elle une forme de santé ? \\[0pt]
La vertu est-elle utile ? \\[0pt]
La vertu peut-elle être excessive ? \\[0pt]
La vertu peut-elle être purement morale ? \\[0pt]
La vertu peut-elle s'enseigner ? \\[0pt]
La vertu requiert-elle l'idée de bien ? \\[0pt]
La vertu s'enseigne-t-elle ? \\[0pt]
L'aveu diminue-t-il la faute ? \\[0pt]
La vie a-t-elle un sens ? \\[0pt]
La vie collective est-elle nécessairement frustrante ? \\[0pt]
La vie en société est-elle naturelle à l'homme ? \\[0pt]
La vie en société impose-t-elle de n'être pas soi-même ? \\[0pt]
La vie en société menace-t-elle la liberté ? \\[0pt]
La vie est-elle la valeur suprême ? \\[0pt]
La vie est-elle le bien le plus précieux ? \\[0pt]
La vie est-elle le bien suprême ? \\[0pt]
La vie est-elle l'objet des sciences de la vie ? \\[0pt]
La vie est-elle objet de science ? \\[0pt]
La vie est-elle sacrée ? \\[0pt]
La vie est-elle trop courte ? \\[0pt]
La vie est-elle une notion métaphysique ? \\[0pt]
La vie est-elle une valeur ? \\[0pt]
La vie est-elle un roman ? \\[0pt]
La vie est-elle un songe ? \\[0pt]
La vie intérieure est-elle une illusion ? \\[0pt]
La vie peut-elle être éternelle ? \\[0pt]
La vie peut-elle être objet de science ? \\[0pt]
La vie peut-elle être sans histoire ? \\[0pt]
La vie politique est-elle aliénante ? \\[0pt]
La vie sexuelle est-elle volontaire ? \\[0pt]
La vie sociale est-elle toujours conflictuelle ? \\[0pt]
La vie sociale est-elle une comédie ? \\[0pt]
La vie sociale relève-t-elle des habitudes ? \\[0pt]
La vie suffit-elle à créer des valeurs ? \\[0pt]
La violence a-t-elle des degrés ? \\[0pt]
La violence a-t-elle un sens dans l'histoire ? \\[0pt]
La violence engendre-t-elle toujours la violence ? \\[0pt]
La violence est-elle la condition du progrès ? \\[0pt]
La violence est-elle le fondement du droit ? \\[0pt]
La violence est-elle le prix du progrès ? \\[0pt]
La violence est-elle toujours destructrice ? \\[0pt]
La violence est-elle toujours injustifiable ? \\[0pt]
La violence peut-elle avoir raison ? \\[0pt]
La violence peut-elle être gratuite ? \\[0pt]
La violence peut-elle être morale ? \\[0pt]
La vision peut-elle être le modèle de toute connaissance ? \\[0pt]
La volonté constitue-t-elle le principe de la politique ? \\[0pt]
La volonté est-elle la somme des inclinations ? \\[0pt]
La volonté est-elle libre ? \\[0pt]
La volonté générale est-elle la volonté de tous ? \\[0pt]
La volonté peut-elle être collective ? \\[0pt]
La volonté peut-elle être générale ? \\[0pt]
La volonté peut-elle être indéterminée ? \\[0pt]
La volonté peut-elle être libre ? \\[0pt]
La volonté peut-elle nous manquer ? \\[0pt]
La volonté politique peut-elle être commune ? \\[0pt]
La vraie morale se moque-t-elle de la morale ? \\[0pt]
Le beau a-t-il une histoire ? \\[0pt]
Le beau est-il aimable ? \\[0pt]
Le beau est-il dicible ? \\[0pt]
Le beau est-il l'objet de l'esthétique ? \\[0pt]
Le beau est-il toujours moral ? \\[0pt]
Le beau est-il une valeur commune ? \\[0pt]
Le beau est-il universel ? \\[0pt]
Le beau et le bien sont-ils, au fond, identiques ? \\[0pt]
Le beau existe-t-il indépendamment du bien ? \\[0pt]
Le beau n'est-il qu'une idée ? \\[0pt]
Le beau peut-il être bizarre ? \\[0pt]
Le beau peut-il être effrayant ? \\[0pt]
Le besoin de métaphysique est-il un besoin de connaissance ? \\[0pt]
Le bien commun est-il une illusion ? \\[0pt]
Le bien détermine-t-il la volonté ? \\[0pt]
Le bien, est-ce l'utile ? \\[0pt]
Le bien est-il conformité à la nature ? \\[0pt]
Le bien est-il l'utile ? \\[0pt]
Le bien est-il relatif ? \\[0pt]
Le bien n'est-il réalisable que comme moindre mal ? \\[0pt]
Le bien suppose-t-il la transcendance ? \\[0pt]
Le bonheur a-t-il nécessairement un objet ? \\[0pt]
Le bonheur comporte-t-il des degrés ? \\[0pt]
Le bonheur de la passion est-il sans lendemain ? \\[0pt]
Le bonheur des citoyens est-il un idéal politique ? \\[0pt]
Le bonheur est-il affaire de calcul ? \\[0pt]
Le bonheur est-il affaire de hasard ou de nécessité ? \\[0pt]
Le bonheur est-il affaire de vertu ? \\[0pt]
Le bonheur est-il affaire de volonté ? \\[0pt]
Le bonheur est-il affaire privée ? \\[0pt]
Le bonheur est-il au nombre de nos devoirs ? \\[0pt]
Le bonheur est-il dans l'inconscience ? \\[0pt]
Le bonheur est-il l'absence de maux ? \\[0pt]
Le bonheur est-il l'affaire du politique ? \\[0pt]
Le bonheur est-il la fin de la vie ? \\[0pt]
Le bonheur est-il le bien suprême ? \\[0pt]
Le bonheur est-il le but de la politique ? \\[0pt]
Le bonheur est-il le prix de la vertu ? \\[0pt]
Le bonheur est-il nécessairement lié au plaisir ? \\[0pt]
Le bonheur est-il un accident ? \\[0pt]
Le bonheur est-il un but politique ? \\[0pt]
Le bonheur est-il un droit ? \\[0pt]
Le bonheur est-il une affaire de circonstances ? \\[0pt]
Le bonheur est-il une affaire privée ? \\[0pt]
Le bonheur est-il une fin morale ? \\[0pt]
Le bonheur est-il une fin politique ? \\[0pt]
Le bonheur est-il une récompense ? \\[0pt]
Le bonheur est-il une valeur morale ? \\[0pt]
Le bonheur est-il un idéal ? \\[0pt]
Le bonheur est-il un principe politique ? \\[0pt]
Le bonheur n'est-il qu'une idée ? \\[0pt]
Le bonheur n'est-il qu'un idéal ? \\[0pt]
Le bonheur peut-il être collectif ? \\[0pt]
Le bonheur peut-il être le but de la politique ? \\[0pt]
Le bonheur peut-il être un droit ? \\[0pt]
Le bonheur peut-il être un objectif politique ? \\[0pt]
Le bonheur s'apprend-il ? \\[0pt]
Le bonheur se calcule-t-il ? \\[0pt]
Le bonheur se fonde-t-il sur un savoir ? \\[0pt]
Le bonheur se mérite-t-il ? \\[0pt]
Le cerveau pense-t-il ? \\[0pt]
L'échange constitue-t-il un lien social ? \\[0pt]
L'échange est-il un facteur de paix ? \\[0pt]
L'échange n'a-t-il de fondement qu'économique ? \\[0pt]
L'échange ne porte-t-il que sur les choses ? \\[0pt]
L'échange peut-il être désintéressé ? \\[0pt]
Le châtiment peut-il ne rien devoir au désir de vengeance ? \\[0pt]
Le châtiment rend-il meilleur ? \\[0pt]
Le choix peut-il être éclairé ? \\[0pt]
Le cinéma, art de la représentation ? \\[0pt]
Le cinéma est-il un art ? \\[0pt]
Le cinéma est-il un art comme les autres ? \\[0pt]
Le cinéma est-il un art narratif ? \\[0pt]
Le cinéma est-il un art ou une industrie ? \\[0pt]
Le cinéma est-il un art populaire ? \\[0pt]
Le citoyen a-t-il perdu toute naturalité ? \\[0pt]
Le citoyen peut-il être à la fois libre et soumis à l'État ? \\[0pt]
L'écologie est-elle aussi une science humaine ? \\[0pt]
L'écologie est-elle une science de la nature ou une science sociale ? \\[0pt]
L'écologie est-elle un problème politique ? \\[0pt]
L'écologie, une science humaine ? \\[0pt]
Le combat contre l'injustice a-t-il une source morale ? \\[0pt]
Le commerce adoucit-il les mœurs ? \\[0pt]
Le commerce est-il pacificateur ? \\[0pt]
Le commerce est-il un facteur de paix ? \\[0pt]
Le commerce peut-il être équitable ? \\[0pt]
Le commerce unit-il les hommes ? \\[0pt]
Le concept de nature est-il un concept scientifique ? \\[0pt]
Le concept de race est-il anthropologique ? \\[0pt]
Le concept de société sans écriture est-il pertinent ? \\[0pt]
Le concept d'inconscient est-il nécessaire en sciences humaines ? \\[0pt]
Le conflit entre la science et la religion est-il inévitable ? \\[0pt]
Le conflit entre l'individu et la société est-il irréductible ? \\[0pt]
Le conflit est-il au fondement de la vie sociale ? \\[0pt]
Le conflit est-il constitutif de la politique ? \\[0pt]
Le conflit est-il la raison d'être de la politique ? \\[0pt]
Le conflit est-il une maladie sociale ? \\[0pt]
L'économie a-t-elle des lois ? \\[0pt]
L'économie a-t-elle ses propres lois ? \\[0pt]
L'économie est-elle politique ? \\[0pt]
L'économie est-elle une science ? \\[0pt]
L'économie est-elle une science humaine ? \\[0pt]
L'économie et le politique obéissent-ils à une même rationalité ? \\[0pt]
L'économie fait-elle partie des sciences humaines ? \\[0pt]
Le consensus peut-il être critère de vérité ? \\[0pt]
Le consensus peut-il faire le vrai ? \\[0pt]
Le contradictoire peut-il exister ? \\[0pt]
Le contrat est-il au fondement de la politique ? \\[0pt]
Le contrat peut-il remplacer la loi ? \\[0pt]
Le corps dit-il quelque chose ? \\[0pt]
Le corps est-il le reflet de l'âme ? \\[0pt]
Le corps est-il négociable ? \\[0pt]
Le corps est-il porteur de valeurs ? \\[0pt]
Le corps est-il respectable ? \\[0pt]
Le corps est-il une entrave pour l'esprit ? \\[0pt]
Le corps est-il un enjeu des sciences humaines ? \\[0pt]
Le corps est-il un tout ? \\[0pt]
Le corps humain est-il naturel ? \\[0pt]
Le corps impose-t-il des perspectives ? \\[0pt]
Le corps n'est-il que matière ? \\[0pt]
Le corps n'est-il qu'un mécanisme ? \\[0pt]
Le corps obéit-il à l'esprit ? \\[0pt]
Le corps pense-t-il ? \\[0pt]
Le corps peut-il être objet d'art ? \\[0pt]
Le cosmopolitisme peut-il devenir réalité ? \\[0pt]
Le cosmopolitisme peut-il être réaliste ? \\[0pt]
Le courage est-il une vertu ? \\[0pt]
Le crime suppose-t-il la loi ? \\[0pt]
L'écriture est-elle une technique parmi d'autres ? \\[0pt]
L'écriture ne sert-elle qu'à consigner la pensée ? \\[0pt]
L'écriture peut-elle porter secours à la pensée ? \\[0pt]
Le désespoir est-il une faute morale ? \\[0pt]
Le désespoir peut-il passer pour une sagesse ? \\[0pt]
Le désir a-t-il un objet ? \\[0pt]
Le désir de savoir est-il comblé par la science ? \\[0pt]
Le désir de savoir est-il naturel ? \\[0pt]
Le désir de vérité peut-il être interprété comme un désir de pouvoir ? \\[0pt]
Le désir du bonheur est-il universel ? \\[0pt]
Le désir est-il aveugle ? \\[0pt]
Le désir est-il ce qui nous fait vivre ? \\[0pt]
Le désir est-il désir de l'autre ? \\[0pt]
Le désir est-il le signe d'un manque ? \\[0pt]
Le désir est-il l'essence de l'homme ? \\[0pt]
Le désir est-il nécessairement l'expression d'un manque ? \\[0pt]
Le désir est-il par nature illimité ? \\[0pt]
Le désir est-il sans limite ? \\[0pt]
Le désir n'est-il pas qu'inquiétude ? \\[0pt]
Le désir n'est-il que l'épreuve d'un manque ? \\[0pt]
Le désir n'est-il que manque ? \\[0pt]
Le désir n'est-il qu'inquiétude ? \\[0pt]
Le désir peut-il atteindre son objet ? \\[0pt]
Le désir peut-il être désintéressé ? \\[0pt]
Le désir peut-il ne pas avoir d'objet ? \\[0pt]
Le désir peut-il nous rendre libre ? \\[0pt]
Le désir peut-il se satisfaire de la réalité ? \\[0pt]
Le désordre est-il étonnant ? \\[0pt]
Le désordre est-il un mal ? \\[0pt]
Le despote peut-il être éclairé ? \\[0pt]
Le despotisme peut-il être éclairé ? \\[0pt]
Le développement de la technique est-il toujours facteur de progrès ? \\[0pt]
Le développement des techniques fait-il reculer la croyance ? \\[0pt]
Le devenir est-il pensable ? \\[0pt]
Le devoir est-il l'expression de la contrainte sociale ? \\[0pt]
Le devoir nous apparaît-il d'autant mieux qu'on ne l'a pas accompli ? \\[0pt]
Le devoir rend-il libre ? \\[0pt]
Le devoir s'apprend-il ? \\[0pt]
Le devoir se présente-t-il avec la force de l'évidence ? \\[0pt]
Le devoir supprime-t-il la liberté ? \\[0pt]
Le dialogue conduit-il à la vérité ? \\[0pt]
Le dialogue suffit-il à rompre la solitude ? \\[0pt]
Le divertissement est-il vain ? \\[0pt]
Le don est-il toujours généreux ? \\[0pt]
Le don est-il une modalité de l'échange ? \\[0pt]
Le don exclut-il le calcul ? \\[0pt]
Le doute est-il le principe de la méthode scientifique ? \\[0pt]
Le doute est-il une faiblesse de la pensée ? \\[0pt]
Le doute peut-il être méthodique ? \\[0pt]
Le droit à la différence met-il en péril l'égalité des droits ? \\[0pt]
Le droit doit-il être indépendant de la morale ? \\[0pt]
Le droit doit-il être le seul régulateur de la vie sociale ? \\[0pt]
Le droit doit-il être moral ? \\[0pt]
Le droit est-il facteur de paix ? \\[0pt]
Le droit est-il le fondement de l'État ? \\[0pt]
Le droit est-il raison du plus fort ? \\[0pt]
Le droit est-il une science ? \\[0pt]
Le droit est-il une science humaine ? \\[0pt]
Le droit exclut-il la force ? \\[0pt]
Le droit ne peut-il se fonder sur des faits ? \\[0pt]
Le droit n'est-il qu'une justice par défaut ? \\[0pt]
Le droit n'est-il qu'un ensemble de conventions ? \\[0pt]
Le droit peut-il échapper à l'histoire ? \\[0pt]
Le droit peut-il être flexible ? \\[0pt]
Le droit peut-il être naturel ? \\[0pt]
Le droit peut-il se fonder sur la force ? \\[0pt]
Le droit peut-il se passer de la morale ? \\[0pt]
Le droit sert-il à établir l'ordre ou la justice ? \\[0pt]
L'éducation du goût est-elle la condition de l'expérience esthétique ? \\[0pt]
L'éducation est-elle affaire de politique ? \\[0pt]
L'éducation est-elle par nature conservatrice ? \\[0pt]
L'éducation peut-elle être sentimentale ? \\[0pt]
L'éducation relève-t-elle du politique ? \\[0pt]
Le fait de vivre constitue-t-il un bien en soi ? \\[0pt]
Le fait de vivre est-il un bien en soi ? \\[0pt]
Le fait social est-il une chose ? \\[0pt]
L'efficacité est-elle une vertu ? \\[0pt]
L'efficacité technique est-elle la norme de toute réussite ? \\[0pt]
Le futur est-il contingent ? \\[0pt]
Le futur est-il nécessairement contingent ? \\[0pt]
Le futur existe-t-il ? \\[0pt]
Le futur nous appartient-il ? \\[0pt]
L'égalité des hommes est-elle un fait ou une idée ? \\[0pt]
L'égalité des hommes et des femmes est-elle une question politique ? \\[0pt]
L'égalité est-elle contre nature ? \\[0pt]
L'égalité est-elle l'ennemie de la liberté ? \\[0pt]
L'égalité est-elle souhaitable ? \\[0pt]
L'égalité est-elle toujours juste ? \\[0pt]
L'égalité est-elle une condition de la liberté ? \\[0pt]
L'égalité peut-elle être une menace pour la liberté ? \\[0pt]
Le génie est-il la marque de l'excellence artistique ? \\[0pt]
Le génie n'a-t-il sa place que dans les arts ? \\[0pt]
Le genre humain : unité ou pluralité ? \\[0pt]
Le geste technique exprime t-il une liberté sans fin ? \\[0pt]
Le goût : certitude ou conviction ? \\[0pt]
Le goût est-il affaire d'éducation ? \\[0pt]
Le goût est-il nécessairement subjectif ? \\[0pt]
Le goût est-il une faculté ? \\[0pt]
Le goût est-il une question de classe ? \\[0pt]
Le goût est-il une vertu sociale ? \\[0pt]
Le goût s'éduque-t-il ? \\[0pt]
Le goût se forme-t-il ? \\[0pt]
Le gouvernement par le peuple est-il nécessairement pour le peuple ? \\[0pt]
Le grand art est-il de plaire ? \\[0pt]
Le hasard est-il injuste ? \\[0pt]
Le hasard existe-t-il ? \\[0pt]
Le hasard fait-il bien les choses ? \\[0pt]
Le hasard gouverne-t-il le monde ? \\[0pt]
Le hasard n'est-il que la mesure de notre ignorance ? \\[0pt]
Le hasard n'est-il que le nom de notre ignorance ? \\[0pt]
Le hasard peut-il être un concept explicatif ?La morale doit-elle s'adapter à la réalité ? \\[0pt]
Le jugement artistique se fait-il sans concept ? \\[0pt]
Le jugement critique peut-il s'exercer sans culture ? \\[0pt]
Le jugement de goût est-il désintéressé ? \\[0pt]
Le jugement de goût est-il universel ? \\[0pt]
Le jugement de valeur est-il indifférent à la vérité ? \\[0pt]
Le langage a-t-il une prétention à la vérité ? \\[0pt]
Le langage est-il assimilable à un outil ? \\[0pt]
Le langage est-il d'essence poétique ? \\[0pt]
Le langage est-il l'auxiliaire de la pensée ? \\[0pt]
Le langage est-il le lieu de la vérité ? \\[0pt]
Le langage est-il le miroir du monde ? \\[0pt]
Le langage est-il le moyen d'expression le plus efficace ? \\[0pt]
Le langage est-il le propre de l'homme ? \\[0pt]
Le langage est-il l'instrument de la pensée ? \\[0pt]
Le langage est-il logique ? \\[0pt]
Le langage est-il naturel ? \\[0pt]
Le langage est-il une prise de possession des choses ? \\[0pt]
Le langage est-il un instrument ? \\[0pt]
Le langage est-il un instrument de connaissance ? \\[0pt]
Le langage est-il un obstacle pour la pensée ? \\[0pt]
Le langage fait-il obstacle à la connaissance ? \\[0pt]
Le langage masque-t-il la pensée ? \\[0pt]
Le langage ne sert-il qu'à communiquer ? \\[0pt]
Le langage n'est-il qu'un instrument de communication ? \\[0pt]
Le langage n'est-il qu'un outil ? \\[0pt]
Le langage nous donne-t-il une connaissance des choses ? \\[0pt]
Le langage nous éloigne-t-il de la réalité ? \\[0pt]
Le langage permet-il seulement de communiquer ? \\[0pt]
Le langage peut-il être un obstacle à la recherche de la vérité ? \\[0pt]
Le langage rapproche-t-il ou sépare-t-il les hommes ? \\[0pt]
Le langage rend-il l'homme plus puissant ? \\[0pt]
Le langage reste-t-il toujours à la surface des choses ? \\[0pt]
Le langage traduit-il la pensée ? \\[0pt]
Le langage trahit-il la pensée ? \\[0pt]
Le langage usuel est-il un obstacle à la connaissance ? \\[0pt]
Le langage voile-t-il les choses ? \\[0pt]
Le libre cours de l'imagination est-il libérateur ? \\[0pt]
Le lien social peut-il être compassionnel ? \\[0pt]
Le lien social relève-t-il de la politique ? \\[0pt]
Le logique est-elle un art de penser ? \\[0pt]
Le loisir caractérise-t-il l'homme libre ? \\[0pt]
Le mal apparaît-il toujours ? \\[0pt]
Le mal a-t-il des raisons ? \\[0pt]
Le mal constitue-t-il une objection à l'existence de Dieu ? \\[0pt]
Le mal est-il une erreur ? \\[0pt]
Le mal est-il une objection à l'existence de Dieu ? \\[0pt]
Le mal est-il une réalité objective ? \\[0pt]
Le mal existe-t-il ? \\[0pt]
Le malheur donne-t-il le droit d'être injuste ? \\[0pt]
Le malheur est-il injuste ? \\[0pt]
Le mal peut-il être absolu ? \\[0pt]
Le mal peut-il être involontaire ? \\[0pt]
Le mariage est-il un contrat ? \\[0pt]
L'embryon humain est-il une personne ? \\[0pt]
L'embryon humain peut-il être l'objet d'expérimentation ? \\[0pt]
Le méchant est-il malheureux ? \\[0pt]
Le méchant peut-il être heureux ? \\[0pt]
Le meilleur est-il l'ennemi du bien ? \\[0pt]
Le meilleur gouvernement est-il le gouvernement des meilleurs ? \\[0pt]
Le mensonge de l'art ? \\[0pt]
Le mensonge est-il la plus grande transgression ? \\[0pt]
Le mensonge est-il une forme d'indifférence à la vérité ? \\[0pt]
Le mensonge peut-il être au service de la vérité ? \\[0pt]
Le mépris peut-il être justifié ? \\[0pt]
Le mérite est-il le critère de la vertu ? \\[0pt]
Le métaphysicien est-il un physicien à sa façon ? \\[0pt]
Le mieux est-il l'ennemi du bien ? \\[0pt]
Le modèle du jeu dans les sciences humaines ? \\[0pt]
Le moi est-il haïssable ? \\[0pt]
Le moi est-il insaisissable ? \\[0pt]
Le moi est-il objet de connaissance ? \\[0pt]
Le moi est-il une collection d'idées ? \\[0pt]
Le moi est-il une fiction ? \\[0pt]
Le moi est-il une illusion ? \\[0pt]
Le moi n'est-il qu'une fiction ? \\[0pt]
Le moi n'est-il qu'une idée ? \\[0pt]
Le moi peut-il être l'objet d'un savoir ? \\[0pt]
Le moi reste-t-il identique à lui-même au cours du temps ? \\[0pt]
Le monde a-t-il besoin de moi ? \\[0pt]
Le monde a-t-il une histoire ? \\[0pt]
Le monde aurait-il un sens si l'homme n'existait pas ? \\[0pt]
Le monde de l'animal nous est-il étranger ? \\[0pt]
Le monde est-il autre chose que ce que j'en dis ? \\[0pt]
Le monde est-il écrit en langage mathématique ? \\[0pt]
Le monde est-il en progrès ? \\[0pt]
Le monde est-il éternel ? \\[0pt]
Le monde est-il ma représentation ? \\[0pt]
Le monde est-il notre représentation ? \\[0pt]
Le monde est-il une marchandise ? \\[0pt]
Le monde est-il un théâtre ? \\[0pt]
Le monde extérieur existe-t-il ? \\[0pt]
Le monde n'est-il que l'ensemble de ce qui existe ? \\[0pt]
Le monde n'est-il que ma représentation ? \\[0pt]
Le monde n'est-il qu'une représentation ? \\[0pt]
Le monde peut-il être nouveau ? \\[0pt]
Le monde se réduit-il à ce que nous en voyons ? \\[0pt]
Le monde se suffit-il à lui-même ? \\[0pt]
L'émotion esthétique peut-elle se communiquer ? \\[0pt]
L'émotion esthétique peut-elle se partager ? \\[0pt]
Le mot vie a-t-il plusieurs sens ? \\[0pt]
L'empathie est-elle nécessaire aux sciences sociales ? \\[0pt]
L'empathie est-elle possible ? \\[0pt]
L'empirisme exclut-il l'abstraction ? \\[0pt]
Le mythe est-il objet de science ? \\[0pt]
Le naturel a-t-il toutes les vertus ? \\[0pt]
Le néant est-il ? \\[0pt]
L'enfance est-elle ce qui doit être surmonté ? \\[0pt]
L'enfance est-elle en nous ce qui doit être abandonné ? \\[0pt]
L'enfance peut-elle servir de modèle ? \\[0pt]
L'enfer est-il véritablement pavé de bonnes intentions ? \\[0pt]
L'enquête empirique rend-elle la métaphysique inutile ? \\[0pt]
L'enseignement peut-il se passer d'exemples ? \\[0pt]
L'enthousiasme est-il moral ? \\[0pt]
L'environnement est-il un nouvel objet pour les sciences humaines ? \\[0pt]
L'environnement est-il un problème politique ? \\[0pt]
Le pardon peut-il être une obligation ? \\[0pt]
Le partage est-il une obligation morale ? \\[0pt]
Le passé a-t-il plus de réalité que l'avenir ? \\[0pt]
Le passé a-t-il une réalité ? \\[0pt]
Le passé a-t-il un intérêt ? \\[0pt]
Le passé détermine-t-il notre présent ? \\[0pt]
Le passé, est-ce du passé ? \\[0pt]
Le passé est-il ce qui a disparu ? \\[0pt]
Le passé est-il indépassable ? \\[0pt]
Le passé est-il objet de science ? \\[0pt]
Le passé est-il perdu ? \\[0pt]
Le passé est-il réel ? \\[0pt]
Le passé existe-t-il ? \\[0pt]
Le passé peut-il être un objet de connaissance ? \\[0pt]
Le passionné mérite-t-il d'être plaint ? \\[0pt]
Le patriotisme est-il une vertu ? \\[0pt]
Le peuple a-t-il toujours raison ? \\[0pt]
Le peuple est-il bête ? \\[0pt]
Le peuple est-il souverain ? \\[0pt]
Le peuple est-il un acteur politique ? \\[0pt]
Le peuple peut-il se tromper ? \\[0pt]
Le peuple possède-t-il une volonté ? \\[0pt]
L'éphémère a-t-il une valeur ? \\[0pt]
Le philosophe a-t-il besoin de l'histoire ? \\[0pt]
Le philosophe a-t-il des leçons à donner au politique ? \\[0pt]
Le philosophe est-il le vrai politique ? \\[0pt]
Le philosophe s'écarte-t-il du réel ? \\[0pt]
Le philosophe se détourne-t-il du réel ? \\[0pt]
L'épistémologie est-elle une logique de la science ? \\[0pt]
Le plaisir artistique est-il affaire de jugement ? \\[0pt]
Le plaisir a-t-il un rôle à jouer dans la morale ? \\[0pt]
Le plaisir esthétique est-il affaire de jugement ? \\[0pt]
Le plaisir esthétique est-il un plaisir ? \\[0pt]
Le plaisir esthétique peut-il se communiquer ? \\[0pt]
Le plaisir esthétique peut-il se partager ? \\[0pt]
Le plaisir esthétique suppose-t-il une culture ? \\[0pt]
Le plaisir esthétique suppose-t-il une culture esthétique ? \\[0pt]
Le plaisir est-il immoral ? \\[0pt]
Le plaisir est-il la fin du désir ? \\[0pt]
Le plaisir est-il tout le bonheur ? \\[0pt]
Le plaisir est-il un bien ? \\[0pt]
Le plaisir peut-il être immoral ? \\[0pt]
Le plaisir peut-il être partagé ? \\[0pt]
Le plaisir se mérite-t-il ? \\[0pt]
Le plaisir suffit-il au bonheur ? \\[0pt]
Le poète réinvente-t-il la langue ? \\[0pt]
Le politique a-t-il à régler les passions ? \\[0pt]
Le politique a-t-il à régler les passions humaines ? \\[0pt]
Le politique doit-il être un technicien ? \\[0pt]
Le politique doit-il s'appuyer sur la science ? \\[0pt]
Le politique doit-il se soucier des émotions ? \\[0pt]
Le politique peut-il faire abstraction de la morale ? \\[0pt]
Le possible existe-t-il ? \\[0pt]
Le pouvoir corrompt-il ? \\[0pt]
Le pouvoir corrompt-il nécessairement ? \\[0pt]
Le pouvoir corrompt-il toujours ? \\[0pt]
Le pouvoir de l'État est-il arbitraire ? \\[0pt]
Le pouvoir est-il une forme de domination ? \\[0pt]
Le pouvoir peut-il arrêter le pouvoir ? \\[0pt]
Le pouvoir peut-il être limité ? \\[0pt]
Le pouvoir peut-il ignorer la morale ? \\[0pt]
Le pouvoir peut-il limiter le pouvoir ? \\[0pt]
Le pouvoir peut-il se déléguer ? \\[0pt]
Le pouvoir peut-il se passer de sa mise en scène ? \\[0pt]
Le pouvoir peut-il se transmettre ? \\[0pt]
Le pouvoir politique est-il nécessairement coercitif ? \\[0pt]
Le pouvoir politique peut-il échapper à l'arbitraire ? \\[0pt]
Le pouvoir politique repose-t-il sur la contrainte ? \\[0pt]
Le pouvoir politique repose-t-il sur un savoir ? \\[0pt]
Le pouvoir se partage-t-il ? \\[0pt]
Le pouvoir tend-il toujours à l'excès ? \\[0pt]
Le premier devoir de l'État est-il de se défendre ? \\[0pt]
Le profit est-il la fin de l'échange ? \\[0pt]
Le profit est-il le moteur de la production économique ? \\[0pt]
Le progrès de l'humanité se réduit-il au progrès technique ? \\[0pt]
Le progrès des sciences infirme-t-il les résultats anciens ? \\[0pt]
Le progrès est-il réversible ? \\[0pt]
Le progrès est-il un mythe ? \\[0pt]
Le progrès scientifique fait-il disparaître la superstition ? \\[0pt]
Le progrès technique a-t-il une fin ? \\[0pt]
Le progrès technique est-il source de bonheur ? \\[0pt]
Le progrès technique peut-il être aliénant ? \\[0pt]
Le projet d'une paix perpétuelle est-il insensé ? \\[0pt]
Le propre du vivant est-il de tomber malade ? \\[0pt]
Le psychisme est-il objet de connaissance ? \\[0pt]
Lequel, de l'art ou du réel, est-il une imitation de l'autre ? \\[0pt]
Le raisonnement suit-il des règles ? \\[0pt]
Le rapport de l'homme à son milieu a-t-il une dimension morale ? \\[0pt]
Le rationalisme peut-il être une passion ? \\[0pt]
Le recours à la force signifie-t-il l'échec de la justice ? \\[0pt]
Le recours au symbole est-il la marque de la finitude de la pensée ? \\[0pt]
Le réel est-il ce que l'on croit ? \\[0pt]
Le réel est-il ce que nous expérimentons ? \\[0pt]
Le réel est-il ce que nous percevons ? \\[0pt]
Le réel est-il ce qui apparaît ? \\[0pt]
Le réel est-il ce qui est perçu ? \\[0pt]
Le réel est-il ce qui résiste ? \\[0pt]
Le réel est-il inaccessible ? \\[0pt]
Le réel est-il l'objet de la science ? \\[0pt]
Le réel est-il objet d'interprétation ? \\[0pt]
Le réel est-il rationnel ? \\[0pt]
Le réel excède-t-il l'intelligible ? \\[0pt]
Le réel n'est-il qu'un ensemble de contraintes ? \\[0pt]
Le réel obéit-il à la raison ? \\[0pt]
Le réel peut-il échapper à la logique ? \\[0pt]
Le réel peut-il être beau ? \\[0pt]
Le réel peut-il être contradictoire ? \\[0pt]
Le réel résiste-t-il à la connaissance ? \\[0pt]
Le réel se donne-t-il à voir ? \\[0pt]
Le réel se limite-t-il à ce que font connaître les théories scientifiques ? \\[0pt]
Le réel se limite-t-il à ce que nous percevons ? \\[0pt]
Le réel se réduit-il à ce que l'on perçoit ? \\[0pt]
Le réel se réduit-il à l'objectivité ? \\[0pt]
Le règlement politique des conflits ? \\[0pt]
Le relativisme culturel est-il indépassable ? \\[0pt]
Le relativisme peut-il être un principe de méthode dans les sciences humaines ? \\[0pt]
Le religieux est-il inutile ? \\[0pt]
Le remords est-il stérile ? \\[0pt]
Le respect n'est-il dû qu'aux personnes ? \\[0pt]
Le retour à la nature est-il souhaitable ? \\[0pt]
Le rôle de l'État est-il de faire régner la justice ? \\[0pt]
Le rôle de l'État est-il de préserver la liberté de l'individu ? \\[0pt]
Le rôle de l'historien est-il de juger ? \\[0pt]
Le rôle des théories est-il d'expliquer ou de décrire ? \\[0pt]
Le roman peut-il être philosophique ? \\[0pt]
L'erreur est-elle humaine ? \\[0pt]
L'erreur peut-elle donner un accès à la vérité ? \\[0pt]
L'erreur peut-elle jouer un rôle dans la connaissance scientifique ? \\[0pt]
Les acteurs de l'histoire en sont-ils les auteurs ? \\[0pt]
Les affects sont-ils déraisonnables ? \\[0pt]
Les affects sont-ils des objets sociologiques ? \\[0pt]
Le sage a-t-il besoin d'autrui ? \\[0pt]
Le sage est-il insensible ? \\[0pt]
Les agents économiques ont-ils un comportement rationnel ? \\[0pt]
Les agents sociaux poursuivent-ils l'utilité ? \\[0pt]
Les agents sociaux sont-ils rationnels ? \\[0pt]
Le salut vient-il de la raison ? \\[0pt]
Les animaux échappent-ils à la moralité ? \\[0pt]
Les animaux ont-ils des droits ? \\[0pt]
Les animaux pensent-ils ? \\[0pt]
Les animaux peuvent-ils avoir des droits ? \\[0pt]
Les animaux révèlent-ils ce que nous sommes ? \\[0pt]
Les animaux sont-ils nos amis ? \\[0pt]
Les apparences font-elles partie du monde ? \\[0pt]
Les apparences sont-elles toujours trompeuses ? \\[0pt]
Les artistes sont-ils sérieux ? \\[0pt]
Les arts admettent-ils une hiérarchie ? \\[0pt]
Les arts communiquent-ils entre eux ? \\[0pt]
Les arts ont-ils besoin de théorie ? \\[0pt]
Les arts ont-ils pour fonction de divertir ? \\[0pt]
Les arts sont-ils des jeux ? \\[0pt]
Le savoir a-t-il besoin d'être fondé ? \\[0pt]
Le savoir a-t-il des degrés ? \\[0pt]
Le savoir émancipe-t-il ? \\[0pt]
Le savoir est-il libérateur ? \\[0pt]
Le savoir est-il une condition du bonheur ? \\[0pt]
Le savoir est-il une forme de pouvoir ? \\[0pt]
Le savoir exclut-il toute forme de croyance ? \\[0pt]
Le savoir peut-il être gratuit ? \\[0pt]
Le savoir rend-il libre ? \\[0pt]
Le savoir se vulgarise-t-il ? \\[0pt]
Les beaux-arts sont-ils compatibles entre eux ? \\[0pt]
Les bêtes travaillent-elles ? \\[0pt]
Les bonnes lois font-elles les bonnes mœurs ? \\[0pt]
Les bons comptes font-ils les bons amis ? \\[0pt]
Les catégories sont-elles définitives ? \\[0pt]
Les catégories sont-elles des effets de langue ? \\[0pt]
Les causes naturelles expliquent-elles la nature ? \\[0pt]
Le scepticisme a-t-il des limites ? \\[0pt]
Les changements de théorie s'expliquent-ils seulement par le résultat des expériences ? \\[0pt]
Les choses ont-elles l'air de ce qu'elles sont ? \\[0pt]
Les choses ont-elles quelque chose en commun ? \\[0pt]
Les choses ont-elles une essence ? \\[0pt]
Les choses ont-elles un sens ? \\[0pt]
Les choses peuvent-elles avoir un sens ? \\[0pt]
Les choses sont-elles dans l'espace ? \\[0pt]
Les coïncidences ont-elles des causes ? \\[0pt]
Les comportements humains s'expliquent-il par l'instinct naturel ? \\[0pt]
Les concepts sont-ils des fictions ? \\[0pt]
Les concitoyens doivent-ils être des amis ? \\[0pt]
Les conflits menacent-ils la société ? \\[0pt]
Les conflits politiques ne sont-ils que des conflits sociaux ? \\[0pt]
Les conflits sociaux sont-ils des conflits de classe ? \\[0pt]
Les conflits sociaux sont-ils des conflits politiques ? \\[0pt]
Les connaissances scientifiques peuvent-elles être à la fois vraies et provisoires ? \\[0pt]
Les connaissances scientifiques peuvent-elles être vulgarisées ? \\[0pt]
Les connaissances scientifiques sont-elles vraies ? \\[0pt]
Les considérations morales ont-elles leur place en politique ? \\[0pt]
Les contradictions sont-elles un échec de la raison ? \\[0pt]
Les convictions d'autrui sont-elles un argument ? \\[0pt]
Les couleurs existent-elles en dehors de notre esprit ? \\[0pt]
Les croyances religieuses sont-elles indiscutables ? \\[0pt]
Les croyances sont-elles utiles ? \\[0pt]
Les cultures sont-elles incommensurables ? \\[0pt]
Les décisions politiques peuvent-elles être collectives ? \\[0pt]
Les démonstrations ont-elles des propriétés esthétiques ? \\[0pt]
Les désaccords politiques peuvent-ils être surmontés ? \\[0pt]
Les désirs ont-ils nécessairement un objet ? \\[0pt]
Les devoirs de l'homme varient-ils selon la culture ? \\[0pt]
Les devoirs de l'homme varient-ils selon les cultures ? \\[0pt]
Les différences entre les sciences humaines sont-elles des différences d'objet ? \\[0pt]
Les différentes preuves de l'existence de Dieu prouvent-elles le même Dieu ? \\[0pt]
Les divers sens du mot culture peuvent-ils être ramenés à l'unité ? \\[0pt]
Les droits de l'homme ont-ils besoin d'être fondés ? \\[0pt]
Les droits de l'homme ont-ils un fondement moral ? \\[0pt]
Les droits de l'homme sont-ils ceux du citoyen ? \\[0pt]
Les droits de l'homme sont-ils les droits de la femme ? \\[0pt]
Les droits de l'homme sont-ils une abstraction ? \\[0pt]
Les droits naturels imposent-ils une limite à la politique ? \\[0pt]
Les échanges économiques sont-ils facteurs de paix ? \\[0pt]
Les échanges, facteurs de paix ? \\[0pt]
Les échanges favorisent-ils la paix ? \\[0pt]
Les échanges sont-ils facteurs de paix ? \\[0pt]
Le sens commun existe-t-il ? \\[0pt]
Le sens de l'opportunité est-il moral ? \\[0pt]
Le sens des mots dépend-il de notre connaissance des choses ? \\[0pt]
Le sens de tout énoncé dépend-il de son interprétation ? \\[0pt]
Le sens du devoir nous fait-il connaître le bien de façon indiscutable ? \\[0pt]
Le sensible est-il communicable ? \\[0pt]
Le sensible est-il irréductible à l'intelligible ? \\[0pt]
Le sensible peut-il être connu ? \\[0pt]
Le sensible peut-il fonder des certitudes ? \\[0pt]
Le sens moral est-il naturel ? \\[0pt]
Le sentiment d'injustice est-il naturel ? \\[0pt]
Les entités mathématiques sont-elles des fictions ? \\[0pt]
Les épreuves ont-elles un sens ? \\[0pt]
Les États ont-ils pour fin d'assurer la paix ? \\[0pt]
Les êtres vivants sont-ils des machines ? \\[0pt]
Les événements historiques sont-ils de nature imprévisible ? \\[0pt]
Les exceptions ont-elles une place en morale ? \\[0pt]
Les faits existent-ils indépendamment de leur établissement par l'esprit humain ? \\[0pt]
Les faits parlent-ils d'eux-mêmes ? \\[0pt]
Les faits peuvent-ils faire autorité ? \\[0pt]
Les faits sociaux se répètent-ils ? \\[0pt]
Les faits sont-ils des preuves ? \\[0pt]
Les faits sont-ils têtus ? \\[0pt]
Les fins de la technique sont-elles techniques ? \\[0pt]
Les fins sont-elles toujours intentionnelles ? \\[0pt]
Les générations futures ont-elles des droits ? \\[0pt]
Les habitudes nous forment-elles ? \\[0pt]
Les hommes croient-ils dans leurs mythes ? \\[0pt]
Les hommes n'agissent-ils que par intérêt ? \\[0pt]
Les hommes naissent-ils libres ? \\[0pt]
Les hommes ont-ils besoin de maîtres ? \\[0pt]
Les hommes peuvent-ils s'associer sans renoncer à leur liberté ? \\[0pt]
Les hommes savent-ils ce qu'ils désirent ? \\[0pt]
Les hommes sont-ils des animaux ? \\[0pt]
Les hommes sont-ils faits pour s'entendre ? \\[0pt]
Les hommes sont-ils frères ? \\[0pt]
Les hommes sont-ils naturellement sociables ? \\[0pt]
Les hommes sont-ils seulement le produit de leur culture ? \\[0pt]
Les hypothèses scientifiques ont-elles pour nature d'être confirmées ou infirmées ? \\[0pt]
Les idées existent-elles ? \\[0pt]
Les idées mènent-elles le monde ? \\[0pt]
Les idées ont-elles une existence éternelle ? \\[0pt]
Les idées ont-elles une histoire ? \\[0pt]
Les idées ont-elles une réalité ? \\[0pt]
Les idées sont-elles vivantes ? \\[0pt]
Le silence a-t-il un sens ? \\[0pt]
Le silence est-il signifiant ? \\[0pt]
Le silence signifie-t-il toujours l'échec du langage ? \\[0pt]
Les images empêchent-elles de penser ? \\[0pt]
Les images nous égarent-elles ? \\[0pt]
Les images ont-elles un sens ? \\[0pt]
Les images peuvent-elles instruire ? \\[0pt]
Les images peuvent-elles mentir ? \\[0pt]
Les individus se réduisent-ils à leurs qualités ? \\[0pt]
Les inégalités de la nature doivent-elles être compensées ? \\[0pt]
Les inégalités menacent-elles la société ? \\[0pt]
Les inégalités sociales sont-elles inévitables ? \\[0pt]
Les inégalités sociales sont-elles naturelles ? \\[0pt]
Le singulier est-il objet de connaissance ? \\[0pt]
Les institutions politiques : qu'ont-elles en propre ? \\[0pt]
Les intérêts particuliers peuvent-ils tempérer l'autorité politique ? \\[0pt]
Les jugements de goût sont-ils susceptibles de vérité ? \\[0pt]
Les langues meurent-elles ? \\[0pt]
Les langues que nous parlons sont-elles imparfaites ? \\[0pt]
Les langues sont-elles condamnées à l'approximation ? \\[0pt]
Les limites de ma langue sont-elles les limites de mon monde ? \\[0pt]
Les lois de la nature sont-elles contingentes ? \\[0pt]
Les lois de la nature sont-elles de simples régularités ? \\[0pt]
Les lois de la nature sont elles nécessaires ? \\[0pt]
Les lois nous rendent-elles meilleurs ? \\[0pt]
Les lois scientifiques sont-elles des lois de la nature ? \\[0pt]
Les lois sont-elles sacrées ? \\[0pt]
Les lois sont-elles seulement utiles ? \\[0pt]
Les lois suffisent-elles à garantir nos libertés ? \\[0pt]
Les machines nous rendent-elles libres ? \\[0pt]
Les machines pensent-elles ? \\[0pt]
Les machines permettent-elles de mieux connaître le corps humain ? \\[0pt]
Les maladies mentales sont-elles des maladies sociales ? \\[0pt]
Les mathématiques consistent-elles seulement en des opérations de l'esprit ? \\[0pt]
Les mathématiques ont-elles affaire au réel ? \\[0pt]
Les mathématiques ont-elles besoin d'un fondement ? \\[0pt]
Les mathématiques parlent-elles du réel ? \\[0pt]
Les mathématiques peuvent-elles se passer de l'intuition ? \\[0pt]
Les mathématiques se réduisent-elles à une pensée cohérente ? \\[0pt]
Les mathématiques sont-elles le modèle de toute science ? \\[0pt]
Les mathématiques sont-elles réductibles à la logique ? \\[0pt]
Les mathématiques sont-elles une découverte ou une invention ? \\[0pt]
Les mathématiques sont-elles une science ? \\[0pt]
Les mathématiques sont-elles une science de l'ordre ? \\[0pt]
Les mathématiques sont-elles un instrument ? \\[0pt]
Les mathématiques sont-elles un jeu de l'esprit ? \\[0pt]
Les mathématiques sont-elles un langage ? \\[0pt]
Les mathématiques sont-elles utiles au philosophe ? \\[0pt]
Les méchants peuvent-ils être amis ? \\[0pt]
Les méchants peuvent-ils faire société ? \\[0pt]
Les mœurs sont-elles objet de science ? \\[0pt]
Les mœurs sont-ils l'objet d'une science ? \\[0pt]
Les mots disent-ils les choses ? \\[0pt]
Les mots expriment-ils les choses ? \\[0pt]
Les mots n'ont-ils que le pouvoir qu'on leur donne ? \\[0pt]
Les mots nous éloignent-ils des autres ? \\[0pt]
Les mots nous éloignent-ils des choses ? \\[0pt]
Les mots ont-ils un pouvoir ? \\[0pt]
Les mots parviennent-ils à tout exprimer ? \\[0pt]
Les mots peuvent-ils nous manquer ? \\[0pt]
Les mots rendent-ils compte de la nature des choses ? \\[0pt]
Les mots sont-ils trompeurs ? \\[0pt]
Les mythes relèvent-ils de l'anthropologie ? \\[0pt]
Les nombres gouvernent-ils le monde ? \\[0pt]
Les noms propres ont-ils une signification ? \\[0pt]
Les normes du savoir sont-elles universelles ? \\[0pt]
Les nouvelles technologies transforment-elles l'idée de l'art ? \\[0pt]
Les objets existent-ils ? \\[0pt]
Les objets sont-ils colorés ? \\[0pt]
Les objets techniques nous imposent-ils une manière de vivre ? \\[0pt]
Les œuvres d'art doivent-elles faire l'objet d'une évaluation morale ? \\[0pt]
Les œuvres d'art ont-elles besoin d'un commentaire ? \\[0pt]
Les œuvres d'art sont-elles des choses ? \\[0pt]
Les œuvres d'art sont-elles des réalités comme les autres ? \\[0pt]
Les œuvres d'art sont-elles éternelles ? \\[0pt]
Le soi est-il inaccessible ? \\[0pt]
Le soleil se lèvera-t-il demain ? \\[0pt]
Le souci d'autrui résume-t-il la morale ? \\[0pt]
Le souci de soi est-il une attitude morale ? \\[0pt]
Le souci du bien-être est-il politique ? \\[0pt]
L'espace nous sépare-t-il ? \\[0pt]
Les passions ont-elles une place en politique ? \\[0pt]
Les passions peuvent-elles être raisonnables ? \\[0pt]
Les passions sont-elles mauvaises ? \\[0pt]
Les passions sont-elles toujours mauvaises ? \\[0pt]
Les passions sont-elles toutes bonnes ? \\[0pt]
Les passions sont-elles un obstacle à la vie sociale ? \\[0pt]
Les passions s'opposent-elles à la raison ? \\[0pt]
L'espérance est-elle une mauvaise passion ? \\[0pt]
L'espérance est-elle une vertu ? \\[0pt]
Les perceptions sont-elles des jugements ? \\[0pt]
Les personnages de fiction peuvent-ils avoir une réalité ? \\[0pt]
Les peuples font-ils l'histoire ? \\[0pt]
Les peuples ont-ils les gouvernements qu'ils méritent ? \\[0pt]
Les phénomènes inconscients sont-ils réductibles à une mécanique cérébrale ? \\[0pt]
Les philosophes doivent-ils être rois ? \\[0pt]
Les philosophies se classent-elles ? \\[0pt]
L'espoir peut-il être raisonnable ? \\[0pt]
Le sport : s'accomplir ou se dépasser ? \\[0pt]
Les principes de la morale dépendent-ils de la culture ? \\[0pt]
Les principes d'une science sont-ils des conventions ? \\[0pt]
Les principes sont-ils indémontrables ? \\[0pt]
L'esprit appartient-il à la nature ? \\[0pt]
L'esprit dépend-il du corps ? \\[0pt]
L'esprit domine-t-il la matière ? \\[0pt]
L'esprit est-il chose ? \\[0pt]
L'esprit est-il irréductible à la matière ? \\[0pt]
L'esprit est-il libre ? \\[0pt]
L'esprit est-il matériel ? \\[0pt]
L'esprit est-il mieux connu que le corps ? \\[0pt]
L'esprit est-il objet de science ? \\[0pt]
L'esprit est-il plus aisé à connaître que le corps ? \\[0pt]
L'esprit est-il plus difficile à connaître que la matière ? \\[0pt]
L'esprit est-il plus facile à connaître que le corps ? \\[0pt]
L'esprit est-il réductible à la matière ? \\[0pt]
L'esprit est-il une chose ? \\[0pt]
L'esprit est-il une machine ? \\[0pt]
L'esprit est-il un ensemble de facultés ? \\[0pt]
L'esprit est-il une partie du corps ? \\[0pt]
L'esprit et la matière peuvent-ils agir l'un sur l'autre ? \\[0pt]
L'esprit humain progresse-t-il ? \\[0pt]
L'esprit n'a-t-il jamais affaire qu'à lui-même ? \\[0pt]
L'esprit peut-il être divisé ? \\[0pt]
L'esprit peut-il être malade ? \\[0pt]
L'esprit peut-il être mesuré ? \\[0pt]
L'esprit peut-il être objet de science ? \\[0pt]
L'esprit s'explique-t-il par une activité cérébrale ? \\[0pt]
Les problèmes politiques peuvent-ils se ramener à des problèmes techniques ? \\[0pt]
Les problèmes politiques sont-ils des problèmes techniques ? \\[0pt]
Les procédures de validation dans les sciences humaines ? \\[0pt]
Les progrès de la technique sont-ils nécessairement des progrès de la raison ? \\[0pt]
Les progrès techniques constituent-ils des progrès de la civilisation ? \\[0pt]
Les propositions métaphysiques sont-elles des illusions ? \\[0pt]
Les proverbes enseignent-ils quelque chose ? \\[0pt]
Les proverbes nous instruisent-ils moralement ? \\[0pt]
Les qualités sensibles sont-elles dans les choses ou dans l'esprit ? \\[0pt]
Les querelles de mots sont-elles toujours stériles ? \\[0pt]
Les questions métaphysiques ont-elles un sens ? \\[0pt]
Les rapports entre les hommes sont-ils des rapports de force ? \\[0pt]
Les rapports sociaux sont-ils des rapports entre individus ? \\[0pt]
Les règles du langage limitent-elles notre liberté d'expression ? \\[0pt]
Les règles du langage sont- elles un obstacle à la communication ? \\[0pt]
Les règles sociales sont-elles répressives ? \\[0pt]
Les relations ont-elles une existence objective ? \\[0pt]
Les religions naissent-elles du besoin de justice ? \\[0pt]
Les religions peuvent-elles être objets de science ? \\[0pt]
Les religions peuvent-elles prétendre libérer les hommes ? \\[0pt]
Les religions sont-elles affaire de foi ? \\[0pt]
Les religions sont-elles des illusions ? \\[0pt]
Les rêves ont-ils un sens ? \\[0pt]
Les rêves sont-ils des pensées ? \\[0pt]
Les révolutions techniques suscitent-elles des révolutions dans l'art ? \\[0pt]
Les scélérats peuvent-ils être heureux ? \\[0pt]
Les sciences décrivent-elles le réel ? \\[0pt]
Les sciences décrivent-elles ou construisent-elles leurs objets ? \\[0pt]
Les sciences de la vie visent-elles un objet irréductible à la matière ? \\[0pt]
Les sciences de l'homme ont-elles inventé leur objet ? \\[0pt]
Les sciences de l'homme permettent-elles d'affiner la notion de responsabilité ? \\[0pt]
Les sciences de l'homme peuvent-elles expliquer l'impuissance de la liberté ? \\[0pt]
Les sciences de l'homme rendent-elles l'homme prévisible ? \\[0pt]
Les sciences de l'homme sont-elles des sciences ? \\[0pt]
Les sciences de l'homme sont-elles utiles à la société ? \\[0pt]
Les sciences doivent-elle prétendre à l'unification ? \\[0pt]
Les sciences forment-elle un système ? \\[0pt]
Les sciences humaines doivent-elles être transdisciplinaires ? \\[0pt]
Les sciences humaines échappent-elles à l'abstraction ? \\[0pt]
Les sciences humaines éliminent-elles la contingence du futur ? \\[0pt]
Les sciences humaines, est-ce la mort de l'homme ? \\[0pt]
Les sciences humaines et sociales n'ont-elles de sciences que le nom ? \\[0pt]
Les sciences humaines finissent-elles de ruiner l'idée de liberté ? \\[0pt]
Les sciences humaines jouent elles un rôle dans l'exploitation de l'homme par l'homme ? \\[0pt]
Les sciences humaines nient-elles la liberté de l'homme ? \\[0pt]
Les sciences humaines nient-elles la liberté du sujet humain ? \\[0pt]
Les sciences humaines nous font-elles connaître l'homme ? \\[0pt]
Les sciences humaines nous obligent-elles à renoncer à l'idée de libre arbitre ? \\[0pt]
Les sciences humaines nous protègent-elles de l'essentialisme ? \\[0pt]
Les sciences humaines ont-elles une utilité ? \\[0pt]
Les sciences humaines ont-elles un objet commun ? \\[0pt]
Les sciences humaines opposent-elles histoire et système ? \\[0pt]
Les sciences humaines permettent-elles de comprendre la vie d'un homme ? \\[0pt]
Les sciences humaines peuvent-elles adopter les méthodes des sciences de la nature ? \\[0pt]
Les sciences humaines peuvent-elles éclairer la réflexion éthique ? \\[0pt]
Les sciences humaines peuvent-elles être objectives ? \\[0pt]
Les sciences humaines peuvent-elles faire abstraction de la nature ? \\[0pt]
Les sciences humaines peuvent-elles s'affranchir de l'ethnocentrisme ? \\[0pt]
Les sciences humaines peuvent-elles se passer de la notion d'inconscient ? \\[0pt]
Les sciences humaines présupposent-elles une définition de l'homme ? \\[0pt]
Les sciences humaines recherchent-elles des invariants ? \\[0pt]
Les sciences humaines rendent-elle l'homme prévisible ? \\[0pt]
Les sciences humaines reposent-elles sur des faits ? \\[0pt]
Les sciences humaines sont-elles des sciences ? \\[0pt]
Les sciences humaines sont-elles des sciences de la nature humaine ? \\[0pt]
Les sciences humaines sont-elles des sciences de la vie humaine ? \\[0pt]
Les sciences humaines sont-elles des sciences de l'esprit ? \\[0pt]
Les sciences humaines sont-elles des sciences de l'interprétation ? \\[0pt]
Les sciences humaines sont-elles des sciences d'interprétation ? \\[0pt]
Les sciences humaines sont-elles explicatives ou compréhensives ? \\[0pt]
Les sciences humaines sont-elles les sciences de l'esprit ? \\[0pt]
Les sciences humaines sont-elles nécessairement empiriques ? \\[0pt]
Les sciences humaines sont-elles normatives ? \\[0pt]
Les sciences humaines sont-elles relativistes ? \\[0pt]
Les sciences humaines sont-elles subversives ? \\[0pt]
Les sciences humaines suffisent-elles à connaître l'homme ? \\[0pt]
Les sciences humaines supposent-elles une définition de l'homme ? \\[0pt]
Les sciences humaines supposent-elles une rupture avec l'ordre naturel ? \\[0pt]
Les sciences humaines traitent-elles de l'homme ? \\[0pt]
Les sciences humaines traitent-elles de l'individu ? \\[0pt]
Les sciences humaines transforment-elles la notion de causalité ? \\[0pt]
Les sciences ne sont-elles qu'une description du monde ? \\[0pt]
Les sciences nous donnent-elles des normes ? \\[0pt]
Les sciences ont-elles à penser leurs fondements ? \\[0pt]
Les sciences ont-elles besoin de principes fondamentaux ? \\[0pt]
Les sciences ont-elles besoin d'une fondation ? \\[0pt]
Les sciences ont-elles besoin d'une fondation métaphysique ? \\[0pt]
Les sciences ont-elles le monopole de la vérité ? \\[0pt]
Les sciences pensent-elles leurs fondements ? \\[0pt]
Les sciences permettent-elles de connaître la réalité-même ? \\[0pt]
Les sciences peuvent-elles exclure toute notion de finalité ? \\[0pt]
Les sciences peuvent-elles penser leurs fondements ? \\[0pt]
Les sciences peuvent-elles penser l'individu ? \\[0pt]
Les sciences peuvent-elles se passer de fondements métaphysiques ? \\[0pt]
Les sciences peuvent-elles se passer d'images ? \\[0pt]
Les sciences produisent-elles des normes ? \\[0pt]
Les sciences sociales ont-elles un objet ? \\[0pt]
Les sciences sociales peuvent-elles être expérimentales ? \\[0pt]
Les sciences sociales peuvent-elles servir à gouverner ? \\[0pt]
Les sciences sociales sont-elles nécessairement inexactes ? \\[0pt]
Les sciences sont-elles une description du monde ? \\[0pt]
Les sensations peuvent-elles se partager ? \\[0pt]
Les sens jugent-ils ? \\[0pt]
Les sens nous trompent-ils ? \\[0pt]
Les sens peuvent-ils nous tromper ? \\[0pt]
Les sens sont-ils source d'illusion ? \\[0pt]
Les sens sont-ils trompeurs ? \\[0pt]
Les sentiments ont-ils une histoire ? \\[0pt]
Les sentiments peuvent-ils s'apprendre ? \\[0pt]
Les sociétés évoluent-elles ? \\[0pt]
Les sociétés humaines se construisent-elles contre la nature ? \\[0pt]
Les sociétés ont-elles besoin de théâtre ? \\[0pt]
Les sociétés ont-elles un inconscient ? \\[0pt]
Les sociétés sont-elles hiérarchisables ? \\[0pt]
Les sociétés sont-elles imprévisibles ? \\[0pt]
Les structures expliquent-elles tout ? \\[0pt]
Les structures que dégagent les sciences humaines agissent-elles ? \\[0pt]
Les structures sociales sont-elles économiques ? \\[0pt]
Les techniques de manipulation des hommes peuvent-elles s'appuyer sur les sciences humaines ? \\[0pt]
Les théories scientifiques décrivent-elles la réalité ? \\[0pt]
Les théories scientifiques sont-elles vraies ? \\[0pt]
L'esthétique est-elle une métaphysique de l'art ? \\[0pt]
Le sublime est-il dans les choses ? \\[0pt]
Le succès peut-il être un critère de vérité ? \\[0pt]
Le sujet n'est-il qu'une fiction ? \\[0pt]
Le sujet peut-il s'aliéner par un libre choix ? \\[0pt]
Les universaux existent-ils ? \\[0pt]
Les valeurs morales ont-elles leur origine dans la raison ? \\[0pt]
Les valeurs ont-elles une histoire ? \\[0pt]
Les valeurs ont-elles une origine ? \\[0pt]
Les vérités scientifiques sont-elles relatives ? \\[0pt]
Les vérités sont-elles intemporelles ? \\[0pt]
Les vérités sont-elles toujours démontrables ? \\[0pt]
Les vertus ne sont-elles que des vices déguisés ? \\[0pt]
Les vertus sont-elles au principe de la morale ? \\[0pt]
Les vices privés peuvent-ils faire le bien public ? \\[0pt]
Les vivants peuvent-ils se passer des morts ? \\[0pt]
Le tableau ? \\[0pt]
Le talent s'enseigne-t-il ? \\[0pt]
L'étant est-il le premier pensable ? \\[0pt]
L'État a-t-il des intérêts propres ? \\[0pt]
L'État a-t-il le droit de contrôler notre habillement ? \\[0pt]
L'État a-t-il pour but de maintenir l'ordre ? \\[0pt]
L'État a-t-il pour finalité de maintenir l'ordre ? \\[0pt]
L'État a-t-il tous les droits ? \\[0pt]
L'État a-t-il un fondement rationnel ? \\[0pt]
L'État contribue-t-il à pacifier les relations entre les hommes ? \\[0pt]
L'État crée-t-il la liberté ? \\[0pt]
L'État de droit ? \\[0pt]
L'État doit-il disparaître ? \\[0pt]
L'État doit-il éduquer le citoyen ? \\[0pt]
L'État doit-il éduquer le peuple ? \\[0pt]
L'État doit-il éduquer les citoyens ? \\[0pt]
L'État doit-il être fort ? \\[0pt]
L'État doit-il être le plus fort ? \\[0pt]
L'État doit-il être neutre ? \\[0pt]
L'État doit-il être sans pitié ? \\[0pt]
L'État doit-il faire le bonheur des citoyens ? \\[0pt]
L'État doit-il nous rendre meilleurs ? \\[0pt]
L'État doit-il préférer l'injustice au désordre ? \\[0pt]
L'État doit-il reconnaître des limites à sa puissance ? \\[0pt]
L'État doit-il se mêler de religion ? \\[0pt]
L'État doit-il se préoccuper des arts ? \\[0pt]
L'État doit-il se préoccuper du bonheur des citoyens ? \\[0pt]
L'État doit-il se soucier de la morale ? \\[0pt]
L'État doit-il veiller au bonheur des individus ? \\[0pt]
L'État est-il appelé à disparaître ? \\[0pt]
L'État est-il au service de la société ? \\[0pt]
L'État est-il ennemi de la liberté ? \\[0pt]
L'État est-il fin ou moyen ? \\[0pt]
L'État est-il la seule source du droit ? \\[0pt]
L'État est-il le garant de la propriété privée ? \\[0pt]
L'État est-il le garant du bien commun ? \\[0pt]
L'État est-il l'ennemi de la liberté ? \\[0pt]
L'État est-il l'ennemi de l'individu ? \\[0pt]
L'État est-il libérateur ? \\[0pt]
L'État est-il nécessaire ? \\[0pt]
L'État est-il notre ennemi ? \\[0pt]
L'État est-il oppresseur ? \\[0pt]
L'État est-il souverain ? \\[0pt]
L'État est-il toujours juste ? \\[0pt]
L'État est-il un arbitre ? \\[0pt]
L'État est-il un mal nécessaire ? \\[0pt]
L'État est-il un moindre mal ? \\[0pt]
L'État est-il un « monstre froid » ? \\[0pt]
L'État est-il un tiers impartial ? \\[0pt]
L'État n'a-t-il d'existence que contractuelle ? \\[0pt]
L'État n'est-il qu'un instrument de domination ? \\[0pt]
L'État n'est-il qu'un moyen ? \\[0pt]
L'État nous rend-il meilleurs ? \\[0pt]
L'État peut-il avoir tous les droits ? \\[0pt]
L'État peut-il créer la liberté ? \\[0pt]
L'État peut-il demeurer indifférent à la religion ? \\[0pt]
L'État peut-il disparaître ? \\[0pt]
L'État peut-il être fondé sur un contrat ? \\[0pt]
L'État peut-il être impartial ? \\[0pt]
L'État peut-il être indifférent à la religion ? \\[0pt]
L'État peut-il être libéral ? \\[0pt]
L'État peut-il fonder sa propre légitimité ? \\[0pt]
L'État peut-il limiter son pouvoir ? \\[0pt]
L'État peut-il poursuivre une autre fin que sa propre puissance ? \\[0pt]
L'État peut-il renoncer à la violence ? \\[0pt]
L'État peut-il se désintéresser de la religion ? \\[0pt]
L'État sert-il l'intérêt général ? \\[0pt]
Le technicien n'est-il qu'un exécutant ? \\[0pt]
Le temps dépend-il de la mémoire ? \\[0pt]
Le temps détruit-il tout ? \\[0pt]
Le temps est-il destructeur ? \\[0pt]
Le temps est-il en nous ou hors de nous ? \\[0pt]
Le temps est-il essentiellement destructeur ? \\[0pt]
Le temps est-il la marque de notre impuissance ? \\[0pt]
Le temps est-il notre allié ? \\[0pt]
Le temps est-il notre ennemi ? \\[0pt]
Le temps est-il une contrainte ? \\[0pt]
Le temps est-il une dimension de la nature ? \\[0pt]
Le temps est-il une expérience de la continuité ou de la discontinuité ? \\[0pt]
Le temps est-il une prison ? \\[0pt]
Le temps est-il une réalité ? \\[0pt]
Le temps existe-t-il ? \\[0pt]
Le temps ne fait-il que passer ? \\[0pt]
Le temps n'est-il pour l'homme que ce qui le limite ? \\[0pt]
Le temps n'existe-t-il que subjectivement ? \\[0pt]
Le temps nous appartient-il ? \\[0pt]
Le temps nous est-il compté ? \\[0pt]
Le temps passe-t-il ? \\[0pt]
Le temps peut-il s'arrêter ? \\[0pt]
Le temps physique est-il comparable au temps psychique ? \\[0pt]
Le temps rend-il tout vain ? \\[0pt]
Le temps s'écoule-t-il ? \\[0pt]
Le temps se laisse-t-il décrire logiquement ? \\[0pt]
Le temps se mesure-t-il ? \\[0pt]
L'éternité est-elle désirable ? \\[0pt]
L'éternité n'est-elle qu'une illusion ? \\[0pt]
Le terrorisme est-il un acte de guerre ? \\[0pt]
L'éthique est-elle affaire de choix ? \\[0pt]
L'éthique suppose-t-elle la liberté ? \\[0pt]
L'ethnologie est-elle une science mélancolique ? \\[0pt]
Le tout est-il la somme de ses parties ? \\[0pt]
Le travail artistique doit-il demeurer caché ? \\[0pt]
Le travail a-t-il une valeur morale ? \\[0pt]
Le travail est-il le propre de l'homme ? \\[0pt]
Le travail est-il libérateur ? \\[0pt]
Le travail est-il nécessaire au bonheur ? \\[0pt]
Le travail est-il toujours une activité productrice ? \\[0pt]
Le travail est-il un besoin ? \\[0pt]
Le travail est-il une fin ? \\[0pt]
Le travail est-il une marchandise ? \\[0pt]
Le travail est-il une valeur ? \\[0pt]
Le travail est-il une valeur morale ? \\[0pt]
Le travail est-il un rapport naturel de l'homme à la nature ? \\[0pt]
Le travail fait-il de l'homme un être moral ? \\[0pt]
Le travail fonde-t-il la propriété ? \\[0pt]
Le travaille libère-t-il ? \\[0pt]
Le travail manuel est-il sans pensée ? \\[0pt]
Le travail nous libère-t-il ? \\[0pt]
Le travail nous rend-il heureux ? \\[0pt]
Le travail nous rend-il solidaires ? \\[0pt]
Le travail peut-il être solitaire ? \\[0pt]
Le travail rapproche-t-il les hommes ? \\[0pt]
Le travail unit-il ou sépare-t-il les hommes ? \\[0pt]
L'être en tant qu'être est-il connaissable ? \\[0pt]
L'être est-il un genre ? \\[0pt]
L'être humain désire-t-il naturellement connaître ? \\[0pt]
L'être humain est-il la mesure de toute chose ? \\[0pt]
L'être se confond-il avec l'être perçu ? \\[0pt]
L'étude de l'histoire conduit-elle à désespérer l'homme ? \\[0pt]
Le vainqueur a-t-il tous les droits ? \\[0pt]
L'événement historique a-t-il un sens par lui-même ? \\[0pt]
L'événement manque-t-il d'être ? \\[0pt]
L'évidence a-t-elle une valeur absolue ? \\[0pt]
L'évidence est-elle critère de vérité ? \\[0pt]
L'évidence est-elle le signe de la vérité ? \\[0pt]
L'évidence est-elle toujours un critère de vérité ? \\[0pt]
L'évidence est-elle un critère de vérité ? \\[0pt]
L'évidence est-elle un obstacle ou un instrument de la recherche de la vérité ? \\[0pt]
L'évidence se passe-t-elle de démonstration ? \\[0pt]
Le virtuel existe-t-il ? \\[0pt]
Le visage n'est-il qu'un masque ? \\[0pt]
Le vivant a-t-il des droits ? \\[0pt]
Le vivant a-t-il une temporalité propre ? \\[0pt]
Le vivant définit-il ses propres normes ? \\[0pt]
Le vivant échappe-t-il à la connaissance ? \\[0pt]
Le vivant échappe-t-il au déterminisme ? \\[0pt]
Le vivant est-il entièrement connaissable ? \\[0pt]
Le vivant est-il entièrement explicable ? \\[0pt]
Le vivant est-il l'affaire du politique ? \\[0pt]
Le vivant est-il réductible au physico-chimique ? \\[0pt]
Le vivant est-il un objet de science comme un autre ? \\[0pt]
Le vivant n'est-il que matière ? \\[0pt]
Le vivant n'est-il qu'une machine ingénieuse ? \\[0pt]
Le vivant obéit-il à des lois ? \\[0pt]
Le vivant obéit-il à une nécessité ? \\[0pt]
Le vivant se définit-il par sa résistance à la mort ? \\[0pt]
Le vrai admet-il des degrés ? \\[0pt]
Le vrai a-t-il une histoire ? \\[0pt]
Le vrai doit-il être démontré ? \\[0pt]
Le vrai est-il à lui-même sa propre marque ? \\[0pt]
Le vrai et le bien sont-ils analogues ? \\[0pt]
Le vrai n'est-il que ce sur quoi l'on s'accorde ? \\[0pt]
Le vrai peut-il rester invérifiable ? \\[0pt]
Le vraisemblable présente-t-il un intérêt pour la pensée ? \\[0pt]
Le vrai se perçoit-il ? \\[0pt]
Le vrai se réduit-il à ce qui est vérifiable ? \\[0pt]
Le vrai se réduit-il à l'utile ? \\[0pt]
L'exception est-elle instructive ? \\[0pt]
L'exception peut-elle confirmer la règle ? \\[0pt]
L'exécution d'une œuvre d'art est-elle toujours une œuvre d'art ? \\[0pt]
L'exercice du pouvoir est-il nécessairement solitaire ? \\[0pt]
L'exigence de vérité a-t-elle un sens moral ? \\[0pt]
L'existence a-t-elle un sens ? \\[0pt]
L'existence de l'État dépend-elle d'un contrat ? \\[0pt]
L'existence du mal met-elle en échec la raison ? \\[0pt]
L'existence est-elle pensable ? \\[0pt]
L'existence est-elle une propriété ? \\[0pt]
L'existence est-elle un jeu ? \\[0pt]
L'existence est-elle vaine ? \\[0pt]
L'existence se démontre-t-elle ? \\[0pt]
L'existence se laisse-t-elle penser ? \\[0pt]
L'existence se prouve-t-elle ? \\[0pt]
L'existence suppose-t-elle la conscience du temps ? \\[0pt]
L'expérience a-t-elle le même sens dans toutes les sciences ? \\[0pt]
L'expérience d'autrui nous est-elle utile ? \\[0pt]
L'expérience de la beauté est-elle naturelle ? \\[0pt]
L'expérience de la vie produit-elle un savoir ? \\[0pt]
L'expérience démontre-t-elle quelque chose ? \\[0pt]
L'expérience directe est-elle une connaissance ? \\[0pt]
L'expérience du beau peut-elle être une expérience métaphysique ? \\[0pt]
L'expérience enseigne-elle quelque chose ? \\[0pt]
L'expérience, est-ce l'observation ? \\[0pt]
L'expérience est-elle la seule source de nos connaissances ? \\[0pt]
L'expérience est-elle un guide suffisant ? \\[0pt]
L'expérience esthétique peut-elle se communiquer ? \\[0pt]
L'expérience esthétique relève-t-elle de la contemplation ? \\[0pt]
L'expérience instruit-elle ? \\[0pt]
L'expérience nous apprend-elle quelque chose ? \\[0pt]
L'expérience permet-elle de départager des théories concurrentes ? \\[0pt]
L'expérience peut-elle avoir raison des principes ? \\[0pt]
L'expérience peut-elle contredire la théorie ? \\[0pt]
L'expérience rend-elle raisonnable ? \\[0pt]
L'expérience rend-elle responsable ? \\[0pt]
L'expérience sensible est-elle la seule source légitime de connaissance ? \\[0pt]
L'expérience suffit-elle pour établir une vérité ? \\[0pt]
L'expérimentation est-elle facteur de vérité ? \\[0pt]
L'expression des citoyens dans le vote suffit-elle à faire la démocratie ? \\[0pt]
L'expression des citoyens dans le vote suffit-elle à faire vivre la démocratie ? \\[0pt]
L'expression « perdre son temps » a-t-elle un sens ? \\[0pt]
L'expression peut-elle être libre ? \\[0pt]
L'extinction du désir est-elle le commencement du bonheur ? \\[0pt]
L'habitude a-t-elle des vertus ? \\[0pt]
L'habitude est-elle notre guide dans la vie ? \\[0pt]
L'habitude est-elle une force ? \\[0pt]
L'habitude est-elle une seconde nature ? \\[0pt]
L'histoire a-t-elle des lois ? \\[0pt]
L'Histoire a-t-elle un commencement ? \\[0pt]
L'histoire a-t-elle un commencement et une fin ? \\[0pt]
L'histoire a-t-elle une fin ? \\[0pt]
L'histoire a-t-elle un sens ? \\[0pt]
L'histoire de l'art est-elle celle des styles ? \\[0pt]
L'histoire de l'art est-elle finie ? \\[0pt]
L'histoire de l'art peut-elle arriver à son terme ? \\[0pt]
L'histoire des arts est-elle liée à l'histoire des techniques ? \\[0pt]
L'histoire des sciences est-elle une histoire ? \\[0pt]
L'histoire du droit est-elle celle du progrès de la justice ? \\[0pt]
L'histoire : enquête ou science ? \\[0pt]
L'histoire est-elle avant tout mémoire ? \\[0pt]
L'histoire est-elle cyclique ? \\[0pt]
L'histoire est-elle déterministe ? \\[0pt]
L'histoire est-elle écrite par les vainqueurs ? \\[0pt]
L'histoire est-elle la connaissance du passé humain ? \\[0pt]
L'histoire est-elle la mémoire de l'humanité ? \\[0pt]
L'histoire est-elle la science de ce qui ne se répète jamais ? \\[0pt]
L'histoire est-elle la science de l'événementiel ? \\[0pt]
L'histoire est-elle la science du passé ? \\[0pt]
L'histoire est-elle la vérité de la mémoire ? \\[0pt]
L'histoire est-elle le récit objectif des faits passés ? \\[0pt]
L'histoire est-elle le règne du hasard ? \\[0pt]
L'histoire est-elle le théâtre des passions ? \\[0pt]
L'histoire est-elle rationnelle ? \\[0pt]
L'histoire est-elle sans fin ? \\[0pt]
L'histoire est-elle tragique ? \\[0pt]
L'histoire est-elle une explication ou une justification du passé ? \\[0pt]
L'histoire est-elle une science ? \\[0pt]
L'histoire est-elle une science comme les autres ? \\[0pt]
L'histoire est-elle une science de l'individuel ? \\[0pt]
L'histoire est-elle un genre littéraire ? \\[0pt]
L'histoire est-elle un récit ? \\[0pt]
L'histoire est-elle un roman ? \\[0pt]
L'histoire est-elle un roman vrai ? \\[0pt]
L'histoire est-elle utile ? \\[0pt]
L'histoire est-elle utile à la politique ? \\[0pt]
« L'histoire jugera » : quel sens faut-il accorder à cette expression ? \\[0pt]
L'histoire jugera-t-elle ? \\[0pt]
L'histoire n'a-t-elle pour objet que le passé ? \\[0pt]
L'histoire n'est-elle que bruit et fureur ? \\[0pt]
L'histoire n'est-elle que la connaissance du passé ? \\[0pt]
L'histoire n'est-elle que le produit de rapports de force ? \\[0pt]
L'histoire n'est-elle qu'un déchaînement de violence ? \\[0pt]
L'histoire n'est-elle qu'une suite d'événements ? \\[0pt]
L'histoire n'est-elle qu'un récit ? \\[0pt]
L'histoire nous appartient-elle ? \\[0pt]
L'histoire obéit-elle à des lois ? \\[0pt]
L'histoire pèse-t-elle comme un destin sur les individus ? \\[0pt]
L'histoire peut-elle être contemporaine ? \\[0pt]
L'histoire peut-elle être objective ? \\[0pt]
L'histoire peut-elle être universelle ? \\[0pt]
L'histoire peut-elle ne pas avoir de sens ? \\[0pt]
L'histoire peut-elle s'écrire au présent ? \\[0pt]
L'histoire peut-elle se répéter ? \\[0pt]
L'histoire : science ou récit ? \\[0pt]
L'histoire se réduit-elle à l'étude du passé ? \\[0pt]
L'histoire se répète-t-elle ? \\[0pt]
L'histoire sert-elle le présent ? \\[0pt]
L'histoire universelle est-elle l'histoire des guerres ? \\[0pt]
L'historien est-il un homme qui se souvient du passé ? \\[0pt]
L'historien peut-il être impartial ? \\[0pt]
L'historien peut-il se passer du concept de causalité ? \\[0pt]
L'homme aime-t-il la justice pour elle-même ? \\[0pt]
L'homme a-t-il besoin de l'art ? \\[0pt]
L'homme a-t-il besoin de religion ? \\[0pt]
L'homme a-t-il une nature ? \\[0pt]
L'homme a-t-il une place dans la nature ? \\[0pt]
L'homme des droits de l'homme n'est-il qu'une fiction ? \\[0pt]
L'homme est-il chez lui dans l'univers ? \\[0pt]
L'homme est-il face à la nature ? \\[0pt]
L'homme est-il fait pour le travail ? \\[0pt]
L'homme est-il la mesure de toute chose ? \\[0pt]
L'homme est-il la mesure de toutes choses ? \\[0pt]
L'homme est-il l'artisan de sa dignité ? \\[0pt]
L'homme est-il le seul être à avoir une histoire ? \\[0pt]
L'homme est-il le seul sujet de droit ? \\[0pt]
L'homme est-il le sujet de son histoire ? \\[0pt]
L'homme est-il libre par nature ? \\[0pt]
L'homme est-il meilleur seul ou en société ? \\[0pt]
L'homme est-il naturellement artiste ? \\[0pt]
L'homme est-il objet de science ? \\[0pt]
L'homme est-il par nature un être religieux ? \\[0pt]
L'homme est-il prisonnier du temps ? \\[0pt]
L'homme est-il raisonnable par nature ? \\[0pt]
L'homme est-il religieux par nature ? \\[0pt]
L'homme est-il un animal ? \\[0pt]
L'homme est-il un animal comme les autres ? \\[0pt]
L'homme est-il un animal comme un autre ? \\[0pt]
L'homme est-il un animal dénaturé ? \\[0pt]
L'homme est-il un animal métaphysique ? \\[0pt]
L'homme est-il un animal politique ? \\[0pt]
L'homme est-il un animal rationnel ? \\[0pt]
L'homme est-il un animal religieux ? \\[0pt]
L'homme est-il un animal social ? \\[0pt]
L'homme est-il un animal symbolique ? \\[0pt]
L'homme est-il un animal technicien ? \\[0pt]
L'homme est-il un corps pensant ? \\[0pt]
L'homme est-il un être de devoir ? \\[0pt]
L'homme est-il un être inachevé ? \\[0pt]
L'homme est-il un être social par nature ? \\[0pt]
L'homme est-il un loup pour l'homme ? \\[0pt]
L'homme est-il vraiment un animal politique ? \\[0pt]
L'homme et la nature sont-ils commensurables ? \\[0pt]
L'homme fait-il partie de la nature ? \\[0pt]
L'homme injuste est-il heureux ? \\[0pt]
L'homme injuste peut-il être heureux ? \\[0pt]
L'homme injuste peut-il être malheureux ? \\[0pt]
L'homme libre est-il un homme seul ? \\[0pt]
L'homme n'est-il qu'un animal comme les autres ? \\[0pt]
L'homme pense-t-il toujours ? \\[0pt]
L'homme peut-il changer ? \\[0pt]
L'homme peut-il être inhumain ? \\[0pt]
L'homme peut-il être libéré du besoin ? \\[0pt]
L'homme peut-il être objet d'expérience ? \\[0pt]
L'homme peut-il faire l'objet d'une science ? \\[0pt]
L'homme peut-il se représenter un monde sans l'homme ? \\[0pt]
L'homme se réalise-t-il dans le travail ? \\[0pt]
L'homme se reconnaît-il mieux dans le travail ou dans le loisir ? \\[0pt]
L'homme trouble-t-il l'ordre de la nature ? \\[0pt]
L'homme vertueux doit-il s'attendre à être malheureux ? \\[0pt]
L'homo œconomicus est-il rationnel ? \\[0pt]
L'hospitalité a-t-elle un sens politique ? \\[0pt]
L'hospitalité est-elle un devoir ? \\[0pt]
L'hospitalité est-elle une vertu politique ? \\[0pt]
L'humain est-il un animal irrationnel ? \\[0pt]
L'humanité a-t-elle eu une enfance ? \\[0pt]
L'humanité a-t-elle une histoire ? \\[0pt]
L'humanité est-elle aimable ? \\[0pt]
L'humanité : un concept normatif ? \\[0pt]
L'humilité est-elle une vertu ? \\[0pt]
L'hypothèse de la liberté est-elle compatible avec les exigences de la raison ? \\[0pt]
Libre arbitre et déterminisme sont-ils compatibles ? \\[0pt]
L'idéal moral est-il vain ? \\[0pt]
L'idée de bonheur collectif a-t-elle un sens ? \\[0pt]
L'idée de destin a-t-elle un sens ? \\[0pt]
L'idée de destin est-elle une idée périmée ? \\[0pt]
L'idée de devoir requiert-elle l'idée de liberté ? \\[0pt]
L'idée de « nature » n'est-elle qu'un mythe ? \\[0pt]
L'idée de progrès historique est-elle devenue désuète ? \\[0pt]
L'idée de rétribution est-elle nécessaire à la morale ? \\[0pt]
L'idée d'une liberté totale a-t- elle un sens ? \\[0pt]
L'idée d'une pensée inconsciente est-elle contradictoire ? \\[0pt]
L'idée d'une religion personnelle a-t-elle un sens ? \\[0pt]
L'identité personnelle est-elle donnée ou construite ? \\[0pt]
L'identité relève-t-elle du champ politique ? \\[0pt]
L'ignorance est-elle la raison du mal ? \\[0pt]
L'ignorance est-elle préférable à l'erreur ? \\[0pt]
L'ignorance est-elle une excuse ? \\[0pt]
L'ignorance nous excuse-t-elle ? \\[0pt]
L'ignorance peut-elle être une excuse ? \\[0pt]
L'illusion est-elle nécessaire au bonheur des hommes ? \\[0pt]
L'imagination a-t-elle des limites ? \\[0pt]
L'imagination a-t-elle sa place dans la connaissance scientifique ? \\[0pt]
L'imagination enrichit-elle la connaissance ? \\[0pt]
L'imagination est-elle créatrice ? \\[0pt]
L'imagination est-elle dangereuse ? \\[0pt]
L'imagination est-elle le refuge de la liberté ? \\[0pt]
L'imagination est-elle libre ? \\[0pt]
L'imagination est-elle maîtresse d'erreur et de fausseté ? \\[0pt]
L'imagination nous éloigne-t-elle du réel ? \\[0pt]
L'imagination nous instruit-elle ? \\[0pt]
L'imitation a-t-elle une fonction morale ? \\[0pt]
L'impartialité est-elle toujours désirable ? \\[0pt]
L'impossible est-il concevable ? \\[0pt]
L'incertitude est-elle dans les choses ou dans les idées ? \\[0pt]
L'incertitude interdit-elle de raisonner ? \\[0pt]
L'inconscience peut-elle être coupable ? \\[0pt]
L'inconscient a-t-il son propre langage ? \\[0pt]
L'inconscient a-t-il une histoire ? \\[0pt]
L'inconscient est-il dans l'âme ou dans le corps ? \\[0pt]
L'inconscient est-il l'animal en nous ? \\[0pt]
L'inconscient est-il pure négation de la conscience ? \\[0pt]
L'inconscient est-il un concept scientifique ? \\[0pt]
L'inconscient est-il un destin ? \\[0pt]
L'inconscient est-il une découverte ou une invention ? \\[0pt]
L'inconscient est-il une dimension de la conscience ? \\[0pt]
L'inconscient est-il une excuse ? \\[0pt]
L'inconscient est-il un obstacle à la liberté ? \\[0pt]
L'inconscient n'est-il qu'un défaut de conscience ? \\[0pt]
L'inconscient n'est-il qu'une hypothèse ? \\[0pt]
L'inconscient nous révèle-t-il à nous-même ? \\[0pt]
L'inconscient peut-il se manifester ? \\[0pt]
L'inconscient psychanalytique réfute-t-il la notion philosophique de conscience ? \\[0pt]
L'indifférence peut-elle être une vertu ? \\[0pt]
L'individualisme a-t-il sa place en politique ? \\[0pt]
L'individualisme est-il un égoïsme ? \\[0pt]
L'individualisme rend-il la politique impossible ? \\[0pt]
L'individu a-t-il des droits ? \\[0pt]
L'individu échappe-t-il aux sciences humaines ? \\[0pt]
L'individu est-il définissable ? \\[0pt]
L'individu est-il ineffable ? \\[0pt]
L'individu particulier peut-il être l'objet d'une connaissance sociologique ? \\[0pt]
L'inégalité a-t-elle des vertus ? \\[0pt]
L'inégalité est-elle nécessairement injuste ? \\[0pt]
L'inexact peut-il avoir une valeur ? \\[0pt]
L'infini est-il la négation du fini ? \\[0pt]
L'infini se réduit-il à l'indéfini ? \\[0pt]
L'infinité de l'univers a-t-elle de quoi nous effrayer ? \\[0pt]
L'injustice est-elle préférable au désordre ? \\[0pt]
L'innovation technique nous impose-t-elle de nouveaux devoirs ? \\[0pt]
L'inquiétude peut-elle définir l'existence humaine ? \\[0pt]
L'inquiétude peut-elle devenir l'existence humaine ? \\[0pt]
L'instant a-t-il une réalité ? \\[0pt]
L'instant de la décision est-il une folie ? \\[0pt]
L'instruction est-elle facteur de moralité ? \\[0pt]
L'insurrection est-elle un droit ? \\[0pt]
L'intelligence peut-elle être artificielle ? \\[0pt]
L'intelligence peut-elle être inhumaine ? \\[0pt]
L'intention morale suffit-elle à constituer la valeur morale de l'action ? \\[0pt]
L'interdit a-t-il une origine sociale ? \\[0pt]
L'interdit est-il au fondement de la culture ? \\[0pt]
L'intérêt constitue-t-il l'unique lien social ? \\[0pt]
L'intérêt de la société l'emporte-t-il sur celui des individus ? \\[0pt]
L'intérêt est-il le principe de tout échange ? \\[0pt]
L'intérêt général est-il la somme des intérêts particuliers ? \\[0pt]
L'intérêt général est-il le bien commun ? \\[0pt]
L'intérêt général est-il un mythe ? \\[0pt]
L'intérêt général n'est-il qu'un mythe ? \\[0pt]
L'intérêt gouverne-t-il le monde ? \\[0pt]
L'intérêt peut-il avoir une valeur morale ? \\[0pt]
L'intérêt peut-il être général ? \\[0pt]
L'intérêt peut-il être une valeur morale ? \\[0pt]
L'intérêt public est-il une illusion ? \\[0pt]
L'intériorité est-elle un mythe ? \\[0pt]
L'interprétation est-elle sans fin ? \\[0pt]
L'interprétation est-elle un art ? \\[0pt]
L'interprétation est-elle un art ou une science ? \\[0pt]
L'interprétation est-elle une activité sans fin ? \\[0pt]
L'interprétation est-elle une forme de connaissance ? \\[0pt]
L'interprétation est-elle une science ? \\[0pt]
L'interprétation : mode d'être ou mode de connaissance ? \\[0pt]
L'interprète est-il un créateur ? \\[0pt]
L'interprète sait-il ce qu'il cherche ? \\[0pt]
L'intime est-il politique ? \\[0pt]
L'introspection est-elle une connaissance ? \\[0pt]
L'intuition a-t-elle une place en logique ? \\[0pt]
L'inutile a-t-il de la valeur ? \\[0pt]
L'inutile est-il sans valeur ? \\[0pt]
L'irrationnel est-il nécessairement absurde ? \\[0pt]
L'irrationnel est-il pensable ? \\[0pt]
L'irrationnel est-il toujours absurde ? \\[0pt]
L'irrationnel existe-t-il ? \\[0pt]
L'obéissance est-elle compatible avec la liberté ? \\[0pt]
L'obéissance peut-elle être un acte de liberté ? \\[0pt]
L'objectivité existe-t-elle ? \\[0pt]
L'objectivité historique est-elle synonyme de neutralité ? \\[0pt]
L'objet du désir en est-il la cause ? \\[0pt]
L'obligation morale peut-elle se réduire à une obligation sociale ? \\[0pt]
L'œuvre d'art a-t-elle un sens ? \\[0pt]
L'œuvre d'art doit-elle être belle ? \\[0pt]
L'œuvre d'art doit-elle nous émouvoir ? \\[0pt]
L'œuvre d'art donne-t-elle à penser ? \\[0pt]
L'œuvre d'art échappe-t-elle au monde ? \\[0pt]
L'œuvre d'art échappe-t-elle au temps ? \\[0pt]
L'œuvre d'art échappe-t-elle nécessairement au temps ? \\[0pt]
L'œuvre d'art est-elle anhistorique ? \\[0pt]
L'œuvre d'art est-elle intemporelle ? \\[0pt]
L'œuvre d'art est-elle l'expression d'une idée ? \\[0pt]
L'œuvre d'art est-elle toujours destinée à un public ? \\[0pt]
L'œuvre d'art est-elle une belle apparence ? \\[0pt]
L'œuvre d'art est-elle une marchandise ? \\[0pt]
L'œuvre d'art est-elle un monde ? \\[0pt]
L'œuvre d'art est-elle un objet d'échange ? \\[0pt]
L'œuvre d'art est-elle un symbole ? \\[0pt]
L'œuvre d'art instruit-elle ? \\[0pt]
L'œuvre d'art nous apprend-elle quelque chose ? \\[0pt]
L'œuvre d'art représente-t-elle quelque chose ? \\[0pt]
L'œuvre d'art traduit-elle une vision du monde ? \\[0pt]
L'ontologie peut-elle être relative ? \\[0pt]
L'opéra est-il un art complet ? \\[0pt]
L'opinion a-t-elle nécessairement tort ? \\[0pt]
L'opinion est-elle un savoir ? \\[0pt]
L'ordinaire est-il ennuyeux ? \\[0pt]
L'ordre des échanges est-il un ordre rationnel ? \\[0pt]
L'ordre du vivant est-il façonné par le hasard ? \\[0pt]
L'ordre est-il dans les choses ? \\[0pt]
L'ordre politique exclut-il la violence ? \\[0pt]
L'ordre politique peut-il exclure la violence ? \\[0pt]
L'ordre social peut-il être juste ? \\[0pt]
L'origine des langues est-elle un faux problème ? \\[0pt]
L'oubli est-il nécessaire à la vie ? \\[0pt]
L'oubli est-il un échec de la mémoire ? \\[0pt]
L'oubli n'est-il qu'une perte de mémoire ? \\[0pt]
L'unanimité est-elle un critère de légitimité ? \\[0pt]
L'unanimité est-elle un critère de vérité ? \\[0pt]
L'unité des sciences humaines ? \\[0pt]
L'universel est-il une illusion ? \\[0pt]
L'universel implique-t-il l'indifférence à la singularité ? \\[0pt]
L'utilité est-elle étrangère à la morale ? \\[0pt]
L'utilité est-elle une valeur morale ? \\[0pt]
L'utilité peut-elle être le principe de la moralité ? \\[0pt]
L'utilité peut-elle être un critère pour juger de la valeur de nos actions ? \\[0pt]
L'utopie a-t-elle une signification politique ? \\[0pt]
L'utopie a-t-elle un lien avec la pratique ? \\[0pt]
Ma conscience est-elle digne de confiance ? \\[0pt]
Ma liberté s'arrête-t-elle où commence celle des autres ? \\[0pt]
Ma parole m'engage-t-elle ? \\[0pt]
Mérite-t-on d'être heureux ? \\[0pt]
Mon corps est-il ma propriété ? \\[0pt]
Mon corps est-il naturel ? \\[0pt]
Mon corps fait-il obstacle à ma liberté ? \\[0pt]
Mon corps m'appartient-il ? \\[0pt]
Mon devoir dépend-il de moi ? \\[0pt]
Mon existence est-elle contingente ? \\[0pt]
Mon passé m'appartient-il ? \\[0pt]
Mon prochain est-il mon semblable ? \\[0pt]
Montrer, est-ce démontrer ? \\[0pt]
Morale et politique sont-elles indépendantes ? \\[0pt]
Naît-on sujet ou le devient-on ? \\[0pt]
N'apprend-on que par l'expérience ? \\[0pt]
N'a-t-on des devoirs qu'envers autrui ? \\[0pt]
N'avons-nous affaire qu'au réel ? \\[0pt]
N'avons-nous de devoir qu'envers autrui ? \\[0pt]
N'avons-nous de devoirs qu'envers autrui ? \\[0pt]
N'échange-t-on que ce qui a de la valeur ? \\[0pt]
N'échange-t-on que des biens ? \\[0pt]
N'échange-t-on que des symboles ? \\[0pt]
N'échange-t-on que par intérêt ? \\[0pt]
Ne désire-t-on que ce dont on manque ? \\[0pt]
Ne désirons-nous que ce qui est bon pour nous ? \\[0pt]
Ne désirons-nous que les choses que nous estimons bonnes ? \\[0pt]
Ne doit-on tenir pour vrai que ce qui est démontré ? \\[0pt]
Ne faut-il pas craindre la liberté ? \\[0pt]
Ne faut-il vivre que dans le présent ? \\[0pt]
Ne faut-il vivre que pour faire son devoir ? \\[0pt]
Ne prêche-t-on que les convertis ? \\[0pt]
Ne sait-on rien que par expérience ? \\[0pt]
Ne sommes-nous justes que par intérêt ? \\[0pt]
Ne sommes-nous véritablement maîtres que de nos pensées ? \\[0pt]
N'est-il d'inférence que démonstrative ? \\[0pt]
N'est-on juste que par crainte du châtiment ? \\[0pt]
Ne travaille-t-on que par nécessité ? \\[0pt]
Ne travaille-t- on que pour un salaire ? \\[0pt]
Ne veut-on que ce qui est désirable ? \\[0pt]
Ne vit-on bien qu'avec ses amis ? \\[0pt]
N'existe-t-il que des individus ? \\[0pt]
N'existe-t-il que le présent ? \\[0pt]
N'existe-t-il qu'un seul temps ? \\[0pt]
N'exprime-t-on que ce dont on a conscience ? \\[0pt]
Nier la liberté de l'homme, est-ce ôter tout sens à la morale ? \\[0pt]
Nier le libre arbitre, est-ce nier la liberté ? \\[0pt]
N'interprète-t-on que ce qui est équivoque ? \\[0pt]
Nos convictions morales sont-elles le simple reflet de notre temps ? \\[0pt]
Nos désirs nous appartiennent-ils ? \\[0pt]
Nos désirs nous opposent-ils ? \\[0pt]
Nos devoirs nous motivent-ils ? \\[0pt]
Nos devoirs se ramènent-ils aux conventions sociales ? \\[0pt]
Nos goûts sont-ils justifiables ? \\[0pt]
Nos habitudes compromettent-elles notre moralité ? \\[0pt]
Nos paroles peuvent-elles dépasser notre pensée ? \\[0pt]
Nos pensées dépendent-elles de nous ? \\[0pt]
Nos pensées sont-elles entièrement en notre pouvoir ? \\[0pt]
Nos sens nous font-ils connaître la matière ? \\[0pt]
Nos sens nous trompent-ils ? \\[0pt]
Nos sens peuvent-ils s'éduquer ? \\[0pt]
Notre connaissance du réel se limite-t-elle au savoir scientifique ? \\[0pt]
Notre corps pense-t-il ? \\[0pt]
Notre existence a-t-elle un sens si l'histoire n'en a pas ? \\[0pt]
Notre ignorance nous excuse-t-elle ? \\[0pt]
Notre liberté de pensée a-t-elle des limites ? \\[0pt]
Notre liberté est-elle toujours relative ? \\[0pt]
Notre pensée est-elle prisonnière de la langue que nous parlons ? \\[0pt]
Notre rapport au monde est-il essentiellement technique ? \\[0pt]
Notre rapport au monde peut-il être exclusivement technique ? \\[0pt]
Notre rapport au monde peut-il n'être que technique ? \\[0pt]
Nous trouvons-nous nous-mêmes dans l'animal ? \\[0pt]
N'y a-t-il d'amitié qu'entre égaux ? \\[0pt]
N'y a-t-il de beauté qu'artistique ? \\[0pt]
N'y a-t-il de bonheur que dans l'instant ? \\[0pt]
N'y a-t-il de bonheur qu'éphémère ? \\[0pt]
N'y a-t-il de certitude que mathématique ? \\[0pt]
N'y a-t-il de connaissance que de l'universel ? \\[0pt]
N'y a-t-il de connaissance que par les causes ? \\[0pt]
N'y a-t-il de connaissance que scientifique ? \\[0pt]
N'y a-t-il de démocratie que représentative ? \\[0pt]
N'y a-t-il de devoirs qu'envers autrui ? \\[0pt]
N'y a-t-il de droit qu'écrit ? \\[0pt]
N'y a-t-il de droit que de l'homme ? \\[0pt]
N'y a-t-il de droits que de l'homme ? \\[0pt]
N'y a-t-il de foi que religieuse ? \\[0pt]
N'y a-t-il de liberté qu'individuelle ? \\[0pt]
N'y a-t-il de propriété que privée ? \\[0pt]
N'y a-t-il de rationalité que scientifique ? \\[0pt]
N'y a-t-il de réalité que de l'individuel ? \\[0pt]
N'y a-t-il de réel que le présent ? \\[0pt]
N'y a-t-il de savoir que livresque ? \\[0pt]
N'y a-t-il de science qu'autant qu'il s'y trouve de mathématique ? \\[0pt]
N'y a-t-il de science que de ce qui est mathématisable ? \\[0pt]
N'y a-t-il de science que du général ? \\[0pt]
N'y a-t-il de science que du mesurable ? \\[0pt]
N'y a-t-il de science que du nécessaire ? \\[0pt]
N'y a-t-il de science que provisoire ? \\[0pt]
N'y a-t-il de science que systématique ? \\[0pt]
N'y a-t-il de science qu'exacte ? \\[0pt]
N'y a-t-il des droits que de l'homme ? \\[0pt]
N'y a-t-il de sens que par le langage ? \\[0pt]
N'y a-t-il d'être que sensible ? \\[0pt]
N'y a-t-il de vérité que scientifique ? \\[0pt]
N'y a-t-il de vérité que vérifiable ? \\[0pt]
N'y a-t-il de vérités que scientifiques ? \\[0pt]
N'y a-t-il de vrai que le vérifiable ? \\[0pt]
N'y a-t-il d'origine que mythique ? \\[0pt]
N'y a-t-il que des individus ? \\[0pt]
N'y a-t-il que du mesurable ? \\[0pt]
N'y a-t-il que l'intention qui compte ? \\[0pt]
N'y a-t-il qu'une substance ? \\[0pt]
N'y a-t-il qu'un seul monde ? \\[0pt]
N'y a-t-il rien au monde de certain ? \\[0pt]
Obéir aux lois, est-ce renoncer à la liberté ? \\[0pt]
Obéir, est-ce se soumettre ? \\[0pt]
Par le langage, peut-on agir sur la réalité ? \\[0pt]
Parler de Dieu, est-ce nécessairement tenir un discours religieux ? \\[0pt]
Parler de soi est-il intéressant ? \\[0pt]
Parler, est-ce agir ? \\[0pt]
Parler, est-ce communiquer ? \\[0pt]
Parler, est-ce donner sa parole ? \\[0pt]
Parler, est-ce ne pas agir ? \\[0pt]
Parler, n'est-ce que désigner ? \\[0pt]
Par où commencer ? \\[0pt]
Par quoi un individu diffère-t-il réellement d'un autre ? \\[0pt]
Par quoi un individu se distingue-t-il d'un autre ? \\[0pt]
« Pas de liberté pour les ennemis de la liberté » ? \\[0pt]
Peindre, est-ce nécessairement feindre ? \\[0pt]
Peindre, est-ce traduire ? \\[0pt]
Peindre un paysage, est-ce représenter la nature ? \\[0pt]
Penser, est-ce aller contre l'évidence ? \\[0pt]
Penser, est-ce calculer ? \\[0pt]
Penser, est-ce comparer ? \\[0pt]
Penser, est-ce désobéir ? \\[0pt]
Penser, est-ce dire non ? \\[0pt]
Penser, est-ce se parler à soi-même ? \\[0pt]
Penser est-il assimilable à un travail ? \\[0pt]
Penser le rien, est-ce ne rien penser ? \\[0pt]
Penser par soi-même, est-ce être l'auteur de ses pensées ? \\[0pt]
Penser peut-il nous rendre heureux ? \\[0pt]
Penser requiert-il d'avoir un corps ? \\[0pt]
Penser requiert-il un corps ? \\[0pt]
Pense-t-on jamais seul ? \\[0pt]
Pensez-vous que vous avez une âme ? \\[0pt]
Percevoir, est-ce connaître ? \\[0pt]
Percevoir, est-ce interpréter ? \\[0pt]
Percevoir, est-ce juger ? \\[0pt]
Percevoir, est-ce nécessaire pour penser ? \\[0pt]
Percevoir, est-ce reconnaître ? \\[0pt]
Percevoir, est-ce savoir ? \\[0pt]
Percevoir, est-ce s'ouvrir au monde ? \\[0pt]
Percevoir s'apprend-il ? \\[0pt]
Percevons-nous le monde extérieur ? \\[0pt]
Percevons-nous les choses telles qu'elles sont ? \\[0pt]
Perçoit-on le réel ? \\[0pt]
Perçoit-on le réel tel qu'il est ? \\[0pt]
Perçoit-on les choses comme elles sont ? \\[0pt]
Peut-il être moral de tuer ? \\[0pt]
Peut-il être préférable de ne pas savoir ? \\[0pt]
Peut-il exister une action désintéressée ? \\[0pt]
Peut-il exister une morale sceptique ? \\[0pt]
Peut-il y avoir conflit entre nos devoirs ? \\[0pt]
Peut-il y avoir de bonnes raisons de croire ? \\[0pt]
Peut-il y avoir de bons tyrans ? \\[0pt]
Peut-il y avoir de la politique sans conflit ? \\[0pt]
Peut-il y avoir des choix collectifs ? \\[0pt]
Peut-il y avoir des conflits de devoirs ? \\[0pt]
Peut-il y avoir des degrés dans la vérité ? \\[0pt]
Peut-il y avoir des échanges équitables ? \\[0pt]
Peut-il y avoir des expériences métaphysiques ? \\[0pt]
Peut-il y avoir des lois de l'histoire ? \\[0pt]
Peut-il y avoir des lois injustes ? \\[0pt]
Peut-il y avoir des modèles en morale ? \\[0pt]
Peut-il y avoir des nations sans État ? \\[0pt]
Peut-il y avoir des vérités partielles ? \\[0pt]
Peut-il y avoir esprit sans corps ? \\[0pt]
Peut-il y avoir plusieurs vérités religieuses ? \\[0pt]
Peut-il y avoir savoir-faire sans savoir ? \\[0pt]
Peut-il y avoir science sans intuition du vrai ? \\[0pt]
Peut-il y avoir un art conceptuel ? \\[0pt]
Peut-il y avoir un bien commun ? \\[0pt]
Peut-il y avoir un droit à désobéir ? \\[0pt]
Peut-il y avoir un droit de la guerre ? \\[0pt]
Peut-il y avoir une anthropologie non anthropocentriste ? \\[0pt]
Peut-il y avoir une ethnologie non ethnocentriste ? \\[0pt]
Peut-il y avoir une histoire universelle ? \\[0pt]
Peut-il y avoir une méthode commune à toutes les sciences ? \\[0pt]
Peut-il y avoir une morale dans les échanges ? \\[0pt]
Peut-il y avoir une pensée sans langage ? \\[0pt]
Peut-il y avoir une philosophie applicable ? \\[0pt]
Peut-il y avoir une philosophie appliquée ? \\[0pt]
Peut-il y avoir une philosophie politique sans Dieu ? \\[0pt]
Peut-il y avoir une politique de la langue ? \\[0pt]
Peut-il y avoir une pratique sans théorie ? \\[0pt]
Peut-il y avoir une science de la morale ? \\[0pt]
Peut-il y avoir une science de l'éducation ? \\[0pt]
Peut-il y avoir une science de l'homme ? \\[0pt]
Peut-il y avoir une science de l'inconscient ? \\[0pt]
Peut-il y avoir une science des signes ? \\[0pt]
Peut-il y avoir une science du désordre ? \\[0pt]
Peut-il y avoir une science du moi ? \\[0pt]
Peut-il y avoir une science politique ? \\[0pt]
Peut-il y avoir une servitude volontaire ? \\[0pt]
Peut-il y avoir une société des nations ? \\[0pt]
Peut-il y avoir une société sans État ? \\[0pt]
Peut-il y avoir un État mondial ? \\[0pt]
Peut-il y avoir une vérité en art ? \\[0pt]
Peut-il y avoir une vérité en politique ? \\[0pt]
Peut-il y avoir une vérité religieuse ? \\[0pt]
Peut-il y avoir un intérêt collectif ? \\[0pt]
Peut-il y avoir un langage universel ? \\[0pt]
Peut-on abolir la religion ? \\[0pt]
Peut-on admettre ce qui n'est pas vérifié ? \\[0pt]
Peut-on admettre un droit à la révolte ? \\[0pt]
Peut-on affirmer que le monde a un ordre ? \\[0pt]
Peut-on affirmer qu'être, c'est être perçu ou percevoir ? \\[0pt]
Peut-on agir de manière désintéressée ? \\[0pt]
Peut-on agir machinalement ? \\[0pt]
Peut-on agir sans prendre de risques ? \\[0pt]
Peut-on agir sans raison ? \\[0pt]
Peut-on aimer ce qu'on ne connaît pas ? \\[0pt]
Peut-on aimer l'autre tel qu'il est ? \\[0pt]
Peut-on aimer la vie plus que tout ? \\[0pt]
Peut-on aimer les animaux ? \\[0pt]
Peut-on aimer l'humanité ? \\[0pt]
Peut-on aimer sans perdre sa liberté ? \\[0pt]
Peut-on aimer son prochain comme soi-même ? \\[0pt]
Peut-on aimer son travail ? \\[0pt]
Peut-on aimer une œuvre d'art sans la comprendre ? \\[0pt]
Peut-on à la fois préserver et dominer la nature ? \\[0pt]
Peut-on aller à l'encontre de la nature ? \\[0pt]
Peut-on appréhender les choses telles qu'elles sont ? \\[0pt]
Peut-on apprendre à être heureux ? \\[0pt]
Peut-on apprendre à être juste ? \\[0pt]
Peut-on apprendre à être libre ? \\[0pt]
Peut-on apprendre à mourir ? \\[0pt]
Peut-on apprendre à penser ? \\[0pt]
Peut-on apprendre à vivre ? \\[0pt]
Peut-on argumenter en morale ? \\[0pt]
Peut-on arrêter le progrès ? \\[0pt]
Peut-on assimiler le vivant à une machine ? \\[0pt]
Peut-on atteindre une certitude ? \\[0pt]
Peut-on attendre de la politique qu'elle soit conforme aux exigences de la raison ? \\[0pt]
Peut-on attribuer à chacun son dû ? \\[0pt]
Peut-on avoir conscience de soi sans avoir conscience d'autrui ? \\[0pt]
Peut-on avoir de bonnes raisons de ne pas dire la vérité ? \\[0pt]
Peut-on avoir des droits sans avoir de devoirs ? \\[0pt]
Peut-on avoir le droit de se révolter ? \\[0pt]
Peut-on avoir peur de la liberté ? \\[0pt]
Peut-on avoir peur de soi-même ? \\[0pt]
Peut-on avoir peur d'être libre ? \\[0pt]
Peut-on avoir raison contre la science ? \\[0pt]
Peut-on avoir raison contre les faits ? \\[0pt]
Peut-on avoir raison contre tous ? \\[0pt]
Peut-on avoir raison contre tout le monde ? \\[0pt]
Peut-on avoir raisons contre les faits ? \\[0pt]
Peut-on avoir raison tout.e seul.e ? \\[0pt]
Peut-on avoir raison tout seul ? \\[0pt]
Peut-on avoir trop d'imagination ? \\[0pt]
Peut-on bien agir sans le savoir ? \\[0pt]
Peut-on cesser de croire ? \\[0pt]
Peut-on cesser de désirer ? \\[0pt]
Peut-on changer au point de ne plus être le même ? \\[0pt]
Peut-on changer de culture ? \\[0pt]
Peut-on changer de logique ? \\[0pt]
Peut-on changer le cours de l'histoire ? \\[0pt]
Peut-on changer le monde ? \\[0pt]
Peut-on changer le passé ? \\[0pt]
Peut-on changer ses désirs ? \\[0pt]
Peut-on changer ses désirs à volonté ? \\[0pt]
Peut-on choisir de renoncer à sa liberté ? \\[0pt]
Peut-on choisir le mal ? \\[0pt]
Peut-on choisir sa vie ? \\[0pt]
Peut-on choisir ses désirs ? \\[0pt]
Peut-on classer les arts ? \\[0pt]
Peut-on combattre une croyance par le raisonnement ? \\[0pt]
Peut-on commander à la nature ? \\[0pt]
Peut-on communiquer ses perceptions à autrui ? \\[0pt]
Peut-on communiquer son expérience ? \\[0pt]
Peut-on comparer deux philosophies ? \\[0pt]
Peut-on comparer les cultures ? \\[0pt]
Peut-on comparer l'organisme à une machine ? \\[0pt]
Peut-on comprendre ce qui est illogique ? \\[0pt]
Peut-on comprendre le présent ? \\[0pt]
Peut-on comprendre les actions humaines ? \\[0pt]
Peut-on comprendre un acte que l'on désapprouve ? \\[0pt]
Peut-on comprendre un auteur mieux qu'il ne s'est compris lui-même ? \\[0pt]
Peut-on concevoir la science achevée ? \\[0pt]
Peut-on concevoir une humanité sans art ? \\[0pt]
Peut-on concevoir une morale sans sanction ni obligation ? \\[0pt]
Peut-on concevoir une religion dans les limites de la simple raison ? \\[0pt]
Peut-on concevoir une science qui ne soit pas démonstrative ? \\[0pt]
Peut-on concevoir une science sans expérience ? \\[0pt]
Peut-on concevoir une société juste sans que les hommes ne le soient ? \\[0pt]
Peut-on concevoir une société qui n'aurait plus besoin du droit ? \\[0pt]
Peut-on concevoir une société sans État ? \\[0pt]
Peut-on concevoir un État mondial ? \\[0pt]
Peut-on concilier bonheur et liberté ? \\[0pt]
Peut-on concilier le droit à la liberté et l'égalité des droits ? \\[0pt]
Peut-on concilier nécessité et liberté ? \\[0pt]
Peut-on conclure de l'être au devoir-être ? \\[0pt]
Peut-on connaître autrui ? \\[0pt]
Peut-on connaître les causes ? \\[0pt]
Peut-on connaître les choses telles qu'elles sont ? \\[0pt]
Peut-on connaître le singulier ? \\[0pt]
Peut-on connaître l'esprit ? \\[0pt]
Peut-on connaître le vivant sans le dénaturer ? \\[0pt]
Peut-on connaître le vivant sans recourir à la notion de finalité ? \\[0pt]
Peut-on connaître l'individuel ? \\[0pt]
Peut-on connaître par intuition ? \\[0pt]
Peut-on connaître sans concept ? \\[0pt]
Peut-on connaître une vie humaine ? \\[0pt]
Peut-on considérer la conscience comme une chose ? \\[0pt]
Peut-on considérer la main comme un outil ? \\[0pt]
Peut-on considérer l'art comme un langage ? \\[0pt]
Peut-on considérer les hommes de science comme des autorités morales ? \\[0pt]
Peut-on contester les droits de l'homme ? \\[0pt]
Peut-on contredire l'expérience ? \\[0pt]
Peut-on convaincre quelqu'un de la beauté d'une œuvre d'art ? \\[0pt]
Peut-on craindre la liberté ? \\[0pt]
Peut-on créer un homme nouveau ? \\[0pt]
Peut-on critiquer la démocratie ? \\[0pt]
Peut-on critiquer la religion ? \\[0pt]
Peut-on croire ce qu'on veut ? \\[0pt]
Peut-on croire en rien ? \\[0pt]
Peut-on croire librement ? \\[0pt]
Peut-on croire sans être crédule ? \\[0pt]
Peut-on croire sans savoir pourquoi ? \\[0pt]
Peut-on décider de croire ? \\[0pt]
Peut-on décider de tout ? \\[0pt]
Peut-on décider d'être heureux ? \\[0pt]
Peut-on définir ce qu'est un progrès technique ? \\[0pt]
Peut-on définir la matière ? \\[0pt]
Peut-on définir la morale comme l'art d'être heureux ? \\[0pt]
Peut-on définir la santé ? \\[0pt]
Peut-on définir la vérité ? \\[0pt]
Peut-on définir la vérité par la correspondance ? \\[0pt]
Peut-on définir la vie ? \\[0pt]
Peut-on définir le bien ? \\[0pt]
Peut-on définir le bonheur ? \\[0pt]
Peut-on délimiter le réel ? \\[0pt]
Peut-on délimiter l'humain ? \\[0pt]
Peut-on démontrer l'existence de la matière ? \\[0pt]
Peut-on démontrer que l'on est libre ? \\[0pt]
Peut-on démontrer qu'on ne rêve pas ? \\[0pt]
Peut-on démontrer une théorie ? \\[0pt]
Peut-on dépasser la subjectivité ? \\[0pt]
Peut-on dériver des concepts de l'expérience ? \\[0pt]
Peut-on désirer autre chose qu'être heureux ? \\[0pt]
Peut-on désirer ce qui est ? \\[0pt]
Peut-on désirer ce qu'on ne veut pas ? \\[0pt]
Peut-on désirer ce qu'on possède ? \\[0pt]
Peut-on désirer ce qu'on sait être impossible ? \\[0pt]
Peut-on désirer l'absolu ? \\[0pt]
Peut-on désirer l'impossible ? \\[0pt]
Peut-on désirer sans souffrir ? \\[0pt]
Peut-on désobéir à l'État ? \\[0pt]
Peut-on désobéir aux lois ? \\[0pt]
Peut-on désobéir par devoir ? \\[0pt]
Peut-on dialoguer avec soi-même ? \\[0pt]
Peut-on dialoguer avec un ordinateur ? \\[0pt]
Peut-on dire ce que l'on pense ? \\[0pt]
Peut-on dire ce qui n'est pas ? \\[0pt]
Peut-on dire de la connaissance scientifique qu'elle procède par approximation ? \\[0pt]
Peut-on dire de l'art qu'il donne un monde en partage ? \\[0pt]
Peut-on dire d'une image qu'elle parle ? \\[0pt]
Peut-on dire d'une œuvre d'art qu'elle est ratée ? \\[0pt]
Peut-on dire d'une théorie scientifique qu'elle n'est jamais plus vraie qu'une autre mais seulement plus commode ? \\[0pt]
Peut-on dire d'un homme qu'il est supérieur à un autre homme ? \\[0pt]
Peut-on dire la vérité ? \\[0pt]
Peut-on dire le singulier ? \\[0pt]
Peut-on dire l'être ? \\[0pt]
Peut-on dire que la science désenchante le monde ? \\[0pt]
Peut-on dire que la science ne nous fait pas connaître les choses mais les rapports entre les choses ? \\[0pt]
Peut-on dire que les hommes font l'histoire ? \\[0pt]
Peut-on dire que les machines travaillent pour nous ? \\[0pt]
Peut-on dire que les mots pensent pour nous ? \\[0pt]
Peut-on dire que l'humanité progresse ? \\[0pt]
Peut-on dire que rien n'échappe à la technique ? \\[0pt]
Peut-on dire qu'est vrai ce qui correspond aux faits ? \\[0pt]
Peut-on dire que toutes les croyances se valent ? \\[0pt]
Peut-on dire que tout est politique ? \\[0pt]
Peut-on dire que tout est relatif ? \\[0pt]
Peut-on dire qu'une théorie physique en contredit une autre ? \\[0pt]
Peut-on dire toute la vérité ? \\[0pt]
Peut-on discuter des goûts et des couleurs ? \\[0pt]
Peut-on disposer de son corps ? \\[0pt]
Peut-on distinguer différents types de causes ? \\[0pt]
Peut-on distinguer entre de bons et de mauvais désirs ? \\[0pt]
Peut-on distinguer entre les bons et les mauvais désirs ? \\[0pt]
Peut-on distinguer la vie et l'existence ? \\[0pt]
Peut-on distinguer le réel de l'imaginaire ? \\[0pt]
Peut-on distinguer les faits de leurs interprétations ? \\[0pt]
Peut-on distinguer le vrai du faux ? \\[0pt]
Peut-on donner un sens à la souffrance ? \\[0pt]
Peut-on donner un sens à l'existence ? \\[0pt]
Peut-on donner un sens à son existence ? \\[0pt]
Peut-on douter de la raison ? \\[0pt]
Peut-on douter de l'humanité d'un homme ? \\[0pt]
Peut-on douter de sa propre existence ? \\[0pt]
Peut-on douter de soi ? \\[0pt]
Peut-on douter de tout ? \\[0pt]
Peut-on douter de toute vérité ? \\[0pt]
Peut-on douter du sens des mots ? \\[0pt]
Peut-on échanger des idées ? \\[0pt]
Peut-on échapper à ses désirs ? \\[0pt]
Peut-on échapper à son temps ? \\[0pt]
Peut-on échapper au cours de l'histoire ? \\[0pt]
Peut-on échapper au temps ? \\[0pt]
Peut-on échapper aux relations de pouvoir ? \\[0pt]
Peut-on éclairer la liberté ? \\[0pt]
Peut-on écrire comme on parle ? \\[0pt]
Peut-on éduquer la conscience ? \\[0pt]
Peut-on éduquer la sensibilité ? \\[0pt]
Peut-on éduquer le goût ? \\[0pt]
Peut-on éduquer quelqu'un à être libre ? \\[0pt]
Peut-on en appeler à la conscience contre la loi ? \\[0pt]
Peut-on en appeler à la conscience contre l'État ? \\[0pt]
Peut-on encore être cosmopolite ? \\[0pt]
Peut-on encore parler d'une nature humaine ? \\[0pt]
Peut-on encore soutenir que l'homme est un animal rationnel ? \\[0pt]
Peut-on en finir avec la violence ? \\[0pt]
Peut-on en finir avec les préjugés ? \\[0pt]
Peut-on en savoir trop ? \\[0pt]
Peut-on entreprendre d'éliminer la métaphysique ? \\[0pt]
Peut-on espérer être libéré du travail ? \\[0pt]
Peut-on établir une hiérarchie des arts ? \\[0pt]
Peut-on être à la fois lucide et heureux ? \\[0pt]
Peut-on être aliéné sans le savoir ? \\[0pt]
Peut-on être amoral ? \\[0pt]
Peut-on être apolitique ? \\[0pt]
Peut-on être assuré d'avoir raison ? \\[0pt]
Peut-on être athée ? \\[0pt]
Peut-on être citoyen du monde ? \\[0pt]
Peut-on être citoyen sans être soldat ? \\[0pt]
Peut-on être complètement athée ? \\[0pt]
Peut-on être content de soi ? \\[0pt]
Peut-on être dans le présent ? \\[0pt]
Peut-on être désintéressé ? \\[0pt]
Peut-on être en avance sur son temps ? \\[0pt]
Peut-on être en conflit avec soi-même ? \\[0pt]
Peut-on être esclave de soi-même ? \\[0pt]
Peut-on être étranger à soi-même ? \\[0pt]
Peut-on être étranger au monde ? \\[0pt]
Peut-on être fidèle à soi-même ? \\[0pt]
Peut-on être heureux dans la solitude ? \\[0pt]
Peut-on être heureux sans être sage ? \\[0pt]
Peut-on être heureux sans le savoir ? \\[0pt]
Peut-on être heureux sans philosophie ? \\[0pt]
Peut-on être heureux sans s'en rendre compte ? \\[0pt]
Peut-on être heureux tout seul ? \\[0pt]
Peut-on être homme sans être citoyen ? \\[0pt]
Peut-on être hors de soi ? \\[0pt]
Peut-on être ignorant ? \\[0pt]
Peut-on être impartial ? \\[0pt]
Peut-on être indifférent à l'histoire ? \\[0pt]
Peut-on être indifférent à son bonheur ? \\[0pt]
Peut-on être injuste avec soi-même ? \\[0pt]
Peut-on être injuste envers soi-même ? \\[0pt]
Peut-on être injuste et heureux ? \\[0pt]
Peut-on être insensible à l'art ? \\[0pt]
Peut-on être insensible au vrai ? \\[0pt]
Peut-on être juste dans une situation injuste ? \\[0pt]
Peut-on être juste dans une société injuste ? \\[0pt]
Peut-on être juste sans être impartial ? \\[0pt]
Peut-on être libre sans le savoir ? \\[0pt]
Peut-on être maître de soi ? \\[0pt]
Peut-on être méchant volontairement ? \\[0pt]
Peut-on être moral sans religion ? \\[0pt]
Peut-on être objectif en matière d'art ? \\[0pt]
Peut-on être obligé d'aimer ? \\[0pt]
Peut-on être plus ou moins libre ? \\[0pt]
Peut-on être prisonnier de soi-même ? \\[0pt]
Peut-on être responsable de ce que l'on n'a pas fait ? \\[0pt]
Peut-on être sage inconsciemment ? \\[0pt]
Peut-on être sage tout seul ? \\[0pt]
Peut-on être sans opinion ? \\[0pt]
Peut-on être sceptique ? \\[0pt]
Peut-on être sceptique de bonne foi ? \\[0pt]
Peut-on être seul ? \\[0pt]
Peut-on être seul avec soi-même ? \\[0pt]
Peut-on être soi-même en société ? \\[0pt]
Peut-on être sûr d'avoir raison ? \\[0pt]
Peut-on être sûr de bien agir ? \\[0pt]
Peut-on être sûr de ne pas se tromper ? \\[0pt]
Peut-on être trop religieux ? \\[0pt]
Peut-on être trop sage ? \\[0pt]
Peut-on être trop sensible ? \\[0pt]
Peut-on étudier le passé de façon objective ? \\[0pt]
Peut-on étudier l'homme de l'extérieur ? \\[0pt]
Peut-on exercer son esprit ? \\[0pt]
Peut-on expérimenter sur le vivant ? \\[0pt]
Peut-on expliquer le mal ? \\[0pt]
Peut-on expliquer le monde par la matière ? \\[0pt]
Peut-on expliquer le vivant ? \\[0pt]
Peut-on expliquer un acte libre ? \\[0pt]
Peut-on expliquer une œuvre d'art ? \\[0pt]
Peut-on faire ce qu'on ne veut pas ? \\[0pt]
Peut-on faire comme si le passé n'existait pas ? \\[0pt]
Peut-on faire de la politique sans supposer les hommes méchants ? \\[0pt]
Peut-on faire de l'art avec tout ? \\[0pt]
Peut-on faire de l'esprit un objet de science ? \\[0pt]
Peut-on faire de sa vie une œuvre d'art ? \\[0pt]
Peut-on faire du dialogue un modèle de relation morale ? \\[0pt]
Peut-on faire la paix ? \\[0pt]
Peut-on faire la philosophie de l'histoire ? \\[0pt]
Peut-on faire le bien d'autrui malgré lui ? \\[0pt]
Peut-on faire le bien de quelqu'un malgré lui ? \\[0pt]
Peut-on faire le bonheur d'autrui ? \\[0pt]
Peut-on faire l'économie de la notion de forme ? \\[0pt]
Peut-on faire l'éloge de l'illusion ? \\[0pt]
Peut-on faire le mal en vue du bien ? \\[0pt]
Peut-on faire le mal innocemment ? \\[0pt]
Peut-on faire l'expérience de la nécessité ? \\[0pt]
Peut-on faire l'inventaire du monde ? \\[0pt]
Peut-on faire son devoir par habitude ? \\[0pt]
Peut-on faire table rase du passé ? \\[0pt]
Peut-on faire un mauvais usage de la raison ? \\[0pt]
Peut-on feindre la vertu ? \\[0pt]
Peut-on fixer des limites à la science ? \\[0pt]
Peut-on fonder la liberté ? \\[0pt]
Peut-on fonder la morale ? \\[0pt]
Peut-on fonder la morale sur la pitié ? \\[0pt]
Peut-on fonder le droit de propriété ? \\[0pt]
Peut-on fonder le droit sur la morale ? \\[0pt]
Peut-on fonder les droits de l'homme ? \\[0pt]
Peut-on fonder les mathématiques ? \\[0pt]
Peut-on fonder un droit de désobéir ? \\[0pt]
Peut-on fonder une éthique sur la biologie ? \\[0pt]
Peut-on fonder une morale sur la nature ? \\[0pt]
Peut-on fonder une morale sur le plaisir ? \\[0pt]
Peut-on forcer quelqu'un à être libre ? \\[0pt]
Peut-on forcer un homme à être libre ? \\[0pt]
Peut-on fuir hors du monde ? \\[0pt]
Peut-on fuir la société ? \\[0pt]
Peut-on gâcher son talent ? \\[0pt]
Peut-on gouverner le développement des innovations techniques ? \\[0pt]
Peut-on gouverner sans lois ? \\[0pt]
Peut-on guérir par la parole ? \\[0pt]
Peut-on haïr la raison ? \\[0pt]
Peut-on haïr la vie ? \\[0pt]
Peut-on haïr les images ? \\[0pt]
Peut-on hiérarchiser les arts ? \\[0pt]
Peut-on hiérarchiser les devoirs ? \\[0pt]
Peut-on hiérarchiser les œuvres d'art ? \\[0pt]
Peut-on hiérarchiser les sociétés ? \\[0pt]
Peut-on identifier le désir au besoin ? \\[0pt]
Peut-on ignorer sa propre liberté ? \\[0pt]
Peut-on ignorer volontairement la vérité ? \\[0pt]
Peut-on imaginer l'avenir ? \\[0pt]
Peut-on imaginer un langage universel ? \\[0pt]
Peut-on imposer la liberté ? \\[0pt]
Peut-on innover en politique ? \\[0pt]
Peut-on interpréter la nature ? \\[0pt]
Peut-on inventer en morale ? \\[0pt]
Peut-on invoquer la nature comme une excuse ? \\[0pt]
Peut-on jamais aimer son prochain ? \\[0pt]
Peut-on jamais avoir la conscience tranquille ? \\[0pt]
Peut-on juger autrui ? \\[0pt]
Peut-on juger de la valeur d'une vie humaine ? \\[0pt]
Peut-on juger des œuvres d'art sans recourir à l'idée de beauté ? \\[0pt]
Peut-on justifier la discrimination ? \\[0pt]
Peut-on justifier la guerre ? \\[0pt]
Peut-on justifier la raison d'État ? \\[0pt]
Peut-on justifier la violence ? \\[0pt]
Peut-on justifier le mal ? \\[0pt]
Peut-on justifier le mensonge ? \\[0pt]
Peut-on justifier le recours à l'idée de finalité ? \\[0pt]
Peut-on justifier ses choix ? \\[0pt]
Peut-on légitimer la violence ? \\[0pt]
Peut-on limiter l'expression de la volonté du peuple ? \\[0pt]
Peut-on lutter contre le destin ? \\[0pt]
Peut-on lutter contre soi-même ? \\[0pt]
Peut-on maîtriser la nature ? \\[0pt]
Peut-on maîtriser la technique ? \\[0pt]
Peut-on maîtriser le risque ? \\[0pt]
Peut-on maîtriser le temps ? \\[0pt]
Peut-on maîtriser l'évolution de la technique ? \\[0pt]
Peut-on maîtriser l'inconscient ? \\[0pt]
Peut-on maîtriser ses désirs ? \\[0pt]
Peut-on manipuler les esprits ? \\[0pt]
Peut-on manquer de culture ? \\[0pt]
Peut-on manquer de volonté ? \\[0pt]
Peut-on me contraindre à faire mon devoir ? \\[0pt]
Peut-on mentir par humanité ? \\[0pt]
Peut-on mesurer les phénomènes sociaux ? \\[0pt]
Peut-on mesurer le temps ? \\[0pt]
Peut-on montrer en cachant ? \\[0pt]
Peut-on moraliser la guerre ? \\[0pt]
Peut-on ne croire en rien ? \\[0pt]
Peut-on ne pas connaître son bonheur ? \\[0pt]
Peut-on ne pas croire ? \\[0pt]
Peut-on ne pas croire à la science ? \\[0pt]
Peut-on ne pas croire au progrès ? \\[0pt]
Peut-on ne pas être de son temps ? \\[0pt]
Peut-on ne pas être égoïste ? \\[0pt]
Peut-on ne pas être matérialiste ? \\[0pt]
Peut-on ne pas être soi-même ? \\[0pt]
Peut-on ne pas interpréter ? \\[0pt]
Peut-on ne pas manquer de temps ? \\[0pt]
Peut-on ne pas perdre son temps ? \\[0pt]
Peut-on ne pas rechercher le bonheur ? \\[0pt]
Peut-on ne pas savoir ce que l'on dit ? \\[0pt]
Peut-on ne pas savoir ce que l'on fait ? \\[0pt]
Peut-on ne pas savoir ce que l'on veut ? \\[0pt]
Peut-on ne pas savoir ce qu'on veut ? \\[0pt]
Peut-on ne pas vouloir être heureux ? \\[0pt]
Peut-on ne penser à rien ? \\[0pt]
Peut-on ne rien aimer ? \\[0pt]
Peut-on ne rien devoir à personne ? \\[0pt]
Peut-on ne rien vouloir ? \\[0pt]
Peut-on ne vivre qu'au présent ? \\[0pt]
Peut-on nier la réalité ? \\[0pt]
Peut-on nier le réel ? \\[0pt]
Peut-on nier l'évidence ? \\[0pt]
Peut-on nier l'existence de la matière ? \\[0pt]
Peut-on nier l'existence du temps ? \\[0pt]
Peut-on objectiver le psychisme ? \\[0pt]
Peut-on opposer connaissance scientifique et création artistique ? \\[0pt]
Peut-on opposer justice et liberté ? \\[0pt]
Peut-on opposer le loisir au travail ? \\[0pt]
Peut-on opposer morale et technique ? \\[0pt]
Peut-on opposer nature et culture ? \\[0pt]
Peut-on opposer religion et raison ? \\[0pt]
Peut-on ôter à l'homme sa liberté ? \\[0pt]
Peut-on oublier ? \\[0pt]
Peut-on oublier de vivre ? \\[0pt]
Peut-on pactiser avec un brigand ? \\[0pt]
Peut-on parler d'art primitif ? \\[0pt]
Peut-on parler de capital culturel ? \\[0pt]
Peut-on parler de ce qui n'existe pas ? \\[0pt]
Peut-on parler de conscience collective ? \\[0pt]
Peut-on parler de corruption des mœurs ? \\[0pt]
Peut-on parler de despotisme démocratique ? \\[0pt]
Peut-on parler de dialogue des cultures ? \\[0pt]
Peut-on parler de droits des animaux ? \\[0pt]
Peut-on parler de mondes imaginaires ? \\[0pt]
Peut-on parler de « nature humaine » ? \\[0pt]
Peut-on parler de nourriture spirituelle ? \\[0pt]
Peut-on parler de problèmes techniques ? \\[0pt]
Peut-on parler de progrès en art ? \\[0pt]
Peut-on parler des miracles de la technique ? \\[0pt]
Peut-on parler des œuvres d'art ? \\[0pt]
Peut-on parler de « travail intellectuel » ? \\[0pt]
Peut-on parler de travail intellectuel ? \\[0pt]
Peut-on parler de vérité en art ? \\[0pt]
Peut-on parler de vérités métaphysiques ? \\[0pt]
Peut-on parler de vérité subjective ? \\[0pt]
Peut-on parler de vérité théâtrale ? \\[0pt]
Peut-on parler de vertu politique ? \\[0pt]
Peut-on parler de violence d'État ? \\[0pt]
Peut-on parler d'un droit de la guerre ? \\[0pt]
Peut-on parler d'un droit de résistance ? \\[0pt]
Peut-on parler d'une expérience religieuse ? \\[0pt]
Peut-on parler d'une morale collective ? \\[0pt]
Peut-on parler d'une religion de l'humanité ? \\[0pt]
Peut-on parler d'une santé de l'âme ? \\[0pt]
Peut-on parler d'une science de l'art ? \\[0pt]
Peut-on parler d'univers symbolique ? \\[0pt]
Peut-on parler d'un progrès dans l'histoire ? \\[0pt]
Peut-on parler d'un progrès de la liberté ? \\[0pt]
Peut-on parler d'un progrès moral ? \\[0pt]
Peut-on parler d'un règne de la technique ? \\[0pt]
Peut-on parler d'un savoir poétique ? \\[0pt]
Peut-on parler d'un sens moral ? \\[0pt]
Peut-on parler d'un style moral ? \\[0pt]
Peut-on parler d'un travail intellectuel ? \\[0pt]
Peut-on parler pour ne rien dire ? \\[0pt]
Peut-on parler sans incohérence d'une libre obéissance ? \\[0pt]
Peut-on partager ses goûts ? \\[0pt]
Peut-on partager une expérience ? \\[0pt]
Peut-on penser ce qu'on ne peut dire ? \\[0pt]
Peut-on penser contre l'expérience ? \\[0pt]
Peut-on penser illogiquement ? \\[0pt]
Peut-on penser la création ? \\[0pt]
Peut-on penser la douleur ? \\[0pt]
Peut-on penser la fin de toute chose ? \\[0pt]
Peut-on penser la justice comme une compétence ? \\[0pt]
Peut-on penser la matière ? \\[0pt]
Peut-on penser la mort ? \\[0pt]
Peut-on penser la nouveauté ? \\[0pt]
Peut-on penser l'art comme un langage ? \\[0pt]
Peut-on penser l'art sans référence au beau ? \\[0pt]
Peut-on penser l'avenir ? \\[0pt]
Peut-on penser la vie ? \\[0pt]
Peut-on penser la vie sans penser la mort ? \\[0pt]
Peut-on penser le changement ? \\[0pt]
Peut-on penser le divin ? \\[0pt]
Peut-on penser le monde sans la technique ? \\[0pt]
Peut-on penser le réel comme un tout ? \\[0pt]
Peut-on penser le social sans la politique ? \\[0pt]
Peut-on penser le temps ? \\[0pt]
Peut-on penser le temps sans l'espace ? \\[0pt]
Peut-on penser l'extériorité ? \\[0pt]
Peut-on penser l'histoire comme progrès ? \\[0pt]
Peut-on penser l'homme à partir de la nature ? \\[0pt]
Peut-on penser l'impossible ? \\[0pt]
Peut-on penser l'infini ? \\[0pt]
Peut-on penser l'irrationnel ? \\[0pt]
Peut-on penser l'œuvre d'art sans référence à l'idée de beauté ? \\[0pt]
Peut-on penser sans concept ? \\[0pt]
Peut-on penser sans concepts ? \\[0pt]
Peut-on penser sans image ? \\[0pt]
Peut-on penser sans images ? \\[0pt]
Peut-on penser sans les mots ? \\[0pt]
Peut-on penser sans les signes ? \\[0pt]
Peut-on penser sans méthode ? \\[0pt]
Peut-on penser sans ordre ? \\[0pt]
Peut-on penser sans préjugé ? \\[0pt]
Peut-on penser sans préjugés ? \\[0pt]
Peut-on penser sans règles ? \\[0pt]
Peut-on penser sans savoir ce que l'on pense ? \\[0pt]
Peut-on penser sans savoir que l'on pense ? \\[0pt]
Peut-on penser sans signes ? \\[0pt]
Peut-on penser sans son corps ? \\[0pt]
Peut-on penser un art sans œuvres ? \\[0pt]
Peut-on penser un droit international ? \\[0pt]
Peut-on penser une conscience sans mémoire ? \\[0pt]
Peut-on penser une conscience sans objet ? \\[0pt]
Peut-on penser une fin de l'histoire ? \\[0pt]
Peut-on penser une métaphysique sans Dieu ? \\[0pt]
Peut-on penser une religion sans le recours au divin ? \\[0pt]
Peut-on penser une société sans État ? \\[0pt]
Peut-on penser un État sans violence ? \\[0pt]
Peut-on penser une volonté diabolique ? \\[0pt]
Peut-on penser un pouvoir absolu ? \\[0pt]
Peut-on percevoir le temps ? \\[0pt]
Peut-on percevoir sans juger ? \\[0pt]
Peut-on percevoir sans s'en apercevoir ? \\[0pt]
Peut-on perdre la raison ? \\[0pt]
Peut-on perdre sa dignité ? \\[0pt]
Peut-on perdre sa liberté ? \\[0pt]
Peut-on perdre son identité ? \\[0pt]
Peut-on perdre son temps ? \\[0pt]
Peut-on permettre le mensonge ? \\[0pt]
Peut-on préconiser, dans les sciences humaines et sociales, l'imitation des sciences de la nature ? \\[0pt]
Peut-on prédire les événements ? \\[0pt]
Peut-on prédire l'histoire ? \\[0pt]
Peut-on préférer le bonheur à la vérité ? \\[0pt]
Peut-on préférer l'injustice au désordre ? \\[0pt]
Peut-on préférer l'ordre à la justice ? \\[0pt]
Peut-on prendre le risque de donner la mort en voulant soulager la souffrance ? \\[0pt]
Peut-on prendre les moyens pour la fin ? \\[0pt]
Peut-on prévoir l'avenir ? \\[0pt]
Peut-on prévoir le futur ? \\[0pt]
Peut-on prévoir les hommes ? \\[0pt]
Peut-on promettre le bonheur ? \\[0pt]
Peut-on protéger les libertés sans les réduire ? \\[0pt]
Peut-on prouver la réalité de l'esprit ? \\[0pt]
Peut-on prouver l'existence ? \\[0pt]
Peut-on prouver l'existence de Dieu ? \\[0pt]
Peut-on prouver l'existence de l'inconscient ? \\[0pt]
Peut-on prouver l'existence du monde ? \\[0pt]
Peut-on prouver une existence ? \\[0pt]
Peut-on raconter sa vie ? \\[0pt]
Peut-on raisonner sans règles ? \\[0pt]
Peut-on ralentir la course du temps ? \\[0pt]
Peut-on recommencer sa vie ? \\[0pt]
Peut-on reconnaître un sens à l'histoire sans lui assigner une fin ? \\[0pt]
Peut-on réduire la pensée à une espèce de comportement ? \\[0pt]
Peut-on réduire la vie à la matière ? \\[0pt]
Peut-on réduire le raisonnement au calcul ? \\[0pt]
Peut-on réduire l'esprit à la matière ? \\[0pt]
Peut-on réduire une métaphysique à une conception du monde ? \\[0pt]
Peut-on réduire un homme à la somme de ses actes ? \\[0pt]
Peut-on refaire le monde ? \\[0pt]
Peut-on refuser de voir la vérité ? \\[0pt]
Peut-on refuser la loi ? \\[0pt]
Peut-on refuser la vérité ? \\[0pt]
Peut-on refuser la violence ? \\[0pt]
Peut-on refuser le bonheur ? \\[0pt]
Peut-on refuser l'évidence ? \\[0pt]
Peut-on refuser le vrai ? \\[0pt]
Peut-on réfuter le scepticisme ? \\[0pt]
Peut-on régner innocemment ? \\[0pt]
Peut-on rejeter le principe de non-contradiction ? \\[0pt]
Peut-on rendre raison des émotions ? \\[0pt]
Peut-on rendre raison de tout ? \\[0pt]
Peut-on rendre raison du réel ? \\[0pt]
Peut-on renoncer à comprendre ? \\[0pt]
Peut-on renoncer à être libre ? \\[0pt]
Peut-on renoncer à la liberté ? \\[0pt]
Peut-on renoncer à la vérité ? \\[0pt]
Peut-on renoncer à sa liberté ? \\[0pt]
Peut-on renoncer à ses droits ? \\[0pt]
Peut-on renoncer à soi ? \\[0pt]
Peut-on renoncer au bonheur ? \\[0pt]
Peut-on réparer le vivant ? \\[0pt]
Peut-on réparer une injustice ? \\[0pt]
Peut-on répondre au sceptique ? \\[0pt]
Peut-on répondre d'autrui ? \\[0pt]
Peut-on représenter le peuple ? \\[0pt]
Peut-on représenter l'espace ? \\[0pt]
Peut-on représenter l'invisible ? \\[0pt]
Peut-on reprocher à la morale d'être abstraite ? \\[0pt]
Peut-on reprocher au langage d'être équivoque ? \\[0pt]
Peut-on reprocher au langage d'être parfait ? \\[0pt]
Peut-on résister à la vérité ? \\[0pt]
Peut-on résister au vrai ? \\[0pt]
Peut-on rester dans le doute ? \\[0pt]
Peut-on rester insensible à la beauté ? \\[0pt]
Peut-on rester sceptique ? \\[0pt]
Peut-on restreindre la logique à la pensée formelle ? \\[0pt]
Peut-on résumer une philosophie ? \\[0pt]
Peut-on retenir le temps ? \\[0pt]
Peut-on réunir des arts différents dans une même œuvre ? \\[0pt]
Peut-on revendiquer la paix comme un droit ? \\[0pt]
Peut-on revendiquer le droit de désobéir ? \\[0pt]
Peut-on revenir sur ses erreurs ? \\[0pt]
Peut-on rire de tout ? \\[0pt]
Peut-on rompre avec la société ? \\[0pt]
Peut-on rompre avec le passé ? \\[0pt]
Peut-on s'abstenir de penser politiquement ? \\[0pt]
Peut-on s'accorder sur des vérités morales ? \\[0pt]
Peut-on s'affranchir de la subjectivité ? \\[0pt]
Peut-on s'affranchir de sa propre nature ? \\[0pt]
Peut-on s'affranchir des lois ? \\[0pt]
Peut-on saisir le temps ? \\[0pt]
Peut-on s'attendre à tout ? \\[0pt]
Peut-on savoir ce que l'on fait ? \\[0pt]
Peut-on savoir ce qui est bien ? \\[0pt]
Peut-on savoir quelque chose de l'avenir ? \\[0pt]
Peut-on savoir sans croire ? \\[0pt]
Peut-on se choisir un destin ? \\[0pt]
Peut-on se connaître soi-même ? \\[0pt]
Peut-on se désintéresser de la politique ? \\[0pt]
Peut-on se désintéresser de son bonheur ? \\[0pt]
Peut-on se duper soi-même ? \\[0pt]
Peut-on se faire une idée de tout ? \\[0pt]
Peut-on se fier à la technique ? \\[0pt]
Peut-on se fier à l'expérience vécue ? \\[0pt]
Peut-on se fier à l'intuition ? \\[0pt]
Peut-on se fier à sa propre raison ? \\[0pt]
Peut-on se fier à son intuition ? \\[0pt]
Peut-on se fier aux apparences ? \\[0pt]
Peut-on se gouverner soi-même ? \\[0pt]
Peut-on se méfier de soi-même ? \\[0pt]
Peut-on se mentir à soi-même ? \\[0pt]
Peut-on se mettre à la place d'autrui ? \\[0pt]
Peut-on se mettre à la place de l'autre ? \\[0pt]
Peut-on se mettre à la place des autres ? \\[0pt]
Peut-on s'en tenir au présent ? \\[0pt]
Peut-on sentir le temps qui passe ? \\[0pt]
Peut-on séparer la politique de l'exigence de vérité ? \\[0pt]
Peut-on séparer l'homme et l'œuvre ? \\[0pt]
Peut-on séparer politique et économie ? \\[0pt]
Peut-on se passer de chef ? \\[0pt]
Peut-on se passer de croire ? \\[0pt]
Peut-on se passer de croyance ? \\[0pt]
Peut-on se passer de croyances ? \\[0pt]
Peut-on se passer de Dieu ? \\[0pt]
Peut-on se passer de frontières ? \\[0pt]
Peut-on se passer de la religion ? \\[0pt]
Peut-on se passer de la technique ? \\[0pt]
Peut-on se passer de l'État ? \\[0pt]
Peut-on se passer de l'idée de cause finale ? \\[0pt]
Peut-on se passer de l'idée de Dieu ? \\[0pt]
Peut-on se passer de l'idée de droit naturel ? \\[0pt]
Peut-on se passer de l'idée de finalité ? \\[0pt]
Peut-on se passer de l'idée de nature humaine ? \\[0pt]
Peut-on se passer de maître ? \\[0pt]
Peut-on se passer de métaphysique ? \\[0pt]
Peut-on se passer de méthode ? \\[0pt]
Peut-on se passer de mythes ? \\[0pt]
Peut-on se passer de principes ? \\[0pt]
Peut-on se passer de religion ? \\[0pt]
Peut-on se passer de représentants ? \\[0pt]
Peut-on se passer des causes finales ? \\[0pt]
Peut-on se passer de spiritualité ? \\[0pt]
Peut-on se passer des relations ? \\[0pt]
Peut-on se passer d'État ? \\[0pt]
Peut-on se passer de technique ? \\[0pt]
Peut-on se passer de techniques de raisonnement ? \\[0pt]
Peut-on se passer de toute religion ? \\[0pt]
Peut-on se passer d'idéal ? \\[0pt]
Peut-on se passer d'ontologie ? \\[0pt]
Peut-on se passer d'un maître ? \\[0pt]
Peut-on se peindre soi-même ? \\[0pt]
Peut-on se prescrire une loi ? \\[0pt]
Peut-on se promettre quelque chose à soi-même ? \\[0pt]
Peut-on se punir soi-même ? \\[0pt]
Peut-on se régler sur des exemples en politique ? \\[0pt]
Peut-on se rendre maître de la technique ? \\[0pt]
Peut-on se retirer du monde ? \\[0pt]
Peut-on servir deux maîtres à la fois ? \\[0pt]
Peut-on se soustraire à son devoir ? \\[0pt]
Peut-on se tromper en se croyant heureux ? \\[0pt]
Peut-on se tromper sur son devoir ? \\[0pt]
Peut-on se vouloir parfait ? \\[0pt]
Peut-on s'exprimer par le silence ? \\[0pt]
Peut-on s'obliger soi-même ? \\[0pt]
Peut-on sortir de la subjectivité ? \\[0pt]
Peut-on sortir de sa conscience ? \\[0pt]
Peut-on sortir de sa culture ? \\[0pt]
Peut-on s'oublier ? \\[0pt]
Peut-on souhaiter le gouvernement des meilleurs ? \\[0pt]
Peut-on souhaiter ne pas être libre ? \\[0pt]
Peut-on suivre une règle ? \\[0pt]
Peut-on suspendre le temps ? \\[0pt]
Peut-on suspendre son jugement ? \\[0pt]
Peut-on sympathiser avec l'ennemi ? \\[0pt]
Peut-on tirer des leçons de l'histoire ? \\[0pt]
Peut-on tolérer l'injustice ? \\[0pt]
Peut-on toujours faire ce qu'on doit ? \\[0pt]
Peut-on toujours raison garder ? \\[0pt]
Peut-on toujours savoir entièrement ce que l'on dit ? \\[0pt]
Peut-on tout analyser ? \\[0pt]
Peut-on tout attendre de l'État ? \\[0pt]
Peut-on tout définir ? \\[0pt]
Peut-on tout démontrer ? \\[0pt]
Peut-on tout désirer ? \\[0pt]
Peut-on tout dire ? \\[0pt]
Peut-on tout donner ? \\[0pt]
Peut-on tout échanger ? \\[0pt]
Peut-on tout enseigner ? \\[0pt]
Peut-on tout expliquer ? \\[0pt]
Peut-on tout exprimer ? \\[0pt]
Peut-on tout imaginer ? \\[0pt]
Peut-on tout imiter ? \\[0pt]
Peut-on tout interpréter ? \\[0pt]
Peut-on tout justifier ? \\[0pt]
Peut-on tout mathématiser ? \\[0pt]
Peut-on tout mesurer ? \\[0pt]
Peut-on tout ordonner ? \\[0pt]
Peut-on tout pardonner ? \\[0pt]
Peut-on tout partager ? \\[0pt]
Peut-on tout prévoir ? \\[0pt]
Peut-on tout prouver ? \\[0pt]
Peut-on tout soumettre à la discussion ? \\[0pt]
Peut-on tout tolérer ? \\[0pt]
Peut-on traiter autrui comme un moyen ? \\[0pt]
Peut-on traiter un être vivant comme une machine ? \\[0pt]
Peut-on trancher les conflits entre interprétations ? \\[0pt]
Peut-on transformer le réel ? \\[0pt]
Peut-on transiger avec les principes ? \\[0pt]
Peut-on trop penser ? \\[0pt]
Peut-on trouver du plaisir à l'ennui ? \\[0pt]
Peut-on vivre avec les autres ? \\[0pt]
Peut-on vivre dans le doute ? \\[0pt]
Peut-on vivre en dehors de l'histoire ? \\[0pt]
Peut-on vivre en marge de la société ? \\[0pt]
Peut-on vivre en paix avec son inconscient ? \\[0pt]
Peut-on vivre en sceptique ? \\[0pt]
Peut-on vivre hors du temps ? \\[0pt]
Peut-on vivre pour la vérité ? \\[0pt]
Peut-on vivre sans aimer ? \\[0pt]
Peut-on vivre sans art ? \\[0pt]
Peut-on vivre sans aucune certitude ? \\[0pt]
Peut-on vivre sans croyance ? \\[0pt]
Peut-on vivre sans croyances ? \\[0pt]
Peut-on vivre sans désir ? \\[0pt]
Peut-on vivre sans échange ? \\[0pt]
Peut-on vivre sans foi ni loi ? \\[0pt]
Peut-on vivre sans illusions ? \\[0pt]
Peut-on vivre sans l'art ? \\[0pt]
Peut-on vivre sans le plaisir de vivre ? \\[0pt]
Peut-on vivre sans lois ? \\[0pt]
Peut-on vivre sans opinions ? \\[0pt]
Peut-on vivre sans passion ? \\[0pt]
Peut-on vivre sans peur ? \\[0pt]
Peut-on vivre sans principes ? \\[0pt]
Peut-on vivre sans réfléchir ? \\[0pt]
Peut-on vivre sans ressentiment ? \\[0pt]
Peut-on vivre sans rien espérer ? \\[0pt]
Peut-on vivre sans sacré ? \\[0pt]
Peut-on vivre sans travailler ? \\[0pt]
Peut-on vivre sans valeurs ? \\[0pt]
Peut-on voir sans croire ? \\[0pt]
Peut-on vouloir ce qu'on ne désire pas ? \\[0pt]
Peut-on vouloir le bien sans le faire ? \\[0pt]
Peut-on vouloir le bonheur d'autrui ? \\[0pt]
Peut-on vouloir le mal ? \\[0pt]
Peut-on vouloir le mal pour le mal ? \\[0pt]
Peut-on vouloir le mal sachant que c'est le mal ? \\[0pt]
Peut-on vouloir l'impossible ? \\[0pt]
Peut-on vouloir ne pas être libre ? \\[0pt]
Peut-on vouloir sans désirer ? \\[0pt]
Peut-on vraiment créer ? \\[0pt]
Peut-on vraiment dire ce qu'on pense ? \\[0pt]
Peut-on vraiment parler de soi ? \\[0pt]
Peut-on vraiment tirer des leçons du passé ? \\[0pt]
Phénomène ou fait social ? \\[0pt]
Philosopher, est-ce apprendre à vivre ? \\[0pt]
Philosophe-t-on pour être heureux ? \\[0pt]
Plusieurs religions valent-elles mieux qu'une seule ? \\[0pt]
Pour agir moralement, faut-il ne pas se soucier de soi ? \\[0pt]
Pour apprécier une œuvre, faut-il être cultivé ? \\[0pt]
Pour bien agir, faut-il vouloir le bien d'autrui ? \\[0pt]
Pour bien penser, faut-il renoncer au langage courant ? \\[0pt]
Pour connaître, suffit-il de démontrer ? \\[0pt]
Pour dialoguer faut-il parler le même langage ? \\[0pt]
Pour être heureux, faut-il ne rien désirer ? \\[0pt]
Pour être heureux, faut-il renoncer à la perfection ? \\[0pt]
Pour être homme, faut-il être citoyen ? \\[0pt]
Pour être libre, faut-il renoncer à être heureux ? \\[0pt]
Pour être un bon observateur faut-il être un bon théoricien ? \\[0pt]
Pour juger, faut-il seulement apprendre à raisonner ? \\[0pt]
Pour penser rigoureusement, faut-il renoncer au langage ordinaire ? \\[0pt]
Pour qui se prend-on ? \\[0pt]
Pourrait-on se passer de l'argent ? \\[0pt]
Pourrait-on vivre sans art ? \\[0pt]
Pourrait-on vivre sans idéal ? \\[0pt]
Pourrions-nous comprendre une pensée non humaine ? \\[0pt]
Pourrions-nous nous passer des musées ? \\[0pt]
Pourrions-nous vivre sans religion ? \\[0pt]
Pour s'aimer, faut-il se connaître ? \\[0pt]
Pour se trouver, faut-il savoir se perdre ? \\[0pt]
Pouvons-nous communiquer ce que nous sentons ? \\[0pt]
Pouvons-nous connaître sans interpréter ? \\[0pt]
Pouvons-nous désirer ce qui nous nuit ? \\[0pt]
Pouvons-nous devenir meilleurs ? \\[0pt]
Pouvons-nous dissocier le réel de nos interprétations ? \\[0pt]
Pouvons-nous être certains que nous ne rêvons pas ? \\[0pt]
Pouvons-nous être objectifs ? \\[0pt]
Pouvons-nous faire l'expérience de la liberté ? \\[0pt]
Pouvons-nous justifier nos croyances ? \\[0pt]
Pouvons-nous réellement tirer des leçons du passé ? \\[0pt]
Pouvons-nous savoir ce que nous ignorons ? \\[0pt]
Pouvons-nous vivre sans préjugés ? \\[0pt]
Prendre la parole, est-ce prendre le pouvoir ? \\[0pt]
Prendre son temps, est-ce le perdre ? \\[0pt]
Primitif ou premier ? \\[0pt]
Prince : nature ou fonction ? \\[0pt]
Promettre, est-ce renoncer à sa liberté ? \\[0pt]
Puis-je aimer tous les hommes ? \\[0pt]
Puis-je avoir la certitude que mes choix sont libres ? \\[0pt]
Puis-je comprendre autrui ? \\[0pt]
Puis-je décider de croire ? \\[0pt]
Puis-je devenir moi-même ? \\[0pt]
Puis-je devenir un autre ? \\[0pt]
Puis-je dire « ceci est mon corps » ? \\[0pt]
Puis-je douter de ma propre existence ? \\[0pt]
Puis-je être dans le vrai sans le savoir ? \\[0pt]
Puis-je être heureux dans un monde chaotique ? \\[0pt]
Puis-je être libre sans être responsable ? \\[0pt]
Puis-je être libre tout seul ? \\[0pt]
Puis-je être sûr de bien agir ? \\[0pt]
Puis-je être sûr de ne pas me tromper ? \\[0pt]
Puis-je être sûr que je ne rêve pas ? \\[0pt]
Puis-je être universel ? \\[0pt]
Puis-je faire ce que je veux de mon corps ? \\[0pt]
Puis-je faire confiance à mes sens ? \\[0pt]
Puis-je invoquer l'inconscient sans ruiner la morale ? \\[0pt]
Puis-je juger une culture à laquelle je n'appartiens pas ? \\[0pt]
Puis-je me connaître sans l'aide des autres ? \\[0pt]
Puis-je me connaître si je change à travers le temps ? \\[0pt]
Puis-je me mettre à la place d'un autre ? \\[0pt]
Puis-je me passer d'imiter autrui ? \\[0pt]
Puis-je me trouver moi-même sans l'aide d'autrui ? \\[0pt]
Puis-je ne croire que ce que je vois ? \\[0pt]
Puis-je ne pas vouloir ce que je désire ? \\[0pt]
Puis-je ne rien croire ? \\[0pt]
Puis-je ne rien devoir à personne ? \\[0pt]
Puis-je répondre des autres ? \\[0pt]
Puis-je savoir avec assurance en quoi consiste mon devoir ? \\[0pt]
Puis-je savoir ce qui m'est propre ? \\[0pt]
Punir ou soigner ? \\[0pt]
Raison et religion sont-elles incompatibles ? \\[0pt]
Rassembler les hommes, est-ce les unir ? \\[0pt]
Rebuts et objets quelconques : une matière pour l'art ? \\[0pt]
Rechercher la vérité, est-ce renoncer à toute opinion ? \\[0pt]
Reconnaissons-nous le bien comme nous reconnaissons le vrai ? \\[0pt]
Recourir au langage, est-ce renoncer à la violence ? \\[0pt]
Résister peut-il être un droit ? \\[0pt]
Respecter autrui, est-ce l'aimer ? \\[0pt]
Respecter la nature, est-ce renoncer à l'exploiter ? \\[0pt]
Respecter la nature, est-ce renoncer à s'en emparer ? \\[0pt]
Ressent-on ou apprécie-t-on l'art ? \\[0pt]
Retenons-nous le temps par le souvenir ? \\[0pt]
Rêver éloigne-t-il de la réalité ? \\[0pt]
Revient-il à l'État d'assurer le bonheur des citoyens ? \\[0pt]
Revient-il à l'État d'assurer votre bonheur ? \\[0pt]
Rêvons-nous ? \\[0pt]
Sait-on ce que l'on veut ? \\[0pt]
Sait-on ce qu'on fait ? \\[0pt]
Sait-on ce qu'on veut ? \\[0pt]
Sait-on nécessairement ce que l'on désire ? \\[0pt]
Sait-on toujours ce que l'on fait ? \\[0pt]
Sait-on toujours ce que l'on veut ? \\[0pt]
Sait-on toujours ce qu'on veut ? \\[0pt]
Sait-on vivre au présent ? \\[0pt]
Sans justice, pas de liberté ? \\[0pt]
Sans l'art, parlerait-on de beauté ? \\[0pt]
Sans les mots, que seraient les choses ? \\[0pt]
Sans référence à la beauté, peut-on rendre compte de l'art ? \\[0pt]
Savoir, est-ce cesser de croire ? \\[0pt]
Savoir, est-ce déjouer le réel ? \\[0pt]
Savoir, est-ce pouvoir ? \\[0pt]
Savoir, est-ce se libérer ? \\[0pt]
Savons-nous ce que nous disons ? \\[0pt]
Savons-nous ce que peut un corps ? \\[0pt]
Savons-nous ce qui nous rend heureux ? \\[0pt]
Savons-nous de ce que nous disons ? \\[0pt]
Savons-nous nous souvenir ? \\[0pt]
Science sans conscience n'est-elle que ruine de l'âme ? \\[0pt]
Sciences humaines et liberté sont-elles compatibles ? \\[0pt]
Sciences humaines ou sciences sociales ? \\[0pt]
Se chercher soi-même, est-ce devenir un autre ? \\[0pt]
Se cultiver, est-ce s'affranchir de son appartenance culturelle ? \\[0pt]
Se force-t-on à faire son devoir ? \\[0pt]
Se mentir à soi-même : est-ce possible ? \\[0pt]
S'engager, est-ce perdre sa liberté ? \\[0pt]
Sentiment et justice sont-ils compatibles ? \\[0pt]
Serait-il immoral d'autoriser le commerce des organes humains ? \\[0pt]
Se réfugier dans la croyance ? \\[0pt]
Se retirer dans la pensée ? \\[0pt]
Serions-nous heureux dans un ordre politique parfait ? \\[0pt]
Serions-nous plus libres sans État ? \\[0pt]
Servir, est-ce nécessairement renoncer à sa liberté ? \\[0pt]
Se sentir libre implique-t-il qu'on le soit ? \\[0pt]
Se souvenir, est-ce préparer l'avenir ? \\[0pt]
« Se trouver » a-t-il un sens ? \\[0pt]
Seul le présent existe-t-il ? \\[0pt]
Seuls les humains sont-ils libres ? \\[0pt]
Si Dieu n'existe pas, tout est-il permis ? \\[0pt]
Si Dieu n'existe pas, tout est-il possible ? \\[0pt]
Si l'esprit n'est pas une table rase, qu'est-il ? \\[0pt]
Si l'État n'existait pas, faudrait-il l'inventer ? \\[0pt]
S'indigner, est-ce un devoir ? \\[0pt]
Si nous étions moraux, le droit serait-il inutile ? \\[0pt]
Si tout est historique, tout est-il relatif ? \\[0pt]
« Sois naturel » : est-ce un bon conseil ? \\[0pt]
« Sois toi-même ! » : un impératif absurde ? \\[0pt]
Sommes-nous adaptés au monde de la technique ? \\[0pt]
Sommes-nous capables d'agir de manière désintéressée ? \\[0pt]
Sommes-nous condamnés à être libres ? \\[0pt]
Sommes-nous conscients de nos mobiles ? \\[0pt]
Sommes-nous dans le temps comme dans l'espace ? \\[0pt]
Sommes-nous des êtres métaphysiques ? \\[0pt]
Sommes-nous des sujets ? \\[0pt]
Sommes-nous déterminés par notre culture ? \\[0pt]
Sommes-nous dominés par la technique ? \\[0pt]
Sommes-nous égaux devant le bonheur ? \\[0pt]
Sommes-nous faits pour être heureux ? \\[0pt]
Sommes-nous faits pour la vérité ? \\[0pt]
Sommes-nous faits pour le bonheur ? \\[0pt]
Sommes-nous faits pour vivre en société ? \\[0pt]
Sommes-nous gouvernés par nos passions ? \\[0pt]
Sommes-nous jamais certains d'avoir choisi librement ? \\[0pt]
Sommes-nous les jouets de l'histoire ? \\[0pt]
Sommes-nous les jouets de nos pulsions ? \\[0pt]
Sommes-nous libres ? \\[0pt]
Sommes-nous libres de nos croyances ? \\[0pt]
Sommes-nous libres de nos pensées ? \\[0pt]
Sommes-nous libres de nos préférences morales ? \\[0pt]
Sommes-nous libres face à l'évidence ? \\[0pt]
Sommes-nous libres lorsque les actes émanent de notre personne ? \\[0pt]
Sommes-nous libres par nature ? \\[0pt]
Sommes-nous maîtres de la nature ? \\[0pt]
Sommes-nous maîtres de nos désirs ? \\[0pt]
Sommes-nous maîtres de nos paroles ? \\[0pt]
Sommes-nous maîtres de nos pensées ? \\[0pt]
Sommes-nous menacés par les progrès techniques ? \\[0pt]
Sommes-nous perfectibles ? \\[0pt]
Sommes-nous portés au bien ? \\[0pt]
Sommes-nous prisonniers de nos désirs ? \\[0pt]
Sommes-nous prisonniers de notre histoire ? \\[0pt]
Sommes-nous prisonniers du temps ? \\[0pt]
Sommes-nous responsables d'autrui ? \\[0pt]
Sommes-nous responsables de ce dont nous n'avons pas conscience ? \\[0pt]
Sommes-nous responsables de ce que nous sommes ? \\[0pt]
Sommes-nous responsables de nos désirs ? \\[0pt]
Sommes-nous responsables de nos erreurs ? \\[0pt]
Sommes-nous responsables de nos opinions ? \\[0pt]
Sommes-nous responsables de nos passions ? \\[0pt]
Sommes-nous responsables de notre être social ? \\[0pt]
Sommes-nous responsables du sens que prennent nos paroles ? \\[0pt]
Sommes-nous séparés du réel par les représentations que nous nous en faisons ? \\[0pt]
Sommes-nous soumis au temps ? \\[0pt]
Sommes-nous sujets de nos désirs ? \\[0pt]
Sommes-nous toujours conscients des causes de nos désirs ? \\[0pt]
Sommes-nous toujours dépendants d'autrui ? \\[0pt]
Sommes-nous tous artistes ? \\[0pt]
Sommes-nous tous contemporains ? \\[0pt]
Sommes-nous vulnérables toute notre vie ? \\[0pt]
S'opposer à l'autorité, est-ce toujours une marque de liberté ? \\[0pt]
Subissons-nous la langue que nous parlons ? \\[0pt]
Suffit-il à Dieu d'être possible pour exister ? \\[0pt]
Suffit-il d'avoir raison ? \\[0pt]
Suffit-il de bien juger pour bien faire ? \\[0pt]
Suffit-il de bien penser pour bien agir ? \\[0pt]
Suffit-il de communiquer pour dialoguer ? \\[0pt]
Suffit-il de démontrer pour convaincre ? \\[0pt]
Suffit-il de faire son devoir ? \\[0pt]
Suffit-il de faire son devoir pour être vertueux ? \\[0pt]
Suffit-il de n'avoir rien fait pour être innocent ? \\[0pt]
Suffit-il de se parler pour éviter la discorde ? \\[0pt]
Suffit-il de suivre ses convictions pour bien agir ? \\[0pt]
Suffit-il de tenir ses promesses pour faire son devoir ? \\[0pt]
Suffit-il d'être conscient de ses actes pour en être responsable ? \\[0pt]
Suffit-il d'être dans le vrai pour savoir ? \\[0pt]
Suffit-il d'être informé pour comprendre ? \\[0pt]
Suffit-il d'être juste ? \\[0pt]
Suffit-il d'être raisonnable pour être moral ? \\[0pt]
Suffit-il d'être soi-même pour être libre ? \\[0pt]
Suffit-il d'être vertueux pour être heureux ? \\[0pt]
Suffit-il de vivre pour exister ? \\[0pt]
Suffit-il de voir le meilleur pour le suivre ? \\[0pt]
Suffit-il de voir pour savoir ? \\[0pt]
Suffit-il de vouloir pour bien faire ? \\[0pt]
Suffit-il, pour croire, de le vouloir ? \\[0pt]
Suffit-il, pour être juste, d'obéir aux lois et aux coutumes de son pays ? \\[0pt]
Suffit-il que nos intentions soient bonnes pour que nos actions le soient aussi ? \\[0pt]
Suis-ce que j'ai conscience d'être ? \\[0pt]
Suis-je au centre de l'espace ? \\[0pt]
Suis-je aussi ce que j'aurais pu être ? \\[0pt]
Suis-je ce que j'ai conscience d'être ? \\[0pt]
Suis-je ce que je fais ? \\[0pt]
Suis-je ce que les autres font de moi ? \\[0pt]
Suis-je ce que mon passé a fait de moi ? \\[0pt]
Suis-je dans le temps comme je suis dans l'espace ? \\[0pt]
Suis-je étranger à moi-même ? \\[0pt]
Suis-je l'auteur de ce que je dis ? \\[0pt]
Suis-je l'auteur de mes actions ? \\[0pt]
Suis-je le même en des temps différents ? \\[0pt]
Suis-je le mieux placé pour me connaître ? \\[0pt]
Suis-je le mieux placé pour savoir qui je suis ? \\[0pt]
Suis-je le sujet de mes pensées ? \\[0pt]
Suis-je libre ? \\[0pt]
Suis-je maître de ma conscience ? \\[0pt]
Suis-je maître de mes pensées ? \\[0pt]
Suis-je ma mémoire ? \\[0pt]
Suis-je mon cerveau ? \\[0pt]
Suis-je mon corps ? \\[0pt]
Suis-je mon passé ? \\[0pt]
Suis-je propriétaire de mon corps ? \\[0pt]
Suis-je responsable de ce dont je n'ai pas conscience ? \\[0pt]
Suis-je responsable de ce que je suis ? \\[0pt]
Suis-je seul au monde ? \\[0pt]
Suis-je toujours autre que moi-même ? \\[0pt]
Suis-je toujours le même ? \\[0pt]
Superstition et fanatisme sont-ils inhérents à la religion ? \\[0pt]
Sur quels fondements distinguer ce qui est public et ce qui est privé ? \\[0pt]
Sur quoi fonder la justice ? \\[0pt]
Sur quoi fonder la légitimité de la loi ? \\[0pt]
Sur quoi fonder la propriété ? \\[0pt]
Sur quoi fonder la société ? \\[0pt]
Sur quoi fonder l'autorité ? \\[0pt]
Sur quoi fonder l'autorité politique ? \\[0pt]
Sur quoi fonder le devoir ? \\[0pt]
Sur quoi fonder le droit de punir ? \\[0pt]
Sur quoi le langage doit-il se régler ? \\[0pt]
Sur quoi l'historien travaille-t-il ? \\[0pt]
Sur quoi peut-on fonder la certitude d'avoir raison ? \\[0pt]
Sur quoi repose l'accord des esprits ? \\[0pt]
Sur quoi repose la croyance au progrès ? \\[0pt]
Sur quoi reposent nos certitudes ? \\[0pt]
Sur quoi se fonde la connaissance scientifique ? \\[0pt]
Sur quoi se fonde la vérité des énoncés ? \\[0pt]
Sur quoi sont fondées les mathématiques ? \\[0pt]
Tester une hypothèse permet-il de la prouver ? \\[0pt]
Toujours plus vite ? \\[0pt]
Tous les arts se valent-ils ? \\[0pt]
Tous les conflits peuvent-ils être résolus par le dialogue ? \\[0pt]
Tous les désirs sont-ils naturels ? \\[0pt]
Tous les droits sont-ils formels ? \\[0pt]
Tous les hommes désirent-ils connaître ? \\[0pt]
Tous les hommes désirent-ils être heureux ? \\[0pt]
Tous les hommes désirent-ils naturellement être heureux ? \\[0pt]
Tous les hommes désirent-ils naturellement savoir ? \\[0pt]
Tous les hommes sont-ils égaux ? \\[0pt]
Tous les paradis sont-ils perdus ? \\[0pt]
Tous les plaisirs se valent-ils ? \\[0pt]
Tous les problèmes peuvent-ils recevoir une solution technique ? \\[0pt]
Tous les rapports humains sont-ils des échanges ? \\[0pt]
Tout art est-il abstrait ? \\[0pt]
Tout art est-il décoratif ? \\[0pt]
Tout art est-il poésie ? \\[0pt]
Tout art est-il symbolique ? \\[0pt]
Tout a-t-il une cause ? \\[0pt]
Tout a-t-il une fin ? \\[0pt]
Tout a-t-il une raison d'être ? \\[0pt]
Tout a-t-il un prix ? \\[0pt]
Tout a-t-il un sens ? \\[0pt]
Tout bien n'est-il qu'un moindre mal ? \\[0pt]
Tout ce qui est excessif est-il insignifiant ? \\[0pt]
Tout ce qui est humain est-il signifiant ? \\[0pt]
Tout ce qui est humain nous est-il familier ? \\[0pt]
Tout ce qui est naturel est-il normal ? \\[0pt]
Tout ce qui est rationnel est-il raisonnable ? \\[0pt]
Tout ce qui est vrai doit-il être prouvé ? \\[0pt]
Tout ce qui existe a-t-il un prix ? \\[0pt]
Tout ce qui existe est-il matériel ? \\[0pt]
Tout change-t-il avec le temps ? \\[0pt]
Tout comprendre, est-ce tout pardonner ? \\[0pt]
Tout désir est-il désir de posséder ? \\[0pt]
Tout désir est-il désir de quelque chose ? \\[0pt]
Tout désir est-il égoïste ? \\[0pt]
Tout désir est-il manque ? \\[0pt]
Tout désir est-il nostalgique ? \\[0pt]
Tout désir est-il une souffrance ? \\[0pt]
Tout devoir est-il l'envers d'un droit ? \\[0pt]
Tout droit est-il un pouvoir ? \\[0pt]
Toute action politique est-elle collective ? \\[0pt]
Toute chose a-t-elle une essence ? \\[0pt]
Toute chose a-t-elle une fin ? \\[0pt]
Toute communauté est-elle politique ? \\[0pt]
Toute compréhension implique-t-elle une interprétation ? \\[0pt]
Toute conception de l'humain est-elle particulière ? \\[0pt]
Toute connaissance autre que scientifique doit-elle être considérée comme une illusion ? \\[0pt]
Toute connaissance commence-t-elle avec l'expérience ? \\[0pt]
Toute connaissance consiste-t-elle en un savoir-faire ? \\[0pt]
Toute connaissance est-elle historique ? \\[0pt]
Toute connaissance est-elle hypothétique ? \\[0pt]
Toute connaissance est-elle relative ? \\[0pt]
Toute connaissance s'enracine-t-elle dans la perception ? \\[0pt]
Toute conscience est-elle conscience de quelque chose ? \\[0pt]
Toute conscience est-elle conscience de soi ? \\[0pt]
Toute conscience est-elle subjective ? \\[0pt]
Toute conscience n'est-elle pas implicitement morale ? \\[0pt]
Toute conversation est-elle futile ? \\[0pt]
Toute définition est-elle une convention ? \\[0pt]
Toute description est-elle une interprétation ? \\[0pt]
Toute existence est-elle indémontrable ? \\[0pt]
Toute expérience appelle-t-elle une interprétation ? \\[0pt]
Toute expression est-elle métaphorique ? \\[0pt]
Toute faute est-elle une erreur ? \\[0pt]
Toute hiérarchie est-elle inégalitaire ? \\[0pt]
Toute inégalité est-elle injuste ? \\[0pt]
Toute interprétation est-elle contestable ? \\[0pt]
Toute interprétation est-elle subjective ? \\[0pt]
Toute loi est-elle arbitraire ? \\[0pt]
Toute métaphysique implique-t-elle une transcendance ? \\[0pt]
Toute morale est-elle rationnelle ? \\[0pt]
Toute morale implique-t-elle l'effort ? \\[0pt]
Toute morale s'oppose-t-elle aux désirs ? \\[0pt]
Tout énoncé admet-il une interprétation ? \\[0pt]
Tout énoncé est-il nécessairement vrai ou faux ? \\[0pt]
Toute notre connaissance dérive-t-elle de l'expérience ? \\[0pt]
Toute œuvre d'art exige-t-elle une interprétation ? \\[0pt]
Toute origine est-elle mythique ? \\[0pt]
Toute passion fait-elle souffrir ? \\[0pt]
Toute pensée revêt-elle nécessairement une forme linguistique ? \\[0pt]
Toute pensée revêt-elle une forme linguistique ? \\[0pt]
Toute peur est-elle irrationnelle ? \\[0pt]
Toute philosophie constitue-t-elle une doctrine ? \\[0pt]
Toute philosophie est-elle systématique ? \\[0pt]
Toute philosophie implique-t-elle une politique ? \\[0pt]
Toute physique exige-t-elle une métaphysique ? \\[0pt]
Toute polémique est-elle stérile ? \\[0pt]
Toute psychologie est-elle normalisatrice ? \\[0pt]
Toute relation humaine est-elle un échange ? \\[0pt]
Toute religion a-t-elle sa vérité ? \\[0pt]
Toute science est-elle naturelle ? \\[0pt]
Toutes les choses sont-elles singulières ? \\[0pt]
Toutes les convictions sont-elles respectables ? \\[0pt]
Toutes les croyances se valent-elles ? \\[0pt]
Toutes les fautes se valent-elles ? \\[0pt]
Toutes les inégalités ont-elles une importance politique ? \\[0pt]
Toutes les inégalités sont-elles des injustices ? \\[0pt]
Toutes les interprétations se valent-elles ? \\[0pt]
Toutes les opinions se valent-elles ? \\[0pt]
Toutes les opinions sont-elles bonnes à dire ? \\[0pt]
Toutes les vérités scientifiques sont-elles révisables ? \\[0pt]
Toute société a-t-elle besoin d'une religion ? \\[0pt]
Tout est-il affaire de point de vue ? \\[0pt]
Tout est-il à vendre ? \\[0pt]
Tout est-il connaissable ? \\[0pt]
Tout est-il contingent ? \\[0pt]
Tout est-il démontrable ? \\[0pt]
Tout est-il digne de mémoire ? \\[0pt]
Tout est-il faux dans la fiction ? \\[0pt]
Tout est-il historique ? \\[0pt]
Tout est-il matière ? \\[0pt]
Tout est-il mesurable ? \\[0pt]
Tout est-il nécessaire ? \\[0pt]
Tout est-il politique ? \\[0pt]
Tout est-il quantifiable ? \\[0pt]
Tout est-il relatif ? \\[0pt]
Tout est-il vraiment permis, si Dieu n'existe pas ? \\[0pt]
Tout être est-il dans l'espace ? \\[0pt]
Tout événement a-t-il une cause ? \\[0pt]
Toute vérité doit-elle être dite ? \\[0pt]
Toute vérité est-elle bonne à dire ? \\[0pt]
Toute vérité est-elle démontrable ? \\[0pt]
Toute vérité est-elle nécessaire ? \\[0pt]
Toute vérité est-elle une erreur rectifiée ? \\[0pt]
Toute vérité est-elle un rapport ? \\[0pt]
Toute vérité est-elle vérifiable ? \\[0pt]
Toute vérité se laisse-t-elle démontrer ? \\[0pt]
Toute vie est-elle intrinsèquement respectable ? \\[0pt]
Toute violence est-elle contre nature ? \\[0pt]
Tout exercice du pouvoir politique implique-t-il l'usage de la violence ? \\[0pt]
Tout fondement de la connaissance est-il métaphysique ? \\[0pt]
Tout futur est-il contingent ? \\[0pt]
Tout malheur est-il une injustice ? \\[0pt]
Tout ordre est-il rationnel ? \\[0pt]
Tout ordre est-il une violence déguisée ? \\[0pt]
Tout passe-t-il avec le temps ? \\[0pt]
Tout peut-il avoir une valeur marchande ? \\[0pt]
Tout peut-il avoir valeur marchande ? \\[0pt]
Tout peut-il être dit ? \\[0pt]
Tout peut-il être expliqué historiquement ? \\[0pt]
Tout peut-il être objet d'échange ? \\[0pt]
Tout peut-il être objet de jugement esthétique ? \\[0pt]
Tout peut-il être objet de science ? \\[0pt]
Tout peut-il n'être qu'apparence ? \\[0pt]
Tout peut-il s'acheter ? \\[0pt]
Tout peut-il se démontrer ? \\[0pt]
Tout peut-il se quantifier ? \\[0pt]
Tout peut-il se vendre ? \\[0pt]
Tout peut-il s'expliquer ? \\[0pt]
Tout pouvoir corrompt-il ? \\[0pt]
Tout pouvoir est-il oppresseur ? \\[0pt]
Tout pouvoir est-il politique ? \\[0pt]
Tout pouvoir est-il usurpé ? \\[0pt]
Tout pouvoir n'est-il pas abusif ? \\[0pt]
Tout pouvoir relève-t-il de la domination ? \\[0pt]
Tout principe est-il un fondement ? \\[0pt]
Tout savoir a-t-il une justification ? \\[0pt]
Tout savoir est-il fondé sur un savoir premier ? \\[0pt]
Tout savoir est-il pouvoir ? \\[0pt]
Tout savoir est-il transmissible ? \\[0pt]
Tout savoir est-il un pouvoir ? \\[0pt]
Tout savoir peut-il se transmettre ? \\[0pt]
Tout s'en va-t-il avec le temps ? \\[0pt]
Tout se prête-il à la mesure ? \\[0pt]
Tout texte est-il à interpréter ? \\[0pt]
Tout travail a-t-il un sens ? \\[0pt]
Tout travail est-il forcé ? \\[0pt]
Tout travail est-il social ? \\[0pt]
Traduire, est-ce trahir ? \\[0pt]
Traiter des faits humains comme des choses, est-ce considérer l'homme comme une chose ? \\[0pt]
Traiter les faits humains comme des choses, est-ce réduire les hommes à des choses ? \\[0pt]
Travailler, est-ce faire œuvre ? \\[0pt]
Travailler, est-ce lutter contre soi-même ? \\[0pt]
Travailler, est-ce seulement produire ? \\[0pt]
Travailler par plaisir, est-ce encore travailler ? \\[0pt]
Travailler sans souffrir peut-il être un idéal ? \\[0pt]
Travaille-t-on pour soi-même ? \\[0pt]
Un acte désintéressé est-il possible ? \\[0pt]
Un acte gratuit est-il possible ? \\[0pt]
Un acte inconscient est-il nécessairement un acte involontaire ? \\[0pt]
Un acte libre est-il un acte imprévisible ? \\[0pt]
Un acte peut-il être inhumain ? \\[0pt]
Un artiste doit-il être génial ? \\[0pt]
Un artiste doit-il être original ? \\[0pt]
Un art peut-il être populaire ? \\[0pt]
Un art peut-il se passer de règles ? \\[0pt]
Un art sans sublimation est-il possible ? \\[0pt]
Un autre monde est-il possible ? \\[0pt]
Un bien peut-il être commun ? \\[0pt]
Un bien peut-il sortir d'un mal ? \\[0pt]
Un chef-d'œuvre est-il immortel ? \\[0pt]
Un choix peut-il être rationnel ? \\[0pt]
Un choix rationnel est-il un choix libre ? \\[0pt]
Un contrat peut-il être injuste ? \\[0pt]
Un contrat peut-il être social ? \\[0pt]
Un corps n'est-il qu'un objet ? \\[0pt]
Un désir peut-il être coupable ? \\[0pt]
Un désir peut-il être inconscient ? \\[0pt]
Un devoir admet-il des exceptions ? \\[0pt]
Un devoir peut-il être absolu ? \\[0pt]
Un Dieu unique ? \\[0pt]
Une action peut-elle être désintéressée ? \\[0pt]
Une action peut-elle être machinale ? \\[0pt]
Une action se juge-t-elle à ses conséquences ? \\[0pt]
Une action vertueuse se reconnaît-elle à sa difficulté ? \\[0pt]
Une activité inutile est-elle sans valeur ? \\[0pt]
Une bonne cité peut-elle se passer du beau ? \\[0pt]
Une cause peut-elle être libre ? \\[0pt]
Une communauté politique n'est-elle qu'une communauté d'intérêt ? \\[0pt]
Une communauté politique n'est-elle qu'une communauté d'intérêts ? \\[0pt]
Une connaissance peut-elle ne pas être relative ? \\[0pt]
Une connaissance scientifique du vivant est-elle possible ? \\[0pt]
Une croyance infondée est-elle illégitime ? \\[0pt]
Une croyance peut-elle être libre ? \\[0pt]
Une croyance peut-elle être rationnelle ? \\[0pt]
Une culture constitue-t-elle un paradigme ? \\[0pt]
Une culture de masse est-elle une culture ? \\[0pt]
Une culture peut-elle être porteuse de valeurs universelles ? \\[0pt]
Une décision politique peut-elle être juste ? \\[0pt]
Une destruction peut-elle être créatrice ? \\[0pt]
Une d'œuvre peut-elle être achevée ? \\[0pt]
Une durée peut-elle être éternelle ? \\[0pt]
Une éducation des sentiments est-elle possible ? \\[0pt]
Une éducation esthétique est-elle possible ? \\[0pt]
Une éducation morale est-elle possible ? \\[0pt]
Une éthique sceptique est-elle possible ? \\[0pt]
Une existence se démontre-t-elle ? \\[0pt]
Une expérience peut-elle être fictive ? \\[0pt]
Une expérience se partage-t-elle ? \\[0pt]
Une explication peut-elle être réductrice ? \\[0pt]
Une fausse science est-elle une science qui commet des erreurs ? \\[0pt]
Une femme politique est-elle un homme politique comme les autres ? \\[0pt]
Une fiction peut-elle être vraie ? \\[0pt]
Une guerre peut-elle être juste ? \\[0pt]
Une idée peut-elle être fausse ? \\[0pt]
Une idée peut-elle être générale ? \\[0pt]
Une imitation peut-elle être parfaite ? \\[0pt]
Une inégalité peut-elle être juste ? \\[0pt]
Une injustice vaut elle mieux qu'un désordre ? \\[0pt]
Une intention peut-elle être coupable ? \\[0pt]
Une interprétation est-elle nécessairement subjective ? \\[0pt]
Une interprétation peut-elle échapper à l'arbitraire ? \\[0pt]
Une interprétation peut-elle être définitive ? \\[0pt]
Une interprétation peut-elle être fausse ? \\[0pt]
Une interprétation peut-elle être objective ? \\[0pt]
Une interprétation peut-elle être vraie ? \\[0pt]
Une interprétation peut-elle prétendre à la vérité ? \\[0pt]
Une justice sans égalité est-elle possible ? \\[0pt]
Une langue n'est-elle faite que de mots ? \\[0pt]
Une ligne de conduite peut-elle tenir lieu de morale ? \\[0pt]
Une logique non-formelle est-elle possible ? \\[0pt]
Une loi n'est-elle qu'une règle ? \\[0pt]
Une loi peut-elle être injuste ? \\[0pt]
Une machine n'est-elle qu'un outil perfectionné ? \\[0pt]
Une machine peut-elle avoir une mémoire ? \\[0pt]
Une machine peut-elle être naturelle ? \\[0pt]
Une machine peut-elle penser ? \\[0pt]
Une machine pourrait-elle penser ? \\[0pt]
Une machine se réduit-elle à un mécanisme ? \\[0pt]
Une matière sans forme est-elle concevable ? \\[0pt]
Une métaphysique athée est-elle possible ? \\[0pt]
Une métaphysique n'est-elle qu'une ontologie ? \\[0pt]
Une métaphysique peut-elle être sceptique ? \\[0pt]
Une morale du plaisir est-elle concevable ? \\[0pt]
Une morale personnelle est-elle une contradiction dans les termes ? \\[0pt]
Une morale peut-elle être dépassée ? \\[0pt]
Une morale peut-elle être permissive ? \\[0pt]
Une morale peut-elle être provisoire ? \\[0pt]
Une morale peut-elle prétendre à l'universalité ? \\[0pt]
Une morale peut-elle se passer de la notion de vertu ? \\[0pt]
Une morale qui ne vise pas l'utilité est-elle rationnelle ? \\[0pt]
Une morale sans devoirs est-elle possible ? \\[0pt]
Une morale sans obligation est-elle possible ? \\[0pt]
Une morale sceptique est-elle possible ? \\[0pt]
Un enfant est-il un citoyen ? \\[0pt]
Une œuvre d'art a-t-elle toujours un sens ? \\[0pt]
Une œuvre d'art a-t-elle un prix ? \\[0pt]
Une œuvre d'art doit-elle avoir un sens ? \\[0pt]
Une œuvre d'art doit-elle nécessairement être belle ? \\[0pt]
Une œuvre d'art doit-elle plaire ? \\[0pt]
Une œuvre d'art est-elle immortelle ? \\[0pt]
Une œuvre d'art est-elle toujours originale ? \\[0pt]
Une œuvre d'art est-elle une marchandise ? \\[0pt]
Une œuvre d'art peut-elle être exemplaire ? \\[0pt]
Une œuvre d'art peut-elle être immorale ? \\[0pt]
Une œuvre d'art peut-elle être laide ? \\[0pt]
Une œuvre d'art produit-elle du beau ? \\[0pt]
Une œuvre d'art s'explique-t-elle à partir de ses influences ? \\[0pt]
Une œuvre doit-elle nécessairement être belle ? \\[0pt]
Une œuvre est-elle nécessairement singulière ? \\[0pt]
Une œuvre est-elle toujours de son temps ? \\[0pt]
Une passion peut-elle résister au temps ? \\[0pt]
Une pensée contradictoire est-elle dénuée de valeur ? \\[0pt]
Une perception peut-elle être illusoire ? \\[0pt]
Une philosophie de l'amour est-elle possible ? \\[0pt]
Une philosophie peut-elle être réactionnaire ? \\[0pt]
Une politique de la nature est-elle concevable ? \\[0pt]
Une politique morale est-elle possible ? \\[0pt]
Une politique peut-elle se réclamer de la vie ? \\[0pt]
Une psychologie peut-elle être matérialiste ? \\[0pt]
Une religion civile est-elle possible ? \\[0pt]
Une religion peut-elle être fausse ? \\[0pt]
Une religion peut-elle être rationnelle ? \\[0pt]
Une religion peut-elle être universelle ? \\[0pt]
Une religion peut-elle prétendre à la vérité ? \\[0pt]
Une religion peut-elle se passer de pratiques ? \\[0pt]
Une religion rationnelle est-elle possible ? \\[0pt]
Une science bien faite est-elle une langue bien faite ? \\[0pt]
Une science de la conscience est-elle possible ? \\[0pt]
Une science de la culture est-elle possible ? \\[0pt]
Une science de la morale est-elle possible ? \\[0pt]
Une science de l'éducation est-elle possible ? \\[0pt]
Une science de l'esprit est-elle possible ? \\[0pt]
Une science de l'homme est-elle possible ? \\[0pt]
Une science de l'imaginaire est-elle possible ? \\[0pt]
Une science des symboles est-elle possible ? \\[0pt]
Une science parfaite est-elle possible ? \\[0pt]
Une science peut-elle se passer d'expérimentation ? \\[0pt]
Une sensation peut-elle être fausse ? \\[0pt]
Une société basée sur le don pourrait-elle être prospère ? \\[0pt]
Une société d'athées est-elle possible ? \\[0pt]
Une société juste, est-ce une société sans conflit ? \\[0pt]
Une société juste est-elle une société sans conflits ? \\[0pt]
Une société n'est-elle qu'un ensemble d'individus ? \\[0pt]
Une société peut-elle être juste ? \\[0pt]
Une société peut-elle se passer de religion ? \\[0pt]
Une société sans conflit est-elle pensable ? \\[0pt]
Une société sans conflit est-elle possible ? \\[0pt]
Une société sans conflits est-elle souhaitable ? \\[0pt]
Une société sans État est-elle possible ? \\[0pt]
Une société sans État est-elle une société sans politique ? \\[0pt]
Une société sans religion est-elle possible ? \\[0pt]
Une société sans travail est-elle souhaitable ? \\[0pt]
Un État mondial ? \\[0pt]
Un État peut-il être trop étendu ? \\[0pt]
Un État peut-il être « voyou » ? \\[0pt]
Une technique ne se réduit-elle pas toujours à une forme de bricolage ? \\[0pt]
Une théorie de la culture peut-elle être scientifique ? \\[0pt]
Une théorie peut-elle être vérifiée ? \\[0pt]
Une théorie sans expérience nous apprend-elle quelque chose ? \\[0pt]
Une théorie scientifique peut-elle devenir fausse ? \\[0pt]
Une théorie scientifique peut-elle être ramenée à des propositions empiriques élémentaires ? \\[0pt]
Une théorie scientifique peut-elle être vraie ? \\[0pt]
Un être vivant peut-il être comparé à une œuvre d'art ? \\[0pt]
Un événement historique est-il toujours imprévisible ? \\[0pt]
Une vérité approchée est-elle pour autant approximative ? \\[0pt]
Une vérité est-elle discutable ? \\[0pt]
Une vérité peut-elle être approximative ? \\[0pt]
Une vérité peut-elle être indicible ? \\[0pt]
Une vérité peut-elle être provisoire ? \\[0pt]
Une vérité sans preuve est-elle une vérité ? \\[0pt]
Une vie heureuse est-elle une vie de plaisirs ? \\[0pt]
Une vie juste est-elle une bonne vie ? \\[0pt]
Une vie libre exclut-elle le travail ? \\[0pt]
Une volonté peut-elle être générale ? \\[0pt]
Un fait existe-t-il sans interprétation ? \\[0pt]
Un fait scientifique doit-il être nécessairement démontré ? \\[0pt]
Un gouvernement de savants est-il souhaitable ? \\[0pt]
Un homme n'est-il que la somme de ses actes ? \\[0pt]
Un jeu peut-il être sérieux ? \\[0pt]
Un jugement de goût est-il culturel ? \\[0pt]
Un jugement moral peut-il relever de la vérité ? \\[0pt]
Un langage peut-il être privé ? \\[0pt]
Un langage universel est-il concevable ? \\[0pt]
Un mensonge peut-il avoir une valeur morale ? \\[0pt]
Un mensonge peut-il être légitime ? \\[0pt]
Un monde sans guerre serait-il un monde sans histoire ? \\[0pt]
Un monde sans nature est-il pensable ? \\[0pt]
Un monde sans travail est-il souhaitable ? \\[0pt]
Un objet technique peut-il être beau ? \\[0pt]
Un organisme n'est-il qu'un objet ? \\[0pt]
Un peuple en paix n'a-t-il plus d'histoire ? \\[0pt]
Un peuple est-il responsable de son histoire ? \\[0pt]
Un peuple est-il un rassemblement d'individus ? \\[0pt]
Un peuple peut-il être sans histoire ? \\[0pt]
Un peuple peut-il être souverain ? \\[0pt]
Un peuple se définit-il par son histoire ? \\[0pt]
Un philosophe a-t-il des devoirs envers la société ? \\[0pt]
Un plaisir peut-il être désintéressé ? \\[0pt]
Un pouvoir a-t-il besoin d'être légitime ? \\[0pt]
Un préjugé n'a-t-il aucun fondement ? \\[0pt]
Un problème moral peut-il recevoir une solution certaine ? \\[0pt]
Un problème philosophique peut-il être périmé ? \\[0pt]
Un problème scientifique peut-il être insoluble ? \\[0pt]
Un savoir peut-il être inconscient ? \\[0pt]
Un savoir scientifique sur l'homme est-il compatible avec l'idée de liberté ? \\[0pt]
Un savoir sur l'homme peut-il être séparé d'un pouvoir sur les hommes ? \\[0pt]
Un sceptique peut-il être logicien ? \\[0pt]
Un seul peut-il avoir raison contre tous ? \\[0pt]
Un souverain élu peut-il être illégitime ? \\[0pt]
Un tableau peut-il être une dénonciation ? \\[0pt]
Un témoin de moralité est-il possible ? \\[0pt]
Un travail bien fait est-il nécessairement un bon travail ? \\[0pt]
Un vice, est-ce un manque ? \\[0pt]
User de violence peut-il être moral ? \\[0pt]
Utiliser les embryons humains ? \\[0pt]
Vaut-il mieux oublier ou pardonner ? \\[0pt]
Vaut-il mieux subir l'injustice que la commettre ? \\[0pt]
Vaut-il mieux subir ou commettre l'injustice ? \\[0pt]
Vérité et cohérence ? \\[0pt]
Veut-on toujours savoir ? \\[0pt]
Vie et matière : une distinction périmée ? \\[0pt]
Vit-on au présent ? \\[0pt]
Vivons-nous au présent ? \\[0pt]
Vivons-nous tous dans le même monde ? \\[0pt]
Vivrait-on mieux sans désirs ? \\[0pt]
Vivrait-on mieux sans morale ? \\[0pt]
Vivre en société, est-ce seulement vivre ensemble ? \\[0pt]
Vivre, est-ce interpréter ? \\[0pt]
Vivre, est-ce lutter contre la mort ? \\[0pt]
Vivre, est-ce lutter pour survivre ? \\[0pt]
Vivre, est-ce résister à la mort ? \\[0pt]
Vivre, est-ce un droit ? \\[0pt]
Vivre et exister, est-ce la même chose ? \\[0pt]
Vivre sans mémoire, est-ce être libre ? \\[0pt]
Vivre sans religion, est-ce vivre sans espoir ? \\[0pt]
Voit-on ce qu'on croit ? \\[0pt]
Vouloir croire, est-ce possible ? \\[0pt]
Vouloir, est-ce encore désirer ? \\[0pt]
Vouloir la paix sociale peut-il aller jusqu'à accepter l'injustice ? \\[0pt]
Voulons-nous la vérité ? \\[0pt]
Vulgariser la science ? \\[0pt]
Y a-t-il à la question « qu'est-ce que l'homme ? » une réponse scientifique ? \\[0pt]
Y a-t-il autant de mondes que de cultures ? \\[0pt]
Y a-t-il continuité entre l'expérience et la science ? \\[0pt]
Y a-t-il continuité ou discontinuité entre la pensée mythique et la science ? \\[0pt]
Y a-t-il d'autres moyens que la démonstration pour établir la vérité ? \\[0pt]
Y a-t-il de bons et de mauvais désirs ? \\[0pt]
Y a-t-il de bons préjugés ? \\[0pt]
Y a-t-il de fausses religions ? \\[0pt]
Y a-t-il de fausses sciences ? \\[0pt]
Y a-t-il de faux besoins ? \\[0pt]
Y a-t-il de faux problèmes ? \\[0pt]
Y a-t-il de justes inégalités ? \\[0pt]
Y a-t-il de la fatalité dans la vie de l'homme ? \\[0pt]
Y a-t-il de la grandeur à être libre ? \\[0pt]
Y a-t-il de la raison dans la perception ? \\[0pt]
Y a-t-il de l'impensable ? \\[0pt]
Y a-t-il de l'incommunicable ? \\[0pt]
Y a-t-il de l'incompréhensible ? \\[0pt]
Y a-t-il de l'inconcevable ? \\[0pt]
Y a-t-il de l'inconnaissable ? \\[0pt]
Y a-t-il de l'indémontrable ? \\[0pt]
Y a-t-il de l'indésirable ? \\[0pt]
Y a-t-il de l'indicible ? \\[0pt]
Y a-t-il de l'inexprimable ? \\[0pt]
Y a-t-il de l'intelligible dans l'art ? \\[0pt]
Y a-t-il de l'irréductible ? \\[0pt]
Y a-t-il de l'irréfutable ? \\[0pt]
Y a-t-il de l'irréparable ? \\[0pt]
Y a-t-il de l'universel ? \\[0pt]
Y a-t-il de mauvais désirs ? \\[0pt]
Y a-t-il de mauvaises raisons ? \\[0pt]
Y a-t-il de mauvais spectateurs ? \\[0pt]
Y a-t-il des acquis définitifs en science ? \\[0pt]
Y a-t-il des actes de pensée ? \\[0pt]
Y a-t-il des actes désintéressés ? \\[0pt]
Y a-t-il des actes gratuits ? \\[0pt]
Y a-t-il des actes intrinsèquement mauvais ? \\[0pt]
Y a-t-il des actes moralement indifférents ? \\[0pt]
Y a-t-il des actions collectives ? \\[0pt]
Y a-t-il des actions désintéressées ? \\[0pt]
Y a-t-il des actions libres ? \\[0pt]
Y a-t-il des aliénations heureuses ? \\[0pt]
Y a-t-il des arts du corps ? \\[0pt]
Y a-t-il des arts majeurs ? \\[0pt]
Y a-t-il des arts mineurs ? \\[0pt]
Y a-t-il des barbares ? \\[0pt]
Y a-t-il des biens communs ? \\[0pt]
Y a-t-il des biens inestimables ? \\[0pt]
Y a-t-il des canons de la beauté ? \\[0pt]
Y a-t-il des certitudes historiques ? \\[0pt]
Y a-t-il des certitudes morales ? \\[0pt]
Y a-t-il des choses à savoir ? \\[0pt]
Y a-t-il des choses dont on ne peut parler ? \\[0pt]
Y a-t-il des choses qui échappent au droit ? \\[0pt]
Y a-t-il des choses qu'on n'échange pas ? \\[0pt]
Y a-t-il des colères justes ? \\[0pt]
Y a-t-il des compétences politiques ? \\[0pt]
Y a-t-il des conflits de devoirs ? \\[0pt]
Y a-t-il des connaissances dangereuses ? \\[0pt]
Y a-t-il des connaissances désintéressées ? \\[0pt]
Y a-t-il des contraintes légitimes ? \\[0pt]
Y a-t-il des convictions philosophiques ? \\[0pt]
Y a-t-il des correspondances entre les arts ? \\[0pt]
Y a-t-il des critères de la scientificité ? \\[0pt]
Y a-t-il des critères de l'humanité ? \\[0pt]
Y a-t-il des critères du beau ? \\[0pt]
Y a-t-il des critères du goût ? \\[0pt]
Y a-t-il des croyances démocratiques ? \\[0pt]
Y a-t-il des croyances nécessaires ? \\[0pt]
Y a-t-il des croyances raisonnables ? \\[0pt]
Y a-t-il des croyances rationnelles ? \\[0pt]
Y a-t-il des degrés dans la certitude ? \\[0pt]
Y a-t-il des degrés de conscience ? \\[0pt]
Y a-t-il des degrés de liberté ? \\[0pt]
Y a-t-il des degrés de réalité ? \\[0pt]
Y a-t-il des degrés de vérité ? \\[0pt]
Y a-t-il des démonstrations en morale ? \\[0pt]
Y a-t-il des démonstrations en philosophie ? \\[0pt]
Y a-t-il des désirs moraux ? \\[0pt]
Y a-t-il des désordres naturels ? \\[0pt]
Y a-t-il des despotes éclairés ? \\[0pt]
Y a-t-il des déterminismes sociaux ? \\[0pt]
Y a-t-il des devoirs envers soi ? \\[0pt]
Y a-t-il des devoirs envers soi-même ? \\[0pt]
Y a-t-il des dilemmes moraux ? \\[0pt]
Y a-t-il des droits naturels ? \\[0pt]
Y a-t-il des droits sans devoirs ? \\[0pt]
Y a-t-il des erreurs de la nature ? \\[0pt]
Y a-t-il des erreurs en politique ? \\[0pt]
Y a-t-il des êtres de raison ? \\[0pt]
Y a-t-il des êtres mathématiques ? \\[0pt]
Y a-t-il des évidences morales ? \\[0pt]
Y a-t-il des excès en art ? \\[0pt]
Y a-t-il des expériences absolument certaines ? \\[0pt]
Y a-t-il des expériences cruciales ? \\[0pt]
Y a-t-il des expériences de la liberté ? \\[0pt]
Y a-t-il des expériences métaphysiques ? \\[0pt]
Y a-t-il des expériences sans théorie ? \\[0pt]
Y a-t-il des facultés dans l'esprit ? \\[0pt]
Y a-t-il des faits moraux ? \\[0pt]
Y a-t-il des faits religieux ? \\[0pt]
Y a-t-il des faits sans essence ? \\[0pt]
Y a-t-il des faits scientifiques ? \\[0pt]
Y a-t-il des faux problèmes ? \\[0pt]
Y a-t-il des fins dans la nature ? \\[0pt]
Y a-t-il des fins de la nature ? \\[0pt]
Y a-t-il des fins dernières ? \\[0pt]
Y a-t-il des fondements naturels à l'ordre social ? \\[0pt]
Y a-t-il des formes élémentaires de la société ? \\[0pt]
Y a-t-il des genres de plaisir ? \\[0pt]
Y a-t-il des genres du plaisir ? \\[0pt]
Y a-t-il des guerres injustes ? \\[0pt]
Y a-t-il des guerres justes ? \\[0pt]
Y a-t-il des héritages philosophiques ? \\[0pt]
Y a-t-il des idées générales ? \\[0pt]
Y a-t-il des idées innées ? \\[0pt]
Y a-t-il des illusions de la conscience ? \\[0pt]
Y a-t-il des illusions de la liberté ? \\[0pt]
Y a-t-il des illusions nécessaires ? \\[0pt]
Y a-t-il des inégalités justes ? \\[0pt]
Y a-t-il des injustices naturelles ? \\[0pt]
Y a-t-il des instincts propres à l'Homme ? \\[0pt]
Y a-t-il des interprétations fausses ? \\[0pt]
Y a-t-il des intuitions morales ? \\[0pt]
Y a-t-il des jours où je n'existe pas ? \\[0pt]
Y a-t-il des leçons de l'histoire ? \\[0pt]
Y a-t-il des liens qui libèrent ? \\[0pt]
Y a-t-il des limites à la connaissance ? \\[0pt]
Y a-t-il des limites à la connaissance de l'homme par les sciences ? \\[0pt]
Y a-t-il des limites à la conscience ? \\[0pt]
Y a-t-il des limites à la critique ? \\[0pt]
Y a-t-il des limites à la description ? \\[0pt]
Y a-t-il des limites à la pensée ? \\[0pt]
Y a-t-il des limites à la responsabilité ? \\[0pt]
Y a-t-il des limites à la tolérance ? \\[0pt]
Y a-t-il des limites à l'exprimable ? \\[0pt]
Y a-t-il des limites au droit ? \\[0pt]
Y a-t-il des limites au pouvoir de la technique ? \\[0pt]
Y a-t-il des limites proprement morales à la discussion ? \\[0pt]
Y a-t-il des lois de la pensée ? \\[0pt]
Y a-t-il des lois de l'histoire ? \\[0pt]
Y a-t-il des lois de l'Histoire ? \\[0pt]
Y a-t-il des lois du comportement humain ? \\[0pt]
Y a-t-il des lois du hasard ? \\[0pt]
Y a-t-il des lois du social ? \\[0pt]
Y a-t-il des lois du vivant ? \\[0pt]
Y a-t-il des lois en histoire ? \\[0pt]
Y a-t-il des lois injustes ? \\[0pt]
Y a-t-il des lois morales ? \\[0pt]
Y a-t-il des lois non écrites ? \\[0pt]
Y a-t-il des lois sociales ? \\[0pt]
Y a-t-il des mécanismes humains ? \\[0pt]
Y a-t-il des mentalités collectives ? \\[0pt]
Y a-t-il des méthodes pour être heureux ? \\[0pt]
Y a-t-il des modèles en morale ? \\[0pt]
Y a-t-il des mondes imaginaires ? \\[0pt]
Y a-t-il des mots vides de sens ? \\[0pt]
Y a-t-il des normes de la vie ? \\[0pt]
Y a-t-il des normes naturelles ? \\[0pt]
Y a-t-il des notions communes ? \\[0pt]
Y a-t-il des objets qui n'existent pas ? \\[0pt]
Y a-t-il des obstacles à la connaissance du vivant ? \\[0pt]
Y a-t-il de sots métiers ? \\[0pt]
Y a-t-il des passions collectives ? \\[0pt]
Y a-t-il des passions intraitables ? \\[0pt]
Y a-t-il des passions raisonnables ? \\[0pt]
Y a-t-il des pathologies du social ? \\[0pt]
Y a-t-il des pathologies sociales ? \\[0pt]
Y a-t-il des pensées folles ? \\[0pt]
Y a-t-il des pensées inconscientes ? \\[0pt]
Y a-t-il des perceptions insensibles ? \\[0pt]
Y a-t-il des petites vertus ? \\[0pt]
Y a-t-il des peuples sans histoire ? \\[0pt]
Y a-t-il des phénomènes psychiques ? \\[0pt]
Y a-t-il des plaisirs innocents ? \\[0pt]
Y a-t-il des plaisirs mauvais ? \\[0pt]
Y a-t-il des plaisirs meilleurs que d'autres ? \\[0pt]
Y a-t-il des plaisirs purs ? \\[0pt]
Y a-t-il des plaisirs simples ? \\[0pt]
Y a-t-il des points de vue en morale ? \\[0pt]
Y a-t-il des preuves d'amour ? \\[0pt]
Y a-t-il des preuves de la liberté ? \\[0pt]
Y a-t-il des preuves de la non-existence de Dieu ? \\[0pt]
Y a-t-il des preuves de l'existence de Dieu ? \\[0pt]
Y a-t-il des preuves en métaphysique ? \\[0pt]
Y a-t-il des preuves en philosophie ? \\[0pt]
Y a-t-il des principes de justice universels ? \\[0pt]
Y a-t-il des progrès dans l'art ? \\[0pt]
Y a-t-il des progrès en art ? \\[0pt]
Y a-t-il des progrès en philosophie ? \\[0pt]
Y a-t-il des propriétés singulières ? \\[0pt]
Y a-t-il des questions sans réponse ? \\[0pt]
Y a-t-il des questions sans réponses ? \\[0pt]
Y a-t-il des raisons d'aimer ? \\[0pt]
Y a-t-il des raisons de douter de la raison ? \\[0pt]
Y a-t-il des raisons de vivre ? \\[0pt]
Y a-t-il des réalités mathématiques ? \\[0pt]
Y a-t-il des règles de la guerre ? \\[0pt]
Y a-t-il des règles de l'art ? \\[0pt]
Y a-t-il des régressions historiques ? \\[0pt]
Y a-t-il des révolutions dans l'art ? \\[0pt]
Y a-t-il des révolutions dans les sciences ? \\[0pt]
Y a-t-il des révolutions en art ? \\[0pt]
Y a-t-il des révolutions scientifiques ? \\[0pt]
Y a-t-il des savoirs immédiats ? \\[0pt]
Y a-t-il des sciences de l'éducation ? \\[0pt]
Y a-t-il des sciences de l'homme ? \\[0pt]
Y a-t-il des sciences exactes ? \\[0pt]
Y a-t-il des secrets de la nature ? \\[0pt]
Y a-t-il des sentiments moraux ? \\[0pt]
Y a-t-il des signes naturels ? \\[0pt]
Y a-t-il des sociétés archaïques ? \\[0pt]
Y a-t-il des sociétés sans État ? \\[0pt]
Y a-t-il des sociétés sans histoire ? \\[0pt]
Y a-t-il des solutions en politique ? \\[0pt]
Y a-t-il des sots métiers ? \\[0pt]
Y a-t-il des substances incorporelles ? \\[0pt]
Y a-t-il des techniques de pensée ? \\[0pt]
Y a-t-il des techniques du corps ? \\[0pt]
Y a-t-il des techniques pour être heureux ? \\[0pt]
Y a-t-il des tyrannies douces ? \\[0pt]
Y a-t-il des universaux anthropologiques ? \\[0pt]
Y a-t-il des valeurs absolues ? \\[0pt]
Y a-t-il des valeurs naturelles ? \\[0pt]
Y a-t-il des valeurs objectives ? \\[0pt]
Y a-t-il des valeurs propres à la science ? \\[0pt]
Y a-t-il des valeurs universelles ? \\[0pt]
Y a-t-il des vérités de fait ? \\[0pt]
Y a-t-il des vérités définitives ? \\[0pt]
Y a-t-il des vérités en art ? \\[0pt]
Y a-t-il des vérités en politique ? \\[0pt]
Y a-t-il des vérités éternelles ? \\[0pt]
Y a-t-il des vérités indémontrables ? \\[0pt]
Y a-t-il des vérités indiscutables ? \\[0pt]
Y a-t-il des vérités métaphysiques ? \\[0pt]
Y a-t-il des vérités morales ? \\[0pt]
Y a-t-il des vérités philosophiques ? \\[0pt]
Y a-t-il des vérités plus importantes que d'autres ? \\[0pt]
Y a-t-il des vérités qui échappent à la raison ? \\[0pt]
Y a-t-il des vérités religieuses ? \\[0pt]
Y a-t-il des vérités sans preuve ? \\[0pt]
Y a-t-il des vertus mineures ? \\[0pt]
Y a-t-il des violences justifiées ? \\[0pt]
Y a-t-il des violences légitimes ? \\[0pt]
Y a-t-il différentes façons d'exister ? \\[0pt]
Y a-t-il différentes manières de connaître ? \\[0pt]
Y a-t-il du déterminisme dans les sciences humaines ? \\[0pt]
Y a-t-il du hasard objectif ? \\[0pt]
Y a-t-il du non-être ? \\[0pt]
Y a-t-il du nouveau dans l'histoire ? \\[0pt]
Y a-t-il du progrès en politique ? \\[0pt]
Y a-t-il du sacré dans la nature ? \\[0pt]
Y a-t-il du synthétique \emph{a priori} ? \\[0pt]
Y a-t-il du vrai dans le beau ? \\[0pt]
Y a-t-il encore des mythologies ? \\[0pt]
Y a-t-il encore une sphère privée ? \\[0pt]
Y a-t-il incompatibilité entre science et religion ? \\[0pt]
Y a-t-il lieu de distinguer entre éthique et morale ? \\[0pt]
Y a-t-il lieu de distinguer instinct et pulsion ? \\[0pt]
Y a-t-il lieu de distinguer le don et l'échange ? \\[0pt]
Y a-t-il lieu d'opposer la philosophie et les sciences humaines ? \\[0pt]
Y a-t-il lieu d'opposer matière et esprit ? \\[0pt]
Y a-t-il nécessairement du religieux dans l'art ? \\[0pt]
Y a-t-il place pour l'idée de vérité en morale ? \\[0pt]
Y a-t-il plusieurs libertés ? \\[0pt]
Y a-t-il plusieurs manières de définir ? \\[0pt]
Y a-t-il plusieurs métaphysiques ? \\[0pt]
Y a-t-il plusieurs morales ? \\[0pt]
Y a-t-il plusieurs nécessités ? \\[0pt]
Y a-t-il plusieurs sens du mot vie ? \\[0pt]
Y a-t-il plusieurs sortes de matières ? \\[0pt]
Y a-t-il plusieurs sortes de vérité ? \\[0pt]
Y a-t-il plusieurs temps ? \\[0pt]
Y a-t-il plusieurs vérités ? \\[0pt]
Y a-t-il progrès en art ? \\[0pt]
Y a-t-il quelque chose de commun aux œuvres d'art ? \\[0pt]
Y a-t-il quelque chose de subversif dans les sciences humaines ? \\[0pt]
Y a-t-il quoi que ce soit de nouveau dans l'histoire ? \\[0pt]
Y a-t-il sens à vouloir justifier le mal ? \\[0pt]
Y a-t-il trop d'images ? \\[0pt]
Y a-t-il un art de gouverner ? \\[0pt]
Y a-t-il un art de penser ? \\[0pt]
Y a-t-il un art de persuader ? \\[0pt]
Y a-t-il un art d'être heureux ? \\[0pt]
Y a-t-il un art de vivre ? \\[0pt]
Y a-t-il un art d'interpréter ? \\[0pt]
Y a-t-il un art d'inventer ? \\[0pt]
Y a-t-il un art du bonheur ? \\[0pt]
Y a-t-il un art populaire ? \\[0pt]
Y a-t-il un art total ? \\[0pt]
Y a-t-il un au-delà de la vérité ? \\[0pt]
Y a-t-il un au-delà du langage ? \\[0pt]
Y a-t-il un auteur de l'histoire ? \\[0pt]
Y a-t-il un autre monde ? \\[0pt]
Y a-t-il un beau idéal ? \\[0pt]
Y a-t-il un beau naturel ? \\[0pt]
Y a-t-il un besoin métaphysique ? \\[0pt]
Y a-t-il un bien commun ? \\[0pt]
Y a-t-il un bien plus précieux que la paix ? \\[0pt]
Y a-t-il un Bien suprême ? \\[0pt]
Y a-t-il un bonheur sans illusion ? \\[0pt]
Y a-t-il un bon usage de la rhétorique ? \\[0pt]
Y a-t-il un bon usage des passions ? \\[0pt]
Y a-t-il un bon usage du temps ? \\[0pt]
Y a-t-il un canon de la beauté ? \\[0pt]
Y a-t-il un commencement à tout ? \\[0pt]
Y a-t-il un critère de vérité ? \\[0pt]
Y a-t-il un critère du vrai ? \\[0pt]
Y a-t-il un désir de connaître ? \\[0pt]
Y a-t-il un déterminisme dans les sciences humaines ? \\[0pt]
Y a-t-il un devoir de désobéissance ? \\[0pt]
Y a-t-il un devoir d'émancipation ? \\[0pt]
Y a-t-il un devoir de mémoire ? \\[0pt]
Y a-t-il un devoir de travailler ? \\[0pt]
Y a-t-il un devoir d'être heureux ? \\[0pt]
Y a-t-il un devoir de vérité ? \\[0pt]
Y a-t-il un devoir d'indignation ? \\[0pt]
Y a-t-il un devoir d'oubli ? \\[0pt]
Y a-t-il un différend entre poésie et philosophie ? \\[0pt]
Y a-t-il un droit à la différence ? \\[0pt]
Y a-t-il un droit à la propriété ? \\[0pt]
Y a-t-il un droit au bonheur ? \\[0pt]
Y a-t-il un droit au travail ? \\[0pt]
Y a-t-il un droit de désobéissance ? \\[0pt]
Y a-t-il un droit de la guerre ? \\[0pt]
Y a-t-il un droit de mentir ? \\[0pt]
Y a-t-il un droit de mourir ? \\[0pt]
Y a-t-il un droit de résistance ? \\[0pt]
Y a-t-il un droit de révolte ? \\[0pt]
Y a-t-il un droit des peuples ? \\[0pt]
Y a-t-il un droit d'ingérence ? \\[0pt]
Y a-t-il un droit du plus faible ? \\[0pt]
Y a-t-il un droit du plus fort ? \\[0pt]
Y a-t-il un droit international ? \\[0pt]
Y a-t-il un droit naturel ? \\[0pt]
Y a-t-il un droit universel au mariage ? \\[0pt]
Y a-t-il une argumentation métaphysique ? \\[0pt]
Y a-t-il une beauté morale ? \\[0pt]
Y a-t-il une beauté naturelle ? \\[0pt]
Y a-t-il une beauté propre à la photographie ? \\[0pt]
Y a-t-il une beauté propre à l'objet technique ? \\[0pt]
Y a-t-il une bonne éducation ? \\[0pt]
Y a-t-il une bonne imitation ? \\[0pt]
Y a-t-il une causalité empirique ? \\[0pt]
Y a-t-il une causalité en histoire ? \\[0pt]
Y a-t-il une causalité historique ? \\[0pt]
Y a-t-il une cause première ? \\[0pt]
Y a-t-il une compétence en politique ? \\[0pt]
Y a-t-il une compétence politique ? \\[0pt]
Y a-t-il une condition humaine ? \\[0pt]
Y a-t-il une connaissance du probable ? \\[0pt]
Y a-t-il une connaissance du singulier ? \\[0pt]
Y a-t-il une connaissance historique ? \\[0pt]
Y a-t-il une connaissance logique ? \\[0pt]
Y a-t-il une connaissance métaphysique ? \\[0pt]
Y a-t-il une connaissance sensible ? \\[0pt]
Y a-t-il une conscience collective ? \\[0pt]
Y a-t-il une correspondance des arts ? \\[0pt]
Y a-t-il une définition du bonheur ? \\[0pt]
Y a-t-il une dimension métaphysique du désir ? \\[0pt]
Y a-t-il une éducation du goût ? \\[0pt]
Y a-t-il une enfance de l'humanité ? \\[0pt]
Y a-t-il une esthétique de la laideur ? \\[0pt]
Y a-t-il une esthétique du laid ? \\[0pt]
Y a-t-il une esthétique industrielle ? \\[0pt]
Y a-t-il une éthique de la technique ? \\[0pt]
Y a-t-il une éthique de l'authenticité ? \\[0pt]
Y a-t-il une éthique des moyens ? \\[0pt]
Y a-t-il une expérience de la liberté ? \\[0pt]
Y a-t-il une expérience de l'éternité ? \\[0pt]
Y a-t-il une expérience du néant ? \\[0pt]
Y a-t-il une expérience du temps ? \\[0pt]
Y a-t-il une expérience métaphysique ? \\[0pt]
Y a-t-il une finalité dans la nature ? \\[0pt]
Y a-t-il une fin de l'histoire ? \\[0pt]
Y a-t-il une fin dernière ? \\[0pt]
Y a-t-il une fonction propre à l'œuvre d'art ? \\[0pt]
Y a-t-il une force des faibles ? \\[0pt]
Y a-t-il une force du droit ? \\[0pt]
Y a-t-il une formation de la pensée logique ? \\[0pt]
Y a-t-il une forme morale de fanatisme ? \\[0pt]
Y a-t-il une hiérarchie des devoirs ? \\[0pt]
Y a-t-il une hiérarchie des êtres ? \\[0pt]
Y a-t-il une hiérarchie des sciences ? \\[0pt]
Y a-t-il une hiérarchie des sciences sociales ? \\[0pt]
Y a-t-il une hiérarchie du vivant ? \\[0pt]
Y a-t-il une histoire de la folie ? \\[0pt]
Y a-t-il une histoire de la nature ? \\[0pt]
Y a-t-il une histoire de la raison ? \\[0pt]
Y a-t-il une histoire de l'art ? \\[0pt]
Y a-t-il une histoire de la vérité ? \\[0pt]
Y a-t-il une histoire de la vie quotidienne ? \\[0pt]
Y a-t-il une histoire universelle ? \\[0pt]
Y a-t-il une intelligence du corps ? \\[0pt]
Y a-t-il une intentionnalité collective ? \\[0pt]
Y a-t-il une irréversibilité du temps ? \\[0pt]
Y a-t-il une justice naturelle ? \\[0pt]
Y a-t-il une justice sans morale ? \\[0pt]
Y a-t-il une langue de la philosophie ? \\[0pt]
Y a-t-il une limite à la connaissance du vivant ? \\[0pt]
Y a-t-il une limite au désir ? \\[0pt]
Y a-t-il une limite au développement scientifique ? \\[0pt]
Y a-t-il une logique dans l'histoire ? \\[0pt]
Y a-t-il une logique de la découverte ? \\[0pt]
Y a-t-il une logique de la découverte scientifique ? \\[0pt]
Y a-t-il une logique de la déraison ? \\[0pt]
Y a-t-il une logique de l'art ? \\[0pt]
Y a-t-il une logique de l'inconscient ? \\[0pt]
Y a-t-il une logique des événements historiques ? \\[0pt]
Y a-t-il une logique du désir ? \\[0pt]
Y a-t-il une logique du mythe ? \\[0pt]
Y a-t-il une loi suprême ? \\[0pt]
Y a-t-il une mathématique universelle ? \\[0pt]
Y a-t-il une mécanique des passions ? \\[0pt]
Y a-t-il une médecine de l'âme ? \\[0pt]
Y a-t-il une métaphysique de l'amour ? \\[0pt]
Y a-t-il une méthode de l'interprétation ? \\[0pt]
Y a-t-il une méthode propre aux sciences humaines ? \\[0pt]
Y a-t-il une morale universelle ? \\[0pt]
Y a-t-il un empire de la technique ? \\[0pt]
Y a-t-il une nature humaine ? \\[0pt]
Y a-t-il une nécessité de l'erreur ? \\[0pt]
Y a-t-il une nécessité de l'Histoire ? \\[0pt]
Y a-t-il une nécessité morale ? \\[0pt]
Y a-t-il une œuvre du temps ? \\[0pt]
Y a-t-il une opinion publique mondiale ? \\[0pt]
Y a-t-il une ou des morales ? \\[0pt]
Y a-t-il une ou plusieurs philosophies ? \\[0pt]
Y a-t-il une pensée sans signes ? \\[0pt]
Y a-t-il une pensée technique ? \\[0pt]
Y a-t-il une perception esthétique ? \\[0pt]
Y a-t-il une philosophie de la nature ? \\[0pt]
Y a-t-il une philosophie de la philosophie ? \\[0pt]
Y a-t-il une philosophie première ? \\[0pt]
Y a-t-il une place dans les sciences humaines pour la catégorie de personne ? \\[0pt]
Y a-t-il une place pour la morale dans l'économie ? \\[0pt]
Y a-t-il une positivité de l'erreur ? \\[0pt]
Y a-t-il une présence du passé ? \\[0pt]
Y a-t-il une primauté du devoir sur le droit ? \\[0pt]
Y a-t-il une psychologie animale ? \\[0pt]
Y a-t-il une raison à tout ? \\[0pt]
Y a-t-il une raison des choses ? \\[0pt]
Y a-t-il une rationalité dans la religion ? \\[0pt]
Y a-t-il une rationalité de la contingence ? \\[0pt]
Y a-t-il une rationalité des sentiments ? \\[0pt]
Y a-t-il une rationalité du hasard ? \\[0pt]
Y a-t-il une rationalité du marché ? \\[0pt]
Y a-t-il une rationalité propre de l'intérêt ? \\[0pt]
Y a-t-il une réalité des idées ? \\[0pt]
Y a-t-il une réalité du hasard ? \\[0pt]
Y a-t-il une réalité du mal ? \\[0pt]
Y a-t-il une religion civile ? \\[0pt]
Y a-t-il une responsabilité de l'artiste ? \\[0pt]
Y a-t-il une sagesse de l'inconscient ? \\[0pt]
Y a-t-il une sagesse populaire ? \\[0pt]
Y a-t-il une science de la vie ? \\[0pt]
Y a-t-il une science de la vie mentale ? \\[0pt]
Y a-t-il une science de l'esprit ? \\[0pt]
Y a-t-il une science de l'être ? \\[0pt]
Y a-t-il une science de l'homme ? \\[0pt]
Y a-t-il une science de l'individuel ? \\[0pt]
Y a-t-il une science des principes ? \\[0pt]
Y a-t-il une science du juste ? \\[0pt]
Y a-t-il une science du moi ? \\[0pt]
Y a-t-il une science du qualitatif ? \\[0pt]
Y a-t-il une science ou des sciences ? \\[0pt]
Y a-t-il une science politique ? \\[0pt]
Y a-t-il une sensibilité esthétique ? \\[0pt]
Y a-t-il une servitude volontaire ? \\[0pt]
Y a-t-il une singularité de l'histoire de l'art ? \\[0pt]
Y a-t-il une spécificité de la délibération politique ? \\[0pt]
Y a-t-il une spécificité des sciences humaines ? \\[0pt]
Y a-t-il une spécificité du vivant ? \\[0pt]
Y a-t-il un esprit scientifique ? \\[0pt]
Y a-t-il un État idéal ? \\[0pt]
Y a-t-il une technique de la nature ? \\[0pt]
Y a-t-il une technique pour tout ? \\[0pt]
Y a-t-il une temporalité proprement morale ? \\[0pt]
Y a-t-il une tyrannie du vrai ? \\[0pt]
Y a-t-il une unité de la science ? \\[0pt]
Y a-t-il une unité des devoirs ? \\[0pt]
Y a-t-il une unité des langages humains ? \\[0pt]
Y a-t-il une unité des sciences ? \\[0pt]
Y a-t-il une unité en psychologie ? \\[0pt]
Y a-t-il une universalité des mathématiques ? \\[0pt]
Y a-t-il une universalité du beau ? \\[0pt]
Y a-t-il une utilité des lieux communs ? \\[0pt]
Y a-t-il une valeur de l'inutile ? \\[0pt]
Y a-t-il une vérité absolue ? \\[0pt]
Y a-t-il une vérité dans les arts ? \\[0pt]
Y a-t-il une vérité de la fiction ? \\[0pt]
Y a-t-il une vérité de l'inconscient ? \\[0pt]
Y a-t-il une vérité de l'œuvre d'art ? \\[0pt]
Y a-t-il une vérité des apparences ? \\[0pt]
Y a-t-il une vérité des représentations ? \\[0pt]
Y a-t-il une vérité des sentiments ? \\[0pt]
Y a-t-il une vérité des symboles ? \\[0pt]
Y a-t-il une vérité du sensible ? \\[0pt]
Y a-t-il une vérité du sentiment ? \\[0pt]
Y a-t-il une vérité en art ? \\[0pt]
Y a-t-il une vérité en histoire ? \\[0pt]
Y a-t-il une vérité philosophique ? \\[0pt]
Y a-t-il une vérité sur le bonheur ? \\[0pt]
Y a-t-il une vertu de l'imitation ? \\[0pt]
Y a-t-il une vertu de l'oubli ? \\[0pt]
Y a-t-il une vie de l'esprit ? \\[0pt]
Y a-t-il une violence du droit ? \\[0pt]
Y a-t-il une vision du monde propre à la pensée technique ? \\[0pt]
Y a-t-il une volonté du mal ? \\[0pt]
Y a-t-il un fondement de la croyance ? \\[0pt]
Y a-t-il un fondement moral aux circonstances atténuantes ? \\[0pt]
Y a-t-il un inconscient collectif ? \\[0pt]
Y a-t-il un inconscient psychique ? \\[0pt]
Y a-t-il un inconscient social ? \\[0pt]
Y a t-il un instinct moral ? \\[0pt]
Y a-t-il un intérêt à faire son devoir ? \\[0pt]
Y a-t-il un jugement de l'histoire ? \\[0pt]
Y a-t-il un langage animal ? \\[0pt]
Y a-t-il un langage commun ? \\[0pt]
Y a-t-il un langage de la musique ? \\[0pt]
Y a-t-il un langage de l'art ? \\[0pt]
Y a-t-il un langage de l'inconscient ? \\[0pt]
Y a-t-il un langage des animaux ? \\[0pt]
Y a-t-il un langage du corps ? \\[0pt]
Y a-t-il un langage unifié de la science ? \\[0pt]
Y a-t-il un langage universel ? \\[0pt]
Y a-t-il un mal absolu ? \\[0pt]
Y a-t-il un monde de l'art ? \\[0pt]
Y a-t-il un monde extérieur ? \\[0pt]
Y a-t-il un monde technique ? \\[0pt]
Y a-t-il un moteur de l'histoire ? \\[0pt]
Y a-t-il un objet du désir ? \\[0pt]
Y a-t-il un ordre dans la nature ? \\[0pt]
Y a-t-il un ordre des choses ? \\[0pt]
Y a-t-il un ordre du monde ? \\[0pt]
Y a-t-il un pouvoir de la technique ? \\[0pt]
Y a-t-il un primat de la nature sur la culture ? \\[0pt]
Y a-t-il un principe du mal ? \\[0pt]
Y a-t-il un progrès dans l'art ? \\[0pt]
Y a-t-il un progrès des/en sciences humaines ? \\[0pt]
Y a-t-il un progrès du droit ? \\[0pt]
Y a-t-il un progrès en art ? \\[0pt]
Y a-t-il un progrès en philosophie ? \\[0pt]
Y a-t-il un progrès moral ? \\[0pt]
Y a-t-il un propre de l'homme ? \\[0pt]
Y a-t-il un rapport moral à soi-même ? \\[0pt]
Y a-t-il un rythme de l'histoire ? \\[0pt]
Y a-t-il un savoir de la justice ? \\[0pt]
Y a-t-il un savoir du bien ? \\[0pt]
Y a-t-il un savoir du contingent ? \\[0pt]
Y a-t-il un savoir du corps ? \\[0pt]
Y a-t-il un savoir du juste ? \\[0pt]
Y a-t-il un savoir du politique ? \\[0pt]
Y a-t-il un savoir immédiat ? \\[0pt]
Y a-t-il un savoir inconscient ? \\[0pt]
Y a-t-il un savoir politique ? \\[0pt]
Y a-t-il un sens à dire que Dieu comprend, veut, fait ? \\[0pt]
Y a-t-il un sens à instituer une langue universelle ? \\[0pt]
Y a-t-il un sens à ne plus rien désirer ? \\[0pt]
Y a-t-il un sens à penser un droit des générations futures ? \\[0pt]
Y a-t-il un sens à se dire « citoyen du monde » ? \\[0pt]
Y a-t-il un sens à se dire citoyen du monde ? \\[0pt]
Y a-t-il un sens à s'opposer à la technique ? \\[0pt]
Y a-t-il un sens du beau ? \\[0pt]
Y a-t-il un sens moral ? \\[0pt]
Y a-t-il un sentiment métaphysique ? \\[0pt]
Y a-t-il un sentiment naturel du beau ? \\[0pt]
Y a-t-il un souverain bien ? \\[0pt]
Y a-t-il un temps des choses ? \\[0pt]
Y a-t-il un temps pour l'action ? \\[0pt]
Y a-t-il un temps pour tout ? \\[0pt]
Y a-t-il un traitement scientifique du langage ? \\[0pt]
Y a-t-il un travail de la pensée ? \\[0pt]
Y a-t-il un tribunal de l'histoire ? \\[0pt]
Y a-t-il un usage moral des passions ? \\[0pt]
Y a-t-il un usage purement instrumental de la raison ? \\[0pt]
Y a-t-il vérité sans interprétation ? \\[0pt]
Y a-t-une science du comportement ? \\[0pt]
Y aura-t-il toujours des religions ? \\[0pt]

\subsection{Question partielle (autre que oui/non)}
\label{sec:org4084eb3}
\noindent
À quelle condition une démarche est-elle scientifique ? \\[0pt]
À quelle condition un travail est-il humain ? \\[0pt]
À quelle expérience l'art nous convie-t-il ? \\[0pt]
À quelle généralité peuvent prétendre les sciences humaines ? \\[0pt]
À quelle objectivité peuvent prétendre les sciences sociales ? \\[0pt]
À quelles conditions est-il acceptable de travailler ? \\[0pt]
À quelles conditions le vivant peut-il être objet de science ? \\[0pt]
À quelles conditions penser une seconde nature ? \\[0pt]
À quelles conditions peut-on dire qu'une action est historique ? \\[0pt]
À quelles conditions un choix peut-il être rationnel ? \\[0pt]
À quelles conditions une activité est-elle un travail ? \\[0pt]
À quelles conditions une croyance devient-elle religieuse ? \\[0pt]
À quelles conditions une démarche est-elle scientifique ? \\[0pt]
À quelles conditions une expérience est-elle possible ? \\[0pt]
À quelles conditions une explication est-elle scientifique ? \\[0pt]
À quelles conditions une hypothèse est-elle scientifique ? \\[0pt]
À quelles conditions un énoncé est-il doué de sens ? \\[0pt]
À quelles conditions une pensée est-elle libre ? \\[0pt]
À quelles conditions un État peut-il être juste ? \\[0pt]
À quelles conditions une théorie est-elle scientifique ? \\[0pt]
À quelles conditions une théorie peut-elle être scientifique ? \\[0pt]
À quelles conditions une vérité s'impose-t-elle à nous ? \\[0pt]
À quelles conditions un jugement est-il moral ? \\[0pt]
À quelles conditions y a-t-il progrès technique ? \\[0pt]
À quelque chose le malheur peut-il être bon ? \\[0pt]
À quelque chose, le malheur peut-il être bon ? \\[0pt]
À quels signes reconnaît-on la vérité ? \\[0pt]
À qui appartient-il de décider du juste et de l'injuste ? \\[0pt]
À qui appartient la Terre ? \\[0pt]
À qui devons-nous obéir ? \\[0pt]
À qui dois-je la vérité ? \\[0pt]
À qui doit-on la vérité ? \\[0pt]
À qui doit-on le respect ? \\[0pt]
À qui doit-on obéir ? \\[0pt]
À qui est mon corps ? \\[0pt]
À qui faut-il obéir ? \\[0pt]
À qui la faute ? \\[0pt]
À qui le pouvoir appartient-il ? \\[0pt]
À qui peut-on faire confiance ? \\[0pt]
À qui profite le crime ? \\[0pt]
À qui profite le travail ? \\[0pt]
À qui se fier ? \\[0pt]
À quoi bon ? \\[0pt]
À quoi bon avoir mauvaise conscience ? \\[0pt]
À quoi bon chercher à prouver l'existence de Dieu ? \\[0pt]
À quoi bon critiquer les autres ? \\[0pt]
À quoi bon culpabiliser ? \\[0pt]
À quoi bon démontrer ? \\[0pt]
À quoi bon des héros ? \\[0pt]
À quoi bon des philosophes ? \\[0pt]
À quoi bon discuter ? \\[0pt]
À quoi bon imaginer la cité idéale ? \\[0pt]
À quoi bon imiter la nature ? \\[0pt]
À quoi bon la morale ? \\[0pt]
À quoi bon la science ? \\[0pt]
À quoi bon les regrets ? \\[0pt]
À quoi bon les romans ? \\[0pt]
À quoi bon les sciences humaines et sociales ? \\[0pt]
À quoi bon les systèmes ? \\[0pt]
À quoi bon parler ? \\[0pt]
À quoi bon penser la fin du monde ? \\[0pt]
À quoi bon raconter des histoires ? \\[0pt]
À quoi bon rêver ? \\[0pt]
À quoi bon se parler ? \\[0pt]
À quoi bon voyager ? \\[0pt]
À quoi est-il impossible de s'habituer ? \\[0pt]
À quoi est-il nécessaire d'obéir ? \\[0pt]
À quoi faut-il être fidèle ? \\[0pt]
À quoi faut-il renoncer ? \\[0pt]
À quoi faut-il renoncer pour dialoguer ? \\[0pt]
À quoi juger l'action d'un gouvernement ? \\[0pt]
À quoi la conscience nous donne-t-elle accès ? \\[0pt]
À quoi la logique peut-elle servir dans les sciences ? \\[0pt]
À quoi la perception donne-t-elle accès ? \\[0pt]
À quoi la religion sert-elle ? \\[0pt]
À quoi l'art est-il bon ? \\[0pt]
À quoi l'art nous rend-il sensibles ? \\[0pt]
À quoi la valeur d'un homme se mesure-t-elle ? \\[0pt]
À quoi les sciences nous forment-elles ? \\[0pt]
À quoi nos illusions tiennent-elles ? \\[0pt]
À quoi peut-on reconnaître la vérité ? \\[0pt]
À quoi peut-on reconnaître une œuvre d'art ? \\[0pt]
À quoi reconnaît-on la rationalité ? \\[0pt]
À quoi reconnaît-on la vérité ? \\[0pt]
À quoi reconnaît-on le faux ? \\[0pt]
À quoi reconnaît-on le réel ? \\[0pt]
À quoi reconnaît-on l'humanité en chaque homme ? \\[0pt]
À quoi reconnaît-on l'injustice ? \\[0pt]
À quoi reconnaît-on qu'une activité est un travail ? \\[0pt]
À quoi reconnaît-on qu'une expérience est scientifique ? \\[0pt]
À quoi reconnaît-on qu'une pensée est vraie ? \\[0pt]
À quoi reconnaît-on qu'une politique est juste ? \\[0pt]
À quoi reconnaît-on qu'une science est une science ? \\[0pt]
À quoi reconnaît-on qu'une théorie est scientifique ? \\[0pt]
À quoi reconnaît-on qu'un événement est historique ? \\[0pt]
À quoi reconnaît-on un acte libre ? \\[0pt]
À quoi reconnaît-on un acte vraiment libre ? \\[0pt]
À quoi reconnaît-on un bon artisan ? \\[0pt]
À quoi reconnaît-on un bon gouvernement ? \\[0pt]
À quoi reconnaît-on un comportement humain ? \\[0pt]
À quoi reconnaît-on un devoir ? \\[0pt]
À quoi reconnaît-on une action morale ? \\[0pt]
À quoi reconnaît-on une attitude religieuse ? \\[0pt]
À quoi reconnaît-on une bonne interprétation ? \\[0pt]
À quoi reconnaît-on une idéologie ? \\[0pt]
À quoi reconnaît-on une œuvre d'art ? \\[0pt]
À quoi reconnaît-on une religion ? \\[0pt]
À quoi reconnaît-on une science ? \\[0pt]
À quoi reconnaît-on une théorie scientifique ? \\[0pt]
À quoi reconnaît-on un être vivant ? \\[0pt]
À quoi renonçons-nous quand nous obéissons ? \\[0pt]
À quoi rêve-t-on ? \\[0pt]
À quoi sert la bioéthique ? \\[0pt]
À quoi sert la connaissance du passé ? \\[0pt]
À quoi sert la critique ? \\[0pt]
À quoi sert la dialectique ? \\[0pt]
À quoi sert la logique ? \\[0pt]
À quoi sert la mémoire ? \\[0pt]
À quoi sert la métaphysique ? \\[0pt]
À quoi sert la négation ? \\[0pt]
À quoi sert la notion de contrat social ? \\[0pt]
À quoi sert la notion d'état de nature ? \\[0pt]
À quoi sert la technique ? \\[0pt]
À quoi sert la théodicée ? \\[0pt]
À quoi sert la vérité ? \\[0pt]
À quoi sert le contrat social ? \\[0pt]
À quoi sert l'écriture ? \\[0pt]
À quoi sert l'État ? \\[0pt]
À quoi sert l'histoire ? \\[0pt]
À quoi sert l'intuition ? \\[0pt]
À quoi sert l'ontologie ? \\[0pt]
À quoi sert un critique d'art ? \\[0pt]
À quoi sert une métaphore ? \\[0pt]
À quoi sert un exemple ? \\[0pt]
À quoi servent les cérémonies ? \\[0pt]
À quoi servent les constitutions ? \\[0pt]
À quoi servent les croyances ? \\[0pt]
À quoi servent les doctrines morales ? \\[0pt]
À quoi servent les élections ? \\[0pt]
À quoi servent les émotions ? \\[0pt]
À quoi servent les encyclopédies ? \\[0pt]
À quoi servent les expériences ? \\[0pt]
À quoi servent les fictions ? \\[0pt]
À quoi servent les hypothèses ? \\[0pt]
À quoi servent les images ? \\[0pt]
À quoi servent les lois ? \\[0pt]
À quoi servent les machines ? \\[0pt]
À quoi servent les mythes ? \\[0pt]
À quoi servent les œuvres d'art ? \\[0pt]
À quoi servent les preuves ? \\[0pt]
À quoi servent les preuves de l'existence de Dieu ? \\[0pt]
À quoi servent les règles ? \\[0pt]
À quoi servent les religions ? \\[0pt]
À quoi servent les sciences ? \\[0pt]
À quoi servent les sciences humaines ? \\[0pt]
À quoi servent les statistiques ? \\[0pt]
À quoi servent les symboles ? \\[0pt]
À quoi servent les théories ? \\[0pt]
À quoi servent les utopies ? \\[0pt]
À quoi servent les voyages ? \\[0pt]
À quoi suis-je obligé ? \\[0pt]
À quoi tenons-nous ? \\[0pt]
À quoi tient la fermeté du vouloir ? \\[0pt]
À quoi tient la force de l'État ? \\[0pt]
À quoi tient la force des religions ? \\[0pt]
À quoi tient l'autorité ? \\[0pt]
À quoi tient la valeur d'une pensée ? \\[0pt]
À quoi tient la vérité d'une interprétation ? \\[0pt]
À quoi tient l'efficacité d'une technique ? \\[0pt]
À quoi tient le pouvoir des mots ? \\[0pt]
À quoi tient le pouvoir des sciences ? \\[0pt]
À quoi tient notre humanité ? \\[0pt]
Comment assumer les conséquences de ses actes ? \\[0pt]
Comment autrui peut-il m'aider à rechercher le bonheur ? \\[0pt]
Comment bien appliquer une règle ? \\[0pt]
Comment bien vivre ? \\[0pt]
Comment caractériser une idée confuse ? \\[0pt]
Comment chercher ce qu'on ignore ? \\[0pt]
Comment comprendre les faits sociaux ? \\[0pt]
Comment comprendre une croyance qu'on ne partage pas ? \\[0pt]
Comment conduire ses pensées ? \\[0pt]
Comment connaissons-nous nos devoirs ? \\[0pt]
Comment connaître l'autre ? \\[0pt]
Comment connaître le passé ? \\[0pt]
Comment connaître l'inconscient ? \\[0pt]
Comment connaître nos devoirs ? \\[0pt]
Comment considérer le devenir ? \\[0pt]
Comment croire au progrès ? \\[0pt]
Comment décider, sinon à la majorité ? \\[0pt]
Comment définir la démocratie ? \\[0pt]
Comment définir la raison ? \\[0pt]
Comment définir le bien commun ? \\[0pt]
Comment définir le laid ? \\[0pt]
Comment définir le peuple ? \\[0pt]
Comment définir le style ? \\[0pt]
Comment définir l'État de droit ? \\[0pt]
Comment définir l'excès ? \\[0pt]
Comment définir l'infini ? \\[0pt]
Comment déterminer le sens d'une conduite ? \\[0pt]
Comment deux personnes peuvent-elles partager la même pensée ? \\[0pt]
Comment devient-on artiste ? \\[0pt]
Comment devient-on citoyen ? \\[0pt]
Comment devient-on raisonnable ? \\[0pt]
Comment dire la vérité ? \\[0pt]
Comment dire l'individuel ? \\[0pt]
Comment distinguer culture et civilisation ? \\[0pt]
Comment distinguer désirs et besoins ? \\[0pt]
Comment distinguer entre l'amour et l'amitié ? \\[0pt]
Comment distinguer l'amour de l'amitié ? \\[0pt]
Comment distinguer le rêvé du perçu ? \\[0pt]
Comment distinguer le vrai du faux ? \\[0pt]
Comment distingue-t-on le vrai du réel ? \\[0pt]
Comment entrer dans l'histoire ? \\[0pt]
Comment établir des critères d'équité ? \\[0pt]
Comment être naturel ? \\[0pt]
Comment évaluer la qualité de la vie ? \\[0pt]
Comment évaluer l'art ? \\[0pt]
Comment expliquer l'erreur ? \\[0pt]
Comment expliquer les phénomènes mentaux ? \\[0pt]
Comment exprimer l'identité ? \\[0pt]
Comment faire de l'homme un objet d'étude ? \\[0pt]
Comment fonder la propriété ? \\[0pt]
Comment fonder nos devoirs ? \\[0pt]
Comment fonder une psychologie sociale ? \\[0pt]
Comment juger de la justesse d'une interprétation ? \\[0pt]
Comment juger de la politique ? \\[0pt]
Comment juger d'une œuvre d'art ? \\[0pt]
Comment juger son éducation ? \\[0pt]
Comment juger une œuvre d'art ? \\[0pt]
Comment justifier l'autonomie des sciences de la vie ? \\[0pt]
Comment justifier une méthode ? \\[0pt]
Comment la psychanalyse use-t-elle de l'interprétation ? \\[0pt]
Comment la réalité peut-elle dépasser la fiction ? \\[0pt]
Comment la science progresse-t-elle ? \\[0pt]
Comment la volonté peut-elle être faible ? \\[0pt]
Comment la volonté peut-elle être indéterminée ? \\[0pt]
Comment le devoir peut-il déterminer l'action ? \\[0pt]
Comment le passé nous est-il présent ? \\[0pt]
Comment le passé peut-il demeurer présent ? \\[0pt]
Comment l'erreur est-elle possible ? \\[0pt]
Comment les mathématiques peuvent-elles s'appliquer au réel ? \\[0pt]
Comment les sciences sociales expliquent-elles l'irrationnel ? \\[0pt]
Comment les sociétés changent-elles ? \\[0pt]
Comment l'homme peut-il se représenter le temps ? \\[0pt]
Comment l'inconscient peut-il se manifester ? \\[0pt]
Comment mesurer ? \\[0pt]
Comment mesurer une sensation ? \\[0pt]
Comment ne pas être humaniste ? \\[0pt]
Comment ne pas être libéral ? \\[0pt]
Comment penser la diversité des langues ? \\[0pt]
Comment penser l'écoulement du temps ? \\[0pt]
Comment penser le futur ? \\[0pt]
Comment penser le hasard ? \\[0pt]
Comment penser le mouvement ? \\[0pt]
Comment penser les universaux ? \\[0pt]
Comment penser l'éternel ? \\[0pt]
Comment penser un pouvoir qui ne corrompe pas ? \\[0pt]
Comment percevons-nous l'espace ? \\[0pt]
Comment peut-on choisir entre différentes hypothèses ? \\[0pt]
Comment peut-on définir la politique ? \\[0pt]
Comment peut-on définir un être vivant ? \\[0pt]
Comment peut-on être heureux ? \\[0pt]
« Comment peut-on être persan ? » \\[0pt]
Comment peut-on être sceptique ? \\[0pt]
Comment peut-on savoir que l'on a raison ? \\[0pt]
Comment peut-on se trahir soi-même ? \\[0pt]
Comment prend-on connaissance de ses devoirs ? \\[0pt]
Comment prouver la liberté ? \\[0pt]
Comment puis-je devenir ce que je suis ? \\[0pt]
Comment reconnaît-on une œuvre d'art ? \\[0pt]
Comment reconnaît-on un vivant ? \\[0pt]
Comment réfuter une thèse métaphysique ? \\[0pt]
Comment rendre la justice ? \\[0pt]
Comment représenter la douleur ? \\[0pt]
Comment retrouver la nature ? \\[0pt]
Comment sait-on qu'on se comprend ? \\[0pt]
Comment sait-on qu'une chose existe ? \\[0pt]
Comment s'assurer de ce qui est réel ? \\[0pt]
Comment s'assurer qu'on est libre ? \\[0pt]
Comment savoir ce qui existe ? \\[0pt]
Comment savoir quand nous sommes libres ? \\[0pt]
Comment savoir que l'on est dans l'erreur ? \\[0pt]
Comment savoir quels sont nos devoirs ? \\[0pt]
Comment se libérer du temps ? \\[0pt]
Comment se mettre à la place d'autrui ? \\[0pt]
Comment s'entendre ? \\[0pt]
Comment s'orienter dans la pensée ? \\[0pt]
Comment traiter les animaux ? \\[0pt]
Comment trancher une controverse ? \\[0pt]
Comment vérifier les hypothèses ? \\[0pt]
Comment vivre ensemble ? \\[0pt]
Comment voyager dans le temps ? \\[0pt]
Connaître les choses, en quoi est-ce déterminer leurs différences ? \\[0pt]
Dans quel but les hommes se donnent-ils des lois ? \\[0pt]
Dans quelle mesure est-on l'auteur de sa propre vie ? \\[0pt]
Dans quelle mesure l'art est-il un fait social ? \\[0pt]
Dans quelle mesure la technique nous libère-t-elle de la nature ? \\[0pt]
Dans quelle mesure la traduction est-elle une activité cognitive ? \\[0pt]
Dans quelle mesure les sciences ne sont-elles pas à l'abri de l'erreur ? \\[0pt]
Dans quelle mesure le temps nous appartient-il ? \\[0pt]
Dans quelle mesure pouvons-nous échapper à notre passé ? \\[0pt]
Dans quelle mesure suis-je responsable de mon inconscient ? \\[0pt]
Dans quelle mesure toute philosophie est-elle critique du langage ? \\[0pt]
Définir et démontrer : à quoi bon ? \\[0pt]
Définir l'art : à quoi bon ? \\[0pt]
De quel bonheur sommes-nous capables ? \\[0pt]
De quel droit ? \\[0pt]
De quel droit l'État exerce-t-il un pouvoir ? \\[0pt]
De quel droit punit-on ? \\[0pt]
De quel homme parlent les sciences humaines ? \\[0pt]
De quelle certitude la science est-elle capable ? \\[0pt]
De quelle expertise les sciences humaines peuvent-elles se prévaloir ? \\[0pt]
De quelle liberté témoigne l'œuvre d'art ? \\[0pt]
De quelle réalité nos perceptions témoignent-elles ? \\[0pt]
De quelle réalité parlons-nous lorsque nous nous exprimons ? \\[0pt]
De quelle réalité témoignent nos perceptions ? \\[0pt]
De quelle science humaine la folie peut-elle être l'objet ? \\[0pt]
De quelle transgression l'art est-il susceptible ? \\[0pt]
De quelle vérité l'art est-il capable ? \\[0pt]
De quelle vérité l'inconscient est-il porteur ? \\[0pt]
De quelle vérité l'opinion est-elle capable ? \\[0pt]
De qui sommes-nous les obligés ? \\[0pt]
De quoi a-t-on besoin ? \\[0pt]
De quoi a-t-on conscience lorsqu'on a conscience de soi ? \\[0pt]
De quoi avons-nous besoin ? \\[0pt]
De quoi avons-nous vraiment besoin ? \\[0pt]
De quoi dépend le bonheur ? \\[0pt]
De quoi dépend notre bonheur ? \\[0pt]
De quoi doute un sceptique ? \\[0pt]
De quoi est fait mon présent ? \\[0pt]
De quoi est fait notre présent ? \\[0pt]
De quoi est-on conscient ? \\[0pt]
De quoi est-on malheureux ? \\[0pt]
De quoi faisons-nous l'expérience ? \\[0pt]
De quoi fait-on l'expérience face à une œuvre ? \\[0pt]
De quoi faut-il attendre son bonheur ? \\[0pt]
De quoi faut-il avoir peur ? \\[0pt]
De quoi « humain » est-il le nom ? \\[0pt]
De quoi la forme est-elle la forme ? \\[0pt]
De quoi la logique est-elle la science ? \\[0pt]
De quoi la musique est-elle l'art ? \\[0pt]
De quoi la philosophie est-elle le désir ? \\[0pt]
De quoi la politique s'occupe-t-elle ? \\[0pt]
De quoi la religion sauve-t-elle ? \\[0pt]
De quoi l'art nous console-t-il ? \\[0pt]
De quoi l'art nous délivre-t-il ? \\[0pt]
De quoi l'art peut-il être contemporain ? \\[0pt]
De quoi l'art peut-il nous libérer ? \\[0pt]
De quoi la technique ne peut-elle pas nous libérer ? \\[0pt]
De quoi la vérité libère-t-elle ? \\[0pt]
De quoi le devoir libère-t-il ? \\[0pt]
De quoi le phénomène est-il l'apparaître ? \\[0pt]
De quoi le réel est-il constitué ? \\[0pt]
De quoi les logiciens parlent-ils ? \\[0pt]
De quoi les métaphysiciens parlent-ils ? \\[0pt]
De quoi les sciences humaines nous instruisent-elles ? \\[0pt]
De quoi l'État doit-il être propriétaire ? \\[0pt]
De quoi l'État ne doit-il pas se mêler ? \\[0pt]
De quoi le tissu social est-il fait ? \\[0pt]
De quoi le tyran est-il libre ? \\[0pt]
De quoi l'expérience esthétique est-elle expérience ? \\[0pt]
De quoi l'expérience esthétique est-elle l'expérience ? \\[0pt]
De quoi l'historien fait-il l'histoire ? \\[0pt]
De quoi l'inconscient peut-il être la cause ? \\[0pt]
De quoi l'outil nous libère-t-il ? \\[0pt]
De quoi n'avons-nous pas conscience ? \\[0pt]
De quoi ne peut-on pas répondre ? \\[0pt]
De quoi parlent les mathématiques ? \\[0pt]
De quoi parlent les théories physiques ? \\[0pt]
De quoi pâtit-on ? \\[0pt]
De quoi peut-il y avoir science ? \\[0pt]
De quoi peut-on être certain ? \\[0pt]
De quoi peut-on être inconscient ? \\[0pt]
De quoi peut-on faire l'expérience ? \\[0pt]
De quoi peut-on réclamer la propriété ? \\[0pt]
De quoi pouvons-nous être certains ? \\[0pt]
De quoi pouvons-nous être sûrs ? \\[0pt]
De quoi puis-je répondre ? \\[0pt]
De quoi rit-on ? \\[0pt]
De quoi sommes-nous coupables ? \\[0pt]
De quoi sommes-nous prisonniers ? \\[0pt]
De quoi sommes-nous responsables ? \\[0pt]
De quoi suis-je inconscient ? \\[0pt]
De quoi suis-je responsable ? \\[0pt]
De quoi une œuvre d'art nous instruit-elle ? \\[0pt]
De quoi y a-t-il expérience ? \\[0pt]
De quoi y a-t-il histoire ? \\[0pt]
Donner, à quoi bon ? \\[0pt]
En quel sens la maladie peut-elle transformer notre vie ? \\[0pt]
En quel sens la métaphysique a-t-elle une histoire ? \\[0pt]
En quel sens la métaphysique est-elle une science ? \\[0pt]
En quel sens l'anthropologie peut-elle être historique ? \\[0pt]
En quel sens la perception engage-t-elle le corps ? \\[0pt]
En quel sens les sciences ont-elles une histoire ? \\[0pt]
En quel sens l'État est-il rationnel ? \\[0pt]
En quel sens le vécu est-il l'objet des sciences humaines ? \\[0pt]
En quel sens le vivant a-t-il une histoire ? \\[0pt]
En quel sens parler de lois de la pensée ? \\[0pt]
En quel sens parler de structure métaphysique ? \\[0pt]
En quel sens parler d'identité culturelle ? \\[0pt]
En quel sens parler d'illusion religieuse ? \\[0pt]
En quel sens peut-on appeler les sciences humaines « sciences de l'esprit » ? \\[0pt]
En quel sens peut-on dire de l'écologie qu'elle est politique ? \\[0pt]
En quel sens peut-on dire que la vérité s'impose ? \\[0pt]
En quel sens peut-on dire que le langage est la condition de la pensée ? \\[0pt]
En quel sens peut-on dire que le mal n'existe pas ? \\[0pt]
En quel sens peut-on dire que l'homme est un animal politique ? \\[0pt]
En quel sens peut-on dire que nos paroles nous trahissent ? \\[0pt]
En quel sens peut-on dire qu'« on expérimente avec sa raison » ? \\[0pt]
En quel sens peut-on parler de la mort de l'art ? \\[0pt]
En quel sens peut-on parler de la vie sociale comme d'un jeu ? \\[0pt]
En quel sens peut-on parler de responsabilité collective ? \\[0pt]
En quel sens peut-on parler de transcendance ? \\[0pt]
En quel sens peut-on parler d'expérience possible ? \\[0pt]
En quel sens peut-on parler d'une « culture technique » ? \\[0pt]
En quel sens peut-on parler d'une culture technique ? \\[0pt]
En quel sens peut-on parler d'une interprétation de la nature ? \\[0pt]
En quel sens une œuvre d'art est-elle un document ? \\[0pt]
En quels sens les corps peuvent-ils être objets de la métaphysique ? \\[0pt]
En quoi la connaissance de la matière peut-elle relever de la métaphysique ? \\[0pt]
En quoi la connaissance du vivant contribue-t-elle à la connaissance de l'homme ? \\[0pt]
En quoi la croyance religieuse se distingue-t-elle des autres croyances ? \\[0pt]
En quoi la justice met-elle fin à la violence ? \\[0pt]
En quoi la liberté n'est-elle pas une illusion ? \\[0pt]
En quoi la linguistique est-elle une science ? \\[0pt]
En quoi la matière s'oppose-t-elle à l'esprit ? \\[0pt]
En quoi la métaphysique est-elle une science ? \\[0pt]
En quoi la méthode est-elle un art de penser ? \\[0pt]
En quoi la nature constitue-t-elle un modèle ? \\[0pt]
En quoi la nature peut-elle être source de création ? \\[0pt]
En quoi l'animal intéresse-t-il les sciences humaines ? \\[0pt]
En quoi la patience est-elle une vertu ? \\[0pt]
En quoi la physique a-t-elle besoin des mathématiques ? \\[0pt]
En quoi la question de Dieu est-elle philosophique ? \\[0pt]
En quoi l'art peut-il intéresser le philosophe ? \\[0pt]
En quoi la sexualité peut-elle être objet de science ? \\[0pt]
En quoi la sociologie est-elle fondamentale ? \\[0pt]
En quoi la technique fait-elle question ? \\[0pt]
En quoi le bien d'autrui m'importe-t-il ? \\[0pt]
En quoi le bonheur est-il l'affaire de l'État ? \\[0pt]
En quoi le langage est-il constitutif de l'homme ? \\[0pt]
En quoi le langage est-il objet de science ? \\[0pt]
En quoi les hommes restent-ils des enfants ? \\[0pt]
En quoi les sciences humaines nous éclairent-elles sur la barbarie ? \\[0pt]
En quoi les sciences humaines sont-elles normatives ? \\[0pt]
En quoi les vivants témoignent-ils d'une histoire ? \\[0pt]
En quoi l'histoire est-elle une science du réel ? \\[0pt]
En quoi l'œuvre d'art donne-t-elle à penser ? \\[0pt]
En quoi une culture peut-elle être la mienne ? \\[0pt]
En quoi une discussion est-elle politique ? \\[0pt]
En quoi une insulte est-elle blessante ? \\[0pt]
En quoi une œuvre d'art est-elle moderne ? \\[0pt]
Entre l'art et la nature, qui imite l'autre ? \\[0pt]
« Je ne voulais pas cela » : en quoi les sciences humaines permettent-elles de comprendre cette excuse ? \\[0pt]
Les institutions politiques : qu'ont-elles en propre ? \\[0pt]
« L'histoire jugera » : quel sens faut-il accorder à cette expression ? \\[0pt]
Où commence la liberté ? \\[0pt]
Où commence la servitude ? \\[0pt]
Où commence la vie privée ? \\[0pt]
Où commence la violence ? \\[0pt]
Où commence le territoire de la psychologie ? \\[0pt]
Où commence l'interprétation ? \\[0pt]
Où commence ma liberté ? \\[0pt]
Où est le danger ? \\[0pt]
Où est le passé ? \\[0pt]
Où est le pouvoir ? \\[0pt]
Où est l'esprit ? \\[0pt]
Où est mon esprit ? \\[0pt]
Où est-on quand on pense ? \\[0pt]
Où s'arrête la responsabilité ? \\[0pt]
Où s'arrête l'espace public ? \\[0pt]
Où sont les relations ? \\[0pt]
Où suis-je ? \\[0pt]
Où suis-je quand je pense ? \\[0pt]
Pourquoi ? \\[0pt]
Pourquoi accomplir son devoir ? \\[0pt]
Pourquoi aimer la liberté ? \\[0pt]
Pourquoi aimons-nous la musique ? \\[0pt]
Pourquoi aller contre son désir ? \\[0pt]
Pourquoi a-t-on peur de la folie ? \\[0pt]
Pourquoi avoir recours à la notion d'inconscient ? \\[0pt]
Pourquoi avons-nous besoin des autres pour être heureux ? \\[0pt]
Pourquoi avons-nous du mal à reconnaître la vérité ? \\[0pt]
Pourquoi avons-nous tant de mal à renoncer à nos illusions ? \\[0pt]
Pourquoi châtier ? \\[0pt]
Pourquoi chercher à connaître le passé ? \\[0pt]
Pourquoi chercher à se connaître soi-même ? \\[0pt]
Pourquoi chercher à se distinguer ? \\[0pt]
Pourquoi chercher à vivre libre ? \\[0pt]
Pourquoi chercher la vérité ? \\[0pt]
Pourquoi chercher un sens à l'histoire ? \\[0pt]
Pourquoi cherche-t-on à connaître ? \\[0pt]
Pourquoi commémorer ? \\[0pt]
Pourquoi communiquer ? \\[0pt]
Pourquoi conserver des œuvres d'art ? \\[0pt]
Pourquoi conserver les œuvres d'art ? \\[0pt]
Pourquoi construire des monuments ? \\[0pt]
Pourquoi critiquer la raison ? \\[0pt]
Pourquoi critiquer le conformisme ? \\[0pt]
Pourquoi croyons-nous ? \\[0pt]
Pourquoi défendre le faible ? \\[0pt]
Pourquoi définir ? \\[0pt]
Pourquoi délibérer ? \\[0pt]
Pourquoi démontrer ? \\[0pt]
Pourquoi démontrer ce que l'on sait être vrai ? \\[0pt]
Pourquoi des artifices ? \\[0pt]
Pourquoi des artistes ? \\[0pt]
Pourquoi des cérémonies ? \\[0pt]
Pourquoi des châtiments ? \\[0pt]
Pourquoi des classifications ? \\[0pt]
Pourquoi des comités d'éthique ? \\[0pt]
Pourquoi des conflits ? \\[0pt]
Pourquoi des constitutions ? \\[0pt]
Pourquoi des corps intermédiaires ? \\[0pt]
Pourquoi des définitions ? \\[0pt]
Pourquoi des devoirs ? \\[0pt]
Pourquoi des élections ? \\[0pt]
Pourquoi des exemples ? \\[0pt]
Pourquoi des fêtes ? \\[0pt]
Pourquoi des fictions ? \\[0pt]
Pourquoi des géométries ? \\[0pt]
Pourquoi des guerres ? \\[0pt]
Pourquoi des historiens ? \\[0pt]
Pourquoi des hommes politiques ? \\[0pt]
Pourquoi des hypothèses ? \\[0pt]
Pourquoi des idoles ? \\[0pt]
Pourquoi des institutions ? \\[0pt]
Pourquoi des interdits ? \\[0pt]
Pourquoi désirer ? \\[0pt]
Pourquoi désirer la sagesse ? \\[0pt]
Pourquoi désirer l'immortalité ? \\[0pt]
Pourquoi désire-t-on ce dont on n'a nul besoin ? \\[0pt]
Pourquoi désirons-nous ? \\[0pt]
Pourquoi des logiciens ? \\[0pt]
Pourquoi des lois ? \\[0pt]
Pourquoi des maîtres ? \\[0pt]
Pourquoi des métaphores ? \\[0pt]
Pourquoi des modèles ? \\[0pt]
Pourquoi des musées ? \\[0pt]
Pourquoi des œuvres d'art ? \\[0pt]
Pourquoi des philosophes ? \\[0pt]
Pourquoi des poètes ? \\[0pt]
Pourquoi des psychologues ? \\[0pt]
Pourquoi des religions ? \\[0pt]
Pourquoi des rites ? \\[0pt]
Pourquoi des sociologues ? \\[0pt]
Pourquoi des syllogismes ? \\[0pt]
Pourquoi des symboles ? \\[0pt]
Pourquoi des traditions ? \\[0pt]
Pourquoi des utopies ? \\[0pt]
Pourquoi dialogue-t-on ? \\[0pt]
Pourquoi Dieu se soucierait-il des affaires humaines ? \\[0pt]
Pourquoi dire la vérité ? \\[0pt]
Pourquoi distinguer nature et culture ? \\[0pt]
Pourquoi domestiquer ? \\[0pt]
Pourquoi donner ? \\[0pt]
Pourquoi donner des leçons de morale ? \\[0pt]
Pourquoi douter ? \\[0pt]
Pourquoi échanger des idées ? \\[0pt]
Pourquoi écrire ? \\[0pt]
Pourquoi écrit-on ? \\[0pt]
Pourquoi écrit-on des lois ? \\[0pt]
Pourquoi écrit-on les lois ? \\[0pt]
Pourquoi écrit-on l'Histoire ? \\[0pt]
Pourquoi est-il difficile de rectifier une erreur ? \\[0pt]
Pourquoi être exigeant ? \\[0pt]
Pourquoi être moral ? \\[0pt]
Pourquoi être raisonnable ? \\[0pt]
Pourquoi étudier le vivant ? \\[0pt]
Pourquoi étudier l'Histoire ? \\[0pt]
Pourquoi exposer les œuvres d'art ? \\[0pt]
Pourquoi faire confiance ? \\[0pt]
Pourquoi faire confiance au dialogue ? \\[0pt]
Pourquoi faire de la métaphysique ? \\[0pt]
Pourquoi faire de la politique ? \\[0pt]
Pourquoi faire de l'histoire ? \\[0pt]
Pourquoi faire la guerre ? \\[0pt]
Pourquoi faire l'hypothèse de l'inconscient ? \\[0pt]
Pourquoi faire son devoir ? \\[0pt]
Pourquoi fait-on le mal ? \\[0pt]
Pourquoi faudrait-il avoir peur de la technique ? \\[0pt]
Pourquoi faudrait-il être cohérent ? \\[0pt]
Pourquoi faudrait-il sauver les apparences ? \\[0pt]
Pourquoi faut-il agir ? \\[0pt]
Pourquoi faut-il diviser le travail ? \\[0pt]
Pourquoi faut-il être cohérent ? \\[0pt]
Pourquoi faut-il être juste ? \\[0pt]
Pourquoi faut-il être poli ? \\[0pt]
Pourquoi faut-il travailler ? \\[0pt]
Pourquoi formaliser des arguments ? \\[0pt]
Pourquoi il n'y a pas de société sans art ? \\[0pt]
Pourquoi imiter ? \\[0pt]
Pourquoi interprète-t-on ? \\[0pt]
Pourquoi joue-t-on ? \\[0pt]
Pourquoi la critique ? \\[0pt]
Pourquoi la curiosité est-elle un vilain défaut ? \\[0pt]
Pourquoi la guerre ? \\[0pt]
Pourquoi la justice a-t-elle besoin d'institutions ? \\[0pt]
Pourquoi la musique intéresse-t-elle le philosophe ? \\[0pt]
Pourquoi la prison ? \\[0pt]
Pourquoi la prohibition de l'inceste ? \\[0pt]
Pourquoi la raison recourt-elle à l'hypothèse ? \\[0pt]
Pourquoi la réalité peut-elle dépasser la fiction ? \\[0pt]
Pourquoi l'art intéresse-t-il les philosophes ? \\[0pt]
Pourquoi l'économie est-elle politique ? \\[0pt]
Pourquoi le droit international est-il imparfait ? \\[0pt]
Pourquoi les droits de l'homme sont-ils universels ? \\[0pt]
Pourquoi les États se font-ils la guerre ? \\[0pt]
Pourquoi les hommes jouent-ils ? \\[0pt]
Pourquoi les hommes mentent-ils ? \\[0pt]
Pourquoi les hommes se soumettent-ils à l'autorité ? \\[0pt]
Pourquoi les mathématiques s'appliquent-elles à la réalité ? \\[0pt]
Pourquoi les œuvres d'art résistent-elles au temps ? \\[0pt]
Pourquoi le sport ? \\[0pt]
Pourquoi les sciences humaines s'intéressent-elles à la parenté ? \\[0pt]
Pourquoi les sciences ont-elles une histoire ? \\[0pt]
Pourquoi les sociétés ont-elles besoin de lois ? \\[0pt]
Pourquoi l'État ? \\[0pt]
Pourquoi le théâtre ? \\[0pt]
Pourquoi l'ethnologue s'intéresse-t-il à la vie urbaine ? \\[0pt]
Pourquoi l'homme a-t-il des droits ? \\[0pt]
Pourquoi l'homme est-il l'objet de plusieurs sciences ? \\[0pt]
Pourquoi l'Homme se pose-t-il des questions métaphysiques ? \\[0pt]
Pourquoi l'homme travaille-t-il ? \\[0pt]
Pourquoi lire des romans ? \\[0pt]
Pourquoi lire les poètes ? \\[0pt]
Pourquoi lit-on des romans ? \\[0pt]
Pourquoi mentir ? \\[0pt]
Pourquoi ne peut-on concevoir la science comme achevée ? \\[0pt]
Pourquoi ne s'entend-on pas sur la nature de ce qui est réel ? \\[0pt]
Pourquoi nous racontons-nous des histoires ? \\[0pt]
Pourquoi nous soucier du sort des générations futures ? \\[0pt]
Pourquoi nous souvenons-nous ? \\[0pt]
Pourquoi nous trompons-nous ? \\[0pt]
Pourquoi n'y aurait-il pas de sots métiers ? \\[0pt]
Pourquoi obéir ? \\[0pt]
Pourquoi obéir aux lois ? \\[0pt]
Pourquoi obéit-on ? \\[0pt]
Pourquoi obéit-on aux lois ? \\[0pt]
Pourquoi oublie-t-on ? \\[0pt]
Pourquoi pardonner ? \\[0pt]
Pourquoi parler de fautes de goût ? \\[0pt]
Pourquoi parler de jugement de goût ? \\[0pt]
Pourquoi parler de « sciences exactes » ? \\[0pt]
Pourquoi parler du travail comme d'un droit ? \\[0pt]
Pourquoi parle-t-on ? \\[0pt]
Pourquoi parle-t-on d'économie politique ? \\[0pt]
Pourquoi parle-t-on d'une « société civile » ? \\[0pt]
Pourquoi parlons-nous ? \\[0pt]
Pourquoi pas ? \\[0pt]
Pourquoi pas plusieurs dieux ? \\[0pt]
Pourquoi peindre la nature ? \\[0pt]
Pourquoi penser à la mort ? \\[0pt]
Pourquoi penser l'impossible ? \\[0pt]
Pourquoi pensons-nous ? \\[0pt]
Pourquoi philosopher ? \\[0pt]
Pourquoi pleure-t-on ? \\[0pt]
Pourquoi pleure-t-on au cinéma ? \\[0pt]
Pourquoi plusieurs sciences ? \\[0pt]
Pourquoi préférer l'original ? \\[0pt]
Pourquoi préférer l'original à la copie ? \\[0pt]
Pourquoi préférer l'original à la reproduction ? \\[0pt]
Pourquoi préférer l'original à sa reproduction ? \\[0pt]
Pourquoi préserver la nature ? \\[0pt]
Pourquoi préserver l'environnement ? \\[0pt]
Pourquoi prier ? \\[0pt]
Pourquoi promettre ? \\[0pt]
Pourquoi prouver l'existence de Dieu ? \\[0pt]
Pourquoi punir ? \\[0pt]
Pourquoi punit-on ? \\[0pt]
Pourquoi raconter des histoires ? \\[0pt]
Pourquoi rechercher la vérité ? \\[0pt]
Pourquoi rechercher le bonheur ? \\[0pt]
Pourquoi refuser de faire son devoir ? \\[0pt]
Pourquoi refuse-t-on la conscience à l'animal ? \\[0pt]
Pourquoi respecter autrui ? \\[0pt]
Pourquoi respecter la nature ? \\[0pt]
Pourquoi respecter le droit ? \\[0pt]
Pourquoi respecter les anciens ? \\[0pt]
Pourquoi rit-on ? \\[0pt]
Pourquoi sauver les apparences ? \\[0pt]
Pourquoi sauver les phénomènes ? \\[0pt]
Pourquoi se confesser ? \\[0pt]
Pourquoi se cultiver ? \\[0pt]
Pourquoi se divertir ? \\[0pt]
Pourquoi se fier à autrui ? \\[0pt]
Pourquoi se mettre à la place d'autrui ? \\[0pt]
Pourquoi s'ennuie-t-on ? \\[0pt]
Pourquoi séparer les pouvoirs ? \\[0pt]
Pourquoi se référer au destin ? \\[0pt]
Pourquoi se révolter ? \\[0pt]
Pourquoi se soucier du futur ? \\[0pt]
Pourquoi se taire ? \\[0pt]
Pourquoi s'étonner ? \\[0pt]
Pourquoi s'exprimer ? \\[0pt]
Pourquoi s'inspirer de l'art antique ? \\[0pt]
Pourquoi s'intéresser à l'histoire ? \\[0pt]
Pourquoi s'intéresser à l'origine ? \\[0pt]
Pourquoi s'interroger sur l'origine du langage ? \\[0pt]
Pourquoi soigner son apparence ? \\[0pt]
Pourquoi sommes-nous déçus par les œuvres d'un faussaire ? \\[0pt]
Pourquoi sommes-nous des êtres moraux ? \\[0pt]
Pourquoi sommes-nous moraux ? \\[0pt]
Pourquoi suivre l'actualité ? \\[0pt]
Pourquoi suspendre son jugement ? \\[0pt]
Pourquoi tenir ses promesses ? \\[0pt]
Pourquoi théoriser ? \\[0pt]
Pourquoi transformer le monde ? \\[0pt]
Pourquoi transmettre ? \\[0pt]
Pourquoi travailler ? \\[0pt]
Pourquoi travaille-t-on ? \\[0pt]
Pourquoi un droit du travail ? \\[0pt]
Pourquoi une instruction publique ? \\[0pt]
Pourquoi un fait devrait-il être établi ? \\[0pt]
Pourquoi veut-on changer le monde ? \\[0pt]
Pourquoi veut-on la vérité ? \\[0pt]
Pourquoi vivons-nous ? \\[0pt]
Pourquoi vivre ensemble ? \\[0pt]
Pourquoi vouloir avoir raison ? \\[0pt]
Pourquoi vouloir devenir « comme maîtres et possesseurs de la nature » ? \\[0pt]
Pourquoi vouloir être libre ? \\[0pt]
Pourquoi vouloir gagner ? \\[0pt]
Pourquoi vouloir la vérité ? \\[0pt]
Pourquoi vouloir s'abstraire du quotidien ? \\[0pt]
Pourquoi vouloir se connaître ? \\[0pt]
Pourquoi voulons-nous savoir ? \\[0pt]
Pourquoi voyager ? \\[0pt]
Pourquoi y a-t-il des conflits insolubles ? \\[0pt]
Pourquoi y a-t-il des institutions ? \\[0pt]
Pourquoi y a-t-il des lois ? \\[0pt]
Pourquoi y a-t-il des religions ? \\[0pt]
Pourquoi y a-t-il du mal dans le monde ? \\[0pt]
Pourquoi y a-t-il plusieurs façons de démontrer ? \\[0pt]
Pourquoi y a-t-il plusieurs langues ? \\[0pt]
Pourquoi y a-t-il plusieurs philosophies ? \\[0pt]
Pourquoi y a-t-il plusieurs sciences ? \\[0pt]
Pourquoi y a-t-il quelque chose plutôt que rien ? \\[0pt]
Pourquoi y a-t-il une philosophie de la vie ? \\[0pt]
Qu'admire-t-on dans une œuvre d'art ? \\[0pt]
Qu'ai-je le droit d'exiger d'autrui ? \\[0pt]
Qu'ai-je le droit d'exiger des autres ? \\[0pt]
Qu'aime-t-on ? \\[0pt]
Qu'aime-t-on dans l'amour ? \\[0pt]
Qu'aime-t-on quand on aime ? \\[0pt]
Qu'aime-t-on quand on aime une œuvre d'art ? \\[0pt]
Qu'aimons-nous dans l'amour ? \\[0pt]
Quand agit-on ? \\[0pt]
Quand est-on stupide ? \\[0pt]
Quand faut-il désobéir ? \\[0pt]
Quand faut-il désobéir aux lois ? \\[0pt]
Quand faut-il mentir ? \\[0pt]
Quand faut-il se taire ? \\[0pt]
Quand la guerre finira-t-elle ? \\[0pt]
Quand l'art est-il abstrait ? \\[0pt]
Quand la technique devient-elle art ? \\[0pt]
Quand le temps passe, que reste-t-il ? \\[0pt]
Quand manquons-nous à nos devoirs ? \\[0pt]
Quand pense-t-on ? \\[0pt]
Quand peut-on se passer de théories ? \\[0pt]
Quand suis-je en faute ? \\[0pt]
Quand suis-je moi-même ? \\[0pt]
Quand une autorité est-elle légitime ? \\[0pt]
Quand y a-t-il art ? \\[0pt]
Quand y a-t-il de l'art ? \\[0pt]
Quand y a-t-il métaphore ? \\[0pt]
Quand y a-t-il œuvre ? \\[0pt]
Quand y a-t-il paysage ? \\[0pt]
Quand y a-t-il peuple ? \\[0pt]
Qu'anticipent les romans d'anticipation ? \\[0pt]
Qu'a perdu le fou ? \\[0pt]
Qu'appelle-t-on chef-d'œuvre ? \\[0pt]
Qu'appelle-t-on destin ? \\[0pt]
Qu'appelle-t-on penser ? \\[0pt]
Qu'apporte au mathématicien l'histoire des mathématiques ? \\[0pt]
Qu'apporte la photographie aux arts ? \\[0pt]
Qu'apporte l'histoire à la vie politique ? \\[0pt]
Qu'apprend-on dans les livres ? \\[0pt]
Qu'apprend-on de ses erreurs ? \\[0pt]
Qu'apprend-on des romans ? \\[0pt]
Qu'apprend-on en apprenant une technique ? \\[0pt]
Qu'apprend-on en commettant une faute ? \\[0pt]
Qu'apprend-on quand on apprend à parler ? \\[0pt]
Qu'apprenons-nous de nos affects ? \\[0pt]
Qu'apprenons-nous en éduquant ? \\[0pt]
Qu'a-t-on le droit de pardonner ? \\[0pt]
Qu'a-t-on le droit d'exiger ? \\[0pt]
Qu'a-t-on le droit d'interdire ? \\[0pt]
Qu'a-t-on le droit d'interpréter ? \\[0pt]
Qu'attendons-nous de la science ? \\[0pt]
Qu'attendons-nous de la technique ? \\[0pt]
Qu'attendons-nous des œuvres d'art ? \\[0pt]
Qu'attendons-nous pour être heureux ? \\[0pt]
Qu'attendre de l'État ? \\[0pt]
Qu'avons-nous à apprendre de nos illusions ? \\[0pt]
Qu'avons-nous à apprendre des historiens ? \\[0pt]
 Qu'avons-nous à attendre de la vie politique ? \\[0pt]
Qu'avons-nous en commun ? \\[0pt]
Que célèbre l'art ? \\[0pt]
Que changent les révolutions ? \\[0pt]
Qu'échange-t-on ? \\[0pt]
Que cherchons-nous dans le regard des autres ? \\[0pt]
Que choisir ? \\[0pt]
Que comprend-on d'une œuvre d'art ? \\[0pt]
Que connaissons-nous du vivant ? \\[0pt]
Que connaît-on de ses passions ? \\[0pt]
Que construit le politique ? \\[0pt]
Que coûte une victoire ? \\[0pt]
Que crée l'artiste ? \\[0pt]
Que déduire d'une contradiction ? \\[0pt]
Que démontrent nos actions ? \\[0pt]
Que désire-t-on ? \\[0pt]
Que désirons-nous ? \\[0pt]
Que désirons-nous quand nous désirons savoir ? \\[0pt]
Que devons-nous à autrui ? \\[0pt]
Que devons-nous à l'animal ? \\[0pt]
Que devons-nous à l'État ? \\[0pt]
Que disent les légendes ? \\[0pt]
Que disent les tables de vérité ? \\[0pt]
Que dit la loi ? \\[0pt]
Que dit la musique ? \\[0pt]
Que dit l'interdit ? \\[0pt]
Que dois-je à autrui ? \\[0pt]
Que dois-je à l'État ? \\[0pt]
Que dois-je faire ? \\[0pt]
Que dois-je respecter en autrui ? \\[0pt]
Que doit la pensée à l'écriture ? \\[0pt]
Que doit l'art à la nature ? \\[0pt]
Que doit la science à la technique ? \\[0pt]
Que doit-on à autrui ? \\[0pt]
Que doit-on à l'État ? \\[0pt]
Que doit-on aux morts ? \\[0pt]
Que doit-on croire ? \\[0pt]
Que doit-on désirer pour ne pas être déçu ? \\[0pt]
Que doit-on faire de ses rêves ? \\[0pt]
Que doit-on protéger ? \\[0pt]
Que doit-on savoir avant d'agir ? \\[0pt]
Que faire ? \\[0pt]
Que faire de la diversité des arts ? \\[0pt]
Que faire de la violence ? \\[0pt]
Que faire de nos émotions ? \\[0pt]
Que faire de nos passions ? \\[0pt]
Que faire de notre cerveau ? \\[0pt]
Que faire des adversaires ? \\[0pt]
Que faire du vraisemblable ? \\[0pt]
Que faire quand la loi est injuste ? \\[0pt]
Que fait aux œuvres d'art leur reproductibilité ? \\[0pt]
Que fait la machine ? \\[0pt]
Que fait la police ? \\[0pt]
Que fait l'art à nos vies ? \\[0pt]
Que fait l'artiste ? \\[0pt]
Que fait le spectateur ? \\[0pt]
Que faut-il absolument savoir ? \\[0pt]
Que faut-il craindre ? \\[0pt]
Que faut-il pour faire un monde ? \\[0pt]
Que faut-il respecter ? \\[0pt]
Que faut-il savoir pour agir ? \\[0pt]
Que faut-il savoir pour bien agir ? \\[0pt]
Que faut-il savoir pour gouverner ? \\[0pt]
Que faut-il savoir pour pouvoir gouverner ? \\[0pt]
Que faut-il vouloir pour être heureux ? \\[0pt]
Que gagne l'art à devenir abstrait ? \\[0pt]
Que gagne l'art à se réfléchir ? \\[0pt]
Que gagne-t-on à se mettre à la place d'autrui ? \\[0pt]
Que gagne-t-on à travailler ? \\[0pt]
Que garantit la séparation des pouvoirs ? \\[0pt]
Que la nature soit explicable, est-ce explicable ? \\[0pt]
Quel besoin avons-nous de chercher la vérité ? \\[0pt]
Quel contrôle a-t-on sur son corps ? \\[0pt]
Quel est le bon nombre d'amis ? \\[0pt]
Quel est le but de la politique ? \\[0pt]
Quel est le but d'une théorie physique ? \\[0pt]
Quel est le but du travail scientifique ? \\[0pt]
Quel est le contraire du travail ? \\[0pt]
Quel est le fondement de la propriété ? \\[0pt]
Quel est le fondement de l'autorité ? \\[0pt]
Quel est le meilleur régime politique ? \\[0pt]
Quel est le poids du passé ? \\[0pt]
Quel est le pouvoir de la beauté ? \\[0pt]
Quel est le pouvoir de l'art ? \\[0pt]
Quel est le pouvoir des métaphores ? \\[0pt]
Quel est le pouvoir des mots ? \\[0pt]
Quel est le problème posé par la pluralité des langues ? \\[0pt]
Quel est le propre de la pensée technique ? \\[0pt]
Quel est le rôle de la créativité dans les sciences ? \\[0pt]
Quel est le rôle du concept en art ? \\[0pt]
Quel est le rôle du médecin ? \\[0pt]
Quel est le sens du progrès technique ? \\[0pt]
Quel est le statut de la preuve en sciences humaines et sociales ? \\[0pt]
Quel est le sujet de la pensée ? \\[0pt]
Quel est le sujet de l'histoire ? \\[0pt]
Quel est le sujet des sciences sociales ? \\[0pt]
Quel est le sujet du devenir ? \\[0pt]
Quel est l'être de l'illusion ? \\[0pt]
Quel est l'homme de l'ethnologie ? \\[0pt]
Quel est l'homme des Droits de l'homme ? \\[0pt]
Quel est l'humain des sciences humaines ? \\[0pt]
Quel est l'objectif des sciences humaines ? \\[0pt]
Quel est l'objet de la biologie ? \\[0pt]
Quel est l'objet de la conscience de soi ? \\[0pt]
Quel est l'objet de la géométrie ? \\[0pt]
Quel est l'objet de la logique ? \\[0pt]
Quel est l'objet de la métaphysique ? \\[0pt]
Quel est l'objet de l'amour ? \\[0pt]
Quel est l'objet de la perception ? \\[0pt]
Quel est l'objet de la philosophie politique ? \\[0pt]
Quel est l'objet de la physique ? \\[0pt]
Quel est l'objet de la psychologie ? \\[0pt]
Quel est l'objet de la psychologie collective ? \\[0pt]
Quel est l'objet de la psychologie sociale ? \\[0pt]
Quel est l'objet de la science ? \\[0pt]
Quel est l'objet de l'échange ? \\[0pt]
Quel est l'objet de l'esthétique ? \\[0pt]
Quel est l'objet de l'histoire ? \\[0pt]
Quel est l'objet de l'ontologie ? \\[0pt]
Quel est l'objet des mathématiques ? \\[0pt]
Quel est l'objet des « sciences de l'esprit » ? \\[0pt]
Quel est l'objet des sciences humaines ? \\[0pt]
Quel est l'objet des sciences politiques ? \\[0pt]
Quel est l'objet du désir ? \\[0pt]
Quel être peut être un sujet de droits ? \\[0pt]
Quel genre de conscience peut-on accorder à l'animal ? \\[0pt]
Quel intérêt y a-t-il à se connaître ? \\[0pt]
Quelle causalité pour le vivant ? \\[0pt]
« Quelle chimère est-ce donc que l'homme » ? \\[0pt]
Quelle confiance accorder au langage ? \\[0pt]
Quelle différence y a-t-il entre un corps vivant et un corps mort ? \\[0pt]
Quelle espèce de vivants sommes-nous ? \\[0pt]
Quelle est la cause du désir ? \\[0pt]
Quelle est la fin de la science ? \\[0pt]
Quelle est la fin de l'État ? \\[0pt]
Quelle est la fonction première de l'État ? \\[0pt]
Quelle est la force de la loi ? \\[0pt]
Quelle est la limite du pouvoir de l'État ? \\[0pt]
Quelle est la matière de l'œuvre d'art ? \\[0pt]
Quelle est la nature du droit naturel ? \\[0pt]
Quelle est la place de l'imagination dans la vie de l'esprit ? \\[0pt]
Quelle est la place du hasard dans l'histoire ? \\[0pt]
Quelle est la place du sujet dans les sciences humaines ? \\[0pt]
Quelle est la portée d'un exemple ? \\[0pt]
Quelle est la réalité de la matière ? \\[0pt]
Quelle est la réalité de l'avenir ? \\[0pt]
Quelle est la réalité des objets mathématiques ? \\[0pt]
Quelle est la réalité d'une idée ? \\[0pt]
Quelle est la réalité du passé ? \\[0pt]
Quelle est la réalité du temps ? \\[0pt]
Quelle est la source de nos devoirs ? \\[0pt]
Quelle est la spécificité de la communauté politique ? \\[0pt]
Quelle est la valeur culturelle de la science ? \\[0pt]
Quelle est la valeur de la connaissance ? \\[0pt]
Quelle est la valeur de l'expérience ? \\[0pt]
Quelle est la valeur des hypothèses ? \\[0pt]
Quelle est la valeur d'une expérimentation ? \\[0pt]
Quelle est la valeur d'une œuvre d'art ? \\[0pt]
Quelle est la valeur d'une promesse ? \\[0pt]
Quelle est la valeur du rêve ? \\[0pt]
Quelle est la valeur du témoignage ? \\[0pt]
Quelle est la valeur du temps ? \\[0pt]
Quelle est la valeur du vivant ? \\[0pt]
Quelle est l'unité du « je » ? \\[0pt]
Quelle expérience la politique requiert-elle ? \\[0pt]
Quelle idée le fanatique se fait-il de la vérité ? \\[0pt]
Quelle maîtrise avons-nous du temps ? \\[0pt]
Quelle peut être la force de nos idées ? \\[0pt]
Quelle place donner au vraisemblable ? \\[0pt]
Quelle place la raison peut-elle faire à la croyance ? \\[0pt]
Quelle place les sciences humaines doivent-elles accorder à la subjectivité ? \\[0pt]
Quelle place pour la mesure dans les sciences de l'homme ? \\[0pt]
Quelle politique fait-on avec les sciences humaines ? \\[0pt]
Quelle réalité attribuer à la matière ? \\[0pt]
Quelle réalité conférer aux fictions ? \\[0pt]
Quelle réalité l'art nous fait-il connaître ? \\[0pt]
Quelle réalité la science décrit-elle ? \\[0pt]
Quelle réalité l'imagination nous fait-elle connaître ? \\[0pt]
Quelle réalité peut-on accorder au temps ? \\[0pt]
Quelle réalité peut-on attribuer au temps ? \\[0pt]
Quelle responsabilité publique pour le chercheur en sciences sociales ? \\[0pt]
Quelles actions permettent d'être heureux ? \\[0pt]
Quelles limites assigner au pouvoir de la parole ? \\[0pt]
Quelles limites poser à l'autorité de l'État ? \\[0pt]
Quelles lois pour les sciences humaines ? \\[0pt]
Quelle sorte d'histoire ont les sciences ? \\[0pt]
Quelles règles la technique dicte-t-elle à l'art ? \\[0pt]
Quelles sont les caractéristiques d'une proposition morale ? \\[0pt]
Quelles sont les caractéristiques d'un être vivant ? \\[0pt]
Quelles sont les limites de la démonstration ? \\[0pt]
Quelles sont les limites de la souveraineté ? \\[0pt]
Quelles sont les limites de mon monde ? \\[0pt]
Quelle valeur accorder à l'expérience ? \\[0pt]
Quelle valeur devons accorder à l'expérience ? \\[0pt]
Quelle valeur devons-nous accorder à l'expérience ? \\[0pt]
Quelle valeur devons-nous accorder à l'intuition ? \\[0pt]
Quelle valeur donner à la notion de « corps social » ? \\[0pt]
Quelle valeur peut-on accorder à l'expérience ? \\[0pt]
Quelle vérité y a-t-il dans la perception ? \\[0pt]
Quel réel pour l'art ? \\[0pt]
Quel rôle attribuer à l'intuition \emph{a priori} dans une philosophie des mathématiques ? \\[0pt]
Quel rôle la logique joue-t-elle en mathématiques ? \\[0pt]
Quel rôle les statistiques jouent-elles dans les sciences humaines ? \\[0pt]
Quel rôle l'imagination joue-t-elle en mathématiques ? \\[0pt]
Quels critères de démarcation pour les sciences sociales ? \\[0pt]
Quels désirs dois-je m'interdire ? \\[0pt]
Quels devoirs les religions peuvent-elles énoncer ? \\[0pt]
Quel sens donner à l'expression « gagner sa vie » ? \\[0pt]
Quel sens donner au projet de forger une langue parfaite ? \\[0pt]
Quels enseignements peut-on tirer de l'histoire des sciences ? \\[0pt]
Quel sens y a-t-il à se demander si les sciences humaines sont vraiment des sciences ? \\[0pt]
Quels matériaux pour les sciences humaines et sociales ? \\[0pt]
Quels sont les droits de la conscience ? \\[0pt]
Quels sont les fondements de l'autorité ? \\[0pt]
Quels sont les moyens légitimes de la politique ? \\[0pt]
Quel statut philosophique pour l'opinion ? \\[0pt]
Quel type de réalité faut-il attribuer à l'inconscient en psychanalyse ? \\[0pt]
Quel type de réalité faut-il attribuer au temps ? \\[0pt]
Quel usage faire de la finalité ? \\[0pt]
Quel usage faut-il faire des exemples ? \\[0pt]
Quel usage peut-on faire des fictions ? \\[0pt]
Quel vivant sommes-nous ? \\[0pt]
Que manque-t-il à une machine pour être vivante ? \\[0pt]
Que manque-t-il aux machines pour être des organismes ? \\[0pt]
Que mesure-t-on du temps ? \\[0pt]
Que montre l'image ? \\[0pt]
Que montre une démonstration ? \\[0pt]
Que montre un tableau ? \\[0pt]
Que m'ordonne ma conscience ? \\[0pt]
Que ne peut-on pas expliquer ? \\[0pt]
Que nous append l'histoire ? \\[0pt]
Que nous apporte l'art ? \\[0pt]
Que nous apporte la technique ? \\[0pt]
Que nous apporte la vérité ? \\[0pt]
Que nous apporte le travail ? \\[0pt]
Que nous apprend la biologie sur l'homme ? \\[0pt]
Que nous apprend la définition de la vérité ? \\[0pt]
Que nous apprend la diversité des langues ? \\[0pt]
Que nous apprend la fiction sur la réalité ? \\[0pt]
Que nous apprend la géométrie ? \\[0pt]
Que nous apprend la grammaire ? \\[0pt]
Que nous apprend la littérature ? \\[0pt]
Que nous apprend la maladie ? \\[0pt]
Que nous apprend la maladie sur la santé ? \\[0pt]
Que nous apprend la métaphysique ? \\[0pt]
Que nous apprend la musique ? \\[0pt]
Que nous apprend l'animal ? \\[0pt]
Que nous apprend l'anthropologie ? \\[0pt]
Que nous apprend la poésie ? \\[0pt]
Que nous apprend l'approche sociologique de l'art ? \\[0pt]
Que nous apprend la psychanalyse de l'homme ? \\[0pt]
Que nous apprend la religion ? \\[0pt]
Que nous apprend la sociologie des sciences ? \\[0pt]
Que nous apprend la technique ? \\[0pt]
Que nous apprend la vie ? \\[0pt]
Que nous apprend le cinéma ? \\[0pt]
Que nous apprend le faux ? \\[0pt]
Que nous apprend le plaisir ? \\[0pt]
Que nous apprend l'erreur ? \\[0pt]
Que nous apprend le témoignage ? \\[0pt]
Que nous apprend le théâtre ? \\[0pt]
Que nous apprend le toucher ? \\[0pt]
Que nous apprend l'étude du cerveau ? \\[0pt]
Que nous apprend l'expérience ? \\[0pt]
Que nous apprend l'expérience esthétique ? \\[0pt]
Que nous apprend l'histoire de l'art ? \\[0pt]
Que nous apprend l'histoire des sciences ? \\[0pt]
Que nous apprend, sur la politique, l'utopie ? \\[0pt]
Que nous apprennent les algorithmes sur nos sociétés ? \\[0pt]
Que nous apprennent les Anciens ? \\[0pt]
Que nous apprennent les animaux ? \\[0pt]
Que nous apprennent les animaux sur nous-mêmes ? \\[0pt]
Que nous apprennent les controverses scientifiques ? \\[0pt]
Que nous apprennent les expériences de pensée ? \\[0pt]
Que nous apprennent les faits divers ? \\[0pt]
Que nous apprennent les illusions d'optique ? \\[0pt]
Que nous apprennent les jeux ? \\[0pt]
Que nous apprennent les langues étrangères ? \\[0pt]
Que nous apprennent les machines ? \\[0pt]
Que nous apprennent les métaphores ? \\[0pt]
Que nous apprennent les mythes ? \\[0pt]
Que nous apprennent les objets techniques ? \\[0pt]
Que nous apprennent les statistiques ? \\[0pt]
Que nous apprennent nos erreurs ? \\[0pt]
Que nous apprennent nos sentiments ? \\[0pt]
Que nous devons-nous ? \\[0pt]
Que nous doit l'État ? \\[0pt]
Que nous enseigne la lecture des romans ? \\[0pt]
Que nous enseigne la monstruosité ? \\[0pt]
Que nous enseigne l'expérience ? \\[0pt]
Que nous enseignent les œuvres d'art ? \\[0pt]
Que nous enseignent les sens ? \\[0pt]
Que nous enseignent nos peurs ? \\[0pt]
Que nous impose la nature ? \\[0pt]
Que nous impose le temps ? \\[0pt]
Que nous montre le cinéma ? \\[0pt]
Que nous montre l'œuvre d'art ? \\[0pt]
Que nous montrent les natures mortes ? \\[0pt]
Que nous réserve l'avenir ? \\[0pt]
Que partage-t-on avec les animaux ? \\[0pt]
Que peindre ? \\[0pt]
Que peint le peintre ? \\[0pt]
Que penser de l'adage : « Que la justice s'accomplisse, le monde dût-il périr » (Fiat justitia pereat mundus) ? \\[0pt]
Que penser de la formule : « il faut suivre la nature » ? \\[0pt]
Que penser de l'opposition travail manuel, travail intellectuel ? \\[0pt]
Que percevons-nous ? \\[0pt]
Que percevons-nous d'autrui ? \\[0pt]
Que percevons-nous du monde extérieur ? \\[0pt]
Que perçoit-on ? \\[0pt]
Que perd la pensée en perdant l'écriture ? \\[0pt]
Que perd-on quand on perd son temps ? \\[0pt]
Que perdrait la pensée en perdant l'écriture ? \\[0pt]
Que peut expliquer l'histoire ? \\[0pt]
Que peut la force ? \\[0pt]
Que peut la musique ? \\[0pt]
Que peut la pensée ? \\[0pt]
Que peut la philosophie ? \\[0pt]
Que peut la politique ? \\[0pt]
Que peut la raison ? \\[0pt]
Que peut la raison contre une croyance ? \\[0pt]
Que peut l'art ? \\[0pt]
Que peut la science ? \\[0pt]
Que peut la théorie ? \\[0pt]
Que peut la volonté ? \\[0pt]
Que peut la volonté contre les passions ? \\[0pt]
Que peut le corps ? \\[0pt]
Que peut le droit ? \\[0pt]
Que peut le politique ? \\[0pt]
Que peut l'esprit ? \\[0pt]
Que peut l'esprit sur la matière ? \\[0pt]
Que peut l'État ? \\[0pt]
Que peut l'intelligence artificielle ? \\[0pt]
Que peut-on apprendre des émotions esthétiques ? \\[0pt]
Que peut-on attendre de l'État ? \\[0pt]
Que peut-on attendre du droit international ? \\[0pt]
Que peut-on attendre d'une philosophie de l'art ? \\[0pt]
Que peut-on calculer ? \\[0pt]
Que peut-on comprendre immédiatement ? \\[0pt]
Que peut-on comprendre qu'on ne puisse expliquer ? \\[0pt]
Que peut-on contre un préjugé ? \\[0pt]
Que peut-on cultiver ? \\[0pt]
Que peut-on démontrer ? \\[0pt]
Que peut-on dire de l'être ? \\[0pt]
Que peut-on échanger ? \\[0pt]
Que peut-on enseigner ? \\[0pt]
Que peut-on espérer ? \\[0pt]
Que peut-on exprimer ? \\[0pt]
Que peut-on interdire ? \\[0pt]
Que peut-on partager ? \\[0pt]
Que peut-on perdre ? \\[0pt]
Que peut-on savoir de l'inconscient ? \\[0pt]
Que peut-on savoir de soi ? \\[0pt]
Que peut-on savoir du réel ? \\[0pt]
Que peut-on savoir par expérience ? \\[0pt]
Que peut-on sur autrui ? \\[0pt]
Que peut-on voir ? \\[0pt]
Que peut prétendre imposer une religion ? \\[0pt]
Que peut signifier « avoir le sens de la réalité » ? \\[0pt]
Que peut signifier : « gérer son temps » ? \\[0pt]
Que peut-signifier « tuer le temps » ? \\[0pt]
Que peut un corps ? \\[0pt]
Que peuvent les idées ? \\[0pt]
Que peuvent les images ? \\[0pt]
Que peuvent les lois ? \\[0pt]
Que peuvent nous apprendre les expériences de pensée ? \\[0pt]
Que pouvons-nous attendre de la technique ? \\[0pt]
Que pouvons-nous aujourd'hui apprendre des sciences d'autrefois ? \\[0pt]
Que pouvons-nous comprendre du monde ? \\[0pt]
Que pouvons-nous connaître ? \\[0pt]
Que pouvons-nous espérer ? \\[0pt]
Que pouvons-nous espérer de la connaissance du vivant ? \\[0pt]
Que pouvons-nous faire de notre passé ? \\[0pt]
Que pouvons-nous partager ? \\[0pt]
Que produit la poésie ? \\[0pt]
Que produit l'inconscient ? \\[0pt]
Que protège la loi ? \\[0pt]
Que prouvent les faits ? \\[0pt]
Que prouvent les preuves de l'existence de Dieu ? \\[0pt]
Que puis-je dire être mien ? \\[0pt]
Que recherche l'artiste ? \\[0pt]
Que rend visible l'art ? \\[0pt]
Que répondre au sceptique ? \\[0pt]
Que reste-t-il d'une existence ? \\[0pt]
Que reste-t-il du passé ? \\[0pt]
Que sais-je d'autrui ? \\[0pt]
Que sais-je de ma souffrance ? \\[0pt]
Que sait la conscience ? \\[0pt]
Que sait la science ? \\[0pt]
Que sait-on de soi ? \\[0pt]
Que sait-on du réel ? \\[0pt]
Que savons-nous de l'avenir ? \\[0pt]
Que savons-nous de l'inconscient ? \\[0pt]
Que savons-nous de notre vie ? \\[0pt]
Que savons-nous de nous-mêmes ? \\[0pt]
Que savons-nous des principes ? \\[0pt]
Que sent le corps ? \\[0pt]
Que serait la fin de l'histoire ? \\[0pt]
Que serait la vie sans l'art ? \\[0pt]
Que serait le meilleur des mondes ? \\[0pt]
Que serait un art total ? \\[0pt]
Que serait une démocratie parfaite ? \\[0pt]
Que serait une pure sensation ? \\[0pt]
Que serions-nous sans l'État ? \\[0pt]
Que signifie apprendre ? \\[0pt]
Que signifie : « avoir de l'esprit » ? \\[0pt]
Que signifie connaître ? \\[0pt]
Que signifie « donner le change » ? \\[0pt]
Que signifie : « échanger des arguments » ? \\[0pt]
Que signifie être dépendant ? \\[0pt]
Que signifie être en guerre ? \\[0pt]
Que signifie être mortel ? \\[0pt]
Que signifie la mort ? \\[0pt]
Que signifie l'angoisse ? \\[0pt]
Que signifie l'expression : « l'histoire jugera » ? \\[0pt]
Que signifie l'idée de technoscience ? \\[0pt]
Que signifient les mots ? \\[0pt]
Que signifie « politiquement correct » ? \\[0pt]
Que signifie pour l'homme être mortel ? \\[0pt]
Que signifie prouver en sciences sociales ? \\[0pt]
Que signifie refuser l'injustice ? \\[0pt]
Que signifier « juger en son âme et conscience » ? \\[0pt]
Que signifie : « se rendre à l'évidence » ? \\[0pt]
Que sondent les sondages d'opinion ? \\[0pt]
Que sont les apparences ? \\[0pt]
Que sont les institutions sans les individus ? \\[0pt]
Que sont les objets de la perception ? \\[0pt]
Qu'est-ce la vérité ? \\[0pt]
Qu'est-ce qu'abstraire ? \\[0pt]
Qu'est-ce qu'agir avec discernement ? \\[0pt]
Qu'est-ce qu'agir ensemble ? \\[0pt]
Qu'est-ce qu'aimer une œuvre d'art ? \\[0pt]
Qu'est-ce qu'apparaître ? \\[0pt]
Qu'est-ce qu'appliquer une règle de droit ? \\[0pt]
Qu'est-ce qu'apprécier une œuvre d'art ? \\[0pt]
Qu'est-ce qu'apprendre ? \\[0pt]
Qu'est-ce qu'argumenter ? \\[0pt]
Qu'est-ce qu'avoir conscience de soi ? \\[0pt]
Qu'est-ce qu'avoir de l'expérience ? \\[0pt]
Qu'est-ce qu'avoir du goût ? \\[0pt]
Qu'est-ce qu'avoir du jugement ? \\[0pt]
Qu'est-ce qu'avoir du style ? \\[0pt]
Qu'est-ce qu'avoir raison ? \\[0pt]
Qu'est-ce qu'avoir un droit ? \\[0pt]
Qu'est-ce qu'avoir une idée ? \\[0pt]
Qu'est-ce qu'avoir un esprit scientifique ? \\[0pt]
Qu'est-ce que bien vivre ? \\[0pt]
Qu'est-ce que calculer ? \\[0pt]
Qu'est-ce que catégoriser ? \\[0pt]
Qu'est-ce que classer ? \\[0pt]
Qu'est-ce que commencer ? \\[0pt]
Qu'est-ce que composer une œuvre ? \\[0pt]
Qu'est-ce que comprendre ? \\[0pt]
Qu'est-ce que comprendre une œuvre d'art ? \\[0pt]
Qu'est-ce que connaître ? \\[0pt]
Qu'est-ce que contempler ? \\[0pt]
Qu'est-ce que créer ? \\[0pt]
Qu'est-ce que croire ? \\[0pt]
Qu'est-ce que décider ? \\[0pt]
Qu'est-ce que définir ? \\[0pt]
Qu'est-ce que démontrer ? \\[0pt]
Qu'est-ce que déraisonner ? \\[0pt]
Qu'est-ce que Dieu pour athée ? \\[0pt]
Qu'est-ce que Dieu pour un athée ? \\[0pt]
Qu'est-ce que discuter ? \\[0pt]
Qu'est-ce que donner sa parole ? \\[0pt]
Qu'est-ce qu'éduquer ? \\[0pt]
Qu'est-ce qu'éduquer le sens esthétique ? \\[0pt]
Qu'est-ce que faire autorité ? \\[0pt]
Qu'est-ce que « faire de la politique » ? \\[0pt]
Qu'est-ce que faire la paix ? \\[0pt]
Qu'est-ce que faire preuve d'humanité ? \\[0pt]
Qu'est-ce que faire une expérience ? \\[0pt]
Qu'est-ce que « faire usage de sa raison » ? \\[0pt]
Qu'est-ce que gouverner ? \\[0pt]
Qu'est-ce que guérir ? \\[0pt]
Qu'est-ce que jouer ? \\[0pt]
Qu'est-ce que juger ? \\[0pt]
Qu'est-ce que la barbarie ? \\[0pt]
Qu'est-ce que la causalité ? \\[0pt]
Qu'est-ce que la critique ? \\[0pt]
Qu'est-ce que la démocratie ? \\[0pt]
Qu'est-ce que la folie ? \\[0pt]
Qu'est-ce que la normalité ? \\[0pt]
Qu'est-ce que l'anthropologie politique ? \\[0pt]
Qu'est-ce que la perception ? \\[0pt]
Qu'est-ce que la politique ? \\[0pt]
Qu'est-ce que la psychanalyse nous apprend sur les sociétés humaines ? \\[0pt]
Qu'est-ce que la psychologie ? \\[0pt]
Qu'est-ce que la raison d'État ? \\[0pt]
Qu'est-ce que la réalité ? \\[0pt]
Qu'est-ce que la religion nous donne à savoir ? \\[0pt]
Qu'est-ce que l'art contemporain ? \\[0pt]
Qu'est-ce que la science doit à l'expérience ? \\[0pt]
Qu'est-ce que la science saisit du vivant ? \\[0pt]
Qu'est-ce que la science, si elle inclut la psychanalyse ? \\[0pt]
Qu'est-ce que la scientificité ? \\[0pt]
Qu'est-ce que la sociologie nous apprend sur la culture ? \\[0pt]
Qu'est-ce que la souveraineté ? \\[0pt]
Qu'est-ce que la technique ? \\[0pt]
Qu'est-ce que la technique doit à la nature ? \\[0pt]
Qu'est-ce que la tragédie ? \\[0pt]
Qu'est-ce que la valeur marchande ? \\[0pt]
Qu'est-ce que la vérité ? \\[0pt]
Qu'est-ce que la vie ? \\[0pt]
Qu'est-ce que la vie bonne ? \\[0pt]
Qu'est-ce que le bonheur ? \\[0pt]
Qu'est-ce que le cinéma a changé dans l'idée que l'on se fait du temps ? \\[0pt]
Qu'est-ce que le cinéma donne à voir ? \\[0pt]
Qu'est-ce que le courage ? \\[0pt]
Qu'est-ce que le désordre ? \\[0pt]
Qu'est-ce que le dogmatisme ? \\[0pt]
Qu'est-ce que le génie ? \\[0pt]
Qu'est-ce que le hasard ? \\[0pt]
Qu'est-ce que le langage ordinaire ? \\[0pt]
Qu'est-ce que le malheur ? \\[0pt]
Qu'est-ce que le mal radical ? \\[0pt]
Qu'est-ce que le mauvais goût ? \\[0pt]
Qu'est-ce que le moi ? \\[0pt]
Qu'est-ce que le naturalisme ? \\[0pt]
Qu'est-ce que l'enfance ? \\[0pt]
Qu'est-ce que le nihilisme ? \\[0pt]
Qu'est-ce que le pathologique nous apprend sur le normal ? \\[0pt]
Qu'est-ce que le présent ? \\[0pt]
Qu'est-ce que le progrès technique ? \\[0pt]
Qu'est-ce que le réel ? \\[0pt]
Qu'est-ce que le sacré ? \\[0pt]
Qu'est-ce que le sens pratique ? \\[0pt]
Qu'est-ce que les sciences humaines apportent à la critique littéraire ? \\[0pt]
Qu'est-ce que les sciences religieuses nous apprennent de la religion ? \\[0pt]
Qu'est-ce que le style ? \\[0pt]
Qu'est-ce que le sublime ? \\[0pt]
Qu'est-ce que le symbolique ? \\[0pt]
Qu'est-ce que le temps ? \\[0pt]
Qu'est-ce que le travail ? \\[0pt]
Qu'est-ce que l'expérience pour un empiriste ? \\[0pt]
Qu'est-ce que l'harmonie ? \\[0pt]
Qu'est-ce que l'identité sociale des individus ? \\[0pt]
Qu'est-ce que l'inconscient ? \\[0pt]
Qu'est-ce que l'indifférence ? \\[0pt]
Qu'est-ce que l'individualisme méthodologique ? \\[0pt]
Qu'est-ce que l'intérêt général ? \\[0pt]
Qu'est-ce que l'interprétation des Écritures nous apprend sur les religions ? \\[0pt]
Qu'est-ce que l'intuition ? \\[0pt]
Qu'est-ce que lire ? \\[0pt]
Qu'est-ce que l'objectivité scientifique ? \\[0pt]
Qu'est-ce que l'ordinaire ? \\[0pt]
Qu'est-ce que maîtriser une technique ? \\[0pt]
Qu'est-ce que manquer de culture ? \\[0pt]
Qu'est-ce que méditer ? \\[0pt]
Qu'est-ce que mesurer le temps ? \\[0pt]
Qu'est-ce que mourir ? \\[0pt]
Qu'est-ce que nous apprennent les animaux sur nous-même ? \\[0pt]
Qu'est-ce qu'enquêter ? \\[0pt]
Qu'est-ce qu'enseigner ? \\[0pt]
Qu'est-ce que parler ? \\[0pt]
Qu'est-ce que parler à-propos ? \\[0pt]
Qu'est-ce que « parler le même langage » ? \\[0pt]
Qu'est-ce que parler le même langage ? \\[0pt]
Qu'est-ce que parler pour ne rien dire ? \\[0pt]
Qu'est-ce que parler veut dire ? \\[0pt]
Qu'est-ce que penser ? \\[0pt]
Qu'est-ce que percevoir ? \\[0pt]
Qu'est-ce que perdre la raison ? \\[0pt]
Qu'est-ce que perdre sa liberté ? \\[0pt]
Qu'est-ce que perdre son temps ? \\[0pt]
Qu'est-ce que peut un corps ? \\[0pt]
Qu'est-ce que prendre conscience ? \\[0pt]
Qu'est-ce que prendre le pouvoir ? \\[0pt]
Qu'est-ce que produire ? \\[0pt]
Qu'est-ce que promettre ? \\[0pt]
Qu'est-ce que prouver ? \\[0pt]
Qu'est-ce que raisonner ? \\[0pt]
Qu'est-ce que réfuter ? \\[0pt]
Qu'est-ce que réfuter une philosophie ? \\[0pt]
Qu'est-ce que rendre raison d'un effet ? \\[0pt]
Qu'est-ce que résister ? \\[0pt]
Qu'est-ce que résoudre une contradiction ? \\[0pt]
Qu'est-ce que « rester soi-même » ? \\[0pt]
Qu'est-ce que rester soi-même ? \\[0pt]
Qu'est-ce que réussir sa vie ? \\[0pt]
Qu'est-ce que savoir ? \\[0pt]
Qu'est-ce que se cultiver ? \\[0pt]
Qu'est-ce que « se rendre maître et possesseur de la nature » ? \\[0pt]
Qu'est-ce que se tromper ? \\[0pt]
Qu'est-ce que s'orienter ? \\[0pt]
Qu'est-ce que témoigner ? \\[0pt]
Qu'est-ce que traduire ? \\[0pt]
Qu'est-ce que travailler ? \\[0pt]
Qu'est-ce qu'être ? \\[0pt]
Qu'est-ce qu'être adulte ? \\[0pt]
Qu'est-ce qu'être artiste ? \\[0pt]
Qu'est-ce qu'être asocial ? \\[0pt]
Qu'est-ce qu'être barbare ? \\[0pt]
Qu'est-ce qu'être chez soi ? \\[0pt]
Qu'est-ce qu'être cohérent ? \\[0pt]
Qu'est-ce qu'être comportementaliste ? \\[0pt]
Qu'est-ce qu'être cultivé ? \\[0pt]
Qu'est-ce qu'être dans le vrai ? \\[0pt]
Qu'est-ce qu'être de bonne volonté ? \\[0pt]
Qu'est-ce qu'être démocrate ? \\[0pt]
Qu'est-ce qu'être de son temps ? \\[0pt]
Qu'est-ce qu'être efficace en politique ? \\[0pt]
Qu'est-ce qu'être ensemble ? \\[0pt]
Qu'est-ce qu'être en vie ? \\[0pt]
Qu'est-ce qu'être esclave ? \\[0pt]
Qu'est-ce qu'être fidèle à soi-même ? \\[0pt]
Qu'est-ce qu'être fou ? \\[0pt]
Qu'est-ce qu'être généreux ? \\[0pt]
Qu'est-ce qu'être hors de soi ? \\[0pt]
Qu'est-ce qu'être idéaliste ? \\[0pt]
Qu'est-ce qu'être inhumain ? \\[0pt]
Qu'est-ce qu'être l'auteur de son acte ? \\[0pt]
Qu'est-ce qu'être libéral ? \\[0pt]
Qu'est-ce qu'être libre ? \\[0pt]
Qu'est-ce qu'être maître de soi ? \\[0pt]
Qu'est-ce qu'être maître de soi-même ? \\[0pt]
Qu'est-ce qu'être malade ? \\[0pt]
Qu'est-ce qu'être matérialiste ? \\[0pt]
Qu'est-ce qu'être moderne ? \\[0pt]
Qu'est-ce qu'être nihiliste ? \\[0pt]
Qu'est-ce qu'être normal ? \\[0pt]
Qu'est-ce qu'être psychologue ? \\[0pt]
Qu'est-ce qu'être rationaliste ? \\[0pt]
Qu'est-ce qu'être rationnel ? \\[0pt]
Qu'est-ce qu'être réaliste ? \\[0pt]
Qu'est-ce qu'être républicain ? \\[0pt]
Qu'est-ce qu'être sceptique ? \\[0pt]
Qu'est-ce qu'être seul ? \\[0pt]
Qu'est-ce qu'être simple ? \\[0pt]
Qu'est-ce qu'être soi-même ? \\[0pt]
Qu'est-ce qu'être souverain ? \\[0pt]
Qu'est-ce qu'être spirituel ? \\[0pt]
Qu'est-ce qu'être témoin ? \\[0pt]
Qu'est-ce qu'être un bon citoyen ? \\[0pt]
Qu'est-ce qu'être un esclave ? \\[0pt]
Qu'est-ce qu'être un sujet ? \\[0pt]
Qu'est-ce qu'être vertueux ? \\[0pt]
Qu'est-ce qu'être visionnaire ? \\[0pt]
Qu'est-ce qu'être vivant ? \\[0pt]
Qu'est-ce que vérifier ? \\[0pt]
Qu'est-ce que vérifier une théorie ? \\[0pt]
Qu'est-ce que vieillir ? \\[0pt]
Qu'est-ce que vivre ? \\[0pt]
Qu'est-ce que vivre au présent ? \\[0pt]
Qu'est-ce que vivre bien ? \\[0pt]
Qu'est-ce que vivre vertueusement ? \\[0pt]
Qu'est-ce que voir ? \\[0pt]
Qu'est-ce qu'exercer un pouvoir ? \\[0pt]
Qu'est-ce qu'exister ? \\[0pt]
Qu'est-ce qu'exister pour un individu ? \\[0pt]
Qu'est-ce qu'expliquer ? \\[0pt]
Qu'est-ce qu'expliquer un comportement ? \\[0pt]
Qu'est-ce qu'habiter ? \\[0pt]
Qu'est-ce qui adoucit les mœurs ? \\[0pt]
Qu'est-ce qui a du sens ? \\[0pt]
Qu'est-ce qui agit ? \\[0pt]
Qu'est-ce qui anime l'animal ? \\[0pt]
Qu'est-ce qui apparaît ? \\[0pt]
Qu'est-ce qui a un sens ? \\[0pt]
Qu'est-ce qu'identifier ? \\[0pt]
Qu'est-ce qui dépend de nous ? \\[0pt]
Qu'est-ce qui distingue un argument d'une démonstration ? \\[0pt]
Qu'est-ce qui distingue un vivant d'une machine ? \\[0pt]
Qu'est-ce qui échappe aux sciences humaines ? \\[0pt]
Qu'est-ce qui est absurde ? \\[0pt]
Qu'est-ce qui est actuel ? \\[0pt]
Qu'est-ce qui est anormal ? \\[0pt]
Qu'est-ce qui est artificiel ? \\[0pt]
Qu'est-ce qui est beau ? \\[0pt]
Qu'est-ce qui est commun à toutes les langues ? \\[0pt]
Qu'est-ce qui est concret ? \\[0pt]
Qu'est-ce qui est contre nature ? \\[0pt]
Qu'est-ce qui est culturel ? \\[0pt]
Qu'est-ce qui est démonstratif ? \\[0pt]
Qu'est-ce qui est donné ? \\[0pt]
Qu'est-ce qui est essentiel ? \\[0pt]
Qu'est-ce qui est étonnant ? \\[0pt]
Qu'est-ce qui est extérieur à ma conscience, ? \\[0pt]
Qu'est-ce qui est historique ? \\[0pt]
Qu'est-ce qui est hors la loi ? \\[0pt]
Qu'est-ce qui est humain ? \\[0pt]
Qu'est-ce qui est immoral ? \\[0pt]
Qu'est-ce qui est impossible ? \\[0pt]
Qu'est-ce qui est indiscutable ? \\[0pt]
Qu'est-ce qui est injuste ? \\[0pt]
Qu'est-ce qui est insignifiant ? \\[0pt]
Qu'est-ce qui est intolérable ? \\[0pt]
Qu'est-ce qui est invérifiable ? \\[0pt]
Qu'est-ce qui est irrationnel ? \\[0pt]
Qu'est-ce qui est irréfutable ? \\[0pt]
Qu'est-ce qui est irréversible ? \\[0pt]
Qu'est-ce qui est le plus à craindre, l'ordre ou le désordre ? \\[0pt]
Qu'est-ce qui est mauvais dans l'égoïsme ? \\[0pt]
Qu'est-ce qui est mien ? \\[0pt]
Qu'est-ce qui est moderne ? \\[0pt]
Qu'est-ce qui est naturel ? \\[0pt]
Qu'est-ce qui est nécessaire ? \\[0pt]
Qu'est-ce qui est noble ? \\[0pt]
Qu'est-ce qui est normal ? \\[0pt]
Qu'est-ce qui est politique ? \\[0pt]
Qu'est-ce qui est possible ? \\[0pt]
Qu'est-ce qui est prométhéen ? \\[0pt]
Qu'est-ce qui est public ? \\[0pt]
Qu'est-ce qui est réel ? \\[0pt]
Qu'est-ce qui est respectable ? \\[0pt]
Qu'est-ce qui est sacré ? \\[0pt]
Qu'est-ce qui est sans raison ? \\[0pt]
Qu'est-ce qui est sauvage ? \\[0pt]
Qu'est-ce qui est scientifique ? \\[0pt]
Qu'est-ce qui est sensible ? \\[0pt]
Qu'est-ce qui est spectaculaire ? \\[0pt]
Qu'est-ce qui est sublime ? \\[0pt]
Qu'est-ce qui est tragique ? \\[0pt]
Qu'est-ce qui est transcendant ? \\[0pt]
Qu'est-ce qui est vital ? \\[0pt]
Qu'est-ce qui est vital pour le vivant ? \\[0pt]
Qu'est-ce qui est vivant ? \\[0pt]
Qu'est-ce qui exige d'être interprété ? \\[0pt]
Qu'est-ce qui existe ? \\[0pt]
Qu'est-ce qui fait autorité ? \\[0pt]
Qu'est-ce qui fait changer les sociétés ? \\[0pt]
Qu'est-ce qui fait des sciences humaines des sciences ? \\[0pt]
Qu'est-ce qui fait d'une activité un travail ? \\[0pt]
Qu'est-ce qui fait du scepticisme une philosophie ? \\[0pt]
Qu'est-ce qui fait la force de la loi ? \\[0pt]
Qu'est-ce qui fait la force des lois ? \\[0pt]
Qu'est-ce qui fait la justice des lois ? \\[0pt]
Qu'est-ce qui fait la légitimité d'une autorité politique ? \\[0pt]
Qu'est-ce qui fait la valeur de la technique ? \\[0pt]
Qu'est-ce qui fait la valeur de l'œuvre d'art ? \\[0pt]
Qu'est-ce qui fait la valeur des objets d'art ? \\[0pt]
Qu'est-ce qui fait la valeur d'une croyance ? \\[0pt]
Qu'est-ce qui fait la valeur d'une existence ? \\[0pt]
Qu'est-ce qui fait la valeur d'une œuvre ? \\[0pt]
Qu'est-ce qui fait la valeur d'une œuvre d'art ? \\[0pt]
Qu'est-ce qui fait le pouvoir des mots ? \\[0pt]
Qu'est-ce qui fait le propre d'un corps propre ? \\[0pt]
Qu'est-ce qui fait l'humanité d'un corps ? \\[0pt]
Qu'est-ce qui fait l'unité d'une science ? \\[0pt]
Qu'est-ce qui fait l'unité d'un organisme ? \\[0pt]
Qu'est-ce qui fait l'unité d'un peuple ? \\[0pt]
Qu'est-ce qui fait l'unité du vivant ? \\[0pt]
Qu'est-ce qui fait mon identité ? \\[0pt]
Qu'est-ce qui fait obstacle à la science ? \\[0pt]
Qu'est-ce qui fait qu'un corps est humain ? \\[0pt]
Qu'est-ce qui fait qu'une théorie est vraie ? \\[0pt]
Qu'est-ce qui fait qu'un peuple est un peuple ? \\[0pt]
Qu'est-ce qui fait un peuple ? \\[0pt]
Qu'est-ce qui fonde la croyance ? \\[0pt]
Qu'est-ce qui fonde le respect d'autrui ? \\[0pt]
Qu'est-ce qu'ignore la science ? \\[0pt]
Qu'est-ce qui importe ? \\[0pt]
Qu'est-ce qui importe dans l'éducation ? \\[0pt]
Qu'est-ce qui innocente le bourreau ? \\[0pt]
Qu'est-ce qui justifie l'hypothèse d'un inconscient ? \\[0pt]
Qu'est-ce qui justifie une croyance ? \\[0pt]
Qu'est-ce qu'imaginer ? \\[0pt]
Qu'est-ce qui m'appartient ? \\[0pt]
Qu'est-ce qui menace la liberté ? \\[0pt]
Qu'est-ce qui me rend plus fort ? \\[0pt]
Qu'est-ce qui mérite l'admiration ? \\[0pt]
Qu'est-ce qui m'est étranger ? \\[0pt]
Qu'est-ce qui mesure la valeur d'un travail ? \\[0pt]
Qu'est-ce qui n'a pas d'histoire ? \\[0pt]
Qu'est-ce qui n'appartient pas au monde ? \\[0pt]
Qu'est-ce qui ne change pas ? \\[0pt]
Qu'est-ce qui ne disparaît jamais ? \\[0pt]
Qu'est-ce qui ne s'achète pas ? \\[0pt]
Qu'est-ce qui ne s'échange pas ? \\[0pt]
Qu'est-ce qui n'est pas démontrable ? \\[0pt]
Qu'est-ce qui n'est pas en mouvement ? \\[0pt]
Qu'est-ce qui n'est pas maîtrisable ? \\[0pt]
Qu'est-ce qui n'est pas mathématisable ? \\[0pt]
Qu'est-ce qui n'est pas normal ? \\[0pt]
Qu'est-ce qui n'est pas politique ? \\[0pt]
Qu'est-ce qui n'est pas technique dans l'activité humaine ? \\[0pt]
Qu'est-ce qui n'existe pas ? \\[0pt]
Qu'est-ce qui nous échappe dans le temps ? \\[0pt]
Qu'est-ce qui nous fait danser ? \\[0pt]
Qu'est-ce qui nous oblige ? \\[0pt]
Qu'est-ce qu'interpréter ? \\[0pt]
Qu'est-ce qu'interpréter une œuvre d'art ? \\[0pt]
Qu'est-ce qu'interpréter un texte ? \\[0pt]
Qu'est-ce qui oppose les hommes ? \\[0pt]
Qu'est-ce qui peut être hors du temps ? \\[0pt]
Qu'est-ce qui peut se transformer ? \\[0pt]
Qu'est-ce qui plaît dans la musique ? \\[0pt]
Qu'est-ce qui rapproche le vivant de la machine ? \\[0pt]
Qu'est-ce qui rend l'objectivité difficile dans les sciences humaines ? \\[0pt]
Qu'est-ce qui rend une parole poétique ? \\[0pt]
Qu'est-ce qui rend vrai un énoncé ? \\[0pt]
Qu'est-ce qu'obéir ? \\[0pt]
Qu'est-ce qu'on attend ? \\[0pt]
Qu'est-ce qu'on ne peut comprendre ? \\[0pt]
Qu'est-ce qu'on ne peut pas partager ? \\[0pt]
Qu'est-ce qu'un abus de langage ? \\[0pt]
Qu'est-ce qu'un abus de pouvoir ? \\[0pt]
Qu'est-ce qu'un accident ? \\[0pt]
Qu'est-ce qu'un acte ? \\[0pt]
Qu'est-ce qu'un acte libre ? \\[0pt]
Qu'est-ce qu'un acte moral ? \\[0pt]
Qu'est-ce qu'un acte symbolique ? \\[0pt]
Qu'est-ce qu'un acteur ? \\[0pt]
Qu'est-ce qu'un adversaire en politique ? \\[0pt]
Qu'est-ce qu'un alter ego ? \\[0pt]
Qu'est-ce qu'un ami ? \\[0pt]
Qu'est-ce qu'un animal ? \\[0pt]
Qu'est-ce qu'un animal domestique ? \\[0pt]
Qu'est-ce qu'un argument ? \\[0pt]
Qu'est-ce qu'un art abstrait ? \\[0pt]
Qu'est-ce qu'un art de vivre ? \\[0pt]
Qu'est-ce qu'un artefact ? \\[0pt]
Qu'est-ce qu'un artiste ? \\[0pt]
Qu'est-ce qu'un art moral ? \\[0pt]
Qu'est-ce qu'un art populaire ? \\[0pt]
Qu'est-ce qu'un auteur ? \\[0pt]
Qu'est-ce qu'un axiome ? \\[0pt]
Qu'est-ce qu'un beau parleur ? \\[0pt]
Qu'est-ce qu'un beau paysage ? \\[0pt]
Qu'est-ce qu'un beau travail ? \\[0pt]
Qu'est-ce qu'un bon argument ? \\[0pt]
Qu'est-ce qu'un bon citoyen ? \\[0pt]
Qu'est-ce qu'un bon commentaire ? \\[0pt]
Qu'est-ce qu'un bon conseil ? \\[0pt]
Qu'est-ce qu'un bon gouvernement ? \\[0pt]
Qu'est-ce qu'un bon jugement ? \\[0pt]
Qu'est-ce qu'un capital culturel ? \\[0pt]
Qu'est-ce qu'un caractère ? \\[0pt]
Qu'est-ce qu'un cas de conscience ? \\[0pt]
Qu'est-ce qu'un cas en morale ? \\[0pt]
Qu'est-ce qu'un « champ artistique » ? \\[0pt]
Qu'est-ce qu'un châtiment ? \\[0pt]
Qu'est-ce qu'un chef ? \\[0pt]
Qu'est-ce qu'un chef-d'œuvre ? \\[0pt]
Qu'est-ce qu'un choix éclairé ? \\[0pt]
Qu'est-ce qu'un choix rationnel ? \\[0pt]
Qu'est-ce qu'un citoyen ? \\[0pt]
Qu'est-ce qu'un citoyen libre ? \\[0pt]
Qu'est-ce qu'un civilisé ? \\[0pt]
Qu'est-ce qu'un classique ? \\[0pt]
Qu'est-ce qu'un code ? \\[0pt]
Qu'est-ce qu'un collectif ? \\[0pt]
Qu'est-ce qu'un concept ? \\[0pt]
Qu'est-ce qu'un concept philosophique ? \\[0pt]
Qu'est-ce qu'un concept scientifique ? \\[0pt]
Qu'est-ce qu'un conflit de devoirs ? \\[0pt]
Qu'est-ce qu'un conflit de générations ? \\[0pt]
Qu'est-ce qu'un conflit de valeurs ? \\[0pt]
Qu'est-ce qu'un conflit politique ? \\[0pt]
Qu'est-ce qu'un consommateur ? \\[0pt]
Qu'est-ce qu'un contenu de conscience ? \\[0pt]
Qu'est-ce qu'un contrat ? \\[0pt]
Qu'est-ce qu'un contre-pouvoir ? \\[0pt]
Qu'est-ce qu'un corps ? \\[0pt]
Qu'est-ce qu'un corps politique ? \\[0pt]
Qu'est-ce qu'un corps social ? \\[0pt]
Qu'est-ce qu'un coup d'État ? \\[0pt]
Qu'est-ce qu'un créateur ? \\[0pt]
Qu'est-ce qu'un crime ? \\[0pt]
Qu'est-ce qu'un crime contre l'humanité ? \\[0pt]
Qu'est-ce qu'un crime politique ? \\[0pt]
Qu'est-ce qu'un critère de vérité ? \\[0pt]
Qu'est-ce qu'un déni ? \\[0pt]
Qu'est-ce qu'un désir satisfait ? \\[0pt]
Qu'est-ce qu'un détail ? \\[0pt]
Qu'est-ce qu'un dialogue ? \\[0pt]
Qu'est-ce qu'un dieu ? \\[0pt]
Qu'est-ce qu'un Dieu ? \\[0pt]
Qu'est-ce qu'un dilemme ? \\[0pt]
Qu'est-ce qu'un document ? \\[0pt]
Qu'est-ce qu'un dogme ? \\[0pt]
Qu'est-ce qu'un doute raisonnable ? \\[0pt]
Qu'est-ce qu'un droit naturel ? \\[0pt]
Qu'est-ce qu'une action intentionnelle ? \\[0pt]
Qu'est-ce qu'une action juste ? \\[0pt]
Qu'est-ce qu'une action politique ? \\[0pt]
Qu'est-ce qu'une action réussie ? \\[0pt]
Qu'est-ce qu'une alternative ? \\[0pt]
Qu'est-ce qu'une âme ? \\[0pt]
Qu'est-ce qu'une analyse ? \\[0pt]
Qu'est-ce qu'une anomalie ? \\[0pt]
Qu'est-ce qu'une aporie ? \\[0pt]
Qu'est-ce qu'une autorité légitime ? \\[0pt]
Qu'est-ce qu'une avant-garde ? \\[0pt]
Qu'est-ce qu'une belle action ? \\[0pt]
Qu'est-ce qu'une belle démonstration ? \\[0pt]
Qu'est-ce qu'une belle forme ? \\[0pt]
Qu'est-ce qu'une belle mort ? \\[0pt]
Qu'est-ce qu'une bête ? \\[0pt]
Qu'est-ce qu'une bonne définition ? \\[0pt]
Qu'est-ce qu'une bonne délibération ? \\[0pt]
Qu'est-ce qu'une bonne éducation ? \\[0pt]
Qu'est-ce qu'une bonne loi ? \\[0pt]
Qu'est-ce qu'une bonne méthode ? \\[0pt]
Qu'est-ce qu'une bonne théorie ? \\[0pt]
Qu'est-ce qu'une bonne traduction ? \\[0pt]
Qu'est-ce qu'une catastrophe ? \\[0pt]
Qu'est-ce qu'une catégorie ? \\[0pt]
Qu'est-ce qu'une catégorie de l'être ? \\[0pt]
Qu'est-ce qu'une cause ? \\[0pt]
Qu'est-ce qu'un échange juste ? \\[0pt]
Qu'est-ce qu'un échange réussi ? \\[0pt]
Qu'est-ce qu'une chimère ? \\[0pt]
Qu'est-ce qu'une chose ? \\[0pt]
Qu'est-ce qu'une chose matérielle ? \\[0pt]
Qu'est-ce qu'une civilisation ? \\[0pt]
Qu'est-ce qu'une classe sociale ? \\[0pt]
Qu'est-ce qu'une classification scientifique ? \\[0pt]
Qu'est-ce qu'une collectivité ? \\[0pt]
Qu'est-ce qu'une comédie ? \\[0pt]
Qu'est-ce qu'une communauté ? \\[0pt]
Qu'est-ce qu'une communauté humaine ? \\[0pt]
Qu'est-ce qu'une communauté politique ? \\[0pt]
Qu'est-ce qu'une communauté scientifique ? \\[0pt]
Qu'est-ce qu'une conception scientifique du monde ? \\[0pt]
Qu'est-ce qu'une condition de possibilité ? \\[0pt]
Qu'est-ce qu'une condition suffisante ? \\[0pt]
Qu'est-ce qu'une conduite irrationnelle ? \\[0pt]
Qu'est-ce qu'une connaissance a priori ? \\[0pt]
Qu'est-ce qu'une connaissance fiable ? \\[0pt]
Qu'est-ce qu'une connaissance non scientifique ? \\[0pt]
Qu'est-ce qu'une connaissance par les faits ? \\[0pt]
Qu'est-ce qu'une conscience collective ? \\[0pt]
Qu'est-ce qu'une constitution ? \\[0pt]
Qu'est-ce qu'une contradiction ? \\[0pt]
Qu'est-ce qu'une contrainte ? \\[0pt]
Qu'est-ce qu'une convention ? \\[0pt]
Qu'est-ce qu'une conviction ? \\[0pt]
Qu'est-ce qu'une crise ? \\[0pt]
Qu'est-ce qu'une crise politique ? \\[0pt]
Qu'est-ce qu'une croyance ? \\[0pt]
Qu'est-ce qu'une croyance rationnelle ? \\[0pt]
Qu'est-ce qu'une croyance vraie ? \\[0pt]
Qu'est-ce qu'une culture ? \\[0pt]
Qu'est-ce qu'une décision politique ? \\[0pt]
Qu'est-ce qu'une décision rationnelle ? \\[0pt]
Qu'est-ce qu'une découverte ? \\[0pt]
Qu'est-ce qu'une découverte scientifique ? \\[0pt]
Qu'est-ce qu'une définition ? \\[0pt]
Qu'est-ce qu'une démocratie ? \\[0pt]
Qu'est-ce qu'une démonstration ? \\[0pt]
Qu'est-ce qu'une discipline savante ? \\[0pt]
Qu'est-ce qu'une disposition ? \\[0pt]
Qu'est-ce qu'une école philosophique ? \\[0pt]
Qu'est-ce qu'une éducation réussie ? \\[0pt]
Qu'est-ce qu'une éducation scientifique ? \\[0pt]
Qu'est-ce qu'une encyclopédie ? \\[0pt]
Qu'est-ce qu'une époque ? \\[0pt]
Qu'est-ce qu'une erreur ? \\[0pt]
Qu'est-ce qu'une espèce naturelle ? \\[0pt]
Qu'est-ce qu'une exception ? \\[0pt]
Qu'est-ce qu'une existence historique ? \\[0pt]
Qu'est-ce qu'une expérience ? \\[0pt]
Qu'est-ce qu'une expérience cruciale ? \\[0pt]
Qu'est-ce qu'une « expérience de pensée » ? \\[0pt]
Qu'est-ce qu'une expérience de pensée ? \\[0pt]
Qu'est-ce qu'une expérience esthétique ? \\[0pt]
Qu'est-ce qu'une expérience politique ? \\[0pt]
Qu'est-ce qu'une expérience religieuse ? \\[0pt]
Qu'est-ce qu'une expérience scientifique ? \\[0pt]
Qu'est-ce qu'une explication matérialiste ? \\[0pt]
Qu'est-ce qu'une exposition ? \\[0pt]
Qu'est-ce qu'une famille ? \\[0pt]
Qu'est-ce qu'une fausse science ? \\[0pt]
Qu'est-ce qu'une faute de goût ? \\[0pt]
Qu'est-ce qu'une fiction ? \\[0pt]
Qu'est-ce qu'une fonction ? \\[0pt]
Qu'est-ce qu'une forme ? \\[0pt]
Qu'est-ce qu'une grande cause ? \\[0pt]
Qu'est-ce qu'une guerre juste ? \\[0pt]
Qu'est-ce qu'une histoire vraie ? \\[0pt]
Qu'est-ce qu'une hypothèse ? \\[0pt]
Qu'est-ce qu'une hypothèse scientifique ? \\[0pt]
Qu'est-ce qu'une idée ? \\[0pt]
Qu'est-ce qu'une idée esthétique ? \\[0pt]
Qu'est-ce qu'une idée fausse ? \\[0pt]
Qu'est-ce qu'une idée incertaine ? \\[0pt]
Qu'est-ce qu'une idée morale ? \\[0pt]
Qu'est-ce qu'une idée vraie ? \\[0pt]
Qu'est-ce qu'une idéologie ? \\[0pt]
Qu'est-ce qu'une idéologie scientifique ? \\[0pt]
Qu'est-ce qu'une illusion ? \\[0pt]
Qu'est-ce qu'une image ? \\[0pt]
Qu'est-ce qu'une impression ? \\[0pt]
Qu'est-ce qu'une inégalité ? \\[0pt]
Qu'est-ce qu'une inférence ? \\[0pt]
Qu'est-ce qu'une injustice ? \\[0pt]
Qu'est-ce qu'une innovation technologique ? \\[0pt]
Qu'est-ce qu'une institution ? \\[0pt]
Qu'est-ce qu'une interprétation ? \\[0pt]
Qu'est-ce qu'une invention technique ? \\[0pt]
Qu'est-ce qu'une langue ? \\[0pt]
Qu'est-ce qu'une langue artificielle ? \\[0pt]
Qu'est-ce qu'une langue bien faite ? \\[0pt]
Qu'est-ce qu'une langue morte ? \\[0pt]
Qu'est-ce qu'un élément ? \\[0pt]
Qu'est-ce qu'une libération ? \\[0pt]
Qu'est-ce qu'une liberté fondamentale ? \\[0pt]
Qu'est-ce qu'une libre interprétation ? \\[0pt]
Qu'est-ce qu'une limite ? \\[0pt]
Qu'est-ce qu'une logique sociale ? \\[0pt]
Qu'est-ce qu'une loi ? \\[0pt]
Qu'est-ce qu'une loi de la nature ? \\[0pt]
Qu'est-ce qu'une loi de la pensée ? \\[0pt]
Qu'est-ce qu'une loi d'exception ? \\[0pt]
Qu'est-ce qu'une loi injuste ? \\[0pt]
Qu'est-ce qu'une loi scientifique ? \\[0pt]
Qu'est-ce qu'une machine ? \\[0pt]
Qu'est-ce qu'une machine ne peut pas faire ? \\[0pt]
Qu'est-ce qu'une maladie ? \\[0pt]
Qu'est-ce qu'une marchandise ? \\[0pt]
Qu'est-ce qu'une mauvaise idée ? \\[0pt]
Qu'est-ce qu'une mauvaise interprétation ? \\[0pt]
Qu'est-ce qu'une méditation ? \\[0pt]
Qu'est-ce qu'une méditation métaphysique ? \\[0pt]
Qu'est-ce qu'une mentalité collective ? \\[0pt]
Qu'est-ce qu'une métaphore ? \\[0pt]
Qu'est-ce qu'une méthode ? \\[0pt]
Qu'est-ce qu'une minorité ? \\[0pt]
Qu'est-ce qu'une morale de la communication ? \\[0pt]
Qu'est-ce qu'un empire ? \\[0pt]
Qu'est-ce qu'une nation ? \\[0pt]
Qu'est-ce qu'un enfant ? \\[0pt]
Qu'est-ce qu'un ennemi ? \\[0pt]
Qu'est-ce qu'un ennemi politique ? \\[0pt]
Qu'est-ce qu'un énoncé scientifique sur l'homme ? \\[0pt]
Qu'est-ce qu'une norme ? \\[0pt]
Qu'est-ce qu'une norme sociale ? \\[0pt]
Qu'est-ce qu'une nouveauté ? \\[0pt]
Qu'est-ce qu'une occasion ? \\[0pt]
Qu'est-ce qu'une œuvre ? \\[0pt]
Qu'est-ce qu'une œuvre classique ? \\[0pt]
Qu'est-ce qu'une œuvre d'art ? \\[0pt]
Qu'est-ce qu'une œuvre d'art authentique ? \\[0pt]
Qu'est-ce qu'une œuvre d'art réaliste ? \\[0pt]
Qu'est-ce qu'une œuvre d'art réussie ? \\[0pt]
Qu'est-ce qu'une œuvre « géniale » ? \\[0pt]
Qu'est-ce qu'une œuvre ratée ? \\[0pt]
Qu'est-ce qu'une parole en l'air ? \\[0pt]
Qu'est-ce qu'une parole juste ? \\[0pt]
Qu'est-ce qu'une parole libre ? \\[0pt]
Qu'est-ce qu'une parole vivante ? \\[0pt]
Qu'est-ce qu'une parole vraie ? \\[0pt]
Qu'est-ce qu'une passion ? \\[0pt]
Qu'est-ce qu'une patrie ? \\[0pt]
Qu'est-ce qu'une peinture abstraite ? \\[0pt]
Qu'est-ce qu'une pensée libre ? \\[0pt]
Qu'est-ce qu'une perception sensible ? \\[0pt]
Qu'est-ce qu'une « performance » ? \\[0pt]
Qu'est-ce qu'une période en histoire ? \\[0pt]
Qu'est-ce qu'une personne ? \\[0pt]
Qu'est-ce qu'une personne morale ? \\[0pt]
Qu'est-ce qu'une pétition de principe ? \\[0pt]
Qu'est-ce qu'une philosophie ? \\[0pt]
Qu'est-ce qu'une philosophie de la vie ? \\[0pt]
Qu'est-ce qu'une philosophie première ? \\[0pt]
Qu'est-ce qu'une phrase ? \\[0pt]
Qu'est-ce qu'une politique sociale ? \\[0pt]
Qu'est-ce qu'une preuve ? \\[0pt]
Qu'est-ce qu'une preuve scientifique ? \\[0pt]
Qu'est-ce qu'une promesse ? \\[0pt]
Qu'est-ce qu'une propriété ? \\[0pt]
Qu'est-ce qu'une propriété essentielle ? \\[0pt]
Qu'est-ce qu'une pseudoscience ? \\[0pt]
Qu'est-ce qu'une psychologie scientifique ? \\[0pt]
Qu'est-ce qu'une question ? \\[0pt]
Qu'est-ce qu'une question dénuée de sens ? \\[0pt]
Qu'est-ce qu'une question métaphysique ? \\[0pt]
Qu'est-ce qu'une raison d'agir ? \\[0pt]
Qu'est-ce qu'une réfutation ? \\[0pt]
Qu'est-ce qu'une règle ? \\[0pt]
Qu'est-ce qu'une règle de grammaire ? \\[0pt]
Qu'est-ce qu'une règle de vie ? \\[0pt]
Qu'est-ce qu'une règle sociale ? \\[0pt]
Qu'est-ce qu'une relation ? \\[0pt]
Qu'est-ce qu'une religion ? \\[0pt]
Qu'est-ce qu'une rencontre ? \\[0pt]
Qu'est-ce qu'une représentation ? \\[0pt]
Qu'est-ce qu'une représentation du monde ? \\[0pt]
Qu'est-ce qu'une représentation réussie ? \\[0pt]
Qu'est-ce qu'une république ? \\[0pt]
Qu'est-ce qu'une révélation ? \\[0pt]
Qu'est-ce qu'une révolution ? \\[0pt]
Qu'est-ce qu'une révolution artistique ? \\[0pt]
Qu'est-ce qu'une révolution politique ? \\[0pt]
Qu'est-ce qu'une révolution scientifique ? \\[0pt]
Qu'est-ce qu'une science exacte ? \\[0pt]
Qu'est-ce qu'une science expérimentale ? \\[0pt]
Qu'est-ce qu'une science humaine ? \\[0pt]
Qu'est-ce qu'une science rigoureuse ? \\[0pt]
Qu'est-ce qu'un esclave ? \\[0pt]
Qu'est-ce qu'une secte ? \\[0pt]
Qu'est-ce qu'une situation ? \\[0pt]
Qu'est-ce qu'une situation tragique ? \\[0pt]
Qu'est-ce qu'une société ? \\[0pt]
Qu'est-ce qu'une société juste ? \\[0pt]
Qu'est-ce qu'une société libre ? \\[0pt]
Qu'est-ce qu'une société mondialisée ? \\[0pt]
Qu'est-ce qu'une société ouverte ? \\[0pt]
Qu'est-ce qu'une société primitive ? \\[0pt]
Qu'est-ce qu'une solution ? \\[0pt]
Qu'est-ce qu'un esprit faux ? \\[0pt]
Qu'est-ce qu'un esprit juste ? \\[0pt]
Qu'est-ce qu'un esprit libre ? \\[0pt]
Qu'est-ce qu'un esprit profond ? \\[0pt]
Qu'est-ce qu'une structure ? \\[0pt]
Qu'est-ce qu'une substance ? \\[0pt]
Qu'est-ce qu'un État de droit ? \\[0pt]
Qu'est-ce qu'un État libre ? \\[0pt]
Qu'est-ce qu'un état mental ? \\[0pt]
Qu'est-ce qu'un État souverain ? \\[0pt]
Qu'est-ce qu'une théorie ? \\[0pt]
Qu'est-ce qu'une théorie scientifique ? \\[0pt]
Qu'est-ce qu'une tradition ? \\[0pt]
Qu'est-ce qu'une tragédie ? \\[0pt]
Qu'est-ce qu'une tragédie historique ? \\[0pt]
Qu'est-ce qu'un être cultivé ? \\[0pt]
Qu'est-ce qu'un « être dégénéré » ? \\[0pt]
Qu'est-ce qu'un être imaginaire ? \\[0pt]
Qu'est-ce qu'un être vivant ? \\[0pt]
Qu'est-ce qu'une valeur ? \\[0pt]
Qu'est-ce qu'une valeur esthétique ? \\[0pt]
Qu'est-ce qu'un événement ? \\[0pt]
Qu'est-ce qu'un événement fondateur ? \\[0pt]
Qu'est-ce qu'un événement historique ? \\[0pt]
Qu'est-ce qu'une vérité contingente ? \\[0pt]
Qu'est-ce qu'une vérité historique ? \\[0pt]
Qu'est-ce qu'une vérité scientifique ? \\[0pt]
Qu'est-ce qu'une vérité subjective ? \\[0pt]
Qu'est-ce qu'une vertu ? \\[0pt]
Qu'est-ce qu'une vie d'artiste ? \\[0pt]
Qu'est-ce qu'une vie heureuse ? \\[0pt]
Qu'est-ce qu'une vie humaine ? \\[0pt]
Qu'est-ce qu'une vie réussie ? \\[0pt]
Qu'est-ce qu'une ville ? \\[0pt]
Qu'est-ce qu'une violence symbolique ? \\[0pt]
Qu'est-ce qu'une vision du monde ? \\[0pt]
Qu'est-ce qu'une vision scientifique du monde ? \\[0pt]
Qu'est-ce qu'une volonté libre ? \\[0pt]
Qu'est-ce qu'une volonté raisonnable ? \\[0pt]
Qu'est-ce qu'un exemple ? \\[0pt]
Qu'est-ce qu'un expérimentateur ? \\[0pt]
Qu'est-ce qu'un expert ? \\[0pt]
Qu'est-ce qu'un fait ? \\[0pt]
Qu'est-ce qu'un fait culturel ? \\[0pt]
Qu'est-ce qu'un fait de culture ? \\[0pt]
Qu'est-ce qu'un fait de société ? \\[0pt]
Qu'est-ce qu'un fait divers ? \\[0pt]
Qu'est-ce qu'un fait historique ? \\[0pt]
Qu'est-ce qu'un fait moral ? \\[0pt]
Qu'est-ce qu'un fait religieux ? \\[0pt]
Qu'est-ce qu'un fait scientifique ? \\[0pt]
Qu'est-ce qu'un fait social ? \\[0pt]
Qu'est-ce qu'un faux ? \\[0pt]
Qu'est-ce qu'un faux problème ? \\[0pt]
Qu'est-ce qu'un faux sentiment ? \\[0pt]
Qu'est-ce qu'un film ? \\[0pt]
Qu'est-ce qu'un génie ? \\[0pt]
Qu'est-ce qu'un geste artistique ? \\[0pt]
Qu'est-ce qu'un geste technique ? \\[0pt]
Qu'est-ce qu'un gouvernement ? \\[0pt]
Qu'est-ce qu'un gouvernement démocratique ? \\[0pt]
Qu'est-ce qu'un gouvernement juste ? \\[0pt]
Qu'est-ce qu'un gouvernement républicain ? \\[0pt]
Qu'est-ce qu'un grand homme ? \\[0pt]
Qu'est-ce qu'un grand homme ou une grande femme ? \\[0pt]
Qu'est-ce qu'un grand philosophe ? \\[0pt]
Qu'est-ce qu'un héros ? \\[0pt]
Qu'est-ce qu'un homme bon ? \\[0pt]
Qu'est-ce qu'un homme d'action ? \\[0pt]
Qu'est-ce qu'un homme d'État ? \\[0pt]
Qu'est-ce qu'un homme d'expérience ? \\[0pt]
Qu'est-ce qu'un homme juste ? \\[0pt]
Qu'est-ce qu'un homme libre ? \\[0pt]
Qu'est-ce qu'un homme méchant ? \\[0pt]
Qu'est-ce qu'un homme normal ? \\[0pt]
Qu'est-ce qu'un homme politique ? \\[0pt]
Qu'est-ce qu'un homme sans éducation ? \\[0pt]
Qu'est-ce qu'un homme seul ? \\[0pt]
Qu'est-ce qu'un idéal ? \\[0pt]
Qu'est-ce qu'un idéaliste ? \\[0pt]
Qu'est-ce qu'un idéal moral ? \\[0pt]
Qu'est-ce qu'un imaginaire social ? \\[0pt]
Qu'est-ce qu'un impératif technique ? \\[0pt]
Qu'est-ce qu'un individu ? \\[0pt]
Qu'est-ce qu'un intellectuel ? \\[0pt]
Qu'est-ce qu'un jeu ? \\[0pt]
Qu'est-ce qu'un juge ? \\[0pt]
Qu'est-ce qu'un jugement analytique ? \\[0pt]
Qu'est-ce qu'un jugement de goût ? \\[0pt]
Qu'est-ce qu'un juste ? \\[0pt]
Qu'est-ce qu'un juste salaire ? \\[0pt]
Qu'est-ce qu'un justicier ? \\[0pt]
Qu'est-ce qu'un laboratoire ? \\[0pt]
Qu'est-ce qu'un langage technique ? \\[0pt]
Qu'est-ce qu'un législateur ? \\[0pt]
Qu'est-ce qu'un lien logique ? \\[0pt]
Qu'est-ce qu'un lieu ? \\[0pt]
Qu'est-ce qu'un lieu commun ? \\[0pt]
Qu'est-ce qu'un livre ? \\[0pt]
Qu'est-ce qu'un maître ? \\[0pt]
Qu'est-ce qu'un marginal ? \\[0pt]
Qu'est-ce qu'un mécanisme social ? \\[0pt]
Qu'est-ce qu'un métaphysicien ? \\[0pt]
Qu'est-ce qu'un métier ? \\[0pt]
Qu'est-ce qu'un mineur ? \\[0pt]
Qu'est-ce qu'un miracle ? \\[0pt]
Qu'est-ce qu'un modèle ? \\[0pt]
Qu'est-ce qu'un moderne ? \\[0pt]
Qu'est-ce qu'un moment ? \\[0pt]
Qu'est-ce qu'un monde ? \\[0pt]
Qu'est-ce qu'un monde commun ? \\[0pt]
Qu'est-ce qu'un monstre ? \\[0pt]
Qu'est-ce qu'un monument ? \\[0pt]
Qu'est-ce qu'un mouvement politique ? \\[0pt]
Qu'est-ce qu'un musée ? \\[0pt]
Qu'est-ce qu'un mythe ? \\[0pt]
Qu'est-ce qu'un nombre ? \\[0pt]
Qu'est-ce qu'un nom propre ? \\[0pt]
Qu'est-ce qu'un objet ? \\[0pt]
Qu'est-ce qu'un objet d'art ? \\[0pt]
Qu'est-ce qu'un objet de connaissance ? \\[0pt]
Qu'est-ce qu'un objet de science ? \\[0pt]
Qu'est-ce qu'un objet esthétique ? \\[0pt]
Qu'est-ce qu'un objet mathématique ? \\[0pt]
Qu'est-ce qu'un objet métaphysique ? \\[0pt]
Qu'est-ce qu'un objet technique ? \\[0pt]
Qu'est-ce qu'un œuvre d'art ? \\[0pt]
Qu'est-ce qu'un ordre ? \\[0pt]
Qu'est-ce qu'un organisme ? \\[0pt]
Qu'est-ce qu'un original ? \\[0pt]
Qu'est-ce qu'un outil ? \\[0pt]
Qu'est-ce qu'un paradoxe ? \\[0pt]
Qu'est-ce qu'un passe-temps ? \\[0pt]
Qu'est-ce qu'un patrimoine ? \\[0pt]
Qu'est-ce qu'un pauvre ? \\[0pt]
Qu'est-ce qu'un paysage ? \\[0pt]
Qu'est-ce qu'un pédant ? \\[0pt]
Qu'est-ce qu'un peuple ? \\[0pt]
Qu'est-ce qu'un peuple libre ? \\[0pt]
Qu'est-ce qu'un phénomène ? \\[0pt]
Qu'est-ce qu'un philosophe ? \\[0pt]
Qu'est-ce qu'un plaisir pur ? \\[0pt]
Qu'est-ce qu'un point de vue ? \\[0pt]
Qu'est-ce qu'un portrait ? \\[0pt]
Qu'est-ce qu'un post-moderne ? \\[0pt]
Qu'est-ce qu'un postulat ? \\[0pt]
Qu'est-ce qu'un pouvoir despotique ? \\[0pt]
Qu'est-ce qu'un pouvoir légitime ? \\[0pt]
Qu'est-ce qu'un précurseur ? \\[0pt]
Qu'est-ce qu'un préjugé ? \\[0pt]
Qu'est-ce qu'un primitif ? \\[0pt]
Qu'est-ce qu'un prince juste ? \\[0pt]
Qu'est-ce qu'un principe ? \\[0pt]
Qu'est-ce qu'un problème ? \\[0pt]
Qu'est-ce qu'un problème éthique ? \\[0pt]
Qu'est-ce qu'un problème insoluble ? \\[0pt]
Qu'est-ce qu'un problème métaphysique ? \\[0pt]
Qu'est-ce qu'un problème philosophique ? \\[0pt]
Qu'est-ce qu'un problème politique ? \\[0pt]
Qu'est-ce qu'un problème scientifique ? \\[0pt]
Qu'est-ce qu'un problème technique ? \\[0pt]
Qu'est-ce qu'un produit culturel ? \\[0pt]
Qu'est-ce qu'un programme ? \\[0pt]
Qu'est-ce qu'un programme politique ? \\[0pt]
Qu'est-ce qu'un programmer ? \\[0pt]
Qu'est-ce qu'un progrès scientifique ? \\[0pt]
Qu'est-ce qu'un progrès technique ? \\[0pt]
Qu'est-ce qu'un projet ? \\[0pt]
Qu'est-ce qu'un prophète ? \\[0pt]
Qu'est-ce qu'un public ? \\[0pt]
Qu'est-ce qu'un pur esprit ? \\[0pt]
Qu'est-ce qu'un rapport de force ? \\[0pt]
Qu'est-ce qu'un récit ? \\[0pt]
Qu'est-ce qu'un récit véridique ? \\[0pt]
Qu'est-ce qu'un réfugié politique ? \\[0pt]
Qu'est-ce qu'un réfutation ? \\[0pt]
Qu'est-ce qu'un régime politique ? \\[0pt]
Qu'est-ce qu'un réseau ? \\[0pt]
Qu'est-ce qu'un rhéteur ? \\[0pt]
Qu'est-ce qu'un rite ? \\[0pt]
Qu'est-ce qu'un rival ? \\[0pt]
Qu'est-ce qu'un sage ? \\[0pt]
Qu'est-ce qu'un savoir-faire ? \\[0pt]
Qu'est-ce qu'un sceptique ? \\[0pt]
Qu'est-ce qu'un sentiment moral ? \\[0pt]
Qu'est-ce qu'un sentiment vrai ? \\[0pt]
Qu'est-ce qu'un signe ? \\[0pt]
Qu'est-ce qu'un sophisme ? \\[0pt]
Qu'est-ce qu'un sophiste ? \\[0pt]
Qu'est-ce qu'un souvenir ? \\[0pt]
Qu'est-ce qu'un spécialiste ? \\[0pt]
Qu'est-ce qu'un spectacle ? \\[0pt]
Qu'est-ce qu'un spectateur ? \\[0pt]
Qu'est-ce qu'un style ? \\[0pt]
Qu'est-ce qu'un sujet de droit ? \\[0pt]
Qu'est-ce qu'un symbole ? \\[0pt]
Qu'est-ce qu'un symptôme ? \\[0pt]
Qu'est-ce qu'un système ? \\[0pt]
Qu'est-ce qu'un système philosophique ? \\[0pt]
Qu'est-ce qu'un tableau ? \\[0pt]
Qu'est-ce qu'un tabou ? \\[0pt]
Qu'est-ce qu'un technicien ? \\[0pt]
Qu'est-ce qu'un témoignage ? \\[0pt]
Qu'est-ce qu'un témoin ? \\[0pt]
Qu'est-ce qu'un temple ? \\[0pt]
Qu'est-ce qu'un territoire ? \\[0pt]
Qu'est-ce qu'un terroriste ? \\[0pt]
Qu'est-ce qu'un texte ? \\[0pt]
Qu'est-ce qu'un tout ? \\[0pt]
Qu'est-ce qu'un traître ? \\[0pt]
Qu'est-ce qu'un travail bien fait ? \\[0pt]
Qu'est-ce qu'un trouble social ? \\[0pt]
Qu'est-ce qu'un tyran ? \\[0pt]
Qu'est-ce qu'un vice ? \\[0pt]
Qu'est-ce qu'un visage ? \\[0pt]
Qu'est-ce qu'un vrai changement ? \\[0pt]
Qu'est-il légitime d'interdire ? \\[0pt]
Qu'est-il utile d'interdire ? \\[0pt]
Qu'est que l'autorité ? \\[0pt]
Qu'est qu'une image ? \\[0pt]
Qu'est qu'un régime politique ? \\[0pt]
Que suis-je ? \\[0pt]
Que suppose le mouvement ? \\[0pt]
Que transmettons-nous ? \\[0pt]
Que trouve-t-on dans ce que l'on trouve beau ? \\[0pt]
Que valent les classifications en sciences humaines ? \\[0pt]
Que valent les décisions de la majorité ? \\[0pt]
Que valent les excuses ? \\[0pt]
Que valent les idées générales ? \\[0pt]
Que valent les mots ? \\[0pt]
Que valent les préjugés ? \\[0pt]
Que valent les preuves ? \\[0pt]
Que valent les théories ? \\[0pt]
« Que va-t-il se passer ? » \\[0pt]
Que vaut en morale la justification par l'utilité ? \\[0pt]
Que vaut la décision de la majorité ? \\[0pt]
Que vaut la définition de l'homme comme animal doué de raison ? \\[0pt]
Que vaut la distinction entre nature et culture ? \\[0pt]
Que vaut la fidélité ? \\[0pt]
Que vaut le conseil : « vivez avec votre temps » ? \\[0pt]
Que vaut le travail ? \\[0pt]
Que vaut l'excuse : c'est plus fort que moi ? \\[0pt]
Que vaut l'excuse : « C'est plus fort que moi » ? \\[0pt]
Que vaut l'excuse : « Je ne l'ai pas fait exprès » ? \\[0pt]
Que vaut l'incertain ? \\[0pt]
Que vaut un argument d'autorité ? \\[0pt]
Que vaut un consensus ? \\[0pt]
Que vaut une connaissance empirique de l'homme ? \\[0pt]
Que vaut une parole ? \\[0pt]
Que vaut une preuve contre un préjugé ? \\[0pt]
Que veut dire avoir raison ? \\[0pt]
Que veut dire « essentiel » ? \\[0pt]
Que veut dire : être athée ? \\[0pt]
Que veut dire : « être cultivé » ? \\[0pt]
Que veut dire introduire à la métaphysique ? \\[0pt]
Que veut dire « je t'aime » ? \\[0pt]
Que veut dire : « je t'aime » ? \\[0pt]
Que veut dire : « le temps passe » ? \\[0pt]
Que veut dire l'expression « aller au fond des choses » ? \\[0pt]
Que veut dire « réel » ? \\[0pt]
Que veut dire « respecter la nature » ? \\[0pt]
Que veut dire : « respecter la nature » ? \\[0pt]
Que veut-on dire quand on dit que « rien n'est sans raison » ? \\[0pt]
Que veut-on dire quand on dit « rien n'est sans raison » ? \\[0pt]
Que voit-on dans une image ? \\[0pt]
Que voit-on dans un miroir ? \\[0pt]
Que voit-on dans un tableau ? \\[0pt]
Que voulons-nous ? \\[0pt]
Que voulons-nous vraiment savoir ? \\[0pt]
Que voyons-nous ? \\[0pt]
Que voyons-nous sur un tableau ? \\[0pt]
Qu'expriment les mythes ? \\[0pt]
Qu'expriment les œuvres d'art ? \\[0pt]
Qu'exprime une œuvre d'art ? \\[0pt]
Qui a des droits ? \\[0pt]
Qui agit ? \\[0pt]
Qui a le droit de juger ? \\[0pt]
Qui a le droit de punir ? \\[0pt]
Qui a une histoire ? \\[0pt]
Qui a une parole politique ? \\[0pt]
Qui commande ? \\[0pt]
Qui connaît le mieux mon corps ? \\[0pt]
Qui croire ? \\[0pt]
Qui détient le pouvoir ? \\[0pt]
Qui doit dire le droit ? \\[0pt]
Qui doit éduquer ? \\[0pt]
Qui doit faire les lois ? \\[0pt]
Qui doit gouverner ? \\[0pt]
Qui domine ? \\[0pt]
Qui donne la norme du goût ? \\[0pt]
Qui écrit l'histoire ? \\[0pt]
Qui est autorisé à me dire « tu dois » ? \\[0pt]
Qui est citoyen ? \\[0pt]
Qui est comédien ? \\[0pt]
Qui est compétent en matière politique ? \\[0pt]
Qui est crédible ? \\[0pt]
Qui est cultivé ? \\[0pt]
Qui est digne du bonheur ? \\[0pt]
Qui est immoral ? \\[0pt]
Qui est l'autre ? \\[0pt]
Qui est le maître ? \\[0pt]
Qui est le peuple ? \\[0pt]
Qui est l'homme des sciences humaines ? \\[0pt]
Qui est libre ? \\[0pt]
Qui est méchant ? \\[0pt]
Qui est métaphysicien ? \\[0pt]
Qui est mon prochain ? \\[0pt]
Qui est mon semblable ? \\[0pt]
Qui est responsable ? \\[0pt]
Qui est riche ? \\[0pt]
Qui est sage ? \\[0pt]
Qui est souverain ? \\[0pt]
Qui est une personne ? \\[0pt]
Qui fait la loi ? \\[0pt]
Qui fait l'histoire ? \\[0pt]
Qui fait l'unité du peuple ? \\[0pt]
Qui faut-il protéger ? \\[0pt]
Qui gouverne ? \\[0pt]
Qui me dit ce que je dois faire ? \\[0pt]
Qui mérite d'être aimé ? \\[0pt]
Qui meurt ? \\[0pt]
Qui nous dicte nos devoirs ? \\[0pt]
Qui parle ? \\[0pt]
Qui parle quand je dis « je » ? \\[0pt]
Qui pense ? \\[0pt]
Qui peut avoir des droits ? \\[0pt]
Qui peut me dire « tu ne dois pas » ? \\[0pt]
Qui peut obliger ? \\[0pt]
Qui peut parler ? \\[0pt]
Qui peut prétendre énoncer des devoirs ? \\[0pt]
Qui peut prétendre imposer des bornes à la technique ? \\[0pt]
Qui peut se passer de religion ? \\[0pt]
Qui pose les fins ? \\[0pt]
Qui sont mes amis ? \\[0pt]
Qui sont mes semblables ? \\[0pt]
Qui sont nos ennemis ? \\[0pt]
« Qui suis-je ? » \\[0pt]
Qui suis-je ? \\[0pt]
Qui suis-je et qui es-tu ? \\[0pt]
Qui suis-je pour me juger ? \\[0pt]
Qui travaille ? \\[0pt]
Qu'oppose-t-on à la vérité ? \\[0pt]
Qu'y a-t-il ? \\[0pt]
Qu'y a-t-il à comprendre dans une œuvre d'art ? \\[0pt]
Qu'y a-t-il à comprendre en histoire ? \\[0pt]
Qu'y a-t-il à craindre de la technique ? \\[0pt]
Qu'y a-t-il à l'origine de toutes choses ? \\[0pt]
Qu'y a-t-il au-delà de l'être ? \\[0pt]
Qu'y a-t-il au-delà du réel ? \\[0pt]
Qu'y a-t-il au-dessus des lois ? \\[0pt]
Qu'y a-t-il au fondement de l'objectivité ? \\[0pt]
Qu'y a-t-il de sacré ? \\[0pt]
Qu'y a-t-il de sérieux dans le jeu ? \\[0pt]
Qu'y a-t-il de technique dans l'art ? \\[0pt]
Qu'y a-t-il d'universel dans la culture ? \\[0pt]
Qu'y a-t-il que la nature fait en vain ? \\[0pt]
Qu'y a-t-il sinon le réel ? \\[0pt]
Sans les mots, que seraient les choses ? \\[0pt]
Si l'esprit n'est pas une table rase, qu'est-il ? \\[0pt]

\subsection{Question en « peut »}
\label{sec:orgdd389a1}

\noindent
Aimer peut-il être un devoir ? \\[0pt]
À quelles conditions le vivant peut-il être objet de science ? \\[0pt]
À quelles conditions peut-on dire qu'une action est historique ? \\[0pt]
À quelles conditions un choix peut-il être rationnel ? \\[0pt]
À quelles conditions un État peut-il être juste ? \\[0pt]
À quelles conditions une théorie peut-elle être scientifique ? \\[0pt]
À quelque chose le malheur peut-il être bon ? \\[0pt]
À quelque chose, le malheur peut-il être bon ? \\[0pt]
À qui peut-on faire confiance ? \\[0pt]
À quoi la logique peut-elle servir dans les sciences ? \\[0pt]
À quoi peut-on reconnaître la vérité ? \\[0pt]
À quoi peut-on reconnaître une œuvre d'art ? \\[0pt]
Ce que la morale autorise, l'État peut-il légitimement l'interdire ? \\[0pt]
Ce que la technique rend possible, peut-on jamais en empêcher la réalisation ? \\[0pt]
Ce que nous avons le devoir de faire peut-il toujours s'exprimer sous forme de loi ? \\[0pt]
Ce qui est contingent peut-il être objet de science ? \\[0pt]
Ce qui est contradictoire peut-il exister ? \\[0pt]
Ce qui ne peut s'acheter est-il dépourvu de valeur ? \\[0pt]
Ce qui n'est pas démontré peut-il être vrai ? \\[0pt]
Ce qui n'est pas matériel peut-il être réel ? \\[0pt]
Ce qu'on ne peut pas vendre \\[0pt]
Comment autrui peut-il m'aider à rechercher le bonheur ? \\[0pt]
Comment la réalité peut-elle dépasser la fiction ? \\[0pt]
Comment la volonté peut-elle être faible ? \\[0pt]
Comment la volonté peut-elle être indéterminée ? \\[0pt]
Comment le devoir peut-il déterminer l'action ? \\[0pt]
Comment le passé peut-il demeurer présent ? \\[0pt]
Comment l'homme peut-il se représenter le temps ? \\[0pt]
Comment l'inconscient peut-il se manifester ? \\[0pt]
Comment peut-on choisir entre différentes hypothèses ? \\[0pt]
Comment peut-on définir la politique ? \\[0pt]
Comment peut-on définir un être vivant ? \\[0pt]
Comment peut-on être heureux ? \\[0pt]
« Comment peut-on être persan ? » \\[0pt]
Comment peut-on être sceptique ? \\[0pt]
Comment peut-on savoir que l'on a raison ? \\[0pt]
Comment peut-on se trahir soi-même ? \\[0pt]
Dans les sciences humaines et sociales, peut-on préconiser l'imitation des sciences de la nature ? \\[0pt]
« Dans un bois aussi courbe que celui dont l'homme est fait on ne peut rien tailler de tout à fait droit » \\[0pt]
De quelle science humaine la folie peut-elle être l'objet ? \\[0pt]
De quoi l'art peut-il être contemporain ? \\[0pt]
De quoi l'art peut-il nous libérer ? \\[0pt]
De quoi la technique ne peut-elle pas nous libérer ? \\[0pt]
De quoi l'inconscient peut-il être la cause ? \\[0pt]
De quoi ne peut-on pas répondre ? \\[0pt]
De quoi peut-il y avoir science ? \\[0pt]
De quoi peut-on être certain ? \\[0pt]
De quoi peut-on être inconscient ? \\[0pt]
De quoi peut-on faire l'expérience ? \\[0pt]
De quoi peut-on réclamer la propriété ? \\[0pt]
Dieu peut-il tout faire ? \\[0pt]
En morale, peut-on dire : « C'est l'intention qui compte » ? \\[0pt]
En politique, peut-on faire table rase du passé ? \\[0pt]
En quel sens la maladie peut-elle transformer notre vie ? \\[0pt]
En quel sens l'anthropologie peut-elle être historique ? \\[0pt]
En quel sens peut-on appeler les sciences humaines « sciences de l'esprit » ? \\[0pt]
En quel sens peut-on dire de l'écologie qu'elle est politique ? \\[0pt]
En quel sens peut-on dire que la vérité s'impose ? \\[0pt]
En quel sens peut-on dire que le langage est la condition de la pensée ? \\[0pt]
En quel sens peut-on dire que le mal n'existe pas ? \\[0pt]
En quel sens peut-on dire que l'homme est un animal politique ? \\[0pt]
En quel sens peut-on dire que nos paroles nous trahissent ? \\[0pt]
En quel sens peut-on dire qu'« on expérimente avec sa raison » ? \\[0pt]
En quel sens peut-on parler de la mort de l'art ? \\[0pt]
En quel sens peut-on parler de la vie sociale comme d'un jeu ? \\[0pt]
En quel sens peut-on parler de responsabilité collective ? \\[0pt]
En quel sens peut-on parler de transcendance ? \\[0pt]
En quel sens peut-on parler d'expérience possible ? \\[0pt]
En quel sens peut-on parler d'une « culture technique » ? \\[0pt]
En quel sens peut-on parler d'une culture technique ? \\[0pt]
En quel sens peut-on parler d'une interprétation de la nature ? \\[0pt]
En quoi la connaissance de la matière peut-elle relever de la métaphysique ? \\[0pt]
En quoi la nature peut-elle être source de création ? \\[0pt]
En quoi l'art peut-il intéresser le philosophe ? \\[0pt]
En quoi la sexualité peut-elle être objet de science ? \\[0pt]
En quoi une culture peut-elle être la mienne ? \\[0pt]
Entre l'utile et l'honnête, peut-on choisir ? \\[0pt]
Est-ce par son objet ou par ses méthodes qu'une science peut se définir ? \\[0pt]
Faut-il douter de ce qu'on ne peut pas démontrer ? \\[0pt]
Jusqu'où peut-on dialoguer ? \\[0pt]
Jusqu'où peut-on soigner ? \\[0pt]
La barbarie peut-elle avoir un visage humain ? \\[0pt]
La beauté peut-elle délivrer une vérité ? \\[0pt]
La beauté peut-elle être cachée ? \\[0pt]
La beauté peut-elle laisser indifférent ? \\[0pt]
La biologie peut-elle se passer de causes finales ? \\[0pt]
L'absurde peut-il avoir un sens pour une pensée 155 \\[0pt]
La certitude ne peut-elle naître que de l'évidence ? \\[0pt]
La colère peut-elle être justifiée ? \\[0pt]
La compétence technique peut-elle fonder l'autorité publique ? \\[0pt]
La connaissance de la nécessité a priori peut-elle évoluer ? \\[0pt]
La connaissance du vivant peut-elle être désintéressée ? \\[0pt]
La connaissance peut-elle être pratique ? \\[0pt]
La connaissance peut-elle se passer de l'imagination ? \\[0pt]
La conscience peut-elle être collective ? \\[0pt]
La conscience peut-elle être objet de science ? \\[0pt]
La conscience peut-elle nous tromper ? \\[0pt]
La contrainte peut-elle être légitime ? \\[0pt]
La critique du pouvoir peut-elle conduire à la désobéissance ? \\[0pt]
La croyance peut-elle être rationnelle ? \\[0pt]
La croyance peut-elle s'accompagner de certitude ? \\[0pt]
La croyance peut-elle tenir lieu de savoir ? \\[0pt]
L'action politique peut-elle se passer de mots ? \\[0pt]
La culture peut-elle être instituée ? \\[0pt]
La culture peut-elle être objet de science ? \\[0pt]
La découverte de la vérité peut-elle être le fait du hasard ? \\[0pt]
La démocratie peut-elle échapper à la démagogie ? \\[0pt]
La démocratie peut-elle être représentative ? \\[0pt]
La démocratie peut-elle se passer de représentation ? \\[0pt]
La démonstration mathématique peut-elle servir de modèle à toute démonstration ? \\[0pt]
La désobéissance peut-elle être légitime ? \\[0pt]
La fonction de penser peut-elle se déléguer ? \\[0pt]
La fraternité peut-elle se passer d'un fondement religieux ? \\[0pt]
La généalogie peut-elle être une méthode historique ? \\[0pt]
La guerre peut-elle être juste ? \\[0pt]
La guerre peut-elle être justifiée ? \\[0pt]
La guerre peut-elle être un droit ? \\[0pt]
La justice peut-elle être fondée en nature ? \\[0pt]
La justice peut-elle se fonder sur le compromis ? \\[0pt]
La justice peut-elle se passer de la force ? \\[0pt]
La justice peut-elle se passer d'institutions ? \\[0pt]
La liberté peut-elle être prouvée ? \\[0pt]
La liberté peut-elle être une illusion ? \\[0pt]
La liberté peut-elle faire peur ? \\[0pt]
La liberté peut-elle s'affirmer sans violence ? \\[0pt]
La liberté peut-elle s'aliéner ? \\[0pt]
La liberté peut-elle se constater ? \\[0pt]
La liberté peut-elle se prouver ? \\[0pt]
La liberté peut-elle se refuser ? \\[0pt]
La linguistique peut-elle se passer de la psychologie ? \\[0pt]
La littérature peut-elle suppléer les sciences de l'homme ? \\[0pt]
La logique peut-elle se passer de la métaphysique ? \\[0pt]
La loi peut-elle changer les mœurs ? \\[0pt]
La loi peut-elle être injuste ? \\[0pt]
La magie peut-elle être efficace ? \\[0pt]
La majorité peut-elle être tyrannique ? \\[0pt]
La matière peut-elle agir ? \\[0pt]
La matière peut-elle être objet de connaissance ? \\[0pt]
La matière peut-elle penser ? \\[0pt]
L'ambiguïté des mots peut-elle être heureuse ? \\[0pt]
L'amélioration des hommes peut-elle être considérée comme un objectif politique ? \\[0pt]
La métaphysique peut-elle être autre chose qu'une science recherchée ? \\[0pt]
La métaphysique peut-elle faire appel à l'expérience ? \\[0pt]
L'amitié peut-elle obliger ? \\[0pt]
La morale peut-elle être fondée sur la science ? \\[0pt]
La morale peut-elle être naturelle ? \\[0pt]
La morale peut-elle être personnelle ? \\[0pt]
La morale peut-elle être un calcul ? \\[0pt]
La morale peut-elle être une science ? \\[0pt]
La morale peut-elle se définir comme l'art d'être heureux ? \\[0pt]
La morale peut-elle se fonder sur les sentiments ? \\[0pt]
La morale peut-elle s'enseigner ? \\[0pt]
La morale peut-elle se passer d'un fondement religieux ? \\[0pt]
L'amour peut-il être absolu ? \\[0pt]
L'amour peut-il être raisonnable ? \\[0pt]
L'amour peut-il être un devoir ? \\[0pt]
La musique peut-elle être narrative ? \\[0pt]
L'analyse du langage ordinaire peut-elle avoir un intérêt philosophique ? \\[0pt]
La nature peut-elle avoir des droits ? \\[0pt]
La nature peut-elle constituer une norme ? \\[0pt]
La nature peut-elle être belle ? \\[0pt]
La nature peut-elle être détruite ? \\[0pt]
La nature peut-elle être un modèle ? \\[0pt]
La nature peut-elle nous indiquer ce que nous devons faire ? \\[0pt]
L'animal peut-il être un sujet moral ? \\[0pt]
La parole peut-elle agir ? \\[0pt]
La parole peut-elle être étudiée comme un acte social ? \\[0pt]
La parole peut-elle être une arme ? \\[0pt]
La passion de la vérité peut-elle être source d'erreur ? \\[0pt]
La peinture peut-elle être un art du temps ? \\[0pt]
La pensée formelle peut-elle avoir un contenu ? \\[0pt]
La pensée peut-elle devenir une technique ? \\[0pt]
La pensée peut-elle s'écrire ? \\[0pt]
La pensée peut-elle se passer de mots ? \\[0pt]
La perception peut-elle être désintéressée ? \\[0pt]
La perception peut-elle s'éduquer ? \\[0pt]
La philosophie peut-elle disparaître ? \\[0pt]
La philosophie peut-elle être expérimentale ? \\[0pt]
La philosophie peut-elle être populaire ? \\[0pt]
La philosophie peut-elle être une science ? \\[0pt]
La philosophie peut-elle ne pas parler de Dieu ? \\[0pt]
La philosophie peut-elle se passer de l'idée de vérité ? \\[0pt]
La philosophie peut-elle se passer de théologie ? \\[0pt]
La pitié peut-elle fonder la morale ? \\[0pt]
La politique peut-elle changer la société ? \\[0pt]
La politique peut-elle changer le monde ? \\[0pt]
La politique peut-elle disparaître ? \\[0pt]
La politique peut-elle échapper au mythe ? \\[0pt]
La politique peut-elle être indépendante de la morale ? \\[0pt]
La politique peut-elle être morale ? \\[0pt]
La politique peut-elle être objet de science ? \\[0pt]
La politique peut-elle être un objet de science ? \\[0pt]
La politique peut-elle n'être qu'une pratique ? \\[0pt]
La politique peut-elle se passer de croyance ? \\[0pt]
La politique peut-elle se passer de croyances ? \\[0pt]
La politique peut-elle se passer des idéologies ? \\[0pt]
La politique peut-elle unir les hommes ? \\[0pt]
La politique peut-elle viser à autre chose qu'à l'efficacité ? \\[0pt]
La précaution peut-elle être un principe ? \\[0pt]
L'a priori peut-il être historique ? \\[0pt]
La question du sens de la vie peut-elle être sans réponses ? \\[0pt]
La raison d'État peut-elle être justifiée ? \\[0pt]
La raison peut-elle entrer en conflit avec elle-même ? \\[0pt]
La raison peut-elle entrer en contradiction avec elle-même ? \\[0pt]
La raison peut-elle errer ? \\[0pt]
La raison peut-elle être immédiatement pratique ? \\[0pt]
La raison peut-elle être pratique ? \\[0pt]
La raison peut-elle nous commander de croire ? \\[0pt]
La raison peut-elle nous égarer ? \\[0pt]
La raison peut-elle nous induire en erreur ? \\[0pt]
La raison peut-elle rendre compte du réel entièrement ? \\[0pt]
La raison peut-elle rendre raison de tout ? \\[0pt]
La raison peut-elle s'aveugler elle-même ? \\[0pt]
La raison peut-elle se contredire ? \\[0pt]
La raison peut-elle servir le mal ? \\[0pt]
La raison peut-elle s'opposer à elle-même ? \\[0pt]
La réalité peut-elle être virtuelle ? \\[0pt]
La recherche de la vérité peut-elle être désintéressée ? \\[0pt]
La recherche de la vérité peut-elle être une passion ? \\[0pt]
La recherche du bonheur peut-elle être un devoir ? \\[0pt]
La religion peut-elle être civile ? \\[0pt]
La religion peut-elle être naturelle ? \\[0pt]
La religion peut-elle faire lien social ? \\[0pt]
La religion peut-elle n'être qu'une affaire privée ? \\[0pt]
La religion peut-elle se passer de transcendant ? \\[0pt]
La religion peut-elle se passer du recours au sacré ? \\[0pt]
La religion peut-elle suppléer la raison ? \\[0pt]
La responsabilité peut-elle être collective ? \\[0pt]
La révolte peut-elle être un droit ? \\[0pt]
L'art a-t-il des vertus thérapeutiques ? \\[0pt]
L'artiste peut-il se passer d'un maître ? \\[0pt]
L'art peut-il changer le monde ? \\[0pt]
L'art peut-il contribuer à éduquer les hommes ? \\[0pt]
L'art peut-il encore imiter la nature ? \\[0pt]
L'art peut-il être abstrait ? \\[0pt]
L'art peut-il être brut ? \\[0pt]
L'art peut-il être conceptuel ? \\[0pt]
L'art peut-il être enseigné ? \\[0pt]
L'art peut-il être populaire ? \\[0pt]
L'art peut-il être réaliste \\[0pt]
L'art peut-il être révolutionnaire ? \\[0pt]
L'art peut-il être sans œuvre ? \\[0pt]
L'art peut-il être utile ? \\[0pt]
L'art peut-il finir ? \\[0pt]
L'art peut-il mourir ? \\[0pt]
L'art peut-il ne pas être sacré ? \\[0pt]
L'art peut-il n'être aucunement mimétique ? \\[0pt]
L'art peut-il n'être pas conceptuel ? \\[0pt]
L'art peut-il nous rendre meilleurs ? \\[0pt]
L'art peut-il prétendre à la vérité ? \\[0pt]
L'art peut-il quelque chose contre la morale ? \\[0pt]
L'art peut-il quelque chose pour la morale ? \\[0pt]
L'art peut-il rendre le mouvement ? \\[0pt]
L'art peut-il s'affranchir des lois ? \\[0pt]
L'art peut-il sauver le monde ? \\[0pt]
L'art peut-il s'enseigner ? \\[0pt]
L'art peut-il se passer de la beauté ? \\[0pt]
L'art peut-il se passer de règles ? \\[0pt]
L'art peut-il se passer des critères du beau et du laid ? \\[0pt]
L'art peut-il se passer d'idéal ? \\[0pt]
L'art peut-il se passer d'œuvres ? \\[0pt]
L'art peut-il tenir lieu de métaphysique ? \\[0pt]
L'art sait-il montrer ce que le langage ne peut pas dire ? \\[0pt]
La science du vivant peut-elle se passer de l'idée de finalité ? \\[0pt]
La science historique peut-elle être prédictive ? \\[0pt]
La science peut-elle errer ? \\[0pt]
La science peut-elle être une métaphysique ? \\[0pt]
La science peut-elle guider notre conduite ? \\[0pt]
La science peut-elle lutter contre les préjugés ? \\[0pt]
La science peut-elle produire des croyances ? \\[0pt]
La science peut-elle se passer de fondement ? \\[0pt]
La science peut-elle se passer de l'idée de finalité ? \\[0pt]
La science peut-elle se passer de métaphysique ? \\[0pt]
La science peut-elle se passer de méthode ? \\[0pt]
La science peut-elle se passer d'hypothèses ? \\[0pt]
La science peut-elle se passer d'institutions ? \\[0pt]
La science peut-elle tout expliquer ? \\[0pt]
La sensibilité peut-elle délivrer une vérité ? \\[0pt]
La servitude peut-elle être volontaire ? \\[0pt]
La société peut-elle être l'objet d'une science ? \\[0pt]
La société peut-elle être objet de connaissance ? \\[0pt]
La société peut-elle se passer de l'État ? \\[0pt]
La souffrance peut-elle avoir un sens ? \\[0pt]
La souffrance peut-elle être partagée ? \\[0pt]
La souffrance peut-elle être un mode de connaissance ? \\[0pt]
La souveraineté peut-elle être déléguée \\[0pt]
La souveraineté peut-elle être limitée ? \\[0pt]
La souveraineté peut-elle se partager ? \\[0pt]
La sympathie peut-elle tenir lieu de morale ? \\[0pt]
La sympathie peut-elle tenir lieu de moralité ? \\[0pt]
La technique peut-elle améliorer l'homme ? \\[0pt]
La technique peut-elle être tenue pour la forme moderne de la culture ? \\[0pt]
La technique peut-elle respecter la nature ? \\[0pt]
La technique peut-elle se déduire de la science ? \\[0pt]
La technique peut-elle se passer de la science ? \\[0pt]
L'athée peut-il être vertueux ? \\[0pt]
La théologie peut-elle être rationnelle ? \\[0pt]
La théorie peut-elle nous égarer ? \\[0pt]
La tolérance peut-elle constituer un problème pour la démocratie ? \\[0pt]
L'avenir peut-il être objet de connaissance ? \\[0pt]
La vérité peut-elle changer avec le temps ? \\[0pt]
La vérité peut-elle être équivoque ? \\[0pt]
La vérité peut-elle être indicible ? \\[0pt]
La vérité peut-elle être relative ? \\[0pt]
La vérité peut-elle être tolérante ? \\[0pt]
La vérité peut-elle faire peur ? \\[0pt]
La vérité peut-elle laisser indifférent ? \\[0pt]
La vérité peut-elle se définir par le consensus ? \\[0pt]
La vérité peut-elle se discuter ? \\[0pt]
La vertu peut-elle être excessive ? \\[0pt]
La vertu peut-elle être purement morale ? \\[0pt]
La vertu peut-elle s'enseigner ? \\[0pt]
La vie peut-elle être éternelle ? \\[0pt]
La vie peut-elle être objet de science ? \\[0pt]
La vie peut-elle être sans histoire ? \\[0pt]
La violence peut-elle avoir raison ? \\[0pt]
La violence peut-elle être gratuite ? \\[0pt]
La violence peut-elle être morale ? \\[0pt]
La vision peut-elle être le modèle de toute connaissance ? \\[0pt]
La volonté peut-elle être collective ? \\[0pt]
La volonté peut-elle être générale ? \\[0pt]
La volonté peut-elle être indéterminée ? \\[0pt]
La volonté peut-elle être libre ? \\[0pt]
La volonté peut-elle nous manquer ? \\[0pt]
La volonté politique peut-elle être commune ? \\[0pt]
Le beau peut-il être bizarre ? \\[0pt]
Le beau peut-il être effrayant ? \\[0pt]
Le bonheur peut-il être collectif ? \\[0pt]
Le bonheur peut-il être le but de la politique ? \\[0pt]
Le bonheur peut-il être un droit ? \\[0pt]
Le bonheur peut-il être un objectif politique ? \\[0pt]
L'échange peut-il être désintéressé ? \\[0pt]
Le châtiment peut-il ne rien devoir au désir de vengeance ? \\[0pt]
Le choix peut-il être éclairé ? \\[0pt]
Le citoyen peut-il être à la fois libre et soumis à l'État ? \\[0pt]
Le commerce peut-il être équitable ? \\[0pt]
Le consensus peut-il être critère de vérité ? \\[0pt]
Le consensus peut-il faire le vrai ? \\[0pt]
Le contradictoire peut-il exister ? \\[0pt]
Le contrat peut-il remplacer la loi ? \\[0pt]
Le corps peut-il être objet d'art ? \\[0pt]
Le cosmopolitisme peut-il devenir réalité ? \\[0pt]
Le cosmopolitisme peut-il être réaliste ? \\[0pt]
L'écriture peut-elle porter secours à la pensée ? \\[0pt]
Le désespoir peut-il passer pour une sagesse ? \\[0pt]
Le désir de vérité peut-il être interprété comme un désir de pouvoir ? \\[0pt]
Le désir peut-il atteindre son objet ? \\[0pt]
Le désir peut-il être désintéressé ? \\[0pt]
Le désir peut-il ne pas avoir d'objet ? \\[0pt]
Le désir peut-il nous rendre libre ? \\[0pt]
Le désir peut-il se satisfaire de la réalité ? \\[0pt]
Le despote peut-il être éclairé ? \\[0pt]
Le despotisme peut-il être éclairé ? \\[0pt]
Le doute peut-il être méthodique ? \\[0pt]
Le droit ne peut-il se fonder sur des faits ? \\[0pt]
Le droit peut-il échapper à l'histoire ? \\[0pt]
Le droit peut-il être flexible ? \\[0pt]
Le droit peut-il être naturel ? \\[0pt]
Le droit peut-il se fonder sur la force ? \\[0pt]
Le droit peut-il se passer de la morale ? \\[0pt]
L'éducation peut-elle être sentimentale ? \\[0pt]
L'efficacité thérapeutique de la psychanalyse \\[0pt]
L'égalité peut-elle être une menace pour la liberté ? \\[0pt]
Le hasard peut-il être un concept explicatif ?La morale doit-elle s'adapter à la réalité ? \\[0pt]
Le jugement critique peut-il s'exercer sans culture ? \\[0pt]
Le langage peut-il être un obstacle à la recherche de la vérité ? \\[0pt]
Le lien social peut-il être compassionnel ? \\[0pt]
Le mal peut-il être absolu ? \\[0pt]
Le mal peut-il être involontaire ? \\[0pt]
L'embryon humain peut-il être l'objet d'expérimentation ? \\[0pt]
Le méchant peut-il être heureux ? \\[0pt]
Le mensonge peut-il être au service de la vérité ? \\[0pt]
Le mépris peut-il être justifié ? \\[0pt]
Le moi peut-il être l'objet d'un savoir ? \\[0pt]
Le monde peut-il être nouveau ? \\[0pt]
L'émotion esthétique peut-elle se communiquer ? \\[0pt]
L'émotion esthétique peut-elle se partager ? \\[0pt]
L'enfance peut-elle servir de modèle ? \\[0pt]
L'enseignement peut-il se passer d'exemples ? \\[0pt]
Le pardon peut-il être une obligation ? \\[0pt]
Le passé peut-il être un objet de connaissance ? \\[0pt]
Le peuple peut-il se tromper ? \\[0pt]
Le plaisir esthétique peut-il se communiquer ? \\[0pt]
Le plaisir esthétique peut-il se partager ? \\[0pt]
Le plaisir peut-il être immoral ? \\[0pt]
Le plaisir peut-il être partagé ? \\[0pt]
Le politique peut-il faire abstraction de la morale ? \\[0pt]
Le pouvoir peut-il arrêter le pouvoir ? \\[0pt]
Le pouvoir peut-il être limité ? \\[0pt]
Le pouvoir peut-il ignorer la morale ? \\[0pt]
Le pouvoir peut-il limiter le pouvoir ? \\[0pt]
Le pouvoir peut-il se déléguer ? \\[0pt]
Le pouvoir peut-il se passer de sa mise en scène ? \\[0pt]
Le pouvoir peut-il se transmettre ? \\[0pt]
Le pouvoir politique peut-il échapper à l'arbitraire ? \\[0pt]
Le progrès technique peut-il être aliénant ? \\[0pt]
Le rationalisme peut-il être une passion ? \\[0pt]
Le réel peut-il échapper à la logique ? \\[0pt]
Le réel peut-il être beau ? \\[0pt]
Le réel peut-il être contradictoire ? \\[0pt]
Le relativisme peut-il être un principe de méthode dans les sciences humaines ? \\[0pt]
Le roman peut-il être philosophique ? \\[0pt]
L'erreur peut-elle donner un accès à la vérité ? \\[0pt]
L'erreur peut-elle jouer un rôle dans la connaissance scientifique ? \\[0pt]
Le savoir peut-il être gratuit ? \\[0pt]
Le sensible peut-il être connu ? \\[0pt]
Le sensible peut-il fonder des certitudes ? \\[0pt]
L'espoir peut-il être raisonnable ? \\[0pt]
L'esprit peut-il être divisé ? \\[0pt]
L'esprit peut-il être malade ? \\[0pt]
L'esprit peut-il être mesuré ? \\[0pt]
L'esprit peut-il être objet de science ? \\[0pt]
Le succès peut-il être un critère de vérité ? \\[0pt]
Le sujet peut-il s'aliéner par un libre choix ? \\[0pt]
L'État peut-il avoir tous les droits ? \\[0pt]
L'État peut-il créer la liberté ? \\[0pt]
L'État peut-il demeurer indifférent à la religion ? \\[0pt]
L'État peut-il disparaître ? \\[0pt]
L'État peut-il être fondé sur un contrat ? \\[0pt]
L'État peut-il être impartial ? \\[0pt]
L'État peut-il être indifférent à la religion ? \\[0pt]
L'État peut-il être libéral ? \\[0pt]
L'État peut-il fonder sa propre légitimité ? \\[0pt]
L'État peut-il limiter son pouvoir ? \\[0pt]
L'État peut-il poursuivre une autre fin que sa propre puissance ? \\[0pt]
L'État peut-il renoncer à la violence ? \\[0pt]
L'État peut-il se désintéresser de la religion ? \\[0pt]
Le temps peut-il s'arrêter ? \\[0pt]
Le travail peut-il être solitaire ? \\[0pt]
Le vrai peut-il rester invérifiable ? \\[0pt]
L'exception peut-elle confirmer la règle ? \\[0pt]
L'expérience du beau peut-elle être une expérience métaphysique ? \\[0pt]
L'expérience esthétique peut-elle se communiquer ? \\[0pt]
L'expérience peut-elle avoir raison des principes ? \\[0pt]
L'expérience peut-elle contredire la théorie ? \\[0pt]
L'expression peut-elle être libre ? \\[0pt]
L'histoire de l'art peut-elle arriver à son terme ? \\[0pt]
L'histoire peut-elle être contemporaine ? \\[0pt]
L'histoire peut-elle être objective ? \\[0pt]
L'histoire peut-elle être universelle ? \\[0pt]
L'histoire peut-elle ne pas avoir de sens ? \\[0pt]
L'histoire peut-elle s'écrire au présent ? \\[0pt]
L'histoire peut-elle se répéter ? \\[0pt]
L'historien peut-il être impartial ? \\[0pt]
L'historien peut-il se passer du concept de causalité ? \\[0pt]
L'homme injuste peut-il être heureux ? \\[0pt]
L'homme injuste peut-il être malheureux ? \\[0pt]
L'homme peut-il changer ? \\[0pt]
L'homme peut-il être inhumain ? \\[0pt]
L'homme peut-il être libéré du besoin ? \\[0pt]
L'homme peut-il être objet d'expérience ? \\[0pt]
L'homme peut-il faire l'objet d'une science ? \\[0pt]
L'homme peut-il se représenter un monde sans l'homme ? \\[0pt]
L'ignorance peut-elle être une excuse ? \\[0pt]
L'inconscience peut-elle être coupable ? \\[0pt]
L'inconscient peut-il se manifester ? \\[0pt]
L'indifférence peut-elle être une vertu ? \\[0pt]
L'individu particulier peut-il être l'objet d'une connaissance sociologique ? \\[0pt]
L'inexact peut-il avoir une valeur ? \\[0pt]
L'inquiétude peut-elle définir l'existence humaine ? \\[0pt]
L'inquiétude peut-elle devenir l'existence humaine ? \\[0pt]
L'intelligence peut-elle être artificielle ? \\[0pt]
L'intelligence peut-elle être inhumaine ? \\[0pt]
L'intérêt peut-il avoir une valeur morale ? \\[0pt]
L'intérêt peut-il être général ? \\[0pt]
L'intérêt peut-il être une valeur morale ? \\[0pt]
L'obéissance peut-elle être un acte de liberté ? \\[0pt]
L'obligation morale peut-elle se réduire à une obligation sociale ? \\[0pt]
L'ontologie peut-elle être relative ? \\[0pt]
L'ordre politique peut-il exclure la violence ? \\[0pt]
L'ordre social peut-il être juste ? \\[0pt]
L'utilité peut-elle être le principe de la moralité ? \\[0pt]
L'utilité peut-elle être un critère pour juger de la valeur de nos actions ? \\[0pt]
Notre rapport au monde peut-il être exclusivement technique ? \\[0pt]
Notre rapport au monde peut-il n'être que technique ? \\[0pt]
Par le langage, peut-on agir sur la réalité ? \\[0pt]
Penser peut-il nous rendre heureux ? \\[0pt]
Peut-il être moral de tuer ? \\[0pt]
Peut-il être préférable de ne pas savoir ? \\[0pt]
Peut-il exister une action désintéressée ? \\[0pt]
Peut-il exister une morale sceptique ? \\[0pt]
Peut-il y avoir conflit entre nos devoirs ? \\[0pt]
Peut-il y avoir de bonnes raisons de croire ? \\[0pt]
Peut-il y avoir de bons tyrans ? \\[0pt]
Peut-il y avoir de la politique sans conflit ? \\[0pt]
Peut-il y avoir des choix collectifs ? \\[0pt]
Peut-il y avoir des conflits de devoirs ? \\[0pt]
Peut-il y avoir des degrés dans la vérité ? \\[0pt]
Peut-il y avoir des échanges équitables ? \\[0pt]
Peut-il y avoir des expériences métaphysiques ? \\[0pt]
Peut-il y avoir des lois de l'histoire ? \\[0pt]
Peut-il y avoir des lois injustes ? \\[0pt]
Peut-il y avoir des modèles en morale ? \\[0pt]
Peut-il y avoir des nations sans État ? \\[0pt]
Peut-il y avoir des vérités partielles ? \\[0pt]
Peut-il y avoir esprit sans corps ? \\[0pt]
Peut-il y avoir plusieurs vérités religieuses ? \\[0pt]
Peut-il y avoir savoir-faire sans savoir ? \\[0pt]
Peut-il y avoir science sans intuition du vrai ? \\[0pt]
Peut-il y avoir un art conceptuel ? \\[0pt]
Peut-il y avoir un bien commun ? \\[0pt]
Peut-il y avoir un droit à désobéir ? \\[0pt]
Peut-il y avoir un droit de la guerre ? \\[0pt]
Peut-il y avoir une anthropologie non anthropocentriste ? \\[0pt]
Peut-il y avoir une ethnologie non ethnocentriste ? \\[0pt]
Peut-il y avoir une histoire universelle ? \\[0pt]
Peut-il y avoir une méthode commune à toutes les sciences ? \\[0pt]
Peut-il y avoir une morale dans les échanges ? \\[0pt]
Peut-il y avoir une pensée sans langage ? \\[0pt]
Peut-il y avoir une philosophie applicable ? \\[0pt]
Peut-il y avoir une philosophie appliquée ? \\[0pt]
Peut-il y avoir une philosophie politique sans Dieu ? \\[0pt]
Peut-il y avoir une politique de la langue ? \\[0pt]
Peut-il y avoir une pratique sans théorie ? \\[0pt]
Peut-il y avoir une science de la morale ? \\[0pt]
Peut-il y avoir une science de l'éducation ? \\[0pt]
Peut-il y avoir une science de l'homme ? \\[0pt]
Peut-il y avoir une science de l'inconscient ? \\[0pt]
Peut-il y avoir une science des signes ? \\[0pt]
Peut-il y avoir une science du désordre ? \\[0pt]
Peut-il y avoir une science du moi ? \\[0pt]
Peut-il y avoir une science politique ? \\[0pt]
Peut-il y avoir une servitude volontaire ? \\[0pt]
Peut-il y avoir une société des nations ? \\[0pt]
Peut-il y avoir une société sans État ? \\[0pt]
Peut-il y avoir un État mondial ? \\[0pt]
Peut-il y avoir une vérité en art ? \\[0pt]
Peut-il y avoir une vérité en politique ? \\[0pt]
Peut-il y avoir une vérité religieuse ? \\[0pt]
Peut-il y avoir un intérêt collectif ? \\[0pt]
Peut-il y avoir un langage universel ? \\[0pt]
Peut-on abolir la religion ? \\[0pt]
Peut-on admettre ce qui n'est pas vérifié ? \\[0pt]
Peut-on admettre un droit à la révolte ? \\[0pt]
Peut-on affirmer que le monde a un ordre ? \\[0pt]
Peut-on affirmer qu'être, c'est être perçu ou percevoir ? \\[0pt]
Peut-on agir de manière désintéressée ? \\[0pt]
Peut-on agir machinalement ? \\[0pt]
Peut-on agir sans prendre de risques ? \\[0pt]
Peut-on agir sans raison ? \\[0pt]
Peut-on aimer ce qu'on ne connaît pas ? \\[0pt]
Peut-on aimer l'autre tel qu'il est ? \\[0pt]
Peut-on aimer la vie plus que tout ? \\[0pt]
Peut-on aimer les animaux ? \\[0pt]
Peut-on aimer l'humanité ? \\[0pt]
Peut-on aimer sans perdre sa liberté ? \\[0pt]
Peut-on aimer son prochain comme soi-même ? \\[0pt]
Peut-on aimer son travail ? \\[0pt]
Peut-on aimer une œuvre d'art sans la comprendre ? \\[0pt]
Peut-on à la fois préserver et dominer la nature ? \\[0pt]
Peut-on aller à l'encontre de la nature ? \\[0pt]
Peut-on appréhender les choses telles qu'elles sont ? \\[0pt]
Peut-on apprendre à être heureux ? \\[0pt]
Peut-on apprendre à être juste ? \\[0pt]
Peut-on apprendre à être libre ? \\[0pt]
Peut-on apprendre à mourir ? \\[0pt]
Peut-on apprendre à penser ? \\[0pt]
Peut-on apprendre à vivre ? \\[0pt]
Peut-on argumenter en morale ? \\[0pt]
Peut-on arrêter le progrès ? \\[0pt]
Peut-on assimiler le vivant à une machine ? \\[0pt]
Peut-on atteindre une certitude ? \\[0pt]
Peut-on attendre de la politique qu'elle soit conforme aux exigences de la raison ? \\[0pt]
Peut-on attribuer à chacun son dû ? \\[0pt]
Peut-on avoir conscience de soi sans avoir conscience d'autrui ? \\[0pt]
Peut-on avoir de bonnes raisons de ne pas dire la vérité ? \\[0pt]
Peut-on avoir des droits sans avoir de devoirs ? \\[0pt]
Peut-on avoir le droit de se révolter ? \\[0pt]
Peut-on avoir peur de la liberté ? \\[0pt]
Peut-on avoir peur de soi-même ? \\[0pt]
Peut-on avoir peur d'être libre ? \\[0pt]
Peut-on avoir raison contre la science ? \\[0pt]
Peut-on avoir raison contre les faits ? \\[0pt]
Peut-on avoir raison contre tous ? \\[0pt]
Peut-on avoir raison contre tout le monde ? \\[0pt]
Peut-on avoir raisons contre les faits ? \\[0pt]
Peut-on avoir raison tout.e seul.e ? \\[0pt]
Peut-on avoir raison tout seul ? \\[0pt]
Peut-on avoir trop d'imagination ? \\[0pt]
Peut-on bien agir sans le savoir ? \\[0pt]
Peut-on cesser de croire ? \\[0pt]
Peut-on cesser de désirer ? \\[0pt]
Peut-on changer au point de ne plus être le même ? \\[0pt]
Peut-on changer de culture ? \\[0pt]
Peut-on changer de logique ? \\[0pt]
Peut-on changer le cours de l'histoire ? \\[0pt]
Peut-on changer le monde ? \\[0pt]
Peut-on changer le passé ? \\[0pt]
Peut-on changer ses désirs ? \\[0pt]
Peut-on changer ses désirs à volonté ? \\[0pt]
Peut-on choisir de renoncer à sa liberté ? \\[0pt]
Peut-on choisir le mal ? \\[0pt]
Peut-on choisir sa vie ? \\[0pt]
Peut-on choisir ses désirs ? \\[0pt]
Peut-on classer les arts ? \\[0pt]
Peut-on combattre une croyance par le raisonnement ? \\[0pt]
Peut-on commander à la nature ? \\[0pt]
Peut-on communiquer ses perceptions à autrui ? \\[0pt]
Peut-on communiquer son expérience ? \\[0pt]
Peut-on comparer deux philosophies ? \\[0pt]
Peut-on comparer les cultures ? \\[0pt]
Peut-on comparer l'organisme à une machine ? \\[0pt]
Peut-on comprendre ce qui est illogique ? \\[0pt]
Peut-on comprendre le présent ? \\[0pt]
Peut-on comprendre les actions humaines ? \\[0pt]
Peut-on comprendre un acte que l'on désapprouve ? \\[0pt]
Peut-on comprendre un auteur mieux qu'il ne s'est compris lui-même ? \\[0pt]
Peut-on concevoir la science achevée ? \\[0pt]
Peut-on concevoir une humanité sans art ? \\[0pt]
Peut-on concevoir une morale sans sanction ni obligation ? \\[0pt]
Peut-on concevoir une religion dans les limites de la simple raison ? \\[0pt]
Peut-on concevoir une science qui ne soit pas démonstrative ? \\[0pt]
Peut-on concevoir une science sans expérience ? \\[0pt]
Peut-on concevoir une société juste sans que les hommes ne le soient ? \\[0pt]
Peut-on concevoir une société qui n'aurait plus besoin du droit ? \\[0pt]
Peut-on concevoir une société sans État ? \\[0pt]
Peut-on concevoir un État mondial ? \\[0pt]
Peut-on concilier bonheur et liberté ? \\[0pt]
Peut-on concilier le droit à la liberté et l'égalité des droits ? \\[0pt]
Peut-on concilier nécessité et liberté ? \\[0pt]
Peut-on conclure de l'être au devoir-être ? \\[0pt]
Peut-on connaître autrui ? \\[0pt]
Peut-on connaître les causes ? \\[0pt]
Peut-on connaître les choses telles qu'elles sont ? \\[0pt]
Peut-on connaître le singulier ? \\[0pt]
Peut-on connaître l'esprit ? \\[0pt]
Peut-on connaître le vivant sans le dénaturer ? \\[0pt]
Peut-on connaître le vivant sans recourir à la notion de finalité ? \\[0pt]
Peut-on connaître l'individuel ? \\[0pt]
Peut-on connaître par intuition ? \\[0pt]
Peut-on connaître sans concept ? \\[0pt]
Peut-on connaître une vie humaine ? \\[0pt]
Peut-on considérer la conscience comme une chose ? \\[0pt]
Peut-on considérer la main comme un outil ? \\[0pt]
Peut-on considérer l'art comme un langage ? \\[0pt]
Peut-on considérer les hommes de science comme des autorités morales ? \\[0pt]
Peut-on contester les droits de l'homme ? \\[0pt]
Peut-on contredire l'expérience ? \\[0pt]
Peut-on convaincre quelqu'un de la beauté d'une œuvre d'art ? \\[0pt]
Peut-on craindre la liberté ? \\[0pt]
Peut-on créer un homme nouveau ? \\[0pt]
Peut-on critiquer la démocratie ? \\[0pt]
Peut-on critiquer la religion ? \\[0pt]
Peut-on croire ce qu'on veut ? \\[0pt]
Peut-on croire en rien ? \\[0pt]
Peut-on croire librement ? \\[0pt]
Peut-on croire sans être crédule ? \\[0pt]
Peut-on croire sans savoir pourquoi ? \\[0pt]
Peut-on décider de croire ? \\[0pt]
Peut-on décider de tout ? \\[0pt]
Peut-on décider d'être heureux ? \\[0pt]
Peut-on définir ce qu'est un progrès technique ? \\[0pt]
Peut-on définir la matière ? \\[0pt]
Peut-on définir la morale comme l'art d'être heureux ? \\[0pt]
Peut-on définir la santé ? \\[0pt]
Peut-on définir la vérité ? \\[0pt]
Peut-on définir la vérité par la correspondance ? \\[0pt]
Peut-on définir la vie ? \\[0pt]
Peut-on définir le bien ? \\[0pt]
Peut-on définir le bonheur ? \\[0pt]
Peut-on délimiter le réel ? \\[0pt]
Peut-on délimiter l'humain ? \\[0pt]
Peut-on démontrer l'existence de la matière ? \\[0pt]
Peut-on démontrer que l'on est libre ? \\[0pt]
Peut-on démontrer qu'on ne rêve pas ? \\[0pt]
Peut-on démontrer une théorie ? \\[0pt]
Peut-on dépasser la subjectivité ? \\[0pt]
Peut-on dériver des concepts de l'expérience ? \\[0pt]
Peut-on désirer autre chose qu'être heureux ? \\[0pt]
Peut-on désirer ce qui est ? \\[0pt]
Peut-on désirer ce qu'on ne veut pas ? \\[0pt]
Peut-on désirer ce qu'on possède ? \\[0pt]
Peut-on désirer ce qu'on sait être impossible ? \\[0pt]
Peut-on désirer l'absolu ? \\[0pt]
Peut-on désirer l'impossible ? \\[0pt]
Peut-on désirer sans souffrir ? \\[0pt]
Peut-on désobéir à l'État ? \\[0pt]
Peut-on désobéir aux lois ? \\[0pt]
Peut-on désobéir par devoir ? \\[0pt]
Peut-on dialoguer avec soi-même ? \\[0pt]
Peut-on dialoguer avec un ordinateur ? \\[0pt]
Peut-on dire ce que l'on pense ? \\[0pt]
Peut-on dire ce qui n'est pas ? \\[0pt]
Peut-on dire de la connaissance scientifique qu'elle procède par approximation ? \\[0pt]
Peut-on dire de l'art qu'il donne un monde en partage ? \\[0pt]
Peut-on dire d'une image qu'elle parle ? \\[0pt]
Peut-on dire d'une œuvre d'art qu'elle est ratée ? \\[0pt]
Peut-on dire d'une théorie scientifique qu'elle n'est jamais plus vraie qu'une autre mais seulement plus commode ? \\[0pt]
Peut-on dire d'un homme qu'il est supérieur à un autre homme ? \\[0pt]
Peut-on dire la vérité ? \\[0pt]
Peut-on dire le singulier ? \\[0pt]
Peut-on dire l'être ? \\[0pt]
Peut-on dire que la science désenchante le monde ? \\[0pt]
Peut-on dire que la science ne nous fait pas connaître les choses mais les rapports entre les choses ? \\[0pt]
Peut-on dire que les hommes font l'histoire ? \\[0pt]
Peut-on dire que les machines travaillent pour nous ? \\[0pt]
Peut-on dire que les mots pensent pour nous ? \\[0pt]
Peut-on dire que l'humanité progresse ? \\[0pt]
Peut-on dire que rien n'échappe à la technique ? \\[0pt]
Peut-on dire qu'est vrai ce qui correspond aux faits ? \\[0pt]
Peut-on dire que toutes les croyances se valent ? \\[0pt]
Peut-on dire que tout est politique ? \\[0pt]
Peut-on dire que tout est relatif ? \\[0pt]
Peut-on dire qu'une théorie physique en contredit une autre ? \\[0pt]
Peut-on dire toute la vérité ? \\[0pt]
Peut-on discuter des goûts et des couleurs ? \\[0pt]
Peut-on disposer de son corps ? \\[0pt]
Peut-on distinguer différents types de causes ? \\[0pt]
Peut-on distinguer entre de bons et de mauvais désirs ? \\[0pt]
Peut-on distinguer entre les bons et les mauvais désirs ? \\[0pt]
Peut-on distinguer la vie et l'existence ? \\[0pt]
Peut-on distinguer le réel de l'imaginaire ? \\[0pt]
Peut-on distinguer les faits de leurs interprétations ? \\[0pt]
Peut-on distinguer le vrai du faux ? \\[0pt]
Peut-on donner un sens à la souffrance ? \\[0pt]
Peut-on donner un sens à l'existence ? \\[0pt]
Peut-on donner un sens à son existence ? \\[0pt]
Peut-on douter de la raison ? \\[0pt]
Peut-on douter de l'humanité d'un homme ? \\[0pt]
Peut-on douter de sa propre existence ? \\[0pt]
Peut-on douter de soi ? \\[0pt]
Peut-on douter de tout ? \\[0pt]
Peut-on douter de toute vérité ? \\[0pt]
Peut-on douter du sens des mots ? \\[0pt]
Peut-on échanger des idées ? \\[0pt]
Peut-on échapper à ses désirs ? \\[0pt]
Peut-on échapper à son temps ? \\[0pt]
Peut-on échapper au cours de l'histoire ? \\[0pt]
Peut-on échapper au temps ? \\[0pt]
Peut-on échapper aux relations de pouvoir ? \\[0pt]
Peut-on éclairer la liberté ? \\[0pt]
Peut-on écrire comme on parle ? \\[0pt]
Peut-on éduquer la conscience ? \\[0pt]
Peut-on éduquer la sensibilité ? \\[0pt]
Peut-on éduquer le goût ? \\[0pt]
Peut-on éduquer quelqu'un à être libre ? \\[0pt]
Peut-on en appeler à la conscience contre la loi ? \\[0pt]
Peut-on en appeler à la conscience contre l'État ? \\[0pt]
Peut-on encore être cosmopolite ? \\[0pt]
Peut-on encore parler d'une nature humaine ? \\[0pt]
Peut-on encore soutenir que l'homme est un animal rationnel ? \\[0pt]
Peut-on en finir avec la violence ? \\[0pt]
Peut-on en finir avec les préjugés ? \\[0pt]
Peut-on en savoir trop ? \\[0pt]
Peut-on entreprendre d'éliminer la métaphysique ? \\[0pt]
Peut-on espérer être libéré du travail ? \\[0pt]
Peut-on établir une hiérarchie des arts ? \\[0pt]
Peut-on être à la fois lucide et heureux ? \\[0pt]
Peut-on être aliéné sans le savoir ? \\[0pt]
Peut-on être amoral ? \\[0pt]
Peut-on être apolitique ? \\[0pt]
Peut-on être assuré d'avoir raison ? \\[0pt]
Peut-on être athée ? \\[0pt]
Peut-on être citoyen du monde ? \\[0pt]
Peut-on être citoyen sans être soldat ? \\[0pt]
Peut-on être complètement athée ? \\[0pt]
Peut-on être content de soi ? \\[0pt]
Peut-on être dans le présent ? \\[0pt]
Peut-on être désintéressé ? \\[0pt]
Peut-on être en avance sur son temps ? \\[0pt]
Peut-on être en conflit avec soi-même ? \\[0pt]
Peut-on être esclave de soi-même ? \\[0pt]
Peut-on être étranger à soi-même ? \\[0pt]
Peut-on être étranger au monde ? \\[0pt]
Peut-on être fidèle à soi-même ? \\[0pt]
Peut-on être heureux dans la solitude ? \\[0pt]
Peut-on être heureux sans être sage ? \\[0pt]
Peut-on être heureux sans le savoir ? \\[0pt]
Peut-on être heureux sans philosophie ? \\[0pt]
Peut-on être heureux sans s'en rendre compte ? \\[0pt]
Peut-on être heureux tout seul ? \\[0pt]
Peut-on être homme sans être citoyen ? \\[0pt]
Peut-on être hors de soi ? \\[0pt]
Peut-on être ignorant ? \\[0pt]
Peut-on être impartial ? \\[0pt]
Peut-on être indifférent à l'histoire ? \\[0pt]
Peut-on être indifférent à son bonheur ? \\[0pt]
Peut-on être injuste avec soi-même ? \\[0pt]
Peut-on être injuste envers soi-même ? \\[0pt]
Peut-on être injuste et heureux ? \\[0pt]
Peut-on être insensible à l'art ? \\[0pt]
Peut-on être insensible au vrai ? \\[0pt]
Peut-on être juste dans une situation injuste ? \\[0pt]
Peut-on être juste dans une société injuste ? \\[0pt]
Peut-on être juste sans être impartial ? \\[0pt]
Peut-on être libre sans le savoir ? \\[0pt]
Peut-on être maître de soi ? \\[0pt]
Peut-on être méchant volontairement ? \\[0pt]
Peut-on être moral sans religion ? \\[0pt]
Peut-on être objectif en matière d'art ? \\[0pt]
Peut-on être obligé d'aimer ? \\[0pt]
Peut-on être plus ou moins libre ? \\[0pt]
Peut-on être prisonnier de soi-même ? \\[0pt]
Peut-on être responsable de ce que l'on n'a pas fait ? \\[0pt]
Peut-on être sage inconsciemment ? \\[0pt]
Peut-on être sage tout seul ? \\[0pt]
Peut-on être sans opinion ? \\[0pt]
Peut-on être sceptique ? \\[0pt]
Peut-on être sceptique de bonne foi ? \\[0pt]
Peut-on être seul ? \\[0pt]
Peut-on être seul avec soi-même ? \\[0pt]
Peut-on être soi-même en société ? \\[0pt]
Peut-on être sûr d'avoir raison ? \\[0pt]
Peut-on être sûr de bien agir ? \\[0pt]
Peut-on être sûr de ne pas se tromper ? \\[0pt]
Peut-on être trop religieux ? \\[0pt]
Peut-on être trop sage ? \\[0pt]
Peut-on être trop sensible ? \\[0pt]
Peut-on étudier le passé de façon objective ? \\[0pt]
Peut-on étudier l'homme de l'extérieur ? \\[0pt]
Peut-on exercer son esprit ? \\[0pt]
Peut-on expérimenter sur le vivant ? \\[0pt]
Peut-on expliquer le mal ? \\[0pt]
Peut-on expliquer le monde par la matière ? \\[0pt]
Peut-on expliquer le vivant ? \\[0pt]
Peut-on expliquer un acte libre ? \\[0pt]
Peut-on expliquer une œuvre d'art ? \\[0pt]
Peut-on faire ce qu'on ne veut pas ? \\[0pt]
Peut-on faire comme si le passé n'existait pas ? \\[0pt]
Peut-on faire de la politique sans supposer les hommes méchants ? \\[0pt]
Peut-on faire de l'art avec tout ? \\[0pt]
Peut-on faire de l'esprit un objet de science ? \\[0pt]
Peut-on faire de sa vie une œuvre d'art ? \\[0pt]
Peut-on faire du dialogue un modèle de relation morale ? \\[0pt]
Peut-on faire la paix ? \\[0pt]
Peut-on faire la philosophie de l'histoire ? \\[0pt]
Peut-on faire le bien d'autrui malgré lui ? \\[0pt]
Peut-on faire le bien de quelqu'un malgré lui ? \\[0pt]
Peut-on faire le bonheur d'autrui ? \\[0pt]
Peut-on faire l'économie de la notion de forme ? \\[0pt]
Peut-on faire l'éloge de l'illusion ? \\[0pt]
Peut-on faire le mal en vue du bien ? \\[0pt]
Peut-on faire le mal innocemment ? \\[0pt]
Peut-on faire l'expérience de la nécessité ? \\[0pt]
Peut-on faire l'inventaire du monde ? \\[0pt]
Peut-on faire son devoir par habitude ? \\[0pt]
Peut-on faire table rase du passé ? \\[0pt]
Peut-on faire un mauvais usage de la raison ? \\[0pt]
Peut-on feindre la vertu ? \\[0pt]
Peut-on fixer des limites à la science ? \\[0pt]
Peut-on fonder la liberté ? \\[0pt]
Peut-on fonder la morale ? \\[0pt]
Peut-on fonder la morale sur la pitié ? \\[0pt]
Peut-on fonder le droit de propriété ? \\[0pt]
Peut-on fonder le droit sur la morale ? \\[0pt]
Peut-on fonder les droits de l'homme ? \\[0pt]
Peut-on fonder les mathématiques ? \\[0pt]
Peut-on fonder un droit de désobéir ? \\[0pt]
Peut-on fonder une éthique sur la biologie ? \\[0pt]
Peut-on fonder une morale sur la nature ? \\[0pt]
Peut-on fonder une morale sur le plaisir ? \\[0pt]
Peut-on forcer quelqu'un à être libre ? \\[0pt]
Peut-on forcer un homme à être libre ? \\[0pt]
Peut-on fuir hors du monde ? \\[0pt]
Peut-on fuir la société ? \\[0pt]
Peut-on gâcher son talent ? \\[0pt]
Peut-on gouverner le développement des innovations techniques ? \\[0pt]
Peut-on gouverner sans lois ? \\[0pt]
Peut-on guérir par la parole ? \\[0pt]
Peut-on haïr la raison ? \\[0pt]
Peut-on haïr la vie ? \\[0pt]
Peut-on haïr les images ? \\[0pt]
Peut-on hiérarchiser les arts ? \\[0pt]
Peut-on hiérarchiser les devoirs ? \\[0pt]
Peut-on hiérarchiser les œuvres d'art ? \\[0pt]
Peut-on hiérarchiser les sociétés ? \\[0pt]
Peut-on identifier le désir au besoin ? \\[0pt]
Peut-on ignorer sa propre liberté ? \\[0pt]
Peut-on ignorer volontairement la vérité ? \\[0pt]
Peut-on imaginer l'avenir ? \\[0pt]
Peut-on imaginer un langage universel ? \\[0pt]
Peut-on imposer la liberté ? \\[0pt]
Peut-on innover en politique ? \\[0pt]
Peut-on interpréter la nature ? \\[0pt]
Peut-on inventer en morale ? \\[0pt]
Peut-on invoquer la nature comme une excuse ? \\[0pt]
Peut-on jamais aimer son prochain ? \\[0pt]
Peut-on jamais avoir la conscience tranquille ? \\[0pt]
Peut-on juger autrui ? \\[0pt]
Peut-on juger de la valeur d'une vie humaine ? \\[0pt]
Peut-on juger des œuvres d'art sans recourir à l'idée de beauté ? \\[0pt]
Peut-on justifier la discrimination ? \\[0pt]
Peut-on justifier la guerre ? \\[0pt]
Peut-on justifier la raison d'État ? \\[0pt]
Peut-on justifier la violence ? \\[0pt]
Peut-on justifier le mal ? \\[0pt]
Peut-on justifier le mensonge ? \\[0pt]
Peut-on justifier le recours à l'idée de finalité ? \\[0pt]
Peut-on justifier ses choix ? \\[0pt]
Peut-on légitimer la violence ? \\[0pt]
Peut-on limiter l'expression de la volonté du peuple ? \\[0pt]
Peut-on lutter contre le destin ? \\[0pt]
Peut-on lutter contre soi-même ? \\[0pt]
Peut-on maîtriser la nature ? \\[0pt]
Peut-on maîtriser la technique ? \\[0pt]
Peut-on maîtriser le risque ? \\[0pt]
Peut-on maîtriser le temps ? \\[0pt]
Peut-on maîtriser l'évolution de la technique ? \\[0pt]
Peut-on maîtriser l'inconscient ? \\[0pt]
Peut-on maîtriser ses désirs ? \\[0pt]
Peut-on manipuler les esprits ? \\[0pt]
Peut-on manquer de culture ? \\[0pt]
Peut-on manquer de volonté ? \\[0pt]
Peut-on me contraindre à faire mon devoir ? \\[0pt]
Peut-on mentir par humanité ? \\[0pt]
Peut-on mesurer les phénomènes sociaux ? \\[0pt]
Peut-on mesurer le temps ? \\[0pt]
Peut-on montrer en cachant ? \\[0pt]
Peut-on moraliser la guerre ? \\[0pt]
Peut-on ne croire en rien \\[0pt]
Peut-on ne croire en rien ? \\[0pt]
Peut-on ne pas connaître son bonheur ? \\[0pt]
Peut-on ne pas croire ? \\[0pt]
Peut-on ne pas croire à la science ? \\[0pt]
Peut-on ne pas croire au progrès ? \\[0pt]
Peut-on ne pas être de son temps ? \\[0pt]
Peut-on ne pas être égoïste ? \\[0pt]
Peut-on ne pas être matérialiste ? \\[0pt]
Peut-on ne pas être soi-même ? \\[0pt]
Peut-on ne pas interpréter ? \\[0pt]
Peut-on ne pas manquer de temps ? \\[0pt]
Peut-on ne pas perdre son temps ? \\[0pt]
Peut-on ne pas rechercher le bonheur ? \\[0pt]
Peut-on ne pas savoir ce que l'on dit ? \\[0pt]
Peut-on ne pas savoir ce que l'on fait ? \\[0pt]
Peut-on ne pas savoir ce que l'on veut ? \\[0pt]
Peut-on ne pas savoir ce qu'on veut ? \\[0pt]
Peut-on ne pas vouloir être heureux ? \\[0pt]
Peut-on ne penser à rien ? \\[0pt]
Peut-on ne rien aimer ? \\[0pt]
Peut-on ne rien devoir à personne ? \\[0pt]
Peut-on ne rien vouloir ? \\[0pt]
Peut-on ne vivre qu'au présent ? \\[0pt]
Peut-on nier la réalité ? \\[0pt]
Peut-on nier le réel ? \\[0pt]
Peut-on nier l'évidence ? \\[0pt]
Peut-on nier l'existence de la matière ? \\[0pt]
Peut-on nier l'existence du temps ? \\[0pt]
Peut-on objectiver le psychisme ? \\[0pt]
Peut-on opposer connaissance scientifique et création artistique ? \\[0pt]
Peut-on opposer justice et liberté ? \\[0pt]
Peut-on opposer le loisir au travail ? \\[0pt]
Peut-on opposer morale et technique ? \\[0pt]
Peut-on opposer nature et culture ? \\[0pt]
Peut-on opposer religion et raison ? \\[0pt]
Peut-on ôter à l'homme sa liberté ? \\[0pt]
Peut-on oublier ? \\[0pt]
Peut-on oublier de vivre ? \\[0pt]
Peut-on pactiser avec un brigand ? \\[0pt]
Peut-on parler d'art primitif ? \\[0pt]
Peut-on parler de capital culturel ? \\[0pt]
Peut-on parler de ce qui n'existe pas ? \\[0pt]
Peut-on parler de conscience collective ? \\[0pt]
Peut-on parler de corruption des mœurs ? \\[0pt]
Peut-on parler de despotisme démocratique ? \\[0pt]
Peut-on parler de dialogue des cultures ? \\[0pt]
Peut-on parler de droits des animaux ? \\[0pt]
Peut-on parler de mondes imaginaires ? \\[0pt]
Peut-on parler de « nature humaine » ? \\[0pt]
Peut-on parler de nourriture spirituelle ? \\[0pt]
Peut-on parler de problèmes techniques ? \\[0pt]
Peut-on parler de progrès en art ? \\[0pt]
Peut-on parler des miracles de la technique ? \\[0pt]
Peut-on parler des œuvres d'art ? \\[0pt]
Peut-on parler de « travail intellectuel » ? \\[0pt]
Peut-on parler de travail intellectuel ? \\[0pt]
Peut-on parler de vérité en art ? \\[0pt]
Peut-on parler de vérités métaphysiques ? \\[0pt]
Peut-on parler de vérité subjective ? \\[0pt]
Peut-on parler de vérité théâtrale ? \\[0pt]
Peut-on parler de vertu politique ? \\[0pt]
Peut-on parler de violence d'État ? \\[0pt]
Peut-on parler d'un droit de la guerre ? \\[0pt]
Peut-on parler d'un droit de résistance ? \\[0pt]
Peut-on parler d'une expérience religieuse ? \\[0pt]
Peut-on parler d'une morale collective ? \\[0pt]
Peut-on parler d'une religion de l'humanité ? \\[0pt]
Peut-on parler d'une santé de l'âme ? \\[0pt]
Peut-on parler d'une science de l'art ? \\[0pt]
Peut-on parler d'univers symbolique ? \\[0pt]
Peut-on parler d'un progrès dans l'histoire ? \\[0pt]
Peut-on parler d'un progrès de la liberté ? \\[0pt]
Peut-on parler d'un progrès moral ? \\[0pt]
Peut-on parler d'un règne de la technique ? \\[0pt]
Peut-on parler d'un savoir poétique ? \\[0pt]
Peut-on parler d'un sens moral ? \\[0pt]
Peut-on parler d'un style moral ? \\[0pt]
Peut-on parler d'un travail intellectuel ? \\[0pt]
Peut-on parler pour ne rien dire ? \\[0pt]
Peut-on parler sans incohérence d'une libre obéissance ? \\[0pt]
Peut-on partager ses goûts ? \\[0pt]
Peut-on partager une expérience ? \\[0pt]
Peut-on penser ce qu'on ne peut dire ? \\[0pt]
Peut-on penser contre l'expérience ? \\[0pt]
Peut-on penser illogiquement ? \\[0pt]
Peut-on penser la création ? \\[0pt]
Peut-on penser la douleur ? \\[0pt]
Peut-on penser la fin de toute chose ? \\[0pt]
Peut-on penser la justice comme une compétence ? \\[0pt]
Peut-on penser la matière ? \\[0pt]
Peut-on penser la mort ? \\[0pt]
Peut-on penser la nouveauté ? \\[0pt]
Peut-on penser l'art comme un langage ? \\[0pt]
Peut-on penser l'art sans référence au beau ? \\[0pt]
Peut-on penser l'avenir ? \\[0pt]
Peut-on penser la vie ? \\[0pt]
Peut-on penser la vie sans penser la mort ? \\[0pt]
Peut-on penser le changement ? \\[0pt]
Peut-on penser le divin ? \\[0pt]
Peut-on penser le monde sans la technique ? \\[0pt]
Peut-on penser le réel comme un tout ? \\[0pt]
Peut-on penser le social sans la politique ? \\[0pt]
Peut-on penser le temps ? \\[0pt]
Peut-on penser le temps sans l'espace ? \\[0pt]
Peut-on penser l'extériorité ? \\[0pt]
Peut-on penser l'histoire comme progrès ? \\[0pt]
Peut-on penser l'homme à partir de la nature ? \\[0pt]
Peut-on penser l'impossible ? \\[0pt]
Peut-on penser l'infini ? \\[0pt]
Peut-on penser l'irrationnel ? \\[0pt]
Peut-on penser l'œuvre d'art sans référence à l'idée de beauté ? \\[0pt]
Peut-on penser sans concept ? \\[0pt]
Peut-on penser sans concepts ? \\[0pt]
Peut-on penser sans image ? \\[0pt]
Peut-on penser sans images ? \\[0pt]
Peut-on penser sans les mots ? \\[0pt]
Peut-on penser sans les signes ? \\[0pt]
Peut-on penser sans méthode ? \\[0pt]
Peut-on penser sans ordre ? \\[0pt]
Peut-on penser sans préjugé ? \\[0pt]
Peut-on penser sans préjugés ? \\[0pt]
Peut-on penser sans règles ? \\[0pt]
Peut-on penser sans savoir ce que l'on pense ? \\[0pt]
Peut-on penser sans savoir que l'on pense ? \\[0pt]
Peut-on penser sans signes ? \\[0pt]
Peut-on penser sans son corps ? \\[0pt]
Peut-on penser un art sans œuvres ? \\[0pt]
Peut-on penser un droit international ? \\[0pt]
Peut-on penser une conscience sans mémoire ? \\[0pt]
Peut-on penser une conscience sans objet ? \\[0pt]
Peut-on penser une fin de l'histoire ? \\[0pt]
Peut-on penser une métaphysique sans Dieu ? \\[0pt]
Peut-on penser une religion sans le recours au divin ? \\[0pt]
Peut-on penser une société sans État ? \\[0pt]
Peut-on penser un État sans violence ? \\[0pt]
Peut-on penser une volonté diabolique ? \\[0pt]
Peut-on penser un pouvoir absolu ? \\[0pt]
Peut-on percevoir le temps ? \\[0pt]
Peut-on percevoir sans juger ? \\[0pt]
Peut-on percevoir sans s'en apercevoir ? \\[0pt]
Peut-on perdre la raison ? \\[0pt]
Peut-on perdre sa dignité ? \\[0pt]
Peut-on perdre sa liberté ? \\[0pt]
Peut-on perdre son identité ? \\[0pt]
Peut-on perdre son temps ? \\[0pt]
Peut-on permettre le mensonge ? \\[0pt]
Peut-on préconiser, dans les sciences humaines et sociales, l'imitation des sciences de la nature ? \\[0pt]
Peut-on prédire les événements ? \\[0pt]
Peut-on prédire l'histoire ? \\[0pt]
Peut-on préférer le bonheur à la vérité ? \\[0pt]
Peut-on préférer l'injustice au désordre ? \\[0pt]
Peut-on préférer l'ordre à la justice ? \\[0pt]
Peut-on prendre le risque de donner la mort en voulant soulager la souffrance ? \\[0pt]
Peut-on prendre les moyens pour la fin ? \\[0pt]
Peut-on prévoir l'avenir ? \\[0pt]
Peut-on prévoir le futur ? \\[0pt]
Peut-on prévoir les hommes ? \\[0pt]
Peut-on promettre le bonheur ? \\[0pt]
Peut-on protéger les libertés sans les réduire ? \\[0pt]
Peut-on prouver la réalité de l'esprit ? \\[0pt]
Peut-on prouver l'existence ? \\[0pt]
Peut-on prouver l'existence de Dieu ? \\[0pt]
Peut-on prouver l'existence de l'inconscient ? \\[0pt]
Peut-on prouver l'existence du monde ? \\[0pt]
Peut-on prouver une existence ? \\[0pt]
Peut-on raconter sa vie ? \\[0pt]
Peut-on raisonner sans règles ? \\[0pt]
Peut-on ralentir la course du temps ? \\[0pt]
Peut-on recommencer sa vie ? \\[0pt]
Peut-on reconnaître un sens à l'histoire sans lui assigner une fin ? \\[0pt]
Peut-on réduire la pensée à une espèce de comportement ? \\[0pt]
Peut-on réduire la vie à la matière ? \\[0pt]
Peut-on réduire le raisonnement au calcul ? \\[0pt]
Peut-on réduire l'esprit à la matière ? \\[0pt]
Peut-on réduire une métaphysique à une conception du monde ? \\[0pt]
Peut-on réduire un homme à la somme de ses actes ? \\[0pt]
Peut-on refaire le monde ? \\[0pt]
Peut-on refuser de voir la vérité ? \\[0pt]
Peut-on refuser la loi ? \\[0pt]
Peut-on refuser la vérité ? \\[0pt]
Peut-on refuser la violence ? \\[0pt]
Peut-on refuser le bonheur ? \\[0pt]
Peut-on refuser l'évidence ? \\[0pt]
Peut-on refuser le vrai ? \\[0pt]
Peut-on réfuter le scepticisme ? \\[0pt]
Peut-on régner innocemment ? \\[0pt]
Peut-on rejeter le principe de non-contradiction ? \\[0pt]
Peut-on rendre raison des émotions ? \\[0pt]
Peut-on rendre raison de tout ? \\[0pt]
Peut-on rendre raison du réel ? \\[0pt]
Peut-on renoncer à comprendre ? \\[0pt]
Peut-on renoncer à être libre ? \\[0pt]
Peut-on renoncer à la liberté ? \\[0pt]
Peut-on renoncer à la vérité ? \\[0pt]
Peut-on renoncer à sa liberté ? \\[0pt]
Peut-on renoncer à ses droits ? \\[0pt]
Peut-on renoncer à soi ? \\[0pt]
Peut-on renoncer au bonheur ? \\[0pt]
Peut-on réparer le vivant ? \\[0pt]
Peut-on réparer une injustice ? \\[0pt]
Peut-on répondre au sceptique ? \\[0pt]
Peut-on répondre d'autrui ? \\[0pt]
Peut-on représenter le peuple ? \\[0pt]
Peut-on représenter l'espace ? \\[0pt]
Peut-on représenter l'invisible ? \\[0pt]
Peut-on reprocher à la morale d'être abstraite ? \\[0pt]
Peut-on reprocher au langage d'être équivoque ? \\[0pt]
Peut-on reprocher au langage d'être parfait ? \\[0pt]
Peut-on résister à la vérité ? \\[0pt]
Peut-on résister au vrai ? \\[0pt]
Peut-on rester dans le doute ? \\[0pt]
Peut-on rester insensible à la beauté ? \\[0pt]
Peut-on rester sceptique ? \\[0pt]
Peut-on restreindre la logique à la pensée formelle ? \\[0pt]
Peut-on résumer une philosophie ? \\[0pt]
Peut-on retenir le temps ? \\[0pt]
Peut-on réunir des arts différents dans une même œuvre ? \\[0pt]
Peut-on revendiquer la paix comme un droit ? \\[0pt]
Peut-on revendiquer le droit de désobéir ? \\[0pt]
Peut-on revenir sur ses erreurs ? \\[0pt]
Peut-on rire de tout ? \\[0pt]
Peut-on rompre avec la société ? \\[0pt]
Peut-on rompre avec le passé ? \\[0pt]
Peut-on s'abstenir de penser politiquement ? \\[0pt]
Peut-on s'accorder sur des vérités morales ? \\[0pt]
Peut-on s'affranchir de la subjectivité ? \\[0pt]
Peut-on s'affranchir de sa propre nature ? \\[0pt]
Peut-on s'affranchir des lois ? \\[0pt]
Peut-on saisir le temps ? \\[0pt]
Peut-on s'attendre à tout ? \\[0pt]
Peut-on savoir ce que l'on fait ? \\[0pt]
Peut-on savoir ce qui est bien ? \\[0pt]
Peut-on savoir quelque chose de l'avenir ? \\[0pt]
Peut-on savoir sans croire ? \\[0pt]
Peut-on se choisir un destin ? \\[0pt]
Peut-on se connaître soi-même ? \\[0pt]
Peut-on se désintéresser de la politique ? \\[0pt]
Peut-on se désintéresser de son bonheur ? \\[0pt]
Peut-on se duper soi-même ? \\[0pt]
Peut-on se faire une idée de tout ? \\[0pt]
Peut-on se fier à la technique ? \\[0pt]
Peut-on se fier à l'expérience vécue ? \\[0pt]
Peut-on se fier à l'intuition ? \\[0pt]
Peut-on se fier à sa propre raison ? \\[0pt]
Peut-on se fier à son intuition ? \\[0pt]
Peut-on se fier aux apparences ? \\[0pt]
Peut-on se gouverner soi-même ? \\[0pt]
Peut-on se méfier de soi-même ? \\[0pt]
Peut-on se mentir à soi-même ? \\[0pt]
Peut-on se mettre à la place d'autrui ? \\[0pt]
Peut-on se mettre à la place de l'autre ? \\[0pt]
Peut-on se mettre à la place des autres ? \\[0pt]
Peut-on s'en tenir au présent ? \\[0pt]
Peut-on sentir le temps qui passe ? \\[0pt]
Peut-on séparer la politique de l'exigence de vérité ? \\[0pt]
Peut-on séparer l'homme et l'œuvre ? \\[0pt]
Peut-on séparer politique et économie ? \\[0pt]
Peut-on se passer de chef ? \\[0pt]
Peut-on se passer de croire ? \\[0pt]
Peut-on se passer de croyance ? \\[0pt]
Peut-on se passer de croyances ? \\[0pt]
Peut-on se passer de Dieu ? \\[0pt]
Peut-on se passer de frontières ? \\[0pt]
Peut-on se passer de la religion ? \\[0pt]
Peut-on se passer de la technique ? \\[0pt]
Peut-on se passer de l'État ? \\[0pt]
Peut-on se passer de l'idée de cause finale ? \\[0pt]
Peut-on se passer de l'idée de Dieu ? \\[0pt]
Peut-on se passer de l'idée de droit naturel ? \\[0pt]
Peut-on se passer de l'idée de finalité ? \\[0pt]
Peut-on se passer de l'idée de nature humaine ? \\[0pt]
Peut-on se passer de maître ? \\[0pt]
Peut-on se passer de métaphysique ? \\[0pt]
Peut-on se passer de méthode ? \\[0pt]
Peut-on se passer de mythes ? \\[0pt]
Peut-on se passer de principes ? \\[0pt]
Peut-on se passer de religion ? \\[0pt]
Peut-on se passer de représentants ? \\[0pt]
Peut-on se passer des causes finales ? \\[0pt]
Peut-on se passer de spiritualité ? \\[0pt]
Peut-on se passer des relations ? \\[0pt]
Peut-on se passer d'État ? \\[0pt]
Peut-on se passer de technique ? \\[0pt]
Peut-on se passer de techniques de raisonnement ? \\[0pt]
Peut-on se passer de toute religion ? \\[0pt]
Peut-on se passer d'idéal ? \\[0pt]
Peut-on se passer d'ontologie ? \\[0pt]
Peut-on se passer d'un maître ? \\[0pt]
Peut-on se peindre soi-même ? \\[0pt]
Peut-on se prescrire une loi ? \\[0pt]
Peut-on se promettre quelque chose à soi-même ? \\[0pt]
Peut-on se punir soi-même ? \\[0pt]
Peut-on se régler sur des exemples en politique ? \\[0pt]
Peut-on se rendre maître de la technique ? \\[0pt]
Peut-on se retirer du monde ? \\[0pt]
Peut-on servir deux maîtres à la fois ? \\[0pt]
Peut-on se soustraire à son devoir ? \\[0pt]
Peut-on se tromper en se croyant heureux ? \\[0pt]
Peut-on se tromper sur son devoir ? \\[0pt]
Peut-on se vouloir parfait ? \\[0pt]
Peut-on s'exprimer par le silence ? \\[0pt]
Peut-on s'obliger soi-même ? \\[0pt]
Peut-on sortir de la subjectivité ? \\[0pt]
Peut-on sortir de sa conscience ? \\[0pt]
Peut-on sortir de sa culture ? \\[0pt]
Peut-on s'oublier ? \\[0pt]
Peut-on souhaiter le gouvernement des meilleurs ? \\[0pt]
Peut-on souhaiter ne pas être libre ? \\[0pt]
Peut-on suivre une règle ? \\[0pt]
Peut-on suspendre le temps ? \\[0pt]
Peut-on suspendre son jugement ? \\[0pt]
Peut-on sympathiser avec l'ennemi ? \\[0pt]
Peut-on tirer des leçons de l'histoire ? \\[0pt]
Peut-on tolérer l'injustice ? \\[0pt]
Peut-on toujours faire ce qu'on doit ? \\[0pt]
Peut-on toujours raison garder ? \\[0pt]
Peut-on toujours savoir entièrement ce que l'on dit ? \\[0pt]
Peut-on tout analyser ? \\[0pt]
Peut-on tout attendre de l'État ? \\[0pt]
Peut-on tout définir ? \\[0pt]
Peut-on tout démontrer ? \\[0pt]
Peut-on tout désirer ? \\[0pt]
Peut-on tout dire ? \\[0pt]
Peut-on tout donner ? \\[0pt]
Peut-on tout échanger ? \\[0pt]
Peut-on tout enseigner ? \\[0pt]
Peut-on tout expliquer ? \\[0pt]
Peut-on tout exprimer ? \\[0pt]
Peut-on tout imaginer ? \\[0pt]
Peut-on tout imiter ? \\[0pt]
Peut-on tout interpréter ? \\[0pt]
Peut-on tout justifier ? \\[0pt]
Peut-on tout mathématiser ? \\[0pt]
Peut-on tout mesurer ? \\[0pt]
Peut-on tout ordonner ? \\[0pt]
Peut-on tout pardonner ? \\[0pt]
Peut-on tout partager ? \\[0pt]
Peut-on tout prévoir ? \\[0pt]
Peut-on tout prouver ? \\[0pt]
Peut-on tout soumettre à la discussion ? \\[0pt]
Peut-on tout tolérer ? \\[0pt]
Peut-on traiter autrui comme un moyen ? \\[0pt]
Peut-on traiter un être vivant comme une machine ? \\[0pt]
Peut-on trancher les conflits entre interprétations ? \\[0pt]
Peut-on transformer le réel ? \\[0pt]
Peut-on transiger avec les principes ? \\[0pt]
Peut-on trop penser ? \\[0pt]
Peut-on trouver du plaisir à l'ennui ? \\[0pt]
Peut-on vivre avec les autres ? \\[0pt]
Peut-on vivre dans le doute ? \\[0pt]
Peut-on vivre en dehors de l'histoire ? \\[0pt]
Peut-on vivre en marge de la société ? \\[0pt]
Peut-on vivre en paix avec son inconscient ? \\[0pt]
Peut-on vivre en sceptique ? \\[0pt]
Peut-on vivre hors du temps ? \\[0pt]
Peut-on vivre pour la vérité ? \\[0pt]
Peut-on vivre sans aimer ? \\[0pt]
Peut-on vivre sans art ? \\[0pt]
Peut-on vivre sans aucune certitude ? \\[0pt]
Peut-on vivre sans croyance ? \\[0pt]
Peut-on vivre sans croyances ? \\[0pt]
Peut-on vivre sans désir ? \\[0pt]
Peut-on vivre sans échange ? \\[0pt]
Peut-on vivre sans foi ni loi ? \\[0pt]
Peut-on vivre sans illusions ? \\[0pt]
Peut-on vivre sans l'art ? \\[0pt]
Peut-on vivre sans le plaisir de vivre ? \\[0pt]
Peut-on vivre sans lois ? \\[0pt]
Peut-on vivre sans opinions ? \\[0pt]
Peut-on vivre sans passion ? \\[0pt]
Peut-on vivre sans peur ? \\[0pt]
Peut-on vivre sans principes ? \\[0pt]
Peut-on vivre sans réfléchir ? \\[0pt]
Peut-on vivre sans ressentiment ? \\[0pt]
Peut-on vivre sans rien espérer ? \\[0pt]
Peut-on vivre sans sacré ? \\[0pt]
Peut-on vivre sans travailler ? \\[0pt]
Peut-on vivre sans valeurs ? \\[0pt]
Peut-on voir sans croire ? \\[0pt]
Peut-on vouloir ce qu'on ne désire pas ? \\[0pt]
Peut-on vouloir le bien sans le faire ? \\[0pt]
Peut-on vouloir le bonheur d'autrui ? \\[0pt]
Peut-on vouloir le mal ? \\[0pt]
Peut-on vouloir le mal pour le mal ? \\[0pt]
Peut-on vouloir le mal sachant que c'est le mal ? \\[0pt]
Peut-on vouloir l'impossible ? \\[0pt]
Peut-on vouloir ne pas être libre ? \\[0pt]
Peut-on vouloir sans désirer ? \\[0pt]
Peut-on vraiment créer ? \\[0pt]
Peut-on vraiment dire ce qu'on pense ? \\[0pt]
Peut-on vraiment parler de soi ? \\[0pt]
Peut-on vraiment tirer des leçons du passé ? \\[0pt]
Pourquoi la réalité peut-elle dépasser la fiction ? \\[0pt]
Pourquoi ne peut-on concevoir la science comme achevée ? \\[0pt]
Quand peut-on se passer de théories ? \\[0pt]
Quel être peut être un sujet de droits ? \\[0pt]
Quel genre de conscience peut-on accorder à l'animal ? \\[0pt]
Quelle peut être la force de nos idées ? \\[0pt]
Quelle place la raison peut-elle faire à la croyance ? \\[0pt]
Quelle réalité peut-on accorder au temps ? \\[0pt]
Quelle réalité peut-on attribuer au temps ? \\[0pt]
Quelle valeur peut-on accorder à l'expérience ? \\[0pt]
Quels enseignements peut-on tirer de l'histoire des sciences ? \\[0pt]
Quel usage peut-on faire des fictions ? \\[0pt]
Que ne peut-on pas expliquer ? \\[0pt]
Que peut expliquer l'histoire ? \\[0pt]
Que peut la force ? \\[0pt]
Que peut la musique ? \\[0pt]
Que peut la pensée ? \\[0pt]
Que peut la philosophie ? \\[0pt]
Que peut la politique ? \\[0pt]
Que peut la raison ? \\[0pt]
Que peut la raison contre une croyance ? \\[0pt]
Que peut l'art ? \\[0pt]
Que peut la science ? \\[0pt]
Que peut la théorie ? \\[0pt]
Que peut la volonté ? \\[0pt]
Que peut la volonté contre les passions ? \\[0pt]
Que peut le corps ? \\[0pt]
Que peut le droit ? \\[0pt]
Que peut le politique ? \\[0pt]
Que peut l'esprit ? \\[0pt]
Que peut l'esprit sur la matière ? \\[0pt]
Que peut l'État ? \\[0pt]
Que peut l'intelligence artificielle ? \\[0pt]
Que peut-on apprendre des émotions esthétiques ? \\[0pt]
Que peut-on attendre de l'État ? \\[0pt]
Que peut-on attendre du droit international ? \\[0pt]
Que peut-on attendre d'une philosophie de l'art ? \\[0pt]
Que peut-on calculer ? \\[0pt]
Que peut-on comprendre immédiatement ? \\[0pt]
Que peut-on comprendre qu'on ne puisse expliquer ? \\[0pt]
Que peut-on contre un préjugé ? \\[0pt]
Que peut-on cultiver ? \\[0pt]
Que peut-on démontrer ? \\[0pt]
Que peut-on dire de l'être ? \\[0pt]
Que peut-on échanger ? \\[0pt]
Que peut-on enseigner ? \\[0pt]
Que peut-on espérer ? \\[0pt]
Que peut-on exprimer ? \\[0pt]
Que peut-on interdire ? \\[0pt]
Que peut-on partager ? \\[0pt]
Que peut-on perdre ? \\[0pt]
Que peut-on savoir de l'inconscient ? \\[0pt]
Que peut-on savoir de soi ? \\[0pt]
Que peut-on savoir du réel ? \\[0pt]
Que peut-on savoir par expérience ? \\[0pt]
Que peut-on sur autrui ? \\[0pt]
Que peut-on voir ? \\[0pt]
Que peut prétendre imposer une religion ? \\[0pt]
Que peut signifier « avoir le sens de la réalité » ? \\[0pt]
Que peut signifier : « gérer son temps » ? \\[0pt]
Que peut-signifier « tuer le temps » ? \\[0pt]
Que peut un corps ? \\[0pt]
Qu'est-ce que peut un corps ? \\[0pt]
Qu'est-ce qui peut être hors du temps ? \\[0pt]
Qu'est-ce qui peut se transformer ? \\[0pt]
Qu'est-ce qu'on ne peut comprendre ? \\[0pt]
Qu'est-ce qu'on ne peut pas partager ? \\[0pt]
Qu'est-ce qu'une machine ne peut pas faire ? \\[0pt]
Qui peut avoir des droits ? \\[0pt]
Qui peut me dire « tu ne dois pas » ? \\[0pt]
Qui peut obliger ? \\[0pt]
Qui peut parler ? \\[0pt]
Qui peut prétendre énoncer des devoirs ? \\[0pt]
Qui peut prétendre imposer des bornes à la technique ? \\[0pt]
Qui peut se passer de religion ? \\[0pt]
Résister peut-il être un droit ? \\[0pt]
Sans référence à la beauté, peut-on rendre compte de l'art ? \\[0pt]
Savons-nous ce que peut un corps ? \\[0pt]
Sur quoi peut-on fonder la certitude d'avoir raison ? \\[0pt]
Tout peut-il avoir une valeur marchande ? \\[0pt]
Tout peut-il avoir valeur marchande ? \\[0pt]
Tout peut-il être dit ? \\[0pt]
Tout peut-il être expliqué historiquement ? \\[0pt]
Tout peut-il être objet d'échange ? \\[0pt]
Tout peut-il être objet de jugement esthétique ? \\[0pt]
Tout peut-il être objet de science ? \\[0pt]
Tout peut-il n'être qu'apparence ? \\[0pt]
Tout peut-il s'acheter ? \\[0pt]
Tout peut-il se démontrer ? \\[0pt]
Tout peut-il se quantifier ? \\[0pt]
Tout peut-il se vendre ? \\[0pt]
Tout peut-il s'expliquer ? \\[0pt]
Tout savoir peut-il se transmettre ? \\[0pt]
Travailler sans souffrir peut-il être un idéal ? \\[0pt]
Un acte peut-il être inhumain ? \\[0pt]
Un art peut-il être populaire ? \\[0pt]
Un art peut-il se passer de règles ? \\[0pt]
Un bien peut-il être commun ? \\[0pt]
Un bien peut-il sortir d'un mal ? \\[0pt]
Un choix peut-il être rationnel ? \\[0pt]
Un contrat peut-il être injuste ? \\[0pt]
Un contrat peut-il être social ? \\[0pt]
Un désir peut-il être coupable ? \\[0pt]
Un désir peut-il être inconscient ? \\[0pt]
Un devoir peut-il être absolu ? \\[0pt]
Une action peut-elle être désintéressée ? \\[0pt]
Une action peut-elle être machinale ? \\[0pt]
Une bonne cité peut-elle se passer du beau ? \\[0pt]
Une cause peut-elle être libre ? \\[0pt]
Une connaissance peut-elle ne pas être relative ? \\[0pt]
Une croyance peut-elle être libre ? \\[0pt]
Une croyance peut-elle être rationnelle ? \\[0pt]
Une culture peut-elle être porteuse de valeurs universelles ? \\[0pt]
Une décision politique peut-elle être juste ? \\[0pt]
Une destruction peut-elle être créatrice ? \\[0pt]
Une d'œuvre peut-elle être achevée ? \\[0pt]
Une durée peut-elle être éternelle ? \\[0pt]
Une expérience peut-elle être fictive ? \\[0pt]
Une explication peut-elle être réductrice ? \\[0pt]
Une fiction peut-elle être vraie ? \\[0pt]
Une guerre peut-elle être juste ? \\[0pt]
Une idée peut-elle être fausse ? \\[0pt]
Une idée peut-elle être générale ? \\[0pt]
Une imitation peut-elle être parfaite ? \\[0pt]
Une inégalité peut-elle être juste ? \\[0pt]
Une intention peut-elle être coupable ? \\[0pt]
Une interprétation peut-elle échapper à l'arbitraire ? \\[0pt]
Une interprétation peut-elle être définitive ? \\[0pt]
Une interprétation peut-elle être fausse ? \\[0pt]
Une interprétation peut-elle être objective ? \\[0pt]
Une interprétation peut-elle être vraie ? \\[0pt]
Une interprétation peut-elle prétendre à la vérité ? \\[0pt]
Une ligne de conduite peut-elle tenir lieu de morale ? \\[0pt]
Une loi peut-elle être injuste ? \\[0pt]
Une machine peut-elle avoir une mémoire ? \\[0pt]
Une machine peut-elle être naturelle ? \\[0pt]
Une machine peut-elle penser ? \\[0pt]
Une métaphysique peut-elle être sceptique ? \\[0pt]
Une morale peut-elle être dépassée ? \\[0pt]
Une morale peut-elle être permissive ? \\[0pt]
Une morale peut-elle être provisoire ? \\[0pt]
Une morale peut-elle prétendre à l'universalité ? \\[0pt]
Une morale peut-elle se passer de la notion de vertu ? \\[0pt]
Une œuvre d'art peut-elle être exemplaire ? \\[0pt]
Une œuvre d'art peut-elle être immorale ? \\[0pt]
Une œuvre d'art peut-elle être laide ? \\[0pt]
Une passion peut-elle résister au temps ? \\[0pt]
Une perception peut-elle être illusoire ? \\[0pt]
Une philosophie peut-elle être réactionnaire ? \\[0pt]
Une politique peut-elle se réclamer de la vie ? \\[0pt]
Une psychologie peut-elle être matérialiste ? \\[0pt]
Une religion peut-elle être fausse ? \\[0pt]
Une religion peut-elle être rationnelle ? \\[0pt]
Une religion peut-elle être universelle ? \\[0pt]
Une religion peut-elle prétendre à la vérité ? \\[0pt]
Une religion peut-elle se passer de pratiques ? \\[0pt]
Une science peut-elle se passer d'expérimentation ? \\[0pt]
Une sensation peut-elle être fausse ? \\[0pt]
Une société peut-elle être juste ? \\[0pt]
Une société peut-elle se passer de religion ? \\[0pt]
Un État peut-il être trop étendu ? \\[0pt]
Un État peut-il être « voyou » ? \\[0pt]
Une théorie de la culture peut-elle être scientifique ? \\[0pt]
Une théorie peut-elle être vérifiée ? \\[0pt]
Une théorie scientifique peut-elle devenir fausse ? \\[0pt]
Une théorie scientifique peut-elle être ramenée à des propositions empiriques élémentaires ? \\[0pt]
Une théorie scientifique peut-elle être vraie ? \\[0pt]
Un être vivant peut-il être comparé à une œuvre d'art ? \\[0pt]
Une vérité peut-elle être approximative ? \\[0pt]
Une vérité peut-elle être indicible ? \\[0pt]
Une vérité peut-elle être provisoire ? \\[0pt]
Une volonté peut-elle être générale ? \\[0pt]
Un jeu peut-il être sérieux ? \\[0pt]
Un jugement moral peut-il relever de la vérité ? \\[0pt]
Un langage peut-il être privé ? \\[0pt]
Un mensonge peut-il avoir une valeur morale ? \\[0pt]
Un mensonge peut-il être légitime ? \\[0pt]
Un objet technique peut-il être beau ? \\[0pt]
Un peuple peut-il être sans histoire ? \\[0pt]
Un peuple peut-il être souverain ? \\[0pt]
Un plaisir peut-il être désintéressé ? \\[0pt]
Un problème moral peut-il recevoir une solution certaine ? \\[0pt]
Un problème philosophique peut-il être périmé ? \\[0pt]
Un problème scientifique peut-il être insoluble ? \\[0pt]
Un savoir peut-il être inconscient ? \\[0pt]
Un savoir sur l'homme peut-il être séparé d'un pouvoir sur les hommes ? \\[0pt]
Un sceptique peut-il être logicien ? \\[0pt]
Un seul peut-il avoir raison contre tous ? \\[0pt]
Un souverain élu peut-il être illégitime ? \\[0pt]
Un tableau peut-il être une dénonciation ? \\[0pt]
User de violence peut-il être moral ? \\[0pt]
Vouloir ce que l'on peut \\[0pt]
Vouloir la paix sociale peut-il aller jusqu'à accepter l'injustice ? \\[0pt]
Y a-t-il des choses dont on ne peut parler ? \\[0pt]


\subsection{Mot unique}
\label{sec:org8469ceb}
\noindent
Abstraire \\[0pt]
Agir \\[0pt]
Analyser \\[0pt]
Apparaître \\[0pt]
Apprendre \\[0pt]
Après-coup \\[0pt]
Arbitrer \\[0pt]
Argumenter \\[0pt]
Attendre \\[0pt]
Au-delà \\[0pt]
Autrui \\[0pt]
Avoir \\[0pt]
Calculer \\[0pt]
Cartographier \\[0pt]
« Ceci » \\[0pt]
Ceci \\[0pt]
Changer \\[0pt]
Chanter \\[0pt]
Choisir \\[0pt]
Classer \\[0pt]
Collectionner \\[0pt]
Commander \\[0pt]
Commémorer \\[0pt]
Commencer \\[0pt]
Communiquer \\[0pt]
Compatir \\[0pt]
Comprendre \\[0pt]
Compter \\[0pt]
Conclure \\[0pt]
Confondre \\[0pt]
Conquérir \\[0pt]
Consentir \\[0pt]
Contempler \\[0pt]
Convaincre \\[0pt]
Correspondre \\[0pt]
Créer \\[0pt]
Critiquer \\[0pt]
Croire \\[0pt]
Danser \\[0pt]
Débattre \\[0pt]
Déchiffrer \\[0pt]
Décider \\[0pt]
Décorer \\[0pt]
Découvrir \\[0pt]
Décrire \\[0pt]
Définir \\[0pt]
Déjouer \\[0pt]
Délibérer \\[0pt]
Dématérialiser \\[0pt]
Démériter \\[0pt]
Démontrer \\[0pt]
Dénaturer \\[0pt]
Déplaire \\[0pt]
Déraisonner \\[0pt]
Désacraliser \\[0pt]
Désirer \\[0pt]
Désobéir \\[0pt]
Dessiner \\[0pt]
Déterminer \\[0pt]
Devenir \\[0pt]
Dialoguer \\[0pt]
Différer \\[0pt]
Distinguer \\[0pt]
Donner \\[0pt]
Douter \\[0pt]
Durer \\[0pt]
Échanger \\[0pt]
Éclairer \\[0pt]
Écouter \\[0pt]
Écrire \\[0pt]
Éduquer \\[0pt]
Élire \\[0pt]
Enquêter \\[0pt]
Enseigner \\[0pt]
Entendre \\[0pt]
Énumérer \\[0pt]
Éprouver \\[0pt]
Errer \\[0pt]
Espérer \\[0pt]
Essayer \\[0pt]
Estimer \\[0pt]
Étudier \\[0pt]
Évaluer \\[0pt]
Exister \\[0pt]
Expérimenter \\[0pt]
Expliquer \\[0pt]
Falsifier \\[0pt]
Fonder \\[0pt]
Gagner \\[0pt]
Gouverner \\[0pt]
Grandir \\[0pt]
Guérir \\[0pt]
Habiter \\[0pt]
Haïr \\[0pt]
Hériter \\[0pt]
Hésiter \\[0pt]
Ignorer \\[0pt]
Imaginer \\[0pt]
Imiter \\[0pt]
Improviser \\[0pt]
Interpréter \\[0pt]
Interroger \\[0pt]
Je \\[0pt]
Jouer \\[0pt]
Juger \\[0pt]
Justifier \\[0pt]
Lire \\[0pt]
Locke \\[0pt]
« Machinalement » \\[0pt]
Manger \\[0pt]
Manifester \\[0pt]
Méditer \\[0pt]
Mentir \\[0pt]
Mesurer \\[0pt]
Mourir \\[0pt]
Naître \\[0pt]
Naviguer \\[0pt]
Nommer \\[0pt]
Obéir \\[0pt]
Objectiver \\[0pt]
Observer \\[0pt]
Œuvrer \\[0pt]
Organiser \\[0pt]
Oublier \\[0pt]
Paraître \\[0pt]
Pardonner \\[0pt]
Parfaire \\[0pt]
Parier \\[0pt]
Partager \\[0pt]
Pâtir \\[0pt]
Peindre \\[0pt]
Percevoir \\[0pt]
Permettre \\[0pt]
Persister \\[0pt]
Persuader \\[0pt]
Plaider \\[0pt]
« Pourquoi » \\[0pt]
Pourquoi ? \\[0pt]
Prévoir \\[0pt]
Promettre \\[0pt]
Protester \\[0pt]
Prouver \\[0pt]
Prouvez-le ! \\[0pt]
Publier \\[0pt]
Punir \\[0pt]
Qu'aime-t-on ? \\[0pt]
Quantifier \\[0pt]
Qu'échange-t-on ? \\[0pt]
Qu'y a-t-il ? \\[0pt]
Raisonner \\[0pt]
Recevoir \\[0pt]
Réfuter \\[0pt]
Regarder \\[0pt]
Réparer \\[0pt]
Répondre \\[0pt]
Représenter \\[0pt]
Reproduire \\[0pt]
Réprouver \\[0pt]
Résister \\[0pt]
Ressembler \\[0pt]
Ressentir \\[0pt]
Révéler \\[0pt]
Rêver \\[0pt]
Rêvons-nous ? \\[0pt]
Rien \\[0pt]
Rire \\[0pt]
S'adapter \\[0pt]
S'aliéner \\[0pt]
S'amuser \\[0pt]
S'associer \\[0pt]
Sculpter \\[0pt]
Séduire \\[0pt]
S'émanciper \\[0pt]
S'engager \\[0pt]
S'ennuyer \\[0pt]
S'entendre \\[0pt]
Sentir \\[0pt]
Servir \\[0pt]
S'étonner \\[0pt]
Seul \\[0pt]
S'exercer \\[0pt]
S'exprimer \\[0pt]
S'indigner \\[0pt]
Socrate \\[0pt]
Soi \\[0pt]
Soigner \\[0pt]
S'opposer \\[0pt]
S'orienter \\[0pt]
Souffrir \\[0pt]
Soupçonner \\[0pt]
Subir \\[0pt]
Survivre \\[0pt]
Témoigner \\[0pt]
Terroriser \\[0pt]
Tolérer \\[0pt]
Toucher \\[0pt]
Traduire \\[0pt]
Trahir \\[0pt]
Transmettre \\[0pt]
Tricher \\[0pt]
Tromper \\[0pt]
Vérifier \\[0pt]
Vieillir \\[0pt]
Vivre \\[0pt]
Voir \\[0pt]
Voyager \\[0pt]


\subsection{Le X}
\label{sec:org38b15c0}
\noindent
La banalité \\[0pt]
L'abandon \\[0pt]
La barbarie \\[0pt]
La bassesse \\[0pt]
La béatitude \\[0pt]
La beauté \\[0pt]
La bestialité \\[0pt]
La bête \\[0pt]
La bêtise \\[0pt]
La bibliothèque \\[0pt]
La bienfaisance \\[0pt]
La bienséance \\[0pt]
La bienveillance \\[0pt]
La bioéthique \\[0pt]
La biographie \\[0pt]
L'abnégation \\[0pt]
L'abondance \\[0pt]
La bonté \\[0pt]
L'absence \\[0pt]
L'absolu \\[0pt]
L'abstention \\[0pt]
L'abstraction \\[0pt]
L'absurde \\[0pt]
La bureaucratie \\[0pt]
L'académisme \\[0pt]
La calomnie \\[0pt]
La caricature \\[0pt]
La casuistique \\[0pt]
La catastrophe \\[0pt]
La catharsis \\[0pt]
La causalité \\[0pt]
La cause \\[0pt]
L'accident \\[0pt]
L'accidentel \\[0pt]
L'accomplissement \\[0pt]
L'accord \\[0pt]
L'accusation \\[0pt]
La censure \\[0pt]
La certitude \\[0pt]
La chair \\[0pt]
La chance \\[0pt]
La charité \\[0pt]
La chose \\[0pt]
La chronologie \\[0pt]
La chute \\[0pt]
La circonspection \\[0pt]
La circonstance \\[0pt]
La citation \\[0pt]
La cité \\[0pt]
La citoyenneté \\[0pt]
La civilisation \\[0pt]
La civilité \\[0pt]
La clarté \\[0pt]
La classification \\[0pt]
La clémence \\[0pt]
La cohérence \\[0pt]
La colère \\[0pt]
La collection \\[0pt]
La comédie \\[0pt]
La communauté \\[0pt]
La communication \\[0pt]
La comparaison \\[0pt]
La compassion \\[0pt]
La compétence \\[0pt]
La composition \\[0pt]
La compréhension \\[0pt]
La concorde \\[0pt]
La concurrence \\[0pt]
La condition \\[0pt]
La confiance \\[0pt]
La confusion \\[0pt]
La conquête \\[0pt]
La conscience \\[0pt]
La conséquence \\[0pt]
La conservation \\[0pt]
La considération \\[0pt]
La consolation \\[0pt]
La consommation \\[0pt]
La constance \\[0pt]
La constitution \\[0pt]
La contemplation \\[0pt]
La contestation \\[0pt]
La contingence \\[0pt]
La continuité \\[0pt]
La contradiction \\[0pt]
La contrainte \\[0pt]
La controverse \\[0pt]
La convalescence \\[0pt]
La convention \\[0pt]
La conversation \\[0pt]
La conversion \\[0pt]
La conviction \\[0pt]
La coopération \\[0pt]
La copie \\[0pt]
La corruption \\[0pt]
La cosmogonie \\[0pt]
La couleur \\[0pt]
La courtoisie \\[0pt]
La coutume \\[0pt]
L'acquis \\[0pt]
La crainte \\[0pt]
La création \\[0pt]
La créativité \\[0pt]
La crédibilité \\[0pt]
La crédulité \\[0pt]
La criminalité \\[0pt]
La crise \\[0pt]
La critique \\[0pt]
La croissance \\[0pt]
La croyance \\[0pt]
La cruauté \\[0pt]
L'acte \\[0pt]
L'acteur \\[0pt]
L'action \\[0pt]
L'activité \\[0pt]
L'actualité \\[0pt]
L'actuel \\[0pt]
La cuisine \\[0pt]
La culpabilité \\[0pt]
La culture \\[0pt]
La curiosité \\[0pt]
La danse \\[0pt]
L'adaptation \\[0pt]
La décadence \\[0pt]
La décence \\[0pt]
La déception \\[0pt]
La décision \\[0pt]
La déduction \\[0pt]
La déficience \\[0pt]
La définition \\[0pt]
La délibération \\[0pt]
La démagogie \\[0pt]
La démence \\[0pt]
La démesure \\[0pt]
La démocratie \\[0pt]
La démonstration \\[0pt]
La déontologie \\[0pt]
La dépendance \\[0pt]
La dépense \\[0pt]
La déraison \\[0pt]
La dérision \\[0pt]
La descendance \\[0pt]
La description \\[0pt]
La désillusion \\[0pt]
La désinvolture \\[0pt]
La désobéissance \\[0pt]
La destruction \\[0pt]
La détermination \\[0pt]
La dette \\[0pt]
La déviance \\[0pt]
La dialectique \\[0pt]
La dictature \\[0pt]
La différence \\[0pt]
La difformité \\[0pt]
La dignité \\[0pt]
La digression \\[0pt]
La discipline \\[0pt]
La discorde \\[0pt]
La discrétion \\[0pt]
La discrimination \\[0pt]
La discursivité \\[0pt]
La discussion \\[0pt]
La disgrâce \\[0pt]
La disharmonie \\[0pt]
La disponibilité \\[0pt]
La disposition \\[0pt]
La dispute \\[0pt]
La dissidence \\[0pt]
La dissimulation \\[0pt]
La distance \\[0pt]
La distinction \\[0pt]
La distraction \\[0pt]
La diversion \\[0pt]
La diversité \\[0pt]
La division \\[0pt]
L'admiration \\[0pt]
La domestication \\[0pt]
La domination \\[0pt]
La douceur \\[0pt]
La douleur \\[0pt]
La droiture \\[0pt]
La dualité \\[0pt]
La duplicité \\[0pt]
La durée \\[0pt]
L'adversité \\[0pt]
La faiblesse \\[0pt]
La familiarité \\[0pt]
La famille \\[0pt]
La fatalité \\[0pt]
La fatigue \\[0pt]
La fausseté \\[0pt]
La faute \\[0pt]
La fermeté \\[0pt]
La fête \\[0pt]
L'affection \\[0pt]
L'affirmation \\[0pt]
La fiction \\[0pt]
La fidélité \\[0pt]
La fierté \\[0pt]
La fièvre \\[0pt]
La figuration \\[0pt]
La fin \\[0pt]
La finalité \\[0pt]
La finitude \\[0pt]
La foi \\[0pt]
La folie \\[0pt]
La fonction \\[0pt]
La force \\[0pt]
La formalisation \\[0pt]
La forme \\[0pt]
La fortune \\[0pt]
La foule \\[0pt]
La fragilité \\[0pt]
La franchise \\[0pt]
La fraternité \\[0pt]
La fraude \\[0pt]
La frivolité \\[0pt]
La frontière \\[0pt]
La futilité \\[0pt]
La généalogie \\[0pt]
La généralisation \\[0pt]
La généralité \\[0pt]
La générosité \\[0pt]
La genèse \\[0pt]
L'agent \\[0pt]
La gentillesse \\[0pt]
La géographie \\[0pt]
La géométrie \\[0pt]
La géopolitique \\[0pt]
La gloire \\[0pt]
La grâce \\[0pt]
La grammaire \\[0pt]
La grandeur \\[0pt]
La gratitude \\[0pt]
La gratuité \\[0pt]
La gravité \\[0pt]
L'agression \\[0pt]
L'agressivité \\[0pt]
La grève \\[0pt]
L'agriculture \\[0pt]
La grossièreté \\[0pt]
La guérison \\[0pt]
La guerre \\[0pt]
La haine \\[0pt]
La hiérarchie \\[0pt]
La honte \\[0pt]
La jalousie \\[0pt]
La jeunesse \\[0pt]
La joie \\[0pt]
La jouissance \\[0pt]
La jurisprudence \\[0pt]
La justice \\[0pt]
La justification \\[0pt]
La lâcheté \\[0pt]
La laïcité \\[0pt]
La laideur \\[0pt]
La lassitude \\[0pt]
L'aléatoire \\[0pt]
La lecture \\[0pt]
La légende \\[0pt]
La légèreté \\[0pt]
La légitimation \\[0pt]
La légitimité \\[0pt]
La liberté \\[0pt]
L'aliénation \\[0pt]
L'aliéné \\[0pt]
La limite \\[0pt]
L'allégorie \\[0pt]
La loi \\[0pt]
La loyauté \\[0pt]
L'altérité \\[0pt]
L'alternative \\[0pt]
L'altruisme \\[0pt]
La lumière \\[0pt]
La machine \\[0pt]
La magie \\[0pt]
La magnanimité \\[0pt]
La main \\[0pt]
La maîtrise \\[0pt]
La majesté \\[0pt]
La majorité \\[0pt]
La maladie \\[0pt]
La malchance \\[0pt]
La malveillance \\[0pt]
La manière \\[0pt]
La manifestation \\[0pt]
La marchandisation \\[0pt]
La marchandise \\[0pt]
La marge \\[0pt]
La marginalité \\[0pt]
La masse \\[0pt]
L'amateur \\[0pt]
L'amateurisme \\[0pt]
La matière \\[0pt]
La maturité \\[0pt]
L'ambiguïté \\[0pt]
L'ambition \\[0pt]
L'âme \\[0pt]
La méchanceté \\[0pt]
La méconnaissance \\[0pt]
La médiation \\[0pt]
La médiocrité \\[0pt]
La médisance \\[0pt]
La méditation \\[0pt]
La méfiance \\[0pt]
La mélancolie \\[0pt]
La mémoire \\[0pt]
La menace \\[0pt]
La mesure \\[0pt]
La métamorphose \\[0pt]
La métaphore \\[0pt]
La méthode \\[0pt]
L'ami \\[0pt]
La minorité \\[0pt]
La misanthropie \\[0pt]
La misère \\[0pt]
La miséricorde \\[0pt]
La misologie \\[0pt]
L'amitié \\[0pt]
La modalité \\[0pt]
La mode \\[0pt]
La modération \\[0pt]
La modernité \\[0pt]
La modestie \\[0pt]
La mondialisation \\[0pt]
La monnaie \\[0pt]
La monumentalité \\[0pt]
La morale \\[0pt]
La mort \\[0pt]
L'amour \\[0pt]
La multiplicité \\[0pt]
La multitude \\[0pt]
L'anachronisme \\[0pt]
La naissance \\[0pt]
La naïveté \\[0pt]
L'analogie \\[0pt]
L'analyse \\[0pt]
L'anarchie \\[0pt]
La nation \\[0pt]
La nature \\[0pt]
L'anéantissement \\[0pt]
L'anecdotique \\[0pt]
La nécessité \\[0pt]
La négation \\[0pt]
La négligence \\[0pt]
La négociation \\[0pt]
La neutralité \\[0pt]
La névrose \\[0pt]
L'angélisme \\[0pt]
L'angoisse \\[0pt]
L'animal \\[0pt]
L'animalité \\[0pt]
L'animisme \\[0pt]
La noblesse \\[0pt]
L'anomalie \\[0pt]
L'anomie \\[0pt]
L'anonymat \\[0pt]
L'anonyme \\[0pt]
L'anormal \\[0pt]
La normalité \\[0pt]
La normativité \\[0pt]
La norme \\[0pt]
La nostalgie \\[0pt]
La nouveauté \\[0pt]
L'antériorité \\[0pt]
L'anthropocentrisme \\[0pt]
L'anticipation \\[0pt]
L'antinomie \\[0pt]
La nuance \\[0pt]
La nudité \\[0pt]
La nuit \\[0pt]
La paix \\[0pt]
La parenté \\[0pt]
La paresse \\[0pt]
La parole \\[0pt]
La participation \\[0pt]
La parure \\[0pt]
La passion \\[0pt]
La passivité \\[0pt]
La paternité \\[0pt]
L'apathie \\[0pt]
La patience \\[0pt]
L'apatride \\[0pt]
La patrie \\[0pt]
La pauvreté \\[0pt]
La peine \\[0pt]
La pensée \\[0pt]
L'aperception \\[0pt]
La perception \\[0pt]
La perfectibilité \\[0pt]
La perfection \\[0pt]
La performance \\[0pt]
La permanence \\[0pt]
La persécution \\[0pt]
La persévérance \\[0pt]
La personnalité \\[0pt]
La personne \\[0pt]
La perspective \\[0pt]
La persuasion \\[0pt]
La pertinence \\[0pt]
La perversion \\[0pt]
La perversité \\[0pt]
La pesanteur \\[0pt]
La peur \\[0pt]
La philanthropie \\[0pt]
La pitié \\[0pt]
La plaisanterie \\[0pt]
La plénitude \\[0pt]
La pluralité \\[0pt]
La poésie \\[0pt]
La polémique \\[0pt]
La police \\[0pt]
La politesse \\[0pt]
La politique \\[0pt]
L'apolitisme \\[0pt]
La populace \\[0pt]
La population \\[0pt]
L'aporie \\[0pt]
La pornographie \\[0pt]
La position \\[0pt]
La possession \\[0pt]
La possibilité \\[0pt]
La postérité \\[0pt]
La potentialité \\[0pt]
L'apparence \\[0pt]
L'appel \\[0pt]
L'apprentissage \\[0pt]
L'appropriation \\[0pt]
L'approximation \\[0pt]
La pratique \\[0pt]
La précarité \\[0pt]
La précaution \\[0pt]
La précision \\[0pt]
La préhistoire \\[0pt]
La présence \\[0pt]
La présomption \\[0pt]
La preuve \\[0pt]
La prévision \\[0pt]
La prévoyance \\[0pt]
La prière \\[0pt]
La prison \\[0pt]
La privation \\[0pt]
La probabilité \\[0pt]
La probité \\[0pt]
La production \\[0pt]
La profondeur \\[0pt]
La promenade \\[0pt]
La promesse \\[0pt]
La propagande \\[0pt]
La proportion \\[0pt]
La proposition \\[0pt]
La propreté \\[0pt]
La propriété \\[0pt]
La protection \\[0pt]
La providence \\[0pt]
La prudence \\[0pt]
La publicité \\[0pt]
La pudeur \\[0pt]
La puissance \\[0pt]
La pulsion \\[0pt]
La punition \\[0pt]
La pureté \\[0pt]
La qualité \\[0pt]
La quantité \\[0pt]
La radicalité \\[0pt]
La raison \\[0pt]
La rareté \\[0pt]
La rationalité \\[0pt]
L'arbitraire \\[0pt]
L'archéologie \\[0pt]
L'archive \\[0pt]
La réaction \\[0pt]
La réalité \\[0pt]
La recherche \\[0pt]
La réciprocité \\[0pt]
La réconciliation \\[0pt]
La reconnaissance \\[0pt]
La rectitude \\[0pt]
La référence \\[0pt]
La réflexion \\[0pt]
La réforme \\[0pt]
La réfutation \\[0pt]
La règle \\[0pt]
La régression \\[0pt]
La régularité \\[0pt]
La régulation \\[0pt]
La relation \\[0pt]
La relativité \\[0pt]
La religion \\[0pt]
La réminiscence \\[0pt]
La renaissance \\[0pt]
La Renaissance \\[0pt]
La rencontre \\[0pt]
La réparation \\[0pt]
La répétition \\[0pt]
La représentation \\[0pt]
La reproduction \\[0pt]
La république \\[0pt]
La réputation \\[0pt]
La résignation \\[0pt]
La résilience \\[0pt]
La résistance \\[0pt]
La résolution \\[0pt]
La responsabilité \\[0pt]
La ressemblance \\[0pt]
La réussite \\[0pt]
La révélation \\[0pt]
La rêverie \\[0pt]
La révolte \\[0pt]
La révolution \\[0pt]
L'argent \\[0pt]
L'argumentation \\[0pt]
La rhétorique \\[0pt]
La richesse \\[0pt]
La rigueur \\[0pt]
L'aristocratie \\[0pt]
La rivalité \\[0pt]
L'art \\[0pt]
L'artifice \\[0pt]
L'artificiel \\[0pt]
L'artiste \\[0pt]
La ruine \\[0pt]
La rumeur \\[0pt]
La rupture \\[0pt]
La ruse \\[0pt]
La sagesse \\[0pt]
La sainteté \\[0pt]
La sanction \\[0pt]
La santé \\[0pt]
La satisfaction \\[0pt]
La scène \\[0pt]
L'ascèse \\[0pt]
L'ascétisme \\[0pt]
La sculpture \\[0pt]
La sécularisation \\[0pt]
La sécurité \\[0pt]
La séduction \\[0pt]
La ségrégation \\[0pt]
La sensation \\[0pt]
La sensibilité \\[0pt]
La séparation \\[0pt]
La sérénité \\[0pt]
La servitude \\[0pt]
La sévérité \\[0pt]
La sexualité \\[0pt]
La signification \\[0pt]
La simplicité \\[0pt]
La simulation \\[0pt]
La sincérité \\[0pt]
La singularité \\[0pt]
La situation \\[0pt]
La sobriété \\[0pt]
La sociabilité \\[0pt]
La socialisation \\[0pt]
La société \\[0pt]
La solidarité \\[0pt]
La solitude \\[0pt]
La sollicitude \\[0pt]
La souffrance \\[0pt]
La soumission \\[0pt]
La souveraineté \\[0pt]
La spéculation \\[0pt]
La spontanéité \\[0pt]
L'assentiment \\[0pt]
L'association \\[0pt]
La standardisation \\[0pt]
La structure \\[0pt]
La subjectivité \\[0pt]
La subsistance \\[0pt]
La substance \\[0pt]
La subtilité \\[0pt]
La superstition \\[0pt]
La sûreté \\[0pt]
La surprise \\[0pt]
La survie \\[0pt]
La sympathie \\[0pt]
La systématicité \\[0pt]
La tautologie \\[0pt]
La technique \\[0pt]
La technocratie \\[0pt]
La téléologie \\[0pt]
La télévision \\[0pt]
La témérité \\[0pt]
La tempérance \\[0pt]
La tendance \\[0pt]
La tentation \\[0pt]
La terre \\[0pt]
La terreur \\[0pt]
L'athéisme \\[0pt]
La théodicée \\[0pt]
La théogonie \\[0pt]
La théorie \\[0pt]
La tolérance \\[0pt]
L'atome \\[0pt]
La totalitarisme \\[0pt]
La totalité \\[0pt]
La trace \\[0pt]
La tradition \\[0pt]
La traduction \\[0pt]
La tragédie \\[0pt]
La trahison \\[0pt]
La tranquillité \\[0pt]
La transcendance \\[0pt]
La transe \\[0pt]
La transgression \\[0pt]
La transmission \\[0pt]
La transparence \\[0pt]
La tristesse \\[0pt]
L'attachement \\[0pt]
L'attente \\[0pt]
L'attention \\[0pt]
L'attraction \\[0pt]
La tyrannie \\[0pt]
L'audace \\[0pt]
L'autarcie \\[0pt]
L'authenticité \\[0pt]
L'autobiographie \\[0pt]
L'autocritique \\[0pt]
L'automate \\[0pt]
L'automatisation \\[0pt]
L'autonomie \\[0pt]
L'autoportrait \\[0pt]
L'autorité \\[0pt]
La valeur \\[0pt]
La validité \\[0pt]
La vanité \\[0pt]
L'avarice \\[0pt]
La variété \\[0pt]
La vénalité \\[0pt]
La vénération \\[0pt]
La vengeance \\[0pt]
L'avenir \\[0pt]
L'aventure \\[0pt]
La véracité \\[0pt]
La vérification \\[0pt]
La vérité \\[0pt]
La vertu \\[0pt]
L'aveu \\[0pt]
L'aveuglement \\[0pt]
La vie \\[0pt]
La vieillesse \\[0pt]
La vigilance \\[0pt]
La ville \\[0pt]
La violence \\[0pt]
La virtualité \\[0pt]
La virtuosité \\[0pt]
La vitesse \\[0pt]
La vocation \\[0pt]
La voix \\[0pt]
La volupté \\[0pt]
La vraisemblance \\[0pt]
La vulgarisation \\[0pt]
La vulgarité \\[0pt]
La vulnérabilité \\[0pt]
L'axiome \\[0pt]
Le barbare \\[0pt]
Le baroque \\[0pt]
Le bavardage \\[0pt]
Le besoin \\[0pt]
Le biologisme \\[0pt]
L'éblouissement \\[0pt]
Le bonheur \\[0pt]
Le bricolage \\[0pt]
Le bruit \\[0pt]
Le cadavre \\[0pt]
Le cadre \\[0pt]
Le calcul \\[0pt]
Le calendrier \\[0pt]
Le cannibalisme \\[0pt]
Le canon \\[0pt]
Le capitalisme \\[0pt]
Le caractère \\[0pt]
L'écart \\[0pt]
L'échange \\[0pt]
Le changement \\[0pt]
Le chant \\[0pt]
Le chaos \\[0pt]
Le charme \\[0pt]
Le châtiment \\[0pt]
L'échec \\[0pt]
Le chef \\[0pt]
Le chemin \\[0pt]
Le choix \\[0pt]
Le citoyen \\[0pt]
Le civisme \\[0pt]
Le classicisme \\[0pt]
L'éclat \\[0pt]
Le cliché \\[0pt]
Le code \\[0pt]
Le cœur \\[0pt]
Le combat \\[0pt]
Le comédien \\[0pt]
Le comique \\[0pt]
Le commencement \\[0pt]
Le commerce \\[0pt]
Le commun \\[0pt]
Le complexe \\[0pt]
Le comportement \\[0pt]
Le compromis \\[0pt]
Le concept \\[0pt]
Le concret \\[0pt]
Le conditionnel \\[0pt]
Le conflit \\[0pt]
Le conformisme \\[0pt]
L'économie \\[0pt]
Le conseil \\[0pt]
Le consensus \\[0pt]
Le consentement \\[0pt]
Le conservatisme \\[0pt]
Le contemporain \\[0pt]
Le contentement \\[0pt]
Le contingent \\[0pt]
Le continu \\[0pt]
Le contrat \\[0pt]
Le convenable \\[0pt]
Le cosmopolitisme \\[0pt]
Le coupable \\[0pt]
Le couple \\[0pt]
Le courage \\[0pt]
Le cri \\[0pt]
Le crime \\[0pt]
Le critère \\[0pt]
L'écriture \\[0pt]
Le cynique \\[0pt]
Le cynisme \\[0pt]
Le dandysme \\[0pt]
Le danger \\[0pt]
Le débat \\[0pt]
Le déchet \\[0pt]
Le déclassement \\[0pt]
Le défaut \\[0pt]
Le dégoût \\[0pt]
Le déguisement \\[0pt]
Le délire \\[0pt]
Le démoniaque \\[0pt]
Le dépaysement \\[0pt]
Le dérèglement \\[0pt]
Le désaccord \\[0pt]
Le désenchantement \\[0pt]
Le désespoir \\[0pt]
Le déshonneur \\[0pt]
Le design \\[0pt]
Le désintéressement \\[0pt]
Le désir \\[0pt]
Le désœuvrement \\[0pt]
Le désordre \\[0pt]
Le despotisme \\[0pt]
Le dessin \\[0pt]
Le destin \\[0pt]
Le désuet \\[0pt]
Le détachement \\[0pt]
Le détail \\[0pt]
Le déterminisme \\[0pt]
Le deuil \\[0pt]
Le devenir \\[0pt]
Le devoir \\[0pt]
Le dévouement \\[0pt]
Le diable \\[0pt]
Le dialogue \\[0pt]
Le dictionnaire \\[0pt]
Le dilemme \\[0pt]
Le discernement \\[0pt]
Le discontinu \\[0pt]
Le divers \\[0pt]
Le divertissement \\[0pt]
Le divin \\[0pt]
Le dogmatisme \\[0pt]
Le domestique \\[0pt]
Le don \\[0pt]
Le donné \\[0pt]
Le double \\[0pt]
Le doute \\[0pt]
Le drame \\[0pt]
Le droit \\[0pt]
Le dualisme \\[0pt]
L'éducation \\[0pt]
Le factice \\[0pt]
Le fait \\[0pt]
Le familier \\[0pt]
Le fanatisme \\[0pt]
Le fantasme \\[0pt]
Le fantastique \\[0pt]
Le fatalisme \\[0pt]
Le faux \\[0pt]
Le fédéralisme \\[0pt]
Le féminin \\[0pt]
Le féminisme \\[0pt]
Le fétichisme \\[0pt]
L'effectivité \\[0pt]
L'efficacité \\[0pt]
L'efficience \\[0pt]
L'effort \\[0pt]
Le finalisme \\[0pt]
Le fini \\[0pt]
Le flegme \\[0pt]
Le folklore \\[0pt]
Le fonctionnaire \\[0pt]
Le fond \\[0pt]
Le fondement \\[0pt]
Le formalisme \\[0pt]
Le fou \\[0pt]
Le fragment \\[0pt]
Le frivole \\[0pt]
L'égalité \\[0pt]
L'égarement \\[0pt]
Le génie \\[0pt]
Le genre \\[0pt]
Le geste \\[0pt]
L'égoïsme \\[0pt]
Le goût \\[0pt]
Le grandiose \\[0pt]
Le grotesque \\[0pt]
Le groupe \\[0pt]
Le handicap \\[0pt]
Le hasard \\[0pt]
Le haut \\[0pt]
Le héros \\[0pt]
Le jeu \\[0pt]
Le juge \\[0pt]
Le jugement \\[0pt]
Le laboratoire \\[0pt]
Le laid \\[0pt]
Le langage \\[0pt]
L'élection \\[0pt]
L'élégance \\[0pt]
Le législateur \\[0pt]
L'élément \\[0pt]
L'élémentaire \\[0pt]
Le libertin \\[0pt]
Le lieu \\[0pt]
Le livre \\[0pt]
Le logique \\[0pt]
Le loisir \\[0pt]
Le luxe \\[0pt]
Le lyrisme \\[0pt]
Le magistrat \\[0pt]
Le maître \\[0pt]
Le mal \\[0pt]
Le malentendu \\[0pt]
Le malheur \\[0pt]
L'émancipation \\[0pt]
Le maniérisme \\[0pt]
Le marché \\[0pt]
Le mariage \\[0pt]
Le masculin \\[0pt]
Le masque \\[0pt]
Le matérialisme \\[0pt]
Le matériel \\[0pt]
L'embarras \\[0pt]
Le mécanisme \\[0pt]
Le mécénat \\[0pt]
Le méchant \\[0pt]
Le meilleur \\[0pt]
Le mélange \\[0pt]
Le ménage \\[0pt]
Le mensonge \\[0pt]
Le mépris \\[0pt]
Le mérite \\[0pt]
Le merveilleux \\[0pt]
Le mesurable \\[0pt]
Le métier \\[0pt]
Le milieu \\[0pt]
Le miracle \\[0pt]
Le miroir \\[0pt]
Le misanthrope \\[0pt]
Le mode \\[0pt]
Le modèle \\[0pt]
Le moi \\[0pt]
Le monde \\[0pt]
Le monstre \\[0pt]
Le monstrueux \\[0pt]
Le moralisme \\[0pt]
Le moraliste \\[0pt]
L'émotion \\[0pt]
Le mouvement \\[0pt]
L'empathie \\[0pt]
L'empire \\[0pt]
L'empirisme \\[0pt]
Le multiculturalisme \\[0pt]
Le multiple \\[0pt]
Le musée \\[0pt]
Le Musée \\[0pt]
Le mystère \\[0pt]
Le mysticisme \\[0pt]
Le mythe \\[0pt]
Le naïf \\[0pt]
Le narcissisme \\[0pt]
Le naturalisme \\[0pt]
Le naturel \\[0pt]
L'encyclopédie \\[0pt]
L'Encyclopédie \\[0pt]
Le néant \\[0pt]
Le négatif \\[0pt]
Le néologisme \\[0pt]
L'énergie \\[0pt]
L'enfance \\[0pt]
L'enfant \\[0pt]
L'engagement \\[0pt]
L'engendrement \\[0pt]
L'énigme \\[0pt]
Le nihilisme \\[0pt]
L'ennemi \\[0pt]
L'ennui \\[0pt]
Le nomade \\[0pt]
Le nomadisme \\[0pt]
Le nombre \\[0pt]
Le nominalisme \\[0pt]
L'enquête \\[0pt]
L'enthousiasme \\[0pt]
L'entraide \\[0pt]
Le nu \\[0pt]
L'envie \\[0pt]
L'environnement \\[0pt]
Le pacifisme \\[0pt]
Le paradigme \\[0pt]
Le paradoxal \\[0pt]
Le paradoxe \\[0pt]
Le pardon \\[0pt]
Le parfait \\[0pt]
Le pari \\[0pt]
Le partage \\[0pt]
Le particulier \\[0pt]
Le passé \\[0pt]
Le paternalisme \\[0pt]
Le pathologique \\[0pt]
Le patriarcat \\[0pt]
Le patrimoine \\[0pt]
Le patriotisme \\[0pt]
Le paysage \\[0pt]
Le péché \\[0pt]
Le pédagogue \\[0pt]
Le pessimisme \\[0pt]
Le peuple \\[0pt]
Le phantasme \\[0pt]
L'éphémère \\[0pt]
Le phénomène \\[0pt]
Le philanthrope \\[0pt]
Le plagiat \\[0pt]
Le plaisir \\[0pt]
Le pluralisme \\[0pt]
Le poétique \\[0pt]
Le populaire \\[0pt]
Le populisme \\[0pt]
Le portrait \\[0pt]
Le possible \\[0pt]
Le pouvoir \\[0pt]
Le pragmatisme \\[0pt]
Le préférable \\[0pt]
Le préjugé \\[0pt]
Le premier \\[0pt]
Le préscientifique \\[0pt]
Le présent \\[0pt]
L'épreuve \\[0pt]
Le primitif \\[0pt]
Le prince \\[0pt]
Le principe \\[0pt]
Le prix \\[0pt]
Le probabilisme \\[0pt]
Le probable \\[0pt]
Le problème \\[0pt]
Le processus \\[0pt]
Le prochain \\[0pt]
Le profane \\[0pt]
Le profit \\[0pt]
Le progrès \\[0pt]
Le projet \\[0pt]
Le propre \\[0pt]
Le propriétaire \\[0pt]
Le protectionnisme \\[0pt]
Le provisoire \\[0pt]
Le psychologisme \\[0pt]
Le public \\[0pt]
Le quelconque \\[0pt]
L'équilibre \\[0pt]
L'équité \\[0pt]
L'équivalence \\[0pt]
L'équivocité \\[0pt]
L'équivoque \\[0pt]
Le quotidien \\[0pt]
Le racisme \\[0pt]
Le raffinement \\[0pt]
Le rationalisme \\[0pt]
Le rationnel \\[0pt]
Le réalisme \\[0pt]
Le récit \\[0pt]
Le reconnaissance \\[0pt]
Le réel \\[0pt]
Le refoulement \\[0pt]
Le refus \\[0pt]
Le regard \\[0pt]
Le regret \\[0pt]
Le relativisme \\[0pt]
Le remords \\[0pt]
Le renoncement \\[0pt]
Le repas \\[0pt]
Le repentir \\[0pt]
Le repos \\[0pt]
Le respect \\[0pt]
Le ressentiment \\[0pt]
Le rêve \\[0pt]
Le révisionnisme \\[0pt]
Le ridicule \\[0pt]
Le rien \\[0pt]
Le rigorisme \\[0pt]
Le rire \\[0pt]
Le risque \\[0pt]
Le rite \\[0pt]
Le rituel \\[0pt]
Le roman \\[0pt]
Le romantisme \\[0pt]
L'érotisme \\[0pt]
L'erreur \\[0pt]
L'érudition \\[0pt]
Le rythme \\[0pt]
Le sacré \\[0pt]
Le sacrifice \\[0pt]
Les affections \\[0pt]
Le salaire \\[0pt]
Le salut \\[0pt]
Les amis \\[0pt]
Les archives \\[0pt]
Les atomes \\[0pt]
Les autorités \\[0pt]
Les autres \\[0pt]
Le sauvage \\[0pt]
Les bêtes \\[0pt]
Les biotechnologies \\[0pt]
Le scandale \\[0pt]
Les caractères \\[0pt]
Les catastrophes \\[0pt]
Les catégories \\[0pt]
Les causes \\[0pt]
Le scepticisme \\[0pt]
Les cérémonies \\[0pt]
Les choses \\[0pt]
Les circonstances \\[0pt]
L'esclavage \\[0pt]
L'esclave \\[0pt]
Les clichés \\[0pt]
Les commencements \\[0pt]
Les conséquences \\[0pt]
Les convenances \\[0pt]
Les conventions \\[0pt]
Les couleurs \\[0pt]
Les coutumes \\[0pt]
Le scrupule \\[0pt]
Les dictionnaires \\[0pt]
Les dogmes \\[0pt]
Les échanges \\[0pt]
Les écrans \\[0pt]
Le secret \\[0pt]
Les éléments \\[0pt]
Les élites \\[0pt]
Le semblable \\[0pt]
Les enfants \\[0pt]
Le sensationnel \\[0pt]
Les ensembles \\[0pt]
Le sensible \\[0pt]
Le sentiment \\[0pt]
Les envieux \\[0pt]
Le sérieux \\[0pt]
Le serment \\[0pt]
Les événements \\[0pt]
Les excuses \\[0pt]
Le sexe \\[0pt]
Les experts \\[0pt]
Les faits \\[0pt]
Les fictions \\[0pt]
Les foules \\[0pt]
Les fous \\[0pt]
Les frontières \\[0pt]
Les générations \\[0pt]
Les héros \\[0pt]
Les idoles \\[0pt]
Le signe \\[0pt]
Le silence \\[0pt]
Le simple \\[0pt]
Le simulacre \\[0pt]
Les individus \\[0pt]
Le singulier \\[0pt]
Les interdits \\[0pt]
Les liens \\[0pt]
Les livres \\[0pt]
Les lois \\[0pt]
Les machines \\[0pt]
Les marginaux \\[0pt]
Les masses \\[0pt]
Les matériaux \\[0pt]
Les miracles \\[0pt]
Les modalités \\[0pt]
Les modèles \\[0pt]
Les mœurs \\[0pt]
Les monstres \\[0pt]
Les morts \\[0pt]
Les muses \\[0pt]
Les noms \\[0pt]
Les normes \\[0pt]
Le sociologisme \\[0pt]
Le soin \\[0pt]
Le soldat \\[0pt]
Le solipsisme \\[0pt]
Le sommeil \\[0pt]
L'ésotérisme \\[0pt]
Le souci \\[0pt]
Le soupçon \\[0pt]
Les outils \\[0pt]
Le souvenir \\[0pt]
L'espace \\[0pt]
Les pauvres \\[0pt]
Le spectacle \\[0pt]
Le spectateur \\[0pt]
L'espérance \\[0pt]
Les phénomènes \\[0pt]
Les plaisirs \\[0pt]
L'espoir \\[0pt]
Le sport \\[0pt]
Les possibles \\[0pt]
Les prêtres \\[0pt]
Les principes \\[0pt]
L'esprit \\[0pt]
Les proverbes \\[0pt]
L'esquisse \\[0pt]
Les relations \\[0pt]
Les remords \\[0pt]
Les reproductions \\[0pt]
Les rites \\[0pt]
Les rituels \\[0pt]
Les robots \\[0pt]
Les ruines \\[0pt]
Les sacrifices \\[0pt]
Les sauvages \\[0pt]
Les secrets \\[0pt]
L'essence \\[0pt]
L'essentiel \\[0pt]
Les sentiments \\[0pt]
Les signes \\[0pt]
Les styles \\[0pt]
Les substances \\[0pt]
Les systèmes \\[0pt]
Les tabous \\[0pt]
L'esthète \\[0pt]
L'esthétique \\[0pt]
L'esthétisme \\[0pt]
L'estime \\[0pt]
Les traditions \\[0pt]
Les transcendantaux \\[0pt]
Le style \\[0pt]
Le sublime \\[0pt]
Le substitut \\[0pt]
Le substrat \\[0pt]
Le succès \\[0pt]
Le suicide \\[0pt]
Le sujet \\[0pt]
Les universaux \\[0pt]
Le superflu \\[0pt]
Le surnaturel \\[0pt]
Les usages \\[0pt]
Les valeurs \\[0pt]
Les vertus \\[0pt]
Les vivants \\[0pt]
Le syllogisme \\[0pt]
Le symbole \\[0pt]
Le symbolique \\[0pt]
Le symbolisme \\[0pt]
Le système \\[0pt]
Le tableau \\[0pt]
Le tacite \\[0pt]
Le tact \\[0pt]
Le talent \\[0pt]
L'État \\[0pt]
Le témoignage \\[0pt]
Le témoin \\[0pt]
Le temps \\[0pt]
L'éternité \\[0pt]
Le terrain \\[0pt]
Le territoire \\[0pt]
Le théâtral \\[0pt]
L'ethnocentrisme \\[0pt]
L'étonnement \\[0pt]
Le totalitarisme \\[0pt]
Le totémisme \\[0pt]
Le toucher \\[0pt]
Le tragique \\[0pt]
L'étranger \\[0pt]
L'étrangeté \\[0pt]
Le transcendantal \\[0pt]
Le travail \\[0pt]
Le troc \\[0pt]
L'étude \\[0pt]
Le tyran \\[0pt]
L'eugénisme \\[0pt]
L'Europe \\[0pt]
L'euthanasie \\[0pt]
L'évaluation \\[0pt]
L'évasion \\[0pt]
Le vécu \\[0pt]
L'événement \\[0pt]
Le verbalisme \\[0pt]
Le verbe \\[0pt]
Le vertige \\[0pt]
Le vestige \\[0pt]
Le vêtement \\[0pt]
Le vice \\[0pt]
Le vide \\[0pt]
L'évidence \\[0pt]
Le virtuel \\[0pt]
Le visage \\[0pt]
Le vivant \\[0pt]
Le vol \\[0pt]
Le volontarisme \\[0pt]
L'évolution \\[0pt]
Le voyage \\[0pt]
Le vraisemblable \\[0pt]
Le vulgaire \\[0pt]
L'exactitude \\[0pt]
L'excellence \\[0pt]
L'exception \\[0pt]
L'excès \\[0pt]
L'exclusion \\[0pt]
L'excuse \\[0pt]
L'exemplaire \\[0pt]
L'exemplarité \\[0pt]
L'exemple \\[0pt]
L'exercice \\[0pt]
L'exil \\[0pt]
L'existence \\[0pt]
L'expérience \\[0pt]
L'expérimentation \\[0pt]
L'expertise \\[0pt]
L'explication \\[0pt]
L'exposition \\[0pt]
L'expression \\[0pt]
L'extériorité \\[0pt]
L'extraordinaire \\[0pt]
L'extrémisme \\[0pt]
L'extrémité \\[0pt]
L'habileté \\[0pt]
L'habitation \\[0pt]
L'habitude \\[0pt]
L'habitus \\[0pt]
L'harmonie \\[0pt]
L'hérédité \\[0pt]
L'hérésie \\[0pt]
L'héritage \\[0pt]
L'héroïsme \\[0pt]
L'hésitation \\[0pt]
L'hétéronomie \\[0pt]
L'histoire \\[0pt]
L'Histoire \\[0pt]
L'historicisme \\[0pt]
L'historicité \\[0pt]
L'historien \\[0pt]
L'homicide \\[0pt]
L'hominisation \\[0pt]
L'honnêteté \\[0pt]
L'honneur \\[0pt]
L'horizon \\[0pt]
L'horreur \\[0pt]
L'horrible \\[0pt]
L'hospitalité \\[0pt]
L'humain \\[0pt]
L'humanité \\[0pt]
L'humiliation \\[0pt]
L'humilité \\[0pt]
L'humour \\[0pt]
L'hypocrisie \\[0pt]
L'hypothèse \\[0pt]
L'idéal \\[0pt]
L'idéalisme \\[0pt]
L'idéaliste \\[0pt]
L'idéalité \\[0pt]
L'identification \\[0pt]
L'identité \\[0pt]
L'idéologie \\[0pt]
L'idiot \\[0pt]
L'idolâtrie \\[0pt]
L'idole \\[0pt]
L'ignoble \\[0pt]
L'ignorance \\[0pt]
L'illimité \\[0pt]
L'illusion \\[0pt]
L'illusoire \\[0pt]
L'illustration \\[0pt]
L'image \\[0pt]
L'imaginaire \\[0pt]
L'imagination \\[0pt]
L'imitation \\[0pt]
L'immanence \\[0pt]
L'immatériel \\[0pt]
L'immédiat \\[0pt]
L'immémorial \\[0pt]
L'immensité \\[0pt]
L'immoralisme \\[0pt]
L'immoraliste \\[0pt]
L'immoralité \\[0pt]
L'immortalité \\[0pt]
L'immuable \\[0pt]
L'immutabilité \\[0pt]
L'impardonnable \\[0pt]
L'imparfait \\[0pt]
L'impartialité \\[0pt]
L'impassibilité \\[0pt]
L'impatience \\[0pt]
L'impensable \\[0pt]
L'impératif \\[0pt]
L'imperceptible \\[0pt]
L'impersonnel \\[0pt]
L'impiété \\[0pt]
L'implicite \\[0pt]
L'impossible \\[0pt]
L'imposteur \\[0pt]
L'imprescriptible \\[0pt]
L'impression \\[0pt]
L'imprévisible \\[0pt]
L'imprévu \\[0pt]
L'improbable \\[0pt]
L'improvisation \\[0pt]
L'imprudence \\[0pt]
L'impuissance \\[0pt]
L'impunité \\[0pt]
L'inacceptable \\[0pt]
L'inaccessible \\[0pt]
L'inachevé \\[0pt]
L'inachèvement \\[0pt]
L'inaction \\[0pt]
L'inaliénable \\[0pt]
L'inaperçu \\[0pt]
L'inapparent \\[0pt]
L'inattendu \\[0pt]
L'incarnation \\[0pt]
L'incertain \\[0pt]
L'incertitude \\[0pt]
L'incommensurabilité \\[0pt]
L'incommensurable \\[0pt]
L'incompréhensible \\[0pt]
L'inconcevable \\[0pt]
L'inconnaissable \\[0pt]
L'inconnu \\[0pt]
L'inconscience \\[0pt]
L'inconscient \\[0pt]
L'inconséquence \\[0pt]
L'inconstance \\[0pt]
L'incorporel \\[0pt]
L'incrédulité \\[0pt]
L'incroyable \\[0pt]
L'inculture \\[0pt]
L'indécence \\[0pt]
L'indécidable \\[0pt]
L'indécision \\[0pt]
L'indéfini \\[0pt]
L'indémontrable \\[0pt]
L'indépassable \\[0pt]
L'indépendance \\[0pt]
L'indescriptible \\[0pt]
L'indésirable \\[0pt]
L'indétermination \\[0pt]
L'indéterminé \\[0pt]
L'indice \\[0pt]
L'indicible \\[0pt]
L'indifférence \\[0pt]
L'indignation \\[0pt]
L'indignité \\[0pt]
L'indiscernable \\[0pt]
L'indiscutable \\[0pt]
L'indistinct \\[0pt]
L'individu \\[0pt]
L'individualisme \\[0pt]
L'individualité \\[0pt]
L'individuation \\[0pt]
L'individuel \\[0pt]
L'indivisible \\[0pt]
L'indubitable \\[0pt]
L'induction \\[0pt]
L'indulgence \\[0pt]
L'ineffable \\[0pt]
L'inéluctable \\[0pt]
L'inerte \\[0pt]
L'inertie \\[0pt]
L'inespéré \\[0pt]
L'inesthétique \\[0pt]
L'inestimable \\[0pt]
L'inexistant \\[0pt]
L'inexpérience \\[0pt]
L'infâme \\[0pt]
L'infamie \\[0pt]
L'inférence \\[0pt]
L'infériorité \\[0pt]
L'infidélité \\[0pt]
L'infini \\[0pt]
L'influence \\[0pt]
L'information \\[0pt]
L'informe \\[0pt]
L'ingénieur \\[0pt]
L'ingéniosité \\[0pt]
L'ingénuité \\[0pt]
L'ingérence \\[0pt]
L'ingratitude \\[0pt]
L'inhibition \\[0pt]
L'inhumain \\[0pt]
L'inhumanité \\[0pt]
L'inimaginable \\[0pt]
L'inimitié \\[0pt]
L'inintelligible \\[0pt]
L'iniquité \\[0pt]
L'initiation \\[0pt]
L'injonction \\[0pt]
L'injure \\[0pt]
L'injustice \\[0pt]
L'injustifiable \\[0pt]
L'innocence \\[0pt]
L'innommable \\[0pt]
L'innovation \\[0pt]
L'inobservable \\[0pt]
L'inoubliable \\[0pt]
L'inquiétant \\[0pt]
L'inquiétude \\[0pt]
L'insatisfaction \\[0pt]
L'insensé \\[0pt]
L'insensibilité \\[0pt]
L'insignifiant \\[0pt]
L'insolite \\[0pt]
L'insoluble \\[0pt]
L'insouciance \\[0pt]
L'insoumission \\[0pt]
L'insoutenable \\[0pt]
L'inspiration \\[0pt]
L'instant \\[0pt]
L'instinct \\[0pt]
L'institution \\[0pt]
L'instruction \\[0pt]
L'instrument \\[0pt]
L'insulte \\[0pt]
L'insurrection \\[0pt]
L'intangible \\[0pt]
L'intégrité \\[0pt]
L'intellect \\[0pt]
L'intellectuel \\[0pt]
L'intelligence \\[0pt]
L'intelligible \\[0pt]
L'intempérance \\[0pt]
L'intemporel \\[0pt]
L'intention \\[0pt]
L'intentionnalité \\[0pt]
L'interaction \\[0pt]
L'interdépendance \\[0pt]
L'interdiction \\[0pt]
L'interdisciplinarité \\[0pt]
L'interdit \\[0pt]
L'intéressant \\[0pt]
L'intérêt \\[0pt]
L'intériorité \\[0pt]
L'interprétation \\[0pt]
L'intersectionnalité \\[0pt]
L'intersubjectivité \\[0pt]
L'intime \\[0pt]
L'intimité \\[0pt]
L'intolérable \\[0pt]
L'intolérance \\[0pt]
L'intraduisible \\[0pt]
L'intransigeance \\[0pt]
L'intransmissible \\[0pt]
L'introspection \\[0pt]
L'intuition \\[0pt]
L'inutile \\[0pt]
L'invention \\[0pt]
L'invérifiable \\[0pt]
L'invisibilité \\[0pt]
L'invisible \\[0pt]
L'invivable \\[0pt]
L'involontaire \\[0pt]
L'invraisemblable \\[0pt]
L'ironie \\[0pt]
L'irrationalité \\[0pt]
L'irrationnel \\[0pt]
L'irréductible \\[0pt]
L'irréel \\[0pt]
L'irréfléchi \\[0pt]
L'irréfutable \\[0pt]
L'irrégularité \\[0pt]
L'irréparable \\[0pt]
L'irreprésentable \\[0pt]
L'irrésolution \\[0pt]
L'irrespect \\[0pt]
L'irresponsabilité \\[0pt]
L'irréversibilité \\[0pt]
L'irréversible \\[0pt]
L'irrévocable \\[0pt]
L'ivresse \\[0pt]
L'obéissance \\[0pt]
L'objectivation \\[0pt]
L'objectivité \\[0pt]
L'objet \\[0pt]
L'obligation \\[0pt]
L'obscène \\[0pt]
L'obscénité \\[0pt]
L'obscur \\[0pt]
L'obscurantisme \\[0pt]
L'obscurité \\[0pt]
L'observation \\[0pt]
L'obsession \\[0pt]
L'obstacle \\[0pt]
L'occasion \\[0pt]
L'œuvre \\[0pt]
L'offense \\[0pt]
L'oisiveté \\[0pt]
L'oligarchie \\[0pt]
L'ombre \\[0pt]
L'omniscience \\[0pt]
L'opinion \\[0pt]
L'opportunisme \\[0pt]
L'opportunité \\[0pt]
L'opposant \\[0pt]
L'opposition \\[0pt]
L'optimisme \\[0pt]
L'ordinaire \\[0pt]
L'ordinateur \\[0pt]
L'ordre \\[0pt]
L'organique \\[0pt]
L'organisation \\[0pt]
L'organisme \\[0pt]
L'orgueil \\[0pt]
L'orientation \\[0pt]
L'originalité \\[0pt]
L'origine \\[0pt]
L'ornement \\[0pt]
L'orthodoxie \\[0pt]
L'oubli \\[0pt]
L'outil \\[0pt]
L'un \\[0pt]
L'Un \\[0pt]
L'unanimité \\[0pt]
L'uniformité \\[0pt]
L'unité \\[0pt]
L'univers \\[0pt]
L'universalisme \\[0pt]
L'universel \\[0pt]
L'urbanité \\[0pt]
L'urgence \\[0pt]
L'usage \\[0pt]
L'usure \\[0pt]
L'utile \\[0pt]
L'utilité \\[0pt]
L'utopie \\[0pt]


\subsection{X et Y}
\label{sec:orgcc9c49a}

\noindent
Abjection et sainteté \\[0pt]
Accuser et défendre \\[0pt]
Acteurs sociaux et usages sociaux \\[0pt]
Action et contemplation \\[0pt]
Action et événement \\[0pt]
Action et fabrication \\[0pt]
Action et production \\[0pt]
Actions et pratiques \\[0pt]
Activité et passivité \\[0pt]
Additionner et soustraire \\[0pt]
Affirmation et négation \\[0pt]
Affirmer et nier \\[0pt]
Agir et faire \\[0pt]
Agir et réagir \\[0pt]
« Aime, et fais ce que tu veux » \\[0pt]
Aimer et être aimé \\[0pt]
Aliénation et servitude \\[0pt]
Ami et ennemi \\[0pt]
Amitié et société \\[0pt]
Amour et amitié \\[0pt]
Amour et inconscient \\[0pt]
Amour et respect \\[0pt]
Analogie et ressemblance \\[0pt]
Analyse et intuition \\[0pt]
Analyse et synthèse \\[0pt]
Anomalie et anomie \\[0pt]
Anthropologie et humanisme \\[0pt]
Anthropologie et ontologie \\[0pt]
Anthropologie et politique \\[0pt]
Apparence et réalité \\[0pt]
Apparition et manifestation \\[0pt]
Apprendre et devenir \\[0pt]
Apprendre et enseigner \\[0pt]
Apprentissage et conditionnement \\[0pt]
\emph{A priori} et \emph{a posteriori} \\[0pt]
À qui appartient-il de décider du juste et de l'injuste ? \\[0pt]
À quoi bon les sciences humaines et sociales ? \\[0pt]
Architecture et religion \\[0pt]
Architecture et utopie \\[0pt]
Argent et liberté \\[0pt]
Argumenter et démontrer \\[0pt]
Art et abstraction \\[0pt]
Art et apparence \\[0pt]
Art et apparences \\[0pt]
Art et artifice \\[0pt]
Art et authenticité \\[0pt]
Art et beauté \\[0pt]
Art et connaissance \\[0pt]
Art et création \\[0pt]
Art et critique \\[0pt]
Art et décadence \\[0pt]
Art et divertissement \\[0pt]
Art et émotion \\[0pt]
Art et expression \\[0pt]
Art et finitude \\[0pt]
Art et folie \\[0pt]
Art et forme \\[0pt]
Art et illusion \\[0pt]
Art et image \\[0pt]
Art et imagination \\[0pt]
Art et industrie \\[0pt]
Art et interdit \\[0pt]
Art et jeu \\[0pt]
Art et langage \\[0pt]
Art et marchandise \\[0pt]
Art et matière \\[0pt]
Art et mélancolie \\[0pt]
Art et mémoire \\[0pt]
Art et métaphysique \\[0pt]
Art et morale \\[0pt]
Art et politique \\[0pt]
Art et pouvoir \\[0pt]
Art et présence \\[0pt]
Art et propagande \\[0pt]
Art et religieux \\[0pt]
Art et religion \\[0pt]
Art et représentation \\[0pt]
Art et société \\[0pt]
Art et Société \\[0pt]
Art et symbole \\[0pt]
Art et technique \\[0pt]
Art et tradition \\[0pt]
Art et transgression \\[0pt]
Art et utilité \\[0pt]
Art et vérité \\[0pt]
Artiste et artisan \\[0pt]
Artistes et ingénieurs \\[0pt]
Art populaire et art savant \\[0pt]
Arts de l'espace et arts du temps \\[0pt]
Arts du temps et arts de l'espace \\[0pt]
Aspect et apparence \\[0pt]
Attente et espérance \\[0pt]
Autorité et autoritarisme \\[0pt]
Autorité et domination \\[0pt]
Autorité et pouvoir \\[0pt]
Autorité et souveraineté \\[0pt]
Avoir et être \\[0pt]
Axiomatique et vérité \\[0pt]
Beauté et moralité \\[0pt]
Beauté et vérité \\[0pt]
Beauté naturelle et beauté artistique \\[0pt]
Besoin et désir \\[0pt]
Besoins et désirs \\[0pt]
Bêtise et méchanceté \\[0pt]
Bien commun et bien public \\[0pt]
Bien commun et droits individuels \\[0pt]
Bien commun et intérêt particulier \\[0pt]
Bonheur et autarcie \\[0pt]
Bonheur et bien-être \\[0pt]
Bonheur et plaisir \\[0pt]
Bonheur et satisfaction \\[0pt]
Bonheur et société \\[0pt]
Bonheur et technique \\[0pt]
Bonheur et vertu \\[0pt]
Bon sens et philosophie \\[0pt]
Bon sens et sens commun \\[0pt]
Calculer et penser \\[0pt]
Calculer et raisonner \\[0pt]
Castes et classes \\[0pt]
Catégories logiques et catégories linguistiques \\[0pt]
Causalité et finalité \\[0pt]
Cause et condition \\[0pt]
Cause et effet \\[0pt]
Cause et loi \\[0pt]
Cause et motivation \\[0pt]
Cause et raison \\[0pt]
Causes et lois \\[0pt]
Causes et motivations \\[0pt]
Causes et raisons \\[0pt]
Causes premières et causes secondes \\[0pt]
Ce qui fut et ce qui sera \\[0pt]
Ce qui passe et ce qui demeure \\[0pt]
Ce qui subsiste et ce qui change \\[0pt]
Certitude et conviction \\[0pt]
Certitude et pari \\[0pt]
Certitude et probabilité \\[0pt]
Certitude et vérité \\[0pt]
Chance et bonheur \\[0pt]
Changement et mouvement \\[0pt]
Chanter et parler \\[0pt]
Chaos et cosmos \\[0pt]
Choix et raison \\[0pt]
Chose et objet \\[0pt]
Chose et personne \\[0pt]
Choses et personnes \\[0pt]
Cinéma et réalité \\[0pt]
Cinéma et vérité \\[0pt]
Citoyen et soldat \\[0pt]
Citoyenneté et solidarité \\[0pt]
Civilisation et barbarie \\[0pt]
Classer et ordonner \\[0pt]
Classes et essences \\[0pt]
Classes et histoire \\[0pt]
Classicisme et romantisme \\[0pt]
Cohérence et vérité \\[0pt]
Colère et indignation \\[0pt]
Comment distinguer culture et civilisation ? \\[0pt]
Comment distinguer désirs et besoins ? \\[0pt]
Comment distinguer entre l'amour et l'amitié ? \\[0pt]
Communauté et conflit \\[0pt]
Communauté et société \\[0pt]
Communiquer et enseigner \\[0pt]
Compétence et autorité \\[0pt]
Comportement et conduite \\[0pt]
Composition et construction \\[0pt]
Compter et mesurer \\[0pt]
Concept et existence \\[0pt]
Concept et image \\[0pt]
Concept et intuition \\[0pt]
Concept et métaphore \\[0pt]
Conception et perception \\[0pt]
Concevoir et expérimenter \\[0pt]
Concevoir et juger \\[0pt]
Concevoir et percevoir \\[0pt]
Concurrence et égalité \\[0pt]
Conduite et comportement \\[0pt]
Confiance et crédulité \\[0pt]
Conflit et démocratie \\[0pt]
Conflit et liberté \\[0pt]
Connaissance commune et connaissance scientifique \\[0pt]
Connaissance de soi et conscience de soi \\[0pt]
Connaissance du futur et connaissance du passé \\[0pt]
Connaissance et croyance \\[0pt]
Connaissance et expérience \\[0pt]
Connaissance et liberté \\[0pt]
Connaissance et perception \\[0pt]
Connaissance et progrès \\[0pt]
Connaissance historique et action politique \\[0pt]
Connaître et agir \\[0pt]
Connaître et comprendre \\[0pt]
Connaître et penser \\[0pt]
Conscience de soi et amour de soi \\[0pt]
Conscience de soi et connaissance de soi \\[0pt]
Conscience et attention \\[0pt]
Conscience et connaissance \\[0pt]
Conscience et conscience de soi \\[0pt]
Conscience et existence \\[0pt]
Conscience et liberté \\[0pt]
Conscience et mauvaise conscience \\[0pt]
Conscience et mémoire \\[0pt]
Conscience et responsabilité \\[0pt]
Conscience et subjectivité \\[0pt]
Conscience et volonté \\[0pt]
Consensus et conflit \\[0pt]
Conservatisme et tradition \\[0pt]
Consistance et précarité \\[0pt]
Constitution et lois \\[0pt]
Consumérisme et démocratie \\[0pt]
Contemplation et action \\[0pt]
Contemplation et création \\[0pt]
Contemplation et distraction \\[0pt]
Contingence et nécessité \\[0pt]
Contingence et rationalité \\[0pt]
Continuité et discontinuité \\[0pt]
Contradiction et opposition \\[0pt]
Contrainte et désobéissance \\[0pt]
Contrainte et obligation \\[0pt]
Contrôle et vigilance \\[0pt]
Convaincre et persuader \\[0pt]
Convention et observation \\[0pt]
Conventions sociales et moralité \\[0pt]
Conviction et certitude \\[0pt]
Conviction et responsabilité \\[0pt]
Convient-il d'opposer explication et interprétation ? \\[0pt]
Convient-il d'opposer liberté et nécessité ? \\[0pt]
Corps et espace \\[0pt]
Corps et esprit \\[0pt]
Corps et identité \\[0pt]
Corps et matière \\[0pt]
Corps et nature \\[0pt]
Couleur et forme \\[0pt]
Crainte et espoir \\[0pt]
Création et critique \\[0pt]
Création et fabrication \\[0pt]
Création et production \\[0pt]
Création et réception \\[0pt]
Créativité et contrainte \\[0pt]
Créer et produire \\[0pt]
Crime et châtiment \\[0pt]
Crimes et châtiments \\[0pt]
Crise et création \\[0pt]
Crise et critique \\[0pt]
Crise et progrès \\[0pt]
Crise et révolution \\[0pt]
Croire et savoir \\[0pt]
Croyance et certitude \\[0pt]
Croyance et choix \\[0pt]
Croyance et confiance \\[0pt]
Croyance et connaissance \\[0pt]
Croyance et passivité \\[0pt]
Croyance et probabilité \\[0pt]
Croyance et société \\[0pt]
Croyance et vérité \\[0pt]
Culpabilité et responsabilité \\[0pt]
Cultes et rituels \\[0pt]
Culture et artifice \\[0pt]
Culture et civilisation \\[0pt]
Culture et communauté \\[0pt]
Culture et conscience \\[0pt]
Culture et différence \\[0pt]
Culture et éducation \\[0pt]
Culture et identité \\[0pt]
Culture et langage \\[0pt]
Culture et moralité \\[0pt]
Culture et savoir \\[0pt]
Culture et sociétés \\[0pt]
Culture et technique \\[0pt]
Culture et violence \\[0pt]
Dans les sciences humaines et sociales, peut-on préconiser l'imitation des sciences de la nature ? \\[0pt]
Débat et démocratie \\[0pt]
Débattre et dialoguer \\[0pt]
Découverte et invention \\[0pt]
Découverte et invention dans les sciences \\[0pt]
Découverte et justification \\[0pt]
Découvrir et inventer \\[0pt]
Décrire et expliquer \\[0pt]
Déduction et démonstration \\[0pt]
Déduction et expérience \\[0pt]
Définir et démontrer : à quoi bon ? \\[0pt]
Définition et démonstration \\[0pt]
Définition nominale et définition réelle \\[0pt]
Démocrates et démagogues \\[0pt]
Démocratie ancienne et démocratie moderne \\[0pt]
Démocratie et anarchie \\[0pt]
Démocratie et consensus \\[0pt]
Démocratie et démagogie \\[0pt]
Démocratie et impérialisme \\[0pt]
Démocratie et opinion \\[0pt]
Démocratie et religion \\[0pt]
Démocratie et représentation \\[0pt]
Démocratie et république \\[0pt]
Démocratie et technocratie \\[0pt]
Démocratie et transparence \\[0pt]
Démocratie et vérité \\[0pt]
Démonstration et argumentation \\[0pt]
Démonstration et déduction \\[0pt]
Démonstration et intuition \\[0pt]
Démontrer et argumenter \\[0pt]
Démontrer et expérimenter \\[0pt]
Description et explication \\[0pt]
Des goûts et des couleurs \\[0pt]
« Des goûts et des couleurs, on ne dispute pas » \\[0pt]
Des hommes et des dieux \\[0pt]
Désintérêt et désintéressement \\[0pt]
Désirer et vouloir \\[0pt]
Désir et besoin \\[0pt]
Désir et bonheur \\[0pt]
Désir et interdit \\[0pt]
Désir et langage \\[0pt]
Désir et manque \\[0pt]
Désir et ordre \\[0pt]
Désir et plaisir \\[0pt]
Désir et politique \\[0pt]
Désir et pouvoir \\[0pt]
Désir et raison \\[0pt]
Désir et réalité \\[0pt]
Désir et société \\[0pt]
Désir et temps \\[0pt]
Désir et volonté \\[0pt]
Désobéissance et résistance \\[0pt]
Désordre et chaos \\[0pt]
Des statistiques et des hommes \\[0pt]
Des structures et des hommes \\[0pt]
Déterminisme et responsabilité \\[0pt]
Déterminisme naturel et déterminisme social \\[0pt]
Déterminisme psychique et déterminisme physique \\[0pt]
Détruire et construire \\[0pt]
Deux et deux font quatre \\[0pt]
Devenir et évolution \\[0pt]
Devoir et bonheur \\[0pt]
Devoir et conformisme \\[0pt]
Devoir et contrainte \\[0pt]
Devoir et intérêt \\[0pt]
Devoir et liberté \\[0pt]
Devoir et plaisir \\[0pt]
Devoir et prudence \\[0pt]
Devoir et utilité \\[0pt]
Devoir et vertu \\[0pt]
Devoirs envers les autres et devoirs envers soi-même \\[0pt]
Devoirs et passions \\[0pt]
Diachronie et synchronie \\[0pt]
Dialectique et Philosophie \\[0pt]
Dialectique et vérité \\[0pt]
Dialogue et délibération en démocratie \\[0pt]
Dialoguer et s'entendre \\[0pt]
Dieu des philosophes et Dieu des croyants \\[0pt]
Dieu est-il une hypothèse dont le savant et le philosophe peuvent se passer ? \\[0pt]
Dieu et César \\[0pt]
Dieu et l'âme \\[0pt]
Dieu et le monde \\[0pt]
Dire et exprimer \\[0pt]
Dire et faire \\[0pt]
Dire et montrer \\[0pt]
Discerner et juger \\[0pt]
Discret et continu \\[0pt]
Discrimination et revendication \\[0pt]
Discussion et conversation \\[0pt]
Discussion et dialogue \\[0pt]
Diversité des sciences humaines et unité de l'homme \\[0pt]
Division du travail et cohésion sociale \\[0pt]
Documents et monuments \\[0pt]
Dogme et opinion \\[0pt]
Doit-on distinguer devoir moral et obligation sociale ? \\[0pt]
Don et échange \\[0pt]
Donner et recevoir \\[0pt]
Doute et raison \\[0pt]
Dressage et éducation \\[0pt]
Droit et coutume \\[0pt]
Droit et démocratie \\[0pt]
Droit et devoir \\[0pt]
Droit et devoir sont-ils liés ? \\[0pt]
Droit et interprétation \\[0pt]
Droit et morale \\[0pt]
Droit et moralité \\[0pt]
Droit et nature \\[0pt]
Droit et protection \\[0pt]
Droit et violence \\[0pt]
Droit naturel et loi naturelle \\[0pt]
« Droits de\ldots{} » et « droits à\ldots{} » \\[0pt]
Droits de l'homme et droits du citoyen \\[0pt]
Droits et devoirs \\[0pt]
Droits et devoirs sont-ils réciproques ? \\[0pt]
Droits et obligations \\[0pt]
Durée et instant \\[0pt]
Échange et don \\[0pt]
Échange et partage \\[0pt]
Échange et valeur \\[0pt]
Économie et politique \\[0pt]
Économie et prédiction \\[0pt]
Économie et société \\[0pt]
Économie politique et politique économique \\[0pt]
Écouter et entendre \\[0pt]
Écrire et parler \\[0pt]
Éducation et citoyenneté \\[0pt]
Éducation et dressage \\[0pt]
Éducation et instruction \\[0pt]
Éduquer et instruire \\[0pt]
Efficacité et justice \\[0pt]
Égalité et différence \\[0pt]
Égalité et identité \\[0pt]
Égalité et liberté \\[0pt]
Égalité et solidarité \\[0pt]
Égoïsme et altruisme \\[0pt]
Égoïsme et individualisme \\[0pt]
Égoïsme et méchanceté \\[0pt]
Élite et démocratie \\[0pt]
Empathie et sympathie \\[0pt]
Empirique et expérimental \\[0pt]
Empirique et transcendantal \\[0pt]
Enfance et moralité \\[0pt]
Enseigner et éduquer \\[0pt]
Entendement et raison \\[0pt]
Entité et identité \\[0pt]
Entre croire et savoir, y a-t-il une différence de nature ? \\[0pt]
Entre l'art et la nature, qui imite l'autre ? \\[0pt]
Entre la sécurité de tous et la liberté de chacun, le politique doit-il choisir ? \\[0pt]
Entre le vrai et le faux y a-t-il une place pour le probable ? \\[0pt]
Entre l'opinion et la science, n'y a-t-il qu'une différence de degré ? \\[0pt]
Entre l'utile et l'honnête, peut-on choisir ? \\[0pt]
Épistémologie et psychologie \\[0pt]
Épistémologie générale et épistémologie des sciences particulières \\[0pt]
Erreur et faute \\[0pt]
Erreur et illusion \\[0pt]
Espace et réalité \\[0pt]
Espace et représentation \\[0pt]
Espace et structure sociale \\[0pt]
Espace mathématique et espace physique \\[0pt]
Espace public et vie privée \\[0pt]
Espèce et genre \\[0pt]
Esprit et intériorité \\[0pt]
Essence et existence \\[0pt]
Essence et nature \\[0pt]
Esthétique et éthique \\[0pt]
Esthétique et poétique \\[0pt]
Esthétique et politique \\[0pt]
Esthétisme et moralité \\[0pt]
Est-il légitime d'opposer liberté et nécessité ? \\[0pt]
Est-il pertinent d'opposer les actes et les paroles ? \\[0pt]
Est-il pertinent d'opposer l'individuel et le collectif ? \\[0pt]
Estime et respect \\[0pt]
Est-on fondé à distinguer la justice et le droit ? \\[0pt]
État et gouvernement \\[0pt]
État et institutions \\[0pt]
État et nation \\[0pt]
État et société \\[0pt]
État et Société \\[0pt]
État et société civile \\[0pt]
État et vie privée \\[0pt]
État et violence \\[0pt]
Éternité et immortalité \\[0pt]
Éternité et perpétuité \\[0pt]
Éthique et authenticité \\[0pt]
Éthique et esthétique \\[0pt]
Éthique et Morale \\[0pt]
Ethnologie et cinéma \\[0pt]
Ethnologie et ethnocentrisme \\[0pt]
Ethnologie et sociologie \\[0pt]
Étonnement et sidération \\[0pt]
Être et apparaître \\[0pt]
Être et avoir \\[0pt]
Être et avoir été \\[0pt]
Être et devenir \\[0pt]
Être et devoir être \\[0pt]
Être et devoir-être \\[0pt]
Être et être pensé \\[0pt]
Être et être perçu \\[0pt]
Être et exister \\[0pt]
Être et ne plus être \\[0pt]
Être et paraître \\[0pt]
Être et penser, est-ce la même chose ? \\[0pt]
Être et représentation \\[0pt]
Être et sens \\[0pt]
Être juge et partie \\[0pt]
Être, vie et pensée \\[0pt]
Événement et structure \\[0pt]
Évidence et certitude \\[0pt]
Évidence et raison \\[0pt]
Évidence et vérité \\[0pt]
Évidences et préjugés \\[0pt]
Évolution biologique et culture \\[0pt]
Évolution et progrès \\[0pt]
Évolution et révolution \\[0pt]
Exactitude et objectivité \\[0pt]
Excuser et pardonner \\[0pt]
Existence et contingence \\[0pt]
Existence et essence \\[0pt]
Existence et religion \\[0pt]
Existence et transcendance \\[0pt]
Expérience esthétique et sens commun \\[0pt]
Expérience et approximation \\[0pt]
Expérience et expérimentation \\[0pt]
Expérience et habitude \\[0pt]
Expérience et interprétation \\[0pt]
Expérience et jugement \\[0pt]
Expérience et phénomène \\[0pt]
Expérience et vérité \\[0pt]
Expérience immédiate et expérimentation scientifique \\[0pt]
Expérimentation et vérification \\[0pt]
Explication et compréhension \\[0pt]
Explication et prévision \\[0pt]
Expliquer et comprendre \\[0pt]
Expliquer et interpréter \\[0pt]
Expliquer et justifier \\[0pt]
Expression et création \\[0pt]
Expression et signification \\[0pt]
Extension et compréhension \\[0pt]
Fabriquer et créer \\[0pt]
Faire et agir \\[0pt]
Faire et laisser faire \\[0pt]
Fait et essence \\[0pt]
Fait et fiction \\[0pt]
Fait et preuve \\[0pt]
Fait et théorie \\[0pt]
Fait et valeur \\[0pt]
Faits et preuves \\[0pt]
Faits et valeurs \\[0pt]
Famille et société \\[0pt]
Famille et tribu \\[0pt]
Faut-il choisir entre être heureux et être libre ? \\[0pt]
Faut-il distinguer culture et civilisation ? \\[0pt]
Faut-il distinguer désir et besoin ? \\[0pt]
Faut-il distinguer devoir moral et obligation sociale ? \\[0pt]
Faut-il distinguer esthétique et philosophie de l'art ? \\[0pt]
Faut-il opposer culture et nature ? \\[0pt]
Faut-il opposer droits et devoirs ? \\[0pt]
Faut-il opposer éduquer et instruire ? \\[0pt]
Faut-il opposer histoire et mémoire ? \\[0pt]
Faut-il opposer la foi et la raison ? \\[0pt]
Faut-il opposer la matière et l'esprit ? \\[0pt]
Faut-il opposer la société et l'État ? \\[0pt]
Faut-il opposer la théorie et la pratique ? \\[0pt]
Faut-il opposer le don et l'échange ? \\[0pt]
Faut-il opposer le réel et l'imaginaire ? \\[0pt]
Faut-il opposer l'esprit et la matière ? \\[0pt]
Faut-il opposer l'État et la société ? \\[0pt]
Faut-il opposer le temps vécu et le temps des choses ? \\[0pt]
Faut-il opposer l'histoire et la fiction ? \\[0pt]
Faut-il opposer nature et culture ? \\[0pt]
Faut-il opposer produire et créer ? \\[0pt]
Faut-il opposer raison et sensation ? \\[0pt]
Faut-il opposer rhétorique et philosophie ? \\[0pt]
Faut-il opposer science et croyance ? \\[0pt]
Faut-il opposer science et métaphysique ? \\[0pt]
Faut-il opposer sciences humaines et sciences de la nature ? \\[0pt]
Faut-il opposer sens et vérité ? \\[0pt]
Faut-il opposer subjectivité et objectivité ? \\[0pt]
Faut-il opposer technique et nature ? \\[0pt]
Faut-il opposer théorie et expérience ? \\[0pt]
Faut-il renoncer à la distinction entre nature et culture ? \\[0pt]
Faut-il séparer la science et la technique ? \\[0pt]
Faut-il séparer morale et politique ? \\[0pt]
Féminité et citoyenneté \\[0pt]
Fiction et mensonge \\[0pt]
Fiction et réalité \\[0pt]
Fiction et vérité \\[0pt]
Fiction et virtualité \\[0pt]
Finalité et causalité \\[0pt]
Fins et moyens \\[0pt]
Foi et bonne foi \\[0pt]
Foi et raison \\[0pt]
Foi et savoir \\[0pt]
Foi et superstition \\[0pt]
Folie et raison \\[0pt]
Folie et société \\[0pt]
Fonction et limites des modèles en sciences sociales \\[0pt]
Fonction et prédicat \\[0pt]
Force et droit \\[0pt]
Force et violence \\[0pt]
Formaliser et axiomatiser \\[0pt]
Forme et contenu \\[0pt]
Forme et fonction \\[0pt]
Forme et matière \\[0pt]
Forme et rythme \\[0pt]
Former et éduquer \\[0pt]
Généralité et universalité \\[0pt]
Genèse et structure \\[0pt]
Génie et technique \\[0pt]
Genre et espèce \\[0pt]
Géographie et géométrie \\[0pt]
Gérer et gouverner \\[0pt]
Gestion et politique \\[0pt]
Gouvernement des hommes et administration des choses \\[0pt]
Gouvernement et société \\[0pt]
Gouverner et se gouverner \\[0pt]
Grammaire et logique \\[0pt]
Grammaire et métaphysique \\[0pt]
Grammaire et philosophie \\[0pt]
Grandeur et décadence \\[0pt]
Grève et politique \\[0pt]
Guerre et paix \\[0pt]
Guerre et politique \\[0pt]
Guerres justes et injustes \\[0pt]
Hasard et destin \\[0pt]
Herméneutique et sciences humaines \\[0pt]
Histoire des techniques et histoire de l'art \\[0pt]
Histoire et anthropologie \\[0pt]
Histoire et causalité \\[0pt]
Histoire et devenir \\[0pt]
Histoire et écriture \\[0pt]
Histoire et ethnologie \\[0pt]
Histoire et fiction \\[0pt]
Histoire et géographie \\[0pt]
Histoire et géographie font-elles couple ? \\[0pt]
Histoire et mémoire \\[0pt]
Histoire et morale \\[0pt]
Histoire et narration \\[0pt]
Histoire et politique \\[0pt]
Histoire et préhistoire \\[0pt]
Histoire et progrès \\[0pt]
Histoire et structure \\[0pt]
Histoire et violence \\[0pt]
Histoire individuelle et histoire collective \\[0pt]
Historicité et vérité \\[0pt]
Humanisation et déshumanisation \\[0pt]
Humour et ironie \\[0pt]
Hypothèse et vérité \\[0pt]
Ici et maintenant \\[0pt]
Idéal et utopie \\[0pt]
Idée et concept \\[0pt]
Idée et réalité \\[0pt]
Identité et appartenance \\[0pt]
Identité et changement \\[0pt]
Identité et communauté \\[0pt]
Identité et différence \\[0pt]
Identité et égalité \\[0pt]
Identité et indiscernabilité \\[0pt]
Identité et représentation politique \\[0pt]
Identité et responsabilité \\[0pt]
Identité et ressemblance \\[0pt]
Idéologie et utopie \\[0pt]
Illégalité et injustice \\[0pt]
Illusion et apparence \\[0pt]
Illusion et hallucination \\[0pt]
Image, concept et signe \\[0pt]
Image et concept \\[0pt]
Image et idée \\[0pt]
Imaginaire et politique \\[0pt]
Imagination et conception \\[0pt]
Imagination et connaissance \\[0pt]
Imagination et culture \\[0pt]
Imagination et fantasme \\[0pt]
Imagination et perception \\[0pt]
Imagination et pouvoir \\[0pt]
Imagination et raison \\[0pt]
Imitation et création \\[0pt]
Imitation et identification \\[0pt]
Imitation et représentation \\[0pt]
Imiter et créer \\[0pt]
Immoralité et amoralité \\[0pt]
Immortalité et éternité \\[0pt]
Incertitude et action \\[0pt]
Inconscient et déterminisme \\[0pt]
Inconscient et identité \\[0pt]
Inconscient et inconscience \\[0pt]
Inconscient et instinct \\[0pt]
Inconscient et langage \\[0pt]
Inconscient et liberté \\[0pt]
Inconscient et mythes \\[0pt]
Indépendance et autonomie \\[0pt]
Indépendance et liberté \\[0pt]
Individualisme et égoïsme \\[0pt]
Individuation et identité \\[0pt]
Individu et citoyen \\[0pt]
Individu et communauté \\[0pt]
Individu et société \\[0pt]
Inégalités et différences \\[0pt]
Inégalités et discriminations \\[0pt]
Infini et indéfini \\[0pt]
Information et communication \\[0pt]
Information et opinion \\[0pt]
Innocence et ignorance \\[0pt]
Instinct et morale \\[0pt]
Instruction et éducation \\[0pt]
Instruire et éduquer \\[0pt]
Intégration et assimilation \\[0pt]
Intégration et exclusion \\[0pt]
Intentions, plans et stratégies \\[0pt]
Interdire et prohiber \\[0pt]
Intérêt général et bien commun \\[0pt]
Intériorité et extériorité \\[0pt]
Interprétation et création \\[0pt]
Interprétation et perception \\[0pt]
Interprétation et scientificité \\[0pt]
Interpréter et comprendre \\[0pt]
Interpréter et expliquer \\[0pt]
Interpréter et formaliser dans les sciences humaines \\[0pt]
Interpréter et traduire \\[0pt]
Interroger et répondre \\[0pt]
Intuition et concept \\[0pt]
Intuition et déduction \\[0pt]
Intuition et intellection \\[0pt]
Invariance et pluralité des langues \\[0pt]
Inventer et découvrir \\[0pt]
Invention et création \\[0pt]
Invention et découverte \\[0pt]
Invention et imitation \\[0pt]
Jugement analytique et jugement synthétique \\[0pt]
Jugement de goût et jugement esthétique \\[0pt]
Jugement esthétique et jugement de valeur \\[0pt]
Jugement esthétique et objectivité \\[0pt]
Jugement et proposition \\[0pt]
Jugement et réflexion \\[0pt]
Jugement et vérité \\[0pt]
Jugement moral et jugement empirique \\[0pt]
Juger et connaître \\[0pt]
Juger et décider \\[0pt]
Juger et raisonner \\[0pt]
Juger et sentir \\[0pt]
Justice et bonheur \\[0pt]
Justice et charité \\[0pt]
Justice et égalité \\[0pt]
Justice et équité \\[0pt]
Justice et fiscalité \\[0pt]
Justice et force \\[0pt]
Justice et impartialité \\[0pt]
Justice et pardon \\[0pt]
Justice et ressentiment \\[0pt]
Justice et utilité \\[0pt]
Justice et vengeance \\[0pt]
Justice et violence \\[0pt]
Justice sociale et charité publique \\[0pt]
Justification et politique \\[0pt]
Justifier et prouver \\[0pt]
La beauté et la grâce \\[0pt]
La beauté et le mal \\[0pt]
La beauté et l'ennui \\[0pt]
La bête et l'animal \\[0pt]
La bêtise et la méchanceté sont-elles liées intrinsèquement ? \\[0pt]
La bêtise et la méchanceté sont-elles liées nécessairement ? \\[0pt]
L'absolu et le relatif \\[0pt]
L'abstrait est-il en dehors de l'espace et du temps ? \\[0pt]
L'abstrait et le concret \\[0pt]
L'abstrait et l'immatériel \\[0pt]
L'académisme et les fins de l'art \\[0pt]
La carte et le territoire \\[0pt]
La cause et la raison \\[0pt]
La cause et l'effet \\[0pt]
La chair et l'esprit \\[0pt]
La chasse et la guerre \\[0pt]
La chose et l'objet \\[0pt]
La connaissance et la croyance \\[0pt]
La connaissance et la morale \\[0pt]
La connaissance et le vivant \\[0pt]
La connexion des choses et la connexion des idées \\[0pt]
La conscience de soi et l'identité personnelle \\[0pt]
La conscience et l'inconscient \\[0pt]
La convention et l'arbitraire \\[0pt]
La couleur et le dessin \\[0pt]
La crainte et l'ignorance \\[0pt]
La critique et la crise \\[0pt]
La croyance et la foi \\[0pt]
La croyance et la raison \\[0pt]
L'acte et la parole \\[0pt]
L'acte et la puissance \\[0pt]
L'acte et l'œuvre \\[0pt]
L'acteur et l'observateur dans les sciences humaines \\[0pt]
L'acteur et son rôle \\[0pt]
L'action et la passion \\[0pt]
L'action et le risque \\[0pt]
L'action et son contexte \\[0pt]
La culture et le langage \\[0pt]
La culture et les cultures \\[0pt]
La culture savante et la culture populaire \\[0pt]
La danse et l'espace \\[0pt]
La démocratie et les experts \\[0pt]
La démocratie et les institutions de la justice \\[0pt]
La démocratie et le statut de la loi \\[0pt]
La différence de l'être et de l'étant \\[0pt]
La disposition. Le possible et le réel \\[0pt]
La distinction de la nature et de la culture est-elle un fait de culture ? \\[0pt]
La distinction du sacré et du profane \\[0pt]
La distinction entre vraie et fausse sciences a-t-elle un sens ? \\[0pt]
La douleur et le plaisir \\[0pt]
La durée et l'instant \\[0pt]
La famille et la cité \\[0pt]
La famille et le droit \\[0pt]
La faute et le péché \\[0pt]
La faute et l'erreur \\[0pt]
La fin et les moyens \\[0pt]
La folie et l'étude du psychisme \\[0pt]
La fonction et l'organe \\[0pt]
La force et la violence \\[0pt]
La force et le droit \\[0pt]
La force et le droit s'opposent-ils nécessairement ? \\[0pt]
La forme et la couleur \\[0pt]
La forme et le fond \\[0pt]
La gauche et la droite \\[0pt]
La grammaire et la logique \\[0pt]
La guerre et la paix \\[0pt]
La haine et le mépris \\[0pt]
La justice et la force \\[0pt]
La justice et la loi \\[0pt]
La justice et la paix \\[0pt]
La justice et le droit \\[0pt]
La justice et l'égalité \\[0pt]
La langue et la parole \\[0pt]
La langue et la pensée \\[0pt]
La lettre et l'esprit \\[0pt]
La liberté et la grâce \\[0pt]
La liberté et la loi \\[0pt]
La liberté et la nécessité \\[0pt]
La liberté et l'égalité sont-elles compatibles ? \\[0pt]
La liberté et le hasard \\[0pt]
La liberté et les libertés \\[0pt]
La liberté et le temps \\[0pt]
La littérature et le mythe \\[0pt]
La logique et le réel \\[0pt]
La loi et la coutume \\[0pt]
La loi et la règle \\[0pt]
La loi et le contrat \\[0pt]
La loi et le règlement \\[0pt]
La loi et les mœurs \\[0pt]
La loi et l'ordre \\[0pt]
La louange et le blâme \\[0pt]
La lumière et les ténèbres \\[0pt]
La main et l'esprit \\[0pt]
La main et l'outil \\[0pt]
La matière et la forme \\[0pt]
La matière et la vie \\[0pt]
La matière et l'esprit \\[0pt]
La matière et l'esprit peuvent-ils constituer une seule chose ? \\[0pt]
L'âme et le corps \\[0pt]
L'âme et le corps sont-ils une seule et même chose ? \\[0pt]
L'âme et l'esprit \\[0pt]
L'âme, le monde et Dieu \\[0pt]
La mémoire et l'histoire \\[0pt]
La mémoire et l'individu \\[0pt]
La mémoire et l'oubli \\[0pt]
L'ami et l'ennemi \\[0pt]
La mise à l'épreuve des hypothèses en sciences humaines et sociales \\[0pt]
La morale et la politique \\[0pt]
La morale et la religion visent-elles les mêmes fins ? \\[0pt]
La morale et le droit \\[0pt]
La morale et les mœurs \\[0pt]
La moralité et le traitement des animaux \\[0pt]
La motivation et le jugement moral \\[0pt]
L'amour et la haine \\[0pt]
L'amour et la justice \\[0pt]
L'amour et l'amitié \\[0pt]
L'amour et la mort \\[0pt]
L'amour et la philosophie \\[0pt]
L'amour et le devoir \\[0pt]
L'amour et le respect \\[0pt]
L'amour et l'humanité \\[0pt]
La musique et la danse \\[0pt]
La musique et le bruit \\[0pt]
La musique et le temps \\[0pt]
L'analyse et la synthèse \\[0pt]
La nation et l'État \\[0pt]
La nature et la grâce \\[0pt]
La nature et l'artifice \\[0pt]
La nature et le beau \\[0pt]
La nature et le monde \\[0pt]
Langage et communication \\[0pt]
Langage et logique \\[0pt]
Langage et passions \\[0pt]
Langage et pensée \\[0pt]
Langage et pouvoir \\[0pt]
Langage et réalité \\[0pt]
Langage et réification \\[0pt]
Langage et société \\[0pt]
Langage, langue et parole \\[0pt]
Langage ordinaire et langage de la science \\[0pt]
L'angoisse et la peur \\[0pt]
Langue et langage \\[0pt]
Langue et parole \\[0pt]
L'animal et la bête \\[0pt]
L'animal et l'homme \\[0pt]
La norme et l'écart \\[0pt]
La norme et le fait \\[0pt]
La norme et son application \\[0pt]
La notion de loi dans les sciences de la nature et dans les sciences de l'homme \\[0pt]
La nuit et le jour \\[0pt]
La panne et la maladie \\[0pt]
La parenté et la famille \\[0pt]
La parole et l'action \\[0pt]
La parole et l'écriture \\[0pt]
La parole et le geste \\[0pt]
La partie et le tout \\[0pt]
La peine de mort est-elle juste, injuste, et pourquoi ? \\[0pt]
La pensée et la conscience sont-elles une seule et même chose ? \\[0pt]
La personne et la mémoire \\[0pt]
La personne et l'individu \\[0pt]
La philosophie et le sens commun \\[0pt]
La philosophie et les sciences \\[0pt]
La philosophie et son histoire \\[0pt]
La physique et la chimie \\[0pt]
La poésie et l'idée \\[0pt]
La police et l'armée \\[0pt]
La politique et la gloire \\[0pt]
La politique et la guerre \\[0pt]
La politique et la ville \\[0pt]
La politique et le bonheur \\[0pt]
La politique et le mal \\[0pt]
La politique et le politique \\[0pt]
La politique et le pouvoir \\[0pt]
La politique et les passions \\[0pt]
La politique et l'opinion \\[0pt]
La politique et l'utopie \\[0pt]
La politique n'est-elle que l'art de conquérir et de conserver le pouvoir ? \\[0pt]
La promesse et le contrat \\[0pt]
La propriété et le travail \\[0pt]
La puissance et l'acte \\[0pt]
La puissance et le possible \\[0pt]
La quantité et la qualité \\[0pt]
La raison et la démonstration \\[0pt]
La raison et la liberté \\[0pt]
La raison et le bonheur \\[0pt]
La raison et le calcul \\[0pt]
La raison et le réel \\[0pt]
La raison et l'existence \\[0pt]
La raison et l'expérience \\[0pt]
La raison et l'irrationnel \\[0pt]
L'architecte et la cité \\[0pt]
L'architecte et le législateur \\[0pt]
L'architecte et l'ingénieur \\[0pt]
L'architecture et l'espace \\[0pt]
La règle et l'exception \\[0pt]
La religion et la croyance \\[0pt]
L'argent et la valeur \\[0pt]
La rime et la raison \\[0pt]
L'art est par-delà beauté et laideur ? \\[0pt]
L'art et la manière \\[0pt]
L'art et la morale \\[0pt]
L'art et la mort \\[0pt]
L'art et la nature \\[0pt]
L'art et la nouveauté \\[0pt]
L'art et la raison \\[0pt]
L'art et l'artifice \\[0pt]
L'art et la société \\[0pt]
L'art et la technique \\[0pt]
L'art et la tradition \\[0pt]
L'art et la vérité \\[0pt]
L'art et la vie \\[0pt]
L'art et le beau \\[0pt]
L'art et le divin \\[0pt]
L'art et le jeu \\[0pt]
L'art et le mouvement \\[0pt]
L'art et le mythe \\[0pt]
L'art et l'éphémère \\[0pt]
L'art et le réel \\[0pt]
L'art et le rêve \\[0pt]
L'art et le sacré \\[0pt]
L'art et les arts \\[0pt]
L'art et l'espace \\[0pt]
L'art et le temps \\[0pt]
L'art et le travail \\[0pt]
L'art et le vivant \\[0pt]
L'art et l'illusion \\[0pt]
L'art et l'immoralité \\[0pt]
L'art et l'interdit \\[0pt]
L'art et l'interprétation \\[0pt]
L'art et l'invisible \\[0pt]
L'art et morale \\[0pt]
L'art et ses institutions \\[0pt]
L'artisan et l'ingénieur \\[0pt]
L'artiste et l'artisan \\[0pt]
L'artiste et la sensation \\[0pt]
L'artiste et la société \\[0pt]
L'artiste et le savant \\[0pt]
L'artiste et son public \\[0pt]
L'art peut-il se passer des critères du beau et du laid ? \\[0pt]
La sagesse et la passion \\[0pt]
La sagesse et l'expérience \\[0pt]
La science a-t-elle besoin d'un critère de démarcation entre science et non science ? \\[0pt]
La science et la foi \\[0pt]
La science et le faux \\[0pt]
La science et le mythe \\[0pt]
La science et les sciences \\[0pt]
La science et l'irrationnel \\[0pt]
La sculpture et la peinture \\[0pt]
La sculpture et le mouvement \\[0pt]
La sculpture et l'espace \\[0pt]
La signification et l'usage \\[0pt]
La société civile et l'État \\[0pt]
La société et le droit \\[0pt]
La société et les échanges \\[0pt]
La société et l'État \\[0pt]
La société et l'individu \\[0pt]
La somme et le tout \\[0pt]
La structure et les éléments \\[0pt]
La structure et le sujet \\[0pt]
La substance et l'accident \\[0pt]
La substance et le substrat \\[0pt]
La surface et la profondeur \\[0pt]
La technique et la liberté \\[0pt]
La technique et la morale \\[0pt]
La technique et le corps \\[0pt]
La technique et le travail \\[0pt]
La terre et le ciel \\[0pt]
La Terre et le Ciel \\[0pt]
La théorie et la pratique \\[0pt]
La théorie et l'expérience \\[0pt]
La trace et l'indice \\[0pt]
L'auteur et le créateur \\[0pt]
L'autre et les autres \\[0pt]
La valeur et le prix \\[0pt]
La veille et le sommeil \\[0pt]
La vérité et les vérités \\[0pt]
La vie et le vivant \\[0pt]
La ville et la campagne \\[0pt]
La vision et le toucher \\[0pt]
La volonté et le désir \\[0pt]
La vue et le toucher \\[0pt]
La vue et l'ouïe \\[0pt]
Le beau et l'agréable \\[0pt]
Le beau et le bien \\[0pt]
Le beau et le bien sont-ils, au fond, identiques ? \\[0pt]
Le beau et le bon \\[0pt]
Le beau et le joli \\[0pt]
Le beau et le sublime \\[0pt]
Le beau et l'utile \\[0pt]
Le besoin et le désir \\[0pt]
Le bien commun et l'intérêt de tous \\[0pt]
Le bien et le beau \\[0pt]
Le bien et le mal \\[0pt]
Le bien et les biens \\[0pt]
Le bien et l'utile \\[0pt]
Le bon et l'utile \\[0pt]
Le bonheur et la raison \\[0pt]
Le bonheur et la technique \\[0pt]
Le bonheur et la vertu \\[0pt]
Le bonheur et le plaisir \\[0pt]
Le bourgeois et le citoyen \\[0pt]
Le bruit et la musique \\[0pt]
Le capital et le travail \\[0pt]
Le certain et le probable \\[0pt]
Le cerveau et la pensée \\[0pt]
L'échange des marchandises et les rapports humains \\[0pt]
L'échange et l'usage \\[0pt]
Le chant et le cri \\[0pt]
Le charme et la grâce \\[0pt]
Le choix et la liberté \\[0pt]
Le ciel et la terre \\[0pt]
Le citoyen peut-il être à la fois libre et soumis à l'État ? \\[0pt]
Le clair et le distinct \\[0pt]
Le clair et l'obscur \\[0pt]
Le cœur et la raison \\[0pt]
Le comique et le tragique \\[0pt]
Le comment et le pourquoi \\[0pt]
Le commun et le propre \\[0pt]
Le concept et la vie \\[0pt]
Le concept et l'exemple \\[0pt]
Le concept et l'image \\[0pt]
Le concret et l'abstrait \\[0pt]
Le conflit entre la science et la religion est-il inévitable ? \\[0pt]
Le conflit entre l'individu et la société est-il irréductible ? \\[0pt]
L'économie et la politique \\[0pt]
L'économie et la valeur humaine \\[0pt]
L'économie et le politique obéissent-ils à une même rationalité ? \\[0pt]
L'économique et le politique \\[0pt]
Le conscient et l'inconscient \\[0pt]
Le corps et la danse \\[0pt]
Le corps et la machine \\[0pt]
Le corps et l'âme \\[0pt]
Le corps et l'esprit \\[0pt]
Le corps et le temps \\[0pt]
Le corps et l'instrument \\[0pt]
Le créé et l'incréé \\[0pt]
L'écrit et l'oral \\[0pt]
L'écriture et la parole \\[0pt]
L'écriture et la pensée \\[0pt]
Le cru et le cuit \\[0pt]
Le dedans et le dehors \\[0pt]
Le déni et la loi \\[0pt]
Le désir et la culpabilité \\[0pt]
Le désir et la liberté \\[0pt]
Le désir et la loi \\[0pt]
Le désir et le besoin \\[0pt]
Le désir et le mal \\[0pt]
Le désir et le manque \\[0pt]
Le désir et le réel \\[0pt]
Le désir et le rêve \\[0pt]
Le désir et le temps \\[0pt]
Le désir et le travail \\[0pt]
Le désir et l'instinct \\[0pt]
Le désir et l'interdit \\[0pt]
Le dessin et la couleur \\[0pt]
Le devoir et la dette \\[0pt]
Le devoir et le bonheur \\[0pt]
Le dire et le faire \\[0pt]
Le don et la dette \\[0pt]
Le don et l'échange \\[0pt]
Le donné et le construit \\[0pt]
Le droit de vie et de mort \\[0pt]
Le droit et la convention \\[0pt]
Le droit et la force \\[0pt]
Le droit et la liberté \\[0pt]
Le droit et la loi \\[0pt]
Le droit et la morale \\[0pt]
Le droit et la violence \\[0pt]
Le droit et le devoir \\[0pt]
Le Droit et l'État \\[0pt]
Le fait et le droit \\[0pt]
Le fait et l'événement \\[0pt]
Le faux et l'absurde \\[0pt]
Le faux et le fictif \\[0pt]
Le féminin et le masculin \\[0pt]
L'effet et la cause \\[0pt]
Le fini et l'infini \\[0pt]
Le fonctionnel et le beau \\[0pt]
Le fond et la forme \\[0pt]
L'égalité des hommes et des femmes est-elle une question politique ? \\[0pt]
Légalité et causalité \\[0pt]
Légalité et légitimité \\[0pt]
Légalité et moralité \\[0pt]
Le général et le particulier \\[0pt]
Le génie et la règle \\[0pt]
Le génie et le savant \\[0pt]
Le genre et l'espèce \\[0pt]
Le geste et la parole \\[0pt]
Légitimité et légalité \\[0pt]
Le goût et la distinction \\[0pt]
Le gouvernement des hommes et l'administration des choses \\[0pt]
Le gouvernement de soi et des autres \\[0pt]
Le hasard et la nécessité \\[0pt]
Le haut et le bas \\[0pt]
Le je et le tu \\[0pt]
Le jeu et la règle \\[0pt]
Le jeu et le divertissement \\[0pt]
Le jeu et le hasard \\[0pt]
Le jeu et le sérieux \\[0pt]
Le juste et le bien \\[0pt]
Le juste et le légal \\[0pt]
Le juste et le vrai \\[0pt]
Le langage et la pensée \\[0pt]
Le langage et la raison \\[0pt]
Le langage et la vérité \\[0pt]
Le langage et le réel \\[0pt]
Le langage et les langues \\[0pt]
Le langage et l'image \\[0pt]
Le latent et le manifeste \\[0pt]
Le légal et le légitime \\[0pt]
Le législateur et le juge \\[0pt]
Le légitime et le légal \\[0pt]
Le lieu et l'espace \\[0pt]
Le littéral et le figuré \\[0pt]
Le maître et le disciple \\[0pt]
Le maître et l'esclave \\[0pt]
Le masculin et le féminin \\[0pt]
Le matériel et le virtuel \\[0pt]
Le mécanisme et la mécanique \\[0pt]
Le médiat et l'immédiat \\[0pt]
Le même et l'autre \\[0pt]
Le mérite et les talents \\[0pt]
Le mien et le tien \\[0pt]
Le modèle et la copie \\[0pt]
Le moi et la conscience \\[0pt]
Le moi et ses images \\[0pt]
Le monde de l'art et l'ordinateur \\[0pt]
Le monde et la nature \\[0pt]
Le monde et la technique \\[0pt]
Le mot et la chose \\[0pt]
Le mot et le geste \\[0pt]
Le multiple et l'un \\[0pt]
Le naturalisme des sciences humaines et sociales \\[0pt]
Le naturel et l'artificiel \\[0pt]
Le naturel et le fabriqué \\[0pt]
Le naturel et le normal \\[0pt]
Le nécessaire et le contingent \\[0pt]
Le nécessaire et le superflu \\[0pt]
L'enfant et l'adulte \\[0pt]
L'énigme, le mystère et le secret \\[0pt]
Le noble et le vil \\[0pt]
Le noir et blanc \\[0pt]
Le nombre et la mesure \\[0pt]
Le nom et le verbe \\[0pt]
Le nom propre et le nom commun \\[0pt]
Le normal et le pathologique \\[0pt]
L'entendement et la volonté \\[0pt]
Le nu et la nudité \\[0pt]
Le pardon et l'oubli \\[0pt]
Le passé et le présent \\[0pt]
Le peintre et le sculpteur \\[0pt]
Le peintre et son modèle \\[0pt]
Le personnage et la personne \\[0pt]
Le peuple et la nation \\[0pt]
Le peuple et les élites \\[0pt]
Le philosophe et l'écrivain \\[0pt]
Le philosophe et le médecin \\[0pt]
Le philosophe et l'enfant \\[0pt]
Le philosophe et le sophiste \\[0pt]
Le physique et le mental \\[0pt]
Le plaisir et la douleur \\[0pt]
Le plaisir et la joie \\[0pt]
Le plaisir et la jouissance \\[0pt]
Le plaisir et la peine \\[0pt]
Le plaisir et le bien \\[0pt]
Le plaisir et le contentement \\[0pt]
Le plus et le moins \\[0pt]
Le politique et l'économique \\[0pt]
Le politique et le religieux \\[0pt]
Le politique et le social \\[0pt]
Le possible et le contingent \\[0pt]
Le possible et le probable \\[0pt]
Le possible et le réel \\[0pt]
Le possible et le virtuel \\[0pt]
Le possible et l'existence \\[0pt]
Le possible et l'impossible \\[0pt]
Le postulat et l'axiome \\[0pt]
Le pour et le contre \\[0pt]
Le pourquoi et le comment \\[0pt]
Le pouvoir des sciences humaines et sociales \\[0pt]
Le pouvoir et l'autorité \\[0pt]
Le pouvoir et la violence \\[0pt]
Le premier et le primitif \\[0pt]
Le privé et le public \\[0pt]
Le prochain et le semblable \\[0pt]
Le proche et le lointain \\[0pt]
Le propre et l'impropre \\[0pt]
Le public et le privé \\[0pt]
Le pur et l'impur \\[0pt]
Le raisonnable et le rationnel \\[0pt]
Le rationnel et le raisonnable \\[0pt]
Le rationnel et l'irrationnel \\[0pt]
Le réel et la fiction \\[0pt]
Le réel et le matériel \\[0pt]
Le réel et le nécessaire \\[0pt]
Le réel et le possible \\[0pt]
Le réel et le virtuel \\[0pt]
Le réel et le vrai \\[0pt]
Le réel et l'idéal \\[0pt]
Le réel et l'imaginaire \\[0pt]
Le réel et l'impossible \\[0pt]
Le réel et l'irréel \\[0pt]
Le rêve et la réalité \\[0pt]
Le rêve et la veille \\[0pt]
Le riche et le pauvre \\[0pt]
L'erreur et la faute \\[0pt]
L'erreur et l'ignorance \\[0pt]
L'erreur et l'illusion \\[0pt]
Le sacré et le profane \\[0pt]
Les anciens et les modernes \\[0pt]
Les Anciens et les Modernes \\[0pt]
Le sauvage et le barbare \\[0pt]
Le sauvage et le civilisé \\[0pt]
Le sauvage et le cultivé \\[0pt]
Le savant et le politique \\[0pt]
Le savant et l'ignorant \\[0pt]
Le savant et l'ingénieur \\[0pt]
Le savant, l'artiste et le politique \\[0pt]
Les besoins et les désirs \\[0pt]
Les causes et les conditions \\[0pt]
Les causes et les effets \\[0pt]
Les causes et les lois \\[0pt]
Les causes et les raisons \\[0pt]
Les causes et les signes \\[0pt]
Les changements scientifiques et la réalité \\[0pt]
Les choses et les événements \\[0pt]
L'esclave et son maître \\[0pt]
Les connaissances scientifiques peuvent-elles être à la fois vraies et provisoires ? \\[0pt]
Les désirs et les valeurs \\[0pt]
Les disciplines scientifiques et leurs interfaces \\[0pt]
Les droits de l'homme et ceux du citoyen \\[0pt]
Les droits et les devoirs \\[0pt]
Le sens et le non-sens \\[0pt]
Le sensible et la science \\[0pt]
Le sensible et l'intelligible \\[0pt]
Le sentiment du juste et de l'injuste \\[0pt]
Les faits et les valeurs \\[0pt]
Les fins et les moyens \\[0pt]
Les fins naturelles et les fins morales \\[0pt]
Les forts et les faibles \\[0pt]
Les hommes et les dieux \\[0pt]
Les hommes et les femmes \\[0pt]
Les idées et les choses \\[0pt]
Le signe et le symbole \\[0pt]
Le signe et l'indice \\[0pt]
Le simple et le complexe \\[0pt]
Le singulier et le pluriel \\[0pt]
Les intentions et les actes \\[0pt]
Les intentions et les conséquences \\[0pt]
Les lettres et les sciences \\[0pt]
Les lois et les armes \\[0pt]
Les lois et les mœurs \\[0pt]
Les mathématiques et la pensée de l'infini \\[0pt]
Les mathématiques et la quantité \\[0pt]
Les mathématiques et le monde sensible \\[0pt]
Les mathématiques et l'expérience \\[0pt]
Les mœurs et la morale \\[0pt]
Les mots et la signification \\[0pt]
Les mots et les choses \\[0pt]
Les mots et les concepts \\[0pt]
Les moyens et la fin \\[0pt]
Les moyens et les fins \\[0pt]
Les moyens et les fins en art \\[0pt]
Les normes et les valeurs \\[0pt]
Le social et le politique \\[0pt]
Le sociologue et l'historien \\[0pt]
Le soi et le je \\[0pt]
Le sommeil et la veille \\[0pt]
Le sophiste et le philosophe \\[0pt]
L'espace et le lieu \\[0pt]
L'espace et le territoire \\[0pt]
Les paroles et les actes \\[0pt]
L'espèce et l'individu \\[0pt]
Les personnes et les choses \\[0pt]
Le spirituel et le temporel \\[0pt]
Les poètes et la cité \\[0pt]
L'espoir et la crainte \\[0pt]
Les principes et les éléments \\[0pt]
L'esprit et la lettre \\[0pt]
L'esprit et la machine \\[0pt]
L'esprit et la matière peuvent-ils agir l'un sur l'autre ? \\[0pt]
L'esprit et le cerveau \\[0pt]
Les riches et les pauvres \\[0pt]
Les sciences de la vie et de la Terre \\[0pt]
Les sciences de l'homme et l'évolution \\[0pt]
Les sciences et le vivant \\[0pt]
Les sciences humaines et le droit \\[0pt]
Les sciences humaines et l'expérience \\[0pt]
Les sciences humaines et sociales n'ont-elles de sciences que le nom ? \\[0pt]
Les sciences humaines opposent-elles histoire et système ? \\[0pt]
L'essence et l'existence \\[0pt]
Les témoignages et la preuve \\[0pt]
L'esthète et l'artiste \\[0pt]
Le style et le beau \\[0pt]
Le sujet et la personne \\[0pt]
Le sujet et l'individu \\[0pt]
Le sujet et l'objet \\[0pt]
Les vivants et les morts \\[0pt]
Le talent et le génie \\[0pt]
Le tas et le tout \\[0pt]
L'État et la culture \\[0pt]
L'État et la guerre \\[0pt]
L'État et la justice \\[0pt]
L'État et la nation \\[0pt]
L'État et la Nation \\[0pt]
L'État et la protection \\[0pt]
L'État et la société \\[0pt]
L'État et la violence \\[0pt]
L'État et le droit \\[0pt]
L'État et le marché \\[0pt]
L'État et le peuple \\[0pt]
L'État et les communautés \\[0pt]
L'État et les Églises \\[0pt]
L'État et l'individu \\[0pt]
Le temporel et le spirituel \\[0pt]
Le temps de la vie et le temps de l'existence \\[0pt]
Le temps du monde et le temps de la conscience \\[0pt]
Le temps et la liberté \\[0pt]
Le temps et le bonheur \\[0pt]
Le temps et l'espace \\[0pt]
Le temps et l'éternité \\[0pt]
Le théâtre et les mœurs \\[0pt]
Le théâtre et l'existence \\[0pt]
Le tout et la partie \\[0pt]
Le tout et les parties \\[0pt]
Le tragique et le comique \\[0pt]
Le travail et la propriété \\[0pt]
Le travail et la technique \\[0pt]
Le travail et le bonheur \\[0pt]
Le travail et le jeu \\[0pt]
Le travail et le labeur \\[0pt]
Le travail et le loisir \\[0pt]
Le travail et le temps \\[0pt]
Le travail et l'œuvre \\[0pt]
L'être et l'apparence \\[0pt]
L'être et la relation \\[0pt]
L'être et la volonté \\[0pt]
L'être et le bien \\[0pt]
L'être et le devenir \\[0pt]
L'être et le devoir-être \\[0pt]
L'être et le néant \\[0pt]
L'être et les êtres \\[0pt]
L'être et l'esprit \\[0pt]
L'être et l'essence \\[0pt]
L'être et l'étant \\[0pt]
L'être et le temps \\[0pt]
L'être imaginaire et l'être de raison \\[0pt]
Le vécu et la vérité \\[0pt]
L'événement et le fait divers \\[0pt]
Le vice et la vertu \\[0pt]
Le vide et le plein \\[0pt]
L'évidence et la démonstration \\[0pt]
Le visible et l'invisible \\[0pt]
Le vivant et la machine \\[0pt]
Le vivant et la mort \\[0pt]
Le vivant et la sensibilité \\[0pt]
Le vivant et la technique \\[0pt]
Le vivant et le vécu \\[0pt]
Le vivant et l'expérimentation \\[0pt]
Le vivant et l'inerte \\[0pt]
Le vivant et ses lois \\[0pt]
Le vivant et son milieu \\[0pt]
Le volontaire et l'involontaire \\[0pt]
Le vrai et le bien \\[0pt]
Le vrai et le bien sont-ils analogues ? \\[0pt]
Le vrai et le faux \\[0pt]
Le vrai et le réel \\[0pt]
Le vrai et le vraisemblable \\[0pt]
Le vrai et l'imaginaire \\[0pt]
Le vraisemblable et le probable \\[0pt]
Le vraisemblable et le romanesque \\[0pt]
L'excès et le défaut \\[0pt]
L'existence et le temps \\[0pt]
L'expérience et la sensation \\[0pt]
L'expérience et l'expérimentation \\[0pt]
L'expert et l'amateur \\[0pt]
L'habileté et la prudence \\[0pt]
L'histoire a-t-elle un commencement et une fin ? \\[0pt]
L'histoire et la géographie \\[0pt]
L'histoire et la raison \\[0pt]
L'histoire n'est-elle que bruit et fureur ? \\[0pt]
L'historien et le politique \\[0pt]
L'homme et la bête \\[0pt]
L'homme et la machine \\[0pt]
L'homme et la nature sont-ils commensurables ? \\[0pt]
L'homme et l'animal \\[0pt]
L'homme et le citoyen \\[0pt]
L'homme et le sacré \\[0pt]
L'humour et l'ironie \\[0pt]
Libéral et libertaire \\[0pt]
Libéralisme et démocratie \\[0pt]
Libéralité et libéralisme \\[0pt]
Liberté et courage \\[0pt]
Liberté et démocratie \\[0pt]
Liberté et déterminisme \\[0pt]
Liberté et éducation \\[0pt]
Liberté et égalité \\[0pt]
Liberté et engagement \\[0pt]
Liberté et existence \\[0pt]
Liberté et habitude \\[0pt]
Liberté et indépendance \\[0pt]
Liberté et libération \\[0pt]
Liberté et licence \\[0pt]
Liberté et nécessité \\[0pt]
Liberté et négativité \\[0pt]
Liberté et pouvoir \\[0pt]
Liberté et propriété \\[0pt]
Liberté et responsabilité \\[0pt]
Liberté et savoir \\[0pt]
Liberté et sécurité \\[0pt]
Liberté et société \\[0pt]
Liberté et solitude \\[0pt]
Liberté et transgression \\[0pt]
Liberté et vérité \\[0pt]
Liberté humaine et liberté divine \\[0pt]
Libertés publiques et culture politique \\[0pt]
Libre arbitre et déterminisme sont-ils compatibles ? \\[0pt]
Libre arbitre et liberté \\[0pt]
Libre et heureux \\[0pt]
L'idéal et le réel \\[0pt]
L'idée et l'idéal \\[0pt]
L'identité et la différence \\[0pt]
Lieu et milieu \\[0pt]
L'image et le modèle \\[0pt]
L'image et le réel \\[0pt]
L'imaginaire et le réel \\[0pt]
L'imagination est-elle maîtresse d'erreur et de fausseté ? \\[0pt]
L'imagination et la raison \\[0pt]
L'inconcevable et l'impossible \\[0pt]
L'inconscient et l'involontaire \\[0pt]
L'inconscient et l'oubli \\[0pt]
L'indice et la preuve \\[0pt]
L'indicible et l'impensable \\[0pt]
L'indicible et l'ineffable \\[0pt]
L'indigène et l'étranger \\[0pt]
L'individuel et le collectif \\[0pt]
L'individu et la multitude \\[0pt]
L'individu et la société \\[0pt]
L'individu et le groupe \\[0pt]
L'individu et les masses \\[0pt]
L'individu et l'espèce \\[0pt]
L'induction et la déduction \\[0pt]
L'ineffable et l'innommable \\[0pt]
L'inexactitude et le savoir scientifique \\[0pt]
L'infini et l'indéfini \\[0pt]
L'informe et le difforme \\[0pt]
L'ingénieur et l'artiste \\[0pt]
L'ingénieur et le politique \\[0pt]
Linguistique et sens \\[0pt]
L'inné et l'acquis \\[0pt]
L'instant et la durée \\[0pt]
L'instrument et la machine \\[0pt]
L'interdit et le sacré \\[0pt]
L'intérieur et l'extérieur \\[0pt]
L'interprète et le créateur \\[0pt]
L'invention et la découverte \\[0pt]
Lire et écrire \\[0pt]
L'irrationnel et l'absurde \\[0pt]
L'irrationnel et le politique \\[0pt]
Littérature et philosophie \\[0pt]
Littérature et réalité \\[0pt]
L'objet et la chose \\[0pt]
L'observable et l'inobservable \\[0pt]
L'œil et l'oreille \\[0pt]
L'œuvre d'art et le plaisir \\[0pt]
L'œuvre d'art et sa reproduction \\[0pt]
L'œuvre d'art et son auteur \\[0pt]
L'œuvre et le produit \\[0pt]
L'œuvre : l'universel et le particulier \\[0pt]
Logique et dialectique \\[0pt]
Logique et écriture \\[0pt]
Logique et existence \\[0pt]
Logique et grammaire \\[0pt]
Logique et logiques \\[0pt]
Logique et mathématique \\[0pt]
Logique et mathématiques \\[0pt]
Logique et métaphysique \\[0pt]
Logique et méthode \\[0pt]
Logique et ontologie \\[0pt]
Logique et psychologie \\[0pt]
Logique et réalité \\[0pt]
Logique et vérité \\[0pt]
Logique générale et logique transcendantale \\[0pt]
Loi et commandement \\[0pt]
Loi morale et loi politique \\[0pt]
Loi naturelle et loi morale \\[0pt]
Loi naturelle et loi politique \\[0pt]
Loi scientifique et causalité \\[0pt]
Lois et coutumes \\[0pt]
Lois et normes \\[0pt]
Lois et règles en logique \\[0pt]
Lois et valeurs \\[0pt]
Loisir et oisiveté \\[0pt]
Lois naturelles et lois civiles \\[0pt]
L'ombre et la lumière \\[0pt]
L'oral et l'écrit \\[0pt]
L'ordre et la beauté \\[0pt]
L'ordre et la mesure \\[0pt]
L'ordre et le désordre \\[0pt]
L'organique et le mécanique \\[0pt]
L'organique et l'inorganique \\[0pt]
L'Orient et l'Occident \\[0pt]
L'original et la copie \\[0pt]
L'origine et le fondement \\[0pt]
L'oubli et le pardon \\[0pt]
L'outil et la machine \\[0pt]
L'ouvrier et l'ingénieur \\[0pt]
Lumière et couleur \\[0pt]
L'un et le multiple \\[0pt]
L'un et le tout \\[0pt]
L'un et l'être \\[0pt]
L'universel et le particulier \\[0pt]
L'universel et le singulier \\[0pt]
L'utile et l'agréable \\[0pt]
L'utile et le beau \\[0pt]
L'utile et le bien \\[0pt]
L'utile et le bon \\[0pt]
L'utile et l'honnête \\[0pt]
L'utile et l'inutile \\[0pt]
L'utopie et l'idéologie \\[0pt]
Machine et organisme \\[0pt]
Machines et liberté \\[0pt]
Machines et mémoire \\[0pt]
Magie et religion \\[0pt]
Maître et disciple \\[0pt]
Maître et serviteur \\[0pt]
Maîtrise et puissance \\[0pt]
Majorité et minorité \\[0pt]
Maladie de l'âme et maladie du corps \\[0pt]
Maladie et convalescence \\[0pt]
Mal et liberté \\[0pt]
Masculin et féminin \\[0pt]
Masculin et Féminin \\[0pt]
Mass-media et création artistique \\[0pt]
Matérialisme et mécanisme \\[0pt]
Mathématique et imagination \\[0pt]
Mathématiques et réalité \\[0pt]
Mathématiques et sciences humaines \\[0pt]
Mathématiques et universel \\[0pt]
Mathématiques pures et mathématiques appliquées \\[0pt]
Matière et corps \\[0pt]
Matière et forme \\[0pt]
Matière et matériaux \\[0pt]
Mécanisme et finalité \\[0pt]
Médecine et morale \\[0pt]
Médecine et philosophie \\[0pt]
Mémoire et anticipation \\[0pt]
Mémoire et fiction \\[0pt]
Mémoire et identité \\[0pt]
Mémoire et imagination \\[0pt]
Mémoire et responsabilité \\[0pt]
Mémoire et souvenir \\[0pt]
Mensonge et politique \\[0pt]
Mérite et démocratie \\[0pt]
Mesure et démesure \\[0pt]
Métaphysique et histoire \\[0pt]
Métaphysique et mystique \\[0pt]
Métaphysique et ontologie \\[0pt]
Métaphysique et poésie \\[0pt]
Métaphysique et politique \\[0pt]
Métaphysique et psychologie \\[0pt]
Métaphysique et religion \\[0pt]
Métaphysique et théologie \\[0pt]
Méthode et métaphysique \\[0pt]
Métier et vocation \\[0pt]
Microscope et télescope \\[0pt]
Milieux naturels et milieux sociaux \\[0pt]
Misère et pauvreté \\[0pt]
Modèle et copie \\[0pt]
Modèle, type et paradigme \\[0pt]
Mœurs et moralité \\[0pt]
Monde et nature \\[0pt]
Monde naturel et monde humain \\[0pt]
Monde perçu et monde conçu \\[0pt]
Monologue et dialogue \\[0pt]
Montrer et démontrer \\[0pt]
Montrer et dire \\[0pt]
Morale et calcul \\[0pt]
Morale et convention \\[0pt]
Morale et dispositions \\[0pt]
Morale et économie \\[0pt]
Morale et éducation \\[0pt]
Morale et histoire \\[0pt]
Morale et identité \\[0pt]
Morale et intérêt \\[0pt]
Morale et liberté \\[0pt]
Morale et politique sont-elles indépendantes ? \\[0pt]
Morale et pratique \\[0pt]
Morale et prudence \\[0pt]
Morale et religion \\[0pt]
Morale et science des mœurs \\[0pt]
Morale et sentiments \\[0pt]
Morale et sexualité \\[0pt]
Morale et société \\[0pt]
Morale et technique \\[0pt]
Morale et violence \\[0pt]
Moralité et connaissance \\[0pt]
Moralité et utilité \\[0pt]
Mouvement et changement \\[0pt]
Murs et frontières \\[0pt]
Musique et architecture \\[0pt]
Musique et bruit \\[0pt]
Musique et émotion \\[0pt]
Musique et poésie \\[0pt]
Musique et rhétorique \\[0pt]
Mythe et connaissance \\[0pt]
Mythe et histoire \\[0pt]
Mythe et pensée \\[0pt]
Mythe et philosophie \\[0pt]
Mythe et symbole \\[0pt]
Mythe et vérité \\[0pt]
Mythes et idéologies \\[0pt]
Naître et mourir \\[0pt]
Narration et identité \\[0pt]
Nation et richesse \\[0pt]
Nature et artifice \\[0pt]
Nature et convention \\[0pt]
Nature et culture \\[0pt]
Nature et fonction du sacrifice \\[0pt]
Nature et histoire \\[0pt]
Nature et institution \\[0pt]
Nature et institutions \\[0pt]
Nature et liberté \\[0pt]
Nature et loi \\[0pt]
Nature et mesure du temps \\[0pt]
Nature et monde \\[0pt]
Nature et morale \\[0pt]
Nature et nature humaine \\[0pt]
Naturel et artificiel \\[0pt]
Nécessité et contingence \\[0pt]
Nécessité et fatalité \\[0pt]
Nécessité et liberté \\[0pt]
Négation et privation \\[0pt]
Névroses et psychoses \\[0pt]
Nomade et sédentaire \\[0pt]
Nom propre et nom commun \\[0pt]
Non-sens et absurdité \\[0pt]
Normalité et normativité \\[0pt]
Norme et moyenne \\[0pt]
Normes et valeurs \\[0pt]
Normes morales et normes vitales \\[0pt]
Nous et les autres \\[0pt]
Nouveauté et tradition \\[0pt]
Obéir et désobéir \\[0pt]
Obéissance et esclavage \\[0pt]
Obéissance et liberté \\[0pt]
Obéissance et servitude \\[0pt]
Obéissance et soumission \\[0pt]
Objecter et répondre \\[0pt]
Objectivé et subjectivité \\[0pt]
Objectivité et impartialité \\[0pt]
Objectivité et subjectivité \\[0pt]
Objet et œuvre \\[0pt]
Observation et expérience \\[0pt]
Observation et expérimentation \\[0pt]
Observer, décrire et expliquer \\[0pt]
Observer et comprendre \\[0pt]
Observer et expérimenter \\[0pt]
Observer et interpréter \\[0pt]
Œuvre et événement \\[0pt]
Offre et demande \\[0pt]
Ontologie et métaphysique \\[0pt]
Ontologie et théologie \\[0pt]
Opinion et ignorance \\[0pt]
Opinion publique et conscience universelle \\[0pt]
Optimisme et pessimisme \\[0pt]
Ordre et chaos \\[0pt]
Ordre et désordre \\[0pt]
Ordre et justice \\[0pt]
Ordre et liberté \\[0pt]
Ordre et progrès \\[0pt]
Organe et fonction \\[0pt]
Organisme et milieu \\[0pt]
Origine des hommes et principe de l'humanité \\[0pt]
Origine et commencement \\[0pt]
Origine et fondement \\[0pt]
Outil et machine \\[0pt]
Outil et organe \\[0pt]
Par-delà beauté et laideur \\[0pt]
Pardonner et oublier \\[0pt]
Parler et agir \\[0pt]
Parler et penser \\[0pt]
Parole et pouvoir \\[0pt]
Paroles et actes \\[0pt]
Participation et imitation \\[0pt]
Passions et intérêts \\[0pt]
Peinture et abstraction \\[0pt]
Peinture et histoire \\[0pt]
Peinture et photographie \\[0pt]
Peinture et réalité \\[0pt]
Peinture et vérité \\[0pt]
Pensée et calcul \\[0pt]
Pensée et réalité \\[0pt]
Penser et calculer \\[0pt]
Penser et connaître \\[0pt]
Penser et être \\[0pt]
Penser et imaginer \\[0pt]
Penser et parler \\[0pt]
Penser et raisonner \\[0pt]
Penser et savoir \\[0pt]
Penser et sentir \\[0pt]
Perception et aperception \\[0pt]
Perception et connaissance \\[0pt]
Perception et création artistique \\[0pt]
Perception et imagination \\[0pt]
Perception et interprétation \\[0pt]
Perception et jugement \\[0pt]
Perception et mémoire \\[0pt]
Perception et mouvement \\[0pt]
Perception et observation \\[0pt]
Perception et passivité \\[0pt]
Perception et sensation \\[0pt]
Perception et souvenir \\[0pt]
Perception et vérité \\[0pt]
Percevoir et concevoir \\[0pt]
Percevoir et imaginer \\[0pt]
Percevoir et juger \\[0pt]
Percevoir et sentir \\[0pt]
Percevoir et témoigner \\[0pt]
Permanence et identité \\[0pt]
Personne et individu \\[0pt]
Persuader et convaincre \\[0pt]
Peuple et culture \\[0pt]
Peuple et masse \\[0pt]
Peuple et multitude \\[0pt]
Peuple et société \\[0pt]
Peuples et masses \\[0pt]
Peut-on à la fois préserver et dominer la nature ? \\[0pt]
Peut-on concilier bonheur et liberté ? \\[0pt]
Peut-on concilier le droit à la liberté et l'égalité des droits ? \\[0pt]
Peut-on concilier nécessité et liberté ? \\[0pt]
Peut-on discuter des goûts et des couleurs ? \\[0pt]
Peut-on distinguer entre de bons et de mauvais désirs ? \\[0pt]
Peut-on distinguer entre les bons et les mauvais désirs ? \\[0pt]
Peut-on distinguer la vie et l'existence ? \\[0pt]
Peut-on être à la fois lucide et heureux ? \\[0pt]
Peut-on être injuste et heureux ? \\[0pt]
Peut-on opposer connaissance scientifique et création artistique ? \\[0pt]
Peut-on opposer justice et liberté ? \\[0pt]
Peut-on opposer morale et technique ? \\[0pt]
Peut-on opposer nature et culture ? \\[0pt]
Peut-on opposer religion et raison ? \\[0pt]
Peut-on préconiser, dans les sciences humaines et sociales, l'imitation des sciences de la nature ? \\[0pt]
Peut-on séparer l'homme et l'œuvre ? \\[0pt]
Peut-on séparer politique et économie ? \\[0pt]
Philosophie et mathématiques \\[0pt]
Philosophie et métaphysique \\[0pt]
Philosophie et poésie \\[0pt]
Philosophie et religion \\[0pt]
Philosophie et système \\[0pt]
Philosophie et théologie \\[0pt]
Physique et cosmologie \\[0pt]
Physique et mathématique \\[0pt]
Physique et mathématiques \\[0pt]
Physique et métaphysique \\[0pt]
Pitié et compassion \\[0pt]
Pitié et cruauté \\[0pt]
Pitié et mépris \\[0pt]
Plaisir et bonheur \\[0pt]
Plaisir et douleur \\[0pt]
Plaisir et préférences \\[0pt]
Pluralisme et politique \\[0pt]
Pluralité et unité \\[0pt]
Poésie et philosophie \\[0pt]
Poésie et vérité \\[0pt]
Poétique et prosaïque \\[0pt]
Point de vue du créateur et point de vue du spectateur \\[0pt]
Police et politique \\[0pt]
Politique et coopération \\[0pt]
Politique et esthétique \\[0pt]
Politique et médias \\[0pt]
Politique et mémoire \\[0pt]
Politique et parole \\[0pt]
Politique et participation \\[0pt]
Politique et passions \\[0pt]
Politique et propagande \\[0pt]
Politique et religion \\[0pt]
Politique et secret \\[0pt]
Politique et technique \\[0pt]
Politique et technologie \\[0pt]
Politique et territoire \\[0pt]
Politique et trahison \\[0pt]
Politique et unité \\[0pt]
Politique et vérité \\[0pt]
Politique et vertu \\[0pt]
Possession et propriété \\[0pt]
Possibilité et intelligibilité \\[0pt]
Pourquoi distinguer nature et culture ? \\[0pt]
Pourquoi vouloir devenir « comme maîtres et possesseurs de la nature » ? \\[0pt]
Pouvoir et autorité \\[0pt]
Pouvoir et contre-pouvoir \\[0pt]
Pouvoir et devoir \\[0pt]
Pouvoir et domination \\[0pt]
Pouvoir et politique \\[0pt]
Pouvoir et puissance \\[0pt]
Pouvoir et savoir \\[0pt]
Pouvoirs et libertés \\[0pt]
Pouvoir temporel et pouvoir spirituel \\[0pt]
Prédicats et relations \\[0pt]
Prédiction et prévision \\[0pt]
Prédiction et probabilité \\[0pt]
Prédire et expliquer \\[0pt]
Prémisses et conclusions \\[0pt]
Prescrire et décrire \\[0pt]
Prescrire et interdire \\[0pt]
Présence et absence \\[0pt]
Présence et représentation \\[0pt]
Preuve et démonstration \\[0pt]
Preuve et résultat dans les sciences humaines \\[0pt]
Principe et axiome \\[0pt]
Principe et cause \\[0pt]
Principe et commencement \\[0pt]
Principe et conséquence \\[0pt]
Principe et fondement \\[0pt]
Principes et stratégie \\[0pt]
Privation et négation \\[0pt]
Probabilité et explication scientifique \\[0pt]
Production et action \\[0pt]
Production et création \\[0pt]
Production et réception de l'œuvre d'art \\[0pt]
Produire et consommer \\[0pt]
Produire et créer \\[0pt]
Proposition et jugement \\[0pt]
Propriétés et dispositions \\[0pt]
Prose et poésie \\[0pt]
Prospérité et sécurité \\[0pt]
Prouver et démontrer \\[0pt]
Prouver et éprouver \\[0pt]
Prouver et justifier \\[0pt]
Prouver et réfuter \\[0pt]
Providence et destin \\[0pt]
Prudence et liberté \\[0pt]
Psychanalyse et critique littéraire \\[0pt]
Psychanalyse et esthétique \\[0pt]
Psychanalyse et interprétation \\[0pt]
Psychologie et contrôle des comportements \\[0pt]
Psychologie et ethnologie \\[0pt]
Psychologie et métaphysique \\[0pt]
Psychologie et neurosciences \\[0pt]
Psychologie et philosophie \\[0pt]
Puissance et pouvoir \\[0pt]
Pulsion et instinct \\[0pt]
Pulsions et passions \\[0pt]
Punition et vengeance \\[0pt]
Qualité et quantité \\[0pt]
Qualités premières et qualités secondes \\[0pt]
Quantification et intelligibilité \\[0pt]
Quantification et pensée scientifique \\[0pt]
Quantité et qualité \\[0pt]
Quel est le statut de la preuve en sciences humaines et sociales ? \\[0pt]
Quelle différence y a-t-il entre un corps vivant et un corps mort ? \\[0pt]
Quels matériaux pour les sciences humaines et sociales ? \\[0pt]
Que signifier « juger en son âme et conscience » ? \\[0pt]
Qu'est-ce que « se rendre maître et possesseur de la nature » ? \\[0pt]
Question et problème \\[0pt]
Que vaut la distinction entre nature et culture ? \\[0pt]
Qui suis-je et qui es-tu ? \\[0pt]
Raison et calcul \\[0pt]
Raison et démonstration \\[0pt]
Raison et déraison \\[0pt]
Raison et dialogue \\[0pt]
Raison et folie \\[0pt]
Raison et fondement \\[0pt]
Raison et langage \\[0pt]
Raison et politique \\[0pt]
Raison et religion sont-elles incompatibles ? \\[0pt]
Raison et révélation \\[0pt]
Raison et technique \\[0pt]
Raison et tradition \\[0pt]
Raisonnable et rationnel \\[0pt]
Raisonnement et expérimentation \\[0pt]
Raisonner et argumenter \\[0pt]
Raisonner et calculer \\[0pt]
Raisonner et décider \\[0pt]
Rapports de la science et de la politique \\[0pt]
Rationalité et irrationalité du travail \\[0pt]
Rationnel et raisonnable \\[0pt]
Réalisme et idéalisme \\[0pt]
Réalité et apparence \\[0pt]
Réalité et idéal \\[0pt]
Réalité et perception \\[0pt]
Réalité et représentation \\[0pt]
Rebuts et objets quelconques : une matière pour l'art ? \\[0pt]
Recherche et militantisme \\[0pt]
Récit et explication en histoire \\[0pt]
Récit et fiction \\[0pt]
Récit et histoire \\[0pt]
Récit et mémoire \\[0pt]
Reconnaissance et inégalité \\[0pt]
Réforme et révolution \\[0pt]
Refuser et réfuter \\[0pt]
Réfutation et confirmation \\[0pt]
Regarder et voir \\[0pt]
Règle et commandement \\[0pt]
Règle morale et norme juridique \\[0pt]
Règles logiques et lois psychologiques \\[0pt]
Règles sociales et loi morale \\[0pt]
Regrets et remords \\[0pt]
Régularité et singularité \\[0pt]
Relativisme et sciences humaines \\[0pt]
Religion et démocratie \\[0pt]
Religion et liberté \\[0pt]
Religion et métaphysique \\[0pt]
Religion et moralité \\[0pt]
Religion et politique \\[0pt]
Religion et raison \\[0pt]
Religion et superstition \\[0pt]
Religion et vérité \\[0pt]
Religion et violence \\[0pt]
Religion naturelle et religion révélée \\[0pt]
Religions et démocratie \\[0pt]
Représentation et expression \\[0pt]
Représentation et illusion \\[0pt]
République et démocratie \\[0pt]
Résistance et obéissance \\[0pt]
Résistance et soumission \\[0pt]
Respect et tolérance \\[0pt]
Responsabilité et culpabilité \\[0pt]
Ressemblance et identité \\[0pt]
Ressemblance et imitation \\[0pt]
Rêve et réalité \\[0pt]
Révolte et révolution \\[0pt]
Rhétorique et vérité \\[0pt]
Richesse et pauvreté \\[0pt]
Rire et pleurer \\[0pt]
Rites et cérémonies \\[0pt]
Rituels et cérémonies \\[0pt]
Roman et vérité \\[0pt]
Sagesse et renoncement \\[0pt]
Santé et politique \\[0pt]
Savoir et croire \\[0pt]
Savoir et démontrer \\[0pt]
Savoir et faire \\[0pt]
Savoir et liberté \\[0pt]
Savoir et objectivité dans les sciences \\[0pt]
Savoir et pouvoir \\[0pt]
Savoir et rectification \\[0pt]
Savoir et savoir faire \\[0pt]
Savoir et savoir-faire \\[0pt]
Savoir et vérifier \\[0pt]
Scepticisme et relativisme \\[0pt]
Science du vivant et finalisme \\[0pt]
Science et abstraction \\[0pt]
Science et certitude \\[0pt]
Science et complexité \\[0pt]
Science et conscience \\[0pt]
Science et croyance \\[0pt]
Science et démocratie \\[0pt]
Science et domination sociale \\[0pt]
Science et expérience \\[0pt]
Science et histoire \\[0pt]
Science et hypothèse \\[0pt]
Science et idéologie \\[0pt]
Science et imagination \\[0pt]
Science et invention \\[0pt]
Science et libération \\[0pt]
Science et magie \\[0pt]
Science et métaphysique \\[0pt]
Science et méthode \\[0pt]
Science et mythe \\[0pt]
Science et objectivité \\[0pt]
Science et opinion \\[0pt]
Science et persuasion \\[0pt]
Science et philosophie \\[0pt]
Science et pouvoir \\[0pt]
Science et réalité \\[0pt]
Science et religion \\[0pt]
Science et sagesse \\[0pt]
Science et société \\[0pt]
Science et subjectivité \\[0pt]
Science et technique \\[0pt]
Science et technologie \\[0pt]
Science pure et science appliquée \\[0pt]
Sciences de la nature et sciences de l'esprit \\[0pt]
Sciences de la nature et sciences humaines \\[0pt]
Sciences empiriques et critères du vrai \\[0pt]
Sciences et philosophie \\[0pt]
Sciences et vérité \\[0pt]
Sciences humaines et déterminisme \\[0pt]
Sciences humaines et herméneutique \\[0pt]
Sciences humaines et idéologie \\[0pt]
Sciences humaines et liberté sont-elles compatibles ? \\[0pt]
Sciences humaines et littérature \\[0pt]
Sciences humaines et naturalisme \\[0pt]
Sciences humaines et nature humaine \\[0pt]
Sciences humaines et objectivité \\[0pt]
Sciences humaines et philosophie \\[0pt]
Sciences humaines et vérité \\[0pt]
Sciences sociales et humanités \\[0pt]
Sciences sociales et questions environnementales \\[0pt]
Sécularisation et laïcisation \\[0pt]
Sécurité et liberté \\[0pt]
Se donner corps et âme \\[0pt]
Sensation et perception \\[0pt]
Sens et existence \\[0pt]
Sens et fait \\[0pt]
Sens et limites de la notion de capital culturel \\[0pt]
Sens et non-sens \\[0pt]
Sens et sensibilité \\[0pt]
Sens et sensible \\[0pt]
Sens et signification \\[0pt]
Sens et structure \\[0pt]
Sens et vérité \\[0pt]
Sensible et intelligible \\[0pt]
Sens propre et sens figuré \\[0pt]
Sentiment et émotion \\[0pt]
Sentiment et justice sont-ils compatibles ? \\[0pt]
Sentir et juger \\[0pt]
Sentir et penser \\[0pt]
Sentir et percevoir \\[0pt]
Se parler et s'entendre \\[0pt]
Sexe et genre \\[0pt]
Sexualité et féminité \\[0pt]
Sexualité et nature \\[0pt]
Signe et symbole \\[0pt]
Signes, traces et indices \\[0pt]
Signification et expression \\[0pt]
Signification et vérité \\[0pt]
Sincérité et authenticité \\[0pt]
Sincérité et vérité \\[0pt]
Société et biologie \\[0pt]
Société et communauté \\[0pt]
Société et contrat \\[0pt]
Société et liberté \\[0pt]
Société et mouvement \\[0pt]
Société et organisme \\[0pt]
Société et réciprocité \\[0pt]
Société et religion \\[0pt]
Sociologie et anthropologie \\[0pt]
Sociologie et économie \\[0pt]
Sociologie et ethnologie \\[0pt]
Solitude et isolement \\[0pt]
Solitude et liberté \\[0pt]
Songe et réalité \\[0pt]
Sophisme et paralogisme \\[0pt]
Sophismes et paradoxes \\[0pt]
Soumission et servitude \\[0pt]
Souveraineté et liberté \\[0pt]
Sphère privée et sphère publique \\[0pt]
Spontanéité et liberté \\[0pt]
Sport et politique \\[0pt]
Structure et événement \\[0pt]
Structure et histoire \\[0pt]
Subjectivité et vérité \\[0pt]
Substance et accident \\[0pt]
Substance et sujet \\[0pt]
Suffit-il, pour être juste, d'obéir aux lois et aux coutumes de son pays ? \\[0pt]
Suicide et homicide \\[0pt]
Sujet et citoyen \\[0pt]
Sujet et prédicat \\[0pt]
Sujet et substance \\[0pt]
Superstition et fanatisme sont-ils inhérents à la religion ? \\[0pt]
Superstition et religion \\[0pt]
Surface et profondeur \\[0pt]
Sur quels fondements distinguer ce qui est public et ce qui est privé ? \\[0pt]
Surveillance et discipline \\[0pt]
Syllogisme et démonstration \\[0pt]
Sympathie et respect \\[0pt]
Système et liberté \\[0pt]
Système et structure \\[0pt]
Talent et génie \\[0pt]
Tautologie et contradiction \\[0pt]
Technique et apprentissage \\[0pt]
Technique et démocratie \\[0pt]
Technique et esthétique \\[0pt]
Technique et idéologie \\[0pt]
Technique et intérêt \\[0pt]
Technique et liberté \\[0pt]
Technique et nature \\[0pt]
Technique et pratiques scientifiques \\[0pt]
Technique et progrès \\[0pt]
Technique et responsabilité \\[0pt]
Technique et savoir-faire \\[0pt]
Technique et technologie \\[0pt]
Technique et violence \\[0pt]
Témoignage et vérité \\[0pt]
Temps et commencement \\[0pt]
Temps et conscience \\[0pt]
Temps et création \\[0pt]
Temps et durée \\[0pt]
Temps et éternité \\[0pt]
Temps et histoire \\[0pt]
Temps et irréversibilité \\[0pt]
Temps et liberté \\[0pt]
Temps et mémoire \\[0pt]
Temps et musique \\[0pt]
Temps et négation \\[0pt]
Temps et réalité \\[0pt]
Temps et vérité \\[0pt]
Temps individuel et temps collectif \\[0pt]
Temps physique et temps biologique \\[0pt]
Tendances et besoins \\[0pt]
Terres et territoires \\[0pt]
Territoire et peuplement \\[0pt]
Thème et variations \\[0pt]
Théologie et morale \\[0pt]
Théorie et expérience \\[0pt]
Théorie et modèle \\[0pt]
Théorie et modélisation \\[0pt]
Théorie et pratique \\[0pt]
Théorie et pratique en médecine \\[0pt]
Tolérance et laïcité \\[0pt]
Tradition et innovation \\[0pt]
Tradition et liberté \\[0pt]
Tradition et nouveauté \\[0pt]
Tradition et raison \\[0pt]
Tradition et transmission \\[0pt]
Tradition et vérité \\[0pt]
Traduire et interpréter \\[0pt]
Tragédie et comédie \\[0pt]
Transcendance et altérité \\[0pt]
Transcendance et immanence \\[0pt]
Travail et aliénation \\[0pt]
Travail et besoin \\[0pt]
Travail et bonheur \\[0pt]
Travail et capital \\[0pt]
Travail et liberté \\[0pt]
Travail et loisir \\[0pt]
Travail et nécessité \\[0pt]
Travail et œuvre \\[0pt]
Travail et plaisir \\[0pt]
Travail et propriété \\[0pt]
Travail et subjectivité \\[0pt]
Travailler et œuvrer \\[0pt]
Travail manuel et travail intellectuel \\[0pt]
Travail servile et travail salarié \\[0pt]
Trouver et prouver \\[0pt]
Tuer et laisser mourir \\[0pt]
Tyrannie et dictature \\[0pt]
Unité et unicité \\[0pt]
Universalité et nécessité dans les sciences \\[0pt]
Univocité et équivocité \\[0pt]
Urbanisme et politique \\[0pt]
Utilité et beauté \\[0pt]
Utopie et tradition \\[0pt]
Vainqueurs et vaincus \\[0pt]
Valeur et évaluation \\[0pt]
Vérité et apparence \\[0pt]
Vérité et certitude \\[0pt]
Vérité et cohérence \\[0pt]
Vérité et cohérence ? \\[0pt]
Vérité et convention \\[0pt]
Vérité et démonstration \\[0pt]
Vérité et efficacité \\[0pt]
Vérité et exactitude \\[0pt]
Vérité et fiction \\[0pt]
Vérité et histoire \\[0pt]
Vérité et illusion \\[0pt]
Vérité et liberté \\[0pt]
Vérité et poésie \\[0pt]
Vérité et réalité \\[0pt]
Vérité et référence \\[0pt]
Vérité et religion \\[0pt]
Vérité et sensibilité \\[0pt]
Vérité et signification \\[0pt]
Vérité et sincérité \\[0pt]
Vérité et subjectivité \\[0pt]
Vérité et totalité \\[0pt]
Vérité et validité \\[0pt]
Vérité et vérification \\[0pt]
Vérité et vraisemblance \\[0pt]
Vérités de fait et vérités de raison \\[0pt]
Vertu et habitude \\[0pt]
Vertu et perfection \\[0pt]
Vice et délice \\[0pt]
Vie et destin \\[0pt]
Vie et existence \\[0pt]
Vie et finalité \\[0pt]
Vie et matière \\[0pt]
Vie et matière : une distinction périmée ? \\[0pt]
Vie et mécanisme \\[0pt]
Vie et volonté \\[0pt]
Vie politique et vie contemplative \\[0pt]
Vie privée et vie publique \\[0pt]
Vie publique et vie privée \\[0pt]
Violence et discours \\[0pt]
Violence et force \\[0pt]
Violence et histoire \\[0pt]
Violence et politique \\[0pt]
Violence et pouvoir \\[0pt]
Vitalisme et mécanique \\[0pt]
Vivre et bien vivre \\[0pt]
Vivre et exister \\[0pt]
Vivre et exister, est-ce la même chose ? \\[0pt]
Voir et concevoir \\[0pt]
Voir et entendre \\[0pt]
Voir et observer \\[0pt]
Voir et savoir \\[0pt]
Voir et toucher \\[0pt]
Voir le meilleur et faire le pire \\[0pt]
Volonté et désir \\[0pt]
Vouloir et pouvoir \\[0pt]
Y a-t-il continuité entre l'expérience et la science ? \\[0pt]
Y a-t-il continuité ou discontinuité entre la pensée mythique et la science ? \\[0pt]
Y a-t-il de bons et de mauvais désirs ? \\[0pt]
Y a-t-il incompatibilité entre science et religion ? \\[0pt]
Y a-t-il lieu de distinguer entre éthique et morale ? \\[0pt]
Y a-t-il lieu de distinguer instinct et pulsion ? \\[0pt]
Y a-t-il lieu de distinguer le don et l'échange ? \\[0pt]
Y a-t-il lieu d'opposer la philosophie et les sciences humaines ? \\[0pt]
Y a-t-il lieu d'opposer matière et esprit ? \\[0pt]
Y a-t-il un différend entre poésie et philosophie ? \\[0pt]


\subsection{X ou Y}
\label{sec:org4b1885e}

\noindent
Ai-je un corps ou suis-je mon corps ? \\[0pt]
Animal politique ou social ? \\[0pt]
Cité juste ou citoyen juste ? \\[0pt]
Connaît-on la vie ou bien connaît-on le vivant ? \\[0pt]
Connaît-on la vie ou connaît-on le vivant ? \\[0pt]
Connaît-on la vie ou le vivant ? \\[0pt]
Connaître la vie ou le vivant ? \\[0pt]
Dieu, prouvé ou éprouvé ? \\[0pt]
Diversité ou universalité \\[0pt]
Droits de l'homme ou droits du citoyen ? \\[0pt]
Est-ce par son objet ou par ses méthodes qu'une science peut se définir ? \\[0pt]
Être ou avoir \\[0pt]
Être ou ne pas être \\[0pt]
Être ou ne pas être ? \\[0pt]
Être ou ne pas être, est-ce la question ? \\[0pt]
Faut-il rire ou pleurer ? \\[0pt]
Faut-il se demander si l'homme est bon ou méchant par nature \\[0pt]
Interpréter ou expliquer \\[0pt]
Juge-t-on un acte ou son auteur ? \\[0pt]
La beauté est-elle dans le regard ou dans la chose vue ? \\[0pt]
L'accord est-il le terme ou le principe du dialogue ? \\[0pt]
La conscience est-elle ou n'est-elle pas ? \\[0pt]
La culture est-elle ce qui unit ou ce qui sépare les hommes ? \\[0pt]
La démocratie est-elle moyen ou fin ? \\[0pt]
La docilité est-elle un vice ou une vertu ? \\[0pt]
La domination technique de la nature doit-elle susciter la crainte ou l'espoir ? \\[0pt]
La justice : moyen ou fin de la politique ? \\[0pt]
La liberté se prouve-t-elle ou s'éprouve-t-elle ? \\[0pt]
La logique : calcul ou langage ? \\[0pt]
La logique : découverte ou invention ? \\[0pt]
« La logique » ou bien « les logiques » ? \\[0pt]
La majorité, force ou droit ? \\[0pt]
La métaphysique relève-t-elle de la philosophie ou de la poésie ? \\[0pt]
La métaphysique se définit-elle par son objet ou sa démarche ? \\[0pt]
La moralité est-elle affaire de principes ou de conséquences ? \\[0pt]
La non-contradiction : principe logique ou ontologique ? \\[0pt]
La notion de caractère relève-t- elle de la morale ou de la psychologie ? \\[0pt]
La ou les vertus ? \\[0pt]
La pédagogie, philosophie pratique ou psychologie appliquée ? \\[0pt]
La perception de l'espace est-elle innée ou acquise ? \\[0pt]
La philosophie est-elle une méditation de la vie ou de la mort ? \\[0pt]
La politique est-elle affaire d'expérience ou de théorie ? \\[0pt]
La politique est-elle un moyen ou une fin ? \\[0pt]
L'architecte : artiste ou ingénieur ? \\[0pt]
L'architecte : artiste ou l'ingénieur \\[0pt]
L'art donne-t-il à rêver ou à penser ? \\[0pt]
L'art : expérience, exercice ou habitude ? \\[0pt]
L'art ou les arts \\[0pt]
La santé est-elle un droit ou un devoir ? \\[0pt]
La science découvre-t-elle ou construit-elle son objet ? \\[0pt]
La vérité est-elle affaire de croyance ou de savoir ? \\[0pt]
Le bonheur est-il affaire de hasard ou de nécessité ? \\[0pt]
Le cinéma est-il un art ou une industrie ? \\[0pt]
L'écologie est-elle une science de la nature ou une science sociale ? \\[0pt]
Le droit sert-il à établir l'ordre ou la justice ? \\[0pt]
L'égalité des hommes est-elle un fait ou une idée ? \\[0pt]
Le genre humain : unité ou pluralité ? \\[0pt]
Le goût : certitude ou conviction ? \\[0pt]
Le langage rapproche-t-il ou sépare-t-il les hommes ? \\[0pt]
Lequel, de l'art ou du réel, est-il une imitation de l'autre ? \\[0pt]
Le rôle des théories est-il d'expliquer ou de décrire ? \\[0pt]
Les hypothèses scientifiques ont-elles pour nature d'être confirmées ou infirmées ? \\[0pt]
Les mathématiques sont-elles une découverte ou une invention ? \\[0pt]
Le sport : s'accomplir ou se dépasser ? \\[0pt]
Les qualités sensibles sont-elles dans les choses ou dans l'esprit ? \\[0pt]
Les sciences décrivent-elles ou construisent-elles leurs objets ? \\[0pt]
Les sciences humaines sont-elles explicatives ou compréhensives ? \\[0pt]
L'État est-il fin ou moyen ? \\[0pt]
Le temps est-il en nous ou hors de nous ? \\[0pt]
Le temps est-il une expérience de la continuité ou de la discontinuité ? \\[0pt]
Le travail unit-il ou sépare-t-il les hommes ? \\[0pt]
L'évidence est-elle un obstacle ou un instrument de la recherche de la vérité ? \\[0pt]
L'histoire : enquête ou science ? \\[0pt]
L'histoire est-elle une explication ou une justification du passé ? \\[0pt]
L'histoire : science ou récit ? \\[0pt]
L'homme est-il meilleur seul ou en société ? \\[0pt]
L'homme se reconnaît-il mieux dans le travail ou dans le loisir ? \\[0pt]
L'identité personnelle est-elle donnée ou construite ? \\[0pt]
L'incertitude est-elle dans les choses ou dans les idées ? \\[0pt]
L'inconscient est-il dans l'âme ou dans le corps ? \\[0pt]
L'inconscient est-il une découverte ou une invention ? \\[0pt]
L'interprétation est-elle un art ou une science ? \\[0pt]
L'interprétation : mode d'être ou mode de connaissance ? \\[0pt]
Naît-on sujet ou le devient-on ? \\[0pt]
Peut-on affirmer qu'être, c'est être perçu ou percevoir ? \\[0pt]
Peut-on être plus ou moins libre ? \\[0pt]
Phénomène ou fait social ? \\[0pt]
Primitif ou premier ? \\[0pt]
Prince : nature ou fonction ? \\[0pt]
Punir ou soigner ? \\[0pt]
Qu'est-ce qui est le plus à craindre, l'ordre ou le désordre ? \\[0pt]
Qu'est-ce qu'un grand homme ou une grande femme ? \\[0pt]
Ressent-on ou apprécie-t-on l'art ? \\[0pt]
Sciences humaines ou sciences sociales ? \\[0pt]
Tout énoncé est-il nécessairement vrai ou faux ? \\[0pt]
Tout ou rien \\[0pt]
Vaut-il mieux oublier ou pardonner ? \\[0pt]
Vaut-il mieux subir ou commettre l'injustice ? \\[0pt]
« Vivre libre ou mourir » \\[0pt]
Y a-t-il continuité ou discontinuité entre la pensée mythique et la science ? \\[0pt]
Y a-t-il une ou des morales ? \\[0pt]
Y a-t-il une ou plusieurs philosophies ? \\[0pt]
Y a-t-il une science ou des sciences ? \\[0pt]


\subsection{Citation}
\label{sec:org51c3a6b}

\noindent
« À chacun sa morale » \\[0pt]
« À chacun sa vérité » \\[0pt]
« Aime, et fais ce que tu veux » \\[0pt]
« Aimer » se dit-il en un seul sens ? \\[0pt]
« Aimez vos ennemis » \\[0pt]
« À l'impossible nul n'est tenu » \\[0pt]
« À l'impossible, nul n'est tenu » \\[0pt]
« Après moi, le déluge » \\[0pt]
« À quelque chose malheur est bon » \\[0pt]
« Aux armes citoyens ! » \\[0pt]
« Bienheureuse faute » \\[0pt]
« Ceci » \\[0pt]
« Ce ne sont que des mots » \\[0pt]
« C'est en forgeant que l'on devient forgeron » \\[0pt]
« C'est humain » \\[0pt]
« C'est la vie » \\[0pt]
« C'est plus fort que moi » \\[0pt]
« C'est pour ton bien » \\[0pt]
« C'est scientifique » \\[0pt]
« C'est tout un art » \\[0pt]
« Chacun ses goûts » \\[0pt]
« Changer le monde » \\[0pt]
« Comment peut-on être persan ? » \\[0pt]
« Connais-toi toi-même » \\[0pt]
« Dans un bois aussi courbe que celui dont l'homme est fait on ne peut rien tailler de tout à fait droit » \\[0pt]
« De la musique avant toute chose » \\[0pt]
De quoi « humain » est-il le nom ? \\[0pt]
« Des goûts et des couleurs, on ne dispute pas » \\[0pt]
« Deviens ce que tu es » \\[0pt]
« Deviens qui tu es » \\[0pt]
« Dieu est mort » \\[0pt]
Dire « je » \\[0pt]
« Droits de\ldots{} » et « droits à\ldots{} » \\[0pt]
« Du passé, faisons table rase » \\[0pt]
En morale, peut-on dire : « C'est l'intention qui compte » ? \\[0pt]
En quel sens peut-on appeler les sciences humaines « sciences de l'esprit » ? \\[0pt]
En quel sens peut-on dire qu'« on expérimente avec sa raison » ? \\[0pt]
En quel sens peut-on parler d'une « culture technique » ? \\[0pt]
Est-il bien vrai qu'« on n'arrête pas le progrès » ? \\[0pt]
Est-il justifié de parler de « corps social » ? \\[0pt]
Est-il vrai qu'en science, « rien n'est donné, tout est construit » ? \\[0pt]
« Et pourtant elle tourne » \\[0pt]
« Être contre » \\[0pt]
« Être négatif » \\[0pt]
« Être » se dit-il en plusieurs sens ? \\[0pt]
« Être soi-même », cela a-t-il un sens ? \\[0pt]
« Expliquer les faits sociaux par des faits sociaux » \\[0pt]
« Faire la paix » \\[0pt]
« Il faudrait rester des années entières pour contempler une telle œuvre » \\[0pt]
« Il faut de tout pour faire un monde » \\[0pt]
« Il ne lui manque que la parole » \\[0pt]
« Il n'y a pas de fait, il n'y a que des interprétations » \\[0pt]
« Il y a un temps pour tout » \\[0pt]
« J'ai le droit » \\[0pt]
« Je mens » \\[0pt]
« Je m'en veux » \\[0pt]
« Je n'ai pas voulu cela » \\[0pt]
« Je ne crois que ce que je vois » \\[0pt]
« Je ne l'ai pas fait exprès » \\[0pt]
« Je ne voulais pas cela » : en quoi les sciences humaines permettent-elles de comprendre cette excuse ? \\[0pt]
« Je préfère une injustice à un désordre » \\[0pt]
« La beauté est dans l'œil de celui qui regarde » \\[0pt]
« La crainte est le commencement de la sagesse » \\[0pt]
« La critique est aisée » \\[0pt]
« La logique » ou bien « les logiques » ? \\[0pt]
« La nature fait bien les choses » \\[0pt]
La question « qu'est-ce que l'homme ? » est-elle scientifique ? \\[0pt]
La question : « qui ? » \\[0pt]
La question « qui suis-je » admet-elle une réponse exacte ? \\[0pt]
« La science ne pense pas » \\[0pt]
« La valeur travail » \\[0pt]
« La vie des formes » \\[0pt]
« La vie est une scène » \\[0pt]
« La vie est un songe » \\[0pt]
« La vieillesse est un naufrage » \\[0pt]
« La vraie morale se moque de la morale » \\[0pt]
Le « je ne sais quoi » \\[0pt]
« L'enfer est pavé de bonnes intentions » \\[0pt]
« Les bons comptes font les bons amis » \\[0pt]
« Le seul problème philosophique vraiment sérieux, c'est le suicide » \\[0pt]
« Les faits, rien que les faits » \\[0pt]
« Les faits sont là » \\[0pt]
Les « forces de l'ordre » \\[0pt]
« Les miracles de la technique » \\[0pt]
« Les paroles s'envolent, les écrits restent » \\[0pt]
« L'État, c'est moi » \\[0pt]
L'État est-il un « monstre froid » ? \\[0pt]
Le « terrain » \\[0pt]
« Le travail rend libre » \\[0pt]
L'expression « perdre son temps » a-t-elle un sens ? \\[0pt]
« L'histoire jugera » \\[0pt]
« L'histoire jugera » : quel sens faut-il accorder à cette expression ? \\[0pt]
« L'homme est la mesure de toute chose » \\[0pt]
« L'homme est la mesure de toutes choses » \\[0pt]
« Liberté, égalité, fraternité » \\[0pt]
L'idée de « nature » n'est-elle qu'un mythe ? \\[0pt]
L'idée de « sciences exactes » \\[0pt]
« L'insociable sociabilité » \\[0pt]
« Machinalement » \\[0pt]
« Malheur aux vaincus » \\[0pt]
« Ne fais pas à autrui ce que tu ne voudrais pas qu'on te fasse » \\[0pt]
« Ni Dieu ni maître » \\[0pt]
« Ni Dieu, ni maître ! » \\[0pt]
« Nos amis les animaux » \\[0pt]
« Nul n'est censé ignorer la loi » \\[0pt]
« Œil pour œil, dent pour dent » \\[0pt]
« On n'arrête pas le progrès » \\[0pt]
« Par hasard » \\[0pt]
« Pas de liberté pour les ennemis de la liberté » \\[0pt]
« Pas de liberté pour les ennemis de la liberté » ? \\[0pt]
« Pauvre bête » \\[0pt]
« Penser, c'est dire non » \\[0pt]
« Petites causes, grands effets » \\[0pt]
Peut-on parler de « nature humaine » ? \\[0pt]
Peut-on parler de « travail intellectuel » ? \\[0pt]
« Plus vite, plus haut, plus fort » \\[0pt]
« Pourquoi » \\[0pt]
Pourquoi parler de « sciences exactes » ? \\[0pt]
Pourquoi parle-t-on d'une « société civile » ? \\[0pt]
Pourquoi vouloir devenir « comme maîtres et possesseurs de la nature » ? \\[0pt]
« Prendre ses désirs pour des réalités » \\[0pt]
Puis-je dire « ceci est mon corps » ? \\[0pt]
Quel est l'objet des « sciences de l'esprit » ? \\[0pt]
« Quelle chimère est-ce donc que l'homme » ? \\[0pt]
Quelle est l'unité du « je » ? \\[0pt]
Quelle valeur donner à la notion de « corps social » ? \\[0pt]
« Quelle vanité que la peinture » \\[0pt]
Quel sens donner à l'expression « gagner sa vie » ? \\[0pt]
« Que nul n'entre ici s'il n'est géomètre » \\[0pt]
Que penser de l'adage : « Que la justice s'accomplisse, le monde dût-il périr » (Fiat justitia pereat mundus) ? \\[0pt]
Que penser de la formule : « il faut suivre la nature » ? \\[0pt]
Que peut signifier « avoir le sens de la réalité » ? \\[0pt]
Que peut signifier : « gérer son temps » ? \\[0pt]
Que peut-signifier « tuer le temps » ? \\[0pt]
Que signifie : « avoir de l'esprit » ? \\[0pt]
Que signifie « donner le change » ? \\[0pt]
Que signifie : « échanger des arguments » ? \\[0pt]
Que signifie l'expression : « l'histoire jugera » ? \\[0pt]
Que signifie « politiquement correct » ? \\[0pt]
Que signifier « juger en son âme et conscience » ? \\[0pt]
Que signifie : « se rendre à l'évidence » ? \\[0pt]
Qu'est-ce que « faire de la politique » ? \\[0pt]
Qu'est-ce que « faire usage de sa raison » ? \\[0pt]
Qu'est-ce que « parler le même langage » ? \\[0pt]
Qu'est-ce que « rester soi-même » ? \\[0pt]
Qu'est-ce que « se rendre maître et possesseur de la nature » ? \\[0pt]
Qu'est-ce qu'un « champ artistique » ? \\[0pt]
Qu'est-ce qu'une « expérience de pensée » ? \\[0pt]
Qu'est-ce qu'une œuvre « géniale » ? \\[0pt]
Qu'est-ce qu'une « performance » ? \\[0pt]
Qu'est-ce qu'un « être dégénéré » ? \\[0pt]
« Que va-t-il se passer ? » \\[0pt]
Que vaut le conseil : « vivez avec votre temps » ? \\[0pt]
Que vaut l'excuse : « C'est plus fort que moi » ? \\[0pt]
Que vaut l'excuse : « Je ne l'ai pas fait exprès » ? \\[0pt]
Que veut dire « essentiel » ? \\[0pt]
Que veut dire : « être cultivé » ? \\[0pt]
Que veut dire « je t'aime » ? \\[0pt]
Que veut dire : « je t'aime » ? \\[0pt]
Que veut dire : « le temps passe » ? \\[0pt]
Que veut dire l'expression « aller au fond des choses » ? \\[0pt]
Que veut dire « réel » ? \\[0pt]
Que veut dire « respecter la nature » ? \\[0pt]
Que veut dire : « respecter la nature » ? \\[0pt]
Que veut-on dire quand on dit que « rien n'est sans raison » ? \\[0pt]
Que veut-on dire quand on dit « rien n'est sans raison » ? \\[0pt]
Qui est autorisé à me dire « tu dois » ? \\[0pt]
« Qui ne dit mot consent » \\[0pt]
Qui parle quand je dis « je » ? \\[0pt]
Qui peut me dire « tu ne dois pas » ? \\[0pt]
« Qui suis-je ? » \\[0pt]
« Rester soi-même » \\[0pt]
« Rien de ce qui est humain ne m'est étranger » \\[0pt]
« Rien de nouveau sous le soleil » \\[0pt]
« Rien n'est sans raison » \\[0pt]
« Rien n'est simple » \\[0pt]
« Sans l'ombre d'un doute » \\[0pt]
« Sans titre » \\[0pt]
« Sauver les apparences » \\[0pt]
« Sauver les phénomènes » \\[0pt]
« Se trouver » a-t-il un sens ? \\[0pt]
« Si tu veux la paix, prépare la guerre » \\[0pt]
« Sois naturel » : est-ce un bon conseil ? \\[0pt]
« Sois toi-même ! » : un impératif absurde ? \\[0pt]
« Toute peine mérite salaire » \\[0pt]
« Tout est politique » \\[0pt]
« Tout est possible » \\[0pt]
« Tout est relatif » \\[0pt]
« Tradition n'est pas raison » \\[0pt]
« Trop beau pour être vrai » \\[0pt]
« Tu dois, donc tu peux » \\[0pt]
« Tu ne tueras point » \\[0pt]
« Une main de fer dans un gant de velours » \\[0pt]
Un État peut-il être « voyou » ? \\[0pt]
« Un instant d'éternité » \\[0pt]
« Vis caché » \\[0pt]
« Vivre libre ou mourir » \\[0pt]
Y a-t-il à la question « qu'est-ce que l'homme ? » une réponse scientifique ? \\[0pt]
Y a-t-il un sens à se dire « citoyen du monde » ? \\[0pt]
\end{document}
