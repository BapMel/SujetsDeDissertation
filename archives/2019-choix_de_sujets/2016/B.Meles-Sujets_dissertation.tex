\documentclass[a4paper,11pt]{article}

\usepackage[francais]{babel}
\usepackage[utf8]{inputenc}
\usepackage[T1]{fontenc}
\usepackage{lmodern} 
\usepackage{multicol} 

\author{Baptiste Mélès}
\title{Sujets de dissertation}
\date{Janvier 2016}


% Mes guillemets
\usepackage[babel]{csquotes}
\MakeAutoQuote{«}{»}

%\newenvironment{liste}{\begin{multicols}{2}\begin{itemize}}{\end{itemize}\end{multicols}}
\newenvironment{liste}{\begin{itemize}}{\end{itemize}}

\begin{document}
\maketitle

\section{L'art}

\begin{liste}
  \item Pourquoi conserver les œuvres d'art~?
  \item L'art imite-t-il la nature~?
  \item L'éducation esthétique
  \item L'inspiration
  \item L'artiste sait-il ce qu'il fait~?
  \item L'art et la morale
  \item Le plaisir esthétique suppose-t-il une culture~?
  \item La virtuosité
  \item Qu'est-ce qu'une œuvre ratée~?
  \item Y a-t-il un progrès en art~?
  \item Le génie
  \item Le mauvais goût
  \item Arts de l'espace et arts du temps
  \item L'art engagé
  \item La pluralité des arts
  \item La vérité de l'œuvre d'art
\end{liste}

\section{Logique et épistémologie}

\begin{liste}
\item Mécanisme et finalité
\item Le symbolisme mathématique
\item Le hasard n'est-il que la mesure de notre ignorance~?
\item Comment choisir entre plusieurs hypothèses~?
\item La logique nous apprend-elle quelque chose sur le langage
  ordinaire~?
\item La causalité
\item Sauver les phénomènes
\item Les genres naturels
\item Qu'est-ce qu'un nombre~?
\item La cohérence est-elle un critère de vérité~?
\item Des événements aléatoires peuvent-ils obéir à des lois~?
\item L'intuition en mathématiques
\item La contradiction
\item La logique a-t-elle une histoire~?
\item Y a-t-il plusieurs logiques~?
\item La méthode
\item Savoir et pouvoir
\end{liste}


\section{La métaphysique}

\begin{liste}
\item L'impossible
\item L'être et le temps
\item Y a-t-il une connaissance métaphysique~?
\item Seul le présent existe-t-il~?
\item N'y a-t-il qu'un seul monde~?
\item L'existence se démontre-t-elle~?
\item Avons-nous une âme~?
\item L'infinité du monde
\item Que prouvent les preuves de l'existence de Dieu~?
\item Le virtuel et le réel
\item Le réel est-il rationnel~?
\item Être et être pensé
\item Penser sans corps
\item Le miracle
\item Logique et métaphysique
\item Dieu a-t-il pu vouloir le mal~?
\end{liste}


\section{La morale}

\begin{liste}
\item Sommes-nous responsables de notre passé~?
\item Le repentir
\item Peut-on conclure de l'être au devoir-être~?
\item L'intolérable
\item Le péché
\item La beauté morale
\item Peut-on vouloir le mal~?
\item La morale peut-elle être fondée sur la science~?
\item Y a-t-il un devoir d'être heureux~?
\item La morale peut-elle se passer d'un fondement religieux~?
\item La moralité n'est-elle que dressage~?
\item La morale peut-elle être un calcul~?
\item Le moi est-il haïssable~?
\end{liste}


\section{La politique}

\begin{liste}
\item Guerre et politique
\item La rationalité politique
\item Qu'est-ce qu'un contre-pouvoir~?
\item Le totalitarisme
\item Que faut-il savoir pour gouverner~?
\item Le législateur
\item Le respect des institutions
\item Les droits de l'homme sont-ils une abstraction~?
\item La meilleure constitution
\item A-t-on des droits contre l'État~?
\item Qu'est-ce qu'un programme politique~?
\item Y a-t-il des erreurs en politique~?
\end{liste}


\section{Les sciences humaines}

\begin{liste}
\item Histoire et ethnologie
\item Les sciences humaines permettent-elles de comprendre la vie d'un
  homme~?
\item Les sciences humaines sont-elles dangereuses~?
\item Expliquer et comprendre
\item Qu'est-ce qui rend l'objectivité difficile dans les sciences
  humaines~?
\item sciences humaines et philosophie
\item L'efficacité thérapeutique de la psychanalyse
\item La psychanalyse est-elle une science~?
\item Sciences humaines et liberté sont-elles compatibles~?
\item Y a-t-il une causalité historique~?
\item L'objectivité de l'historien
\item L'arbitraire du signe
\item Machines et mémoire
\item Les sciences humaines permettent-elles d'affiner la notion de
  responsabilité~?
\item L'économie a-t-elle des lois~?
\item L'argent
\item Y a-t-il un inconscient collectif~? 
\end{liste}

\end{document}
