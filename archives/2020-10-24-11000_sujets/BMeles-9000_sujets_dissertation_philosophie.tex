% Created 2020-10-24 sam. 14:26
\documentclass[a4paper,12pt]{article}
\usepackage[utf8]{inputenc}
\usepackage[T1]{fontenc}
\usepackage{fixltx2e}
\usepackage{graphicx}
\usepackage{longtable}
\usepackage{float}
\usepackage{wrapfig}
\usepackage{rotating}
\usepackage[normalem]{ulem}
\usepackage{amsmath}
\usepackage{textcomp}
\usepackage{marvosym}
\usepackage{wasysym}
\usepackage{amssymb}
\usepackage{hyperref}
\tolerance=1000
\usepackage[frenchb]{babel}
\usepackage[frenchb]{babel}
\usepackage{lmodern}
\DeclareUnicodeCharacter{00A0}{~}
\DeclareUnicodeCharacter{200B}{}
\author{Baptiste Mélès}
\date{24 octobre 2020}
\title{11 000 sujets de dissertation de philosophie}
\hypersetup{
  pdfkeywords={},
  pdfsubject={},
  pdfcreator={Emacs 24.5.1 (Org mode 8.2.10)}}
\begin{document}

\maketitle

On trouvera ici, classés par ordre alphabétique, quelque 11 000 sujets
de dissertation de philosophie.

Cette liste est la compilation de tous les sujets donnés aux épreuves
écrites et orales de plus de 80 concours :
\begin{enumerate}
\item agrégation externe (2008-2020) ;
\item agrégation interne (2009-2019) ;
\item CAPES externe (2010-2020) ;
\item CAPES interne (2011-2019) ;
\item École normale supérieure de Paris, concours A​/​L (2002-2020) ;
\item École normale supérieure de Paris, concours B​/​L (2002-2020).
\end{enumerate}

\bigskip

\noindent
2+2 = 4 \\
2+2 pourrait-il ne pas être égal à 4 ? \\
Abolir la propriété \\
Abstraire, est-ce se couper du réel ? \\
Abuser du pouvoir \\
À chacun sa morale \\
À chacun son dû \\
Acteurs sociaux et usages sociaux \\
Action et contemplation \\
Action et événement \\
Action et production \\
Activité et passivité \\
Affirmer et nier \\
Agir \\
Agir et faire \\
Agir et réagir \\
Agir justement fait-il de moi un homme juste ? \\
Agir moralement, est-ce lutter contre ses idées ? \\
Agir sans raison \\
Aide-toi, le ciel t'aidera \\
Ai-je des devoirs envers moi-même ? \\
Ai-je un corps ou suis-je mon corps ? \\
Ai-je un corps ? \\
Ai-je une âme ? \\
Aimer ce qui est beau \\
Aimer, est-ce vraiment connaître ? \\
Aimer la nature \\
Aimer la vie \\
Aimer les lois \\
Aimer peut-il être un devoir ? \\
Aimer ses proches \\
Aimer son prochain \\
Aimer son prochain comme soi-même \\
Aimer une œuvre d'art \\
Aime ton prochain comme toi-même \\
À l'impossible nul n'est tenu \\
Ami et ennemi \\
Amitié et société \\
Amour et amitié \\
Amour et inconscient \\
Analyse et synthèse \\
Analyser \\
Analyser les mœurs \\
Animal politique ou social ? \\
Anomalie et anomie \\
Anthropologie et ontologie \\
Anthropologie et politique \\
Apparaître \\
Apparence et réalité \\
Appartenons-nous à une culture ? \\
Apprend-on à aimer ? \\
Apprend-on à être artiste ? \\
Apprend-on à penser ? \\
Apprend-on à percevoir ? \\
Apprend-on à voir ? \\
Apprendre à gouverner \\
Apprendre à parler \\
Apprendre à penser \\
Apprendre à philosopher \\
Apprendre à vivre \\
Apprendre à voir \\
Apprendre et enseigner \\
Apprendre s'apprend-il ? \\
Apprentissage et conditionnement \\
Après-coup \\
Après moi le déluge \\
A priori et a posteriori \\
À quelle expérience l'art nous convie-t-il ? \\
À quelles conditions une démarche est-elle scientifique ? \\
À quelles conditions une expérience est-elle possible ? \\
À quelles conditions une hypothèse est-elle scientifique ? \\
À quelles conditions un énoncé est-il doué de sens ? \\
À quoi bon discuter ? \\
À quoi bon imiter la nature ? \\
À quoi bon les sciences humaines et sociales ? \\
À quoi bon penser la fin du monde ? \\
À quoi bon voyager ? \\
À quoi bon ? \\
À quoi faut-il renoncer ? \\
À quoi juger l'action d'un gouvernement ? \\
À quoi la conscience nous donne-t-elle accès ? \\
À quoi la logique peut-elle servir dans les sciences ? \\
À quoi nos illusions tiennent-elles ? \\
À quoi reconnaît-on la vérité ? \\
À quoi reconnaît-on qu'une expérience est scientifique ? \\
À quoi reconnaît-on qu'une théorie est scientifique ? \\
À quoi reconnaît-on qu'un événement est historique ? \\
À quoi reconnaît-on une œuvre d'art ? \\
À quoi reconnaît-on une religion ? \\
À quoi reconnaît-on un être vivant ? \\
À quoi sert la dialectique ? \\
À quoi sert la négation ? \\
À quoi sert la technique ? \\
À quoi sert l'écriture ? \\
À quoi sert l'État ? \\
À quoi servent les mythes ? \\
À quoi servent les preuves de l'existence de Dieu ? \\
À quoi servent les religions ? \\
À quoi servent les sciences ? \\
À quoi servent les utopies ? \\
À quoi tient la fermeté du vouloir ? \\
À quoi tient la force de l'État ? \\
À quoi tient la force des religions ? \\
À quoi tient la vérité d'une interprétation ? \\
Argent et liberté \\
Argumenter \\
Argumenter et démontrer \\
Art et apparences \\
Art et authenticité \\
Art et beauté \\
Art et connaissance \\
Art et création \\
Art et critique \\
Art et divertissement \\
Art et émotion \\
Art et finitude \\
Art et folie \\
Art et forme \\
Art et illusion \\
Art et image \\
Art et imagination \\
Art et interdit \\
Art et jeu \\
Art et marchandise \\
Art et matière \\
Art et mélancolie \\
Art et mémoire \\
Art et métaphysique \\
Art et morale \\
Art et politique \\
Art et pouvoir \\
Art et propagande \\
Art et religieux \\
Art et religion \\
Art et représentation \\
Art et société \\
Art et Société \\
Art et symbole \\
Art et technique \\
Art et transgression \\
Art et vérité \\
Artiste et artisan \\
Art populaire et art savant \\
Arts de l'espace et arts du temps \\
A-t-on besoin de certitudes ? \\
A-t-on besoin de fonder la connaissance ? \\
A-t-on besoin de spécialistes en politique ? \\
A-t-on besoin d'experts ? \\
A-t-on besoin d'un chef ? \\
A-t-on des devoirs envers soi-même ? \\
A-t-on des droits contre l'État ? \\
A-t-on des raisons de croire ce qu'on croit ? \\
A-t-on des raisons de croire ? \\
A-t-on intérêt à tout savoir ? \\
A-t-on le droit de mentir ? \\
A-t-on le droit de résister ? \\
A-t-on le droit de se révolter ? \\
A-t-on le droit de s'évader ? \\
A-t-on l'obligation de pardonner ? \\
Attente et espérance \\
Au-delà \\
Au-delà de la nature ? \\
Au nom de qui rend-on justice ? \\
Au nom de quoi le plaisir serait-il condamnable ? \\
Au nom de quoi rend-on justice ? \\
Aussitôt dit, aussitôt fait \\
Autorité et pouvoir \\
Autorité et souveraineté \\
Autrui \\
Autrui, est-ce n'importe quel autre ? \\
Autrui est-il aimable ? \\
Autrui est-il inconnaissable ? \\
Autrui est-il mon semblable ? \\
Autrui est-il pour moi un mystère ? \\
Autrui est-il un autre moi-même ? \\
Autrui est-il un autre moi ? \\
Autrui me connaît-il mieux que moi-même ? \\
Autrui m'est-il étranger ? \\
Aux armes, citoyens ! \\
Avez-vous une âme ? \\
Avoir \\
Avoir bonne conscience \\
Avoir confiance \\
Avoir de la suite dans les idées \\
Avoir de l'autorité \\
Avoir de l'esprit \\
Avoir de l'expérience \\
Avoir des ennemis \\
Avoir des principes \\
Avoir des valeurs \\
Avoir du goût \\
Avoir du jugement \\
Avoir du métier \\
Avoir du pouvoir \\
Avoir du style \\
Avoir le choix \\
Avoir le sens de la situation \\
Avoir le sens du devoir \\
Avoir le temps \\
Avoir mauvaise conscience \\
Avoir peur \\
Avoir peur des mots \\
Avoir raison \\
Avoir un corps \\
Avoir un destin \\
Avoir une bonne mémoire \\
Avoir une idée \\
Avons-nous à apprendre des images ? \\
Avons-nous besoin d'amis ? \\
Avons-nous besoin de cérémonies ? \\
Avons-nous besoin de héros ? \\
Avons-nous besoin de maîtres ? \\
Avons-nous besoin de métaphysique ? \\
Avons-nous besoin de partis politiques ? \\
Avons-nous besoin de rêver ? \\
Avons-nous besoin de spectacles ? \\
Avons-nous besoin de traditions ? \\
Avons-nous besoin d'experts en matière d'art ? \\
Avons-nous besoin d'un libre arbitre ? \\
Avons-nous besoin d'utopies ? \\
Avons-nous des devoirs à l'égard de la vérité ? \\
Avons-nous des devoirs envers la nature ? \\
Avons-nous des devoirs envers les animaux ? \\
Avons-nous des devoirs envers les autres êtres vivants ? \\
Avons-nous des devoirs envers les générations futures ? \\
Avons-nous des devoirs envers les morts ? \\
Avons-nous des devoirs envers le vivant ? \\
Avons-nous des devoirs envers nous-mêmes ? \\
Avons-nous des droits sur la nature ? \\
Avons-nous des raisons d'espérer ? \\
Avons-nous intérêt à la liberté d'autrui ? \\
Avons-nous le devoir d'être heureux ? \\
Avons-nous le devoir de vivre ? \\
Avons-nous le droit de juger autrui ? \\
Avons-nous le droit d'être heureux ? \\
Avons-nous peur de la liberté ? \\
Avons-nous raison d'exiger toujours des raisons ? \\
Avons-nous un corps ? \\
Avons-nous un devoir de vérité ? \\
Avons-nous un droit au droit ? \\
Avons-nous une âme ? \\
Avons-nous une identité ? \\
Avons-nous une intuition du temps ? \\
Avons-nous une obligation envers les générations à venir ? \\
Avons-nous une responsabilité envers le passé ? \\
Avons-nous un libre arbitre ? \\
Avons-nous un monde commun ? \\
Axiomatiser, est-ce fonder ? \\
À chacun selon son mérite \\
À chacun ses goûts \\
À chacun son dû \\
À l'impossible nul n'est tenu \\
À quelle condition un travail est-il humain ? \\
À quelles conditions est-il acceptable de travailler ? \\
À quelles conditions le vivant peut-il être objet de science ? \\
À quelles conditions peut-on dire qu'une action est historique ? \\
À quelles conditions un choix peut-il être rationnel ? \\
À quelles conditions une démarche est-elle scientifique ? \\
À quelles conditions une explication est-elle scientifique ? \\
À quelles conditions une hypothèse est-elle scientifique ? \\
À quelles conditions une pensée est-elle libre ? \\
À quelles conditions une théorie est-elle scientifique ? \\
À quelles conditions une théorie peut-elle être scientifique ? \\
À quelles conditions un jugement est-il moral ? \\
À quelque chose malheur est bon \\
À quels signes reconnaît-on la vérité ? \\
À qui devons-nous obéir ? \\
À qui dois-je la vérité ? \\
À qui doit-on le respect ? \\
À qui doit-on obéir ? \\
À qui est mon corps ? \\
À qui faut-il obéir ? \\
À qui la faute ? \\
À qui profite le crime ? \\
À qui profite le travail ? \\
À quoi bon avoir mauvaise conscience ? \\
À quoi bon critiquer les autres ? \\
À quoi bon démontrer ? \\
À quoi bon les regrets ? \\
À quoi bon les romans ? \\
À quoi bon raconter des histoires ? \\
À quoi bon se parler ? \\
À quoi bon voyager ? \\
À quoi est-il impossible de s'habituer ? \\
À quoi faut-il être fidèle ? \\
À quoi la perception donne-t-elle accès ? \\
À quoi l'art nous rend-il sensibles ? \\
À quoi la valeur d'un homme se mesure-t-elle ? \\
À quoi peut-on reconnaître une œuvre d'art ? \\
À quoi reconnaît-on la rationalité ? \\
À quoi reconnaît-on la vérité ? \\
À quoi reconnaît-on le réel ? \\
À quoi reconnaît-on l'injustice ? \\
À quoi reconnaît-on qu'une activité est un travail ? \\
À quoi reconnaît-on qu'une expérience est scientifique ? \\
À quoi reconnaît-on qu'une pensée est vraie ? \\
À quoi reconnaît-on qu'une politique est juste ? \\
À quoi reconnaît-on un acte libre ? \\
À quoi reconnaît-on un bon gouvernement ? \\
À quoi reconnaît-on une attitude religieuse ? \\
À quoi reconnaît-on une bonne interprétation ? \\
À quoi reconnaît-on une idéologie ? \\
À quoi reconnaît-on une œuvre d'art ? \\
À quoi sert la connaissance du passé ? \\
À quoi sert la logique ? \\
À quoi sert la mémoire ? \\
À quoi sert la notion de contrat social ? \\
À quoi sert la notion d'état de nature ? \\
À quoi sert la technique ? \\
À quoi sert le contrat social ? \\
À quoi sert l'État ? \\
À quoi sert l'histoire ? \\
À quoi sert un exemple ? \\
À quoi servent les doctrines morales ? \\
À quoi servent les élections ? \\
À quoi servent les émotions ? \\
À quoi servent les expériences ? \\
À quoi servent les fictions ? \\
À quoi servent les images ? \\
À quoi servent les lois ? \\
À quoi servent les machines ? \\
À quoi servent les mythes ? \\
À quoi servent les preuves ? \\
À quoi servent les règles ? \\
À quoi servent les statistiques ? \\
À quoi servent les symboles ? \\
À quoi servent les théories ? \\
À quoi servent les utopies ? \\
À quoi servent les voyages ? \\
À quoi tenons-nous ? \\
À quoi tient la force des religions ? \\
À quoi tient l'autorité ? \\
À quoi tient la valeur d'une pensée ? \\
À quoi tient le pouvoir des mots ? \\
À quoi tient notre humanité ? \\
À science nouvelle, nouvelle philosophie ? \\
À t-on le droit de faire tout ce qui est permis par la loi ? \\
Bâtir un monde \\
Beauté et moralité \\
Beauté et vérité \\
Beauté naturelle et beauté artistique \\
Beauté réelle, beauté idéale \\
Besoin et désir \\
Besoins et désirs \\
Bêtise et méchanceté \\
Bien agir, est-ce toujours être moral ? \\
Bien commun et bien public \\
Bien commun et intérêt particulier \\
Bien jouer son rôle \\
Bien juger \\
Bien parler \\
Bonheur de chacun bonheur de tous \\
Bonheur et autarcie \\
Bonheur et satisfaction \\
Bonheur et société \\
Bonheur et technique \\
Bonheur et vertu \\
Calculer \\
Calculer et penser \\
Calculer et raisonner \\
Cartographier \\
Castes et classes \\
Catégories de langue, catégories de pensée \\
Catégories de l'être, catégories de langue \\
Catégories de pensée, catégories de langue \\
Catégories logiques et catégories linguistiques \\
Cause et condition \\
Cause et effet \\
Cause et loi \\
Cause et raison \\
Causes et motivations \\
Causes et raisons \\
Causes premières et causes secondes \\
Ce que je pense est-il nécessairement vrai ? \\
Ce que la morale autorise, l'État peut-il légitimement l'interdire ? \\
Ce que la technique rend possible, peut-on jamais en empêcher la réalisation ? \\
Ce que sait le poète \\
Ce qui dépasse la raison est-il nécessairement irréel ? \\
Ce qui dépend de moi \\
Ce qui est démontré est-il nécessairement vrai ? \\
Ce qui est faux est-il dénué de sens ? \\
Ce qui est ordinaire est-il normal ? \\
Ce qui est subjectif est-il arbitraire ? \\
Ce qui est vrai est-il toujours vérifiable ? \\
Ce qui fut et ce qui sera \\
Ce qu'il y a \\
Ce qui n'a pas de prix \\
Ce qui ne peut s'acheter est-il dépourvu de valeur ? \\
Ce qui n'est pas démontré peut-il être vrai ? \\
Ce qui n'est pas matériel peut-il être réel ? \\
Ce qui n'est pas réel est-il impossible ? \\
Ce qui passe et ce qui demeure \\
Ce qui vaut en théorie vaut-il toujours en pratique ? \\
Ce qu'on ne peut pas vendre \\
Certaines œuvres d'art ont-elles plus de valeur que d'autres ? \\
Certitude et conviction \\
Certitude et probabilité \\
Certitude et vérité \\
Cesser d'espérer \\
C'est pour ton bien \\
C'est trop beau pour être vrai ! \\
Ceux qui savent doivent-ils gouverner ? \\
Chance et bonheur \\
Changer \\
Changer d'opinion \\
Changer, est-ce devenir un autre ? \\
Changer la vie \\
Changer le monde \\
Changer ses désirs plutôt que l'ordre du monde \\
Change-t-on avec le temps ? \\
Chaque science porte-t-elle une métaphysique qui lui est propre ? \\
Châtier, est ce faire honneur au criminel ? \\
Chercher ses mots \\
Chercher son intérêt, est-ce être immoral ? \\
Choisir \\
Choisir, est-ce renoncer ? \\
Choisir ses souvenirs ? \\
Choisissons-nous qui nous sommes ? \\
Choisit-on ses souvenirs ? \\
Choisit-on son corps ? \\
Choix et raison \\
Chose et objet \\
Chose et personne \\
Choses et personnes \\
Cinéma et réalité \\
Cité juste ou citoyen juste ? \\
Citoyen du monde ? \\
Citoyen et soldat \\
Civilisation et barbarie \\
Civilisé, barbare, sauvage \\
Classer \\
Classer et ordonner \\
Classes et histoire \\
Classicisme et romantisme \\
Colère et indignation \\
Collectionner \\
Commander \\
Commémorer \\
Commencer \\
Commencer en philosophie \\
Comment assumer les conséquences de ses actes ? \\
Comment autrui peut-il m'aider à rechercher le bonheur ? \\
Comment bien vivre ? \\
Comment chercher ce qu'on ignore ? \\
Comment comprendre les faits sociaux ? \\
Comment comprendre une croyance qu'on ne partage pas ? \\
Comment conduire ses pensées ? \\
Comment connaître nos devoirs ? \\
Comment croire au progrès ? \\
Comment décider, sinon à la majorité ? \\
Comment définir la raison ? \\
Comment définir la signification \\
Comment deux personnes peuvent-elles partager la même pensée ? \\
Comment devient-on artiste ? \\
Comment devient-on raisonnable ? \\
Comment dire la vérité ? \\
Comment dire l'individuel ? \\
Comment distinguer désirs et besoins ? \\
Comment distinguer entre l'amour et l'amitié ? \\
Comment distinguer l'amour de l'amitié ? \\
Comment distinguer le rêvé du perçu ? \\
Comment distinguer le vrai du faux ? \\
Comment établir des critères d'équité ? \\
Comment être naturel ? \\
Comment évaluer la qualité de la vie ? \\
Comment expliquer les phénomènes mentaux ? \\
Comment exprimer l'identité ? \\
Comment fonder la propriété ? \\
Comment juger de la justesse d'une interprétation ? \\
Comment juger de la politique ? \\
Comment juger d'une œuvre d'art ? \\
Comment justifier l'autonomie des sciences de la vie ? \\
Comment la science progresse-t-elle ? \\
Comment le passé peut-il demeurer présent ? \\
Comment l'erreur est-elle possible ? \\
Comment les sociétés changent-elles ? \\
Comment l'homme peut-il se représenter le temps ? \\
Comment mesurer une sensation ? \\
Comment mesurer ? \\
Comment ne pas être humaniste ? \\
Comment ne pas être libéral ? \\
Comment penser la diversité des langues ? \\
Comment penser le hasard ? \\
Comment penser le mouvement ? \\
Comment penser l'éternel ? \\
Comment penser un pouvoir qui ne corrompe pas ? \\
Comment percevons-nous l'espace ? \\
Comment peut-on choisir entre différentes hypothèses ? \\
Comment peut-on définir la politique ? \\
Comment peut-on définir un être vivant ? \\
Comment peut-on être heureux ? \\
Comment peut-on être sceptique ? \\
Comment peut-on se trahir soi-même ? \\
Comment puis-je devenir ce que je suis ? \\
Comment reconnaît-on une œuvre d'art ? \\
Comment reconnaît-on un vivant ? \\
Comment retrouver la nature ? \\
Comment sait-on qu'on se comprend ? \\
Comment sait-on qu'une chose existe ? \\
Comment savoir que l'on est dans l'erreur ? \\
Comment se mettre à la place d'autrui ? \\
Comment s'entendre ? \\
Comment s'orienter dans la pensée ? \\
Comment traiter les animaux ? \\
Comment trancher une controverse ? \\
Comment vivre ensemble ? \\
Comment voyager dans le temps ? \\
Comme on dit \\
Communauté, collectivité, société \\
Communauté et société \\
Communiquer \\
Communiquer et enseigner \\
Comparaison n'est pas raison \\
Comparer les cultures \\
Compatir \\
Compétence et autorité \\
Composer avec les circonstances \\
Composition et construction \\
Comprendre \\
Comprendre autrui \\
Comprendre, est-ce interpréter ? \\
Comprendre le réel est-ce le dominer ? \\
Comprendre le sens d'un texte \\
Comprendre l'inconscient \\
Comprendre une démonstration \\
Compter sur soi \\
Concept et image \\
Concept et intuition \\
Concept et métaphore \\
Conception et perception \\
Concevoir et juger \\
Conclure \\
Concurrence et égalité \\
Conduire sa vie \\
Conduire ses pensées \\
Conflit et démocratie \\
Conflit et liberté \\
Connaissance commune et connaissance scientifique \\
Connaissance, croyance, conjecture \\
Connaissance de soi et conscience de soi \\
Connaissance du futur et connaissance du passé \\
Connaissance et croyance \\
Connaissance et expérience \\
Connaissance et perception \\
Connaissance historique et action politique \\
Connaissons-nous la réalité des choses ? \\
Connaissons-nous la réalité telle qu'elle est ? \\
Connaissons-nous mieux le présent que le passé ? \\
Connais-toi toi-même \\
Connaît-on la vie ou bien connaît-on le vivant ? \\
Connaît-on la vie ou connaît-on le vivant ? \\
Connaît-on la vie ou le vivant ? \\
Connaît-on les choses telles qu'elles sont ? \\
Connaître autrui \\
Connaître, est-ce connaître par les causes ? \\
Connaître est-ce découvrir le réel ? \\
Connaître, est-ce dépasser les apparences ? \\
Connaître et comprendre \\
Connaître et penser \\
Connaître la vie ou le vivant ? \\
Connaître l'infini \\
Connaître par les causes \\
Connaître ses limites \\
Connaître ses origines \\
Conquérir \\
Conscience de soi et amour de soi \\
Conscience de soi et connaissance de soi \\
Conscience et attention \\
Conscience et connaissance \\
Conscience et conscience de soi \\
Conscience et existence \\
Conscience et liberté \\
Conscience et mémoire \\
Conscience et responsabilité \\
Conscience et subjectivité \\
Conscience et volonté \\
Conseiller le prince \\
Consensus et conflit \\
Conservatisme et tradition \\
Considère-t-on jamais le temps en lui-même ? \\
Consistance et précarité \\
Constitution et lois \\
Construire l'espace \\
Consumérisme et démocratie \\
Contemplation et distraction \\
Contempler \\
Contingence et nécessité \\
Continuité et discontinuité \\
Contradiction et opposition \\
Contrainte et désobéissance \\
Contrainte et obligation \\
Contrôle et vigilance \\
Convaincre et persuader \\
Convention et observation \\
Conventions sociales et moralité \\
Conviction et certitude \\
Conviction et responsabilité \\
Convient-il d'opposer explication et interprétation ? \\
Corps et espace \\
Corps et esprit \\
Corps et identité \\
Corps et matière \\
Corps et nature \\
Correspondre \\
Crainte et espoir \\
Création et fabrication \\
Création et production \\
Créativité et contrainte \\
Créer \\
Créer et produire \\
Crime et châtiment \\
Crise et progrès \\
Critiquer \\
Critiquer la démocratie \\
Croire aux fictions \\
Croire en Dieu \\
Croire, est-ce être faible ? \\
Croire, est-ce obéir ? \\
Croire, est-ce renoncer au savoir ? \\
Croire et savoir \\
Croire pour savoir \\
Croire que Dieu existe, est-ce croire en lui ? \\
Croire savoir \\
Croit-on ce que l'on veut ? \\
Croit-on comme on veut ? \\
Croyance et certitude \\
Croyance et choix \\
Croyance et connaissance \\
Croyance et probabilité \\
Croyance et vérité \\
Culpabilité et responsabilité \\
Cultes et rituels \\
Cultivons notre jardin \\
Culture et civilisation \\
Culture et communauté \\
Culture et conscience \\
Culture et différence \\
Culture et éducation \\
Culture et langage \\
Culture et savoir \\
Culture et technique \\
Culture et violence \\
Dans l'action, est-ce l'intention qui compte ? \\
Dans quel but les hommes se donnent-ils des lois ? \\
Dans quelle mesure est-on l'auteur de sa propre vie ? \\
Dans quelle mesure l'art est-il un fait social ? \\
Dans quelle mesure le temps nous appartient-il ? \\
Dans quelle mesure toute philosophie est-elle critique du langage ? \\
Débattre et dialoguer \\
Déchiffrer \\
Décider \\
Décomposer les choses \\
Découverte et invention \\
Découverte et invention dans les sciences \\
Découverte et justification \\
Découvrir \\
Décrire \\
Décrire, est-ce déjà expliquer ? \\
Déduction et expérience \\
Défendre son honneur \\
Définir \\
Définir, est-ce déterminer l'essence ? \\
Définir l'art : à quoi bon ? \\
Définir la vérité, est-ce la connaître ? \\
Définition et démonstration \\
Définition nominale et définition réelle \\
Définitions, axiomes, postulats \\
Déjouer \\
Délibérer, est-ce être dans l'incertitude ? \\
De l'utilité des voyages \\
Dématérialiser \\
Démêler le vrai du faux \\
Démériter \\
Démocrates et démagogues \\
Démocratie ancienne et démocratie moderne \\
Démocratie et anarchie \\
Démocratie et démagogie \\
Démocratie et impérialisme \\
Démocratie et opinion \\
Démocratie et religion \\
Démocratie et représentation \\
Démocratie et république \\
Démocratie et transparence \\
Démonstration et argumentation \\
Démonstration et déduction \\
Démontrer, argumenter, expérimenter \\
Démontrer est-il le privilège du mathématicien ? \\
Démontrer et argumenter \\
Démontrer par l'absurde \\
Dénaturer \\
Dépasser les apparences ? \\
Dépasser l'humain \\
Dépend-il de soi d'être heureux ? \\
De quel bonheur sommes-nous capables ? \\
De quel droit l'État exerce-t-il un pouvoir ? \\
De quel droit punit-on ? \\
De quel droit ? \\
De quelle certitude la science est-elle capable ? \\
De quelle liberté témoigne l'œuvre d'art ? \\
De quelle réalité nos perceptions témoignent-elles ? \\
De quelle réalité témoignent nos perceptions ? \\
De quelle science humaine la folie peut-elle être l'objet ? \\
De quelle vérité l'art est-il capable ? \\
De quelle vérité l'opinion est-elle capable ? \\
De quoi a-t-on conscience lorsqu'on a conscience de soi ? \\
De quoi avons-nous besoin ? \\
De quoi avons-nous vraiment besoin ? \\
De quoi dépend le bonheur ? \\
De quoi dépend notre bonheur ? \\
De quoi doute un sceptique ? \\
De quoi est fait mon présent ? \\
De quoi est-fait notre présent ? \\
De quoi est-on conscient ? \\
De quoi est-on malheureux ? \\
De quoi la forme est-elle la forme ? \\
De quoi la logique est-elle la science ? \\
De quoi la philosophie est-elle le désir ? \\
De quoi l'art nous délivre-t-il ? \\
De quoi la vérité libère-t-elle ? \\
De quoi le devoir libère-t-il ? \\
De quoi les logiciens parlent-ils ? \\
De quoi les métaphysiciens parlent-ils ? \\
De quoi les sciences humaines nous instruisent-elles ? \\
De quoi l'État doit-il être propriétaire ? \\
De quoi l'État ne doit-il pas se mêler ? \\
De quoi l'expérience esthétique est-elle l'expérience ? \\
De quoi n'avons-nous pas conscience ? \\
De quoi ne peut-on pas répondre ? \\
De quoi parlent les mathématiques ? \\
De quoi parlent les théories physiques ? \\
De quoi pâtit-on ? \\
De quoi peut-il y avoir science ? \\
De quoi peut-on être inconscient ? \\
De quoi peut-on faire l'expérience ? \\
De quoi pouvons-nous être sûrs ? \\
De quoi puis-je répondre ? \\
De quoi rit-on ? \\
De quoi somme-nous prisonniers ? \\
De quoi sommes-nous coupables ? \\
De quoi sommes-nous responsables ? \\
De quoi suis-je inconscient ? \\
De quoi suis-je responsable ? \\
De quoi y a-t-il expérience ? \\
De quoi y a-t-il histoire ? \\
Déraisonner \\
Déraisonner, est-ce perdre de vue le réel ? \\
Désacraliser \\
Des comportements économiques peuvent-ils être rationnels ? \\
Description et explication \\
Des événements aléatoires peuvent-ils obéir à des lois ? \\
Des inégalités peuvent-elles être justes ? \\
Désintérêt et désintéressement \\
Désirer \\
Désirer, est-ce être aliéné ? \\
Désirer et vouloir \\
Désir et besoin \\
Désir et bonheur \\
Désir et interdit \\
Désir et langage \\
Désir et manque \\
Désire-t-on la reconnaissance ? \\
Désir et ordre \\
Désir et politique \\
Désir et pouvoir \\
Désir et raison \\
Désir et réalité \\
Désir et volonté \\
Des lois justes suffisent-elles à assurer la justice ? \\
Des motivations peuvent-elles être sociales ? \\
Des nations peuvent-elles former une société ? \\
Désobéir \\
Désobéir aux lois \\
Désobéissance et résistance \\
Des peuples sans histoire \\
Des sociétés sans État sont-elles des sociétés politiques ? \\
Déterminisme et responsabilité \\
Déterminisme psychique et déterminisme physique \\
Détruire et construire \\
Détruire pour reconstruire \\
Devant qui est-on responsable ? \\
Devant qui sommes-nous responsables ? \\
Devenir autre \\
Devenir citoyen \\
Devenir et évolution \\
Devient-on raisonnable ? \\
Devoir et bonheur \\
Devoir et conformisme \\
Devoir et contrainte \\
Devoir et intérêt \\
Devoir et liberté \\
Devoir et plaisir \\
Devoir et prudence \\
Devoir et vertu \\
Devoirs envers les autres et devoirs envers soi-même \\
Devoirs et passions \\
Devons-nous dire la vérité ? \\
Devons-nous nous libérer de nos désirs ? \\
Devons-nous quelque chose à la nature ? \\
Devons-nous tenir certaines connaissances pour acquises ? \\
Devons-nous vivre comme si nous ne devions jamais mourir ? \\
Dialectique et Philosophie \\
Dialogue et délibération en démocratie \\
Dialoguer \\
Dieu aurait-il pu mieux faire ? \\
Dieu est-il mortel ? \\
Dieu est-il une invention humaine ? \\
Dieu est-il une limite de la pensée ? \\
Dieu est mort \\
Dieu et César \\
Dieu pense-t-il ? \\
Dieu peut-il tout faire ? \\
Dieu, prouvé ou éprouvé ? \\
Dieu tout-puissant \\
Dire ce qui est \\
Dire, est-ce faire ? \\
Dire et exprimer \\
Dire et faire \\
Dire et montrer \\
Dire je \\
Dire le monde \\
Dire l'individuel \\
Dire oui \\
Dire « je » \\
Diriger son esprit \\
Discrimination et revendication \\
Discussion et conversation \\
Discussion et dialogue \\
Disposer de son corps \\
Distinguer \\
Division du travail et cohésion sociale \\
Documents et monuments \\
Dogme et opinion \\
Dois-je mériter mon bonheur ? \\
Dois-je obéir à la loi ? \\
Doit-on apprendre à percevoir ? \\
Doit-on apprendre à vivre ? \\
Doit-on bien juger pour bien faire ? \\
Doit-on changer ses désirs, plutôt que l'ordre du monde ? \\
Doit-on chasser les artistes de la cité ? \\
Doit-on corriger les inégalités sociales ? \\
Doit-on croire en l'humanité ? \\
Doit-on distinguer devoir moral et obligation sociale ? \\
Doit-on identifier l'âme à la conscience ? \\
Doit-on interpréter les rêves ? \\
Doit-on justifier les inégalités ? \\
Doit-on le respect au vivant ? \\
Doit-on mûrir pour la liberté ? \\
Doit-on rechercher le bonheur ? \\
Doit-on rechercher l'harmonie ? \\
Doit-on refuser d'interpréter ? \\
Doit-on répondre de ce qu'on est devenu ? \\
Doit-on respecter la nature ? \\
Doit-on respecter les êtres vivants ? \\
Doit-on se faire l'avocat du diable ? \\
Doit-on se justifier d'exister ? \\
Doit-on se mettre à la place d'autrui ? \\
Doit-on se passer des utopies ? \\
Doit-on souffrir de n'être pas compris ? \\
Doit-on tenir le plaisir pour une fin ? \\
Doit-on toujours dire la vérité ? \\
Doit-on toujours rechercher la vérité ? \\
Doit-on tout accepter de l'État ? \\
Doit-on tout attendre de l'État ? \\
Doit-on tout calculer ? \\
Doit-on tout contrôler ? \\
Dominer la nature \\
Don et échange \\
Don Juan \\
Donner \\
Donner à chacun son dû \\
Donner, à quoi bon ? \\
Donner des exemples \\
Donner des preuves \\
Donner des raisons \\
Donner du sens \\
Donner et recevoir \\
Donner raison \\
Donner raison, rendre raison \\
Donner sa parole \\
Donner son assentiment \\
Donner une représentation \\
Donner un exemple \\
D'où la politique tire-t-elle sa légitimité ? \\
Doute et raison \\
Douter \\
D'où viennent les concepts ? \\
D'où viennent les idées générales ? \\
D'où viennent les préjugés ? \\
D'où viennent nos idées ? \\
D'où vient aux objets techniques leur beauté ? \\
D'où vient la certitude dans les sciences ? \\
D'où vient la certitude ? \\
D'où vient la servitude ? \\
D'où vient la signification des mots ? \\
D'où vient le mal ? \\
D'où vient le plaisir de lire ? \\
D'où vient que l'histoire soit autre chose qu'un chaos ? \\
Dressage et éducation \\
Droit et coutume \\
Droit et démocratie \\
Droit et devoir \\
Droit et devoir sont-ils liés ? \\
Droit et morale \\
Droit et protection \\
Droit et violence \\
Droit naturel et loi naturelle \\
Droits de l'homme et droits du citoyen \\
Droits de l'homme ou droits du citoyen ? \\
Droits et devoirs \\
Droits et devoirs sont-ils réciproques ? \\
Droits, garanties, protection \\
Du passé pouvons-nous faire table rase ? \\
Durée et instant \\
Durer \\
Échange et don \\
Échange et partage \\
Échange et valeur \\
Échanger \\
Échanger des idées \\
Échanger, est-ce créer de la valeur ? \\
Échanger, est-ce partager ? \\
Échanger, est-ce risquer ? \\
Éclairer \\
Économie et politique \\
Économie et société \\
Économie politique et politique économique \\
Écouter \\
Écouter et entendre \\
Écrire \\
Écrire et parler \\
Écrire l'histoire \\
Éducation de l'homme, éducation du citoyen \\
Éducation et instruction \\
Éduquer et instruire \\
Éduquer le citoyen \\
Efficacité et justice \\
Égalité des droits, égalité des conditions \\
Égalité et différence \\
Égalité et solidarité \\
Égoïsme et altruisme \\
Égoïsme et individualisme \\
Égoïsme et méchanceté \\
Empirique et expérimental \\
Enfance et moralité \\
En finir avec les préjugés \\
En histoire, tout est-il affaire d'interprétation ? \\
En morale, est-ce seulement l'intention qui compte ? \\
En politique, faut-il refuser l'utopie ? \\
En politique, nécessité fait loi \\
En politique, ne faut-il croire qu'aux rapports de force ? \\
En politique n'y a-t-il que des rapports de force ? \\
En politique, peut-on faire table rase du passé ? \\
En politique, y a-t-il des modèles ? \\
En quel sens la métaphysique a-t-elle une histoire ? \\
En quel sens la métaphysique est-elle une science ? \\
En quel sens l'anthropologie peut-elle être historique ? \\
En quel sens les sciences ont-elles une histoire ? \\
En quel sens l'État est-il rationnel ? \\
En quel sens le vivant a-t-il une histoire ? \\
En quel sens parler de lois de la pensée ? \\
En quel sens parler de structure métaphysique ? \\
En quel sens parler d'identité culturelle ? \\
En quel sens peut-on dire que la vérité s'impose ? \\
En quel sens peut-on dire que le mal n'existe pas ? \\
En quel sens peut-on dire que l'homme est un animal politique ? \\
En quel sens peut-on dire qu' « on expérimente avec sa raison » ? \\
En quel sens peut-on parler de la mort de l'art ? \\
En quel sens peut-on parler de la vie sociale comme d'un jeu ? \\
En quel sens peut-on parler de transcendance ? \\
En quel sens peut-on parler d'expérience possible ? \\
En quel sens peut-on parler d'une culture technique ? \\
En quel sens peut-on parler d'une interprétation de la nature ? \\
En quel sens une œuvre d'art est-elle un document ? \\
Enquêter \\
En quoi la connaissance de la matière peut-elle relever de la métaphysique ? \\
En quoi la connaissance du vivant contribue-t-elle à la connaissance de l'homme ? \\
En quoi la justice met-elle fin à la violence ? \\
En quoi la matière s'oppose-t-elle à l'esprit ? \\
En quoi la méthode est-elle un art de penser ? \\
En quoi la nature constitue-t-elle un modèle ? \\
En quoi la patience est-elle une vertu ? \\
En quoi la physique a-t-elle besoin des mathématiques ? \\
En quoi l'art peut-il intéresser le philosophe ? \\
En quoi la sociologie est-elle fondamentale ? \\
En quoi la technique fait-elle question ? \\
En quoi le bien d'autrui m'importe-t-il ? \\
En quoi le bonheur est-il l'affaire de l'État ? \\
En quoi le langage est-il constitutif de l'homme ? \\
En quoi les hommes restent-ils des enfants ? \\
En quoi les sciences humaines nous éclairent-elles sur la barbarie ? \\
En quoi les sciences humaines sont-elles normatives ? \\
En quoi les vivants témoignent-ils d'une histoire ? \\
En quoi l'œuvre d'art donne-t-elle à penser ? \\
En quoi une discussion est-elle politique ? \\
En quoi une insulte est-elle blessante ? \\
En quoi une œuvre d'art est-elle moderne ? \\
Enseigner \\
Enseigner, est-ce transmettre un savoir ? \\
Enseigner et éduquer \\
Enseigner, instruire, éduquer \\
Enseigner l'art \\
Entendement et raison \\
Entendre \\
Entendre raison \\
Entre l'art et la nature, qui imite l'autre ? \\
Entre l'opinion et la science, n'y a-t-il qu'une différence de degré ? \\
Entrer en scène \\
Énumérer \\
Épistémologie générale et épistémologie des sciences particulières \\
Éprouver sa valeur \\
Erreur et faute \\
Erreur et illusion \\
Espace et représentation \\
Espace et structure sociale \\
Espace mathématique et espace physique \\
Espace public et vie privée \\
Esprit et intériorité \\
Essayer \\
Essence et existence \\
Est-ce à la fin que le sens apparaît ? \\
Est-ce à la raison de déterminer ce qui est réel ? \\
Est-ce de la force que l'État tient son autorité ? \\
Est-ce la certitude qui fait la science ? \\
Est-ce la démonstration qui fait la science ? \\
Est-ce la majorité qui doit décider ? \\
Est-ce la mémoire qui constitue mon identité ? \\
Est-ce l'autorité qui fait la loi ? \\
Est-ce le cerveau qui pense ? \\
Est-ce l'échange utilitaire qui fait le lien social ? \\
Est-ce le corps qui perçoit ? \\
Est-ce l'ignorance qui rend les hommes croyants ? \\
Est-ce l'intérêt qui fonde le lien social ? \\
Est-ce l'utilité qui définit un objet technique ? \\
Est-ce par désir de la vérité que l'homme cherche à savoir ? \\
Est-ce par son objet ou par ses méthodes qu'une science peut se définir ? \\
Est-ce pour des raisons morales qu'il faut protéger l'environnement ? \\
Est-ce seulement l'intention qui compte ? \\
Est-ce un devoir d'aimer son prochain ? \\
Esthétique et éthique \\
Esthétique et poétique \\
Esthétisme et moralité \\
Est-il bon qu'un seul commande ? \\
Est-il difficile de savoir ce que l'on veut ? \\
Est-il difficile de savoir ce qu'on veut ? \\
Est-il difficile d'être heureux ? \\
Est-il difficile de vivre en société ? \\
Est-il immoral de se rendre heureux ? \\
Est-il judicieux de revenir sur ses décisions ? \\
Est-il juste de payer l'impôt ? \\
Est-il juste d'interpréter la loi ? \\
Est-il légitime d'affirmer que seul le présent existe ? \\
Est-il légitime d'opposer liberté et nécessité ? \\
Est-il mauvais de suivre son désir ? \\
Est-il naturel à l'homme de parler ? \\
Est-il naturel de s'aimer soi-même ? \\
Est-il nécessaire d'espérer pour entreprendre ? \\
Est-il parfois bon de mentir ? \\
Est-il possible d'améliorer l'homme ? \\
Est-il possible de croire en la vie éternelle ? \\
Est-il possible de douter de tout ? \\
Est-il possible de ne croire à rien ? \\
Est-il possible de préparer l'avenir ? \\
Est-il possible de tout avoir pour être heureux ? \\
Est-il possible d'être immoral sans le savoir ? \\
Est-il possible d'être neutre politiquement ? \\
Est-il raisonnable d'aimer ? \\
Est-il raisonnable d'être rationnel ? \\
Est-il raisonnable de vouloir maîtriser la nature ? \\
Est-il toujours avantageux de promouvoir son propre intérêt ? \\
Est-il toujours meilleur d'avoir le choix ? \\
Est-il utile d'avoir mal ? \\
Est-il vrai que les animaux ne pensent pas ? \\
Est-il vrai que l'ignorant n'est pas libre ? \\
Est-il vrai que ma liberté s'arrête là où commence celle des autres ? \\
Est-il vrai que nous ne nous tenons jamais au temps présent ? \\
Est-il vrai qu'en science, « rien n'est donné, tout est construit » ? \\
Est-il vrai que plus on échange, moins on se bat ? \\
Est-il vrai qu'on apprenne de ses erreurs ? \\
Estime et respect \\
Estimer \\
Est-on fondé à distinguer la justice et le droit ? \\
Est-on l'auteur de sa propre vie ? \\
Est-on le produit d'une culture ? \\
Est-on libre de ne pas vouloir ce que l'on veut ? \\
Est-on libre face à la vérité ? \\
Est-on responsable de ce qu'on n'a pas voulu ? \\
Est-on responsable de l'avenir de l'humanité \\
Est-on responsable de son passé ? \\
Est-on sociable par nature ? \\
Établir la vérité, est-ce nécessairement démontrer ? \\
État et institutions \\
État et nation \\
État et société \\
État et Société \\
État et société civile \\
Éternité et immortalité \\
Éthique et authenticité \\
Éthique et esthétique \\
Éthique et Morale \\
Ethnologie et cinéma \\
Ethnologie et ethnocentrisme \\
Ethnologie et sociologie \\
Étonnement et sidération \\
Être acteur \\
Être affairé \\
Être à l'écoute de son désir, est-ce nier le désir de l'autre ? \\
Être aliéné \\
Être au monde \\
Être bon juge \\
Être cause de soi \\
Être, c'est agir \\
Être chez soi \\
Être citoyen \\
Être citoyen du monde \\
Être compris \\
Être conscient de soi, est-ce être maître de soi ? \\
Être conscient, est-ce être maître de soi ? \\
Être conséquent avec soi-même \\
Être content de soi \\
Être cultivé, est-ce tout connaître ? \\
Être cultivé rend-il meilleur ? \\
Être cynique \\
Être dans l'esprit \\
Être dans le temps \\
Être dans son bon droit \\
Être dans son droit \\
Être de mauvaise humeur \\
Être de son temps \\
Être déterminé \\
Être dogmatique \\
Être égal à soi-même \\
Être en bonne santé \\
Être en désaccord \\
Être en règle avec soi-même \\
Être ensemble \\
Être équitable \\
Être est-ce agir ? \\
Être et apparaître \\
Être et avoir \\
Être et avoir été \\
Être et devenir \\
Être et devoir être \\
Être et devoir-être \\
Être et être pensé \\
Être et exister \\
Être et ne plus être \\
Être et paraître \\
Être et penser, est-ce la même chose ? \\
Être et représentation \\
Être et sens \\
Être exemplaire \\
Être heureux \\
Être heureux, est-ce devoir ? \\
Être hors de soi \\
Être impossible \\
Être juge et partie \\
Être là \\
Être l'entrepreneur de soi-même \\
Être libre, est-ce dire non ? \\
Être libre est-ce faire ce que l'on veut ? \\
Être libre, est-ce n'obéir qu'à soi-même ? \\
Être libre, est-ce pouvoir choisir ? \\
Être libre, est-ce se suffire à soi-même ? \\
Être libre, même dans les fers \\
Être logique \\
Être logique avec soi-même \\
Être maître de soi \\
Être majeur \\
Être malade \\
Être matérialiste \\
Être méchant \\
Être méchant volontairement \\
Être mère \\
Être moderne \\
Être né \\
Être ou avoir \\
Être ou ne pas être \\
Être ou ne pas être, est-ce la question ? \\
Être par soi \\
Être pauvre \\
Être père \\
Être précurseur \\
Être quelqu'un \\
Être raisonnable, est-ce accepter la réalité telle qu'elle est ? \\
Être raisonnable, est-ce renoncer à ses désirs ? \\
Être réaliste \\
Être relativiste \\
Être sans cause \\
Être sans cœur \\
Être sans scrupule \\
Être sceptique \\
Être seul avec sa conscience \\
Être seul avec soi-même \\
Être soi \\
Être soi-même \\
Être spectateur \\
Être spirituel \\
Être systématique \\
Être un artiste \\
Être un corps \\
Être une chose qui pense \\
Être un sujet, est-ce être maître de soi ? \\
Être vertueux \\
Être, vie et pensée \\
Étudier \\
Évidence et certitude \\
Évidence et raison \\
Évidence et vérité \\
Évidences et préjugés \\
Évolution biologique et culture \\
Évolution et progrès \\
Évolution et révolution \\
Excuser et pardonner \\
Existence et contingence \\
Existence et essence \\
Exister \\
Exister, est-ce simplement vivre ? \\
Existe-t-il de faux besoins ? \\
Existe-t-il des choses en soi ? \\
Existe-t-il des choses sans prix ? \\
Existe-t-il des croyances collectives ? \\
Existe-t-il des désirs coupables ? \\
Existe-t-il des devoirs envers soi-même ? \\
Existe-t-il des dilemmes moraux ? \\
Existe-t-il des questions sans réponse ? \\
Existe-t-il des sciences de différentes natures ? \\
Existe-t-il des signes naturels ? \\
Existe-t-il un art de penser ? \\
Existe-t-il un bien commun qui soit la norme de la vie politique ? \\
Existe-t-il un droit de mentir ? \\
Existe-t-il une méthode pour rechercher la vérité ? \\
Existe-t-il une méthode pour trouver la vérité ? \\
Existe-t-il une opinion publique ? \\
Existe-t-il un vocabulaire neutre des droits fondamentaux ? \\
Expérience esthétique et sens commun \\
Expérience et approximation \\
Expérience et expérimentation \\
Expérience et habitude \\
Expérience et interprétation \\
Expérience et phénomène \\
Expérience et vérité \\
Expérience, expérimentation \\
Expérience immédiate et expérimentation scientifique \\
Expérimentation et vérification \\
Expérimenter \\
Explication et prévision \\
Expliquer \\
Expliquer, est-ce interpréter ? \\
Expliquer et comprendre \\
Expliquer et interpréter \\
Expliquer et justifier \\
Expression et création \\
Expression et signification \\
Extension et compréhension \\
Fabriquer et créer \\
Faire apprendre \\
Faire ce que l'on dit \\
Faire ce qu'on dit \\
Faire comme si \\
Faire confiance \\
Faire corps \\
Faire de la métaphysique, est-ce se détourner du monde ? \\
Faire de la politique \\
Faire de nécessité vertu \\
Faire de sa vie une œuvre d'art \\
Faire des choix \\
Faire douter \\
Faire école \\
Faire et laisser faire \\
Faire justice \\
Faire la loi \\
Faire la morale \\
Faire la paix \\
Faire la part des choses \\
Faire la révolution \\
Faire le mal \\
Faire l'histoire \\
Faire son devoir \\
Faire son devoir, est-ce là toute la morale ? \\
Faire table rase \\
Faire une expérience \\
Faire voir \\
Faisons-nous l'histoire ? \\
Fait et essence \\
Fait et fiction \\
Fait et preuve \\
Fait et théorie \\
Fait et valeur \\
Fait-on de la politique pour changer les choses ? \\
Faits et preuves \\
Faits et valeurs \\
Famille et tribu \\
Familles, je vous hais \\
Faudrait-il ne rien oublier ? \\
Faudrait-il vivre sans passion ? \\
Faut-avoir peur de la technique ? \\
Faut-il accepter sa condition ? \\
Faut-il accorder de l'importance aux mots ? \\
Faut-il accorder l'esprit aux bêtes ? \\
Faut-il affirmer son identité ? \\
Faut-il aimer autrui pour le respecter ? \\
Faut-il aimer la vie ? \\
Faut-il aimer son prochain comme soi-même ? \\
Faut-il aimer son prochain ? \\
Faut-il aller au-delà des apparences ? \\
Faut-il aller toujours plus vite ? \\
Faut-il apprendre à être libre ? \\
Faut-il apprendre à vivre en renonçant au bonheur ? \\
Faut-il apprendre à voir ? \\
Faut-il avoir des ennemis ? \\
Faut-il avoir des principes ? \\
Faut-il avoir peur de la liberté ? \\
Faut-il avoir peur de la nature ? \\
Faut-il avoir peur de la technique ? \\
Faut-il avoir peur des habitudes ? \\
Faut-il avoir peur des machines ? \\
Faut-il avoir peur d'être libre ? \\
Faut-il avoir peur du désordre ? \\
Faut-il changer le monde ? \\
Faut-il changer ses désirs plutôt que l'ordre du monde ? \\
Faut-il chasser les poètes ? \\
Faut-il chercher à satisfaire tous nos désirs ? \\
Faut-il chercher à se connaître ? \\
Faut-il chercher la paix à tout prix ? \\
Faut-il chercher le bonheur à tout prix ? \\
Faut-il chercher un sens à l'histoire ? \\
Faut-il choisir entre être heureux et être libre ? \\
Faut-il concilier les contraires ? \\
Faut-il condamner la fiction ? \\
Faut-il condamner la rhétorique ? \\
Faut-il condamner le luxe ? \\
Faut-il condamner les illusions ? \\
Faut-il connaître l'Histoire pour gouverner ? \\
Faut-il considérer le droit pénal comme instituant une violence légitime ? \\
Faut-il considérer les faits sociaux comme des choses ? \\
Faut-il contrôler les mœurs ? \\
Faut-il craindre la mort ? \\
Faut-il craindre la révolution ? \\
Faut-il craindre le développement des techniques ? \\
Faut-il craindre le pire ? \\
Faut-il craindre le regard d'autrui ? \\
Faut-il craindre les foules ? \\
Faut-il craindre les machines ? \\
Faut-il craindre les masses ? \\
Faut-il craindre l'État ? \\
Faut-il craindre l'ordre ? \\
Faut-il croire au progrès ? \\
Faut-il croire en la science ? \\
Faut-il croire en quelque chose ? \\
Faut-il croire les historiens ? \\
Faut-il croire que l'histoire a un sens ? \\
Faut-il défendre la démocratie ? \\
Faut-il défendre l'ordre à tout prix ? \\
Faut-il défendre ses convictions \\
Faut-il dépasser les apparences ? \\
Faut-il désespérer de l'humanité ? \\
Faut-il des frontières ? \\
Faut-il des héros ? \\
Faut-il désirer la vérité ? \\
Faut-il des outils pour penser ? \\
Faut-il détruire l'État ? \\
Faut-il détruire pour créer ? \\
Faut-il dire de la justice qu'elle n'existe pas ? \\
Faut-il dire tout haut ce que les autres pensent tout bas ? \\
Faut-il diriger l'économie ? \\
Faut-il distinguer ce qui est de ce qui doit être ? \\
Faut-il distinguer désir et besoin ? \\
Faut-il distinguer esthétique et philosophie de l'art ? \\
Faut-il donner un sens à la souffrance ? \\
Faut-il douter de ce qu'on ne peut pas démontrer ? \\
Faut-il douter de l'évidence \\
Faut-il du passé faire table rase ? \\
Faut-il enfermer ? \\
Faut-il espérer pour agir ? \\
Faut-il être à l'écoute du corps ? \\
Faut-il être bon ? \\
Faut-il être cohérent ? \\
Faut-il être connaisseur pour apprécier une œuvre d'art ? \\
Faut-il être cosmopolite ? \\
Faut-il être fidèle à soi-même ? \\
Faut-il être idéaliste ? \\
Faut-il être libre pour être heureux ? \\
Faut-il être logique avec soi-même ? \\
Faut-il être mesuré en toutes choses ? \\
Faut-il être modéré ? \\
Faut-il être objectif ? \\
Faut-il être original ? \\
Faut-il être positif ? \\
Faut-il être pragmatique ? \\
Faut-il être réaliste en politique ? \\
Faut-il être réaliste ? \\
Faut-il expliquer la morale par son utilité ? \\
Faut-il faire confiance au progrès technique ? \\
Faut-il faire de nécessité vertu ? \\
Faut-il faire table rase du passé ? \\
Faut-il forcer les gens à participer à la vie politique ? \\
Faut-il fuir la politique ? \\
Faut-il garder ses illusions ? \\
Faut-il hiérarchiser les désirs ? \\
Faut-il hiérarchiser les formes de vie ? \\
Faut-il imaginer que nous sommes heureux ? \\
Faut-il imposer la vérité ? \\
Faut-il interpréter la loi ? \\
Faut-il laisser parler la nature ? \\
Faut-il libérer l'humanité du travail ? \\
Faut-il limiter la souveraineté de l'État ? \\
Faut-il limiter la souveraineté ? \\
Faut-il limiter le pouvoir de l'État ? \\
Faut-il limiter l'exercice de la puissance publique ? \\
Faut-il lire des romans ? \\
Faut-il ménager les apparences ? \\
Faut-il mépriser le luxe ? \\
Faut-il mieux vivre comme si nous ne devions jamais mourir ? \\
Faut-il ne manquer de rien pour être heureux ? \\
Faut-il n'être jamais méchant ? \\
Faut-il obéir à la voix de sa conscience ? \\
Faut-il opposer à la politique la souveraineté du droit ? \\
Faut-il opposer histoire et mémoire ? \\
Faut-il opposer la matière et l'esprit ? \\
Faut-il opposer l'art à la connaissance ? \\
Faut-il opposer la théorie et la pratique ? \\
Faut-il opposer le don et l'échange ? \\
Faut-il opposer l'État et la société ? \\
Faut-il opposer le temps vécu et le temps des choses ? \\
Faut-il opposer l'histoire et la fiction ? \\
Faut-il opposer nature et culture ? \\
Faut-il opposer raison et sensation ? \\
Faut-il opposer rhétorique et philosophie ? \\
Faut-il oublier le passé pour se donner un avenir ? \\
Faut-il parler pour avoir des idées générales ? \\
Faut-il partager la souveraineté ? \\
Faut-il penser l'État comme un corps ? \\
Faut-il perdre ses illusions ? \\
Faut-il perdre son temps ? \\
Faut-il poser des limites à l'activité rationnelle ? \\
Faut-il pour le connaître faire du vivant un objet ? \\
Faut-il préférer l'art à la nature ? \\
Faut-il préférer le bonheur à la vérité ? \\
Faut-il préférer une injustice au désordre ? \\
Faut-il prendre soin de soi ? \\
Faut-il protéger la dignité humaine ? \\
Faut-il protéger la nature ? \\
Faut-il protéger les faibles contre les forts ? \\
Faut-il que le réel ait un sens ? \\
Faut-il que les meilleurs gouvernent ? \\
Faut-il rechercher la certitude ? \\
Faut-il rechercher la simplicité ? \\
Faut-il rechercher le bonheur ? \\
Faut-il rechercher l'harmonie ? \\
Faut-il reconnaître pour connaître ? \\
Faut-il regretter l'équivocité du langage ? \\
Faut-il rejeter tous les préjugés ? \\
Faut-il rejeter toute norme ? \\
Faut-il renoncer à faire du travail une valeur ? \\
Faut-il renoncer à la certitude ? \\
Faut-il renoncer à l'idée d'âme ? \\
Faut-il renoncer à l'impossible ? \\
Faut-il renoncer à son désir ? \\
Faut-il résister à la peur de mourir ? \\
Faut-il respecter la nature ? \\
Faut-il respecter les convenances ? \\
Faut-il respecter le vivant ? \\
Faut-il rester impartial ? \\
Faut-il rester naturel ? \\
Faut-il rire ou pleurer ? \\
Faut-il rompre avec le passé ? \\
Faut-il s'adapter ? \\
Faut-il s'affranchir des désirs ? \\
Faut-il s'aimer soi-même ? \\
Faut-il sauver des vies à tout prix ? \\
Faut-il sauver les apparences ? \\
Faut-il savoir mentir ? \\
Faut-il savoir obéir pour gouverner ? \\
Faut-il savoir pour agir ? \\
Faut-il savoir prendre des risques ? \\
Faut-il se contenter de peu ? \\
Faut-il se cultiver ? \\
Faut-il se délivrer de la peur ? \\
Faut-il se délivrer des passions ? \\
Faut-il se détacher du monde ? \\
Faut-il s'efforcer d'être moins personnel ? \\
Faut-il se fier à ce que l'on ressent ? \\
Faut-il se fier à la majorité ? \\
Faut-il se fier à sa propre raison ? \\
Faut-il se fier aux apparences ? \\
Faut-il se libérer du travail ? \\
Faut-il se méfier de l'écriture ? \\
Faut-il se méfier de l'imagination ? \\
Faut-il se méfier de l'intuition ? \\
Faut-il se méfier des apparences ? \\
Faut-il se méfier de ses désirs ? \\
Faut-il se méfier du volontarisme politique ? \\
Faut-il s'en remettre à l'État pour limiter le pouvoir de l'État ? \\
Faut-il s'en tenir aux faits ? \\
Faut-il séparer la science et la technique ? \\
Faut-il séparer morale et politique ? \\
Faut-il se poser des questions métaphysiques ? \\
Faut-il se réjouir d'exister ? \\
Faut-il se rendre à l'évidence ? \\
Faut-il se ressembler pour former une société ? \\
Faut-il suivre ses intuitions ? \\
Faut-il surmonter son enfance ? \\
Faut-il tolérer les intolérants ? \\
Faut-il toujours avoir raison ? \\
Faut-il toujours dire la vérité ? \\
Faut-il toujours être en accord avec soi-même ? \\
Faut-il toujours éviter de se contredire ? \\
Faut-il toujours faire son devoir ? \\
Faut-il toujours garder espoir ? \\
Faut-il tout critiquer ? \\
Faut-il tout démontrer ? \\
Faut-il tout interpréter ? \\
Faut-il un commencement à tout ? \\
Faut-il un corps pour penser ? \\
Faut-il une guerre pour mettre fin à toutes les guerres ? \\
Faut-il une théorie de la connaissance ? \\
Faut-il vaincre ses désirs plutôt que l'ordre du monde ? \\
Faut-il vivre avec son temps ? \\
Faut-il vivre comme si l'on ne devait jamais mourir ? \\
Faut-il vivre comme si nous étions immortels ? \\
Faut-il vivre comme si nous ne devions jamais mourir ? \\
Faut-il vivre comme si on ne devait jamais mourir ? \\
Faut-il vivre dangereusement ? \\
Faut-il vivre hors de la société pour être heureux ? \\
Faut-il voir pour croire ? \\
Faut-il vouloir changer le monde ? \\
Faut-il vouloir être heureux ? \\
Faut-il vouloir la paix de l'âme ? \\
Faut-il vouloir la paix ? \\
Faut-il vouloir la transparence ? \\
Fiction et virtualité \\
Foi et bonne foi \\
Foi et raison \\
Foi et savoir \\
Foi et superstition \\
Folie et raison \\
Folie et société \\
Fonction et prédicat \\
Fonder \\
Fonder la justice \\
Fonder une cite \\
Fonder une cité \\
Force et violence \\
Forcer à être libre \\
Forger des hypothèses \\
Formaliser et axiomatiser \\
Forme et contenu \\
Forme et matière \\
Forme et rythme \\
Former et éduquer \\
Former les esprits \\
Forme-t-on son esprit en transformant la matière ? \\
Fuir la civilisation \\
Gagner \\
Garder la mesure \\
Génie et technique \\
Genre et espèce \\
Gérer et gouverner \\
Gouvernement des hommes et administration des choses \\
Gouvernement et société \\
Gouverner \\
Gouverner, administrer, gérer \\
Gouverner, est-ce prévoir ? \\
Gouverner, est-ce régner ? \\
Gouverner et se gouverner \\
Grammaire et métaphysique \\
Grammaire et philosophie \\
Grandeur et décadence \\
Groupe, classe, société \\
Guérir \\
Guerre et politique \\
Guerres justes et injustes \\
Habiter \\
Habiter le monde \\
Habiter sur la terre \\
Haïr \\
Haïr la raison \\
Hasard et destin \\
Hériter \\
Hésiter \\
Hier a-t-il plus de réalité que demain ? \\
Histoire et anthropologie \\
Histoire et devenir \\
Histoire et écriture \\
Histoire et ethnologie \\
Histoire et fiction \\
Histoire et géographie \\
Histoire et mémoire \\
Histoire et morale \\
Histoire et politique \\
Histoire et progrès \\
Histoire et structure \\
Histoire et violence \\
Histoire individuelle et histoire collective \\
Homo religiosus \\
Honte, pudeur, embarras \\
Humour et ironie \\
Hypothèse et vérité \\
Ici et maintenant \\
Idéal et utopie \\
Idée et réalité \\
Identité et changement \\
Identité et communauté \\
Identité et différence \\
Identité et égalité \\
Identité et indiscernabilité \\
Ignorer \\
Illégalité et injustice \\
Illusion et apparence \\
Il y a \\
Image et concept \\
Image et idée \\
Image, signe, symbole \\
Imaginaire et politique \\
Imagination et conception \\
Imagination et culture \\
Imagination et pouvoir \\
Imagination et raison \\
Imaginer \\
Imitation et création \\
Imitation et identification \\
Imitation et représentation \\
Imiter \\
Imiter, est-ce copier ? \\
Incertitude et action \\
Inconscient et déterminisme \\
Inconscient et identité \\
Inconscient et inconscience \\
Inconscient et instinct \\
Inconscient et langage \\
Inconscient et liberté \\
Inconscient et mythes \\
Indépendance et autonomie \\
Indépendance et liberté \\
Individualisme et égoïsme \\
Individuation et identité \\
Individu et citoyen \\
Individu et communauté \\
Individu et société \\
Infini et indéfini \\
Information et communication \\
Information et opinion \\
Innocence et ignorance \\
Innocenter le devenir \\
Instinct et morale \\
Instruction et éducation \\
Instruire et éduquer \\
Intentions, plans et stratégies \\
Interdire et prohiber \\
Intérêt général et bien commun \\
Interprétation et création \\
Interpréter \\
Interpréter, est-ce connaître ? \\
Interpréter, est-ce renoncer à prouver ? \\
Interpréter, est-ce savoir ? \\
Interpréter est-il subjectif ? \\
Interpréter et expliquer \\
Interpréter et formaliser dans les sciences humaines \\
Interpréter et traduire \\
Interpréter ou expliquer \\
Interpréter une œuvre d'art \\
Interprète-t-on à défaut de connaître ? \\
Interroger \\
Interroger et répondre \\
Intuition et concept \\
Intuition et déduction \\
Intuition et intellection \\
Invention et création \\
Invention et découverte \\
Invention et imitation \\
J'ai un corps \\
Je \\
Je est un autre \\
Je mens \\
Je ne l'ai pas fait exprès \\
Je sens, donc je suis \\
Je, tu, il \\
Jouer \\
Jouer son rôle \\
Jouer un rôle \\
Jouir sans entraves \\
Jugement analytique et jugement synthétique \\
Jugement de goût et jugement esthétique \\
Jugement esthétique et jugement de valeur \\
Jugement et réflexion \\
Jugement et vérité \\
Jugement moral et jugement empirique \\
Juger \\
Juger en conscience \\
Juger et connaître \\
Juger et décider \\
Juger et raisonner \\
Juger et sentir \\
Jusqu'à quel point la nature est-elle objet de science ? \\
Jusqu'à quel point pouvons-nous juger autrui ? \\
Jusqu'à quel point sommes-nous responsables de nos passions ? \\
Jusqu'à quel point suis-je mon propre maître ? \\
Jusqu'où interpréter ? \\
Jusqu'où peut-on dialoguer ? \\
Jusqu'où peut-on soigner ? \\
Justice et charité \\
Justice et égalité \\
Justice et équité \\
Justice et force \\
Justice et pardon \\
Justice et utilité \\
Justice et vengeance \\
Justice et violence \\
Justification et politique \\
Justifier \\
Justifier et prouver \\
Justifier le mensonge \\
La banalité \\
L'abandon \\
La barbarie \\
La barbarie de la technique \\
La bassesse \\
La béatitude \\
La beauté \\
La beauté a-t-elle une histoire ? \\
La beauté de la nature \\
La beauté des corps \\
La beauté des ruines \\
La beauté du diable \\
La beauté du geste \\
La beauté du monde \\
La beauté est-elle affaire de goût ? \\
La beauté est-elle dans le regard ou dans la chose vue ? \\
La beauté est-elle dans les choses ? \\
La beauté est-elle intemporelle ? \\
La beauté est-elle l'objet d'une connaissance ? \\
La beauté est-elle partout ? \\
La beauté est-elle sensible ? \\
La beauté est-elle une promesse de bonheur ? \\
La beauté et la grâce \\
La beauté idéale \\
La beauté morale \\
La beauté naturelle \\
La beauté nous rend-elle meilleurs ? \\
La beauté peut-elle délivrer une vérité ? \\
La beauté s'explique-t-elle ? \\
La belle âme \\
La belle nature \\
La bestialité \\
La bête \\
La bête et l'animal \\
La bêtise \\
La bêtise et la méchanceté sont-elles liées intrinsèquement ? \\
La bêtise et la méchanceté sont-elles liées nécessairement ? \\
La bêtise n'est-elle pas proprement humaine ? \\
La bibliothèque \\
La bienfaisance \\
La bienséance \\
La bienveillance \\
La biographie \\
La biologie peut-elle se passer de causes finales ? \\
L'abondance \\
La bonne conscience \\
La bonne éducation \\
La bonne intention \\
La bonne volonté \\
La bonté \\
L'absence \\
L'absence de fondement \\
L'absence de générosité \\
L'absence d'œuvre \\
L'absolu \\
L'absolu et le relatif \\
L'abstraction \\
L'abstraction en art \\
L'abstraction est-elle toujours utile à la science empirique ? \\
L'abstrait est-il en dehors de l'espace et du temps ? \\
L'abstrait et le concret \\
L'absurde \\
L'abus de pouvoir \\
L'académisme \\
L'académisme dans l'art \\
L'académisme et les fins de l'art \\
La calomnie \\
La casuistique \\
La catharsis \\
La causalité \\
La causalité en histoire \\
La causalité historique \\
La causalité suppose-t-elle des lois ? \\
La cause \\
La cause efficiente \\
La cause et la raison \\
La cause et l'effet \\
La cause première \\
L'accès à la vérité \\
L'accident \\
L'accidentel \\
L'accomplissement \\
L'accomplissement de soi \\
L'accord \\
La censure \\
La certitude \\
La certitude de mourir \\
La chair \\
La chance \\
La charité \\
La charité est-elle une vertu ? \\
La chasse et la guerre \\
L'achèvement de l'œuvre \\
La chose \\
La chose en soi \\
La chose publique \\
La chronologie \\
La chute \\
La circonspection \\
La citation \\
La cité \\
La cité idéale \\
La cité sans dieux \\
La citoyenneté \\
La civilisation \\
La civilité \\
La clarté \\
La classe moyenne \\
La classification \\
La classification des arts \\
La classification des sciences \\
La clause de conscience \\
La clémence \\
La coexistence des libertés \\
La cohérence \\
La cohérence est-elle la norme du vrai ? \\
La cohérence est-elle un critère de la vérité ? \\
La cohérence est-elle un critère de vérité ? \\
La cohérence est-elle une vertu ? \\
La cohérence logique est-elle une condition suffisante de la démonstration ? \\
La colère \\
La collection \\
La comédie \\
La comédie du pouvoir \\
La comédie humaine \\
La comédie sociale \\
La communauté \\
La communauté des savants \\
La communauté internationale \\
La communauté morale \\
La communauté scientifique \\
La communication \\
La communication est-elle nécessaire à la démocratie ? \\
La comparaison \\
La compassion \\
La compassion risque-t-elle d'abolir l'exigence politique ? \\
La compétence \\
La compétence technique peut-elle fonder l'autorité publique ? \\
La composition \\
La compréhension \\
La concorde \\
La concurrence \\
La condition \\
La condition de mortel \\
La condition humaine \\
La condition sociale \\
La confiance \\
La confiance en la raison \\
La confiance est-elle une vertu ? \\
La confusion \\
La connaissance adéquate \\
La connaissance animale \\
La connaissance a-t-elle des limites ? \\
La connaissance commune est-elle le point de départ de la science ? \\
La connaissance commune fait-elle obstacle à la vérité ? \\
La connaissance de Dieu \\
La connaissance de la vie \\
La connaissance de la vie se confond-elle avec celle du vivant ? \\
La connaissance de l'histoire est-elle utile à l'action ? \\
La connaissance des causes \\
La connaissance de soi \\
La connaissance des passions \\
La connaissance des principes \\
La connaissance du bien \\
La connaissance du futur \\
La connaissance du monde \\
La connaissance du passé \\
La connaissance du singulier \\
La connaissance du vivant \\
La connaissance du vivant est-elle désintéressée ? \\
La connaissance du vivant peut-elle être désintéressée  ? \\
La connaissance est-elle une contemplation ? \\
La connaissance est-elle une croyance justifiée ? \\
La connaissance et la croyance \\
La connaissance et la morale \\
La connaissance et le vivant \\
La connaissance historique \\
La connaissance historique est-elle une interprétation des faits ? \\
La connaissance historique est-elle utile à l'homme ? \\
La connaissance intuitive \\
La connaissance mathématique \\
La connaissance objective \\
La connaissance objective doit-elle s'interdire toute interprétation ? \\
La connaissance objective exclut-elle toute forme de subjectivité ? \\
La connaissance peut-elle être pratique ? \\
La connaissance peut-elle se passer de l'imagination ? \\
La connaissance scientifique \\
La connaissance scientifique abolit-elle toute croyance ? \\
La connaissance scientifique est-elle désintéressée ? \\
La connaissance scientifique n'est-elle qu'une croyance argumentée ? \\
La connaissance sensible \\
La connaissance s'interdit-elle tout recours à l'imagination ? \\
La connaissance suppose-t-elle une éthique ? \\
La conquête \\
La conquête de l'espace \\
La conscience \\
La conscience a-t-elle des degrés ? \\
La conscience a-t-elle des moments ? \\
La conscience collective \\
La conscience d'autrui est-elle impénétrable ? \\
La conscience de la mort est-elle une condition de la sagesse ? \\
La conscience de soi \\
La conscience de soi de l'art \\
La conscience de soi est-elle une donnée immédiate ? \\
La conscience de soi et l'identité personnelle \\
La conscience de soi suppose-t-elle autrui ? \\
La conscience du temps rend-elle l'existence tragique ? \\
La conscience entrave-t-elle l'action ? \\
La conscience est-elle ce qui fait le sujet ? \\
La conscience est-elle intrinsèquement morale ? \\
La conscience est-elle nécessairement malheureuse ? \\
La conscience est-elle ou n'est-elle pas ? \\
La conscience est-elle source d'illusions ? \\
La conscience est-elle toujours morale ? \\
La conscience est-elle une activité ? \\
La conscience est-elle une illusion ? \\
La conscience et l'inconscient \\
La conscience historique \\
La conscience morale \\
La conscience morale est-elle innée ? \\
La conscience morale n'est-elle que le fruit de l'éducation ? \\
La conscience morale n'est-elle que le produit de l'éducation ? \\
La conscience peut-elle être collective ? \\
La conscience peut-elle nous tromper ? \\
La conscience politique \\
La conscience universelle \\
La conséquence \\
La conservation \\
La considération de l'utilité doit-elle déterminer toutes nos actions ? \\
La consolation \\
La constance \\
La constitution \\
La contemplation \\
La contestation \\
La contingence \\
La contingence de l'existence \\
La contingence des lois de la nature \\
La contingence du futur \\
La contingence du monde \\
La contingence est-elle la condition de la liberté ? \\
La continuité \\
La contradiction \\
La contradiction réside-t-elle dans les choses ? \\
La contrainte \\
La contrainte déontologique \\
La contrainte des lois est-elle une violence ? \\
La contrainte en art \\
La contrainte peut-elle être légitime ? \\
La contrainte supprime-t-elle la responsabilité ? \\
La contrôle social \\
La controverse scientifique \\
La convalescence \\
La convention et l'arbitraire \\
La conversation \\
La conversion \\
La conviction \\
La coopération \\
La copie \\
La corruption \\
La corruption politique \\
La cosmogonie \\
La couleur \\
La courtoisie \\
La coutume \\
La crainte des Dieux \\
La crainte et l'ignorance \\
La création \\
La création artistique \\
La création dans l'art \\
La création de l'humanité \\
La création de valeur \\
La créativité \\
La crédibilité \\
La crédulité \\
La criminalité \\
La crise \\
La crise sociale \\
La critique \\
La critique d'art \\
La critique de l'État \\
La critique des théories \\
La critique du pouvoir peut-elle conduire à la désobéissance ? \\
La croissance \\
La croissance du savoir \\
La croyance \\
La croyance est-elle l'asile de l'ignorance ? \\
La croyance est-elle signe de faiblesse ? \\
La croyance est-elle une opinion comme les autres ? \\
La croyance est-elle une opinion ? \\
La croyance et la foi \\
La croyance et la raison \\
La croyance peut-elle être rationnelle ? \\
La croyance peut-elle tenir lieu de savoir ? \\
La croyance religieuse échappe-t-elle à toute logique ? \\
La croyance religieuse se distingue-t-elle des autres formes de croyance ? \\
La cruauté \\
L'acte \\
L'acte et la parole \\
L'acte et la puissance \\
L'acte et l'œuvre \\
L'acte gratuit \\
L'acteur \\
L'acteur et son rôle \\
L'action \\
L'action collective \\
L'action du temps \\
L'action et le risque \\
L'action et son contexte \\
L'action humaine nécessite-t-elle la contingence du monde ? \\
L'action intentionnelle \\
L'action politique \\
L'action politique a-t-elle un fondement rationnel ? \\
L'action politique peut-elle se passer de mots ? \\
L'activité \\
L'activité se laisse-t-elle programmer ? \\
L'actualité \\
L'actuel \\
La cuisine \\
La culpabilité \\
La culture \\
La culture artistique \\
La culture de masse \\
La culture démocratique \\
La culture d'entreprise \\
La culture est-elle affaire de politique ? \\
La culture est-elle la négation de la nature ? \\
La culture est-elle nécessaire à l'appréciation d'une œuvre d'art ? \\
La culture est-elle une question politique ? \\
La culture est-elle une seconde nature ? \\
La culture est-elle un luxe ? \\
La culture et les cultures \\
La culture garantit-elle l'excellence humaine ? \\
La culture générale \\
La culture libère-t-elle des préjugés ? \\
La culture morale \\
La culture nous rend-elle meilleurs ? \\
La culture nous rend-elle plus humains ? \\
La culture nous unit-elle ? \\
La culture peut-elle être instituée ? \\
La culture peut-elle être objet de science ? \\
La culture rend-elle plus humain ? \\
La culture savante et la culture populaire \\
La culture scientifique \\
La culture technique \\
La culture : pour quoi faire ? \\
La curiosité \\
La curiosité est-elle à l'origine du savoir ? \\
La danse \\
La danse est-elle l'œuvre du corps ? \\
La décadence \\
La décence \\
La déception \\
La décision \\
La décision a-t-elle besoin de raisons ? \\
La décision morale \\
La décision politique \\
La découverte de la vérité peut-elle être le fait du hasard ? \\
La déduction \\
La défense de la liberté \\
La défense de l'intérêt général est-il la fin dernière de la politique ? \\
La défense nationale \\
La déficience \\
La définition \\
La délibération \\
La délibération en morale \\
La délibération politique \\
La démagogie \\
La démarche scientifique exclut-elle tout recours à l'imagination ? \\
La démence \\
La démesure \\
La démocratie \\
La démocratie a-t-elle des limites ? \\
La démocratie a-t-elle une histoire ? \\
La démocratie conduit-elle au règne de l'opinion ? \\
La démocratie est-ce la fin du despotisme ? \\
La démocratie, est-ce le pouvoir du plus grand nombre ? \\
La démocratie est-elle la loi du plus fort ? \\
La démocratie est-elle le pire des régimes politiques ? \\
La démocratie est-elle le règne de l'opinion ? \\
La démocratie est-elle moyen ou fin ? \\
La démocratie est-elle nécessairement libérale ? \\
La démocratie est-elle possible ? \\
La démocratie et les experts \\
La démocratie et les institutions de la justice \\
La démocratie et le statut de la loi \\
La démocratie n'est-elle que la force des faibles ? \\
La démocratie participative \\
La démocratie peut-elle échapper à la démagogie ? \\
La démocratie peut-elle être représentative ? \\
La démocratie peut-elle se passer de représentation ? \\
La démonstration \\
La démonstration nous garantit-elle l'accès à la vérité ? \\
La démonstration obéit-elle à des lois ? \\
La démonstration supprime-t-elle le doute ? \\
La déontologie \\
La dépendance \\
La dépense \\
La déraison \\
La dérision \\
La descendance \\
La description \\
La désillusion \\
La désinvolture \\
La désobéissance \\
La désobéissance civile \\
La destruction \\
La détermination \\
La dette \\
La deuxième chance \\
La déviance \\
La dialectique \\
La dialectique est-elle une science ? \\
La dictature \\
La différence \\
La différence culturelle \\
La différence des arts \\
La différence des sexes \\
La différence des sexes est-elle une question philosophique ? \\
La différence des sexes est-elle un problème philosophique ? \\
La différence homme-femme \\
La différence sexuelle \\
La difformité \\
La dignité \\
La dignité humaine \\
La digression \\
La direction de l'esprit \\
La discipline \\
La discorde \\
La discrétion \\
La discrimination \\
La discursivité \\
La discussion \\
La disgrâce \\
La disharmonie \\
La disponibilité \\
La disposition \\
La disposition morale \\
La dispute \\
La dissidence \\
La dissimulation \\
La distance \\
La distinction \\
La distinction de genre \\
La distinction de la nature et de la culture est-elle un fait de culture ? \\
La distinction sociale \\
La distraction \\
La diversion \\
La diversité \\
La diversité des cultures \\
La diversité des langues \\
La diversité des langues est-elle une diversité des pensées ? \\
La diversité des opinions conduit-elle à douter de tout ? \\
La diversité des perceptions \\
La diversité des religions \\
La diversité des sciences \\
La diversité humaine \\
La division \\
La division de la volonté \\
La division des pouvoirs \\
La division des tâches \\
La division du travail \\
L'admiration \\
La docilité est-elle un vice ou une vertu ? \\
La domestication \\
La domination \\
La domination du corps \\
La domination sociale \\
L'adoucissement des mœurs \\
La douleur \\
La douleur est-elle utile ? \\
La douleur nous apprend-elle quelque chose ? \\
La droit de conquête \\
La droiture \\
La dualité \\
La duplicité \\
La durée \\
L'adversité \\
La faiblesse \\
La faiblesse de croire \\
La faiblesse de la démocratie \\
La faiblesse de la volonté \\
La faiblesse d'esprit \\
La familiarité \\
La famille \\
La famille est-elle le lieu de la formation morale ? \\
La famille est-elle naturelle ? \\
La famille est-elle une communauté naturelle ? \\
La famille est-elle une institution politique ? \\
La famille est-elle un modèle de société ? \\
La famille et la cité \\
La famille et le droit \\
La fatalité \\
La fatigue \\
La fausseté \\
La faute \\
La faute et le péché \\
La faute et l'erreur \\
La femme est-elle l'avenir de l'homme ? \\
La fermeté \\
La fête \\
L'affirmation \\
La fiction \\
La fidélité \\
La fidélité à soi \\
La fierté \\
La fièvre \\
La figuration \\
La figure de l'ennemi \\
La figure humaine \\
La fin \\
La finalité \\
La finalité des sciences humaines \\
La finalité est-elle nécessaire pour penser le vivant ? \\
La fin de la guerre \\
La fin de la métaphysique \\
La fin de la politique \\
La fin de la politique est-elle l'établissement de la justice ? \\
La fin de l'art \\
La fin de l'État \\
La fin de l'histoire \\
La fin de l'homme \\
La fin des désirs \\
La fin des guerres \\
La fin du monde \\
La fin du mythe \\
La fin du travail \\
La fin et les moyens \\
La finitude \\
La fin justifie-t-elle les moyens ? \\
La foi \\
La foi est-elle aveugle ? \\
La foi est-elle irrationnelle ? \\
La foi est-elle rationnelle ? \\
La folie \\
La folie des grandeurs \\
La fonction \\
La fonction de l'art \\
La fonction de penser peut-elle se déléguer ? \\
La fonction des exemples \\
La fonction du philosophe est-elle de diriger l'État ? \\
La fonction et l'organe \\
La fonction première de l'État est-elle de durer ? \\
La force \\
La force d'âme \\
La force de conviction \\
La force de la croyance \\
La force de la loi \\
La force de l'art \\
La force de la vérité \\
La force de l'esprit \\
La force de l'État est-elle nécessaire à la liberté des citoyens ? \\
La force de l'expérience \\
La force de l'habitude \\
La force de l'idée \\
La force de l'inconscient \\
La force des choses \\
La force des faibles \\
La force des idées \\
La force des lois \\
La force du droit \\
La force du pouvoir \\
La force du social \\
La force est-elle une vertu ? \\
La force et le droit \\
La force fait-elle le droit ? \\
La force publique \\
La formalisation \\
La formation de l'esprit \\
La formation des citoyens \\
La formation du goût \\
La formation d'une conscience \\
La forme \\
La fortune \\
La foule \\
La fragilité \\
La franchise \\
La franchise est-elle une vertu ? \\
La fraternité \\
La fraternité a-t-elle un sens politique ? \\
La fraternité est-elle un idéal moral ? \\
La fraternité peut-elle se passer d'un fondement religieux ? \\
La fraude \\
La frivolité \\
La frontière \\
La fuite du temps est-elle nécessairement un malheur ? \\
La futilité \\
La gauche et la droite \\
L'âge atomique \\
L'âge d'or \\
La généalogie \\
La généralisation \\
La générosité \\
La genèse \\
La genèse de l'œuvre \\
La gentillesse \\
La géographie \\
La géométrie \\
La gloire \\
La gloire est-elle un bien ? \\
La grâce \\
La grammaire \\
La grammaire contraint-elle la pensée ? \\
La grammaire contraint-elle notre pensée ? \\
La grammaire et la logique \\
La grandeur \\
La grandeur d'âme \\
La grandeur d'une culture \\
La gratitude \\
La gratuité \\
L'agression \\
L'agressivité \\
L'agriculture \\
La grossièreté \\
La guérison \\
La guerre \\
La guerre civile \\
La guerre est-elle la continuation de la politique par d'autres moyens ? \\
La guerre est-elle la continuation de la politique ? \\
La guerre est-elle la politique continuée par d'autres moyens ? \\
La guerre est-elle l'essentiel de toute politique ? \\
La guerre et la paix \\
La guerre juste \\
La guerre met-elle fin au droit ? \\
La guerre mondiale \\
La guerre peut-elle être juste ? \\
La guerre totale \\
La haine \\
La haine de la pensée \\
La haine de la raison \\
La haine des images \\
La haine des machines \\
La haine de soi \\
La haine et le mépris \\
La hiérarchie \\
La hiérarchie des arts \\
La hiérarchie des énoncés scientifiques \\
La honte \\
Laisser mourir, est-ce tuer ? \\
La jalousie \\
La jeunesse \\
La jeunesse est mécontente \\
La joie \\
La joie de vivre \\
La jouissance \\
La jurisprudence \\
La juste colère \\
La juste mesure \\
La juste peine \\
La justice \\
La justice a-t-elle besoin des institutions ? \\
La justice a-t-elle un fondement rationnel ? \\
La justice consiste-t-elle à traiter tout le monde de la même manière ? \\
La justice consiste-t-elle dans l'application de la loi ? \\
La justice de l'État \\
La justice divine \\
La justice entre les générations \\
La justice est-elle l'affaire de l'État ? \\
La justice est-elle une notion morale ? \\
La justice est-elle une vertu ? \\
La justice et la force \\
La justice et la loi \\
La justice et la paix \\
La justice et le droit \\
La justice et l'égalité \\
La justice internationale \\
La justice n'est-elle qu'une institution ? \\
La justice n'est-elle qu'un idéal ? \\
La justice peut-elle se fonder sur le compromis ? \\
La justice peut-elle se passer de la force ? \\
La justice peut-elle se passer d'institutions ? \\
La justice sociale \\
La justice suppose-t-elle l'égalité ? \\
La justice : moyen ou fin de la politique ? \\
La justification \\
La lâcheté \\
La laïcité \\
La laideur \\
La langue de la raison \\
La langue et la parole \\
La langue maternelle \\
La lassitude \\
L'aléatoire \\
La leçon des choses \\
La lecture \\
La légende \\
La légèreté \\
La légitimation \\
La légitime défense \\
La légitimité \\
La légitimité démocratique \\
La lettre et l'esprit \\
La libération des mœurs \\
La liberté \\
La liberté artistique \\
La liberté a-t-elle un prix ? \\
La liberté civile \\
La liberté comporte-t-elle des degrés ? \\
La liberté connaît-elle des excès ? \\
La liberté créatrice \\
La liberté d'autrui \\
La liberté de croire \\
La liberté de culte \\
La liberté de l'artiste \\
La liberté de la science \\
La liberté de la volonté \\
La liberté de l'interprète \\
La liberté de parole \\
La liberté de penser \\
La liberté des autres \\
La liberté des citoyens \\
La liberté d'expression \\
La liberté d'expression est-elle nécessaire à la liberté de pensée ? \\
La liberté d'indifférence \\
La liberté d'opinion \\
La liberté du choix \\
La liberté du savant \\
La liberté, est-ce l'indépendance à l'égard des passions ? \\
La liberté est-elle le pouvoir de refuser ? \\
La liberté est-elle un fait ? \\
La liberté et l'égalité sont-elles compatibles ? \\
La liberté et le hasard \\
La liberté et le temps \\
La liberté fait-elle de nous des êtres meilleurs ? \\
La liberté implique-t-elle l'indifférence ? \\
La liberté individuelle \\
La liberté intéresse-t-elle les sciences humaines ? \\
La liberté morale \\
La liberté n'est-elle qu'une illusion ? \\
La liberté nous rend-elle inexcusables ? \\
La liberté peut-elle être prouvée ? \\
La liberté peut-elle être une illusion ? \\
La liberté peut-elle faire peur ? \\
La liberté peut-elle s'affirmer sans violence ? \\
La liberté peut-elle s'aliéner ? \\
La liberté peut-elle se constater ? \\
La liberté peut-elle se prouver ? \\
La liberté peut-elle se refuser ? \\
La liberté politique \\
La liberté requiert-elle le libre échange ? \\
La liberté s'achète-t-elle ? \\
La liberté se mérite-t-elle ? \\
La liberté se prouve-t-elle ? \\
La liberté se réduit-elle au libre-arbitre ? \\
La libre interprétation \\
L'aliénation \\
La limite \\
La littérature peut-elle suppléer les sciences de l'homme ? \\
L'allégorie \\
La logique a-t-elle une histoire ? \\
La logique a-t-elle un intérêt philosophique ? \\
La logique décrit-elle le monde ? \\
La logique du pire \\
La logique est-elle indépendante de la psychologie ? \\
La logique est-elle la norme du vrai ? \\
La logique est-elle l'art de penser ? \\
La logique est-elle un art de penser ? \\
La logique est-elle un art de raisonner ? \\
La logique est-elle une discipline normative ? \\
La logique est-elle une forme de calcul ? \\
La logique est-elle une science de la vérité ? \\
La logique est-elle une science ? \\
La logique est-elle utile à la métaphysique ? \\
La logique et le réel \\
La logique nous apprend-elle quelque chose sur le langage ordinaire ? \\
La logique peut-elle se passer de la métaphysique ? \\
La logique pourrait-elle nous surprendre ? \\
La logique : découverte ou invention ? \\
La loi \\
La loi dit-elle ce qui est juste ? \\
La loi du désir \\
La loi du genre \\
La loi du marché \\
La loi du plus fort \\
La loi éduque-t-elle ? \\
La loi est-elle une garantie contre l'injustice ? \\
La loi et la coutume \\
La loi et la règle \\
La loi et le règlement \\
La loi et les mœurs \\
La loi et l'ordre \\
La loi peut-elle changer les mœurs ? \\
La loi peut-elle être injuste ? \\
La louange et le blâme \\
La loyauté \\
L'alter ego \\
L'altérité \\
L'altruisme \\
L'altruisme n'est-il qu'un égoïsme bien compris ? \\
La lumière de la vérité \\
La lumière naturelle \\
La lutte des classes \\
La machine \\
La magie \\
La magie des mots \\
La magie peut-elle être efficace ? \\
La magnanimité \\
La main \\
La main et l'esprit \\
La main et l'outil \\
La maîtrise \\
La maîtrise de la langue \\
La maîtrise de la nature \\
La maîtrise de soi \\
La maîtrise du feu \\
La maîtrise du temps \\
La majesté \\
La majorité \\
La majorité doit-elle toujours l'emporter ? \\
La majorité, force ou droit ? \\
La majorité peut-elle être tyrannique ? \\
La maladie \\
La maladie est-elle à l'organisme vivant ce que la panne est à la machine ? \\
La maladie est-elle indispensable à la connaissance du vivant ? \\
La malchance \\
La malveillance \\
La manière \\
La manifestation \\
La marchandise \\
La marge \\
La marginalité \\
L'amateur \\
L'amateurisme \\
La mathématique est-elle une ontologie ? \\
La mathématisation du réel \\
La matière \\
La matière de la pensée \\
La matière de l'œuvre \\
La matière, est-ce le mal ? \\
La matière, est-ce l'informe ? \\
La matière est-elle amorphe ? \\
La matière est-elle plus facile à connaître que l'esprit ? \\
La matière est-elle une vue de l'esprit ? \\
La matière et la forme \\
La matière et la vie \\
La matière et l'esprit \\
La matière n'est-elle que ce que l'on perçoit ? \\
La matière n'est-elle qu'une idée ? \\
La matière n'est-elle qu'un obstacle ? \\
La matière pense-t-elle ? \\
La matière peut-elle être objet de connaissance ? \\
La matière première \\
La matière sensible \\
La matière vivante \\
La maturité \\
La mauvaise conscience \\
La mauvaise foi \\
La mauvaise volonté \\
L'ambiguïté \\
L'ambiguïté des mots peut-elle être heureuse ? \\
L'ambition \\
L'ambition politique \\
L'âme \\
La méchanceté \\
L'âme concerne-t-elle les sciences humaines ? \\
La méconnaissance de soi \\
La médecine est-elle une science ? \\
L'âme des bêtes \\
La médiation \\
La médiocrité \\
La méditation \\
L'âme est-elle immortelle ? \\
L'âme et le corps \\
L'âme et le corps sont-ils une seule et même chose ? \\
L'âme et l'esprit \\
La méfiance \\
La meilleure constitution \\
L'âme jouit-elle d'une vie propre ? \\
La mélancolie \\
L'âme, le monde et Dieu \\
L'amélioration des hommes peut-elle être considérée comme un objectif politique ? \\
La mémoire \\
La mémoire collective \\
La mémoire et l'histoire \\
La mémoire et l'individu \\
La mémoire et l'oubli \\
La mémoire sélective \\
La menace \\
La mesure \\
La mesure de l'intelligence \\
La mesure des choses \\
La mesure du temps \\
La métamorphose \\
La métaphore \\
La métaphysique a-t-elle ses fictions ? \\
La métaphysique est-elle le fondement de la morale ? \\
La métaphysique est-elle nécessairement une réflexion sur Dieu ? \\
La métaphysique est-elle une science ? \\
La métaphysique peut-elle être autre chose qu'une science recherchée ? \\
La métaphysique peut-elle faire appel à l'expérience ? \\
La métaphysique se définit-elle par son objet ou sa démarche ? \\
La méthode \\
La méthode de la science \\
La méthode expérimentale est-elle appropriée à l'étude du vivant ? \\
L'ami \\
L'ami du prince \\
La minorité \\
La misanthropie \\
La misère \\
La misologie \\
L'amitié \\
L'amitié est-elle une vertu ? \\
L'amitié est-elle un principe politique ? \\
L'amitié peut-elle obliger ? \\
L'amitié relève-t-elle d'une décision ? \\
La modalité \\
La mode \\
La modélisation en sciences sociales \\
La modération \\
La modération est-elle l'essence de la vertu ? \\
La modération est-elle une vertu politique ? \\
La modernité \\
La modernité dans les arts \\
La modestie \\
La mondialisation \\
La monnaie \\
La monumentalité \\
La morale a-t-elle à décider de la sexualité ? \\
La morale a-t-elle besoin de la notion de sainteté ? \\
La morale a-t-elle besoin d'être fondée ? \\
La morale a-t-elle besoin d'un au-delà ? \\
La morale a-t-elle besoin d'un fondement ? \\
La morale a-t-elle sa place dans l'économie ? \\
La morale commune \\
La morale consiste-t-elle à respecter le droit ? \\
La morale consiste-t-elle à suivre la nature ? \\
La morale de l'athée \\
La morale de l'intérêt \\
La morale dépend-elle de la culture ? \\
La morale des fables \\
La morale doit-elle en appeler à la nature ? \\
La morale doit-elle être rationnelle ? \\
La morale doit-elle fournir des préceptes ? \\
La morale du citoyen \\
La morale du plus fort \\
La morale est-elle affaire de convention ? \\
La morale est-elle affaire de jugement ? \\
La morale est-elle affaire de sentiments ? \\
La morale est-elle affaire de sentiment ? \\
La morale est-elle condamnée à n'être qu'un champ de bataille ? \\
La morale est-elle désintéressée ? \\
La morale est-elle en conflit avec le désir ? \\
La morale est-elle ennemie du bonheur ? \\
La morale est-elle fondée sur la liberté ? \\
La morale est-elle incompatible avec le déterminisme ? \\
La morale est-elle l'ennemie de la vie ? \\
La morale est-elle nécessairement répressive ? \\
La morale est-elle objet de science ? \\
La morale est-elle un art de vivre ? \\
La morale est-elle une affaire de raison ? \\
La morale est-elle une affaire d'habitude ? \\
La morale est-elle une affaire solitaire ? \\
La morale est-elle un fait de culture ? \\
La morale est-elle un fait social ? \\
La morale et la politique \\
La morale et la religion visent-elles les mêmes fins ? \\
La morale et le droit \\
La morale et les mœurs \\
La morale n'est-elle qu'un ensemble de conventions ? \\
La morale peut-elle être fondée sur la science ? \\
La morale peut-elle être naturelle ? \\
La morale peut-elle être un calcul ? \\
La morale peut-elle être une science ? \\
La morale peut-elle se fonder sur les sentiments ? \\
La morale peut-elle s'enseigner ? \\
La morale peut-elle se passer d'un fondement religieux ? \\
La morale politique \\
La morale s'apprend-elle ? \\
La morale s'enseigne-t-elle ? \\
La morale s'oppose-t-elle à la politique ? \\
La morale suppose-t-elle le libre arbitre ? \\
La moralité consiste-t-elle à se contraindre soi-même ? \\
La moralité des lois \\
La moralité est-elle affaire de principes ou de conséquences ? \\
La moralité et le traitement des animaux \\
La moralité n'est-elle que dressage ? \\
La moralité réside-t-elle dans l'intention ? \\
La moralité se réduit-elle aux sentiments ? \\
La mort \\
La mort a-t-elle un sens ? \\
La mort dans l'âme \\
La mort d'autrui \\
La mort de Dieu \\
La mort de l'art \\
La mort de l'homme \\
La mort fait-elle partie de la vie ? \\
L'amour a-t-il des raisons ? \\
L'amour de la liberté \\
L'amour de l'argent \\
L'amour de l'art \\
L'amour de la vérité \\
L'amour de la vie \\
L'amour de l'humanité \\
L'amour des lois \\
L'amour de soi \\
L'amour de soi est-il immoral ? \\
L'amour du destin \\
L'amour du travail \\
L'amour est-il désir ? \\
L'amour est-il une vertu ? \\
L'amour et la haine \\
L'amour et la justice \\
L'amour et l'amitié \\
L'amour et la mort \\
L'amour et le devoir \\
L'amour et le respect \\
L'amour fou \\
L'amour implique-t-il le respect ? \\
L'amour maternel \\
L'amour peut-il être absolu ? \\
L'amour peut-il être raisonnable ? \\
L'amour peut-il être un devoir ? \\
L'amour propre \\
L'amour-propre \\
L'amour vrai \\
La multiplicité \\
La multitude \\
La musique a-t-elle une essence ? \\
La musique de film \\
La musique est-elle un langage ? \\
La musique et le bruit \\
L'anachronisme \\
La naissance \\
La naissance de la science \\
La naissance de l'homme \\
La naïveté \\
La naïveté est-elle une vertu ? \\
L'analogie \\
L'analyse \\
L'analyse du langage ordinaire peut-elle avoir un intérêt philosophique ? \\
L'analyse du vécu \\
L'analyse et la synthèse \\
L'anarchie \\
La nation \\
La nation est-elle dépassée ? \\
La nation et l'État \\
La nature \\
La nature artiste \\
La nature a-t-elle des droits ? \\
La nature a-t-elle une histoire ? \\
La nature a-t-elle un langage ? \\
La nature des choses \\
La nature du bien \\
La nature du fait moral \\
La nature est-elle artiste ? \\
La nature est-elle belle ? \\
La nature est-elle bien faite ? \\
La nature est-elle digne de respect ? \\
La nature est-elle écrite en langage mathématique ? \\
La nature est-elle muette ? \\
La nature est-elle politique ? \\
La nature est-elle prévisible ? \\
La nature est-elle sacrée ? \\
La nature est-elle sans histoire ? \\
La nature est-elle sauvage ? \\
La nature est-elle une norme ? \\
La nature est-elle une ressource ? \\
La nature est-elle un modèle ? \\
La nature est-elle un système ? \\
La nature et la grâce \\
La nature et le beau \\
La nature et le monde \\
La nature existe-t-elle ? \\
La nature fait-elle bien les choses ? \\
La nature morte \\
La nature ne fait pas de saut \\
La nature parle-t-elle le langage des mathématiques ? \\
La nature peut-elle avoir des droits ? \\
La nature peut-elle constituer une norme ? \\
La nature peut-elle être belle ? \\
La nature peut-elle être un modèle ? \\
La nature peut-elle nous indiquer ce que nous devons faire ? \\
La nature se donne-t-elle à penser ? \\
L'anéantissement \\
L'anecdotique \\
La nécessité \\
La nécessité de l'oubli \\
La nécessité des contradictions \\
La nécessité des signes \\
La nécessité fait-elle loi ? \\
La nécessité historique \\
La négation \\
La négligence \\
La négligence est-elle une faute ? \\
La négociation \\
La neige est-elle blanche ? \\
La neutralité \\
La neutralité de l'État \\
Langage et communication \\
Langage et logique \\
Langage et passions \\
Langage et pensée \\
Langage et pouvoir \\
Langage et réalité \\
Langage et société \\
Langage, langue et parole \\
Langage ordinaire et langage de la science \\
L'angélisme \\
L'angoisse \\
Langue et parole \\
L'animal \\
L'animal a-t-il des droits ? \\
L'animal et la bête \\
L'animal et l'homme \\
L'animalité \\
L'animalité de l'animal, l'animalité de l'homme \\
L'animal nous apprend-il quelque chose sur l'homme ? \\
L'animal peut-il être un sujet moral ? \\
L'animal politique \\
L'animisme \\
La noblesse \\
L'anomalie \\
La non-violence \\
L'anonymat \\
L'anormal \\
La normalité \\
La norme \\
La norme et le fait \\
La nostalgie \\
La notion d'administration \\
La notion de barbarie a-t-elle un sens ? \\
La notion de civilisation \\
La notion de classe dominante \\
La notion de classe sociale \\
La notion de comportement \\
La notion de corps social \\
La notion de finalité a-t-elle de l'intérêt pour le savant ? \\
La notion de genre littéraire \\
La notion de loi a-t-elle une unité ? \\
La notion de loi dans les sciences de la nature et dans les sciences de l'homme \\
La notion de monde \\
La notion de nature humaine \\
La notion de nature humaine a-t-elle un sens ? \\
La notion de peuple \\
La notion de point de vue \\
La notion de possible \\
La notion de progrès a-t-elle un sens en politique ? \\
La notion de sujet en politique \\
La notion de système \\
La notion d'évolution \\
La notion d'intérêt \\
La notion d'ordre \\
La notion physique de force \\
La nouveauté \\
La nouveauté en art \\
L'antériorité \\
L'anthropocentrisme \\
L'anthropologie est-elle une ontologie ? \\
L'anticipation \\
L'antinomie \\
La nuance \\
La nudité \\
La nuit \\
La nuit et le jour \\
La ou les vertus ? \\
La paix \\
La paix civile \\
La paix de la conscience \\
La paix est-elle l'absence de guerres ? \\
La paix est-elle l'absence de guerre ? \\
La paix est-elle le plus grand des biens ? \\
La paix est-elle moins naturelle que la guerre ? \\
La paix est-elle possible ? \\
La paix n'est-elle que l'absence de conflit ? \\
La paix n'est-elle que l'absence de guerre ? \\
La paix perpétuelle \\
La paix sociale \\
La paix sociale est-elle la finalité de la politique ? \\
La paix sociale est-elle le but de la politique ? \\
La paix sociale est-elle une fin en soi ? \\
La panne et la maladie \\
La parenté \\
La parenté et la famille \\
La paresse \\
La parole \\
La parole donnée \\
La parole et l'écriture \\
La parole et le geste \\
La parole intérieure \\
La parole peut-elle être une arme ? \\
La parole publique \\
La part de l'ombre \\
La participation \\
La participation des citoyens \\
La partie et le tout \\
La passion \\
La passion amoureuse \\
La passion de la connaissance \\
La passion de la justice \\
La passion de la liberté \\
La passion de la vérité \\
La passion de la vérité peut-elle être source d'erreur ? \\
La passion de l'égalité \\
La passion du juste \\
La passion est-elle immorale ? \\
La passion est-elle l'ennemi de la raison ? \\
La passion exclut-elle la lucidité ? \\
La passion n'est-elle que souffrance ? \\
La passivité \\
La paternité \\
L'apathie \\
La patience \\
La patience est-elle une vertu ? \\
La patrie \\
La pauvreté \\
La pauvreté est-elle une injustice ? \\
La peine \\
La peine capitale \\
La peine de mort \\
La peine de mort est-elle juste, injuste, et pourquoi ? \\
La peinture apprend-elle à voir ? \\
La peinture des mœurs \\
La peinture est-elle une poésie muette ? \\
La peinture peut-elle être un art du temps ? \\
La pénibilité du travail \\
La pensée \\
La pensée a-t-elle une histoire ? \\
La pensée collective \\
La pensée de l'espace \\
La pensée doit-elle se soumettre aux règles de la logique ? \\
La pensée échappe-t-elle à la grammaire ? \\
La pensée est-elle en lutte avec le langage ? \\
La pensée est-elle une activité assimilable à un travail ? \\
La pensée et la conscience sont-elles une seule et même chose ? \\
La pensée formelle \\
La pensée formelle est-elle privée d'objet ? \\
La pensée formelle est-elle une pensée vide ? \\
La pensée formelle peut-elle avoir un contenu ? \\
La pensée magique \\
La pensée obéit-elle à des lois ? \\
La pensée peut-elle s'écrire ? \\
La pensée peut-elle se passer de mots ? \\
La perception \\
La perception construit-elle son objet ? \\
La perception de l'espace est-elle innée ou acquise ? \\
La perception est-elle le premier degré de la connaissance ? \\
La perception est-elle l'interprétation du réel ? \\
La perception est-elle source de connaissance ? \\
La perception est-elle une interprétation ? \\
La perception me donne-t-elle le réel ? \\
La perception peut-elle être désintéressée ? \\
La perception peut-elle s'éduquer ? \\
La perfectibilité \\
La perfection \\
La perfection en art \\
La perfection est-elle désirable ? \\
La perfection morale \\
La performance \\
La permanence \\
La personnalité \\
La personne \\
La personne et l'individu \\
La perspective \\
La persuasion \\
La pertinence \\
La perversion \\
La perversion morale \\
La perversité \\
La peur \\
La peur de la mort \\
La peur de la nature \\
La peur de la science \\
La peur de l'autre \\
La peur de la vérité \\
La peur des machines \\
La peur des mots \\
La peur du châtiment \\
La peur du désordre \\
La philanthropie \\
La philosophie a-t-elle une histoire ? \\
La philosophie doit-elle être une science ? \\
La philosophie doit-elle se préoccuper du salut ? \\
La philosophie est-elle abstraite ? \\
La philosophie est-elle une science ? \\
La philosophie et le sens commun \\
La philosophie et les sciences \\
La philosophie et son histoire \\
La philosophie peut-elle disparaître ? \\
La philosophie peut-elle être expérimentale ? \\
La philosophie peut-elle être populaire ? \\
La philosophie peut-elle être une science ? \\
La philosophie peut-elle se passer de théologie ? \\
La philosophie première \\
La philosophie rend-elle inefficace la propagande ? \\
La photographie est-elle un art ? \\
La physique et la chimie \\
La pitié \\
La pitié a-t-elle une valeur ? \\
La pitié est-elle morale ? \\
La pitié peut-elle fonder la morale ? \\
La place d'autrui \\
La place de la philosophie dans la culture \\
La place du hasard dans la science \\
La place du sujet dans la science \\
La place publique \\
La plaisanterie \\
La plénitude \\
La pluralité \\
La pluralité des arts \\
La pluralité des cultures \\
La pluralité des interprétations \\
La pluralité des langues \\
La pluralité des mondes \\
La pluralité des opinions \\
La pluralité des pouvoirs \\
La pluralité des religions \\
La pluralité des sciences de la nature \\
La pluralité des sens de l'être \\
La poésie \\
La poésie et l'idée \\
La poésie pense-t-elle ? \\
La polémique \\
La police \\
La politesse \\
La politesse est-elle une vertu ? \\
La politique \\
La politique a-t-elle besoin de héros ? \\
La politique a-t-elle besoin de modèles ? \\
La politique a-t-elle besoin d'experts ? \\
La politique a-t-elle pour fin d'éliminer la violence ? \\
La politique consiste-t-elle à faire cause commune ? \\
La politique consiste-t-elle à faire des compromis ? \\
La politique de la santé \\
La politique doit-elle être morale ? \\
La politique doit-elle être rationnelle ? \\
La politique doit-elle refuser l'utopie ? \\
La politique doit-elle se mêler de l'art ? \\
La politique doit-elle se mêler du bonheur ? \\
La politique doit-elle viser la concorde ? \\
La politique doit-elle viser le consensus ? \\
La politique échappe-telle à l'exigence de vérité ? \\
La politique est-elle affaire de compétence ? \\
La politique est-elle affaire de décision ? \\
La politique est-elle affaire de jugement ? \\
La politique est-elle architectonique ? \\
La politique est-elle extérieure au droit ? \\
La politique est-elle la continuation de la guerre ? \\
La politique est-elle l'affaire des spécialistes ? \\
La politique est-elle l'affaire de tous ? \\
La politique est-elle l'art des possibles ? \\
La politique est-elle l'art du possible ? \\
La politique est-elle naturelle ? \\
La politique est-elle par nature sujette à dispute ? \\
La politique est-elle plus importante que tout ? \\
La politique est-elle un art ? \\
La politique est-elle une affaire d'experts ? \\
La politique est-elle une science ? \\
La politique est-elle une technique ? \\
La politique est-elle un métier ? \\
La politique et la gloire \\
La politique et la guerre \\
La politique et la ville \\
La politique et le bonheur \\
La politique et le mal \\
La politique et le politique \\
La politique et les passions \\
La politique et l'opinion \\
La politique exclut-elle le désordre ? \\
La politique implique-t-elle la hiérarchie ? \\
La politique n'est-elle que l'art de conquérir et de conserver le pouvoir ? \\
La politique peut-elle changer la société \\
La politique peut-elle changer le monde ? \\
La politique peut-elle être indépendante de la morale ? \\
La politique peut-elle être objet de science ? \\
La politique peut-elle être un objet de science ? \\
La politique peut-elle n'être qu'une pratique ? \\
La politique peut-elle se passer de croyances ? \\
La politique peut-elle se passer de croyance ? \\
La politique peut-elle unir les hommes ? \\
La politique repose-t-elle sur un contrat ? \\
La politique requière-t-elle le compromis \\
La politique scientifique \\
La politique suppose-t-elle la morale ? \\
La politique suppose-t-elle une idée de l'homme ? \\
L'apolitisme \\
La populace \\
La population \\
La pornographie \\
La possession \\
La possibilité \\
La possibilité logique \\
La possibilité métaphysique \\
La possibilité réelle \\
La poursuite de mon intérêt m'oppose-t-elle aux autres ? \\
L'apparence \\
L'apparence du pouvoir \\
L'apparence est-elle toujours trompeuse ? \\
L'appartenance sociale \\
L'appel \\
L'appréciation de la nature \\
L'apprentissage \\
L'apprentissage de la langue \\
L'apprentissage de la liberté \\
L'appropriation \\
L'approximation \\
La pratique de l'espace \\
La pratique des sciences met-elle à l'abri des préjugés ? \\
La précarité \\
La précaution \\
La précaution peut-elle être un principe ? \\
La précision \\
La première fois \\
La première vérité \\
La présence \\
La présence de l'œuvre d'art \\
La présence d'esprit \\
La présence du passé \\
La présomption \\
La pression du groupe \\
La preuve \\
La preuve de l'existence de Dieu \\
La preuve expérimentale \\
La prévision \\
La prière \\
L'a priori \\
La prise de parti est-elle essentielle en politique ? \\
La prise du pouvoir \\
La prison \\
La prison est-elle utile ? \\
La privation \\
La privation de liberté \\
La probabilité \\
La probité \\
La productivité de l'art \\
La profondeur \\
La prohibition de l'inceste \\
La promenade \\
La promesse \\
La promesse et le contrat \\
L'à propos \\
La proposition \\
La propriété \\
La propriété, est-ce un vol ? \\
La propriété est-elle un droit ? \\
La propriété est-elle une garantie de liberté ? \\
La propriété et le travail \\
La prose du monde \\
La protection \\
La protection sociale \\
La providence \\
La prudence \\
La psychanalyse est-elle une science ? \\
La psychologie est-elle une science de la nature ? \\
La psychologie est-elle une science ? \\
La publicité \\
La pudeur \\
La puissance \\
La puissance de la technique \\
La puissance de l'État \\
La puissance de l'image \\
La puissance de l'imagination \\
La puissance des contraires \\
La puissance des images \\
La puissance du langage \\
La puissance du peuple \\
La puissance et l'acte \\
La pulsion \\
La punition \\
La pureté \\
La qualité \\
La quantité \\
La quantité et la qualité \\
La question de l'essence \\
La question de l'œuvre d'art \\
La question de l'origine \\
La question des origines \\
La question sociale \\
La question « qui suis-je » admet-elle une réponse exacte ? \\
La question: « qui ? » \\
La question : « qui ? » \\
La radicalité \\
La radicalité est-elle une exigence philosophique ? \\
La raison \\
La raison a-t-elle des limites ? \\
La raison a-t-elle le droit d'expliquer ce que morale condamne ? \\
La raison a-t-elle pour fin la connaissance ? \\
La raison a-t-elle une histoire ? \\
La raison des mythes \\
La raison d'état \\
La raison d'État \\
La raison d'État peut-elle être justifiée ? \\
La raison d'être \\
La raison doit-elle critiquer la croyance ? \\
La raison doit-elle être cultivée ? \\
La raison doit-elle être notre guide ? \\
La raison doit-elle se soumettre au réel ? \\
La raison du plus fort \\
La raison engendre-t-elle des illusions ? \\
La raison épuise-t-elle le réel ? \\
La raison est-elle le pouvoir de distinguer le vrai du faux ? \\
La raison est-elle l'esclave des passions ? \\
La raison est-elle l'esclave du désir ? \\
La raison est-elle morale par elle-même ? \\
La raison est-elle plus fiable que l'expérience ? \\
La raison est-elle seulement affaire de logique ? \\
La raison est-elle suffisante ? \\
La raison est-elle toujours raisonnable ? \\
La raison et le réel \\
La raison et l'expérience \\
La raison et l'irrationnel \\
La raison ne connaît-elle du réel que ce qu'elle y met d'elle-même ? \\
La raison ne veut-elle que connaître ? \\
La raison peut-elle entrer en conflit avec elle-même ? \\
La raison peut-elle errer ? \\
La raison peut-elle être immédiatement pratique ? \\
La raison peut-elle être pratique ? \\
La raison peut-elle nous commander de croire ? \\
La raison peut-elle se contredire ? \\
La raison peut-elle servir le mal ? \\
La raison pratique \\
La raison s'oppose-t-elle aux passions ? \\
La raison suffisante \\
La raison transforme-t-elle le réel ? \\
La rareté \\
La rationalité \\
La rationalité des choix politiques \\
La rationalité des comportements économiques \\
La rationalité des émotions \\
La rationalité du langage \\
La rationalité du marché \\
La rationalité en sciences sociales \\
La rationalité politique \\
L'arbitraire \\
L'arbitraire du signe \\
L'archéologie \\
L'architecte et l'ingénieur \\
L'architecture est-elle un art ? \\
L'archive \\
La réaction \\
La réaction en politique \\
La réalité \\
La réalité a-t-elle une forme logique ? \\
La réalité décrite par la science s'oppose-t-elle à la démonstration ? \\
La réalité de la vie s'épuise-t-elle dans celle des vivants ? \\
La réalité de l'espace \\
La réalité de l'idéal \\
La réalité de l'idée \\
La réalité des idées \\
La réalité des phénomènes \\
La réalité du beau \\
La réalité du désordre \\
La réalité du futur \\
La réalité du monde extérieur \\
La réalité du mouvement \\
La réalité du passé \\
La réalité du possible \\
La réalité du rêve \\
La réalité du sensible \\
La réalité du temps \\
La réalité du temps se réduit-elle à la conscience que nous en avons ? \\
La réalité est-elle une idée ? \\
La réalité n'est-elle qu'une construction ? \\
La réalité nourrit-elle la fiction ? \\
La réalité peut-elle être virtuelle ? \\
La réalité sensible \\
La réalité sociale \\
La réalité virtuelle \\
La réception de l'œuvre d'art \\
La recherche \\
La recherche de l'absolu \\
La recherche de la perfection \\
La recherche de l'authenticité \\
La recherche de la vérité \\
La recherche de la vérité dans les sciences humaines \\
La recherche de la vérité peut-elle être désintéressée ? \\
La recherche des causes \\
La recherche des invariants \\
La recherche des origines \\
La recherche d'identité \\
La recherche du bonheur \\
La recherche du bonheur est-elle un idéal égoïste ? \\
La recherche du bonheur suffit-elle à déterminer une morale ? \\
La recherche scientifique est-elle désintéressée ? \\
La réciprocité \\
La réciprocité est-elle indispensable à la communauté politique ? \\
La réconciliation \\
La reconnaissance \\
La rectitude \\
La rectitude du droit \\
La référence \\
La référence aux faits suffit-elle à garantir l'objectivité de la connaissance ? \\
La réflexion \\
La réflexion sur l'expérience participe-t-elle de l'expérience ? \\
La réforme \\
La réforme des institutions \\
La réfutation \\
La règle \\
La règle du jeu \\
La règle et l'exception \\
La régression \\
La régression à l'infini \\
La régularité \\
La relation \\
La relation de cause à effet \\
La relation de nécessité \\
La relativité \\
La religion \\
La religion a-t-elle besoin d'un dieu ? \\
La religion civile \\
La religion conduit-elle l'homme au-delà de lui-même ? \\
La religion divise-t-elle les hommes ? \\
La religion est-elle contraire à la raison ? \\
La religion est-elle fondée sur la peur de la mort ? \\
La religion est-elle la sagesse des pauvres ? \\
La religion est-elle l'asile de l'ignorance ? \\
La religion est-elle l'opium du peuple ? \\
La religion est-elle une affaire privée ? \\
La religion est-elle une consolation pour les hommes ? \\
La religion est-elle un instrument de pouvoir ? \\
La religion et la croyance \\
La religion implique-t-elle la croyance en un être divin ? \\
La religion naturelle \\
La religion n'est-elle que l'affaire des croyants ? \\
La religion n'est-elle qu'une affaire privée ? \\
La religion n'est-elle qu'un fait de culture ? \\
La religion peut-elle être civile ? \\
La religion peut-elle être naturelle ? \\
La religion peut-elle faire lien social ? \\
La religion peut-elle n'être qu'une affaire privée ? \\
La religion peut-elle suppléer la raison ? \\
La religion relie-t-elle les hommes ? \\
La religion rend-elle l'homme heureux ? \\
La religion rend-elle meilleur ? \\
La religion repose-t-elle sur une illusion ? \\
La religion se distingue-t-elle de la superstition ? \\
La réminiscence \\
La renaissance \\
La Renaissance \\
La rencontre \\
La rencontre d'autrui \\
La réparation \\
La répétition \\
La représentation \\
La représentation en politique \\
La représentation politique \\
La reproduction \\
La reproduction des œuvres d'art \\
La reproduction sociale \\
La république \\
La réputation \\
La résignation \\
La résilience \\
La résistance \\
La résistance à l'oppression \\
La résistance de la matière \\
La résolution \\
La responsabilité \\
La responsabilité collective \\
La responsabilité de l'artiste \\
La responsabilité peut-elle être collective ? \\
La responsabilité politique \\
La responsabilité politique n'est-elle le fait que de ceux qui gouvernent ? \\
La ressemblance \\
La restauration des œuvres d'art \\
La réussite \\
La révélation \\
La rêverie \\
La révolte \\
La révolte peut-elle être un droit ? \\
La révolution \\
L'argent \\
L'argent est-il la mesure de tout échange ? \\
L'argent est-il un mal nécessaire ? \\
L'argent et la valeur \\
L'argumentation \\
L'argumentation morale \\
L'argument d'autorité \\
La rhétorique \\
La rhétorique a-t-elle une valeur ? \\
La rhétorique est-elle un art ? \\
La richesse \\
La richesse du sensible \\
La richesse intérieure \\
La rigueur \\
La rigueur de la loi \\
La rigueur des lois ? \\
La rigueur morale \\
L'aristocratie \\
La rivalité \\
L'arme rhétorique \\
L'art \\
L'art abstrait \\
L'art apprend-il à percevoir ? \\
L'art a-t-il besoin de théorie ? \\
L'art a-t-il des vertus thérapeutiques ? \\
L'art a-t-il plus de valeur que la vérité ? \\
L'art a-t-il pour fin le plaisir ? \\
L'art a-t-il une fin morale ? \\
L'art a-t-il une histoire ? \\
L'art a-t-il une valeur sociale ? \\
L'art a-t-il un rôle à jouer dans l'éducation ? \\
L'art change-t-il la vie ? \\
L'art cinématographique \\
L'art contre la beauté ? \\
L'art décoratif \\
L'art d'écrire \\
L'art de faire croire \\
L'art de gouverner \\
L'art de juger \\
L'art de la discussion \\
L'art de masse \\
L'art de persuader \\
L'art des images \\
L'art de vivre \\
L'art de vivre est-il un art ? \\
L'art d'interpréter \\
L'art d'inventer \\
L'art doit-il divertir ? \\
L'art doit-il être critique ? \\
L'art doit-il nous étonner ? \\
L'art doit-il refaire le monde ? \\
L'art donne-t-il à penser ? \\
L'art donne-t-il nécessairement lieu à la production d'une œuvre ? \\
L'art dramatique \\
L'art du comédien \\
L'art du corps \\
L'art du mensonge \\
L'art du portrait \\
L'art échappe-t-il à la raison ? \\
L'art éduque-t-il la perception ? \\
L'art éduque-t-il l'homme ? \\
L'art engagé \\
L'art est-il affaire d'apparence ? \\
L'art est-il affaire de goût ? \\
L'art est-il affaire d'imagination ? \\
L'art est-il à lui-même son propre but ? \\
L'art est-il au service du beau ? \\
L'art est-il destiné à embellir ? \\
L'art est-il imitatif ? \\
L'art est-il le produit de l'inconscient ? \\
L'art est-il le règne des apparences ? \\
L'art est-il mensonger ? \\
L'art est-il méthodique ? \\
L'art est-il moins nécessaire que la science ? \\
L'art est-il subversif ? \\
L'art est-il une affaire sérieuse ? \\
L'art est-il une critique de la culture ? \\
L'art est-il une expérience de la liberté ? \\
L'art est-il une histoire ? \\
L'art est-il une promesse de bonheur ? \\
L'art est-il universel ? \\
L'art est-il un jeu ? \\
L'art est-il un langage ? \\
L'art est-il un luxe ? \\
L'art est-il un mode de connaissance ? \\
L'art est-il un modèle pour la philosophie ? \\
L'art est-il un moyen de connaître ? \\
L'art est-il un refuge ? \\
L'art et la manière \\
L'art et la morale \\
L'art et la nature \\
L'art et la nouveauté \\
L'art et la technique \\
L'art et la tradition \\
L'art et la vie \\
L'art et le beau \\
L'art et le divin \\
L'art et le jeu \\
L'art et le mouvement \\
L'art et l'éphémère \\
L'art et le réel \\
L'art et le rêve \\
L'art et le sacré \\
L'art et les arts \\
L'art et l'espace \\
L'art et le temps \\
L'art et l'illusion \\
L'art et l'immoralité \\
L'art et l'invisible \\
L'art et morale \\
L'art exprime-t-il ce que nous ne saurions dire ? \\
L'art fait-il penser ? \\
L'artifice \\
L'artificiel \\
L'art imite-t-il la nature ? \\
L'artiste \\
L'artiste a-t-il besoin de modèle ? \\
L'artiste a-t-il besoin d'une idée de l'art ? \\
L'artiste a-t-il besoin d'un public ? \\
L'artiste a-t-il une méthode ? \\
L'artiste de soi-même \\
L'artiste doit-il être de son temps ? \\
L'artiste doit-il être original ? \\
L'artiste doit-il se donner des modèles ? \\
L'artiste doit-il se soucier du goût du public ? \\
L'artiste est-il le mieux placé pour comprendre son œuvre ? \\
L'artiste est-il maître de son œuvre ? \\
L'artiste est-il souverain ? \\
L'artiste est-il un créateur ? \\
L'artiste est-il un travailleur ? \\
L'artiste et l'artisan \\
L'artiste et la sensation \\
L'artiste et la société \\
L'artiste et le savant \\
L'artiste exprime-t-il quelque chose ? \\
L'artiste peut-il se passer d'un maître ? \\
L'artiste recherche-t-il le beau ? \\
L'artiste sait-il ce qu'il fait ? \\
L'artiste travaille-t-il ? \\
L'art modifie-t-il notre rapport au réel ? \\
L'art n'est-il pas toujours politique ? \\
L'art n'est-il pas toujours religieux ? \\
L'art n'est-il qu'une question de sentiment ? \\
L'art n'est-il qu'un mode d'expression subjectif ? \\
L'art n'est qu'une affaire de goût ? \\
L'art nous détourne-t-il de la réalité ? \\
L'art nous fait-il mieux percevoir le réel ? \\
L'art nous mène-t-il au vrai ? \\
L'art nous réconcilie-t-il avec le monde ? \\
L'art ou les arts \\
L'art parachève-t-il la nature ? \\
L'art participe-t-il à la vie politique ? \\
L'art permet-il un accès au divin ? \\
L'art peut-il changer le monde \\
L'art peut-il contribuer à éduquer les hommes ? \\
L'art peut-il encore imiter la nature ? \\
L'art peut-il être abstrait ? \\
L'art peut-il être brut ? \\
L'art peut-il être conceptuel ? \\
L'art peut-il être populaire ? \\
L'art peut-il être réaliste \\
L'art peut-il être sans œuvre ? \\
L'art peut-il être utile ? \\
L'art peut-il ne pas être sacré ? \\
L'art peut-il n'être pas conceptuel ? \\
L'art peut-il nous rendre meilleurs ? \\
L'art peut-il prétendre à la vérité ? \\
L'art peut-il quelque chose contre la morale ? \\
L'art peut-il quelque chose pour la morale ? \\
L'art peut-il rendre le mouvement ? \\
L'art peut-il s'affranchir des lois ? \\
L'art peut-il s'enseigner ? \\
L'art peut-il se passer de la beauté ? \\
L'art peut-il se passer de règles ? \\
L'art peut-il se passer d'idéal ? \\
L'art peut-il se passer d'œuvres ? \\
L'art politique \\
L'art populaire \\
L'art pour l'art \\
L'art produit-il nécessairement des œuvres ? \\
L'art rend-il les hommes meilleurs ? \\
L'art s'adresse-t-il à la sensibilité ? \\
L'art s'adresse-t-il à tous ? \\
L'art sait-il montrer ce que le langage ne peut pas dire ? \\
L'art s'apparente-t-il à la philosophie ? \\
L'art s'apprend-il ? \\
L'art vise-t-il le beau ? \\
L'art : expérience, exercice ou habitude ? \\
L'art : une arithmétique sensible ? \\
La ruine \\
La rumeur \\
La rupture \\
La ruse \\
La ruse technique \\
La sacralisation de l'œuvre \\
La sagesse \\
La sagesse et la passion \\
La sagesse et l'expérience \\
La sagesse rend-elle heureux ? \\
La sainteté \\
La sanction \\
La santé \\
La santé est-elle un devoir ? \\
La santé est-elle un droit ou un devoir ? \\
La santé mentale \\
La satisfaction \\
La satisfaction des penchants \\
La scène du monde \\
La scène théâtrale \\
L'ascèse \\
L'ascétisme \\
La science admet-elle des degrés de croyance ? \\
La science a-t-elle besoin d'un critère de démarcation entre science et non science ? \\
La science a-t-elle besoin d'une méthode ? \\
La science a-t-elle besoin du principe de causalité ? \\
La science a-t-elle des limites ? \\
La science a-t-elle le monopole de la raison ? \\
La science a-t-elle le monopole de la vérité ? \\
La science a-t-elle une histoire ? \\
La science commence-t-elle avec la perception ? \\
La science commence-telle avec la perception ? \\
La science découvre-t-elle ou construit-elle son objet ? \\
La science de l'être \\
La science de l'individuel \\
La science des mœurs \\
La science dévoile-t-elle le réel ? \\
La science doit-elle se fonder sur une idée de la nature ? \\
La science doit-elle se passer de l'idée de finalité ? \\
La science du vivant peut-elle se passer de l'idée de finalité ? \\
La science est-elle austère ? \\
La science est-elle indépendante de toute métaphysique ? \\
La science est-elle le lieu de la vérité ? \\
La science est-elle une connaissance du réel ? \\
La science est-elle une langue bien faite ? \\
La science est-elle un jeu ? \\
La science et la foi \\
La science et le mythe \\
La science et les sciences \\
La science et l'irrationnel \\
La science exclut-elle l'imagination ? \\
La science n'est-elle qu'une activité théorique ? \\
La science n'est-elle qu'une fiction ? \\
La science nous éloigne-t-elle de la religion ? \\
La science nous éloigne-t-elle des choses ? \\
La science nous indique-t-elle ce que nous devons faire ? \\
La science pense-t-elle ? \\
La science permet-elle de comprendre le monde ? \\
La science peut-elle être une métaphysique ? \\
La science peut-elle guider notre conduite ? \\
La science peut-elle lutter contre les préjugés ? \\
La science peut-elle produire des croyances ? \\
La science peut-elle se passer de fondement ? \\
La science peut-elle se passer de l'idée de finalité ? \\
La science peut-elle se passer de métaphysique ? \\
La science peut-elle se passer d'hypothèses ? \\
La science peut-elle se passer d'institutions ? \\
La science peut-elle tout expliquer ? \\
La science politique \\
La science porte-elle au scepticisme ? \\
La science procède-t-elle par rectification ? \\
La science se limite-t-elle à constater les faits ? \\
La sculpture \\
La seconde nature \\
La sécularisation \\
La sécurité \\
La sécurité nationale \\
La sécurité publique \\
La séduction \\
La ségrégation \\
La sensation \\
La sensibilité \\
La séparation \\
La séparation des pouvoirs \\
La sérénité \\
La servitude \\
La servitude peut-elle être volontaire ? \\
La servitude volontaire \\
La sévérité \\
La sexualité \\
La signification \\
La signification dans l'œuvre \\
La signification des mots \\
La signification en musique \\
L'asile de l'ignorance \\
La simplicité \\
La simplicité du bien \\
La simulation \\
La sincérité \\
La singularité \\
La situation \\
La sobriété \\
La sociabilité \\
La socialisation des comportements \\
La société \\
La société civile \\
La société civile et l'État \\
La société contre l'État \\
La société des nations \\
La société des savants \\
La société doit-elle reconnaître les désirs individuels ? \\
La société du genre humain \\
La société est-elle concevable sans le travail ? \\
La société est-elle un organisme ? \\
La société et les échanges \\
La société et l'État \\
La société et l'individu \\
La société existe-t-elle ? \\
La société fait-elle l'homme ? \\
La société peut-elle être l'objet d'une science ? \\
La société peut-elle se passer de l'État ? \\
La société précède-t-elle l'individu ? \\
La société repose-t-elle sur l'altruisme ? \\
La société sans l'État \\
La sociologie de l'art nous permet-elle de comprendre l'art ? \\
La sociologie relativise-t-elle la valeur des œuvres d'art ? \\
La solidarité \\
La solidarité est-elle naturelle ? \\
La solitude \\
La solitude constitue-t-elle un obstacle à la citoyenneté ? \\
La sollicitude \\
La somme et le tout \\
La souffrance \\
La souffrance a-t-elle une valeur morale ? \\
La souffrance a-t-elle un sens moral ? \\
La souffrance a-t-elle un sens ? \\
La souffrance au travail \\
La souffrance d'autrui \\
La souffrance d'autrui m'importe-t-elle ? \\
La souffrance des animaux \\
La souffrance morale \\
La souffrance peut-elle être un mode de connaissance ? \\
La soumission à l'autorité \\
La souveraineté \\
La souveraineté de l'État \\
La souveraineté du peuple \\
La souveraineté peut-elle être déléguée \\
La souveraineté peut-elle être limitée ? \\
La souveraineté peut-elle se partager ? \\
La souveraineté populaire \\
La spécificité des sciences humaines \\
La spéculation \\
La sphère privée échappe-t-elle au politique ? \\
L'aspiration esthétique \\
La spontanéité \\
L'assentiment \\
L'association \\
L'association des idées \\
La standardisation \\
La structure \\
La structure et le sujet \\
La subjectivité \\
La substance \\
La subtilité \\
La succession des théories scientifiques \\
La superstition \\
La sûreté \\
La surface et la profondeur \\
La surprise \\
La surveillance de la société \\
La survie \\
La sympathie \\
La sympathie peut-elle tenir lieu de moralité ? \\
La table rase \\
La tâche d'exister \\
La tautologie \\
La technique \\
La technique accroît-elle notre liberté ? \\
La technique a-t-elle sa place en politique ? \\
La technique a-t-elle une histoire ? \\
La technique crée-t-elle son propre monde ? \\
La technique est-elle civilisatrice ? \\
La technique est-elle contre-nature ? \\
La technique est-elle dangereuse ? \\
La technique est-elle l'application de la science ? \\
La technique est-elle le propre de l'homme ? \\
La technique est-elle libératrice ? \\
La technique est-elle moralement neutre ? \\
La technique est-elle neutre ? \\
La technique est-elle un savoir ? \\
La technique et le corps \\
La technique et le travail \\
La technique fait-elle des miracles ? \\
La technique fait-elle violence à la nature ? \\
La technique libère-t-elle les hommes ? \\
La technique ne fait-elle qu'appliquer la science ? \\
La technique ne pose-t-elle que des problèmes techniques ? \\
La technique n'est-elle pour l'homme qu'un moyen ? \\
La technique n'est-elle qu'une application de la science ? \\
La technique n'est-elle qu'un outil au service de l'homme ? \\
La technique n'existe-elle que pour satisfaire des besoins ? \\
La technique nous éloigne-t-elle de la nature ? \\
La technique nous éloigne-t-elle de la réalité ? \\
La technique nous libère-t-elle ? \\
La technique nous oppose-t-elle à la nature ? \\
La technique nous permet-elle de comprendre la nature ? \\
La technique permet-elle de réaliser tous les désirs ? \\
La technique peut-elle améliorer l'homme ? \\
La technique peut-elle se déduire de la science ? \\
La technique peut-elle se passer de la science ? \\
La technique repose-t-elle sur le génie du technicien ? \\
La technique sert-elle nos désirs ? \\
La technocratie \\
La technologie modifie-t-elle les rapports sociaux ? \\
La téléologie \\
La télévision \\
La tempérance \\
La temporalité de l'œuvre d'art \\
La tendance \\
La tentation \\
La tentation réductionniste \\
La terre \\
La Terre et le Ciel \\
La terreur \\
La terreur morale \\
L'athéisme \\
La théogonie \\
La théologie rationnelle \\
La théorie \\
La théorie et la pratique \\
La théorie et l'expérience \\
La théorie nous éloigne-t-elle de la réalité ? \\
La théorie peut-elle nous égarer ? \\
La théorie scientifique \\
La tolérance \\
La tolérance a-t-elle des limites ? \\
La tolérance envers les intolérants \\
La tolérance est-elle un concept politique ? \\
La tolérance est-elle une vertu ? \\
La tolérance peut-elle constituer un problème pour la démocratie ? \\
L'atome \\
La totalitarisme \\
La totalité \\
La toute puissance \\
La toute-puissance \\
La toute puissance de la pensée \\
La trace \\
La trace et l'indice \\
La tradition \\
La traduction \\
La tragédie \\
La trahison \\
La tranquillité \\
La transcendance \\
La transe \\
La transgression \\
La transgression des règles \\
La transmission \\
La transmission de pensée \\
La transparence \\
La transparence des consciences \\
La transparence est-elle un idéal démocratique ? \\
La tristesse \\
L'attachement \\
L'attente \\
L'attention \\
L'attention caractérise-t-elle la conscience ? \\
L'attraction \\
L'attrait du beau \\
La tyrannie \\
La tyrannie de la majorité \\
La tyrannie des désirs \\
La tyrannie du bonheur \\
L'audace \\
L'audace politique \\
L'au-delà \\
L'au-delà de l'être \\
L'autarcie \\
L'auteur et le créateur \\
L'authenticité \\
L'authenticité de l'œuvre d'art \\
L'autobiographie \\
L'autocritique \\
L'automate \\
L'automatisation \\
L'automatisation du raisonnement \\
L'autonomie \\
L'autonomie de l'art \\
L'autonomie de l'œuvre d'art \\
L'autonomie du théorique \\
L'autoportrait \\
L'autorité \\
L'autorité de la loi \\
L'autorité de la parole \\
L'autorité de la science \\
L'autorité de l'écrit \\
L'autorité de l'État \\
L'autorité des lois \\
L'autorité du droit \\
L'autorité morale \\
L'autorité politique \\
L'autre est-il le fondement de la conscience morale ? \\
L'autre et les autres \\
L'autre monde \\
La valeur \\
La valeur d'échange \\
La valeur de la pitié \\
La valeur de l'argent \\
La valeur de l'art \\
La valeur de la science \\
La valeur de la vérité \\
La valeur de la vie \\
La valeur de l'échange \\
La valeur de l'exemple \\
La valeur de l'hypothèse \\
La valeur des choses \\
La valeur des images \\
La valeur du consensus \\
La valeur du consentement \\
La valeur d'une action se mesure-t-elle à sa réussite ? \\
La valeur d'une œuvre \\
La valeur d'une théorie scientifique se mesure-t-elle à son efficacité ? \\
La valeur du plaisir \\
La valeur du témoignage \\
La valeur du temps \\
La valeur du travail \\
La valeur et le prix \\
La valeur morale \\
La valeur morale de l'amour \\
La valeur morale d'une action se juge-t-elle à ses conséquences ? \\
La validité \\
La vanité \\
La vanité est-elle toujours sans objet ? \\
L'avant-garde \\
L'avarice \\
La variété \\
La veille et le sommeil \\
La vénalité \\
La vengeance \\
L'avenir \\
L'avenir a-t-il une réalité ? \\
L'avenir de l'humanité \\
L'avenir est-il imaginable ? \\
L'avenir existe-t-il ? \\
L'avenir peut-il être objet de connaissance ? \\
L'aventure \\
La véracité \\
La vérification \\
La vérification expérimentale \\
La vérité \\
La vérité admet-elle des degrés ? \\
La vérité a-t-elle une histoire ? \\
La vérité de la fiction \\
La vérité de la perception \\
La vérité de l'apparence \\
La vérité de la religion \\
La vérité demande-t-elle du courage ? \\
La vérité des arts \\
La vérité des images \\
La vérité des sciences \\
La vérité doit-elle toujours être démontrée ? \\
La vérité donne-t-elle le droit d'être injuste ? \\
La vérité du déterminisme \\
La vérité d'une théorie dépend-elle de sa correspondance avec les faits ? \\
La vérité du roman \\
La vérité échappe-t-elle au temps ? \\
La vérité en art \\
La vérité est-elle affaire de cohérence ? \\
La vérité est-elle affaire de croyance ou de savoir ? \\
La vérité est-elle contraignante ? \\
La vérité est-elle éternelle ? \\
La vérité est-elle intemporelle ? \\
La vérité est-elle libératrice ? \\
La vérité est-elle morale ? \\
La vérité est-elle objective ? \\
La vérité est-elle triste ? \\
La vérité est-elle une construction ? \\
La vérité est-elle une valeur ? \\
La vérité est-elle une ? \\
La vérité historique \\
La vérité mathématique \\
La vérité n'est-elle qu'une erreur rectifiée ? \\
La vérité nous contraint-elle ? \\
La vérité peut-elle être équivoque ? \\
La vérité peut-elle être tolérante ? \\
La vérité peut-elle laisser indifférent ? \\
La vérité peut-elle se définir par le consensus ? \\
La vérité philosophique \\
La vérité rend-elle heureux ? \\
La vérité scientifique est-elle relative ? \\
La vérité se communique-t-elle ? \\
La vertu \\
La vertu de l'homme politique \\
La vertu du citoyen \\
La vertu du plaisir \\
La vertu, les vertus \\
La vertu peut-elle être excessive ? \\
La vertu peut-elle être purement morale ? \\
La vertu peut-elle s'enseigner ? \\
La vertu politique \\
La vertu s'enseigne-t-elle ? \\
L'aveu \\
L'aveu diminue-t-il la faute ? \\
L'aveuglement \\
La vie \\
La vie active \\
La vie a-t-elle un sens ? \\
La vie collective est-elle nécessairement frustrante ? \\
La vie de la langue \\
La vie de l'esprit \\
La vie de plaisirs \\
La vie des machines \\
La vie des rêves \\
La vie du droit \\
La vie en société est-elle naturelle à l'homme ? \\
La vie en société impose-t-elle de n'être pas soi-même ? \\
La vie est-elle la valeur suprême ? \\
La vie est-elle le bien le plus précieux ? \\
La vie est-elle l'objet des sciences de la vie ? \\
La vie est-elle objet de science ? \\
La vie est-elle sacrée ? \\
La vie est-elle une valeur ? \\
La vie est-elle un roman ? \\
La vie est-elle un songe ? \\
La vie éternelle \\
La vie heureuse \\
La vieillesse \\
La vie intérieure \\
La vie moderne \\
La vie morale \\
La vie ordinaire \\
La vie peut-elle être éternelle ? \\
La vie peut-elle être objet de science ? \\
La vie politique \\
La vie politique est-elle aliénante ? \\
La vie privée \\
La vie psychique \\
La vie quotidienne \\
La vie sauvage \\
La vie sexuelle est-elle volontaire ? \\
La vie sociale \\
La vie sociale est-elle toujours conflictuelle ? \\
La vie sociale est-elle une comédie ? \\
La vigilance \\
La ville \\
La ville et la campagne \\
La violence \\
La violence a-t-elle des degrés ? \\
La violence de l'État \\
La violence d'État \\
La violence est-elle toujours destructrice ? \\
La violence peut-elle avoir raison ? \\
La violence peut-elle être gratuite ? \\
La violence politique \\
La violence révolutionnaire \\
La violence sociale \\
La violence verbale \\
La virtualité \\
La virtuosité \\
La vision et le toucher \\
La vision peut-elle être le modèle de toute connaissance ? \\
La vitesse \\
L'avocat du diable \\
La vocation \\
La voix \\
La voix de la conscience \\
La voix de la raison \\
La voix du peuple \\
La volonté constitue-t-elle le principe de la politique ? \\
La volonté de savoir \\
La volonté du peuple \\
La volonté et le désir \\
La volonté générale \\
La volonté générale est-elle la volonté de tous ? \\
La volonté peut-elle être collective ? \\
La volonté peut-elle être générale ? \\
La volonté peut-elle être indéterminée ? \\
La volonté peut-elle nous manquer ? \\
La volupté \\
La vraie morale se moque-t-elle de la morale ? \\
La vraie vie \\
La vraisemblance \\
La vue et le toucher \\
La vue et l'ouïe \\
La vulgarisation \\
La vulgarité \\
La vulnérabilité \\
L'axiome \\
Le barbare \\
Le baroque \\
Le bavardage \\
Le beau a-t-il une histoire ? \\
Le beau est-il aimable ? \\
Le beau est-il toujours moral ? \\
Le beau est-il une valeur commune ? \\
Le beau est-il universel ? \\
Le beau et l'agréable \\
Le beau et le bien \\
Le beau et le bien sont-ils, au fond, identiques ? \\
Le beau et le joli \\
Le beau et le sublime \\
Le beau et l'utile \\
Le beau geste \\
Le beau naturel \\
Le beau peut-il être bizarre ? \\
Le bénéfice du doute \\
Le besoin \\
Le besoin d'absolu \\
Le besoin de métaphysique est-il un besoin de connaissance ? \\
Le besoin de philosophie \\
Le besoin de reconnaissance \\
Le besoin de sens \\
Le besoin de signes \\
Le besoin de théorie \\
Le besoin de vérité ? \\
Le besoin et le désir \\
Le bien commun \\
Le bien commun est-il une illusion ? \\
Le bien commun et l'intérêt de tous \\
Le bien d'autrui \\
Le bien est-ce l'utile ? \\
Le bien est-il relatif ? \\
Le bien et le beau \\
Le bien et le mal \\
Le bien et les biens \\
Le bien et l'utile \\
Le bien-être \\
Le bien public \\
Le bien suppose-t-il la transcendance ? \\
L'éblouissement \\
Le bon Dieu \\
Le bon et l'utile \\
Le bon goût \\
Le bon gouvernement \\
Le bonheur \\
Le bonheur collectif \\
Le bonheur dans le mal \\
Le bonheur de la passion est-il sans lendemain ? \\
Le bonheur des autres \\
Le bonheur des citoyens est-il un idéal politique ? \\
Le bonheur des méchants \\
Le bonheur des sens \\
Le bonheur des uns, le malheur des autres \\
Le bonheur du juste \\
Le bonheur est-il affaire de vertu ? \\
Le bonheur est-il affaire de volonté ? \\
Le bonheur est-il affaire privée ? \\
Le bonheur est-il au nombre de nos devoirs ? \\
Le bonheur est-il dans l'inconscience ? \\
Le bonheur est-il l'absence de maux ? \\
Le bonheur est-il la fin de la vie ? \\
Le bonheur est-il le bien suprême ? \\
Le bonheur est-il le but de la politique ? \\
Le bonheur est-il le prix de la vertu ? \\
Le bonheur est-il un accident ? \\
Le bonheur est-il un but politique ? \\
Le bonheur est-il un droit ? \\
Le bonheur est-il une affaire privée ? \\
Le bonheur est-il une fin morale ? \\
Le bonheur est-il une fin politique ? \\
Le bonheur est-il une valeur morale ? \\
Le bonheur est-il un idéal ? \\
Le bonheur est-il un principe politique ? \\
Le bonheur et la raison \\
Le bonheur et la technique \\
Le bonheur et la vertu \\
Le bonheur n'est-il qu'une idée ? \\
Le bonheur n'est-il qu'un idéal ? \\
Le bonheur peut-il être collectif ? \\
Le bonheur peut-il être le but de la politique ? \\
Le bonheur peut-il être un droit ? \\
Le bonheur se calcule-t-il ? \\
Le bonheur se mérite-t-il ? \\
Le bon régime \\
Le bon sens \\
Le bon usage des passions \\
Le bouc émissaire \\
Le bourgeois et le citoyen \\
Le bricolage \\
Le bruit \\
Le bruit et la musique \\
Le but de l'association politique \\
Le cadavre \\
Le calcul \\
Le calcul des plaisirs \\
Le calendrier \\
Le cannibalisme \\
Le capitalisme \\
Le capital social \\
Le caractère \\
Le caractère sacré de la vie \\
L'écart \\
Le cas de conscience \\
Le cas particulier \\
Le catéchisme moral \\
Le cerveau et la pensée \\
Le cerveau pense-t-il ? \\
L'échange \\
L'échange constitue-t-il un lien social ? \\
L'échange des marchandises et les rapports humains \\
L'échange économique fonde-t-il la société humaine \\
L'échange est-il un facteur de paix ? \\
L'échange et l'usage \\
Le changement \\
L'échange n'a-t-il de fondement qu'économique ? \\
L'échange ne porte-t-il que sur les choses ? \\
L'échange peut-il être désintéressé ? \\
L'échange symbolique \\
Le chant \\
Le chaos \\
Le charisme en politique \\
Le charme \\
Le charme et la grâce \\
Le châtiment \\
Le chef \\
Le chef d'œuvre \\
Le chef-d'œuvre \\
Le chemin \\
Le choc esthétique \\
Le choix \\
Le choix de philosopher \\
Le choix des moyens \\
Le choix d'un destin \\
Le choix d'un métier \\
Le choix et la liberté \\
Le choix peut-il être éclairé ? \\
Le ciel et la terre \\
Le cinéma, art de la représentation ? \\
Le cinéma est-il un art comme les autres ? \\
Le cinéma est-il un art ou une industrie ? \\
Le cinéma est-il un art populaire ? \\
Le cinéma est-il un art ? \\
Le citoyen \\
Le citoyen a-t-il perdu toute naturalité ?L'étranger \\
Le citoyen peut-il être à la fois libre et soumis à l'État ? \\
Le clair et l'obscur \\
Le classicisme \\
L'éclat \\
Le cliché \\
Le cœur \\
Le cœur et la raison \\
L'école de la vie \\
L'école des vertus \\
L'écologie est-elle un problème politique ? \\
L'écologie politique \\
L'écologie, une science humaine ? \\
Le combat \\
Le combat contre l'injustice a-t-il une source morale ? \\
Le comédien \\
Le comique \\
Le comique et le tragique \\
Le commencement \\
Le comment et le pourquoi \\
Le commerce \\
Le commerce adoucit-il les mœurs ? \\
Le commerce des hommes \\
Le commerce des idées \\
Le commerce équitable \\
Le commerce est-il pacificateur ? \\
Le commerce peut-il être équitable ? \\
Le commerce unit-il les hommes ? \\
Le commun \\
Le commun et le propre \\
Le comparatisme dans les sciences humaines \\
Le complexe \\
Le comportement \\
Le compromis \\
Le concept \\
Le concept de matière \\
Le concept de nature est-il un concept scientifique ? \\
Le concept de pulsion \\
Le concept de structure \\
Le concept de structure sociale \\
Le concept d'inconscient est-il nécessaire en sciences humaines ? \\
Le concept et l'exemple \\
Le concret \\
Le concret et l'abstrait \\
Le conditionnel \\
Le conflit \\
Le conflit de devoirs \\
Le conflit des devoirs \\
Le conflit des interprétations \\
Le conflit entre la science et la religion est-il inévitable ? \\
Le conflit est-il constitutif de la politique ? \\
Le conflit est-il la raison d'être de la politique ? \\
Le conflit est-il une maladie sociale ? \\
Le conflit ? \\
Le conformisme \\
Le conformisme moral \\
Le conformisme social \\
Le confort intellectuel \\
L'économie \\
L'économie a-t-elle des lois ? \\
L'économie des moyens \\
L'économie est-elle une science humaine ? \\
L'économie est-elle une science ? \\
L'économie et la politique \\
L'économie politique \\
L'économie psychique \\
L'économique et le politique \\
Le conscient et l'inconscient \\
Le conseil \\
Le conseiller du prince \\
Le consensus \\
Le consentement \\
Le consentement des gouvernés \\
Le conservatisme \\
Le contentement \\
Le contingent \\
Le continu \\
Le contrat \\
Le contrat de travail \\
Le contrat est-il au fondement de la politique ? \\
Le contrôle social \\
Le convenable \\
Le corps dansant \\
Le corps dit-il quelque chose ? \\
Le corps du travailleur \\
Le corps est-il le reflet de l'âme ? \\
Le corps est-il négociable ? \\
Le corps est-il porteur de valeurs ? \\
Le corps est-il respectable ? \\
Le corps et la machine \\
Le corps et l'âme \\
Le corps et l'esprit \\
Le corps et le temps \\
Le corps et l'instrument \\
Le corps humain \\
Le corps humain est-il naturel ? \\
Le corps impose-t-il des perspectives ? \\
Le corps n'est-il que matière ? \\
Le corps n'est-il qu'un mécanisme ? \\
Le corps obéit-il à l'esprit ? \\
Le corps pense-t-il ? \\
Le corps peut-il être objet d'art ? \\
Le corps politique \\
Le corps propre \\
Le cosmopolitisme \\
Le cosmopolitisme peut-il devenir réalité ? \\
Le cosmopolitisme peut-il être réaliste ? \\
Le coup d'État \\
Le courage \\
Le courage de penser \\
Le courage politique \\
Le cours des choses \\
Le cours du temps \\
Le créé et l'incréé \\
Le cri \\
Le crime \\
Le crime contre l'humanité \\
Le crime inexpiable \\
Le critère \\
L'écrit et l'oral \\
Le critique d'art \\
L'écriture \\
L'écriture des lois \\
L'écriture et la parole \\
L'écriture et la pensée \\
L'écriture ne sert-elle qu'à consigner la pensée ? \\
L'écriture peut-elle porter secours à la pensée ? \\
Le culte des ancêtres \\
Le cynisme \\
Le dandysme \\
Le danger \\
Le débat \\
Le débat politique \\
Le déchet \\
Le dedans et le dehors \\
Le défaut \\
Le dégoût \\
Le déguisement \\
Le délire \\
Le démoniaque \\
Le dépaysement \\
Le dérèglement \\
Le dernier mot \\
Le désaccord \\
Le désenchantement \\
Le désespoir \\
Le désespoir est-il une faute morale ? \\
Le déshonneur \\
Le design \\
Le désintéressement \\
Le désintéressement esthétique \\
Le désir \\
Le désir a-t-il un objet ? \\
Le désir d'absolu \\
Le désir de connaître \\
Le désir d'égalité \\
Le désir de gloire \\
Le désir de l'autre \\
Le désir de pouvoir \\
Le désir de reconnaissance \\
Le désir de savoir \\
Le désir de savoir est-il naturel ? \\
Le désir d'éternité \\
Le désir d'être autre \\
Le désir de vérité \\
Le désir de vivre \\
Le désir d'immortalité \\
Le désir d'originalité \\
Le désir du bonheur est-il universel ? \\
Le désir est-il aveugle ? \\
Le désir est-il ce qui nous fait vivre ? \\
Le désir est-il désir de l'autre ? \\
Le désir est-il le signe d'un manque ? \\
Le désir est-il l'essence de l'homme ? \\
Le désir est-il nécessairement l'expression d'un manque ? \\
Le désir est-il par nature illimité ? \\
Le désir est-il sans limite ? \\
Le désir et la culpabilité \\
Le désir et la loi \\
Le désir et le besoin \\
Le désir et le mal \\
Le désir et le manque \\
Le désir et le rêve \\
Le désir et le temps \\
Le désir et le travail \\
Le désir et l'interdit \\
Le désir métaphysique \\
Le désir n'est-il que l'épreuve d'un manque ? \\
Le désir n'est-il que manque ? \\
Le désir n'est-il qu'inquiétude ? \\
Le désir peut-il être désintéressé ? \\
Le désir peut-il ne pas avoir d'objet ? \\
Le désir peut-il nous rendre libre ? \\
Le désir peut-il se satisfaire de la réalité ? \\
Le désœuvrement \\
Le désordre \\
Le désordre des choses \\
Le despote peut-il être éclairé ? \\
Le despotisme \\
Le dessin et la couleur \\
Le destin \\
Le désuet \\
Le détachement \\
Le détail \\
Le déterminisme \\
Le déterminisme social \\
Le deuil \\
Le devenir \\
Le devoir \\
Le devoir d'aimer \\
Le devoir de mémoire \\
Le devoir d'obéissance \\
Le devoir est-il l'expression de la contrainte sociale ? \\
Le devoir et la dette \\
Le devoir et le bonheur \\
Le devoir-être \\
Le devoir rend-il libre ? \\
Le devoir se présente-t-il avec la force de l'évidence ? \\
Le devoir supprime-t-il la liberté ? \\
Le dévouement \\
Le diable \\
Le dialogue \\
Le dialogue des philosophes \\
Le dialogue entre les cultures \\
Le dialogue entre nations \\
Le dialogue suffit-il à rompre la solitude ? \\
Le dictionnaire \\
Le dieu artiste \\
Le dieu des philosophes \\
Le Dieu des philosophes \\
L'édification morale \\
Le dilemme \\
Le dire et le faire \\
Le discernement \\
Le discontinu \\
Le discours politique \\
Le divers \\
Le divertissement \\
Le divin \\
Le dogmatisme \\
Le don \\
Le don de soi \\
Le don est-il toujours généreux ? \\
Le don est-il une modalité de l'échange ? \\
Le don et la dette \\
Le don et l'échange \\
Le donné \\
Le double \\
Le doute \\
Le doute dans les sciences \\
Le doute est-il le principe de la méthode scientifique ? \\
Le doute est-il une faiblesse de la pensée ? \\
Le doute peut-il être méthodique ? \\
Le drame \\
Le droit \\
Le droit à la citoyenneté \\
Le droit à la différence met-il en péril l'égalité des droits ? \\
Le droit à la paresse \\
Le droit à la révolte \\
Le droit à l'erreur \\
Le droit au bonheur \\
Le droit au Bonheur \\
Le droit au respect de la vie privée \\
Le droit au travail \\
Le droit d'auteur \\
Le droit de la guerre \\
Le droit de mentir \\
Le droit de propriété \\
Le droit de punir \\
Le droit de résistance \\
Le droit de révolte \\
Le droit des animaux \\
Le droit des gens \\
Le droit des peuples à disposer d'eux-mêmes \\
Le droit de veto \\
Le droit de vie et de mort \\
Le droit de vivre \\
Le droit d'ingérence \\
Le droit d'intervention \\
Le droit divin \\
Le droit doit-il être indépendant de la morale ? \\
Le droit doit-il être le seul régulateur de la vie sociale ? \\
Le droit du plus faible \\
Le droit du plus fort \\
Le droit du premier occupant \\
Le droit est-il facteur de paix ? \\
Le droit est-il une science humaine ? \\
Le droit est-il une science ? \\
Le droit et la convention \\
Le droit et la force \\
Le droit et la liberté \\
Le droit et la loi \\
Le droit et la morale \\
Le droit et le devoir \\
Le Droit et l'État \\
Le droit humanitaire \\
Le droit international \\
Le droit naturel \\
Le droit n'est-il qu'une justice par défaut ? \\
Le droit peut-il échapper à l'histoire ? \\
Le droit peut-il être flexible ? \\
Le droit peut-il être naturel ? \\
Le droit peut-il se fonder sur la force ? \\
Le droit peut-il se passer de la morale ? \\
Le droit positif \\
Le droit sert-il à établir l'ordre ou la justice ? \\
Le dualisme \\
L'éducation artistique \\
L'éducation civique \\
L'éducation des esprits \\
L'éducation du goût \\
L'éducation esthétique \\
L'éducation peut-elle être sentimentale ? \\
L'éducation physique \\
L'éducation politique \\
Le factice \\
Le fait \\
Le fait de vivre constitue-t-il un bien en soi ? \\
Le fait de vivre est-il un bien en soi ? \\
Le fait d'exister \\
Le fait divers \\
Le fait et le droit \\
Le fait et l'événement \\
Le fait religieux \\
Le fait scientifique \\
Le fait social est-il une chose ? \\
Le familier \\
Le fanatisme \\
Le fantasme \\
Le fantastique \\
Le fatalisme \\
Le fatalisme l'incarnation \\
Le faux \\
Le faux en art \\
Le faux et l'absurde \\
Le faux et le fictif \\
Le féminin \\
Le féminin et le masculin \\
Le féminisme \\
Le fétichisme \\
Le fétichisme de la marchandise \\
L'effectivité \\
L'effet et la cause \\
L'efficacité \\
L'efficacité des discours \\
L'efficacité thérapeutique de la psychanalyse \\
L'efficience \\
L'effort \\
L'effort moral \\
Le fil conducteur \\
Le finalisme \\
Le fini \\
Le fini et l'infini \\
Le fin mot de l'histoire \\
Le flegme \\
Le fond \\
Le fondement \\
Le fondement de l'autorité \\
Le fondement de l'induction \\
Le fond et la forme \\
Le for intérieur \\
Le formalisme \\
Le formalisme moral \\
Le fou \\
Le fragment \\
Le frivole \\
Le futur est-il contingent ? \\
L'égalité \\
L'égalité civile \\
L'égalité des chances \\
L'égalité des citoyens \\
L'égalité des conditions \\
L'égalité des hommes et des femmes est-elle une question politique ? \\
L'égalité des sexes \\
L'égalité devant la loi \\
L'égalité est-elle souhaitable ? \\
L'égalité est-elle toujours juste ? \\
L'égalité est-elle une condition de la liberté ? \\
Légalité et causalité \\
Légalité et légitimité \\
Légalité et moralité \\
L'égalité peut-elle être une menace pour la liberté ? \\
L'égarement \\
Le général et le particulier \\
Le génie \\
Le génie du lieu \\
Le génie du mal \\
Le génie est-il la marque de l'excellence artistique ? \\
Le génie et la règle \\
Le génie et le savant \\
Le genre \\
Le genre et l'espèce \\
Le genre humain \\
Le geste \\
Le geste créateur \\
Le geste et la parole \\
Légitimité et légalité \\
L'égoïsme \\
Le goût \\
Le goût de la liberté \\
Le goût de la polémique \\
Le goût des autres \\
Le goût du beau \\
Le goût du pouvoir \\
Le goût du risque \\
Le goût est-il affaire d'éducation ? \\
Le goût est-il une faculté ? \\
Le goût est-il une vertu sociale ? \\
Le goût s'éduque-t-il ? \\
Le goût se forme-t-il ? \\
Le goût : certitude ou conviction ? \\
Le gouvernement des experts \\
Le gouvernement des hommes et l'administration des choses \\
Le gouvernement des hommes libres \\
Le gouvernement des meilleurs \\
Le gouvernement de soi et des autres \\
Le grand art est-il de plaire ? \\
Le handicap \\
Le hasard \\
Le hasard est-il injuste ? \\
Le hasard et la nécessité \\
Le hasard existe-t-il ? \\
Le hasard fait-il bien les choses ? \\
Le hasard n'est il que la mesure de notre ignorance ? \\
Le hasard n'est-il que la mesure de notre ignorance ? \\
Le haut \\
Le haut et le bas \\
Le héros \\
Le héros moral \\
Le hors-la-loi \\
Le je et le tu \\
Le je ne sais quoi \\
Le jeu \\
Le jeu et le divertissement \\
Le jeu et le hasard \\
Le jeu et le sérieux \\
Le jeu social \\
Le joli, le beau \\
Le juge \\
Le jugement \\
Le jugement critique peut-il s'exercer sans culture ? \\
Le jugement de goût \\
Le jugement de goût est-il désintéressé ? \\
Le jugement de goût est-il universel ? \\
Le jugement dernier \\
Le jugement de valeur \\
Le jugement de valeur est-il indifférent à la vérité ? \\
Le jugement moral \\
Le jugement politique \\
Le juste et le bien \\
Le juste et le légal \\
Le juste milieu \\
Le laboratoire \\
Le langage \\
Le langage animal \\
Le langage de la morale \\
Le langage de la peinture \\
Le langage de la pensée \\
Le langage de l'art \\
Le langage de la science \\
Le langage des sciences \\
Le langage du corps \\
Le langage est-il assimilable à un outil ? \\
Le langage est-il d'essence poétique ? \\
Le langage est-il l'auxiliaire de la pensée ? \\
Le langage est-il le lieu de la vérité ? \\
Le langage est-il logique ? \\
Le langage est-il une prise de possession des choses ? \\
Le langage est-il un instrument de connaissance ? \\
Le langage est-il un instrument ? \\
Le langage est-il un obstacle pour la pensée ? \\
Le langage et la pensée \\
Le langage et le réel \\
Le langage masque-t-il la pensée ? \\
Le langage mathématique \\
Le langage ne sert-il qu'à communiquer ? \\
Le langage n'est-il qu'un instrument de communication ? \\
Le langage peut-il être un obstacle à la recherche de la vérité ? \\
Le langage rapproche-t-il ou sépare-t-il les hommes ? \\
Le langage rend-il l'homme plus puissant ? \\
Le langage traduit-il la pensée ? \\
Le langage trahit-il la pensée ? \\
L'élection \\
Le légal et le légitime \\
L'élégance \\
Le législateur \\
Le légitime et le légal \\
L'élémentaire \\
Le libre arbitre \\
Le libre-arbitre \\
Le libre échange \\
Le lien causal \\
Le lien politique \\
Le lien social \\
Le lien social peut-il être compassionnel ? \\
Le lieu \\
Le lieu commun \\
Le lieu de la pensée \\
Le lieu de l'esprit \\
Le lieu et l'espace \\
Le littéral et le figuré \\
Le livre \\
Le livre de la nature \\
L'éloge de la démesure \\
Le logique \\
Le loisir \\
Le luxe \\
Le lyrisme \\
Le maître \\
Le maître et l'esclave \\
Le mal \\
Le mal a-t-il des raisons ? \\
Le mal constitue-t-il une objection à l'existence de Dieu ? \\
Le malentendu \\
Le mal être \\
Le mal existe-t-il ? \\
Le malheur \\
Le malheur est-il injuste ? \\
Le malin plaisir \\
Le mal métaphysique \\
L'émancipation \\
L'émancipation des femmes \\
Le maniérisme \\
Le manifeste politique \\
Le manque de culture \\
Le marché \\
Le marché de l'art \\
Le marché du travail \\
Le mariage \\
Le mariage est-il un contrat ? \\
Le masculin \\
Le masculin et le féminin \\
Le masque \\
Le matérialisme \\
Le matériel \\
Le matériel et le virtuel \\
Le mauvais goût \\
L'embarras du choix \\
Le mécanisme \\
Le mécanisme et la mécanique \\
Le méchant \\
Le méchant est-il malheureux ? \\
Le méchant peut-il être heureux ? \\
Le médiat et l'immédiat \\
Le meilleur \\
Le meilleur des mondes \\
Le meilleur des mondes possible \\
Le meilleur est-il l'ennemi du bien ? \\
Le meilleur régime \\
Le meilleur régime politique \\
Le même et l'autre \\
Le mensonge \\
Le mensonge de l'art ? \\
Le mensonge en politique \\
Le mensonge est-il la plus grande transgression ? \\
Le mensonge peut-il être au service de la vérité ? \\
Le mensonge politique \\
Le mépris \\
Le mépris peut-il être justifié ? \\
Le mérite \\
Le mérite est-il le critère de la vertu ? \\
Le mérite et les talents \\
Le merveilleux \\
Le métaphysicien est-il un physicien à sa façon ? \\
Le métier \\
Le métier de philosophe \\
Le métier de politique \\
Le métier de savant \\
Le métier d'homme \\
Le mien et le tien \\
Le mieux est-il l'ennemi du bien ? \\
Le milieu \\
Le miracle \\
Le miroir \\
Le misanthrope \\
Le mode \\
Le mode d'existence de l'œuvre d'art \\
Le modèle \\
Le modèle en morale \\
Le modèle organiciste \\
Le moi \\
Le moi est-il haïssable ? \\
Le moi est-il objet de connaissance ? \\
Le moi est-il une fiction ? \\
Le moi est-il une illusion ? \\
Le moi et la conscience \\
Le moindre mal \\
Le moi n'est-il qu'une fiction ? \\
Le moi n'est-il qu'une idée ? \\
Le monde \\
Le monde à l'envers \\
Le monde a-t-il besoin de moi ? \\
Le monde de l'animal \\
Le monde de l'art \\
Le monde de la technique \\
Le monde de la vie \\
Le monde de l'entreprise \\
Le monde des idées \\
Le monde des images \\
Le monde des machines \\
Le monde des œuvres \\
Le monde des physiciens \\
Le monde des rêves \\
Le monde des sens \\
Le monde du rêve \\
Le monde du travail \\
Le monde est-il écrit en langage mathématique ? \\
Le monde est-il éternel ? \\
Le monde est-il ma représentation ? \\
Le monde est-il une marchandise ? \\
Le monde est-il un théâtre ? \\
Le monde extérieur \\
Le monde extérieur existe-t-il ? \\
Le monde intelligible \\
Le monde intérieur \\
Le monde politique \\
Le monde sensible \\
Le monde se réduit-il à ce que nous en voyons ? \\
Le monde vrai \\
Le monopole de la violence légitime \\
Le monstre \\
Le monstrueux \\
Le moralisme \\
Le moraliste \\
Le mot d'esprit \\
Le mot et la chose \\
Le mot et le geste \\
L'émotion \\
L'émotion esthétique \\
L'émotion esthétique peut-elle se communiquer ? \\
Le mot juste \\
Le mot vie a-t-il plusieurs sens ? \\
Le mouvement \\
Le mouvement de la pensée \\
L'empathie \\
L'empathie est-elle nécessaire aux sciences sociales ? \\
L'empathie est-elle possible ? \\
L'empire \\
L'empire sur soi \\
L'empirisme \\
L'empirisme exclut-il l'abstraction ? \\
L'emploi du temps \\
Le multiculturalisme \\
Le multiple \\
Le multiple et l'un \\
Le musée \\
Le Musée \\
Le mystère \\
Le mysticisme \\
Le mythe \\
Le mythe est-il objet de science ? \\
Le naïf \\
Le narcissisme \\
Le naturalisme des sciences humaines et sociales \\
Le naturel \\
Le naturel et l'artificiel \\
Le naturel et le fabriqué \\
L'encyclopédie \\
L'Encyclopédie \\
Le néant \\
Le néant est-il ? \\
Le nécessaire et le contingent \\
Le nécessaire et le superflu \\
Le négatif \\
Le néologisme \\
L'énergie \\
L'énergie du désespoir \\
L'enfance \\
L'enfance de l'art \\
L'enfance est-elle ce qui doit être surmonté ? \\
L'enfance est-elle en nous ce qui doit être abandonné ? \\
L'enfant \\
L'enfant et l'adulte \\
L'enfer est pavé de bonnes intentions \\
L'engagement \\
L'engagement dans l'art \\
L'engagement politique \\
L'engendrement \\
L'énigme \\
Le nihilisme \\
L'ennemi \\
L'ennemi intérieur \\
L'ennui \\
Le noble et le vil \\
Le nomade \\
Le nomadisme \\
Le nombre \\
Le nombre et la mesure \\
Le nom et le verbe \\
Le nominalisme \\
Le nom propre \\
Le non-être \\
Le non-sens \\
Le normal et le pathologique \\
L'enquête \\
L'enquête de terrain \\
L'enquête empirique rend-elle la métaphysique inutile ? \\
L'enquête sociale \\
L'enseignement peut-il se passer d'exemples ? \\
L'entendement et la volonté \\
L'enthousiasme \\
L'enthousiasme est-il moral ? \\
L'entraide \\
Le nu \\
Le nu et la nudité \\
L'envie \\
L'environnement \\
L'environnement est-il un nouvel objet pour les sciences humaines ? \\
L'environnement est-il un problème politique ? \\
Le oui-dire \\
Le pacifisme \\
Le paradigme \\
Le paradoxe \\
Le pardon \\
Le pardon et l'oubli \\
Le pardon peut-il être une obligation ? \\
Le pari \\
Le partage \\
Le partage des biens \\
Le partage des connaissances \\
Le partage des savoirs \\
Le partage est-il une obligation morale ? \\
Le particulier \\
Le passage à l'acte \\
Le passé \\
Le passé a-t-il plus de réalité que l'avenir ? \\
Le passé a-t-il un intérêt ? \\
Le passé détermine-t-il notre présent ? \\
Le passé est-il ce qui a disparu ? \\
Le passé est-il réel ? \\
Le passé et le présent \\
Le passé existe-t-il ? \\
Le passé peut-il être un objet de connaissance ? \\
Le paternalisme \\
Le pathologique \\
Le patriarcat \\
Le patrimoine \\
Le patrimoine artistique \\
Le patrimoine de l'humanité \\
Le patriotisme \\
Le patriotisme est-il une vertu ? \\
Le paysage \\
Le pays natal \\
Le péché \\
Le pédagogue \\
Le personnage et la personne \\
Le pessimisme \\
Le peuple \\
Le peuple est-il bête ? \\
Le peuple et la nation \\
Le peuple et les élites \\
Le peuple peut-il se tromper ? \\
Le phantasme \\
L'éphémère \\
Le phénomène \\
Le philanthrope \\
Le philosophe a-t-il besoin de l'histoire ?Prouver et justifier \\
Le philosophe a-t-il des leçons à donner au politique ? \\
Le philosophe est-il le vrai politique ? \\
Le philosophe et le sophiste \\
Le philosophe-roi \\
Le philosophe s'écarte-t-il du réel ? \\
L'épistémologie est-elle une logique de la science ? \\
Le plagiat \\
Le plaisir \\
Le plaisir a-t-il un rôle à jouer dans la morale ? \\
Le plaisir d'avoir mal \\
Le plaisir de l'art \\
Le plaisir de parler \\
Le plaisir des sens \\
Le plaisir d'être libre \\
Le plaisir d'imiter \\
Le plaisir esthétique \\
Le plaisir esthétique peut-il se partager ? \\
Le plaisir esthétique suppose-t-il une culture ? \\
Le plaisir est-il immoral ? \\
Le plaisir est-il la fin du désir ? \\
Le plaisir est-il tout le bonheur ? \\
Le plaisir est-il un bien ? \\
Le plaisir et la douleur \\
Le plaisir et la joie \\
Le plaisir et la jouissance \\
Le plaisir et la peine \\
Le plaisir et le bien \\
Le plaisir peut-il être immoral ? \\
Le plaisir peut-il être partagé ? \\
Le plaisir suffit-il au bonheur ? \\
Le pluralisme \\
Le pluralisme politique \\
Le plus grand bonheur pour le plus grand nombre \\
Le poète réinvente-t-il la langue ? \\
Le poétique \\
Le poids de la culture \\
Le poids de la société \\
Le poids du passé \\
Le poids du préjugé en politique \\
Le point de vue \\
Le point de vue d'autrui \\
Le point de vue de l'auteur \\
Le politique a-t-il à régler les passions humaines ? \\
Le politique doit-il être un technicien ? \\
Le politique doit-il se soucier des émotions ? \\
Le politique et le religieux \\
Le politique peut-il faire abstraction de la morale ? \\
Le populaire \\
Le populisme \\
Le portrait \\
Le possible \\
Le possible et le probable \\
Le possible et le réel \\
Le possible et le virtuel \\
Le possible et l'impossible \\
Le possible existe-t-il ? \\
Le pour et le contre \\
Le pourquoi et le comment \\
Le pouvoir \\
Le pouvoir absolu \\
Le pouvoir causal de l'inconscient \\
Le pouvoir corrompt-il nécessairement ? \\
Le pouvoir corrompt-il toujours ? \\
Le pouvoir corrompt-il ? \\
Le pouvoir de la science \\
Le pouvoir de l'État est-il arbitraire ? \\
Le pouvoir de l'habitude \\
Le pouvoir de l'imagination \\
Le pouvoir de l'opinion \\
Le pouvoir des images \\
Le pouvoir des mots \\
Le pouvoir des paroles \\
Le pouvoir des sciences humaines et sociales \\
Le pouvoir du concept \\
Le pouvoir du peuple \\
Le pouvoir et l'autorité \\
Le pouvoir et la violence \\
Le pouvoir législatif \\
Le pouvoir magique \\
Le pouvoir peut-il être limité ? \\
Le pouvoir peut-il limiter le pouvoir ? \\
Le pouvoir peut-il se déléguer ? \\
Le pouvoir peut-il se passer de sa mise en scène ? \\
Le pouvoir politique est-il nécessairement coercitif ? \\
Le pouvoir politique peut-il échapper à l'arbitraire ? \\
Le pouvoir politique repose-t-il sur un savoir ? \\
Le pouvoir souverain \\
Le pouvoir traditionnel \\
Le pragmatisme \\
Le préférable \\
Le préjugé \\
Le premier \\
Le premier devoir de l'État est-il de se défendre ? \\
Le premier et le primitif \\
Le premier principe \\
Le présent \\
L'épreuve \\
L'épreuve de la liberté \\
L'épreuve du réel \\
Le primitif \\
Le primitivisme en art \\
Le prince \\
Le principe \\
Le principe de causalité \\
Le principe de contradiction \\
Le principe d'égalité \\
Le principe de non-contradiction \\
Le principe de raison \\
Le principe de raison suffisante \\
Le principe de réalité \\
Le principe de réciprocité \\
Le principe d'identité \\
Le privé et le public \\
Le privilège de l'original \\
Le prix de la liberté \\
Le prix des choses \\
Le prix du travail \\
Le probable \\
Le problème de l'être \\
Le procès d'intention \\
Le processus \\
Le processus de civilisation \\
Le prochain \\
Le proche et le lointain \\
Le profane \\
Le profit \\
Le profit est-il la fin de l'échange ? \\
Le progrès \\
Le progrès des sciences \\
Le progrès des sciences infirme-t-il les résultats anciens ? \\
Le progrès en logique \\
Le progrès est-il réversible ? \\
Le progrès est-il un mythe ? \\
Le progrès moral \\
Le progrès scientifique fait-il disparaître la superstition ? \\
Le progrès technique \\
Le progrès technique peut-il être aliénant ? \\
Le projet \\
Le projet d'une paix perpétuelle est-il insensé ? \\
Le propre \\
Le propre de la musique \\
Le propre de l'homme \\
Le propre du vivant est-il de tomber malade ? \\
Le propriétaire \\
Le provisoire \\
Le psychisme est-il objet de connaissance ? \\
Le public \\
Le public et le privé \\
L e pur et l'impur \\
Le pur et l'impur \\
Le quelconque \\
Lequel, de l'art ou du réel, est-il une imitation de l'autre ? \\
L'équilibre des pouvoirs \\
L'équité \\
L'équivalence \\
L'équivocité \\
L'équivocité du langage \\
L'équivoque \\
Le quotidien \\
Le racisme \\
Le raffinement \\
Le raisonnable et le rationnel \\
Le raisonnement par l'absurde \\
Le raisonnement scientifique \\
Le raisonnement suit-il des règles ? \\
Le rapport de l'homme à son milieu a-t-il une dimension morale ? \\
Le rationalisme \\
Le rationalisme peut-il être une passion ? \\
Le rationnel \\
Le rationnel et le raisonnable \\
Le rationnel et l'irrationnel \\
Le réalisme \\
Le réalisme de la science \\
Le récit \\
Le récit en histoire \\
Le récit historique \\
Le reconnaissance \\
Le recours à l'Histoire \\
Le réel \\
Le réel est-il ce que l'on croit ? \\
Le réel est-il ce que nous expérimentons ? \\
Le réel est-il ce que nous percevons ? \\
Le réel est-il ce qui apparaît ? \\
Le réel est-il ce qui est perçu ? \\
Le réel est-il ce qui résiste ? \\
Le réel est-il inaccessible ? \\
Le réel est-il l'objet de la science ? \\
Le réel est-il objet d'interprétation ? \\
Le réel est-il rationnel ? \\
Le réel et la fiction \\
Le réel et le matériel \\
Le réel et le nécessaire \\
Le réel et le possible \\
Le réel et le virtuel \\
Le réel et le vrai \\
Le réel et l'idéal \\
Le réel et l'imaginaire \\
Le réel et l'irréel \\
Le réel n'est-il qu'un ensemble de contraintes ? \\
Le réel obéit-il à la raison ? \\
Le réel peut-il échapper à la logique ? \\
Le réel peut-il être contradictoire ? \\
Le réel résiste-t-il à la connaissance ? \\
Le réel se limite-t-il à ce que font connaître les théories scientifiques ? \\
Le réel se limite-t-il à ce que nous percevons ? \\
Le réel se réduit-il à ce que l'on perçoit ? \\
Le réel se réduit-il à l'objectivité ? \\
Le refoulement \\
Le refus \\
Le refus de la vérité \\
Le regard \\
Le regard du photographe \\
Le règlement politique des conflits ? \\
Le règne de l'homme \\
Le règne des passions \\
Le regret \\
Le relativisme \\
Le relativisme culturel \\
Le relativisme moral \\
Le religieux est-il inutile ? \\
Le remords \\
Le renoncement \\
Le repentir \\
Le repos \\
Le respect \\
Le respect de la nature \\
Le respect des convenances \\
Le respect des institutions \\
Le respect de soi \\
Le respect de soi-même \\
Le ressentiment \\
Le retour à la nature \\
Le retour à l'expérience \\
Le rêve \\
Le rêve et la réalité \\
Le rêve et la veille \\
Le riche et le pauvre \\
Le ridicule \\
Le rien \\
Le rigorisme \\
Le rire \\
Le risque \\
Le risque de la liberté \\
Le risque technique \\
Le rite \\
Le rituel \\
Le rôle de la théorie dans l'expérience scientifique \\
Le rôle de l'État est-il de faire régner la justice ? \\
Le rôle de l'État est-il de préserver la liberté de l'individu ? \\
Le rôle de l'historien est-il de juger ? \\
Le rôle des institutions \\
Le rôle des théories est-il d'expliquer ou de décrire ? \\
Le roman \\
Le roman peut-il être philosophique ? \\
Le romantisme \\
L'érotisme \\
Le royaume du possible \\
L'erreur \\
L'erreur est-elle humaine ? \\
L'erreur et la faute \\
L'erreur et l'ignorance \\
L'erreur et l'illusion \\
L'erreur peut-elle jouer un rôle dans la connaissance scientifique ? \\
L'erreur politique, la faute politique \\
L'erreur scientifique \\
L'érudition \\
Le rythme \\
Le sacré \\
Le sacré et le profane \\
Le sacrifice \\
Le sacrifice de soi \\
Les acteurs de l'histoire en sont-ils les auteurs ? \\
Les affaires publiques \\
Les affects sont-ils déraisonnables ? \\
Les affects sont-ils des objets sociologiques ? \\
Le sage a-t-il besoin d'autrui ? \\
Les agents sociaux poursuivent-ils l'utilité ? \\
Les agents sociaux sont-ils rationnels ? \\
Les âges de la vie \\
Les âges de l'humanité \\
Le salaire \\
Le salut \\
Le salut vient-il de la raison ? \\
Les amis \\
Les analogies dans les sciences humaines \\
Les anciens et les modernes \\
Les Anciens et les Modernes \\
Les animaux échappent-ils à la moralité ? \\
Les animaux ont-ils des droits ? \\
Les animaux pensent-ils ? \\
Les animaux peuvent-ils avoir des droits ? \\
Les antagonismes sociaux \\
Les apparences font-elles partie du monde ? \\
Les apparences sont-elles toujours trompeuses ? \\
Les archives \\
Les arts admettent-ils une hiérarchie ? \\
Les arts appliqués \\
Les arts communiquent-ils entre eux ? \\
Les arts de la mémoire \\
Les arts industriels \\
Les arts mineurs \\
Les arts nobles \\
Les arts ont-ils besoin de théorie ? \\
Les arts populaires \\
Les arts vivants \\
Le sauvage \\
Le sauvage et le barbare \\
Le sauvage et le cultivé \\
Le savant et le politique \\
Le savant et l'ignorant \\
Les avant-gardes \\
Le savoir absolu \\
Le savoir a-t-il besoin d'être fondé ? \\
Le savoir a-t-il des degrés ? \\
Le savoir du peintre \\
Le savoir émancipe-t-il ? \\
Le savoir est-il libérateur ? \\
Le savoir exclut-il toute forme de croyance ? \\
Le savoir-faire \\
Le savoir rend-il libre ? \\
Le savoir se vulgarise-t-il ? \\
Le savoir utile au citoyen \\
Les beautés de la nature \\
Les beaux-arts \\
Les beaux-arts sont-ils compatibles entre eux ? \\
Les belles choses \\
Les bénéfices du doute \\
Les bénéfices moraux \\
Les besoins et les désirs \\
Les bêtes travaillent-elles ? \\
Les bienfaits de la coopération \\
Les biens communs \\
Les biotechnologies \\
Les blessures de l'esprit \\
Les bonnes intentions \\
Les bonnes manières \\
Les bonnes mœurs \\
Les bonnes résolutions \\
Les bons sentiments \\
Le scandale \\
Les caractères \\
Les caractères moraux \\
Les catastrophes \\
Les catégories \\
Les causes et les effets \\
Les causes et les lois \\
Les causes et les raisons \\
Les causes et les signes \\
Les causes finales \\
Le scepticisme \\
Le scepticisme a-t-il des limites ? \\
Les cérémonies \\
Les changements scientifiques et la réalité \\
Les chemins de traverse \\
Les choses \\
Les choses et les événements \\
Les choses ont-elles une essence ? \\
Les cinq sens \\
Les circonstances \\
Les classes sociales \\
L'esclavage \\
L'esclavage des passions \\
L'esclave \\
L'esclave et son maître \\
Les coïncidences ont-elles des causes ? \\
Les commandements divins \\
Les commencements \\
Les conditions de la démocratie \\
Les conditions d'existence \\
Les conditions du dialogue \\
Les conflits menacent-ils la société ? \\
Les conflits politiques \\
Les conflits politiques ne sont-ils que des conflits sociaux ? \\
Les conflits sociaux \\
Les conflits sociaux sont-ils des conflits de classe ? \\
Les conflits sociaux sont-ils des conflits politiques ? \\
Les connaissances scientifiques peuvent-elles être à la fois vraies et provisoires ? \\
Les connaissances scientifiques peuvent-elles être vulgarisées ? \\
Les conquêtes de la science \\
Les conséquences \\
Les conséquences de l'action \\
Les considérations morales ont-elles leur place en politique ? \\
Les convictions d'autrui sont-elles un argument ? \\
Les coutumes \\
Les critères de vérité dans les sciences humaines \\
Les croyances politiques \\
Le scrupule \\
Les cultures sont-elles incommensurables ? \\
Les degrés de conscience \\
Les degrés de la beauté \\
Les désirs et les valeurs \\
Les désirs ont-ils nécessairement un objet \\
Les devoirs à l'égard de la nature \\
Les devoirs de l'État \\
Les devoirs de l'homme varient-ils selon la culture ? \\
Les devoirs de l'homme varient-ils selon les cultures ? \\
Les devoirs du citoyen \\
Les devoirs envers soi-même \\
Les dictionnaires \\
Les dilemmes moraux \\
Les disciplines scientifiques et leurs interfaces \\
Les dispositions sociales \\
Les distinctions sociales \\
Les divisions sociales \\
Les droits de l'enfant \\
Les droits de l'homme \\
Les droits de l'homme et ceux du citoyen \\
Les droits de l'homme ont-ils un fondement moral ? \\
Les droits de l'homme sont-ils les droits de la femme ? \\
Les droits de l'homme sont-ils une abstraction ? \\
Les droits de l'individu \\
Les droits des animaux \\
Les droits et les devoirs \\
Les droits naturels imposent-ils une limite à la politique ? \\
Les échanges \\
Les échanges économiques sont-ils facteurs de paix ? \\
Les échanges, facteurs de paix ? \\
Les échanges favorisent-ils la paix ? \\
Les échanges sont-ils facteurs de paix ? \\
Les écrans \\
Le secret \\
Le secret d'État \\
Les effets de l'esclavage \\
Les éléments \\
Les élites \\
Le semblable \\
Les enfants \\
Le sensationnel \\
Le sens caché \\
Le sens commun \\
Le sens de la justice \\
Le sens de la mesure \\
Le sens de la situation \\
Le sens de la vie \\
Le sens de l'État \\
Le sens de l'existence \\
Le sens de l'histoire \\
Le sens de l'Histoire \\
Le sens de l'humour \\
Le sens des mots \\
Le sens du destin \\
Le sens du devoir \\
Le sens du silence \\
Le sens du travail \\
Les ensembles \\
Le sensible \\
Le sensible est-il communicable ? \\
Le sensible est-il irréductible à l'intelligible ? \\
Le sensible et la science \\
Le sensible et l'intelligible \\
Le sensible peut-il être connu ? \\
Le sens interne \\
Le sens moral \\
Le sens moral est-il naturel ? \\
Le sens musical \\
Le sentiment \\
Le sentiment de culpabilité \\
Le sentiment de l'existence \\
Le sentiment de liberté \\
Le sentiment de l'injustice \\
Le sentiment d'injustice \\
Le sentiment d'injustice est-il naturel ? \\
Le sentiment du juste et de l'injuste \\
Le sentiment esthétique \\
Le sentiment moral \\
Les entités mathématiques sont-elles des fictions ? \\
Les envieux \\
Le sérieux \\
Le serment \\
Le service de l'État \\
Les êtres vivants sont-ils des machines ? \\
Les études comparatives \\
Les événements historiques sont-ils de nature imprévisible ? \\
Les excuses \\
Les factions politiques \\
Les faits et les valeurs \\
Les faits existent-ils indépendamment de leur établissement par l'esprit humain ? \\
Les faits parlent-ils d'eux-mêmes ? \\
Les faits peuvent-ils faire autorité ? \\
Les fausses sciences \\
Les fins de la culture \\
Les fins de l'art \\
Les fins de la science \\
Les fins de la technique sont-elles techniques ? \\
Les fins de l'éducation \\
Les fins dernières \\
Les fins et les moyens \\
Les fins naturelles et les fins morales \\
Les fonctions de l'image \\
Les fondements de l'État \\
Les formes du vivant \\
Les forts et les faibles \\
Les foules \\
Les fous \\
Les frontières \\
Les frontières de l'art \\
Les fruits du travail \\
Les générations \\
Les genres de Dieu \\
Les genres de vie \\
Les genres esthétiques \\
Les genres naturels \\
Les grands hommes \\
Les habitudes nous forment-elles ? \\
Les hasards de la vie \\
Les héros \\
Les hommes de pouvoir \\
Les hommes et les dieux \\
Les hommes et les femmes \\
Les hommes n'agissent-ils que par intérêt ? \\
Les hommes naissent-ils libres ? \\
Les hommes ont-ils besoin de maîtres ? \\
Les hommes savent-ils ce qu'ils désirent ? \\
Les hommes sont-ils des animaux ? \\
Les hommes sont-ils faits pour s'entendre ? \\
Les hommes sont-ils frères ? \\
Les hommes sont-ils naturellement sociables ? \\
Les hommes sont-ils seulement le produit de leur culture ? \\
Les hors-la-loi \\
Les hypothèses scientifiques ont-elles pour nature d'être confirmées ou infirmées ? \\
Les idées et les choses \\
Les idées ont-elles une existence éternelle ? \\
Les idées ont-elles une histoire ? \\
Les idées ont-elles une réalité ? \\
Les idées politiques \\
Les idoles \\
Le signe \\
Le signe et le symbole \\
Le silence \\
Le silence a-t-il un sens ? \\
Le silence des lois \\
Le silence signifie-t-il toujours l'échec du langage ? \\
Les images empêchent-elles de penser ? \\
Les images ont-elles un sens ? \\
Le simple \\
Le simple et le complexe \\
Le simulacre \\
Les individus \\
Les industries culturelles \\
Les inégalités de la nature doivent-elles être compensées ? \\
Les inégalités menacent-elles la société ? \\
Les inégalités sociales \\
Les inégalités sociales sont-elles inévitables ? \\
Les inégalités sociales sont-elles naturelles ? \\
Le singulier \\
Le singulier est-il objet de connaissance ? \\
Le singulier et le pluriel \\
Les institutions artistiques \\
Les instruments de la pensée \\
Les intentions de l'artiste \\
Les intentions et les actes \\
Les intentions et les conséquences \\
Les interdits \\
Les intérêts particuliers peuvent-ils tempérer l'autorité politique ? \\
Les invariants culturels \\
Les jeux du pouvoir \\
Les jugements analytiques \\
Les langues que nous parlons sont-elles imparfaites ? \\
Les leçons de l'expérience \\
Les leçons de l'histoire \\
Les leçons de morale \\
Les lettres et les sciences \\
Les libertés civiles \\
Les libertés fondamentales \\
Les liens sociaux \\
Les lieux du pouvoir \\
Les limites de la connaissance \\
Les limites de la connaissance scientifique \\
Les limites de la démocratie \\
Les limites de la description \\
Les limites de la discussion \\
Les limites de la raison \\
Les limites de la science \\
Les limites de la tolérance \\
Les limites de la vérité \\
Les limites de la vertu \\
Les limites de l'État \\
Les limites de l'expérience \\
Les limites de l'humain \\
Les limites de l'imaginaire \\
Les limites de l'imagination \\
Les limites de l'interprétation \\
Les limites de l'obéissance \\
Les limites du corps \\
Les limites du langage \\
Les limites du pouvoir \\
Les limites du pouvoir politique \\
Les limites du réel \\
Les limites du vivant \\
Les livres \\
Les lois \\
Les lois causales \\
Les lois de la guerre \\
Les lois de la nature \\
Les lois de la nature sont-elles contingentes ? \\
Les lois de la nature sont-elles de simples régularités ? \\
Les lois de la nature sont elles nécessaires ? \\
Les lois de la pensée \\
Les lois de l'art \\
Les lois de l'histoire \\
Les lois de l'hospitalité \\
Les lois du sang \\
Les lois et les armes \\
Les lois et les mœurs \\
Les lois nous rendent-elles meilleurs ? \\
Les lois scientifiques sont-elles des lois de la nature ? \\
Les lois sont-elles seulement utiles ? \\
Les machines \\
Les machines nous rendent-elles libres ? \\
Les machines pensent-elles ? \\
Les maladies de l'âme \\
Les maladies de l'esprit \\
Les marginaux \\
Les matériaux \\
Les mathématiques consistent-elles seulement en des opérations de l'esprit ? \\
Les mathématiques du mouvement \\
Les mathématiques et la pensée de l'infini \\
Les mathématiques et la quantité \\
Les mathématiques et l'expérience \\
Les mathématiques ont-elles affaire au réel ? \\
Les mathématiques ont-elles besoin d'un fondement ? \\
Les mathématiques parlent-elles du réel ? \\
Les mathématiques se réduisent-elles à une pensée cohérente ? \\
Les mathématiques sont-elles réductibles à la logique ? \\
Les mathématiques sont-elles un instrument ? \\
Les mathématiques sont-elles un jeu de l'esprit ? \\
Les mathématiques sont-elles un langage ? \\
Les mathématiques sont-elles utiles au philosophe ? \\
Les mécanismes cérébraux \\
Les méchants peuvent-ils être amis ? \\
Les modalités \\
Les modèles \\
Les mœurs \\
Les mœurs et la morale \\
Les mondes possibles \\
Les monstres \\
Les morts \\
Les mots disent-ils les choses ? \\
Les mots et la signification \\
Les mots et les choses \\
Les mots et les concepts \\
Les mots expriment-ils les choses ? \\
Les mots justes \\
Les mots nous éloignent-ils des choses ? \\
Les mots parviennent-ils à tout exprimer ? \\
Les mots sont-ils trompeurs ? \\
Les moyens de l'autorité \\
Les moyens et la fin \\
Les moyens et les fins \\
Les moyens et les fins en art \\
Les nombres gouvernent-ils le monde ? \\
Les noms \\
Les noms propres \\
Les noms propres ont-ils une signification ? \\
Les normes \\
Les normes du vivant \\
Les normes esthétiques \\
Les normes et les valeurs \\
Les nouvelles technologies transforment-elles l'idée de l'art ? \\
Les objets de la pensée \\
Les objets scientifiques \\
Les objets sont-ils colorés ? \\
Le social et le politique \\
Les œuvres d'art ont-elles besoin d'un commentaire ? \\
Les œuvres d'art sont-elles des choses ? \\
Les œuvres d'art sont-elles des réalités comme les autres ? \\
Les œuvres d'art sont-elles éternelles ? \\
Le soi et le je \\
Le soin \\
Le soldat \\
Le soleil se lèvera-t-il demain ? \\
Le solipsisme \\
Le sommeil \\
Le sommeil de la raison \\
Le sommeil et la veille \\
Les opérations de la pensée \\
Les opérations de l'esprit \\
Le sophiste et le philosophe \\
Les opinions politiques \\
L'ésotérisme \\
Le souci \\
Le souci d'autrui résume-t-il la morale ? \\
Le souci de l'avenir \\
Le souci de soi \\
Le souci de soi est-il une attitude morale ? \\
Le souci du bien-être est-il politique ? \\
Le soupçon \\
Les outils \\
Le souvenir \\
Le souverain bien \\
L'espace \\
L'espace de la perception \\
L'espace et le lieu \\
L'espace et le territoire \\
L'espace nous sépare-t-il ? \\
L'espace public \\
Les paroles et les actes \\
Les parties de l'âme \\
Les passions ont-elles une place en politique ? \\
Les passions peuvent-elles être raisonnables ? \\
Les passions politiques \\
Les passions sont-elles toujours mauvaises ? \\
Les passions sont-elles toutes bonnes ? \\
Les passions sont-elles un obstacle à la vie sociale ? \\
Les passions s'opposent-elles à la raison ? \\
Les pauvres \\
L'espèce et l'individu \\
L'espèce humaine \\
Le spectacle \\
Le spectacle de la nature \\
Le spectacle de la pensée \\
Le spectacle du monde \\
L'espérance \\
L'espérance est-elle une vertu ? \\
Les personnages de fiction peuvent-ils avoir une réalité ? \\
Les peuples font-ils l'histoire ? \\
Les peuples ont-ils les gouvernements qu'ils méritent ? \\
Les phénomènes \\
Les phénomènes inconscients sont-ils réductibles à une mécanique cérébrale ? \\
Les philosophes doivent-ils être rois ? \\
Les philosophies se classent-elles ? \\
Le spirituel et le temporel \\
Les plaisirs \\
Les plaisirs de l'amitié \\
Les poètes et la cité \\
L'espoir \\
L'espoir et la crainte \\
L'espoir peut-il être raisonnable ? \\
Le sport \\
Le sport : s'accomplir ou se dépasser ? \\
Les possibles \\
Les pouvoirs de la religion \\
Les préjugés moraux \\
Les prêtres \\
Les preuves de la liberté \\
Les preuves de l'existence de Dieu \\
Les principes \\
Les principes de la démonstration \\
Les principes de la morale dépendent-ils de la culture ? \\
Les principes d'une science sont-ils des conventions ? \\
Les principes et les éléments \\
Les principes moraux \\
Les principes sont-ils indémontrables ?Qu'est-ce qu'être ensemble ? \\
L'esprit \\
L'esprit critique \\
L'esprit de corps \\
L'esprit de finesse \\
L'esprit dépend-il du corps ? \\
L'esprit des lois \\
L'esprit de système \\
L'esprit d'invention \\
L'esprit domine-t-il la matière ? \\
L'esprit du christianisme \\
L'esprit est-il matériel ? \\
L'esprit est-il mieux connu que le corps ? \\
L'esprit est-il objet de science ? \\
L'esprit est-il plus aisé à connaître que le corps ? \\
L'esprit est-il plus difficile à connaître que la matière ? \\
L'esprit est-il une chose ? \\
L'esprit est-il une machine ? \\
L'esprit est-il un ensemble de facultés ? \\
L'esprit est-il une partie du corps ? \\
L'esprit et la lettre \\
L'esprit et la machine \\
L'esprit et le cerveau \\
L'esprit humain progresse-t-il ? \\
L'esprit n'a-t-il jamais affaire qu'à lui-même ? \\
L'esprit peut-il être divisé ? \\
L'esprit peut-il être malade ? \\
L'esprit peut-il être mesuré ? \\
L'esprit peut-il être objet de science ? \\
L'esprit scientifique \\
L'esprit s'explique-t-il par une activité cérébrale ? \\
L'esprit tranquille \\
Les problèmes politiques peuvent-ils se ramener à des problèmes techniques ? \\
Les problèmes politiques sont-ils des problèmes techniques ? \\
Les progrès de la technique sont-ils nécessairement des progrès de la raison ? \\
Les progrès techniques constituent-ils des progrès de la civilisation ? \\
Les propositions métaphysiques sont-elles des illusions ? \\
Les proverbes \\
Les proverbes enseignent-ils quelque chose ? \\
Les proverbes nous instruisent-ils moralement ? \\
Les qualités esthétiques \\
Les qualités sensibles sont-elles dans les choses ou dans l'esprit ? \\
Les questions métaphysiques ont-elles un sens ? \\
L'esquisse \\
Les raisons de croire \\
Les raisons d'espérer \\
Les raisons de vivre \\
Les raisons du choix \\
Les rapports entre les hommes sont-ils des rapports de force ? \\
Les règles de l'art \\
Les règles du jeu \\
Les règles d'un bon gouvernement \\
Les règles sociales \\
Les relations \\
Les religions peuvent-elles être objets de science ? \\
Les religions sont-elles des illusions ? \\
Les représentants du peuple \\
Les reproductions \\
Les ressources humaines \\
Les ressources naturelles \\
Les révolutions scientifiques \\
Les révolutions techniques suscitent-elles des révolutions dans l'art ? \\
Les riches et les pauvres \\
Les rituels \\
Les robots \\
Les rôles sociaux \\
Les ruines \\
Les sacrifices \\
Les sauvages \\
Les scélérats peuvent-ils être heureux ? \\
Les sciences appliquées \\
Les sciences décrivent-elles le réel ? \\
Les sciences de la vie et de la Terre \\
Les sciences de la vie visent-elles un objet irréductible à la matière ? \\
Les sciences de l'éducation \\
Les sciences de l'esprit \\
Les sciences de l'homme et l'évolution \\
Les sciences de l'homme ont-elles inventé leur objet ? \\
Les sciences de l'homme permettent-elles d'affiner la notion de responsabilité ? \\
Les sciences de l'homme peuvent-elles expliquer l'impuissance de la liberté ? \\
Les sciences de l'homme rendent-elles l'homme prévisible ? \\
Les sciences doivent-elle prétendre à l'unification ? \\
Les sciences du comportement \\
Les sciences et le vivant \\
Les sciences exactes \\
Les sciences forment-elle un système ? \\
Les sciences historiques \\
Les sciences humaines doivent-elles être transdisciplinaires ? \\
Les sciences humaines éliminent-elles la contingence du futur ? \\
Les sciences humaines et le droit \\
Les sciences humaines nous protègent-elles de l'essentialisme ? \\
Les sciences humaines ont-elles un objet commun ? \\
Les sciences humaines permettent-elles de comprendre la vie d'un homme ? \\
Les sciences humaines peuvent-elles adopter les méthodes des sciences de la nature ? \\
Les sciences humaines peuvent-elles se passer de la notion d'inconscient ? \\
Les sciences humaines présupposent-elles une définition de l'homme ? \\
Les sciences humaines sont-elles des sciences de la nature humaine ? \\
Les sciences humaines sont-elles des sciences de la vie humaine ? \\
Les sciences humaines sont-elles des sciences d'interprétation ? \\
Les sciences humaines sont-elles des sciences ? \\
Les sciences humaines sont-elles explicatives ou compréhensives ? \\
Les sciences humaines sont-elles normatives ? \\
Les sciences humaines sont-elles relativistes ? \\
Les sciences humaines sont-elles subversives ? \\
Les sciences humaines traitent-elles de l'homme ? \\
Les sciences humaines traitent-elles de l'individu ? \\
Les sciences humaines transforment-elles la notion de causalité ? \\
Les sciences naturelles \\
Les sciences ne sont-elles qu'une description du monde ? \\
Les sciences nous donnent-elles des normes ? \\
Les sciences ont-elles besoin de principes fondamentaux ? \\
Les sciences ont-elles besoin d'une fondation métaphysique ? \\
Les sciences permettent-elles de connaître la réalité-même ? \\
Les sciences peuvent-elles exclure toute notion de finalité ? \\
Les sciences peuvent-elles penser l'individu ? \\
Les sciences sociales \\
Les sciences sociales ont-elles un objet ? \\
Les sciences sociales peuvent-elles être expérimentales ? \\
Les sciences sociales sont-elles nécessairement inexactes ? \\
Les sciences sont-elles une description du monde ? \\
L'essence \\
L'essence de la technique \\
L'essence et l'existence \\
Les sens jugent-ils ? \\
Les sens nous trompent-ils ? \\
Les sens peuvent-ils nous tromper ? \\
Les sens sont-ils source d'illusion ? \\
Les sens sont-ils trompeurs ? \\
L'essentiel \\
Les sentiments \\
Les sentiments ont-ils une histoire ? \\
Les sentiments peuvent-ils s'apprendre ? \\
Les services publics \\
Les signes de l'intelligence \\
Les sociétés évoluent-elles ? \\
Les sociétés ont-elles un inconscient ? \\
Les sociétés sont-elles hiérarchisables ? \\
Les sociétés sont-elles imprévisibles ? \\
Les structures expliquent-elles tout ? \\
Les systèmes \\
Le statut de l'axiome \\
Le statut des hypothèses dans la démarche scientifique \\
Les techniques artistiques \\
Les techniques du corps \\
Les témoignages et la preuve \\
Les théories scientifiques décrivent-elles la réalité ? \\
Les théories scientifiques sont-elles vraies ? \\
L'esthète \\
L'esthétique \\
L'esthétique est-elle une métaphysique de l'art ? \\
L'esthétisme \\
L'estime de soi \\
Les traditions \\
Le style \\
Le style et le beau \\
Le sublime \\
Le substitut \\
Le succès \\
Le suffrage universel \\
Le sujet \\
Le sujet de droit \\
Le sujet de l'action \\
Le sujet de la pensée \\
Le sujet de l'histoire \\
Le sujet et l'individu \\
Le sujet et l'objet \\
Le sujet moral \\
Le sujet n'est-il qu'une fiction ? \\
Le sujet peut-il s'aliéner par un libre choix ? \\
Les universaux \\
Le surnaturel \\
Les usages de l'art \\
Les valeurs de la République \\
Les valeurs morales ont-elles leur origine dans la raison ? \\
Les valeurs universelles \\
Les vérités empiriques \\
Les vérités éternelles \\
Les vérités scientifiques sont-elles relatives ? \\
Les vérités sont-elles intemporelles ? \\
Les vertus \\
Les vertus cardinales \\
Les vertus de l'amour \\
Les vertus du commerce \\
Les vertus ne sont-elles que des vices déguisés ? \\
Les vertus politiques \\
Les vices privés peuvent-ils faire le bien public ? \\
Les visages du mal \\
Les vivants \\
Les vivants et les morts \\
Les vivants peuvent-ils se passer des morts ? \\
Le syllogisme \\
Le symbole \\
Le symbolisme \\
Le symbolisme mathématique \\
Le système \\
Le système des arts \\
Le système des beaux-arts \\
Le système des besoins \\
Le tableau \\
Le tableau ? \\
Le tacite \\
Le tact \\
Le talent \\
Le talent et le génie \\
Le tas et le tout \\
L'État \\
L'État a-t-il des intérêts propres ? \\
L'État a-t-il le droit de contrôler notre habillement ? \\
L'État a-t-il pour but de maintenir l'ordre ? \\
L'État a-t-il pour finalité de maintenir l'ordre ? \\
L'État a-t-il tous les droits ? \\
L'État contribue-t-il à pacifier les relations entre les hommes ? \\
L'état de droit \\
L'État de droit \\
L'état de guerre \\
L'état de la nature \\
L'état de nature \\
L'état d'exception \\
L'État doit-il disparaître ? \\
L'État doit-il éduquer le citoyen ? \\
L'État doit-il éduquer le peuple ? \\
L'État doit-il éduquer les citoyens ? \\
L'État doit-il être fort ? \\
L'État doit-il être le plus fort ? \\
L'État doit-il être neutre ? \\
L'État doit-il être sans pitié ? \\
L'État doit-il faire le bonheur des citoyens ? \\
L'État doit-il nous rendre meilleurs ? \\
L'État doit-il préférer l'injustice au désordre ? \\
L'État doit-il reconnaître des limites à sa puissance ? \\
L'État doit-il se mêler de religion ? \\
L'État doit-il se préoccuper des arts ? \\
L'État doit-il se préoccuper du bonheur des citoyens ? \\
L'État doit-il se soucier de la morale ? \\
L'État doit-il veiller au bonheur des individus  ? \\
L'État est-il appelé à disparaître ? \\
L'État est-il au service de la société ? \\
L'État est-il fin ou moyen ? \\
L'État est-il le garant de la propriété privée ? \\
L'État est-il l'ennemi de la liberté ? \\
L'État est-il l'ennemi de l'individu ? \\
L'État est-il libérateur ? \\
L'État est-il toujours juste ? \\
L'État est-il un arbitre ? \\
L'État est-il un mal nécessaire ? \\
L'État est-il un moindre mal ? \\
L'État est-il un tiers impartial ? \\
L'État est-il un « monstre froid » ? \\
L'État et la culture \\
L'État et la guerre \\
L'État et la justice \\
L'État et la nation \\
L'État et la Nation \\
L'État et la protection \\
L'État et la société \\
L'État et le droit \\
L'État et le marché \\
L'État et le peuple \\
L'État et les communautés \\
L'État et les Églises \\
L'État et l'individu \\
L'État libéral \\
L'État mondial \\
L'État n'est-il qu'un instrument de domination ? \\
L'État nous rend-il meilleurs ? \\
L'État peut-il créer la liberté ? \\
L'État peut-il être impartial ? \\
L'État peut-il être indifférent à la religion ? \\
L'État peut-il être libéral ? \\
L'État peut-il poursuivre une autre fin que sa propre puissance ? \\
L'État peut-il renoncer à la violence ? \\
L'État providence \\
L'État-providence \\
L'État universel \\
Le technicien n'est-il qu'un exécutant ? \\
Le témoignage \\
Le témoignage des sens \\
Le témoin \\
Le temps \\
Le temps de la liberté \\
Le temps de la réflexion \\
Le temps de la science \\
Le temps de l'histoire \\
Le temps détruit-il tout ? \\
Le temps du bonheur \\
Le temps du désir \\
Le temps est-il destructeur ? \\
Le temps est-il en nous ou hors de nous ? \\
Le temps est-il essentiellement destructeur ? \\
Le temps est-il la marque de notre impuissance ? \\
Le temps est-il notre allié ? \\
Le temps est-il notre ennemi ? \\
Le temps est-il une contrainte ? \\
Le temps est-il une dimension de la nature ? \\
Le temps est-il une prison ? \\
Le temps est-il une réalité ? \\
Le temps et l'espace \\
Le temps existe-t-il ? \\
Le temps libre \\
Le temps ne fait-il que passer ? \\
Le temps n'est-il pour l'homme que ce qui le limite ? \\
Le temps nous appartient-il ? \\
Le temps nous est-il compté ? \\
Le temps passe-t-il ? \\
Le temps perdu \\
Le temps s'écoule-t-il ? \\
Le temps se laisse-t-il décrire logiquement ? \\
L'éternel présent \\
L'éternel retour \\
L'éternité \\
L'éternité n'est-elle qu'une illusion ? \\
Le terrain \\
Le territoire \\
Le terrorisme est-il un acte de guerre ? \\
Le théâtral \\
Le théâtre de l'histoire \\
Le théâtre du monde \\
Le théâtre et l'existence \\
L'éthique à l'épreuve du tragique \\
L'éthique des plaisirs \\
L'éthique est-elle affaire de choix ? \\
L'éthique suppose-t-elle la liberté ? \\
L'ethnocentrisme \\
Le tiers exclu \\
L'étonnement \\
Le totalitarisme \\
Le totémisme \\
Le toucher \\
Le tourment moral \\
Le tout est-il la somme de ses parties ? \\
Le tout et les parties \\
Le tragique \\
Le tragique et le comique \\
Le trait d'esprit \\
L'étranger \\
L'étrangeté \\
Le travail \\
Le travail artistique \\
Le travail a-t-il une valeur morale ? \\
Le travail de la raison \\
Le travail du négatif \\
Le travail est-il le propre de l'homme ? \\
Le travail est-il libérateur ? \\
Le travail est-il nécessaire au bonheur ? \\
Le travail est-il toujours une activité productrice ? \\
Le travail est-il un besoin ? \\
Le travail est-il une fin ? \\
Le travail est-il une marchandise ? \\
Le travail est-il une valeur morale ? \\
Le travail est-il une valeur ? \\
Le travail est-il un rapport naturel de l'homme à la nature ? \\
Le travail et la propriété \\
Le travail et la technique \\
Le travail et le labeur \\
Le travail et le temps \\
Le travail et l'œuvre \\
Le travail fait-il de l'homme un être moral ? \\
Le travail fonde-t-il la propriété ? \\
Le travaille libère-t-il ? \\
Le travail manuel \\
Le travail manuel est-il sans pensée ? \\
Le travail nous rend-il solidaires ? \\
Le travail rapproche-t-il les hommes ? \\
Le travail sur le terrain \\
Le travail sur soi \\
Le travail unit-il ou sépare-t-il les hommes ? \\
L'être de la conscience \\
L'être de la vérité \\
L'être de l'image \\
L'être du possible \\
L'être en tant qu'être \\
L'être en tant qu'être est-il connaissable ? \\
L'être et la relation \\
L'être et la volonté \\
L'être et le bien \\
L'être et le devoir-être \\
L'être et le néant \\
L'être et les êtres \\
L'être et l'essence \\
L'être et l'étant \\
L'être et le temps \\
L'être humain désire-t-il naturellement connaître ? \\
L'être humain est-il la mesure de toute chose ? \\
L'être humain est-il par nature un être religieux \\
L'être imaginaire et l'être de raison \\
L'être se confond-il avec l'être perçu ? \\
Le tribunal de l'histoire \\
Le troc \\
Le trompe-l'œil \\
L'étude \\
L'étude de l'histoire conduit-elle à désespérer l'homme ? \\
Le tyran \\
L'eugénisme \\
L'Europe \\
L'euthanasie \\
Le vainqueur a-t-il tous les droits ? \\
L'évaluation \\
L'évasion \\
Le vécu \\
Le vécu et la vérité \\
L'événement \\
L'événement et le fait divers \\
L'événement historique a-t-il un sens par lui-même ? \\
L'événement manque-t-il d'être ? \\
Le verbalisme \\
Le verbe \\
Le vertige \\
Le vertige de la liberté \\
Le vestige \\
Le vêtement \\
Le vice et la vertu \\
Le vide \\
Le vide et le plein \\
L'évidence \\
L'évidence a-t-elle une valeur absolue ? \\
L'évidence est-elle critère de vérité ? \\
L'évidence et la démonstration \\
L'évidence se passe-t-elle de démonstration ? \\
Le village global \\
Le virtuel \\
Le visage \\
Le visible et l'invisible \\
Le vivant \\
Le vivant a-t-il des droits ? \\
Le vivant comme problème pour la philosophie des sciences \\
Le vivant est-il entièrement connaissable ? \\
Le vivant est-il entièrement explicable ? \\
Le vivant est-il réductible au physico-chimique ? \\
Le vivant est-il un objet de science comme un autre ? \\
Le vivant et la machine \\
Le vivant et la mort \\
Le vivant et la sensibilité \\
Le vivant et la technique \\
Le vivant et le vécu \\
Le vivant et l'expérimentation \\
Le vivant et l'inerte \\
Le vivant n'est-il que matière ? \\
Le vivant n'est-il qu'une machine ingénieuse ? \\
Le vivant obéit-il à des lois ? \\
Le vivant obéit-il à une nécessité ? \\
Le volontaire et l'involontaire \\
Le volontarisme \\
L'évolution \\
L'évolution des langues \\
Le voyage \\
Le voyage dans le temps \\
Le vrai a-t-il une histoire ? \\
Le vrai doit-il être démontré ? \\
Le vrai est-il à lui-même sa propre marque ? \\
Le vrai et le bien \\
Le vrai et le bien sont-ils analogues ? \\
Le vrai et le faux \\
Le vrai et le vraisemblable \\
Le vrai peut-il rester invérifiable ? \\
Le vraisemblable \\
Le vraisemblable et le romanesque \\
Le vrai se perçoit-il ? \\
Le vrai se réduit-il à ce qui est vérifiable ? \\
Le vrai se réduit-il à l'utile ? \\
Le vulgaire \\
L'exactitude \\
L'excellence \\
L'excellence des sens \\
L'exception \\
L'excès \\
L'excès et le défaut \\
L'exclusion \\
L'excuse \\
L'exécution d'une œuvre d'art est-elle toujours une œuvre d'art ? \\
L'exemplaire \\
L'exemplarité \\
L'exemple \\
L'exemple en morale \\
L'exercice \\
L'exercice de la vertu \\
L'exercice de la volonté \\
L'exercice du pouvoir \\
L'exercice solitaire du pouvoir \\
L'exigence de vérité a-t-elle un sens moral ? \\
L'exigence morale \\
L'exil \\
L'existence \\
L'existence a-t-elle un sens ? \\
L'existence d'autrui \\
L'existence de Dieu \\
L'existence de l'État dépend-elle d'un contrat ? \\
L'existence des idées \\
L'existence du mal \\
L'existence du mal met-elle en échec la raison ? \\
L'existence du passé \\
L'existence est-elle pensable ? \\
L'existence est-elle une propriété ? \\
L'existence est-elle un jeu ? \\
L'existence est-elle vaine ? \\
L'existence et le temps \\
L'existence se démontre-t-elle ? \\
L'existence se laisse-t-elle penser ? \\
L'expérience \\
L'expérience artistique \\
L'expérience a-t-elle le même sens dans toutes les sciences ? \\
L'expérience cruciale \\
L'expérience d'autrui nous est-elle utile ? \\
L'expérience de la beauté \\
L'expérience de la liberté \\
L'expérience de la maladie \\
L'expérience de la vie \\
L'expérience de l'injustice \\
L'expérience démontre-t-elle quelque chose ? \\
L'expérience de pensée \\
L'expérience directe est-elle une connaissance ? \\
L'expérience du danger \\
L'expérience du désir \\
L'expérience du mal \\
L'expérience du temps \\
L'expérience en sciences humaines \\
L'expérience enseigne-elle quelque chose ? \\
L'expérience, est-ce l'observation ? \\
L'expérience et la sensation \\
L'expérience et l'expérimentation \\
L'expérience imaginaire \\
L'expérience instruit-elle ? \\
L'expérience métaphysique \\
L'expérience morale \\
L'expérience nous apprend-elle quelque chose ? \\
L'expérience peut-elle avoir raison des principes ? \\
L'expérience peut-elle contredire la théorie ? \\
L'expérience rend-elle raisonnable ? \\
L'expérience rend-elle responsable ? \\
L'expérience sensible est-elle la seule source légitime de connaissance ? \\
L'expérience suffit-elle pour établir une vérité ? \\
L'expérimentation \\
L'expérimentation en psychologie \\
L'expérimentation en sciences sociales \\
L'expérimentation sur l'être humain \\
L'expérimentation sur le vivant \\
L'expert et l'amateur \\
L'expertise \\
L'expertise politique \\
L'explication \\
L'explication scientifique \\
L'exploitation de l'homme par l'homme \\
L'exposition \\
L'exposition de l'œuvre d'art \\
L'expression \\
L'expression artistique \\
L'expression de l'inconscient \\
L'expression du désir \\
L'expressivité musicale \\
L'extériorité \\
L'extinction du désir \\
L'extraordinaire \\
L'extrémisme \\
Le « je ne sais quoi » \\
L'habileté \\
L'habileté et la prudence \\
L'habitation \\
L'habitude \\
L'habitude a-t-elle des vertus ? \\
L'habitude est-elle notre guide dans la vie ? \\
L'harmonie \\
L'harmonie du monde \\
L'hégémonie politique \\
L'hérédité \\
L'hérésie \\
L'héritage \\
L'héroïsme \\
L'hésitation \\
L'hétérogénéité sociale \\
L'hétéronomie \\
L'hétéronomie de l'art \\
L'histoire a-t-elle des lois ? \\
L'histoire a-t-elle un commencement et une fin ? \\
L'Histoire a-t-elle un commencement ? \\
L'histoire a-t-elle une fin ? \\
L'histoire a-t-elle un sens ? \\
L'histoire de l'art \\
L'histoire de l'art est-elle celle des styles ? \\
L'histoire de l'art est-elle finie ? \\
L'histoire de l'art peut-elle arriver à son terme ? \\
L'histoire des arts est-elle liée à l'histoire des techniques ? \\
L'histoire des civilisations \\
L'histoire des sciences \\
L'Histoire des sciences \\
L'histoire des sciences est-elle une histoire ? \\
L'histoire du droit est-elle celle du progrès de la justice ? \\
L'histoire est-elle avant tout mémoire ? \\
L'histoire est-elle déterministe ? \\
L'histoire est-elle écrite par les vainqueurs ? \\
L'histoire est-elle la connaissance du passé humain ? \\
L'histoire est-elle la mémoire de l'humanité ? \\
L'histoire est-elle la science de ce qui ne se répète jamais ? \\
L'histoire est-elle la science du passé ? \\
L'histoire est-elle le récit objectif des faits passés  ? \\
L'histoire est-elle le règne du hasard ? \\
L'histoire est-elle le théâtre des passions ? \\
L'histoire est-elle rationnelle ? \\
L'histoire est-elle tragique ? \\
L'histoire est-elle une explication ou une justification du passé ? \\
L'histoire est-elle une science comme les autres ? \\
L'histoire est-elle une science ? \\
L'histoire est-elle un genre littéraire ? \\
L'histoire est-elle un roman vrai ? \\
L'histoire est-elle utile à la politique ? \\
L'histoire est-elle utile ? \\
L'histoire et la géographie \\
L'histoire jugera \\
L'histoire jugera-t-elle ? \\
L'histoire n'a-t-elle pour objet que le passé ? \\
L'histoire naturelle \\
L'histoire n'est-elle que la connaissance du passé ? \\
L'histoire n'est-elle qu'un récit ? \\
L'histoire nous appartient-elle ? \\
L'histoire obéit-elle à des lois ? \\
L'histoire peut-elle être contemporaine ? \\
L'histoire peut-elle être universelle ? \\
L'histoire peut-elle se répéter ? \\
L'histoire se répète-t-elle ? \\
L'histoire universelle est-elle l'histoire des guerres ? \\
L'histoire : enquête ou science ? \\
L'histoire : science ou récit ? \\
L'historicité des sciences \\
L'historien \\
L'historien peut-il être impartial ? \\
L'historien peut-il se passer du concept de causalité ? \\
L'homme aime-t-il la justice pour elle-même ? \\
L'homme a-t-il besoin de l'art ? \\
L'homme a-t-il besoin de religion ? \\
L'homme a-t-il une nature ? \\
L'homme a-t-il une place dans la nature ? \\
L'homme de l'art \\
L'homme de la rue \\
L'homme des droits de l'homme \\
L'homme des droits de l'homme n'est-il qu'une fiction ? \\
L'homme des foules \\
L'homme des sciences de l'homme ? \\
L'homme des sciences humaines \\
L'homme d'État \\
L'homme est-il chez lui dans l'univers ? \\
L'homme est-il fait pour le travail ? \\
L'homme est-il la mesure de toute chose ? \\
L'homme est-il la mesure de toutes choses ? \\
L'homme est-il l'artisan de sa dignité ? \\
L'homme est-il le seul être à avoir une histoire ? \\
L'homme est-il le sujet de son histoire ? \\
L'homme est-il objet de science ? \\
L'homme est-il prisonnier du temps ? \\
L'homme est-il raisonnable par nature ? \\
L'homme est-il un animal comme les autres ? \\
L'homme est-il un animal comme un autre ? \\
L'homme est-il un animal dénaturé ? \\
L'homme est-il un animal politique ? \\
L'homme est-il un animal rationnel ? \\
L'homme est-il un animal social ? \\
L'homme est-il un animal ? \\
L'homme est-il un corps pensant ? \\
L'homme est-il un être de devoir ? \\
L'homme est-il un être social par nature ? \\
L'homme est-il un loup pour l'homme ? \\
L'homme et la bête \\
L'homme et la machine \\
L'homme et la nature sont-ils commensurables ? \\
L'homme et l'animal \\
L'homme et le citoyen \\
L'homme injuste peut-il être heureux ? \\
L'homme intérieur \\
L'homme, le citoyen, le soldat \\
L'homme libre est-il un homme seul ? \\
L'homme-machine \\
L'homme n'est-il qu'un animal comme les autres ? \\
L'homme pense-t-il toujours ? \\
L'homme peut-il changer ? \\
L'homme peut-il être libéré du besoin ? \\
L'homme peut-il se représenter un monde sans l'homme ? \\
L'homme se réalise-t-il dans le travail ? \\
L'homme se reconnaît-il mieux dans le travail ou dans le loisir ? \\
L'honnêteté \\
L'honneur \\
L'honneur ? \\
L'horizon \\
L'horreur \\
L'horrible \\
L'hospitalité \\
L'hospitalité a-t-elle un sens politique ? \\
L'hospitalité est-elle un devoir ? \\
L'humain \\
L'humanité \\
L'humanité est-elle aimable ? \\
L'humiliation \\
L'humilité \\
L'humour \\
L'humour et l'ironie \\
L'hybridation des arts \\
L'hypocrisie \\
L'hypothèse \\
L'hypothèse de l'inconscient \\
Libéral et libertaire \\
Libéralité et libéralisme \\
Liberté d'agir, liberté de penser \\
Liberté, égalité, fraternité \\
Liberté et courage \\
Liberté et démocratie \\
Liberté et déterminisme \\
Liberté et éducation \\
Liberté et égalité \\
Liberté et engagement \\
Liberté et existence \\
Liberté et habitude \\
Liberté et indépendance \\
Liberté et libération \\
Liberté et licence \\
Liberté et nécessité \\
Liberté et pouvoir \\
Liberté et responsabilité \\
Liberté et savoir \\
Liberté et sécurité \\
Liberté et société \\
Liberté et solitude \\
Liberté humaine et liberté divine \\
Liberté réelle, liberté formelle \\
Libertés publiques et culture politique \\
Libre arbitre et déterminisme sont-ils compatibles ? \\
Libre arbitre et liberté \\
Libre et heureux \\
L'idéal \\
L'idéal de l'art \\
L'idéal démonstratif \\
L'idéal de vérité \\
L'idéal et le réel \\
L'idéalisme \\
L'idéaliste \\
L'idéalité \\
L'idéal moral est-il vain ? \\
L'idéal systématique \\
L'idéal-type \\
L'idée d'anthropologie \\
L'idée de beaux arts \\
L'idée de bonheur \\
L'idée de bonheur collectif a-t-elle un sens ? \\
L'idée de civilisation \\
L'idée de communauté \\
L'idée de communisme \\
L'idée de connaissance approchée \\
L'idée de conscience collective \\
L'idée de continuité \\
L'idée de contrat social \\
L'idée de création \\
L'idée de crise \\
L'idée de destin a-t-elle un sens ? \\
L'idée de déterminisme \\
L'idée de Dieu \\
L'idée de domination \\
L'idée de forme sociale \\
L'idée de justice \\
L'idée de langue universelle \\
L'idée de logique \\
L'idée de logique transcendantale \\
L'idée de logique universelle \\
L'idée de loi de la nature \\
L'idée de loi logique \\
L'idée de loi naturelle \\
L'idée de mal nécessaire \\
L'idée de mathesis universalis \\
L'idée de métier \\
L'idée de modernité \\
L'idée de monde \\
L'idée de morale appliquée \\
L'idée de nation \\
L'idée d'encyclopédie \\
L'idée de norme \\
L'idée de paix \\
L'idée de perfection \\
L'idée de progrès \\
L'idée de république \\
L'idée de rétribution est-elle nécessaire à la morale ? \\
L'idée de révolution \\
L'idée de science \\
L'idée de science expérimentale \\
L'idée de substance \\
L'idée de système \\
L'idée d'éternité \\
L'idée d'Europe \\
L'idée d'exactitude \\
L'idée de « sciences exactes » \\
L'idée d'histoire universelle \\
L'idée d'ordre \\
L'idée d'organisme \\
L'idée d'origine \\
L'idée d'un commencement absolu \\
L'idée d'une langue universelle \\
L'idée d'une science bien faite \\
L'idée d'univers \\
L'idée d'université \\
L'idée esthétique \\
L'identification \\
L'identité \\
L'identité collective \\
L'identité et la différence \\
L'identité personnelle \\
L'identité personnelle est-elle donnée ou construite ? \\
L'identité relève-telle du champ politique ? \\
L'idéologie \\
L'idiot \\
L'idolâtrie \\
L'idole \\
L'ignoble \\
L'ignorance \\
L'ignorance est-elle préférable à l'erreur ? \\
L'ignorance nous excuse-t-elle ? \\
L'ignorance peut-elle être une excuse ? \\
L'illimité \\
L'illusion \\
L'illusion de la liberté \\
L'illustration \\
L'image \\
L'image et le réel \\
L'imaginaire \\
L'imaginaire et le réel \\
L'imagination \\
L'imagination a-t-elle des limites ? \\
L'imagination dans l'art \\
L'imagination dans les sciences \\
L'imagination enrichit-elle la connaissance ? \\
L'imagination est-elle libre ? \\
L'imagination est-elle maîtresse d'erreur et de fausseté ? \\
L'imagination esthétique \\
L'imagination et la raison \\
L'imagination nous éloigne-t-elle du réel ? \\
L'imagination politique \\
L'imitation \\
L'imitation a-t-elle une fonction morale ? \\
L'immanence \\
L'immatériel \\
L'immédiat \\
L'immensité \\
L'immortalité \\
L'immortalité de l'âme \\
L'immortalité des œuvres d'art \\
L'immuable \\
L'immutabilité \\
L'impardonnable \\
L'impartialité \\
L'impartialité des historiens \\
L'impartialité est-elle toujours désirable ? \\
L'impassibilité \\
L'impatience \\
L'impensable \\
L'impératif \\
L'impératif d'impartialité \\
L'impératif hypothétique \\
L'imperceptible \\
L'impersonnel \\
L'implicite \\
L'importance des détails \\
L'impossible \\
L'imposteur \\
L'imprescriptible \\
L'impression \\
L'imprévisible \\
L'improbable \\
L'improvisation \\
L'improvisation dans l'art \\
L'imprudence \\
L'impuissance \\
L'impuissance de la raison \\
L'impuissance de l'État \\
L'impunité \\
L'inaccessible \\
L'inachevé \\
L'inaction \\
L'inaliénable \\
L'inaperçu \\
L'inapparent \\
L'inattendu \\
L'incarnation \\
L'incertitude \\
L'incertitude du passé \\
L'incertitude est-elle dans les choses ou dans les idées ? \\
L'incertitude interdit-elle de raisonner ? \\
L'incommensurabilité \\
L'incommensurable \\
L'incompréhensible \\
L'inconcevable \\
L'inconnaissable \\
L'inconnu \\
L'inconscience \\
L'inconscient \\
L'inconscient a-t-il une histoire ? \\
L'inconscient collectif \\
L'inconscient de l'art \\
L'inconscient est-il dans l'âme ou dans le corps ? \\
L'inconscient est-il l'animal en nous ? \\
L'inconscient est-il un concept scientifique ? \\
L'inconscient est-il une dimension de la conscience ? \\
L'inconscient est-il une excuse ? \\
L'inconscient et l'involontaire \\
L'inconscient et l'oubli \\
L'inconscient n'est-il qu'un défaut de conscience ? \\
L'inconscient n'est-il qu'une hypothèse ? \\
L'inconscient peut-il se manifester ? \\
L'inconséquence \\
L'inconstance \\
L'incorporel \\
L'incrédulité \\
L'incroyable \\
L'inculture \\
L'indécence \\
L'indécidable \\
L'indécision \\
L'indéfini \\
L'indémontrable \\
L'indépassable \\
L'indépendance \\
L'indescriptible \\
L'indésirable \\
L'indétermination \\
L'indéterminé \\
L'indice \\
L'indice et la preuve \\
L'indicible \\
L'indicible et l'impensable \\
L'indicible et l'ineffable \\
L'indifférence \\
L'indifférence à la politique \\
L'indifférence peut-elle être une vertu ? \\
L'indignation \\
L'indignité \\
L'indiscutable \\
L'individu \\
L'individualisme \\
L'individualisme a-t-il sa place en politique ? \\
L'individualisme est-il un égoïsme ? \\
L'individualisme méthodologique \\
L'individualité \\
L'individu a-t-il des droits ? \\
L'individuel et le collectif \\
L'individu et la multitude \\
L'individu et le groupe \\
L'individu et l'espèce \\
L'individu face à L'État \\
L'indivisible \\
L'induction \\
L'induction et la déduction \\
L'indulgence \\
L'industrie culturelle \\
L'industrie du beau \\
L'ineffable et l'innommable \\
L'inégalité des chances \\
L'inégalité entre les hommes \\
L'inégalité naturelle \\
L'inéluctable \\
L'inertie \\
L'inesthétique \\
L'inestimable \\
L'inexactitude et le savoir scientifique \\
L'inexistant \\
L'inexpérience \\
L'infâme \\
L'infamie \\
L'inférence \\
L'infidélité \\
L'infini \\
L'infini et l'indéfini \\
L'infinité de l'espace \\
L'infinité de l'univers a-t-elle de quoi nous effrayer ? \\
L'influence \\
L'information \\
L'informe \\
L'informe et le difforme \\
L'ingénieur \\
L'ingénuité \\
L'ingratitude \\
L'inhibition \\
L'inhumain \\
L'inhumanité \\
L'inimaginable \\
L'inimitié \\
L'inintelligible \\
L'initiation \\
L'injonction \\
L'injustice \\
L'injustice est-elle préférable au désordre ? \\
L'injustifiable \\
L'inné et l'acquis \\
L'innocence \\
L'innommable \\
L'innovation \\
L'inobservable \\
L'inquiétant \\
L'inquiétude \\
L'inquiétude peut-elle définir l'existence humaine ? \\
L'inquiétude peut-elle devenir l'existence humaine ? \\
L'insatisfaction \\
L'insensé \\
L'insensibilité \\
L'insignifiant \\
L'insociable sociabilité \\
L'insolite \\
L'insouciance \\
L'insoumission \\
L'insoutenable \\
L'inspiration \\
L'instant \\
L'instant de la décision est-il une folie ? \\
L'instant et la durée \\
L'instinct \\
L'institution \\
L'institutionnalisation des conduites \\
L'institution scientifique \\
L'institution scolaire \\
L'instruction \\
L'instruction est-elle facteur de moralité ? \\
L'instrument \\
L'instrument et la machine \\
L'instrument mathématique en sciences humaines \\
L'instrument scientifique \\
L'insulte \\
L'insurrection \\
L'insurrection est-elle un droit ? \\
L'intangible \\
L'intellect \\
L'intellectuel \\
L'intelligence \\
L'intelligence artificielle \\
L'intelligence de la main \\
L'intelligence de la matière \\
L'intelligence de la technique \\
L'intelligence des bêtes \\
L'intelligence des foules \\
L'intelligence du sensible \\
L'intelligence du vivant \\
L'intelligence politique \\
L'intelligible \\
L'intempérance \\
L'intemporel \\
L'intention \\
L'intention morale \\
L'intention morale suffit-elle à constituer la valeur morale de l'action ? \\
L'intentionnalité \\
L'interaction \\
L'interdisciplinarité \\
L'interdit \\
L'interdit est-il au fondement de la culture ? \\
L'intéressant \\
L'intérêt \\
L'intérêt bien compris \\
L'intérêt commun \\
L'intérêt constitue-t-il l'unique lien social ? \\
L'intérêt de la société l'emporte-t-il sur celui des individus ? \\
L'intérêt de l'État \\
L'intérêt est-il le principe de tout échange ? \\
L'intérêt général est-il la somme des intérêts particuliers ? \\
L'intérêt général est-il le bien commun ? \\
L'intérêt gouverne-t-il le monde ? \\
L'intérêt peut-il être une valeur morale ? \\
L'intérêt public est-il une illusion ? \\
L'intérieur et l'extérieur \\
L'intériorisation des normes \\
L'intériorité \\
L'intériorité de l'œuvre \\
L'intériorité est-elle un mythe ? \\
L'interprétation \\
L'interprétation de la loi \\
L'interprétation de la nature \\
L'interprétation des œuvres \\
L'interprétation est-elle sans fin ? \\
L'interprétation est-elle un art ? \\
L'interprétation est-elle une activité sans fin ? \\
L'interprétation est-elle une science ? \\
L'interprète et le créateur \\
L'interprète sait-il ce qu'il cherche ? \\
L'interrogation humaine \\
L'intersubjectivité \\
L'intime conviction \\
L'intimité \\
L'intolérable \\
L'intolérance \\
L'intraduisible \\
L'intransigeance \\
L'introspection \\
L'introspection est-elle une connaissance ? \\
L'intuition \\
L'intuition a-t-elle une place en logique ? \\
L'intuition en mathématiques \\
L'intuition intellectuelle \\
L'intuition morale \\
L'inutile \\
L'inutile a-t-il de la valeur ? \\
L'inutile est-il sans valeur ? \\
L'invention \\
L'invention de soi \\
L'invention et la découverte \\
L'invention technique \\
L'invérifiable \\
L'invisibilité \\
L'invisible \\
L'involontaire \\
Lire et écrire \\
L'ironie \\
L'irrationalité \\
L'irrationnel \\
L'irrationnel est-il pensable ? \\
L'irrationnel est-il toujours absurde ? \\
L'irrationnel et le politique \\
L'irrationnel existe-t-il ? \\
L'irréductible \\
L'irréel \\
L'irréfléchi \\
L'irréfutable \\
L'irrégularité \\
L'irréparable \\
L'irreprésentable \\
L'irrésolution \\
L'irresponsabilité \\
L'irréversibilité \\
L'irréversible \\
L'irrévocable \\
Littérature et réalité \\
L'ivresse \\
L'obéissance \\
L'obéissance à l'autorité \\
L'obéissance est-elle compatible avec la liberté ? \\
L'objection de conscience \\
L'objectivation \\
L'objectivité \\
L'objectivité de l'art \\
L'objectivité de l'historien \\
L'objectivité historique \\
L'objectivité historique est-elle synonyme de neutralité ? \\
L'objectivité scientifique \\
L'objet \\
L'objet d'amour \\
L'objet de culte \\
L'objet de la littérature \\
L'objet de l'amour \\
L'objet de la politique \\
L'objet de la psychologie \\
L'objet de la réflexion \\
L'objet de l'art \\
L'objet de l'intention \\
L'objet des mathématiques \\
L'objet du désir \\
L'objet du désir en est-il la cause ? \\
L'objet et la chose \\
L'obligation \\
L'obligation d'échanger \\
L'obligation morale \\
L'obligation morale peut-elle se réduire à une obligation sociale ? \\
L'obscène \\
L'obscénité \\
L'obscur \\
L'obscurantisme \\
L'obscurité \\
L'observation \\
L'observation participante \\
L'obsession \\
L'obstacle \\
L'obstacle épistémologique \\
L'occasion \\
L'œil et l'oreille \\
L'œuvre \\
L'œuvre anonyme \\
L'œuvre d'art a-t-elle un sens ? \\
L'œuvre d'art doit-elle être belle ? \\
L'œuvre d'art doit-elle nous émouvoir ? \\
L'œuvre d'art donne-t-elle à penser ? \\
L'œuvre d'art échappe-t-elle au temps ? \\
L'œuvre d'art échappe-t-elle nécessairement au temps ? \\
L'œuvre d'art est-elle l'expression d'une idée ? \\
L'œuvre d'art est-elle toujours destinée à un public ? \\
L'œuvre d'art est-elle une belle apparence ? \\
L'œuvre d'art est-elle une marchandise ? \\
L'œuvre d'art est-elle un objet d'échange ? \\
L'œuvre d'art est-elle un symbole ? \\
L'œuvre d'art et le plaisir \\
L'œuvre d'art et sa reproduction \\
L'œuvre d'art et son auteur \\
L'œuvre d'art instruit-elle ? \\
L'œuvre d'art nous apprend-elle quelque chose ? \\
L'œuvre d'art représente-t-elle quelque chose ? \\
L'œuvre d'art totale \\
L'œuvre de fiction \\
L'œuvre de l'historien \\
L'œuvre du temps \\
L'œuvre et le produit \\
L'œuvre inachevée \\
L'offense \\
Logique et dialectique \\
Logique et existence \\
Logique et grammaire \\
Logique et logiques \\
Logique et mathématique \\
Logique et mathématiques \\
Logique et métaphysique \\
Logique et méthode \\
Logique et ontologie \\
Logique et psychologie \\
Logique et réalité \\
Logique et vérité \\
Logique générale et logique transcendantale \\
Loi morale et loi politique \\
Loi naturelle et loi politique \\
Lois et coutumes \\
Lois et règles en logique \\
Loisir et oisiveté \\
L'oisiveté \\
Lois naturelles et lois civiles \\
L'oligarchie \\
L'ombre et la lumière \\
L'omniscience \\
L'opinion \\
L'opinion a-t-elle nécessairement tort ? \\
L'opinion droite \\
L'opinion du citoyen \\
L'opinion est-elle un savoir ? \\
L'opinion publique \\
L'opinion vraie \\
L'opportunisme \\
L'opportunité \\
L'opposant \\
L'opposition \\
L'optimisme \\
L'oral et l'écrit \\
L'ordinaire \\
L'ordinaire est-il ennuyeux ? \\
L'ordre \\
L'ordre des choses \\
L'ordre du monde \\
L'ordre du temps \\
L'ordre du vivant est-il façonné par le hasard ? \\
L'ordre est-il dans les choses ? \\
L'ordre établi \\
L'ordre et la mesure \\
L'ordre et le désordre \\
L'ordre international \\
L'ordre moral \\
L'ordre politique exclut-il la violence ? \\
L'ordre politique peut-il exclure la violence ? \\
L'ordre public \\
L'ordre social \\
L'ordre social peut-il être juste ? \\
L'organique \\
L'organique et le mécanique \\
L'organique et l'inorganique \\
L'organisation \\
L'organisation du travail \\
L'organisation du vivant \\
L'organisme \\
L'orgueil \\
L'orientation \\
L'Orient et l'Occident \\
L'original et la copie \\
L'originalité \\
L'originalité en art \\
L'origine \\
L'origine de la culpabilité \\
L'origine de la négation \\
L'origine de l'art \\
L'origine de la violence \\
L'origine des croyances \\
L'origine des idées \\
L'origine des langues \\
L'origine des langues est-elle un faux problème ? \\
L'origine des valeurs \\
L'origine des vertus \\
L'origine du droit \\
L'origine et le fondement \\
L'ornement \\
L'oubli \\
L'oubli des fautes \\
L'oubli est-il un échec de la mémoire ? \\
L'oubli et le pardon \\
L'outil \\
L'outil et la machine \\
L'ouverture d'esprit \\
L'un \\
L'unanimité est-elle un critère de légitimité ? \\
L'unanimité est-elle un critère de vérité ? \\
L'un est le multiple \\
L'un et le multiple \\
L'un et l'être \\
L'uniformité \\
L'unité \\
L'unité dans le beau \\
L'unité de l'art \\
L'unité de la science \\
L'unité de l'État \\
L'unité de l'œuvre d'art \\
L'unité des arts \\
L'unité des contraires \\
L'unité des langues \\
L'unité des sciences \\
L'unité des sciences humaines \\
L'unité des sciences humaines ? \\
L'unité du corps politique \\
L'unité du genre humain \\
L'unité du vivant \\
L'univers \\
L'universalisme \\
L'universel \\
L'universel et le particulier \\
L'universel et le singulier \\
L'univocité de l'étant \\
L'urbanité \\
L'urgence \\
L'usage \\
L'usage des fictions \\
L'usage des généalogies \\
L'usage des mots \\
L'usage des passions \\
L'usage du monde \\
L'usure \\
L'usure des mots \\
L'utile \\
L'utile et l'agréable \\
L'utile et le beau \\
L'utile et le bien \\
L'utile et l'honnête \\
L'utile et l'inutile \\
L'utilité \\
L'utilité de croire \\
L'utilité de la poésie \\
L'utilité de l'art \\
L'utilité des préjugés \\
L'utilité des sciences humaines \\
L'utilité est-elle étrangère à la morale ? \\
L'utilité est-elle une valeur morale ? \\
L'utilité peut-elle être le principe de la moralité ? \\
L'utilité peut-elle être un critère pour juger de la valeur de nos actions ? \\
L'utilité publique \\
L'utopie \\
L'utopie a-t-elle une signification politique ? \\
L'utopie en politique \\
L'utopie et l'idéologie \\
Machine et organisme \\
Machines et liberté \\
Machines et mémoire \\
Magie et religion \\
Maître et disciple \\
Maître et serviteur \\
Maîtrise et puissance \\
Maîtriser l'absence \\
Maîtriser la technique \\
Maîtriser le vivant \\
Majorité et minorité \\
Maladie et convalescence \\
Maladies du corps, maladies de l'âme \\
Malaise dans la civilisation \\
Mal et liberté \\
Malheur aux vaincus \\
Ma liberté s'arrête-t-elle où commence celle des autres ? \\
Manger \\
Manquer de jugement \\
Ma parole m'engage-t-elle ? \\
Masculin, féminin \\
Mathématiques et réalité \\
Mathématiques pures et mathématiques appliquées \\
Matière et corps \\
Matière et matériaux \\
Ma vraie nature \\
Mécanisme et finalité \\
Médecine et philosophie \\
Méditer \\
Mémoire et fiction \\
Mémoire et identité \\
Mémoire et responsabilité \\
Mémoire et souvenir \\
Ménager les apparences \\
Mensonge et politique \\
Mentir \\
Mesure et démesure \\
Mesurer \\
Métaphysique et histoire \\
Métaphysique et ontologie \\
Métaphysique et politique \\
Métaphysique et religion \\
Métaphysique spéciale, métaphysique générale \\
Métier et vocation \\
Mettre en ordre \\
Microscope et télescope \\
Mieux vaut subir que commettre l'injustice \\
Misère et pauvreté \\
Modèle et copie \\
Mœurs, coutumes, lois \\
Moi d'abord \\
Mon corps \\
Mon corps est-il ma propriété ? \\
Mon corps est-il naturel ? \\
Mon corps fait-il obstacle à ma liberté ? \\
Mon corps m'appartient-il ? \\
Monde et nature \\
Monologue et dialogue \\
Mon prochain est-il mon semblable ? \\
Mon semblable \\
Montrer, est-ce démontrer ? \\
Montrer et démontrer \\
Morale et calcul \\
Morale et convention \\
Morale et économie \\
Morale et éducation \\
Morale et histoire \\
Morale et identité \\
Morale et intérêt \\
Morale et liberté \\
Morale et politique sont-elles indépendantes ? \\
Morale et pratique \\
Morale et prudence \\
Morale et religion \\
Morale et sexualité \\
Morale et société \\
Morale et technique \\
Morale et violence \\
Moralité et connaissance \\
Moralité et utilité \\
Mourir \\
Mourir dans la dignité \\
Mourir pour des principes \\
Mourir pour la patrie \\
Murs et frontières \\
Musique et bruit \\
Mythe et connaissance \\
Mythe et histoire \\
Mythe et pensée \\
Mythe et philosophie \\
Mythe et symbole \\
Mythe et vérité \\
Mythes et idéologies \\
Naît-on sujet ou le devient-on ? \\
Naître \\
N'apprend-on que par l'expérience ? \\
Narration et identité \\
Nation et richesse \\
Nature et artifice \\
Nature et convention \\
Nature et culture \\
Nature et fonction du sacrifice \\
Nature et histoire \\
Nature et institution \\
Nature et institutions \\
Nature et loi \\
Nature et morale \\
Naturel et artificiel \\
Nature, monde, univers \\
Naviguer \\
N'avons-nous affaire qu'au réel ? \\
N'avons-nous de devoir qu'envers autrui ? \\
Nécessité et contingence \\
Nécessité et liberté \\
Nécessité fait loi \\
N'échange-t-on que ce qui a de la valeur ? \\
N'échange-t-on que des symboles ? \\
N'échange-t-on que par intérêt ? \\
Ne désire-t-on que ce dont on manque ? \\
Ne désirons-nous que ce qui est bon pour nous ? \\
Ne désirons-nous que les choses que nous estimons bonnes ? \\
Ne faire que son devoir \\
Négation et privation \\
Ne lèse personne \\
Ne pas raconter d'histoires \\
Ne pas rire, ne pas pleurer, mais comprendre \\
Ne pas savoir ce que l'on fait \\
Ne penser à rien \\
Ne penser qu'à soi \\
Ne prêche-t-on que les convertis ? \\
Ne rien devoir à personne \\
Ne sait-on rien que par expérience ? \\
Ne sommes-nous véritablement maîtres que de nos pensées ? \\
N'est-on juste que par crainte du châtiment ? \\
Ne veut-on que ce qui est désirable ? \\
Ne vit-on bien qu'avec ses amis ? \\
Névroses et psychoses \\
N'existe-t-il que des individus ? \\
N'existe-t-il que le présent ? \\
N'exprime t-on que ce dont on a conscience ? \\
Ni Dieu ni maître \\
Ni Dieu, ni maître \\
Nier la vérité \\
Nier le monde \\
Nier l'évidence \\
N'interprète-t-on que ce qui est équivoque ? \\
Ni regrets, ni remords \\
Nomade et sédentaire \\
Nommer \\
Nom propre et nom commun \\
Normes et valeurs \\
Normes morales et normes vitales \\
Nos convictions morales sont-elles le simple reflet de notre temps ? \\
Nos désirs nous appartiennent-ils ? \\
Nos désirs nous opposent-ils ? \\
Nos pensées dépendent-elles de nous ? \\
Nos pensées sont-elles entièrement en notre pouvoir ? \\
Nos sens nous trompent-ils ? \\
Notre besoin de fictions \\
Notre connaissance du réel se limite-t-elle au savoir scientifique ? \\
Notre corps pense-t-il ? \\
Notre existence a-t-elle un sens si l'histoire n'en a pas ? \\
Notre ignorance nous excuse-t-elle ? \\
Notre liberté de pensée a-t-elle des limites ? \\
Notre rapport au monde est-il essentiellement technique ? \\
Notre rapport au monde peut-il n'être que technique ? \\
Nous et les autres \\
Nouveauté et tradition \\
Nul n'est censé ignorer la loi \\
Nul n'est méchant volontairement \\
N'y a-t-il d'amitié qu'entre égaux ? \\
N'y a-t-il de beauté qu'artistique ? \\
N'y a t-il de bonheur que dans l'instant ? \\
N'y a-t-il de bonheur qu'éphémère ? \\
N'y a-t-il de certitude que mathématique ? \\
N'y a-t-il de connaissance que de l'universel ? \\
N'y a-t-il de démocratie que représentative ? \\
N'y a-t-il de devoirs qu'envers autrui ? \\
N'y a-t-il de droit qu'écrit ? \\
N'y a-t-il de droits que de l'homme ? \\
N'y a-t-il de foi que religieuse ? \\
N'y a-t-il de propriété que privée ? \\
N'y a-t-il de rationalité que scientifique ? \\
N'y a-t-il de réalité que de l'individuel ? \\
N'y a-t-il de savoir que livresque ? \\
N'y a-t-il de science qu'autant qu'il s'y trouve de mathématique ? \\
N'y a-t-il de science que de ce qui est mathématisable ? \\
N'y a-t-il de science que du général ? \\
N'y a-t-il de science que du mesurable ? \\
N'y a-t-il de science qu'exacte ? \\
N'y a-t-il de sens que par le langage ? \\
N'y a-t-il de vérité que scientifique ? \\
N'y a-t-il de vérité que vérifiable ? \\
N'y a-t-il de vérités que scientifiques ? \\
N'y a-t-il de vrai que le vérifiable ? \\
N'y a-t-il que des individus ? \\
N'y a-t-il qu'une substance ? \\
N'y a-t-il qu'un seul monde ? \\
Obéir \\
Obéir, est-ce se soumettre ? \\
Obéissance et liberté \\
Obéissance et servitude \\
Obéissance et soumission \\
Objectivé et subjectivité \\
Objectiver \\
Objet et œuvre \\
Observation et expérience \\
Observation et expérimentation \\
Observer \\
Observer et comprendre \\
Observer et expérimenter \\
Observer et interpréter \\
Œil pour œil, dent pour dent \\
Œuvre et événement \\
Opinion et ignorance \\
Optimisme et pessimisme \\
Ordre et désordre \\
Ordre et justice \\
Ordre et liberté \\
Organisme et milieu \\
Origine et commencement \\
Origine et fondement \\
Où commence la servitude ? \\
Où commence la violence ? \\
Où commence l'interprétation ? \\
Où commence ma liberté ? \\
Où est le passé ? \\
Où est le pouvoir ? \\
Où est l'esprit ? \\
Où est mon esprit ? \\
Où est-on quand on pense ? \\
Où s'arrête l'espace public ? \\
Où sont les relations ? \\
Où suis-je quand je pense ? \\
Où suis-je ? \\
Outil et machine \\
Outil et organe \\
Paraître \\
Par-delà beauté et laideur \\
Pardonner et oublier \\
Parfaire \\
Parier \\
Par le langage, peut-on agir sur la réalité ? \\
Parler, est-ce agir ? \\
Parler, est-ce communiquer ? \\
Parler, est-ce donner sa parole ? \\
Parler, est-ce ne pas agir ? \\
Parler et agir \\
Parler, n'est-ce que désigner ? \\
Parler pour ne rien dire \\
Parler pour quelqu'un \\
Parole et pouvoir \\
Paroles et actes \\
Par où commencer ? \\
Par quoi un individu se distingue-t-il d'un autre ? \\
Partager \\
Partager les richesses \\
Partager sa vie \\
Partager ses sentiments \\
Passer du fait au droit \\
Passer le temps \\
Passions et intérêts \\
Passions, intérêt, raison \\
Pâtir \\
Peindre \\
Peindre d'après nature \\
Peindre, est-ce nécessairement feindre ? \\
Peinture et histoire \\
Peinture et réalité \\
Pensée et réalité \\
Penser à rien \\
Penser est-ce calculer ? \\
Penser, est-ce calculer ? \\
Penser, est-ce désobéir ? \\
Penser, est-ce dire non ? \\
Penser, est-ce se parler à soi-même ? \\
Penser est-il assimilable à un travail ? \\
Penser et calculer \\
Penser et imaginer \\
Penser et parler \\
Penser et raisonner \\
Penser et savoir \\
Penser et sentir \\
Penser la matière \\
Penser la technique \\
Penser l'avenir \\
Penser le changement \\
Penser le réel \\
Penser le rien, est-ce ne rien penser ? \\
Penser les sociétés comme des organismes \\
Penser l'impossible \\
Penser par soi-même \\
Penser par soi-même, est-ce être l'auteur de ses pensées ? \\
Penser peut-il nous rendre heureux ? \\
Penser requiert-il un corps ? \\
Penser sans corps \\
Penser sans sujet \\
Pense-t-on jamais seul ? \\
Pensez-vous que vous avez une âme ? \\
Perception et aperception \\
Perception et connaissance \\
Perception et création artistique \\
Perception et imagination \\
Perception et jugement \\
Perception et mouvement \\
Perception et passivité \\
Perception et sensation \\
Perception et souvenir \\
Perception et vérité \\
Percevoir \\
Percevoir est-ce connaître ? \\
Percevoir, est-ce connaître ? \\
Percevoir, est-ce interpréter ? \\
Percevoir, est-ce juger ? \\
Percevoir, est-ce reconnaître ? \\
Percevoir, est-ce savoir ? \\
Percevoir, est-ce s'ouvrir au monde ? \\
Percevoir et concevoir \\
Percevoir et imaginer \\
Percevoir et juger \\
Percevoir et sentir \\
Percevoir s'apprend-il ? \\
Percevons-nous les choses telles qu'elles sont ? \\
Perçoit-on le réel tel qu'il est ? \\
Perçoit-on le réel ? \\
Perçoit-on les choses comme elles sont ? \\
Perdre la mémoire \\
Perdre la raison \\
Perdre le contrôle \\
Perdre ses habitudes \\
Perdre ses illusions \\
Perdre son âme \\
Perdre son temps \\
Permanence et identité \\
Persévérer dans son être \\
Personne et individu \\
Personne n'est innocent \\
Persuader \\
Persuader et convaincre \\
Peuple et culture \\
Peuple et masse \\
Peuple et multitude \\
Peuple et société \\
Peuples et masses \\
Peut-être se mettre à la place de l'autre ? \\
Peut-il être moral de tuer ? \\
Peut-il être préférable de ne pas savoir ? \\
Peut-il exister une action désintéressée ? \\
Peut-il y avoir conflit entre nos devoirs ? \\
Peut-il y avoir de bons tyrans ? \\
Peut-il y avoir de la politique sans conflit ? \\
Peut-il y avoir des échanges équitables ? \\
Peut-il y avoir des expériences métaphysiques ? \\
Peut-il y avoir des lois de l'histoire ? \\
Peut-il y avoir des lois injustes ? \\
Peut-il y avoir des modèles en morale ? \\
Peut-il y avoir des vérités partielles ? \\
Peut-il y avoir esprit sans corps ? \\
Peut-il y avoir savoir-faire sans savoir ? \\
Peut-il y avoir science sans intuition du vrai ? \\
Peut-il y avoir un art conceptuel ? \\
Peut-il y avoir un droit à désobéir ? \\
Peut-il y avoir un droit de la guerre ? \\
Peut-il y avoir une histoire universelle ? \\
Peut-il y avoir une philosophie applicable ? \\
Peut-il y avoir une philosophie appliquée ? \\
Peut-il y avoir une philosophie politique sans Dieu ? \\
Peut-il y avoir une science de l'éducation ? \\
Peut-il y avoir une science politique ? \\
Peut-il y avoir une société des nations ? \\
Peut-il y avoir une société sans État ? \\
Peut-il y avoir un État mondial ? \\
Peut-il y avoir une vérité en art ? \\
Peut-il y avoir une vérité en politique ? \\
Peut-il y avoir un intérêt collectif ? \\
Peut-il y avoir un langage universel ? \\
Peut-on admettre un droit à la révolte ? \\
Peut-on agir machinalement ? \\
Peut-on aimer ce qu'on ne connaît pas ? \\
Peut-on aimer l'autre tel qu'il est ? \\
Peut-on aimer la vie plus que tout ? \\
Peut-on aimer les animaux ? \\
Peut-on aimer l'humanité ? \\
Peut-on aimer sans perdre sa liberté ? \\
Peut-on aimer son prochain comme soi-même ? \\
Peut-on aimer son travail ? \\
Peut-on aimer une œuvre d'art sans la comprendre ? \\
Peut-on apprendre à être heureux ? \\
Peut-on apprendre à être juste ? \\
Peut-on apprendre à être libre ? \\
Peut-on apprendre à mourir ? \\
Peut-on apprendre à penser ? \\
Peut-on apprendre à vivre ? \\
Peut-on argumenter en morale ? \\
Peut-on assimiler le vivant à une machine ? \\
Peut-on atteindre une certitude ? \\
Peut-on attribuer à chacun son dû ? \\
Peut-on avoir conscience de soi sans avoir conscience d'autrui ? \\
Peut-on avoir de bonnes raisons de ne pas dire la vérité ? \\
Peut-on avoir le droit de se révolter ? \\
Peut-on avoir peur de soi-même ? \\
Peut-on avoir raison contre les faits ? \\
Peut-on avoir raison contre tous ? \\
Peut-on avoir raison contre tout le monde ? \\
Peut-on avoir raisons contre les faits ? \\
Peut-on avoir raison tout.e seul.e ? \\
Peut-on avoir raison tout seul ? \\
Peut-on cesser de croire ? \\
Peut-on cesser de désirer ? \\
Peut-on changer de culture ? \\
Peut-on changer de logique ? \\
Peut-on changer le cours de l'histoire ? \\
Peut-on changer le monde ? \\
Peut-on changer le passé ? \\
Peut-on changer ses désirs ? \\
Peut-on choisir le mal ? \\
Peut-on choisir sa vie ? \\
Peut-on choisir ses désirs ? \\
Peut-on classer les arts ? \\
Peut-on commander à la nature ? \\
Peut-on communiquer ses perceptions à autrui ? \\
Peut-on communiquer son expérience ? \\
Peut-on comparer deux philosophies ? \\
Peut-on comparer les cultures ? \\
Peut-on comparer l'organisme à une machine ? \\
Peut-on comprendre ce qui est illogique ? \\
Peut-on comprendre le présent ? \\
Peut-on comprendre un acte que l'on désapprouve ? \\
Peut-on concevoir une humanité sans art ? \\
Peut-on concevoir une morale sans sanction ni obligation ? \\
Peut-on concevoir une science qui ne soit pas démonstrative ? \\
Peut-on concevoir une science sans expérience ? \\
Peut-on concevoir une société juste sans que les hommes ne le soient ? \\
Peut-on concevoir une société qui n'aurait plus besoin du droit ? \\
Peut-on concevoir une société sans État ? \\
Peut-on concevoir un État mondial ? \\
Peut-on concilier bonheur et liberté ? \\
Peut-on conclure de l'être au devoir-être ? \\
Peut-on connaître autrui ? \\
Peut-on connaître les choses telles qu'elles sont ? \\
Peut-on connaître le singulier ? \\
Peut-on connaître l'esprit ? \\
Peut-on connaître le vivant sans le dénaturer ? \\
Peut-on connaître le vivant sans recourir à la notion de finalité ? \\
Peut-on connaître l'individuel ? \\
Peut-on connaître par intuition ? \\
Peut-on considérer l'art comme un langage ? \\
Peut-on contester les droits de l'homme ? \\
Peut-on contredire l'expérience ? \\
Peut-on convaincre quelqu'un de la beauté d'une œuvre d'art ? \\
Peut-on craindre la liberté ? \\
Peut-on créer un homme nouveau ? \\
Peut-on critiquer la démocratie ? \\
Peut-on critiquer la religion ? \\
Peut-on croire ce qu'on veut ? \\
Peut-on croire en rien ? \\
Peut-on croire sans être crédule ? \\
Peut-on croire sans savoir pourquoi ? \\
Peut-on décider de croire ? \\
Peut-on décider de tout ? \\
Peut-on décider d'être heureux ? \\
Peut-on définir la morale comme l'art d'être heureux ? \\
Peut-on définir la vérité ? \\
Peut-on définir la vie ? \\
Peut-on définir le bien ? \\
Peut-on définir le bonheur ? \\
Peut-on délimiter le réel ? \\
Peut-on délimiter l'humain ? \\
Peut-on démontrer qu'on ne rêve pas ? \\
Peut-on dépasser la subjectivité ? \\
Peut-on désirer ce qui est ? \\
Peut-on désirer ce qu'on ne veut pas ? \\
Peut-on désirer ce qu'on possède ? \\
Peut-on désirer l'absolu ? \\
Peut-on désirer l'impossible ? \\
Peut-on désirer sans souffrir ? \\
Peut-on désobéir à l'État ? \\
Peut-on désobéir aux lois ? \\
Peut-on désobéir par devoir ? \\
Peut-on dialoguer avec un ordinateur ? \\
Peut-on dire ce que l'on pense ? \\
Peut-on dire ce qui n'est pas ? \\
Peut-on dire de la connaissance scientifique qu'elle procède par approximation ? \\
Peut-on dire de l'art qu'il donne un monde en partage ? \\
Peut-on dire d'une image qu'elle parle ? \\
Peut-on dire d'une œuvre d'art qu'elle est ratée ? \\
Peut-on dire d'une théorie scientifique qu'elle n'est jamais plus vraie qu'une autre mais seulement plus commode ? \\
Peut-on dire d'un homme qu'il est supérieur à un autre homme ? \\
Peut-on dire la vérité ? \\
Peut-on dire le singulier ? \\
Peut-on dire que la science ne nous fait pas connaître les choses mais les rapports entre les choses ? \\
Peut-on dire que les hommes font l'histoire ? \\
Peut-on dire que les machines travaillent pour nous ? \\
Peut-on dire que les mots pensent pour nous ? \\
Peut-on dire que l'humanité progresse ? \\
Peut-on dire que rien n'échappe à la technique ? \\
Peut-on dire qu'est vrai ce qui correspond aux faits ? \\
Peut-on dire que toutes les croyances se valent ? \\
Peut-on dire que tout est relatif ? \\
Peut-on dire qu'une théorie physique en contredit une autre ? \\
Peut-on dire toute la vérité ? \\
Peut-on discuter des goûts et des couleurs ? \\
Peut-on disposer de son corps ? \\
Peut-on distinguer différents types de causes ? \\
Peut-on distinguer entre de bons et de mauvais désirs ? \\
Peut-on distinguer entre les bons et les mauvais désirs ? \\
Peut-on distinguer le réel de l'imaginaire ? \\
Peut-on distinguer les faits de leurs interprétations ? \\
Peut-on donner un sens à son existence ? \\
Peut-on douter de sa propre existence ? \\
Peut-on douter de soi ? \\
Peut-on douter de toute vérité ? \\
Peut-on douter de tout ? \\
Peut-on échanger des idées ? \\
Peut-on échapper à ses désirs ? \\
Peut-on échapper à son temps ? \\
Peut-on échapper au cours de l'histoire ? \\
Peut-on échapper au temps ? \\
Peut-on éclairer la liberté ? \\
Peut-on écrire comme on parle ? \\
Peut-on éduquer la conscience ? \\
Peut-on éduquer le goût ? \\
Peut-on éduquer quelqu'un à être libre ? \\
Peut-on en appeler à la conscience contre la loi ? \\
Peut-on en appeler à la conscience contre l'État ? \\
Peut-on en savoir trop ? \\
Peut-on entreprendre d'éliminer la métaphysique ? \\
Peut-on établir une hiérarchie des arts ? \\
Peut-on être amoral ? \\
Peut-on être apolitique ? \\
Peut-on être citoyen du monde ? \\
Peut-on être complètement athée ? \\
Peut-on être dans le présent ? \\
Peut-on être en avance sur son temps ? \\
Peut-on être en conflit avec soi-même ? \\
Peut-on être esclave de soi-même ? \\
Peut-on être heureux dans la solitude ? \\
Peut-on être heureux sans être sage ? \\
Peut-on être heureux sans philosophie ? \\
Peut-on être heureux sans s'en rendre compte ? \\
Peut-on être heureux tout seul ? \\
Peut-on être homme sans être citoyen ? \\
Peut-on être hors de soi ? \\
Peut-on être ignorant ? \\
Peut-on être impartial ? \\
Peut-on être indifférent à l'histoire ? \\
Peut-on être indifférent à son bonheur ? \\
Peut-on être injuste et heureux ? \\
Peut-on être insensible à l'art ? \\
Peut-on être juste dans une situation injuste ? \\
Peut-on être juste dans une société injuste ? \\
Peut-on être juste sans être impartial ? \\
Peut-on être maître de soi ? \\
Peut-on être méchant volontairement ? \\
Peut-on être obligé d'aimer ? \\
Peut-on être plus ou moins libre ? \\
Peut-on être sans opinion ? \\
Peut-on être sceptique de bonne foi ? \\
Peut-on être sceptique ? \\
Peut-on être seul avec soi-même ? \\
Peut-on être seul ? \\
Peut-on être soi-même en société ? \\
Peut-on être sûr d'avoir raison ? \\
Peut-on être sûr de bien agir ? \\
Peut-on être sûr de ne pas se tromper ? \\
Peut-on être trop sage ? \\
Peut-on être trop sensible ? \\
Peut-on étudier le passé de façon objective ? \\
Peut-on exercer son esprit ? \\
Peut-on expérimenter sur le vivant ? \\
Peut-on expliquer le mal ? \\
Peut-on expliquer le monde par la matière ? \\
Peut-on expliquer le vivant ? \\
Peut-on expliquer une œuvre d'art ? \\
Peut-on faire de la politique sans supposer les hommes méchants ? \\
Peut-on faire de l'art avec tout ? \\
Peut-on faire de l'esprit un objet de science ? \\
Peut-on faire du dialogue un modèle de relation morale ? \\
Peut-on faire la paix ? \\
Peut-on faire la philosophie de l'histoire ? \\
Peut-on faire le bien d'autrui malgré lui ? \\
Peut-on faire l'économie de la notion de forme ? \\
Peut-on faire le mal en vue du bien ? \\
Peut-on faire le mal innocemment ? \\
Peut-on faire l'expérience de la nécessité ? \\
Peut-on faire l'inventaire du monde ? \\
Peut-on faire table rase du passé ? \\
Peut-on fixer des limites à la science ? \\
Peut-on fonder la morale sur la pitié ? \\
Peut-on fonder la morale ? \\
Peut-on fonder le droit sur la morale ? \\
Peut-on fonder les droits de l'homme ? \\
Peut-on fonder les mathématiques ? \\
Peut-on fonder un droit de désobéir ? \\
Peut-on fonder une éthique sur la biologie ? \\
Peut-on fonder une morale sur la nature ? \\
Peut-on fonder une morale sur le plaisir ? \\
Peut-on forcer quelqu'un à être libre ? \\
Peut-on forcer un homme à être libre ? \\
Peut-on fuir hors du monde ? \\
Peut-on fuir la société ? \\
Peut-on gâcher son talent ? \\
Peut-on gouverner sans lois ? \\
Peut-on haïr la raison ? \\
Peut-on haïr la vie ? \\
Peut-on haïr les images ? \\
Peut-on hiérarchiser les arts ? \\
Peut-on hiérarchiser les œuvres d'art ? \\
Peut-on identifier le désir au besoin ? \\
Peut-on ignorer sa propre liberté ? \\
Peut-on ignorer volontairement la vérité ? \\
Peut-on imaginer l'avenir ? \\
Peut-on imposer la liberté ? \\
Peut-on innover en politique ? \\
Peut-on interpréter la nature ? \\
Peut-on inventer en morale ? \\
Peut-on jamais aimer son prochain ? \\
Peut-on juger des œuvres d'art sans recourir à l'idée de beauté ? \\
Peut-on justifier la discrimination ? \\
Peut-on justifier la guerre ? \\
Peut-on justifier la raison d'État ? \\
Peut-on justifier le mal ? \\
Peut-on justifier le mensonge ? \\
Peut-on justifier ses choix ? \\
Peut-on légitimer la violence ? \\
Peut-on limiter l'expression de la volonté du peuple ? \\
Peut-on lutter contre le destin ? \\
Peut-on lutter contre soi-même ? \\
Peut-on maîtriser la nature ? \\
Peut-on maîtriser la technique ? \\
Peut-on maîtriser le risque ? \\
Peut-on maîtriser le temps ? \\
Peut-on maîtriser l'évolution de la technique ? \\
Peut-on maîtriser l'inconscient ? \\
Peut-on maîtriser ses désirs ? \\
Peut-on manipuler les esprits ? \\
Peut-on manquer de culture ? \\
Peut-on manquer de volonté ? \\
Peut-on mentir par humanité ? \\
Peut-on mesurer les phénomènes sociaux ? \\
Peut-on mesurer le temps ? \\
Peut-on montrer en cachant ? \\
Peut-on moraliser la guerre ? \\
Peut-on ne croire en rien ? \\
Peut-on ne pas connaître son bonheur ? \\
Peut-on ne pas croire au progrès ? \\
Peut-on ne pas croire ? \\
Peut-on ne pas être de son temps ? \\
Peut-on ne pas être égoïste ? \\
Peut-on ne pas être matérialiste ? \\
Peut-on ne pas être soi-même ? \\
Peut-on ne pas interpréter ? \\
Peut-on ne pas savoir ce que l'on dit ? \\
Peut-on ne pas savoir ce que l'on fait ? \\
Peut-on ne pas savoir ce que l'on veut ? \\
Peut-on ne pas savoir ce qu'on veut ? \\
Peut-on ne pas vouloir être heureux ? \\
Peut-on ne penser à rien ? \\
Peut-on ne rien devoir à personne ? \\
Peut-on ne rien vouloir ? \\
Peut-on ne vivre qu'au présent ? \\
Peut-on nier la réalité ? \\
Peut-on nier le réel ? \\
Peut-on nier l'évidence ? \\
Peut-on nier l'existence de la matière ? \\
Peut-on objectiver le psychisme ? \\
Peut-on opposer justice et liberté ? \\
Peut-on opposer le loisir au travail ? \\
Peut-on opposer morale et technique ? \\
Peut-on opposer nature et culture ? \\
Peut-on ôter à l'homme sa liberté ? \\
Peut-on oublier de vivre ? \\
Peut-on parler d'art primitif ? \\
Peut-on parler de ce qui n'existe pas ? \\
Peut-on parler de corruption des mœurs ? \\
Peut-on parler de dialogue des cultures ? \\
Peut-on parler de droits des animaux ? \\
Peut-on parler de mondes imaginaires ? \\
Peut-on parler de nourriture spirituelle ? \\
Peut-on parler de problèmes techniques ? \\
Peut-on parler des miracles de la technique ? \\
Peut-on parler des œuvres d'art ? \\
Peut-on parler de travail intellectuel ? \\
Peut-on parler de vérités métaphysiques ? \\
Peut-on parler de vérité subjective ? \\
Peut-on parler de vertu politique ? \\
Peut-on parler de violence d'État ? \\
Peut-on parler de « travail intellectuel » ? \\
Peut-on parler d'un droit de la guerre ? \\
Peut-on parler d'un droit de résistance ? \\
Peut-on parler d'une expérience religieuse ? \\
Peut-on parler d'une morale collective ? \\
Peut-on parler d'une religion de l'humanité ? \\
Peut-on parler d'une santé de l'âme ? \\
Peut-on parler d'une science de l'art ? \\
Peut-on parler d'un progrès dans l'histoire ? \\
Peut-on parler d'un progrès de la liberté ? \\
Peut-on parler d'un règne de la technique ? \\
Peut-on parler d'un savoir poétique ? \\
Peut-on parler d'un travail intellectuel ? \\
Peut-on parler pour en rien dire ? \\
Peut-on parler pour ne rien dire ? \\
Peut-on penser ce qu'on ne peut dire ? \\
Peut-on penser contre l'expérience ? \\
Peut-on penser illogiquement ? \\
Peut-on penser la douleur ? \\
Peut-on penser la matière ? \\
Peut-on penser la mort ? \\
Peut-on penser la nouveauté ? \\
Peut-on penser l'art sans référence au beau ? \\
Peut-on penser la vie sans penser la mort ? \\
Peut-on penser la vie ? \\
Peut-on penser le changement ? \\
Peut-on penser le monde sans la technique ? \\
Peut-on penser le temps sans l'espace ? \\
Peut-on penser l'extériorité ? \\
Peut-on penser l'impossible ? \\
Peut-on penser l'infini ? \\
Peut-on penser l'irrationnel ? \\
Peut-on penser l'œuvre d'art sans référence à l'idée de beauté ? \\
Peut-on penser sans concepts ? \\
Peut-on penser sans concept ? \\
Peut-on penser sans images ? \\
Peut-on penser sans image ? \\
Peut-on penser sans les mots ? \\
Peut-on penser sans les signes ? \\
Peut-on penser sans méthode ? \\
Peut-on penser sans ordre ? \\
Peut-on penser sans préjugés ? \\
Peut-on penser sans préjugé ? \\
Peut-on penser sans règles ? \\
Peut-on penser sans savoir que l'on pense ? \\
Peut-on penser sans signes ? \\
Peut-on penser sans son corps ? \\
Peut-on penser un art sans œuvres ? \\
Peut-on penser un droit international ? \\
Peut-on penser une conscience sans objet ? \\
Peut-on penser une métaphysique sans Dieu ? \\
Peut-on penser une société sans État ? \\
Peut-on penser un État sans violence ? \\
Peut-on penser une volonté diabolique ? \\
Peut-on percevoir sans juger ? \\
Peut-on percevoir sans s'en apercevoir ? \\
Peut-on perdre la raison ? \\
Peut-on perdre sa dignité ? \\
Peut-on perdre sa liberté ? \\
Peut-on perdre son identité ? \\
Peut-on perdre son temps ? \\
Peut-on préconiser, dans les sciences humaines et sociales, l'imitation des sciences de la nature ? \\
Peut-on prédire les événements ? \\
Peut-on prédire l'histoire ? \\
Peut-on préférer le bonheur à la vérité ? \\
Peut-on préférer l'injustice au désordre ? \\
Peut-on préférer l'ordre à la justice ? \\
Peut-on prévoir l'avenir ? \\
Peut-on prévoir le futur ? \\
Peut-on promettre le bonheur ? \\
Peut-on protéger les libertés sans les réduire ? \\
Peut-on prouver l'existence de Dieu ? \\
Peut-on prouver l'existence de l'inconscient ? \\
Peut-on prouver l'existence du monde ? \\
Peut-on prouver l'existence ? \\
Peut-on prouver une existence ? \\
Peut-on raconter sa vie ? \\
Peut-on raisonner sans règles ? \\
Peut-on ralentir la course du temps ? \\
Peut-on recommencer sa vie ? \\
Peut-on reconnaître un sens à l'histoire sans lui assigner une fin ? \\
Peut-on réduire la pensée à une espèce de comportement ? \\
Peut-on réduire le raisonnement au calcul ? \\
Peut-on réduire l'esprit à la matière ? \\
Peut-on réduire une métaphysique à une conception du monde ? \\
Peut-on réduire un homme à la somme de ses actes ? \\
Peut-on refuser de voir la vérité ? \\
Peut-on refuser la loi ? \\
Peut-on refuser la violence ? \\
Peut-on refuser le vrai ? \\
Peut-on régner innocemment ? \\
Peut-on rendre raison de tout ? \\
Peut-on rendre raison du réel ? \\
Peut-on renoncer à comprendre ? \\
Peut-on renoncer à ses droits ? \\
Peut-on renoncer à soi ? \\
Peut-on renoncer au bonheur ? \\
Peut-on réparer le vivant ? \\
Peut-on répondre d'autrui ? \\
Peut-on représenter le peuple ? \\
Peut-on représenter l'espace ? \\
Peut-on reprocher à la morale d'être abstraite ? \\
Peut-on reprocher au langage d'être équivoque ? \\
Peut-on reprocher au langage d'être parfait ? \\
Peut-on résister au vrai ? \\
Peut-on rester dans le doute ? \\
Peut-on rester insensible à la beauté ? \\
Peut-on rester sceptique ? \\
Peut-on restreindre la logique à la pensée formelle ? \\
Peut-on réunir des arts différents dans une même œuvre ? \\
Peut-on revendiquer la paix comme un droit ? \\
Peut-on revenir sur ses erreurs ? \\
Peut-on rire de tout ? \\
Peut-on rompre avec la société ? \\
Peut-on rompre avec le passé ? \\
Peut-on s'abstenir de penser politiquement ? \\
Peut-on s'accorder sur des vérités morales ? \\
Peut-on s'affranchir des lois ? \\
Peut-on s'attendre à tout ? \\
Peut-on savoir ce qui est bien ? \\
Peut-on savoir quelque chose de l'avenir ? \\
Peut-on savoir sans croire ? \\
Peut-on se choisir un destin ? \\
Peut-on se connaître soi-même ? \\
Peut-on se désintéresser de la politique ? \\
Peut-on se désintéresser de son bonheur ? \\
Peut-on se duper soi-même ? \\
Peut-on se faire une idée de tout ? \\
Peut-on se fier à l'expérience vécue ? \\
Peut-on se fier à l'intuition ? \\
Peut-on se fier à son intuition ? \\
Peut-on se gouverner soi-même ? \\
Peut-on se méfier de soi-même ? \\
Peut-on se mentir à soi-même \\
Peut-on se mentir à soi-même ? \\
Peut-on se mettre à la place d'autrui ? \\
Peut-on se mettre à la place de l'autre ? \\
Peut-on s'en tenir au présent ? \\
Peut-on séparer l'homme et l'œuvre ? \\
Peut-on séparer politique et économie ? \\
Peut-on se passer de chef ? \\
Peut-on se passer de croire ? \\
Peut-on se passer de croyances ? \\
Peut-on se passer de croyance ? \\
Peut-on se passer de Dieu ? \\
Peut-on se passer de frontières ? \\
Peut-on se passer de la religion ? \\
Peut-on se passer de l'État ? \\
Peut-on se passer de méthode ? \\
Peut-on se passer de mythes ? \\
Peut-on se passer de principes ? \\
Peut-on se passer de religion ? \\
Peut-on se passer de représentants ? \\
Peut-on se passer de spiritualité ? \\
Peut-on se passer des relations ? \\
Peut-on se passer d'État ? \\
Peut-on se passer de techniques de raisonnement ? \\
Peut-on se passer de technique ? \\
Peut-on se passer de toute religion ? \\
Peut-on se passer d'idéal ? \\
Peut-on se passer d'un maître ? \\
Peut-on se prescrire une loi ? \\
Peut-on se promettre quelque chose à soi-même ? \\
Peut-on se punir soi-même ? \\
Peut-on se régler sur des exemples en politique ? \\
Peut-on se retirer du monde ? \\
Peut-on se tromper en se croyant heureux ? \\
Peut-on se vouloir parfait ? \\
Peut-on sortir de la subjectivité ? \\
Peut-on sortir de sa conscience ? \\
Peut-on souhaiter le gouvernement des meilleurs ? \\
Peut-on suivre une règle ? \\
Peut-on suspendre le temps ? \\
Peut-on suspendre son jugement ? \\
Peut-on sympathiser avec l'ennemi ? \\
Peut-on tirer des leçons de l'histoire ? \\
Peut-on toujours faire ce qu'on doit ? \\
Peut-on toujours savoir entièrement ce que l'on dit ? \\
Peut-on tout analyser ? \\
Peut-on tout attendre de l'État ? \\
Peut-on tout définir ? \\
Peut-on tout démontrer ? \\
Peut-on tout désirer ? \\
Peut-on tout dire ? \\
Peut-on tout donner ? \\
Peut-on tout échanger ? \\
Peut-on tout enseigner ? \\
Peut-on tout expliquer ? \\
Peut-on tout exprimer ? \\
Peut-on tout imaginer ? \\
Peut-on tout imiter ? \\
Peut-on tout interpréter ? \\
Peut-on tout mathématiser ? \\
Peut-on tout mesurer ? \\
Peut-on tout ordonner ? \\
Peut-on tout pardonner ? \\
Peut-on tout partager ? \\
Peut-on tout prévoir ? \\
Peut-on tout prouver ? \\
Peut-on tout soumettre à la discussion ? \\
Peut-on tout tolérer ? \\
Peut-on traiter autrui comme un moyen ? \\
Peut-on traiter un être vivant comme une machine ? \\
Peut-on transformer le réel ? \\
Peut-on transiger avec les principes ? \\
Peut-on trouver du plaisir à l'ennui ? \\
Peut-on vivre avec les autres ? \\
Peut-on vivre dans le doute ? \\
Peut-on vivre en marge de la société ? \\
Peut-on vivre en sceptique ? \\
Peut-on vivre hors du temps ? \\
Peut-on vivre pour la vérité ? \\
Peut-on vivre sans aimer ? \\
Peut-on vivre sans art ? \\
Peut-on vivre sans aucune certitude ? \\
Peut-on vivre sans croyances ? \\
Peut-on vivre sans désir ? \\
Peut-on vivre sans échange ? \\
Peut-on vivre sans illusions ? \\
Peut-on vivre sans l'art ? \\
Peut-on vivre sans le plaisir de vivre ? \\
Peut-on vivre sans lois ? \\
Peut-on vivre sans passion ? \\
Peut-on vivre sans peur ? \\
Peut-on vivre sans principes ? \\
Peut-on vivre sans réfléchir ? \\
Peut-on vivre sans ressentiment ? \\
Peut-on vivre sans rien espérer ? \\
Peut-on vivre sans sacré ? \\
Peut-on voir sans croire ? \\
Peut-on vouloir ce qu'on ne désire pas ? \\
Peut-on vouloir le bonheur d'autrui ? \\
Peut-on vouloir le mal pour le mal ? \\
Peut-on vouloir le mal ? \\
Peut-on vouloir l'impossible ? \\
Peut-on vouloir sans désirer ? \\
Philosopher, est-ce apprendre à vivre ? \\
Philosophe-t-on pour être heureux ? \\
Philosophie et mathématiques \\
Philosophie et métaphysique \\
Philosophie et poésie \\
Philosophie et système \\
Photographier le réel \\
Physique et mathématiques \\
Physique et métaphysique \\
Pitié et compassion \\
Pitié et cruauté \\
Pitié et mépris \\
Plaider \\
Plaisir et bonheur \\
Plaisir et douleur \\
Plaisirs, honneurs, richesses \\
Pluralisme et politique \\
Pluralité et unité \\
Plusieurs religions valent-elles mieux qu'une seule ? \\
Poésie et philosophie \\
Poésie et vérité \\
Poétique et prosaïque \\
Point de vue du créateur et point de vue du spectateur \\
Police et politique \\
Politique et coopération \\
Politique et esthétique \\
Politique et mémoire \\
Politique et parole \\
Politique et participation \\
Politique et passions \\
Politique et propagande \\
Politique et secret \\
Politique et technologie \\
Politique et territoire \\
Politique et trahison \\
Politique et unité \\
Politique et vérité \\
Possession et propriété \\
Pour connaître, suffit-il de démontrer ? \\
Pour être heureux, faut-il renoncer à la perfection ? \\
Pour être homme, faut-il être citoyen ? \\
Pour être libre, faut-il renoncer à être heureux ? \\
Pour être un bon observateur faut-il être un bon théoricien ? \\
Pour juger, faut-il seulement apprendre à raisonner ? \\
Pour qui se prend-on ? \\
Pourquoi aimons-nous la musique ? \\
Pourquoi aller contre son désir ? \\
Pourquoi a-t-on peur de la folie ? \\
Pourquoi avoir recours à la notion d'inconscient ? \\
Pourquoi châtier ? \\
Pourquoi chercher à connaître le passé ? \\
Pourquoi chercher à se distinguer ? \\
Pourquoi chercher la vérité ? \\
Pourquoi chercher un sens à l'histoire ? \\
Pourquoi cherche-t-on à connaître ? \\
Pourquoi commémorer ? \\
Pourquoi communiquer ? \\
Pourquoi conserver les œuvres d'art ? \\
Pourquoi construire des monuments ? \\
Pourquoi critiquer la raison ? \\
Pourquoi critiquer le conformisme ? \\
Pourquoi croyons-nous ? \\
Pourquoi défendre le faible ? \\
Pourquoi définir ? \\
Pourquoi délibérer ? \\
Pourquoi démontrer ce que l'on sait être vrai ? \\
Pourquoi démontrer ? \\
Pourquoi des artifices ? \\
Pourquoi des artistes ? \\
Pourquoi des cérémonies ? \\
Pourquoi des châtiments ? \\
Pourquoi des classifications ? \\
Pourquoi des conflits ? \\
Pourquoi des devoirs ? \\
Pourquoi des élections ? \\
Pourquoi des exemples ? \\
Pourquoi des fictions ? \\
Pourquoi des géométries ? \\
Pourquoi des guerres ? \\
Pourquoi des historiens ? \\
Pourquoi des hypothèses ? \\
Pourquoi des idoles ? \\
Pourquoi des institutions ? \\
Pourquoi des interdits ? \\
Pourquoi désirer la sagesse ? \\
Pourquoi désirer l'immortalité ? \\
Pourquoi désire-t-on ce dont on n'a nul besoin ? \\
Pourquoi désirons-nous ? \\
Pourquoi des logiciens ? \\
Pourquoi des lois ? \\
Pourquoi des maîtres ? \\
Pourquoi des métaphores ? \\
Pourquoi des modèles ? \\
Pourquoi des musées ? \\
Pourquoi des œuvres d'art ? \\
Pourquoi des philosophes ? \\
Pourquoi des poètes ? \\
Pourquoi des psychologues ? \\
Pourquoi des religions ? \\
Pourquoi des rites ? \\
Pourquoi des sociologues ? \\
Pourquoi des traditions ? \\
Pourquoi des utopies ? \\
Pourquoi dialogue-t-on ? \\
Pourquoi Dieu se soucierait-il des affaires humaines ? \\
Pourquoi dire la vérité ? \\
Pourquoi domestiquer ? \\
Pourquoi donner des leçons de morale ? \\
Pourquoi donner ? \\
Pourquoi échanger des idées ? \\
Pourquoi écrire ? \\
Pourquoi écrit-on des lois ? \\
Pourquoi écrit-on les lois ? \\
Pourquoi écrit-on l'Histoire ? \\
Pourquoi écrit-on ? \\
Pourquoi est-il difficile de rectifier une erreur ? \\
Pourquoi être exigeant ? \\
Pourquoi être moral ? \\
Pourquoi être raisonnable ? \\
Pourquoi étudier le vivant ? \\
Pourquoi étudier l'Histoire ? \\
Pourquoi exiger la cohérence \\
Pourquoi exposer les œuvres d'art ? \\
Pourquoi faire confiance ? \\
Pourquoi faire de la politique ? \\
Pourquoi faire de l'histoire ? \\
Pourquoi faire la guerre ? \\
Pourquoi faire son devoir ? \\
Pourquoi fait-on le mal ? \\
Pourquoi faudrait-il être cohérent ? \\
Pourquoi faut-il diviser le travail ? \\
Pourquoi faut-il être cohérent ? \\
Pourquoi faut-il être juste ? \\
Pourquoi faut-il être poli ? \\
Pourquoi faut-il travailler ? \\
Pourquoi formaliser des arguments ? \\
Pourquoi imiter ? \\
Pourquoi interprète-t-on ? \\
Pourquoi joue-t-on ? \\
Pourquoi la critique ? \\
Pourquoi la curiosité est-elle un vilain défaut ? \\
Pourquoi la guerre ? \\
Pourquoi la justice a-t-elle besoin d'institutions ? \\
Pourquoi la musique intéresse-t-elle le philosophe ? \\
Pourquoi la prison ? \\
Pourquoi la prohibition de l'inceste ? \\
Pourquoi la raison recourt-elle à l'hypothèse ? \\
Pourquoi la réalité peut-elle dépasser la fiction ? \\
Pourquoi l'art intéresse-t-il les philosophes ? \\
Pourquoi l'économie est-elle politique ? \\
Pourquoi le droit international est-il imparfait ? \\
Pourquoi les droits de l'homme sont-ils universels ? \\
Pourquoi les États se font-ils la guerre ? \\
Pourquoi les hommes mentent-ils ? \\
Pourquoi les mathématiques s'appliquent-elles à la réalité ? \\
Pourquoi les œuvres d'art résistent-elles au temps ? \\
Pourquoi le sport ? \\
Pourquoi les sciences ont-elles une histoire ? \\
Pourquoi les sociétés ont-elles besoin de lois ? \\
Pourquoi le théâtre ? \\
Pourquoi l'ethnologue s'intéresse-t-il à la vie urbaine ? \\
Pourquoi l'homme a-t-il des droits ? \\
Pourquoi l'homme est-il l'objet de plusieurs sciences ? \\
Pourquoi l'homme travaille-t-il ? \\
Pourquoi lire des romans ? \\
Pourquoi lire les poètes ? \\
Pourquoi lit-on des romans ? \\
Pourquoi mentir ? \\
Pourquoi ne s'entend-on pas sur la nature de ce qui est réel ? \\
Pourquoi nous soucier du sort des générations futures ? \\
Pourquoi nous souvenons-nous ? \\
Pourquoi nous trompons-nous ? \\
Pourquoi nous-trompons nous ? \\
Pourquoi obéir aux lois ? \\
Pourquoi obéir ? \\
Pourquoi obéit-on aux lois ? \\
Pourquoi obéit-on ? \\
Pourquoi parler de fautes de goût ? \\
Pourquoi parler du travail comme d'un droit ? \\
Pourquoi parle-t-on d'économie politique ? \\
Pourquoi parle-t-on d'une « société civile » ? \\
Pourquoi parle-t-on ? \\
Pourquoi parlons-nous ? \\
Pourquoi pas plusieurs dieux ? \\
Pourquoi pas ? \\
Pourquoi penser à la mort ? \\
Pourquoi pensons-nous ? \\
Pourquoi philosopher ? \\
Pourquoi pleure-t-on au cinéma ? \\
Pourquoi pleure-t-on ? \\
Pourquoi plusieurs sciences ? \\
Pourquoi préférer l'original à la reproduction ? \\
Pourquoi préférer l'original à sa reproduction ? \\
Pourquoi préférer l'original ? \\
Pourquoi préserver l'environnement ? \\
Pourquoi prier ? \\
Pourquoi prouver l'existence de Dieu ? \\
Pourquoi punir ? \\
Pourquoi punit-on ? \\
Pourquoi raconter des histoires ? \\
Pourquoi rechercher la vérité ? \\
Pourquoi rechercher le bonheur ? \\
Pourquoi refuse-t-on la conscience à l'animal ? \\
Pourquoi respecter autrui ? \\
Pourquoi respecter le droit ? \\
Pourquoi respecter les anciens ? \\
Pourquoi rit-on ? \\
Pourquoi sauver les apparences ? \\
Pourquoi sauver les phénomènes ? \\
Pourquoi se confesser ? \\
Pourquoi se divertir ? \\
Pourquoi se fier à autrui ? \\
Pourquoi se mettre à la place d'autrui ? \\
Pourquoi séparer les pouvoirs ? \\
Pourquoi se révolter ? \\
Pourquoi se soucier du futur ? \\
Pourquoi s'étonner ? \\
Pourquoi s'exprimer ? \\
Pourquoi s'inspirer de l'art antique ? \\
Pourquoi s'intéresser à l'histoire ? \\
Pourquoi s'intéresser à l'origine ? \\
Pourquoi s'interroger sur l'origine du langage ? \\
Pourquoi soigner son apparence ? \\
Pourquoi sommes-nous déçus par les œuvres d'un faussaire ? \\
Pourquoi sommes-nous des êtres moraux ? \\
Pourquoi sommes-nous moraux ? \\
Pourquoi suivre l'actualité ? \\
Pourquoi tenir ses promesses ? \\
Pourquoi théoriser ? \\
Pourquoi transformer le monde ? \\
Pourquoi transmettre ? \\
Pourquoi travailler ? \\
Pourquoi travaille-t-on ? \\
Pourquoi un droit du travail ? \\
Pourquoi une instruction publique ? \\
Pourquoi un fait devrait-il être établi ? \\
Pourquoi veut-on changer le monde ? \\
Pourquoi veut-on la vérité ? \\
Pourquoi vivons-nous ? \\
Pourquoi vivre ensemble ? \\
Pourquoi vouloir avoir raison ? \\
Pourquoi vouloir se connaître ? \\
Pourquoi voulons-nous savoir ? \\
Pourquoi voyager ? \\
Pourquoi y a-t-il des conflits insolubles ? \\
Pourquoi y a-t-il des institutions ? \\
Pourquoi y a-t-il des religions ? \\
Pourquoi y a-t-il du mal dans le monde ? \\
Pourquoi y a-t-il plusieurs façons de démontrer ? \\
Pourquoi y a-t-il plusieurs langues ? \\
Pourquoi y a-t-il plusieurs sciences ? \\
Pourquoi y a-t-il quelque chose plutôt que rien ? \\
Pourquoi y a-t-il des lois ? \\
Pourquoi y a-t-il plusieurs philosophies ? \\
Pourquoi ? \\
Pourrait-on se passer de l'argent ? \\
Pourrions-nous comprendre une pensée non humaine ? \\
Pour vivre heureux, vivons cachés \\
Pouvoir et autorité \\
Pouvoir et contre-pouvoir \\
Pouvoir et devoir \\
Pouvoir et politique \\
Pouvoir et puissance \\
Pouvoir et savoir \\
Pouvoir, magie, secret \\
Pouvoirs et libertés \\
Pouvoir temporel et pouvoir spirituel \\
Pouvons-nous communiquer ce que nous sentons ? \\
Pouvons-nous connaître sans interpréter ? \\
Pouvons-nous désirer ce qui nous nuit ? \\
Pouvons-nous devenir meilleurs ? \\
Pouvons-nous dissocier le réel de nos interprétations ? \\
Pouvons-nous être certains que nous ne rêvons pas ? \\
Pouvons-nous être objectifs ? \\
Pouvons-nous faire l'expérience de la liberté ? \\
Pouvons-nous justifier nos croyances ? \\
Pouvons-nous savoir ce que nous ignorons ? \\
Prédicats et relations \\
Prédiction et probabilité \\
Prédire et expliquer \\
Prémisses et conclusions \\
Prendre conscience \\
Prendre des risques \\
Prendre la parole \\
Prendre le pouvoir \\
Prendre les armes \\
Prendre ses désirs pour des réalités \\
Prendre ses responsabilités \\
Prendre soin \\
Prendre son temps \\
Prendre son temps, est-ce le perdre ? \\
Prendre une décision \\
Prendre une décision politique \\
Présence et absence \\
Présence et représentation \\
Preuve et démonstration \\
Prévoir \\
Prévoir les comportements humains \\
Primitif ou premier ? \\
Principe et cause \\
Principe et commencement \\
Principe et fondement \\
Principes et stratégie \\
Probabilité et explication scientifique \\
Production et création \\
Produire et créer \\
Promettre \\
Proposition et jugement \\
Propriétés artistiques, propriétés esthétiques \\
Prose et poésie \\
Prospérité et sécurité \\
Protester \\
Prouver \\
Prouver Dieu \\
Prouver en métaphysique \\
Prouver et démontrer \\
Prouver et éprouver \\
Prouver et réfuter \\
Prouver la force d'âme \\
Prouver l'existence de Dieu \\
Prouvez-le ! \\
Providence et destin \\
Prudence et liberté \\
Psychologie et contrôle des comportements \\
Psychologie et métaphysique \\
Psychologie et neurosciences \\
Publier \\
Puis-je aimer tous les hommes ? \\
Puis-je comprendre autrui ? \\
Puis-je décider de croire ? \\
Puis-je dire « ceci est mon corps » ? \\
Puis-je douter de ma propre existence ? \\
Puis-je être dans le vrai sans le savoir ? \\
Puis-je être heureux dans un monde chaotique ? \\
Puis-je être libre sans être responsable ? \\
Puis-je être sûr de bien agir ? \\
Puis-je être sûr de ne pas me tromper ? \\
Puis-je être sûr que je ne rêve pas ? \\
Puis-je être universel ? \\
Puis-je faire ce que je veux de mon corps ? \\
Puis-je faire confiance à mes sens ? \\
Puis-je invoquer l'inconscient sans ruiner la morale ? \\
Puis-je me passer d'imiter autrui ? \\
Puis-je ne croire que ce que je vois ? \\
Puis-je ne pas vouloir ce que je désire ? \\
Puis-je ne rien croire ? \\
Puis-je répondre des autres ? \\
Puis-je savoir ce qui m'est propre ? \\
Pulsion et instinct \\
Pulsions et passions \\
Punir \\
Punir ou soigner ? \\
Punition et vengeance \\
Qu'ai-je le droit d'exiger d'autrui ? \\
Qu'ai-je le droit d'exiger des autres ? \\
Qu'aime-t-on dans l'amour ? \\
Qu'aime-t-on ? \\
Qualité et quantité \\
Qualités premières, qualités secondes \\
Quand agit-on ? \\
Quand faut-il désobéir aux lois ? \\
Quand faut-il désobéir ? \\
Quand faut-il mentir ? \\
Quand la guerre finira-t-elle ? \\
Quand la technique devient-elle art ? \\
Quand le temps passe, que reste-t-il ? \\
Quand pense-t-on ? \\
Quand peut-on se passer de théories ? \\
Quand suis-je en faute ? \\
Quand une autorité est-elle légitime ? \\
Quand y a-t-il de l'art ? \\
Quand y a-t-il œuvre ? \\
Quand y a-t-il paysage ? \\
Quand y a-t-il peuple ? \\
Qu'anticipent les romans d'anticipation ? \\
Quantification et pensée scientifique \\
Quantité et qualité \\
Qu'a perdu le fou ? \\
Qu'appelle-t-on chef-d'œuvre ? \\
Qu'appelle-t-on destin ? \\
Qu'appelle-t-on penser ? \\
Qu'apprend-on dans les livres ? \\
Qu'apprend-on des romans ? \\
Qu'apprend-on en commettant une faute ? \\
Qu'apprend-on quand on apprend à parler ? \\
Qu'apprenons-nous de nos affects ? \\
Qu'a-t-on le droit de pardonner ? \\
Qu'a-t-on le droit d'exiger ? \\
Qu'a-t-on le droit d'interpréter ? \\
Qu'attendons-nous de la technique ? \\
Qu'attendons-nous pour être heureux ? \\
Qu'avons-nous à apprendre des historiens ? \\
Qu'avons-nous en commun ? \\
Que célèbre l'art ? \\
Que cherchons-nous dans le regard des autres ? \\
Que choisir ? \\
Que connaissons-nous du vivant ? \\
Que construit le politique ? \\
Que coûte une victoire ? \\
Que crée l'artiste ? \\
Que déduire d'une contradiction ? \\
Que démontrent nos actions ? \\
Que désire-t-on ? \\
Que désirons-nous quand nous désirons savoir ?Qu'est-ce qu'un événement historique ? \\
Que désirons-nous ? \\
Que devons-nous à autrui ? \\
Que devons-nous à l'État ? \\
Que disent les légendes ? \\
Que disent les tables de vérité ? \\
Que dit la loi ? \\
Que dit la musique ? \\
Que dois-je à autrui ? \\
Que dois-je à l'État ? \\
Que dois-je respecter en autrui ? \\
Que doit la pensée à l'écriture ? \\
Que doit la science à la technique ? \\
Que doit-on aux morts ? \\
Que doit-on croire ? \\
Que doit-on désirer pour ne pas être déçu ? \\
Que doit-on faire de ses rêves ? \\
Que doit-on savoir avant d'agir ? \\
Que faire de la diversité des arts ? \\
Que faire de nos émotions ? \\
Que faire de nos passions ? \\
Que faire de notre cerveau ? \\
Que faire des adversaires ? \\
Que faire ? \\
Que fait la police ? \\
Que faut-il absolument savoir ? \\
Que faut-il craindre ? \\
Que faut-il pour faire un monde ? \\
Que faut-il respecter ? \\
Que faut-il savoir pour agir ? \\
Que faut-il savoir pour gouverner ? \\
Que gagne-t-on à travailler ? \\
Que la nature soit explicable, est-ce explicable ? \\
Quel contrôle a-t-on sur son corps ? \\
Quel est le bon nombre d'amis ? \\
Quel est le but de la politique ? \\
Quel est le but d'une théorie physique ? \\
Quel est le but du travail scientifique ? \\
Quel est le contraire du travail ? \\
Quel est le fondement de la propriété ? \\
Quel est le fondement de l'autorité ? \\
Quel est le poids du passé ? \\
Quel est le pouvoir de l'art ? \\
Quel est le pouvoir des mots ? La prévoyance \\
Quel est le rôle de la créativité dans les sciences ? \\
Quel est le rôle du concept en art ? \\
Quel est le rôle du médecin ? \\
Quel est le sens du progrès technique ? \\
Quel est le sujet de la pensée ? \\
Quel est le sujet de l'histoire ? \\
Quel est le sujet du devenir ? \\
Quel est l'être de l'illusion ? \\
Quel est l'homme des Droits de l'homme ? \\
Quel est l'objet de la biologie ? \\
Quel est l'objet de la métaphysique ? \\
Quel est l'objet de l'amour ? \\
Quel est l'objet de la perception ? \\
Quel est l'objet de la philosophie politique ? \\
Quel est l'objet de la science ? \\
Quel est l'objet de l'échange ? \\
Quel est l'objet de l'esthétique ? \\
Quel est l'objet de l'histoire ? \\
Quel est l'objet des mathématiques ? \\
Quel est l'objet des sciences humaines ? \\
Quel est l'objet des sciences politiques ? \\
Quel est l'objet du désir ? \\
Quel être peut être un sujet de droits ? \\
Quelle causalité pour le vivant ? \\
Quelle confiance accorder au langage ? \\
Quelle est la cause du désir ? \\
Quelle est la fin de la science ? \\
Quelle est la fin de l'État ? \\
Quelle est la fonction première de l'État ? \\
Quelle est la force de la loi ? \\
Quelle est la matière de l'œuvre d'art ? \\
Quelle est la place de l'imagination dans la vie de l'esprit ? \\
Quelle est la place du hasard dans l'histoire ? \\
Quelle est la portée d'un exemple ? \\
Quelle est la réalité de la matière ? \\
Quelle est la réalité de l'avenir ? \\
Quelle est la réalité d'une idée ? \\
Quelle est la réalité du passé ? \\
Quelle est la spécificité de la communauté politique ? \\
Quelle est la valeur de l'expérience ? \\
Quelle est la valeur des hypothèses ? \\
Quelle est la valeur d'une expérimentation ? \\
Quelle est la valeur d'une œuvre d'art ? \\
Quelle est la valeur du rêve ? \\
Quelle est la valeur du témoignage ? \\
Quelle est la valeur du vivant ? \\
Quelle est l'unité du « je » ? \\
Quelle idée le fanatique se fait-il de la vérité ? \\
Quelle politique fait-on avec les sciences humaines ? \\
Quelle réalité attribuer à la matière ? \\
Quelle réalité l'art nous fait-il connaître ? \\
Quelle réalité la science décrit-elle ? \\
Quelle réalité peut-on accorder au temps ? \\
Quelles actions permettent d'être heureux ? \\
Quelle sorte d'histoire ont les sciences ? \\
Quelles règles la technique dicte-t-elle à l'art ? \\
Quelles sont les caractéristiques d'une proposition morale ? \\
Quelles sont les caractéristiques d'un être vivant ? \\
Quelles sont les limites de la démonstration ? \\
Quelles sont les limites de la souveraineté ? \\
Quelle valeur accorder à l'expérience ? \\
Quelle valeur devons accorder à l'expérience ? \\
Quelle valeur devons-nous accorder à l'expérience ? \\
Quelle valeur devons-nous accorder à l'intuition ? \\
Quelle valeur donner à la notion de « corps social » ? \\
Quelle valeur peut-on accorder à l'expérience ? \\
Quelle vérité y-a-t-il dans la perception ? \\
Quel réel pour l'art ? \\
Quel rôle attribuer à l'intuition a priori dans une philosophie des mathématiques ? \\
Quel rôle la logique joue-t-elle en mathématiques ? \\
Quel rôle l'imagination joue-t-elle en mathématiques ? \\
Quels désirs dois-je m'interdire ? \\
Quel sens donner à l'expression « gagner sa vie » ? \\
Quel sens y a-t-il à se demander si les sciences humaines sont vraiment des sciences ? \\
Quels sont les droits de la conscience ? \\
Quels sont les fondements de l'autorité ? \\
Quels sont les moyens légitimes de la politique ? \\
Quel usage faut-il faire des exemples ? \\
Quel usage peut-on faire des fictions ? \\
Que manque-t-il à une machine pour être vivante ? \\
Que manque-t-il aux machines pour être des organismes ? \\
Que mesure-t-on du temps ? \\
Que montre l'image ? \\
Que montre une démonstration ? \\
Que montre un tableau ? \\
Que ne peut-on pas expliquer ? \\
Que nous append l'histoire ? \\
Que nous apporte l'art ? \\
Que nous apporte la vérité ? \\
Que nous apprend la définition de la vérité ? \\
Que nous apprend la diversité des langues ? \\
Que nous apprend la fiction sur la réalité ? \\
Que nous apprend la grammaire ? \\
Que nous apprend la maladie sur la santé ? \\
Que nous apprend la musique ? \\
Que nous apprend la poésie ? \\
Que nous apprend la psychanalyse de l'homme ? \\
Que nous apprend la sociologie des sciences ? \\
Que nous apprend la vie ? \\
Que nous apprend le cinéma ? \\
Que nous apprend le faux ? \\
Que nous apprend le plaisir ? \\
Que nous apprend le toucher ? \\
Que nous apprend l'étude du cerveau ? \\
Que nous apprend l'expérience ? \\
Que nous apprend l'histoire de l'art ? \\
Que nous apprend l'histoire des sciences ? \\
Que nous apprend, sur la politique, l'utopie ? \\
Que nous apprennent les algorithmes sur nos sociétés ? \\
Que nous apprennent les animaux sur nous-mêmes ? \\
Que nous apprennent les animaux ? \\
Que nous apprennent les controverses scientifiques ? \\
Que nous apprennent les expériences de pensée ? \\
Que nous apprennent les faits divers ? \\
Que nous apprennent les illusions d'optique ? \\
Que nous apprennent les jeux ? \\
Que nous apprennent les langues étrangères ? \\
Que nous apprennent les machines ? \\
Que nous apprennent les métaphores ? \\
Que nous apprennent les mythes ? \\
Que nous enseigne l'expérience ? \\
Que nous enseignent les œuvres d'art ? \\
Que nous enseignent les sens ? \\
Que nous montre le cinéma ? \\
Que nous montrent les natures mortes ? \\
Que nous réserve l'avenir ? \\
Que nul n'entre ici s'il n'est géomètre \\
Que partage-t-on avec les animaux ? \\
Que peindre ? \\
Que peint le peintre ? \\
Que penser de l'adage : « Que la justice s'accomplisse, le monde dût-il périr » (Fiat justitia pereat mundus) ? \\
Que penser de la formule : « il faut suivre la nature » ? \\
Que penser de l'opposition travail manuel, travail intellectuel ? \\
Que percevons-nous d'autrui ? \\
Que percevons-nous du monde extérieur ? \\
Que percevons-nous ? \\
Que perçoit-on ? \\
Que perd la pensée en perdant l'écriture ? \\
Que perdrait la pensée en perdant l'écriture ? \\
Que peut expliquer l'histoire ? \\
Que peut la force ? \\
Que peut la musique ? \\
Que peut la philosophie ? \\
Que peut la politique ? \\
Que peut la raison ? \\
Que peut l'art ? \\
Que peut la science ? \\
Que peut la théorie ? \\
Que peut la volonté ? \\
Que peut le corps ? \\
Que peut le politique ? \\
Que peut l'esprit sur la matière ? \\
Que peut l'esprit ? \\
Que peut l'État ? \\
Que peut-on attendre de l'État ? \\
Que peut-on attendre du droit international ? \\
Que peut-on calculer ? \\
Que peut-on comprendre immédiatement ? \\
Que peut-on comprendre qu'on ne puisse expliquer ? \\
Que peut-on contre un préjugé ? \\
Que peut-on cultiver ? \\
Que peut-on démontrer ? \\
Que peut-on dire de l'être ? \\
Que peut-on échanger ? \\
Que peut-on interdire ? \\
Que peut-on partager ? \\
Que peut-on savoir de l'inconscient ? \\
Que peut-on savoir de soi ? \\
Que peut-on savoir du réel ? \\
Que peut-on savoir par expérience ? \\
Que peut-on sur autrui ? \\
Que peut-on voir ? \\
Que peut un corps ? \\
Que pouvons-nous aujourd'hui apprendre des sciences d'autrefois ? \\
Que pouvons-nous espérer de la connaissance du vivant ? \\
Que pouvons-nous faire de notre passé ? \\
Que produit l'inconscient ? \\
Que prouvent les faits ? \\
Que prouvent les preuves de l'existence de Dieu ? \\
Que recherche l'artiste ? \\
Que répondre au sceptique ? \\
Que reste-t-il d'une existence ? \\
Que sais-je d'autrui ? \\
Que sais-je de ma souffrance ? \\
Que sait la conscience ? \\
Que sait-on de soi ? \\
Que sait-on du réel ? \\
Que savons-nous de l'inconscient ? \\
Que serait la vie sans l'art ? \\
Que serait le meilleur des mondes ? \\
Que serait un art total ? \\
Que serait une démocratie parfaite ? \\
Que serions-nous sans l'État ? \\
Que signifie apprendre ? \\
Que signifie connaître ? \\
Que signifie être en guerre ? \\
Que signifie être mortel ? \\
Que signifie la mort ? \\
Que signifie l'idée de technoscience ? \\
Que signifient les mots ? \\
Que signifie pour l'homme être mortel ? \\
Que signifier « juger en son âme et conscience » ? \\
Que signifie « donner le change » ? \\
Que sondent les sondages d'opinion ? \\
Que sont les apparences ? \\
Qu'est-ce la technique ? \\
Qu'est-ce le mal radical ? \\
Qu'est-ce qu'agir ensemble ? \\
Qu'est-ce qu'aimer une œuvre d'art ? \\
Qu'est-ce qu'apprendre ? \\
Qu'est-ce qu'argumenter ? \\
Qu'est-ce qu'avoir conscience de soi ? \\
Qu'est-ce qu'avoir de l'expérience ? \\
Qu'est-ce qu'avoir du goût ? \\
Qu'est-ce qu'avoir du style ? \\
Qu'est-ce qu'avoir un droit ? \\
Qu'est-ce que calculer ? \\
Qu'est-ce que catégoriser ? \\
Qu'est-ce que commencer ? \\
Qu'est-ce que composer une œuvre ? \\
Qu'est-ce que comprendre une œuvre d'art ? \\
Qu'est-ce que comprendre ? \\
Qu'est-ce que créer ? \\
Qu'est-ce que croire ? \\
Qu'est-ce que décider ? \\
Qu'est-ce que définir ? \\
Qu'est-ce que démontrer ? \\
Qu'est-ce que déraisonner ? \\
Qu'est-ce que Dieu pour athée ? \\
Qu'est-ce que Dieu pour un athée ? \\
Qu'est-ce que discuter ? \\
Qu'est-ce qu'éduquer ? \\
Qu'est-ce que faire autorité ? \\
Qu'est-ce que faire preuve d'humanité ? \\
Qu'est-ce que faire une expérience ? \\
Qu'est-ce que gouverner ? \\
Qu'est-ce que guérir ? \\
Qu'est-ce que jouer ? \\
Qu'est-ce que juger ? \\
Qu'est-ce que la barbarie ? \\
Qu'est-ce que la causalité ? \\
Qu'est-ce que la critique ? \\
Qu'est-ce que la culture générale \\
Qu'est-ce que la démocratie ? \\
Qu'est-ce que la folie ? \\
Qu'est-ce que la normalité ? \\
Qu'est-ce que la politique ? \\
Qu'est-ce que la psychologie ? \\
Qu'est-ce que la raison d'État ? \\
Qu'est-ce que l'art contemporain ? \\
Qu'est-ce que la science saisit du vivant ? \\
Qu'est-ce que la science, si elle inclut la psychanalyse ? \\
Qu'est-ce que la scientificité ? \\
Qu'est-ce que la souveraineté ? \\
Qu'est-ce que la tragédie ? \\
Qu'est-ce que la valeur marchande ? \\
Qu'est-ce que la vie bonne ? \\
Qu'est-ce que la vie ? \\
Qu'est-ce que le bonheur ? \\
Qu'est-ce que le cinéma a changé dans l'idée que l'on se fait du temps ? \\
Qu'est-ce que le cinéma donne à voir ? \\
Qu'est-ce que le courage ? \\
Qu'est-ce que le désordre ? \\
Qu'est-ce que le dogmatisme ? \\
Qu'est-ce que le hasard ? \\
Qu'est-ce que le langage ordinaire ? \\
Qu'est-ce que le malheur ? \\
Qu'est-ce que le moi ? \\
Qu'est-ce que le naturalisme ? \\
Qu'est-ce que l'enfance ? \\
Qu'est-ce que le nihilisme ? \\
Qu'est-ce que le pathologique nous apprend sur le normal ? \\
Qu'est-ce que le présent ? \\
Qu'est-ce que le réel ? \\
Qu'est-ce que le sacré ? \\
Qu'est-ce que le sens pratique ? \\
Qu'est-ce que le sublime ? \\
Qu'est-ce que le travail ? \\
Qu'est-ce que l'harmonie ? \\
Qu'est-ce que l'inconscient ? \\
Qu'est-ce que l'indifférence ? \\
Qu'est-ce que l'intérêt général ? \\
Qu'est-ce que l'intuition ? \\
Qu'est ce que lire ? \\
Qu'est-ce que lire ? \\
Qu'est-ce que l'ordinaire ? \\
Qu'est-ce que maîtriser une technique ? \\
Qu'est-ce que manquer de culture ? \\
Qu'est-ce que méditer ? \\
Qu'est-ce que mourir ? \\
Qu'est-ce qu'enquêter ? \\
Qu'est-ce qu'enseigner ? \\
Qu'est-ce que parler le même langage ? \\
Qu'est-ce que parler ? \\
Qu'est-ce que penser ? \\
Qu'est-ce que percevoir ? \\
Qu'est-ce que perdre la raison ? \\
Qu'est-ce que perdre sa liberté ? \\
Qu'est-ce que perdre son temps ? \\
Qu'est-ce que prendre conscience ? \\
Qu'est-ce que prendre le pouvoir ? \\
Qu'est-ce que promettre ? \\
Qu'est-ce que prouver ? \\
Qu'est-ce que raisonner ? \\
Qu'est-ce que réfuter une philosophie ? \\
Qu'est-ce que réfuter ? \\
Qu'est-ce que résister ? \\
Qu'est-ce que résoudre une contradiction ? \\
Qu'est-ce que rester soi-même ? \\
Qu'est-ce que réussir sa vie ? \\
Qu'est-ce que s'orienter ? \\
Qu'est-ce que témoigner ? \\
Qu'est-ce que traduire ? \\
Qu'est-ce que travailler ? \\
Qu'est-ce qu'être adulte ? \\
Qu'est-ce qu'être artiste ? \\
Qu'est-ce qu'être asocial ? \\
Qu'est-ce qu'être barbare ? \\
Qu'est-ce qu'être chez soi ? \\
Qu'est-ce qu'être cohérent ? \\
Qu'est-ce qu'être comportementaliste ? \\
Qu'est-ce qu'être cultivé ? \\
Qu'est-ce qu'être dans le vrai ? \\
Qu'est-ce qu'être de son temps ? \\
Qu'est-ce qu'être efficace en politique ? \\
Qu'est-ce qu'être en vie ? \\
Qu'est-ce qu'être esclave ? \\
Qu'est-ce qu'être fidèle à soi-même ? \\
Qu'est-ce qu'être généreux ? \\
Qu'est-ce qu'être idéaliste ? \\
Qu'est-ce qu'être inhumain ? \\
Qu'est-ce qu'être l'auteur de son acte ? \\
Qu'est-ce qu'être libéral ? \\
Qu'est-ce qu'être libre ? \\
Qu'est-ce qu'être maître de soi-même ? \\
Qu'est-ce qu'être malade ? \\
Qu'est-ce qu'être moderne ? \\
Qu'est-ce qu'être nihiliste ? \\
Qu'est-ce qu'être normal ? \\
Qu'est-ce qu'être rationnel ? \\
Qu'est-ce qu'être réaliste ? \\
Qu'est-ce qu'être républicain ? \\
Qu'est-ce qu'être sceptique ? \\
Qu'est-ce qu'être seul ? \\
Qu'est-ce qu'être soi-même ? \\
Qu'est-ce qu'être souverain ? \\
Qu'est-ce qu'être spirituel ? \\
Qu'est-ce qu'être témoin ? \\
Qu'est-ce qu'être un bon citoyen ? \\
Qu'est-ce qu'être un esclave ? \\
Qu'est-ce qu'être un sujet ? \\
Qu'est-ce qu'être vivant ? \\
Qu'est-ce qu'être ? \\
Qu'est-ce que un individu \\
Qu'est-ce que vérifier une théorie ? \\
Qu'est-ce que vérifier ? \\
Qu'est-ce que vivre bien ? \\
Qu'est-ce que vivre ? \\
Qu'est-ce qu'exister pour un individu ? \\
Qu'est-ce qu'exister ? \\
Qu'est-ce qu'expliquer ? \\
Qu'est-ce que « parler le même langage » ? \\
Qu'est-ce que « se rendre maître et possesseur de la nature » ? \\
Qu'est-ce qu'habiter ? \\
Qu'est-ce qui agit ? \\
Qu'est-ce qui apparaît ? \\
Qu'est-ce qui dépend de nous ? \\
Qu'est-ce qui distingue un vivant d'une machine ? \\
Qu'est-ce qui est absurde ? \\
Qu'est-ce qui est actuel ? \\
Qu'est-ce qui est beau ? \\
Qu'est ce qui est concret ? \\
Qu'est-ce qui est concret ? \\
Qu'est ce qui est contre nature ? \\
Qu'est-ce qui est contre nature ? \\
Qu'est ce qui est culturel ? \\
Qu'est-ce qui est culturel ? \\
Qu'est-ce qui est donné ? \\
Qu'est-ce qui est essentiel ? \\
Qu'est-ce qui est extérieur à ma conscience, ? \\
Qu'est-ce qui est historique ? \\
Qu'est-ce qui est hors la loi ? \\
Qu'est-ce qui est hors-la-loi ? \\
Qu'est-ce qui est immoral ? \\
Qu'est-ce qui est impossible ? \\
Qu'est-ce qui est indiscutable ? \\
Qu'est-ce qui est invérifiable ? \\
Qu'est-ce qui est irrationnel ? \\
Qu'est ce qui est irréfutable ? \\
Qu'est-ce qui est irréversible ? \\
Qu'est-ce qui est le plus à craindre, l'ordre ou le désordre ? \\
Qu'est-ce qui est mauvais dans l'égoïsme ? \\
Qu'est-ce qui est mien ? \\
Qu'est-ce qui est moderne ? \\
Qu'est-ce qui est naturel ? \\
Qu'est-ce qui est noble ? \\
Qu'est-ce qui est politique ? \\
Qu'est-ce qui est possible ? \\
Qu'est-ce qui est public ? \\
Qu'est-ce qui est réel ? \\
Qu'est-ce qui est respectable ? \\
Qu'est ce qui est sacré ? \\
Qu'est-ce qui est sauvage ? \\
Qu'est-ce qui est scientifique ? \\
Qu'est-ce qui est spectaculaire ? \\
Qu'est-ce qui est sublime ? \\
Qu'est-ce qui est tragique ? \\
Qu'est-ce qui est vital pour le vivant ? \\
Qu'est-ce qui est vital ? \\
Qu'est ce qui existe ? \\
Qu'est-ce qui existe ? \\
Qu'est-ce qui fait changer les sociétés ? \\
Qu'est-ce qui fait d'une activité un travail ? \\
Qu'est-ce qui fait la force de la loi ? \\
Qu'est-ce qui fait la force des lois ? \\
Qu'est-ce qui fait la justice des lois ? \\
Qu'est-ce qui fait la légitimité d'une autorité politique ? \\
Qu'est-ce qui fait la valeur de la technique ? \\
Qu'est-ce qui fait la valeur de l'œuvre d'art ? \\
Qu'est-ce qui fait la valeur d'une croyance ? \\
Qu'est-ce qui fait la valeur d'une existence ? \\
Qu'est-ce qui fait la valeur d'une œuvre d'art ? \\
Qu'est-ce qui fait le pouvoir des mots ? \\
Qu'est-ce qui fait le propre d'un corps propre ? \\
Qu'est-ce qui fait l'humanité d'un corps ? \\
Qu'est-ce qui fait l'unité d'une science ? \\
Qu'est-ce qui fait l'unité d'un organisme ? \\
Qu'est-ce qui fait l'unité d'un peuple ? \\
Qu'est-ce qui fait l'unité du vivant ? \\
Qu'est-ce qui fait mon identité ? \\
Qu'est-ce qui fait qu'une théorie est vraie ? \\
Qu'est-ce qui fait un peuple ? \\
Qu'est-ce qui fonde la croyance ? \\
Qu'est-ce qui fonde le respect d'autrui ? \\
Qu'est-ce qu'ignore la science ? \\
Qu'est-ce qui importe ? \\
Qu'est-ce qui innocente le bourreau ? \\
Qu'est-ce qui justifie l'hypothèse d'un inconscient ? \\
Qu'est-ce qui justifie une croyance ? \\
Qu'est-ce qu'imaginer ? \\
Qu'est-ce qui menace la liberté ? \\
Qu'est-ce qui mesure la valeur d'un travail ? \\
Qu'est-ce qui n'a pas d'histoire ? \\
Qu'est-ce qui ne disparaît jamais ?/ \\
Qu'est-ce qui ne s'achète pas ? \\
Qu'est-ce qui ne s'échange pas ? \\
Qu'est-ce qui n'est pas démontrable ? \\
Qu'est-ce qui n'est pas politique ? \\
Qu'est-ce qui n'existe pas ? \\
Qu'est-ce qui nous fait danser ? \\
Qu'est-ce qu'interpréter une œuvre d'art ? \\
Qu'est-ce qu'interpréter ? \\
Qu'est-ce qui peut se transformer ? \\
Qu'est-ce qui plaît dans la musique ? \\
Qu'est ce qui rapproche le vivant de la machine ? \\
Qu'est-ce qui rend l'objectivité difficile dans les sciences humaines ? \\
Qu'est-ce qui rend vrai un énoncé ? \\
Qu'est-ce qu'obéir ? \\
Qu'est-ce qu'on attend ? \\
Qu'est-ce qu'on ne peut comprendre ? \\
Qu'est-ce qu'un abus de langage ? \\
Qu'est-ce qu'un abus de pouvoir ? \\
Qu'est-ce qu'un accident ? \\
Qu'est-ce qu'un acte libre ? \\
Qu'est-ce qu'un acte moral ? \\
Qu'est-ce qu'un acte symbolique ? \\
Qu'est-ce qu'un acteur ? \\
Qu'est-ce qu'un acte ? \\
Qu'est-ce qu'un adversaire en politique ? \\
Qu'est-ce qu'un alter ego \\
Qu'est-ce qu'un alter ego ? \\
Qu'est-ce qu'un ami ? \\
Qu'est-ce qu'un animal domestique ? \\
Qu'est-ce qu'un animal ? \\
Qu'est-ce qu'un argument ? \\
Qu'est-ce qu'un art de vivre ? \\
Qu'est-ce qu'un artiste ? \\
Qu'est-ce qu'un art moral ? \\
Qu'est-ce qu'un auteur ? \\
Qu'est-ce qu'un axiome ? \\
Qu'est-ce qu'un bon citoyen ? \\
Qu'est-ce qu'un bon conseil ? \\
Qu'est-ce qu'un bon gouvernement ? \\
Qu'est-ce qu'un bon jugement ? \\
Qu'est-ce qu'un capital culturel ? \\
Qu'est-ce qu'un caractère ? \\
Qu'est-ce qu'un cas de conscience ? \\
Qu'est-ce qu'un châtiment ? \\
Qu'est-ce qu'un chef d'œuvre ? \\
Qu'est-ce qu'un chef-d'œuvre ? \\
Qu'est-ce qu'un chef ? \\
Qu'est-ce qu'un choix éclairé ? \\
Qu'est-ce qu'un citoyen libre ? \\
Qu'est-ce qu'un citoyen ? \\
Qu'est-ce qu'un civilisé ? \\
Qu'est-ce qu'un classique ? \\
Qu'est-ce qu'un code ? \\
Qu'est-ce qu'un concept philosophique ? \\
Qu'est-ce qu'un concept scientifique ? \\
Qu'est-ce qu'un concept ? \\
Qu'est-ce qu'un conflit de générations ? \\
Qu'est-ce qu'un conflit politique ? \\
Qu'est-ce qu'un consommateur ? \\
Qu'est-ce qu'un contenu de conscience ? \\
Qu'est-ce qu'un contrat ? \\
Qu'est-ce qu'un contre-pouvoir ? \\
Qu'est-ce qu'un corps social ? \\
Qu'est-ce qu'un coup d'État ? \\
Qu'est-ce qu'un créateur ? \\
Qu'est-ce qu'un crime contre l'humanité ? \\
Qu'est-ce qu'un crime politique ? \\
Qu'est-ce qu'un crime ? \\
Qu'est-ce qu'un critère de vérité ? \\
Qu'est-ce qu'un déni ? \\
Qu'est-ce qu'un désir satisfait ? \\
Qu'est-ce qu'un détail ? \\
Qu'est-ce qu'un dialogue ? \\
Qu'est-ce qu'un dieu ? \\
Qu'est-ce qu'un Dieu ? \\
Qu'est-ce qu'un dilemme ? \\
Qu'est-ce qu'un document ? \\
Qu'est-ce qu'un dogme ? \\
Qu'est-ce qu'une action intentionnelle ? \\
Qu'est-ce qu'une action juste ? \\
Qu'est-ce qu'une action politique ? \\
Qu'est-ce qu'une action réussie ? \\
Qu'est-ce qu'une alternative ? \\
Qu'est-ce qu'une âme ? \\
Qu'est-ce qu'une analyse ? \\
Qu'est-ce qu'une aporie ? \\
Qu'est-ce qu'une autorité légitime ? \\
Qu'est-ce qu'une avant-garde ? \\
Qu'est-ce qu'une belle démonstration ? \\
Qu'est-ce qu'une belle forme ? \\
Qu'est-ce qu'une belle mort ? \\
Qu'est-ce qu'une bête ? \\
Qu'est-ce qu'une bonne définition ? \\
Qu'est-ce qu'une bonne délibération ? \\
Qu'est-ce qu'une bonne éducation ? \\
Qu'est-ce qu'une bonne loi ? \\
Qu'est-ce qu'une bonne méthode ? \\
Qu'est-ce qu'une bonne traduction ? \\
Qu'est-ce qu'une catastrophe ? \\
Qu'est-ce qu'une catégorie de l'être ? \\
Qu'est-ce qu'une catégorie ? \\
Qu'est-ce qu'une cause ? \\
Qu'est-ce qu'un échange juste ? \\
Qu'est-ce qu'un échange réussi ? \\
Qu'est-ce qu'une chose matérielle ? \\
Qu'est-ce qu'une chose ? \\
Qu'est-ce qu'une civilisation ? \\
Qu'est-ce qu'une collectivité ? \\
Qu'est-ce qu'une comédie ? \\
Qu'est-ce qu'une communauté politique ? \\
Qu'est-ce qu'une communauté ? \\
Qu'est-ce qu'une conception scientifique du monde ? \\
Qu'est-ce qu'une condition suffisante ? \\
Qu'est-ce qu'une conduite irrationnelle ? \\
Qu'est ce qu'une connaissance fiable ? \\
Qu'est-ce qu'une connaissance non scientifique ? \\
Qu'est-ce qu'une connaissance par les faits ? \\
Qu'est-ce qu'une constitution ? \\
Qu'est-ce qu'une contrainte ? \\
Qu'est-ce qu'une convention ? \\
Qu'est-ce qu'une conviction ? \\
Qu'est-ce qu'une crise politique ? \\
Qu'est-ce qu'une crise ? \\
Qu'est-ce qu'une croyance rationnelle ? \\
Qu'est-ce qu'une croyance vraie ? \\
Qu'est-ce qu'une croyance ? \\
Qu'est-ce qu'une culture ? \\
Qu'est-ce qu'une décision politique ? \\
Qu'est-ce qu'une décision rationnelle ? \\
Qu'est-ce qu'une découverte scientifique ? \\
Qu'est-ce qu'une découverte ? \\
Qu'est-ce qu'une définition ? \\
Qu'est-ce qu'une démocratie ? \\
Qu'est-ce qu'une démonstration ? \\
Qu'est-ce qu'une discipline savante ? \\
Qu'est-ce qu'une école philosophique ? \\
Qu'est-ce qu'une éducation réussie ? \\
Qu'est-ce qu'une éducation scientifique ? \\
Qu'est-ce qu'une époque ? \\
Qu'est-ce qu'une erreur ? \\
Qu'est-ce qu'une exception ? \\
Qu'est-ce qu'une existence historique ? \\
Qu'est-ce qu'une expérience cruciale ? \\
Qu'est-ce qu'une expérience de pensée ? \\
Qu'est-ce qu'une expérience religieuse ? \\
Qu'est-ce qu'une expérience scientifique ? \\
Qu'est-ce qu'une expérience ? \\
Qu'est-ce qu'une explication matérialiste ? \\
Qu'est-ce qu'une exposition ? \\
Qu'est-ce qu'une famille ? \\
Qu'est-ce qu'une fausse science ? \\
Qu'est-ce qu'une faute de goût ? \\
Qu'est-ce qu'une fiction ? \\
Qu'est-ce qu'une fonction ? \\
Qu'est-ce qu'une forme ? \\
Qu'est-ce qu'une grande cause ? \\
Qu'est-ce qu'une guerre juste ? \\
Qu'est-ce qu'une histoire vraie ? \\
Qu'est-ce qu'une hypothèse scientifique ? \\
Qu'est-ce qu'une hypothèse ? \\
Qu'est-ce qu'une idée esthétique ? \\
Qu'est-ce qu'une idée incertaine ? \\
Qu'est-ce qu'une idée morale ? \\
Qu'est-ce qu'une idée vraie ? \\
Qu'est-ce qu'une idée ? \\
Qu'est-ce qu'une idéologie ? \\
Qu'est-ce qu'une illusion ? \\
Qu'est-ce qu'une image ? \\
Qu'est-ce qu'une inégalité ? \\
Qu'est-ce qu'une injustice ? \\
Qu'est-ce qu'une institution ? \\
Qu'est-ce qu'une interprétation ? \\
Qu'est-ce qu'une invention technique ? \\
Qu'est-ce qu'une langue artificielle ? \\
Qu'est-ce qu'une langue bien faite ? \\
Qu'est-ce qu'une langue morte ? \\
Qu'est-ce qu'une langue ? \\
Qu'est-ce qu'un élément ? \\
Qu'est-ce qu'une libération ? \\
Qu'est-ce qu'une libre interprétation ? \\
Qu'est-ce qu'une limite ? \\
Qu'est-ce qu'une logique sociale ? \\
Qu'est-ce qu'une loi de la nature ? \\
Qu'est ce qu'une loi de la pensée ? \\
Qu'est ce qu'une loi scientifique ? \\
Qu'est-ce qu'une loi scientifique ? \\
Qu'est-ce qu'une loi ? \\
Qu'est-ce qu'une machine ? \\
Qu'est-ce qu'une maladie ? \\
Qu'est-ce qu'une marchandise ? \\
Qu'est ce qu'une mauvaise idée ? \\
Qu'est-ce qu'une mauvaise interprétation ? \\
Qu'est-ce qu'une méditation métaphysique ? \\
Qu'est-ce qu'une méditation ? \\
Qu'est-ce qu'une mentalité collective ? \\
Qu'est-ce qu'une métaphore ? \\
Qu'est-ce qu'une méthode ? \\
Qu'est-ce qu'une morale de la communication ? \\
Qu'est-ce qu'un empire ? \\
Qu'est-ce qu'une nation ? \\
Qu'est-ce qu'un enfant ? \\
Qu'est-ce qu'un ennemi ? \\
Qu'est-ce qu'une norme sociale ? \\
Qu'est-ce qu'une norme ? \\
Qu'est-ce qu'une nouveauté ? \\
Qu'est-ce qu'une œuvre d'art authentique ? \\
Qu'est-ce qu'une œuvre d'art réaliste ? \\
Qu'est-ce qu'une œuvre d'art ? \\
Qu'est-ce qu'une œuvre ratée ? \\
Qu'est-ce qu'une œuvre ? \\
Qu'est-ce qu'une œuvre « géniale » ? \\
Qu'est-ce qu'une parole libre ? \\
Qu'est-ce qu'une parole vraie ? \\
Qu'est-ce qu'une passion ? \\
Qu'est-ce qu'une patrie ? \\
Qu'est-ce qu'une pensée libre ? \\
Qu'est-ce qu'une période en histoire ? \\
Qu'est-ce qu'une personne morale ? \\
Qu'est-ce qu'une personne ? \\
Qu'est-ce qu‘une philosophie première ? \\
Qu'est-ce qu'une philosophie ? \\
Qu'est-ce qu'une phrase ? \\
Qu'est-ce qu'une politique sociale ? \\
Qu'est-ce qu'une preuve ? \\
Qu'est-ce qu'une promesse ? \\
Qu'est-ce qu'une propriété essentielle ? \\
Qu'est-ce qu'une propriété ? \\
Qu'est-ce qu'une psychologie scientifique ? \\
Qu'est-ce qu'une question dénuée de sens ? \\
Qu'est-ce qu'une question métaphysique ? \\
Qu'est-ce qu'une question ? \\
Qu'est-ce qu'une raison d'agir ? \\
Qu'est-ce qu'une réfutation ? \\
Qu'est-ce qu'une règle de vie ? \\
Qu'est-ce qu'une règle ? \\
Qu'est-ce qu'une relation ? \\
Qu'est ce qu'une religion ? \\
Qu'est-ce qu'une rencontre ? \\
Qu'est-ce qu'une représentation réussie ? \\
Qu'est-ce qu'une république ? \\
Qu'est-ce qu'une révélation ? \\
Qu'est-ce qu'une révolution politique ? \\
Qu'est-ce qu'une révolution scientifique ? \\
Qu'est-ce qu'une révolution ? \\
Qu'est-ce qu'une science exacte ? \\
Qu'est-ce qu'une science expérimentale ? \\
Qu'est-ce qu'une science humaine ? \\
Qu'est-ce qu'une science rigoureuse ? \\
Qu'est-ce qu'un esclave ? \\
Qu'est-ce qu'une situation tragique ? \\
Qu'est-ce qu'une société juste ? \\
Qu'est-ce qu'une société libre ? \\
Qu'est-ce qu'une société mondialisée ? \\
Qu'est-ce qu'une société ouverte ? \\
Qu'est-ce qu'une solution ? \\
Qu'est-ce qu'un esprit faux ? \\
Qu'est-ce qu'un esprit juste ? \\
Qu'est ce qu'un esprit libre ? \\
Qu'est-ce qu'un esprit libre ? \\
Qu'est-ce qu'un esprit profond ? \\
Qu'est-ce qu'une structure ? \\
Qu'est-ce qu'une substance ? \\
Qu'est-ce qu'un état de droit ? \\
Qu'est-ce qu'un État de droit ? \\
Qu'est-ce qu'un État libre ? \\
Qu'est-ce qu'un état mental ? \\
Qu'est-ce qu'une théorie scientifique ? \\
Qu'est-ce qu'une théorie ? \\
Qu'est-ce qu'une tradition ? \\
Qu'est-ce qu'une tragédie historique ? \\
Qu'est-ce qu'une tragédie ? \\
Qu'est-ce qu'un être cultivé ? \\
Qu'est-ce qu'un être vivant ? \\
Qu'est-ce qu'une valeur ? \\
Qu'est-ce qu'un événement fondateur ? \\
Qu'est-ce qu'un événement historique ? \\
Qu'est-ce qu'un événement ? \\
Qu'est-ce qu'une vérité contingente ? \\
Qu'est-ce qu'une vérité historique ? \\
Qu'est-ce qu'une vérité scientifique ? \\
Qu'est-ce qu'une vérité subjective ? \\
Qu'est-ce qu'une vertu ? \\
Qu'est-ce qu'une vie heureuse ? \\
Qu'est-ce qu'une vie humaine ? \\
Qu'est-ce qu'une vie réussie ? \\
Qu'est-ce qu'une ville ? \\
Qu'est-ce qu'une violence symbolique ? \\
Qu'est-ce qu'une vision du monde ? \\
Qu'est-ce qu'une vision scientifique du monde ? \\
Qu'est-ce qu'une volonté libre ? \\
Qu'est-ce qu'une volonté raisonnable ? \\
Qu'est-ce qu'un exemple ? \\
Qu'est-ce qu'un expérimentateur ? \\
Qu'est-ce qu'un expert ? \\
Qu'est-ce qu'une « expérience de pensée » ? \\
Qu'est-ce qu'une « performance » ? \\
Qu'est-ce qu'un fait de culture ? \\
Qu'est-ce qu'un fait de société ? \\
Qu'est-ce qu'un fait divers ? \\
Qu'est-ce qu'un fait historique ? \\
Qu'est-ce qu'un fait moral ? \\
Qu'est ce qu'un fait scientifique ? \\
Qu'est-ce qu'un fait scientifique ? \\
Qu'est-ce qu'un fait social ? \\
Qu'est-ce qu'un fait ? \\
Qu'est-ce qu'un faux problème ? \\
Qu'est-ce qu'un faux sentiment ? \\
Qu'est-ce qu'un faux ? \\
Qu'est-ce qu'un film ? \\
Qu'est-ce qu'un génie ? \\
Qu'est-ce qu'un geste artistique ? \\
Qu'est-ce qu'un geste technique ? \\
Qu'est-ce qu'un gouvernement démocratique ? \\
Qu'est-ce qu'un gouvernement juste ? \\
Qu'est-ce qu'un gouvernement républicain ? \\
Qu'est-ce qu'un gouvernement ? \\
Qu'est-ce qu'un grand homme ou une grande femme ? \\
Qu'est-ce qu'un grand homme ? \\
Qu'est-ce qu'un grand philosophe ? \\
Qu'est-ce qu'un héros ? \\
Qu'est-ce qu'un homme bon ? \\
Qu'est-ce qu'un homme d'action ? \\
Qu'est-ce qu'un homme d'État ? \\
Qu'est-ce qu'un homme d'expérience ? \\
Qu'est-ce qu'un homme juste ? \\
Qu'est-ce qu'un homme libre ? \\
Qu'est-ce qu'un homme méchant ? \\
Qu'est-ce qu'un homme normal ? \\
Qu'est-ce qu'un homme politique ? \\
Qu'est-ce qu'un homme sans éducation ? \\
Qu'est-ce qu'un homme seul ? \\
Qu'est-ce qu'un idéaliste ? \\
Qu'est-ce qu'un idéal moral ? \\
Qu'est-ce qu'un idéal ? \\
Qu'est-ce qu'un individu ? \\
Qu'est-ce qu'un intellectuel ? \\
Qu'est-ce qu'un jeu ? \\
Qu'est-ce qu'un jugement analytique ? \\
Qu'est-ce qu'un jugement de goût ? \\
Qu'est-ce qu'un justicier ? \\
Qu'est-ce qu'un laboratoire ? \\
Qu'est-ce qu'un langage technique ? \\
Qu'est-ce qu'un législateur ? \\
Qu'est-ce qu'un lieu commun ? \\
Qu'est-ce qu'un livre ? \\
Qu'est-ce qu'un maître ? \\
Qu'est-ce qu'un marginal ? \\
Qu'est-ce qu'un mécanisme social ? \\
Qu'est-ce qu'un métaphysicien ? \\
Qu'est-ce qu'un mineur ? \\
Qu'est-ce qu'un miracle ? \\
Qu'est-ce qu'un modèle ? \\
Qu'est-ce qu'un moderne ? \\
Qu'est-ce qu'un monde \\
Qu'est-ce qu'un monde ? \\
Qu'est-ce qu'un monstre ? \\
Qu'est-ce qu'un monument ? \\
Qu'est-ce qu'un mouvement politique \\
Qu'est-ce qu'un musée ? \\
Qu'est-ce qu'un mythe ? \\
Qu'est-ce qu'un nombre ? \\
Qu'est-ce qu'un nom propre ? \\
Qu'est-ce qu'un objet d'art ? \\
Qu'est-ce qu'un objet esthétique ? \\
Qu'est-ce qu'un objet mathématique ? \\
Qu'est-ce qu'un objet métaphysique ? \\
Qu'est-ce qu'un objet technique ? \\
Qu'est-ce qu'un objet ? \\
Qu'est-ce qu'un œuvre d'art ? \\
Qu'est-ce qu'un ordre ? \\
Qu'est-ce qu'un organisme ? \\
Qu'est-ce qu'un original ? \\
Qu'est-ce qu'un outil ? \\
Qu'est ce qu'un paradoxe ? \\
Qu'est-ce qu'un paradoxe ? \\
Qu'est-ce qu'un patrimoine ? \\
Qu'est-ce qu'un pauvre ? \\
Qu'est-ce qu'un paysage ? \\
Qu'est-ce qu'un pédant ? \\
Qu'est-ce qu'un peuple \\
Qu'est-ce qu'un peuple libre ? \\
Qu'est-ce qu'un peuple ? \\
Qu'est-ce qu'un phénomène ? \\
Qu'est-ce qu'un philosophe ? \\
Qu'est-ce qu'un plaisir pur ? \\
Qu'est-ce qu'un point de vue ? \\
Qu'est-ce qu'un portrait ? \\
Qu'est-ce qu'un post-moderne ? \\
Qu'est-ce qu'un précurseur ? \\
Qu'est-ce qu'un préjugé ? \\
Qu'est-ce qu'un primitif ? \\
Qu'est-ce qu'un prince juste ? \\
Qu'est-ce qu'un principe ? \\
Qu'est-ce qu'un problème éthique ? \\
Qu'est-ce qu'un problème métaphysique ? \\
Qu'est-ce qu'un problème philosophique ? \\
Qu'est-ce qu'un problème politique ? \\
Qu'est-ce qu'un problème scientifique ? \\
Qu'est-ce qu'un problème technique ? \\
Qu'est-ce qu'un problème ? \\
Qu'est-ce qu'un produit culturel ? \\
Qu'est-ce qu'un programme politique ? \\
Qu'est-ce qu'un programmer ? \\
Qu'est-ce qu'un programme ? \\
Qu'est-ce qu'un progrès scientifique ? \\
Qu'est-ce qu'un progrès technique ? \\
Qu'est-ce qu'un prophète ? \\
Qu'est-ce qu'un public ? \\
Qu'est-ce qu'un rapport de force ? \\
Qu'est-ce qu'un récit véridique ? \\
Qu'est-ce qu'un récit ? \\
Qu'est-ce qu'un réfutation ? \\
Qu'est-ce qu'un régime politique ? \\
Qu'est-ce qu'un réseau ? \\
Qu'est-ce qu'un rhéteur ? \\
Qu'est-ce qu'un rite ? \\
Qu'est-ce qu'un rival ? \\
Qu'est-ce qu'un sage ? \\
Qu'est-ce qu'un savoir-faire ? \\
Qu'est-ce qu'un sceptique ? \\
Qu'est-ce qu'un sentiment moral ? \\
Qu'est-ce qu'un sentiment vrai ? \\
Qu'est-ce qu'un signe ? \\
Qu'est-ce qu'un sophisme ? \\
Qu'est-ce qu'un sophiste ? \\
Qu'est-ce qu'un souvenir ? \\
Qu'est-ce qu'un spécialiste ? \\
Qu'est-ce qu'un spectateur ? \\
Qu'est-ce qu'un style ? \\
Qu'est-ce qu'un symbole ? \\
Qu'est-ce qu'un symptôme ? \\
Qu'est-ce qu'un système philosophique ? \\
Qu'est-ce qu'un système ? \\
Qu'est-ce qu'un tableau \\
Qu'est-ce qu'un tableau ? \\
Qu'est-ce qu'un tabou ? \\
Qu'est-ce qu'un technicien ? \\
Qu'est-ce qu'un témoin ? \\
Qu'est-ce qu'un temple ? \\
Qu'est-ce qu'un texte ? \\
Qu'est-ce qu'un tout ? \\
Qu'est-ce qu'un traître ? \\
Qu'est-ce qu'un travail bien fait ? \\
Qu'est-ce qu'un trouble social ? \\
Qu'est-ce qu'un tyran ? \\
Qu'est-ce qu'un vice ? \\
Qu'est-ce qu'un visage ? \\
Qu'est-ce qu'un vrai changement ? \\
Qu'est-ce qu'un « champ artistique » ? \\
Qu'est-ce qu'un « être dégénéré » ? \\
Question et problème \\
Qu'est qu'une image ? \\
Qu'est qu'un régime politique ? \\
Que suis-je ? \\
Que suppose le mouvement ? \\
Que valent les excuses ? \\
Que valent les idées générales ? \\
Que valent les mots ? \\
Que valent les préjugés ? \\
Que valent les théories ? \\
Que vaut en morale la justification par l'utilité ? \\
Que vaut la décision de la majorité ? \\
Que vaut la définition de l'homme comme animal doué de raison ? \\
Que vaut la distinction entre nature et culture ? \\
Que vaut l'excuse : « c'est plus fort que moi » ? \\
Que vaut l'excuse : « C'est plus fort que moi » ? \\
Que vaut l'excuse : « Je ne l'ai pas fait exprès» ? \\
Que vaut l'incertain ? \\
Que vaut une preuve contre un préjugé ? \\
Que veut dire avoir raison ? \\
Que veut dire introduire à la métaphysique ? \\
Que veut dire l'expression « aller au fond des choses » ? \\
Que veut dire « essentiel » ? \\
Que veut dire « je t'aime » ? \\
Que veut dire « réel » ? \\
Que veut dire « respecter la nature » ? \\
Que veut dire : « être cultivé » ? \\
Que veut dire : « je t'aime » ? \\
Que veut dire : « le temps passe » ? \\
Que veut dire : « respecter la nature » ? \\
Que voit-on dans une image ? \\
Que voit-on dans un miroir \\
Que voit-on dans un miroir ? \\
Que voit-on dans un tableau ? \\
Que voyons-nous ? \\
Qu'expriment les mythes ? \\
Qu'exprime une œuvre d'art ? \\
Qui accroît son savoir accroît sa douleur \\
Qui agit ? \\
Qui a le droit de juger ? \\
Qui a une histoire ? \\
Qui a une parole politique ? \\
Qui commande ? \\
Qui connaît le mieux mon corps ? \\
Qui croire ? \\
Qui doit faire les lois ? \\
Qui écrit l'histoire ? \\
Qui est autorisé à me dire « tu dois » ? \\
Qui est citoyen ? \\
Qui est compétent en matière politique ? \\
Qui est digne du bonheur ? \\
Qui est immoral ? \\
Qui est l'autre ? \\
Qui est le maître ? \\
Qui est l'homme des sciences humaines ? \\
Qui est libre ? \\
Qui est métaphysicien ? \\
Qui est mon prochain ? \\
Qui est mon semblable ? \\
Qui est riche ? \\
Qui est sage ? \\
Qui est souverain ? \\
Qui fait la loi ? \\
Qui fait l'histoire ? \\
Qui gouverne ? \\
Qui mérite d'être aimé ? \\
Qui meurt ? \\
Qui nous dicte nos devoirs ? \\
Qui parle quand je dis « je » ? \\
Qui parle ? \\
Qui pense ? \\
Qui peut avoir des droits ? \\
Qui peut me dire « tu ne dois pas » ? \\
Qui peut parler ? \\
Qui suis-je et qui es-tu ? \\
Qui suis-je ? \\
Qui travaille ? \\
Qui veut la fin veut les moyens \\
Qu'oppose-t-on à la vérité ? \\
Qu'y a-t-il à comprendre dans une œuvre d'art ? \\
Qu'y a-t-il à comprendre en histoire ? \\
Qu'y a-t-il à l'origine de toutes choses ? \\
Qu'y a-t-il au-delà du réel ? \\
Qu'y a-t-il au fondement de l'objectivité ? \\
Qu'y a-t-il de sérieux dans le jeu ? \\
Qu'y a-t-il d'universel dans la culture ? \\
Qu'y a-t-il ? \\
Raconter sa vie \\
Raconter son histoire \\
Raison et dialogue \\
Raison et folie \\
Raison et fondement \\
Raison et langage \\
Raison et politique \\
Raison et révélation \\
Raison et tradition \\
Raisonnable et rationnel \\
Raisonnement et expérimentation \\
Raisonner \\
Raisonner et calculer \\
Raisonner par l'absurde \\
Rapports de force, rapport de pouvoir \\
Rassembler les hommes, est-ce les unir ? \\
Rationnel et raisonnable \\
Réalisme et idéalisme \\
Réalité et apparence \\
Réalité et idéal \\
Réalité et perception \\
Réalité et représentation \\
Rebuts et objets quelconques : une matière pour l'art ? \\
Recevoir \\
Récit et histoire \\
Récit et mémoire \\
Reconnaissance et inégalité \\
Reconnaissons-nous le bien comme nous reconnaissons le vrai ? \\
Recourir au langage, est-ce renoncer à la violence ? \\
Refaire sa vie \\
Réforme et révolution \\
Refuser et réfuter \\
Réfutation et confirmation \\
Réfuter \\
Réfuter une théorie \\
Regarder \\
Regarder un tableau \\
Règle et commandement \\
Règle morale et norme juridique \\
Règles sociales et loi morale \\
Regrets et remords \\
Religion et démocratie \\
Religion et liberté \\
Religion et moralité \\
Religion et politique \\
Religion et violence \\
Religion naturelle et religion révélée \\
Religions et démocratie \\
Rendre justice \\
Rendre la justice \\
Rendre raison \\
Rendre visible l'invisible \\
Renoncer au passé \\
Rentrer en soi-même \\
Répondre \\
Répondre de soi \\
Représentation et illusion \\
Représenter \\
Reproduire, copier, imiter \\
Réprouver \\
République et démocratie \\
Résistance et obéissance \\
Résistance et soumission \\
Résister \\
Résister à l'oppression \\
Résister peut-il être un droit ? \\
Respecter la nature, est-ce renoncer à l'exploiter ? \\
Respect et tolérance \\
Rester soi-même \\
Réussir sa vie \\
Revenir à la nature \\
Rêver \\
Revient-il à l'État d'assurer le bonheur des citoyens ? \\
Revient-il à l'État d'assurer votre bonheur ? \\
Révolte et révolution \\
Rêvons-nous ? \\
Rhétorique et vérité \\
Richesse et pauvreté \\
Rien \\
Rien de nouveau sous le soleil \\
Rien n'est sans raison \\
Rire \\
Rire et pleurer \\
Rites et cérémonies \\
Rituels et cérémonies \\
Roman et vérité \\
Rythmes sociaux, rythmes naturels \\
Sait-on ce que l'on veut ? \\
Sait-on ce qu'on fait ? \\
Sait-on ce qu'on veut ? \\
Sait-on nécessairement ce que l'on désire ? \\
Sait-on toujours ce que l'on fait ? \\
Sait-on toujours ce que l'on veut ? \\
Sait-on toujours ce qu'on veut ? \\
S'aliéner \\
S'amuser \\
Sans foi ni loi \\
S'approprier une œuvre d'art \\
Sauver les apparences \\
Sauver les phénomènes \\
Savoir ce qu'on dit \\
Savoir démontrer \\
Savoir de quoi on parle \\
Savoir est-ce cesser de croire ? \\
Savoir, est-ce pouvoir ? \\
Savoir est-ce se libérer ? \\
Savoir et croire \\
Savoir et démontrer \\
Savoir et liberté \\
Savoir et objectivité dans les sciences \\
Savoir et pouvoir \\
Savoir et rectification \\
Savoir être heureux \\
Savoir et savoir faire \\
Savoir et savoir-faire \\
Savoir et vérifier \\
Savoir faire \\
Savoir pour prévoir \\
Savoir, pouvoir \\
Savoir renoncer \\
Savoir s'arrêter \\
Savoir se décider \\
Savoir tout \\
Savoir vivre \\
Savons-nous ce que nous disons ? \\
Science du vivant et finalisme \\
Science du vivant, science de l'inerte \\
Science et abstraction \\
Science et certitude \\
Science et complexité \\
Science et croyance \\
Science et démocratie \\
Science et domination sociale \\
Science et expérience \\
Science et histoire \\
Science et hypothèse \\
Science et idéologie \\
Science et imagination \\
Science et invention \\
Science et libération \\
Science et magie \\
Science et métaphysique \\
Science et méthode \\
Science et mythe \\
Science et objectivité \\
Science et opinion \\
Science et persuasion \\
Science et philosophie \\
Science et réalité \\
Science et religion \\
Science et sagesse \\
Science et société \\
Science et technique \\
Science et technologie \\
Science pure et science appliquée \\
Sciences de la nature et sciences de l'esprit \\
Sciences de la nature et sciences humaines \\
Sciences empiriques et critères du vrai \\
Sciences et philosophie \\
Sciences humaines et déterminisme \\
Sciences humaines et herméneutique \\
Sciences humaines et idéologie \\
Sciences humaines et liberté sont-elles compatibles ? \\
Sciences humaines et littérature \\
Sciences humaines et naturalisme \\
Sciences humaines et nature humaine \\
Sciences humaines et objectivité \\
Sciences humaines et philosophie \\
Sciences humaines, sciences de l'homme \\
Sciences sociales et humanités \\
Se connaître soi-même \\
Se conserver \\
Se convertir \\
Se cultiver \\
Se cultiver, est-ce s'affranchir de son appartenance culturelle ? \\
Sécurité et liberté \\
Se décider \\
Se défendre \\
Se détacher des sens \\
Se faire comprendre \\
Se faire justice \\
Se mentir à soi-même \\
Se mentir à soi-même : est-ce possible ? \\
Se mettre à la place d'autrui \\
S'engager \\
S'ennuyer \\
Se nourrir \\
Sensation et perception \\
Sens et existence \\
Sens et fait \\
Sens et limites de la notion de capital culturel \\
Sens et sensibilité \\
Sens et sensible \\
Sens et signification \\
Sens et structure \\
Sens et vérité \\
Sensible et intelligible \\
Sens propre et sens figuré \\
Sentir \\
Sentir et juger \\
Sentir et penser \\
Se parler et s'entendre \\
Se passer de philosophie \\
Se prendre au sérieux \\
Se raconter des histoires \\
Serait-il immoral d'autoriser le commerce des organes humains ? \\
Se retirer dans la pensée ? \\
Se retirer du monde \\
Se révolter \\
Serions-nous heureux dans un ordre politique parfait ? \\
Serions-nous plus libres sans État ? \\
Servir \\
Servir, est-ce nécessairement renoncer à sa liberté ? \\
Servir l'État \\
Se savoir mortel \\
Se suffire à soi-même \\
Se taire \\
Seul \\
Seul le présent existe-t-il ? \\
Se voiler la face \\
Sexe et genre \\
S'exercer \\
S'exprimer \\
Sexualité et féminité \\
Sexualité et nature \\
Si\ldots{} alors \\
Si Dieu n'existe pas, tout est-il permis ? \\
Si Dieu n'existe pas, tout est-il possible ? \\
Signe et symbole \\
Signes, traces et indices \\
Signification et expression \\
Signification et vérité \\
Si l'esprit n'est pas une table rase, qu'est-il ? \\
Si l'État n'existait pas, faudrait-il l'inventer ? \\
Sincérité et vérité \\
S'indigner \\
S'indigner, est-ce un devoir ? \\
Si nous étions moraux, le droit serait-il inutile ? \\
S'intéresser à l'art \\
Si tout est historique, tout est-il relatif ? \\
Si tu veux, tu peux \\
Société et biologie \\
Société et communauté \\
Société et organisme \\
Société et religion \\
Société humaines, sociétés animales \\
Socrate \\
Soi \\
Soigner \\
Solitude et isolement \\
Sommes-nous capables d'agir de manière désintéressée ? \\
Sommes-nous conscients de nos mobiles ? \\
Sommes-nous dans le temps comme dans l'espace ? \\
Sommes-nous des êtres métaphysiques ? \\
Sommes-nous des sujets ? \\
Sommes-nous déterminés par notre culture ? \\
Sommes-nous dominés par la technique ? \\
Sommes-nous faits pour la vérité ? \\
Sommes-nous faits pour le bonheur ? \\
Sommes-nous gouvernés par nos passions ? \\
Sommes-nous jamais certains d'avoir choisi librement ? \\
Sommes-nous les jouets de l'histoire ? \\
Sommes-nous les jouets de nos pulsions ? \\
Sommes-nous libres de nos croyances ? \\
Sommes-nous libres de nos pensées ? \\
Sommes-nous libres de nos préférences morales ? \\
Sommes-nous libres face à l'évidence ? \\
Sommes-nous maîtres de nos désirs ? \\
Sommes-nous maîtres de nos paroles ? \\
Sommes-nous perfectibles ? \\
Sommes-nous portés au bien ? \\
Sommes-nous prisonniers de nos désirs ? \\
Sommes-nous prisonniers de notre histoire ? \\
Sommes-nous prisonniers du temps ? \\
Sommes-nous responsables de ce dont nous n'avons pas conscience ? \\
Sommes-nous responsables de ce que nous sommes ? \\
Sommes-nous responsables de nos désirs ? \\
Sommes-nous responsables de nos erreurs ? \\
Sommes-nous responsables de nos opinions ? \\
Sommes-nous responsables de nos passions ? \\
Sommes-nous soumis au temps ? \\
Sommes-nous sujets de nos désirs ? \\
Sommes-nous toujours conscients des causes de nos désirs ?` \\
Sommes-nous toujours dépendants d'autrui ? \\
Sommes-nous tous contemporains ? \\
Sophismes et paradoxes \\
S'orienter \\
Sortir de soi \\
Soumission et servitude \\
Soutenir une thèse \\
Soyez naturel ! \\
Sport et politique \\
Structure et événement \\
Subir \\
Substance et accident \\
Substance et sujet \\
Suffit-il d'avoir raison ? \\
Suffit-il de bien juger pour bien faire ? \\
Suffit-il de faire son devoir pour être vertueux ? \\
Suffit-il de faire son devoir ? \\
Suffit-il de n'avoir rien fait pour être innocent ? \\
Suffit-il d'être informé pour comprendre ? \\
Suffit-il d'être juste ? \\
Suffit-il d'être vertueux pour être heureux ? \\
Suffit-il de voir le meilleur pour le suivre ? \\
Suffit-il de vouloir pour bien faire ? \\
Suffit-il, pour croire, de le vouloir ? \\
Suffit-il pour être juste d'obéir aux lois et aux coutumes de son pays ? \\
Suffit-il que nos intentions soient bonnes pour que nos actions le soient aussi ? \\
Suis-ce que j'ai conscience d'être ? \\
Suis-je aussi ce que j'aurais pu être ? \\
Suis-je ce que j'ai conscience d'être ? \\
Suis-je ce que je fais ? \\
Suis-je dans le temps comme je suis dans l'espace ? \\
Suis-je étranger à moi-même ? \\
Suis-je l'auteur de ce que je dis ? \\
Suis-je le même en des temps différents ? \\
Suis-je le mieux placé pour me connaître ? \\
Suis-je libre ? \\
Suis-je maître de ma conscience ? \\
Suis-je maître de mes pensées ? \\
Suis-je ma mémoire ? \\
Suis-je mon corps ? \\
Suis-je mon passé ? \\
Suis-je propriétaire de mon corps ? \\
Suis-je responsable de ce dont je n'ai pas conscience ? \\
Suis-je responsable de ce que je suis ? \\
Suis-je seul au monde ? \\
Suis-je toujours autre que moi-même ? \\
Suivre la coutume \\
Suivre son intuition \\
Suivre une règle \\
Sujet et citoyen \\
Sujet et prédicat \\
Sujet et substance \\
Superstition et religion \\
Surface et profondeur \\
Sur quoi fonder la justice ? \\
Sur quoi fonder la légitimité de la loi ? \\
Sur quoi fonder la propriété ? \\
Sur quoi fonder la société ? \\
Sur quoi fonder l'autorité politique ? \\
Sur quoi fonder l'autorité ? \\
Sur quoi fonder le devoir ? \\
Sur quoi fonder le droit de punir ? \\
Sur quoi le langage doit-il se régler ? \\
Sur quoi l'historien travaille-t-il ? \\
Sur quoi repose l'accord des esprits ? \\
Sur quoi repose la croyance au progrès ? \\
Sur quoi reposent nos certitudes ? \\
Sur quoi se fonde la connaissance scientifique ? \\
Surveillance et discipline \\
Surveiller son comportement \\
Survivre \\
Suspendre son assentiment \\
Suspendre son jugement \\
Syllogisme et démonstration \\
Sympathie et respect \\
Système et structure \\
Talent et génie \\
Tantôt je pense, tantôt je suis \\
Tautologie et contradiction \\
Technique et apprentissage \\
Technique et idéologie \\
Technique et intérêt \\
Technique et nature \\
Technique et pratiques scientifiques \\
Technique et progrès \\
Technique et responsabilité \\
Technique et savoir-faire \\
Technique et violence \\
Témoigner \\
Temps et commencement \\
Temps et conscience \\
Temps et création \\
Temps et éternité \\
Temps et histoire \\
Temps et irréversibilité \\
Temps et liberté \\
Temps et mémoire \\
Temps et réalité \\
Temps et vérité \\
Tenir parole \\
Tenir pour vrai \\
Tenir sa parole \\
Thème et variations \\
Théorie et expérience \\
Théorie et modèle \\
Théorie et modélisation \\
Théorie et pratique \\
Toucher \\
Toucher, sentir, goûter \\
Toujours plus vite ? \\
Tous les conflits peuvent-ils être résolus par le dialogue ? \\
Tous les désirs sont-ils naturels ? \\
Tous les droits sont-ils formels ? \\
Tous les hommes désirent-ils connaître ? \\
Tous les hommes désirent-ils être heureux ? \\
Tous les hommes désirent-ils naturellement être heureux ? \\
Tous les hommes désirent-ils naturellement savoir ? \\
Tous les hommes sont-ils égaux ? \\
Tous les paradis sont-ils perdus ? \\
Tous les plaisirs se valent-ils ? \\
Tous les rapports humains sont-ils des échanges ? \\
Tout art est-il poésie ? \\
Tout a-t-il une cause ? \\
Tout a-t-il une raison d'être ? \\
Tout a-t-il un prix ? \\
Tout a-t-il un sens ? \\
Tout ce qui est excessif est-il insignifiant ? \\
Tout ce qui est naturel est-il normal ? \\
Tout ce qui est rationnel est-il raisonnable ? \\
Tout ce qui est vrai doit-il être prouvé ? \\
Tout ce qui existe a-t-il un prix ? \\
Tout comprendre, est-ce tout pardonner ? \\
Tout définir, tout démontrer \\
Tout démontrer \\
Tout désir est-il désir de posséder ? \\
Tout désir est-il égoïste ? \\
Tout désir est-il manque ? \\
Tout désir est-il une souffrance ? \\
Tout devoir est-il l'envers d'un droit ? \\
Tout dire \\
Tout droit est-il un pouvoir ? \\
Toute action politique est-elle collective ? \\
Toute chose a-t-elle une essence ? \\
Toute communauté est-elle politique ? \\
Toute compréhension implique-t-elle une interprétation ? \\
Toute connaissance autre que scientifique doit-elle être considérée comme une illusion ? \\
Toute connaissance commence-t-elle avec l'expérience ? \\
Toute connaissance consiste-t-elle en un savoir-faire ? \\
Toute connaissance est-elle historique ? \\
Toute connaissance est-elle hypothétique ? \\
Toute connaissance s'enracine-t-elle dans la perception ? \\
Toute conscience est-elle conscience de soi ? \\
Toute conscience est-elle subjective ? \\
Toute conscience n'est-elle pas implicitement morale ? \\
Toute description est-elle une interprétation ? \\
Toute existence est-elle indémontrable ? \\
Toute expérience appelle-t-elle une interprétation ? \\
Toute expression est-elle métaphorique ? \\
Toute faute est-elle une erreur ? \\
Toute hiérarchie est-elle inégalitaire ? \\
Toute inégalité est-elle injuste ? \\
Toute interprétation est-elle contestable ? \\
Toute interprétation est-elle subjective ? \\
Toute métaphysique implique-t-elle une transcendance ? \\
Toute morale implique-t-elle l'effort ? \\
Toute morale s'oppose-t-elle aux désirs ? \\
Tout énoncé est-il nécessairement vrai ou faux ? \\
Toute notre connaissance dérive-t-elle de l'expérience ? \\
Toute origine est-elle mythique ? \\
Toute passion fait-elle souffrir ? \\
Toute pensée revêt-elle nécessairement une forme linguistique ? \\
Toute peur est-elle irrationnelle ? \\
Toute philosophie constitue-t-elle une doctrine ? \\
Toute philosophie est-elle systématique ? \\
Toute philosophie implique-t-elle une politique ? \\
Toute relation humaine est-elle un échange ? \\
Toute science est-elle naturelle ? \\
Toutes les choses sont-elles singulières ? \\
Toutes les convictions sont-elles respectables ? \\
Toutes les croyances se valent-elles ? \\
Toutes les fautes se valent-elles ? \\
Toutes les inégalités ont-elles une importance politique ? \\
Toutes les inégalités sont-elles des injustices ? \\
Toutes les interprétations se valent-elles ? \\
Toutes les opinions se valent-elles ? \\
Toutes les opinions sont-elles bonnes à dire ? \\
Toutes les vérités  scientifiques sont-elles révisables ? \\
Toute société a-t-elle besoin d'une religion ? \\
Tout est corps \\
Tout est-il affaire de point de vue ? \\
Tout est-il à vendre ? \\
Tout est-il connaissable ? \\
Tout est-il faux dans la fiction ? \\
Tout est-il historique ? \\
Tout est-il matière ? \\
Tout est-il mesurable ? \\
Tout est-il nécessaire ? \\
Tout est-il politique ? \\
Tout est-il quantifiable ? \\
Tout est-il relatif ? \\
Tout est permis \\
Tout est relatif \\
Tout est vanité \\
Tout être est-il dans l'espace ? \\
Toute vérité est-elle bonne à dire ? \\
Toute vérité est-elle démontrable ? \\
Toute vérité est-elle vérifiable ? \\
Toute vie est-elle intrinsèquement respectable ? \\
Toute violence est-elle contre nature ? \\
Tout futur est-il contingent ? \\
Tout ordre est-il une violence déguisée ? \\
Tout ou rien \\
Tout peut-il être objet d'échange ? \\
Tout peut-il être objet de jugement esthétique ? \\
Tout peut-il être objet de science ? \\
Tout peut-il n'être qu'apparence ? \\
Tout peut-il s'acheter ? \\
Tout peut-il se démontrer ? \\
Tout peut-il se vendre ? \\
Tout peut-il s'expliquer ? \\
Tout pouvoir corrompt-il ? \\
Tout pouvoir est-il oppresseur ? \\
Tout pouvoir est-il politique ? \\
Tout pouvoir n'est-il pas abusif ? \\
Tout principe est-il un fondement ? \\
Tout savoir \\
Tout savoir a-t-il une justification ? \\
Tout savoir est-il fondé sur un savoir premier ? \\
Tout savoir est-il pouvoir ? \\
Tout savoir est-il transmissible ? \\
Tout savoir est-il un pouvoir ? \\
Tout savoir peut-il se transmettre ? \\
Tout s'en va-t-il avec le temps ? \\
Tout se prête-il à la mesure ? \\
Tout travail est-il forcé ? \\
Tout travail est-il social ? \\
Tout travail mérite salaire \\
Tout vouloir \\
Tradition et innovation \\
Tradition et liberté \\
Tradition et nouveauté \\
Tradition et raison \\
Tradition et transmission \\
Tradition et vérité \\
Traduction, Trahison \\
Traduire \\
Traduire, est-ce trahir ? \\
Traduire et interpréter \\
Tragédie et comédie \\
Trahir \\
Traiter autrui comme une chose \\
Traiter des faits humains comme des choses, est-ce considérer l'homme comme une chose ? \\
Traiter les faits humains comme des choses, est-ce réduire les hommes à des choses ? \\
Transcendance et altérité \\
Transcendance et immanence \\
Transmettre \\
Travail, besoin, désir \\
Travail et aliénation \\
Travail et besoin \\
Travail et bonheur \\
Travail et capital \\
Travail et liberté \\
Travail et loisir \\
Travail et nécessité \\
Travail et œuvre \\
Travail et subjectivité \\
Travailler, est-ce faire œuvre ? \\
Travailler et œuvrer \\
Travailler par plaisir, est-ce encore travailler ? \\
Travaille-t-on pour soi-même ? \\
Travail manuel et travail intellectuel \\
Travail manuel, travail intellectuel \\
Tricher \\
Trouver sa voie \\
Tu aimeras ton prochain comme toi-même \\
Tuer et laisser mourir \\
Tuer le temps \\
Un acte désintéressé est-il possible ? \\
Un acte gratuit est-il possible ? \\
Un acte libre est-il un acte imprévisible ? \\
Un acte peut-il être inhumain ? \\
Un artiste doit-il être original ? \\
Un art peut-il être populaire ? \\
Un art sans sublimation est-il possible ? \\
Un bien peut-il être commun ? \\
Un bien peut-il sortir d'un mal ? \\
Un choix peut-il être rationnel ? \\
Un contrat peut-il être injuste ? \\
Un contrat peut-il être social ? \\
Un désir peut-il être coupable ? \\
Un désir peut-il être inconscient ? \\
Un Dieu unique ? \\
Une action peut-elle être désintéressée ? \\
Une action peut-elle être machinale ? \\
Une action vertueuse se reconnaît-elle à sa difficulté ? \\
Une activité inutile est-elle sans valeur ? \\
Une cause peut-elle être libre ? \\
Une communauté politique n'est-elle qu'une communauté d'intérêt ? \\
Une connaissance peut-elle ne pas être relative ? \\
Une connaissance scientifique du vivant est-elle possible ? \\
Une croyance infondée est-elle illégitime ? \\
Une croyance peut-elle être libre ? \\
Une croyance peut-elle être rationnelle ? \\
Une culture de masse est-elle une culture ? \\
Une culture peut-elle être porteuse de valeurs universelles ? \\
Une décision politique peut-elle être juste ? \\
Une destruction peut-elle être créatrice ? \\
Une éducation esthétique est-elle possible ? \\
Une éducation morale est-elle possible ? \\
Une éthique sceptique est-elle possible ? \\
Une existence se démontre-t-elle ? \\
Une expérience peut-elle être fictive ? \\
Une explication peut-elle être réductrice ? \\
Une fiction peut-elle être vraie ? \\
Une foi rationnelle \\
Une guerre peut-elle être juste ? \\
Une idée peut-elle être fausse ? \\
Une idée peut-elle être générale ? \\
Une imitation peut-elle être parfaite ? \\
Une intention peut-elle être coupable ? \\
Une interprétation est-elle nécessairement subjective ? \\
Une interprétation peut-elle échapper à l'arbitraire ? \\
Une interprétation peut-elle être définitive ? \\
Une interprétation peut-elle être objective ? \\
Une interprétation peut-elle prétendre à la vérité ? \\
Une langue n'est-elle faite que de mots ? \\
Une ligne de conduite peut-elle tenir lieu de morale ? \\
Une logique non-formelle est-elle possible ? \\
Une loi n'est-elle qu'une règle ? \\
Une loi peut-elle être injuste ? \\
Une machine peut-elle avoir une mémoire ? \\
Une machine peut-elle penser ? \\
Une machine pourrait-elle penser ? \\
Une métaphysique athée est-elle possible ? \\
Une métaphysique peut-elle être sceptique ? \\
Une morale du plaisir est-elle concevable ? \\
Une morale peut-elle être dépassée ? \\
Une morale peut-elle être provisoire ? \\
Une morale peut-elle prétendre à l'universalité ? \\
Une morale sans Dieu \\
Une morale sans obligation est-elle possible ? \\
Une morale sceptique est-elle possible ? \\
Une œuvre d'art a-t-elle toujours un sens ? \\
Une œuvre d'art doit-elle avoir un sens ? \\
Une œuvre d'art doit-elle nécessairement être belle ? \\
Une œuvre d'art doit-elle plaire ? \\
Une œuvre d'art est-elle une marchandise ? \\
Une œuvre d'art peut-elle être immorale ? \\
Une œuvre d'art peut-elle être laide ? \\
Une œuvre d'art s'explique-t-elle à partir de ses influences ? \\
Une œuvre doit-elle nécessairement être belle ? \\
Une œuvre est-elle nécessairement singulière ? \\
Une œuvre est-elle toujours de son temps ? \\
Une pensée contradictoire est-elle dénuée de valeur ? \\
Une perception peut-elle être illusoire ? \\
Une philosophie de l'amour est-elle possible ? \\
Une philosophie peut-elle être réactionnaire ? \\
Une politique peut-elle se réclamer de la vie ? \\
Une psychologie peut-elle être matérialiste ? \\
Une religion civile est-elle possible ? \\
Une religion peut-elle être fausse ? \\
Une religion peut-elle être rationnelle ? \\
Une religion peut-elle se passer de pratiques ? \\
Une religion rationnelle est-elle possible ? \\
Une science de la conscience est-elle possible ? \\
Une science de la culture est-elle possible ? \\
Une science de la morale est-elle possible ? \\
Une science de l'éducation est-elle possible ? \\
Une science de l'esprit est-elle possible ? \\
Une science des symboles est-elle possible ? \\
Une sensation peut-elle être fausse ? \\
Une société d'athées est-elle possible ? \\
Une société juste est-ce une société sans conflit ? \\
Une société juste est-elle une société sans conflits ? \\
Une société n'est-elle qu'un ensemble d'individus ? \\
Une société sans conflit est-elle possible ? \\
Une société sans État est-elle possible ? \\
Une société sans État est-elle une société sans politique ? \\
Une société sans religion est-elle possible ? \\
Une société sans travail est-elle souhaitable ? \\
Un État mondial ? \\
Un État peut-il être trop étendu ? \\
Une théorie peut-elle être vérifiée ? \\
Une théorie scientifique peut-elle devenir fausse ? \\
Une théorie scientifique peut-elle être ramenée à des propositions empiriques élémentaires ? \\
Une théorie scientifique peut-elle être vraie ? \\
Un être vivant peut-il être comparé à une œuvre d'art ? \\
Un événement historique est-il toujours imprévisible ? \\
Une vérité peut-elle être indicible ? \\
Une vérité peut-elle être provisoire ? \\
Une vie heureuse est-elle une vie de plaisirs ? \\
Une vie libre exclut-elle le travail ? \\
Une volonté peut-elle être générale ? \\
Un fait existe-t-il sans interprétation ? \\
Un gouvernement de savants est-il souhaitable ? \\
Un homme n'est-il que la somme de ses actes ? \\
Universalité et nécessité dans les sciences \\
Univocité et équivocité \\
Un jeu peut-il être sérieux ? \\
Un jugement de goût est-il culturel ? \\
Un langage universel est-il concevable ? \\
Un mensonge peut-il avoir une valeur morale ? \\
Un moment d'éternité \\
Un monde meilleur \\
Un monde sans beauté \\
Un monde sans nature est-il pensable ? \\
Un objet technique peut-il être beau ? \\
Un peuple est-il responsable de son histoire ? \\
Un peuple est-il un rassemblement d'individus ? \\
Un peuple se définit-il par son histoire ? \\
Un philosophe a-t-il des devoirs envers la société ? \\
Un pouvoir a-t-il besoin d'être légitime ? \\
Un problème moral peut-il recevoir une solution certaine ? \\
Un problème scientifique peut-il être insoluble ? \\
Un savoir peut-il être inconscient ? \\
Un sceptique peut-il être logicien ? \\
Un seul peut-il avoir raison contre tous ? \\
Un tableau peut-il être une dénonciation ? \\
Un vice, est-ce un manque ? \\
User de violence peut-il être moral ? \\
Utilité et beauté \\
Utopie et tradition \\
Vaincre la mort \\
Valeur artistique, valeur esthétique \\
Vanité des vanités \\
Vaut-il mieux oublier ou pardonner ? \\
Vaut-il mieux subir l'injustice que la commettre ? \\
Vaut-il mieux subir ou commettre l'injustice ? \\
Vendre son corps \\
Vérité et apparence \\
Vérité et certitude \\
Vérité et cohérence \\
Vérité et efficacité \\
Vérité et exactitude \\
Vérité et fiction \\
Vérité et histoire \\
Vérité et illusion \\
Vérité et liberté \\
Vérité et poésie \\
Vérité et réalité \\
Vérité et religion \\
Vérité et sensibilité \\
Vérité et signification \\
Vérité et sincérité \\
Vérité et subjectivité \\
Vérité et vérification \\
Vérité et vraisemblance \\
Vérités de fait et vérités de raison \\
Vérités mathématiques, vérités philosophiques \\
Vérité théorique, vérité pratique \\
Vertu et habitude \\
Vertu et perfection \\
Vice et délice \\
Vices privés, vertus publiques \\
Vie active, vie contemplative \\
Vie et existence \\
Vie et volonté \\
Vieillir \\
Vie politique et vie contemplative \\
Vie privée et vie publique \\
Vie publique et vie privée \\
Violence et discours \\
Violence et force \\
Violence et histoire \\
Violence et politique \\
Violence et pouvoir \\
Vitalisme et mécanique \\
Vit-on au présent ? \\
Vivons-nous tous dans le même monde ? \\
Vivrait-on mieux sans désirs ? \\
Vivre \\
Vivre au présent \\
Vivre caché \\
Vivre comme si nous ne devions pas mourir \\
Vivre en société, est-ce seulement vivre ensemble ? \\
Vivre, est-ce interpréter ? \\
Vivre, est-ce lutter contre la mort ? \\
Vivre, est-ce lutter pour survivre ? \\
Vivre, est-ce résister à la mort ? \\
Vivre, est-ce un droit ? \\
Vivre et bien vivre \\
Vivre et exister \\
Vivre libre \\
Vivre sans morale \\
Vivre sans religion, est-ce vivre sans espoir ? \\
Vivre sa vie \\
Vivre selon la nature \\
Vivre sous la conduite de la raison \\
Vivre vertueusement \\
Voir \\
Voir et entendre \\
Voir et savoir \\
Voir et toucher \\
Voir le meilleur et faire le pire \\
Voir le meilleur, faire le pire \\
Voir, observer, penser \\
Voir un tableau \\
Voit-on ce qu'on croit ? \\
Vouloir ce que l'on peut \\
Vouloir dire \\
Vouloir, est-ce encore désirer ? \\
Vouloir et pouvoir \\
Vouloir être immortel \\
Vouloir la paix sociale peut-il aller jusqu'à accepter l'injustice ? \\
Vouloir la solitude \\
Vouloir le bien \\
Vouloir l'égalité \\
Vouloir le mal \\
Vouloir l'impossible \\
Vouloir oublier \\
Voyager \\
Vulgariser la science ? \\
Y a-t-il continuité entre l'expérience et la science ? \\
Y a-t-il continuité ou discontinuité entre la pensée mythique et la science ? \\
Y a-t-il d'autres moyens que la démonstration pour établir la vérité ? \\
Y a-t-il de bons et de mauvais désirs ? \\
Y a-t-il de bons préjugés ? \\
Y a-t-il de fausses religions ? \\
Y a-t-il de faux besoins ? \\
Y a-t-il de faux problèmes ? \\
Y a-t-il de justes inégalités ? \\
Y a-t-il de la fatalité dans la vie de l'homme ? \\
Y a-t-il de la raison dans la perception ? \\
Y a-t-il de l'impensable ? \\
Y a-t-il de l'incommunicable ? \\
Y a-t-il de l'inconcevable ? \\
Y a-t-il de l'inconnaissable ? \\
Y a-t-il de l'indémontrable ? \\
Y a-t-il de l'indésirable ? \\
Y a-t-il de l'indicible ? \\
Y a-t-il de l'inexprimable ? \\
Y a-t-il de l'irréductible ? \\
Y a-t-il de l'irréfutable ? \\
Y a-t-il de l'irréparable ? \\
Y a-t-il de l'universel ? \\
Y a-t-il de mauvais désirs ? \\
Y a-t-il des acquis définitifs en science ? \\
Y a-t-il des actes de pensée ? \\
Y a-t-il des actes désintéressés ? \\
Y a-t-il des actes gratuits ? \\
Y a-t-il des actes moralement indifférents ? \\
Y a-t-il des actions désintéressées ? \\
Y a-t-il des arts mineurs ? \\
Y a-t-il des barbares ? \\
Y a-t-il des biens inestimables ? \\
Y a-t-il des canons de la beauté ? \\
Y a-t-il des certitudes historiques ? \\
Y a-t-il des choses qui échappent au droit ? \\
Y a-t-il des choses qu'on n'échange pas ? \\
Y a-t-il des compétences politiques ? \\
Y a-t-il des connaissances dangereuses ? \\
Y a-t-il des connaissances désintéressées ? \\
Y a-t-il des contraintes légitimes ? \\
Y a-t-il des convictions philosophiques ? \\
Y a-t-il des correspondances entre les arts ? \\
Y a-t-il des critères de l'humanité ? \\
Y a-t-il des critères du beau ? \\
Y a-t-il des critères du goût ? \\
Y a-t-il des croyances démocratiques ? \\
Y a-t-il des croyances nécessaires ? \\
Y a-t-il des croyances rationnelles ? \\
Y a-t-il des degrés dans la certitude ? \\
Y a-t-il des degrés de conscience ? \\
Y a-t-il des degrés de réalité ? \\
Y a-t-il des degrés de vérité ? \\
Y a-t-il des démonstrations en philosophie ? \\
Y a-t-il des despotes éclairés ? \\
Y a-t-il des déterminismes sociaux ? \\
Y a-t-il des devoirs envers soi-même ? \\
Y a-t-il des devoirs envers soi ? \\
Y a-t-il des dilemmes moraux ? \\
Y a-t-il des droits sans devoirs ? \\
Y a-t-il des erreurs de la nature ? \\
Y a-t-il des erreurs en politique ? \\
Y a-t-il des êtres mathématiques ? \\
Y a-t-il des évidences morales ? \\
Y a-t-il des expériences absolument certaines ? \\
Y a-t-il des expériences cruciales ? \\
Y a-t-il des expériences de la liberté ? \\
Y a-t-il des expériences métaphysiques ? \\
Y a-t-il des expériences sans théorie ? \\
Y a-t-il des facultés dans l'esprit ? \\
Y a-t-il des faits moraux ? \\
Y a-t-il des faits sans essence ? \\
Y a-t-il des faits scientifiques ? \\
Y a-t-il des faux problèmes ? \\
Y a-t-il des fins dans la nature ? \\
Y a-t-il des fins de la nature ? \\
Y a-t-il des fins dernières ? \\
Y a-t-il des fondements naturels à l'ordre social ? \\
Y a-t-il des genres de plaisir ? \\
Y a-t-il des genres du plaisir ? \\
Y a-t-il des guerres justes ? \\
Y a-t-il des héritages philosophiques ? \\
Y a-t-il des illusions de la conscience ? \\
Y a-t-il des illusions nécessaires ? \\
Y a-t-il des inégalités justes ? \\
Y a-t-il des injustices naturelles ? \\
Y a-t-il des instincts propres à l'Homme ? \\
Y a-t-il des interprétations fausses ? \\
Y a-t-il des intuitions morales ? \\
Y a-t-il des leçons de l'histoire ? \\
Y a-t-il des limites à la connaissance ? \\
Y a-t-il des limites à la conscience ? \\
Y a-t-il des limites à la tolérance ? \\
Y a-t-il des limites à l'exprimable ? \\
Y a-t-il des limites au droit ? \\
Y a-t-il des limites proprement morales à la discussion ? \\
Y a-t-il des lois de la pensée ? \\
Y a-t-il des lois de l'histoire ? \\
Y a-t-il des lois de l'Histoire ? \\
Y a-t-il des lois du hasard ? \\
Y a-t-il des lois du social ? \\
Y a-t-il des lois du vivant ? \\
Y a-t-il des lois en histoire ? \\
Y a-t-il des lois injustes ? \\
Y a-t-il des lois morales ? \\
Y a-t-il des lois non écrites ? \\
Y a-t-il des mentalités collectives ? \\
Y a-t-il des modèles en morale ? \\
Y a-t-il des mondes imaginaires ? \\
Y a-t-il des normes naturelles ? \\
Y a-t-il des objets qui n'existent pas ? \\
Y a-t-il des obstacles à la connaissance du vivant ? \\
Y a-t-il des passions collectives ? \\
Y a-t-il des passions intraitables ? \\
Y a-t-il des passions raisonnables ? \\
Y a-t-il des pathologies sociales ? \\
Y a-t-il des pensées folles ? \\
Y a-t-il des pensées inconscientes ? \\
Y a-t-il des perceptions insensibles ? \\
Y a-t-il des peuples sans histoire ? \\
Y a-t-il des plaisirs meilleurs que d'autres ? \\
Y a-t-il des plaisirs purs ? \\
Y a-t-il des preuves d'amour ? \\
Y a-t-il des principes de justice universels ? \\
Y a-t-il des progrès en art ? \\
Y a-t-il des propriétés singulières ? \\
Y a-t-il des questions sans réponse ? \\
Y a-t-il des raisons de douter de la raison ? \\
Y a-t-il des règles de la guerre ? \\
Y a-t-il des règles de l'art ? \\
Y a-t-il des régressions historiques ? \\
Y a-t-il des révolutions en art ? \\
Y a-t-il des révolutions scientifiques ? \\
Y a-t-il des sciences de l'homme ? \\
Y a-t-il des sciences exactes ? \\
Y a-t-il des secrets de la nature ? \\
Y a-t-il des sentiments moraux ? \\
Y a-t-il des signes naturels ? \\
Y a-t-il des sociétés sans État ? \\
Y a-t-il des sociétés sans histoire ? \\
Y a-t-il des solutions en politique ? \\
Y a-t-il des sots métiers ? \\
Y a-t-il des substances incorporelles ? \\
Y a-t-il des techniques de pensée ? \\
Y a-t-il des techniques du corps ? \\
Y a-t-il des valeurs absolues ? \\
Y a-t-il des valeurs naturelles ? \\
Y a-t-il des valeurs objectives ? \\
Y a-t-il des valeurs propres à la science ? \\
Y a-t-il des valeurs universelles ? \\
Y a-t-il des vérités de fait ? \\
Y a-t-il des vérités définitives ? \\
Y a-t-il des vérités en art ? \\
Y a-t-il des vérités éternelles ? \\
Y a-t-il des vérités indémontrables ? \\
Y a-t-il des vérités indiscutables ? \\
Y a-t-il des vérités morales ? \\
Y a-t-il des vérités philosophiques ? \\
Y a-t-il des vérités qui échappent à la raison ? \\
Y a-t-il des vérités sans preuve ? \\
Y a-t-il des vertus mineures ? \\
Y a-t-il des violences justifiées ? \\
Y a-t-il des violences légitimes ? \\
Y a-t-il différentes façons d'exister ? \\
Y a-t-il différentes manières de connaître ? \\
Y a-t-il du non être ? \\
Y a-t-il du non-être ? \\
Y a-t-il du nouveau dans l'histoire ? \\
Y a-t-il du sacré dans la nature ? \\
Y a-t-il du synthétique a priori ? \\
Y a-t-il encore des mythologies ? \\
Y a-t-il encore une sphère privée ? \\
Y a-t-il lieu d'opposer matière et esprit ? \\
Y a-t-il nécessairement du religieux dans l'art ? \\
Y a-t-il place pour l'idée de vérité en morale ? \\
Y a-t-il plusieurs libertés ? \\
Y a-t-il plusieurs manières de définir ? \\
Y a-t-il plusieurs morales ? \\
Y a-t-il plusieurs nécessités ? \\
Y a-t-il plusieurs sortes de matières ? \\
Y a-t-il plusieurs sortes de vérité ? \\
Y a-t-il progrès en art ? \\
Y a-t-il quoi que ce soit de nouveau dans l'histoire ? \\
Y a-t-il trop d'images ? \\
Y a-t-il un art de gouverner ? \\
Y a-t-il un art de penser ? \\
Y a-t-il un art d'être heureux ? \\
Y a-t-il un art de vivre ? \\
Y a-t-il un art d'interpréter ? \\
Y a-t-il un art d'inventer ? \\
Y a-t-il un art du bonheur ? \\
Y a-t-il un au-delà de la vérité ? \\
Y a-t-il un au-delà du langage ? \\
Y a-t-il un auteur de l'histoire ? \\
Y a-t-il un autre monde ? \\
Y a-t-il un beau idéal ? \\
Y a-t-il un beau naturel ? \\
Y a-t-il un besoin métaphysique ? \\
Y a-t-il un bien commun ? \\
Y a-t-il un bien plus précieux que la paix ? \\
Y a-t-il un bonheur sans illusion ? \\
Y a-t-il un bon usage des passions ? \\
Y a-t-il un bon usage du temps ? \\
Y a-t-il un canon de la beauté ? \\
Y a-t-il un critère de vérité ? \\
Y a-t-il un critère du vrai ? \\
Y a-t-il un devoir de mémoire ? \\
Y a-t-il un devoir d'être heureux ? \\
Y a-t-il un devoir d'indignation ? \\
Y a-t-il un droit à la différence ? \\
Y a-t-il un droit au bonheur ? \\
Y a-t-il un droit au travail ? \\
Y a-t-il un droit de désobéissance ? \\
Y a-t-il un droit de la guerre ? \\
Y a-t-il un droit de mentir ? \\
Y a-t-il un droit de mourir ? \\
Y a-t-il un droit de résistance ? \\
Y a-t-il un droit de révolte ? \\
Y a-t-il un droit d'ingérence ? \\
Y a-t-il un droit du plus faible ? \\
Y a-t-il un droit du plus fort ? \\
Y a-t-il un droit international ? \\
Y a-t-il un droit naturel ? \\
Y a-t-il un droit universel au mariage ? \\
Y a-t-il une argumentation métaphysique ? \\
Y a-t-il une beauté morale ? \\
Y a-t-il une beauté naturelle ? \\
Y a-t-il une beauté propre à l'objet technique ? \\
Y a-t-il une bonne imitation ? \\
Y a-t-il une causalité empirique ? \\
Y a-t-il une causalité en histoire ? \\
Y a-t-il une causalité historique ? \\
Y a-t-il une cause première ? \\
Y a-t-il une compétence en politique ? \\
Y a-t-il une compétence politique ? \\
Y a-t-il une condition humaine ? \\
Y a-t-il une connaissance du probable ? \\
Y a-t-il une connaissance du singulier ? \\
Y a-t-il une connaissance historique ? \\
Y a-t-il une connaissance métaphysique ? \\
Y a-t-il une connaissance sensible ? \\
Y a-t-il une conscience collective ? \\
Y a-t-il une correspondance des arts ? \\
Y a-t-il une éducation du goût ? \\
Y a-t-il une enfance de l'humanité ? \\
Y a-t-il une esthétique de la laideur ? \\
Y a-t-il une éthique de l'authenticité ? \\
Y a-t-il une éthique des moyens ? \\
Y a-t-il une expérience de la liberté ? \\
Y a-t-il une expérience de l'éternité ? \\
Y a-t-il une expérience du néant ? \\
Y a-t-il une expérience du temps ? \\
Y a-t-il une fin de l'histoire ? \\
Y a-t-il une fin dernière ? \\
Y a-t-il une fonction propre à l'œuvre d'art ? \\
Y a-t-il une force du droit ? \\
Y a-t-il une forme morale de fanatisme ? \\
Y a-t-il une hiérarchie des êtres ? \\
Y a-t-il une hiérarchie des sciences ? \\
Y a-t-il une hiérarchie du vivant ? \\
Y a-t-il une histoire de la nature ? \\
Y a-t-il une histoire de la raison ? \\
Y a-t-il une histoire de la vérité ? \\
Y a-t-il une histoire universelle ? \\
Y a-t-il une intelligence du corps ? \\
Y a-t-il une intentionnalité collective ? \\
Y a-t-il une irréversibilité du temps ? \\
Y a-t-il une justice naturelle ? \\
Y a-t-il une justice sans morale ? \\
Y a-t-il une langue de la philosophie ? \\
Y a-t-il une limite à la connaissance du vivant ? \\
Y a-t-il une limite au désir ? \\
Y a-t-il une limite au développement scientifique ? \\
Y a-t-il une logique dans l'histoire ? \\
Y a-t-il une logique de la découverte scientifique ? \\
Y a-t-il une logique de la découverte ? \\
Y a-t-il une logique des événements historiques ? \\
Y a-t-il une logique du désir ? \\
Y a-t-il une mécanique des passions ? \\
Y a-t-il une médecine de l'âme ? \\
Y a-t-il une métaphysique de l'amour ? \\
Y a-t-il une méthode de l'interprétation ? \\
Y a-t-il une méthode propre aux sciences humaines ? \\
Y a-t-il une morale universelle ? \\
Y a-t-il un empire de la technique ? \\
Y a-t-il une nature humaine ? \\
Y a-t-il une nécessité de l'erreur ? \\
Y a-t-il une nécessité de l'Histoire ? \\
Y a-t-il une nécessité morale ? \\
Y a-t-il une œuvre du temps ? \\
Y a-t-il une opinion publique mondiale ? \\
Y a-t-il une ou des morales ? \\
Y a-t-il une ou plusieurs philosophies ? \\
Y a-t-il une pensée sans signes ? \\
Y a-t-il une pensée technique ? \\
Y a-t-il une philosophie de la nature ? \\
Y a-t-il une philosophie de la philosophie ? \\
Y a-t-il une philosophie première ? \\
Y a-t-il une place pour la morale dans l'économie ? \\
Y a-t-il une positivité de l'erreur ? \\
Y a-t-il une présence du passé ? \\
Y a-t-il une primauté du devoir sur le droit ? \\
Y a-t-il une rationalité des sentiments ? \\
Y a-t-il une rationalité du hasard ? \\
Y a-t-il une réalité du hasard ? \\
Y a-t-il une responsabilité de l'artiste ? \\
Y a-t-il une sagesse populaire ? \\
Y a-t-il une science de la vie mentale ? \\
Y a-t-il une science de l'esprit ? \\
Y a-t-il une science de l'être ? \\
Y a-t-il une science de l'homme ? \\
Y a-t-il une science de l'individuel ? \\
Y a-t-il une science des principes ? \\
Y a-t-il une science du moi ? \\
Y a-t-il une science du qualitatif ? \\
Y a-t-il une science politique ? \\
Y a-t-il une sensibilité esthétique ? \\
Y a-t-il une servitude volontaire ? \\
Y a-t-il une spécificité de la délibération politique ? \\
Y a-t-il une spécificité des sciences humaines ? \\
Y a-t-il une spécificité du vivant ? \\
Y a-t-il un esprit scientifique ? \\
Y a-t-il un État idéal ? \\
Y a-t-il une technique de la nature ? \\
Y a-t-il une technique pour tout ? \\
Y a-t-il une unité de la science ? \\
Y a-t-il une unité des devoirs ? \\
Y a-t-il une unité des langages humains ? \\
Y a-t-il une unité des sciences ? \\
Y a-t-il une unité en psychologie ? \\
Y a-t-il une universalité des mathématiques ? \\
Y a-t-il une universalité du beau ? \\
Y a-t-il une valeur de l'inutile ? \\
Y a-t-il une vérité de l'œuvre d'art ? \\
Y a-t-il une vérité des apparences ? \\
Y a-t-il une vérité des représentations ? \\
Y a-t-il une vérité des sentiments ? \\
Y a-t-il une vérité des symboles ? \\
Y a-t-il une vérité du sensible ? \\
Y a-t-il une vérité du sentiment ? \\
Y a-t-il une vérité en histoire ? \\
Y a-t-il une vérité philosophique ? \\
Y a-t-il une vertu de l'imitation ? \\
Y a-t-il une vertu de l'oubli ? \\
Y a-t-il une vie de l'esprit ? \\
Y a-t-il une violence du droit ? \\
Y a-t-il un fondement de la croyance ? \\
Y a-t-il un inconscient collectif ? \\
Y a-t-il un inconscient psychique ? \\
Y a-t-il un inconscient social ? \\
Y a-t-il un jugement de l'histoire ? \\
Y a-t-il un langage animal ? \\
Y a-t-il un langage commun ? \\
Y a-t-il un langage de la musique ? \\
Y a-t-il un langage de l'art ? \\
Y a-t-il un langage de l'inconscient ? \\
Y a-t-il un langage du corps ? \\
Y a-t-il un langage unifié de la science ? \\
Y a-t-il un mal absolu ? \\
Y a-t-il un monde de l'art ? \\
Y a-t-il un monde extérieur ? \\
Y a-t-il un moteur de l'histoire ? \\
Y a-t-il un objet du désir ? \\
Y a-t-il un ordre dans la nature ? \\
Y a-t-il un ordre des choses ? \\
Y a-t-il un ordre du monde ? \\
Y a-t-il un primat de la nature sur la culture ? \\
Y a-t-il un principe du mal ? \\
Y a-t-il un progrès du droit ? \\
Y a-t-il un progrès en art ? \\
Y a-t-il un progrès en philosophie ? \\
Y a-t-il un progrès moral ? \\
Y a-t-il un propre de l'homme ? \\
Y a-t-il un rapport moral à soi-même ? \\
Y a-t-il un rythme de l'histoire ? \\
Y a-t-il un savoir de la justice ? \\
Y a-t-il un savoir du contingent ? \\
Y a-t-il un savoir du corps ? \\
Y a-t-il un savoir du juste ? \\
Y a-t-il un savoir du politique ? \\
Y a-t-il un savoir immédiat ? \\
Y a-t-il un savoir politique ? \\
Y a-t-il un savoir pratique \\
Y a-t-il un sens du beau ? \\
Y a-t-il un sens moral ? \\
Y a-t-il un souverain bien ? \\
Y a-t-il un temps des choses ? \\
Y a-t-il un temps pour tout ? \\
Y a-t-il un travail de la pensée ? \\
Y a-t-il un usage moral des passions ? \\
Y a-t-il un usage purement instrumental de la raison ? \\
Y aura-t-il toujours des religions ? \\
« Aime, et fais ce que tu veux » \\
« Aimer » se dit-il en un seul sens ? \\
« Aimez vos ennemis » \\
« Après moi, le déluge » \\
« Aux armes citoyens ! » \\
« À l'impossible, nul n'est tenu » \\
« À quelque chose malheur est bon » \\
« Bienheureuse faute » \\
« Ceci » \\
« Ce ne sont que des mots » \\
« C'est humain » \\
« C'est la vie » \\
« Comment peut-on être persan ? » \\
« Connais-toi toi-même » \\
« Dans un bois aussi courbe que celui dont l'homme est fait on ne peut rien tailler de tout à fait droit » \\
« De la musique avant toute chose » \\
« Deviens qui tu es » \\
« Dieu est mort » \\
« Être contre » \\
« Expliquer les faits sociaux par des faits sociaux » \\
« Il faudrait rester des années entières pour contempler une telle œuvre » \\
« Il ne lui manque que la parole » \\
« Je mens » \\
« Je n'ai pas voulu cela » \\
« Je ne crois que ce que je vois » \\
« Je ne l'ai pas fait exprès » \\
« Je ne voulais pas cela » : en quoi les sciences humaines permettent-elles de comprendre cette excuse ? \\
« Je préfère une injustice à un désordre » \\
« La crainte est le commencement de la sagesse » \\
« La critique est aisée » \\
« La logique » ou bien « les logiques » ? \\
« La science ne pense pas » \\
« La vie des formes » \\
« La vie est une scène » \\
« La vie est un songe » \\
« La vraie morale se moque de la morale » \\
« L'enfer est pavé de bonnes intentions » \\
« Les bons comptes font les bons amis » \\
« Le seul problème philosophique vraiment sérieux, c'est le suicide » \\
« Les faits, rien que les faits » \\
« Les faits sont là » \\
« Le travail rend libre » \\
« L'histoire jugera » \\
« L'histoire jugera » : quel sens faut-il accorder à cette expression ? \\
« L'homme est la mesure de toute chose » \\
« L'homme est la mesure de toutes choses » \\
« Liberté, égalité, fraternité » \\
« Malheur aux vaincus » \\
« Ne fais pas à autrui ce que tu ne voudrais pas qu'on te fasse » \\
« Nul n'est censé ignorer la loi » \\
« Œil pour œil, dent pour dent » \\
« Pas de liberté pour les ennemis de la liberté » ? \\
« Pauvre bête » \\
« Petites causes, grands effets » \\
« Pourquoi » \\
« Prendre ses désirs pour des réalités » \\
« Quelle vanité que la peinture » \\
« Que nul n'entre ici s'il n'est géomètre » \\
« Que va-t-il se passer ? » \\
« Rien de ce qui est humain ne m'est étranger » \\
« Rien n'est sans raison » \\
« Rien n'est simple » \\
« Sans titre » \\
« Sauver les apparences » \\
« Sauver les phénomènes » \\
« Toute peine mérite salaire » \\
« Tout est relatif » \\
« Tradition n'est pas raison » \\
« Trop beau pour être vrai » \\
« Tu ne tueras point » \\
« Un instant d'éternité » \\
« Vis caché » 
% Emacs 24.5.1 (Org mode 8.2.10)
\end{document}
