% Created 2020-10-28 mer. 18:40
\documentclass[a4paper,12pt]{article}
\usepackage[utf8]{inputenc}
\usepackage[T1]{fontenc}
\usepackage{fixltx2e}
\usepackage{graphicx}
\usepackage{longtable}
\usepackage{float}
\usepackage{wrapfig}
\usepackage{rotating}
\usepackage[normalem]{ulem}
\usepackage{amsmath}
\usepackage{textcomp}
\usepackage{marvosym}
\usepackage{wasysym}
\usepackage{amssymb}
\usepackage{hyperref}
\tolerance=1000
\usepackage[frenchb]{babel}
\usepackage[frenchb]{babel}
\usepackage{lmodern}
\DeclareUnicodeCharacter{00A0}{~}
\DeclareUnicodeCharacter{200B}{}
\author{Baptiste Mélès}
\date{28 octobre 2020}
\title{11 000 sujets de dissertation de philosophie}
\hypersetup{
  pdfkeywords={},
  pdfsubject={},
  pdfcreator={Emacs 24.5.1 (Org mode 8.2.10)}}
\begin{document}

\maketitle


On trouvera ici, classés par ordre alphabétique puis par concours et
par type, quelque 11 000 sujets de dissertation de philosophie.

Cette liste est la compilation de tous les sujets donnés aux épreuves
écrites et orales de plus de 80 concours :
\begin{enumerate}
\item agrégation externe (13 années : 2008-2020) ;
\item agrégation interne (11 années : 2009-2019) ;
\item CAPES externe (11 années : 2010-2020) ;
\item CAPES interne (9 années : 2011-2019) ;
\item École normale supérieure de Paris, concours A​/​L (19 années :
2002-2020) ;
\item École normale supérieure de Paris, concours B​/​L (19 années :
2002-2020).
\end{enumerate}

L'auteur vous sera reconnaissant de lui signaler toute coquille ou erreur\footnote{Documents pédagogiques de Baptiste Mélès (CNRS, AHP-PReST,
Université de Lorraine) :
\begin{enumerate}
\item « Méthode de la dissertation philosophique »
(\url{http://baptiste.meles.free.fr/site/B.Meles-Methode_dissertation.pdf},
2010-2020) ;
\item « Méthode du commentaire de texte philosophique »
(\url{http://baptiste.meles.free.fr/site/B.Meles-Methode_commentaire_texte.pdf},
2007-2016)
\item « Méthodologie du mémoire de Master »
(\url{http://baptiste.meles.free.fr/site/B.Meles-Memoire_Master.pdf},
2014-2019) ;
\item « Le travail personnel en philosophie, de la licence à l'agrégation »
(\url{http://baptiste.meles.free.fr/site/B.Meles-Travail_perso.pdf},
2008-2016) ;
\item « Les tables de vérité en braille »
(\url{http://baptiste.meles.free.fr/site/B.Meles-Table_verite_braille.pdf},
2011).
\end{enumerate}}.


\newpage

\tableofcontents

\newpage

\section{Liste complète}
\label{sec-1}

\noindent
2+2 = 4 \\
2+2 pourrait-il ne pas être égal à 4 ? \\
Abolir la propriété \\
Abstraire, est-ce se couper du réel ? \\
Abuser du pouvoir \\
À chacun sa morale \\
À chacun son dû \\
Acteurs sociaux et usages sociaux \\
Action et contemplation \\
Action et événement \\
Action et production \\
Activité et passivité \\
Affirmer et nier \\
Agir \\
Agir et faire \\
Agir et réagir \\
Agir justement fait-il de moi un homme juste ? \\
Agir moralement, est-ce lutter contre ses idées ? \\
Agir sans raison \\
Aide-toi, le ciel t'aidera \\
Ai-je des devoirs envers moi-même ? \\
Ai-je un corps ou suis-je mon corps ? \\
Ai-je un corps ? \\
Ai-je une âme ? \\
Aimer ce qui est beau \\
Aimer, est-ce vraiment connaître ? \\
Aimer la nature \\
Aimer la vie \\
Aimer les lois \\
Aimer peut-il être un devoir ? \\
Aimer ses proches \\
Aimer son prochain \\
Aimer son prochain comme soi-même \\
Aimer une œuvre d'art \\
Aime ton prochain comme toi-même \\
À l'impossible nul n'est tenu \\
Ami et ennemi \\
Amitié et société \\
Amour et amitié \\
Amour et inconscient \\
Analyse et synthèse \\
Analyser \\
Analyser les mœurs \\
Animal politique ou social ? \\
Anomalie et anomie \\
Anthropologie et ontologie \\
Anthropologie et politique \\
Apparaître \\
Apparence et réalité \\
Appartenons-nous à une culture ? \\
Apprend-on à aimer ? \\
Apprend-on à être artiste ? \\
Apprend-on à penser ? \\
Apprend-on à percevoir ? \\
Apprend-on à voir ? \\
Apprendre à gouverner \\
Apprendre à parler \\
Apprendre à penser \\
Apprendre à philosopher \\
Apprendre à vivre \\
Apprendre à voir \\
Apprendre et enseigner \\
Apprendre s'apprend-il ? \\
Apprentissage et conditionnement \\
Après-coup \\
Après moi le déluge \\
\emph{A priori} et \emph{a posteriori} \\
À quelle expérience l'art nous convie-t-il ? \\
À quelles conditions une démarche est-elle scientifique ? \\
À quelles conditions une expérience est-elle possible ? \\
À quelles conditions une hypothèse est-elle scientifique ? \\
À quelles conditions un énoncé est-il doué de sens ? \\
À quoi bon discuter ? \\
À quoi bon imiter la nature ? \\
À quoi bon les sciences humaines et sociales ? \\
À quoi bon penser la fin du monde ? \\
À quoi bon voyager ? \\
À quoi bon ? \\
À quoi faut-il renoncer ? \\
À quoi juger l'action d'un gouvernement ? \\
À quoi la conscience nous donne-t-elle accès ? \\
À quoi la logique peut-elle servir dans les sciences ? \\
À quoi nos illusions tiennent-elles ? \\
À quoi reconnaît-on la vérité ? \\
À quoi reconnaît-on qu'une expérience est scientifique ? \\
À quoi reconnaît-on qu'une théorie est scientifique ? \\
À quoi reconnaît-on qu'un événement est historique ? \\
À quoi reconnaît-on une œuvre d'art ? \\
À quoi reconnaît-on une religion ? \\
À quoi reconnaît-on un être vivant ? \\
À quoi sert la dialectique ? \\
À quoi sert la négation ? \\
À quoi sert la technique ? \\
À quoi sert l'écriture ? \\
À quoi sert l'État ? \\
À quoi servent les mythes ? \\
À quoi servent les preuves de l'existence de Dieu ? \\
À quoi servent les religions ? \\
À quoi servent les sciences ? \\
À quoi servent les utopies ? \\
À quoi tient la fermeté du vouloir ? \\
À quoi tient la force de l'État ? \\
À quoi tient la force des religions ? \\
À quoi tient la vérité d'une interprétation ? \\
Argent et liberté \\
Argumenter \\
Argumenter et démontrer \\
Art et apparences \\
Art et authenticité \\
Art et beauté \\
Art et connaissance \\
Art et création \\
Art et critique \\
Art et divertissement \\
Art et émotion \\
Art et finitude \\
Art et folie \\
Art et forme \\
Art et illusion \\
Art et image \\
Art et imagination \\
Art et interdit \\
Art et jeu \\
Art et marchandise \\
Art et matière \\
Art et mélancolie \\
Art et mémoire \\
Art et métaphysique \\
Art et morale \\
Art et politique \\
Art et pouvoir \\
Art et propagande \\
Art et religieux \\
Art et religion \\
Art et représentation \\
Art et société \\
Art et Société \\
Art et symbole \\
Art et technique \\
Art et transgression \\
Art et vérité \\
Artiste et artisan \\
Art populaire et art savant \\
Arts de l'espace et arts du temps \\
A-t-on besoin de certitudes ? \\
A-t-on besoin de fonder la connaissance ? \\
A-t-on besoin de spécialistes en politique ? \\
A-t-on besoin d'experts ? \\
A-t-on besoin d'un chef ? \\
A-t-on des devoirs envers soi-même ? \\
A-t-on des droits contre l'État ? \\
A-t-on des raisons de croire ce qu'on croit ? \\
A-t-on des raisons de croire ? \\
A-t-on intérêt à tout savoir ? \\
A-t-on le droit de mentir ? \\
A-t-on le droit de résister ? \\
A-t-on le droit de se révolter ? \\
A-t-on le droit de s'évader ? \\
A-t-on l'obligation de pardonner ? \\
Attente et espérance \\
Au-delà \\
Au-delà de la nature ? \\
Au nom de qui rend-on justice ? \\
Au nom de quoi le plaisir serait-il condamnable ? \\
Au nom de quoi rend-on justice ? \\
Aussitôt dit, aussitôt fait \\
Autorité et pouvoir \\
Autorité et souveraineté \\
Autrui \\
Autrui, est-ce n'importe quel autre ? \\
Autrui est-il aimable ? \\
Autrui est-il inconnaissable ? \\
Autrui est-il mon semblable ? \\
Autrui est-il pour moi un mystère ? \\
Autrui est-il un autre moi-même ? \\
Autrui est-il un autre moi ? \\
Autrui me connaît-il mieux que moi-même ? \\
Autrui m'est-il étranger ? \\
Aux armes, citoyens ! \\
Avez-vous une âme ? \\
Avoir \\
Avoir bonne conscience \\
Avoir confiance \\
Avoir de la suite dans les idées \\
Avoir de l'autorité \\
Avoir de l'esprit \\
Avoir de l'expérience \\
Avoir des ennemis \\
Avoir des principes \\
Avoir des valeurs \\
Avoir du goût \\
Avoir du jugement \\
Avoir du métier \\
Avoir du pouvoir \\
Avoir du style \\
Avoir le choix \\
Avoir le sens de la situation \\
Avoir le sens du devoir \\
Avoir le temps \\
Avoir mauvaise conscience \\
Avoir peur \\
Avoir peur des mots \\
Avoir raison \\
Avoir un corps \\
Avoir un destin \\
Avoir une bonne mémoire \\
Avoir une idée \\
Avons-nous à apprendre des images ? \\
Avons-nous besoin d'amis ? \\
Avons-nous besoin de cérémonies ? \\
Avons-nous besoin de héros ? \\
Avons-nous besoin de maîtres ? \\
Avons-nous besoin de métaphysique ? \\
Avons-nous besoin de partis politiques ? \\
Avons-nous besoin de rêver ? \\
Avons-nous besoin de spectacles ? \\
Avons-nous besoin de traditions ? \\
Avons-nous besoin d'experts en matière d'art ? \\
Avons-nous besoin d'un libre arbitre ? \\
Avons-nous besoin d'utopies ? \\
Avons-nous des devoirs à l'égard de la vérité ? \\
Avons-nous des devoirs envers la nature ? \\
Avons-nous des devoirs envers les animaux ? \\
Avons-nous des devoirs envers les autres êtres vivants ? \\
Avons-nous des devoirs envers les générations futures ? \\
Avons-nous des devoirs envers les morts ? \\
Avons-nous des devoirs envers le vivant ? \\
Avons-nous des devoirs envers nous-mêmes ? \\
Avons-nous des droits sur la nature ? \\
Avons-nous des raisons d'espérer ? \\
Avons-nous intérêt à la liberté d'autrui ? \\
Avons-nous le devoir d'être heureux ? \\
Avons-nous le devoir de vivre ? \\
Avons-nous le droit de juger autrui ? \\
Avons-nous le droit d'être heureux ? \\
Avons-nous peur de la liberté ? \\
Avons-nous raison d'exiger toujours des raisons ? \\
Avons-nous un corps ? \\
Avons-nous un devoir de vérité ? \\
Avons-nous un droit au droit ? \\
Avons-nous une âme ? \\
Avons-nous une identité ? \\
Avons-nous une intuition du temps ? \\
Avons-nous une obligation envers les générations à venir ? \\
Avons-nous une responsabilité envers le passé ? \\
Avons-nous un libre arbitre ? \\
Avons-nous un monde commun ? \\
Axiomatiser, est-ce fonder ? \\
À chacun selon son mérite \\
À chacun ses goûts \\
À chacun son dû \\
À l'impossible nul n'est tenu \\
À quelle condition un travail est-il humain ? \\
À quelles conditions est-il acceptable de travailler ? \\
À quelles conditions le vivant peut-il être objet de science ? \\
À quelles conditions peut-on dire qu'une action est historique ? \\
À quelles conditions un choix peut-il être rationnel ? \\
À quelles conditions une démarche est-elle scientifique ? \\
À quelles conditions une explication est-elle scientifique ? \\
À quelles conditions une hypothèse est-elle scientifique ? \\
À quelles conditions une pensée est-elle libre ? \\
À quelles conditions une théorie est-elle scientifique ? \\
À quelles conditions une théorie peut-elle être scientifique ? \\
À quelles conditions un jugement est-il moral ? \\
À quelque chose malheur est bon \\
À quels signes reconnaît-on la vérité ? \\
À qui devons-nous obéir ? \\
À qui dois-je la vérité ? \\
À qui doit-on le respect ? \\
À qui doit-on obéir ? \\
À qui est mon corps ? \\
À qui faut-il obéir ? \\
À qui la faute ? \\
À qui profite le crime ? \\
À qui profite le travail ? \\
À quoi bon avoir mauvaise conscience ? \\
À quoi bon critiquer les autres ? \\
À quoi bon démontrer ? \\
À quoi bon les regrets ? \\
À quoi bon les romans ? \\
À quoi bon raconter des histoires ? \\
À quoi bon se parler ? \\
À quoi bon voyager ? \\
À quoi est-il impossible de s'habituer ? \\
À quoi faut-il être fidèle ? \\
À quoi la perception donne-t-elle accès ? \\
À quoi l'art nous rend-il sensibles ? \\
À quoi la valeur d'un homme se mesure-t-elle ? \\
À quoi peut-on reconnaître une œuvre d'art ? \\
À quoi reconnaît-on la rationalité ? \\
À quoi reconnaît-on la vérité ? \\
À quoi reconnaît-on le réel ? \\
À quoi reconnaît-on l'injustice ? \\
À quoi reconnaît-on qu'une activité est un travail ? \\
À quoi reconnaît-on qu'une expérience est scientifique ? \\
À quoi reconnaît-on qu'une pensée est vraie ? \\
À quoi reconnaît-on qu'une politique est juste ? \\
À quoi reconnaît-on un acte libre ? \\
À quoi reconnaît-on un bon gouvernement ? \\
À quoi reconnaît-on une attitude religieuse ? \\
À quoi reconnaît-on une bonne interprétation ? \\
À quoi reconnaît-on une idéologie ? \\
À quoi reconnaît-on une œuvre d'art ? \\
À quoi sert la connaissance du passé ? \\
À quoi sert la logique ? \\
À quoi sert la mémoire ? \\
À quoi sert la notion de contrat social ? \\
À quoi sert la notion d'état de nature ? \\
À quoi sert la technique ? \\
À quoi sert le contrat social ? \\
À quoi sert l'État ? \\
À quoi sert l'histoire ? \\
À quoi sert un exemple ? \\
À quoi servent les doctrines morales ? \\
À quoi servent les élections ? \\
À quoi servent les émotions ? \\
À quoi servent les expériences ? \\
À quoi servent les fictions ? \\
À quoi servent les images ? \\
À quoi servent les lois ? \\
À quoi servent les machines ? \\
À quoi servent les mythes ? \\
À quoi servent les preuves ? \\
À quoi servent les règles ? \\
À quoi servent les statistiques ? \\
À quoi servent les symboles ? \\
À quoi servent les théories ? \\
À quoi servent les utopies ? \\
À quoi servent les voyages ? \\
À quoi tenons-nous ? \\
À quoi tient la force des religions ? \\
À quoi tient l'autorité ? \\
À quoi tient la valeur d'une pensée ? \\
À quoi tient le pouvoir des mots ? \\
À quoi tient notre humanité ? \\
À science nouvelle, nouvelle philosophie ? \\
À t-on le droit de faire tout ce qui est permis par la loi ? \\
Bâtir un monde \\
Beauté et moralité \\
Beauté et vérité \\
Beauté naturelle et beauté artistique \\
Beauté réelle, beauté idéale \\
Besoin et désir \\
Besoins et désirs \\
Bêtise et méchanceté \\
Bien agir, est-ce toujours être moral ? \\
Bien commun et bien public \\
Bien commun et intérêt particulier \\
Bien jouer son rôle \\
Bien juger \\
Bien parler \\
Bonheur de chacun bonheur de tous \\
Bonheur et autarcie \\
Bonheur et satisfaction \\
Bonheur et société \\
Bonheur et technique \\
Bonheur et vertu \\
Calculer \\
Calculer et penser \\
Calculer et raisonner \\
Cartographier \\
Castes et classes \\
Catégories de langue, catégories de pensée \\
Catégories de l'être, catégories de langue \\
Catégories de pensée, catégories de langue \\
Catégories logiques et catégories linguistiques \\
Cause et condition \\
Cause et effet \\
Cause et loi \\
Cause et raison \\
Causes et motivations \\
Causes et raisons \\
Causes premières et causes secondes \\
Ce que je pense est-il nécessairement vrai ? \\
Ce que la morale autorise, l'État peut-il légitimement l'interdire ? \\
Ce que la technique rend possible, peut-on jamais en empêcher la réalisation ? \\
Ce que sait le poète \\
Ce qui dépasse la raison est-il nécessairement irréel ? \\
Ce qui dépend de moi \\
Ce qui est démontré est-il nécessairement vrai ? \\
Ce qui est faux est-il dénué de sens ? \\
Ce qui est ordinaire est-il normal ? \\
Ce qui est subjectif est-il arbitraire ? \\
Ce qui est vrai est-il toujours vérifiable ? \\
Ce qui fut et ce qui sera \\
Ce qu'il y a \\
Ce qui n'a pas de prix \\
Ce qui ne peut s'acheter est-il dépourvu de valeur ? \\
Ce qui n'est pas démontré peut-il être vrai ? \\
Ce qui n'est pas matériel peut-il être réel ? \\
Ce qui n'est pas réel est-il impossible ? \\
Ce qui passe et ce qui demeure \\
Ce qui vaut en théorie vaut-il toujours en pratique ? \\
Ce qu'on ne peut pas vendre \\
Certaines œuvres d'art ont-elles plus de valeur que d'autres ? \\
Certitude et conviction \\
Certitude et probabilité \\
Certitude et vérité \\
Cesser d'espérer \\
C'est pour ton bien \\
C'est trop beau pour être vrai ! \\
Ceux qui savent doivent-ils gouverner ? \\
Chance et bonheur \\
Changer \\
Changer d'opinion \\
Changer, est-ce devenir un autre ? \\
Changer la vie \\
Changer le monde \\
Changer ses désirs plutôt que l'ordre du monde \\
Change-t-on avec le temps ? \\
Chaque science porte-t-elle une métaphysique qui lui est propre ? \\
Châtier, est ce faire honneur au criminel ? \\
Chercher ses mots \\
Chercher son intérêt, est-ce être immoral ? \\
Choisir \\
Choisir, est-ce renoncer ? \\
Choisir ses souvenirs ? \\
Choisissons-nous qui nous sommes ? \\
Choisit-on ses souvenirs ? \\
Choisit-on son corps ? \\
Choix et raison \\
Chose et objet \\
Chose et personne \\
Choses et personnes \\
Cinéma et réalité \\
Cité juste ou citoyen juste ? \\
Citoyen du monde ? \\
Citoyen et soldat \\
Civilisation et barbarie \\
Civilisé, barbare, sauvage \\
Classer \\
Classer et ordonner \\
Classes et histoire \\
Classicisme et romantisme \\
Colère et indignation \\
Collectionner \\
Commander \\
Commémorer \\
Commencer \\
Commencer en philosophie \\
Comment assumer les conséquences de ses actes ? \\
Comment autrui peut-il m'aider à rechercher le bonheur ? \\
Comment bien vivre ? \\
Comment chercher ce qu'on ignore ? \\
Comment comprendre les faits sociaux ? \\
Comment comprendre une croyance qu'on ne partage pas ? \\
Comment conduire ses pensées ? \\
Comment connaître nos devoirs ? \\
Comment croire au progrès ? \\
Comment décider, sinon à la majorité ? \\
Comment définir la raison ? \\
Comment définir la signification \\
Comment deux personnes peuvent-elles partager la même pensée ? \\
Comment devient-on artiste ? \\
Comment devient-on raisonnable ? \\
Comment dire la vérité ? \\
Comment dire l'individuel ? \\
Comment distinguer désirs et besoins ? \\
Comment distinguer entre l'amour et l'amitié ? \\
Comment distinguer l'amour de l'amitié ? \\
Comment distinguer le rêvé du perçu ? \\
Comment distinguer le vrai du faux ? \\
Comment établir des critères d'équité ? \\
Comment être naturel ? \\
Comment évaluer la qualité de la vie ? \\
Comment expliquer les phénomènes mentaux ? \\
Comment exprimer l'identité ? \\
Comment fonder la propriété ? \\
Comment juger de la justesse d'une interprétation ? \\
Comment juger de la politique ? \\
Comment juger d'une œuvre d'art ? \\
Comment justifier l'autonomie des sciences de la vie ? \\
Comment la science progresse-t-elle ? \\
Comment le passé peut-il demeurer présent ? \\
Comment l'erreur est-elle possible ? \\
Comment les sociétés changent-elles ? \\
Comment l'homme peut-il se représenter le temps ? \\
Comment mesurer une sensation ? \\
Comment mesurer ? \\
Comment ne pas être humaniste ? \\
Comment ne pas être libéral ? \\
Comment penser la diversité des langues ? \\
Comment penser le hasard ? \\
Comment penser le mouvement ? \\
Comment penser l'éternel ? \\
Comment penser un pouvoir qui ne corrompe pas ? \\
Comment percevons-nous l'espace ? \\
Comment peut-on choisir entre différentes hypothèses ? \\
Comment peut-on définir la politique ? \\
Comment peut-on définir un être vivant ? \\
Comment peut-on être heureux ? \\
Comment peut-on être sceptique ? \\
Comment peut-on se trahir soi-même ? \\
Comment puis-je devenir ce que je suis ? \\
Comment reconnaît-on une œuvre d'art ? \\
Comment reconnaît-on un vivant ? \\
Comment retrouver la nature ? \\
Comment sait-on qu'on se comprend ? \\
Comment sait-on qu'une chose existe ? \\
Comment savoir que l'on est dans l'erreur ? \\
Comment se mettre à la place d'autrui ? \\
Comment s'entendre ? \\
Comment s'orienter dans la pensée ? \\
Comment traiter les animaux ? \\
Comment trancher une controverse ? \\
Comment vivre ensemble ? \\
Comment voyager dans le temps ? \\
Comme on dit \\
Communauté, collectivité, société \\
Communauté et société \\
Communiquer \\
Communiquer et enseigner \\
Comparaison n'est pas raison \\
Comparer les cultures \\
Compatir \\
Compétence et autorité \\
Composer avec les circonstances \\
Composition et construction \\
Comprendre \\
Comprendre autrui \\
Comprendre, est-ce interpréter ? \\
Comprendre le réel est-ce le dominer ? \\
Comprendre le sens d'un texte \\
Comprendre l'inconscient \\
Comprendre une démonstration \\
Compter sur soi \\
Concept et image \\
Concept et intuition \\
Concept et métaphore \\
Conception et perception \\
Concevoir et juger \\
Conclure \\
Concurrence et égalité \\
Conduire sa vie \\
Conduire ses pensées \\
Conflit et démocratie \\
Conflit et liberté \\
Connaissance commune et connaissance scientifique \\
Connaissance, croyance, conjecture \\
Connaissance de soi et conscience de soi \\
Connaissance du futur et connaissance du passé \\
Connaissance et croyance \\
Connaissance et expérience \\
Connaissance et perception \\
Connaissance historique et action politique \\
Connaissons-nous la réalité des choses ? \\
Connaissons-nous la réalité telle qu'elle est ? \\
Connaissons-nous mieux le présent que le passé ? \\
Connais-toi toi-même \\
Connaît-on la vie ou bien connaît-on le vivant ? \\
Connaît-on la vie ou connaît-on le vivant ? \\
Connaît-on la vie ou le vivant ? \\
Connaît-on les choses telles qu'elles sont ? \\
Connaître autrui \\
Connaître, est-ce connaître par les causes ? \\
Connaître est-ce découvrir le réel ? \\
Connaître, est-ce dépasser les apparences ? \\
Connaître et comprendre \\
Connaître et penser \\
Connaître la vie ou le vivant ? \\
Connaître l'infini \\
Connaître par les causes \\
Connaître ses limites \\
Connaître ses origines \\
Conquérir \\
Conscience de soi et amour de soi \\
Conscience de soi et connaissance de soi \\
Conscience et attention \\
Conscience et connaissance \\
Conscience et conscience de soi \\
Conscience et existence \\
Conscience et liberté \\
Conscience et mémoire \\
Conscience et responsabilité \\
Conscience et subjectivité \\
Conscience et volonté \\
Conseiller le prince \\
Consensus et conflit \\
Conservatisme et tradition \\
Considère-t-on jamais le temps en lui-même ? \\
Consistance et précarité \\
Constitution et lois \\
Construire l'espace \\
Consumérisme et démocratie \\
Contemplation et distraction \\
Contempler \\
Contingence et nécessité \\
Continuité et discontinuité \\
Contradiction et opposition \\
Contrainte et désobéissance \\
Contrainte et obligation \\
Contrôle et vigilance \\
Convaincre et persuader \\
Convention et observation \\
Conventions sociales et moralité \\
Conviction et certitude \\
Conviction et responsabilité \\
Convient-il d'opposer explication et interprétation ? \\
Corps et espace \\
Corps et esprit \\
Corps et identité \\
Corps et matière \\
Corps et nature \\
Correspondre \\
Crainte et espoir \\
Création et fabrication \\
Création et production \\
Créativité et contrainte \\
Créer \\
Créer et produire \\
Crime et châtiment \\
Crise et progrès \\
Critiquer \\
Critiquer la démocratie \\
Croire aux fictions \\
Croire en Dieu \\
Croire, est-ce être faible ? \\
Croire, est-ce obéir ? \\
Croire, est-ce renoncer au savoir ? \\
Croire et savoir \\
Croire pour savoir \\
Croire que Dieu existe, est-ce croire en lui ? \\
Croire savoir \\
Croit-on ce que l'on veut ? \\
Croit-on comme on veut ? \\
Croyance et certitude \\
Croyance et choix \\
Croyance et connaissance \\
Croyance et probabilité \\
Croyance et vérité \\
Culpabilité et responsabilité \\
Cultes et rituels \\
Cultivons notre jardin \\
Culture et civilisation \\
Culture et communauté \\
Culture et conscience \\
Culture et différence \\
Culture et éducation \\
Culture et langage \\
Culture et savoir \\
Culture et technique \\
Culture et violence \\
Dans l'action, est-ce l'intention qui compte ? \\
Dans quel but les hommes se donnent-ils des lois ? \\
Dans quelle mesure est-on l'auteur de sa propre vie ? \\
Dans quelle mesure l'art est-il un fait social ? \\
Dans quelle mesure le temps nous appartient-il ? \\
Dans quelle mesure toute philosophie est-elle critique du langage ? \\
Débattre et dialoguer \\
Déchiffrer \\
Décider \\
Décomposer les choses \\
Découverte et invention \\
Découverte et invention dans les sciences \\
Découverte et justification \\
Découvrir \\
Décrire \\
Décrire, est-ce déjà expliquer ? \\
Déduction et expérience \\
Défendre son honneur \\
Définir \\
Définir, est-ce déterminer l'essence ? \\
Définir l'art : à quoi bon ? \\
Définir la vérité, est-ce la connaître ? \\
Définition et démonstration \\
Définition nominale et définition réelle \\
Définitions, axiomes, postulats \\
Déjouer \\
Délibérer, est-ce être dans l'incertitude ? \\
De l'utilité des voyages \\
Dématérialiser \\
Démêler le vrai du faux \\
Démériter \\
Démocrates et démagogues \\
Démocratie ancienne et démocratie moderne \\
Démocratie et anarchie \\
Démocratie et démagogie \\
Démocratie et impérialisme \\
Démocratie et opinion \\
Démocratie et religion \\
Démocratie et représentation \\
Démocratie et république \\
Démocratie et transparence \\
Démonstration et argumentation \\
Démonstration et déduction \\
Démontrer, argumenter, expérimenter \\
Démontrer est-il le privilège du mathématicien ? \\
Démontrer et argumenter \\
Démontrer par l'absurde \\
Dénaturer \\
Dépasser les apparences ? \\
Dépasser l'humain \\
Dépend-il de soi d'être heureux ? \\
De quel bonheur sommes-nous capables ? \\
De quel droit l'État exerce-t-il un pouvoir ? \\
De quel droit punit-on ? \\
De quel droit ? \\
De quelle certitude la science est-elle capable ? \\
De quelle liberté témoigne l'œuvre d'art ? \\
De quelle réalité nos perceptions témoignent-elles ? \\
De quelle réalité témoignent nos perceptions ? \\
De quelle science humaine la folie peut-elle être l'objet ? \\
De quelle vérité l'art est-il capable ? \\
De quelle vérité l'opinion est-elle capable ? \\
De quoi a-t-on conscience lorsqu'on a conscience de soi ? \\
De quoi avons-nous besoin ? \\
De quoi avons-nous vraiment besoin ? \\
De quoi dépend le bonheur ? \\
De quoi dépend notre bonheur ? \\
De quoi doute un sceptique ? \\
De quoi est fait mon présent ? \\
De quoi est-fait notre présent ? \\
De quoi est-on conscient ? \\
De quoi est-on malheureux ? \\
De quoi la forme est-elle la forme ? \\
De quoi la logique est-elle la science ? \\
De quoi la philosophie est-elle le désir ? \\
De quoi l'art nous délivre-t-il ? \\
De quoi la vérité libère-t-elle ? \\
De quoi le devoir libère-t-il ? \\
De quoi les logiciens parlent-ils ? \\
De quoi les métaphysiciens parlent-ils ? \\
De quoi les sciences humaines nous instruisent-elles ? \\
De quoi l'État doit-il être propriétaire ? \\
De quoi l'État ne doit-il pas se mêler ? \\
De quoi l'expérience esthétique est-elle l'expérience ? \\
De quoi n'avons-nous pas conscience ? \\
De quoi ne peut-on pas répondre ? \\
De quoi parlent les mathématiques ? \\
De quoi parlent les théories physiques ? \\
De quoi pâtit-on ? \\
De quoi peut-il y avoir science ? \\
De quoi peut-on être inconscient ? \\
De quoi peut-on faire l'expérience ? \\
De quoi pouvons-nous être sûrs ? \\
De quoi puis-je répondre ? \\
De quoi rit-on ? \\
De quoi somme-nous prisonniers ? \\
De quoi sommes-nous coupables ? \\
De quoi sommes-nous responsables ? \\
De quoi suis-je inconscient ? \\
De quoi suis-je responsable ? \\
De quoi y a-t-il expérience ? \\
De quoi y a-t-il histoire ? \\
Déraisonner \\
Déraisonner, est-ce perdre de vue le réel ? \\
Désacraliser \\
Des comportements économiques peuvent-ils être rationnels ? \\
Description et explication \\
Des événements aléatoires peuvent-ils obéir à des lois ? \\
Des inégalités peuvent-elles être justes ? \\
Désintérêt et désintéressement \\
Désirer \\
Désirer, est-ce être aliéné ? \\
Désirer et vouloir \\
Désir et besoin \\
Désir et bonheur \\
Désir et interdit \\
Désir et langage \\
Désir et manque \\
Désire-t-on la reconnaissance ? \\
Désir et ordre \\
Désir et politique \\
Désir et pouvoir \\
Désir et raison \\
Désir et réalité \\
Désir et volonté \\
Des lois justes suffisent-elles à assurer la justice ? \\
Des motivations peuvent-elles être sociales ? \\
Des nations peuvent-elles former une société ? \\
Désobéir \\
Désobéir aux lois \\
Désobéissance et résistance \\
Des peuples sans histoire \\
Des sociétés sans État sont-elles des sociétés politiques ? \\
Déterminisme et responsabilité \\
Déterminisme psychique et déterminisme physique \\
Détruire et construire \\
Détruire pour reconstruire \\
Devant qui est-on responsable ? \\
Devant qui sommes-nous responsables ? \\
Devenir autre \\
Devenir citoyen \\
Devenir et évolution \\
Devient-on raisonnable ? \\
Devoir et bonheur \\
Devoir et conformisme \\
Devoir et contrainte \\
Devoir et intérêt \\
Devoir et liberté \\
Devoir et plaisir \\
Devoir et prudence \\
Devoir et vertu \\
Devoirs envers les autres et devoirs envers soi-même \\
Devoirs et passions \\
Devons-nous dire la vérité ? \\
Devons-nous nous libérer de nos désirs ? \\
Devons-nous quelque chose à la nature ? \\
Devons-nous tenir certaines connaissances pour acquises ? \\
Devons-nous vivre comme si nous ne devions jamais mourir ? \\
Dialectique et Philosophie \\
Dialogue et délibération en démocratie \\
Dialoguer \\
Dieu aurait-il pu mieux faire ? \\
Dieu est-il mortel ? \\
Dieu est-il une invention humaine ? \\
Dieu est-il une limite de la pensée ? \\
Dieu est mort \\
Dieu et César \\
Dieu pense-t-il ? \\
Dieu peut-il tout faire ? \\
Dieu, prouvé ou éprouvé ? \\
Dieu tout-puissant \\
Dire ce qui est \\
Dire, est-ce faire ? \\
Dire et exprimer \\
Dire et faire \\
Dire et montrer \\
Dire je \\
Dire le monde \\
Dire l'individuel \\
Dire oui \\
Dire « je » \\
Diriger son esprit \\
Discrimination et revendication \\
Discussion et conversation \\
Discussion et dialogue \\
Disposer de son corps \\
Distinguer \\
Division du travail et cohésion sociale \\
Documents et monuments \\
Dogme et opinion \\
Dois-je mériter mon bonheur ? \\
Dois-je obéir à la loi ? \\
Doit-on apprendre à percevoir ? \\
Doit-on apprendre à vivre ? \\
Doit-on bien juger pour bien faire ? \\
Doit-on changer ses désirs, plutôt que l'ordre du monde ? \\
Doit-on chasser les artistes de la cité ? \\
Doit-on corriger les inégalités sociales ? \\
Doit-on croire en l'humanité ? \\
Doit-on distinguer devoir moral et obligation sociale ? \\
Doit-on identifier l'âme à la conscience ? \\
Doit-on interpréter les rêves ? \\
Doit-on justifier les inégalités ? \\
Doit-on le respect au vivant ? \\
Doit-on mûrir pour la liberté ? \\
Doit-on rechercher le bonheur ? \\
Doit-on rechercher l'harmonie ? \\
Doit-on refuser d'interpréter ? \\
Doit-on répondre de ce qu'on est devenu ? \\
Doit-on respecter la nature ? \\
Doit-on respecter les êtres vivants ? \\
Doit-on se faire l'avocat du diable ? \\
Doit-on se justifier d'exister ? \\
Doit-on se mettre à la place d'autrui ? \\
Doit-on se passer des utopies ? \\
Doit-on souffrir de n'être pas compris ? \\
Doit-on tenir le plaisir pour une fin ? \\
Doit-on toujours dire la vérité ? \\
Doit-on toujours rechercher la vérité ? \\
Doit-on tout accepter de l'État ? \\
Doit-on tout attendre de l'État ? \\
Doit-on tout calculer ? \\
Doit-on tout contrôler ? \\
Dominer la nature \\
Don et échange \\
Don Juan \\
Donner \\
Donner à chacun son dû \\
Donner, à quoi bon ? \\
Donner des exemples \\
Donner des preuves \\
Donner des raisons \\
Donner du sens \\
Donner et recevoir \\
Donner raison \\
Donner raison, rendre raison \\
Donner sa parole \\
Donner son assentiment \\
Donner une représentation \\
Donner un exemple \\
D'où la politique tire-t-elle sa légitimité ? \\
Doute et raison \\
Douter \\
D'où viennent les concepts ? \\
D'où viennent les idées générales ? \\
D'où viennent les préjugés ? \\
D'où viennent nos idées ? \\
D'où vient aux objets techniques leur beauté ? \\
D'où vient la certitude dans les sciences ? \\
D'où vient la certitude ? \\
D'où vient la servitude ? \\
D'où vient la signification des mots ? \\
D'où vient le mal ? \\
D'où vient le plaisir de lire ? \\
D'où vient que l'histoire soit autre chose qu'un chaos ? \\
Dressage et éducation \\
Droit et coutume \\
Droit et démocratie \\
Droit et devoir \\
Droit et devoir sont-ils liés ? \\
Droit et morale \\
Droit et protection \\
Droit et violence \\
Droit naturel et loi naturelle \\
Droits de l'homme et droits du citoyen \\
Droits de l'homme ou droits du citoyen ? \\
Droits et devoirs \\
Droits et devoirs sont-ils réciproques ? \\
Droits, garanties, protection \\
Du passé pouvons-nous faire table rase ? \\
Durée et instant \\
Durer \\
Échange et don \\
Échange et partage \\
Échange et valeur \\
Échanger \\
Échanger des idées \\
Échanger, est-ce créer de la valeur ? \\
Échanger, est-ce partager ? \\
Échanger, est-ce risquer ? \\
Éclairer \\
Économie et politique \\
Économie et société \\
Économie politique et politique économique \\
Écouter \\
Écouter et entendre \\
Écrire \\
Écrire et parler \\
Écrire l'histoire \\
Éducation de l'homme, éducation du citoyen \\
Éducation et instruction \\
Éduquer et instruire \\
Éduquer le citoyen \\
Efficacité et justice \\
Égalité des droits, égalité des conditions \\
Égalité et différence \\
Égalité et solidarité \\
Égoïsme et altruisme \\
Égoïsme et individualisme \\
Égoïsme et méchanceté \\
Empirique et expérimental \\
Enfance et moralité \\
En finir avec les préjugés \\
En histoire, tout est-il affaire d'interprétation ? \\
En morale, est-ce seulement l'intention qui compte ? \\
En politique, faut-il refuser l'utopie ? \\
En politique, nécessité fait loi \\
En politique, ne faut-il croire qu'aux rapports de force ? \\
En politique n'y a-t-il que des rapports de force ? \\
En politique, peut-on faire table rase du passé ? \\
En politique, y a-t-il des modèles ? \\
En quel sens la métaphysique a-t-elle une histoire ? \\
En quel sens la métaphysique est-elle une science ? \\
En quel sens l'anthropologie peut-elle être historique ? \\
En quel sens les sciences ont-elles une histoire ? \\
En quel sens l'État est-il rationnel ? \\
En quel sens le vivant a-t-il une histoire ? \\
En quel sens parler de lois de la pensée ? \\
En quel sens parler de structure métaphysique ? \\
En quel sens parler d'identité culturelle ? \\
En quel sens peut-on dire que la vérité s'impose ? \\
En quel sens peut-on dire que le mal n'existe pas ? \\
En quel sens peut-on dire que l'homme est un animal politique ? \\
En quel sens peut-on dire qu' « on expérimente avec sa raison » ? \\
En quel sens peut-on parler de la mort de l'art ? \\
En quel sens peut-on parler de la vie sociale comme d'un jeu ? \\
En quel sens peut-on parler de transcendance ? \\
En quel sens peut-on parler d'expérience possible ? \\
En quel sens peut-on parler d'une culture technique ? \\
En quel sens peut-on parler d'une interprétation de la nature ? \\
En quel sens une œuvre d'art est-elle un document ? \\
Enquêter \\
En quoi la connaissance de la matière peut-elle relever de la métaphysique ? \\
En quoi la connaissance du vivant contribue-t-elle à la connaissance de l'homme ? \\
En quoi la justice met-elle fin à la violence ? \\
En quoi la matière s'oppose-t-elle à l'esprit ? \\
En quoi la méthode est-elle un art de penser ? \\
En quoi la nature constitue-t-elle un modèle ? \\
En quoi la patience est-elle une vertu ? \\
En quoi la physique a-t-elle besoin des mathématiques ? \\
En quoi l'art peut-il intéresser le philosophe ? \\
En quoi la sociologie est-elle fondamentale ? \\
En quoi la technique fait-elle question ? \\
En quoi le bien d'autrui m'importe-t-il ? \\
En quoi le bonheur est-il l'affaire de l'État ? \\
En quoi le langage est-il constitutif de l'homme ? \\
En quoi les hommes restent-ils des enfants ? \\
En quoi les sciences humaines nous éclairent-elles sur la barbarie ? \\
En quoi les sciences humaines sont-elles normatives ? \\
En quoi les vivants témoignent-ils d'une histoire ? \\
En quoi l'œuvre d'art donne-t-elle à penser ? \\
En quoi une discussion est-elle politique ? \\
En quoi une insulte est-elle blessante ? \\
En quoi une œuvre d'art est-elle moderne ? \\
Enseigner \\
Enseigner, est-ce transmettre un savoir ? \\
Enseigner et éduquer \\
Enseigner, instruire, éduquer \\
Enseigner l'art \\
Entendement et raison \\
Entendre \\
Entendre raison \\
Entre l'art et la nature, qui imite l'autre ? \\
Entre l'opinion et la science, n'y a-t-il qu'une différence de degré ? \\
Entrer en scène \\
Énumérer \\
Épistémologie générale et épistémologie des sciences particulières \\
Éprouver sa valeur \\
Erreur et faute \\
Erreur et illusion \\
Espace et représentation \\
Espace et structure sociale \\
Espace mathématique et espace physique \\
Espace public et vie privée \\
Esprit et intériorité \\
Essayer \\
Essence et existence \\
Est-ce à la fin que le sens apparaît ? \\
Est-ce à la raison de déterminer ce qui est réel ? \\
Est-ce de la force que l'État tient son autorité ? \\
Est-ce la certitude qui fait la science ? \\
Est-ce la démonstration qui fait la science ? \\
Est-ce la majorité qui doit décider ? \\
Est-ce la mémoire qui constitue mon identité ? \\
Est-ce l'autorité qui fait la loi ? \\
Est-ce le cerveau qui pense ? \\
Est-ce l'échange utilitaire qui fait le lien social ? \\
Est-ce le corps qui perçoit ? \\
Est-ce l'ignorance qui rend les hommes croyants ? \\
Est-ce l'intérêt qui fonde le lien social ? \\
Est-ce l'utilité qui définit un objet technique ? \\
Est-ce par désir de la vérité que l'homme cherche à savoir ? \\
Est-ce par son objet ou par ses méthodes qu'une science peut se définir ? \\
Est-ce pour des raisons morales qu'il faut protéger l'environnement ? \\
Est-ce seulement l'intention qui compte ? \\
Est-ce un devoir d'aimer son prochain ? \\
Esthétique et éthique \\
Esthétique et poétique \\
Esthétisme et moralité \\
Est-il bon qu'un seul commande ? \\
Est-il difficile de savoir ce que l'on veut ? \\
Est-il difficile de savoir ce qu'on veut ? \\
Est-il difficile d'être heureux ? \\
Est-il difficile de vivre en société ? \\
Est-il immoral de se rendre heureux ? \\
Est-il judicieux de revenir sur ses décisions ? \\
Est-il juste de payer l'impôt ? \\
Est-il juste d'interpréter la loi ? \\
Est-il légitime d'affirmer que seul le présent existe ? \\
Est-il légitime d'opposer liberté et nécessité ? \\
Est-il mauvais de suivre son désir ? \\
Est-il naturel à l'homme de parler ? \\
Est-il naturel de s'aimer soi-même ? \\
Est-il nécessaire d'espérer pour entreprendre ? \\
Est-il parfois bon de mentir ? \\
Est-il possible d'améliorer l'homme ? \\
Est-il possible de croire en la vie éternelle ? \\
Est-il possible de douter de tout ? \\
Est-il possible de ne croire à rien ? \\
Est-il possible de préparer l'avenir ? \\
Est-il possible de tout avoir pour être heureux ? \\
Est-il possible d'être immoral sans le savoir ? \\
Est-il possible d'être neutre politiquement ? \\
Est-il raisonnable d'aimer ? \\
Est-il raisonnable d'être rationnel ? \\
Est-il raisonnable de vouloir maîtriser la nature ? \\
Est-il toujours avantageux de promouvoir son propre intérêt ? \\
Est-il toujours meilleur d'avoir le choix ? \\
Est-il utile d'avoir mal ? \\
Est-il vrai que les animaux ne pensent pas ? \\
Est-il vrai que l'ignorant n'est pas libre ? \\
Est-il vrai que ma liberté s'arrête là où commence celle des autres ? \\
Est-il vrai que nous ne nous tenons jamais au temps présent ? \\
Est-il vrai qu'en science, « rien n'est donné, tout est construit » ? \\
Est-il vrai que plus on échange, moins on se bat ? \\
Est-il vrai qu'on apprenne de ses erreurs ? \\
Estime et respect \\
Estimer \\
Est-on fondé à distinguer la justice et le droit ? \\
Est-on l'auteur de sa propre vie ? \\
Est-on le produit d'une culture ? \\
Est-on libre de ne pas vouloir ce que l'on veut ? \\
Est-on libre face à la vérité ? \\
Est-on responsable de ce qu'on n'a pas voulu ? \\
Est-on responsable de l'avenir de l'humanité \\
Est-on responsable de son passé ? \\
Est-on sociable par nature ? \\
Établir la vérité, est-ce nécessairement démontrer ? \\
État et institutions \\
État et nation \\
État et société \\
État et Société \\
État et société civile \\
Éternité et immortalité \\
Éthique et authenticité \\
Éthique et esthétique \\
Éthique et Morale \\
Ethnologie et cinéma \\
Ethnologie et ethnocentrisme \\
Ethnologie et sociologie \\
Étonnement et sidération \\
Être acteur \\
Être affairé \\
Être à l'écoute de son désir, est-ce nier le désir de l'autre ? \\
Être aliéné \\
Être au monde \\
Être bon juge \\
Être cause de soi \\
Être, c'est agir \\
Être chez soi \\
Être citoyen \\
Être citoyen du monde \\
Être compris \\
Être conscient de soi, est-ce être maître de soi ? \\
Être conscient, est-ce être maître de soi ? \\
Être conséquent avec soi-même \\
Être content de soi \\
Être cultivé, est-ce tout connaître ? \\
Être cultivé rend-il meilleur ? \\
Être cynique \\
Être dans l'esprit \\
Être dans le temps \\
Être dans son bon droit \\
Être dans son droit \\
Être de mauvaise humeur \\
Être de son temps \\
Être déterminé \\
Être dogmatique \\
Être égal à soi-même \\
Être en bonne santé \\
Être en désaccord \\
Être en règle avec soi-même \\
Être ensemble \\
Être équitable \\
Être est-ce agir ? \\
Être et apparaître \\
Être et avoir \\
Être et avoir été \\
Être et devenir \\
Être et devoir être \\
Être et devoir-être \\
Être et être pensé \\
Être et exister \\
Être et ne plus être \\
Être et paraître \\
Être et penser, est-ce la même chose ? \\
Être et représentation \\
Être et sens \\
Être exemplaire \\
Être heureux \\
Être heureux, est-ce devoir ? \\
Être hors de soi \\
Être impossible \\
Être juge et partie \\
Être là \\
Être l'entrepreneur de soi-même \\
Être libre, est-ce dire non ? \\
Être libre est-ce faire ce que l'on veut ? \\
Être libre, est-ce n'obéir qu'à soi-même ? \\
Être libre, est-ce pouvoir choisir ? \\
Être libre, est-ce se suffire à soi-même ? \\
Être libre, même dans les fers \\
Être logique \\
Être logique avec soi-même \\
Être maître de soi \\
Être majeur \\
Être malade \\
Être matérialiste \\
Être méchant \\
Être méchant volontairement \\
Être mère \\
Être moderne \\
Être né \\
Être ou avoir \\
Être ou ne pas être \\
Être ou ne pas être, est-ce la question ? \\
Être par soi \\
Être pauvre \\
Être père \\
Être précurseur \\
Être quelqu'un \\
Être raisonnable, est-ce accepter la réalité telle qu'elle est ? \\
Être raisonnable, est-ce renoncer à ses désirs ? \\
Être réaliste \\
Être relativiste \\
Être sans cause \\
Être sans cœur \\
Être sans scrupule \\
Être sceptique \\
Être seul avec sa conscience \\
Être seul avec soi-même \\
Être soi \\
Être soi-même \\
Être spectateur \\
Être spirituel \\
Être systématique \\
Être un artiste \\
Être un corps \\
Être une chose qui pense \\
Être un sujet, est-ce être maître de soi ? \\
Être vertueux \\
Être, vie et pensée \\
Étudier \\
Évidence et certitude \\
Évidence et raison \\
Évidence et vérité \\
Évidences et préjugés \\
Évolution biologique et culture \\
Évolution et progrès \\
Évolution et révolution \\
Excuser et pardonner \\
Existence et contingence \\
Existence et essence \\
Exister \\
Exister, est-ce simplement vivre ? \\
Existe-t-il de faux besoins ? \\
Existe-t-il des choses en soi ? \\
Existe-t-il des choses sans prix ? \\
Existe-t-il des croyances collectives ? \\
Existe-t-il des désirs coupables ? \\
Existe-t-il des devoirs envers soi-même ? \\
Existe-t-il des dilemmes moraux ? \\
Existe-t-il des questions sans réponse ? \\
Existe-t-il des sciences de différentes natures ? \\
Existe-t-il des signes naturels ? \\
Existe-t-il un art de penser ? \\
Existe-t-il un bien commun qui soit la norme de la vie politique ? \\
Existe-t-il un droit de mentir ? \\
Existe-t-il une méthode pour rechercher la vérité ? \\
Existe-t-il une méthode pour trouver la vérité ? \\
Existe-t-il une opinion publique ? \\
Existe-t-il un vocabulaire neutre des droits fondamentaux ? \\
Expérience esthétique et sens commun \\
Expérience et approximation \\
Expérience et expérimentation \\
Expérience et habitude \\
Expérience et interprétation \\
Expérience et phénomène \\
Expérience et vérité \\
Expérience, expérimentation \\
Expérience immédiate et expérimentation scientifique \\
Expérimentation et vérification \\
Expérimenter \\
Explication et prévision \\
Expliquer \\
Expliquer, est-ce interpréter ? \\
Expliquer et comprendre \\
Expliquer et interpréter \\
Expliquer et justifier \\
Expression et création \\
Expression et signification \\
Extension et compréhension \\
Fabriquer et créer \\
Faire apprendre \\
Faire ce que l'on dit \\
Faire ce qu'on dit \\
Faire comme si \\
Faire confiance \\
Faire corps \\
Faire de la métaphysique, est-ce se détourner du monde ? \\
Faire de la politique \\
Faire de nécessité vertu \\
Faire de sa vie une œuvre d'art \\
Faire des choix \\
Faire douter \\
Faire école \\
Faire et laisser faire \\
Faire justice \\
Faire la loi \\
Faire la morale \\
Faire la paix \\
Faire la part des choses \\
Faire la révolution \\
Faire le mal \\
Faire l'histoire \\
Faire son devoir \\
Faire son devoir, est-ce là toute la morale ? \\
Faire table rase \\
Faire une expérience \\
Faire voir \\
Faisons-nous l'histoire ? \\
Fait et essence \\
Fait et fiction \\
Fait et preuve \\
Fait et théorie \\
Fait et valeur \\
Fait-on de la politique pour changer les choses ? \\
Faits et preuves \\
Faits et valeurs \\
Famille et tribu \\
Familles, je vous hais \\
Faudrait-il ne rien oublier ? \\
Faudrait-il vivre sans passion ? \\
Faut-avoir peur de la technique ? \\
Faut-il accepter sa condition ? \\
Faut-il accorder de l'importance aux mots ? \\
Faut-il accorder l'esprit aux bêtes ? \\
Faut-il affirmer son identité ? \\
Faut-il aimer autrui pour le respecter ? \\
Faut-il aimer la vie ? \\
Faut-il aimer son prochain comme soi-même ? \\
Faut-il aimer son prochain ? \\
Faut-il aller au-delà des apparences ? \\
Faut-il aller toujours plus vite ? \\
Faut-il apprendre à être libre ? \\
Faut-il apprendre à vivre en renonçant au bonheur ? \\
Faut-il apprendre à voir ? \\
Faut-il avoir des ennemis ? \\
Faut-il avoir des principes ? \\
Faut-il avoir peur de la liberté ? \\
Faut-il avoir peur de la nature ? \\
Faut-il avoir peur de la technique ? \\
Faut-il avoir peur des habitudes ? \\
Faut-il avoir peur des machines ? \\
Faut-il avoir peur d'être libre ? \\
Faut-il avoir peur du désordre ? \\
Faut-il changer le monde ? \\
Faut-il changer ses désirs plutôt que l'ordre du monde ? \\
Faut-il chasser les poètes ? \\
Faut-il chercher à satisfaire tous nos désirs ? \\
Faut-il chercher à se connaître ? \\
Faut-il chercher la paix à tout prix ? \\
Faut-il chercher le bonheur à tout prix ? \\
Faut-il chercher un sens à l'histoire ? \\
Faut-il choisir entre être heureux et être libre ? \\
Faut-il concilier les contraires ? \\
Faut-il condamner la fiction ? \\
Faut-il condamner la rhétorique ? \\
Faut-il condamner le luxe ? \\
Faut-il condamner les illusions ? \\
Faut-il connaître l'Histoire pour gouverner ? \\
Faut-il considérer le droit pénal comme instituant une violence légitime ? \\
Faut-il considérer les faits sociaux comme des choses ? \\
Faut-il contrôler les mœurs ? \\
Faut-il craindre la mort ? \\
Faut-il craindre la révolution ? \\
Faut-il craindre le développement des techniques ? \\
Faut-il craindre le pire ? \\
Faut-il craindre le regard d'autrui ? \\
Faut-il craindre les foules ? \\
Faut-il craindre les machines ? \\
Faut-il craindre les masses ? \\
Faut-il craindre l'État ? \\
Faut-il craindre l'ordre ? \\
Faut-il croire au progrès ? \\
Faut-il croire en la science ? \\
Faut-il croire en quelque chose ? \\
Faut-il croire les historiens ? \\
Faut-il croire que l'histoire a un sens ? \\
Faut-il défendre la démocratie ? \\
Faut-il défendre l'ordre à tout prix ? \\
Faut-il défendre ses convictions \\
Faut-il dépasser les apparences ? \\
Faut-il désespérer de l'humanité ? \\
Faut-il des frontières ? \\
Faut-il des héros ? \\
Faut-il désirer la vérité ? \\
Faut-il des outils pour penser ? \\
Faut-il détruire l'État ? \\
Faut-il détruire pour créer ? \\
Faut-il dire de la justice qu'elle n'existe pas ? \\
Faut-il dire tout haut ce que les autres pensent tout bas ? \\
Faut-il diriger l'économie ? \\
Faut-il distinguer ce qui est de ce qui doit être ? \\
Faut-il distinguer désir et besoin ? \\
Faut-il distinguer esthétique et philosophie de l'art ? \\
Faut-il donner un sens à la souffrance ? \\
Faut-il douter de ce qu'on ne peut pas démontrer ? \\
Faut-il douter de l'évidence \\
Faut-il du passé faire table rase ? \\
Faut-il enfermer ? \\
Faut-il espérer pour agir ? \\
Faut-il être à l'écoute du corps ? \\
Faut-il être bon ? \\
Faut-il être cohérent ? \\
Faut-il être connaisseur pour apprécier une œuvre d'art ? \\
Faut-il être cosmopolite ? \\
Faut-il être fidèle à soi-même ? \\
Faut-il être idéaliste ? \\
Faut-il être libre pour être heureux ? \\
Faut-il être logique avec soi-même ? \\
Faut-il être mesuré en toutes choses ? \\
Faut-il être modéré ? \\
Faut-il être objectif ? \\
Faut-il être original ? \\
Faut-il être positif ? \\
Faut-il être pragmatique ? \\
Faut-il être réaliste en politique ? \\
Faut-il être réaliste ? \\
Faut-il expliquer la morale par son utilité ? \\
Faut-il faire confiance au progrès technique ? \\
Faut-il faire de nécessité vertu ? \\
Faut-il faire table rase du passé ? \\
Faut-il forcer les gens à participer à la vie politique ? \\
Faut-il fuir la politique ? \\
Faut-il garder ses illusions ? \\
Faut-il hiérarchiser les désirs ? \\
Faut-il hiérarchiser les formes de vie ? \\
Faut-il imaginer que nous sommes heureux ? \\
Faut-il imposer la vérité ? \\
Faut-il interpréter la loi ? \\
Faut-il laisser parler la nature ? \\
Faut-il libérer l'humanité du travail ? \\
Faut-il limiter la souveraineté de l'État ? \\
Faut-il limiter la souveraineté ? \\
Faut-il limiter le pouvoir de l'État ? \\
Faut-il limiter l'exercice de la puissance publique ? \\
Faut-il lire des romans ? \\
Faut-il ménager les apparences ? \\
Faut-il mépriser le luxe ? \\
Faut-il mieux vivre comme si nous ne devions jamais mourir ? \\
Faut-il ne manquer de rien pour être heureux ? \\
Faut-il n'être jamais méchant ? \\
Faut-il obéir à la voix de sa conscience ? \\
Faut-il opposer à la politique la souveraineté du droit ? \\
Faut-il opposer histoire et mémoire ? \\
Faut-il opposer la matière et l'esprit ? \\
Faut-il opposer l'art à la connaissance ? \\
Faut-il opposer la théorie et la pratique ? \\
Faut-il opposer le don et l'échange ? \\
Faut-il opposer l'État et la société ? \\
Faut-il opposer le temps vécu et le temps des choses ? \\
Faut-il opposer l'histoire et la fiction ? \\
Faut-il opposer nature et culture ? \\
Faut-il opposer raison et sensation ? \\
Faut-il opposer rhétorique et philosophie ? \\
Faut-il oublier le passé pour se donner un avenir ? \\
Faut-il parler pour avoir des idées générales ? \\
Faut-il partager la souveraineté ? \\
Faut-il penser l'État comme un corps ? \\
Faut-il perdre ses illusions ? \\
Faut-il perdre son temps ? \\
Faut-il poser des limites à l'activité rationnelle ? \\
Faut-il pour le connaître faire du vivant un objet ? \\
Faut-il préférer l'art à la nature ? \\
Faut-il préférer le bonheur à la vérité ? \\
Faut-il préférer une injustice au désordre ? \\
Faut-il prendre soin de soi ? \\
Faut-il protéger la dignité humaine ? \\
Faut-il protéger la nature ? \\
Faut-il protéger les faibles contre les forts ? \\
Faut-il que le réel ait un sens ? \\
Faut-il que les meilleurs gouvernent ? \\
Faut-il rechercher la certitude ? \\
Faut-il rechercher la simplicité ? \\
Faut-il rechercher le bonheur ? \\
Faut-il rechercher l'harmonie ? \\
Faut-il reconnaître pour connaître ? \\
Faut-il regretter l'équivocité du langage ? \\
Faut-il rejeter tous les préjugés ? \\
Faut-il rejeter toute norme ? \\
Faut-il renoncer à faire du travail une valeur ? \\
Faut-il renoncer à la certitude ? \\
Faut-il renoncer à l'idée d'âme ? \\
Faut-il renoncer à l'impossible ? \\
Faut-il renoncer à son désir ? \\
Faut-il résister à la peur de mourir ? \\
Faut-il respecter la nature ? \\
Faut-il respecter les convenances ? \\
Faut-il respecter le vivant ? \\
Faut-il rester impartial ? \\
Faut-il rester naturel ? \\
Faut-il rire ou pleurer ? \\
Faut-il rompre avec le passé ? \\
Faut-il s'adapter ? \\
Faut-il s'affranchir des désirs ? \\
Faut-il s'aimer soi-même ? \\
Faut-il sauver des vies à tout prix ? \\
Faut-il sauver les apparences ? \\
Faut-il savoir mentir ? \\
Faut-il savoir obéir pour gouverner ? \\
Faut-il savoir pour agir ? \\
Faut-il savoir prendre des risques ? \\
Faut-il se contenter de peu ? \\
Faut-il se cultiver ? \\
Faut-il se délivrer de la peur ? \\
Faut-il se délivrer des passions ? \\
Faut-il se détacher du monde ? \\
Faut-il s'efforcer d'être moins personnel ? \\
Faut-il se fier à ce que l'on ressent ? \\
Faut-il se fier à la majorité ? \\
Faut-il se fier à sa propre raison ? \\
Faut-il se fier aux apparences ? \\
Faut-il se libérer du travail ? \\
Faut-il se méfier de l'écriture ? \\
Faut-il se méfier de l'imagination ? \\
Faut-il se méfier de l'intuition ? \\
Faut-il se méfier des apparences ? \\
Faut-il se méfier de ses désirs ? \\
Faut-il se méfier du volontarisme politique ? \\
Faut-il s'en remettre à l'État pour limiter le pouvoir de l'État ? \\
Faut-il s'en tenir aux faits ? \\
Faut-il séparer la science et la technique ? \\
Faut-il séparer morale et politique ? \\
Faut-il se poser des questions métaphysiques ? \\
Faut-il se réjouir d'exister ? \\
Faut-il se rendre à l'évidence ? \\
Faut-il se ressembler pour former une société ? \\
Faut-il suivre ses intuitions ? \\
Faut-il surmonter son enfance ? \\
Faut-il tolérer les intolérants ? \\
Faut-il toujours avoir raison ? \\
Faut-il toujours dire la vérité ? \\
Faut-il toujours être en accord avec soi-même ? \\
Faut-il toujours éviter de se contredire ? \\
Faut-il toujours faire son devoir ? \\
Faut-il toujours garder espoir ? \\
Faut-il tout critiquer ? \\
Faut-il tout démontrer ? \\
Faut-il tout interpréter ? \\
Faut-il un commencement à tout ? \\
Faut-il un corps pour penser ? \\
Faut-il une guerre pour mettre fin à toutes les guerres ? \\
Faut-il une théorie de la connaissance ? \\
Faut-il vaincre ses désirs plutôt que l'ordre du monde ? \\
Faut-il vivre avec son temps ? \\
Faut-il vivre comme si l'on ne devait jamais mourir ? \\
Faut-il vivre comme si nous étions immortels ? \\
Faut-il vivre comme si nous ne devions jamais mourir ? \\
Faut-il vivre comme si on ne devait jamais mourir ? \\
Faut-il vivre dangereusement ? \\
Faut-il vivre hors de la société pour être heureux ? \\
Faut-il voir pour croire ? \\
Faut-il vouloir changer le monde ? \\
Faut-il vouloir être heureux ? \\
Faut-il vouloir la paix de l'âme ? \\
Faut-il vouloir la paix ? \\
Faut-il vouloir la transparence ? \\
Fiction et virtualité \\
Foi et bonne foi \\
Foi et raison \\
Foi et savoir \\
Foi et superstition \\
Folie et raison \\
Folie et société \\
Fonction et prédicat \\
Fonder \\
Fonder la justice \\
Fonder une cite \\
Fonder une cité \\
Force et violence \\
Forcer à être libre \\
Forger des hypothèses \\
Formaliser et axiomatiser \\
Forme et contenu \\
Forme et matière \\
Forme et rythme \\
Former et éduquer \\
Former les esprits \\
Forme-t-on son esprit en transformant la matière ? \\
Fuir la civilisation \\
Gagner \\
Garder la mesure \\
Génie et technique \\
Genre et espèce \\
Gérer et gouverner \\
Gouvernement des hommes et administration des choses \\
Gouvernement et société \\
Gouverner \\
Gouverner, administrer, gérer \\
Gouverner, est-ce prévoir ? \\
Gouverner, est-ce régner ? \\
Gouverner et se gouverner \\
Grammaire et métaphysique \\
Grammaire et philosophie \\
Grandeur et décadence \\
Groupe, classe, société \\
Guérir \\
Guerre et politique \\
Guerres justes et injustes \\
Habiter \\
Habiter le monde \\
Habiter sur la terre \\
Haïr \\
Haïr la raison \\
Hasard et destin \\
Hériter \\
Hésiter \\
Hier a-t-il plus de réalité que demain ? \\
Histoire et anthropologie \\
Histoire et devenir \\
Histoire et écriture \\
Histoire et ethnologie \\
Histoire et fiction \\
Histoire et géographie \\
Histoire et mémoire \\
Histoire et morale \\
Histoire et politique \\
Histoire et progrès \\
Histoire et structure \\
Histoire et violence \\
Histoire individuelle et histoire collective \\
Homo religiosus \\
Honte, pudeur, embarras \\
Humour et ironie \\
Hypothèse et vérité \\
Ici et maintenant \\
Idéal et utopie \\
Idée et réalité \\
Identité et changement \\
Identité et communauté \\
Identité et différence \\
Identité et égalité \\
Identité et indiscernabilité \\
Ignorer \\
Illégalité et injustice \\
Illusion et apparence \\
Il y a \\
Image et concept \\
Image et idée \\
Image, signe, symbole \\
Imaginaire et politique \\
Imagination et conception \\
Imagination et culture \\
Imagination et pouvoir \\
Imagination et raison \\
Imaginer \\
Imitation et création \\
Imitation et identification \\
Imitation et représentation \\
Imiter \\
Imiter, est-ce copier ? \\
Incertitude et action \\
Inconscient et déterminisme \\
Inconscient et identité \\
Inconscient et inconscience \\
Inconscient et instinct \\
Inconscient et langage \\
Inconscient et liberté \\
Inconscient et mythes \\
Indépendance et autonomie \\
Indépendance et liberté \\
Individualisme et égoïsme \\
Individuation et identité \\
Individu et citoyen \\
Individu et communauté \\
Individu et société \\
Infini et indéfini \\
Information et communication \\
Information et opinion \\
Innocence et ignorance \\
Innocenter le devenir \\
Instinct et morale \\
Instruction et éducation \\
Instruire et éduquer \\
Intentions, plans et stratégies \\
Interdire et prohiber \\
Intérêt général et bien commun \\
Interprétation et création \\
Interpréter \\
Interpréter, est-ce connaître ? \\
Interpréter, est-ce renoncer à prouver ? \\
Interpréter, est-ce savoir ? \\
Interpréter est-il subjectif ? \\
Interpréter et expliquer \\
Interpréter et formaliser dans les sciences humaines \\
Interpréter et traduire \\
Interpréter ou expliquer \\
Interpréter une œuvre d'art \\
Interprète-t-on à défaut de connaître ? \\
Interroger \\
Interroger et répondre \\
Intuition et concept \\
Intuition et déduction \\
Intuition et intellection \\
Invention et création \\
Invention et découverte \\
Invention et imitation \\
J'ai un corps \\
Je \\
Je est un autre \\
Je mens \\
Je ne l'ai pas fait exprès \\
Je sens, donc je suis \\
Je, tu, il \\
Jouer \\
Jouer son rôle \\
Jouer un rôle \\
Jouir sans entraves \\
Jugement analytique et jugement synthétique \\
Jugement de goût et jugement esthétique \\
Jugement esthétique et jugement de valeur \\
Jugement et réflexion \\
Jugement et vérité \\
Jugement moral et jugement empirique \\
Juger \\
Juger en conscience \\
Juger et connaître \\
Juger et décider \\
Juger et raisonner \\
Juger et sentir \\
Jusqu'à quel point la nature est-elle objet de science ? \\
Jusqu'à quel point pouvons-nous juger autrui ? \\
Jusqu'à quel point sommes-nous responsables de nos passions ? \\
Jusqu'à quel point suis-je mon propre maître ? \\
Jusqu'où interpréter ? \\
Jusqu'où peut-on dialoguer ? \\
Jusqu'où peut-on soigner ? \\
Justice et charité \\
Justice et égalité \\
Justice et équité \\
Justice et force \\
Justice et pardon \\
Justice et utilité \\
Justice et vengeance \\
Justice et violence \\
Justification et politique \\
Justifier \\
Justifier et prouver \\
Justifier le mensonge \\
La banalité \\
L'abandon \\
La barbarie \\
La barbarie de la technique \\
La bassesse \\
La béatitude \\
La beauté \\
La beauté a-t-elle une histoire ? \\
La beauté de la nature \\
La beauté des corps \\
La beauté des ruines \\
La beauté du diable \\
La beauté du geste \\
La beauté du monde \\
La beauté est-elle affaire de goût ? \\
La beauté est-elle dans le regard ou dans la chose vue ? \\
La beauté est-elle dans les choses ? \\
La beauté est-elle intemporelle ? \\
La beauté est-elle l'objet d'une connaissance ? \\
La beauté est-elle partout ? \\
La beauté est-elle sensible ? \\
La beauté est-elle une promesse de bonheur ? \\
La beauté et la grâce \\
La beauté idéale \\
La beauté morale \\
La beauté naturelle \\
La beauté nous rend-elle meilleurs ? \\
La beauté peut-elle délivrer une vérité ? \\
La beauté s'explique-t-elle ? \\
La belle âme \\
La belle nature \\
La bestialité \\
La bête \\
La bête et l'animal \\
La bêtise \\
La bêtise et la méchanceté sont-elles liées intrinsèquement ? \\
La bêtise et la méchanceté sont-elles liées nécessairement ? \\
La bêtise n'est-elle pas proprement humaine ? \\
La bibliothèque \\
La bienfaisance \\
La bienséance \\
La bienveillance \\
La biographie \\
La biologie peut-elle se passer de causes finales ? \\
L'abondance \\
La bonne conscience \\
La bonne éducation \\
La bonne intention \\
La bonne volonté \\
La bonté \\
L'absence \\
L'absence de fondement \\
L'absence de générosité \\
L'absence d'œuvre \\
L'absolu \\
L'absolu et le relatif \\
L'abstraction \\
L'abstraction en art \\
L'abstraction est-elle toujours utile à la science empirique ? \\
L'abstrait est-il en dehors de l'espace et du temps ? \\
L'abstrait et le concret \\
L'absurde \\
L'abus de pouvoir \\
L'académisme \\
L'académisme dans l'art \\
L'académisme et les fins de l'art \\
La calomnie \\
La casuistique \\
La catharsis \\
La causalité \\
La causalité en histoire \\
La causalité historique \\
La causalité suppose-t-elle des lois ? \\
La cause \\
La cause efficiente \\
La cause et la raison \\
La cause et l'effet \\
La cause première \\
L'accès à la vérité \\
L'accident \\
L'accidentel \\
L'accomplissement \\
L'accomplissement de soi \\
L'accord \\
La censure \\
La certitude \\
La certitude de mourir \\
La chair \\
La chance \\
La charité \\
La charité est-elle une vertu ? \\
La chasse et la guerre \\
L'achèvement de l'œuvre \\
La chose \\
La chose en soi \\
La chose publique \\
La chronologie \\
La chute \\
La circonspection \\
La citation \\
La cité \\
La cité idéale \\
La cité sans dieux \\
La citoyenneté \\
La civilisation \\
La civilité \\
La clarté \\
La classe moyenne \\
La classification \\
La classification des arts \\
La classification des sciences \\
La clause de conscience \\
La clémence \\
La coexistence des libertés \\
La cohérence \\
La cohérence est-elle la norme du vrai ? \\
La cohérence est-elle un critère de la vérité ? \\
La cohérence est-elle un critère de vérité ? \\
La cohérence est-elle une vertu ? \\
La cohérence logique est-elle une condition suffisante de la démonstration ? \\
La colère \\
La collection \\
La comédie \\
La comédie du pouvoir \\
La comédie humaine \\
La comédie sociale \\
La communauté \\
La communauté des savants \\
La communauté internationale \\
La communauté morale \\
La communauté scientifique \\
La communication \\
La communication est-elle nécessaire à la démocratie ? \\
La comparaison \\
La compassion \\
La compassion risque-t-elle d'abolir l'exigence politique ? \\
La compétence \\
La compétence technique peut-elle fonder l'autorité publique ? \\
La composition \\
La compréhension \\
La concorde \\
La concurrence \\
La condition \\
La condition de mortel \\
La condition humaine \\
La condition sociale \\
La confiance \\
La confiance en la raison \\
La confiance est-elle une vertu ? \\
La confusion \\
La connaissance adéquate \\
La connaissance animale \\
La connaissance a-t-elle des limites ? \\
La connaissance commune est-elle le point de départ de la science ? \\
La connaissance commune fait-elle obstacle à la vérité ? \\
La connaissance de Dieu \\
La connaissance de la vie \\
La connaissance de la vie se confond-elle avec celle du vivant ? \\
La connaissance de l'histoire est-elle utile à l'action ? \\
La connaissance des causes \\
La connaissance de soi \\
La connaissance des passions \\
La connaissance des principes \\
La connaissance du bien \\
La connaissance du futur \\
La connaissance du monde \\
La connaissance du passé \\
La connaissance du singulier \\
La connaissance du vivant \\
La connaissance du vivant est-elle désintéressée ? \\
La connaissance du vivant peut-elle être désintéressée  ? \\
La connaissance est-elle une contemplation ? \\
La connaissance est-elle une croyance justifiée ? \\
La connaissance et la croyance \\
La connaissance et la morale \\
La connaissance et le vivant \\
La connaissance historique \\
La connaissance historique est-elle une interprétation des faits ? \\
La connaissance historique est-elle utile à l'homme ? \\
La connaissance intuitive \\
La connaissance mathématique \\
La connaissance objective \\
La connaissance objective doit-elle s'interdire toute interprétation ? \\
La connaissance objective exclut-elle toute forme de subjectivité ? \\
La connaissance peut-elle être pratique ? \\
La connaissance peut-elle se passer de l'imagination ? \\
La connaissance scientifique \\
La connaissance scientifique abolit-elle toute croyance ? \\
La connaissance scientifique est-elle désintéressée ? \\
La connaissance scientifique n'est-elle qu'une croyance argumentée ? \\
La connaissance sensible \\
La connaissance s'interdit-elle tout recours à l'imagination ? \\
La connaissance suppose-t-elle une éthique ? \\
La conquête \\
La conquête de l'espace \\
La conscience \\
La conscience a-t-elle des degrés ? \\
La conscience a-t-elle des moments ? \\
La conscience collective \\
La conscience d'autrui est-elle impénétrable ? \\
La conscience de la mort est-elle une condition de la sagesse ? \\
La conscience de soi \\
La conscience de soi de l'art \\
La conscience de soi est-elle une donnée immédiate ? \\
La conscience de soi et l'identité personnelle \\
La conscience de soi suppose-t-elle autrui ? \\
La conscience du temps rend-elle l'existence tragique ? \\
La conscience entrave-t-elle l'action ? \\
La conscience est-elle ce qui fait le sujet ? \\
La conscience est-elle intrinsèquement morale ? \\
La conscience est-elle nécessairement malheureuse ? \\
La conscience est-elle ou n'est-elle pas ? \\
La conscience est-elle source d'illusions ? \\
La conscience est-elle toujours morale ? \\
La conscience est-elle une activité ? \\
La conscience est-elle une illusion ? \\
La conscience et l'inconscient \\
La conscience historique \\
La conscience morale \\
La conscience morale est-elle innée ? \\
La conscience morale n'est-elle que le fruit de l'éducation ? \\
La conscience morale n'est-elle que le produit de l'éducation ? \\
La conscience peut-elle être collective ? \\
La conscience peut-elle nous tromper ? \\
La conscience politique \\
La conscience universelle \\
La conséquence \\
La conservation \\
La considération de l'utilité doit-elle déterminer toutes nos actions ? \\
La consolation \\
La constance \\
La constitution \\
La contemplation \\
La contestation \\
La contingence \\
La contingence de l'existence \\
La contingence des lois de la nature \\
La contingence du futur \\
La contingence du monde \\
La contingence est-elle la condition de la liberté ? \\
La continuité \\
La contradiction \\
La contradiction réside-t-elle dans les choses ? \\
La contrainte \\
La contrainte déontologique \\
La contrainte des lois est-elle une violence ? \\
La contrainte en art \\
La contrainte peut-elle être légitime ? \\
La contrainte supprime-t-elle la responsabilité ? \\
La contrôle social \\
La controverse scientifique \\
La convalescence \\
La convention et l'arbitraire \\
La conversation \\
La conversion \\
La conviction \\
La coopération \\
La copie \\
La corruption \\
La corruption politique \\
La cosmogonie \\
La couleur \\
La courtoisie \\
La coutume \\
La crainte des Dieux \\
La crainte et l'ignorance \\
La création \\
La création artistique \\
La création dans l'art \\
La création de l'humanité \\
La création de valeur \\
La créativité \\
La crédibilité \\
La crédulité \\
La criminalité \\
La crise \\
La crise sociale \\
La critique \\
La critique d'art \\
La critique de l'État \\
La critique des théories \\
La critique du pouvoir peut-elle conduire à la désobéissance ? \\
La croissance \\
La croissance du savoir \\
La croyance \\
La croyance est-elle l'asile de l'ignorance ? \\
La croyance est-elle signe de faiblesse ? \\
La croyance est-elle une opinion comme les autres ? \\
La croyance est-elle une opinion ? \\
La croyance et la foi \\
La croyance et la raison \\
La croyance peut-elle être rationnelle ? \\
La croyance peut-elle tenir lieu de savoir ? \\
La croyance religieuse échappe-t-elle à toute logique ? \\
La croyance religieuse se distingue-t-elle des autres formes de croyance ? \\
La cruauté \\
L'acte \\
L'acte et la parole \\
L'acte et la puissance \\
L'acte et l'œuvre \\
L'acte gratuit \\
L'acteur \\
L'acteur et son rôle \\
L'action \\
L'action collective \\
L'action du temps \\
L'action et le risque \\
L'action et son contexte \\
L'action humaine nécessite-t-elle la contingence du monde ? \\
L'action intentionnelle \\
L'action politique \\
L'action politique a-t-elle un fondement rationnel ? \\
L'action politique peut-elle se passer de mots ? \\
L'activité \\
L'activité se laisse-t-elle programmer ? \\
L'actualité \\
L'actuel \\
La cuisine \\
La culpabilité \\
La culture \\
La culture artistique \\
La culture de masse \\
La culture démocratique \\
La culture d'entreprise \\
La culture est-elle affaire de politique ? \\
La culture est-elle la négation de la nature ? \\
La culture est-elle nécessaire à l'appréciation d'une œuvre d'art ? \\
La culture est-elle une question politique ? \\
La culture est-elle une seconde nature ? \\
La culture est-elle un luxe ? \\
La culture et les cultures \\
La culture garantit-elle l'excellence humaine ? \\
La culture générale \\
La culture libère-t-elle des préjugés ? \\
La culture morale \\
La culture nous rend-elle meilleurs ? \\
La culture nous rend-elle plus humains ? \\
La culture nous unit-elle ? \\
La culture peut-elle être instituée ? \\
La culture peut-elle être objet de science ? \\
La culture rend-elle plus humain ? \\
La culture savante et la culture populaire \\
La culture scientifique \\
La culture technique \\
La culture : pour quoi faire ? \\
La curiosité \\
La curiosité est-elle à l'origine du savoir ? \\
La danse \\
La danse est-elle l'œuvre du corps ? \\
La décadence \\
La décence \\
La déception \\
La décision \\
La décision a-t-elle besoin de raisons ? \\
La décision morale \\
La décision politique \\
La découverte de la vérité peut-elle être le fait du hasard ? \\
La déduction \\
La défense de la liberté \\
La défense de l'intérêt général est-il la fin dernière de la politique ? \\
La défense nationale \\
La déficience \\
La définition \\
La délibération \\
La délibération en morale \\
La délibération politique \\
La démagogie \\
La démarche scientifique exclut-elle tout recours à l'imagination ? \\
La démence \\
La démesure \\
La démocratie \\
La démocratie a-t-elle des limites ? \\
La démocratie a-t-elle une histoire ? \\
La démocratie conduit-elle au règne de l'opinion ? \\
La démocratie est-ce la fin du despotisme ? \\
La démocratie, est-ce le pouvoir du plus grand nombre ? \\
La démocratie est-elle la loi du plus fort ? \\
La démocratie est-elle le pire des régimes politiques ? \\
La démocratie est-elle le règne de l'opinion ? \\
La démocratie est-elle moyen ou fin ? \\
La démocratie est-elle nécessairement libérale ? \\
La démocratie est-elle possible ? \\
La démocratie et les experts \\
La démocratie et les institutions de la justice \\
La démocratie et le statut de la loi \\
La démocratie n'est-elle que la force des faibles ? \\
La démocratie participative \\
La démocratie peut-elle échapper à la démagogie ? \\
La démocratie peut-elle être représentative ? \\
La démocratie peut-elle se passer de représentation ? \\
La démonstration \\
La démonstration nous garantit-elle l'accès à la vérité ? \\
La démonstration obéit-elle à des lois ? \\
La démonstration supprime-t-elle le doute ? \\
La déontologie \\
La dépendance \\
La dépense \\
La déraison \\
La dérision \\
La descendance \\
La description \\
La désillusion \\
La désinvolture \\
La désobéissance \\
La désobéissance civile \\
La destruction \\
La détermination \\
La dette \\
La deuxième chance \\
La déviance \\
La dialectique \\
La dialectique est-elle une science ? \\
La dictature \\
La différence \\
La différence culturelle \\
La différence des arts \\
La différence des sexes \\
La différence des sexes est-elle une question philosophique ? \\
La différence des sexes est-elle un problème philosophique ? \\
La différence homme-femme \\
La différence sexuelle \\
La difformité \\
La dignité \\
La dignité humaine \\
La digression \\
La direction de l'esprit \\
La discipline \\
La discorde \\
La discrétion \\
La discrimination \\
La discursivité \\
La discussion \\
La disgrâce \\
La disharmonie \\
La disponibilité \\
La disposition \\
La disposition morale \\
La dispute \\
La dissidence \\
La dissimulation \\
La distance \\
La distinction \\
La distinction de genre \\
La distinction de la nature et de la culture est-elle un fait de culture ? \\
La distinction sociale \\
La distraction \\
La diversion \\
La diversité \\
La diversité des cultures \\
La diversité des langues \\
La diversité des langues est-elle une diversité des pensées ? \\
La diversité des opinions conduit-elle à douter de tout ? \\
La diversité des perceptions \\
La diversité des religions \\
La diversité des sciences \\
La diversité humaine \\
La division \\
La division de la volonté \\
La division des pouvoirs \\
La division des tâches \\
La division du travail \\
L'admiration \\
La docilité est-elle un vice ou une vertu ? \\
La domestication \\
La domination \\
La domination du corps \\
La domination sociale \\
L'adoucissement des mœurs \\
La douleur \\
La douleur est-elle utile ? \\
La douleur nous apprend-elle quelque chose ? \\
La droit de conquête \\
La droiture \\
La dualité \\
La duplicité \\
La durée \\
L'adversité \\
La faiblesse \\
La faiblesse de croire \\
La faiblesse de la démocratie \\
La faiblesse de la volonté \\
La faiblesse d'esprit \\
La familiarité \\
La famille \\
La famille est-elle le lieu de la formation morale ? \\
La famille est-elle naturelle ? \\
La famille est-elle une communauté naturelle ? \\
La famille est-elle une institution politique ? \\
La famille est-elle un modèle de société ? \\
La famille et la cité \\
La famille et le droit \\
La fatalité \\
La fatigue \\
La fausseté \\
La faute \\
La faute et le péché \\
La faute et l'erreur \\
La femme est-elle l'avenir de l'homme ? \\
La fermeté \\
La fête \\
L'affirmation \\
La fiction \\
La fidélité \\
La fidélité à soi \\
La fierté \\
La fièvre \\
La figuration \\
La figure de l'ennemi \\
La figure humaine \\
La fin \\
La finalité \\
La finalité des sciences humaines \\
La finalité est-elle nécessaire pour penser le vivant ? \\
La fin de la guerre \\
La fin de la métaphysique \\
La fin de la politique \\
La fin de la politique est-elle l'établissement de la justice ? \\
La fin de l'art \\
La fin de l'État \\
La fin de l'histoire \\
La fin de l'homme \\
La fin des désirs \\
La fin des guerres \\
La fin du monde \\
La fin du mythe \\
La fin du travail \\
La fin et les moyens \\
La finitude \\
La fin justifie-t-elle les moyens ? \\
La foi \\
La foi est-elle aveugle ? \\
La foi est-elle irrationnelle ? \\
La foi est-elle rationnelle ? \\
La folie \\
La folie des grandeurs \\
La fonction \\
La fonction de l'art \\
La fonction de penser peut-elle se déléguer ? \\
La fonction des exemples \\
La fonction du philosophe est-elle de diriger l'État ? \\
La fonction et l'organe \\
La fonction première de l'État est-elle de durer ? \\
La force \\
La force d'âme \\
La force de conviction \\
La force de la croyance \\
La force de la loi \\
La force de l'art \\
La force de la vérité \\
La force de l'esprit \\
La force de l'État est-elle nécessaire à la liberté des citoyens ? \\
La force de l'expérience \\
La force de l'habitude \\
La force de l'idée \\
La force de l'inconscient \\
La force des choses \\
La force des faibles \\
La force des idées \\
La force des lois \\
La force du droit \\
La force du pouvoir \\
La force du social \\
La force est-elle une vertu ? \\
La force et le droit \\
La force fait-elle le droit ? \\
La force publique \\
La formalisation \\
La formation de l'esprit \\
La formation des citoyens \\
La formation du goût \\
La formation d'une conscience \\
La forme \\
La fortune \\
La foule \\
La fragilité \\
La franchise \\
La franchise est-elle une vertu ? \\
La fraternité \\
La fraternité a-t-elle un sens politique ? \\
La fraternité est-elle un idéal moral ? \\
La fraternité peut-elle se passer d'un fondement religieux ? \\
La fraude \\
La frivolité \\
La frontière \\
La fuite du temps est-elle nécessairement un malheur ? \\
La futilité \\
La gauche et la droite \\
L'âge atomique \\
L'âge d'or \\
La généalogie \\
La généralisation \\
La générosité \\
La genèse \\
La genèse de l'œuvre \\
La gentillesse \\
La géographie \\
La géométrie \\
La gloire \\
La gloire est-elle un bien ? \\
La grâce \\
La grammaire \\
La grammaire contraint-elle la pensée ? \\
La grammaire contraint-elle notre pensée ? \\
La grammaire et la logique \\
La grandeur \\
La grandeur d'âme \\
La grandeur d'une culture \\
La gratitude \\
La gratuité \\
L'agression \\
L'agressivité \\
L'agriculture \\
La grossièreté \\
La guérison \\
La guerre \\
La guerre civile \\
La guerre est-elle la continuation de la politique par d'autres moyens ? \\
La guerre est-elle la continuation de la politique ? \\
La guerre est-elle la politique continuée par d'autres moyens ? \\
La guerre est-elle l'essentiel de toute politique ? \\
La guerre et la paix \\
La guerre juste \\
La guerre met-elle fin au droit ? \\
La guerre mondiale \\
La guerre peut-elle être juste ? \\
La guerre totale \\
La haine \\
La haine de la pensée \\
La haine de la raison \\
La haine des images \\
La haine des machines \\
La haine de soi \\
La haine et le mépris \\
La hiérarchie \\
La hiérarchie des arts \\
La hiérarchie des énoncés scientifiques \\
La honte \\
Laisser mourir, est-ce tuer ? \\
La jalousie \\
La jeunesse \\
La jeunesse est mécontente \\
La joie \\
La joie de vivre \\
La jouissance \\
La jurisprudence \\
La juste colère \\
La juste mesure \\
La juste peine \\
La justice \\
La justice a-t-elle besoin des institutions ? \\
La justice a-t-elle un fondement rationnel ? \\
La justice consiste-t-elle à traiter tout le monde de la même manière ? \\
La justice consiste-t-elle dans l'application de la loi ? \\
La justice de l'État \\
La justice divine \\
La justice entre les générations \\
La justice est-elle l'affaire de l'État ? \\
La justice est-elle une notion morale ? \\
La justice est-elle une vertu ? \\
La justice et la force \\
La justice et la loi \\
La justice et la paix \\
La justice et le droit \\
La justice et l'égalité \\
La justice internationale \\
La justice n'est-elle qu'une institution ? \\
La justice n'est-elle qu'un idéal ? \\
La justice peut-elle se fonder sur le compromis ? \\
La justice peut-elle se passer de la force ? \\
La justice peut-elle se passer d'institutions ? \\
La justice sociale \\
La justice suppose-t-elle l'égalité ? \\
La justice : moyen ou fin de la politique ? \\
La justification \\
La lâcheté \\
La laïcité \\
La laideur \\
La langue de la raison \\
La langue et la parole \\
La langue maternelle \\
La lassitude \\
L'aléatoire \\
La leçon des choses \\
La lecture \\
La légende \\
La légèreté \\
La légitimation \\
La légitime défense \\
La légitimité \\
La légitimité démocratique \\
La lettre et l'esprit \\
La libération des mœurs \\
La liberté \\
La liberté artistique \\
La liberté a-t-elle un prix ? \\
La liberté civile \\
La liberté comporte-t-elle des degrés ? \\
La liberté connaît-elle des excès ? \\
La liberté créatrice \\
La liberté d'autrui \\
La liberté de croire \\
La liberté de culte \\
La liberté de l'artiste \\
La liberté de la science \\
La liberté de la volonté \\
La liberté de l'interprète \\
La liberté de parole \\
La liberté de penser \\
La liberté des autres \\
La liberté des citoyens \\
La liberté d'expression \\
La liberté d'expression est-elle nécessaire à la liberté de pensée ? \\
La liberté d'indifférence \\
La liberté d'opinion \\
La liberté du choix \\
La liberté du savant \\
La liberté, est-ce l'indépendance à l'égard des passions ? \\
La liberté est-elle le pouvoir de refuser ? \\
La liberté est-elle un fait ? \\
La liberté et l'égalité sont-elles compatibles ? \\
La liberté et le hasard \\
La liberté et le temps \\
La liberté fait-elle de nous des êtres meilleurs ? \\
La liberté implique-t-elle l'indifférence ? \\
La liberté individuelle \\
La liberté intéresse-t-elle les sciences humaines ? \\
La liberté morale \\
La liberté n'est-elle qu'une illusion ? \\
La liberté nous rend-elle inexcusables ? \\
La liberté peut-elle être prouvée ? \\
La liberté peut-elle être une illusion ? \\
La liberté peut-elle faire peur ? \\
La liberté peut-elle s'affirmer sans violence ? \\
La liberté peut-elle s'aliéner ? \\
La liberté peut-elle se constater ? \\
La liberté peut-elle se prouver ? \\
La liberté peut-elle se refuser ? \\
La liberté politique \\
La liberté requiert-elle le libre échange ? \\
La liberté s'achète-t-elle ? \\
La liberté se mérite-t-elle ? \\
La liberté se prouve-t-elle ? \\
La liberté se réduit-elle au libre-arbitre ? \\
La libre interprétation \\
L'aliénation \\
La limite \\
La littérature peut-elle suppléer les sciences de l'homme ? \\
L'allégorie \\
La logique a-t-elle une histoire ? \\
La logique a-t-elle un intérêt philosophique ? \\
La logique décrit-elle le monde ? \\
La logique du pire \\
La logique est-elle indépendante de la psychologie ? \\
La logique est-elle la norme du vrai ? \\
La logique est-elle l'art de penser ? \\
La logique est-elle un art de penser ? \\
La logique est-elle un art de raisonner ? \\
La logique est-elle une discipline normative ? \\
La logique est-elle une forme de calcul ? \\
La logique est-elle une science de la vérité ? \\
La logique est-elle une science ? \\
La logique est-elle utile à la métaphysique ? \\
La logique et le réel \\
La logique nous apprend-elle quelque chose sur le langage ordinaire ? \\
La logique peut-elle se passer de la métaphysique ? \\
La logique pourrait-elle nous surprendre ? \\
La logique : découverte ou invention ? \\
La loi \\
La loi dit-elle ce qui est juste ? \\
La loi du désir \\
La loi du genre \\
La loi du marché \\
La loi du plus fort \\
La loi éduque-t-elle ? \\
La loi est-elle une garantie contre l'injustice ? \\
La loi et la coutume \\
La loi et la règle \\
La loi et le règlement \\
La loi et les mœurs \\
La loi et l'ordre \\
La loi peut-elle changer les mœurs ? \\
La loi peut-elle être injuste ? \\
La louange et le blâme \\
La loyauté \\
L'alter ego \\
L'altérité \\
L'altruisme \\
L'altruisme n'est-il qu'un égoïsme bien compris ? \\
La lumière de la vérité \\
La lumière naturelle \\
La lutte des classes \\
La machine \\
La magie \\
La magie des mots \\
La magie peut-elle être efficace ? \\
La magnanimité \\
La main \\
La main et l'esprit \\
La main et l'outil \\
La maîtrise \\
La maîtrise de la langue \\
La maîtrise de la nature \\
La maîtrise de soi \\
La maîtrise du feu \\
La maîtrise du temps \\
La majesté \\
La majorité \\
La majorité doit-elle toujours l'emporter ? \\
La majorité, force ou droit ? \\
La majorité peut-elle être tyrannique ? \\
La maladie \\
La maladie est-elle à l'organisme vivant ce que la panne est à la machine ? \\
La maladie est-elle indispensable à la connaissance du vivant ? \\
La malchance \\
La malveillance \\
La manière \\
La manifestation \\
La marchandise \\
La marge \\
La marginalité \\
L'amateur \\
L'amateurisme \\
La mathématique est-elle une ontologie ? \\
La mathématisation du réel \\
La matière \\
La matière de la pensée \\
La matière de l'œuvre \\
La matière, est-ce le mal ? \\
La matière, est-ce l'informe ? \\
La matière est-elle amorphe ? \\
La matière est-elle plus facile à connaître que l'esprit ? \\
La matière est-elle une vue de l'esprit ? \\
La matière et la forme \\
La matière et la vie \\
La matière et l'esprit \\
La matière n'est-elle que ce que l'on perçoit ? \\
La matière n'est-elle qu'une idée ? \\
La matière n'est-elle qu'un obstacle ? \\
La matière pense-t-elle ? \\
La matière peut-elle être objet de connaissance ? \\
La matière première \\
La matière sensible \\
La matière vivante \\
La maturité \\
La mauvaise conscience \\
La mauvaise foi \\
La mauvaise volonté \\
L'ambiguïté \\
L'ambiguïté des mots peut-elle être heureuse ? \\
L'ambition \\
L'ambition politique \\
L'âme \\
La méchanceté \\
L'âme concerne-t-elle les sciences humaines ? \\
La méconnaissance de soi \\
La médecine est-elle une science ? \\
L'âme des bêtes \\
La médiation \\
La médiocrité \\
La méditation \\
L'âme est-elle immortelle ? \\
L'âme et le corps \\
L'âme et le corps sont-ils une seule et même chose ? \\
L'âme et l'esprit \\
La méfiance \\
La meilleure constitution \\
L'âme jouit-elle d'une vie propre ? \\
La mélancolie \\
L'âme, le monde et Dieu \\
L'amélioration des hommes peut-elle être considérée comme un objectif politique ? \\
La mémoire \\
La mémoire collective \\
La mémoire et l'histoire \\
La mémoire et l'individu \\
La mémoire et l'oubli \\
La mémoire sélective \\
La menace \\
La mesure \\
La mesure de l'intelligence \\
La mesure des choses \\
La mesure du temps \\
La métamorphose \\
La métaphore \\
La métaphysique a-t-elle ses fictions ? \\
La métaphysique est-elle le fondement de la morale ? \\
La métaphysique est-elle nécessairement une réflexion sur Dieu ? \\
La métaphysique est-elle une science ? \\
La métaphysique peut-elle être autre chose qu'une science recherchée ? \\
La métaphysique peut-elle faire appel à l'expérience ? \\
La métaphysique se définit-elle par son objet ou sa démarche ? \\
La méthode \\
La méthode de la science \\
La méthode expérimentale est-elle appropriée à l'étude du vivant ? \\
L'ami \\
L'ami du prince \\
La minorité \\
La misanthropie \\
La misère \\
La misologie \\
L'amitié \\
L'amitié est-elle une vertu ? \\
L'amitié est-elle un principe politique ? \\
L'amitié peut-elle obliger ? \\
L'amitié relève-t-elle d'une décision ? \\
La modalité \\
La mode \\
La modélisation en sciences sociales \\
La modération \\
La modération est-elle l'essence de la vertu ? \\
La modération est-elle une vertu politique ? \\
La modernité \\
La modernité dans les arts \\
La modestie \\
La mondialisation \\
La monnaie \\
La monumentalité \\
La morale a-t-elle à décider de la sexualité ? \\
La morale a-t-elle besoin de la notion de sainteté ? \\
La morale a-t-elle besoin d'être fondée ? \\
La morale a-t-elle besoin d'un au-delà ? \\
La morale a-t-elle besoin d'un fondement ? \\
La morale a-t-elle sa place dans l'économie ? \\
La morale commune \\
La morale consiste-t-elle à respecter le droit ? \\
La morale consiste-t-elle à suivre la nature ? \\
La morale de l'athée \\
La morale de l'intérêt \\
La morale dépend-elle de la culture ? \\
La morale des fables \\
La morale doit-elle en appeler à la nature ? \\
La morale doit-elle être rationnelle ? \\
La morale doit-elle fournir des préceptes ? \\
La morale du citoyen \\
La morale du plus fort \\
La morale est-elle affaire de convention ? \\
La morale est-elle affaire de jugement ? \\
La morale est-elle affaire de sentiments ? \\
La morale est-elle affaire de sentiment ? \\
La morale est-elle condamnée à n'être qu'un champ de bataille ? \\
La morale est-elle désintéressée ? \\
La morale est-elle en conflit avec le désir ? \\
La morale est-elle ennemie du bonheur ? \\
La morale est-elle fondée sur la liberté ? \\
La morale est-elle incompatible avec le déterminisme ? \\
La morale est-elle l'ennemie de la vie ? \\
La morale est-elle nécessairement répressive ? \\
La morale est-elle objet de science ? \\
La morale est-elle un art de vivre ? \\
La morale est-elle une affaire de raison ? \\
La morale est-elle une affaire d'habitude ? \\
La morale est-elle une affaire solitaire ? \\
La morale est-elle un fait de culture ? \\
La morale est-elle un fait social ? \\
La morale et la politique \\
La morale et la religion visent-elles les mêmes fins ? \\
La morale et le droit \\
La morale et les mœurs \\
La morale n'est-elle qu'un ensemble de conventions ? \\
La morale peut-elle être fondée sur la science ? \\
La morale peut-elle être naturelle ? \\
La morale peut-elle être un calcul ? \\
La morale peut-elle être une science ? \\
La morale peut-elle se fonder sur les sentiments ? \\
La morale peut-elle s'enseigner ? \\
La morale peut-elle se passer d'un fondement religieux ? \\
La morale politique \\
La morale s'apprend-elle ? \\
La morale s'enseigne-t-elle ? \\
La morale s'oppose-t-elle à la politique ? \\
La morale suppose-t-elle le libre arbitre ? \\
La moralité consiste-t-elle à se contraindre soi-même ? \\
La moralité des lois \\
La moralité est-elle affaire de principes ou de conséquences ? \\
La moralité et le traitement des animaux \\
La moralité n'est-elle que dressage ? \\
La moralité réside-t-elle dans l'intention ? \\
La moralité se réduit-elle aux sentiments ? \\
La mort \\
La mort a-t-elle un sens ? \\
La mort dans l'âme \\
La mort d'autrui \\
La mort de Dieu \\
La mort de l'art \\
La mort de l'homme \\
La mort fait-elle partie de la vie ? \\
L'amour a-t-il des raisons ? \\
L'amour de la liberté \\
L'amour de l'argent \\
L'amour de l'art \\
L'amour de la vérité \\
L'amour de la vie \\
L'amour de l'humanité \\
L'amour des lois \\
L'amour de soi \\
L'amour de soi est-il immoral ? \\
L'amour du destin \\
L'amour du travail \\
L'amour est-il désir ? \\
L'amour est-il une vertu ? \\
L'amour et la haine \\
L'amour et la justice \\
L'amour et l'amitié \\
L'amour et la mort \\
L'amour et le devoir \\
L'amour et le respect \\
L'amour fou \\
L'amour implique-t-il le respect ? \\
L'amour maternel \\
L'amour peut-il être absolu ? \\
L'amour peut-il être raisonnable ? \\
L'amour peut-il être un devoir ? \\
L'amour propre \\
L'amour-propre \\
L'amour vrai \\
La multiplicité \\
La multitude \\
La musique a-t-elle une essence ? \\
La musique de film \\
La musique est-elle un langage ? \\
La musique et le bruit \\
L'anachronisme \\
La naissance \\
La naissance de la science \\
La naissance de l'homme \\
La naïveté \\
La naïveté est-elle une vertu ? \\
L'analogie \\
L'analyse \\
L'analyse du langage ordinaire peut-elle avoir un intérêt philosophique ? \\
L'analyse du vécu \\
L'analyse et la synthèse \\
L'anarchie \\
La nation \\
La nation est-elle dépassée ? \\
La nation et l'État \\
La nature \\
La nature artiste \\
La nature a-t-elle des droits ? \\
La nature a-t-elle une histoire ? \\
La nature a-t-elle un langage ? \\
La nature des choses \\
La nature du bien \\
La nature du fait moral \\
La nature est-elle artiste ? \\
La nature est-elle belle ? \\
La nature est-elle bien faite ? \\
La nature est-elle digne de respect ? \\
La nature est-elle écrite en langage mathématique ? \\
La nature est-elle muette ? \\
La nature est-elle politique ? \\
La nature est-elle prévisible ? \\
La nature est-elle sacrée ? \\
La nature est-elle sans histoire ? \\
La nature est-elle sauvage ? \\
La nature est-elle une norme ? \\
La nature est-elle une ressource ? \\
La nature est-elle un modèle ? \\
La nature est-elle un système ? \\
La nature et la grâce \\
La nature et le beau \\
La nature et le monde \\
La nature existe-t-elle ? \\
La nature fait-elle bien les choses ? \\
La nature morte \\
La nature ne fait pas de saut \\
La nature parle-t-elle le langage des mathématiques ? \\
La nature peut-elle avoir des droits ? \\
La nature peut-elle constituer une norme ? \\
La nature peut-elle être belle ? \\
La nature peut-elle être un modèle ? \\
La nature peut-elle nous indiquer ce que nous devons faire ? \\
La nature se donne-t-elle à penser ? \\
L'anéantissement \\
L'anecdotique \\
La nécessité \\
La nécessité de l'oubli \\
La nécessité des contradictions \\
La nécessité des signes \\
La nécessité fait-elle loi ? \\
La nécessité historique \\
La négation \\
La négligence \\
La négligence est-elle une faute ? \\
La négociation \\
La neige est-elle blanche ? \\
La neutralité \\
La neutralité de l'État \\
Langage et communication \\
Langage et logique \\
Langage et passions \\
Langage et pensée \\
Langage et pouvoir \\
Langage et réalité \\
Langage et société \\
Langage, langue et parole \\
Langage ordinaire et langage de la science \\
L'angélisme \\
L'angoisse \\
Langue et parole \\
L'animal \\
L'animal a-t-il des droits ? \\
L'animal et la bête \\
L'animal et l'homme \\
L'animalité \\
L'animalité de l'animal, l'animalité de l'homme \\
L'animal nous apprend-il quelque chose sur l'homme ? \\
L'animal peut-il être un sujet moral ? \\
L'animal politique \\
L'animisme \\
La noblesse \\
L'anomalie \\
La non-violence \\
L'anonymat \\
L'anormal \\
La normalité \\
La norme \\
La norme et le fait \\
La nostalgie \\
La notion d'administration \\
La notion de barbarie a-t-elle un sens ? \\
La notion de civilisation \\
La notion de classe dominante \\
La notion de classe sociale \\
La notion de comportement \\
La notion de corps social \\
La notion de finalité a-t-elle de l'intérêt pour le savant ? \\
La notion de genre littéraire \\
La notion de loi a-t-elle une unité ? \\
La notion de loi dans les sciences de la nature et dans les sciences de l'homme \\
La notion de monde \\
La notion de nature humaine \\
La notion de nature humaine a-t-elle un sens ? \\
La notion de peuple \\
La notion de point de vue \\
La notion de possible \\
La notion de progrès a-t-elle un sens en politique ? \\
La notion de sujet en politique \\
La notion de système \\
La notion d'évolution \\
La notion d'intérêt \\
La notion d'ordre \\
La notion physique de force \\
La nouveauté \\
La nouveauté en art \\
L'antériorité \\
L'anthropocentrisme \\
L'anthropologie est-elle une ontologie ? \\
L'anticipation \\
L'antinomie \\
La nuance \\
La nudité \\
La nuit \\
La nuit et le jour \\
La ou les vertus ? \\
La paix \\
La paix civile \\
La paix de la conscience \\
La paix est-elle l'absence de guerres ? \\
La paix est-elle l'absence de guerre ? \\
La paix est-elle le plus grand des biens ? \\
La paix est-elle moins naturelle que la guerre ? \\
La paix est-elle possible ? \\
La paix n'est-elle que l'absence de conflit ? \\
La paix n'est-elle que l'absence de guerre ? \\
La paix perpétuelle \\
La paix sociale \\
La paix sociale est-elle la finalité de la politique ? \\
La paix sociale est-elle le but de la politique ? \\
La paix sociale est-elle une fin en soi ? \\
La panne et la maladie \\
La parenté \\
La parenté et la famille \\
La paresse \\
La parole \\
La parole donnée \\
La parole et l'écriture \\
La parole et le geste \\
La parole intérieure \\
La parole peut-elle être une arme ? \\
La parole publique \\
La part de l'ombre \\
La participation \\
La participation des citoyens \\
La partie et le tout \\
La passion \\
La passion amoureuse \\
La passion de la connaissance \\
La passion de la justice \\
La passion de la liberté \\
La passion de la vérité \\
La passion de la vérité peut-elle être source d'erreur ? \\
La passion de l'égalité \\
La passion du juste \\
La passion est-elle immorale ? \\
La passion est-elle l'ennemi de la raison ? \\
La passion exclut-elle la lucidité ? \\
La passion n'est-elle que souffrance ? \\
La passivité \\
La paternité \\
L'apathie \\
La patience \\
La patience est-elle une vertu ? \\
La patrie \\
La pauvreté \\
La pauvreté est-elle une injustice ? \\
La peine \\
La peine capitale \\
La peine de mort \\
La peine de mort est-elle juste, injuste, et pourquoi ? \\
La peinture apprend-elle à voir ? \\
La peinture des mœurs \\
La peinture est-elle une poésie muette ? \\
La peinture peut-elle être un art du temps ? \\
La pénibilité du travail \\
La pensée \\
La pensée a-t-elle une histoire ? \\
La pensée collective \\
La pensée de l'espace \\
La pensée doit-elle se soumettre aux règles de la logique ? \\
La pensée échappe-t-elle à la grammaire ? \\
La pensée est-elle en lutte avec le langage ? \\
La pensée est-elle une activité assimilable à un travail ? \\
La pensée et la conscience sont-elles une seule et même chose ? \\
La pensée formelle \\
La pensée formelle est-elle privée d'objet ? \\
La pensée formelle est-elle une pensée vide ? \\
La pensée formelle peut-elle avoir un contenu ? \\
La pensée magique \\
La pensée obéit-elle à des lois ? \\
La pensée peut-elle s'écrire ? \\
La pensée peut-elle se passer de mots ? \\
La perception \\
La perception construit-elle son objet ? \\
La perception de l'espace est-elle innée ou acquise ? \\
La perception est-elle le premier degré de la connaissance ? \\
La perception est-elle l'interprétation du réel ? \\
La perception est-elle source de connaissance ? \\
La perception est-elle une interprétation ? \\
La perception me donne-t-elle le réel ? \\
La perception peut-elle être désintéressée ? \\
La perception peut-elle s'éduquer ? \\
La perfectibilité \\
La perfection \\
La perfection en art \\
La perfection est-elle désirable ? \\
La perfection morale \\
La performance \\
La permanence \\
La personnalité \\
La personne \\
La personne et l'individu \\
La perspective \\
La persuasion \\
La pertinence \\
La perversion \\
La perversion morale \\
La perversité \\
La peur \\
La peur de la mort \\
La peur de la nature \\
La peur de la science \\
La peur de l'autre \\
La peur de la vérité \\
La peur des machines \\
La peur des mots \\
La peur du châtiment \\
La peur du désordre \\
La philanthropie \\
La philosophie a-t-elle une histoire ? \\
La philosophie doit-elle être une science ? \\
La philosophie doit-elle se préoccuper du salut ? \\
La philosophie est-elle abstraite ? \\
La philosophie est-elle une science ? \\
La philosophie et le sens commun \\
La philosophie et les sciences \\
La philosophie et son histoire \\
La philosophie peut-elle disparaître ? \\
La philosophie peut-elle être expérimentale ? \\
La philosophie peut-elle être populaire ? \\
La philosophie peut-elle être une science ? \\
La philosophie peut-elle se passer de théologie ? \\
La philosophie première \\
La philosophie rend-elle inefficace la propagande ? \\
La photographie est-elle un art ? \\
La physique et la chimie \\
La pitié \\
La pitié a-t-elle une valeur ? \\
La pitié est-elle morale ? \\
La pitié peut-elle fonder la morale ? \\
La place d'autrui \\
La place de la philosophie dans la culture \\
La place du hasard dans la science \\
La place du sujet dans la science \\
La place publique \\
La plaisanterie \\
La plénitude \\
La pluralité \\
La pluralité des arts \\
La pluralité des cultures \\
La pluralité des interprétations \\
La pluralité des langues \\
La pluralité des mondes \\
La pluralité des opinions \\
La pluralité des pouvoirs \\
La pluralité des religions \\
La pluralité des sciences de la nature \\
La pluralité des sens de l'être \\
La poésie \\
La poésie et l'idée \\
La poésie pense-t-elle ? \\
La polémique \\
La police \\
La politesse \\
La politesse est-elle une vertu ? \\
La politique \\
La politique a-t-elle besoin de héros ? \\
La politique a-t-elle besoin de modèles ? \\
La politique a-t-elle besoin d'experts ? \\
La politique a-t-elle pour fin d'éliminer la violence ? \\
La politique consiste-t-elle à faire cause commune ? \\
La politique consiste-t-elle à faire des compromis ? \\
La politique de la santé \\
La politique doit-elle être morale ? \\
La politique doit-elle être rationnelle ? \\
La politique doit-elle refuser l'utopie ? \\
La politique doit-elle se mêler de l'art ? \\
La politique doit-elle se mêler du bonheur ? \\
La politique doit-elle viser la concorde ? \\
La politique doit-elle viser le consensus ? \\
La politique échappe-telle à l'exigence de vérité ? \\
La politique est-elle affaire de compétence ? \\
La politique est-elle affaire de décision ? \\
La politique est-elle affaire de jugement ? \\
La politique est-elle architectonique ? \\
La politique est-elle extérieure au droit ? \\
La politique est-elle la continuation de la guerre ? \\
La politique est-elle l'affaire des spécialistes ? \\
La politique est-elle l'affaire de tous ? \\
La politique est-elle l'art des possibles ? \\
La politique est-elle l'art du possible ? \\
La politique est-elle naturelle ? \\
La politique est-elle par nature sujette à dispute ? \\
La politique est-elle plus importante que tout ? \\
La politique est-elle un art ? \\
La politique est-elle une affaire d'experts ? \\
La politique est-elle une science ? \\
La politique est-elle une technique ? \\
La politique est-elle un métier ? \\
La politique et la gloire \\
La politique et la guerre \\
La politique et la ville \\
La politique et le bonheur \\
La politique et le mal \\
La politique et le politique \\
La politique et les passions \\
La politique et l'opinion \\
La politique exclut-elle le désordre ? \\
La politique implique-t-elle la hiérarchie ? \\
La politique n'est-elle que l'art de conquérir et de conserver le pouvoir ? \\
La politique peut-elle changer la société \\
La politique peut-elle changer le monde ? \\
La politique peut-elle être indépendante de la morale ? \\
La politique peut-elle être objet de science ? \\
La politique peut-elle être un objet de science ? \\
La politique peut-elle n'être qu'une pratique ? \\
La politique peut-elle se passer de croyances ? \\
La politique peut-elle se passer de croyance ? \\
La politique peut-elle unir les hommes ? \\
La politique repose-t-elle sur un contrat ? \\
La politique requière-t-elle le compromis \\
La politique scientifique \\
La politique suppose-t-elle la morale ? \\
La politique suppose-t-elle une idée de l'homme ? \\
L'apolitisme \\
La populace \\
La population \\
La pornographie \\
La possession \\
La possibilité \\
La possibilité logique \\
La possibilité métaphysique \\
La possibilité réelle \\
La poursuite de mon intérêt m'oppose-t-elle aux autres ? \\
L'apparence \\
L'apparence du pouvoir \\
L'apparence est-elle toujours trompeuse ? \\
L'appartenance sociale \\
L'appel \\
L'appréciation de la nature \\
L'apprentissage \\
L'apprentissage de la langue \\
L'apprentissage de la liberté \\
L'appropriation \\
L'approximation \\
La pratique de l'espace \\
La pratique des sciences met-elle à l'abri des préjugés ? \\
La précarité \\
La précaution \\
La précaution peut-elle être un principe ? \\
La précision \\
La première fois \\
La première vérité \\
La présence \\
La présence de l'œuvre d'art \\
La présence d'esprit \\
La présence du passé \\
La présomption \\
La pression du groupe \\
La preuve \\
La preuve de l'existence de Dieu \\
La preuve expérimentale \\
La prévision \\
La prière \\
L'\emph{a priori} \\
La prise de parti est-elle essentielle en politique ? \\
La prise du pouvoir \\
La prison \\
La prison est-elle utile ? \\
La privation \\
La privation de liberté \\
La probabilité \\
La probité \\
La productivité de l'art \\
La profondeur \\
La prohibition de l'inceste \\
La promenade \\
La promesse \\
La promesse et le contrat \\
L'à propos \\
La proposition \\
La propriété \\
La propriété, est-ce un vol ? \\
La propriété est-elle un droit ? \\
La propriété est-elle une garantie de liberté ? \\
La propriété et le travail \\
La prose du monde \\
La protection \\
La protection sociale \\
La providence \\
La prudence \\
La psychanalyse est-elle une science ? \\
La psychologie est-elle une science de la nature ? \\
La psychologie est-elle une science ? \\
La publicité \\
La pudeur \\
La puissance \\
La puissance de la technique \\
La puissance de l'État \\
La puissance de l'image \\
La puissance de l'imagination \\
La puissance des contraires \\
La puissance des images \\
La puissance du langage \\
La puissance du peuple \\
La puissance et l'acte \\
La pulsion \\
La punition \\
La pureté \\
La qualité \\
La quantité \\
La quantité et la qualité \\
La question de l'essence \\
La question de l'œuvre d'art \\
La question de l'origine \\
La question des origines \\
La question sociale \\
La question « qui suis-je » admet-elle une réponse exacte ? \\
La question : « qui ? » \\
La radicalité \\
La radicalité est-elle une exigence philosophique ? \\
La raison \\
La raison a-t-elle des limites ? \\
La raison a-t-elle le droit d'expliquer ce que morale condamne ? \\
La raison a-t-elle pour fin la connaissance ? \\
La raison a-t-elle une histoire ? \\
La raison des mythes \\
La raison d'état \\
La raison d'État \\
La raison d'État peut-elle être justifiée ? \\
La raison d'être \\
La raison doit-elle critiquer la croyance ? \\
La raison doit-elle être cultivée ? \\
La raison doit-elle être notre guide ? \\
La raison doit-elle se soumettre au réel ? \\
La raison du plus fort \\
La raison engendre-t-elle des illusions ? \\
La raison épuise-t-elle le réel ? \\
La raison est-elle le pouvoir de distinguer le vrai du faux ? \\
La raison est-elle l'esclave des passions ? \\
La raison est-elle l'esclave du désir ? \\
La raison est-elle morale par elle-même ? \\
La raison est-elle plus fiable que l'expérience ? \\
La raison est-elle seulement affaire de logique ? \\
La raison est-elle suffisante ? \\
La raison est-elle toujours raisonnable ? \\
La raison et le réel \\
La raison et l'expérience \\
La raison et l'irrationnel \\
La raison ne connaît-elle du réel que ce qu'elle y met d'elle-même ? \\
La raison ne veut-elle que connaître ? \\
La raison peut-elle entrer en conflit avec elle-même ? \\
La raison peut-elle errer ? \\
La raison peut-elle être immédiatement pratique ? \\
La raison peut-elle être pratique ? \\
La raison peut-elle nous commander de croire ? \\
La raison peut-elle se contredire ? \\
La raison peut-elle servir le mal ? \\
La raison pratique \\
La raison s'oppose-t-elle aux passions ? \\
La raison suffisante \\
La raison transforme-t-elle le réel ? \\
La rareté \\
La rationalité \\
La rationalité des choix politiques \\
La rationalité des comportements économiques \\
La rationalité des émotions \\
La rationalité du langage \\
La rationalité du marché \\
La rationalité en sciences sociales \\
La rationalité politique \\
L'arbitraire \\
L'arbitraire du signe \\
L'archéologie \\
L'architecte et l'ingénieur \\
L'architecture est-elle un art ? \\
L'archive \\
La réaction \\
La réaction en politique \\
La réalité \\
La réalité a-t-elle une forme logique ? \\
La réalité décrite par la science s'oppose-t-elle à la démonstration ? \\
La réalité de la vie s'épuise-t-elle dans celle des vivants ? \\
La réalité de l'espace \\
La réalité de l'idéal \\
La réalité de l'idée \\
La réalité des idées \\
La réalité des phénomènes \\
La réalité du beau \\
La réalité du désordre \\
La réalité du futur \\
La réalité du monde extérieur \\
La réalité du mouvement \\
La réalité du passé \\
La réalité du possible \\
La réalité du rêve \\
La réalité du sensible \\
La réalité du temps \\
La réalité du temps se réduit-elle à la conscience que nous en avons ? \\
La réalité est-elle une idée ? \\
La réalité n'est-elle qu'une construction ? \\
La réalité nourrit-elle la fiction ? \\
La réalité peut-elle être virtuelle ? \\
La réalité sensible \\
La réalité sociale \\
La réalité virtuelle \\
La réception de l'œuvre d'art \\
La recherche \\
La recherche de l'absolu \\
La recherche de la perfection \\
La recherche de l'authenticité \\
La recherche de la vérité \\
La recherche de la vérité dans les sciences humaines \\
La recherche de la vérité peut-elle être désintéressée ? \\
La recherche des causes \\
La recherche des invariants \\
La recherche des origines \\
La recherche d'identité \\
La recherche du bonheur \\
La recherche du bonheur est-elle un idéal égoïste ? \\
La recherche du bonheur suffit-elle à déterminer une morale ? \\
La recherche scientifique est-elle désintéressée ? \\
La réciprocité \\
La réciprocité est-elle indispensable à la communauté politique ? \\
La réconciliation \\
La reconnaissance \\
La rectitude \\
La rectitude du droit \\
La référence \\
La référence aux faits suffit-elle à garantir l'objectivité de la connaissance ? \\
La réflexion \\
La réflexion sur l'expérience participe-t-elle de l'expérience ? \\
La réforme \\
La réforme des institutions \\
La réfutation \\
La règle \\
La règle du jeu \\
La règle et l'exception \\
La régression \\
La régression à l'infini \\
La régularité \\
La relation \\
La relation de cause à effet \\
La relation de nécessité \\
La relativité \\
La religion \\
La religion a-t-elle besoin d'un dieu ? \\
La religion civile \\
La religion conduit-elle l'homme au-delà de lui-même ? \\
La religion divise-t-elle les hommes ? \\
La religion est-elle contraire à la raison ? \\
La religion est-elle fondée sur la peur de la mort ? \\
La religion est-elle la sagesse des pauvres ? \\
La religion est-elle l'asile de l'ignorance ? \\
La religion est-elle l'opium du peuple ? \\
La religion est-elle une affaire privée ? \\
La religion est-elle une consolation pour les hommes ? \\
La religion est-elle un instrument de pouvoir ? \\
La religion et la croyance \\
La religion implique-t-elle la croyance en un être divin ? \\
La religion naturelle \\
La religion n'est-elle que l'affaire des croyants ? \\
La religion n'est-elle qu'une affaire privée ? \\
La religion n'est-elle qu'un fait de culture ? \\
La religion peut-elle être civile ? \\
La religion peut-elle être naturelle ? \\
La religion peut-elle faire lien social ? \\
La religion peut-elle n'être qu'une affaire privée ? \\
La religion peut-elle suppléer la raison ? \\
La religion relie-t-elle les hommes ? \\
La religion rend-elle l'homme heureux ? \\
La religion rend-elle meilleur ? \\
La religion repose-t-elle sur une illusion ? \\
La religion se distingue-t-elle de la superstition ? \\
La réminiscence \\
La renaissance \\
La Renaissance \\
La rencontre \\
La rencontre d'autrui \\
La réparation \\
La répétition \\
La représentation \\
La représentation en politique \\
La représentation politique \\
La reproduction \\
La reproduction des œuvres d'art \\
La reproduction sociale \\
La république \\
La réputation \\
La résignation \\
La résilience \\
La résistance \\
La résistance à l'oppression \\
La résistance de la matière \\
La résolution \\
La responsabilité \\
La responsabilité collective \\
La responsabilité de l'artiste \\
La responsabilité peut-elle être collective ? \\
La responsabilité politique \\
La responsabilité politique n'est-elle le fait que de ceux qui gouvernent ? \\
La ressemblance \\
La restauration des œuvres d'art \\
La réussite \\
La révélation \\
La rêverie \\
La révolte \\
La révolte peut-elle être un droit ? \\
La révolution \\
L'argent \\
L'argent est-il la mesure de tout échange ? \\
L'argent est-il un mal nécessaire ? \\
L'argent et la valeur \\
L'argumentation \\
L'argumentation morale \\
L'argument d'autorité \\
La rhétorique \\
La rhétorique a-t-elle une valeur ? \\
La rhétorique est-elle un art ? \\
La richesse \\
La richesse du sensible \\
La richesse intérieure \\
La rigueur \\
La rigueur de la loi \\
La rigueur des lois ? \\
La rigueur morale \\
L'aristocratie \\
La rivalité \\
L'arme rhétorique \\
L'art \\
L'art abstrait \\
L'art apprend-il à percevoir ? \\
L'art a-t-il besoin de théorie ? \\
L'art a-t-il des vertus thérapeutiques ? \\
L'art a-t-il plus de valeur que la vérité ? \\
L'art a-t-il pour fin le plaisir ? \\
L'art a-t-il une fin morale ? \\
L'art a-t-il une histoire ? \\
L'art a-t-il une valeur sociale ? \\
L'art a-t-il un rôle à jouer dans l'éducation ? \\
L'art change-t-il la vie ? \\
L'art cinématographique \\
L'art contre la beauté ? \\
L'art décoratif \\
L'art d'écrire \\
L'art de faire croire \\
L'art de gouverner \\
L'art de juger \\
L'art de la discussion \\
L'art de masse \\
L'art de persuader \\
L'art des images \\
L'art de vivre \\
L'art de vivre est-il un art ? \\
L'art d'interpréter \\
L'art d'inventer \\
L'art doit-il divertir ? \\
L'art doit-il être critique ? \\
L'art doit-il nous étonner ? \\
L'art doit-il refaire le monde ? \\
L'art donne-t-il à penser ? \\
L'art donne-t-il nécessairement lieu à la production d'une œuvre ? \\
L'art dramatique \\
L'art du comédien \\
L'art du corps \\
L'art du mensonge \\
L'art du portrait \\
L'art échappe-t-il à la raison ? \\
L'art éduque-t-il la perception ? \\
L'art éduque-t-il l'homme ? \\
L'art engagé \\
L'art est-il affaire d'apparence ? \\
L'art est-il affaire de goût ? \\
L'art est-il affaire d'imagination ? \\
L'art est-il à lui-même son propre but ? \\
L'art est-il au service du beau ? \\
L'art est-il destiné à embellir ? \\
L'art est-il imitatif ? \\
L'art est-il le produit de l'inconscient ? \\
L'art est-il le règne des apparences ? \\
L'art est-il mensonger ? \\
L'art est-il méthodique ? \\
L'art est-il moins nécessaire que la science ? \\
L'art est-il subversif ? \\
L'art est-il une affaire sérieuse ? \\
L'art est-il une critique de la culture ? \\
L'art est-il une expérience de la liberté ? \\
L'art est-il une histoire ? \\
L'art est-il une promesse de bonheur ? \\
L'art est-il universel ? \\
L'art est-il un jeu ? \\
L'art est-il un langage ? \\
L'art est-il un luxe ? \\
L'art est-il un mode de connaissance ? \\
L'art est-il un modèle pour la philosophie ? \\
L'art est-il un moyen de connaître ? \\
L'art est-il un refuge ? \\
L'art et la manière \\
L'art et la morale \\
L'art et la nature \\
L'art et la nouveauté \\
L'art et la technique \\
L'art et la tradition \\
L'art et la vie \\
L'art et le beau \\
L'art et le divin \\
L'art et le jeu \\
L'art et le mouvement \\
L'art et l'éphémère \\
L'art et le réel \\
L'art et le rêve \\
L'art et le sacré \\
L'art et les arts \\
L'art et l'espace \\
L'art et le temps \\
L'art et l'illusion \\
L'art et l'immoralité \\
L'art et l'invisible \\
L'art et morale \\
L'art exprime-t-il ce que nous ne saurions dire ? \\
L'art fait-il penser ? \\
L'artifice \\
L'artificiel \\
L'art imite-t-il la nature ? \\
L'artiste \\
L'artiste a-t-il besoin de modèle ? \\
L'artiste a-t-il besoin d'une idée de l'art ? \\
L'artiste a-t-il besoin d'un public ? \\
L'artiste a-t-il une méthode ? \\
L'artiste de soi-même \\
L'artiste doit-il être de son temps ? \\
L'artiste doit-il être original ? \\
L'artiste doit-il se donner des modèles ? \\
L'artiste doit-il se soucier du goût du public ? \\
L'artiste est-il le mieux placé pour comprendre son œuvre ? \\
L'artiste est-il maître de son œuvre ? \\
L'artiste est-il souverain ? \\
L'artiste est-il un créateur ? \\
L'artiste est-il un travailleur ? \\
L'artiste et l'artisan \\
L'artiste et la sensation \\
L'artiste et la société \\
L'artiste et le savant \\
L'artiste exprime-t-il quelque chose ? \\
L'artiste peut-il se passer d'un maître ? \\
L'artiste recherche-t-il le beau ? \\
L'artiste sait-il ce qu'il fait ? \\
L'artiste travaille-t-il ? \\
L'art modifie-t-il notre rapport au réel ? \\
L'art n'est-il pas toujours politique ? \\
L'art n'est-il pas toujours religieux ? \\
L'art n'est-il qu'une question de sentiment ? \\
L'art n'est-il qu'un mode d'expression subjectif ? \\
L'art n'est qu'une affaire de goût ? \\
L'art nous détourne-t-il de la réalité ? \\
L'art nous fait-il mieux percevoir le réel ? \\
L'art nous mène-t-il au vrai ? \\
L'art nous réconcilie-t-il avec le monde ? \\
L'art ou les arts \\
L'art parachève-t-il la nature ? \\
L'art participe-t-il à la vie politique ? \\
L'art permet-il un accès au divin ? \\
L'art peut-il changer le monde \\
L'art peut-il contribuer à éduquer les hommes ? \\
L'art peut-il encore imiter la nature ? \\
L'art peut-il être abstrait ? \\
L'art peut-il être brut ? \\
L'art peut-il être conceptuel ? \\
L'art peut-il être populaire ? \\
L'art peut-il être réaliste \\
L'art peut-il être sans œuvre ? \\
L'art peut-il être utile ? \\
L'art peut-il ne pas être sacré ? \\
L'art peut-il n'être pas conceptuel ? \\
L'art peut-il nous rendre meilleurs ? \\
L'art peut-il prétendre à la vérité ? \\
L'art peut-il quelque chose contre la morale ? \\
L'art peut-il quelque chose pour la morale ? \\
L'art peut-il rendre le mouvement ? \\
L'art peut-il s'affranchir des lois ? \\
L'art peut-il s'enseigner ? \\
L'art peut-il se passer de la beauté ? \\
L'art peut-il se passer de règles ? \\
L'art peut-il se passer d'idéal ? \\
L'art peut-il se passer d'œuvres ? \\
L'art politique \\
L'art populaire \\
L'art pour l'art \\
L'art produit-il nécessairement des œuvres ? \\
L'art rend-il les hommes meilleurs ? \\
L'art s'adresse-t-il à la sensibilité ? \\
L'art s'adresse-t-il à tous ? \\
L'art sait-il montrer ce que le langage ne peut pas dire ? \\
L'art s'apparente-t-il à la philosophie ? \\
L'art s'apprend-il ? \\
L'art vise-t-il le beau ? \\
L'art : expérience, exercice ou habitude ? \\
L'art : une arithmétique sensible ? \\
La ruine \\
La rumeur \\
La rupture \\
La ruse \\
La ruse technique \\
La sacralisation de l'œuvre \\
La sagesse \\
La sagesse et la passion \\
La sagesse et l'expérience \\
La sagesse rend-elle heureux ? \\
La sainteté \\
La sanction \\
La santé \\
La santé est-elle un devoir ? \\
La santé est-elle un droit ou un devoir ? \\
La santé mentale \\
La satisfaction \\
La satisfaction des penchants \\
La scène du monde \\
La scène théâtrale \\
L'ascèse \\
L'ascétisme \\
La science admet-elle des degrés de croyance ? \\
La science a-t-elle besoin d'un critère de démarcation entre science et non science ? \\
La science a-t-elle besoin d'une méthode ? \\
La science a-t-elle besoin du principe de causalité ? \\
La science a-t-elle des limites ? \\
La science a-t-elle le monopole de la raison ? \\
La science a-t-elle le monopole de la vérité ? \\
La science a-t-elle une histoire ? \\
La science commence-t-elle avec la perception ? \\
La science commence-telle avec la perception ? \\
La science découvre-t-elle ou construit-elle son objet ? \\
La science de l'être \\
La science de l'individuel \\
La science des mœurs \\
La science dévoile-t-elle le réel ? \\
La science doit-elle se fonder sur une idée de la nature ? \\
La science doit-elle se passer de l'idée de finalité ? \\
La science du vivant peut-elle se passer de l'idée de finalité ? \\
La science est-elle austère ? \\
La science est-elle indépendante de toute métaphysique ? \\
La science est-elle le lieu de la vérité ? \\
La science est-elle une connaissance du réel ? \\
La science est-elle une langue bien faite ? \\
La science est-elle un jeu ? \\
La science et la foi \\
La science et le mythe \\
La science et les sciences \\
La science et l'irrationnel \\
La science exclut-elle l'imagination ? \\
La science n'est-elle qu'une activité théorique ? \\
La science n'est-elle qu'une fiction ? \\
La science nous éloigne-t-elle de la religion ? \\
La science nous éloigne-t-elle des choses ? \\
La science nous indique-t-elle ce que nous devons faire ? \\
La science pense-t-elle ? \\
La science permet-elle de comprendre le monde ? \\
La science peut-elle être une métaphysique ? \\
La science peut-elle guider notre conduite ? \\
La science peut-elle lutter contre les préjugés ? \\
La science peut-elle produire des croyances ? \\
La science peut-elle se passer de fondement ? \\
La science peut-elle se passer de l'idée de finalité ? \\
La science peut-elle se passer de métaphysique ? \\
La science peut-elle se passer d'hypothèses ? \\
La science peut-elle se passer d'institutions ? \\
La science peut-elle tout expliquer ? \\
La science politique \\
La science porte-elle au scepticisme ? \\
La science procède-t-elle par rectification ? \\
La science se limite-t-elle à constater les faits ? \\
La sculpture \\
La seconde nature \\
La sécularisation \\
La sécurité \\
La sécurité nationale \\
La sécurité publique \\
La séduction \\
La ségrégation \\
La sensation \\
La sensibilité \\
La séparation \\
La séparation des pouvoirs \\
La sérénité \\
La servitude \\
La servitude peut-elle être volontaire ? \\
La servitude volontaire \\
La sévérité \\
La sexualité \\
La signification \\
La signification dans l'œuvre \\
La signification des mots \\
La signification en musique \\
L'asile de l'ignorance \\
La simplicité \\
La simplicité du bien \\
La simulation \\
La sincérité \\
La singularité \\
La situation \\
La sobriété \\
La sociabilité \\
La socialisation des comportements \\
La société \\
La société civile \\
La société civile et l'État \\
La société contre l'État \\
La société des nations \\
La société des savants \\
La société doit-elle reconnaître les désirs individuels ? \\
La société du genre humain \\
La société est-elle concevable sans le travail ? \\
La société est-elle un organisme ? \\
La société et les échanges \\
La société et l'État \\
La société et l'individu \\
La société existe-t-elle ? \\
La société fait-elle l'homme ? \\
La société peut-elle être l'objet d'une science ? \\
La société peut-elle se passer de l'État ? \\
La société précède-t-elle l'individu ? \\
La société repose-t-elle sur l'altruisme ? \\
La société sans l'État \\
La sociologie de l'art nous permet-elle de comprendre l'art ? \\
La sociologie relativise-t-elle la valeur des œuvres d'art ? \\
La solidarité \\
La solidarité est-elle naturelle ? \\
La solitude \\
La solitude constitue-t-elle un obstacle à la citoyenneté ? \\
La sollicitude \\
La somme et le tout \\
La souffrance \\
La souffrance a-t-elle une valeur morale ? \\
La souffrance a-t-elle un sens moral ? \\
La souffrance a-t-elle un sens ? \\
La souffrance au travail \\
La souffrance d'autrui \\
La souffrance d'autrui m'importe-t-elle ? \\
La souffrance des animaux \\
La souffrance morale \\
La souffrance peut-elle être un mode de connaissance ? \\
La soumission à l'autorité \\
La souveraineté \\
La souveraineté de l'État \\
La souveraineté du peuple \\
La souveraineté peut-elle être déléguée \\
La souveraineté peut-elle être limitée ? \\
La souveraineté peut-elle se partager ? \\
La souveraineté populaire \\
La spécificité des sciences humaines \\
La spéculation \\
La sphère privée échappe-t-elle au politique ? \\
L'aspiration esthétique \\
La spontanéité \\
L'assentiment \\
L'association \\
L'association des idées \\
La standardisation \\
La structure \\
La structure et le sujet \\
La subjectivité \\
La substance \\
La subtilité \\
La succession des théories scientifiques \\
La superstition \\
La sûreté \\
La surface et la profondeur \\
La surprise \\
La surveillance de la société \\
La survie \\
La sympathie \\
La sympathie peut-elle tenir lieu de moralité ? \\
La table rase \\
La tâche d'exister \\
La tautologie \\
La technique \\
La technique accroît-elle notre liberté ? \\
La technique a-t-elle sa place en politique ? \\
La technique a-t-elle une histoire ? \\
La technique crée-t-elle son propre monde ? \\
La technique est-elle civilisatrice ? \\
La technique est-elle contre-nature ? \\
La technique est-elle dangereuse ? \\
La technique est-elle l'application de la science ? \\
La technique est-elle le propre de l'homme ? \\
La technique est-elle libératrice ? \\
La technique est-elle moralement neutre ? \\
La technique est-elle neutre ? \\
La technique est-elle un savoir ? \\
La technique et le corps \\
La technique et le travail \\
La technique fait-elle des miracles ? \\
La technique fait-elle violence à la nature ? \\
La technique libère-t-elle les hommes ? \\
La technique ne fait-elle qu'appliquer la science ? \\
La technique ne pose-t-elle que des problèmes techniques ? \\
La technique n'est-elle pour l'homme qu'un moyen ? \\
La technique n'est-elle qu'une application de la science ? \\
La technique n'est-elle qu'un outil au service de l'homme ? \\
La technique n'existe-elle que pour satisfaire des besoins ? \\
La technique nous éloigne-t-elle de la nature ? \\
La technique nous éloigne-t-elle de la réalité ? \\
La technique nous libère-t-elle ? \\
La technique nous oppose-t-elle à la nature ? \\
La technique nous permet-elle de comprendre la nature ? \\
La technique permet-elle de réaliser tous les désirs ? \\
La technique peut-elle améliorer l'homme ? \\
La technique peut-elle se déduire de la science ? \\
La technique peut-elle se passer de la science ? \\
La technique repose-t-elle sur le génie du technicien ? \\
La technique sert-elle nos désirs ? \\
La technocratie \\
La technologie modifie-t-elle les rapports sociaux ? \\
La téléologie \\
La télévision \\
La tempérance \\
La temporalité de l'œuvre d'art \\
La tendance \\
La tentation \\
La tentation réductionniste \\
La terre \\
La Terre et le Ciel \\
La terreur \\
La terreur morale \\
L'athéisme \\
La théogonie \\
La théologie rationnelle \\
La théorie \\
La théorie et la pratique \\
La théorie et l'expérience \\
La théorie nous éloigne-t-elle de la réalité ? \\
La théorie peut-elle nous égarer ? \\
La théorie scientifique \\
La tolérance \\
La tolérance a-t-elle des limites ? \\
La tolérance envers les intolérants \\
La tolérance est-elle un concept politique ? \\
La tolérance est-elle une vertu ? \\
La tolérance peut-elle constituer un problème pour la démocratie ? \\
L'atome \\
La totalitarisme \\
La totalité \\
La toute puissance \\
La toute-puissance \\
La toute puissance de la pensée \\
La trace \\
La trace et l'indice \\
La tradition \\
La traduction \\
La tragédie \\
La trahison \\
La tranquillité \\
La transcendance \\
La transe \\
La transgression \\
La transgression des règles \\
La transmission \\
La transmission de pensée \\
La transparence \\
La transparence des consciences \\
La transparence est-elle un idéal démocratique ? \\
La tristesse \\
L'attachement \\
L'attente \\
L'attention \\
L'attention caractérise-t-elle la conscience ? \\
L'attraction \\
L'attrait du beau \\
La tyrannie \\
La tyrannie de la majorité \\
La tyrannie des désirs \\
La tyrannie du bonheur \\
L'audace \\
L'audace politique \\
L'au-delà \\
L'au-delà de l'être \\
L'autarcie \\
L'auteur et le créateur \\
L'authenticité \\
L'authenticité de l'œuvre d'art \\
L'autobiographie \\
L'autocritique \\
L'automate \\
L'automatisation \\
L'automatisation du raisonnement \\
L'autonomie \\
L'autonomie de l'art \\
L'autonomie de l'œuvre d'art \\
L'autonomie du théorique \\
L'autoportrait \\
L'autorité \\
L'autorité de la loi \\
L'autorité de la parole \\
L'autorité de la science \\
L'autorité de l'écrit \\
L'autorité de l'État \\
L'autorité des lois \\
L'autorité du droit \\
L'autorité morale \\
L'autorité politique \\
L'autre est-il le fondement de la conscience morale ? \\
L'autre et les autres \\
L'autre monde \\
La valeur \\
La valeur d'échange \\
La valeur de la pitié \\
La valeur de l'argent \\
La valeur de l'art \\
La valeur de la science \\
La valeur de la vérité \\
La valeur de la vie \\
La valeur de l'échange \\
La valeur de l'exemple \\
La valeur de l'hypothèse \\
La valeur des choses \\
La valeur des images \\
La valeur du consensus \\
La valeur du consentement \\
La valeur d'une action se mesure-t-elle à sa réussite ? \\
La valeur d'une œuvre \\
La valeur d'une théorie scientifique se mesure-t-elle à son efficacité ? \\
La valeur du plaisir \\
La valeur du témoignage \\
La valeur du temps \\
La valeur du travail \\
La valeur et le prix \\
La valeur morale \\
La valeur morale de l'amour \\
La valeur morale d'une action se juge-t-elle à ses conséquences ? \\
La validité \\
La vanité \\
La vanité est-elle toujours sans objet ? \\
L'avant-garde \\
L'avarice \\
La variété \\
La veille et le sommeil \\
La vénalité \\
La vengeance \\
L'avenir \\
L'avenir a-t-il une réalité ? \\
L'avenir de l'humanité \\
L'avenir est-il imaginable ? \\
L'avenir existe-t-il ? \\
L'avenir peut-il être objet de connaissance ? \\
L'aventure \\
La véracité \\
La vérification \\
La vérification expérimentale \\
La vérité \\
La vérité admet-elle des degrés ? \\
La vérité a-t-elle une histoire ? \\
La vérité de la fiction \\
La vérité de la perception \\
La vérité de l'apparence \\
La vérité de la religion \\
La vérité demande-t-elle du courage ? \\
La vérité des arts \\
La vérité des images \\
La vérité des sciences \\
La vérité doit-elle toujours être démontrée ? \\
La vérité donne-t-elle le droit d'être injuste ? \\
La vérité du déterminisme \\
La vérité d'une théorie dépend-elle de sa correspondance avec les faits ? \\
La vérité du roman \\
La vérité échappe-t-elle au temps ? \\
La vérité en art \\
La vérité est-elle affaire de cohérence ? \\
La vérité est-elle affaire de croyance ou de savoir ? \\
La vérité est-elle contraignante ? \\
La vérité est-elle éternelle ? \\
La vérité est-elle intemporelle ? \\
La vérité est-elle libératrice ? \\
La vérité est-elle morale ? \\
La vérité est-elle objective ? \\
La vérité est-elle triste ? \\
La vérité est-elle une construction ? \\
La vérité est-elle une valeur ? \\
La vérité est-elle une ? \\
La vérité historique \\
La vérité mathématique \\
La vérité n'est-elle qu'une erreur rectifiée ? \\
La vérité nous contraint-elle ? \\
La vérité peut-elle être équivoque ? \\
La vérité peut-elle être tolérante ? \\
La vérité peut-elle laisser indifférent ? \\
La vérité peut-elle se définir par le consensus ? \\
La vérité philosophique \\
La vérité rend-elle heureux ? \\
La vérité scientifique est-elle relative ? \\
La vérité se communique-t-elle ? \\
La vertu \\
La vertu de l'homme politique \\
La vertu du citoyen \\
La vertu du plaisir \\
La vertu, les vertus \\
La vertu peut-elle être excessive ? \\
La vertu peut-elle être purement morale ? \\
La vertu peut-elle s'enseigner ? \\
La vertu politique \\
La vertu s'enseigne-t-elle ? \\
L'aveu \\
L'aveu diminue-t-il la faute ? \\
L'aveuglement \\
La vie \\
La vie active \\
La vie a-t-elle un sens ? \\
La vie collective est-elle nécessairement frustrante ? \\
La vie de la langue \\
La vie de l'esprit \\
La vie de plaisirs \\
La vie des machines \\
La vie des rêves \\
La vie du droit \\
La vie en société est-elle naturelle à l'homme ? \\
La vie en société impose-t-elle de n'être pas soi-même ? \\
La vie est-elle la valeur suprême ? \\
La vie est-elle le bien le plus précieux ? \\
La vie est-elle l'objet des sciences de la vie ? \\
La vie est-elle objet de science ? \\
La vie est-elle sacrée ? \\
La vie est-elle une valeur ? \\
La vie est-elle un roman ? \\
La vie est-elle un songe ? \\
La vie éternelle \\
La vie heureuse \\
La vieillesse \\
La vie intérieure \\
La vie moderne \\
La vie morale \\
La vie ordinaire \\
La vie peut-elle être éternelle ? \\
La vie peut-elle être objet de science ? \\
La vie politique \\
La vie politique est-elle aliénante ? \\
La vie privée \\
La vie psychique \\
La vie quotidienne \\
La vie sauvage \\
La vie sexuelle est-elle volontaire ? \\
La vie sociale \\
La vie sociale est-elle toujours conflictuelle ? \\
La vie sociale est-elle une comédie ? \\
La vigilance \\
La ville \\
La ville et la campagne \\
La violence \\
La violence a-t-elle des degrés ? \\
La violence de l'État \\
La violence d'État \\
La violence est-elle toujours destructrice ? \\
La violence peut-elle avoir raison ? \\
La violence peut-elle être gratuite ? \\
La violence politique \\
La violence révolutionnaire \\
La violence sociale \\
La violence verbale \\
La virtualité \\
La virtuosité \\
La vision et le toucher \\
La vision peut-elle être le modèle de toute connaissance ? \\
La vitesse \\
L'avocat du diable \\
La vocation \\
La voix \\
La voix de la conscience \\
La voix de la raison \\
La voix du peuple \\
La volonté constitue-t-elle le principe de la politique ? \\
La volonté de savoir \\
La volonté du peuple \\
La volonté et le désir \\
La volonté générale \\
La volonté générale est-elle la volonté de tous ? \\
La volonté peut-elle être collective ? \\
La volonté peut-elle être générale ? \\
La volonté peut-elle être indéterminée ? \\
La volonté peut-elle nous manquer ? \\
La volupté \\
La vraie morale se moque-t-elle de la morale ? \\
La vraie vie \\
La vraisemblance \\
La vue et le toucher \\
La vue et l'ouïe \\
La vulgarisation \\
La vulgarité \\
La vulnérabilité \\
L'axiome \\
Le barbare \\
Le baroque \\
Le bavardage \\
Le beau a-t-il une histoire ? \\
Le beau est-il aimable ? \\
Le beau est-il toujours moral ? \\
Le beau est-il une valeur commune ? \\
Le beau est-il universel ? \\
Le beau et l'agréable \\
Le beau et le bien \\
Le beau et le bien sont-ils, au fond, identiques ? \\
Le beau et le joli \\
Le beau et le sublime \\
Le beau et l'utile \\
Le beau geste \\
Le beau naturel \\
Le beau peut-il être bizarre ? \\
Le bénéfice du doute \\
Le besoin \\
Le besoin d'absolu \\
Le besoin de métaphysique est-il un besoin de connaissance ? \\
Le besoin de philosophie \\
Le besoin de reconnaissance \\
Le besoin de sens \\
Le besoin de signes \\
Le besoin de théorie \\
Le besoin de vérité ? \\
Le besoin et le désir \\
Le bien commun \\
Le bien commun est-il une illusion ? \\
Le bien commun et l'intérêt de tous \\
Le bien d'autrui \\
Le bien est-ce l'utile ? \\
Le bien est-il relatif ? \\
Le bien et le beau \\
Le bien et le mal \\
Le bien et les biens \\
Le bien et l'utile \\
Le bien-être \\
Le bien public \\
Le bien suppose-t-il la transcendance ? \\
L'éblouissement \\
Le bon Dieu \\
Le bon et l'utile \\
Le bon goût \\
Le bon gouvernement \\
Le bonheur \\
Le bonheur collectif \\
Le bonheur dans le mal \\
Le bonheur de la passion est-il sans lendemain ? \\
Le bonheur des autres \\
Le bonheur des citoyens est-il un idéal politique ? \\
Le bonheur des méchants \\
Le bonheur des sens \\
Le bonheur des uns, le malheur des autres \\
Le bonheur du juste \\
Le bonheur est-il affaire de vertu ? \\
Le bonheur est-il affaire de volonté ? \\
Le bonheur est-il affaire privée ? \\
Le bonheur est-il au nombre de nos devoirs ? \\
Le bonheur est-il dans l'inconscience ? \\
Le bonheur est-il l'absence de maux ? \\
Le bonheur est-il la fin de la vie ? \\
Le bonheur est-il le bien suprême ? \\
Le bonheur est-il le but de la politique ? \\
Le bonheur est-il le prix de la vertu ? \\
Le bonheur est-il un accident ? \\
Le bonheur est-il un but politique ? \\
Le bonheur est-il un droit ? \\
Le bonheur est-il une affaire privée ? \\
Le bonheur est-il une fin morale ? \\
Le bonheur est-il une fin politique ? \\
Le bonheur est-il une valeur morale ? \\
Le bonheur est-il un idéal ? \\
Le bonheur est-il un principe politique ? \\
Le bonheur et la raison \\
Le bonheur et la technique \\
Le bonheur et la vertu \\
Le bonheur n'est-il qu'une idée ? \\
Le bonheur n'est-il qu'un idéal ? \\
Le bonheur peut-il être collectif ? \\
Le bonheur peut-il être le but de la politique ? \\
Le bonheur peut-il être un droit ? \\
Le bonheur se calcule-t-il ? \\
Le bonheur se mérite-t-il ? \\
Le bon régime \\
Le bon sens \\
Le bon usage des passions \\
Le bouc émissaire \\
Le bourgeois et le citoyen \\
Le bricolage \\
Le bruit \\
Le bruit et la musique \\
Le but de l'association politique \\
Le cadavre \\
Le calcul \\
Le calcul des plaisirs \\
Le calendrier \\
Le cannibalisme \\
Le capitalisme \\
Le capital social \\
Le caractère \\
Le caractère sacré de la vie \\
L'écart \\
Le cas de conscience \\
Le cas particulier \\
Le catéchisme moral \\
Le cerveau et la pensée \\
Le cerveau pense-t-il ? \\
L'échange \\
L'échange constitue-t-il un lien social ? \\
L'échange des marchandises et les rapports humains \\
L'échange économique fonde-t-il la société humaine \\
L'échange est-il un facteur de paix ? \\
L'échange et l'usage \\
Le changement \\
L'échange n'a-t-il de fondement qu'économique ? \\
L'échange ne porte-t-il que sur les choses ? \\
L'échange peut-il être désintéressé ? \\
L'échange symbolique \\
Le chant \\
Le chaos \\
Le charisme en politique \\
Le charme \\
Le charme et la grâce \\
Le châtiment \\
Le chef \\
Le chef d'œuvre \\
Le chef-d'œuvre \\
Le chemin \\
Le choc esthétique \\
Le choix \\
Le choix de philosopher \\
Le choix des moyens \\
Le choix d'un destin \\
Le choix d'un métier \\
Le choix et la liberté \\
Le choix peut-il être éclairé ? \\
Le ciel et la terre \\
Le cinéma, art de la représentation ? \\
Le cinéma est-il un art comme les autres ? \\
Le cinéma est-il un art ou une industrie ? \\
Le cinéma est-il un art populaire ? \\
Le cinéma est-il un art ? \\
Le citoyen \\
Le citoyen a-t-il perdu toute naturalité ?L'étranger \\
Le citoyen peut-il être à la fois libre et soumis à l'État ? \\
Le clair et l'obscur \\
Le classicisme \\
L'éclat \\
Le cliché \\
Le cœur \\
Le cœur et la raison \\
L'école de la vie \\
L'école des vertus \\
L'écologie est-elle un problème politique ? \\
L'écologie politique \\
L'écologie, une science humaine ? \\
Le combat \\
Le combat contre l'injustice a-t-il une source morale ? \\
Le comédien \\
Le comique \\
Le comique et le tragique \\
Le commencement \\
Le comment et le pourquoi \\
Le commerce \\
Le commerce adoucit-il les mœurs ? \\
Le commerce des hommes \\
Le commerce des idées \\
Le commerce équitable \\
Le commerce est-il pacificateur ? \\
Le commerce peut-il être équitable ? \\
Le commerce unit-il les hommes ? \\
Le commun \\
Le commun et le propre \\
Le comparatisme dans les sciences humaines \\
Le complexe \\
Le comportement \\
Le compromis \\
Le concept \\
Le concept de matière \\
Le concept de nature est-il un concept scientifique ? \\
Le concept de pulsion \\
Le concept de structure \\
Le concept de structure sociale \\
Le concept d'inconscient est-il nécessaire en sciences humaines ? \\
Le concept et l'exemple \\
Le concret \\
Le concret et l'abstrait \\
Le conditionnel \\
Le conflit \\
Le conflit de devoirs \\
Le conflit des devoirs \\
Le conflit des interprétations \\
Le conflit entre la science et la religion est-il inévitable ? \\
Le conflit est-il constitutif de la politique ? \\
Le conflit est-il la raison d'être de la politique ? \\
Le conflit est-il une maladie sociale ? \\
Le conflit ? \\
Le conformisme \\
Le conformisme moral \\
Le conformisme social \\
Le confort intellectuel \\
L'économie \\
L'économie a-t-elle des lois ? \\
L'économie des moyens \\
L'économie est-elle une science humaine ? \\
L'économie est-elle une science ? \\
L'économie et la politique \\
L'économie politique \\
L'économie psychique \\
L'économique et le politique \\
Le conscient et l'inconscient \\
Le conseil \\
Le conseiller du prince \\
Le consensus \\
Le consentement \\
Le consentement des gouvernés \\
Le conservatisme \\
Le contentement \\
Le contingent \\
Le continu \\
Le contrat \\
Le contrat de travail \\
Le contrat est-il au fondement de la politique ? \\
Le contrôle social \\
Le convenable \\
Le corps dansant \\
Le corps dit-il quelque chose ? \\
Le corps du travailleur \\
Le corps est-il le reflet de l'âme ? \\
Le corps est-il négociable ? \\
Le corps est-il porteur de valeurs ? \\
Le corps est-il respectable ? \\
Le corps et la machine \\
Le corps et l'âme \\
Le corps et l'esprit \\
Le corps et le temps \\
Le corps et l'instrument \\
Le corps humain \\
Le corps humain est-il naturel ? \\
Le corps impose-t-il des perspectives ? \\
Le corps n'est-il que matière ? \\
Le corps n'est-il qu'un mécanisme ? \\
Le corps obéit-il à l'esprit ? \\
Le corps pense-t-il ? \\
Le corps peut-il être objet d'art ? \\
Le corps politique \\
Le corps propre \\
Le cosmopolitisme \\
Le cosmopolitisme peut-il devenir réalité ? \\
Le cosmopolitisme peut-il être réaliste ? \\
Le coup d'État \\
Le courage \\
Le courage de penser \\
Le courage politique \\
Le cours des choses \\
Le cours du temps \\
Le créé et l'incréé \\
Le cri \\
Le crime \\
Le crime contre l'humanité \\
Le crime inexpiable \\
Le critère \\
L'écrit et l'oral \\
Le critique d'art \\
L'écriture \\
L'écriture des lois \\
L'écriture et la parole \\
L'écriture et la pensée \\
L'écriture ne sert-elle qu'à consigner la pensée ? \\
L'écriture peut-elle porter secours à la pensée ? \\
Le culte des ancêtres \\
Le cynisme \\
Le dandysme \\
Le danger \\
Le débat \\
Le débat politique \\
Le déchet \\
Le dedans et le dehors \\
Le défaut \\
Le dégoût \\
Le déguisement \\
Le délire \\
Le démoniaque \\
Le dépaysement \\
Le dérèglement \\
Le dernier mot \\
Le désaccord \\
Le désenchantement \\
Le désespoir \\
Le désespoir est-il une faute morale ? \\
Le déshonneur \\
Le design \\
Le désintéressement \\
Le désintéressement esthétique \\
Le désir \\
Le désir a-t-il un objet ? \\
Le désir d'absolu \\
Le désir de connaître \\
Le désir d'égalité \\
Le désir de gloire \\
Le désir de l'autre \\
Le désir de pouvoir \\
Le désir de reconnaissance \\
Le désir de savoir \\
Le désir de savoir est-il naturel ? \\
Le désir d'éternité \\
Le désir d'être autre \\
Le désir de vérité \\
Le désir de vivre \\
Le désir d'immortalité \\
Le désir d'originalité \\
Le désir du bonheur est-il universel ? \\
Le désir est-il aveugle ? \\
Le désir est-il ce qui nous fait vivre ? \\
Le désir est-il désir de l'autre ? \\
Le désir est-il le signe d'un manque ? \\
Le désir est-il l'essence de l'homme ? \\
Le désir est-il nécessairement l'expression d'un manque ? \\
Le désir est-il par nature illimité ? \\
Le désir est-il sans limite ? \\
Le désir et la culpabilité \\
Le désir et la loi \\
Le désir et le besoin \\
Le désir et le mal \\
Le désir et le manque \\
Le désir et le rêve \\
Le désir et le temps \\
Le désir et le travail \\
Le désir et l'interdit \\
Le désir métaphysique \\
Le désir n'est-il que l'épreuve d'un manque ? \\
Le désir n'est-il que manque ? \\
Le désir n'est-il qu'inquiétude ? \\
Le désir peut-il être désintéressé ? \\
Le désir peut-il ne pas avoir d'objet ? \\
Le désir peut-il nous rendre libre ? \\
Le désir peut-il se satisfaire de la réalité ? \\
Le désœuvrement \\
Le désordre \\
Le désordre des choses \\
Le despote peut-il être éclairé ? \\
Le despotisme \\
Le dessin et la couleur \\
Le destin \\
Le désuet \\
Le détachement \\
Le détail \\
Le déterminisme \\
Le déterminisme social \\
Le deuil \\
Le devenir \\
Le devoir \\
Le devoir d'aimer \\
Le devoir de mémoire \\
Le devoir d'obéissance \\
Le devoir est-il l'expression de la contrainte sociale ? \\
Le devoir et la dette \\
Le devoir et le bonheur \\
Le devoir-être \\
Le devoir rend-il libre ? \\
Le devoir se présente-t-il avec la force de l'évidence ? \\
Le devoir supprime-t-il la liberté ? \\
Le dévouement \\
Le diable \\
Le dialogue \\
Le dialogue des philosophes \\
Le dialogue entre les cultures \\
Le dialogue entre nations \\
Le dialogue suffit-il à rompre la solitude ? \\
Le dictionnaire \\
Le dieu artiste \\
Le dieu des philosophes \\
Le Dieu des philosophes \\
L'édification morale \\
Le dilemme \\
Le dire et le faire \\
Le discernement \\
Le discontinu \\
Le discours politique \\
Le divers \\
Le divertissement \\
Le divin \\
Le dogmatisme \\
Le don \\
Le don de soi \\
Le don est-il toujours généreux ? \\
Le don est-il une modalité de l'échange ? \\
Le don et la dette \\
Le don et l'échange \\
Le donné \\
Le double \\
Le doute \\
Le doute dans les sciences \\
Le doute est-il le principe de la méthode scientifique ? \\
Le doute est-il une faiblesse de la pensée ? \\
Le doute peut-il être méthodique ? \\
Le drame \\
Le droit \\
Le droit à la citoyenneté \\
Le droit à la différence met-il en péril l'égalité des droits ? \\
Le droit à la paresse \\
Le droit à la révolte \\
Le droit à l'erreur \\
Le droit au bonheur \\
Le droit au Bonheur \\
Le droit au respect de la vie privée \\
Le droit au travail \\
Le droit d'auteur \\
Le droit de la guerre \\
Le droit de mentir \\
Le droit de propriété \\
Le droit de punir \\
Le droit de résistance \\
Le droit de révolte \\
Le droit des animaux \\
Le droit des gens \\
Le droit des peuples à disposer d'eux-mêmes \\
Le droit de veto \\
Le droit de vie et de mort \\
Le droit de vivre \\
Le droit d'ingérence \\
Le droit d'intervention \\
Le droit divin \\
Le droit doit-il être indépendant de la morale ? \\
Le droit doit-il être le seul régulateur de la vie sociale ? \\
Le droit du plus faible \\
Le droit du plus fort \\
Le droit du premier occupant \\
Le droit est-il facteur de paix ? \\
Le droit est-il une science humaine ? \\
Le droit est-il une science ? \\
Le droit et la convention \\
Le droit et la force \\
Le droit et la liberté \\
Le droit et la loi \\
Le droit et la morale \\
Le droit et le devoir \\
Le Droit et l'État \\
Le droit humanitaire \\
Le droit international \\
Le droit naturel \\
Le droit n'est-il qu'une justice par défaut ? \\
Le droit peut-il échapper à l'histoire ? \\
Le droit peut-il être flexible ? \\
Le droit peut-il être naturel ? \\
Le droit peut-il se fonder sur la force ? \\
Le droit peut-il se passer de la morale ? \\
Le droit positif \\
Le droit sert-il à établir l'ordre ou la justice ? \\
Le dualisme \\
L'éducation artistique \\
L'éducation civique \\
L'éducation des esprits \\
L'éducation du goût \\
L'éducation esthétique \\
L'éducation peut-elle être sentimentale ? \\
L'éducation physique \\
L'éducation politique \\
Le factice \\
Le fait \\
Le fait de vivre constitue-t-il un bien en soi ? \\
Le fait de vivre est-il un bien en soi ? \\
Le fait d'exister \\
Le fait divers \\
Le fait et le droit \\
Le fait et l'événement \\
Le fait religieux \\
Le fait scientifique \\
Le fait social est-il une chose ? \\
Le familier \\
Le fanatisme \\
Le fantasme \\
Le fantastique \\
Le fatalisme \\
Le fatalisme l'incarnation \\
Le faux \\
Le faux en art \\
Le faux et l'absurde \\
Le faux et le fictif \\
Le féminin \\
Le féminin et le masculin \\
Le féminisme \\
Le fétichisme \\
Le fétichisme de la marchandise \\
L'effectivité \\
L'effet et la cause \\
L'efficacité \\
L'efficacité des discours \\
L'efficacité thérapeutique de la psychanalyse \\
L'efficience \\
L'effort \\
L'effort moral \\
Le fil conducteur \\
Le finalisme \\
Le fini \\
Le fini et l'infini \\
Le fin mot de l'histoire \\
Le flegme \\
Le fond \\
Le fondement \\
Le fondement de l'autorité \\
Le fondement de l'induction \\
Le fond et la forme \\
Le for intérieur \\
Le formalisme \\
Le formalisme moral \\
Le fou \\
Le fragment \\
Le frivole \\
Le futur est-il contingent ? \\
L'égalité \\
L'égalité civile \\
L'égalité des chances \\
L'égalité des citoyens \\
L'égalité des conditions \\
L'égalité des hommes et des femmes est-elle une question politique ? \\
L'égalité des sexes \\
L'égalité devant la loi \\
L'égalité est-elle souhaitable ? \\
L'égalité est-elle toujours juste ? \\
L'égalité est-elle une condition de la liberté ? \\
Légalité et causalité \\
Légalité et légitimité \\
Légalité et moralité \\
L'égalité peut-elle être une menace pour la liberté ? \\
L'égarement \\
Le général et le particulier \\
Le génie \\
Le génie du lieu \\
Le génie du mal \\
Le génie est-il la marque de l'excellence artistique ? \\
Le génie et la règle \\
Le génie et le savant \\
Le genre \\
Le genre et l'espèce \\
Le genre humain \\
Le geste \\
Le geste créateur \\
Le geste et la parole \\
Légitimité et légalité \\
L'égoïsme \\
Le goût \\
Le goût de la liberté \\
Le goût de la polémique \\
Le goût des autres \\
Le goût du beau \\
Le goût du pouvoir \\
Le goût du risque \\
Le goût est-il affaire d'éducation ? \\
Le goût est-il une faculté ? \\
Le goût est-il une vertu sociale ? \\
Le goût s'éduque-t-il ? \\
Le goût se forme-t-il ? \\
Le goût : certitude ou conviction ? \\
Le gouvernement des experts \\
Le gouvernement des hommes et l'administration des choses \\
Le gouvernement des hommes libres \\
Le gouvernement des meilleurs \\
Le gouvernement de soi et des autres \\
Le grand art est-il de plaire ? \\
Le handicap \\
Le hasard \\
Le hasard est-il injuste ? \\
Le hasard et la nécessité \\
Le hasard existe-t-il ? \\
Le hasard fait-il bien les choses ? \\
Le hasard n'est il que la mesure de notre ignorance ? \\
Le hasard n'est-il que la mesure de notre ignorance ? \\
Le haut \\
Le haut et le bas \\
Le héros \\
Le héros moral \\
Le hors-la-loi \\
Le je et le tu \\
Le je ne sais quoi \\
Le jeu \\
Le jeu et le divertissement \\
Le jeu et le hasard \\
Le jeu et le sérieux \\
Le jeu social \\
Le joli, le beau \\
Le juge \\
Le jugement \\
Le jugement critique peut-il s'exercer sans culture ? \\
Le jugement de goût \\
Le jugement de goût est-il désintéressé ? \\
Le jugement de goût est-il universel ? \\
Le jugement dernier \\
Le jugement de valeur \\
Le jugement de valeur est-il indifférent à la vérité ? \\
Le jugement moral \\
Le jugement politique \\
Le juste et le bien \\
Le juste et le légal \\
Le juste milieu \\
Le laboratoire \\
Le langage \\
Le langage animal \\
Le langage de la morale \\
Le langage de la peinture \\
Le langage de la pensée \\
Le langage de l'art \\
Le langage de la science \\
Le langage des sciences \\
Le langage du corps \\
Le langage est-il assimilable à un outil ? \\
Le langage est-il d'essence poétique ? \\
Le langage est-il l'auxiliaire de la pensée ? \\
Le langage est-il le lieu de la vérité ? \\
Le langage est-il logique ? \\
Le langage est-il une prise de possession des choses ? \\
Le langage est-il un instrument de connaissance ? \\
Le langage est-il un instrument ? \\
Le langage est-il un obstacle pour la pensée ? \\
Le langage et la pensée \\
Le langage et le réel \\
Le langage masque-t-il la pensée ? \\
Le langage mathématique \\
Le langage ne sert-il qu'à communiquer ? \\
Le langage n'est-il qu'un instrument de communication ? \\
Le langage peut-il être un obstacle à la recherche de la vérité ? \\
Le langage rapproche-t-il ou sépare-t-il les hommes ? \\
Le langage rend-il l'homme plus puissant ? \\
Le langage traduit-il la pensée ? \\
Le langage trahit-il la pensée ? \\
L'élection \\
Le légal et le légitime \\
L'élégance \\
Le législateur \\
Le légitime et le légal \\
L'élémentaire \\
Le libre arbitre \\
Le libre-arbitre \\
Le libre échange \\
Le lien causal \\
Le lien politique \\
Le lien social \\
Le lien social peut-il être compassionnel ? \\
Le lieu \\
Le lieu commun \\
Le lieu de la pensée \\
Le lieu de l'esprit \\
Le lieu et l'espace \\
Le littéral et le figuré \\
Le livre \\
Le livre de la nature \\
L'éloge de la démesure \\
Le logique \\
Le loisir \\
Le luxe \\
Le lyrisme \\
Le maître \\
Le maître et l'esclave \\
Le mal \\
Le mal a-t-il des raisons ? \\
Le mal constitue-t-il une objection à l'existence de Dieu ? \\
Le malentendu \\
Le mal être \\
Le mal existe-t-il ? \\
Le malheur \\
Le malheur est-il injuste ? \\
Le malin plaisir \\
Le mal métaphysique \\
L'émancipation \\
L'émancipation des femmes \\
Le maniérisme \\
Le manifeste politique \\
Le manque de culture \\
Le marché \\
Le marché de l'art \\
Le marché du travail \\
Le mariage \\
Le mariage est-il un contrat ? \\
Le masculin \\
Le masculin et le féminin \\
Le masque \\
Le matérialisme \\
Le matériel \\
Le matériel et le virtuel \\
Le mauvais goût \\
L'embarras du choix \\
Le mécanisme \\
Le mécanisme et la mécanique \\
Le méchant \\
Le méchant est-il malheureux ? \\
Le méchant peut-il être heureux ? \\
Le médiat et l'immédiat \\
Le meilleur \\
Le meilleur des mondes \\
Le meilleur des mondes possible \\
Le meilleur est-il l'ennemi du bien ? \\
Le meilleur régime \\
Le meilleur régime politique \\
Le même et l'autre \\
Le mensonge \\
Le mensonge de l'art ? \\
Le mensonge en politique \\
Le mensonge est-il la plus grande transgression ? \\
Le mensonge peut-il être au service de la vérité ? \\
Le mensonge politique \\
Le mépris \\
Le mépris peut-il être justifié ? \\
Le mérite \\
Le mérite est-il le critère de la vertu ? \\
Le mérite et les talents \\
Le merveilleux \\
Le métaphysicien est-il un physicien à sa façon ? \\
Le métier \\
Le métier de philosophe \\
Le métier de politique \\
Le métier de savant \\
Le métier d'homme \\
Le mien et le tien \\
Le mieux est-il l'ennemi du bien ? \\
Le milieu \\
Le miracle \\
Le miroir \\
Le misanthrope \\
Le mode \\
Le mode d'existence de l'œuvre d'art \\
Le modèle \\
Le modèle en morale \\
Le modèle organiciste \\
Le moi \\
Le moi est-il haïssable ? \\
Le moi est-il objet de connaissance ? \\
Le moi est-il une fiction ? \\
Le moi est-il une illusion ? \\
Le moi et la conscience \\
Le moindre mal \\
Le moi n'est-il qu'une fiction ? \\
Le moi n'est-il qu'une idée ? \\
Le monde \\
Le monde à l'envers \\
Le monde a-t-il besoin de moi ? \\
Le monde de l'animal \\
Le monde de l'art \\
Le monde de la technique \\
Le monde de la vie \\
Le monde de l'entreprise \\
Le monde des idées \\
Le monde des images \\
Le monde des machines \\
Le monde des œuvres \\
Le monde des physiciens \\
Le monde des rêves \\
Le monde des sens \\
Le monde du rêve \\
Le monde du travail \\
Le monde est-il écrit en langage mathématique ? \\
Le monde est-il éternel ? \\
Le monde est-il ma représentation ? \\
Le monde est-il une marchandise ? \\
Le monde est-il un théâtre ? \\
Le monde extérieur \\
Le monde extérieur existe-t-il ? \\
Le monde intelligible \\
Le monde intérieur \\
Le monde politique \\
Le monde sensible \\
Le monde se réduit-il à ce que nous en voyons ? \\
Le monde vrai \\
Le monopole de la violence légitime \\
Le monstre \\
Le monstrueux \\
Le moralisme \\
Le moraliste \\
Le mot d'esprit \\
Le mot et la chose \\
Le mot et le geste \\
L'émotion \\
L'émotion esthétique \\
L'émotion esthétique peut-elle se communiquer ? \\
Le mot juste \\
Le mot vie a-t-il plusieurs sens ? \\
Le mouvement \\
Le mouvement de la pensée \\
L'empathie \\
L'empathie est-elle nécessaire aux sciences sociales ? \\
L'empathie est-elle possible ? \\
L'empire \\
L'empire sur soi \\
L'empirisme \\
L'empirisme exclut-il l'abstraction ? \\
L'emploi du temps \\
Le multiculturalisme \\
Le multiple \\
Le multiple et l'un \\
Le musée \\
Le Musée \\
Le mystère \\
Le mysticisme \\
Le mythe \\
Le mythe est-il objet de science ? \\
Le naïf \\
Le narcissisme \\
Le naturalisme des sciences humaines et sociales \\
Le naturel \\
Le naturel et l'artificiel \\
Le naturel et le fabriqué \\
L'encyclopédie \\
L'Encyclopédie \\
Le néant \\
Le néant est-il ? \\
Le nécessaire et le contingent \\
Le nécessaire et le superflu \\
Le négatif \\
Le néologisme \\
L'énergie \\
L'énergie du désespoir \\
L'enfance \\
L'enfance de l'art \\
L'enfance est-elle ce qui doit être surmonté ? \\
L'enfance est-elle en nous ce qui doit être abandonné ? \\
L'enfant \\
L'enfant et l'adulte \\
L'enfer est pavé de bonnes intentions \\
L'engagement \\
L'engagement dans l'art \\
L'engagement politique \\
L'engendrement \\
L'énigme \\
Le nihilisme \\
L'ennemi \\
L'ennemi intérieur \\
L'ennui \\
Le noble et le vil \\
Le nomade \\
Le nomadisme \\
Le nombre \\
Le nombre et la mesure \\
Le nom et le verbe \\
Le nominalisme \\
Le nom propre \\
Le non-être \\
Le non-sens \\
Le normal et le pathologique \\
L'enquête \\
L'enquête de terrain \\
L'enquête empirique rend-elle la métaphysique inutile ? \\
L'enquête sociale \\
L'enseignement peut-il se passer d'exemples ? \\
L'entendement et la volonté \\
L'enthousiasme \\
L'enthousiasme est-il moral ? \\
L'entraide \\
Le nu \\
Le nu et la nudité \\
L'envie \\
L'environnement \\
L'environnement est-il un nouvel objet pour les sciences humaines ? \\
L'environnement est-il un problème politique ? \\
Le oui-dire \\
Le pacifisme \\
Le paradigme \\
Le paradoxe \\
Le pardon \\
Le pardon et l'oubli \\
Le pardon peut-il être une obligation ? \\
Le pari \\
Le partage \\
Le partage des biens \\
Le partage des connaissances \\
Le partage des savoirs \\
Le partage est-il une obligation morale ? \\
Le particulier \\
Le passage à l'acte \\
Le passé \\
Le passé a-t-il plus de réalité que l'avenir ? \\
Le passé a-t-il un intérêt ? \\
Le passé détermine-t-il notre présent ? \\
Le passé est-il ce qui a disparu ? \\
Le passé est-il réel ? \\
Le passé et le présent \\
Le passé existe-t-il ? \\
Le passé peut-il être un objet de connaissance ? \\
Le paternalisme \\
Le pathologique \\
Le patriarcat \\
Le patrimoine \\
Le patrimoine artistique \\
Le patrimoine de l'humanité \\
Le patriotisme \\
Le patriotisme est-il une vertu ? \\
Le paysage \\
Le pays natal \\
Le péché \\
Le pédagogue \\
Le personnage et la personne \\
Le pessimisme \\
Le peuple \\
Le peuple est-il bête ? \\
Le peuple et la nation \\
Le peuple et les élites \\
Le peuple peut-il se tromper ? \\
Le phantasme \\
L'éphémère \\
Le phénomène \\
Le philanthrope \\
Le philosophe a-t-il besoin de l'histoire ?Prouver et justifier \\
Le philosophe a-t-il des leçons à donner au politique ? \\
Le philosophe est-il le vrai politique ? \\
Le philosophe et le sophiste \\
Le philosophe-roi \\
Le philosophe s'écarte-t-il du réel ? \\
L'épistémologie est-elle une logique de la science ? \\
Le plagiat \\
Le plaisir \\
Le plaisir a-t-il un rôle à jouer dans la morale ? \\
Le plaisir d'avoir mal \\
Le plaisir de l'art \\
Le plaisir de parler \\
Le plaisir des sens \\
Le plaisir d'être libre \\
Le plaisir d'imiter \\
Le plaisir esthétique \\
Le plaisir esthétique peut-il se partager ? \\
Le plaisir esthétique suppose-t-il une culture ? \\
Le plaisir est-il immoral ? \\
Le plaisir est-il la fin du désir ? \\
Le plaisir est-il tout le bonheur ? \\
Le plaisir est-il un bien ? \\
Le plaisir et la douleur \\
Le plaisir et la joie \\
Le plaisir et la jouissance \\
Le plaisir et la peine \\
Le plaisir et le bien \\
Le plaisir peut-il être immoral ? \\
Le plaisir peut-il être partagé ? \\
Le plaisir suffit-il au bonheur ? \\
Le pluralisme \\
Le pluralisme politique \\
Le plus grand bonheur pour le plus grand nombre \\
Le poète réinvente-t-il la langue ? \\
Le poétique \\
Le poids de la culture \\
Le poids de la société \\
Le poids du passé \\
Le poids du préjugé en politique \\
Le point de vue \\
Le point de vue d'autrui \\
Le point de vue de l'auteur \\
Le politique a-t-il à régler les passions humaines ? \\
Le politique doit-il être un technicien ? \\
Le politique doit-il se soucier des émotions ? \\
Le politique et le religieux \\
Le politique peut-il faire abstraction de la morale ? \\
Le populaire \\
Le populisme \\
Le portrait \\
Le possible \\
Le possible et le probable \\
Le possible et le réel \\
Le possible et le virtuel \\
Le possible et l'impossible \\
Le possible existe-t-il ? \\
Le pour et le contre \\
Le pourquoi et le comment \\
Le pouvoir \\
Le pouvoir absolu \\
Le pouvoir causal de l'inconscient \\
Le pouvoir corrompt-il nécessairement ? \\
Le pouvoir corrompt-il toujours ? \\
Le pouvoir corrompt-il ? \\
Le pouvoir de la science \\
Le pouvoir de l'État est-il arbitraire ? \\
Le pouvoir de l'habitude \\
Le pouvoir de l'imagination \\
Le pouvoir de l'opinion \\
Le pouvoir des images \\
Le pouvoir des mots \\
Le pouvoir des paroles \\
Le pouvoir des sciences humaines et sociales \\
Le pouvoir du concept \\
Le pouvoir du peuple \\
Le pouvoir et l'autorité \\
Le pouvoir et la violence \\
Le pouvoir législatif \\
Le pouvoir magique \\
Le pouvoir peut-il être limité ? \\
Le pouvoir peut-il limiter le pouvoir ? \\
Le pouvoir peut-il se déléguer ? \\
Le pouvoir peut-il se passer de sa mise en scène ? \\
Le pouvoir politique est-il nécessairement coercitif ? \\
Le pouvoir politique peut-il échapper à l'arbitraire ? \\
Le pouvoir politique repose-t-il sur un savoir ? \\
Le pouvoir souverain \\
Le pouvoir traditionnel \\
Le pragmatisme \\
Le préférable \\
Le préjugé \\
Le premier \\
Le premier devoir de l'État est-il de se défendre ? \\
Le premier et le primitif \\
Le premier principe \\
Le présent \\
L'épreuve \\
L'épreuve de la liberté \\
L'épreuve du réel \\
Le primitif \\
Le primitivisme en art \\
Le prince \\
Le principe \\
Le principe de causalité \\
Le principe de contradiction \\
Le principe d'égalité \\
Le principe de non-contradiction \\
Le principe de raison \\
Le principe de raison suffisante \\
Le principe de réalité \\
Le principe de réciprocité \\
Le principe d'identité \\
Le privé et le public \\
Le privilège de l'original \\
Le prix de la liberté \\
Le prix des choses \\
Le prix du travail \\
Le probable \\
Le problème de l'être \\
Le procès d'intention \\
Le processus \\
Le processus de civilisation \\
Le prochain \\
Le proche et le lointain \\
Le profane \\
Le profit \\
Le profit est-il la fin de l'échange ? \\
Le progrès \\
Le progrès des sciences \\
Le progrès des sciences infirme-t-il les résultats anciens ? \\
Le progrès en logique \\
Le progrès est-il réversible ? \\
Le progrès est-il un mythe ? \\
Le progrès moral \\
Le progrès scientifique fait-il disparaître la superstition ? \\
Le progrès technique \\
Le progrès technique peut-il être aliénant ? \\
Le projet \\
Le projet d'une paix perpétuelle est-il insensé ? \\
Le propre \\
Le propre de la musique \\
Le propre de l'homme \\
Le propre du vivant est-il de tomber malade ? \\
Le propriétaire \\
Le provisoire \\
Le psychisme est-il objet de connaissance ? \\
Le public \\
Le public et le privé \\
L e pur et l'impur \\
Le pur et l'impur \\
Le quelconque \\
Lequel, de l'art ou du réel, est-il une imitation de l'autre ? \\
L'équilibre des pouvoirs \\
L'équité \\
L'équivalence \\
L'équivocité \\
L'équivocité du langage \\
L'équivoque \\
Le quotidien \\
Le racisme \\
Le raffinement \\
Le raisonnable et le rationnel \\
Le raisonnement par l'absurde \\
Le raisonnement scientifique \\
Le raisonnement suit-il des règles ? \\
Le rapport de l'homme à son milieu a-t-il une dimension morale ? \\
Le rationalisme \\
Le rationalisme peut-il être une passion ? \\
Le rationnel \\
Le rationnel et le raisonnable \\
Le rationnel et l'irrationnel \\
Le réalisme \\
Le réalisme de la science \\
Le récit \\
Le récit en histoire \\
Le récit historique \\
Le reconnaissance \\
Le recours à l'Histoire \\
Le réel \\
Le réel est-il ce que l'on croit ? \\
Le réel est-il ce que nous expérimentons ? \\
Le réel est-il ce que nous percevons ? \\
Le réel est-il ce qui apparaît ? \\
Le réel est-il ce qui est perçu ? \\
Le réel est-il ce qui résiste ? \\
Le réel est-il inaccessible ? \\
Le réel est-il l'objet de la science ? \\
Le réel est-il objet d'interprétation ? \\
Le réel est-il rationnel ? \\
Le réel et la fiction \\
Le réel et le matériel \\
Le réel et le nécessaire \\
Le réel et le possible \\
Le réel et le virtuel \\
Le réel et le vrai \\
Le réel et l'idéal \\
Le réel et l'imaginaire \\
Le réel et l'irréel \\
Le réel n'est-il qu'un ensemble de contraintes ? \\
Le réel obéit-il à la raison ? \\
Le réel peut-il échapper à la logique ? \\
Le réel peut-il être contradictoire ? \\
Le réel résiste-t-il à la connaissance ? \\
Le réel se limite-t-il à ce que font connaître les théories scientifiques ? \\
Le réel se limite-t-il à ce que nous percevons ? \\
Le réel se réduit-il à ce que l'on perçoit ? \\
Le réel se réduit-il à l'objectivité ? \\
Le refoulement \\
Le refus \\
Le refus de la vérité \\
Le regard \\
Le regard du photographe \\
Le règlement politique des conflits ? \\
Le règne de l'homme \\
Le règne des passions \\
Le regret \\
Le relativisme \\
Le relativisme culturel \\
Le relativisme moral \\
Le religieux est-il inutile ? \\
Le remords \\
Le renoncement \\
Le repentir \\
Le repos \\
Le respect \\
Le respect de la nature \\
Le respect des convenances \\
Le respect des institutions \\
Le respect de soi \\
Le respect de soi-même \\
Le ressentiment \\
Le retour à la nature \\
Le retour à l'expérience \\
Le rêve \\
Le rêve et la réalité \\
Le rêve et la veille \\
Le riche et le pauvre \\
Le ridicule \\
Le rien \\
Le rigorisme \\
Le rire \\
Le risque \\
Le risque de la liberté \\
Le risque technique \\
Le rite \\
Le rituel \\
Le rôle de la théorie dans l'expérience scientifique \\
Le rôle de l'État est-il de faire régner la justice ? \\
Le rôle de l'État est-il de préserver la liberté de l'individu ? \\
Le rôle de l'historien est-il de juger ? \\
Le rôle des institutions \\
Le rôle des théories est-il d'expliquer ou de décrire ? \\
Le roman \\
Le roman peut-il être philosophique ? \\
Le romantisme \\
L'érotisme \\
Le royaume du possible \\
L'erreur \\
L'erreur est-elle humaine ? \\
L'erreur et la faute \\
L'erreur et l'ignorance \\
L'erreur et l'illusion \\
L'erreur peut-elle jouer un rôle dans la connaissance scientifique ? \\
L'erreur politique, la faute politique \\
L'erreur scientifique \\
L'érudition \\
Le rythme \\
Le sacré \\
Le sacré et le profane \\
Le sacrifice \\
Le sacrifice de soi \\
Les acteurs de l'histoire en sont-ils les auteurs ? \\
Les affaires publiques \\
Les affects sont-ils déraisonnables ? \\
Les affects sont-ils des objets sociologiques ? \\
Le sage a-t-il besoin d'autrui ? \\
Les agents sociaux poursuivent-ils l'utilité ? \\
Les agents sociaux sont-ils rationnels ? \\
Les âges de la vie \\
Les âges de l'humanité \\
Le salaire \\
Le salut \\
Le salut vient-il de la raison ? \\
Les amis \\
Les analogies dans les sciences humaines \\
Les anciens et les modernes \\
Les Anciens et les Modernes \\
Les animaux échappent-ils à la moralité ? \\
Les animaux ont-ils des droits ? \\
Les animaux pensent-ils ? \\
Les animaux peuvent-ils avoir des droits ? \\
Les antagonismes sociaux \\
Les apparences font-elles partie du monde ? \\
Les apparences sont-elles toujours trompeuses ? \\
Les archives \\
Les arts admettent-ils une hiérarchie ? \\
Les arts appliqués \\
Les arts communiquent-ils entre eux ? \\
Les arts de la mémoire \\
Les arts industriels \\
Les arts mineurs \\
Les arts nobles \\
Les arts ont-ils besoin de théorie ? \\
Les arts populaires \\
Les arts vivants \\
Le sauvage \\
Le sauvage et le barbare \\
Le sauvage et le cultivé \\
Le savant et le politique \\
Le savant et l'ignorant \\
Les avant-gardes \\
Le savoir absolu \\
Le savoir a-t-il besoin d'être fondé ? \\
Le savoir a-t-il des degrés ? \\
Le savoir du peintre \\
Le savoir émancipe-t-il ? \\
Le savoir est-il libérateur ? \\
Le savoir exclut-il toute forme de croyance ? \\
Le savoir-faire \\
Le savoir rend-il libre ? \\
Le savoir se vulgarise-t-il ? \\
Le savoir utile au citoyen \\
Les beautés de la nature \\
Les beaux-arts \\
Les beaux-arts sont-ils compatibles entre eux ? \\
Les belles choses \\
Les bénéfices du doute \\
Les bénéfices moraux \\
Les besoins et les désirs \\
Les bêtes travaillent-elles ? \\
Les bienfaits de la coopération \\
Les biens communs \\
Les biotechnologies \\
Les blessures de l'esprit \\
Les bonnes intentions \\
Les bonnes manières \\
Les bonnes mœurs \\
Les bonnes résolutions \\
Les bons sentiments \\
Le scandale \\
Les caractères \\
Les caractères moraux \\
Les catastrophes \\
Les catégories \\
Les causes et les effets \\
Les causes et les lois \\
Les causes et les raisons \\
Les causes et les signes \\
Les causes finales \\
Le scepticisme \\
Le scepticisme a-t-il des limites ? \\
Les cérémonies \\
Les changements scientifiques et la réalité \\
Les chemins de traverse \\
Les choses \\
Les choses et les événements \\
Les choses ont-elles une essence ? \\
Les cinq sens \\
Les circonstances \\
Les classes sociales \\
L'esclavage \\
L'esclavage des passions \\
L'esclave \\
L'esclave et son maître \\
Les coïncidences ont-elles des causes ? \\
Les commandements divins \\
Les commencements \\
Les conditions de la démocratie \\
Les conditions d'existence \\
Les conditions du dialogue \\
Les conflits menacent-ils la société ? \\
Les conflits politiques \\
Les conflits politiques ne sont-ils que des conflits sociaux ? \\
Les conflits sociaux \\
Les conflits sociaux sont-ils des conflits de classe ? \\
Les conflits sociaux sont-ils des conflits politiques ? \\
Les connaissances scientifiques peuvent-elles être à la fois vraies et provisoires ? \\
Les connaissances scientifiques peuvent-elles être vulgarisées ? \\
Les conquêtes de la science \\
Les conséquences \\
Les conséquences de l'action \\
Les considérations morales ont-elles leur place en politique ? \\
Les convictions d'autrui sont-elles un argument ? \\
Les coutumes \\
Les critères de vérité dans les sciences humaines \\
Les croyances politiques \\
Le scrupule \\
Les cultures sont-elles incommensurables ? \\
Les degrés de conscience \\
Les degrés de la beauté \\
Les désirs et les valeurs \\
Les désirs ont-ils nécessairement un objet \\
Les devoirs à l'égard de la nature \\
Les devoirs de l'État \\
Les devoirs de l'homme varient-ils selon la culture ? \\
Les devoirs de l'homme varient-ils selon les cultures ? \\
Les devoirs du citoyen \\
Les devoirs envers soi-même \\
Les dictionnaires \\
Les dilemmes moraux \\
Les disciplines scientifiques et leurs interfaces \\
Les dispositions sociales \\
Les distinctions sociales \\
Les divisions sociales \\
Les droits de l'enfant \\
Les droits de l'homme \\
Les droits de l'homme et ceux du citoyen \\
Les droits de l'homme ont-ils un fondement moral ? \\
Les droits de l'homme sont-ils les droits de la femme ? \\
Les droits de l'homme sont-ils une abstraction ? \\
Les droits de l'individu \\
Les droits des animaux \\
Les droits et les devoirs \\
Les droits naturels imposent-ils une limite à la politique ? \\
Les échanges \\
Les échanges économiques sont-ils facteurs de paix ? \\
Les échanges, facteurs de paix ? \\
Les échanges favorisent-ils la paix ? \\
Les échanges sont-ils facteurs de paix ? \\
Les écrans \\
Le secret \\
Le secret d'État \\
Les effets de l'esclavage \\
Les éléments \\
Les élites \\
Le semblable \\
Les enfants \\
Le sensationnel \\
Le sens caché \\
Le sens commun \\
Le sens de la justice \\
Le sens de la mesure \\
Le sens de la situation \\
Le sens de la vie \\
Le sens de l'État \\
Le sens de l'existence \\
Le sens de l'histoire \\
Le sens de l'Histoire \\
Le sens de l'humour \\
Le sens des mots \\
Le sens du destin \\
Le sens du devoir \\
Le sens du silence \\
Le sens du travail \\
Les ensembles \\
Le sensible \\
Le sensible est-il communicable ? \\
Le sensible est-il irréductible à l'intelligible ? \\
Le sensible et la science \\
Le sensible et l'intelligible \\
Le sensible peut-il être connu ? \\
Le sens interne \\
Le sens moral \\
Le sens moral est-il naturel ? \\
Le sens musical \\
Le sentiment \\
Le sentiment de culpabilité \\
Le sentiment de l'existence \\
Le sentiment de liberté \\
Le sentiment de l'injustice \\
Le sentiment d'injustice \\
Le sentiment d'injustice est-il naturel ? \\
Le sentiment du juste et de l'injuste \\
Le sentiment esthétique \\
Le sentiment moral \\
Les entités mathématiques sont-elles des fictions ? \\
Les envieux \\
Le sérieux \\
Le serment \\
Le service de l'État \\
Les êtres vivants sont-ils des machines ? \\
Les études comparatives \\
Les événements historiques sont-ils de nature imprévisible ? \\
Les excuses \\
Les factions politiques \\
Les faits et les valeurs \\
Les faits existent-ils indépendamment de leur établissement par l'esprit humain ? \\
Les faits parlent-ils d'eux-mêmes ? \\
Les faits peuvent-ils faire autorité ? \\
Les fausses sciences \\
Les fins de la culture \\
Les fins de l'art \\
Les fins de la science \\
Les fins de la technique sont-elles techniques ? \\
Les fins de l'éducation \\
Les fins dernières \\
Les fins et les moyens \\
Les fins naturelles et les fins morales \\
Les fonctions de l'image \\
Les fondements de l'État \\
Les formes du vivant \\
Les forts et les faibles \\
Les foules \\
Les fous \\
Les frontières \\
Les frontières de l'art \\
Les fruits du travail \\
Les générations \\
Les genres de Dieu \\
Les genres de vie \\
Les genres esthétiques \\
Les genres naturels \\
Les grands hommes \\
Les habitudes nous forment-elles ? \\
Les hasards de la vie \\
Les héros \\
Les hommes de pouvoir \\
Les hommes et les dieux \\
Les hommes et les femmes \\
Les hommes n'agissent-ils que par intérêt ? \\
Les hommes naissent-ils libres ? \\
Les hommes ont-ils besoin de maîtres ? \\
Les hommes savent-ils ce qu'ils désirent ? \\
Les hommes sont-ils des animaux ? \\
Les hommes sont-ils faits pour s'entendre ? \\
Les hommes sont-ils frères ? \\
Les hommes sont-ils naturellement sociables ? \\
Les hommes sont-ils seulement le produit de leur culture ? \\
Les hors-la-loi \\
Les hypothèses scientifiques ont-elles pour nature d'être confirmées ou infirmées ? \\
Les idées et les choses \\
Les idées ont-elles une existence éternelle ? \\
Les idées ont-elles une histoire ? \\
Les idées ont-elles une réalité ? \\
Les idées politiques \\
Les idoles \\
Le signe \\
Le signe et le symbole \\
Le silence \\
Le silence a-t-il un sens ? \\
Le silence des lois \\
Le silence signifie-t-il toujours l'échec du langage ? \\
Les images empêchent-elles de penser ? \\
Les images ont-elles un sens ? \\
Le simple \\
Le simple et le complexe \\
Le simulacre \\
Les individus \\
Les industries culturelles \\
Les inégalités de la nature doivent-elles être compensées ? \\
Les inégalités menacent-elles la société ? \\
Les inégalités sociales \\
Les inégalités sociales sont-elles inévitables ? \\
Les inégalités sociales sont-elles naturelles ? \\
Le singulier \\
Le singulier est-il objet de connaissance ? \\
Le singulier et le pluriel \\
Les institutions artistiques \\
Les instruments de la pensée \\
Les intentions de l'artiste \\
Les intentions et les actes \\
Les intentions et les conséquences \\
Les interdits \\
Les intérêts particuliers peuvent-ils tempérer l'autorité politique ? \\
Les invariants culturels \\
Les jeux du pouvoir \\
Les jugements analytiques \\
Les langues que nous parlons sont-elles imparfaites ? \\
Les leçons de l'expérience \\
Les leçons de l'histoire \\
Les leçons de morale \\
Les lettres et les sciences \\
Les libertés civiles \\
Les libertés fondamentales \\
Les liens sociaux \\
Les lieux du pouvoir \\
Les limites de la connaissance \\
Les limites de la connaissance scientifique \\
Les limites de la démocratie \\
Les limites de la description \\
Les limites de la discussion \\
Les limites de la raison \\
Les limites de la science \\
Les limites de la tolérance \\
Les limites de la vérité \\
Les limites de la vertu \\
Les limites de l'État \\
Les limites de l'expérience \\
Les limites de l'humain \\
Les limites de l'imaginaire \\
Les limites de l'imagination \\
Les limites de l'interprétation \\
Les limites de l'obéissance \\
Les limites du corps \\
Les limites du langage \\
Les limites du pouvoir \\
Les limites du pouvoir politique \\
Les limites du réel \\
Les limites du vivant \\
Les livres \\
Les lois \\
Les lois causales \\
Les lois de la guerre \\
Les lois de la nature \\
Les lois de la nature sont-elles contingentes ? \\
Les lois de la nature sont-elles de simples régularités ? \\
Les lois de la nature sont elles nécessaires ? \\
Les lois de la pensée \\
Les lois de l'art \\
Les lois de l'histoire \\
Les lois de l'hospitalité \\
Les lois du sang \\
Les lois et les armes \\
Les lois et les mœurs \\
Les lois nous rendent-elles meilleurs ? \\
Les lois scientifiques sont-elles des lois de la nature ? \\
Les lois sont-elles seulement utiles ? \\
Les machines \\
Les machines nous rendent-elles libres ? \\
Les machines pensent-elles ? \\
Les maladies de l'âme \\
Les maladies de l'esprit \\
Les marginaux \\
Les matériaux \\
Les mathématiques consistent-elles seulement en des opérations de l'esprit ? \\
Les mathématiques du mouvement \\
Les mathématiques et la pensée de l'infini \\
Les mathématiques et la quantité \\
Les mathématiques et l'expérience \\
Les mathématiques ont-elles affaire au réel ? \\
Les mathématiques ont-elles besoin d'un fondement ? \\
Les mathématiques parlent-elles du réel ? \\
Les mathématiques se réduisent-elles à une pensée cohérente ? \\
Les mathématiques sont-elles réductibles à la logique ? \\
Les mathématiques sont-elles un instrument ? \\
Les mathématiques sont-elles un jeu de l'esprit ? \\
Les mathématiques sont-elles un langage ? \\
Les mathématiques sont-elles utiles au philosophe ? \\
Les mécanismes cérébraux \\
Les méchants peuvent-ils être amis ? \\
Les modalités \\
Les modèles \\
Les mœurs \\
Les mœurs et la morale \\
Les mondes possibles \\
Les monstres \\
Les morts \\
Les mots disent-ils les choses ? \\
Les mots et la signification \\
Les mots et les choses \\
Les mots et les concepts \\
Les mots expriment-ils les choses ? \\
Les mots justes \\
Les mots nous éloignent-ils des choses ? \\
Les mots parviennent-ils à tout exprimer ? \\
Les mots sont-ils trompeurs ? \\
Les moyens de l'autorité \\
Les moyens et la fin \\
Les moyens et les fins \\
Les moyens et les fins en art \\
Les nombres gouvernent-ils le monde ? \\
Les noms \\
Les noms propres \\
Les noms propres ont-ils une signification ? \\
Les normes \\
Les normes du vivant \\
Les normes esthétiques \\
Les normes et les valeurs \\
Les nouvelles technologies transforment-elles l'idée de l'art ? \\
Les objets de la pensée \\
Les objets scientifiques \\
Les objets sont-ils colorés ? \\
Le social et le politique \\
Les œuvres d'art ont-elles besoin d'un commentaire ? \\
Les œuvres d'art sont-elles des choses ? \\
Les œuvres d'art sont-elles des réalités comme les autres ? \\
Les œuvres d'art sont-elles éternelles ? \\
Le soi et le je \\
Le soin \\
Le soldat \\
Le soleil se lèvera-t-il demain ? \\
Le solipsisme \\
Le sommeil \\
Le sommeil de la raison \\
Le sommeil et la veille \\
Les opérations de la pensée \\
Les opérations de l'esprit \\
Le sophiste et le philosophe \\
Les opinions politiques \\
L'ésotérisme \\
Le souci \\
Le souci d'autrui résume-t-il la morale ? \\
Le souci de l'avenir \\
Le souci de soi \\
Le souci de soi est-il une attitude morale ? \\
Le souci du bien-être est-il politique ? \\
Le soupçon \\
Les outils \\
Le souvenir \\
Le souverain bien \\
L'espace \\
L'espace de la perception \\
L'espace et le lieu \\
L'espace et le territoire \\
L'espace nous sépare-t-il ? \\
L'espace public \\
Les paroles et les actes \\
Les parties de l'âme \\
Les passions ont-elles une place en politique ? \\
Les passions peuvent-elles être raisonnables ? \\
Les passions politiques \\
Les passions sont-elles toujours mauvaises ? \\
Les passions sont-elles toutes bonnes ? \\
Les passions sont-elles un obstacle à la vie sociale ? \\
Les passions s'opposent-elles à la raison ? \\
Les pauvres \\
L'espèce et l'individu \\
L'espèce humaine \\
Le spectacle \\
Le spectacle de la nature \\
Le spectacle de la pensée \\
Le spectacle du monde \\
L'espérance \\
L'espérance est-elle une vertu ? \\
Les personnages de fiction peuvent-ils avoir une réalité ? \\
Les peuples font-ils l'histoire ? \\
Les peuples ont-ils les gouvernements qu'ils méritent ? \\
Les phénomènes \\
Les phénomènes inconscients sont-ils réductibles à une mécanique cérébrale ? \\
Les philosophes doivent-ils être rois ? \\
Les philosophies se classent-elles ? \\
Le spirituel et le temporel \\
Les plaisirs \\
Les plaisirs de l'amitié \\
Les poètes et la cité \\
L'espoir \\
L'espoir et la crainte \\
L'espoir peut-il être raisonnable ? \\
Le sport \\
Le sport : s'accomplir ou se dépasser ? \\
Les possibles \\
Les pouvoirs de la religion \\
Les préjugés moraux \\
Les prêtres \\
Les preuves de la liberté \\
Les preuves de l'existence de Dieu \\
Les principes \\
Les principes de la démonstration \\
Les principes de la morale dépendent-ils de la culture ? \\
Les principes d'une science sont-ils des conventions ? \\
Les principes et les éléments \\
Les principes moraux \\
Les principes sont-ils indémontrables ?Qu'est-ce qu'être ensemble ? \\
L'esprit \\
L'esprit critique \\
L'esprit de corps \\
L'esprit de finesse \\
L'esprit dépend-il du corps ? \\
L'esprit des lois \\
L'esprit de système \\
L'esprit d'invention \\
L'esprit domine-t-il la matière ? \\
L'esprit du christianisme \\
L'esprit est-il matériel ? \\
L'esprit est-il mieux connu que le corps ? \\
L'esprit est-il objet de science ? \\
L'esprit est-il plus aisé à connaître que le corps ? \\
L'esprit est-il plus difficile à connaître que la matière ? \\
L'esprit est-il une chose ? \\
L'esprit est-il une machine ? \\
L'esprit est-il un ensemble de facultés ? \\
L'esprit est-il une partie du corps ? \\
L'esprit et la lettre \\
L'esprit et la machine \\
L'esprit et le cerveau \\
L'esprit humain progresse-t-il ? \\
L'esprit n'a-t-il jamais affaire qu'à lui-même ? \\
L'esprit peut-il être divisé ? \\
L'esprit peut-il être malade ? \\
L'esprit peut-il être mesuré ? \\
L'esprit peut-il être objet de science ? \\
L'esprit scientifique \\
L'esprit s'explique-t-il par une activité cérébrale ? \\
L'esprit tranquille \\
Les problèmes politiques peuvent-ils se ramener à des problèmes techniques ? \\
Les problèmes politiques sont-ils des problèmes techniques ? \\
Les progrès de la technique sont-ils nécessairement des progrès de la raison ? \\
Les progrès techniques constituent-ils des progrès de la civilisation ? \\
Les propositions métaphysiques sont-elles des illusions ? \\
Les proverbes \\
Les proverbes enseignent-ils quelque chose ? \\
Les proverbes nous instruisent-ils moralement ? \\
Les qualités esthétiques \\
Les qualités sensibles sont-elles dans les choses ou dans l'esprit ? \\
Les questions métaphysiques ont-elles un sens ? \\
L'esquisse \\
Les raisons de croire \\
Les raisons d'espérer \\
Les raisons de vivre \\
Les raisons du choix \\
Les rapports entre les hommes sont-ils des rapports de force ? \\
Les règles de l'art \\
Les règles du jeu \\
Les règles d'un bon gouvernement \\
Les règles sociales \\
Les relations \\
Les religions peuvent-elles être objets de science ? \\
Les religions sont-elles des illusions ? \\
Les représentants du peuple \\
Les reproductions \\
Les ressources humaines \\
Les ressources naturelles \\
Les révolutions scientifiques \\
Les révolutions techniques suscitent-elles des révolutions dans l'art ? \\
Les riches et les pauvres \\
Les rituels \\
Les robots \\
Les rôles sociaux \\
Les ruines \\
Les sacrifices \\
Les sauvages \\
Les scélérats peuvent-ils être heureux ? \\
Les sciences appliquées \\
Les sciences décrivent-elles le réel ? \\
Les sciences de la vie et de la Terre \\
Les sciences de la vie visent-elles un objet irréductible à la matière ? \\
Les sciences de l'éducation \\
Les sciences de l'esprit \\
Les sciences de l'homme et l'évolution \\
Les sciences de l'homme ont-elles inventé leur objet ? \\
Les sciences de l'homme permettent-elles d'affiner la notion de responsabilité ? \\
Les sciences de l'homme peuvent-elles expliquer l'impuissance de la liberté ? \\
Les sciences de l'homme rendent-elles l'homme prévisible ? \\
Les sciences doivent-elle prétendre à l'unification ? \\
Les sciences du comportement \\
Les sciences et le vivant \\
Les sciences exactes \\
Les sciences forment-elle un système ? \\
Les sciences historiques \\
Les sciences humaines doivent-elles être transdisciplinaires ? \\
Les sciences humaines éliminent-elles la contingence du futur ? \\
Les sciences humaines et le droit \\
Les sciences humaines nous protègent-elles de l'essentialisme ? \\
Les sciences humaines ont-elles un objet commun ? \\
Les sciences humaines permettent-elles de comprendre la vie d'un homme ? \\
Les sciences humaines peuvent-elles adopter les méthodes des sciences de la nature ? \\
Les sciences humaines peuvent-elles se passer de la notion d'inconscient ? \\
Les sciences humaines présupposent-elles une définition de l'homme ? \\
Les sciences humaines sont-elles des sciences de la nature humaine ? \\
Les sciences humaines sont-elles des sciences de la vie humaine ? \\
Les sciences humaines sont-elles des sciences d'interprétation ? \\
Les sciences humaines sont-elles des sciences ? \\
Les sciences humaines sont-elles explicatives ou compréhensives ? \\
Les sciences humaines sont-elles normatives ? \\
Les sciences humaines sont-elles relativistes ? \\
Les sciences humaines sont-elles subversives ? \\
Les sciences humaines traitent-elles de l'homme ? \\
Les sciences humaines traitent-elles de l'individu ? \\
Les sciences humaines transforment-elles la notion de causalité ? \\
Les sciences naturelles \\
Les sciences ne sont-elles qu'une description du monde ? \\
Les sciences nous donnent-elles des normes ? \\
Les sciences ont-elles besoin de principes fondamentaux ? \\
Les sciences ont-elles besoin d'une fondation métaphysique ? \\
Les sciences permettent-elles de connaître la réalité-même ? \\
Les sciences peuvent-elles exclure toute notion de finalité ? \\
Les sciences peuvent-elles penser l'individu ? \\
Les sciences sociales \\
Les sciences sociales ont-elles un objet ? \\
Les sciences sociales peuvent-elles être expérimentales ? \\
Les sciences sociales sont-elles nécessairement inexactes ? \\
Les sciences sont-elles une description du monde ? \\
L'essence \\
L'essence de la technique \\
L'essence et l'existence \\
Les sens jugent-ils ? \\
Les sens nous trompent-ils ? \\
Les sens peuvent-ils nous tromper ? \\
Les sens sont-ils source d'illusion ? \\
Les sens sont-ils trompeurs ? \\
L'essentiel \\
Les sentiments \\
Les sentiments ont-ils une histoire ? \\
Les sentiments peuvent-ils s'apprendre ? \\
Les services publics \\
Les signes de l'intelligence \\
Les sociétés évoluent-elles ? \\
Les sociétés ont-elles un inconscient ? \\
Les sociétés sont-elles hiérarchisables ? \\
Les sociétés sont-elles imprévisibles ? \\
Les structures expliquent-elles tout ? \\
Les systèmes \\
Le statut de l'axiome \\
Le statut des hypothèses dans la démarche scientifique \\
Les techniques artistiques \\
Les techniques du corps \\
Les témoignages et la preuve \\
Les théories scientifiques décrivent-elles la réalité ? \\
Les théories scientifiques sont-elles vraies ? \\
L'esthète \\
L'esthétique \\
L'esthétique est-elle une métaphysique de l'art ? \\
L'esthétisme \\
L'estime de soi \\
Les traditions \\
Le style \\
Le style et le beau \\
Le sublime \\
Le substitut \\
Le succès \\
Le suffrage universel \\
Le sujet \\
Le sujet de droit \\
Le sujet de l'action \\
Le sujet de la pensée \\
Le sujet de l'histoire \\
Le sujet et l'individu \\
Le sujet et l'objet \\
Le sujet moral \\
Le sujet n'est-il qu'une fiction ? \\
Le sujet peut-il s'aliéner par un libre choix ? \\
Les universaux \\
Le surnaturel \\
Les usages de l'art \\
Les valeurs de la République \\
Les valeurs morales ont-elles leur origine dans la raison ? \\
Les valeurs universelles \\
Les vérités empiriques \\
Les vérités éternelles \\
Les vérités scientifiques sont-elles relatives ? \\
Les vérités sont-elles intemporelles ? \\
Les vertus \\
Les vertus cardinales \\
Les vertus de l'amour \\
Les vertus du commerce \\
Les vertus ne sont-elles que des vices déguisés ? \\
Les vertus politiques \\
Les vices privés peuvent-ils faire le bien public ? \\
Les visages du mal \\
Les vivants \\
Les vivants et les morts \\
Les vivants peuvent-ils se passer des morts ? \\
Le syllogisme \\
Le symbole \\
Le symbolisme \\
Le symbolisme mathématique \\
Le système \\
Le système des arts \\
Le système des beaux-arts \\
Le système des besoins \\
Le tableau \\
Le tableau ? \\
Le tacite \\
Le tact \\
Le talent \\
Le talent et le génie \\
Le tas et le tout \\
L'État \\
L'État a-t-il des intérêts propres ? \\
L'État a-t-il le droit de contrôler notre habillement ? \\
L'État a-t-il pour but de maintenir l'ordre ? \\
L'État a-t-il pour finalité de maintenir l'ordre ? \\
L'État a-t-il tous les droits ? \\
L'État contribue-t-il à pacifier les relations entre les hommes ? \\
L'état de droit \\
L'État de droit \\
L'état de guerre \\
L'état de la nature \\
L'état de nature \\
L'état d'exception \\
L'État doit-il disparaître ? \\
L'État doit-il éduquer le citoyen ? \\
L'État doit-il éduquer le peuple ? \\
L'État doit-il éduquer les citoyens ? \\
L'État doit-il être fort ? \\
L'État doit-il être le plus fort ? \\
L'État doit-il être neutre ? \\
L'État doit-il être sans pitié ? \\
L'État doit-il faire le bonheur des citoyens ? \\
L'État doit-il nous rendre meilleurs ? \\
L'État doit-il préférer l'injustice au désordre ? \\
L'État doit-il reconnaître des limites à sa puissance ? \\
L'État doit-il se mêler de religion ? \\
L'État doit-il se préoccuper des arts ? \\
L'État doit-il se préoccuper du bonheur des citoyens ? \\
L'État doit-il se soucier de la morale ? \\
L'État doit-il veiller au bonheur des individus  ? \\
L'État est-il appelé à disparaître ? \\
L'État est-il au service de la société ? \\
L'État est-il fin ou moyen ? \\
L'État est-il le garant de la propriété privée ? \\
L'État est-il l'ennemi de la liberté ? \\
L'État est-il l'ennemi de l'individu ? \\
L'État est-il libérateur ? \\
L'État est-il toujours juste ? \\
L'État est-il un arbitre ? \\
L'État est-il un mal nécessaire ? \\
L'État est-il un moindre mal ? \\
L'État est-il un tiers impartial ? \\
L'État est-il un « monstre froid » ? \\
L'État et la culture \\
L'État et la guerre \\
L'État et la justice \\
L'État et la nation \\
L'État et la Nation \\
L'État et la protection \\
L'État et la société \\
L'État et le droit \\
L'État et le marché \\
L'État et le peuple \\
L'État et les communautés \\
L'État et les Églises \\
L'État et l'individu \\
L'État libéral \\
L'État mondial \\
L'État n'est-il qu'un instrument de domination ? \\
L'État nous rend-il meilleurs ? \\
L'État peut-il créer la liberté ? \\
L'État peut-il être impartial ? \\
L'État peut-il être indifférent à la religion ? \\
L'État peut-il être libéral ? \\
L'État peut-il poursuivre une autre fin que sa propre puissance ? \\
L'État peut-il renoncer à la violence ? \\
L'État providence \\
L'État-providence \\
L'État universel \\
Le technicien n'est-il qu'un exécutant ? \\
Le témoignage \\
Le témoignage des sens \\
Le témoin \\
Le temps \\
Le temps de la liberté \\
Le temps de la réflexion \\
Le temps de la science \\
Le temps de l'histoire \\
Le temps détruit-il tout ? \\
Le temps du bonheur \\
Le temps du désir \\
Le temps est-il destructeur ? \\
Le temps est-il en nous ou hors de nous ? \\
Le temps est-il essentiellement destructeur ? \\
Le temps est-il la marque de notre impuissance ? \\
Le temps est-il notre allié ? \\
Le temps est-il notre ennemi ? \\
Le temps est-il une contrainte ? \\
Le temps est-il une dimension de la nature ? \\
Le temps est-il une prison ? \\
Le temps est-il une réalité ? \\
Le temps et l'espace \\
Le temps existe-t-il ? \\
Le temps libre \\
Le temps ne fait-il que passer ? \\
Le temps n'est-il pour l'homme que ce qui le limite ? \\
Le temps nous appartient-il ? \\
Le temps nous est-il compté ? \\
Le temps passe-t-il ? \\
Le temps perdu \\
Le temps s'écoule-t-il ? \\
Le temps se laisse-t-il décrire logiquement ? \\
L'éternel présent \\
L'éternel retour \\
L'éternité \\
L'éternité n'est-elle qu'une illusion ? \\
Le terrain \\
Le territoire \\
Le terrorisme est-il un acte de guerre ? \\
Le théâtral \\
Le théâtre de l'histoire \\
Le théâtre du monde \\
Le théâtre et l'existence \\
L'éthique à l'épreuve du tragique \\
L'éthique des plaisirs \\
L'éthique est-elle affaire de choix ? \\
L'éthique suppose-t-elle la liberté ? \\
L'ethnocentrisme \\
Le tiers exclu \\
L'étonnement \\
Le totalitarisme \\
Le totémisme \\
Le toucher \\
Le tourment moral \\
Le tout est-il la somme de ses parties ? \\
Le tout et les parties \\
Le tragique \\
Le tragique et le comique \\
Le trait d'esprit \\
L'étranger \\
L'étrangeté \\
Le travail \\
Le travail artistique \\
Le travail a-t-il une valeur morale ? \\
Le travail de la raison \\
Le travail du négatif \\
Le travail est-il le propre de l'homme ? \\
Le travail est-il libérateur ? \\
Le travail est-il nécessaire au bonheur ? \\
Le travail est-il toujours une activité productrice ? \\
Le travail est-il un besoin ? \\
Le travail est-il une fin ? \\
Le travail est-il une marchandise ? \\
Le travail est-il une valeur morale ? \\
Le travail est-il une valeur ? \\
Le travail est-il un rapport naturel de l'homme à la nature ? \\
Le travail et la propriété \\
Le travail et la technique \\
Le travail et le labeur \\
Le travail et le temps \\
Le travail et l'œuvre \\
Le travail fait-il de l'homme un être moral ? \\
Le travail fonde-t-il la propriété ? \\
Le travaille libère-t-il ? \\
Le travail manuel \\
Le travail manuel est-il sans pensée ? \\
Le travail nous rend-il solidaires ? \\
Le travail rapproche-t-il les hommes ? \\
Le travail sur le terrain \\
Le travail sur soi \\
Le travail unit-il ou sépare-t-il les hommes ? \\
L'être de la conscience \\
L'être de la vérité \\
L'être de l'image \\
L'être du possible \\
L'être en tant qu'être \\
L'être en tant qu'être est-il connaissable ? \\
L'être et la relation \\
L'être et la volonté \\
L'être et le bien \\
L'être et le devoir-être \\
L'être et le néant \\
L'être et les êtres \\
L'être et l'essence \\
L'être et l'étant \\
L'être et le temps \\
L'être humain désire-t-il naturellement connaître ? \\
L'être humain est-il la mesure de toute chose ? \\
L'être humain est-il par nature un être religieux \\
L'être imaginaire et l'être de raison \\
L'être se confond-il avec l'être perçu ? \\
Le tribunal de l'histoire \\
Le troc \\
Le trompe-l'œil \\
L'étude \\
L'étude de l'histoire conduit-elle à désespérer l'homme ? \\
Le tyran \\
L'eugénisme \\
L'Europe \\
L'euthanasie \\
Le vainqueur a-t-il tous les droits ? \\
L'évaluation \\
L'évasion \\
Le vécu \\
Le vécu et la vérité \\
L'événement \\
L'événement et le fait divers \\
L'événement historique a-t-il un sens par lui-même ? \\
L'événement manque-t-il d'être ? \\
Le verbalisme \\
Le verbe \\
Le vertige \\
Le vertige de la liberté \\
Le vestige \\
Le vêtement \\
Le vice et la vertu \\
Le vide \\
Le vide et le plein \\
L'évidence \\
L'évidence a-t-elle une valeur absolue ? \\
L'évidence est-elle critère de vérité ? \\
L'évidence et la démonstration \\
L'évidence se passe-t-elle de démonstration ? \\
Le village global \\
Le virtuel \\
Le visage \\
Le visible et l'invisible \\
Le vivant \\
Le vivant a-t-il des droits ? \\
Le vivant comme problème pour la philosophie des sciences \\
Le vivant est-il entièrement connaissable ? \\
Le vivant est-il entièrement explicable ? \\
Le vivant est-il réductible au physico-chimique ? \\
Le vivant est-il un objet de science comme un autre ? \\
Le vivant et la machine \\
Le vivant et la mort \\
Le vivant et la sensibilité \\
Le vivant et la technique \\
Le vivant et le vécu \\
Le vivant et l'expérimentation \\
Le vivant et l'inerte \\
Le vivant n'est-il que matière ? \\
Le vivant n'est-il qu'une machine ingénieuse ? \\
Le vivant obéit-il à des lois ? \\
Le vivant obéit-il à une nécessité ? \\
Le volontaire et l'involontaire \\
Le volontarisme \\
L'évolution \\
L'évolution des langues \\
Le voyage \\
Le voyage dans le temps \\
Le vrai a-t-il une histoire ? \\
Le vrai doit-il être démontré ? \\
Le vrai est-il à lui-même sa propre marque ? \\
Le vrai et le bien \\
Le vrai et le bien sont-ils analogues ? \\
Le vrai et le faux \\
Le vrai et le vraisemblable \\
Le vrai peut-il rester invérifiable ? \\
Le vraisemblable \\
Le vraisemblable et le romanesque \\
Le vrai se perçoit-il ? \\
Le vrai se réduit-il à ce qui est vérifiable ? \\
Le vrai se réduit-il à l'utile ? \\
Le vulgaire \\
L'exactitude \\
L'excellence \\
L'excellence des sens \\
L'exception \\
L'excès \\
L'excès et le défaut \\
L'exclusion \\
L'excuse \\
L'exécution d'une œuvre d'art est-elle toujours une œuvre d'art ? \\
L'exemplaire \\
L'exemplarité \\
L'exemple \\
L'exemple en morale \\
L'exercice \\
L'exercice de la vertu \\
L'exercice de la volonté \\
L'exercice du pouvoir \\
L'exercice solitaire du pouvoir \\
L'exigence de vérité a-t-elle un sens moral ? \\
L'exigence morale \\
L'exil \\
L'existence \\
L'existence a-t-elle un sens ? \\
L'existence d'autrui \\
L'existence de Dieu \\
L'existence de l'État dépend-elle d'un contrat ? \\
L'existence des idées \\
L'existence du mal \\
L'existence du mal met-elle en échec la raison ? \\
L'existence du passé \\
L'existence est-elle pensable ? \\
L'existence est-elle une propriété ? \\
L'existence est-elle un jeu ? \\
L'existence est-elle vaine ? \\
L'existence et le temps \\
L'existence se démontre-t-elle ? \\
L'existence se laisse-t-elle penser ? \\
L'expérience \\
L'expérience artistique \\
L'expérience a-t-elle le même sens dans toutes les sciences ? \\
L'expérience cruciale \\
L'expérience d'autrui nous est-elle utile ? \\
L'expérience de la beauté \\
L'expérience de la liberté \\
L'expérience de la maladie \\
L'expérience de la vie \\
L'expérience de l'injustice \\
L'expérience démontre-t-elle quelque chose ? \\
L'expérience de pensée \\
L'expérience directe est-elle une connaissance ? \\
L'expérience du danger \\
L'expérience du désir \\
L'expérience du mal \\
L'expérience du temps \\
L'expérience en sciences humaines \\
L'expérience enseigne-elle quelque chose ? \\
L'expérience, est-ce l'observation ? \\
L'expérience et la sensation \\
L'expérience et l'expérimentation \\
L'expérience imaginaire \\
L'expérience instruit-elle ? \\
L'expérience métaphysique \\
L'expérience morale \\
L'expérience nous apprend-elle quelque chose ? \\
L'expérience peut-elle avoir raison des principes ? \\
L'expérience peut-elle contredire la théorie ? \\
L'expérience rend-elle raisonnable ? \\
L'expérience rend-elle responsable ? \\
L'expérience sensible est-elle la seule source légitime de connaissance ? \\
L'expérience suffit-elle pour établir une vérité ? \\
L'expérimentation \\
L'expérimentation en psychologie \\
L'expérimentation en sciences sociales \\
L'expérimentation sur l'être humain \\
L'expérimentation sur le vivant \\
L'expert et l'amateur \\
L'expertise \\
L'expertise politique \\
L'explication \\
L'explication scientifique \\
L'exploitation de l'homme par l'homme \\
L'exposition \\
L'exposition de l'œuvre d'art \\
L'expression \\
L'expression artistique \\
L'expression de l'inconscient \\
L'expression du désir \\
L'expressivité musicale \\
L'extériorité \\
L'extinction du désir \\
L'extraordinaire \\
L'extrémisme \\
Le « je ne sais quoi » \\
L'habileté \\
L'habileté et la prudence \\
L'habitation \\
L'habitude \\
L'habitude a-t-elle des vertus ? \\
L'habitude est-elle notre guide dans la vie ? \\
L'harmonie \\
L'harmonie du monde \\
L'hégémonie politique \\
L'hérédité \\
L'hérésie \\
L'héritage \\
L'héroïsme \\
L'hésitation \\
L'hétérogénéité sociale \\
L'hétéronomie \\
L'hétéronomie de l'art \\
L'histoire a-t-elle des lois ? \\
L'histoire a-t-elle un commencement et une fin ? \\
L'Histoire a-t-elle un commencement ? \\
L'histoire a-t-elle une fin ? \\
L'histoire a-t-elle un sens ? \\
L'histoire de l'art \\
L'histoire de l'art est-elle celle des styles ? \\
L'histoire de l'art est-elle finie ? \\
L'histoire de l'art peut-elle arriver à son terme ? \\
L'histoire des arts est-elle liée à l'histoire des techniques ? \\
L'histoire des civilisations \\
L'histoire des sciences \\
L'Histoire des sciences \\
L'histoire des sciences est-elle une histoire ? \\
L'histoire du droit est-elle celle du progrès de la justice ? \\
L'histoire est-elle avant tout mémoire ? \\
L'histoire est-elle déterministe ? \\
L'histoire est-elle écrite par les vainqueurs ? \\
L'histoire est-elle la connaissance du passé humain ? \\
L'histoire est-elle la mémoire de l'humanité ? \\
L'histoire est-elle la science de ce qui ne se répète jamais ? \\
L'histoire est-elle la science du passé ? \\
L'histoire est-elle le récit objectif des faits passés  ? \\
L'histoire est-elle le règne du hasard ? \\
L'histoire est-elle le théâtre des passions ? \\
L'histoire est-elle rationnelle ? \\
L'histoire est-elle tragique ? \\
L'histoire est-elle une explication ou une justification du passé ? \\
L'histoire est-elle une science comme les autres ? \\
L'histoire est-elle une science ? \\
L'histoire est-elle un genre littéraire ? \\
L'histoire est-elle un roman vrai ? \\
L'histoire est-elle utile à la politique ? \\
L'histoire est-elle utile ? \\
L'histoire et la géographie \\
L'histoire jugera \\
L'histoire jugera-t-elle ? \\
L'histoire n'a-t-elle pour objet que le passé ? \\
L'histoire naturelle \\
L'histoire n'est-elle que la connaissance du passé ? \\
L'histoire n'est-elle qu'un récit ? \\
L'histoire nous appartient-elle ? \\
L'histoire obéit-elle à des lois ? \\
L'histoire peut-elle être contemporaine ? \\
L'histoire peut-elle être universelle ? \\
L'histoire peut-elle se répéter ? \\
L'histoire se répète-t-elle ? \\
L'histoire universelle est-elle l'histoire des guerres ? \\
L'histoire : enquête ou science ? \\
L'histoire : science ou récit ? \\
L'historicité des sciences \\
L'historien \\
L'historien peut-il être impartial ? \\
L'historien peut-il se passer du concept de causalité ? \\
L'homme aime-t-il la justice pour elle-même ? \\
L'homme a-t-il besoin de l'art ? \\
L'homme a-t-il besoin de religion ? \\
L'homme a-t-il une nature ? \\
L'homme a-t-il une place dans la nature ? \\
L'homme de l'art \\
L'homme de la rue \\
L'homme des droits de l'homme \\
L'homme des droits de l'homme n'est-il qu'une fiction ? \\
L'homme des foules \\
L'homme des sciences de l'homme ? \\
L'homme des sciences humaines \\
L'homme d'État \\
L'homme est-il chez lui dans l'univers ? \\
L'homme est-il fait pour le travail ? \\
L'homme est-il la mesure de toute chose ? \\
L'homme est-il la mesure de toutes choses ? \\
L'homme est-il l'artisan de sa dignité ? \\
L'homme est-il le seul être à avoir une histoire ? \\
L'homme est-il le sujet de son histoire ? \\
L'homme est-il objet de science ? \\
L'homme est-il prisonnier du temps ? \\
L'homme est-il raisonnable par nature ? \\
L'homme est-il un animal comme les autres ? \\
L'homme est-il un animal comme un autre ? \\
L'homme est-il un animal dénaturé ? \\
L'homme est-il un animal politique ? \\
L'homme est-il un animal rationnel ? \\
L'homme est-il un animal social ? \\
L'homme est-il un animal ? \\
L'homme est-il un corps pensant ? \\
L'homme est-il un être de devoir ? \\
L'homme est-il un être social par nature ? \\
L'homme est-il un loup pour l'homme ? \\
L'homme et la bête \\
L'homme et la machine \\
L'homme et la nature sont-ils commensurables ? \\
L'homme et l'animal \\
L'homme et le citoyen \\
L'homme injuste peut-il être heureux ? \\
L'homme intérieur \\
L'homme, le citoyen, le soldat \\
L'homme libre est-il un homme seul ? \\
L'homme-machine \\
L'homme n'est-il qu'un animal comme les autres ? \\
L'homme pense-t-il toujours ? \\
L'homme peut-il changer ? \\
L'homme peut-il être libéré du besoin ? \\
L'homme peut-il se représenter un monde sans l'homme ? \\
L'homme se réalise-t-il dans le travail ? \\
L'homme se reconnaît-il mieux dans le travail ou dans le loisir ? \\
L'honnêteté \\
L'honneur \\
L'honneur ? \\
L'horizon \\
L'horreur \\
L'horrible \\
L'hospitalité \\
L'hospitalité a-t-elle un sens politique ? \\
L'hospitalité est-elle un devoir ? \\
L'humain \\
L'humanité \\
L'humanité est-elle aimable ? \\
L'humiliation \\
L'humilité \\
L'humour \\
L'humour et l'ironie \\
L'hybridation des arts \\
L'hypocrisie \\
L'hypothèse \\
L'hypothèse de l'inconscient \\
Libéral et libertaire \\
Libéralité et libéralisme \\
Liberté d'agir, liberté de penser \\
Liberté, égalité, fraternité \\
Liberté et courage \\
Liberté et démocratie \\
Liberté et déterminisme \\
Liberté et éducation \\
Liberté et égalité \\
Liberté et engagement \\
Liberté et existence \\
Liberté et habitude \\
Liberté et indépendance \\
Liberté et libération \\
Liberté et licence \\
Liberté et nécessité \\
Liberté et pouvoir \\
Liberté et responsabilité \\
Liberté et savoir \\
Liberté et sécurité \\
Liberté et société \\
Liberté et solitude \\
Liberté humaine et liberté divine \\
Liberté réelle, liberté formelle \\
Libertés publiques et culture politique \\
Libre arbitre et déterminisme sont-ils compatibles ? \\
Libre arbitre et liberté \\
Libre et heureux \\
L'idéal \\
L'idéal de l'art \\
L'idéal démonstratif \\
L'idéal de vérité \\
L'idéal et le réel \\
L'idéalisme \\
L'idéaliste \\
L'idéalité \\
L'idéal moral est-il vain ? \\
L'idéal systématique \\
L'idéal-type \\
L'idée d'anthropologie \\
L'idée de beaux arts \\
L'idée de bonheur \\
L'idée de bonheur collectif a-t-elle un sens ? \\
L'idée de civilisation \\
L'idée de communauté \\
L'idée de communisme \\
L'idée de connaissance approchée \\
L'idée de conscience collective \\
L'idée de continuité \\
L'idée de contrat social \\
L'idée de création \\
L'idée de crise \\
L'idée de destin a-t-elle un sens ? \\
L'idée de déterminisme \\
L'idée de Dieu \\
L'idée de domination \\
L'idée de forme sociale \\
L'idée de justice \\
L'idée de langue universelle \\
L'idée de logique \\
L'idée de logique transcendantale \\
L'idée de logique universelle \\
L'idée de loi de la nature \\
L'idée de loi logique \\
L'idée de loi naturelle \\
L'idée de mal nécessaire \\
L'idée de mathesis universalis \\
L'idée de métier \\
L'idée de modernité \\
L'idée de monde \\
L'idée de morale appliquée \\
L'idée de nation \\
L'idée d'encyclopédie \\
L'idée de norme \\
L'idée de paix \\
L'idée de perfection \\
L'idée de progrès \\
L'idée de république \\
L'idée de rétribution est-elle nécessaire à la morale ? \\
L'idée de révolution \\
L'idée de science \\
L'idée de science expérimentale \\
L'idée de substance \\
L'idée de système \\
L'idée d'éternité \\
L'idée d'Europe \\
L'idée d'exactitude \\
L'idée de « sciences exactes » \\
L'idée d'histoire universelle \\
L'idée d'ordre \\
L'idée d'organisme \\
L'idée d'origine \\
L'idée d'un commencement absolu \\
L'idée d'une langue universelle \\
L'idée d'une science bien faite \\
L'idée d'univers \\
L'idée d'université \\
L'idée esthétique \\
L'identification \\
L'identité \\
L'identité collective \\
L'identité et la différence \\
L'identité personnelle \\
L'identité personnelle est-elle donnée ou construite ? \\
L'identité relève-telle du champ politique ? \\
L'idéologie \\
L'idiot \\
L'idolâtrie \\
L'idole \\
L'ignoble \\
L'ignorance \\
L'ignorance est-elle préférable à l'erreur ? \\
L'ignorance nous excuse-t-elle ? \\
L'ignorance peut-elle être une excuse ? \\
L'illimité \\
L'illusion \\
L'illusion de la liberté \\
L'illustration \\
L'image \\
L'image et le réel \\
L'imaginaire \\
L'imaginaire et le réel \\
L'imagination \\
L'imagination a-t-elle des limites ? \\
L'imagination dans l'art \\
L'imagination dans les sciences \\
L'imagination enrichit-elle la connaissance ? \\
L'imagination est-elle libre ? \\
L'imagination est-elle maîtresse d'erreur et de fausseté ? \\
L'imagination esthétique \\
L'imagination et la raison \\
L'imagination nous éloigne-t-elle du réel ? \\
L'imagination politique \\
L'imitation \\
L'imitation a-t-elle une fonction morale ? \\
L'immanence \\
L'immatériel \\
L'immédiat \\
L'immensité \\
L'immortalité \\
L'immortalité de l'âme \\
L'immortalité des œuvres d'art \\
L'immuable \\
L'immutabilité \\
L'impardonnable \\
L'impartialité \\
L'impartialité des historiens \\
L'impartialité est-elle toujours désirable ? \\
L'impassibilité \\
L'impatience \\
L'impensable \\
L'impératif \\
L'impératif d'impartialité \\
L'impératif hypothétique \\
L'imperceptible \\
L'impersonnel \\
L'implicite \\
L'importance des détails \\
L'impossible \\
L'imposteur \\
L'imprescriptible \\
L'impression \\
L'imprévisible \\
L'improbable \\
L'improvisation \\
L'improvisation dans l'art \\
L'imprudence \\
L'impuissance \\
L'impuissance de la raison \\
L'impuissance de l'État \\
L'impunité \\
L'inaccessible \\
L'inachevé \\
L'inaction \\
L'inaliénable \\
L'inaperçu \\
L'inapparent \\
L'inattendu \\
L'incarnation \\
L'incertitude \\
L'incertitude du passé \\
L'incertitude est-elle dans les choses ou dans les idées ? \\
L'incertitude interdit-elle de raisonner ? \\
L'incommensurabilité \\
L'incommensurable \\
L'incompréhensible \\
L'inconcevable \\
L'inconnaissable \\
L'inconnu \\
L'inconscience \\
L'inconscient \\
L'inconscient a-t-il une histoire ? \\
L'inconscient collectif \\
L'inconscient de l'art \\
L'inconscient est-il dans l'âme ou dans le corps ? \\
L'inconscient est-il l'animal en nous ? \\
L'inconscient est-il un concept scientifique ? \\
L'inconscient est-il une dimension de la conscience ? \\
L'inconscient est-il une excuse ? \\
L'inconscient et l'involontaire \\
L'inconscient et l'oubli \\
L'inconscient n'est-il qu'un défaut de conscience ? \\
L'inconscient n'est-il qu'une hypothèse ? \\
L'inconscient peut-il se manifester ? \\
L'inconséquence \\
L'inconstance \\
L'incorporel \\
L'incrédulité \\
L'incroyable \\
L'inculture \\
L'indécence \\
L'indécidable \\
L'indécision \\
L'indéfini \\
L'indémontrable \\
L'indépassable \\
L'indépendance \\
L'indescriptible \\
L'indésirable \\
L'indétermination \\
L'indéterminé \\
L'indice \\
L'indice et la preuve \\
L'indicible \\
L'indicible et l'impensable \\
L'indicible et l'ineffable \\
L'indifférence \\
L'indifférence à la politique \\
L'indifférence peut-elle être une vertu ? \\
L'indignation \\
L'indignité \\
L'indiscutable \\
L'individu \\
L'individualisme \\
L'individualisme a-t-il sa place en politique ? \\
L'individualisme est-il un égoïsme ? \\
L'individualisme méthodologique \\
L'individualité \\
L'individu a-t-il des droits ? \\
L'individuel et le collectif \\
L'individu et la multitude \\
L'individu et le groupe \\
L'individu et l'espèce \\
L'individu face à L'État \\
L'indivisible \\
L'induction \\
L'induction et la déduction \\
L'indulgence \\
L'industrie culturelle \\
L'industrie du beau \\
L'ineffable et l'innommable \\
L'inégalité des chances \\
L'inégalité entre les hommes \\
L'inégalité naturelle \\
L'inéluctable \\
L'inertie \\
L'inesthétique \\
L'inestimable \\
L'inexactitude et le savoir scientifique \\
L'inexistant \\
L'inexpérience \\
L'infâme \\
L'infamie \\
L'inférence \\
L'infidélité \\
L'infini \\
L'infini et l'indéfini \\
L'infinité de l'espace \\
L'infinité de l'univers a-t-elle de quoi nous effrayer ? \\
L'influence \\
L'information \\
L'informe \\
L'informe et le difforme \\
L'ingénieur \\
L'ingénuité \\
L'ingratitude \\
L'inhibition \\
L'inhumain \\
L'inhumanité \\
L'inimaginable \\
L'inimitié \\
L'inintelligible \\
L'initiation \\
L'injonction \\
L'injustice \\
L'injustice est-elle préférable au désordre ? \\
L'injustifiable \\
L'inné et l'acquis \\
L'innocence \\
L'innommable \\
L'innovation \\
L'inobservable \\
L'inquiétant \\
L'inquiétude \\
L'inquiétude peut-elle définir l'existence humaine ? \\
L'inquiétude peut-elle devenir l'existence humaine ? \\
L'insatisfaction \\
L'insensé \\
L'insensibilité \\
L'insignifiant \\
L'insociable sociabilité \\
L'insolite \\
L'insouciance \\
L'insoumission \\
L'insoutenable \\
L'inspiration \\
L'instant \\
L'instant de la décision est-il une folie ? \\
L'instant et la durée \\
L'instinct \\
L'institution \\
L'institutionnalisation des conduites \\
L'institution scientifique \\
L'institution scolaire \\
L'instruction \\
L'instruction est-elle facteur de moralité ? \\
L'instrument \\
L'instrument et la machine \\
L'instrument mathématique en sciences humaines \\
L'instrument scientifique \\
L'insulte \\
L'insurrection \\
L'insurrection est-elle un droit ? \\
L'intangible \\
L'intellect \\
L'intellectuel \\
L'intelligence \\
L'intelligence artificielle \\
L'intelligence de la main \\
L'intelligence de la matière \\
L'intelligence de la technique \\
L'intelligence des bêtes \\
L'intelligence des foules \\
L'intelligence du sensible \\
L'intelligence du vivant \\
L'intelligence politique \\
L'intelligible \\
L'intempérance \\
L'intemporel \\
L'intention \\
L'intention morale \\
L'intention morale suffit-elle à constituer la valeur morale de l'action ? \\
L'intentionnalité \\
L'interaction \\
L'interdisciplinarité \\
L'interdit \\
L'interdit est-il au fondement de la culture ? \\
L'intéressant \\
L'intérêt \\
L'intérêt bien compris \\
L'intérêt commun \\
L'intérêt constitue-t-il l'unique lien social ? \\
L'intérêt de la société l'emporte-t-il sur celui des individus ? \\
L'intérêt de l'État \\
L'intérêt est-il le principe de tout échange ? \\
L'intérêt général est-il la somme des intérêts particuliers ? \\
L'intérêt général est-il le bien commun ? \\
L'intérêt gouverne-t-il le monde ? \\
L'intérêt peut-il être une valeur morale ? \\
L'intérêt public est-il une illusion ? \\
L'intérieur et l'extérieur \\
L'intériorisation des normes \\
L'intériorité \\
L'intériorité de l'œuvre \\
L'intériorité est-elle un mythe ? \\
L'interprétation \\
L'interprétation de la loi \\
L'interprétation de la nature \\
L'interprétation des œuvres \\
L'interprétation est-elle sans fin ? \\
L'interprétation est-elle un art ? \\
L'interprétation est-elle une activité sans fin ? \\
L'interprétation est-elle une science ? \\
L'interprète et le créateur \\
L'interprète sait-il ce qu'il cherche ? \\
L'interrogation humaine \\
L'intersubjectivité \\
L'intime conviction \\
L'intimité \\
L'intolérable \\
L'intolérance \\
L'intraduisible \\
L'intransigeance \\
L'introspection \\
L'introspection est-elle une connaissance ? \\
L'intuition \\
L'intuition a-t-elle une place en logique ? \\
L'intuition en mathématiques \\
L'intuition intellectuelle \\
L'intuition morale \\
L'inutile \\
L'inutile a-t-il de la valeur ? \\
L'inutile est-il sans valeur ? \\
L'invention \\
L'invention de soi \\
L'invention et la découverte \\
L'invention technique \\
L'invérifiable \\
L'invisibilité \\
L'invisible \\
L'involontaire \\
Lire et écrire \\
L'ironie \\
L'irrationalité \\
L'irrationnel \\
L'irrationnel est-il pensable ? \\
L'irrationnel est-il toujours absurde ? \\
L'irrationnel et le politique \\
L'irrationnel existe-t-il ? \\
L'irréductible \\
L'irréel \\
L'irréfléchi \\
L'irréfutable \\
L'irrégularité \\
L'irréparable \\
L'irreprésentable \\
L'irrésolution \\
L'irresponsabilité \\
L'irréversibilité \\
L'irréversible \\
L'irrévocable \\
Littérature et réalité \\
L'ivresse \\
L'obéissance \\
L'obéissance à l'autorité \\
L'obéissance est-elle compatible avec la liberté ? \\
L'objection de conscience \\
L'objectivation \\
L'objectivité \\
L'objectivité de l'art \\
L'objectivité de l'historien \\
L'objectivité historique \\
L'objectivité historique est-elle synonyme de neutralité ? \\
L'objectivité scientifique \\
L'objet \\
L'objet d'amour \\
L'objet de culte \\
L'objet de la littérature \\
L'objet de l'amour \\
L'objet de la politique \\
L'objet de la psychologie \\
L'objet de la réflexion \\
L'objet de l'art \\
L'objet de l'intention \\
L'objet des mathématiques \\
L'objet du désir \\
L'objet du désir en est-il la cause ? \\
L'objet et la chose \\
L'obligation \\
L'obligation d'échanger \\
L'obligation morale \\
L'obligation morale peut-elle se réduire à une obligation sociale ? \\
L'obscène \\
L'obscénité \\
L'obscur \\
L'obscurantisme \\
L'obscurité \\
L'observation \\
L'observation participante \\
L'obsession \\
L'obstacle \\
L'obstacle épistémologique \\
L'occasion \\
L'œil et l'oreille \\
L'œuvre \\
L'œuvre anonyme \\
L'œuvre d'art a-t-elle un sens ? \\
L'œuvre d'art doit-elle être belle ? \\
L'œuvre d'art doit-elle nous émouvoir ? \\
L'œuvre d'art donne-t-elle à penser ? \\
L'œuvre d'art échappe-t-elle au temps ? \\
L'œuvre d'art échappe-t-elle nécessairement au temps ? \\
L'œuvre d'art est-elle l'expression d'une idée ? \\
L'œuvre d'art est-elle toujours destinée à un public ? \\
L'œuvre d'art est-elle une belle apparence ? \\
L'œuvre d'art est-elle une marchandise ? \\
L'œuvre d'art est-elle un objet d'échange ? \\
L'œuvre d'art est-elle un symbole ? \\
L'œuvre d'art et le plaisir \\
L'œuvre d'art et sa reproduction \\
L'œuvre d'art et son auteur \\
L'œuvre d'art instruit-elle ? \\
L'œuvre d'art nous apprend-elle quelque chose ? \\
L'œuvre d'art représente-t-elle quelque chose ? \\
L'œuvre d'art totale \\
L'œuvre de fiction \\
L'œuvre de l'historien \\
L'œuvre du temps \\
L'œuvre et le produit \\
L'œuvre inachevée \\
L'offense \\
Logique et dialectique \\
Logique et existence \\
Logique et grammaire \\
Logique et logiques \\
Logique et mathématique \\
Logique et mathématiques \\
Logique et métaphysique \\
Logique et méthode \\
Logique et ontologie \\
Logique et psychologie \\
Logique et réalité \\
Logique et vérité \\
Logique générale et logique transcendantale \\
Loi morale et loi politique \\
Loi naturelle et loi politique \\
Lois et coutumes \\
Lois et règles en logique \\
Loisir et oisiveté \\
L'oisiveté \\
Lois naturelles et lois civiles \\
L'oligarchie \\
L'ombre et la lumière \\
L'omniscience \\
L'opinion \\
L'opinion a-t-elle nécessairement tort ? \\
L'opinion droite \\
L'opinion du citoyen \\
L'opinion est-elle un savoir ? \\
L'opinion publique \\
L'opinion vraie \\
L'opportunisme \\
L'opportunité \\
L'opposant \\
L'opposition \\
L'optimisme \\
L'oral et l'écrit \\
L'ordinaire \\
L'ordinaire est-il ennuyeux ? \\
L'ordre \\
L'ordre des choses \\
L'ordre du monde \\
L'ordre du temps \\
L'ordre du vivant est-il façonné par le hasard ? \\
L'ordre est-il dans les choses ? \\
L'ordre établi \\
L'ordre et la mesure \\
L'ordre et le désordre \\
L'ordre international \\
L'ordre moral \\
L'ordre politique exclut-il la violence ? \\
L'ordre politique peut-il exclure la violence ? \\
L'ordre public \\
L'ordre social \\
L'ordre social peut-il être juste ? \\
L'organique \\
L'organique et le mécanique \\
L'organique et l'inorganique \\
L'organisation \\
L'organisation du travail \\
L'organisation du vivant \\
L'organisme \\
L'orgueil \\
L'orientation \\
L'Orient et l'Occident \\
L'original et la copie \\
L'originalité \\
L'originalité en art \\
L'origine \\
L'origine de la culpabilité \\
L'origine de la négation \\
L'origine de l'art \\
L'origine de la violence \\
L'origine des croyances \\
L'origine des idées \\
L'origine des langues \\
L'origine des langues est-elle un faux problème ? \\
L'origine des valeurs \\
L'origine des vertus \\
L'origine du droit \\
L'origine et le fondement \\
L'ornement \\
L'oubli \\
L'oubli des fautes \\
L'oubli est-il un échec de la mémoire ? \\
L'oubli et le pardon \\
L'outil \\
L'outil et la machine \\
L'ouverture d'esprit \\
L'un \\
L'unanimité est-elle un critère de légitimité ? \\
L'unanimité est-elle un critère de vérité ? \\
L'un est le multiple \\
L'un et le multiple \\
L'un et l'être \\
L'uniformité \\
L'unité \\
L'unité dans le beau \\
L'unité de l'art \\
L'unité de la science \\
L'unité de l'État \\
L'unité de l'œuvre d'art \\
L'unité des arts \\
L'unité des contraires \\
L'unité des langues \\
L'unité des sciences \\
L'unité des sciences humaines \\
L'unité des sciences humaines ? \\
L'unité du corps politique \\
L'unité du genre humain \\
L'unité du vivant \\
L'univers \\
L'universalisme \\
L'universel \\
L'universel et le particulier \\
L'universel et le singulier \\
L'univocité de l'étant \\
L'urbanité \\
L'urgence \\
L'usage \\
L'usage des fictions \\
L'usage des généalogies \\
L'usage des mots \\
L'usage des passions \\
L'usage du monde \\
L'usure \\
L'usure des mots \\
L'utile \\
L'utile et l'agréable \\
L'utile et le beau \\
L'utile et le bien \\
L'utile et l'honnête \\
L'utile et l'inutile \\
L'utilité \\
L'utilité de croire \\
L'utilité de la poésie \\
L'utilité de l'art \\
L'utilité des préjugés \\
L'utilité des sciences humaines \\
L'utilité est-elle étrangère à la morale ? \\
L'utilité est-elle une valeur morale ? \\
L'utilité peut-elle être le principe de la moralité ? \\
L'utilité peut-elle être un critère pour juger de la valeur de nos actions ? \\
L'utilité publique \\
L'utopie \\
L'utopie a-t-elle une signification politique ? \\
L'utopie en politique \\
L'utopie et l'idéologie \\
Machine et organisme \\
Machines et liberté \\
Machines et mémoire \\
Magie et religion \\
Maître et disciple \\
Maître et serviteur \\
Maîtrise et puissance \\
Maîtriser l'absence \\
Maîtriser la technique \\
Maîtriser le vivant \\
Majorité et minorité \\
Maladie et convalescence \\
Maladies du corps, maladies de l'âme \\
Malaise dans la civilisation \\
Mal et liberté \\
Malheur aux vaincus \\
Ma liberté s'arrête-t-elle où commence celle des autres ? \\
Manger \\
Manquer de jugement \\
Ma parole m'engage-t-elle ? \\
Masculin, féminin \\
Mathématiques et réalité \\
Mathématiques pures et mathématiques appliquées \\
Matière et corps \\
Matière et matériaux \\
Ma vraie nature \\
Mécanisme et finalité \\
Médecine et philosophie \\
Méditer \\
Mémoire et fiction \\
Mémoire et identité \\
Mémoire et responsabilité \\
Mémoire et souvenir \\
Ménager les apparences \\
Mensonge et politique \\
Mentir \\
Mesure et démesure \\
Mesurer \\
Métaphysique et histoire \\
Métaphysique et ontologie \\
Métaphysique et politique \\
Métaphysique et religion \\
Métaphysique spéciale, métaphysique générale \\
Métier et vocation \\
Mettre en ordre \\
Microscope et télescope \\
Mieux vaut subir que commettre l'injustice \\
Misère et pauvreté \\
Modèle et copie \\
Mœurs, coutumes, lois \\
Moi d'abord \\
Mon corps \\
Mon corps est-il ma propriété ? \\
Mon corps est-il naturel ? \\
Mon corps fait-il obstacle à ma liberté ? \\
Mon corps m'appartient-il ? \\
Monde et nature \\
Monologue et dialogue \\
Mon prochain est-il mon semblable ? \\
Mon semblable \\
Montrer, est-ce démontrer ? \\
Montrer et démontrer \\
Morale et calcul \\
Morale et convention \\
Morale et économie \\
Morale et éducation \\
Morale et histoire \\
Morale et identité \\
Morale et intérêt \\
Morale et liberté \\
Morale et politique sont-elles indépendantes ? \\
Morale et pratique \\
Morale et prudence \\
Morale et religion \\
Morale et sexualité \\
Morale et société \\
Morale et technique \\
Morale et violence \\
Moralité et connaissance \\
Moralité et utilité \\
Mourir \\
Mourir dans la dignité \\
Mourir pour des principes \\
Mourir pour la patrie \\
Murs et frontières \\
Musique et bruit \\
Mythe et connaissance \\
Mythe et histoire \\
Mythe et pensée \\
Mythe et philosophie \\
Mythe et symbole \\
Mythe et vérité \\
Mythes et idéologies \\
Naît-on sujet ou le devient-on ? \\
Naître \\
N'apprend-on que par l'expérience ? \\
Narration et identité \\
Nation et richesse \\
Nature et artifice \\
Nature et convention \\
Nature et culture \\
Nature et fonction du sacrifice \\
Nature et histoire \\
Nature et institution \\
Nature et institutions \\
Nature et loi \\
Nature et morale \\
Naturel et artificiel \\
Nature, monde, univers \\
Naviguer \\
N'avons-nous affaire qu'au réel ? \\
N'avons-nous de devoir qu'envers autrui ? \\
Nécessité et contingence \\
Nécessité et liberté \\
Nécessité fait loi \\
N'échange-t-on que ce qui a de la valeur ? \\
N'échange-t-on que des symboles ? \\
N'échange-t-on que par intérêt ? \\
Ne désire-t-on que ce dont on manque ? \\
Ne désirons-nous que ce qui est bon pour nous ? \\
Ne désirons-nous que les choses que nous estimons bonnes ? \\
Ne faire que son devoir \\
Négation et privation \\
Ne lèse personne \\
Ne pas raconter d'histoires \\
Ne pas rire, ne pas pleurer, mais comprendre \\
Ne pas savoir ce que l'on fait \\
Ne penser à rien \\
Ne penser qu'à soi \\
Ne prêche-t-on que les convertis ? \\
Ne rien devoir à personne \\
Ne sait-on rien que par expérience ? \\
Ne sommes-nous véritablement maîtres que de nos pensées ? \\
N'est-on juste que par crainte du châtiment ? \\
Ne veut-on que ce qui est désirable ? \\
Ne vit-on bien qu'avec ses amis ? \\
Névroses et psychoses \\
N'existe-t-il que des individus ? \\
N'existe-t-il que le présent ? \\
N'exprime t-on que ce dont on a conscience ? \\
Ni Dieu ni maître \\
Ni Dieu, ni maître \\
Nier la vérité \\
Nier le monde \\
Nier l'évidence \\
N'interprète-t-on que ce qui est équivoque ? \\
Ni regrets, ni remords \\
Nomade et sédentaire \\
Nommer \\
Nom propre et nom commun \\
Normes et valeurs \\
Normes morales et normes vitales \\
Nos convictions morales sont-elles le simple reflet de notre temps ? \\
Nos désirs nous appartiennent-ils ? \\
Nos désirs nous opposent-ils ? \\
Nos pensées dépendent-elles de nous ? \\
Nos pensées sont-elles entièrement en notre pouvoir ? \\
Nos sens nous trompent-ils ? \\
Notre besoin de fictions \\
Notre connaissance du réel se limite-t-elle au savoir scientifique ? \\
Notre corps pense-t-il ? \\
Notre existence a-t-elle un sens si l'histoire n'en a pas ? \\
Notre ignorance nous excuse-t-elle ? \\
Notre liberté de pensée a-t-elle des limites ? \\
Notre rapport au monde est-il essentiellement technique ? \\
Notre rapport au monde peut-il n'être que technique ? \\
Nous et les autres \\
Nouveauté et tradition \\
Nul n'est censé ignorer la loi \\
Nul n'est méchant volontairement \\
N'y a-t-il d'amitié qu'entre égaux ? \\
N'y a-t-il de beauté qu'artistique ? \\
N'y a t-il de bonheur que dans l'instant ? \\
N'y a-t-il de bonheur qu'éphémère ? \\
N'y a-t-il de certitude que mathématique ? \\
N'y a-t-il de connaissance que de l'universel ? \\
N'y a-t-il de démocratie que représentative ? \\
N'y a-t-il de devoirs qu'envers autrui ? \\
N'y a-t-il de droit qu'écrit ? \\
N'y a-t-il de droits que de l'homme ? \\
N'y a-t-il de foi que religieuse ? \\
N'y a-t-il de propriété que privée ? \\
N'y a-t-il de rationalité que scientifique ? \\
N'y a-t-il de réalité que de l'individuel ? \\
N'y a-t-il de savoir que livresque ? \\
N'y a-t-il de science qu'autant qu'il s'y trouve de mathématique ? \\
N'y a-t-il de science que de ce qui est mathématisable ? \\
N'y a-t-il de science que du général ? \\
N'y a-t-il de science que du mesurable ? \\
N'y a-t-il de science qu'exacte ? \\
N'y a-t-il de sens que par le langage ? \\
N'y a-t-il de vérité que scientifique ? \\
N'y a-t-il de vérité que vérifiable ? \\
N'y a-t-il de vérités que scientifiques ? \\
N'y a-t-il de vrai que le vérifiable ? \\
N'y a-t-il que des individus ? \\
N'y a-t-il qu'une substance ? \\
N'y a-t-il qu'un seul monde ? \\
Obéir \\
Obéir, est-ce se soumettre ? \\
Obéissance et liberté \\
Obéissance et servitude \\
Obéissance et soumission \\
Objectivé et subjectivité \\
Objectiver \\
Objet et œuvre \\
Observation et expérience \\
Observation et expérimentation \\
Observer \\
Observer et comprendre \\
Observer et expérimenter \\
Observer et interpréter \\
Œil pour œil, dent pour dent \\
Œuvre et événement \\
Opinion et ignorance \\
Optimisme et pessimisme \\
Ordre et désordre \\
Ordre et justice \\
Ordre et liberté \\
Organisme et milieu \\
Origine et commencement \\
Origine et fondement \\
Où commence la servitude ? \\
Où commence la violence ? \\
Où commence l'interprétation ? \\
Où commence ma liberté ? \\
Où est le passé ? \\
Où est le pouvoir ? \\
Où est l'esprit ? \\
Où est mon esprit ? \\
Où est-on quand on pense ? \\
Où s'arrête l'espace public ? \\
Où sont les relations ? \\
Où suis-je quand je pense ? \\
Où suis-je ? \\
Outil et machine \\
Outil et organe \\
Paraître \\
Par-delà beauté et laideur \\
Pardonner et oublier \\
Parfaire \\
Parier \\
Par le langage, peut-on agir sur la réalité ? \\
Parler, est-ce agir ? \\
Parler, est-ce communiquer ? \\
Parler, est-ce donner sa parole ? \\
Parler, est-ce ne pas agir ? \\
Parler et agir \\
Parler, n'est-ce que désigner ? \\
Parler pour ne rien dire \\
Parler pour quelqu'un \\
Parole et pouvoir \\
Paroles et actes \\
Par où commencer ? \\
Par quoi un individu se distingue-t-il d'un autre ? \\
Partager \\
Partager les richesses \\
Partager sa vie \\
Partager ses sentiments \\
Passer du fait au droit \\
Passer le temps \\
Passions et intérêts \\
Passions, intérêt, raison \\
Pâtir \\
Peindre \\
Peindre d'après nature \\
Peindre, est-ce nécessairement feindre ? \\
Peinture et histoire \\
Peinture et réalité \\
Pensée et réalité \\
Penser à rien \\
Penser est-ce calculer ? \\
Penser, est-ce calculer ? \\
Penser, est-ce désobéir ? \\
Penser, est-ce dire non ? \\
Penser, est-ce se parler à soi-même ? \\
Penser est-il assimilable à un travail ? \\
Penser et calculer \\
Penser et imaginer \\
Penser et parler \\
Penser et raisonner \\
Penser et savoir \\
Penser et sentir \\
Penser la matière \\
Penser la technique \\
Penser l'avenir \\
Penser le changement \\
Penser le réel \\
Penser le rien, est-ce ne rien penser ? \\
Penser les sociétés comme des organismes \\
Penser l'impossible \\
Penser par soi-même \\
Penser par soi-même, est-ce être l'auteur de ses pensées ? \\
Penser peut-il nous rendre heureux ? \\
Penser requiert-il un corps ? \\
Penser sans corps \\
Penser sans sujet \\
Pense-t-on jamais seul ? \\
Pensez-vous que vous avez une âme ? \\
Perception et aperception \\
Perception et connaissance \\
Perception et création artistique \\
Perception et imagination \\
Perception et jugement \\
Perception et mouvement \\
Perception et passivité \\
Perception et sensation \\
Perception et souvenir \\
Perception et vérité \\
Percevoir \\
Percevoir est-ce connaître ? \\
Percevoir, est-ce connaître ? \\
Percevoir, est-ce interpréter ? \\
Percevoir, est-ce juger ? \\
Percevoir, est-ce reconnaître ? \\
Percevoir, est-ce savoir ? \\
Percevoir, est-ce s'ouvrir au monde ? \\
Percevoir et concevoir \\
Percevoir et imaginer \\
Percevoir et juger \\
Percevoir et sentir \\
Percevoir s'apprend-il ? \\
Percevons-nous les choses telles qu'elles sont ? \\
Perçoit-on le réel tel qu'il est ? \\
Perçoit-on le réel ? \\
Perçoit-on les choses comme elles sont ? \\
Perdre la mémoire \\
Perdre la raison \\
Perdre le contrôle \\
Perdre ses habitudes \\
Perdre ses illusions \\
Perdre son âme \\
Perdre son temps \\
Permanence et identité \\
Persévérer dans son être \\
Personne et individu \\
Personne n'est innocent \\
Persuader \\
Persuader et convaincre \\
Peuple et culture \\
Peuple et masse \\
Peuple et multitude \\
Peuple et société \\
Peuples et masses \\
Peut-être se mettre à la place de l'autre ? \\
Peut-il être moral de tuer ? \\
Peut-il être préférable de ne pas savoir ? \\
Peut-il exister une action désintéressée ? \\
Peut-il y avoir conflit entre nos devoirs ? \\
Peut-il y avoir de bons tyrans ? \\
Peut-il y avoir de la politique sans conflit ? \\
Peut-il y avoir des échanges équitables ? \\
Peut-il y avoir des expériences métaphysiques ? \\
Peut-il y avoir des lois de l'histoire ? \\
Peut-il y avoir des lois injustes ? \\
Peut-il y avoir des modèles en morale ? \\
Peut-il y avoir des vérités partielles ? \\
Peut-il y avoir esprit sans corps ? \\
Peut-il y avoir savoir-faire sans savoir ? \\
Peut-il y avoir science sans intuition du vrai ? \\
Peut-il y avoir un art conceptuel ? \\
Peut-il y avoir un droit à désobéir ? \\
Peut-il y avoir un droit de la guerre ? \\
Peut-il y avoir une histoire universelle ? \\
Peut-il y avoir une philosophie applicable ? \\
Peut-il y avoir une philosophie appliquée ? \\
Peut-il y avoir une philosophie politique sans Dieu ? \\
Peut-il y avoir une science de l'éducation ? \\
Peut-il y avoir une science politique ? \\
Peut-il y avoir une société des nations ? \\
Peut-il y avoir une société sans État ? \\
Peut-il y avoir un État mondial ? \\
Peut-il y avoir une vérité en art ? \\
Peut-il y avoir une vérité en politique ? \\
Peut-il y avoir un intérêt collectif ? \\
Peut-il y avoir un langage universel ? \\
Peut-on admettre un droit à la révolte ? \\
Peut-on agir machinalement ? \\
Peut-on aimer ce qu'on ne connaît pas ? \\
Peut-on aimer l'autre tel qu'il est ? \\
Peut-on aimer la vie plus que tout ? \\
Peut-on aimer les animaux ? \\
Peut-on aimer l'humanité ? \\
Peut-on aimer sans perdre sa liberté ? \\
Peut-on aimer son prochain comme soi-même ? \\
Peut-on aimer son travail ? \\
Peut-on aimer une œuvre d'art sans la comprendre ? \\
Peut-on apprendre à être heureux ? \\
Peut-on apprendre à être juste ? \\
Peut-on apprendre à être libre ? \\
Peut-on apprendre à mourir ? \\
Peut-on apprendre à penser ? \\
Peut-on apprendre à vivre ? \\
Peut-on argumenter en morale ? \\
Peut-on assimiler le vivant à une machine ? \\
Peut-on atteindre une certitude ? \\
Peut-on attribuer à chacun son dû ? \\
Peut-on avoir conscience de soi sans avoir conscience d'autrui ? \\
Peut-on avoir de bonnes raisons de ne pas dire la vérité ? \\
Peut-on avoir le droit de se révolter ? \\
Peut-on avoir peur de soi-même ? \\
Peut-on avoir raison contre les faits ? \\
Peut-on avoir raison contre tous ? \\
Peut-on avoir raison contre tout le monde ? \\
Peut-on avoir raisons contre les faits ? \\
Peut-on avoir raison tout.e seul.e ? \\
Peut-on avoir raison tout seul ? \\
Peut-on cesser de croire ? \\
Peut-on cesser de désirer ? \\
Peut-on changer de culture ? \\
Peut-on changer de logique ? \\
Peut-on changer le cours de l'histoire ? \\
Peut-on changer le monde ? \\
Peut-on changer le passé ? \\
Peut-on changer ses désirs ? \\
Peut-on choisir le mal ? \\
Peut-on choisir sa vie ? \\
Peut-on choisir ses désirs ? \\
Peut-on classer les arts ? \\
Peut-on commander à la nature ? \\
Peut-on communiquer ses perceptions à autrui ? \\
Peut-on communiquer son expérience ? \\
Peut-on comparer deux philosophies ? \\
Peut-on comparer les cultures ? \\
Peut-on comparer l'organisme à une machine ? \\
Peut-on comprendre ce qui est illogique ? \\
Peut-on comprendre le présent ? \\
Peut-on comprendre un acte que l'on désapprouve ? \\
Peut-on concevoir une humanité sans art ? \\
Peut-on concevoir une morale sans sanction ni obligation ? \\
Peut-on concevoir une science qui ne soit pas démonstrative ? \\
Peut-on concevoir une science sans expérience ? \\
Peut-on concevoir une société juste sans que les hommes ne le soient ? \\
Peut-on concevoir une société qui n'aurait plus besoin du droit ? \\
Peut-on concevoir une société sans État ? \\
Peut-on concevoir un État mondial ? \\
Peut-on concilier bonheur et liberté ? \\
Peut-on conclure de l'être au devoir-être ? \\
Peut-on connaître autrui ? \\
Peut-on connaître les choses telles qu'elles sont ? \\
Peut-on connaître le singulier ? \\
Peut-on connaître l'esprit ? \\
Peut-on connaître le vivant sans le dénaturer ? \\
Peut-on connaître le vivant sans recourir à la notion de finalité ? \\
Peut-on connaître l'individuel ? \\
Peut-on connaître par intuition ? \\
Peut-on considérer l'art comme un langage ? \\
Peut-on contester les droits de l'homme ? \\
Peut-on contredire l'expérience ? \\
Peut-on convaincre quelqu'un de la beauté d'une œuvre d'art ? \\
Peut-on craindre la liberté ? \\
Peut-on créer un homme nouveau ? \\
Peut-on critiquer la démocratie ? \\
Peut-on critiquer la religion ? \\
Peut-on croire ce qu'on veut ? \\
Peut-on croire en rien ? \\
Peut-on croire sans être crédule ? \\
Peut-on croire sans savoir pourquoi ? \\
Peut-on décider de croire ? \\
Peut-on décider de tout ? \\
Peut-on décider d'être heureux ? \\
Peut-on définir la morale comme l'art d'être heureux ? \\
Peut-on définir la vérité ? \\
Peut-on définir la vie ? \\
Peut-on définir le bien ? \\
Peut-on définir le bonheur ? \\
Peut-on délimiter le réel ? \\
Peut-on délimiter l'humain ? \\
Peut-on démontrer qu'on ne rêve pas ? \\
Peut-on dépasser la subjectivité ? \\
Peut-on désirer ce qui est ? \\
Peut-on désirer ce qu'on ne veut pas ? \\
Peut-on désirer ce qu'on possède ? \\
Peut-on désirer l'absolu ? \\
Peut-on désirer l'impossible ? \\
Peut-on désirer sans souffrir ? \\
Peut-on désobéir à l'État ? \\
Peut-on désobéir aux lois ? \\
Peut-on désobéir par devoir ? \\
Peut-on dialoguer avec un ordinateur ? \\
Peut-on dire ce que l'on pense ? \\
Peut-on dire ce qui n'est pas ? \\
Peut-on dire de la connaissance scientifique qu'elle procède par approximation ? \\
Peut-on dire de l'art qu'il donne un monde en partage ? \\
Peut-on dire d'une image qu'elle parle ? \\
Peut-on dire d'une œuvre d'art qu'elle est ratée ? \\
Peut-on dire d'une théorie scientifique qu'elle n'est jamais plus vraie qu'une autre mais seulement plus commode ? \\
Peut-on dire d'un homme qu'il est supérieur à un autre homme ? \\
Peut-on dire la vérité ? \\
Peut-on dire le singulier ? \\
Peut-on dire que la science ne nous fait pas connaître les choses mais les rapports entre les choses ? \\
Peut-on dire que les hommes font l'histoire ? \\
Peut-on dire que les machines travaillent pour nous ? \\
Peut-on dire que les mots pensent pour nous ? \\
Peut-on dire que l'humanité progresse ? \\
Peut-on dire que rien n'échappe à la technique ? \\
Peut-on dire qu'est vrai ce qui correspond aux faits ? \\
Peut-on dire que toutes les croyances se valent ? \\
Peut-on dire que tout est relatif ? \\
Peut-on dire qu'une théorie physique en contredit une autre ? \\
Peut-on dire toute la vérité ? \\
Peut-on discuter des goûts et des couleurs ? \\
Peut-on disposer de son corps ? \\
Peut-on distinguer différents types de causes ? \\
Peut-on distinguer entre de bons et de mauvais désirs ? \\
Peut-on distinguer entre les bons et les mauvais désirs ? \\
Peut-on distinguer le réel de l'imaginaire ? \\
Peut-on distinguer les faits de leurs interprétations ? \\
Peut-on donner un sens à son existence ? \\
Peut-on douter de sa propre existence ? \\
Peut-on douter de soi ? \\
Peut-on douter de toute vérité ? \\
Peut-on douter de tout ? \\
Peut-on échanger des idées ? \\
Peut-on échapper à ses désirs ? \\
Peut-on échapper à son temps ? \\
Peut-on échapper au cours de l'histoire ? \\
Peut-on échapper au temps ? \\
Peut-on éclairer la liberté ? \\
Peut-on écrire comme on parle ? \\
Peut-on éduquer la conscience ? \\
Peut-on éduquer le goût ? \\
Peut-on éduquer quelqu'un à être libre ? \\
Peut-on en appeler à la conscience contre la loi ? \\
Peut-on en appeler à la conscience contre l'État ? \\
Peut-on en savoir trop ? \\
Peut-on entreprendre d'éliminer la métaphysique ? \\
Peut-on établir une hiérarchie des arts ? \\
Peut-on être amoral ? \\
Peut-on être apolitique ? \\
Peut-on être citoyen du monde ? \\
Peut-on être complètement athée ? \\
Peut-on être dans le présent ? \\
Peut-on être en avance sur son temps ? \\
Peut-on être en conflit avec soi-même ? \\
Peut-on être esclave de soi-même ? \\
Peut-on être heureux dans la solitude ? \\
Peut-on être heureux sans être sage ? \\
Peut-on être heureux sans philosophie ? \\
Peut-on être heureux sans s'en rendre compte ? \\
Peut-on être heureux tout seul ? \\
Peut-on être homme sans être citoyen ? \\
Peut-on être hors de soi ? \\
Peut-on être ignorant ? \\
Peut-on être impartial ? \\
Peut-on être indifférent à l'histoire ? \\
Peut-on être indifférent à son bonheur ? \\
Peut-on être injuste et heureux ? \\
Peut-on être insensible à l'art ? \\
Peut-on être juste dans une situation injuste ? \\
Peut-on être juste dans une société injuste ? \\
Peut-on être juste sans être impartial ? \\
Peut-on être maître de soi ? \\
Peut-on être méchant volontairement ? \\
Peut-on être obligé d'aimer ? \\
Peut-on être plus ou moins libre ? \\
Peut-on être sans opinion ? \\
Peut-on être sceptique de bonne foi ? \\
Peut-on être sceptique ? \\
Peut-on être seul avec soi-même ? \\
Peut-on être seul ? \\
Peut-on être soi-même en société ? \\
Peut-on être sûr d'avoir raison ? \\
Peut-on être sûr de bien agir ? \\
Peut-on être sûr de ne pas se tromper ? \\
Peut-on être trop sage ? \\
Peut-on être trop sensible ? \\
Peut-on étudier le passé de façon objective ? \\
Peut-on exercer son esprit ? \\
Peut-on expérimenter sur le vivant ? \\
Peut-on expliquer le mal ? \\
Peut-on expliquer le monde par la matière ? \\
Peut-on expliquer le vivant ? \\
Peut-on expliquer une œuvre d'art ? \\
Peut-on faire de la politique sans supposer les hommes méchants ? \\
Peut-on faire de l'art avec tout ? \\
Peut-on faire de l'esprit un objet de science ? \\
Peut-on faire du dialogue un modèle de relation morale ? \\
Peut-on faire la paix ? \\
Peut-on faire la philosophie de l'histoire ? \\
Peut-on faire le bien d'autrui malgré lui ? \\
Peut-on faire l'économie de la notion de forme ? \\
Peut-on faire le mal en vue du bien ? \\
Peut-on faire le mal innocemment ? \\
Peut-on faire l'expérience de la nécessité ? \\
Peut-on faire l'inventaire du monde ? \\
Peut-on faire table rase du passé ? \\
Peut-on fixer des limites à la science ? \\
Peut-on fonder la morale sur la pitié ? \\
Peut-on fonder la morale ? \\
Peut-on fonder le droit sur la morale ? \\
Peut-on fonder les droits de l'homme ? \\
Peut-on fonder les mathématiques ? \\
Peut-on fonder un droit de désobéir ? \\
Peut-on fonder une éthique sur la biologie ? \\
Peut-on fonder une morale sur la nature ? \\
Peut-on fonder une morale sur le plaisir ? \\
Peut-on forcer quelqu'un à être libre ? \\
Peut-on forcer un homme à être libre ? \\
Peut-on fuir hors du monde ? \\
Peut-on fuir la société ? \\
Peut-on gâcher son talent ? \\
Peut-on gouverner sans lois ? \\
Peut-on haïr la raison ? \\
Peut-on haïr la vie ? \\
Peut-on haïr les images ? \\
Peut-on hiérarchiser les arts ? \\
Peut-on hiérarchiser les œuvres d'art ? \\
Peut-on identifier le désir au besoin ? \\
Peut-on ignorer sa propre liberté ? \\
Peut-on ignorer volontairement la vérité ? \\
Peut-on imaginer l'avenir ? \\
Peut-on imposer la liberté ? \\
Peut-on innover en politique ? \\
Peut-on interpréter la nature ? \\
Peut-on inventer en morale ? \\
Peut-on jamais aimer son prochain ? \\
Peut-on juger des œuvres d'art sans recourir à l'idée de beauté ? \\
Peut-on justifier la discrimination ? \\
Peut-on justifier la guerre ? \\
Peut-on justifier la raison d'État ? \\
Peut-on justifier le mal ? \\
Peut-on justifier le mensonge ? \\
Peut-on justifier ses choix ? \\
Peut-on légitimer la violence ? \\
Peut-on limiter l'expression de la volonté du peuple ? \\
Peut-on lutter contre le destin ? \\
Peut-on lutter contre soi-même ? \\
Peut-on maîtriser la nature ? \\
Peut-on maîtriser la technique ? \\
Peut-on maîtriser le risque ? \\
Peut-on maîtriser le temps ? \\
Peut-on maîtriser l'évolution de la technique ? \\
Peut-on maîtriser l'inconscient ? \\
Peut-on maîtriser ses désirs ? \\
Peut-on manipuler les esprits ? \\
Peut-on manquer de culture ? \\
Peut-on manquer de volonté ? \\
Peut-on mentir par humanité ? \\
Peut-on mesurer les phénomènes sociaux ? \\
Peut-on mesurer le temps ? \\
Peut-on montrer en cachant ? \\
Peut-on moraliser la guerre ? \\
Peut-on ne croire en rien ? \\
Peut-on ne pas connaître son bonheur ? \\
Peut-on ne pas croire au progrès ? \\
Peut-on ne pas croire ? \\
Peut-on ne pas être de son temps ? \\
Peut-on ne pas être égoïste ? \\
Peut-on ne pas être matérialiste ? \\
Peut-on ne pas être soi-même ? \\
Peut-on ne pas interpréter ? \\
Peut-on ne pas savoir ce que l'on dit ? \\
Peut-on ne pas savoir ce que l'on fait ? \\
Peut-on ne pas savoir ce que l'on veut ? \\
Peut-on ne pas savoir ce qu'on veut ? \\
Peut-on ne pas vouloir être heureux ? \\
Peut-on ne penser à rien ? \\
Peut-on ne rien devoir à personne ? \\
Peut-on ne rien vouloir ? \\
Peut-on ne vivre qu'au présent ? \\
Peut-on nier la réalité ? \\
Peut-on nier le réel ? \\
Peut-on nier l'évidence ? \\
Peut-on nier l'existence de la matière ? \\
Peut-on objectiver le psychisme ? \\
Peut-on opposer justice et liberté ? \\
Peut-on opposer le loisir au travail ? \\
Peut-on opposer morale et technique ? \\
Peut-on opposer nature et culture ? \\
Peut-on ôter à l'homme sa liberté ? \\
Peut-on oublier de vivre ? \\
Peut-on parler d'art primitif ? \\
Peut-on parler de ce qui n'existe pas ? \\
Peut-on parler de corruption des mœurs ? \\
Peut-on parler de dialogue des cultures ? \\
Peut-on parler de droits des animaux ? \\
Peut-on parler de mondes imaginaires ? \\
Peut-on parler de nourriture spirituelle ? \\
Peut-on parler de problèmes techniques ? \\
Peut-on parler des miracles de la technique ? \\
Peut-on parler des œuvres d'art ? \\
Peut-on parler de travail intellectuel ? \\
Peut-on parler de vérités métaphysiques ? \\
Peut-on parler de vérité subjective ? \\
Peut-on parler de vertu politique ? \\
Peut-on parler de violence d'État ? \\
Peut-on parler de « travail intellectuel » ? \\
Peut-on parler d'un droit de la guerre ? \\
Peut-on parler d'un droit de résistance ? \\
Peut-on parler d'une expérience religieuse ? \\
Peut-on parler d'une morale collective ? \\
Peut-on parler d'une religion de l'humanité ? \\
Peut-on parler d'une santé de l'âme ? \\
Peut-on parler d'une science de l'art ? \\
Peut-on parler d'un progrès dans l'histoire ? \\
Peut-on parler d'un progrès de la liberté ? \\
Peut-on parler d'un règne de la technique ? \\
Peut-on parler d'un savoir poétique ? \\
Peut-on parler d'un travail intellectuel ? \\
Peut-on parler pour en rien dire ? \\
Peut-on parler pour ne rien dire ? \\
Peut-on penser ce qu'on ne peut dire ? \\
Peut-on penser contre l'expérience ? \\
Peut-on penser illogiquement ? \\
Peut-on penser la douleur ? \\
Peut-on penser la matière ? \\
Peut-on penser la mort ? \\
Peut-on penser la nouveauté ? \\
Peut-on penser l'art sans référence au beau ? \\
Peut-on penser la vie sans penser la mort ? \\
Peut-on penser la vie ? \\
Peut-on penser le changement ? \\
Peut-on penser le monde sans la technique ? \\
Peut-on penser le temps sans l'espace ? \\
Peut-on penser l'extériorité ? \\
Peut-on penser l'impossible ? \\
Peut-on penser l'infini ? \\
Peut-on penser l'irrationnel ? \\
Peut-on penser l'œuvre d'art sans référence à l'idée de beauté ? \\
Peut-on penser sans concepts ? \\
Peut-on penser sans concept ? \\
Peut-on penser sans images ? \\
Peut-on penser sans image ? \\
Peut-on penser sans les mots ? \\
Peut-on penser sans les signes ? \\
Peut-on penser sans méthode ? \\
Peut-on penser sans ordre ? \\
Peut-on penser sans préjugés ? \\
Peut-on penser sans préjugé ? \\
Peut-on penser sans règles ? \\
Peut-on penser sans savoir que l'on pense ? \\
Peut-on penser sans signes ? \\
Peut-on penser sans son corps ? \\
Peut-on penser un art sans œuvres ? \\
Peut-on penser un droit international ? \\
Peut-on penser une conscience sans objet ? \\
Peut-on penser une métaphysique sans Dieu ? \\
Peut-on penser une société sans État ? \\
Peut-on penser un État sans violence ? \\
Peut-on penser une volonté diabolique ? \\
Peut-on percevoir sans juger ? \\
Peut-on percevoir sans s'en apercevoir ? \\
Peut-on perdre la raison ? \\
Peut-on perdre sa dignité ? \\
Peut-on perdre sa liberté ? \\
Peut-on perdre son identité ? \\
Peut-on perdre son temps ? \\
Peut-on préconiser, dans les sciences humaines et sociales, l'imitation des sciences de la nature ? \\
Peut-on prédire les événements ? \\
Peut-on prédire l'histoire ? \\
Peut-on préférer le bonheur à la vérité ? \\
Peut-on préférer l'injustice au désordre ? \\
Peut-on préférer l'ordre à la justice ? \\
Peut-on prévoir l'avenir ? \\
Peut-on prévoir le futur ? \\
Peut-on promettre le bonheur ? \\
Peut-on protéger les libertés sans les réduire ? \\
Peut-on prouver l'existence de Dieu ? \\
Peut-on prouver l'existence de l'inconscient ? \\
Peut-on prouver l'existence du monde ? \\
Peut-on prouver l'existence ? \\
Peut-on prouver une existence ? \\
Peut-on raconter sa vie ? \\
Peut-on raisonner sans règles ? \\
Peut-on ralentir la course du temps ? \\
Peut-on recommencer sa vie ? \\
Peut-on reconnaître un sens à l'histoire sans lui assigner une fin ? \\
Peut-on réduire la pensée à une espèce de comportement ? \\
Peut-on réduire le raisonnement au calcul ? \\
Peut-on réduire l'esprit à la matière ? \\
Peut-on réduire une métaphysique à une conception du monde ? \\
Peut-on réduire un homme à la somme de ses actes ? \\
Peut-on refuser de voir la vérité ? \\
Peut-on refuser la loi ? \\
Peut-on refuser la violence ? \\
Peut-on refuser le vrai ? \\
Peut-on régner innocemment ? \\
Peut-on rendre raison de tout ? \\
Peut-on rendre raison du réel ? \\
Peut-on renoncer à comprendre ? \\
Peut-on renoncer à ses droits ? \\
Peut-on renoncer à soi ? \\
Peut-on renoncer au bonheur ? \\
Peut-on réparer le vivant ? \\
Peut-on répondre d'autrui ? \\
Peut-on représenter le peuple ? \\
Peut-on représenter l'espace ? \\
Peut-on reprocher à la morale d'être abstraite ? \\
Peut-on reprocher au langage d'être équivoque ? \\
Peut-on reprocher au langage d'être parfait ? \\
Peut-on résister au vrai ? \\
Peut-on rester dans le doute ? \\
Peut-on rester insensible à la beauté ? \\
Peut-on rester sceptique ? \\
Peut-on restreindre la logique à la pensée formelle ? \\
Peut-on réunir des arts différents dans une même œuvre ? \\
Peut-on revendiquer la paix comme un droit ? \\
Peut-on revenir sur ses erreurs ? \\
Peut-on rire de tout ? \\
Peut-on rompre avec la société ? \\
Peut-on rompre avec le passé ? \\
Peut-on s'abstenir de penser politiquement ? \\
Peut-on s'accorder sur des vérités morales ? \\
Peut-on s'affranchir des lois ? \\
Peut-on s'attendre à tout ? \\
Peut-on savoir ce qui est bien ? \\
Peut-on savoir quelque chose de l'avenir ? \\
Peut-on savoir sans croire ? \\
Peut-on se choisir un destin ? \\
Peut-on se connaître soi-même ? \\
Peut-on se désintéresser de la politique ? \\
Peut-on se désintéresser de son bonheur ? \\
Peut-on se duper soi-même ? \\
Peut-on se faire une idée de tout ? \\
Peut-on se fier à l'expérience vécue ? \\
Peut-on se fier à l'intuition ? \\
Peut-on se fier à son intuition ? \\
Peut-on se gouverner soi-même ? \\
Peut-on se méfier de soi-même ? \\
Peut-on se mentir à soi-même \\
Peut-on se mentir à soi-même ? \\
Peut-on se mettre à la place d'autrui ? \\
Peut-on se mettre à la place de l'autre ? \\
Peut-on s'en tenir au présent ? \\
Peut-on séparer l'homme et l'œuvre ? \\
Peut-on séparer politique et économie ? \\
Peut-on se passer de chef ? \\
Peut-on se passer de croire ? \\
Peut-on se passer de croyances ? \\
Peut-on se passer de croyance ? \\
Peut-on se passer de Dieu ? \\
Peut-on se passer de frontières ? \\
Peut-on se passer de la religion ? \\
Peut-on se passer de l'État ? \\
Peut-on se passer de méthode ? \\
Peut-on se passer de mythes ? \\
Peut-on se passer de principes ? \\
Peut-on se passer de religion ? \\
Peut-on se passer de représentants ? \\
Peut-on se passer de spiritualité ? \\
Peut-on se passer des relations ? \\
Peut-on se passer d'État ? \\
Peut-on se passer de techniques de raisonnement ? \\
Peut-on se passer de technique ? \\
Peut-on se passer de toute religion ? \\
Peut-on se passer d'idéal ? \\
Peut-on se passer d'un maître ? \\
Peut-on se prescrire une loi ? \\
Peut-on se promettre quelque chose à soi-même ? \\
Peut-on se punir soi-même ? \\
Peut-on se régler sur des exemples en politique ? \\
Peut-on se retirer du monde ? \\
Peut-on se tromper en se croyant heureux ? \\
Peut-on se vouloir parfait ? \\
Peut-on sortir de la subjectivité ? \\
Peut-on sortir de sa conscience ? \\
Peut-on souhaiter le gouvernement des meilleurs ? \\
Peut-on suivre une règle ? \\
Peut-on suspendre le temps ? \\
Peut-on suspendre son jugement ? \\
Peut-on sympathiser avec l'ennemi ? \\
Peut-on tirer des leçons de l'histoire ? \\
Peut-on toujours faire ce qu'on doit ? \\
Peut-on toujours savoir entièrement ce que l'on dit ? \\
Peut-on tout analyser ? \\
Peut-on tout attendre de l'État ? \\
Peut-on tout définir ? \\
Peut-on tout démontrer ? \\
Peut-on tout désirer ? \\
Peut-on tout dire ? \\
Peut-on tout donner ? \\
Peut-on tout échanger ? \\
Peut-on tout enseigner ? \\
Peut-on tout expliquer ? \\
Peut-on tout exprimer ? \\
Peut-on tout imaginer ? \\
Peut-on tout imiter ? \\
Peut-on tout interpréter ? \\
Peut-on tout mathématiser ? \\
Peut-on tout mesurer ? \\
Peut-on tout ordonner ? \\
Peut-on tout pardonner ? \\
Peut-on tout partager ? \\
Peut-on tout prévoir ? \\
Peut-on tout prouver ? \\
Peut-on tout soumettre à la discussion ? \\
Peut-on tout tolérer ? \\
Peut-on traiter autrui comme un moyen ? \\
Peut-on traiter un être vivant comme une machine ? \\
Peut-on transformer le réel ? \\
Peut-on transiger avec les principes ? \\
Peut-on trouver du plaisir à l'ennui ? \\
Peut-on vivre avec les autres ? \\
Peut-on vivre dans le doute ? \\
Peut-on vivre en marge de la société ? \\
Peut-on vivre en sceptique ? \\
Peut-on vivre hors du temps ? \\
Peut-on vivre pour la vérité ? \\
Peut-on vivre sans aimer ? \\
Peut-on vivre sans art ? \\
Peut-on vivre sans aucune certitude ? \\
Peut-on vivre sans croyances ? \\
Peut-on vivre sans désir ? \\
Peut-on vivre sans échange ? \\
Peut-on vivre sans illusions ? \\
Peut-on vivre sans l'art ? \\
Peut-on vivre sans le plaisir de vivre ? \\
Peut-on vivre sans lois ? \\
Peut-on vivre sans passion ? \\
Peut-on vivre sans peur ? \\
Peut-on vivre sans principes ? \\
Peut-on vivre sans réfléchir ? \\
Peut-on vivre sans ressentiment ? \\
Peut-on vivre sans rien espérer ? \\
Peut-on vivre sans sacré ? \\
Peut-on voir sans croire ? \\
Peut-on vouloir ce qu'on ne désire pas ? \\
Peut-on vouloir le bonheur d'autrui ? \\
Peut-on vouloir le mal pour le mal ? \\
Peut-on vouloir le mal ? \\
Peut-on vouloir l'impossible ? \\
Peut-on vouloir sans désirer ? \\
Philosopher, est-ce apprendre à vivre ? \\
Philosophe-t-on pour être heureux ? \\
Philosophie et mathématiques \\
Philosophie et métaphysique \\
Philosophie et poésie \\
Philosophie et système \\
Photographier le réel \\
Physique et mathématiques \\
Physique et métaphysique \\
Pitié et compassion \\
Pitié et cruauté \\
Pitié et mépris \\
Plaider \\
Plaisir et bonheur \\
Plaisir et douleur \\
Plaisirs, honneurs, richesses \\
Pluralisme et politique \\
Pluralité et unité \\
Plusieurs religions valent-elles mieux qu'une seule ? \\
Poésie et philosophie \\
Poésie et vérité \\
Poétique et prosaïque \\
Point de vue du créateur et point de vue du spectateur \\
Police et politique \\
Politique et coopération \\
Politique et esthétique \\
Politique et mémoire \\
Politique et parole \\
Politique et participation \\
Politique et passions \\
Politique et propagande \\
Politique et secret \\
Politique et technologie \\
Politique et territoire \\
Politique et trahison \\
Politique et unité \\
Politique et vérité \\
Possession et propriété \\
Pour connaître, suffit-il de démontrer ? \\
Pour être heureux, faut-il renoncer à la perfection ? \\
Pour être homme, faut-il être citoyen ? \\
Pour être libre, faut-il renoncer à être heureux ? \\
Pour être un bon observateur faut-il être un bon théoricien ? \\
Pour juger, faut-il seulement apprendre à raisonner ? \\
Pour qui se prend-on ? \\
Pourquoi aimons-nous la musique ? \\
Pourquoi aller contre son désir ? \\
Pourquoi a-t-on peur de la folie ? \\
Pourquoi avoir recours à la notion d'inconscient ? \\
Pourquoi châtier ? \\
Pourquoi chercher à connaître le passé ? \\
Pourquoi chercher à se distinguer ? \\
Pourquoi chercher la vérité ? \\
Pourquoi chercher un sens à l'histoire ? \\
Pourquoi cherche-t-on à connaître ? \\
Pourquoi commémorer ? \\
Pourquoi communiquer ? \\
Pourquoi conserver les œuvres d'art ? \\
Pourquoi construire des monuments ? \\
Pourquoi critiquer la raison ? \\
Pourquoi critiquer le conformisme ? \\
Pourquoi croyons-nous ? \\
Pourquoi défendre le faible ? \\
Pourquoi définir ? \\
Pourquoi délibérer ? \\
Pourquoi démontrer ce que l'on sait être vrai ? \\
Pourquoi démontrer ? \\
Pourquoi des artifices ? \\
Pourquoi des artistes ? \\
Pourquoi des cérémonies ? \\
Pourquoi des châtiments ? \\
Pourquoi des classifications ? \\
Pourquoi des conflits ? \\
Pourquoi des devoirs ? \\
Pourquoi des élections ? \\
Pourquoi des exemples ? \\
Pourquoi des fictions ? \\
Pourquoi des géométries ? \\
Pourquoi des guerres ? \\
Pourquoi des historiens ? \\
Pourquoi des hypothèses ? \\
Pourquoi des idoles ? \\
Pourquoi des institutions ? \\
Pourquoi des interdits ? \\
Pourquoi désirer la sagesse ? \\
Pourquoi désirer l'immortalité ? \\
Pourquoi désire-t-on ce dont on n'a nul besoin ? \\
Pourquoi désirons-nous ? \\
Pourquoi des logiciens ? \\
Pourquoi des lois ? \\
Pourquoi des maîtres ? \\
Pourquoi des métaphores ? \\
Pourquoi des modèles ? \\
Pourquoi des musées ? \\
Pourquoi des œuvres d'art ? \\
Pourquoi des philosophes ? \\
Pourquoi des poètes ? \\
Pourquoi des psychologues ? \\
Pourquoi des religions ? \\
Pourquoi des rites ? \\
Pourquoi des sociologues ? \\
Pourquoi des traditions ? \\
Pourquoi des utopies ? \\
Pourquoi dialogue-t-on ? \\
Pourquoi Dieu se soucierait-il des affaires humaines ? \\
Pourquoi dire la vérité ? \\
Pourquoi domestiquer ? \\
Pourquoi donner des leçons de morale ? \\
Pourquoi donner ? \\
Pourquoi échanger des idées ? \\
Pourquoi écrire ? \\
Pourquoi écrit-on des lois ? \\
Pourquoi écrit-on les lois ? \\
Pourquoi écrit-on l'Histoire ? \\
Pourquoi écrit-on ? \\
Pourquoi est-il difficile de rectifier une erreur ? \\
Pourquoi être exigeant ? \\
Pourquoi être moral ? \\
Pourquoi être raisonnable ? \\
Pourquoi étudier le vivant ? \\
Pourquoi étudier l'Histoire ? \\
Pourquoi exiger la cohérence \\
Pourquoi exposer les œuvres d'art ? \\
Pourquoi faire confiance ? \\
Pourquoi faire de la politique ? \\
Pourquoi faire de l'histoire ? \\
Pourquoi faire la guerre ? \\
Pourquoi faire son devoir ? \\
Pourquoi fait-on le mal ? \\
Pourquoi faudrait-il être cohérent ? \\
Pourquoi faut-il diviser le travail ? \\
Pourquoi faut-il être cohérent ? \\
Pourquoi faut-il être juste ? \\
Pourquoi faut-il être poli ? \\
Pourquoi faut-il travailler ? \\
Pourquoi formaliser des arguments ? \\
Pourquoi imiter ? \\
Pourquoi interprète-t-on ? \\
Pourquoi joue-t-on ? \\
Pourquoi la critique ? \\
Pourquoi la curiosité est-elle un vilain défaut ? \\
Pourquoi la guerre ? \\
Pourquoi la justice a-t-elle besoin d'institutions ? \\
Pourquoi la musique intéresse-t-elle le philosophe ? \\
Pourquoi la prison ? \\
Pourquoi la prohibition de l'inceste ? \\
Pourquoi la raison recourt-elle à l'hypothèse ? \\
Pourquoi la réalité peut-elle dépasser la fiction ? \\
Pourquoi l'art intéresse-t-il les philosophes ? \\
Pourquoi l'économie est-elle politique ? \\
Pourquoi le droit international est-il imparfait ? \\
Pourquoi les droits de l'homme sont-ils universels ? \\
Pourquoi les États se font-ils la guerre ? \\
Pourquoi les hommes mentent-ils ? \\
Pourquoi les mathématiques s'appliquent-elles à la réalité ? \\
Pourquoi les œuvres d'art résistent-elles au temps ? \\
Pourquoi le sport ? \\
Pourquoi les sciences ont-elles une histoire ? \\
Pourquoi les sociétés ont-elles besoin de lois ? \\
Pourquoi le théâtre ? \\
Pourquoi l'ethnologue s'intéresse-t-il à la vie urbaine ? \\
Pourquoi l'homme a-t-il des droits ? \\
Pourquoi l'homme est-il l'objet de plusieurs sciences ? \\
Pourquoi l'homme travaille-t-il ? \\
Pourquoi lire des romans ? \\
Pourquoi lire les poètes ? \\
Pourquoi lit-on des romans ? \\
Pourquoi mentir ? \\
Pourquoi ne s'entend-on pas sur la nature de ce qui est réel ? \\
Pourquoi nous soucier du sort des générations futures ? \\
Pourquoi nous souvenons-nous ? \\
Pourquoi nous trompons-nous ? \\
Pourquoi nous-trompons nous ? \\
Pourquoi obéir aux lois ? \\
Pourquoi obéir ? \\
Pourquoi obéit-on aux lois ? \\
Pourquoi obéit-on ? \\
Pourquoi parler de fautes de goût ? \\
Pourquoi parler du travail comme d'un droit ? \\
Pourquoi parle-t-on d'économie politique ? \\
Pourquoi parle-t-on d'une « société civile » ? \\
Pourquoi parle-t-on ? \\
Pourquoi parlons-nous ? \\
Pourquoi pas plusieurs dieux ? \\
Pourquoi pas ? \\
Pourquoi penser à la mort ? \\
Pourquoi pensons-nous ? \\
Pourquoi philosopher ? \\
Pourquoi pleure-t-on au cinéma ? \\
Pourquoi pleure-t-on ? \\
Pourquoi plusieurs sciences ? \\
Pourquoi préférer l'original à la reproduction ? \\
Pourquoi préférer l'original à sa reproduction ? \\
Pourquoi préférer l'original ? \\
Pourquoi préserver l'environnement ? \\
Pourquoi prier ? \\
Pourquoi prouver l'existence de Dieu ? \\
Pourquoi punir ? \\
Pourquoi punit-on ? \\
Pourquoi raconter des histoires ? \\
Pourquoi rechercher la vérité ? \\
Pourquoi rechercher le bonheur ? \\
Pourquoi refuse-t-on la conscience à l'animal ? \\
Pourquoi respecter autrui ? \\
Pourquoi respecter le droit ? \\
Pourquoi respecter les anciens ? \\
Pourquoi rit-on ? \\
Pourquoi sauver les apparences ? \\
Pourquoi sauver les phénomènes ? \\
Pourquoi se confesser ? \\
Pourquoi se divertir ? \\
Pourquoi se fier à autrui ? \\
Pourquoi se mettre à la place d'autrui ? \\
Pourquoi séparer les pouvoirs ? \\
Pourquoi se révolter ? \\
Pourquoi se soucier du futur ? \\
Pourquoi s'étonner ? \\
Pourquoi s'exprimer ? \\
Pourquoi s'inspirer de l'art antique ? \\
Pourquoi s'intéresser à l'histoire ? \\
Pourquoi s'intéresser à l'origine ? \\
Pourquoi s'interroger sur l'origine du langage ? \\
Pourquoi soigner son apparence ? \\
Pourquoi sommes-nous déçus par les œuvres d'un faussaire ? \\
Pourquoi sommes-nous des êtres moraux ? \\
Pourquoi sommes-nous moraux ? \\
Pourquoi suivre l'actualité ? \\
Pourquoi tenir ses promesses ? \\
Pourquoi théoriser ? \\
Pourquoi transformer le monde ? \\
Pourquoi transmettre ? \\
Pourquoi travailler ? \\
Pourquoi travaille-t-on ? \\
Pourquoi un droit du travail ? \\
Pourquoi une instruction publique ? \\
Pourquoi un fait devrait-il être établi ? \\
Pourquoi veut-on changer le monde ? \\
Pourquoi veut-on la vérité ? \\
Pourquoi vivons-nous ? \\
Pourquoi vivre ensemble ? \\
Pourquoi vouloir avoir raison ? \\
Pourquoi vouloir se connaître ? \\
Pourquoi voulons-nous savoir ? \\
Pourquoi voyager ? \\
Pourquoi y a-t-il des conflits insolubles ? \\
Pourquoi y a-t-il des institutions ? \\
Pourquoi y a-t-il des religions ? \\
Pourquoi y a-t-il du mal dans le monde ? \\
Pourquoi y a-t-il plusieurs façons de démontrer ? \\
Pourquoi y a-t-il plusieurs langues ? \\
Pourquoi y a-t-il plusieurs sciences ? \\
Pourquoi y a-t-il quelque chose plutôt que rien ? \\
Pourquoi y a-t-il des lois ? \\
Pourquoi y a-t-il plusieurs philosophies ? \\
Pourquoi ? \\
Pourrait-on se passer de l'argent ? \\
Pourrions-nous comprendre une pensée non humaine ? \\
Pour vivre heureux, vivons cachés \\
Pouvoir et autorité \\
Pouvoir et contre-pouvoir \\
Pouvoir et devoir \\
Pouvoir et politique \\
Pouvoir et puissance \\
Pouvoir et savoir \\
Pouvoir, magie, secret \\
Pouvoirs et libertés \\
Pouvoir temporel et pouvoir spirituel \\
Pouvons-nous communiquer ce que nous sentons ? \\
Pouvons-nous connaître sans interpréter ? \\
Pouvons-nous désirer ce qui nous nuit ? \\
Pouvons-nous devenir meilleurs ? \\
Pouvons-nous dissocier le réel de nos interprétations ? \\
Pouvons-nous être certains que nous ne rêvons pas ? \\
Pouvons-nous être objectifs ? \\
Pouvons-nous faire l'expérience de la liberté ? \\
Pouvons-nous justifier nos croyances ? \\
Pouvons-nous savoir ce que nous ignorons ? \\
Prédicats et relations \\
Prédiction et probabilité \\
Prédire et expliquer \\
Prémisses et conclusions \\
Prendre conscience \\
Prendre des risques \\
Prendre la parole \\
Prendre le pouvoir \\
Prendre les armes \\
Prendre ses désirs pour des réalités \\
Prendre ses responsabilités \\
Prendre soin \\
Prendre son temps \\
Prendre son temps, est-ce le perdre ? \\
Prendre une décision \\
Prendre une décision politique \\
Présence et absence \\
Présence et représentation \\
Preuve et démonstration \\
Prévoir \\
Prévoir les comportements humains \\
Primitif ou premier ? \\
Principe et cause \\
Principe et commencement \\
Principe et fondement \\
Principes et stratégie \\
Probabilité et explication scientifique \\
Production et création \\
Produire et créer \\
Promettre \\
Proposition et jugement \\
Propriétés artistiques, propriétés esthétiques \\
Prose et poésie \\
Prospérité et sécurité \\
Protester \\
Prouver \\
Prouver Dieu \\
Prouver en métaphysique \\
Prouver et démontrer \\
Prouver et éprouver \\
Prouver et réfuter \\
Prouver la force d'âme \\
Prouver l'existence de Dieu \\
Prouvez-le ! \\
Providence et destin \\
Prudence et liberté \\
Psychologie et contrôle des comportements \\
Psychologie et métaphysique \\
Psychologie et neurosciences \\
Publier \\
Puis-je aimer tous les hommes ? \\
Puis-je comprendre autrui ? \\
Puis-je décider de croire ? \\
Puis-je dire « ceci est mon corps » ? \\
Puis-je douter de ma propre existence ? \\
Puis-je être dans le vrai sans le savoir ? \\
Puis-je être heureux dans un monde chaotique ? \\
Puis-je être libre sans être responsable ? \\
Puis-je être sûr de bien agir ? \\
Puis-je être sûr de ne pas me tromper ? \\
Puis-je être sûr que je ne rêve pas ? \\
Puis-je être universel ? \\
Puis-je faire ce que je veux de mon corps ? \\
Puis-je faire confiance à mes sens ? \\
Puis-je invoquer l'inconscient sans ruiner la morale ? \\
Puis-je me passer d'imiter autrui ? \\
Puis-je ne croire que ce que je vois ? \\
Puis-je ne pas vouloir ce que je désire ? \\
Puis-je ne rien croire ? \\
Puis-je répondre des autres ? \\
Puis-je savoir ce qui m'est propre ? \\
Pulsion et instinct \\
Pulsions et passions \\
Punir \\
Punir ou soigner ? \\
Punition et vengeance \\
Qu'ai-je le droit d'exiger d'autrui ? \\
Qu'ai-je le droit d'exiger des autres ? \\
Qu'aime-t-on dans l'amour ? \\
Qu'aime-t-on ? \\
Qualité et quantité \\
Qualités premières, qualités secondes \\
Quand agit-on ? \\
Quand faut-il désobéir aux lois ? \\
Quand faut-il désobéir ? \\
Quand faut-il mentir ? \\
Quand la guerre finira-t-elle ? \\
Quand la technique devient-elle art ? \\
Quand le temps passe, que reste-t-il ? \\
Quand pense-t-on ? \\
Quand peut-on se passer de théories ? \\
Quand suis-je en faute ? \\
Quand une autorité est-elle légitime ? \\
Quand y a-t-il de l'art ? \\
Quand y a-t-il œuvre ? \\
Quand y a-t-il paysage ? \\
Quand y a-t-il peuple ? \\
Qu'anticipent les romans d'anticipation ? \\
Quantification et pensée scientifique \\
Quantité et qualité \\
Qu'a perdu le fou ? \\
Qu'appelle-t-on chef-d'œuvre ? \\
Qu'appelle-t-on destin ? \\
Qu'appelle-t-on penser ? \\
Qu'apprend-on dans les livres ? \\
Qu'apprend-on des romans ? \\
Qu'apprend-on en commettant une faute ? \\
Qu'apprend-on quand on apprend à parler ? \\
Qu'apprenons-nous de nos affects ? \\
Qu'a-t-on le droit de pardonner ? \\
Qu'a-t-on le droit d'exiger ? \\
Qu'a-t-on le droit d'interpréter ? \\
Qu'attendons-nous de la technique ? \\
Qu'attendons-nous pour être heureux ? \\
Qu'avons-nous à apprendre des historiens ? \\
Qu'avons-nous en commun ? \\
Que célèbre l'art ? \\
Que cherchons-nous dans le regard des autres ? \\
Que choisir ? \\
Que connaissons-nous du vivant ? \\
Que construit le politique ? \\
Que coûte une victoire ? \\
Que crée l'artiste ? \\
Que déduire d'une contradiction ? \\
Que démontrent nos actions ? \\
Que désire-t-on ? \\
Que désirons-nous quand nous désirons savoir ?Qu'est-ce qu'un événement historique ? \\
Que désirons-nous ? \\
Que devons-nous à autrui ? \\
Que devons-nous à l'État ? \\
Que disent les légendes ? \\
Que disent les tables de vérité ? \\
Que dit la loi ? \\
Que dit la musique ? \\
Que dois-je à autrui ? \\
Que dois-je à l'État ? \\
Que dois-je respecter en autrui ? \\
Que doit la pensée à l'écriture ? \\
Que doit la science à la technique ? \\
Que doit-on aux morts ? \\
Que doit-on croire ? \\
Que doit-on désirer pour ne pas être déçu ? \\
Que doit-on faire de ses rêves ? \\
Que doit-on savoir avant d'agir ? \\
Que faire de la diversité des arts ? \\
Que faire de nos émotions ? \\
Que faire de nos passions ? \\
Que faire de notre cerveau ? \\
Que faire des adversaires ? \\
Que faire ? \\
Que fait la police ? \\
Que faut-il absolument savoir ? \\
Que faut-il craindre ? \\
Que faut-il pour faire un monde ? \\
Que faut-il respecter ? \\
Que faut-il savoir pour agir ? \\
Que faut-il savoir pour gouverner ? \\
Que gagne-t-on à travailler ? \\
Que la nature soit explicable, est-ce explicable ? \\
Quel contrôle a-t-on sur son corps ? \\
Quel est le bon nombre d'amis ? \\
Quel est le but de la politique ? \\
Quel est le but d'une théorie physique ? \\
Quel est le but du travail scientifique ? \\
Quel est le contraire du travail ? \\
Quel est le fondement de la propriété ? \\
Quel est le fondement de l'autorité ? \\
Quel est le poids du passé ? \\
Quel est le pouvoir de l'art ? \\
Quel est le pouvoir des mots ? La prévoyance \\
Quel est le rôle de la créativité dans les sciences ? \\
Quel est le rôle du concept en art ? \\
Quel est le rôle du médecin ? \\
Quel est le sens du progrès technique ? \\
Quel est le sujet de la pensée ? \\
Quel est le sujet de l'histoire ? \\
Quel est le sujet du devenir ? \\
Quel est l'être de l'illusion ? \\
Quel est l'homme des Droits de l'homme ? \\
Quel est l'objet de la biologie ? \\
Quel est l'objet de la métaphysique ? \\
Quel est l'objet de l'amour ? \\
Quel est l'objet de la perception ? \\
Quel est l'objet de la philosophie politique ? \\
Quel est l'objet de la science ? \\
Quel est l'objet de l'échange ? \\
Quel est l'objet de l'esthétique ? \\
Quel est l'objet de l'histoire ? \\
Quel est l'objet des mathématiques ? \\
Quel est l'objet des sciences humaines ? \\
Quel est l'objet des sciences politiques ? \\
Quel est l'objet du désir ? \\
Quel être peut être un sujet de droits ? \\
Quelle causalité pour le vivant ? \\
Quelle confiance accorder au langage ? \\
Quelle est la cause du désir ? \\
Quelle est la fin de la science ? \\
Quelle est la fin de l'État ? \\
Quelle est la fonction première de l'État ? \\
Quelle est la force de la loi ? \\
Quelle est la matière de l'œuvre d'art ? \\
Quelle est la place de l'imagination dans la vie de l'esprit ? \\
Quelle est la place du hasard dans l'histoire ? \\
Quelle est la portée d'un exemple ? \\
Quelle est la réalité de la matière ? \\
Quelle est la réalité de l'avenir ? \\
Quelle est la réalité d'une idée ? \\
Quelle est la réalité du passé ? \\
Quelle est la spécificité de la communauté politique ? \\
Quelle est la valeur de l'expérience ? \\
Quelle est la valeur des hypothèses ? \\
Quelle est la valeur d'une expérimentation ? \\
Quelle est la valeur d'une œuvre d'art ? \\
Quelle est la valeur du rêve ? \\
Quelle est la valeur du témoignage ? \\
Quelle est la valeur du vivant ? \\
Quelle est l'unité du « je » ? \\
Quelle idée le fanatique se fait-il de la vérité ? \\
Quelle politique fait-on avec les sciences humaines ? \\
Quelle réalité attribuer à la matière ? \\
Quelle réalité l'art nous fait-il connaître ? \\
Quelle réalité la science décrit-elle ? \\
Quelle réalité peut-on accorder au temps ? \\
Quelles actions permettent d'être heureux ? \\
Quelle sorte d'histoire ont les sciences ? \\
Quelles règles la technique dicte-t-elle à l'art ? \\
Quelles sont les caractéristiques d'une proposition morale ? \\
Quelles sont les caractéristiques d'un être vivant ? \\
Quelles sont les limites de la démonstration ? \\
Quelles sont les limites de la souveraineté ? \\
Quelle valeur accorder à l'expérience ? \\
Quelle valeur devons accorder à l'expérience ? \\
Quelle valeur devons-nous accorder à l'expérience ? \\
Quelle valeur devons-nous accorder à l'intuition ? \\
Quelle valeur donner à la notion de « corps social » ? \\
Quelle valeur peut-on accorder à l'expérience ? \\
Quelle vérité y-a-t-il dans la perception ? \\
Quel réel pour l'art ? \\
Quel rôle attribuer à l'intuition \emph{a priori} dans une philosophie des mathématiques ? \\
Quel rôle la logique joue-t-elle en mathématiques ? \\
Quel rôle l'imagination joue-t-elle en mathématiques ? \\
Quels désirs dois-je m'interdire ? \\
Quel sens donner à l'expression « gagner sa vie » ? \\
Quel sens y a-t-il à se demander si les sciences humaines sont vraiment des sciences ? \\
Quels sont les droits de la conscience ? \\
Quels sont les fondements de l'autorité ? \\
Quels sont les moyens légitimes de la politique ? \\
Quel usage faut-il faire des exemples ? \\
Quel usage peut-on faire des fictions ? \\
Que manque-t-il à une machine pour être vivante ? \\
Que manque-t-il aux machines pour être des organismes ? \\
Que mesure-t-on du temps ? \\
Que montre l'image ? \\
Que montre une démonstration ? \\
Que montre un tableau ? \\
Que ne peut-on pas expliquer ? \\
Que nous append l'histoire ? \\
Que nous apporte l'art ? \\
Que nous apporte la vérité ? \\
Que nous apprend la définition de la vérité ? \\
Que nous apprend la diversité des langues ? \\
Que nous apprend la fiction sur la réalité ? \\
Que nous apprend la grammaire ? \\
Que nous apprend la maladie sur la santé ? \\
Que nous apprend la musique ? \\
Que nous apprend la poésie ? \\
Que nous apprend la psychanalyse de l'homme ? \\
Que nous apprend la sociologie des sciences ? \\
Que nous apprend la vie ? \\
Que nous apprend le cinéma ? \\
Que nous apprend le faux ? \\
Que nous apprend le plaisir ? \\
Que nous apprend le toucher ? \\
Que nous apprend l'étude du cerveau ? \\
Que nous apprend l'expérience ? \\
Que nous apprend l'histoire de l'art ? \\
Que nous apprend l'histoire des sciences ? \\
Que nous apprend, sur la politique, l'utopie ? \\
Que nous apprennent les algorithmes sur nos sociétés ? \\
Que nous apprennent les animaux sur nous-mêmes ? \\
Que nous apprennent les animaux ? \\
Que nous apprennent les controverses scientifiques ? \\
Que nous apprennent les expériences de pensée ? \\
Que nous apprennent les faits divers ? \\
Que nous apprennent les illusions d'optique ? \\
Que nous apprennent les jeux ? \\
Que nous apprennent les langues étrangères ? \\
Que nous apprennent les machines ? \\
Que nous apprennent les métaphores ? \\
Que nous apprennent les mythes ? \\
Que nous enseigne l'expérience ? \\
Que nous enseignent les œuvres d'art ? \\
Que nous enseignent les sens ? \\
Que nous montre le cinéma ? \\
Que nous montrent les natures mortes ? \\
Que nous réserve l'avenir ? \\
Que nul n'entre ici s'il n'est géomètre \\
Que partage-t-on avec les animaux ? \\
Que peindre ? \\
Que peint le peintre ? \\
Que penser de l'adage : « Que la justice s'accomplisse, le monde dût-il périr » (Fiat justitia pereat mundus) ? \\
Que penser de la formule : « il faut suivre la nature » ? \\
Que penser de l'opposition travail manuel, travail intellectuel ? \\
Que percevons-nous d'autrui ? \\
Que percevons-nous du monde extérieur ? \\
Que percevons-nous ? \\
Que perçoit-on ? \\
Que perd la pensée en perdant l'écriture ? \\
Que perdrait la pensée en perdant l'écriture ? \\
Que peut expliquer l'histoire ? \\
Que peut la force ? \\
Que peut la musique ? \\
Que peut la philosophie ? \\
Que peut la politique ? \\
Que peut la raison ? \\
Que peut l'art ? \\
Que peut la science ? \\
Que peut la théorie ? \\
Que peut la volonté ? \\
Que peut le corps ? \\
Que peut le politique ? \\
Que peut l'esprit sur la matière ? \\
Que peut l'esprit ? \\
Que peut l'État ? \\
Que peut-on attendre de l'État ? \\
Que peut-on attendre du droit international ? \\
Que peut-on calculer ? \\
Que peut-on comprendre immédiatement ? \\
Que peut-on comprendre qu'on ne puisse expliquer ? \\
Que peut-on contre un préjugé ? \\
Que peut-on cultiver ? \\
Que peut-on démontrer ? \\
Que peut-on dire de l'être ? \\
Que peut-on échanger ? \\
Que peut-on interdire ? \\
Que peut-on partager ? \\
Que peut-on savoir de l'inconscient ? \\
Que peut-on savoir de soi ? \\
Que peut-on savoir du réel ? \\
Que peut-on savoir par expérience ? \\
Que peut-on sur autrui ? \\
Que peut-on voir ? \\
Que peut un corps ? \\
Que pouvons-nous aujourd'hui apprendre des sciences d'autrefois ? \\
Que pouvons-nous espérer de la connaissance du vivant ? \\
Que pouvons-nous faire de notre passé ? \\
Que produit l'inconscient ? \\
Que prouvent les faits ? \\
Que prouvent les preuves de l'existence de Dieu ? \\
Que recherche l'artiste ? \\
Que répondre au sceptique ? \\
Que reste-t-il d'une existence ? \\
Que sais-je d'autrui ? \\
Que sais-je de ma souffrance ? \\
Que sait la conscience ? \\
Que sait-on de soi ? \\
Que sait-on du réel ? \\
Que savons-nous de l'inconscient ? \\
Que serait la vie sans l'art ? \\
Que serait le meilleur des mondes ? \\
Que serait un art total ? \\
Que serait une démocratie parfaite ? \\
Que serions-nous sans l'État ? \\
Que signifie apprendre ? \\
Que signifie connaître ? \\
Que signifie être en guerre ? \\
Que signifie être mortel ? \\
Que signifie la mort ? \\
Que signifie l'idée de technoscience ? \\
Que signifient les mots ? \\
Que signifie pour l'homme être mortel ? \\
Que signifier « juger en son âme et conscience » ? \\
Que signifie « donner le change » ? \\
Que sondent les sondages d'opinion ? \\
Que sont les apparences ? \\
Qu'est-ce la technique ? \\
Qu'est-ce le mal radical ? \\
Qu'est-ce qu'agir ensemble ? \\
Qu'est-ce qu'aimer une œuvre d'art ? \\
Qu'est-ce qu'apprendre ? \\
Qu'est-ce qu'argumenter ? \\
Qu'est-ce qu'avoir conscience de soi ? \\
Qu'est-ce qu'avoir de l'expérience ? \\
Qu'est-ce qu'avoir du goût ? \\
Qu'est-ce qu'avoir du style ? \\
Qu'est-ce qu'avoir un droit ? \\
Qu'est-ce que calculer ? \\
Qu'est-ce que catégoriser ? \\
Qu'est-ce que commencer ? \\
Qu'est-ce que composer une œuvre ? \\
Qu'est-ce que comprendre une œuvre d'art ? \\
Qu'est-ce que comprendre ? \\
Qu'est-ce que créer ? \\
Qu'est-ce que croire ? \\
Qu'est-ce que décider ? \\
Qu'est-ce que définir ? \\
Qu'est-ce que démontrer ? \\
Qu'est-ce que déraisonner ? \\
Qu'est-ce que Dieu pour athée ? \\
Qu'est-ce que Dieu pour un athée ? \\
Qu'est-ce que discuter ? \\
Qu'est-ce qu'éduquer ? \\
Qu'est-ce que faire autorité ? \\
Qu'est-ce que faire preuve d'humanité ? \\
Qu'est-ce que faire une expérience ? \\
Qu'est-ce que gouverner ? \\
Qu'est-ce que guérir ? \\
Qu'est-ce que jouer ? \\
Qu'est-ce que juger ? \\
Qu'est-ce que la barbarie ? \\
Qu'est-ce que la causalité ? \\
Qu'est-ce que la critique ? \\
Qu'est-ce que la culture générale \\
Qu'est-ce que la démocratie ? \\
Qu'est-ce que la folie ? \\
Qu'est-ce que la normalité ? \\
Qu'est-ce que la politique ? \\
Qu'est-ce que la psychologie ? \\
Qu'est-ce que la raison d'État ? \\
Qu'est-ce que l'art contemporain ? \\
Qu'est-ce que la science saisit du vivant ? \\
Qu'est-ce que la science, si elle inclut la psychanalyse ? \\
Qu'est-ce que la scientificité ? \\
Qu'est-ce que la souveraineté ? \\
Qu'est-ce que la tragédie ? \\
Qu'est-ce que la valeur marchande ? \\
Qu'est-ce que la vie bonne ? \\
Qu'est-ce que la vie ? \\
Qu'est-ce que le bonheur ? \\
Qu'est-ce que le cinéma a changé dans l'idée que l'on se fait du temps ? \\
Qu'est-ce que le cinéma donne à voir ? \\
Qu'est-ce que le courage ? \\
Qu'est-ce que le désordre ? \\
Qu'est-ce que le dogmatisme ? \\
Qu'est-ce que le hasard ? \\
Qu'est-ce que le langage ordinaire ? \\
Qu'est-ce que le malheur ? \\
Qu'est-ce que le moi ? \\
Qu'est-ce que le naturalisme ? \\
Qu'est-ce que l'enfance ? \\
Qu'est-ce que le nihilisme ? \\
Qu'est-ce que le pathologique nous apprend sur le normal ? \\
Qu'est-ce que le présent ? \\
Qu'est-ce que le réel ? \\
Qu'est-ce que le sacré ? \\
Qu'est-ce que le sens pratique ? \\
Qu'est-ce que le sublime ? \\
Qu'est-ce que le travail ? \\
Qu'est-ce que l'harmonie ? \\
Qu'est-ce que l'inconscient ? \\
Qu'est-ce que l'indifférence ? \\
Qu'est-ce que l'intérêt général ? \\
Qu'est-ce que l'intuition ? \\
Qu'est ce que lire ? \\
Qu'est-ce que lire ? \\
Qu'est-ce que l'ordinaire ? \\
Qu'est-ce que maîtriser une technique ? \\
Qu'est-ce que manquer de culture ? \\
Qu'est-ce que méditer ? \\
Qu'est-ce que mourir ? \\
Qu'est-ce qu'enquêter ? \\
Qu'est-ce qu'enseigner ? \\
Qu'est-ce que parler le même langage ? \\
Qu'est-ce que parler ? \\
Qu'est-ce que penser ? \\
Qu'est-ce que percevoir ? \\
Qu'est-ce que perdre la raison ? \\
Qu'est-ce que perdre sa liberté ? \\
Qu'est-ce que perdre son temps ? \\
Qu'est-ce que prendre conscience ? \\
Qu'est-ce que prendre le pouvoir ? \\
Qu'est-ce que promettre ? \\
Qu'est-ce que prouver ? \\
Qu'est-ce que raisonner ? \\
Qu'est-ce que réfuter une philosophie ? \\
Qu'est-ce que réfuter ? \\
Qu'est-ce que résister ? \\
Qu'est-ce que résoudre une contradiction ? \\
Qu'est-ce que rester soi-même ? \\
Qu'est-ce que réussir sa vie ? \\
Qu'est-ce que s'orienter ? \\
Qu'est-ce que témoigner ? \\
Qu'est-ce que traduire ? \\
Qu'est-ce que travailler ? \\
Qu'est-ce qu'être adulte ? \\
Qu'est-ce qu'être artiste ? \\
Qu'est-ce qu'être asocial ? \\
Qu'est-ce qu'être barbare ? \\
Qu'est-ce qu'être chez soi ? \\
Qu'est-ce qu'être cohérent ? \\
Qu'est-ce qu'être comportementaliste ? \\
Qu'est-ce qu'être cultivé ? \\
Qu'est-ce qu'être dans le vrai ? \\
Qu'est-ce qu'être de son temps ? \\
Qu'est-ce qu'être efficace en politique ? \\
Qu'est-ce qu'être en vie ? \\
Qu'est-ce qu'être esclave ? \\
Qu'est-ce qu'être fidèle à soi-même ? \\
Qu'est-ce qu'être généreux ? \\
Qu'est-ce qu'être idéaliste ? \\
Qu'est-ce qu'être inhumain ? \\
Qu'est-ce qu'être l'auteur de son acte ? \\
Qu'est-ce qu'être libéral ? \\
Qu'est-ce qu'être libre ? \\
Qu'est-ce qu'être maître de soi-même ? \\
Qu'est-ce qu'être malade ? \\
Qu'est-ce qu'être moderne ? \\
Qu'est-ce qu'être nihiliste ? \\
Qu'est-ce qu'être normal ? \\
Qu'est-ce qu'être rationnel ? \\
Qu'est-ce qu'être réaliste ? \\
Qu'est-ce qu'être républicain ? \\
Qu'est-ce qu'être sceptique ? \\
Qu'est-ce qu'être seul ? \\
Qu'est-ce qu'être soi-même ? \\
Qu'est-ce qu'être souverain ? \\
Qu'est-ce qu'être spirituel ? \\
Qu'est-ce qu'être témoin ? \\
Qu'est-ce qu'être un bon citoyen ? \\
Qu'est-ce qu'être un esclave ? \\
Qu'est-ce qu'être un sujet ? \\
Qu'est-ce qu'être vivant ? \\
Qu'est-ce qu'être ? \\
Qu'est-ce que un individu \\
Qu'est-ce que vérifier une théorie ? \\
Qu'est-ce que vérifier ? \\
Qu'est-ce que vivre bien ? \\
Qu'est-ce que vivre ? \\
Qu'est-ce qu'exister pour un individu ? \\
Qu'est-ce qu'exister ? \\
Qu'est-ce qu'expliquer ? \\
Qu'est-ce que « parler le même langage » ? \\
Qu'est-ce que « se rendre maître et possesseur de la nature » ? \\
Qu'est-ce qu'habiter ? \\
Qu'est-ce qui agit ? \\
Qu'est-ce qui apparaît ? \\
Qu'est-ce qui dépend de nous ? \\
Qu'est-ce qui distingue un vivant d'une machine ? \\
Qu'est-ce qui est absurde ? \\
Qu'est-ce qui est actuel ? \\
Qu'est-ce qui est beau ? \\
Qu'est ce qui est concret ? \\
Qu'est-ce qui est concret ? \\
Qu'est ce qui est contre nature ? \\
Qu'est-ce qui est contre nature ? \\
Qu'est ce qui est culturel ? \\
Qu'est-ce qui est culturel ? \\
Qu'est-ce qui est donné ? \\
Qu'est-ce qui est essentiel ? \\
Qu'est-ce qui est extérieur à ma conscience, ? \\
Qu'est-ce qui est historique ? \\
Qu'est-ce qui est hors la loi ? \\
Qu'est-ce qui est hors-la-loi ? \\
Qu'est-ce qui est immoral ? \\
Qu'est-ce qui est impossible ? \\
Qu'est-ce qui est indiscutable ? \\
Qu'est-ce qui est invérifiable ? \\
Qu'est-ce qui est irrationnel ? \\
Qu'est ce qui est irréfutable ? \\
Qu'est-ce qui est irréversible ? \\
Qu'est-ce qui est le plus à craindre, l'ordre ou le désordre ? \\
Qu'est-ce qui est mauvais dans l'égoïsme ? \\
Qu'est-ce qui est mien ? \\
Qu'est-ce qui est moderne ? \\
Qu'est-ce qui est naturel ? \\
Qu'est-ce qui est noble ? \\
Qu'est-ce qui est politique ? \\
Qu'est-ce qui est possible ? \\
Qu'est-ce qui est public ? \\
Qu'est-ce qui est réel ? \\
Qu'est-ce qui est respectable ? \\
Qu'est ce qui est sacré ? \\
Qu'est-ce qui est sauvage ? \\
Qu'est-ce qui est scientifique ? \\
Qu'est-ce qui est spectaculaire ? \\
Qu'est-ce qui est sublime ? \\
Qu'est-ce qui est tragique ? \\
Qu'est-ce qui est vital pour le vivant ? \\
Qu'est-ce qui est vital ? \\
Qu'est ce qui existe ? \\
Qu'est-ce qui existe ? \\
Qu'est-ce qui fait changer les sociétés ? \\
Qu'est-ce qui fait d'une activité un travail ? \\
Qu'est-ce qui fait la force de la loi ? \\
Qu'est-ce qui fait la force des lois ? \\
Qu'est-ce qui fait la justice des lois ? \\
Qu'est-ce qui fait la légitimité d'une autorité politique ? \\
Qu'est-ce qui fait la valeur de la technique ? \\
Qu'est-ce qui fait la valeur de l'œuvre d'art ? \\
Qu'est-ce qui fait la valeur d'une croyance ? \\
Qu'est-ce qui fait la valeur d'une existence ? \\
Qu'est-ce qui fait la valeur d'une œuvre d'art ? \\
Qu'est-ce qui fait le pouvoir des mots ? \\
Qu'est-ce qui fait le propre d'un corps propre ? \\
Qu'est-ce qui fait l'humanité d'un corps ? \\
Qu'est-ce qui fait l'unité d'une science ? \\
Qu'est-ce qui fait l'unité d'un organisme ? \\
Qu'est-ce qui fait l'unité d'un peuple ? \\
Qu'est-ce qui fait l'unité du vivant ? \\
Qu'est-ce qui fait mon identité ? \\
Qu'est-ce qui fait qu'une théorie est vraie ? \\
Qu'est-ce qui fait un peuple ? \\
Qu'est-ce qui fonde la croyance ? \\
Qu'est-ce qui fonde le respect d'autrui ? \\
Qu'est-ce qu'ignore la science ? \\
Qu'est-ce qui importe ? \\
Qu'est-ce qui innocente le bourreau ? \\
Qu'est-ce qui justifie l'hypothèse d'un inconscient ? \\
Qu'est-ce qui justifie une croyance ? \\
Qu'est-ce qu'imaginer ? \\
Qu'est-ce qui menace la liberté ? \\
Qu'est-ce qui mesure la valeur d'un travail ? \\
Qu'est-ce qui n'a pas d'histoire ? \\
Qu'est-ce qui ne disparaît jamais ?/ \\
Qu'est-ce qui ne s'achète pas ? \\
Qu'est-ce qui ne s'échange pas ? \\
Qu'est-ce qui n'est pas démontrable ? \\
Qu'est-ce qui n'est pas politique ? \\
Qu'est-ce qui n'existe pas ? \\
Qu'est-ce qui nous fait danser ? \\
Qu'est-ce qu'interpréter une œuvre d'art ? \\
Qu'est-ce qu'interpréter ? \\
Qu'est-ce qui peut se transformer ? \\
Qu'est-ce qui plaît dans la musique ? \\
Qu'est ce qui rapproche le vivant de la machine ? \\
Qu'est-ce qui rend l'objectivité difficile dans les sciences humaines ? \\
Qu'est-ce qui rend vrai un énoncé ? \\
Qu'est-ce qu'obéir ? \\
Qu'est-ce qu'on attend ? \\
Qu'est-ce qu'on ne peut comprendre ? \\
Qu'est-ce qu'un abus de langage ? \\
Qu'est-ce qu'un abus de pouvoir ? \\
Qu'est-ce qu'un accident ? \\
Qu'est-ce qu'un acte libre ? \\
Qu'est-ce qu'un acte moral ? \\
Qu'est-ce qu'un acte symbolique ? \\
Qu'est-ce qu'un acteur ? \\
Qu'est-ce qu'un acte ? \\
Qu'est-ce qu'un adversaire en politique ? \\
Qu'est-ce qu'un alter ego \\
Qu'est-ce qu'un alter ego ? \\
Qu'est-ce qu'un ami ? \\
Qu'est-ce qu'un animal domestique ? \\
Qu'est-ce qu'un animal ? \\
Qu'est-ce qu'un argument ? \\
Qu'est-ce qu'un art de vivre ? \\
Qu'est-ce qu'un artiste ? \\
Qu'est-ce qu'un art moral ? \\
Qu'est-ce qu'un auteur ? \\
Qu'est-ce qu'un axiome ? \\
Qu'est-ce qu'un bon citoyen ? \\
Qu'est-ce qu'un bon conseil ? \\
Qu'est-ce qu'un bon gouvernement ? \\
Qu'est-ce qu'un bon jugement ? \\
Qu'est-ce qu'un capital culturel ? \\
Qu'est-ce qu'un caractère ? \\
Qu'est-ce qu'un cas de conscience ? \\
Qu'est-ce qu'un châtiment ? \\
Qu'est-ce qu'un chef d'œuvre ? \\
Qu'est-ce qu'un chef-d'œuvre ? \\
Qu'est-ce qu'un chef ? \\
Qu'est-ce qu'un choix éclairé ? \\
Qu'est-ce qu'un citoyen libre ? \\
Qu'est-ce qu'un citoyen ? \\
Qu'est-ce qu'un civilisé ? \\
Qu'est-ce qu'un classique ? \\
Qu'est-ce qu'un code ? \\
Qu'est-ce qu'un concept philosophique ? \\
Qu'est-ce qu'un concept scientifique ? \\
Qu'est-ce qu'un concept ? \\
Qu'est-ce qu'un conflit de générations ? \\
Qu'est-ce qu'un conflit politique ? \\
Qu'est-ce qu'un consommateur ? \\
Qu'est-ce qu'un contenu de conscience ? \\
Qu'est-ce qu'un contrat ? \\
Qu'est-ce qu'un contre-pouvoir ? \\
Qu'est-ce qu'un corps social ? \\
Qu'est-ce qu'un coup d'État ? \\
Qu'est-ce qu'un créateur ? \\
Qu'est-ce qu'un crime contre l'humanité ? \\
Qu'est-ce qu'un crime politique ? \\
Qu'est-ce qu'un crime ? \\
Qu'est-ce qu'un critère de vérité ? \\
Qu'est-ce qu'un déni ? \\
Qu'est-ce qu'un désir satisfait ? \\
Qu'est-ce qu'un détail ? \\
Qu'est-ce qu'un dialogue ? \\
Qu'est-ce qu'un dieu ? \\
Qu'est-ce qu'un Dieu ? \\
Qu'est-ce qu'un dilemme ? \\
Qu'est-ce qu'un document ? \\
Qu'est-ce qu'un dogme ? \\
Qu'est-ce qu'une action intentionnelle ? \\
Qu'est-ce qu'une action juste ? \\
Qu'est-ce qu'une action politique ? \\
Qu'est-ce qu'une action réussie ? \\
Qu'est-ce qu'une alternative ? \\
Qu'est-ce qu'une âme ? \\
Qu'est-ce qu'une analyse ? \\
Qu'est-ce qu'une aporie ? \\
Qu'est-ce qu'une autorité légitime ? \\
Qu'est-ce qu'une avant-garde ? \\
Qu'est-ce qu'une belle démonstration ? \\
Qu'est-ce qu'une belle forme ? \\
Qu'est-ce qu'une belle mort ? \\
Qu'est-ce qu'une bête ? \\
Qu'est-ce qu'une bonne définition ? \\
Qu'est-ce qu'une bonne délibération ? \\
Qu'est-ce qu'une bonne éducation ? \\
Qu'est-ce qu'une bonne loi ? \\
Qu'est-ce qu'une bonne méthode ? \\
Qu'est-ce qu'une bonne traduction ? \\
Qu'est-ce qu'une catastrophe ? \\
Qu'est-ce qu'une catégorie de l'être ? \\
Qu'est-ce qu'une catégorie ? \\
Qu'est-ce qu'une cause ? \\
Qu'est-ce qu'un échange juste ? \\
Qu'est-ce qu'un échange réussi ? \\
Qu'est-ce qu'une chose matérielle ? \\
Qu'est-ce qu'une chose ? \\
Qu'est-ce qu'une civilisation ? \\
Qu'est-ce qu'une collectivité ? \\
Qu'est-ce qu'une comédie ? \\
Qu'est-ce qu'une communauté politique ? \\
Qu'est-ce qu'une communauté ? \\
Qu'est-ce qu'une conception scientifique du monde ? \\
Qu'est-ce qu'une condition suffisante ? \\
Qu'est-ce qu'une conduite irrationnelle ? \\
Qu'est ce qu'une connaissance fiable ? \\
Qu'est-ce qu'une connaissance non scientifique ? \\
Qu'est-ce qu'une connaissance par les faits ? \\
Qu'est-ce qu'une constitution ? \\
Qu'est-ce qu'une contrainte ? \\
Qu'est-ce qu'une convention ? \\
Qu'est-ce qu'une conviction ? \\
Qu'est-ce qu'une crise politique ? \\
Qu'est-ce qu'une crise ? \\
Qu'est-ce qu'une croyance rationnelle ? \\
Qu'est-ce qu'une croyance vraie ? \\
Qu'est-ce qu'une croyance ? \\
Qu'est-ce qu'une culture ? \\
Qu'est-ce qu'une décision politique ? \\
Qu'est-ce qu'une décision rationnelle ? \\
Qu'est-ce qu'une découverte scientifique ? \\
Qu'est-ce qu'une découverte ? \\
Qu'est-ce qu'une définition ? \\
Qu'est-ce qu'une démocratie ? \\
Qu'est-ce qu'une démonstration ? \\
Qu'est-ce qu'une discipline savante ? \\
Qu'est-ce qu'une école philosophique ? \\
Qu'est-ce qu'une éducation réussie ? \\
Qu'est-ce qu'une éducation scientifique ? \\
Qu'est-ce qu'une époque ? \\
Qu'est-ce qu'une erreur ? \\
Qu'est-ce qu'une exception ? \\
Qu'est-ce qu'une existence historique ? \\
Qu'est-ce qu'une expérience cruciale ? \\
Qu'est-ce qu'une expérience de pensée ? \\
Qu'est-ce qu'une expérience religieuse ? \\
Qu'est-ce qu'une expérience scientifique ? \\
Qu'est-ce qu'une expérience ? \\
Qu'est-ce qu'une explication matérialiste ? \\
Qu'est-ce qu'une exposition ? \\
Qu'est-ce qu'une famille ? \\
Qu'est-ce qu'une fausse science ? \\
Qu'est-ce qu'une faute de goût ? \\
Qu'est-ce qu'une fiction ? \\
Qu'est-ce qu'une fonction ? \\
Qu'est-ce qu'une forme ? \\
Qu'est-ce qu'une grande cause ? \\
Qu'est-ce qu'une guerre juste ? \\
Qu'est-ce qu'une histoire vraie ? \\
Qu'est-ce qu'une hypothèse scientifique ? \\
Qu'est-ce qu'une hypothèse ? \\
Qu'est-ce qu'une idée esthétique ? \\
Qu'est-ce qu'une idée incertaine ? \\
Qu'est-ce qu'une idée morale ? \\
Qu'est-ce qu'une idée vraie ? \\
Qu'est-ce qu'une idée ? \\
Qu'est-ce qu'une idéologie ? \\
Qu'est-ce qu'une illusion ? \\
Qu'est-ce qu'une image ? \\
Qu'est-ce qu'une inégalité ? \\
Qu'est-ce qu'une injustice ? \\
Qu'est-ce qu'une institution ? \\
Qu'est-ce qu'une interprétation ? \\
Qu'est-ce qu'une invention technique ? \\
Qu'est-ce qu'une langue artificielle ? \\
Qu'est-ce qu'une langue bien faite ? \\
Qu'est-ce qu'une langue morte ? \\
Qu'est-ce qu'une langue ? \\
Qu'est-ce qu'un élément ? \\
Qu'est-ce qu'une libération ? \\
Qu'est-ce qu'une libre interprétation ? \\
Qu'est-ce qu'une limite ? \\
Qu'est-ce qu'une logique sociale ? \\
Qu'est-ce qu'une loi de la nature ? \\
Qu'est ce qu'une loi de la pensée ? \\
Qu'est ce qu'une loi scientifique ? \\
Qu'est-ce qu'une loi scientifique ? \\
Qu'est-ce qu'une loi ? \\
Qu'est-ce qu'une machine ? \\
Qu'est-ce qu'une maladie ? \\
Qu'est-ce qu'une marchandise ? \\
Qu'est ce qu'une mauvaise idée ? \\
Qu'est-ce qu'une mauvaise interprétation ? \\
Qu'est-ce qu'une méditation métaphysique ? \\
Qu'est-ce qu'une méditation ? \\
Qu'est-ce qu'une mentalité collective ? \\
Qu'est-ce qu'une métaphore ? \\
Qu'est-ce qu'une méthode ? \\
Qu'est-ce qu'une morale de la communication ? \\
Qu'est-ce qu'un empire ? \\
Qu'est-ce qu'une nation ? \\
Qu'est-ce qu'un enfant ? \\
Qu'est-ce qu'un ennemi ? \\
Qu'est-ce qu'une norme sociale ? \\
Qu'est-ce qu'une norme ? \\
Qu'est-ce qu'une nouveauté ? \\
Qu'est-ce qu'une œuvre d'art authentique ? \\
Qu'est-ce qu'une œuvre d'art réaliste ? \\
Qu'est-ce qu'une œuvre d'art ? \\
Qu'est-ce qu'une œuvre ratée ? \\
Qu'est-ce qu'une œuvre ? \\
Qu'est-ce qu'une œuvre « géniale » ? \\
Qu'est-ce qu'une parole libre ? \\
Qu'est-ce qu'une parole vraie ? \\
Qu'est-ce qu'une passion ? \\
Qu'est-ce qu'une patrie ? \\
Qu'est-ce qu'une pensée libre ? \\
Qu'est-ce qu'une période en histoire ? \\
Qu'est-ce qu'une personne morale ? \\
Qu'est-ce qu'une personne ? \\
Qu'est-ce qu‘une philosophie première ? \\
Qu'est-ce qu'une philosophie ? \\
Qu'est-ce qu'une phrase ? \\
Qu'est-ce qu'une politique sociale ? \\
Qu'est-ce qu'une preuve ? \\
Qu'est-ce qu'une promesse ? \\
Qu'est-ce qu'une propriété essentielle ? \\
Qu'est-ce qu'une propriété ? \\
Qu'est-ce qu'une psychologie scientifique ? \\
Qu'est-ce qu'une question dénuée de sens ? \\
Qu'est-ce qu'une question métaphysique ? \\
Qu'est-ce qu'une question ? \\
Qu'est-ce qu'une raison d'agir ? \\
Qu'est-ce qu'une réfutation ? \\
Qu'est-ce qu'une règle de vie ? \\
Qu'est-ce qu'une règle ? \\
Qu'est-ce qu'une relation ? \\
Qu'est ce qu'une religion ? \\
Qu'est-ce qu'une rencontre ? \\
Qu'est-ce qu'une représentation réussie ? \\
Qu'est-ce qu'une république ? \\
Qu'est-ce qu'une révélation ? \\
Qu'est-ce qu'une révolution politique ? \\
Qu'est-ce qu'une révolution scientifique ? \\
Qu'est-ce qu'une révolution ? \\
Qu'est-ce qu'une science exacte ? \\
Qu'est-ce qu'une science expérimentale ? \\
Qu'est-ce qu'une science humaine ? \\
Qu'est-ce qu'une science rigoureuse ? \\
Qu'est-ce qu'un esclave ? \\
Qu'est-ce qu'une situation tragique ? \\
Qu'est-ce qu'une société juste ? \\
Qu'est-ce qu'une société libre ? \\
Qu'est-ce qu'une société mondialisée ? \\
Qu'est-ce qu'une société ouverte ? \\
Qu'est-ce qu'une solution ? \\
Qu'est-ce qu'un esprit faux ? \\
Qu'est-ce qu'un esprit juste ? \\
Qu'est ce qu'un esprit libre ? \\
Qu'est-ce qu'un esprit libre ? \\
Qu'est-ce qu'un esprit profond ? \\
Qu'est-ce qu'une structure ? \\
Qu'est-ce qu'une substance ? \\
Qu'est-ce qu'un état de droit ? \\
Qu'est-ce qu'un État de droit ? \\
Qu'est-ce qu'un État libre ? \\
Qu'est-ce qu'un état mental ? \\
Qu'est-ce qu'une théorie scientifique ? \\
Qu'est-ce qu'une théorie ? \\
Qu'est-ce qu'une tradition ? \\
Qu'est-ce qu'une tragédie historique ? \\
Qu'est-ce qu'une tragédie ? \\
Qu'est-ce qu'un être cultivé ? \\
Qu'est-ce qu'un être vivant ? \\
Qu'est-ce qu'une valeur ? \\
Qu'est-ce qu'un événement fondateur ? \\
Qu'est-ce qu'un événement historique ? \\
Qu'est-ce qu'un événement ? \\
Qu'est-ce qu'une vérité contingente ? \\
Qu'est-ce qu'une vérité historique ? \\
Qu'est-ce qu'une vérité scientifique ? \\
Qu'est-ce qu'une vérité subjective ? \\
Qu'est-ce qu'une vertu ? \\
Qu'est-ce qu'une vie heureuse ? \\
Qu'est-ce qu'une vie humaine ? \\
Qu'est-ce qu'une vie réussie ? \\
Qu'est-ce qu'une ville ? \\
Qu'est-ce qu'une violence symbolique ? \\
Qu'est-ce qu'une vision du monde ? \\
Qu'est-ce qu'une vision scientifique du monde ? \\
Qu'est-ce qu'une volonté libre ? \\
Qu'est-ce qu'une volonté raisonnable ? \\
Qu'est-ce qu'un exemple ? \\
Qu'est-ce qu'un expérimentateur ? \\
Qu'est-ce qu'un expert ? \\
Qu'est-ce qu'une « expérience de pensée » ? \\
Qu'est-ce qu'une « performance » ? \\
Qu'est-ce qu'un fait de culture ? \\
Qu'est-ce qu'un fait de société ? \\
Qu'est-ce qu'un fait divers ? \\
Qu'est-ce qu'un fait historique ? \\
Qu'est-ce qu'un fait moral ? \\
Qu'est ce qu'un fait scientifique ? \\
Qu'est-ce qu'un fait scientifique ? \\
Qu'est-ce qu'un fait social ? \\
Qu'est-ce qu'un fait ? \\
Qu'est-ce qu'un faux problème ? \\
Qu'est-ce qu'un faux sentiment ? \\
Qu'est-ce qu'un faux ? \\
Qu'est-ce qu'un film ? \\
Qu'est-ce qu'un génie ? \\
Qu'est-ce qu'un geste artistique ? \\
Qu'est-ce qu'un geste technique ? \\
Qu'est-ce qu'un gouvernement démocratique ? \\
Qu'est-ce qu'un gouvernement juste ? \\
Qu'est-ce qu'un gouvernement républicain ? \\
Qu'est-ce qu'un gouvernement ? \\
Qu'est-ce qu'un grand homme ou une grande femme ? \\
Qu'est-ce qu'un grand homme ? \\
Qu'est-ce qu'un grand philosophe ? \\
Qu'est-ce qu'un héros ? \\
Qu'est-ce qu'un homme bon ? \\
Qu'est-ce qu'un homme d'action ? \\
Qu'est-ce qu'un homme d'État ? \\
Qu'est-ce qu'un homme d'expérience ? \\
Qu'est-ce qu'un homme juste ? \\
Qu'est-ce qu'un homme libre ? \\
Qu'est-ce qu'un homme méchant ? \\
Qu'est-ce qu'un homme normal ? \\
Qu'est-ce qu'un homme politique ? \\
Qu'est-ce qu'un homme sans éducation ? \\
Qu'est-ce qu'un homme seul ? \\
Qu'est-ce qu'un idéaliste ? \\
Qu'est-ce qu'un idéal moral ? \\
Qu'est-ce qu'un idéal ? \\
Qu'est-ce qu'un individu ? \\
Qu'est-ce qu'un intellectuel ? \\
Qu'est-ce qu'un jeu ? \\
Qu'est-ce qu'un jugement analytique ? \\
Qu'est-ce qu'un jugement de goût ? \\
Qu'est-ce qu'un justicier ? \\
Qu'est-ce qu'un laboratoire ? \\
Qu'est-ce qu'un langage technique ? \\
Qu'est-ce qu'un législateur ? \\
Qu'est-ce qu'un lieu commun ? \\
Qu'est-ce qu'un livre ? \\
Qu'est-ce qu'un maître ? \\
Qu'est-ce qu'un marginal ? \\
Qu'est-ce qu'un mécanisme social ? \\
Qu'est-ce qu'un métaphysicien ? \\
Qu'est-ce qu'un mineur ? \\
Qu'est-ce qu'un miracle ? \\
Qu'est-ce qu'un modèle ? \\
Qu'est-ce qu'un moderne ? \\
Qu'est-ce qu'un monde \\
Qu'est-ce qu'un monde ? \\
Qu'est-ce qu'un monstre ? \\
Qu'est-ce qu'un monument ? \\
Qu'est-ce qu'un mouvement politique \\
Qu'est-ce qu'un musée ? \\
Qu'est-ce qu'un mythe ? \\
Qu'est-ce qu'un nombre ? \\
Qu'est-ce qu'un nom propre ? \\
Qu'est-ce qu'un objet d'art ? \\
Qu'est-ce qu'un objet esthétique ? \\
Qu'est-ce qu'un objet mathématique ? \\
Qu'est-ce qu'un objet métaphysique ? \\
Qu'est-ce qu'un objet technique ? \\
Qu'est-ce qu'un objet ? \\
Qu'est-ce qu'un œuvre d'art ? \\
Qu'est-ce qu'un ordre ? \\
Qu'est-ce qu'un organisme ? \\
Qu'est-ce qu'un original ? \\
Qu'est-ce qu'un outil ? \\
Qu'est ce qu'un paradoxe ? \\
Qu'est-ce qu'un paradoxe ? \\
Qu'est-ce qu'un patrimoine ? \\
Qu'est-ce qu'un pauvre ? \\
Qu'est-ce qu'un paysage ? \\
Qu'est-ce qu'un pédant ? \\
Qu'est-ce qu'un peuple \\
Qu'est-ce qu'un peuple libre ? \\
Qu'est-ce qu'un peuple ? \\
Qu'est-ce qu'un phénomène ? \\
Qu'est-ce qu'un philosophe ? \\
Qu'est-ce qu'un plaisir pur ? \\
Qu'est-ce qu'un point de vue ? \\
Qu'est-ce qu'un portrait ? \\
Qu'est-ce qu'un post-moderne ? \\
Qu'est-ce qu'un précurseur ? \\
Qu'est-ce qu'un préjugé ? \\
Qu'est-ce qu'un primitif ? \\
Qu'est-ce qu'un prince juste ? \\
Qu'est-ce qu'un principe ? \\
Qu'est-ce qu'un problème éthique ? \\
Qu'est-ce qu'un problème métaphysique ? \\
Qu'est-ce qu'un problème philosophique ? \\
Qu'est-ce qu'un problème politique ? \\
Qu'est-ce qu'un problème scientifique ? \\
Qu'est-ce qu'un problème technique ? \\
Qu'est-ce qu'un problème ? \\
Qu'est-ce qu'un produit culturel ? \\
Qu'est-ce qu'un programme politique ? \\
Qu'est-ce qu'un programmer ? \\
Qu'est-ce qu'un programme ? \\
Qu'est-ce qu'un progrès scientifique ? \\
Qu'est-ce qu'un progrès technique ? \\
Qu'est-ce qu'un prophète ? \\
Qu'est-ce qu'un public ? \\
Qu'est-ce qu'un rapport de force ? \\
Qu'est-ce qu'un récit véridique ? \\
Qu'est-ce qu'un récit ? \\
Qu'est-ce qu'un réfutation ? \\
Qu'est-ce qu'un régime politique ? \\
Qu'est-ce qu'un réseau ? \\
Qu'est-ce qu'un rhéteur ? \\
Qu'est-ce qu'un rite ? \\
Qu'est-ce qu'un rival ? \\
Qu'est-ce qu'un sage ? \\
Qu'est-ce qu'un savoir-faire ? \\
Qu'est-ce qu'un sceptique ? \\
Qu'est-ce qu'un sentiment moral ? \\
Qu'est-ce qu'un sentiment vrai ? \\
Qu'est-ce qu'un signe ? \\
Qu'est-ce qu'un sophisme ? \\
Qu'est-ce qu'un sophiste ? \\
Qu'est-ce qu'un souvenir ? \\
Qu'est-ce qu'un spécialiste ? \\
Qu'est-ce qu'un spectateur ? \\
Qu'est-ce qu'un style ? \\
Qu'est-ce qu'un symbole ? \\
Qu'est-ce qu'un symptôme ? \\
Qu'est-ce qu'un système philosophique ? \\
Qu'est-ce qu'un système ? \\
Qu'est-ce qu'un tableau \\
Qu'est-ce qu'un tableau ? \\
Qu'est-ce qu'un tabou ? \\
Qu'est-ce qu'un technicien ? \\
Qu'est-ce qu'un témoin ? \\
Qu'est-ce qu'un temple ? \\
Qu'est-ce qu'un texte ? \\
Qu'est-ce qu'un tout ? \\
Qu'est-ce qu'un traître ? \\
Qu'est-ce qu'un travail bien fait ? \\
Qu'est-ce qu'un trouble social ? \\
Qu'est-ce qu'un tyran ? \\
Qu'est-ce qu'un vice ? \\
Qu'est-ce qu'un visage ? \\
Qu'est-ce qu'un vrai changement ? \\
Qu'est-ce qu'un « champ artistique » ? \\
Qu'est-ce qu'un « être dégénéré » ? \\
Question et problème \\
Qu'est qu'une image ? \\
Qu'est qu'un régime politique ? \\
Que suis-je ? \\
Que suppose le mouvement ? \\
Que valent les excuses ? \\
Que valent les idées générales ? \\
Que valent les mots ? \\
Que valent les préjugés ? \\
Que valent les théories ? \\
Que vaut en morale la justification par l'utilité ? \\
Que vaut la décision de la majorité ? \\
Que vaut la définition de l'homme comme animal doué de raison ? \\
Que vaut la distinction entre nature et culture ? \\
Que vaut l'excuse : « C'est plus fort que moi » ? \\
Que vaut l'excuse : « Je ne l'ai pas fait exprès» ? \\
Que vaut l'incertain ? \\
Que vaut une preuve contre un préjugé ? \\
Que veut dire avoir raison ? \\
Que veut dire introduire à la métaphysique ? \\
Que veut dire l'expression « aller au fond des choses » ? \\
Que veut dire « essentiel » ? \\
Que veut dire « je t'aime » ? \\
Que veut dire « réel » ? \\
Que veut dire « respecter la nature » ? \\
Que veut dire : « être cultivé » ? \\
Que veut dire : « je t'aime » ? \\
Que veut dire : « le temps passe » ? \\
Que veut dire : « respecter la nature » ? \\
Que voit-on dans une image ? \\
Que voit-on dans un miroir \\
Que voit-on dans un miroir ? \\
Que voit-on dans un tableau ? \\
Que voyons-nous ? \\
Qu'expriment les mythes ? \\
Qu'exprime une œuvre d'art ? \\
Qui accroît son savoir accroît sa douleur \\
Qui agit ? \\
Qui a le droit de juger ? \\
Qui a une histoire ? \\
Qui a une parole politique ? \\
Qui commande ? \\
Qui connaît le mieux mon corps ? \\
Qui croire ? \\
Qui doit faire les lois ? \\
Qui écrit l'histoire ? \\
Qui est autorisé à me dire « tu dois » ? \\
Qui est citoyen ? \\
Qui est compétent en matière politique ? \\
Qui est digne du bonheur ? \\
Qui est immoral ? \\
Qui est l'autre ? \\
Qui est le maître ? \\
Qui est l'homme des sciences humaines ? \\
Qui est libre ? \\
Qui est métaphysicien ? \\
Qui est mon prochain ? \\
Qui est mon semblable ? \\
Qui est riche ? \\
Qui est sage ? \\
Qui est souverain ? \\
Qui fait la loi ? \\
Qui fait l'histoire ? \\
Qui gouverne ? \\
Qui mérite d'être aimé ? \\
Qui meurt ? \\
Qui nous dicte nos devoirs ? \\
Qui parle quand je dis « je » ? \\
Qui parle ? \\
Qui pense ? \\
Qui peut avoir des droits ? \\
Qui peut me dire « tu ne dois pas » ? \\
Qui peut parler ? \\
Qui suis-je et qui es-tu ? \\
Qui suis-je ? \\
Qui travaille ? \\
Qui veut la fin veut les moyens \\
Qu'oppose-t-on à la vérité ? \\
Qu'y a-t-il à comprendre dans une œuvre d'art ? \\
Qu'y a-t-il à comprendre en histoire ? \\
Qu'y a-t-il à l'origine de toutes choses ? \\
Qu'y a-t-il au-delà du réel ? \\
Qu'y a-t-il au fondement de l'objectivité ? \\
Qu'y a-t-il de sérieux dans le jeu ? \\
Qu'y a-t-il d'universel dans la culture ? \\
Qu'y a-t-il ? \\
Raconter sa vie \\
Raconter son histoire \\
Raison et dialogue \\
Raison et folie \\
Raison et fondement \\
Raison et langage \\
Raison et politique \\
Raison et révélation \\
Raison et tradition \\
Raisonnable et rationnel \\
Raisonnement et expérimentation \\
Raisonner \\
Raisonner et calculer \\
Raisonner par l'absurde \\
Rapports de force, rapport de pouvoir \\
Rassembler les hommes, est-ce les unir ? \\
Rationnel et raisonnable \\
Réalisme et idéalisme \\
Réalité et apparence \\
Réalité et idéal \\
Réalité et perception \\
Réalité et représentation \\
Rebuts et objets quelconques : une matière pour l'art ? \\
Recevoir \\
Récit et histoire \\
Récit et mémoire \\
Reconnaissance et inégalité \\
Reconnaissons-nous le bien comme nous reconnaissons le vrai ? \\
Recourir au langage, est-ce renoncer à la violence ? \\
Refaire sa vie \\
Réforme et révolution \\
Refuser et réfuter \\
Réfutation et confirmation \\
Réfuter \\
Réfuter une théorie \\
Regarder \\
Regarder un tableau \\
Règle et commandement \\
Règle morale et norme juridique \\
Règles sociales et loi morale \\
Regrets et remords \\
Religion et démocratie \\
Religion et liberté \\
Religion et moralité \\
Religion et politique \\
Religion et violence \\
Religion naturelle et religion révélée \\
Religions et démocratie \\
Rendre justice \\
Rendre la justice \\
Rendre raison \\
Rendre visible l'invisible \\
Renoncer au passé \\
Rentrer en soi-même \\
Répondre \\
Répondre de soi \\
Représentation et illusion \\
Représenter \\
Reproduire, copier, imiter \\
Réprouver \\
République et démocratie \\
Résistance et obéissance \\
Résistance et soumission \\
Résister \\
Résister à l'oppression \\
Résister peut-il être un droit ? \\
Respecter la nature, est-ce renoncer à l'exploiter ? \\
Respect et tolérance \\
Rester soi-même \\
Réussir sa vie \\
Revenir à la nature \\
Rêver \\
Revient-il à l'État d'assurer le bonheur des citoyens ? \\
Revient-il à l'État d'assurer votre bonheur ? \\
Révolte et révolution \\
Rêvons-nous ? \\
Rhétorique et vérité \\
Richesse et pauvreté \\
Rien \\
Rien de nouveau sous le soleil \\
Rien n'est sans raison \\
Rire \\
Rire et pleurer \\
Rites et cérémonies \\
Rituels et cérémonies \\
Roman et vérité \\
Rythmes sociaux, rythmes naturels \\
Sait-on ce que l'on veut ? \\
Sait-on ce qu'on fait ? \\
Sait-on ce qu'on veut ? \\
Sait-on nécessairement ce que l'on désire ? \\
Sait-on toujours ce que l'on fait ? \\
Sait-on toujours ce que l'on veut ? \\
Sait-on toujours ce qu'on veut ? \\
S'aliéner \\
S'amuser \\
Sans foi ni loi \\
S'approprier une œuvre d'art \\
Sauver les apparences \\
Sauver les phénomènes \\
Savoir ce qu'on dit \\
Savoir démontrer \\
Savoir de quoi on parle \\
Savoir est-ce cesser de croire ? \\
Savoir, est-ce pouvoir ? \\
Savoir est-ce se libérer ? \\
Savoir et croire \\
Savoir et démontrer \\
Savoir et liberté \\
Savoir et objectivité dans les sciences \\
Savoir et pouvoir \\
Savoir et rectification \\
Savoir être heureux \\
Savoir et savoir faire \\
Savoir et savoir-faire \\
Savoir et vérifier \\
Savoir faire \\
Savoir pour prévoir \\
Savoir, pouvoir \\
Savoir renoncer \\
Savoir s'arrêter \\
Savoir se décider \\
Savoir tout \\
Savoir vivre \\
Savons-nous ce que nous disons ? \\
Science du vivant et finalisme \\
Science du vivant, science de l'inerte \\
Science et abstraction \\
Science et certitude \\
Science et complexité \\
Science et croyance \\
Science et démocratie \\
Science et domination sociale \\
Science et expérience \\
Science et histoire \\
Science et hypothèse \\
Science et idéologie \\
Science et imagination \\
Science et invention \\
Science et libération \\
Science et magie \\
Science et métaphysique \\
Science et méthode \\
Science et mythe \\
Science et objectivité \\
Science et opinion \\
Science et persuasion \\
Science et philosophie \\
Science et réalité \\
Science et religion \\
Science et sagesse \\
Science et société \\
Science et technique \\
Science et technologie \\
Science pure et science appliquée \\
Sciences de la nature et sciences de l'esprit \\
Sciences de la nature et sciences humaines \\
Sciences empiriques et critères du vrai \\
Sciences et philosophie \\
Sciences humaines et déterminisme \\
Sciences humaines et herméneutique \\
Sciences humaines et idéologie \\
Sciences humaines et liberté sont-elles compatibles ? \\
Sciences humaines et littérature \\
Sciences humaines et naturalisme \\
Sciences humaines et nature humaine \\
Sciences humaines et objectivité \\
Sciences humaines et philosophie \\
Sciences humaines, sciences de l'homme \\
Sciences sociales et humanités \\
Se connaître soi-même \\
Se conserver \\
Se convertir \\
Se cultiver \\
Se cultiver, est-ce s'affranchir de son appartenance culturelle ? \\
Sécurité et liberté \\
Se décider \\
Se défendre \\
Se détacher des sens \\
Se faire comprendre \\
Se faire justice \\
Se mentir à soi-même \\
Se mentir à soi-même : est-ce possible ? \\
Se mettre à la place d'autrui \\
S'engager \\
S'ennuyer \\
Se nourrir \\
Sensation et perception \\
Sens et existence \\
Sens et fait \\
Sens et limites de la notion de capital culturel \\
Sens et sensibilité \\
Sens et sensible \\
Sens et signification \\
Sens et structure \\
Sens et vérité \\
Sensible et intelligible \\
Sens propre et sens figuré \\
Sentir \\
Sentir et juger \\
Sentir et penser \\
Se parler et s'entendre \\
Se passer de philosophie \\
Se prendre au sérieux \\
Se raconter des histoires \\
Serait-il immoral d'autoriser le commerce des organes humains ? \\
Se retirer dans la pensée ? \\
Se retirer du monde \\
Se révolter \\
Serions-nous heureux dans un ordre politique parfait ? \\
Serions-nous plus libres sans État ? \\
Servir \\
Servir, est-ce nécessairement renoncer à sa liberté ? \\
Servir l'État \\
Se savoir mortel \\
Se suffire à soi-même \\
Se taire \\
Seul \\
Seul le présent existe-t-il ? \\
Se voiler la face \\
Sexe et genre \\
S'exercer \\
S'exprimer \\
Sexualité et féminité \\
Sexualité et nature \\
Si\ldots{} alors \\
Si Dieu n'existe pas, tout est-il permis ? \\
Si Dieu n'existe pas, tout est-il possible ? \\
Signe et symbole \\
Signes, traces et indices \\
Signification et expression \\
Signification et vérité \\
Si l'esprit n'est pas une table rase, qu'est-il ? \\
Si l'État n'existait pas, faudrait-il l'inventer ? \\
Sincérité et vérité \\
S'indigner \\
S'indigner, est-ce un devoir ? \\
Si nous étions moraux, le droit serait-il inutile ? \\
S'intéresser à l'art \\
Si tout est historique, tout est-il relatif ? \\
Si tu veux, tu peux \\
Société et biologie \\
Société et communauté \\
Société et organisme \\
Société et religion \\
Société humaines, sociétés animales \\
Socrate \\
Soi \\
Soigner \\
Solitude et isolement \\
Sommes-nous capables d'agir de manière désintéressée ? \\
Sommes-nous conscients de nos mobiles ? \\
Sommes-nous dans le temps comme dans l'espace ? \\
Sommes-nous des êtres métaphysiques ? \\
Sommes-nous des sujets ? \\
Sommes-nous déterminés par notre culture ? \\
Sommes-nous dominés par la technique ? \\
Sommes-nous faits pour la vérité ? \\
Sommes-nous faits pour le bonheur ? \\
Sommes-nous gouvernés par nos passions ? \\
Sommes-nous jamais certains d'avoir choisi librement ? \\
Sommes-nous les jouets de l'histoire ? \\
Sommes-nous les jouets de nos pulsions ? \\
Sommes-nous libres de nos croyances ? \\
Sommes-nous libres de nos pensées ? \\
Sommes-nous libres de nos préférences morales ? \\
Sommes-nous libres face à l'évidence ? \\
Sommes-nous maîtres de nos désirs ? \\
Sommes-nous maîtres de nos paroles ? \\
Sommes-nous perfectibles ? \\
Sommes-nous portés au bien ? \\
Sommes-nous prisonniers de nos désirs ? \\
Sommes-nous prisonniers de notre histoire ? \\
Sommes-nous prisonniers du temps ? \\
Sommes-nous responsables de ce dont nous n'avons pas conscience ? \\
Sommes-nous responsables de ce que nous sommes ? \\
Sommes-nous responsables de nos désirs ? \\
Sommes-nous responsables de nos erreurs ? \\
Sommes-nous responsables de nos opinions ? \\
Sommes-nous responsables de nos passions ? \\
Sommes-nous soumis au temps ? \\
Sommes-nous sujets de nos désirs ? \\
Sommes-nous toujours conscients des causes de nos désirs ?` \\
Sommes-nous toujours dépendants d'autrui ? \\
Sommes-nous tous contemporains ? \\
Sophismes et paradoxes \\
S'orienter \\
Sortir de soi \\
Soumission et servitude \\
Soutenir une thèse \\
Soyez naturel ! \\
Sport et politique \\
Structure et événement \\
Subir \\
Substance et accident \\
Substance et sujet \\
Suffit-il d'avoir raison ? \\
Suffit-il de bien juger pour bien faire ? \\
Suffit-il de faire son devoir pour être vertueux ? \\
Suffit-il de faire son devoir ? \\
Suffit-il de n'avoir rien fait pour être innocent ? \\
Suffit-il d'être informé pour comprendre ? \\
Suffit-il d'être juste ? \\
Suffit-il d'être vertueux pour être heureux ? \\
Suffit-il de voir le meilleur pour le suivre ? \\
Suffit-il de vouloir pour bien faire ? \\
Suffit-il, pour croire, de le vouloir ? \\
Suffit-il pour être juste d'obéir aux lois et aux coutumes de son pays ? \\
Suffit-il que nos intentions soient bonnes pour que nos actions le soient aussi ? \\
Suis-ce que j'ai conscience d'être ? \\
Suis-je aussi ce que j'aurais pu être ? \\
Suis-je ce que j'ai conscience d'être ? \\
Suis-je ce que je fais ? \\
Suis-je dans le temps comme je suis dans l'espace ? \\
Suis-je étranger à moi-même ? \\
Suis-je l'auteur de ce que je dis ? \\
Suis-je le même en des temps différents ? \\
Suis-je le mieux placé pour me connaître ? \\
Suis-je libre ? \\
Suis-je maître de ma conscience ? \\
Suis-je maître de mes pensées ? \\
Suis-je ma mémoire ? \\
Suis-je mon corps ? \\
Suis-je mon passé ? \\
Suis-je propriétaire de mon corps ? \\
Suis-je responsable de ce dont je n'ai pas conscience ? \\
Suis-je responsable de ce que je suis ? \\
Suis-je seul au monde ? \\
Suis-je toujours autre que moi-même ? \\
Suivre la coutume \\
Suivre son intuition \\
Suivre une règle \\
Sujet et citoyen \\
Sujet et prédicat \\
Sujet et substance \\
Superstition et religion \\
Surface et profondeur \\
Sur quoi fonder la justice ? \\
Sur quoi fonder la légitimité de la loi ? \\
Sur quoi fonder la propriété ? \\
Sur quoi fonder la société ? \\
Sur quoi fonder l'autorité politique ? \\
Sur quoi fonder l'autorité ? \\
Sur quoi fonder le devoir ? \\
Sur quoi fonder le droit de punir ? \\
Sur quoi le langage doit-il se régler ? \\
Sur quoi l'historien travaille-t-il ? \\
Sur quoi repose l'accord des esprits ? \\
Sur quoi repose la croyance au progrès ? \\
Sur quoi reposent nos certitudes ? \\
Sur quoi se fonde la connaissance scientifique ? \\
Surveillance et discipline \\
Surveiller son comportement \\
Survivre \\
Suspendre son assentiment \\
Suspendre son jugement \\
Syllogisme et démonstration \\
Sympathie et respect \\
Système et structure \\
Talent et génie \\
Tantôt je pense, tantôt je suis \\
Tautologie et contradiction \\
Technique et apprentissage \\
Technique et idéologie \\
Technique et intérêt \\
Technique et nature \\
Technique et pratiques scientifiques \\
Technique et progrès \\
Technique et responsabilité \\
Technique et savoir-faire \\
Technique et violence \\
Témoigner \\
Temps et commencement \\
Temps et conscience \\
Temps et création \\
Temps et éternité \\
Temps et histoire \\
Temps et irréversibilité \\
Temps et liberté \\
Temps et mémoire \\
Temps et réalité \\
Temps et vérité \\
Tenir parole \\
Tenir pour vrai \\
Tenir sa parole \\
Thème et variations \\
Théorie et expérience \\
Théorie et modèle \\
Théorie et modélisation \\
Théorie et pratique \\
Toucher \\
Toucher, sentir, goûter \\
Toujours plus vite ? \\
Tous les conflits peuvent-ils être résolus par le dialogue ? \\
Tous les désirs sont-ils naturels ? \\
Tous les droits sont-ils formels ? \\
Tous les hommes désirent-ils connaître ? \\
Tous les hommes désirent-ils être heureux ? \\
Tous les hommes désirent-ils naturellement être heureux ? \\
Tous les hommes désirent-ils naturellement savoir ? \\
Tous les hommes sont-ils égaux ? \\
Tous les paradis sont-ils perdus ? \\
Tous les plaisirs se valent-ils ? \\
Tous les rapports humains sont-ils des échanges ? \\
Tout art est-il poésie ? \\
Tout a-t-il une cause ? \\
Tout a-t-il une raison d'être ? \\
Tout a-t-il un prix ? \\
Tout a-t-il un sens ? \\
Tout ce qui est excessif est-il insignifiant ? \\
Tout ce qui est naturel est-il normal ? \\
Tout ce qui est rationnel est-il raisonnable ? \\
Tout ce qui est vrai doit-il être prouvé ? \\
Tout ce qui existe a-t-il un prix ? \\
Tout comprendre, est-ce tout pardonner ? \\
Tout définir, tout démontrer \\
Tout démontrer \\
Tout désir est-il désir de posséder ? \\
Tout désir est-il égoïste ? \\
Tout désir est-il manque ? \\
Tout désir est-il une souffrance ? \\
Tout devoir est-il l'envers d'un droit ? \\
Tout dire \\
Tout droit est-il un pouvoir ? \\
Toute action politique est-elle collective ? \\
Toute chose a-t-elle une essence ? \\
Toute communauté est-elle politique ? \\
Toute compréhension implique-t-elle une interprétation ? \\
Toute connaissance autre que scientifique doit-elle être considérée comme une illusion ? \\
Toute connaissance commence-t-elle avec l'expérience ? \\
Toute connaissance consiste-t-elle en un savoir-faire ? \\
Toute connaissance est-elle historique ? \\
Toute connaissance est-elle hypothétique ? \\
Toute connaissance s'enracine-t-elle dans la perception ? \\
Toute conscience est-elle conscience de soi ? \\
Toute conscience est-elle subjective ? \\
Toute conscience n'est-elle pas implicitement morale ? \\
Toute description est-elle une interprétation ? \\
Toute existence est-elle indémontrable ? \\
Toute expérience appelle-t-elle une interprétation ? \\
Toute expression est-elle métaphorique ? \\
Toute faute est-elle une erreur ? \\
Toute hiérarchie est-elle inégalitaire ? \\
Toute inégalité est-elle injuste ? \\
Toute interprétation est-elle contestable ? \\
Toute interprétation est-elle subjective ? \\
Toute métaphysique implique-t-elle une transcendance ? \\
Toute morale implique-t-elle l'effort ? \\
Toute morale s'oppose-t-elle aux désirs ? \\
Tout énoncé est-il nécessairement vrai ou faux ? \\
Toute notre connaissance dérive-t-elle de l'expérience ? \\
Toute origine est-elle mythique ? \\
Toute passion fait-elle souffrir ? \\
Toute pensée revêt-elle nécessairement une forme linguistique ? \\
Toute peur est-elle irrationnelle ? \\
Toute philosophie constitue-t-elle une doctrine ? \\
Toute philosophie est-elle systématique ? \\
Toute philosophie implique-t-elle une politique ? \\
Toute relation humaine est-elle un échange ? \\
Toute science est-elle naturelle ? \\
Toutes les choses sont-elles singulières ? \\
Toutes les convictions sont-elles respectables ? \\
Toutes les croyances se valent-elles ? \\
Toutes les fautes se valent-elles ? \\
Toutes les inégalités ont-elles une importance politique ? \\
Toutes les inégalités sont-elles des injustices ? \\
Toutes les interprétations se valent-elles ? \\
Toutes les opinions se valent-elles ? \\
Toutes les opinions sont-elles bonnes à dire ? \\
Toutes les vérités  scientifiques sont-elles révisables ? \\
Toute société a-t-elle besoin d'une religion ? \\
Tout est corps \\
Tout est-il affaire de point de vue ? \\
Tout est-il à vendre ? \\
Tout est-il connaissable ? \\
Tout est-il faux dans la fiction ? \\
Tout est-il historique ? \\
Tout est-il matière ? \\
Tout est-il mesurable ? \\
Tout est-il nécessaire ? \\
Tout est-il politique ? \\
Tout est-il quantifiable ? \\
Tout est-il relatif ? \\
Tout est permis \\
Tout est relatif \\
Tout est vanité \\
Tout être est-il dans l'espace ? \\
Toute vérité est-elle bonne à dire ? \\
Toute vérité est-elle démontrable ? \\
Toute vérité est-elle vérifiable ? \\
Toute vie est-elle intrinsèquement respectable ? \\
Toute violence est-elle contre nature ? \\
Tout futur est-il contingent ? \\
Tout ordre est-il une violence déguisée ? \\
Tout ou rien \\
Tout peut-il être objet d'échange ? \\
Tout peut-il être objet de jugement esthétique ? \\
Tout peut-il être objet de science ? \\
Tout peut-il n'être qu'apparence ? \\
Tout peut-il s'acheter ? \\
Tout peut-il se démontrer ? \\
Tout peut-il se vendre ? \\
Tout peut-il s'expliquer ? \\
Tout pouvoir corrompt-il ? \\
Tout pouvoir est-il oppresseur ? \\
Tout pouvoir est-il politique ? \\
Tout pouvoir n'est-il pas abusif ? \\
Tout principe est-il un fondement ? \\
Tout savoir \\
Tout savoir a-t-il une justification ? \\
Tout savoir est-il fondé sur un savoir premier ? \\
Tout savoir est-il pouvoir ? \\
Tout savoir est-il transmissible ? \\
Tout savoir est-il un pouvoir ? \\
Tout savoir peut-il se transmettre ? \\
Tout s'en va-t-il avec le temps ? \\
Tout se prête-il à la mesure ? \\
Tout travail est-il forcé ? \\
Tout travail est-il social ? \\
Tout travail mérite salaire \\
Tout vouloir \\
Tradition et innovation \\
Tradition et liberté \\
Tradition et nouveauté \\
Tradition et raison \\
Tradition et transmission \\
Tradition et vérité \\
Traduction, Trahison \\
Traduire \\
Traduire, est-ce trahir ? \\
Traduire et interpréter \\
Tragédie et comédie \\
Trahir \\
Traiter autrui comme une chose \\
Traiter des faits humains comme des choses, est-ce considérer l'homme comme une chose ? \\
Traiter les faits humains comme des choses, est-ce réduire les hommes à des choses ? \\
Transcendance et altérité \\
Transcendance et immanence \\
Transmettre \\
Travail, besoin, désir \\
Travail et aliénation \\
Travail et besoin \\
Travail et bonheur \\
Travail et capital \\
Travail et liberté \\
Travail et loisir \\
Travail et nécessité \\
Travail et œuvre \\
Travail et subjectivité \\
Travailler, est-ce faire œuvre ? \\
Travailler et œuvrer \\
Travailler par plaisir, est-ce encore travailler ? \\
Travaille-t-on pour soi-même ? \\
Travail manuel et travail intellectuel \\
Travail manuel, travail intellectuel \\
Tricher \\
Trouver sa voie \\
Tu aimeras ton prochain comme toi-même \\
Tuer et laisser mourir \\
Tuer le temps \\
Un acte désintéressé est-il possible ? \\
Un acte gratuit est-il possible ? \\
Un acte libre est-il un acte imprévisible ? \\
Un acte peut-il être inhumain ? \\
Un artiste doit-il être original ? \\
Un art peut-il être populaire ? \\
Un art sans sublimation est-il possible ? \\
Un bien peut-il être commun ? \\
Un bien peut-il sortir d'un mal ? \\
Un choix peut-il être rationnel ? \\
Un contrat peut-il être injuste ? \\
Un contrat peut-il être social ? \\
Un désir peut-il être coupable ? \\
Un désir peut-il être inconscient ? \\
Un Dieu unique ? \\
Une action peut-elle être désintéressée ? \\
Une action peut-elle être machinale ? \\
Une action vertueuse se reconnaît-elle à sa difficulté ? \\
Une activité inutile est-elle sans valeur ? \\
Une cause peut-elle être libre ? \\
Une communauté politique n'est-elle qu'une communauté d'intérêt ? \\
Une connaissance peut-elle ne pas être relative ? \\
Une connaissance scientifique du vivant est-elle possible ? \\
Une croyance infondée est-elle illégitime ? \\
Une croyance peut-elle être libre ? \\
Une croyance peut-elle être rationnelle ? \\
Une culture de masse est-elle une culture ? \\
Une culture peut-elle être porteuse de valeurs universelles ? \\
Une décision politique peut-elle être juste ? \\
Une destruction peut-elle être créatrice ? \\
Une éducation esthétique est-elle possible ? \\
Une éducation morale est-elle possible ? \\
Une éthique sceptique est-elle possible ? \\
Une existence se démontre-t-elle ? \\
Une expérience peut-elle être fictive ? \\
Une explication peut-elle être réductrice ? \\
Une fiction peut-elle être vraie ? \\
Une foi rationnelle \\
Une guerre peut-elle être juste ? \\
Une idée peut-elle être fausse ? \\
Une idée peut-elle être générale ? \\
Une imitation peut-elle être parfaite ? \\
Une intention peut-elle être coupable ? \\
Une interprétation est-elle nécessairement subjective ? \\
Une interprétation peut-elle échapper à l'arbitraire ? \\
Une interprétation peut-elle être définitive ? \\
Une interprétation peut-elle être objective ? \\
Une interprétation peut-elle prétendre à la vérité ? \\
Une langue n'est-elle faite que de mots ? \\
Une ligne de conduite peut-elle tenir lieu de morale ? \\
Une logique non-formelle est-elle possible ? \\
Une loi n'est-elle qu'une règle ? \\
Une loi peut-elle être injuste ? \\
Une machine peut-elle avoir une mémoire ? \\
Une machine peut-elle penser ? \\
Une machine pourrait-elle penser ? \\
Une métaphysique athée est-elle possible ? \\
Une métaphysique peut-elle être sceptique ? \\
Une morale du plaisir est-elle concevable ? \\
Une morale peut-elle être dépassée ? \\
Une morale peut-elle être provisoire ? \\
Une morale peut-elle prétendre à l'universalité ? \\
Une morale sans Dieu \\
Une morale sans obligation est-elle possible ? \\
Une morale sceptique est-elle possible ? \\
Une œuvre d'art a-t-elle toujours un sens ? \\
Une œuvre d'art doit-elle avoir un sens ? \\
Une œuvre d'art doit-elle nécessairement être belle ? \\
Une œuvre d'art doit-elle plaire ? \\
Une œuvre d'art est-elle une marchandise ? \\
Une œuvre d'art peut-elle être immorale ? \\
Une œuvre d'art peut-elle être laide ? \\
Une œuvre d'art s'explique-t-elle à partir de ses influences ? \\
Une œuvre doit-elle nécessairement être belle ? \\
Une œuvre est-elle nécessairement singulière ? \\
Une œuvre est-elle toujours de son temps ? \\
Une pensée contradictoire est-elle dénuée de valeur ? \\
Une perception peut-elle être illusoire ? \\
Une philosophie de l'amour est-elle possible ? \\
Une philosophie peut-elle être réactionnaire ? \\
Une politique peut-elle se réclamer de la vie ? \\
Une psychologie peut-elle être matérialiste ? \\
Une religion civile est-elle possible ? \\
Une religion peut-elle être fausse ? \\
Une religion peut-elle être rationnelle ? \\
Une religion peut-elle se passer de pratiques ? \\
Une religion rationnelle est-elle possible ? \\
Une science de la conscience est-elle possible ? \\
Une science de la culture est-elle possible ? \\
Une science de la morale est-elle possible ? \\
Une science de l'éducation est-elle possible ? \\
Une science de l'esprit est-elle possible ? \\
Une science des symboles est-elle possible ? \\
Une sensation peut-elle être fausse ? \\
Une société d'athées est-elle possible ? \\
Une société juste est-ce une société sans conflit ? \\
Une société juste est-elle une société sans conflits ? \\
Une société n'est-elle qu'un ensemble d'individus ? \\
Une société sans conflit est-elle possible ? \\
Une société sans État est-elle possible ? \\
Une société sans État est-elle une société sans politique ? \\
Une société sans religion est-elle possible ? \\
Une société sans travail est-elle souhaitable ? \\
Un État mondial ? \\
Un État peut-il être trop étendu ? \\
Une théorie peut-elle être vérifiée ? \\
Une théorie scientifique peut-elle devenir fausse ? \\
Une théorie scientifique peut-elle être ramenée à des propositions empiriques élémentaires ? \\
Une théorie scientifique peut-elle être vraie ? \\
Un être vivant peut-il être comparé à une œuvre d'art ? \\
Un événement historique est-il toujours imprévisible ? \\
Une vérité peut-elle être indicible ? \\
Une vérité peut-elle être provisoire ? \\
Une vie heureuse est-elle une vie de plaisirs ? \\
Une vie libre exclut-elle le travail ? \\
Une volonté peut-elle être générale ? \\
Un fait existe-t-il sans interprétation ? \\
Un gouvernement de savants est-il souhaitable ? \\
Un homme n'est-il que la somme de ses actes ? \\
Universalité et nécessité dans les sciences \\
Univocité et équivocité \\
Un jeu peut-il être sérieux ? \\
Un jugement de goût est-il culturel ? \\
Un langage universel est-il concevable ? \\
Un mensonge peut-il avoir une valeur morale ? \\
Un moment d'éternité \\
Un monde meilleur \\
Un monde sans beauté \\
Un monde sans nature est-il pensable ? \\
Un objet technique peut-il être beau ? \\
Un peuple est-il responsable de son histoire ? \\
Un peuple est-il un rassemblement d'individus ? \\
Un peuple se définit-il par son histoire ? \\
Un philosophe a-t-il des devoirs envers la société ? \\
Un pouvoir a-t-il besoin d'être légitime ? \\
Un problème moral peut-il recevoir une solution certaine ? \\
Un problème scientifique peut-il être insoluble ? \\
Un savoir peut-il être inconscient ? \\
Un sceptique peut-il être logicien ? \\
Un seul peut-il avoir raison contre tous ? \\
Un tableau peut-il être une dénonciation ? \\
Un vice, est-ce un manque ? \\
User de violence peut-il être moral ? \\
Utilité et beauté \\
Utopie et tradition \\
Vaincre la mort \\
Valeur artistique, valeur esthétique \\
Vanité des vanités \\
Vaut-il mieux oublier ou pardonner ? \\
Vaut-il mieux subir l'injustice que la commettre ? \\
Vaut-il mieux subir ou commettre l'injustice ? \\
Vendre son corps \\
Vérité et apparence \\
Vérité et certitude \\
Vérité et cohérence \\
Vérité et efficacité \\
Vérité et exactitude \\
Vérité et fiction \\
Vérité et histoire \\
Vérité et illusion \\
Vérité et liberté \\
Vérité et poésie \\
Vérité et réalité \\
Vérité et religion \\
Vérité et sensibilité \\
Vérité et signification \\
Vérité et sincérité \\
Vérité et subjectivité \\
Vérité et vérification \\
Vérité et vraisemblance \\
Vérités de fait et vérités de raison \\
Vérités mathématiques, vérités philosophiques \\
Vérité théorique, vérité pratique \\
Vertu et habitude \\
Vertu et perfection \\
Vice et délice \\
Vices privés, vertus publiques \\
Vie active, vie contemplative \\
Vie et existence \\
Vie et volonté \\
Vieillir \\
Vie politique et vie contemplative \\
Vie privée et vie publique \\
Vie publique et vie privée \\
Violence et discours \\
Violence et force \\
Violence et histoire \\
Violence et politique \\
Violence et pouvoir \\
Vitalisme et mécanique \\
Vit-on au présent ? \\
Vivons-nous tous dans le même monde ? \\
Vivrait-on mieux sans désirs ? \\
Vivre \\
Vivre au présent \\
Vivre caché \\
Vivre comme si nous ne devions pas mourir \\
Vivre en société, est-ce seulement vivre ensemble ? \\
Vivre, est-ce interpréter ? \\
Vivre, est-ce lutter contre la mort ? \\
Vivre, est-ce lutter pour survivre ? \\
Vivre, est-ce résister à la mort ? \\
Vivre, est-ce un droit ? \\
Vivre et bien vivre \\
Vivre et exister \\
Vivre libre \\
Vivre sans morale \\
Vivre sans religion, est-ce vivre sans espoir ? \\
Vivre sa vie \\
Vivre selon la nature \\
Vivre sous la conduite de la raison \\
Vivre vertueusement \\
Voir \\
Voir et entendre \\
Voir et savoir \\
Voir et toucher \\
Voir le meilleur et faire le pire \\
Voir le meilleur, faire le pire \\
Voir, observer, penser \\
Voir un tableau \\
Voit-on ce qu'on croit ? \\
Vouloir ce que l'on peut \\
Vouloir dire \\
Vouloir, est-ce encore désirer ? \\
Vouloir et pouvoir \\
Vouloir être immortel \\
Vouloir la paix sociale peut-il aller jusqu'à accepter l'injustice ? \\
Vouloir la solitude \\
Vouloir le bien \\
Vouloir l'égalité \\
Vouloir le mal \\
Vouloir l'impossible \\
Vouloir oublier \\
Voyager \\
Vulgariser la science ? \\
Y a-t-il continuité entre l'expérience et la science ? \\
Y a-t-il continuité ou discontinuité entre la pensée mythique et la science ? \\
Y a-t-il d'autres moyens que la démonstration pour établir la vérité ? \\
Y a-t-il de bons et de mauvais désirs ? \\
Y a-t-il de bons préjugés ? \\
Y a-t-il de fausses religions ? \\
Y a-t-il de faux besoins ? \\
Y a-t-il de faux problèmes ? \\
Y a-t-il de justes inégalités ? \\
Y a-t-il de la fatalité dans la vie de l'homme ? \\
Y a-t-il de la raison dans la perception ? \\
Y a-t-il de l'impensable ? \\
Y a-t-il de l'incommunicable ? \\
Y a-t-il de l'inconcevable ? \\
Y a-t-il de l'inconnaissable ? \\
Y a-t-il de l'indémontrable ? \\
Y a-t-il de l'indésirable ? \\
Y a-t-il de l'indicible ? \\
Y a-t-il de l'inexprimable ? \\
Y a-t-il de l'irréductible ? \\
Y a-t-il de l'irréfutable ? \\
Y a-t-il de l'irréparable ? \\
Y a-t-il de l'universel ? \\
Y a-t-il de mauvais désirs ? \\
Y a-t-il des acquis définitifs en science ? \\
Y a-t-il des actes de pensée ? \\
Y a-t-il des actes désintéressés ? \\
Y a-t-il des actes gratuits ? \\
Y a-t-il des actes moralement indifférents ? \\
Y a-t-il des actions désintéressées ? \\
Y a-t-il des arts mineurs ? \\
Y a-t-il des barbares ? \\
Y a-t-il des biens inestimables ? \\
Y a-t-il des canons de la beauté ? \\
Y a-t-il des certitudes historiques ? \\
Y a-t-il des choses qui échappent au droit ? \\
Y a-t-il des choses qu'on n'échange pas ? \\
Y a-t-il des compétences politiques ? \\
Y a-t-il des connaissances dangereuses ? \\
Y a-t-il des connaissances désintéressées ? \\
Y a-t-il des contraintes légitimes ? \\
Y a-t-il des convictions philosophiques ? \\
Y a-t-il des correspondances entre les arts ? \\
Y a-t-il des critères de l'humanité ? \\
Y a-t-il des critères du beau ? \\
Y a-t-il des critères du goût ? \\
Y a-t-il des croyances démocratiques ? \\
Y a-t-il des croyances nécessaires ? \\
Y a-t-il des croyances rationnelles ? \\
Y a-t-il des degrés dans la certitude ? \\
Y a-t-il des degrés de conscience ? \\
Y a-t-il des degrés de réalité ? \\
Y a-t-il des degrés de vérité ? \\
Y a-t-il des démonstrations en philosophie ? \\
Y a-t-il des despotes éclairés ? \\
Y a-t-il des déterminismes sociaux ? \\
Y a-t-il des devoirs envers soi-même ? \\
Y a-t-il des devoirs envers soi ? \\
Y a-t-il des dilemmes moraux ? \\
Y a-t-il des droits sans devoirs ? \\
Y a-t-il des erreurs de la nature ? \\
Y a-t-il des erreurs en politique ? \\
Y a-t-il des êtres mathématiques ? \\
Y a-t-il des évidences morales ? \\
Y a-t-il des expériences absolument certaines ? \\
Y a-t-il des expériences cruciales ? \\
Y a-t-il des expériences de la liberté ? \\
Y a-t-il des expériences métaphysiques ? \\
Y a-t-il des expériences sans théorie ? \\
Y a-t-il des facultés dans l'esprit ? \\
Y a-t-il des faits moraux ? \\
Y a-t-il des faits sans essence ? \\
Y a-t-il des faits scientifiques ? \\
Y a-t-il des faux problèmes ? \\
Y a-t-il des fins dans la nature ? \\
Y a-t-il des fins de la nature ? \\
Y a-t-il des fins dernières ? \\
Y a-t-il des fondements naturels à l'ordre social ? \\
Y a-t-il des genres de plaisir ? \\
Y a-t-il des genres du plaisir ? \\
Y a-t-il des guerres justes ? \\
Y a-t-il des héritages philosophiques ? \\
Y a-t-il des illusions de la conscience ? \\
Y a-t-il des illusions nécessaires ? \\
Y a-t-il des inégalités justes ? \\
Y a-t-il des injustices naturelles ? \\
Y a-t-il des instincts propres à l'Homme ? \\
Y a-t-il des interprétations fausses ? \\
Y a-t-il des intuitions morales ? \\
Y a-t-il des leçons de l'histoire ? \\
Y a-t-il des limites à la connaissance ? \\
Y a-t-il des limites à la conscience ? \\
Y a-t-il des limites à la tolérance ? \\
Y a-t-il des limites à l'exprimable ? \\
Y a-t-il des limites au droit ? \\
Y a-t-il des limites proprement morales à la discussion ? \\
Y a-t-il des lois de la pensée ? \\
Y a-t-il des lois de l'histoire ? \\
Y a-t-il des lois de l'Histoire ? \\
Y a-t-il des lois du hasard ? \\
Y a-t-il des lois du social ? \\
Y a-t-il des lois du vivant ? \\
Y a-t-il des lois en histoire ? \\
Y a-t-il des lois injustes ? \\
Y a-t-il des lois morales ? \\
Y a-t-il des lois non écrites ? \\
Y a-t-il des mentalités collectives ? \\
Y a-t-il des modèles en morale ? \\
Y a-t-il des mondes imaginaires ? \\
Y a-t-il des normes naturelles ? \\
Y a-t-il des objets qui n'existent pas ? \\
Y a-t-il des obstacles à la connaissance du vivant ? \\
Y a-t-il des passions collectives ? \\
Y a-t-il des passions intraitables ? \\
Y a-t-il des passions raisonnables ? \\
Y a-t-il des pathologies sociales ? \\
Y a-t-il des pensées folles ? \\
Y a-t-il des pensées inconscientes ? \\
Y a-t-il des perceptions insensibles ? \\
Y a-t-il des peuples sans histoire ? \\
Y a-t-il des plaisirs meilleurs que d'autres ? \\
Y a-t-il des plaisirs purs ? \\
Y a-t-il des preuves d'amour ? \\
Y a-t-il des principes de justice universels ? \\
Y a-t-il des progrès en art ? \\
Y a-t-il des propriétés singulières ? \\
Y a-t-il des questions sans réponse ? \\
Y a-t-il des raisons de douter de la raison ? \\
Y a-t-il des règles de la guerre ? \\
Y a-t-il des règles de l'art ? \\
Y a-t-il des régressions historiques ? \\
Y a-t-il des révolutions en art ? \\
Y a-t-il des révolutions scientifiques ? \\
Y a-t-il des sciences de l'homme ? \\
Y a-t-il des sciences exactes ? \\
Y a-t-il des secrets de la nature ? \\
Y a-t-il des sentiments moraux ? \\
Y a-t-il des signes naturels ? \\
Y a-t-il des sociétés sans État ? \\
Y a-t-il des sociétés sans histoire ? \\
Y a-t-il des solutions en politique ? \\
Y a-t-il des sots métiers ? \\
Y a-t-il des substances incorporelles ? \\
Y a-t-il des techniques de pensée ? \\
Y a-t-il des techniques du corps ? \\
Y a-t-il des valeurs absolues ? \\
Y a-t-il des valeurs naturelles ? \\
Y a-t-il des valeurs objectives ? \\
Y a-t-il des valeurs propres à la science ? \\
Y a-t-il des valeurs universelles ? \\
Y a-t-il des vérités de fait ? \\
Y a-t-il des vérités définitives ? \\
Y a-t-il des vérités en art ? \\
Y a-t-il des vérités éternelles ? \\
Y a-t-il des vérités indémontrables ? \\
Y a-t-il des vérités indiscutables ? \\
Y a-t-il des vérités morales ? \\
Y a-t-il des vérités philosophiques ? \\
Y a-t-il des vérités qui échappent à la raison ? \\
Y a-t-il des vérités sans preuve ? \\
Y a-t-il des vertus mineures ? \\
Y a-t-il des violences justifiées ? \\
Y a-t-il des violences légitimes ? \\
Y a-t-il différentes façons d'exister ? \\
Y a-t-il différentes manières de connaître ? \\
Y a-t-il du non-être ? \\
Y a-t-il du nouveau dans l'histoire ? \\
Y a-t-il du sacré dans la nature ? \\
Y a-t-il du synthétique \emph{a priori} ? \\
Y a-t-il encore des mythologies ? \\
Y a-t-il encore une sphère privée ? \\
Y a-t-il lieu d'opposer matière et esprit ? \\
Y a-t-il nécessairement du religieux dans l'art ? \\
Y a-t-il place pour l'idée de vérité en morale ? \\
Y a-t-il plusieurs libertés ? \\
Y a-t-il plusieurs manières de définir ? \\
Y a-t-il plusieurs morales ? \\
Y a-t-il plusieurs nécessités ? \\
Y a-t-il plusieurs sortes de matières ? \\
Y a-t-il plusieurs sortes de vérité ? \\
Y a-t-il progrès en art ? \\
Y a-t-il quoi que ce soit de nouveau dans l'histoire ? \\
Y a-t-il trop d'images ? \\
Y a-t-il un art de gouverner ? \\
Y a-t-il un art de penser ? \\
Y a-t-il un art d'être heureux ? \\
Y a-t-il un art de vivre ? \\
Y a-t-il un art d'interpréter ? \\
Y a-t-il un art d'inventer ? \\
Y a-t-il un art du bonheur ? \\
Y a-t-il un au-delà de la vérité ? \\
Y a-t-il un au-delà du langage ? \\
Y a-t-il un auteur de l'histoire ? \\
Y a-t-il un autre monde ? \\
Y a-t-il un beau idéal ? \\
Y a-t-il un beau naturel ? \\
Y a-t-il un besoin métaphysique ? \\
Y a-t-il un bien commun ? \\
Y a-t-il un bien plus précieux que la paix ? \\
Y a-t-il un bonheur sans illusion ? \\
Y a-t-il un bon usage des passions ? \\
Y a-t-il un bon usage du temps ? \\
Y a-t-il un canon de la beauté ? \\
Y a-t-il un critère de vérité ? \\
Y a-t-il un critère du vrai ? \\
Y a-t-il un devoir de mémoire ? \\
Y a-t-il un devoir d'être heureux ? \\
Y a-t-il un devoir d'indignation ? \\
Y a-t-il un droit à la différence ? \\
Y a-t-il un droit au bonheur ? \\
Y a-t-il un droit au travail ? \\
Y a-t-il un droit de désobéissance ? \\
Y a-t-il un droit de la guerre ? \\
Y a-t-il un droit de mentir ? \\
Y a-t-il un droit de mourir ? \\
Y a-t-il un droit de résistance ? \\
Y a-t-il un droit de révolte ? \\
Y a-t-il un droit d'ingérence ? \\
Y a-t-il un droit du plus faible ? \\
Y a-t-il un droit du plus fort ? \\
Y a-t-il un droit international ? \\
Y a-t-il un droit naturel ? \\
Y a-t-il un droit universel au mariage ? \\
Y a-t-il une argumentation métaphysique ? \\
Y a-t-il une beauté morale ? \\
Y a-t-il une beauté naturelle ? \\
Y a-t-il une beauté propre à l'objet technique ? \\
Y a-t-il une bonne imitation ? \\
Y a-t-il une causalité empirique ? \\
Y a-t-il une causalité en histoire ? \\
Y a-t-il une causalité historique ? \\
Y a-t-il une cause première ? \\
Y a-t-il une compétence en politique ? \\
Y a-t-il une compétence politique ? \\
Y a-t-il une condition humaine ? \\
Y a-t-il une connaissance du probable ? \\
Y a-t-il une connaissance du singulier ? \\
Y a-t-il une connaissance historique ? \\
Y a-t-il une connaissance métaphysique ? \\
Y a-t-il une connaissance sensible ? \\
Y a-t-il une conscience collective ? \\
Y a-t-il une correspondance des arts ? \\
Y a-t-il une éducation du goût ? \\
Y a-t-il une enfance de l'humanité ? \\
Y a-t-il une esthétique de la laideur ? \\
Y a-t-il une éthique de l'authenticité ? \\
Y a-t-il une éthique des moyens ? \\
Y a-t-il une expérience de la liberté ? \\
Y a-t-il une expérience de l'éternité ? \\
Y a-t-il une expérience du néant ? \\
Y a-t-il une expérience du temps ? \\
Y a-t-il une fin de l'histoire ? \\
Y a-t-il une fin dernière ? \\
Y a-t-il une fonction propre à l'œuvre d'art ? \\
Y a-t-il une force du droit ? \\
Y a-t-il une forme morale de fanatisme ? \\
Y a-t-il une hiérarchie des êtres ? \\
Y a-t-il une hiérarchie des sciences ? \\
Y a-t-il une hiérarchie du vivant ? \\
Y a-t-il une histoire de la nature ? \\
Y a-t-il une histoire de la raison ? \\
Y a-t-il une histoire de la vérité ? \\
Y a-t-il une histoire universelle ? \\
Y a-t-il une intelligence du corps ? \\
Y a-t-il une intentionnalité collective ? \\
Y a-t-il une irréversibilité du temps ? \\
Y a-t-il une justice naturelle ? \\
Y a-t-il une justice sans morale ? \\
Y a-t-il une langue de la philosophie ? \\
Y a-t-il une limite à la connaissance du vivant ? \\
Y a-t-il une limite au désir ? \\
Y a-t-il une limite au développement scientifique ? \\
Y a-t-il une logique dans l'histoire ? \\
Y a-t-il une logique de la découverte scientifique ? \\
Y a-t-il une logique de la découverte ? \\
Y a-t-il une logique des événements historiques ? \\
Y a-t-il une logique du désir ? \\
Y a-t-il une mécanique des passions ? \\
Y a-t-il une médecine de l'âme ? \\
Y a-t-il une métaphysique de l'amour ? \\
Y a-t-il une méthode de l'interprétation ? \\
Y a-t-il une méthode propre aux sciences humaines ? \\
Y a-t-il une morale universelle ? \\
Y a-t-il un empire de la technique ? \\
Y a-t-il une nature humaine ? \\
Y a-t-il une nécessité de l'erreur ? \\
Y a-t-il une nécessité de l'Histoire ? \\
Y a-t-il une nécessité morale ? \\
Y a-t-il une œuvre du temps ? \\
Y a-t-il une opinion publique mondiale ? \\
Y a-t-il une ou des morales ? \\
Y a-t-il une ou plusieurs philosophies ? \\
Y a-t-il une pensée sans signes ? \\
Y a-t-il une pensée technique ? \\
Y a-t-il une philosophie de la nature ? \\
Y a-t-il une philosophie de la philosophie ? \\
Y a-t-il une philosophie première ? \\
Y a-t-il une place pour la morale dans l'économie ? \\
Y a-t-il une positivité de l'erreur ? \\
Y a-t-il une présence du passé ? \\
Y a-t-il une primauté du devoir sur le droit ? \\
Y a-t-il une rationalité des sentiments ? \\
Y a-t-il une rationalité du hasard ? \\
Y a-t-il une réalité du hasard ? \\
Y a-t-il une responsabilité de l'artiste ? \\
Y a-t-il une sagesse populaire ? \\
Y a-t-il une science de la vie mentale ? \\
Y a-t-il une science de l'esprit ? \\
Y a-t-il une science de l'être ? \\
Y a-t-il une science de l'homme ? \\
Y a-t-il une science de l'individuel ? \\
Y a-t-il une science des principes ? \\
Y a-t-il une science du moi ? \\
Y a-t-il une science du qualitatif ? \\
Y a-t-il une science politique ? \\
Y a-t-il une sensibilité esthétique ? \\
Y a-t-il une servitude volontaire ? \\
Y a-t-il une spécificité de la délibération politique ? \\
Y a-t-il une spécificité des sciences humaines ? \\
Y a-t-il une spécificité du vivant ? \\
Y a-t-il un esprit scientifique ? \\
Y a-t-il un État idéal ? \\
Y a-t-il une technique de la nature ? \\
Y a-t-il une technique pour tout ? \\
Y a-t-il une unité de la science ? \\
Y a-t-il une unité des devoirs ? \\
Y a-t-il une unité des langages humains ? \\
Y a-t-il une unité des sciences ? \\
Y a-t-il une unité en psychologie ? \\
Y a-t-il une universalité des mathématiques ? \\
Y a-t-il une universalité du beau ? \\
Y a-t-il une valeur de l'inutile ? \\
Y a-t-il une vérité de l'œuvre d'art ? \\
Y a-t-il une vérité des apparences ? \\
Y a-t-il une vérité des représentations ? \\
Y a-t-il une vérité des sentiments ? \\
Y a-t-il une vérité des symboles ? \\
Y a-t-il une vérité du sensible ? \\
Y a-t-il une vérité du sentiment ? \\
Y a-t-il une vérité en histoire ? \\
Y a-t-il une vérité philosophique ? \\
Y a-t-il une vertu de l'imitation ? \\
Y a-t-il une vertu de l'oubli ? \\
Y a-t-il une vie de l'esprit ? \\
Y a-t-il une violence du droit ? \\
Y a-t-il un fondement de la croyance ? \\
Y a-t-il un inconscient collectif ? \\
Y a-t-il un inconscient psychique ? \\
Y a-t-il un inconscient social ? \\
Y a-t-il un jugement de l'histoire ? \\
Y a-t-il un langage animal ? \\
Y a-t-il un langage commun ? \\
Y a-t-il un langage de la musique ? \\
Y a-t-il un langage de l'art ? \\
Y a-t-il un langage de l'inconscient ? \\
Y a-t-il un langage du corps ? \\
Y a-t-il un langage unifié de la science ? \\
Y a-t-il un mal absolu ? \\
Y a-t-il un monde de l'art ? \\
Y a-t-il un monde extérieur ? \\
Y a-t-il un moteur de l'histoire ? \\
Y a-t-il un objet du désir ? \\
Y a-t-il un ordre dans la nature ? \\
Y a-t-il un ordre des choses ? \\
Y a-t-il un ordre du monde ? \\
Y a-t-il un primat de la nature sur la culture ? \\
Y a-t-il un principe du mal ? \\
Y a-t-il un progrès du droit ? \\
Y a-t-il un progrès en art ? \\
Y a-t-il un progrès en philosophie ? \\
Y a-t-il un progrès moral ? \\
Y a-t-il un propre de l'homme ? \\
Y a-t-il un rapport moral à soi-même ? \\
Y a-t-il un rythme de l'histoire ? \\
Y a-t-il un savoir de la justice ? \\
Y a-t-il un savoir du contingent ? \\
Y a-t-il un savoir du corps ? \\
Y a-t-il un savoir du juste ? \\
Y a-t-il un savoir du politique ? \\
Y a-t-il un savoir immédiat ? \\
Y a-t-il un savoir politique ? \\
Y a-t-il un savoir pratique \\
Y a-t-il un sens du beau ? \\
Y a-t-il un sens moral ? \\
Y a-t-il un souverain bien ? \\
Y a-t-il un temps des choses ? \\
Y a-t-il un temps pour tout ? \\
Y a-t-il un travail de la pensée ? \\
Y a-t-il un usage moral des passions ? \\
Y a-t-il un usage purement instrumental de la raison ? \\
Y aura-t-il toujours des religions ? \\
« Aime, et fais ce que tu veux » \\
« Aimer » se dit-il en un seul sens ? \\
« Aimez vos ennemis » \\
« Après moi, le déluge » \\
« Aux armes citoyens ! » \\
« À l'impossible, nul n'est tenu » \\
« À quelque chose malheur est bon » \\
« Bienheureuse faute » \\
« Ceci » \\
« Ce ne sont que des mots » \\
« C'est humain » \\
« C'est la vie » \\
« Comment peut-on être persan ? » \\
« Connais-toi toi-même » \\
« Dans un bois aussi courbe que celui dont l'homme est fait on ne peut rien tailler de tout à fait droit » \\
« De la musique avant toute chose » \\
« Deviens qui tu es » \\
« Dieu est mort » \\
« Être contre » \\
« Expliquer les faits sociaux par des faits sociaux » \\
« Il faudrait rester des années entières pour contempler une telle œuvre » \\
« Il ne lui manque que la parole » \\
« Je mens » \\
« Je n'ai pas voulu cela » \\
« Je ne crois que ce que je vois » \\
« Je ne l'ai pas fait exprès » \\
« Je ne voulais pas cela » : en quoi les sciences humaines permettent-elles de comprendre cette excuse ? \\
« Je préfère une injustice à un désordre » \\
« La crainte est le commencement de la sagesse » \\
« La critique est aisée » \\
« La logique » ou bien « les logiques » ? \\
« La science ne pense pas » \\
« La vie des formes » \\
« La vie est une scène » \\
« La vie est un songe » \\
« La vraie morale se moque de la morale » \\
« L'enfer est pavé de bonnes intentions » \\
« Les bons comptes font les bons amis » \\
« Le seul problème philosophique vraiment sérieux, c'est le suicide » \\
« Les faits, rien que les faits » \\
« Les faits sont là » \\
« Le travail rend libre » \\
« L'histoire jugera » \\
« L'histoire jugera » : quel sens faut-il accorder à cette expression ? \\
« L'homme est la mesure de toute chose » \\
« L'homme est la mesure de toutes choses » \\
« Liberté, égalité, fraternité » \\
« Malheur aux vaincus » \\
« Ne fais pas à autrui ce que tu ne voudrais pas qu'on te fasse » \\
« Nul n'est censé ignorer la loi » \\
« Œil pour œil, dent pour dent » \\
« Pas de liberté pour les ennemis de la liberté » ? \\
« Pauvre bête » \\
« Petites causes, grands effets » \\
« Pourquoi » \\
« Prendre ses désirs pour des réalités » \\
« Quelle vanité que la peinture » \\
« Que nul n'entre ici s'il n'est géomètre » \\
« Que va-t-il se passer ? » \\
« Rien de ce qui est humain ne m'est étranger » \\
« Rien n'est sans raison » \\
« Rien n'est simple » \\
« Sans titre » \\
« Sauver les apparences » \\
« Sauver les phénomènes » \\
« Toute peine mérite salaire » \\
« Tout est relatif » \\
« Tradition n'est pas raison » \\
« Trop beau pour être vrai » \\
« Tu ne tueras point » \\
« Un instant d'éternité » \\
« Vis caché » \\

\section{Tri par concours}
\label{sec-2}

\subsection{Agrégation}
\label{sec-2-1}

\noindent
Abolir la propriété \\
À chacun sa morale \\
À chacun son dû \\
Acteurs sociaux et usages sociaux \\
Action et événement \\
Action et production \\
Affirmer et nier \\
Agir \\
Agir justement fait-il de moi un homme juste ? \\
Agir moralement, est-ce lutter contre ses idées ? \\
Agir sans raison \\
Ai-je une âme ? \\
Aimer la nature \\
Aimer la vie \\
Aimer les lois \\
Aimer ses proches \\
Aimer son prochain comme soi-même \\
Aimer une œuvre d'art \\
À l'impossible nul n'est tenu \\
Amitié et société \\
Analyse et synthèse \\
Analyser les mœurs \\
Animal politique ou social ? \\
Anomalie et anomie \\
Anthropologie et ontologie \\
Anthropologie et politique \\
Apparaître \\
Apparence et réalité \\
Appartenons-nous à une culture ? \\
Apprend-on à penser ? \\
Apprend-on à voir ? \\
Apprendre à gouverner \\
Apprendre à penser \\
Apprendre à vivre \\
Apprendre à voir \\
Apprendre s'apprend-il ? \\
Apprentissage et conditionnement \\
Après-coup \\
\emph{A priori} et \emph{a posteriori} \\
À quelles conditions une démarche est-elle scientifique ? \\
À quelles conditions une hypothèse est-elle scientifique ? \\
À quelles conditions un énoncé est-il doué de sens ? \\
À quoi bon discuter ? \\
À quoi bon les sciences humaines et sociales ? \\
À quoi bon ? \\
À quoi faut-il renoncer ? \\
À quoi juger l'action d'un gouvernement ? \\
À quoi la conscience nous donne-t-elle accès ? \\
À quoi la logique peut-elle servir dans les sciences ? \\
À quoi reconnaît-on la vérité ? \\
À quoi reconnaît-on qu'une théorie est scientifique ? \\
À quoi sert la dialectique ? \\
À quoi sert la négation ? \\
À quoi sert l'écriture ? \\
À quoi servent les sciences ? \\
À quoi tient la fermeté du vouloir ? \\
À quoi tient la force des religions ? \\
À quoi tient la vérité d'une interprétation ? \\
Argumenter \\
Art et authenticité \\
Art et critique \\
Art et divertissement \\
Art et émotion \\
Art et finitude \\
Art et folie \\
Art et forme \\
Art et illusion \\
Art et image \\
Art et interdit \\
Art et jeu \\
Art et marchandise \\
Art et mélancolie \\
Art et mémoire \\
Art et métaphysique \\
Art et politique \\
Art et propagande \\
Art et religieux \\
Art et religion \\
Art et représentation \\
Art et technique \\
Art et transgression \\
Art et vérité \\
Arts de l'espace et arts du temps \\
A-t-on des droits contre l'État ? \\
A-t-on des raisons de croire ce qu'on croit ? \\
A-t-on des raisons de croire ? \\
A-t-on le droit de se révolter ? \\
Attente et espérance \\
Au-delà \\
Au-delà de la nature ? \\
Aussitôt dit, aussitôt fait \\
Autorité et pouvoir \\
Autrui, est-ce n'importe quel autre ? \\
Avoir bonne conscience \\
Avoir de l'autorité \\
Avoir de l'esprit \\
Avoir de l'expérience \\
Avoir des ennemis \\
Avoir des principes \\
Avoir des valeurs \\
Avoir du goût \\
Avoir du style \\
Avoir le sens du devoir \\
Avoir le temps \\
Avoir mauvaise conscience \\
Avoir peur \\
Avoir un corps \\
Avoir une bonne mémoire \\
Avoir une idée \\
Avons-nous à apprendre des images ? \\
Avons-nous besoin de métaphysique ? \\
Avons-nous besoin de partis politiques ? \\
Avons-nous besoin de spectacles ? \\
Avons-nous besoin de traditions ? \\
Avons-nous besoin d'experts en matière d'art ? \\
Avons-nous besoin d'un libre arbitre ? \\
Avons-nous des devoirs envers les animaux ? \\
Avons-nous des devoirs envers les générations futures ? \\
Avons-nous des devoirs envers les morts ? \\
Avons-nous des devoirs envers le vivant ? \\
Avons-nous le devoir d'être heureux ? \\
Avons-nous une identité ? \\
Avons-nous une intuition du temps ? \\
Avons-nous une responsabilité envers le passé ? \\
Avons-nous un monde commun ? \\
À chacun ses goûts \\
À quelles conditions est-il acceptable de travailler ? \\
À quelles conditions une explication est-elle scientifique ? \\
À quelles conditions une hypothèse est-elle scientifique ? \\
À quelles conditions une pensée est-elle libre ? \\
À qui dois-je la vérité ? \\
À qui la faute ? \\
À qui profite le travail ? \\
À quoi bon avoir mauvaise conscience ? \\
À quoi bon critiquer les autres ? \\
À quoi faut-il être fidèle ? \\
À quoi l'art nous rend-il sensibles ? \\
À quoi reconnaît-on la rationalité ? \\
À quoi reconnaît-on qu'une politique est juste ? \\
À quoi reconnaît-on un bon gouvernement ? \\
À quoi reconnaît-on une œuvre d'art ? \\
À quoi sert la logique ? \\
À quoi sert la notion de contrat social ? \\
À quoi sert la notion d'état de nature ? \\
À quoi sert la technique ? \\
À quoi sert l'État ? \\
À quoi sert un exemple ? \\
À quoi servent les doctrines morales ? \\
À quoi servent les élections ? \\
À quoi servent les règles ? \\
À quoi servent les statistiques ? \\
À quoi servent les utopies ? \\
À science nouvelle, nouvelle philosophie ? \\
À t-on le droit de faire tout ce qui est permis par la loi ? \\
Bâtir un monde \\
Beauté et vérité \\
Beauté naturelle et beauté artistique \\
Beauté réelle, beauté idéale \\
Bien commun et bien public \\
Bien jouer son rôle \\
Bien juger \\
Bien parler \\
Bonheur et autarcie \\
Calculer \\
Calculer et penser \\
Cartographier \\
Castes et classes \\
Catégories de l'être, catégories de langue \\
Catégories de pensée, catégories de langue \\
Catégories logiques et catégories linguistiques \\
Cause et loi \\
Causes et motivations \\
Causes premières et causes secondes \\
Ce que la morale autorise, l'État peut-il légitimement l'interdire ? \\
Ce que la technique rend possible, peut-on jamais en empêcher la réalisation ? \\
Ce que sait le poète \\
Ce qui dépend de moi \\
Ce qui est démontré est-il nécessairement vrai ? \\
Ce qui est faux est-il dénué de sens ? \\
Ce qui est subjectif est-il arbitraire ? \\
Ce qui fut et ce qui sera \\
Ce qu'il y a \\
Ce qui n'a pas de prix \\
Ce qui n'est pas démontré peut-il être vrai ? \\
Ce qui n'est pas matériel peut-il être réel ? \\
Ce qui n'est pas réel est-il impossible ? \\
Ce qui passe et ce qui demeure \\
Ce qu'on ne peut pas vendre \\
Certaines œuvres d'art ont-elles plus de valeur que d'autres ? \\
Certitude et vérité \\
Cesser d'espérer \\
C'est pour ton bien \\
C'est trop beau pour être vrai ! \\
Ceux qui savent doivent-ils gouverner ? \\
Changer \\
Changer le monde \\
Changer ses désirs plutôt que l'ordre du monde \\
Chaque science porte-t-elle une métaphysique qui lui est propre ? \\
Chercher ses mots \\
Chercher son intérêt, est-ce être immoral ? \\
Choisir \\
Choisir ses souvenirs ? \\
Choisit-on ses souvenirs ? \\
Choisit-on son corps ? \\
Chose et objet \\
Cinéma et réalité \\
Cité juste ou citoyen juste ? \\
Citoyen et soldat \\
Classer \\
Classer et ordonner \\
Classes et histoire \\
Collectionner \\
Commémorer \\
Commencer \\
Commencer en philosophie \\
Comment assumer les conséquences de ses actes ? \\
Comment bien vivre ? \\
Comment comprendre une croyance qu'on ne partage pas ? \\
Comment décider, sinon à la majorité ? \\
Comment définir la raison ? \\
Comment définir la signification \\
Comment devient-on artiste ? \\
Comment devient-on raisonnable ? \\
Comment expliquer les phénomènes mentaux ? \\
Comment juger de la politique ? \\
Comment juger d'une œuvre d'art ? \\
Comment justifier l'autonomie des sciences de la vie ? \\
Comment les sociétés changent-elles ? \\
Comment penser la diversité des langues ? \\
Comment penser un pouvoir qui ne corrompe pas ? \\
Comment percevons-nous l'espace ? \\
Comment peut-on choisir entre différentes hypothèses ? \\
Comment peut-on être sceptique ? \\
Comment reconnaît-on une œuvre d'art ? \\
Comment reconnaît-on un vivant ? \\
Comment sait-on qu'on se comprend ? \\
Comment traiter les animaux ? \\
Comment trancher une controverse ? \\
Comment vivre ensemble ? \\
Comme on dit \\
Communauté et société \\
Communiquer \\
Communiquer et enseigner \\
Compatir \\
Compétence et autorité \\
Composer avec les circonstances \\
Composition et construction \\
Comprendre autrui \\
Comprendre, est-ce interpréter ? \\
Comprendre l'inconscient \\
Compter sur soi \\
Concept et image \\
Concept et métaphore \\
Conception et perception \\
Concevoir et juger \\
Conduire sa vie \\
Conduire ses pensées \\
Conflit et démocratie \\
Connaissance commune et connaissance scientifique \\
Connaissance, croyance, conjecture \\
Connaissance du futur et connaissance du passé \\
Connaissance et croyance \\
Connaissance et expérience \\
Connaissance historique et action politique \\
Connaissons-nous mieux le présent que le passé ? \\
Connaître, est-ce connaître par les causes ? \\
Connaître et comprendre \\
Connaître et penser \\
Connaître par les causes \\
Connaître ses limites \\
Conscience et attention \\
Conscience et mémoire \\
Conseiller le prince \\
Consensus et conflit \\
Conservatisme et tradition \\
Considère-t-on jamais le temps en lui-même ? \\
Consistance et précarité \\
Constitution et lois \\
Consumérisme et démocratie \\
Contemplation et distraction \\
Contempler \\
Contingence et nécessité \\
Contradiction et opposition \\
Contrainte et désobéissance \\
Convention et observation \\
Conviction et responsabilité \\
Corps et identité \\
Correspondre \\
Création et production \\
Créativité et contrainte \\
Créer \\
Crime et châtiment \\
Crise et progrès \\
Critiquer \\
Critiquer la démocratie \\
Croire aux fictions \\
Croire en Dieu \\
Croire, est-ce être faible ? \\
Croire, est-ce obéir ? \\
Croire et savoir \\
Croire pour savoir \\
Croire savoir \\
Croyance et certitude \\
Croyance et probabilité \\
Croyance et vérité \\
Cultes et rituels \\
Cultivons notre jardin \\
Culture et civilisation \\
Culture et conscience \\
Décider \\
Décomposer les choses \\
Découverte et invention \\
Découverte et invention dans les sciences \\
Découvrir \\
Décrire \\
Décrire, est-ce déjà expliquer ? \\
Déduction et expérience \\
Défendre son honneur \\
Définir, est-ce déterminer l'essence ? \\
Définir l'art : à quoi bon ? \\
Définir la vérité, est-ce la connaître ? \\
Définition et démonstration \\
Définition nominale et définition réelle \\
Définitions, axiomes, postulats \\
Déjouer \\
Délibérer, est-ce être dans l'incertitude ? \\
De l'utilité des voyages \\
Démêler le vrai du faux \\
Démériter \\
Démocrates et démagogues \\
Démocratie ancienne et démocratie moderne \\
Démocratie et anarchie \\
Démocratie et démagogie \\
Démocratie et impérialisme \\
Démocratie et religion \\
Démocratie et représentation \\
Démocratie et république \\
Démocratie et transparence \\
Démonstration et argumentation \\
Démonstration et déduction \\
Démontrer, argumenter, expérimenter \\
Démontrer est-il le privilège du mathématicien ? \\
Démontrer et argumenter \\
Dénaturer \\
Dépasser les apparences ? \\
Dépasser l'humain \\
De quel bonheur sommes-nous capables ? \\
De quel droit ? \\
De quelle certitude la science est-elle capable ? \\
De quelle liberté témoigne l'œuvre d'art ? \\
De quelle réalité nos perceptions témoignent-elles ? \\
De quelle réalité témoignent nos perceptions ? \\
De quelle science humaine la folie peut-elle être l'objet ? \\
De quelle vérité l'opinion est-elle capable ? \\
De quoi doute un sceptique ? \\
De quoi est-on conscient ? \\
De quoi est-on malheureux ? \\
De quoi la forme est-elle la forme ? \\
De quoi la logique est-elle la science ? \\
De quoi l'art nous délivre-t-il ? \\
De quoi les métaphysiciens parlent-ils ? \\
De quoi les sciences humaines nous instruisent-elles ? \\
De quoi l'État doit-il être propriétaire ? \\
De quoi l'expérience esthétique est-elle l'expérience ? \\
De quoi n'avons-nous pas conscience ? \\
De quoi ne peut-on pas répondre ? \\
De quoi parlent les mathématiques ? \\
De quoi parlent les théories physiques ? \\
De quoi pâtit-on ? \\
De quoi peut-il y avoir science ? \\
De quoi somme-nous prisonniers ? \\
De quoi sommes-nous responsables ? \\
De quoi y a-t-il expérience ? \\
De quoi y a-t-il histoire ? \\
Déraisonner \\
Désacraliser \\
Des comportements économiques peuvent-ils être rationnels ? \\
Des événements aléatoires peuvent-ils obéir à des lois ? \\
Désintérêt et désintéressement \\
Désirer \\
Désire-t-on la reconnaissance ? \\
Désir et politique \\
Des motivations peuvent-elles être sociales ? \\
Des nations peuvent-elles former une société ? \\
Désobéir \\
Désobéir aux lois \\
Désobéissance et résistance \\
Des peuples sans histoire \\
Des sociétés sans État sont-elles des sociétés politiques ? \\
Déterminisme psychique et déterminisme physique \\
Détruire pour reconstruire \\
Devant qui sommes-nous responsables ? \\
Devenir autre \\
Devenir citoyen \\
Devenir et évolution \\
Devient-on raisonnable ? \\
Devoir et conformisme \\
Dialectique et Philosophie \\
Dialoguer \\
Dieu aurait-il pu mieux faire ? \\
Dieu est-il une limite de la pensée ? \\
Dieu est mort \\
Dieu et César \\
Dieu pense-t-il ? \\
Dieu peut-il tout faire ? \\
Dieu, prouvé ou éprouvé ? \\
Dieu tout-puissant \\
Dire ce qui est \\
Dire, est-ce faire ? \\
Dire et faire \\
Dire et montrer \\
Dire le monde \\
Dire oui \\
Dire « je » \\
Diriger son esprit \\
Discussion et dialogue \\
Distinguer \\
Division du travail et cohésion sociale \\
Documents et monuments \\
Doit-on bien juger pour bien faire ? \\
Doit-on corriger les inégalités sociales ? \\
Doit-on croire en l'humanité ? \\
Doit-on distinguer devoir moral et obligation sociale ? \\
Doit-on justifier les inégalités ? \\
Doit-on répondre de ce qu'on est devenu ? \\
Doit-on respecter la nature ? \\
Doit-on se faire l'avocat du diable ? \\
Doit-on se justifier d'exister ? \\
Doit-on toujours dire la vérité ? \\
Doit-on tout calculer ? \\
Donner \\
Donner à chacun son dû \\
Donner, à quoi bon ? \\
Donner des exemples \\
Donner des raisons \\
Donner du sens \\
Donner raison \\
Donner raison, rendre raison \\
Donner sa parole \\
Donner une représentation \\
Donner un exemple \\
D'où la politique tire-t-elle sa légitimité ? \\
D'où vient aux objets techniques leur beauté ? \\
D'où vient la certitude dans les sciences ? \\
D'où vient la signification des mots ? \\
D'où vient le mal ? \\
D'où vient le plaisir de lire ? \\
D'où vient que l'histoire soit autre chose qu'un chaos ? \\
Droit et démocratie \\
Droit et devoir sont-ils liés ? \\
Droit et morale \\
Droit et protection \\
Droit naturel et loi naturelle \\
Droits de l'homme et droits du citoyen \\
Droits et devoirs \\
Droits et devoirs sont-ils réciproques ? \\
Échange et don \\
Échange et partage \\
Échange et valeur \\
Échanger, est-ce risquer ? \\
Éclairer \\
Économie et politique \\
Économie politique et politique économique \\
Écouter \\
Écrire \\
Écrire l'histoire \\
Éducation de l'homme, éducation du citoyen \\
Éduquer le citoyen \\
Égalité des droits, égalité des conditions \\
Égoïsme et altruisme \\
Égoïsme et individualisme \\
Égoïsme et méchanceté \\
Empirique et expérimental \\
Enfance et moralité \\
En histoire, tout est-il affaire d'interprétation ? \\
En morale, est-ce seulement l'intention qui compte ? \\
En politique, faut-il refuser l'utopie ? \\
En politique, nécessité fait loi \\
En politique n'y a-t-il que des rapports de force ? \\
En politique, peut-on faire table rase du passé ? \\
En politique, y a-t-il des modèles ? \\
En quel sens la métaphysique a-t-elle une histoire ? \\
En quel sens la métaphysique est-elle une science ? \\
En quel sens l'anthropologie peut-elle être historique ? \\
En quel sens parler de structure métaphysique ? \\
En quel sens peut-on parler de la vie sociale comme d'un jeu ? \\
En quel sens peut-on parler de transcendance ? \\
En quel sens peut-on parler d'expérience possible ? \\
En quel sens une œuvre d'art est-elle un document ? \\
Enquêter \\
En quoi la connaissance de la matière peut-elle relever de la métaphysique ? \\
En quoi la matière s'oppose-t-elle à l'esprit ? \\
En quoi la technique fait-elle question ? \\
En quoi les sciences humaines nous éclairent-elles sur la barbarie ? \\
En quoi les sciences humaines sont-elles normatives ? \\
En quoi l'œuvre d'art donne-t-elle à penser ? \\
En quoi une discussion est-elle politique ? \\
En quoi une insulte est-elle blessante ? \\
Enseigner \\
Enseigner, instruire, éduquer \\
Enseigner l'art \\
Entendement et raison \\
Entendre raison \\
Entrer en scène \\
Énumérer \\
Épistémologie générale et épistémologie des sciences particulières \\
Éprouver sa valeur \\
Erreur et illusion \\
Espace et structure sociale \\
Espace mathématique et espace physique \\
Espace public et vie privée \\
Esprit et intériorité \\
Essayer \\
Essence et existence \\
Est-ce la certitude qui fait la science ? \\
Est-ce la démonstration qui fait la science ? \\
Est-ce le corps qui perçoit ? \\
Est-ce l'utilité qui définit un objet technique ? \\
Est-ce par son objet ou par ses méthodes qu'une science peut se définir ? \\
Est-ce pour des raisons morales qu'il faut protéger l'environnement ? \\
Esthétique et éthique \\
Esthétique et poétique \\
Esthétisme et moralité \\
Est-il bon qu'un seul commande ? \\
Est-il difficile de savoir ce que l'on veut ? \\
Est-il difficile d'être heureux ? \\
Est-il judicieux de revenir sur ses décisions ? \\
Est-il légitime d'affirmer que seul le présent existe ? \\
Est-il mauvais de suivre son désir ? \\
Est-il parfois bon de mentir ? \\
Est-il possible de croire en la vie éternelle ? \\
Est-il possible de douter de tout ? \\
Est-il possible d'être neutre politiquement ? \\
Est-il vrai qu'en science, « rien n'est donné, tout est construit » ? \\
Est-il vrai qu'on apprenne de ses erreurs ? \\
Estimer \\
Est-on fondé à distinguer la justice et le droit ? \\
Est-on le produit d'une culture ? \\
Est-on responsable de ce qu'on n'a pas voulu ? \\
Est-on responsable de l'avenir de l'humanité \\
Établir la vérité, est-ce nécessairement démontrer ? \\
État et nation \\
État et société \\
Éternité et immortalité \\
Éthique et authenticité \\
Ethnologie et cinéma \\
Ethnologie et ethnocentrisme \\
Ethnologie et sociologie \\
Être acteur \\
Être affairé \\
Être aliéné \\
Être au monde \\
Être cause de soi \\
Être, c'est agir \\
Être chez soi \\
Être citoyen \\
Être citoyen du monde \\
Être compris \\
Être conscient de soi, est-ce être maître de soi ? \\
Être content de soi \\
Être dans l'esprit \\
Être dans le temps \\
Être dans son bon droit \\
Être dans son droit \\
Être de mauvaise humeur \\
Être de son temps \\
Être déterminé \\
Être égal à soi-même \\
Être en bonne santé \\
Être en désaccord \\
Être en règle avec soi-même \\
Être ensemble \\
Être est-ce agir ? \\
Être et avoir \\
Être et devenir \\
Être et devoir être \\
Être et être pensé \\
Être et ne plus être \\
Être et représentation \\
Être et sens \\
Être exemplaire \\
Être heureux \\
Être hors de soi \\
Être juge et partie \\
Être là \\
Être l'entrepreneur de soi-même \\
Être libre, est-ce n'obéir qu'à soi-même ? \\
Être libre, même dans les fers \\
Être logique \\
Être maître de soi \\
Être malade \\
Être matérialiste \\
Être méchant \\
Être méchant volontairement \\
Être mère \\
Être né \\
Être par soi \\
Être pauvre \\
Être père \\
Être quelqu'un \\
Être réaliste \\
Être sans cause \\
Être sans cœur \\
Être sans scrupule \\
Être sceptique \\
Être seul avec sa conscience \\
Être seul avec soi-même \\
Être soi-même \\
Être spirituel \\
Être systématique \\
Être un artiste \\
Être un corps \\
Être une chose qui pense \\
Être, vie et pensée \\
Étudier \\
Évidence et certitude \\
Évolution et progrès \\
Existence et essence \\
Exister \\
Existe-t-il des dilemmes moraux ? \\
Existe-t-il des questions sans réponse ? \\
Existe-t-il des sciences de différentes natures ? \\
Existe-t-il un bien commun qui soit la norme de la vie politique ? \\
Existe-t-il une opinion publique ? \\
Expérience esthétique et sens commun \\
Expérience et approximation \\
Expérience et expérimentation \\
Expérience et habitude \\
Expérience et interprétation \\
Expérience et phénomène \\
Expérimentation et vérification \\
Expérimenter \\
Explication et prévision \\
Expliquer \\
Expliquer, est-ce interpréter ? \\
Expliquer et comprendre \\
Expliquer et interpréter \\
Expliquer et justifier \\
Expression et création \\
Expression et signification \\
Extension et compréhension \\
Faire ce qu'on dit \\
Faire confiance \\
Faire corps \\
Faire de la métaphysique, est-ce se détourner du monde ? \\
Faire de la politique \\
Faire de nécessité vertu \\
Faire de sa vie une œuvre d'art \\
Faire des choix \\
Faire école \\
Faire et laisser faire \\
Faire justice \\
Faire la loi \\
Faire la morale \\
Faire la paix \\
Faire la révolution \\
Faire l'histoire \\
Faire son devoir \\
Faire une expérience \\
Fait et essence \\
Fait et valeur \\
Famille et tribu \\
Faudrait-il ne rien oublier ? \\
Faut-il accorder l'esprit aux bêtes ? \\
Faut-il aimer la vie ? \\
Faut-il aimer son prochain comme soi-même ? \\
Faut-il aller au-delà des apparences ? \\
Faut-il apprendre à voir ? \\
Faut-il avoir des ennemis ? \\
Faut-il avoir des principes ? \\
Faut-il avoir peur de la liberté ? \\
Faut-il avoir peur des habitudes ? \\
Faut-il concilier les contraires ? \\
Faut-il condamner la rhétorique ? \\
Faut-il condamner les illusions ? \\
Faut-il considérer le droit pénal comme instituant une violence légitime ? \\
Faut-il considérer les faits sociaux comme des choses ? \\
Faut-il craindre la révolution ? \\
Faut-il craindre le pire ? \\
Faut-il craindre les foules ? \\
Faut-il craindre les masses ? \\
Faut-il croire au progrès ? \\
Faut-il croire en quelque chose ? \\
Faut-il défendre l'ordre à tout prix ? \\
Faut-il détruire l'État ? \\
Faut-il dire tout haut ce que les autres pensent tout bas ? \\
Faut-il diriger l'économie ? \\
Faut-il distinguer ce qui est de ce qui doit être ? \\
Faut-il distinguer esthétique et philosophie de l'art ? \\
Faut-il enfermer ? \\
Faut-il être bon ? \\
Faut-il être cosmopolite ? \\
Faut-il être fidèle à soi-même ? \\
Faut-il être mesuré en toutes choses ? \\
Faut-il être objectif ? \\
Faut-il être réaliste en politique ? \\
Faut-il expliquer la morale par son utilité ? \\
Faut-il fuir la politique ? \\
Faut-il interpréter la loi ? \\
Faut-il laisser parler la nature ? \\
Faut-il limiter la souveraineté ? \\
Faut-il limiter l'exercice de la puissance publique ? \\
Faut-il ménager les apparences ? \\
Faut-il mépriser le luxe ? \\
Faut-il mieux vivre comme si nous ne devions jamais mourir ? \\
Faut-il n'être jamais méchant ? \\
Faut-il opposer à la politique la souveraineté du droit ? \\
Faut-il opposer l'art à la connaissance ? \\
Faut-il opposer la théorie et la pratique ? \\
Faut-il opposer l'histoire et la fiction ? \\
Faut-il opposer rhétorique et philosophie ? \\
Faut-il parler pour avoir des idées générales ? \\
Faut-il penser l'État comme un corps ? \\
Faut-il perdre ses illusions ? \\
Faut-il préférer le bonheur à la vérité ? \\
Faut-il préférer une injustice au désordre ? \\
Faut-il prendre soin de soi ? \\
Faut-il que le réel ait un sens ? \\
Faut-il que les meilleurs gouvernent ? \\
Faut-il rechercher la certitude ? \\
Faut-il renoncer à son désir ? \\
Faut-il respecter la nature ? \\
Faut-il respecter les convenances ? \\
Faut-il rompre avec le passé ? \\
Faut-il sauver les apparences ? \\
Faut-il se délivrer des passions ? \\
Faut-il se fier à la majorité ? \\
Faut-il se méfier de l'imagination ? \\
Faut-il se méfier du volontarisme politique ? \\
Faut-il suivre ses intuitions ? \\
Faut-il tolérer les intolérants ? \\
Faut-il tout démontrer ? \\
Faut-il tout interpréter ? \\
Faut-il vivre comme si l'on ne devait jamais mourir ? \\
Faut-il vivre comme si nous étions immortels ? \\
Faut-il vouloir changer le monde ? \\
Faut-il vouloir la paix ? \\
Foi et bonne foi \\
Folie et société \\
Fonction et prédicat \\
Fonder \\
Fonder la justice \\
Fonder une cite \\
Fonder une cité \\
Force et violence \\
Forcer à être libre \\
Forger des hypothèses \\
Formaliser et axiomatiser \\
Forme et rythme \\
Former les esprits \\
Garder la mesure \\
Gérer et gouverner \\
Gouvernement des hommes et administration des choses \\
Gouverner \\
Gouverner, administrer, gérer \\
Gouverner, est-ce prévoir ? \\
Gouverner et se gouverner \\
Grammaire et métaphysique \\
Grammaire et philosophie \\
Grandeur et décadence \\
Groupe, classe, société \\
Guérir \\
Guerre et politique \\
Habiter \\
Habiter le monde \\
Haïr \\
Haïr la raison \\
Histoire et anthropologie \\
Histoire et écriture \\
Histoire et ethnologie \\
Histoire et fiction \\
Histoire et géographie \\
Histoire et mémoire \\
Histoire et morale \\
Homo religiosus \\
Humour et ironie \\
Ici et maintenant \\
Identité et communauté \\
Identité et différence \\
Identité et indiscernabilité \\
Illégalité et injustice \\
Il y a \\
Imaginaire et politique \\
Imaginer \\
Imitation et création \\
Imitation et identification \\
Imiter, est-ce copier ? \\
Inconscient et identité \\
Inconscient et inconscience \\
Inconscient et langage \\
Indépendance et autonomie \\
Indépendance et liberté \\
Individuation et identité \\
Individu et société \\
Infini et indéfini \\
Information et communication \\
Innocenter le devenir \\
Instinct et morale \\
Instruire et éduquer \\
Interdire et prohiber \\
Interprétation et création \\
Interpréter \\
Interpréter, est-ce renoncer à prouver ? \\
Interpréter, est-ce savoir ? \\
Interpréter et expliquer \\
Interpréter et formaliser dans les sciences humaines \\
Interpréter ou expliquer \\
Interpréter une œuvre d'art \\
Intuition et concept \\
Intuition et déduction \\
Invention et création \\
J'ai un corps \\
Je mens \\
Je ne l'ai pas fait exprès \\
Je sens, donc je suis \\
Je, tu, il \\
Jouer \\
Jouer un rôle \\
Jouir sans entraves \\
Jugement analytique et jugement synthétique \\
Jugement esthétique et jugement de valeur \\
Juger \\
Juger en conscience \\
Juger et connaître \\
Juger et raisonner \\
Jusqu'à quel point pouvons-nous juger autrui ? \\
Jusqu'où peut-on soigner ? \\
Justice et égalité \\
Justice et force \\
Justice et vengeance \\
Justice et violence \\
Justifier \\
Justifier et prouver \\
Justifier le mensonge \\
La banalité \\
L'abandon \\
La barbarie \\
La barbarie de la technique \\
La bassesse \\
La béatitude \\
La beauté \\
La beauté a-t-elle une histoire ? \\
La beauté de la nature \\
La beauté des corps \\
La beauté des ruines \\
La beauté du diable \\
La beauté du geste \\
La beauté du monde \\
La beauté est-elle dans le regard ou dans la chose vue ? \\
La beauté est-elle l'objet d'une connaissance ? \\
La beauté est-elle partout ? \\
La beauté est-elle sensible ? \\
La beauté est-elle une promesse de bonheur ? \\
La beauté et la grâce \\
La beauté idéale \\
La beauté morale \\
La beauté naturelle \\
La beauté peut-elle délivrer une vérité ? \\
La belle âme \\
La belle nature \\
La bestialité \\
La bêtise \\
La bêtise n'est-elle pas proprement humaine ? \\
La bibliothèque \\
La bienfaisance \\
La bienveillance \\
La biologie peut-elle se passer de causes finales ? \\
La bonne conscience \\
La bonne volonté \\
L'absence \\
L'absence de fondement \\
L'absence de générosité \\
L'absence d'œuvre \\
L'absolu \\
L'abstraction \\
L'abstraction en art \\
L'abstrait est-il en dehors de l'espace et du temps ? \\
L'abstrait et le concret \\
L'abus de pouvoir \\
L'académisme \\
L'académisme dans l'art \\
La casuistique \\
La catharsis \\
La causalité \\
La causalité en histoire \\
La causalité historique \\
La causalité suppose-t-elle des lois ? \\
La cause \\
La cause et la raison \\
La cause première \\
L'accident \\
L'accidentel \\
L'accomplissement \\
L'accomplissement de soi \\
L'accord \\
La censure \\
La certitude \\
La chair \\
La chance \\
La charité \\
La charité est-elle une vertu ? \\
La chasse et la guerre \\
L'achèvement de l'œuvre \\
La chose \\
La chose en soi \\
La chose publique \\
La chronologie \\
La chute \\
La circonspection \\
La citation \\
La cité idéale \\
La citoyenneté \\
La civilisation \\
La civilité \\
La clarté \\
La classification \\
La classification des arts \\
La classification des sciences \\
La clause de conscience \\
La clémence \\
La cohérence \\
La cohérence est-elle un critère de la vérité ? \\
La colère \\
La comédie du pouvoir \\
La comédie humaine \\
La comédie sociale \\
La communauté internationale \\
La communauté morale \\
La communauté scientifique \\
La communication \\
La communication est-elle nécessaire à la démocratie ? \\
La comparaison \\
La compassion \\
La compassion risque-t-elle d'abolir l'exigence politique ? \\
La compétence \\
La compétence technique peut-elle fonder l'autorité publique ? \\
La composition \\
La compréhension \\
La concorde \\
La concurrence \\
La condition \\
La condition humaine \\
La condition sociale \\
La confiance \\
La connaissance adéquate \\
La connaissance animale \\
La connaissance a-t-elle des limites ? \\
La connaissance commune est-elle le point de départ de la science ? \\
La connaissance de la vie \\
La connaissance de la vie se confond-elle avec celle du vivant ? \\
La connaissance de l'histoire est-elle utile à l'action ? \\
La connaissance des causes \\
La connaissance des principes \\
La connaissance du futur \\
La connaissance du singulier \\
La connaissance du vivant \\
La connaissance est-elle une croyance justifiée ? \\
La connaissance mathématique \\
La connaissance objective \\
La connaissance scientifique abolit-elle toute croyance ? \\
La connaissance scientifique est-elle désintéressée ? \\
La connaissance scientifique n'est-elle qu'une croyance argumentée ? \\
La connaissance suppose-t-elle une éthique ? \\
La conquête \\
La conscience de soi \\
La conscience de soi de l'art \\
La conscience entrave-t-elle l'action ? \\
La conscience est-elle ce qui fait le sujet ? \\
La conscience est-elle intrinsèquement morale ? \\
La conscience est-elle ou n'est-elle pas ? \\
La conscience historique \\
La conscience morale \\
La conscience morale est-elle innée ? \\
La conscience peut-elle être collective ? \\
La conscience politique \\
La conséquence \\
La constance \\
La constitution \\
La contemplation \\
La contestation \\
La contingence \\
La contingence des lois de la nature \\
La contingence du futur \\
La continuité \\
La contradiction \\
La contradiction réside-t-elle dans les choses ? \\
La contrainte \\
La contrainte des lois est-elle une violence ? \\
La contrainte supprime-t-elle la responsabilité ? \\
La contrôle social \\
La conversation \\
La conversion \\
La conviction \\
La coopération \\
La corruption \\
La corruption politique \\
La cosmogonie \\
La couleur \\
La coutume \\
La crainte des Dieux \\
La crainte et l'ignorance \\
La création \\
La création artistique \\
La création dans l'art \\
La création de l'humanité \\
La créativité \\
La criminalité \\
La crise sociale \\
La critique \\
La critique d'art \\
La critique du pouvoir peut-elle conduire à la désobéissance ? \\
La croissance \\
La croyance est-elle l'asile de l'ignorance ? \\
La croyance est-elle signe de faiblesse ? \\
La croyance est-elle une opinion comme les autres ? \\
La croyance est-elle une opinion ? \\
La croyance peut-elle être rationnelle ? \\
La cruauté \\
L'acte \\
L'acteur \\
L'acteur et son rôle \\
L'action collective \\
L'action du temps \\
L'action politique \\
L'action politique a-t-elle un fondement rationnel ? \\
L'action politique peut-elle se passer de mots ? \\
L'actualité \\
L'actuel \\
La cuisine \\
La culpabilité \\
La culture artistique \\
La culture de masse \\
La culture démocratique \\
La culture d'entreprise \\
La culture est-elle affaire de politique ? \\
La culture est-elle nécessaire à l'appréciation d'une œuvre d'art ? \\
La culture est-elle une seconde nature ? \\
La culture et les cultures \\
La culture générale \\
La culture libère-t-elle des préjugés ? \\
La culture morale \\
La culture nous rend-elle meilleurs ? \\
La culture peut-elle être instituée ? \\
La culture peut-elle être objet de science ? \\
La culture scientifique \\
La curiosité \\
La curiosité est-elle à l'origine du savoir ? \\
La danse \\
La danse est-elle l'œuvre du corps ? \\
La décadence \\
La décence \\
La déception \\
La décision morale \\
La décision politique \\
La déduction \\
La défense nationale \\
La déficience \\
La définition \\
La délibération \\
La délibération en morale \\
La délibération politique \\
La démagogie \\
La démence \\
La démesure \\
La démocratie conduit-elle au règne de l'opinion ? \\
La démocratie est-elle le pire des régimes politiques ? \\
La démocratie est-elle moyen ou fin ? \\
La démocratie est-elle nécessairement libérale ? \\
La démocratie est-elle possible ? \\
La démocratie et les experts \\
La démocratie n'est-elle que la force des faibles ? \\
La démocratie participative \\
La démocratie peut-elle se passer de représentation ? \\
La démonstration \\
La démonstration obéit-elle à des lois ? \\
La déontologie \\
La dépendance \\
La déraison \\
La descendance \\
La désillusion \\
La désinvolture \\
La désobéissance civile \\
La destruction \\
La détermination \\
La dette \\
La dialectique \\
La dialectique est-elle une science ? \\
La dictature \\
La différence \\
La différence des arts \\
La différence des sexes \\
La différence sexuelle \\
La difformité \\
La dignité \\
La dignité humaine \\
La digression \\
La discipline \\
La discrétion \\
La discrimination \\
La discursivité \\
La discussion \\
La disposition morale \\
La distance \\
La distinction \\
La distinction de genre \\
La distinction de la nature et de la culture est-elle un fait de culture ? \\
La distinction sociale \\
La distraction \\
La diversion \\
La diversité des cultures \\
La diversité des langues \\
La diversité des perceptions \\
La diversité des religions \\
La diversité des sciences \\
La diversité humaine \\
La division \\
La division des pouvoirs \\
La division des tâches \\
La division du travail \\
L'admiration \\
La docilité est-elle un vice ou une vertu ? \\
La domination \\
La domination du corps \\
La domination sociale \\
L'adoucissement des mœurs \\
La douleur \\
La droit de conquête \\
La droiture \\
La dualité \\
La duplicité \\
La durée \\
La faiblesse de la démocratie \\
La faiblesse de la volonté \\
La faiblesse d'esprit \\
La famille \\
La famille est-elle le lieu de la formation morale ? \\
La fatalité \\
La fatigue \\
La faute \\
La fête \\
La fiction \\
La fidélité \\
La fidélité à soi \\
La figuration \\
La figure de l'ennemi \\
La fin \\
La finalité \\
La finalité des sciences humaines \\
La fin de la métaphysique \\
La fin de la politique \\
La fin de la politique est-elle l'établissement de la justice ? \\
La fin de l'art \\
La fin de l'État \\
La fin de l'histoire \\
La fin des désirs \\
La fin des guerres \\
La fin du monde \\
La fin du travail \\
La finitude \\
La fin justifie-t-elle les moyens ? \\
La folie \\
La folie des grandeurs \\
La fonction de l'art \\
La fonction de penser peut-elle se déléguer ? \\
La fonction du philosophe est-elle de diriger l'État ? \\
La fonction première de l'État est-elle de durer ? \\
La force \\
La force d'âme \\
La force de la loi \\
La force de l'art \\
La force de la vérité \\
La force de l'expérience \\
La force de l'habitude \\
La force de l'idée \\
La force de l'inconscient \\
La force des idées \\
La force des lois \\
La force du pouvoir \\
La force du social \\
La force est-elle une vertu ? \\
La force fait-elle le droit ? \\
La formation de l'esprit \\
La formation des citoyens \\
La formation du goût \\
La formation d'une conscience \\
La fortune \\
La foule \\
La fragilité \\
La franchise \\
La franchise est-elle une vertu ? \\
La fraternité \\
La fraternité a-t-elle un sens politique ? \\
La fraternité est-elle un idéal moral ? \\
La fraternité peut-elle se passer d'un fondement religieux ? \\
La fraude \\
La frivolité \\
La frontière \\
La futilité \\
L'âge d'or \\
La générosité \\
La genèse \\
La genèse de l'œuvre \\
La gentillesse \\
La géographie \\
La géométrie \\
La grâce \\
La grammaire \\
La grammaire contraint-elle la pensée ? \\
La grammaire contraint-elle notre pensée ? \\
La grammaire et la logique \\
La grandeur \\
La grandeur d'âme \\
La grandeur d'une culture \\
La gratitude \\
La gratuité \\
L'agressivité \\
L'agriculture \\
La guérison \\
La guerre civile \\
La guerre est-elle la continuation de la politique par d'autres moyens ? \\
La guerre est-elle la continuation de la politique ? \\
La guerre et la paix \\
La guerre juste \\
La guerre totale \\
La haine de la pensée \\
La haine de la raison \\
La haine des machines \\
La haine de soi \\
La hiérarchie \\
La hiérarchie des arts \\
La hiérarchie des énoncés scientifiques \\
La honte \\
La jalousie \\
La jeunesse \\
La joie \\
La joie de vivre \\
La jouissance \\
La jurisprudence \\
La juste colère \\
La juste mesure \\
La juste peine \\
La justice \\
La justice a-t-elle besoin des institutions ? \\
La justice consiste-t-elle à traiter tout le monde de la même manière ? \\
La justice consiste-t-elle dans l'application de la loi ? \\
La justice de l'État \\
La justice divine \\
La justice entre les générations \\
La justice est-elle une notion morale ? \\
La justice peut-elle se passer de la force ? \\
La justice peut-elle se passer d'institutions ? \\
La justice sociale \\
La justice : moyen ou fin de la politique ? \\
La justification \\
La laïcité \\
La laideur \\
La langue de la raison \\
La langue maternelle \\
L'aléatoire \\
La leçon des choses \\
La lecture \\
La légende \\
La légitimation \\
La légitimité \\
La lettre et l'esprit \\
La libération des mœurs \\
La liberté artistique \\
La liberté civile \\
La liberté créatrice \\
La liberté de culte \\
La liberté de l'artiste \\
La liberté de la science \\
La liberté de parole \\
La liberté des autres \\
La liberté des citoyens \\
La liberté d'expression \\
La liberté d'opinion \\
La liberté du savant \\
La liberté, est-ce l'indépendance à l'égard des passions ? \\
La liberté est-elle un fait ? \\
La liberté implique-t-elle l'indifférence ? \\
La liberté individuelle \\
La liberté intéresse-t-elle les sciences humaines ? \\
La liberté morale \\
La liberté peut-elle faire peur ? \\
La liberté peut-elle se constater ? \\
La liberté peut-elle se prouver ? \\
La liberté peut-elle se refuser ? \\
La liberté politique \\
La liberté se prouve-t-elle ? \\
La liberté se réduit-elle au libre-arbitre ? \\
L'aliénation \\
La limite \\
La littérature peut-elle suppléer les sciences de l'homme ? \\
L'allégorie \\
La logique a-t-elle une histoire ? \\
La logique a-t-elle un intérêt philosophique ? \\
La logique décrit-elle le monde ? \\
La logique est-elle indépendante de la psychologie ? \\
La logique est-elle la norme du vrai ? \\
La logique est-elle l'art de penser ? \\
La logique est-elle un art de penser ? \\
La logique est-elle un art de raisonner ? \\
La logique est-elle une discipline normative ? \\
La logique est-elle une forme de calcul ? \\
La logique est-elle une science de la vérité ? \\
La logique est-elle utile à la métaphysique ? \\
La logique et le réel \\
La logique nous apprend-elle quelque chose sur le langage ordinaire ? \\
La logique peut-elle se passer de la métaphysique ? \\
La logique pourrait-elle nous surprendre ? \\
La logique : découverte ou invention ? \\
La loi \\
La loi du désir \\
La loi du genre \\
La loi du marché \\
La loi éduque-t-elle ? \\
La loi et le règlement \\
La loi peut-elle changer les mœurs ? \\
La louange et le blâme \\
La loyauté \\
L'altérité \\
L'altruisme \\
La lumière de la vérité \\
La lumière naturelle \\
La lutte des classes \\
La machine \\
La magie \\
La magnanimité \\
La main \\
La main et l'outil \\
La maîtrise \\
La maîtrise de la langue \\
La maîtrise de la nature \\
La maîtrise de soi \\
La maîtrise du feu \\
La maîtrise du temps \\
La majesté \\
La majorité \\
La majorité peut-elle être tyrannique ? \\
La maladie \\
La malchance \\
La manière \\
La manifestation \\
L'amateur \\
L'amateurisme \\
La mathématique est-elle une ontologie ? \\
La matière \\
La matière de la pensée \\
La matière de l'œuvre \\
La matière, est-ce le mal ? \\
La matière, est-ce l'informe ? \\
La matière est-elle amorphe ? \\
La matière est-elle une vue de l'esprit ? \\
La matière et la forme \\
La matière n'est-elle qu'une idée ? \\
La matière n'est-elle qu'un obstacle ? \\
La matière pense-t-elle ? \\
La matière peut-elle être objet de connaissance ? \\
La matière première \\
La matière sensible \\
La matière vivante \\
La maturité \\
La mauvaise conscience \\
La mauvaise foi \\
La mauvaise volonté \\
L'ambiguïté \\
L'ambition politique \\
La méchanceté \\
L'âme concerne-t-elle les sciences humaines ? \\
La méconnaissance de soi \\
La médecine est-elle une science ? \\
L'âme des bêtes \\
La médiation \\
L'âme est-elle immortelle ? \\
La méfiance \\
La meilleure constitution \\
La mélancolie \\
L'âme, le monde et Dieu \\
La mémoire \\
La mémoire collective \\
La mémoire et l'histoire \\
La mémoire et l'individu \\
La mémoire sélective \\
La mesure \\
La mesure de l'intelligence \\
La mesure des choses \\
La mesure du temps \\
La métaphore \\
La métaphysique a-t-elle ses fictions ? \\
La métaphysique est-elle le fondement de la morale ? \\
La métaphysique est-elle nécessairement une réflexion sur Dieu ? \\
La métaphysique peut-elle être autre chose qu'une science recherchée ? \\
La métaphysique peut-elle faire appel à l'expérience ? \\
La métaphysique se définit-elle par son objet ou sa démarche ? \\
La méthode \\
La méthode de la science \\
La minorité \\
La misanthropie \\
La misère \\
La misologie \\
L'amitié \\
L'amitié est-elle une vertu ? \\
L'amitié est-elle un principe politique ? \\
L'amitié peut-elle obliger ? \\
La modalité \\
La mode \\
La modélisation en sciences sociales \\
La modération \\
La modération est-elle l'essence de la vertu ? \\
La modération est-elle une vertu politique ? \\
La modernité \\
La modernité dans les arts \\
La mondialisation \\
La monnaie \\
La monumentalité \\
La morale a-t-elle besoin d'être fondée ? \\
La morale a-t-elle besoin d'un au-delà ? \\
La morale a-t-elle besoin d'un fondement ? \\
La morale commune \\
La morale consiste-t-elle à suivre la nature ? \\
La morale de l'athée \\
La morale de l'intérêt \\
La morale des fables \\
La morale doit-elle en appeler à la nature ? \\
La morale doit-elle fournir des préceptes ? \\
La morale du citoyen \\
La morale du plus fort \\
La morale est-elle affaire de jugement ? \\
La morale est-elle affaire de sentiments ? \\
La morale est-elle affaire de sentiment ? \\
La morale est-elle ennemie du bonheur ? \\
La morale est-elle fondée sur la liberté ? \\
La morale est-elle incompatible avec le déterminisme ? \\
La morale est-elle l'ennemie de la vie ? \\
La morale est-elle nécessairement répressive ? \\
La morale est-elle un art de vivre ? \\
La morale est-elle une affaire d'habitude ? \\
La morale est-elle un fait social ? \\
La morale et le droit \\
La morale peut-elle être fondée sur la science ? \\
La morale peut-elle être naturelle ? \\
La morale peut-elle être un calcul ? \\
La morale peut-elle être une science ? \\
La morale peut-elle se passer d'un fondement religieux ? \\
La morale politique \\
La morale suppose-t-elle le libre arbitre ? \\
La moralité des lois \\
La moralité n'est-elle que dressage ? \\
La moralité réside-t-elle dans l'intention ? \\
La mort dans l'âme \\
La mort de Dieu \\
La mort de l'art \\
La mort fait-elle partie de la vie ? \\
L'amour de la liberté \\
L'amour de l'art \\
L'amour de l'humanité \\
L'amour des lois \\
L'amour de soi \\
L'amour est-il désir ? \\
L'amour est-il une vertu ? \\
L'amour et la haine \\
L'amour et la justice \\
L'amour et l'amitié \\
L'amour et la mort \\
L'amour maternel \\
L'amour peut-il être absolu ? \\
L'amour-propre \\
L'amour vrai \\
La multiplicité \\
La multitude \\
La musique a-t-elle une essence ? \\
La musique de film \\
La musique est-elle un langage ? \\
La musique et le bruit \\
La naissance \\
La naissance de la science \\
La naissance de l'homme \\
La naïveté \\
L'analogie \\
L'analyse \\
L'analyse du vécu \\
L'anarchie \\
La nation \\
La nation est-elle dépassée ? \\
La nation et l'État \\
La nature a-t-elle une histoire ? \\
La nature des choses \\
La nature du bien \\
La nature est-elle artiste ? \\
La nature est-elle digne de respect ? \\
La nature est-elle écrite en langage mathématique ? \\
La nature est-elle muette ? \\
La nature est-elle sacrée ? \\
La nature est-elle sans histoire ? \\
La nature est-elle sauvage ? \\
La nature et la grâce \\
La nature et le monde \\
La nature morte \\
La nature parle-t-elle le langage des mathématiques ? \\
La nature peut-elle être belle ? \\
L'anecdotique \\
La nécessité \\
La nécessité des contradictions \\
La nécessité des signes \\
La nécessité historique \\
La négation \\
La négligence \\
La négligence est-elle une faute ? \\
La neutralité \\
La neutralité de l'État \\
Langage et communication \\
Langage et réalité \\
Langage, langue et parole \\
Langage ordinaire et langage de la science \\
L'angélisme \\
L'angoisse \\
Langue et parole \\
L'animal a-t-il des droits ? \\
L'animalité \\
L'animal nous apprend-il quelque chose sur l'homme ? \\
L'animal peut-il être un sujet moral ? \\
L'animal politique \\
L'animisme \\
La noblesse \\
L'anomalie \\
L'anonymat \\
L'anormal \\
La normalité \\
La norme et le fait \\
La nostalgie \\
La notion d'administration \\
La notion de barbarie a-t-elle un sens ? \\
La notion de civilisation \\
La notion de classe dominante \\
La notion de classe sociale \\
La notion de corps social \\
La notion de loi a-t-elle une unité ? \\
La notion de loi dans les sciences de la nature et dans les sciences de l'homme \\
La notion de peuple \\
La notion de point de vue \\
La notion de possible \\
La notion de progrès a-t-elle un sens en politique ? \\
La notion de sujet en politique \\
La notion d'évolution \\
La notion d'intérêt \\
La notion d'ordre \\
La notion physique de force \\
La nouveauté \\
La nouveauté en art \\
L'antériorité \\
L'anthropocentrisme \\
L'anthropologie est-elle une ontologie ? \\
L'anticipation \\
La nuance \\
La nudité \\
La paix \\
La paix civile \\
La paix de la conscience \\
La paix est-elle moins naturelle que la guerre ? \\
La paix est-elle possible ? \\
La paix n'est-elle que l'absence de conflit ? \\
La paix n'est-elle que l'absence de guerre ? \\
La paix perpétuelle \\
La paix sociale est-elle la finalité de la politique ? \\
La paix sociale est-elle une fin en soi ? \\
La panne et la maladie \\
La parenté \\
La parenté et la famille \\
La paresse \\
La parole \\
La parole publique \\
La participation \\
La participation des citoyens \\
La passion de la vérité \\
La passion de l'égalité \\
La passion du juste \\
La passion n'est-elle que souffrance ? \\
La paternité \\
L'apathie \\
La patience \\
La patience est-elle une vertu ? \\
La patrie \\
La pauvreté \\
La peine capitale \\
La peinture est-elle une poésie muette ? \\
La peinture peut-elle être un art du temps ? \\
La pensée a-t-elle une histoire ? \\
La pensée collective \\
La pensée est-elle en lutte avec le langage ? \\
La pensée formelle peut-elle avoir un contenu ? \\
La pensée magique \\
La pensée peut-elle s'écrire ? \\
La perception est-elle l'interprétation du réel ? \\
La perception peut-elle être désintéressée ? \\
La perfectibilité \\
La perfection \\
La perfection en art \\
La perfection morale \\
La personnalité \\
La personne \\
La perspective \\
La persuasion \\
La pertinence \\
La perversion morale \\
La perversité \\
La peur de la mort \\
La peur de la nature \\
La peur de l'autre \\
La peur des mots \\
La peur du châtiment \\
La peur du désordre \\
La philanthropie \\
La philosophie doit-elle se préoccuper du salut ? \\
La philosophie peut-elle disparaître ? \\
La philosophie peut-elle être expérimentale ? \\
La philosophie peut-elle se passer de théologie ? \\
La philosophie première \\
La physique et la chimie \\
La pitié \\
La pitié est-elle morale ? \\
La pitié peut-elle fonder la morale ? \\
La place d'autrui \\
La place du hasard dans la science \\
La place du sujet dans la science \\
La plénitude \\
La pluralité \\
La pluralité des arts \\
La pluralité des cultures \\
La pluralité des langues \\
La pluralité des mondes \\
La pluralité des opinions \\
La pluralité des pouvoirs \\
La pluralité des sciences de la nature \\
La pluralité des sens de l'être \\
La poésie \\
La poésie et l'idée \\
La poésie pense-t-elle ? \\
La polémique \\
La politesse \\
La politique a-t-elle besoin de héros ? \\
La politique a-t-elle besoin de modèles ? \\
La politique a-t-elle besoin d'experts ? \\
La politique a-t-elle pour fin d'éliminer la violence ? \\
La politique consiste-t-elle à faire des compromis ? \\
La politique de la santé \\
La politique doit-elle être morale ? \\
La politique doit-elle être rationnelle ? \\
La politique doit-elle refuser l'utopie ? \\
La politique doit-elle se mêler de l'art ? \\
La politique doit-elle viser la concorde ? \\
La politique doit-elle viser le consensus ? \\
La politique échappe-telle à l'exigence de vérité ? \\
La politique est-elle affaire de décision ? \\
La politique est-elle affaire de jugement ? \\
La politique est-elle architectonique ? \\
La politique est-elle extérieure au droit ? \\
La politique est-elle la continuation de la guerre ? \\
La politique est-elle l'affaire de tous ? \\
La politique est-elle l'art des possibles ? \\
La politique est-elle l'art du possible ? \\
La politique est-elle naturelle ? \\
La politique est-elle par nature sujette à dispute ? \\
La politique est-elle plus importante que tout ? \\
La politique est-elle un art ? \\
La politique est-elle une science ? \\
La politique est-elle une technique ? \\
La politique est-elle un métier ? \\
La politique et la gloire \\
La politique et la ville \\
La politique et le mal \\
La politique et le politique \\
La politique et l'opinion \\
La politique exclut-elle le désordre ? \\
La politique implique-t-elle la hiérarchie ? \\
La politique peut-elle changer la société \\
La politique peut-elle changer le monde ? \\
La politique peut-elle être indépendante de la morale ? \\
La politique peut-elle être objet de science ? \\
La politique peut-elle être un objet de science ? \\
La politique peut-elle n'être qu'une pratique ? \\
La politique peut-elle se passer de croyances ? \\
La politique peut-elle unir les hommes ? \\
La politique repose-t-elle sur un contrat ? \\
La politique requière-t-elle le compromis \\
La politique scientifique \\
La politique suppose-t-elle la morale ? \\
La politique suppose-t-elle une idée de l'homme ? \\
L'apolitisme \\
La populace \\
La population \\
La possibilité \\
La possibilité logique \\
La possibilité métaphysique \\
La possibilité réelle \\
L'apparence \\
L'appartenance sociale \\
L'appel \\
L'appréciation de la nature \\
L'apprentissage \\
L'apprentissage de la langue \\
L'appropriation \\
L'approximation \\
La pratique de l'espace \\
La pratique des sciences met-elle à l'abri des préjugés ? \\
La précaution peut-elle être un principe ? \\
La précision \\
La première fois \\
La première vérité \\
La présence \\
La présence d'esprit \\
La présence du passé \\
La présomption \\
La preuve \\
La preuve de l'existence de Dieu \\
La prévision \\
L'\emph{a priori} \\
La prise de parti est-elle essentielle en politique ? \\
La prison \\
La prison est-elle utile ? \\
La privation \\
La probabilité \\
La probité \\
La productivité de l'art \\
La profondeur \\
La prohibition de l'inceste \\
La promenade \\
La promesse \\
La promesse et le contrat \\
La proposition \\
La propriété \\
La propriété est-elle une garantie de liberté ? \\
La protection \\
La protection sociale \\
La providence \\
La prudence \\
La psychologie est-elle une science de la nature ? \\
La psychologie est-elle une science ? \\
La publicité \\
La pudeur \\
La puissance \\
La puissance de la technique \\
La puissance de l'État \\
La puissance des contraires \\
La puissance des images \\
La puissance du peuple \\
La puissance et l'acte \\
La pulsion \\
La punition \\
La pureté \\
La qualité \\
La question de l'œuvre d'art \\
La question sociale \\
La question : « qui ? » \\
La radicalité \\
La raison a-t-elle le droit d'expliquer ce que morale condamne ? \\
La raison a-t-elle une histoire ? \\
La raison d'état \\
La raison d'État \\
La raison doit-elle être cultivée ? \\
La raison du plus fort \\
La raison est-elle le pouvoir de distinguer le vrai du faux ? \\
La raison est-elle morale par elle-même ? \\
La raison est-elle suffisante ? \\
La raison peut-elle errer ? \\
La raison peut-elle être immédiatement pratique ? \\
La raison peut-elle être pratique ? \\
La raison peut-elle nous commander de croire ? \\
La raison pratique \\
La raison suffisante \\
La rationalité des choix politiques \\
La rationalité des comportements économiques \\
La rationalité du langage \\
La rationalité du marché \\
La rationalité en sciences sociales \\
La rationalité politique \\
L'arbitraire \\
L'arbitraire du signe \\
L'archéologie \\
L'architecte et l'ingénieur \\
L'architecture est-elle un art ? \\
La réaction en politique \\
La réalité \\
La réalité a-t-elle une forme logique ? \\
La réalité décrite par la science s'oppose-t-elle à la démonstration ? \\
La réalité de la vie s'épuise-t-elle dans celle des vivants ? \\
La réalité de l'idéal \\
La réalité de l'idée \\
La réalité du beau \\
La réalité du futur \\
La réalité du possible \\
La réalité du sensible \\
La réalité du temps \\
La réalité du temps se réduit-elle à la conscience que nous en avons ? \\
La réalité est-elle une idée ? \\
La réalité peut-elle être virtuelle ? \\
La réalité sociale \\
La réception de l'œuvre d'art \\
La recherche de l'absolu \\
La recherche de la vérité \\
La recherche de la vérité dans les sciences humaines \\
La recherche des invariants \\
La recherche des origines \\
La recherche du bonheur \\
La recherche du bonheur suffit-elle à déterminer une morale ? \\
La recherche scientifique est-elle désintéressée ? \\
La réciprocité \\
La réciprocité est-elle indispensable à la communauté politique ? \\
La reconnaissance \\
La rectitude \\
La rectitude du droit \\
La référence \\
La référence aux faits suffit-elle à garantir l'objectivité de la connaissance ? \\
La réflexion \\
La réflexion sur l'expérience participe-t-elle de l'expérience ? \\
La réforme \\
La réforme des institutions \\
La réfutation \\
La règle et l'exception \\
La relation \\
La relation de cause à effet \\
La relation de nécessité \\
La religion \\
La religion a-t-elle besoin d'un dieu ? \\
La religion civile \\
La religion est-elle la sagesse des pauvres ? \\
La religion peut-elle faire lien social ? \\
La religion peut-elle suppléer la raison ? \\
La réminiscence \\
La renaissance \\
La rencontre \\
La rencontre d'autrui \\
La réparation \\
La répétition \\
La représentation \\
La représentation en politique \\
La représentation politique \\
La reproduction \\
La reproduction sociale \\
La république \\
La réputation \\
La résignation \\
La résilience \\
La résistance à l'oppression \\
La résistance de la matière \\
La résolution \\
La responsabilité \\
La responsabilité collective \\
La responsabilité de l'artiste \\
La responsabilité politique \\
La ressemblance \\
La restauration des œuvres d'art \\
La révélation \\
La rêverie \\
La révolte \\
La révolte peut-elle être un droit ? \\
La révolution \\
L'argent \\
L'argent et la valeur \\
L'argumentation \\
L'argumentation morale \\
L'argument d'autorité \\
La rhétorique \\
La rhétorique a-t-elle une valeur ? \\
La rhétorique est-elle un art ? \\
La richesse \\
La richesse du sensible \\
La richesse intérieure \\
La rigueur \\
La rigueur de la loi \\
La rigueur morale \\
L'aristocratie \\
La rivalité \\
L'arme rhétorique \\
L'art apprend-il à percevoir ? \\
L'art a-t-il des vertus thérapeutiques ? \\
L'art a-t-il plus de valeur que la vérité ? \\
L'art a-t-il une fin morale ? \\
L'art a-t-il une histoire ? \\
L'art a-t-il une valeur sociale ? \\
L'art contre la beauté ? \\
L'art d'écrire \\
L'art de faire croire \\
L'art de gouverner \\
L'art de masse \\
L'art des images \\
L'art de vivre est-il un art ? \\
L'art doit-il être critique ? \\
L'art doit-il nous étonner ? \\
L'art doit-il refaire le monde ? \\
L'art dramatique \\
L'art du comédien \\
L'art échappe-t-il à la raison ? \\
L'art éduque-t-il l'homme ? \\
L'art engagé \\
L'art est-il affaire de goût ? \\
L'art est-il affaire d'imagination ? \\
L'art est-il à lui-même son propre but ? \\
L'art est-il destiné à embellir ? \\
L'art est-il le produit de l'inconscient ? \\
L'art est-il le règne des apparences ? \\
L'art est-il subversif ? \\
L'art est-il une critique de la culture ? \\
L'art est-il une expérience de la liberté ? \\
L'art est-il un jeu ? \\
L'art est-il un langage ? \\
L'art est-il un mode de connaissance ? \\
L'art est-il un modèle pour la philosophie ? \\
L'art et la manière \\
L'art et la nature \\
L'art et la tradition \\
L'art et la vie \\
L'art et le divin \\
L'art et le mouvement \\
L'art et l'éphémère \\
L'art et le rêve \\
L'art et le sacré \\
L'art et les arts \\
L'art et le temps \\
L'art et l'immoralité \\
L'art et morale \\
L'art exprime-t-il ce que nous ne saurions dire ? \\
L'art fait-il penser ? \\
L'artifice \\
L'artificiel \\
L'art imite-t-il la nature ? \\
L'artiste a-t-il besoin d'une idée de l'art ? \\
L'artiste a-t-il besoin d'un public ? \\
L'artiste a-t-il une méthode ? \\
L'artiste est-il le mieux placé pour comprendre son œuvre ? \\
L'artiste est-il maître de son œuvre ? \\
L'artiste et l'artisan \\
L'artiste et la société \\
L'artiste et le savant \\
L'artiste exprime-t-il quelque chose ? \\
L'artiste peut-il se passer d'un maître ? \\
L'artiste sait-il ce qu'il fait ? \\
L'art modifie-t-il notre rapport au réel ? \\
L'art n'est-il pas toujours politique ? \\
L'art n'est-il pas toujours religieux ? \\
L'art n'est-il qu'une question de sentiment ? \\
L'art nous fait-il mieux percevoir le réel ? \\
L'art ou les arts \\
L'art peut-il changer le monde \\
L'art peut-il contribuer à éduquer les hommes ? \\
L'art peut-il encore imiter la nature ? \\
L'art peut-il être abstrait ? \\
L'art peut-il être utile ? \\
L'art peut-il n'être pas conceptuel ? \\
L'art peut-il nous rendre meilleurs ? \\
L'art peut-il prétendre à la vérité ? \\
L'art peut-il quelque chose contre la morale ? \\
L'art peut-il quelque chose pour la morale ? \\
L'art peut-il rendre le mouvement ? \\
L'art peut-il s'affranchir des lois ? \\
L'art peut-il s'enseigner ? \\
L'art peut-il se passer d'idéal ? \\
L'art politique \\
L'art pour l'art \\
L'art produit-il nécessairement des œuvres ? \\
L'art s'adresse-t-il à la sensibilité ? \\
L'art sait-il montrer ce que le langage ne peut pas dire ? \\
L'art s'apparente-t-il à la philosophie ? \\
L'art : expérience, exercice ou habitude ? \\
L'art : une arithmétique sensible ? \\
La ruine \\
La rumeur \\
La rupture \\
La ruse \\
La ruse technique \\
La sacralisation de l'œuvre \\
La sagesse \\
La sagesse et l'expérience \\
La sagesse rend-elle heureux ? \\
La sainteté \\
La sanction \\
La santé \\
La santé est-elle un devoir ? \\
La satisfaction des penchants \\
La scène théâtrale \\
L'ascèse \\
L'ascétisme \\
La science admet-elle des degrés de croyance ? \\
La science a-t-elle besoin du principe de causalité ? \\
La science a-t-elle des limites ? \\
La science a-t-elle le monopole de la vérité ? \\
La science a-t-elle une histoire ? \\
La science commence-t-elle avec la perception ? \\
La science commence-telle avec la perception ? \\
La science découvre-t-elle ou construit-elle son objet ? \\
La science de l'être \\
La science de l'individuel \\
La science des mœurs \\
La science dévoile-t-elle le réel ? \\
La science doit-elle se fonder sur une idée de la nature ? \\
La science doit-elle se passer de l'idée de finalité ? \\
La science est-elle indépendante de toute métaphysique ? \\
La science est-elle une langue bien faite ? \\
La science et le mythe \\
La science et les sciences \\
La science et l'irrationnel \\
La science n'est-elle qu'une activité théorique ? \\
La science n'est-elle qu'une fiction ? \\
La science nous éloigne-t-elle des choses ? \\
La science nous indique-t-elle ce que nous devons faire ? \\
La science pense-t-elle ? \\
La science peut-elle guider notre conduite ? \\
La science peut-elle lutter contre les préjugés ? \\
La science peut-elle se passer de fondement ? \\
La science peut-elle se passer de métaphysique ? \\
La science peut-elle se passer d'hypothèses ? \\
La science peut-elle se passer d'institutions ? \\
La science peut-elle tout expliquer ? \\
La science politique \\
La science porte-elle au scepticisme ? \\
La science procède-t-elle par rectification ? \\
La sculpture \\
La sécularisation \\
La sécurité \\
La sécurité nationale \\
La sécurité publique \\
La séduction \\
La ségrégation \\
La sensibilité \\
La séparation \\
La séparation des pouvoirs \\
La sérénité \\
La servitude \\
La servitude peut-elle être volontaire ? \\
La servitude volontaire \\
La sévérité \\
La sexualité \\
La signification \\
La signification dans l'œuvre \\
La signification en musique \\
La simplicité \\
La simplicité du bien \\
La sincérité \\
La singularité \\
La situation \\
La sobriété \\
La socialisation des comportements \\
La société civile \\
La société civile et l'État \\
La société contre l'État \\
La société des nations \\
La société des savants \\
La société du genre humain \\
La société est-elle concevable sans le travail ? \\
La société et l'État \\
La société existe-t-elle ? \\
La société peut-elle se passer de l'État ? \\
La société sans l'État \\
La sociologie de l'art nous permet-elle de comprendre l'art ? \\
La sociologie relativise-t-elle la valeur des œuvres d'art ? \\
La solidarité \\
La solitude \\
La solitude constitue-t-elle un obstacle à la citoyenneté ? \\
La sollicitude \\
La somme et le tout \\
La souffrance \\
La souffrance a-t-elle un sens moral ? \\
La souffrance a-t-elle un sens ? \\
La souffrance au travail \\
La souffrance d'autrui \\
La souffrance morale \\
La souveraineté \\
La souveraineté de l'État \\
La souveraineté du peuple \\
La souveraineté peut-elle se partager ? \\
La souveraineté populaire \\
La spécificité des sciences humaines \\
La spéculation \\
La sphère privée échappe-t-elle au politique ? \\
L'aspiration esthétique \\
La spontanéité \\
L'association \\
L'association des idées \\
La structure et le sujet \\
La subjectivité \\
La succession des théories scientifiques \\
La superstition \\
La sûreté \\
La surveillance de la société \\
La survie \\
La sympathie \\
La sympathie peut-elle tenir lieu de moralité ? \\
La table rase \\
La technique a-t-elle une histoire ? \\
La technique est-elle moralement neutre ? \\
La technique fait-elle des miracles ? \\
La technique fait-elle violence à la nature ? \\
La technique n'est-elle qu'une application de la science ? \\
La technique permet-elle de réaliser tous les désirs ? \\
La technique peut-elle améliorer l'homme ? \\
La technocratie \\
La technologie modifie-t-elle les rapports sociaux ? \\
La téléologie \\
La tempérance \\
La temporalité de l'œuvre d'art \\
La tendance \\
La tentation \\
La tentation réductionniste \\
La terre \\
La Terre et le Ciel \\
La terreur \\
La terreur morale \\
L'athéisme \\
La théogonie \\
La théologie rationnelle \\
La théorie et la pratique \\
La théorie et l'expérience \\
La tolérance \\
La tolérance a-t-elle des limites ? \\
La tolérance envers les intolérants \\
La tolérance est-elle un concept politique ? \\
La tolérance peut-elle constituer un problème pour la démocratie ? \\
L'atome \\
La totalitarisme \\
La totalité \\
La toute puissance \\
La toute-puissance \\
La toute puissance de la pensée \\
La trace \\
La trace et l'indice \\
La tradition \\
La traduction \\
La tranquillité \\
La transcendance \\
La transe \\
La transgression \\
La transmission \\
La transparence est-elle un idéal démocratique ? \\
La tristesse \\
L'attachement \\
L'attente \\
L'attention \\
L'attrait du beau \\
La tyrannie \\
La tyrannie de la majorité \\
La tyrannie du bonheur \\
L'audace \\
L'audace politique \\
L'au-delà \\
L'au-delà de l'être \\
L'autarcie \\
L'auteur et le créateur \\
L'authenticité \\
L'authenticité de l'œuvre d'art \\
L'autobiographie \\
L'autonomie \\
L'autonomie de l'art \\
L'autonomie de l'œuvre d'art \\
L'autonomie du théorique \\
L'autoportrait \\
L'autorité \\
L'autorité de la parole \\
L'autorité de la science \\
L'autorité de l'écrit \\
L'autorité de l'État \\
L'autorité morale \\
L'autorité politique \\
L'autre est-il le fondement de la conscience morale ? \\
L'autre et les autres \\
L'autre monde \\
La valeur d'échange \\
La valeur de la science \\
La valeur de l'échange \\
La valeur de l'exemple \\
La valeur des choses \\
La valeur du consentement \\
La valeur d'une action se mesure-t-elle à sa réussite ? \\
La valeur d'une théorie scientifique se mesure-t-elle à son efficacité ? \\
La valeur du plaisir \\
La valeur du témoignage \\
La valeur du temps \\
La valeur du travail \\
La valeur morale \\
La validité \\
La vanité \\
La vanité est-elle toujours sans objet ? \\
L'avant-garde \\
L'avarice \\
La variété \\
La vénalité \\
La vengeance \\
L'avenir \\
L'avenir a-t-il une réalité ? \\
L'avenir de l'humanité \\
L'avenir est-il imaginable ? \\
L'avenir existe-t-il ? \\
L'aventure \\
La véracité \\
La vérification \\
La vérité admet-elle des degrés ? \\
La vérité a-t-elle une histoire ? \\
La vérité de la fiction \\
La vérité de l'apparence \\
La vérité de la religion \\
La vérité demande-t-elle du courage ? \\
La vérité des images \\
La vérité des sciences \\
La vérité doit-elle toujours être démontrée ? \\
La vérité du déterminisme \\
La vérité d'une théorie dépend-elle de sa correspondance avec les faits ? \\
La vérité du roman \\
La vérité est-elle éternelle ? \\
La vérité est-elle morale ? \\
La vérité est-elle une construction ? \\
La vérité historique \\
La vérité n'est-elle qu'une erreur rectifiée ? \\
La vérité nous contraint-elle ? \\
La vérité peut-elle être équivoque ? \\
La vérité philosophique \\
La vérité scientifique est-elle relative ? \\
La vertu \\
La vertu de l'homme politique \\
La vertu du citoyen \\
La vertu du plaisir \\
La vertu, les vertus \\
La vertu peut-elle être excessive ? \\
La vertu peut-elle être purement morale ? \\
La vertu peut-elle s'enseigner ? \\
La vertu politique \\
L'aveu \\
L'aveu diminue-t-il la faute ? \\
L'aveuglement \\
La vie active \\
La vie collective est-elle nécessairement frustrante ? \\
La vie de la langue \\
La vie de l'esprit \\
La vie des machines \\
La vie du droit \\
La vie est-elle la valeur suprême ? \\
La vie est-elle le bien le plus précieux ? \\
La vie éternelle \\
La vie intérieure \\
La vie ordinaire \\
La vie peut-elle être éternelle ? \\
La vie politique \\
La vie politique est-elle aliénante ? \\
La vie privée \\
La vie psychique \\
La vie quotidienne \\
La vigilance \\
La ville \\
La ville et la campagne \\
La violence \\
La violence a-t-elle des degrés ? \\
La violence de l'État \\
La violence d'État \\
La violence politique \\
La violence révolutionnaire \\
La violence sociale \\
La violence verbale \\
La virtualité \\
La virtuosité \\
La vocation \\
La voix \\
La voix de la conscience \\
La voix du peuple \\
La volonté constitue-t-elle le principe de la politique ? \\
La volonté du peuple \\
La volonté générale \\
La volonté peut-elle être collective ? \\
La volonté peut-elle être indéterminée ? \\
La volonté peut-elle nous manquer ? \\
La volupté \\
La vraie morale se moque-t-elle de la morale ? \\
La vraisemblance \\
La vulgarité \\
La vulnérabilité \\
L'axiome \\
Le barbare \\
Le baroque \\
Le beau a-t-il une histoire ? \\
Le beau est-il aimable ? \\
Le beau est-il une valeur commune ? \\
Le beau et l'agréable \\
Le beau et le bien \\
Le beau et le sublime \\
Le beau naturel \\
Le besoin \\
Le besoin d'absolu \\
Le besoin de métaphysique est-il un besoin de connaissance ? \\
Le besoin de philosophie \\
Le besoin de vérité ? \\
Le bien commun \\
Le bien et le mal \\
Le bien et les biens \\
Le bien public \\
Le bien suppose-t-il la transcendance ? \\
Le bon et l'utile \\
Le bon goût \\
Le bonheur dans le mal \\
Le bonheur de la passion est-il sans lendemain ? \\
Le bonheur des autres \\
Le bonheur des citoyens est-il un idéal politique ? \\
Le bonheur des uns, le malheur des autres \\
Le bonheur est-il un accident ? \\
Le bonheur est-il une fin morale ? \\
Le bonheur est-il une fin politique ? \\
Le bonheur est-il une valeur morale ? \\
Le bonheur est-il un principe politique ? \\
Le bonheur et la vertu \\
Le bonheur peut-il être un droit ? \\
Le bon sens \\
Le bon usage des passions \\
Le bourgeois et le citoyen \\
Le bricolage \\
Le bruit \\
Le cadavre \\
Le calcul \\
Le calcul des plaisirs \\
Le cannibalisme \\
Le capitalisme \\
Le capital social \\
Le caractère \\
L'écart \\
Le cas de conscience \\
Le catéchisme moral \\
Le cerveau et la pensée \\
L'échange des marchandises et les rapports humains \\
L'échange est-il un facteur de paix ? \\
Le changement \\
L'échange symbolique \\
Le chant \\
Le chaos \\
Le charisme en politique \\
Le charme et la grâce \\
Le châtiment \\
Le chemin \\
Le choix \\
Le choix d'un destin \\
Le choix peut-il être éclairé ? \\
Le ciel et la terre \\
Le cinéma, art de la représentation ? \\
Le cinéma est-il un art comme les autres ? \\
Le cinéma est-il un art ou une industrie ? \\
Le cinéma est-il un art populaire ? \\
Le cinéma est-il un art ? \\
Le citoyen \\
Le citoyen a-t-il perdu toute naturalité ?L'étranger \\
Le citoyen peut-il être à la fois libre et soumis à l'État ? \\
Le classicisme \\
Le cœur \\
L'école des vertus \\
L'écologie est-elle un problème politique ? \\
L'écologie politique \\
L'écologie, une science humaine ? \\
Le combat contre l'injustice a-t-il une source morale ? \\
Le comique et le tragique \\
Le commencement \\
Le commerce adoucit-il les mœurs ? \\
Le commerce des hommes \\
Le commerce équitable \\
Le commerce est-il pacificateur ? \\
Le commerce peut-il être équitable ? \\
Le commun \\
Le comparatisme dans les sciences humaines \\
Le complexe \\
Le comportement \\
Le compromis \\
Le concept \\
Le concept de nature est-il un concept scientifique ? \\
Le concept de pulsion \\
Le concept de structure sociale \\
Le concept d'inconscient est-il nécessaire en sciences humaines ? \\
Le concret \\
Le conflit de devoirs \\
Le conflit des devoirs \\
Le conflit des interprétations \\
Le conflit entre la science et la religion est-il inévitable ? \\
Le conflit est-il constitutif de la politique ? \\
Le conflit est-il la raison d'être de la politique ? \\
Le conformisme \\
Le conformisme moral \\
Le conformisme social \\
L'économie a-t-elle des lois ? \\
L'économie est-elle une science humaine ? \\
L'économie politique \\
L'économie psychique \\
L'économique et le politique \\
Le conseil \\
Le conseiller du prince \\
Le consensus \\
Le consentement des gouvernés \\
Le contingent \\
Le continu \\
Le contrat \\
Le contrôle social \\
Le convenable \\
Le corps dansant \\
Le corps dit-il quelque chose ? \\
Le corps du travailleur \\
Le corps est-il porteur de valeurs ? \\
Le corps est-il respectable ? \\
Le corps et la machine \\
Le corps et l'âme \\
Le corps et l'esprit \\
Le corps et le temps \\
Le corps et l'instrument \\
Le corps humain \\
Le corps humain est-il naturel ? \\
Le corps n'est-il que matière ? \\
Le corps pense-t-il ? \\
Le corps politique \\
Le corps propre \\
Le cosmopolitisme \\
Le cosmopolitisme peut-il devenir réalité ? \\
Le cosmopolitisme peut-il être réaliste ? \\
Le coup d'État \\
Le courage \\
Le courage politique \\
Le cours des choses \\
Le cours du temps \\
Le créé et l'incréé \\
Le cri \\
Le critère \\
L'écrit et l'oral \\
Le critique d'art \\
L'écriture des lois \\
L'écriture et la parole \\
L'écriture et la pensée \\
L'écriture ne sert-elle qu'à consigner la pensée ? \\
Le culte des ancêtres \\
Le cynisme \\
Le danger \\
Le débat \\
Le débat politique \\
Le défaut \\
Le déguisement \\
Le dérèglement \\
Le dernier mot \\
Le désaccord \\
Le désespoir \\
Le désespoir est-il une faute morale ? \\
Le déshonneur \\
Le design \\
Le désintéressement \\
Le désintéressement esthétique \\
Le désir de gloire \\
Le désir de pouvoir \\
Le désir de reconnaissance \\
Le désir de savoir \\
Le désir d'éternité \\
Le désir de vérité \\
Le désir d'immortalité \\
Le désir d'originalité \\
Le désir est-il l'essence de l'homme ? \\
Le désir est-il sans limite ? \\
Le désir et la loi \\
Le désir et le manque \\
Le désir métaphysique \\
Le désir n'est-il que l'épreuve d'un manque ? \\
Le désir n'est-il qu'inquiétude ? \\
Le désir peut-il se satisfaire de la réalité ? \\
Le désœuvrement \\
Le désordre \\
Le désordre des choses \\
Le despote peut-il être éclairé ? \\
Le despotisme \\
Le dessin et la couleur \\
Le destin \\
Le désuet \\
Le détachement \\
Le détail \\
Le déterminisme \\
Le déterminisme social \\
Le deuil \\
Le devenir \\
Le devoir d'aimer \\
Le devoir d'obéissance \\
Le devoir et la dette \\
Le devoir-être \\
Le devoir se présente-t-il avec la force de l'évidence ? \\
Le dévouement \\
Le dialogue des philosophes \\
Le dialogue entre les cultures \\
Le dieu artiste \\
L'édification morale \\
Le dilemme \\
Le dire et le faire \\
Le discernement \\
Le discontinu \\
Le discours politique \\
Le divers \\
Le divertissement \\
Le divin \\
Le dogmatisme \\
Le don \\
Le don de soi \\
Le don et l'échange \\
Le donné \\
Le double \\
Le doute dans les sciences \\
Le doute est-il une faiblesse de la pensée ? \\
Le drame \\
Le droit à la citoyenneté \\
Le droit à l'erreur \\
Le droit au bonheur \\
Le droit au Bonheur \\
Le droit au respect de la vie privée \\
Le droit au travail \\
Le droit de la guerre \\
Le droit de propriété \\
Le droit de punir \\
Le droit de révolte \\
Le droit des animaux \\
Le droit des gens \\
Le droit des peuples à disposer d'eux-mêmes \\
Le droit de veto \\
Le droit de vie et de mort \\
Le droit de vivre \\
Le droit d'ingérence \\
Le droit d'intervention \\
Le droit doit-il être le seul régulateur de la vie sociale ? \\
Le droit du plus faible \\
Le droit du plus fort \\
Le droit du premier occupant \\
Le droit est-il une science humaine ? \\
Le droit humanitaire \\
Le droit international \\
Le droit peut-il être naturel ? \\
Le droit peut-il se fonder sur la force ? \\
Le dualisme \\
L'éducation artistique \\
L'éducation civique \\
L'éducation des esprits \\
L'éducation du goût \\
L'éducation esthétique \\
L'éducation peut-elle être sentimentale ? \\
L'éducation physique \\
L'éducation politique \\
Le fait de vivre est-il un bien en soi ? \\
Le fait d'exister \\
Le fait divers \\
Le fait et le droit \\
Le fait religieux \\
Le fait scientifique \\
Le fait social est-il une chose ? \\
Le fanatisme \\
Le fantastique \\
Le faux en art \\
Le faux et l'absurde \\
Le faux et le fictif \\
Le féminin et le masculin \\
Le féminisme \\
Le fétichisme \\
Le fétichisme de la marchandise \\
L'efficacité thérapeutique de la psychanalyse \\
L'efficience \\
Le finalisme \\
Le flegme \\
Le fond \\
Le fondement \\
Le fondement de l'induction \\
Le fond et la forme \\
Le formalisme \\
Le formalisme moral \\
Le fou \\
Le fragment \\
Le frivole \\
Le futur est-il contingent ? \\
L'égalité \\
L'égalité civile \\
L'égalité des chances \\
L'égalité des conditions \\
L'égalité des hommes et des femmes est-elle une question politique ? \\
L'égalité des sexes \\
L'égalité devant la loi \\
Légalité et légitimité \\
Légalité et moralité \\
L'égalité peut-elle être une menace pour la liberté ? \\
Le génie \\
Le génie du lieu \\
Le génie du mal \\
Le genre et l'espèce \\
Le genre humain \\
Le geste \\
Le geste créateur \\
Le geste et la parole \\
Légitimité et légalité \\
L'égoïsme \\
Le goût \\
Le goût de la polémique \\
Le goût des autres \\
Le goût du beau \\
Le goût du pouvoir \\
Le goût est-il une faculté ? \\
Le goût est-il une vertu sociale ? \\
Le goût se forme-t-il ? \\
Le goût : certitude ou conviction ? \\
Le gouvernement des experts \\
Le gouvernement des hommes et l'administration des choses \\
Le gouvernement des hommes libres \\
Le gouvernement des meilleurs \\
Le gouvernement de soi et des autres \\
Le hasard \\
Le hasard existe-t-il ? \\
Le hasard fait-il bien les choses ? \\
Le hasard n'est il que la mesure de notre ignorance ? \\
Le hasard n'est-il que la mesure de notre ignorance ? \\
Le haut et le bas \\
Le héros moral \\
Le je ne sais quoi \\
Le jeu \\
Le jeu social \\
Le joli, le beau \\
Le jugement \\
Le jugement critique peut-il s'exercer sans culture ? \\
Le jugement de goût \\
Le jugement de goût est-il universel ? \\
Le jugement dernier \\
Le jugement de valeur est-il indifférent à la vérité ? \\
Le jugement moral \\
Le jugement politique \\
Le juste et le bien \\
Le juste milieu \\
Le laboratoire \\
Le langage animal \\
Le langage de la pensée \\
Le langage de l'art \\
Le langage des sciences \\
Le langage du corps \\
Le langage est-il d'essence poétique ? \\
Le langage ne sert-il qu'à communiquer ? \\
L'élégance \\
Le législateur \\
Le libre arbitre \\
Le libre-arbitre \\
Le libre échange \\
Le lien politique \\
Le lien social \\
Le lien social peut-il être compassionnel ? \\
Le lieu \\
Le lieu commun \\
Le lieu de la pensée \\
Le lieu de l'esprit \\
Le littéral et le figuré \\
L'éloge de la démesure \\
Le loisir \\
Le luxe \\
Le lyrisme \\
Le mal \\
Le mal constitue-t-il une objection à l'existence de Dieu ? \\
Le malentendu \\
Le malheur \\
Le malin plaisir \\
Le mal métaphysique \\
L'émancipation \\
L'émancipation des femmes \\
Le maniérisme \\
Le manifeste politique \\
Le manque de culture \\
Le marché \\
Le marché de l'art \\
Le mariage \\
Le masque \\
Le matériel \\
Le mauvais goût \\
L'embarras du choix \\
Le mécanisme et la mécanique \\
Le méchant peut-il être heureux ? \\
Le meilleur \\
Le meilleur des mondes possible \\
Le meilleur régime \\
Le meilleur régime politique \\
Le même et l'autre \\
Le mensonge \\
Le mensonge de l'art ? \\
Le mensonge en politique \\
Le mensonge politique \\
Le mépris \\
Le mépris peut-il être justifié ? \\
Le mérite \\
Le mérite est-il le critère de la vertu ? \\
Le métaphysicien est-il un physicien à sa façon ? \\
Le métier \\
Le métier de politique \\
Le métier d'homme \\
Le mien et le tien \\
Le milieu \\
Le miracle \\
Le miroir \\
Le mode \\
Le mode d'existence de l'œuvre d'art \\
Le modèle en morale \\
Le modèle organiciste \\
Le moi \\
Le moi est-il haïssable ? \\
Le moindre mal \\
Le monde à l'envers \\
Le monde de l'animal \\
Le monde de l'art \\
Le monde de la technique \\
Le monde de la vie \\
Le monde de l'entreprise \\
Le monde des machines \\
Le monde des œuvres \\
Le monde des physiciens \\
Le monde des rêves \\
Le monde des sens \\
Le monde du rêve \\
Le monde du travail \\
Le monde est-il éternel ? \\
Le monde intérieur \\
Le monde politique \\
Le monde vrai \\
Le monopole de la violence légitime \\
Le monstre \\
Le monstrueux \\
Le moralisme \\
Le moraliste \\
Le mot et la chose \\
L'émotion \\
L'émotion esthétique peut-elle se communiquer ? \\
Le mot juste \\
Le mouvement \\
Le mouvement de la pensée \\
L'empathie \\
L'empathie est-elle nécessaire aux sciences sociales ? \\
L'empire \\
L'empire sur soi \\
L'emploi du temps \\
Le multiculturalisme \\
Le musée \\
Le mystère \\
Le mysticisme \\
Le mythe est-il objet de science ? \\
Le naïf \\
Le narcissisme \\
Le naturalisme des sciences humaines et sociales \\
Le naturel \\
Le naturel et l'artificiel \\
L'encyclopédie \\
Le néant \\
Le nécessaire et le contingent \\
Le négatif \\
L'énergie \\
L'enfance \\
L'enfance de l'art \\
L'enfance est-elle ce qui doit être surmonté ? \\
L'enfant \\
L'engagement \\
L'engagement dans l'art \\
L'engagement politique \\
L'engendrement \\
Le nihilisme \\
L'ennemi \\
L'ennemi intérieur \\
L'ennui \\
Le noble et le vil \\
Le nomade \\
Le nomadisme \\
Le nombre \\
Le nombre et la mesure \\
Le nom propre \\
Le non-sens \\
Le normal et le pathologique \\
L'enquête de terrain \\
L'enquête sociale \\
L'enthousiasme \\
L'enthousiasme est-il moral ? \\
L'entraide \\
Le nu \\
L'envie \\
L'environnement est-il un nouvel objet pour les sciences humaines ? \\
Le oui-dire \\
Le pacifisme \\
Le paradigme \\
Le paradoxe \\
Le pardon \\
Le pardon et l'oubli \\
Le partage \\
Le partage des biens \\
Le partage des connaissances \\
Le partage des savoirs \\
Le partage est-il une obligation morale ? \\
Le particulier \\
Le passage à l'acte \\
Le passé a-t-il plus de réalité que l'avenir ? \\
Le passé peut-il être un objet de connaissance ? \\
Le paternalisme \\
Le patriarcat \\
Le patrimoine \\
Le patrimoine artistique \\
Le patriotisme \\
Le paysage \\
Le pays natal \\
Le péché \\
Le pédagogue \\
Le pessimisme \\
Le peuple et les élites \\
Le phantasme \\
L'éphémère \\
Le phénomène \\
Le philanthrope \\
Le philosophe a-t-il besoin de l'histoire ?Prouver et justifier \\
Le philosophe a-t-il des leçons à donner au politique ? \\
Le philosophe est-il le vrai politique ? \\
Le philosophe-roi \\
L'épistémologie est-elle une logique de la science ? \\
Le plaisir \\
Le plaisir a-t-il un rôle à jouer dans la morale ? \\
Le plaisir de l'art \\
Le plaisir d'imiter \\
Le plaisir esthétique \\
Le plaisir esthétique suppose-t-il une culture ? \\
Le plaisir est-il la fin du désir ? \\
Le plaisir est-il un bien ? \\
Le plaisir et le bien \\
Le pluralisme \\
Le pluralisme politique \\
Le poète réinvente-t-il la langue ? \\
Le poids du passé \\
Le poids du préjugé en politique \\
Le point de vue \\
Le point de vue de l'auteur \\
Le politique a-t-il à régler les passions humaines ? \\
Le politique doit-il être un technicien ? \\
Le politique doit-il se soucier des émotions ? \\
Le politique et le religieux \\
Le politique peut-il faire abstraction de la morale ? \\
Le populaire \\
Le populisme \\
Le portrait \\
Le possible \\
Le possible et le probable \\
Le possible et le réel \\
Le pour et le contre \\
Le pouvoir absolu \\
Le pouvoir causal de l'inconscient \\
Le pouvoir corrompt-il nécessairement ? \\
Le pouvoir corrompt-il ? \\
Le pouvoir de la science \\
Le pouvoir de l'opinion \\
Le pouvoir des images \\
Le pouvoir des mots \\
Le pouvoir des sciences humaines et sociales \\
Le pouvoir du peuple \\
Le pouvoir législatif \\
Le pouvoir peut-il limiter le pouvoir ? \\
Le pouvoir peut-il se déléguer ? \\
Le pouvoir peut-il se passer de sa mise en scène ? \\
Le pouvoir politique est-il nécessairement coercitif ? \\
Le pouvoir politique peut-il échapper à l'arbitraire ? \\
Le pouvoir politique repose-t-il sur un savoir ? \\
Le pouvoir souverain \\
Le pouvoir traditionnel \\
Le préférable \\
Le préjugé \\
Le premier devoir de l'État est-il de se défendre ? \\
Le premier et le primitif \\
Le premier principe \\
Le présent \\
L'épreuve de la liberté \\
Le primitivisme en art \\
Le prince \\
Le principe de causalité \\
Le principe de contradiction \\
Le principe d'égalité \\
Le principe de raison \\
Le principe de réalité \\
Le principe de réciprocité \\
Le principe d'identité \\
Le privé et le public \\
Le privilège de l'original \\
Le prix de la liberté \\
Le probable \\
Le problème de l'être \\
Le processus \\
Le processus de civilisation \\
Le prochain \\
Le proche et le lointain \\
Le profane \\
Le profit est-il la fin de l'échange ? \\
Le progrès \\
Le progrès des sciences \\
Le progrès des sciences infirme-t-il les résultats anciens ? \\
Le progrès en logique \\
Le progrès moral \\
Le progrès scientifique fait-il disparaître la superstition ? \\
Le progrès technique \\
Le projet \\
Le projet d'une paix perpétuelle est-il insensé ? \\
Le propre \\
Le propre de la musique \\
Le propriétaire \\
Le psychisme est-il objet de connaissance ? \\
Le public \\
Le public et le privé \\
Le pur et l'impur \\
Lequel, de l'art ou du réel, est-il une imitation de l'autre ? \\
L'équilibre des pouvoirs \\
L'équité \\
L'équivalence \\
L'équivocité \\
L'équivocité du langage \\
L'équivoque \\
Le quotidien \\
Le raffinement \\
Le raisonnement par l'absurde \\
Le raisonnement scientifique \\
Le raisonnement suit-il des règles ? \\
Le rapport de l'homme à son milieu a-t-il une dimension morale ? \\
Le rationnel et le raisonnable \\
Le réalisme \\
Le réalisme de la science \\
Le récit \\
Le récit en histoire \\
Le réel est-il ce qui résiste ? \\
Le réel est-il rationnel ? \\
Le réel et le virtuel \\
Le réel et l'idéal \\
Le réel peut-il être contradictoire ? \\
Le refoulement \\
Le refus \\
Le regard \\
Le regard du photographe \\
Le règlement politique des conflits ? \\
Le règne de l'homme \\
Le relativisme \\
Le relativisme culturel \\
Le relativisme moral \\
Le remords \\
Le renoncement \\
Le repentir \\
Le repos \\
Le respect \\
Le respect des convenances \\
Le respect des institutions \\
Le ressentiment \\
Le retour à l'expérience \\
Le rêve \\
Le rêve et la veille \\
Le rien \\
Le rigorisme \\
Le risque \\
Le risque technique \\
Le rôle de la théorie dans l'expérience scientifique \\
Le rôle des institutions \\
L'érotisme \\
Le royaume du possible \\
L'erreur \\
L'erreur et la faute \\
L'erreur et l'ignorance \\
L'erreur peut-elle jouer un rôle dans la connaissance scientifique ? \\
L'erreur politique, la faute politique \\
L'erreur scientifique \\
L'érudition \\
Le rythme \\
Le sacré \\
Le sacré et le profane \\
Le sacrifice \\
Le sacrifice de soi \\
Les affaires publiques \\
Les affects sont-ils des objets sociologiques ? \\
Les agents sociaux poursuivent-ils l'utilité ? \\
Les agents sociaux sont-ils rationnels ? \\
Le salut \\
Les amis \\
Les analogies dans les sciences humaines \\
Les anciens et les modernes \\
Les animaux échappent-ils à la moralité ? \\
Les animaux ont-ils des droits ? \\
Les animaux pensent-ils ? \\
Les antagonismes sociaux \\
Les apparences font-elles partie du monde ? \\
Les archives \\
Les arts appliqués \\
Les arts communiquent-ils entre eux ? \\
Les arts de la mémoire \\
Les arts industriels \\
Les arts mineurs \\
Les arts nobles \\
Les arts ont-ils besoin de théorie ? \\
Les arts populaires \\
Les arts vivants \\
Le sauvage et le barbare \\
Le sauvage et le cultivé \\
Le savant et le politique \\
Le savoir a-t-il besoin d'être fondé ? \\
Le savoir du peintre \\
Le savoir émancipe-t-il ? \\
Le savoir est-il libérateur ? \\
Le savoir-faire \\
Le savoir se vulgarise-t-il ? \\
Le savoir utile au citoyen \\
Les beautés de la nature \\
Les beaux-arts sont-ils compatibles entre eux ? \\
Les bénéfices du doute \\
Les bénéfices moraux \\
Les biens communs \\
Les blessures de l'esprit \\
Les bonnes intentions \\
Les bonnes mœurs \\
Les bons sentiments \\
Le scandale \\
Les caractères moraux \\
Les catégories \\
Les causes et les effets \\
Les causes et les lois \\
Les causes finales \\
Les cérémonies \\
Les changements scientifiques et la réalité \\
Les chemins de traverse \\
Les choses \\
Les choses ont-elles une essence ? \\
Les cinq sens \\
Les circonstances \\
Les classes sociales \\
L'esclavage \\
L'esclave \\
L'esclave et son maître \\
Les commandements divins \\
Les conditions de la démocratie \\
Les conflits politiques \\
Les conflits politiques ne sont-ils que des conflits sociaux ? \\
Les conflits sociaux \\
Les conflits sociaux sont-ils des conflits de classe ? \\
Les conflits sociaux sont-ils des conflits politiques ? \\
Les connaissances scientifiques peuvent-elles être à la fois vraies et provisoires ? \\
Les connaissances scientifiques peuvent-elles être vulgarisées ? \\
Les conquêtes de la science \\
Les conséquences de l'action \\
Les coutumes \\
Les critères de vérité dans les sciences humaines \\
Les croyances politiques \\
Le scrupule \\
Les cultures sont-elles incommensurables ? \\
Les degrés de conscience \\
Les degrés de la beauté \\
Les devoirs à l'égard de la nature \\
Les devoirs de l'État \\
Les devoirs envers soi-même \\
Les dictionnaires \\
Les dilemmes moraux \\
Les dispositions sociales \\
Les distinctions sociales \\
Les divisions sociales \\
Les droits de l'enfant \\
Les droits de l'homme \\
Les droits de l'homme et ceux du citoyen \\
Les droits de l'homme ont-ils un fondement moral ? \\
Les droits de l'homme sont-ils une abstraction ? \\
Les droits et les devoirs \\
Les droits naturels imposent-ils une limite à la politique ? \\
Les échanges, facteurs de paix ? \\
Le secret \\
Le secret d'État \\
Les effets de l'esclavage \\
Les éléments \\
Le sens commun \\
Le sens de la mesure \\
Le sens de la situation \\
Le sens de l'État \\
Le sens de l'existence \\
Le sens de l'histoire \\
Le sens de l'Histoire \\
Le sens de l'humour \\
Le sens des mots \\
Le sens du silence \\
Les ensembles \\
Le sensible \\
Le sensible est-il communicable ? \\
Le sensible est-il irréductible à l'intelligible ? \\
Le sens interne \\
Le sens moral \\
Le sens musical \\
Le sentiment de l'existence \\
Le sentiment de l'injustice \\
Le sentiment esthétique \\
Le sentiment moral \\
Les entités mathématiques sont-elles des fictions ? \\
Les envieux \\
Le sérieux \\
Le serment \\
Les études comparatives \\
Les factions politiques \\
Les faits parlent-ils d'eux-mêmes ? \\
Les fausses sciences \\
Les fins de l'art \\
Les fins de l'éducation \\
Les fins dernières \\
Les fins naturelles et les fins morales \\
Les fonctions de l'image \\
Les fondements de l'État \\
Les forts et les faibles \\
Les foules \\
Les fous \\
Les frontières \\
Les frontières de l'art \\
Les fruits du travail \\
Les genres de Dieu \\
Les genres esthétiques \\
Les genres naturels \\
Les grands hommes \\
Les hasards de la vie \\
Les hommes de pouvoir \\
Les hommes et les dieux \\
Les hommes et les femmes \\
Les hommes n'agissent-ils que par intérêt ? \\
Les hommes sont-ils faits pour s'entendre ? \\
Les hommes sont-ils naturellement sociables ? \\
Les idées et les choses \\
Les idées ont-elles une histoire ? \\
Les idées ont-elles une réalité ? \\
Les idées politiques \\
Les idoles \\
Le silence \\
Le silence des lois \\
Les images empêchent-elles de penser ? \\
Le simple \\
Le simulacre \\
Les individus \\
Les industries culturelles \\
Les inégalités sociales \\
Les inégalités sociales sont-elles inévitables ? \\
Le singulier \\
Le singulier est-il objet de connaissance ? \\
Le singulier et le pluriel \\
Les institutions artistiques \\
Les instruments de la pensée \\
Les intentions de l'artiste \\
Les intentions et les conséquences \\
Les interdits \\
Les intérêts particuliers peuvent-ils tempérer l'autorité politique ? \\
Les invariants culturels \\
Les jeux du pouvoir \\
Les jugements analytiques \\
Les leçons de l'expérience \\
Les leçons de morale \\
Les libertés civiles \\
Les libertés fondamentales \\
Les liens sociaux \\
Les lieux du pouvoir \\
Les limites de la connaissance scientifique \\
Les limites de la démocratie \\
Les limites de la description \\
Les limites de la raison \\
Les limites de la tolérance \\
Les limites de la vérité \\
Les limites de la vertu \\
Les limites de l'État \\
Les limites de l'expérience \\
Les limites de l'humain \\
Les limites de l'imagination \\
Les limites de l'interprétation \\
Les limites de l'obéissance \\
Les limites du corps \\
Les limites du pouvoir \\
Les limites du pouvoir politique \\
Les limites du réel \\
Les limites du vivant \\
Les lois causales \\
Les lois de la guerre \\
Les lois de la nature sont-elles contingentes ? \\
Les lois de la nature sont-elles de simples régularités ? \\
Les lois de la nature sont elles nécessaires ? \\
Les lois de l'art \\
Les lois de l'histoire \\
Les lois de l'hospitalité \\
Les lois du sang \\
Les lois nous rendent-elles meilleurs ? \\
Les lois scientifiques sont-elles des lois de la nature ? \\
Les lois sont-elles seulement utiles ? \\
Les machines \\
Les maladies de l'âme \\
Les maladies de l'esprit \\
Les marginaux \\
Les matériaux \\
Les mathématiques du mouvement \\
Les mathématiques et la pensée de l'infini \\
Les mathématiques sont-elles réductibles à la logique ? \\
Les mathématiques sont-elles un langage ? \\
Les mathématiques sont-elles utiles au philosophe ? \\
Les mécanismes cérébraux \\
Les modalités \\
Les modèles \\
Les mœurs \\
Les mœurs et la morale \\
Les mondes possibles \\
Les mots disent-ils les choses ? \\
Les mots et les choses \\
Les mots et les concepts \\
Les mots justes \\
Les moyens de l'autorité \\
Les moyens et la fin \\
Les moyens et les fins en art \\
Les nombres gouvernent-ils le monde ? \\
Les noms \\
Les noms propres \\
Les normes \\
Les normes du vivant \\
Les normes esthétiques \\
Les normes et les valeurs \\
Les nouvelles technologies transforment-elles l'idée de l'art ? \\
Les objets scientifiques \\
Le social et le politique \\
Les œuvres d'art ont-elles besoin d'un commentaire ? \\
Le sommeil de la raison \\
Le sommeil et la veille \\
Les opérations de la pensée \\
Les opinions politiques \\
Le souci d'autrui résume-t-il la morale ? \\
Le souci de l'avenir \\
Le souci de soi \\
Le souci de soi est-il une attitude morale ? \\
Le souci du bien-être est-il politique ? \\
Le souverain bien \\
L'espace de la perception \\
L'espace et le lieu \\
L'espace et le territoire \\
L'espace public \\
Les paroles et les actes \\
Les parties de l'âme \\
Les passions peuvent-elles être raisonnables ? \\
Les passions politiques \\
Les passions sont-elles un obstacle à la vie sociale ? \\
Les pauvres \\
L'espèce et l'individu \\
Le spectacle \\
Le spectacle de la nature \\
Le spectacle de la pensée \\
L'espérance est-elle une vertu ? \\
Les peuples ont-ils les gouvernements qu'ils méritent ? \\
Les phénomènes inconscients sont-ils réductibles à une mécanique cérébrale ? \\
Le spirituel et le temporel \\
Les plaisirs \\
Les plaisirs de l'amitié \\
Les poètes et la cité \\
Les pouvoirs de la religion \\
Les préjugés moraux \\
Les prêtres \\
Les principes de la démonstration \\
Les principes d'une science sont-ils des conventions ? \\
Les principes et les éléments \\
Les principes moraux \\
Les principes sont-ils indémontrables ?Qu'est-ce qu'être ensemble ? \\
L'esprit critique \\
L'esprit de finesse \\
L'esprit de système \\
L'esprit d'invention \\
L'esprit est-il matériel ? \\
L'esprit est-il objet de science ? \\
L'esprit est-il plus aisé à connaître que le corps ? \\
L'esprit est-il une machine ? \\
L'esprit est-il un ensemble de facultés ? \\
L'esprit et la machine \\
L'esprit n'a-t-il jamais affaire qu'à lui-même ? \\
L'esprit peut-il être malade ? \\
L'esprit peut-il être mesuré ? \\
L'esprit peut-il être objet de science ? \\
L'esprit scientifique \\
L'esprit s'explique-t-il par une activité cérébrale ? \\
L'esprit tranquille \\
Les problèmes politiques peuvent-ils se ramener à des problèmes techniques ? \\
Les problèmes politiques sont-ils des problèmes techniques ? \\
Les propositions métaphysiques sont-elles des illusions ? \\
Les proverbes \\
Les proverbes enseignent-ils quelque chose ? \\
Les proverbes nous instruisent-ils moralement ? \\
Les qualités esthétiques \\
Les questions métaphysiques ont-elles un sens ? \\
L'esquisse \\
Les raisons de vivre \\
Les règles de l'art \\
Les règles du jeu \\
Les règles d'un bon gouvernement \\
Les règles sociales \\
Les relations \\
Les religions sont-elles des illusions ? \\
Les représentants du peuple \\
Les reproductions \\
Les ressources humaines \\
Les révolutions scientifiques \\
Les révolutions techniques suscitent-elles des révolutions dans l'art ? \\
Les riches et les pauvres \\
Les rituels \\
Les rôles sociaux \\
Les ruines \\
Les sacrifices \\
Les sauvages \\
Les sciences décrivent-elles le réel ? \\
Les sciences de la vie et de la Terre \\
Les sciences de la vie visent-elles un objet irréductible à la matière ? \\
Les sciences de l'éducation \\
Les sciences de l'esprit \\
Les sciences de l'homme et l'évolution \\
Les sciences de l'homme ont-elles inventé leur objet ? \\
Les sciences de l'homme permettent-elles d'affiner la notion de responsabilité ? \\
Les sciences de l'homme peuvent-elles expliquer l'impuissance de la liberté ? \\
Les sciences de l'homme rendent-elles l'homme prévisible ? \\
Les sciences doivent-elle prétendre à l'unification ? \\
Les sciences du comportement \\
Les sciences et le vivant \\
Les sciences exactes \\
Les sciences forment-elle un système ? \\
Les sciences historiques \\
Les sciences humaines doivent-elles être transdisciplinaires ? \\
Les sciences humaines éliminent-elles la contingence du futur ? \\
Les sciences humaines et le droit \\
Les sciences humaines nous protègent-elles de l'essentialisme ? \\
Les sciences humaines ont-elles un objet commun ? \\
Les sciences humaines permettent-elles de comprendre la vie d'un homme ? \\
Les sciences humaines peuvent-elles adopter les méthodes des sciences de la nature ? \\
Les sciences humaines peuvent-elles se passer de la notion d'inconscient ? \\
Les sciences humaines présupposent-elles une définition de l'homme ? \\
Les sciences humaines sont-elles des sciences de la nature humaine ? \\
Les sciences humaines sont-elles des sciences de la vie humaine ? \\
Les sciences humaines sont-elles des sciences d'interprétation ? \\
Les sciences humaines sont-elles des sciences ? \\
Les sciences humaines sont-elles explicatives ou compréhensives ? \\
Les sciences humaines sont-elles normatives ? \\
Les sciences humaines sont-elles relativistes ? \\
Les sciences humaines sont-elles subversives ? \\
Les sciences humaines traitent-elles de l'individu ? \\
Les sciences humaines transforment-elles la notion de causalité ? \\
Les sciences naturelles \\
Les sciences ne sont-elles qu'une description du monde ? \\
Les sciences ont-elles besoin d'une fondation métaphysique ? \\
Les sciences peuvent-elles penser l'individu ? \\
Les sciences sociales \\
Les sciences sociales peuvent-elles être expérimentales ? \\
Les sciences sociales sont-elles nécessairement inexactes ? \\
L'essence \\
Les sens peuvent-ils nous tromper ? \\
Les sentiments \\
Les sentiments peuvent-ils s'apprendre ? \\
Les services publics \\
Les signes de l'intelligence \\
Les sociétés évoluent-elles ? \\
Les sociétés ont-elles un inconscient ? \\
Les sociétés sont-elles hiérarchisables ? \\
Les sociétés sont-elles imprévisibles ? \\
Les structures expliquent-elles tout ? \\
Les systèmes \\
Le statut de l'axiome \\
Le statut des hypothèses dans la démarche scientifique \\
Les techniques artistiques \\
Les théories scientifiques décrivent-elles la réalité ? \\
Les théories scientifiques sont-elles vraies ? \\
L'esthète \\
L'esthétique est-elle une métaphysique de l'art ? \\
L'esthétisme \\
L'estime de soi \\
Les traditions \\
Le style \\
Le sublime \\
Le succès \\
Le sujet \\
Le sujet de droit \\
Le sujet de l'action \\
Le sujet de la pensée \\
Le sujet et l'objet \\
Le sujet moral \\
Les universaux \\
Les usages de l'art \\
Les valeurs de la République \\
Les vérités éternelles \\
Les vérités scientifiques sont-elles relatives ? \\
Les vertus \\
Les vertus de l'amour \\
Les vertus politiques \\
Les visages du mal \\
Les vivants \\
Les vivants et les morts \\
Le syllogisme \\
Le symbole \\
Le symbolisme \\
Le symbolisme mathématique \\
Le système des beaux-arts \\
Le système des besoins \\
Le tableau \\
Le talent \\
Le talent et le génie \\
L'État a-t-il pour finalité de maintenir l'ordre ? \\
L'État de droit \\
L'état de la nature \\
L'état de nature \\
L'état d'exception \\
L'État doit-il disparaître ? \\
L'État doit-il éduquer le citoyen ? \\
L'État doit-il éduquer les citoyens ? \\
L'État doit-il faire le bonheur des citoyens ? \\
L'État est-il appelé à disparaître ? \\
L'État est-il fin ou moyen ? \\
L'État est-il le garant de la propriété privée ? \\
L'État est-il un moindre mal ? \\
L'État et la culture \\
L'État et la guerre \\
L'État et la Nation \\
L'État et le marché \\
L'État et les Églises \\
L'État libéral \\
L'État mondial \\
L'État peut-il créer la liberté ? \\
L'État peut-il être indifférent à la religion ? \\
L'État peut-il être libéral ? \\
L'État providence \\
L'État-providence \\
L'État universel \\
Le témoignage \\
Le temps de la liberté \\
Le temps de la science \\
Le temps du désir \\
Le temps est-il essentiellement destructeur ? \\
Le temps est-il une dimension de la nature ? \\
Le temps ne fait-il que passer ? \\
Le temps perdu \\
Le temps se laisse-t-il décrire logiquement ? \\
L'éternel présent \\
L'éternité \\
L'éternité n'est-elle qu'une illusion ? \\
Le terrain \\
Le territoire \\
Le théâtre du monde \\
L'éthique à l'épreuve du tragique \\
L'éthique des plaisirs \\
L'éthique est-elle affaire de choix ? \\
L'éthique suppose-t-elle la liberté ? \\
L'ethnocentrisme \\
Le tiers exclu \\
L'étonnement \\
Le totalitarisme \\
Le totémisme \\
Le toucher \\
Le tourment moral \\
Le tout est-il la somme de ses parties ? \\
Le tragique \\
Le trait d'esprit \\
L'étranger \\
L'étrangeté \\
Le travail \\
Le travail artistique \\
Le travail est-il une fin ? \\
Le travail est-il une valeur morale ? \\
Le travail et l'œuvre \\
Le travail nous rend-il solidaires ? \\
Le travail rapproche-t-il les hommes ? \\
Le travail sur le terrain \\
Le travail sur soi \\
L'être de la conscience \\
L'être de la vérité \\
L'être de l'image \\
L'être du possible \\
L'être en tant qu'être \\
L'être en tant qu'être est-il connaissable ? \\
L'être et la volonté \\
L'être et le bien \\
L'être et le devoir-être \\
L'être et le néant \\
L'être et les êtres \\
L'être et l'essence \\
L'être et l'étant \\
L'être et le temps \\
L'être se confond-il avec l'être perçu ? \\
Le tribunal de l'histoire \\
L'eugénisme \\
Le vainqueur a-t-il tous les droits ? \\
Le vécu \\
L'événement \\
L'événement et le fait divers \\
L'événement manque-t-il d'être ? \\
Le verbalisme \\
Le verbe \\
Le vide \\
L'évidence \\
L'évidence a-t-elle une valeur absolue ? \\
Le village global \\
Le virtuel \\
Le visage \\
Le visible et l'invisible \\
Le vivant a-t-il des droits ? \\
Le vivant comme problème pour la philosophie des sciences \\
Le vivant est-il entièrement connaissable ? \\
Le volontaire et l'involontaire \\
L'évolution \\
L'évolution des langues \\
Le voyage \\
Le vrai a-t-il une histoire ? \\
Le vrai doit-il être démontré ? \\
Le vrai est-il à lui-même sa propre marque ? \\
Le vrai et le vraisemblable \\
Le vrai peut-il rester invérifiable ? \\
Le vraisemblable \\
Le vrai se perçoit-il ? \\
Le vrai se réduit-il à l'utile ? \\
Le vulgaire \\
L'exactitude \\
L'excellence \\
L'exception \\
L'excès et le défaut \\
L'exclusion \\
L'excuse \\
L'exécution d'une œuvre d'art est-elle toujours une œuvre d'art ? \\
L'exemplaire \\
L'exemplarité \\
L'exemple \\
L'exercice de la vertu \\
L'exercice du pouvoir \\
L'exercice solitaire du pouvoir \\
L'exigence de vérité a-t-elle un sens moral ? \\
L'exigence morale \\
L'exil \\
L'existence de Dieu \\
L'existence de l'État dépend-elle d'un contrat ? \\
L'existence du mal \\
L'existence est-elle un jeu ? \\
L'existence se démontre-t-elle ? \\
L'expérience \\
L'expérience artistique \\
L'expérience cruciale \\
L'expérience de la liberté \\
L'expérience directe est-elle une connaissance ? \\
L'expérience en sciences humaines \\
L'expérience enseigne-elle quelque chose ? \\
L'expérience, est-ce l'observation ? \\
L'expérience et l'expérimentation \\
L'expérience métaphysique \\
L'expérience morale \\
L'expérience sensible est-elle la seule source légitime de connaissance ? \\
L'expérimentation \\
L'expérimentation en psychologie \\
L'expérimentation en sciences sociales \\
L'expérimentation sur l'être humain \\
L'expérimentation sur le vivant \\
L'expert et l'amateur \\
L'expertise \\
L'expertise politique \\
L'explication scientifique \\
L'exploitation de l'homme par l'homme \\
L'exposition \\
L'exposition de l'œuvre d'art \\
L'expression \\
L'expression artistique \\
L'expression de l'inconscient \\
L'expression du désir \\
L'expressivité musicale \\
L'extériorité \\
L'habileté \\
L'habileté et la prudence \\
L'habitation \\
L'habitude \\
L'harmonie \\
L'hégémonie politique \\
L'héritage \\
L'hésitation \\
L'hétérogénéité sociale \\
L'hétéronomie \\
L'hétéronomie de l'art \\
L'histoire a-t-elle un sens ? \\
L'histoire de l'art \\
L'histoire de l'art est-elle celle des styles ? \\
L'histoire de l'art est-elle finie ? \\
L'histoire des arts est-elle liée à l'histoire des techniques ? \\
L'histoire des civilisations \\
L'histoire des sciences \\
L'histoire des sciences est-elle une histoire ? \\
L'histoire est-elle avant tout mémoire ? \\
L'histoire est-elle déterministe ? \\
L'histoire est-elle le règne du hasard ? \\
L'histoire est-elle tragique ? \\
L'histoire est-elle un genre littéraire ? \\
L'histoire est-elle un roman vrai ? \\
L'histoire est-elle utile à la politique ? \\
L'histoire est-elle utile ? \\
L'histoire et la géographie \\
L'histoire peut-elle être universelle ? \\
L'histoire peut-elle se répéter ? \\
L'histoire universelle est-elle l'histoire des guerres ? \\
L'histoire : enquête ou science ? \\
L'histoire : science ou récit ? \\
L'historien peut-il se passer du concept de causalité ? \\
L'homme a-t-il une nature ? \\
L'homme de la rue \\
L'homme des droits de l'homme n'est-il qu'une fiction ? \\
L'homme des foules \\
L'homme des sciences de l'homme ? \\
L'homme des sciences humaines \\
L'homme d'État \\
L'homme est-il la mesure de toutes choses ? \\
L'homme est-il objet de science ? \\
L'homme est-il prisonnier du temps ? \\
L'homme est-il un animal politique ? \\
L'homme est-il un être de devoir ? \\
L'homme est-il un être social par nature ? \\
L'homme et la bête \\
L'homme et la machine \\
L'homme et la nature sont-ils commensurables ? \\
L'homme et le citoyen \\
L'homme injuste peut-il être heureux ? \\
L'homme, le citoyen, le soldat \\
L'homme libre est-il un homme seul ? \\
L'homme peut-il changer ? \\
L'honnêteté \\
L'honneur \\
L'horizon \\
L'horreur \\
L'horrible \\
L'hospitalité \\
L'hospitalité a-t-elle un sens politique ? \\
L'hospitalité est-elle un devoir ? \\
L'humiliation \\
L'humilité \\
L'humour \\
L'humour et l'ironie \\
L'hybridation des arts \\
L'hypocrisie \\
L'hypothèse \\
L'hypothèse de l'inconscient \\
Libéral et libertaire \\
Liberté, égalité, fraternité \\
Liberté et habitude \\
Liberté et libération \\
Liberté et nécessité \\
Liberté humaine et liberté divine \\
Liberté réelle, liberté formelle \\
Libertés publiques et culture politique \\
Libre arbitre et liberté \\
L'idéal de l'art \\
L'idéal démonstratif \\
L'idéal de vérité \\
L'idéalisme \\
L'idéaliste \\
L'idéalité \\
L'idéal moral est-il vain ? \\
L'idéal-type \\
L'idée d'anthropologie \\
L'idée de beaux arts \\
L'idée de communauté \\
L'idée de connaissance approchée \\
L'idée de conscience collective \\
L'idée de continuité \\
L'idée de contrat social \\
L'idée de création \\
L'idée de crise \\
L'idée de Dieu \\
L'idée de domination \\
L'idée de forme sociale \\
L'idée de langue universelle \\
L'idée de logique \\
L'idée de logique transcendantale \\
L'idée de logique universelle \\
L'idée de loi logique \\
L'idée de loi naturelle \\
L'idée de mathesis universalis \\
L'idée de morale appliquée \\
L'idée de nation \\
L'idée d'encyclopédie \\
L'idée de norme \\
L'idée de perfection \\
L'idée de république \\
L'idée de rétribution est-elle nécessaire à la morale ? \\
L'idée de révolution \\
L'idée de science expérimentale \\
L'idée de substance \\
L'idée d'éternité \\
L'idée d'exactitude \\
L'idée de « sciences exactes » \\
L'idée d'un commencement absolu \\
L'idée d'une langue universelle \\
L'idée d'une science bien faite \\
L'idée esthétique \\
L'identité \\
L'identité et la différence \\
L'identité personnelle \\
L'identité personnelle est-elle donnée ou construite ? \\
L'identité relève-telle du champ politique ? \\
L'idéologie \\
L'idolâtrie \\
L'ignoble \\
L'ignorance nous excuse-t-elle ? \\
L'illimité \\
L'illusion \\
L'illustration \\
L'image \\
L'imaginaire \\
L'imaginaire et le réel \\
L'imagination dans l'art \\
L'imagination dans les sciences \\
L'imagination esthétique \\
L'imagination et la raison \\
L'imagination nous éloigne-t-elle du réel ? \\
L'imagination politique \\
L'imitation \\
L'imitation a-t-elle une fonction morale ? \\
L'immanence \\
L'immatériel \\
L'immédiat \\
L'immensité \\
L'immortalité de l'âme \\
L'immortalité des œuvres d'art \\
L'immuable \\
L'immutabilité \\
L'impardonnable \\
L'impartialité \\
L'impartialité des historiens \\
L'impensable \\
L'impératif \\
L'imperceptible \\
L'implicite \\
L'importance des détails \\
L'impossible \\
L'imposteur \\
L'imprescriptible \\
L'impression \\
L'imprévisible \\
L'improbable \\
L'improvisation \\
L'improvisation dans l'art \\
L'imprudence \\
L'impuissance \\
L'impuissance de la raison \\
L'impuissance de l'État \\
L'impunité \\
L'inachevé \\
L'inaction \\
L'inapparent \\
L'incarnation \\
L'incertitude est-elle dans les choses ou dans les idées ? \\
L'incommensurabilité \\
L'incommensurable \\
L'incompréhensible \\
L'inconcevable \\
L'inconnu \\
L'inconscience \\
L'inconscient \\
L'inconscient a-t-il une histoire ? \\
L'inconscient collectif \\
L'inconscient de l'art \\
L'inconscient est-il l'animal en nous ? \\
L'inconscient est-il une dimension de la conscience ? \\
L'inconscient n'est-il qu'un défaut de conscience ? \\
L'inconséquence \\
L'incorporel \\
L'incrédulité \\
L'inculture \\
L'indécidable \\
L'indécision \\
L'indéfini \\
L'indépassable \\
L'indépendance \\
L'indésirable \\
L'indétermination \\
L'indéterminé \\
L'indice \\
L'indicible \\
L'indifférence \\
L'indifférence à la politique \\
L'indiscutable \\
L'individu \\
L'individualisme \\
L'individualisme a-t-il sa place en politique ? \\
L'individualisme méthodologique \\
L'individuel et le collectif \\
L'individu et la multitude \\
L'individu et le groupe \\
L'individu face à L'État \\
L'indivisible \\
L'induction \\
L'induction et la déduction \\
L'indulgence \\
L'industrie culturelle \\
L'industrie du beau \\
L'inégalité des chances \\
L'inégalité entre les hommes \\
L'inégalité naturelle \\
L'inertie \\
L'inesthétique \\
L'inexactitude et le savoir scientifique \\
L'infâme \\
L'infamie \\
L'inférence \\
L'infini \\
L'infinité de l'espace \\
L'influence \\
L'information \\
L'informe \\
L'informe et le difforme \\
L'ingratitude \\
L'inhibition \\
L'inhumain \\
L'inimaginable \\
L'inimitié \\
L'inintelligible \\
L'initiation \\
L'injonction \\
L'injustice \\
L'injustifiable \\
L'innocence \\
L'innommable \\
L'inobservable \\
L'inquiétant \\
L'inquiétude \\
L'insatisfaction \\
L'insensé \\
L'insignifiant \\
L'insociable sociabilité \\
L'insouciance \\
L'insoumission \\
L'insoutenable \\
L'inspiration \\
L'instant \\
L'instinct \\
L'institution \\
L'institutionnalisation des conduites \\
L'institution scientifique \\
L'institution scolaire \\
L'instruction est-elle facteur de moralité ? \\
L'instrument \\
L'instrument mathématique en sciences humaines \\
L'instrument scientifique \\
L'insulte \\
L'insurrection \\
L'intangible \\
L'intellectuel \\
L'intelligence \\
L'intelligence de la main \\
L'intelligence de la matière \\
L'intelligence des bêtes \\
L'intelligence des foules \\
L'intelligence du sensible \\
L'intelligence du vivant \\
L'intelligence politique \\
L'intelligible \\
L'intempérance \\
L'intemporel \\
L'intention \\
L'intention morale \\
L'intention morale suffit-elle à constituer la valeur morale de l'action ? \\
L'intentionnalité \\
L'interdit \\
L'intérêt \\
L'intérêt bien compris \\
L'intérêt commun \\
L'intérêt général est-il le bien commun ? \\
L'intérêt peut-il être une valeur morale ? \\
L'intérêt public est-il une illusion ? \\
L'intérieur et l'extérieur \\
L'intériorisation des normes \\
L'intériorité \\
L'intériorité de l'œuvre \\
L'intériorité est-elle un mythe ? \\
L'interprétation de la loi \\
L'interprétation de la nature \\
L'interprétation des œuvres \\
L'interprétation est-elle sans fin ? \\
L'interprétation est-elle un art ? \\
L'interrogation humaine \\
L'intime conviction \\
L'intimité \\
L'intolérable \\
L'intolérance \\
L'intraduisible \\
L'intransigeance \\
L'introspection \\
L'intuition \\
L'intuition a-t-elle une place en logique ? \\
L'intuition en mathématiques \\
L'intuition morale \\
L'inutile \\
L'invention \\
L'invention de soi \\
L'invisibilité \\
L'invisible \\
L'involontaire \\
Lire et écrire \\
L'ironie \\
L'irrationnel \\
L'irrationnel et le politique \\
L'irréel \\
L'irréfutable \\
L'irrégularité \\
L'irréparable \\
L'irreprésentable \\
L'irrésolution \\
L'irresponsabilité \\
L'irréversible \\
L'irrévocable \\
Littérature et réalité \\
L'ivresse \\
L'obéissance \\
L'obéissance à l'autorité \\
L'objectivité \\
L'objectivité de l'art \\
L'objectivité historique \\
L'objet \\
L'objet d'amour \\
L'objet de culte \\
L'objet de la littérature \\
L'objet de l'amour \\
L'objet de la politique \\
L'objet de la psychologie \\
L'objet de la réflexion \\
L'objet de l'art \\
L'objet du désir \\
L'objet du désir en est-il la cause ? \\
L'obligation \\
L'obligation d'échanger \\
L'obligation morale \\
L'obligation morale peut-elle se réduire à une obligation sociale ? \\
L'obscène \\
L'obscénité \\
L'obscurité \\
L'observation \\
L'observation participante \\
L'obsession \\
L'obstacle \\
L'obstacle épistémologique \\
L'occasion \\
L'œil et l'oreille \\
L'œuvre anonyme \\
L'œuvre d'art est-elle l'expression d'une idée ? \\
L'œuvre d'art est-elle toujours destinée à un public ? \\
L'œuvre d'art est-elle une belle apparence ? \\
L'œuvre d'art et le plaisir \\
L'œuvre d'art et sa reproduction \\
L'œuvre d'art et son auteur \\
L'œuvre d'art nous apprend-elle quelque chose ? \\
L'œuvre d'art représente-t-elle quelque chose ? \\
L'œuvre d'art totale \\
L'œuvre de fiction \\
L'œuvre de l'historien \\
L'œuvre du temps \\
L'œuvre et le produit \\
L'œuvre inachevée \\
L'offense \\
Logique et dialectique \\
Logique et existence \\
Logique et logiques \\
Logique et mathématique \\
Logique et mathématiques \\
Logique et métaphysique \\
Logique et méthode \\
Logique et ontologie \\
Logique et psychologie \\
Logique et réalité \\
Logique et vérité \\
Logique générale et logique transcendantale \\
Loi morale et loi politique \\
Loi naturelle et loi politique \\
Lois et règles en logique \\
L'oisiveté \\
Lois naturelles et lois civiles \\
L'oligarchie \\
L'ombre et la lumière \\
L'omniscience \\
L'opinion droite \\
L'opinion du citoyen \\
L'opinion publique \\
L'opinion vraie \\
L'opportunisme \\
L'opposant \\
L'opposition \\
L'oral et l'écrit \\
L'ordinaire est-il ennuyeux ? \\
L'ordre \\
L'ordre des choses \\
L'ordre du monde \\
L'ordre du temps \\
L'ordre établi \\
L'ordre et la mesure \\
L'ordre moral \\
L'ordre politique exclut-il la violence ? \\
L'ordre politique peut-il exclure la violence ? \\
L'ordre public \\
L'ordre social \\
L'organique et le mécanique \\
L'organisation \\
L'organisation du vivant \\
L'orgueil \\
L'orientation \\
L'original et la copie \\
L'originalité \\
L'originalité en art \\
L'origine \\
L'origine de la culpabilité \\
L'origine de la négation \\
L'origine de l'art \\
L'origine des croyances \\
L'origine des langues \\
L'origine des langues est-elle un faux problème ? \\
L'origine des valeurs \\
L'origine des vertus \\
L'origine et le fondement \\
L'ornement \\
L'oubli \\
L'oubli des fautes \\
L'oubli est-il un échec de la mémoire ? \\
L'outil \\
L'outil et la machine \\
L'un \\
L'unanimité est-elle un critère de légitimité ? \\
L'un est le multiple \\
L'un et le multiple \\
L'un et l'être \\
L'unité \\
L'unité dans le beau \\
L'unité de l'art \\
L'unité de la science \\
L'unité de l'œuvre d'art \\
L'unité des contraires \\
L'unité des langues \\
L'unité des sciences \\
L'unité des sciences humaines \\
L'unité des sciences humaines ? \\
L'unité du corps politique \\
L'unité du vivant \\
L'univers \\
L'universel \\
L'universel et le particulier \\
L'universel et le singulier \\
L'univocité de l'étant \\
L'urbanité \\
L'urgence \\
L'usage \\
L'usage des fictions \\
L'usage des généalogies \\
L'usage des mots \\
L'usage des passions \\
L'usage du monde \\
L'usure des mots \\
L'utile et l'agréable \\
L'utilité de la poésie \\
L'utilité de l'art \\
L'utilité des préjugés \\
L'utilité des sciences humaines \\
L'utilité est-elle étrangère à la morale ? \\
L'utilité publique \\
L'utopie \\
L'utopie a-t-elle une signification politique ? \\
L'utopie en politique \\
Machine et organisme \\
Machines et liberté \\
Machines et mémoire \\
Magie et religion \\
Maître et serviteur \\
Maîtriser l'absence \\
Maîtriser le vivant \\
Manger \\
Manquer de jugement \\
Masculin, féminin \\
Mathématiques et réalité \\
Mathématiques pures et mathématiques appliquées \\
Matière et corps \\
Matière et matériaux \\
Ma vraie nature \\
Mécanisme et finalité \\
Mémoire et fiction \\
Mémoire et responsabilité \\
Ménager les apparences \\
Mensonge et politique \\
Mentir \\
Mesurer \\
Métaphysique et histoire \\
Métaphysique et ontologie \\
Métaphysique et religion \\
Métaphysique spéciale, métaphysique générale \\
Métier et vocation \\
Mettre en ordre \\
Microscope et télescope \\
Misère et pauvreté \\
Mœurs, coutumes, lois \\
Moi d'abord \\
Mon corps \\
Mon corps est-il ma propriété ? \\
Mon corps m'appartient-il ? \\
Monde et nature \\
Montrer et démontrer \\
Morale et convention \\
Morale et éducation \\
Morale et histoire \\
Morale et identité \\
Morale et liberté \\
Morale et politique sont-elles indépendantes ? \\
Morale et pratique \\
Morale et prudence \\
Morale et religion \\
Morale et sexualité \\
Morale et société \\
Morale et violence \\
Mourir \\
Mourir dans la dignité \\
Mourir pour des principes \\
Mourir pour la patrie \\
Murs et frontières \\
Musique et bruit \\
Mythe et histoire \\
Mythe et philosophie \\
Mythe et symbole \\
Mythes et idéologies \\
Naître \\
Nation et richesse \\
Nature et fonction du sacrifice \\
Nature et histoire \\
Nature et institutions \\
Naturel et artificiel \\
Naviguer \\
N'avons-nous affaire qu'au réel ? \\
Nécessité fait loi \\
N'échange-t-on que des symboles ? \\
Négation et privation \\
Ne lèse personne \\
Ne pas raconter d'histoires \\
Ne pas savoir ce que l'on fait \\
Ne penser à rien \\
Ne penser qu'à soi \\
Ne prêche-t-on que les convertis ? \\
Ne sait-on rien que par expérience ? \\
Ne sommes-nous véritablement maîtres que de nos pensées ? \\
N'est-on juste que par crainte du châtiment ? \\
Névroses et psychoses \\
N'exprime t-on que ce dont on a conscience ? \\
Ni Dieu ni maître \\
Ni Dieu, ni maître \\
Nier la vérité \\
Nier le monde \\
Nier l'évidence \\
Ni regrets, ni remords \\
Nomade et sédentaire \\
Nommer \\
Normes morales et normes vitales \\
Notre besoin de fictions \\
Notre connaissance du réel se limite-t-elle au savoir scientifique ? \\
Notre corps pense-t-il ? \\
Notre ignorance nous excuse-t-elle ? \\
Notre rapport au monde peut-il n'être que technique ? \\
Nul n'est censé ignorer la loi \\
N'y a-t-il de beauté qu'artistique ? \\
N'y a t-il de bonheur que dans l'instant ? \\
N'y a-t-il de rationalité que scientifique ? \\
N'y a-t-il de science qu'autant qu'il s'y trouve de mathématique ? \\
N'y a-t-il de science que du général ? \\
N'y a-t-il de sens que par le langage ? \\
N'y a-t-il de vérité que scientifique ? \\
N'y a-t-il qu'une substance ? \\
N'y a-t-il qu'un seul monde ? \\
Obéir \\
Obéir, est-ce se soumettre ? \\
Obéissance et servitude \\
Observation et expérimentation \\
Observer \\
Œuvre et événement \\
Ordre et désordre \\
Ordre et liberté \\
Organisme et milieu \\
Origine et commencement \\
Où commence la servitude ? \\
Où commence l'interprétation ? \\
Où est le passé ? \\
Où est le pouvoir ? \\
Où est-on quand on pense ? \\
Où s'arrête l'espace public ? \\
Où sont les relations ? \\
Où suis-je quand je pense ? \\
Où suis-je ? \\
Par-delà beauté et laideur \\
Pardonner et oublier \\
Parfaire \\
Parier \\
Parler, est-ce communiquer ? \\
Parler, est-ce ne pas agir ? \\
Parler pour ne rien dire \\
Par où commencer ? \\
Par quoi un individu se distingue-t-il d'un autre ? \\
Partager les richesses \\
Partager sa vie \\
Partager ses sentiments \\
Passer du fait au droit \\
Pâtir \\
Peindre \\
Peindre d'après nature \\
Peinture et histoire \\
Peinture et réalité \\
Pensée et réalité \\
Penser est-ce calculer ? \\
Penser, est-ce calculer ? \\
Penser, est-ce dire non ? \\
Penser et calculer \\
Penser et parler \\
Penser la matière \\
Penser la technique \\
Penser le réel \\
Penser les sociétés comme des organismes \\
Penser par soi-même \\
Penser requiert-il un corps ? \\
Penser sans corps \\
Perception et aperception \\
Perception et connaissance \\
Perception et création artistique \\
Perception et jugement \\
Perception et mouvement \\
Perception et passivité \\
Perception et souvenir \\
Perception et vérité \\
Percevoir est-ce connaître ? \\
Percevoir, est-ce connaître ? \\
Percevoir, est-ce interpréter ? \\
Percevoir, est-ce juger ? \\
Percevoir, est-ce reconnaître ? \\
Percevoir et imaginer \\
Percevoir et juger \\
Percevoir et sentir \\
Percevoir s'apprend-il ? \\
Perçoit-on les choses comme elles sont ? \\
Perdre la mémoire \\
Perdre la raison \\
Perdre ses habitudes \\
Perdre ses illusions \\
Perdre son âme \\
Persévérer dans son être \\
Persuader \\
Persuader et convaincre \\
Peuple et culture \\
Peuple et masse \\
Peuple et société \\
Peuples et masses \\
Peut-il être moral de tuer ? \\
Peut-il y avoir de bons tyrans ? \\
Peut-il y avoir de la politique sans conflit ? \\
Peut-il y avoir science sans intuition du vrai ? \\
Peut-il y avoir un droit à désobéir ? \\
Peut-il y avoir une philosophie applicable ? \\
Peut-il y avoir une philosophie politique sans Dieu ? \\
Peut-il y avoir une science politique ? \\
Peut-il y avoir une société des nations ? \\
Peut-il y avoir une société sans État ? \\
Peut-il y avoir une vérité en politique ? \\
Peut-on admettre un droit à la révolte ? \\
Peut-on aimer les animaux ? \\
Peut-on aimer l'humanité ? \\
Peut-on apprendre à être heureux ? \\
Peut-on apprendre à être libre ? \\
Peut-on apprendre à vivre ? \\
Peut-on avoir le droit de se révolter ? \\
Peut-on avoir raison tout seul ? \\
Peut-on changer de culture ? \\
Peut-on changer de logique ? \\
Peut-on changer le passé ? \\
Peut-on comparer deux philosophies ? \\
Peut-on concevoir une morale sans sanction ni obligation ? \\
Peut-on concevoir une science qui ne soit pas démonstrative ? \\
Peut-on concevoir une société qui n'aurait plus besoin du droit ? \\
Peut-on concevoir un État mondial ? \\
Peut-on conclure de l'être au devoir-être ? \\
Peut-on connaître autrui ? \\
Peut-on considérer l'art comme un langage ? \\
Peut-on convaincre quelqu'un de la beauté d'une œuvre d'art ? \\
Peut-on critiquer la démocratie ? \\
Peut-on croire ce qu'on veut ? \\
Peut-on croire sans être crédule ? \\
Peut-on croire sans savoir pourquoi ? \\
Peut-on décider de croire ? \\
Peut-on définir la vérité ? \\
Peut-on définir la vie ? \\
Peut-on définir le bien ? \\
Peut-on démontrer qu'on ne rêve pas ? \\
Peut-on désirer ce qu'on possède ? \\
Peut-on dire ce qui n'est pas ? \\
Peut-on dire de la connaissance scientifique qu'elle procède par approximation ? \\
Peut-on dire de l'art qu'il donne un monde en partage ? \\
Peut-on dire d'une image qu'elle parle ? \\
Peut-on dire d'une théorie scientifique qu'elle n'est jamais plus vraie qu'une autre mais seulement plus commode ? \\
Peut-on dire que la science ne nous fait pas connaître les choses mais les rapports entre les choses ? \\
Peut-on dire que rien n'échappe à la technique ? \\
Peut-on dire qu'est vrai ce qui correspond aux faits ? \\
Peut-on dire qu'une théorie physique en contredit une autre ? \\
Peut-on dire toute la vérité ? \\
Peut-on disposer de son corps ? \\
Peut-on distinguer différents types de causes ? \\
Peut-on distinguer le réel de l'imaginaire ? \\
Peut-on distinguer les faits de leurs interprétations ? \\
Peut-on douter de sa propre existence ? \\
Peut-on éclairer la liberté ? \\
Peut-on en appeler à la conscience contre la loi ? \\
Peut-on en savoir trop ? \\
Peut-on entreprendre d'éliminer la métaphysique ? \\
Peut-on établir une hiérarchie des arts ? \\
Peut-on être amoral ? \\
Peut-on être apolitique ? \\
Peut-on être citoyen du monde ? \\
Peut-on être en conflit avec soi-même ? \\
Peut-on être heureux tout seul ? \\
Peut-on être hors de soi ? \\
Peut-on être injuste et heureux ? \\
Peut-on être insensible à l'art ? \\
Peut-on être plus ou moins libre ? \\
Peut-on être sans opinion ? \\
Peut-on être seul ? \\
Peut-on être soi-même en société ? \\
Peut-on être trop sage ? \\
Peut-on expliquer le mal ? \\
Peut-on expliquer le monde par la matière ? \\
Peut-on expliquer une œuvre d'art ? \\
Peut-on faire de l'art avec tout ? \\
Peut-on faire du dialogue un modèle de relation morale ? \\
Peut-on faire l'économie de la notion de forme ? \\
Peut-on faire le mal en vue du bien ? \\
Peut-on faire l'inventaire du monde ? \\
Peut-on fixer des limites à la science ? \\
Peut-on fonder la morale sur la pitié ? \\
Peut-on fonder les droits de l'homme ? \\
Peut-on fonder les mathématiques ? \\
Peut-on fonder une morale sur la nature ? \\
Peut-on gouverner sans lois ? \\
Peut-on hiérarchiser les œuvres d'art ? \\
Peut-on innover en politique ? \\
Peut-on interpréter la nature ? \\
Peut-on jamais aimer son prochain ? \\
Peut-on juger des œuvres d'art sans recourir à l'idée de beauté ? \\
Peut-on justifier la discrimination ? \\
Peut-on justifier la guerre ? \\
Peut-on justifier la raison d'État ? \\
Peut-on justifier le mensonge ? \\
Peut-on lutter contre le destin ? \\
Peut-on lutter contre soi-même ? \\
Peut-on maîtriser l'inconscient ? \\
Peut-on manquer de culture ? \\
Peut-on mesurer les phénomènes sociaux ? \\
Peut-on ne pas être matérialiste ? \\
Peut-on ne pas interpréter ? \\
Peut-on ne pas savoir ce que l'on fait ? \\
Peut-on ne pas savoir ce que l'on veut ? \\
Peut-on ne rien devoir à personne ? \\
Peut-on ne rien vouloir ? \\
Peut-on nier la réalité ? \\
Peut-on nier l'évidence ? \\
Peut-on nier l'existence de la matière ? \\
Peut-on objectiver le psychisme ? \\
Peut-on opposer justice et liberté ? \\
Peut-on opposer morale et technique ? \\
Peut-on opposer nature et culture ? \\
Peut-on parler d'art primitif ? \\
Peut-on parler de corruption des mœurs ? \\
Peut-on parler de droits des animaux ? \\
Peut-on parler des œuvres d'art ? \\
Peut-on parler de vérités métaphysiques ? \\
Peut-on parler de vertu politique ? \\
Peut-on parler d'un droit de la guerre ? \\
Peut-on parler d'un droit de résistance ? \\
Peut-on parler d'une science de l'art ? \\
Peut-on parler d'un savoir poétique ? \\
Peut-on parler d'un travail intellectuel ? \\
Peut-on penser contre l'expérience ? \\
Peut-on penser illogiquement ? \\
Peut-on penser la douleur ? \\
Peut-on penser la matière ? \\
Peut-on penser la mort ? \\
Peut-on penser l'art sans référence au beau ? \\
Peut-on penser le monde sans la technique ? \\
Peut-on penser le temps sans l'espace ? \\
Peut-on penser l'extériorité ? \\
Peut-on penser l'irrationnel ? \\
Peut-on penser l'œuvre d'art sans référence à l'idée de beauté ? \\
Peut-on penser sans concepts ? \\
Peut-on penser sans concept ? \\
Peut-on penser sans règles ? \\
Peut-on penser sans son corps ? \\
Peut-on penser un art sans œuvres ? \\
Peut-on penser une conscience sans objet ? \\
Peut-on penser une métaphysique sans Dieu ? \\
Peut-on penser une volonté diabolique ? \\
Peut-on percevoir sans s'en apercevoir ? \\
Peut-on perdre la raison ? \\
Peut-on perdre sa dignité ? \\
Peut-on perdre sa liberté ? \\
Peut-on perdre son identité ? \\
Peut-on préconiser, dans les sciences humaines et sociales, l'imitation des sciences de la nature ? \\
Peut-on préférer l'ordre à la justice ? \\
Peut-on prouver l'existence ? \\
Peut-on recommencer sa vie ? \\
Peut-on reconnaître un sens à l'histoire sans lui assigner une fin ? \\
Peut-on réduire la pensée à une espèce de comportement ? \\
Peut-on réduire l'esprit à la matière ? \\
Peut-on réduire une métaphysique à une conception du monde ? \\
Peut-on réduire un homme à la somme de ses actes ? \\
Peut-on refuser la loi ? \\
Peut-on régner innocemment ? \\
Peut-on renoncer à ses droits ? \\
Peut-on renoncer au bonheur ? \\
Peut-on représenter l'espace ? \\
Peut-on reprocher à la morale d'être abstraite ? \\
Peut-on rester insensible à la beauté ? \\
Peut-on rester sceptique ? \\
Peut-on restreindre la logique à la pensée formelle ? \\
Peut-on réunir des arts différents dans une même œuvre ? \\
Peut-on revendiquer la paix comme un droit ? \\
Peut-on rire de tout ? \\
Peut-on s'abstenir de penser politiquement ? \\
Peut-on s'accorder sur des vérités morales ? \\
Peut-on savoir ce qui est bien ? \\
Peut-on se connaître soi-même ? \\
Peut-on se désintéresser de la politique ? \\
Peut-on se faire une idée de tout ? \\
Peut-on séparer politique et économie ? \\
Peut-on se passer de chef ? \\
Peut-on se passer de croire ? \\
Peut-on se passer de Dieu ? \\
Peut-on se passer de l'État ? \\
Peut-on se passer de représentants ? \\
Peut-on se passer de spiritualité ? \\
Peut-on se passer des relations ? \\
Peut-on se passer d'un maître ? \\
Peut-on se punir soi-même ? \\
Peut-on se régler sur des exemples en politique ? \\
Peut-on se retirer du monde ? \\
Peut-on souhaiter le gouvernement des meilleurs ? \\
Peut-on suspendre le temps ? \\
Peut-on toujours savoir entièrement ce que l'on dit ? \\
Peut-on tout définir ? \\
Peut-on tout démontrer ? \\
Peut-on tout dire ? \\
Peut-on tout échanger ? \\
Peut-on tout mesurer ? \\
Peut-on tout soumettre à la discussion ? \\
Peut-on transformer le réel ? \\
Peut-on transiger avec les principes ? \\
Peut-on trouver du plaisir à l'ennui ? \\
Peut-on vivre avec les autres ? \\
Peut-on vivre dans le doute ? \\
Peut-on vivre en marge de la société ? \\
Peut-on vivre hors du temps ? \\
Peut-on vivre sans art ? \\
Peut-on vivre sans aucune certitude ? \\
Peut-on vivre sans ressentiment ? \\
Peut-on vivre sans rien espérer ? \\
Peut-on vouloir le mal ? \\
Peut-on vouloir sans désirer ? \\
Philosopher, est-ce apprendre à vivre ? \\
Philosophie et mathématiques \\
Philosophie et métaphysique \\
Photographier le réel \\
Physique et mathématiques \\
Physique et métaphysique \\
Plaider \\
Poésie et philosophie \\
Poésie et vérité \\
Point de vue du créateur et point de vue du spectateur \\
Politique et esthétique \\
Politique et mémoire \\
Politique et parole \\
Politique et participation \\
Politique et passions \\
Politique et propagande \\
Politique et secret \\
Politique et technologie \\
Politique et territoire \\
Politique et trahison \\
Pour être heureux, faut-il renoncer à la perfection ? \\
Pourquoi a-t-on peur de la folie ? \\
Pourquoi avoir recours à la notion d'inconscient ? \\
Pourquoi châtier ? \\
Pourquoi chercher un sens à l'histoire ? \\
Pourquoi commémorer ? \\
Pourquoi conserver les œuvres d'art ? \\
Pourquoi croyons-nous ? \\
Pourquoi définir ? \\
Pourquoi démontrer ce que l'on sait être vrai ? \\
Pourquoi démontrer ? \\
Pourquoi des artifices ? \\
Pourquoi des artistes ? \\
Pourquoi des cérémonies ? \\
Pourquoi des classifications ? \\
Pourquoi des conflits ? \\
Pourquoi des élections ? \\
Pourquoi des exemples ? \\
Pourquoi des géométries ? \\
Pourquoi des historiens ? \\
Pourquoi des hypothèses ? \\
Pourquoi des institutions ? \\
Pourquoi des logiciens ? \\
Pourquoi des lois ? \\
Pourquoi des métaphores ? \\
Pourquoi des modèles ? \\
Pourquoi des musées ? \\
Pourquoi des œuvres d'art ? \\
Pourquoi des poètes ? \\
Pourquoi des rites ? \\
Pourquoi des utopies ? \\
Pourquoi Dieu se soucierait-il des affaires humaines ? \\
Pourquoi dire la vérité ? \\
Pourquoi donner ? \\
Pourquoi écrire ? \\
Pourquoi écrit-on des lois ? \\
Pourquoi écrit-on les lois ? \\
Pourquoi est-il difficile de rectifier une erreur ? \\
Pourquoi être exigeant ? \\
Pourquoi être moral ? \\
Pourquoi exiger la cohérence \\
Pourquoi faire de la politique ? \\
Pourquoi faire de l'histoire ? \\
Pourquoi faire la guerre ? \\
Pourquoi fait-on le mal ? \\
Pourquoi faudrait-il être cohérent ? \\
Pourquoi faut-il être cohérent ? \\
Pourquoi formaliser des arguments ? \\
Pourquoi la musique intéresse-t-elle le philosophe ? \\
Pourquoi la raison recourt-elle à l'hypothèse ? \\
Pourquoi la réalité peut-elle dépasser la fiction ? \\
Pourquoi l'art intéresse-t-il les philosophes ? \\
Pourquoi l'économie est-elle politique ? \\
Pourquoi le droit international est-il imparfait ? \\
Pourquoi les États se font-ils la guerre ? \\
Pourquoi les hommes mentent-ils ? \\
Pourquoi les mathématiques s'appliquent-elles à la réalité ? \\
Pourquoi les œuvres d'art résistent-elles au temps ? \\
Pourquoi l'ethnologue s'intéresse-t-il à la vie urbaine ? \\
Pourquoi lire des romans ? \\
Pourquoi nous souvenons-nous ? \\
Pourquoi obéir aux lois ? \\
Pourquoi obéir ? \\
Pourquoi obéit-on aux lois ? \\
Pourquoi parler de fautes de goût ? \\
Pourquoi parlons-nous ? \\
Pourquoi penser à la mort ? \\
Pourquoi pensons-nous ? \\
Pourquoi plusieurs sciences ? \\
Pourquoi préférer l'original à sa reproduction ? \\
Pourquoi punir ? \\
Pourquoi rechercher le bonheur ? \\
Pourquoi sauver les phénomènes ? \\
Pourquoi se mettre à la place d'autrui ? \\
Pourquoi séparer les pouvoirs ? \\
Pourquoi s'exprimer ? \\
Pourquoi s'inspirer de l'art antique ? \\
Pourquoi sommes-nous déçus par les œuvres d'un faussaire ? \\
Pourquoi sommes-nous moraux ? \\
Pourquoi transformer le monde ? \\
Pourquoi travailler ? \\
Pourquoi un droit du travail ? \\
Pourquoi une instruction publique ? \\
Pourquoi y a-t-il des conflits insolubles ? \\
Pourquoi y a-t-il des religions ? \\
Pourquoi y a-t-il quelque chose plutôt que rien ? \\
Pourquoi y a-t-il plusieurs philosophies ? \\
Pouvoir et autorité \\
Pouvoir et contre-pouvoir \\
Pouvoir et politique \\
Pouvoir et puissance \\
Pouvoir et savoir \\
Pouvoirs et libertés \\
Pouvoir temporel et pouvoir spirituel \\
Pouvons-nous communiquer ce que nous sentons ? \\
Pouvons-nous devenir meilleurs ? \\
Prédicats et relations \\
Prémisses et conclusions \\
Prendre des risques \\
Prendre le pouvoir \\
Prendre les armes \\
Prendre soin \\
Prendre son temps \\
Prendre une décision \\
Prendre une décision politique \\
Présence et absence \\
Présence et représentation \\
Prévoir \\
Prévoir les comportements humains \\
Primitif ou premier ? \\
Principe et cause \\
Principe et commencement \\
Probabilité et explication scientifique \\
Production et création \\
Proposition et jugement \\
Propriétés artistiques, propriétés esthétiques \\
Prose et poésie \\
Prospérité et sécurité \\
Protester \\
Prouver \\
Prouver en métaphysique \\
Prouvez-le ! \\
Providence et destin \\
Psychologie et contrôle des comportements \\
Psychologie et métaphysique \\
Psychologie et neurosciences \\
Publier \\
Puis-je comprendre autrui ? \\
Puis-je douter de ma propre existence ? \\
Puis-je être sûr de bien agir ? \\
Puis-je être universel ? \\
Puis-je ne rien croire ? \\
Pulsion et instinct \\
Pulsions et passions \\
Qualités premières, qualités secondes \\
Quand agit-on ? \\
Quand faut-il désobéir ? \\
Quand faut-il mentir ? \\
Quand la guerre finira-t-elle ? \\
Quand la technique devient-elle art ? \\
Quand pense-t-on ? \\
Quand peut-on se passer de théories ? \\
Quand suis-je en faute ? \\
Quand y a-t-il de l'art ? \\
Quand y a-t-il œuvre ? \\
Quand y a-t-il paysage ? \\
Quand y a-t-il peuple ? \\
Quantification et pensée scientifique \\
Quantité et qualité \\
Qu'a perdu le fou ? \\
Qu'appelle-t-on chef-d'œuvre ? \\
Qu'apprend-on dans les livres ? \\
Qu'apprenons-nous de nos affects ? \\
Qu'a-t-on le droit de pardonner ? \\
Qu'a-t-on le droit d'interpréter ? \\
Qu'avons-nous à apprendre des historiens ? \\
Que cherchons-nous dans le regard des autres ? \\
Que connaissons-nous du vivant ? \\
Que construit le politique ? \\
Que crée l'artiste ? \\
Que déduire d'une contradiction ? \\
Que désirons-nous ? \\
Que devons-nous à l'État ? \\
Que disent les tables de vérité ? \\
Que dit la loi ? \\
Que dois-je à autrui ? \\
Que dois-je à l'État ? \\
Que doit-on aux morts ? \\
Que doit-on faire de ses rêves ? \\
Que fait la police ? \\
Que faut-il craindre ? \\
Que faut-il savoir pour gouverner ? \\
Quel est le but d'une théorie physique ? \\
Quel est le but du travail scientifique ? \\
Quel est le pouvoir de l'art ? \\
Quel est le rôle de la créativité dans les sciences ? \\
Quel est le rôle du médecin ? \\
Quel est le sujet de l'histoire ? \\
Quel est le sujet du devenir ? \\
Quel est l'être de l'illusion ? \\
Quel est l'objet de la métaphysique ? \\
Quel est l'objet de la perception ? \\
Quel est l'objet de la philosophie politique ? \\
Quel est l'objet de la science ? \\
Quel est l'objet de l'échange ? \\
Quel est l'objet de l'esthétique ? \\
Quel est l'objet des sciences politiques ? \\
Quel est l'objet du désir ? \\
Quel être peut être un sujet de droits ? \\
Quelle confiance accorder au langage ? \\
Quelle est la fin de la science ? \\
Quelle est la matière de l'œuvre d'art ? \\
Quelle est la réalité de la matière ? \\
Quelle est la spécificité de la communauté politique ? \\
Quelle idée le fanatique se fait-il de la vérité ? \\
Quelle politique fait-on avec les sciences humaines ? \\
Quelle réalité la science décrit-elle ? \\
Quelles actions permettent d'être heureux ? \\
Quelles règles la technique dicte-t-elle à l'art ? \\
Quelles sont les caractéristiques d'une proposition morale ? \\
Quelle valeur donner à la notion de « corps social » ? \\
Quelle vérité y-a-t-il dans la perception ? \\
Quel réel pour l'art ? \\
Quel rôle attribuer à l'intuition \emph{a priori} dans une philosophie des mathématiques ? \\
Quel rôle la logique joue-t-elle en mathématiques ? \\
Quel rôle l'imagination joue-t-elle en mathématiques ? \\
Quel sens y a-t-il à se demander si les sciences humaines sont vraiment des sciences ? \\
Quels sont les moyens légitimes de la politique ? \\
Que montre l'image ? \\
Que montre un tableau ? \\
Que ne peut-on pas expliquer ? \\
Que nous apporte l'art ? \\
Que nous apprend la poésie ? \\
Que nous apprend la psychanalyse de l'homme ? \\
Que nous apprend la sociologie des sciences ? \\
Que nous apprend le plaisir ? \\
Que nous apprend le toucher ? \\
Que nous apprend l'expérience ? \\
Que nous apprend l'histoire de l'art ? \\
Que nous apprend l'histoire des sciences ? \\
Que nous apprend, sur la politique, l'utopie ? \\
Que nous apprennent les algorithmes sur nos sociétés ? \\
Que nous apprennent les controverses scientifiques ? \\
Que nous apprennent les faits divers ? \\
Que nous apprennent les langues étrangères ? \\
Que nous montrent les natures mortes ? \\
Que peindre ? \\
Que peint le peintre ? \\
Que percevons-nous du monde extérieur ? \\
Que percevons-nous ? \\
Que perçoit-on ? \\
Que perd la pensée en perdant l'écriture ? \\
Que peut expliquer l'histoire ? \\
Que peut la force ? \\
Que peut la politique ? \\
Que peut la raison ? \\
Que peut l'art ? \\
Que peut le politique ? \\
Que peut-on attendre de l'État ? \\
Que peut-on attendre du droit international ? \\
Que peut-on calculer ? \\
Que peut-on comprendre qu'on ne puisse expliquer ? \\
Que peut-on cultiver ? \\
Que peut-on démontrer ? \\
Que peut-on dire de l'être ? \\
Que peut-on partager ? \\
Que peut-on sur autrui ? \\
Que peut un corps ? \\
Que pouvons-nous aujourd'hui apprendre des sciences d'autrefois ? \\
Que produit l'inconscient ? \\
Que sais-je de ma souffrance ? \\
Que sait-on de soi ? \\
Que savons-nous de l'inconscient ? \\
Que serait le meilleur des mondes ? \\
Que serait un art total ? \\
Que serait une démocratie parfaite ? \\
Que signifie apprendre ? \\
Que sondent les sondages d'opinion ? \\
Qu'est-ce le mal radical ? \\
Qu'est-ce qu'avoir conscience de soi ? \\
Qu'est-ce qu'avoir de l'expérience ? \\
Qu'est-ce qu'avoir du goût ? \\
Qu'est-ce qu'avoir du style ? \\
Qu'est-ce que calculer ? \\
Qu'est-ce que démontrer ? \\
Qu'est-ce que déraisonner ? \\
Qu'est-ce que Dieu pour un athée ? \\
Qu'est-ce que discuter ? \\
Qu'est-ce que faire autorité ? \\
Qu'est-ce que faire preuve d'humanité ? \\
Qu'est-ce que gouverner ? \\
Qu'est-ce que guérir ? \\
Qu'est-ce que juger ? \\
Qu'est-ce que la culture générale \\
Qu'est-ce que la psychologie ? \\
Qu'est-ce que l'art contemporain ? \\
Qu'est-ce que le cinéma a changé dans l'idée que l'on se fait du temps ? \\
Qu'est-ce que le désordre ? \\
Qu'est-ce que le moi ? \\
Qu'est-ce que le naturalisme ? \\
Qu'est-ce que l'enfance ? \\
Qu'est-ce que l'harmonie ? \\
Qu'est-ce que lire ? \\
Qu'est-ce que méditer ? \\
Qu'est-ce qu'enquêter ? \\
Qu'est-ce que parler ? \\
Qu'est-ce que perdre son temps ? \\
Qu'est-ce que prendre conscience ? \\
Qu'est-ce que prendre le pouvoir ? \\
Qu'est-ce que raisonner ? \\
Qu'est-ce que résoudre une contradiction ? \\
Qu'est-ce que réussir sa vie ? \\
Qu'est-ce que traduire ? \\
Qu'est-ce qu'être asocial ? \\
Qu'est-ce qu'être chez soi ? \\
Qu'est-ce qu'être comportementaliste ? \\
Qu'est-ce qu'être libéral ? \\
Qu'est-ce qu'être malade ? \\
Qu'est-ce qu'être moderne ? \\
Qu'est-ce qu'être républicain ? \\
Qu'est-ce qu'être sceptique ? \\
Qu'est-ce qu'être seul ? \\
Qu'est-ce qu'être souverain ? \\
Qu'est-ce qu'être un esclave ? \\
Qu'est-ce qu'être un sujet ? \\
Qu'est-ce qu'être vivant ? \\
Qu'est-ce que un individu \\
Qu'est-ce qu'habiter ? \\
Qu'est-ce qui agit ? \\
Qu'est-ce qui apparaît ? \\
Qu'est-ce qui dépend de nous ? \\
Qu'est-ce qui est beau ? \\
Qu'est ce qui est concret ? \\
Qu'est ce qui est contre nature ? \\
Qu'est-ce qui est contre nature ? \\
Qu'est ce qui est culturel ? \\
Qu'est-ce qui est donné ? \\
Qu'est-ce qui est historique ? \\
Qu'est-ce qui est hors-la-loi ? \\
Qu'est-ce qui est impossible ? \\
Qu'est-ce qui est indiscutable ? \\
Qu'est-ce qui est invérifiable ? \\
Qu'est ce qui est irréfutable ? \\
Qu'est-ce qui est mien ? \\
Qu'est-ce qui est moderne ? \\
Qu'est-ce qui est noble ? \\
Qu'est-ce qui est politique ? \\
Qu'est-ce qui est public ? \\
Qu'est-ce qui est réel ? \\
Qu'est-ce qui est respectable ? \\
Qu'est ce qui est sacré ? \\
Qu'est-ce qui est spectaculaire ? \\
Qu'est-ce qui est sublime ? \\
Qu'est-ce qui est vital pour le vivant ? \\
Qu'est ce qui existe ? \\
Qu'est-ce qui existe ? \\
Qu'est-ce qui fait la force des lois ? \\
Qu'est-ce qui fait la justice des lois ? \\
Qu'est-ce qui fait la légitimité d'une autorité politique ? \\
Qu'est-ce qui fait la valeur de l'œuvre d'art ? \\
Qu'est-ce qui fait la valeur d'une croyance ? \\
Qu'est-ce qui fait la valeur d'une œuvre d'art ? \\
Qu'est-ce qui fait le propre d'un corps propre ? \\
Qu'est-ce qui fait l'humanité d'un corps ? \\
Qu'est-ce qui fait l'unité d'une science ? \\
Qu'est-ce qui fait l'unité d'un peuple ? \\
Qu'est-ce qui fait qu'une théorie est vraie ? \\
Qu'est-ce qui fait un peuple ? \\
Qu'est-ce qui fonde la croyance ? \\
Qu'est-ce qu'ignore la science ? \\
Qu'est-ce qui justifie l'hypothèse d'un inconscient ? \\
Qu'est-ce qui ne s'achète pas ? \\
Qu'est-ce qui ne s'échange pas ? \\
Qu'est-ce qui n'est pas démontrable ? \\
Qu'est-ce qui n'est pas politique ? \\
Qu'est-ce qui n'existe pas ? \\
Qu'est-ce qu'interpréter une œuvre d'art ? \\
Qu'est-ce qu'interpréter ? \\
Qu'est-ce qui rend l'objectivité difficile dans les sciences humaines ? \\
Qu'est-ce qui rend vrai un énoncé ? \\
Qu'est-ce qu'obéir ? \\
Qu'est-ce qu'on attend ? \\
Qu'est-ce qu'un abus de pouvoir ? \\
Qu'est-ce qu'un acte libre ? \\
Qu'est-ce qu'un acte moral ? \\
Qu'est-ce qu'un acte symbolique ? \\
Qu'est-ce qu'un acteur ? \\
Qu'est-ce qu'un adversaire en politique ? \\
Qu'est-ce qu'un alter ego \\
Qu'est-ce qu'un animal domestique ? \\
Qu'est-ce qu'un animal ? \\
Qu'est-ce qu'un argument ? \\
Qu'est-ce qu'un artiste ? \\
Qu'est-ce qu'un art moral ? \\
Qu'est-ce qu'un auteur ? \\
Qu'est-ce qu'un axiome ? \\
Qu'est-ce qu'un bon citoyen ? \\
Qu'est-ce qu'un bon conseil ? \\
Qu'est-ce qu'un capital culturel ? \\
Qu'est-ce qu'un cas de conscience ? \\
Qu'est-ce qu'un chef d'œuvre ? \\
Qu'est-ce qu'un chef-d'œuvre ? \\
Qu'est-ce qu'un chef ? \\
Qu'est-ce qu'un citoyen ? \\
Qu'est-ce qu'un civilisé ? \\
Qu'est-ce qu'un concept scientifique ? \\
Qu'est-ce qu'un concept ? \\
Qu'est-ce qu'un conflit politique ? \\
Qu'est-ce qu'un contenu de conscience ? \\
Qu'est-ce qu'un contrat ? \\
Qu'est-ce qu'un contre-pouvoir ? \\
Qu'est-ce qu'un corps social ? \\
Qu'est-ce qu'un coup d'État ? \\
Qu'est-ce qu'un crime contre l'humanité ? \\
Qu'est-ce qu'un crime politique ? \\
Qu'est-ce qu'un déni ? \\
Qu'est-ce qu'un dieu ? \\
Qu'est-ce qu'un Dieu ? \\
Qu'est-ce qu'un document ? \\
Qu'est-ce qu'un dogme ? \\
Qu'est-ce qu'une alternative ? \\
Qu'est-ce qu'une âme ? \\
Qu'est-ce qu'une aporie ? \\
Qu'est-ce qu'une belle démonstration ? \\
Qu'est-ce qu'une bonne loi ? \\
Qu'est-ce qu'une bonne méthode ? \\
Qu'est-ce qu'une catégorie de l'être ? \\
Qu'est-ce qu'une catégorie ? \\
Qu'est-ce qu'une cause ? \\
Qu'est-ce qu'un échange juste ? \\
Qu'est-ce qu'une chose ? \\
Qu'est-ce qu'une collectivité ? \\
Qu'est-ce qu'une communauté politique ? \\
Qu'est-ce qu'une communauté ? \\
Qu'est-ce qu'une conception scientifique du monde ? \\
Qu'est-ce qu'une conduite irrationnelle ? \\
Qu'est-ce qu'une connaissance non scientifique ? \\
Qu'est-ce qu'une constitution ? \\
Qu'est-ce qu'une crise politique ? \\
Qu'est-ce qu'une crise ? \\
Qu'est-ce qu'une croyance rationnelle ? \\
Qu'est-ce qu'une croyance vraie ? \\
Qu'est-ce qu'une culture ? \\
Qu'est-ce qu'une découverte scientifique ? \\
Qu'est-ce qu'une découverte ? \\
Qu'est-ce qu'une discipline savante ? \\
Qu'est-ce qu'une école philosophique ? \\
Qu'est-ce qu'une éducation scientifique ? \\
Qu'est-ce qu'une époque ? \\
Qu'est-ce qu'une erreur ? \\
Qu'est-ce qu'une existence historique ? \\
Qu'est-ce qu'une expérience cruciale ? \\
Qu'est-ce qu'une expérience de pensée ? \\
Qu'est-ce qu'une explication matérialiste ? \\
Qu'est-ce qu'une exposition ? \\
Qu'est-ce qu'une famille ? \\
Qu'est-ce qu'une fonction ? \\
Qu'est-ce qu'une forme ? \\
Qu'est-ce qu'une guerre juste ? \\
Qu'est-ce qu'une histoire vraie ? \\
Qu'est-ce qu'une hypothèse scientifique ? \\
Qu'est-ce qu'une hypothèse ? \\
Qu'est-ce qu'une idée esthétique ? \\
Qu'est-ce qu'une idée incertaine ? \\
Qu'est-ce qu'une idée morale ? \\
Qu'est-ce qu'une idée vraie ? \\
Qu'est-ce qu'une idée ? \\
Qu'est-ce qu'une idéologie ? \\
Qu'est-ce qu'une image ? \\
Qu'est-ce qu'une institution ? \\
Qu'est-ce qu'une langue bien faite ? \\
Qu'est-ce qu'une langue morte ? \\
Qu'est-ce qu'un élément ? \\
Qu'est-ce qu'une limite ? \\
Qu'est-ce qu'une logique sociale ? \\
Qu'est ce qu'une loi scientifique ? \\
Qu'est-ce qu'une loi scientifique ? \\
Qu'est-ce qu'une loi ? \\
Qu'est-ce qu'une machine ? \\
Qu'est-ce qu'une marchandise ? \\
Qu'est-ce qu'une mauvaise interprétation ? \\
Qu'est-ce qu'une méditation métaphysique ? \\
Qu'est-ce qu'une méditation ? \\
Qu'est-ce qu'une mentalité collective ? \\
Qu'est-ce qu'une méthode ? \\
Qu'est-ce qu'un empire ? \\
Qu'est-ce qu'une nation ? \\
Qu'est-ce qu'un enfant ? \\
Qu'est-ce qu'une norme sociale ? \\
Qu'est-ce qu'une norme ? \\
Qu'est-ce qu'une œuvre d'art authentique ? \\
Qu'est-ce qu'une œuvre d'art ? \\
Qu'est-ce qu'une œuvre ratée ? \\
Qu'est-ce qu'une œuvre ? \\
Qu'est-ce qu'une œuvre « géniale » ? \\
Qu'est-ce qu'une patrie ? \\
Qu'est-ce qu'une période en histoire ? \\
Qu'est-ce qu'une personne morale ? \\
Qu'est-ce qu‘une philosophie première ? \\
Qu'est-ce qu'une phrase ? \\
Qu'est-ce qu'une politique sociale ? \\
Qu'est-ce qu'une preuve ? \\
Qu'est-ce qu'une promesse ? \\
Qu'est-ce qu'une propriété essentielle ? \\
Qu'est-ce qu'une psychologie scientifique ? \\
Qu'est-ce qu'une question dénuée de sens ? \\
Qu'est-ce qu'une question métaphysique ? \\
Qu'est-ce qu'une réfutation ? \\
Qu'est ce qu'une religion ? \\
Qu'est-ce qu'une représentation réussie ? \\
Qu'est-ce qu'une révolution politique ? \\
Qu'est-ce qu'une révolution scientifique ? \\
Qu'est-ce qu'une révolution ? \\
Qu'est-ce qu'une science exacte ? \\
Qu'est-ce qu'une science rigoureuse ? \\
Qu'est-ce qu'une situation tragique ? \\
Qu'est-ce qu'une société juste ? \\
Qu'est-ce qu'une société mondialisée ? \\
Qu'est-ce qu'une société ouverte ? \\
Qu'est-ce qu'un esprit faux ? \\
Qu'est-ce qu'une substance ? \\
Qu'est-ce qu'une théorie ? \\
Qu'est-ce qu'une tradition ? \\
Qu'est-ce qu'une tragédie historique ? \\
Qu'est-ce qu'un être cultivé ? \\
Qu'est-ce qu'une valeur ? \\
Qu'est-ce qu'un événement historique ? \\
Qu'est-ce qu'un événement ? \\
Qu'est-ce qu'une vérité scientifique ? \\
Qu'est-ce qu'une vie réussie ? \\
Qu'est-ce qu'une ville ? \\
Qu'est-ce qu'une violence symbolique ? \\
Qu'est-ce qu'une vision du monde ? \\
Qu'est-ce qu'une vision scientifique du monde ? \\
Qu'est-ce qu'une volonté libre ? \\
Qu'est-ce qu'un exemple ? \\
Qu'est-ce qu'une « performance » ? \\
Qu'est-ce qu'un fait de société ? \\
Qu'est-ce qu'un fait historique ? \\
Qu'est-ce qu'un fait moral ? \\
Qu'est ce qu'un fait scientifique ? \\
Qu'est-ce qu'un fait scientifique ? \\
Qu'est-ce qu'un fait social ? \\
Qu'est-ce qu'un faux problème ? \\
Qu'est-ce qu'un faux sentiment ? \\
Qu'est-ce qu'un film ? \\
Qu'est-ce qu'un geste artistique ? \\
Qu'est-ce qu'un geste technique ? \\
Qu'est-ce qu'un gouvernement juste ? \\
Qu'est-ce qu'un gouvernement républicain ? \\
Qu'est-ce qu'un gouvernement ? \\
Qu'est-ce qu'un grand philosophe ? \\
Qu'est-ce qu'un héros ? \\
Qu'est-ce qu'un homme bon ? \\
Qu'est-ce qu'un homme juste ? \\
Qu'est-ce qu'un homme seul ? \\
Qu'est-ce qu'un idéal moral ? \\
Qu'est-ce qu'un individu ? \\
Qu'est-ce qu'un jeu ? \\
Qu'est-ce qu'un laboratoire ? \\
Qu'est-ce qu'un langage technique ? \\
Qu'est-ce qu'un législateur ? \\
Qu'est-ce qu'un lieu commun ? \\
Qu'est-ce qu'un livre ? \\
Qu'est-ce qu'un maître ? \\
Qu'est-ce qu'un marginal ? \\
Qu'est-ce qu'un mécanisme social ? \\
Qu'est-ce qu'un métaphysicien ? \\
Qu'est-ce qu'un mineur ? \\
Qu'est-ce qu'un modèle ? \\
Qu'est-ce qu'un moderne ? \\
Qu'est-ce qu'un monde ? \\
Qu'est-ce qu'un monstre ? \\
Qu'est-ce qu'un monument ? \\
Qu'est-ce qu'un mouvement politique \\
Qu'est-ce qu'un mythe ? \\
Qu'est-ce qu'un nombre ? \\
Qu'est-ce qu'un nom propre ? \\
Qu'est-ce qu'un objet d'art ? \\
Qu'est-ce qu'un objet esthétique ? \\
Qu'est-ce qu'un objet mathématique ? \\
Qu'est-ce qu'un objet métaphysique ? \\
Qu'est-ce qu'un organisme ? \\
Qu'est-ce qu'un original ? \\
Qu'est-ce qu'un outil ? \\
Qu'est ce qu'un paradoxe ? \\
Qu'est-ce qu'un paradoxe ? \\
Qu'est-ce qu'un patrimoine ? \\
Qu'est-ce qu'un pédant ? \\
Qu'est-ce qu'un peuple \\
Qu'est-ce qu'un peuple libre ? \\
Qu'est-ce qu'un peuple ? \\
Qu'est-ce qu'un phénomène ? \\
Qu'est-ce qu'un plaisir pur ? \\
Qu'est-ce qu'un point de vue ? \\
Qu'est-ce qu'un primitif ? \\
Qu'est-ce qu'un prince juste ? \\
Qu'est-ce qu'un principe ? \\
Qu'est-ce qu'un problème éthique ? \\
Qu'est-ce qu'un problème métaphysique ? \\
Qu'est-ce qu'un problème philosophique ? \\
Qu'est-ce qu'un problème politique ? \\
Qu'est-ce qu'un problème scientifique ? \\
Qu'est-ce qu'un problème ? \\
Qu'est-ce qu'un produit culturel ? \\
Qu'est-ce qu'un programme politique ? \\
Qu'est-ce qu'un rapport de force ? \\
Qu'est-ce qu'un rival ? \\
Qu'est-ce qu'un sage ? \\
Qu'est-ce qu'un sentiment moral ? \\
Qu'est-ce qu'un sentiment vrai ? \\
Qu'est-ce qu'un signe ? \\
Qu'est-ce qu'un sophisme ? \\
Qu'est-ce qu'un sophiste ? \\
Qu'est-ce qu'un souvenir ? \\
Qu'est-ce qu'un spécialiste ? \\
Qu'est-ce qu'un spectateur ? \\
Qu'est-ce qu'un style ? \\
Qu'est-ce qu'un symptôme ? \\
Qu'est-ce qu'un système philosophique ? \\
Qu'est-ce qu'un système ? \\
Qu'est-ce qu'un tableau \\
Qu'est-ce qu'un témoin ? \\
Qu'est-ce qu'un tout ? \\
Qu'est-ce qu'un trouble social ? \\
Qu'est-ce qu'un visage ? \\
Qu'est-ce qu'un vrai changement ? \\
Qu'est-ce qu'un « champ artistique » ? \\
Question et problème \\
Qu'est qu'un régime politique ? \\
Que suppose le mouvement ? \\
Que valent les excuses ? \\
Que valent les idées générales ? \\
Que vaut en morale la justification par l'utilité ? \\
Que vaut la distinction entre nature et culture ? \\
Que vaut l'excuse : « C'est plus fort que moi » ? \\
Que vaut l'incertain ? \\
Que vaut une preuve contre un préjugé ? \\
Que veut dire introduire à la métaphysique ? \\
Que veut dire l'expression « aller au fond des choses » ? \\
Que veut dire : « être cultivé » ? \\
Que voit-on dans une image ? \\
Que voit-on dans un miroir ? \\
Qu'exprime une œuvre d'art ? \\
Qui agit ? \\
Qui a le droit de juger ? \\
Qui a une histoire ? \\
Qui a une parole politique ? \\
Qui connaît le mieux mon corps ? \\
Qui croire ? \\
Qui doit faire les lois ? \\
Qui écrit l'histoire ? \\
Qui est autorisé à me dire « tu dois » ? \\
Qui est le maître ? \\
Qui est métaphysicien ? \\
Qui est mon prochain ? \\
Qui est souverain ? \\
Qui fait la loi ? \\
Qui gouverne ? \\
Qui mérite d'être aimé ? \\
Qui parle ? \\
Qui pense ? \\
Qui peut parler ? \\
Qui suis-je ? \\
Qui travaille ? \\
Qu'oppose-t-on à la vérité ? \\
Qu'y a-t-il à comprendre dans une œuvre d'art ? \\
Qu'y a-t-il à comprendre en histoire ? \\
Qu'y a-t-il à l'origine de toutes choses ? \\
Qu'y a-t-il au-delà du réel ? \\
Qu'y a-t-il au fondement de l'objectivité ? \\
Qu'y a-t-il d'universel dans la culture ? \\
Raconter son histoire \\
Raison et langage \\
Raison et politique \\
Raison et révélation \\
Raisonner et calculer \\
Raisonner par l'absurde \\
Rapports de force, rapport de pouvoir \\
Rassembler les hommes, est-ce les unir ? \\
Réalisme et idéalisme \\
Réalité et idéal \\
Rebuts et objets quelconques : une matière pour l'art ? \\
Recevoir \\
Récit et mémoire \\
Reconnaissons-nous le bien comme nous reconnaissons le vrai ? \\
Réforme et révolution \\
Réfutation et confirmation \\
Réfuter \\
Réfuter une théorie \\
Regarder \\
Regarder un tableau \\
Règle et commandement \\
Religion et liberté \\
Religion naturelle et religion révélée \\
Rendre la justice \\
Rendre raison \\
Rendre visible l'invisible \\
Renoncer au passé \\
Rentrer en soi-même \\
Répondre \\
Représentation et illusion \\
Reproduire, copier, imiter \\
Réprouver \\
République et démocratie \\
Résistance et soumission \\
Résister \\
Résister à l'oppression \\
Résister peut-il être un droit ? \\
Rester soi-même \\
Revenir à la nature \\
Rêver \\
Revient-il à l'État d'assurer votre bonheur ? \\
Rien de nouveau sous le soleil \\
Rire \\
Rire et pleurer \\
Rites et cérémonies \\
Rythmes sociaux, rythmes naturels \\
Sait-on toujours ce que l'on fait ? \\
Sait-on toujours ce qu'on veut ? \\
S'aliéner \\
Sans foi ni loi \\
S'approprier une œuvre d'art \\
Sauver les apparences \\
Sauver les phénomènes \\
Savoir ce qu'on dit \\
Savoir démontrer \\
Savoir de quoi on parle \\
Savoir, est-ce pouvoir ? \\
Savoir et liberté \\
Savoir et objectivité dans les sciences \\
Savoir et pouvoir \\
Savoir et rectification \\
Savoir être heureux \\
Savoir et vérifier \\
Savoir pour prévoir \\
Savoir, pouvoir \\
Savoir s'arrêter \\
Savoir vivre \\
Savons-nous ce que nous disons ? \\
Science et complexité \\
Science et démocratie \\
Science et histoire \\
Science et idéologie \\
Science et imagination \\
Science et magie \\
Science et opinion \\
Science et persuasion \\
Science et philosophie \\
Science et réalité \\
Science et religion \\
Science et sagesse \\
Science et société \\
Science et technique \\
Science et technologie \\
Science pure et science appliquée \\
Sciences de la nature et sciences de l'esprit \\
Sciences de la nature et sciences humaines \\
Sciences et philosophie \\
Sciences humaines et déterminisme \\
Sciences humaines et herméneutique \\
Sciences humaines et idéologie \\
Sciences humaines et liberté sont-elles compatibles ? \\
Sciences humaines et littérature \\
Sciences humaines et naturalisme \\
Sciences humaines et nature humaine \\
Sciences humaines et objectivité \\
Sciences humaines et philosophie \\
Sciences humaines, sciences de l'homme \\
Sciences sociales et humanités \\
Se connaître soi-même \\
Se conserver \\
Se cultiver \\
Sécurité et liberté \\
Se défendre \\
Se détacher des sens \\
Se faire justice \\
Se mentir à soi-même \\
Se mettre à la place d'autrui \\
S'ennuyer \\
Sensation et perception \\
Sens et fait \\
Sens et limites de la notion de capital culturel \\
Sens et sensibilité \\
Sens et sensible \\
Sens et structure \\
Sens et vérité \\
Sensible et intelligible \\
Sentir \\
Se parler et s'entendre \\
Se passer de philosophie \\
Se prendre au sérieux \\
Se retirer dans la pensée ? \\
Se retirer du monde \\
Se révolter \\
Servir \\
Servir l'État \\
Se savoir mortel \\
Se taire \\
Seul le présent existe-t-il ? \\
Se voiler la face \\
Sexe et genre \\
S'exercer \\
S'exprimer \\
Sexualité et nature \\
Si Dieu n'existe pas, tout est-il permis ? \\
Signe et symbole \\
Signes, traces et indices \\
Signification et expression \\
Signification et vérité \\
Si l'esprit n'est pas une table rase, qu'est-il ? \\
S'indigner \\
S'indigner, est-ce un devoir ? \\
S'intéresser à l'art \\
Si tu veux, tu peux \\
Société et organisme \\
Soigner \\
Sommes-nous capables d'agir de manière désintéressée ? \\
Sommes-nous conscients de nos mobiles ? \\
Sommes-nous des êtres métaphysiques ? \\
Sommes-nous faits pour la vérité ? \\
Sommes-nous les jouets de l'histoire ? \\
Sommes-nous libres de nos croyances ? \\
Sommes-nous libres de nos pensées ? \\
Sommes-nous libres de nos préférences morales ? \\
Sommes-nous responsables de ce que nous sommes ? \\
Sommes-nous toujours dépendants d'autrui ? \\
Sommes-nous tous contemporains ? \\
Sophismes et paradoxes \\
S'orienter \\
Sortir de soi \\
Soutenir une thèse \\
Structure et événement \\
Subir \\
Substance et sujet \\
Suffit-il de bien juger pour bien faire ? \\
Suffit-il de faire son devoir ? \\
Suffit-il d'être juste ? \\
Suffit-il de vouloir pour bien faire ? \\
Suffit-il, pour croire, de le vouloir ? \\
Suffit-il pour être juste d'obéir aux lois et aux coutumes de son pays ? \\
Suis-je aussi ce que j'aurais pu être ? \\
Suis-je maître de ma conscience ? \\
Suis-je ma mémoire ? \\
Suis-je mon corps ? \\
Sujet et citoyen \\
Sujet et prédicat \\
Sujet et substance \\
Sur quoi fonder la légitimité de la loi ? \\
Sur quoi fonder la propriété ? \\
Sur quoi fonder la société ? \\
Sur quoi fonder l'autorité ? \\
Sur quoi fonder le droit de punir ? \\
Sur quoi reposent nos certitudes ? \\
Sur quoi se fonde la connaissance scientifique ? \\
Surveillance et discipline \\
Surveiller son comportement \\
Survivre \\
Suspendre son assentiment \\
Suspendre son jugement \\
Syllogisme et démonstration \\
Système et structure \\
Tantôt je pense, tantôt je suis \\
Tautologie et contradiction \\
Technique et pratiques scientifiques \\
Témoigner \\
Temps et conscience \\
Temps et éternité \\
Temps et réalité \\
Tenir parole \\
Thème et variations \\
Théorie et modélisation \\
Tous les droits sont-ils formels ? \\
Tous les hommes désirent-ils connaître ? \\
Tous les hommes désirent-ils être heureux ? \\
Tout art est-il poésie ? \\
Tout a-t-il une raison d'être ? \\
Tout a-t-il un sens ? \\
Tout définir, tout démontrer \\
Tout devoir est-il l'envers d'un droit ? \\
Toute action politique est-elle collective ? \\
Toute chose a-t-elle une essence ? \\
Toute communauté est-elle politique ? \\
Toute connaissance autre que scientifique doit-elle être considérée comme une illusion ? \\
Toute connaissance est-elle historique ? \\
Toute conscience n'est-elle pas implicitement morale ? \\
Toute existence est-elle indémontrable ? \\
Toute expérience appelle-t-elle une interprétation ? \\
Toute expression est-elle métaphorique ? \\
Toute hiérarchie est-elle inégalitaire ? \\
Toute métaphysique implique-t-elle une transcendance ? \\
Tout énoncé est-il nécessairement vrai ou faux ? \\
Toute origine est-elle mythique ? \\
Toute passion fait-elle souffrir ? \\
Toute peur est-elle irrationnelle ? \\
Toute philosophie est-elle systématique ? \\
Toute philosophie implique-t-elle une politique ? \\
Toutes les choses sont-elles singulières ? \\
Toutes les croyances se valent-elles ? \\
Toutes les vérités  scientifiques sont-elles révisables ? \\
Tout est corps \\
Tout est-il connaissable ? \\
Tout est-il mesurable ? \\
Tout est-il nécessaire ? \\
Tout est-il politique ? \\
Tout est-il relatif ? \\
Tout est permis \\
Tout être est-il dans l'espace ? \\
Toute vérité est-elle vérifiable ? \\
Toute violence est-elle contre nature ? \\
Tout ou rien \\
Tout peut-il être objet de jugement esthétique ? \\
Tout peut-il n'être qu'apparence ? \\
Tout peut-il se vendre ? \\
Tout pouvoir est-il oppresseur ? \\
Tout pouvoir est-il politique ? \\
Tout pouvoir n'est-il pas abusif ? \\
Tout savoir \\
Tout savoir est-il fondé sur un savoir premier ? \\
Tout savoir est-il pouvoir ? \\
Tout savoir est-il transmissible ? \\
Tout travail mérite salaire \\
Tradition et innovation \\
Tradition et raison \\
Traduire \\
Trahir \\
Traiter autrui comme une chose \\
Traiter des faits humains comme des choses, est-ce considérer l'homme comme une chose ? \\
Traiter les faits humains comme des choses, est-ce réduire les hommes à des choses ? \\
Transcendance et altérité \\
Travail et subjectivité \\
Travailler par plaisir, est-ce encore travailler ? \\
Travail manuel et travail intellectuel \\
Travail manuel, travail intellectuel \\
Tricher \\
Tuer le temps \\
Un acte désintéressé est-il possible ? \\
Un acte libre est-il un acte imprévisible ? \\
Un art sans sublimation est-il possible ? \\
Un bien peut-il sortir d'un mal ? \\
Un contrat peut-il être social ? \\
Un Dieu unique ? \\
Une action vertueuse se reconnaît-elle à sa difficulté ? \\
Une cause peut-elle être libre ? \\
Une croyance infondée est-elle illégitime ? \\
Une culture de masse est-elle une culture ? \\
Une décision politique peut-elle être juste ? \\
Une éducation esthétique est-elle possible ? \\
Une éthique sceptique est-elle possible ? \\
Une existence se démontre-t-elle ? \\
Une explication peut-elle être réductrice ? \\
Une fiction peut-elle être vraie ? \\
Une foi rationnelle \\
Une guerre peut-elle être juste ? \\
Une idée peut-elle être fausse ? \\
Une interprétation est-elle nécessairement subjective ? \\
Une ligne de conduite peut-elle tenir lieu de morale ? \\
Une logique non-formelle est-elle possible ? \\
Une loi n'est-elle qu'une règle ? \\
Une machine peut-elle penser ? \\
Une machine pourrait-elle penser ? \\
Une métaphysique athée est-elle possible ? \\
Une métaphysique peut-elle être sceptique ? \\
Une morale du plaisir est-elle concevable ? \\
Une morale peut-elle être dépassée ? \\
Une morale peut-elle être provisoire ? \\
Une morale peut-elle prétendre à l'universalité ? \\
Une morale sans Dieu \\
Une morale sans obligation est-elle possible ? \\
Une œuvre d'art doit-elle avoir un sens ? \\
Une œuvre d'art est-elle une marchandise ? \\
Une œuvre d'art peut-elle être immorale ? \\
Une œuvre d'art peut-elle être laide ? \\
Une œuvre d'art s'explique-t-elle à partir de ses influences ? \\
Une œuvre est-elle toujours de son temps ? \\
Une perception peut-elle être illusoire ? \\
Une philosophie de l'amour est-elle possible ? \\
Une politique peut-elle se réclamer de la vie ? \\
Une religion civile est-elle possible ? \\
Une religion peut-elle être fausse ? \\
Une religion peut-elle être rationnelle ? \\
Une science de la culture est-elle possible ? \\
Une science de la morale est-elle possible ? \\
Une science des symboles est-elle possible ? \\
Une société juste est-elle une société sans conflits ? \\
Une société n'est-elle qu'un ensemble d'individus ? \\
Une société sans conflit est-elle possible ? \\
Une société sans État est-elle une société sans politique ? \\
Une société sans religion est-elle possible ? \\
Un État mondial ? \\
Un État peut-il être trop étendu ? \\
Une théorie scientifique peut-elle devenir fausse ? \\
Une théorie scientifique peut-elle être ramenée à des propositions empiriques élémentaires ? \\
Une volonté peut-elle être générale ? \\
Un homme n'est-il que la somme de ses actes ? \\
Universalité et nécessité dans les sciences \\
Univocité et équivocité \\
Un jugement de goût est-il culturel ? \\
Un moment d'éternité \\
Un monde sans beauté \\
Un monde sans nature est-il pensable ? \\
Un pouvoir a-t-il besoin d'être légitime ? \\
Un problème scientifique peut-il être insoluble ? \\
Un sceptique peut-il être logicien ? \\
Un tableau peut-il être une dénonciation ? \\
Un vice, est-ce un manque ? \\
Utopie et tradition \\
Valeur artistique, valeur esthétique \\
Vanité des vanités \\
Vérité et certitude \\
Vérité et fiction \\
Vérité et histoire \\
Vérité et réalité \\
Vérité et sensibilité \\
Vérité et signification \\
Vérités de fait et vérités de raison \\
Vérités mathématiques, vérités philosophiques \\
Vertu et habitude \\
Vices privés, vertus publiques \\
Vie et existence \\
Vie et volonté \\
Vieillir \\
Violence et discours \\
Violence et histoire \\
Violence et politique \\
Vitalisme et mécanique \\
Vit-on au présent ? \\
Vivons-nous tous dans le même monde ? \\
Vivre au présent \\
Vivre comme si nous ne devions pas mourir \\
Vivre, est-ce interpréter ? \\
Vivre, est-ce lutter contre la mort ? \\
Vivre sans morale \\
Vivre sans religion, est-ce vivre sans espoir ? \\
Vivre sa vie \\
Vivre sous la conduite de la raison \\
Vivre vertueusement \\
Voir \\
Voir et entendre \\
Voir et toucher \\
Voir le meilleur et faire le pire \\
Voir le meilleur, faire le pire \\
Vouloir ce que l'on peut \\
Vouloir le bien \\
Vouloir l'égalité \\
Vouloir le mal \\
Voyager \\
Vulgariser la science ? \\
Y a-t-il continuité entre l'expérience et la science ? \\
Y a-t-il continuité ou discontinuité entre la pensée mythique et la science ? \\
Y a-t-il de fausses religions ? \\
Y a-t-il de la raison dans la perception ? \\
Y a-t-il de l'impensable ? \\
Y a-t-il de l'inconcevable ? \\
Y a-t-il de l'indémontrable ? \\
Y a-t-il de l'inexprimable ? \\
Y a-t-il de l'irréparable ? \\
Y a-t-il des actes désintéressés ? \\
Y a-t-il des actes moralement indifférents ? \\
Y a-t-il des actions désintéressées ? \\
Y a-t-il des arts mineurs ? \\
Y a-t-il des canons de la beauté ? \\
Y a-t-il des certitudes historiques ? \\
Y a-t-il des choses qui échappent au droit ? \\
Y a-t-il des compétences politiques ? \\
Y a-t-il des critères du beau ? \\
Y a-t-il des croyances démocratiques ? \\
Y a-t-il des degrés dans la certitude ? \\
Y a-t-il des degrés de réalité ? \\
Y a-t-il des démonstrations en philosophie ? \\
Y a-t-il des devoirs envers soi-même ? \\
Y a-t-il des erreurs en politique ? \\
Y a-t-il des êtres mathématiques ? \\
Y a-t-il des expériences absolument certaines ? \\
Y a-t-il des expériences cruciales ? \\
Y a-t-il des faits moraux ? \\
Y a-t-il des faits sans essence ? \\
Y a-t-il des fins de la nature ? \\
Y a-t-il des fins dernières ? \\
Y a-t-il des fondements naturels à l'ordre social ? \\
Y a-t-il des guerres justes ? \\
Y a-t-il des héritages philosophiques ? \\
Y a-t-il des inégalités justes ? \\
Y a-t-il des leçons de l'histoire ? \\
Y a-t-il des limites à l'exprimable ? \\
Y a-t-il des limites au droit ? \\
Y a-t-il des limites proprement morales à la discussion ? \\
Y a-t-il des lois du hasard ? \\
Y a-t-il des lois du vivant ? \\
Y a-t-il des lois en histoire ? \\
Y a-t-il des lois injustes ? \\
Y a-t-il des lois morales ? \\
Y a-t-il des lois non écrites ? \\
Y a-t-il des mentalités collectives ? \\
Y a-t-il des modèles en morale ? \\
Y a-t-il des normes naturelles ? \\
Y a-t-il des passions collectives ? \\
Y a-t-il des pathologies sociales ? \\
Y a-t-il des pensées folles ? \\
Y a-t-il des pensées inconscientes ? \\
Y a-t-il des peuples sans histoire ? \\
Y a-t-il des preuves d'amour ? \\
Y a-t-il des propriétés singulières ? \\
Y a-t-il des règles de l'art ? \\
Y a-t-il des révolutions en art ? \\
Y a-t-il des révolutions scientifiques ? \\
Y a-t-il des secrets de la nature ? \\
Y a-t-il des sociétés sans État ? \\
Y a-t-il des sociétés sans histoire ? \\
Y a-t-il des substances incorporelles ? \\
Y a-t-il des vérités philosophiques ? \\
Y a-t-il des vérités sans preuve ? \\
Y a-t-il des vertus mineures ? \\
Y a-t-il des violences justifiées ? \\
Y a-t-il des violences légitimes ? \\
Y a-t-il différentes manières de connaître ? \\
Y a-t-il du non-être ? \\
Y a-t-il du sacré dans la nature ? \\
Y a-t-il du synthétique \emph{a priori} ? \\
Y a-t-il encore des mythologies ? \\
Y a-t-il encore une sphère privée ? \\
Y a-t-il lieu d'opposer matière et esprit ? \\
Y a-t-il place pour l'idée de vérité en morale ? \\
Y a-t-il plusieurs manières de définir ? \\
Y a-t-il plusieurs nécessités ? \\
Y a-t-il un art de gouverner ? \\
Y a-t-il un art d'inventer ? \\
Y a-t-il un au-delà de la vérité ? \\
Y a-t-il un auteur de l'histoire ? \\
Y a-t-il un autre monde ? \\
Y a-t-il un beau idéal ? \\
Y a-t-il un besoin métaphysique ? \\
Y a-t-il un bien commun ? \\
Y a-t-il un bien plus précieux que la paix ? \\
Y a-t-il un canon de la beauté ? \\
Y a-t-il un critère du vrai ? \\
Y a-t-il un devoir d'être heureux ? \\
Y a-t-il un droit à la différence ? \\
Y a-t-il un droit de mourir ? \\
Y a-t-il un droit de résistance ? \\
Y a-t-il un droit du plus fort ? \\
Y a-t-il un droit international ? \\
Y a-t-il une argumentation métaphysique ? \\
Y a-t-il une beauté morale ? \\
Y a-t-il une beauté naturelle ? \\
Y a-t-il une bonne imitation ? \\
Y a-t-il une causalité historique ? \\
Y a-t-il une compétence en politique ? \\
Y a-t-il une connaissance du singulier ? \\
Y a-t-il une connaissance métaphysique ? \\
Y a-t-il une connaissance sensible ? \\
Y a-t-il une correspondance des arts ? \\
Y a-t-il une éthique de l'authenticité ? \\
Y a-t-il une expérience de la liberté ? \\
Y a-t-il une expérience de l'éternité ? \\
Y a-t-il une expérience du néant ? \\
Y a-t-il une fin dernière ? \\
Y a-t-il une force du droit ? \\
Y a-t-il une forme morale de fanatisme ? \\
Y a-t-il une hiérarchie des êtres ? \\
Y a-t-il une hiérarchie des sciences ? \\
Y a-t-il une histoire de la vérité ? \\
Y a-t-il une intentionnalité collective ? \\
Y a-t-il une justice sans morale ? \\
Y a-t-il une logique de la découverte scientifique ? \\
Y a-t-il une logique de la découverte ? \\
Y a-t-il une logique des événements historiques ? \\
Y a-t-il une métaphysique de l'amour ? \\
Y a-t-il une opinion publique mondiale ? \\
Y a-t-il une ou plusieurs philosophies ? \\
Y a-t-il une philosophie de la philosophie ? \\
Y a-t-il une philosophie première ? \\
Y a-t-il une place pour la morale dans l'économie ? \\
Y a-t-il une réalité du hasard ? \\
Y a-t-il une science de la vie mentale ? \\
Y a-t-il une science de l'être ? \\
Y a-t-il une science de l'homme ? \\
Y a-t-il une science de l'individuel ? \\
Y a-t-il une science des principes ? \\
Y a-t-il une science du qualitatif ? \\
Y a-t-il une science politique ? \\
Y a-t-il une sensibilité esthétique ? \\
Y a-t-il une spécificité de la délibération politique ? \\
Y a-t-il une spécificité des sciences humaines ? \\
Y a-t-il un esprit scientifique ? \\
Y a-t-il une technique de la nature ? \\
Y a-t-il une unité de la science ? \\
Y a-t-il une unité en psychologie ? \\
Y a-t-il une universalité des mathématiques ? \\
Y a-t-il une vérité des sentiments ? \\
Y a-t-il une vérité des symboles ? \\
Y a-t-il une vérité du sentiment ? \\
Y a-t-il une vérité en histoire ? \\
Y a-t-il une vérité philosophique ? \\
Y a-t-il une vie de l'esprit ? \\
Y a-t-il un inconscient collectif ? \\
Y a-t-il un inconscient psychique ? \\
Y a-t-il un inconscient social ? \\
Y a-t-il un langage commun ? \\
Y a-t-il un langage de l'inconscient ? \\
Y a-t-il un langage du corps ? \\
Y a-t-il un mal absolu ? \\
Y a-t-il un monde extérieur ? \\
Y a-t-il un ordre des choses ? \\
Y a-t-il un principe du mal ? \\
Y a-t-il un progrès en art ? \\
Y a-t-il un progrès moral ? \\
Y a-t-il un savoir du contingent ? \\
Y a-t-il un savoir du corps ? \\
Y a-t-il un savoir du politique ? \\
Y a-t-il un savoir immédiat ? \\
Y a-t-il un sens du beau ? \\
Y a-t-il un sens moral ? \\
Y a-t-il un temps des choses ? \\
Y a-t-il un temps pour tout ? \\
Y a-t-il un usage moral des passions ? \\
« Aime, et fais ce que tu veux » \\
« Aimez vos ennemis » \\
« Après moi, le déluge » \\
« À l'impossible, nul n'est tenu » \\
« Bienheureuse faute » \\
« C'est humain » \\
« C'est la vie » \\
« Comment peut-on être persan ? » \\
« De la musique avant toute chose » \\
« Deviens qui tu es » \\
« Dieu est mort » \\
« Être contre » \\
« Expliquer les faits sociaux par des faits sociaux » \\
« Il faudrait rester des années entières pour contempler une telle œuvre » \\
« Je mens » \\
« Je n'ai pas voulu cela » \\
« Je ne voulais pas cela » : en quoi les sciences humaines permettent-elles de comprendre cette excuse ? \\
« La logique » ou bien « les logiques » ? \\
« La vie des formes » \\
« La vie est un songe » \\
« La vraie morale se moque de la morale » \\
« L'enfer est pavé de bonnes intentions » \\
« L'histoire jugera » \\
« L'homme est la mesure de toute chose » \\
« Malheur aux vaincus » \\
« Ne fais pas à autrui ce que tu ne voudrais pas qu'on te fasse » \\
« Œil pour œil, dent pour dent » \\
« Petites causes, grands effets » \\
« Prendre ses désirs pour des réalités » \\
« Que nul n'entre ici s'il n'est géomètre » \\
« Rien n'est sans raison » \\
« Rien n'est simple » \\
« Sans titre » \\
« Sauver les phénomènes » \\
« Toute peine mérite salaire » \\
« Trop beau pour être vrai » \\
« Tu ne tueras point » \\


\subsection{Agrégation externe}
\label{sec-2-2}

\noindent
Abolir la propriété \\
À chacun son dû \\
Acteurs sociaux et usages sociaux \\
Action et événement \\
Affirmer et nier \\
Agir \\
Agir moralement, est-ce lutter contre ses idées ? \\
Ai-je une âme ? \\
Aimer la nature \\
Aimer la vie \\
Aimer les lois \\
Aimer ses proches \\
Aimer son prochain comme soi-même \\
Aimer une œuvre d'art \\
À l'impossible nul n'est tenu \\
Amitié et société \\
Analyse et synthèse \\
Analyser les mœurs \\
Animal politique ou social ? \\
Anomalie et anomie \\
Anthropologie et ontologie \\
Anthropologie et politique \\
Apparaître \\
Apparence et réalité \\
Apprend-on à penser ? \\
Apprend-on à voir ? \\
Apprendre à gouverner \\
Apprendre à penser \\
Apprendre à vivre \\
Apprendre à voir \\
Apprendre s'apprend-il ? \\
Apprentissage et conditionnement \\
Après-coup \\
\emph{A priori} et \emph{a posteriori} \\
À quelles conditions une démarche est-elle scientifique ? \\
À quelles conditions une hypothèse est-elle scientifique ? \\
À quelles conditions un énoncé est-il doué de sens ? \\
À quoi bon discuter ? \\
À quoi bon les sciences humaines et sociales ? \\
À quoi bon ? \\
À quoi faut-il renoncer ? \\
À quoi juger l'action d'un gouvernement ? \\
À quoi la conscience nous donne-t-elle accès ? \\
À quoi la logique peut-elle servir dans les sciences ? \\
À quoi reconnaît-on la vérité ? \\
À quoi reconnaît-on qu'une théorie est scientifique ? \\
À quoi sert la dialectique ? \\
À quoi sert la négation ? \\
À quoi sert l'écriture ? \\
À quoi servent les sciences ? \\
À quoi tient la fermeté du vouloir ? \\
À quoi tient la vérité d'une interprétation ? \\
Argumenter \\
Art et authenticité \\
Art et critique \\
Art et divertissement \\
Art et émotion \\
Art et finitude \\
Art et folie \\
Art et forme \\
Art et illusion \\
Art et image \\
Art et interdit \\
Art et jeu \\
Art et marchandise \\
Art et mélancolie \\
Art et mémoire \\
Art et métaphysique \\
Art et politique \\
Art et propagande \\
Art et religieux \\
Art et religion \\
Art et représentation \\
Art et technique \\
Art et transgression \\
Art et vérité \\
Arts de l'espace et arts du temps \\
A-t-on des droits contre l'État ? \\
A-t-on des raisons de croire ce qu'on croit ? \\
A-t-on des raisons de croire ? \\
Attente et espérance \\
Au-delà \\
Au-delà de la nature ? \\
Aussitôt dit, aussitôt fait \\
Autorité et pouvoir \\
Avoir de l'autorité \\
Avoir de l'esprit \\
Avoir de l'expérience \\
Avoir des ennemis \\
Avoir des principes \\
Avoir du goût \\
Avoir du style \\
Avoir mauvaise conscience \\
Avoir peur \\
Avoir un corps \\
Avoir une bonne mémoire \\
Avoir une idée \\
Avons-nous à apprendre des images ? \\
Avons-nous besoin de métaphysique ? \\
Avons-nous besoin de partis politiques ? \\
Avons-nous besoin de spectacles ? \\
Avons-nous besoin de traditions ? \\
Avons-nous besoin d'experts en matière d'art ? \\
Avons-nous besoin d'un libre arbitre ? \\
Avons-nous des devoirs envers les animaux ? \\
Avons-nous des devoirs envers les morts ? \\
Avons-nous des devoirs envers le vivant ? \\
Avons-nous une identité ? \\
Avons-nous une responsabilité envers le passé ? \\
Avons-nous un monde commun ? \\
À chacun ses goûts \\
À quelles conditions est-il acceptable de travailler ? \\
À quelles conditions une explication est-elle scientifique ? \\
À quelles conditions une hypothèse est-elle scientifique ? \\
À quelles conditions une pensée est-elle libre ? \\
À qui la faute ? \\
À quoi faut-il être fidèle ? \\
À quoi reconnaît-on qu'une politique est juste ? \\
À quoi reconnaît-on un bon gouvernement ? \\
À quoi reconnaît-on une œuvre d'art ? \\
À quoi sert la logique ? \\
À quoi sert la notion de contrat social ? \\
À quoi sert la notion d'état de nature ? \\
À quoi sert l'État ? \\
À quoi sert un exemple ? \\
À quoi servent les doctrines morales ? \\
À quoi servent les élections ? \\
À quoi servent les règles ? \\
À quoi servent les statistiques ? \\
À quoi servent les utopies ? \\
À science nouvelle, nouvelle philosophie ? \\
Bâtir un monde \\
Beauté et vérité \\
Beauté naturelle et beauté artistique \\
Beauté réelle, beauté idéale \\
Bien jouer son rôle \\
Bien juger \\
Bien parler \\
Calculer \\
Calculer et penser \\
Cartographier \\
Castes et classes \\
Catégories de l'être, catégories de langue \\
Catégories logiques et catégories linguistiques \\
Causes et motivations \\
Causes premières et causes secondes \\
Ce que la morale autorise, l'État peut-il légitimement l'interdire ? \\
Ce que sait le poète \\
Ce qui dépend de moi \\
Ce qui est faux est-il dénué de sens ? \\
Ce qui est subjectif est-il arbitraire ? \\
Ce qui fut et ce qui sera \\
Ce qu'il y a \\
Ce qui n'est pas réel est-il impossible ? \\
Ce qui passe et ce qui demeure \\
Ce qu'on ne peut pas vendre \\
Certaines œuvres d'art ont-elles plus de valeur que d'autres ? \\
Certitude et vérité \\
Cesser d'espérer \\
C'est pour ton bien \\
C'est trop beau pour être vrai ! \\
Ceux qui savent doivent-ils gouverner ? \\
Changer \\
Changer ses désirs plutôt que l'ordre du monde \\
Chaque science porte-t-elle une métaphysique qui lui est propre ? \\
Chercher son intérêt, est-ce être immoral ? \\
Choisir \\
Choisir ses souvenirs ? \\
Choisit-on ses souvenirs ? \\
Choisit-on son corps ? \\
Chose et objet \\
Cinéma et réalité \\
Cité juste ou citoyen juste ? \\
Citoyen et soldat \\
Classer \\
Classer et ordonner \\
Classes et histoire \\
Collectionner \\
Commémorer \\
Commencer \\
Commencer en philosophie \\
Comment assumer les conséquences de ses actes ? \\
Comment bien vivre ? \\
Comment comprendre une croyance qu'on ne partage pas ? \\
Comment décider, sinon à la majorité ? \\
Comment définir la raison ? \\
Comment définir la signification \\
Comment devient-on artiste ? \\
Comment devient-on raisonnable ? \\
Comment juger de la politique ? \\
Comment juger d'une œuvre d'art ? \\
Comment justifier l'autonomie des sciences de la vie ? \\
Comment les sociétés changent-elles ? \\
Comment penser la diversité des langues ? \\
Comment penser un pouvoir qui ne corrompe pas ? \\
Comment peut-on choisir entre différentes hypothèses ? \\
Comment peut-on être sceptique ? \\
Comment reconnaît-on une œuvre d'art ? \\
Comment sait-on qu'on se comprend ? \\
Comment traiter les animaux ? \\
Comment trancher une controverse ? \\
Comment vivre ensemble ? \\
Comme on dit \\
Communauté et société \\
Communiquer \\
Communiquer et enseigner \\
Compatir \\
Compétence et autorité \\
Composer avec les circonstances \\
Composition et construction \\
Compter sur soi \\
Concept et image \\
Concept et métaphore \\
Conception et perception \\
Concevoir et juger \\
Conduire sa vie \\
Conduire ses pensées \\
Connaissance commune et connaissance scientifique \\
Connaissance, croyance, conjecture \\
Connaissance du futur et connaissance du passé \\
Connaissance et croyance \\
Connaissance et expérience \\
Connaissance historique et action politique \\
Connaître, est-ce connaître par les causes ? \\
Connaître et comprendre \\
Connaître et penser \\
Connaître par les causes \\
Connaître ses limites \\
Conscience et mémoire \\
Conseiller le prince \\
Consensus et conflit \\
Conservatisme et tradition \\
Considère-t-on jamais le temps en lui-même ? \\
Consistance et précarité \\
Constitution et lois \\
Consumérisme et démocratie \\
Contemplation et distraction \\
Contempler \\
Contingence et nécessité \\
Contradiction et opposition \\
Contrainte et désobéissance \\
Convention et observation \\
Conviction et responsabilité \\
Corps et identité \\
Correspondre \\
Création et production \\
Créativité et contrainte \\
Créer \\
Crime et châtiment \\
Crise et progrès \\
Critiquer \\
Critiquer la démocratie \\
Croire aux fictions \\
Croire en Dieu \\
Croire, est-ce être faible ? \\
Croire et savoir \\
Croire savoir \\
Croyance et probabilité \\
Cultes et rituels \\
Cultivons notre jardin \\
Culture et civilisation \\
Culture et conscience \\
Décider \\
Décomposer les choses \\
Découverte et invention \\
Découverte et invention dans les sciences \\
Découvrir \\
Décrire \\
Décrire, est-ce déjà expliquer ? \\
Déduction et expérience \\
Défendre son honneur \\
Définir, est-ce déterminer l'essence ? \\
Définir l'art : à quoi bon ? \\
Définir la vérité, est-ce la connaître ? \\
Définition et démonstration \\
Définition nominale et définition réelle \\
Définitions, axiomes, postulats \\
Déjouer \\
Délibérer, est-ce être dans l'incertitude ? \\
De l'utilité des voyages \\
Démêler le vrai du faux \\
Démériter \\
Démocrates et démagogues \\
Démocratie ancienne et démocratie moderne \\
Démocratie et anarchie \\
Démocratie et démagogie \\
Démocratie et impérialisme \\
Démocratie et représentation \\
Démocratie et république \\
Démocratie et transparence \\
Démonstration et argumentation \\
Démonstration et déduction \\
Dénaturer \\
Dépasser les apparences ? \\
Dépasser l'humain \\
De quel bonheur sommes-nous capables ? \\
De quel droit ? \\
De quelle certitude la science est-elle capable ? \\
De quelle réalité témoignent nos perceptions ? \\
De quelle science humaine la folie peut-elle être l'objet ? \\
De quelle vérité l'opinion est-elle capable ? \\
De quoi doute un sceptique ? \\
De quoi est-on conscient ? \\
De quoi est-on malheureux ? \\
De quoi la forme est-elle la forme ? \\
De quoi la logique est-elle la science ? \\
De quoi l'art nous délivre-t-il ? \\
De quoi les métaphysiciens parlent-ils ? \\
De quoi les sciences humaines nous instruisent-elles ? \\
De quoi l'État doit-il être propriétaire ? \\
De quoi l'expérience esthétique est-elle l'expérience ? \\
De quoi n'avons-nous pas conscience ? \\
De quoi ne peut-on pas répondre ? \\
De quoi parlent les mathématiques ? \\
De quoi parlent les théories physiques ? \\
De quoi pâtit-on ? \\
De quoi somme-nous prisonniers ? \\
De quoi sommes-nous responsables ? \\
De quoi y a-t-il expérience ? \\
De quoi y a-t-il histoire ? \\
Désacraliser \\
Des comportements économiques peuvent-ils être rationnels ? \\
Des événements aléatoires peuvent-ils obéir à des lois ? \\
Désintérêt et désintéressement \\
Désirer \\
Désire-t-on la reconnaissance ? \\
Désir et politique \\
Des motivations peuvent-elles être sociales ? \\
Des nations peuvent-elles former une société ? \\
Désobéir \\
Désobéir aux lois \\
Désobéissance et résistance \\
Des peuples sans histoire \\
Des sociétés sans État sont-elles des sociétés politiques ? \\
Déterminisme psychique et déterminisme physique \\
Détruire pour reconstruire \\
Devant qui sommes-nous responsables ? \\
Devenir autre \\
Devenir citoyen \\
Devenir et évolution \\
Devient-on raisonnable ? \\
Dialectique et Philosophie \\
Dialoguer \\
Dieu aurait-il pu mieux faire ? \\
Dieu est-il une limite de la pensée ? \\
Dieu est mort \\
Dieu et César \\
Dieu pense-t-il ? \\
Dieu peut-il tout faire ? \\
Dieu, prouvé ou éprouvé ? \\
Dire ce qui est \\
Dire, est-ce faire ? \\
Dire et faire \\
Dire et montrer \\
Dire le monde \\
Dire oui \\
Dire « je » \\
Diriger son esprit \\
Discussion et dialogue \\
Distinguer \\
Division du travail et cohésion sociale \\
Documents et monuments \\
Doit-on bien juger pour bien faire ? \\
Doit-on répondre de ce qu'on est devenu ? \\
Doit-on respecter la nature ? \\
Doit-on se faire l'avocat du diable ? \\
Doit-on toujours dire la vérité ? \\
Doit-on tout calculer ? \\
Donner \\
Donner des exemples \\
Donner des raisons \\
Donner du sens \\
Donner raison \\
Donner raison, rendre raison \\
Donner sa parole \\
Donner une représentation \\
Donner un exemple \\
D'où la politique tire-t-elle sa légitimité ? \\
D'où vient aux objets techniques leur beauté ? \\
D'où vient la certitude dans les sciences ? \\
D'où vient la signification des mots ? \\
D'où vient le mal ? \\
D'où vient le plaisir de lire ? \\
D'où vient que l'histoire soit autre chose qu'un chaos ? \\
Droit naturel et loi naturelle \\
Droits de l'homme et droits du citoyen \\
Droits et devoirs \\
Droits et devoirs sont-ils réciproques ? \\
Échange et don \\
Éclairer \\
Économie et politique \\
Économie politique et politique économique \\
Écouter \\
Écrire \\
Écrire l'histoire \\
Éducation de l'homme, éducation du citoyen \\
Éduquer le citoyen \\
Égalité des droits, égalité des conditions \\
Égoïsme et altruisme \\
Égoïsme et individualisme \\
Égoïsme et méchanceté \\
Empirique et expérimental \\
Enfance et moralité \\
En politique n'y a-t-il que des rapports de force ? \\
En politique, peut-on faire table rase du passé ? \\
En politique, y a-t-il des modèles ? \\
En quel sens la métaphysique a-t-elle une histoire ? \\
En quel sens la métaphysique est-elle une science ? \\
En quel sens l'anthropologie peut-elle être historique ? \\
En quel sens parler de structure métaphysique ? \\
En quel sens peut-on parler de la vie sociale comme d'un jeu ? \\
En quel sens peut-on parler de transcendance ? \\
En quel sens une œuvre d'art est-elle un document ? \\
Enquêter \\
En quoi la connaissance de la matière peut-elle relever de la métaphysique ? \\
En quoi la matière s'oppose-t-elle à l'esprit ? \\
En quoi la technique fait-elle question ? \\
En quoi les sciences humaines nous éclairent-elles sur la barbarie ? \\
En quoi les sciences humaines sont-elles normatives ? \\
En quoi l'œuvre d'art donne-t-elle à penser ? \\
En quoi une discussion est-elle politique ? \\
En quoi une insulte est-elle blessante ? \\
Enseigner \\
Enseigner, instruire, éduquer \\
Enseigner l'art \\
Entendement et raison \\
Entendre raison \\
Entrer en scène \\
Énumérer \\
Épistémologie générale et épistémologie des sciences particulières \\
Éprouver sa valeur \\
Erreur et illusion \\
Espace et structure sociale \\
Espace mathématique et espace physique \\
Espace public et vie privée \\
Essayer \\
Essence et existence \\
Est-ce par son objet ou par ses méthodes qu'une science peut se définir ? \\
Est-ce pour des raisons morales qu'il faut protéger l'environnement ? \\
Esthétique et poétique \\
Esthétisme et moralité \\
Est-il bon qu'un seul commande ? \\
Est-il difficile de savoir ce que l'on veut ? \\
Est-il difficile d'être heureux ? \\
Est-il judicieux de revenir sur ses décisions ? \\
Est-il mauvais de suivre son désir ? \\
Est-il parfois bon de mentir ? \\
Est-il possible de croire en la vie éternelle ? \\
Est-il possible de douter de tout ? \\
Est-il possible d'être neutre politiquement ? \\
Est-il vrai qu'en science, « rien n'est donné, tout est construit » ? \\
Est-il vrai qu'on apprenne de ses erreurs ? \\
Estimer \\
Est-on fondé à distinguer la justice et le droit ? \\
Est-on le produit d'une culture ? \\
Est-on responsable de ce qu'on n'a pas voulu ? \\
Est-on responsable de l'avenir de l'humanité \\
État et nation \\
État et société \\
Éternité et immortalité \\
Éthique et authenticité \\
Ethnologie et cinéma \\
Ethnologie et ethnocentrisme \\
Ethnologie et sociologie \\
Être acteur \\
Être affairé \\
Être cause de soi \\
Être, c'est agir \\
Être chez soi \\
Être citoyen du monde \\
Être compris \\
Être content de soi \\
Être dans l'esprit \\
Être dans le temps \\
Être dans son bon droit \\
Être de mauvaise humeur \\
Être de son temps \\
Être égal à soi-même \\
Être en bonne santé \\
Être en désaccord \\
Être en règle avec soi-même \\
Être ensemble \\
Être est-ce agir ? \\
Être et avoir \\
Être et devenir \\
Être et devoir être \\
Être et être pensé \\
Être et ne plus être \\
Être et représentation \\
Être et sens \\
Être exemplaire \\
Être hors de soi \\
Être juge et partie \\
Être l'entrepreneur de soi-même \\
Être logique \\
Être maître de soi \\
Être malade \\
Être méchant \\
Être méchant volontairement \\
Être mère \\
Être né \\
Être par soi \\
Être pauvre \\
Être père \\
Être réaliste \\
Être sans cause \\
Être sans cœur \\
Être sans scrupule \\
Être sceptique \\
Être seul avec sa conscience \\
Être seul avec soi-même \\
Être soi-même \\
Être spirituel \\
Être systématique \\
Être un artiste \\
Être un corps \\
Être une chose qui pense \\
Être, vie et pensée \\
Étudier \\
Évidence et certitude \\
Évolution et progrès \\
Existence et essence \\
Exister \\
Existe-t-il des dilemmes moraux ? \\
Existe-t-il des questions sans réponse ? \\
Existe-t-il des sciences de différentes natures ? \\
Existe-t-il un bien commun qui soit la norme de la vie politique ? \\
Existe-t-il une opinion publique ? \\
Expérience esthétique et sens commun \\
Expérience et approximation \\
Expérience et expérimentation \\
Expérimenter \\
Explication et prévision \\
Expliquer \\
Expliquer et comprendre \\
Expliquer et interpréter \\
Expliquer et justifier \\
Expression et création \\
Expression et signification \\
Extension et compréhension \\
Faire ce qu'on dit \\
Faire corps \\
Faire de la métaphysique, est-ce se détourner du monde ? \\
Faire de la politique \\
Faire de nécessité vertu \\
Faire de sa vie une œuvre d'art \\
Faire des choix \\
Faire école \\
Faire et laisser faire \\
Faire justice \\
Faire la loi \\
Faire la morale \\
Faire la paix \\
Faire la révolution \\
Faire l'histoire \\
Faire une expérience \\
Fait et essence \\
Fait et valeur \\
Famille et tribu \\
Faut-il aimer la vie ? \\
Faut-il aimer son prochain comme soi-même ? \\
Faut-il aller au-delà des apparences ? \\
Faut-il avoir des ennemis ? \\
Faut-il avoir des principes ? \\
Faut-il avoir peur de la liberté ? \\
Faut-il avoir peur des habitudes ? \\
Faut-il concilier les contraires ? \\
Faut-il condamner la rhétorique ? \\
Faut-il condamner les illusions ? \\
Faut-il considérer le droit pénal comme instituant une violence légitime ? \\
Faut-il considérer les faits sociaux comme des choses ? \\
Faut-il craindre la révolution ? \\
Faut-il craindre le pire ? \\
Faut-il craindre les foules ? \\
Faut-il craindre les masses ? \\
Faut-il croire au progrès ? \\
Faut-il croire en quelque chose ? \\
Faut-il détruire l'État ? \\
Faut-il diriger l'économie ? \\
Faut-il distinguer ce qui est de ce qui doit être ? \\
Faut-il distinguer esthétique et philosophie de l'art ? \\
Faut-il enfermer ? \\
Faut-il être bon ? \\
Faut-il être cosmopolite ? \\
Faut-il être mesuré en toutes choses ? \\
Faut-il être objectif ? \\
Faut-il être réaliste en politique ? \\
Faut-il expliquer la morale par son utilité ? \\
Faut-il fuir la politique ? \\
Faut-il laisser parler la nature ? \\
Faut-il limiter la souveraineté ? \\
Faut-il limiter l'exercice de la puissance publique ? \\
Faut-il ménager les apparences ? \\
Faut-il mépriser le luxe ? \\
Faut-il mieux vivre comme si nous ne devions jamais mourir ? \\
Faut-il n'être jamais méchant ? \\
Faut-il opposer à la politique la souveraineté du droit ? \\
Faut-il opposer l'art à la connaissance ? \\
Faut-il opposer l'histoire et la fiction ? \\
Faut-il opposer rhétorique et philosophie ? \\
Faut-il parler pour avoir des idées générales ? \\
Faut-il penser l'État comme un corps ? \\
Faut-il perdre ses illusions ? \\
Faut-il préférer le bonheur à la vérité ? \\
Faut-il préférer une injustice au désordre ? \\
Faut-il prendre soin de soi ? \\
Faut-il rechercher la certitude ? \\
Faut-il renoncer à son désir ? \\
Faut-il respecter la nature ? \\
Faut-il respecter les convenances ? \\
Faut-il rompre avec le passé ? \\
Faut-il se délivrer des passions ? \\
Faut-il se fier à la majorité ? \\
Faut-il se méfier de l'imagination ? \\
Faut-il se méfier du volontarisme politique ? \\
Faut-il suivre ses intuitions ? \\
Faut-il tolérer les intolérants ? \\
Faut-il vivre comme si l'on ne devait jamais mourir ? \\
Faut-il vouloir changer le monde ? \\
Faut-il vouloir la paix ? \\
Folie et société \\
Fonction et prédicat \\
Fonder \\
Fonder la justice \\
Fonder une cite \\
Fonder une cité \\
Force et violence \\
Forger des hypothèses \\
Formaliser et axiomatiser \\
Forme et rythme \\
Garder la mesure \\
Gérer et gouverner \\
Gouvernement des hommes et administration des choses \\
Gouverner \\
Gouverner, administrer, gérer \\
Gouverner, est-ce prévoir ? \\
Gouverner et se gouverner \\
Grammaire et métaphysique \\
Grammaire et philosophie \\
Grandeur et décadence \\
Groupe, classe, société \\
Guérir \\
Guerre et politique \\
Habiter \\
Habiter le monde \\
Haïr \\
Haïr la raison \\
Histoire et anthropologie \\
Histoire et ethnologie \\
Histoire et fiction \\
Histoire et géographie \\
Histoire et mémoire \\
Homo religiosus \\
Humour et ironie \\
Ici et maintenant \\
Identité et communauté \\
Identité et différence \\
Identité et indiscernabilité \\
Illégalité et injustice \\
Il y a \\
Imaginaire et politique \\
Imaginer \\
Imitation et création \\
Imitation et identification \\
Imiter, est-ce copier ? \\
Inconscient et langage \\
Indépendance et autonomie \\
Indépendance et liberté \\
Individuation et identité \\
Individu et société \\
Infini et indéfini \\
Information et communication \\
Innocenter le devenir \\
Instinct et morale \\
Instruire et éduquer \\
Interdire et prohiber \\
Interpréter \\
Interpréter et expliquer \\
Interpréter et formaliser dans les sciences humaines \\
Interpréter une œuvre d'art \\
Intuition et concept \\
Intuition et déduction \\
Invention et création \\
J'ai un corps \\
Je mens \\
Je ne l'ai pas fait exprès \\
Je sens, donc je suis \\
Je, tu, il \\
Jouer \\
Jouer un rôle \\
Jouir sans entraves \\
Jugement analytique et jugement synthétique \\
Jugement esthétique et jugement de valeur \\
Juger \\
Juger en conscience \\
Juger et raisonner \\
Jusqu'à quel point pouvons-nous juger autrui ? \\
Jusqu'où peut-on soigner ? \\
Justice et vengeance \\
Justice et violence \\
Justifier \\
Justifier et prouver \\
La banalité \\
L'abandon \\
La bassesse \\
La béatitude \\
La beauté \\
La beauté a-t-elle une histoire ? \\
La beauté de la nature \\
La beauté des corps \\
La beauté des ruines \\
La beauté du diable \\
La beauté du geste \\
La beauté du monde \\
La beauté est-elle dans le regard ou dans la chose vue ? \\
La beauté est-elle l'objet d'une connaissance ? \\
La beauté est-elle partout ? \\
La beauté est-elle sensible ? \\
La beauté est-elle une promesse de bonheur ? \\
La beauté et la grâce \\
La beauté idéale \\
La beauté morale \\
La beauté naturelle \\
La beauté peut-elle délivrer une vérité ? \\
La belle âme \\
La belle nature \\
La bestialité \\
La bêtise \\
La bêtise n'est-elle pas proprement humaine ? \\
La bibliothèque \\
La bienfaisance \\
La bienveillance \\
La biologie peut-elle se passer de causes finales ? \\
La bonne conscience \\
La bonne volonté \\
L'absence \\
L'absence de fondement \\
L'absence d'œuvre \\
L'absolu \\
L'abstraction \\
L'abstraction en art \\
L'abstrait est-il en dehors de l'espace et du temps ? \\
L'abstrait et le concret \\
L'abus de pouvoir \\
L'académisme \\
L'académisme dans l'art \\
La casuistique \\
La catharsis \\
La causalité \\
La causalité en histoire \\
La causalité historique \\
La causalité suppose-t-elle des lois ? \\
La cause \\
La cause et la raison \\
La cause première \\
L'accident \\
L'accidentel \\
L'accomplissement \\
L'accord \\
La censure \\
La certitude \\
La chair \\
La chance \\
La charité \\
La charité est-elle une vertu ? \\
La chasse et la guerre \\
L'achèvement de l'œuvre \\
La chose \\
La chose en soi \\
La chose publique \\
La chronologie \\
La circonspection \\
La citation \\
La cité idéale \\
La citoyenneté \\
La civilisation \\
La civilité \\
La clarté \\
La classification \\
La classification des arts \\
La classification des sciences \\
La clause de conscience \\
La clémence \\
La cohérence est-elle un critère de la vérité ? \\
La colère \\
La comédie du pouvoir \\
La comédie humaine \\
La comédie sociale \\
La communauté internationale \\
La communauté morale \\
La communauté scientifique \\
La communication \\
La communication est-elle nécessaire à la démocratie ? \\
La compassion \\
La compassion risque-t-elle d'abolir l'exigence politique ? \\
La compétence \\
La compétence technique peut-elle fonder l'autorité publique ? \\
La composition \\
La compréhension \\
La concorde \\
La concurrence \\
La condition \\
La condition humaine \\
La condition sociale \\
La confiance \\
La connaissance adéquate \\
La connaissance animale \\
La connaissance a-t-elle des limites ? \\
La connaissance commune est-elle le point de départ de la science ? \\
La connaissance de la vie \\
La connaissance des causes \\
La connaissance des principes \\
La connaissance du futur \\
La connaissance du singulier \\
La connaissance du vivant \\
La connaissance est-elle une croyance justifiée ? \\
La connaissance mathématique \\
La connaissance objective \\
La connaissance scientifique abolit-elle toute croyance ? \\
La connaissance scientifique n'est-elle qu'une croyance argumentée ? \\
La connaissance suppose-t-elle une éthique ? \\
La conquête \\
La conscience de soi \\
La conscience de soi de l'art \\
La conscience est-elle intrinsèquement morale ? \\
La conscience historique \\
La conscience morale \\
La conscience morale est-elle innée ? \\
La conscience politique \\
La conséquence \\
La constance \\
La constitution \\
La contemplation \\
La contestation \\
La contingence \\
La contingence des lois de la nature \\
La contingence du futur \\
La continuité \\
La contradiction \\
La contradiction réside-t-elle dans les choses ? \\
La contrainte \\
La contrôle social \\
La conversation \\
La conversion \\
La conviction \\
La coopération \\
La corruption \\
La corruption politique \\
La cosmogonie \\
La couleur \\
La coutume \\
La crainte des Dieux \\
La crainte et l'ignorance \\
La création \\
La création artistique \\
La création dans l'art \\
La création de l'humanité \\
La créativité \\
La criminalité \\
La crise sociale \\
La critique \\
La critique d'art \\
La croissance \\
La croyance est-elle l'asile de l'ignorance ? \\
La cruauté \\
L'acte \\
L'acteur \\
L'acteur et son rôle \\
L'action collective \\
L'action politique \\
L'action politique a-t-elle un fondement rationnel ? \\
L'action politique peut-elle se passer de mots ? \\
L'actualité \\
L'actuel \\
La cuisine \\
La culpabilité \\
La culture artistique \\
La culture de masse \\
La culture démocratique \\
La culture d'entreprise \\
La culture est-elle affaire de politique ? \\
La culture est-elle nécessaire à l'appréciation d'une œuvre d'art ? \\
La culture générale \\
La culture morale \\
La culture scientifique \\
La curiosité \\
La curiosité est-elle à l'origine du savoir ? \\
La danse \\
La danse est-elle l'œuvre du corps ? \\
La décadence \\
La décence \\
La déception \\
La décision morale \\
La décision politique \\
La déduction \\
La défense nationale \\
La déficience \\
La définition \\
La délibération \\
La délibération en morale \\
La délibération politique \\
La démagogie \\
La démence \\
La démesure \\
La démocratie conduit-elle au règne de l'opinion ? \\
La démocratie est-elle le pire des régimes politiques ? \\
La démocratie est-elle moyen ou fin ? \\
La démocratie est-elle nécessairement libérale ? \\
La démocratie est-elle possible ? \\
La démocratie et les experts \\
La démocratie n'est-elle que la force des faibles ? \\
La démocratie participative \\
La démonstration \\
La déontologie \\
La dépendance \\
La déraison \\
La descendance \\
La désillusion \\
La désinvolture \\
La désobéissance civile \\
La destruction \\
La détermination \\
La dette \\
La dialectique \\
La dialectique est-elle une science ? \\
La dictature \\
La différence \\
La différence des arts \\
La différence des sexes \\
La différence sexuelle \\
La difformité \\
La dignité \\
La dignité humaine \\
La digression \\
La discipline \\
La discrétion \\
La discrimination \\
La discussion \\
La disposition morale \\
La distance \\
La distinction \\
La distinction de genre \\
La distinction de la nature et de la culture est-elle un fait de culture ? \\
La distraction \\
La diversion \\
La diversité des cultures \\
La diversité des langues \\
La diversité des perceptions \\
La diversité des religions \\
La diversité des sciences \\
La diversité humaine \\
La division \\
La division des pouvoirs \\
La division des tâches \\
La division du travail \\
L'admiration \\
La docilité est-elle un vice ou une vertu ? \\
La domination \\
La domination du corps \\
La domination sociale \\
L'adoucissement des mœurs \\
La douleur \\
La droit de conquête \\
La droiture \\
La dualité \\
La duplicité \\
La faiblesse de la démocratie \\
La faiblesse de la volonté \\
La famille \\
La famille est-elle le lieu de la formation morale ? \\
La fatalité \\
La fatigue \\
La faute \\
La fête \\
La fiction \\
La fidélité \\
La fidélité à soi \\
La figuration \\
La fin \\
La finalité \\
La finalité des sciences humaines \\
La fin de la métaphysique \\
La fin de la politique \\
La fin de la politique est-elle l'établissement de la justice ? \\
La fin de l'art \\
La fin de l'État \\
La fin de l'histoire \\
La fin du monde \\
La finitude \\
La fin justifie-t-elle les moyens ? \\
La folie \\
La fonction de l'art \\
La fonction de penser peut-elle se déléguer ? \\
La fonction première de l'État est-elle de durer ? \\
La force \\
La force d'âme \\
La force de la loi \\
La force de l'art \\
La force de l'habitude \\
La force de l'idée \\
La force des idées \\
La force des lois \\
La force du pouvoir \\
La force du social \\
La force est-elle une vertu ? \\
La force fait-elle le droit ? \\
La formation des citoyens \\
La formation du goût \\
La formation d'une conscience \\
La fortune \\
La foule \\
La fragilité \\
La franchise \\
La franchise est-elle une vertu ? \\
La fraternité a-t-elle un sens politique ? \\
La fraternité est-elle un idéal moral ? \\
La fraude \\
La frivolité \\
La frontière \\
La futilité \\
L'âge d'or \\
La générosité \\
La genèse \\
La genèse de l'œuvre \\
La gentillesse \\
La géographie \\
La géométrie \\
La grâce \\
La grammaire \\
La grammaire contraint-elle notre pensée ? \\
La grammaire et la logique \\
La grandeur \\
La grandeur d'âme \\
La gratitude \\
La gratuité \\
L'agressivité \\
L'agriculture \\
La guérison \\
La guerre civile \\
La guerre est-elle la continuation de la politique par d'autres moyens ? \\
La guerre est-elle la continuation de la politique ? \\
La guerre et la paix \\
La guerre juste \\
La guerre totale \\
La haine de la pensée \\
La haine de la raison \\
La haine de soi \\
La hiérarchie \\
La hiérarchie des arts \\
La hiérarchie des énoncés scientifiques \\
La honte \\
La jalousie \\
La jeunesse \\
La joie \\
La jouissance \\
La jurisprudence \\
La juste colère \\
La juste mesure \\
La juste peine \\
La justice \\
La justice a-t-elle besoin des institutions ? \\
La justice consiste-t-elle à traiter tout le monde de la même manière ? \\
La justice de l'État \\
La justice divine \\
La justice entre les générations \\
La justice est-elle une notion morale ? \\
La justice peut-elle se passer d'institutions ? \\
La justice sociale \\
La justice : moyen ou fin de la politique ? \\
La justification \\
La laïcité \\
La laideur \\
La langue maternelle \\
L'aléatoire \\
La leçon des choses \\
La lecture \\
La légende \\
La légitimation \\
La légitimité \\
La lettre et l'esprit \\
La libération des mœurs \\
La liberté civile \\
La liberté créatrice \\
La liberté de culte \\
La liberté de l'artiste \\
La liberté de la science \\
La liberté de parole \\
La liberté des citoyens \\
La liberté d'opinion \\
La liberté, est-ce l'indépendance à l'égard des passions ? \\
La liberté intéresse-t-elle les sciences humaines ? \\
La liberté morale \\
La liberté politique \\
La liberté se prouve-t-elle ? \\
L'aliénation \\
La limite \\
La littérature peut-elle suppléer les sciences de l'homme ? \\
L'allégorie \\
La logique a-t-elle une histoire ? \\
La logique a-t-elle un intérêt philosophique ? \\
La logique décrit-elle le monde ? \\
La logique est-elle indépendante de la psychologie ? \\
La logique est-elle un art de penser ? \\
La logique est-elle un art de raisonner ? \\
La logique est-elle une discipline normative ? \\
La logique est-elle une forme de calcul ? \\
La logique est-elle une science de la vérité ? \\
La logique est-elle utile à la métaphysique ? \\
La logique et le réel \\
La logique nous apprend-elle quelque chose sur le langage ordinaire ? \\
La logique peut-elle se passer de la métaphysique ? \\
La logique pourrait-elle nous surprendre ? \\
La logique : découverte ou invention ? \\
La loi \\
La loi du genre \\
La loi du marché \\
La loi et le règlement \\
La loi peut-elle changer les mœurs ? \\
La louange et le blâme \\
La loyauté \\
L'altérité \\
L'altruisme \\
La lumière de la vérité \\
La lumière naturelle \\
La lutte des classes \\
La machine \\
La magie \\
La magnanimité \\
La main \\
La main et l'outil \\
La maîtrise \\
La maîtrise de la langue \\
La maîtrise de la nature \\
La maîtrise de soi \\
La maîtrise du feu \\
La maîtrise du temps \\
La majesté \\
La majorité \\
La majorité peut-elle être tyrannique ? \\
La maladie \\
La malchance \\
La manière \\
La manifestation \\
L'amateur \\
L'amateurisme \\
La mathématique est-elle une ontologie ? \\
La matière \\
La matière, est-ce l'informe ? \\
La matière n'est-elle qu'une idée ? \\
La matière première \\
La matière sensible \\
La maturité \\
La mauvaise conscience \\
La mauvaise foi \\
La mauvaise volonté \\
L'ambiguïté \\
L'ambition politique \\
La méchanceté \\
L'âme concerne-t-elle les sciences humaines ? \\
La médecine est-elle une science ? \\
L'âme des bêtes \\
La médiation \\
L'âme est-elle immortelle ? \\
La meilleure constitution \\
La mélancolie \\
L'âme, le monde et Dieu \\
La mémoire \\
La mémoire collective \\
La mémoire et l'individu \\
La mémoire sélective \\
La mesure \\
La mesure de l'intelligence \\
La mesure des choses \\
La mesure du temps \\
La métaphore \\
La métaphysique a-t-elle ses fictions ? \\
La métaphysique est-elle le fondement de la morale ? \\
La métaphysique est-elle nécessairement une réflexion sur Dieu ? \\
La métaphysique peut-elle être autre chose qu'une science recherchée ? \\
La métaphysique peut-elle faire appel à l'expérience ? \\
La métaphysique se définit-elle par son objet ou sa démarche ? \\
La méthode \\
La méthode de la science \\
La minorité \\
La misanthropie \\
La misère \\
La misologie \\
L'amitié \\
L'amitié est-elle une vertu ? \\
L'amitié est-elle un principe politique ? \\
La modalité \\
La mode \\
La modélisation en sciences sociales \\
La modération \\
La modération est-elle l'essence de la vertu ? \\
La modération est-elle une vertu politique ? \\
La modernité \\
La modernité dans les arts \\
La mondialisation \\
La monnaie \\
La monumentalité \\
La morale a-t-elle besoin d'être fondée ? \\
La morale a-t-elle besoin d'un au-delà ? \\
La morale commune \\
La morale consiste-t-elle à suivre la nature ? \\
La morale de l'athée \\
La morale de l'intérêt \\
La morale des fables \\
La morale doit-elle en appeler à la nature ? \\
La morale doit-elle fournir des préceptes ? \\
La morale du citoyen \\
La morale du plus fort \\
La morale est-elle affaire de jugement ? \\
La morale est-elle affaire de sentiment ? \\
La morale est-elle ennemie du bonheur ? \\
La morale est-elle fondée sur la liberté ? \\
La morale est-elle incompatible avec le déterminisme ? \\
La morale est-elle l'ennemie de la vie ? \\
La morale est-elle nécessairement répressive ? \\
La morale est-elle un art de vivre ? \\
La morale est-elle une affaire d'habitude ? \\
La morale est-elle un fait social ? \\
La morale et le droit \\
La morale peut-elle être fondée sur la science ? \\
La morale peut-elle être naturelle ? \\
La morale peut-elle être un calcul ? \\
La morale peut-elle être une science ? \\
La morale peut-elle se passer d'un fondement religieux ? \\
La morale politique \\
La morale suppose-t-elle le libre arbitre ? \\
La moralité des lois \\
La moralité n'est-elle que dressage ? \\
La moralité réside-t-elle dans l'intention ? \\
La mort dans l'âme \\
La mort de l'art \\
La mort fait-elle partie de la vie ? \\
L'amour de la liberté \\
L'amour de l'art \\
L'amour de l'humanité \\
L'amour des lois \\
L'amour de soi \\
L'amour et la haine \\
L'amour et la justice \\
L'amour et l'amitié \\
L'amour et la mort \\
L'amour peut-il être absolu ? \\
L'amour-propre \\
L'amour vrai \\
La multiplicité \\
La multitude \\
La musique a-t-elle une essence ? \\
La musique de film \\
La musique est-elle un langage ? \\
La musique et le bruit \\
La naissance \\
La naissance de la science \\
La naissance de l'homme \\
La naïveté \\
L'analogie \\
L'analyse \\
L'analyse du vécu \\
L'anarchie \\
La nation \\
La nation est-elle dépassée ? \\
La nation et l'État \\
La nature a-t-elle une histoire ? \\
La nature des choses \\
La nature du bien \\
La nature est-elle artiste ? \\
La nature est-elle digne de respect ? \\
La nature est-elle écrite en langage mathématique ? \\
La nature est-elle muette ? \\
La nature est-elle sacrée ? \\
La nature est-elle sauvage ? \\
La nature et la grâce \\
La nature et le monde \\
La nature morte \\
La nature parle-t-elle le langage des mathématiques ? \\
La nature peut-elle être belle ? \\
L'anecdotique \\
La nécessité \\
La nécessité des contradictions \\
La nécessité des signes \\
La nécessité historique \\
La négation \\
La négligence est-elle une faute ? \\
La neutralité \\
La neutralité de l'État \\
Langage et communication \\
Langage et réalité \\
Langage, langue et parole \\
Langage ordinaire et langage de la science \\
L'angélisme \\
L'angoisse \\
Langue et parole \\
L'animalité \\
L'animal nous apprend-il quelque chose sur l'homme ? \\
L'animal peut-il être un sujet moral ? \\
L'animal politique \\
L'animisme \\
La noblesse \\
L'anomalie \\
L'anonymat \\
L'anormal \\
La normalité \\
La norme et le fait \\
La nostalgie \\
La notion d'administration \\
La notion de barbarie a-t-elle un sens ? \\
La notion de civilisation \\
La notion de classe dominante \\
La notion de classe sociale \\
La notion de corps social \\
La notion de loi a-t-elle une unité ? \\
La notion de loi dans les sciences de la nature et dans les sciences de l'homme \\
La notion de peuple \\
La notion de point de vue \\
La notion de possible \\
La notion de progrès a-t-elle un sens en politique ? \\
La notion de sujet en politique \\
La notion d'évolution \\
La notion d'intérêt \\
La notion d'ordre \\
La notion physique de force \\
La nouveauté \\
La nouveauté en art \\
L'antériorité \\
L'anthropocentrisme \\
L'anthropologie est-elle une ontologie ? \\
L'anticipation \\
La nuance \\
La nudité \\
La paix \\
La paix civile \\
La paix est-elle moins naturelle que la guerre ? \\
La paix est-elle possible ? \\
La paix n'est-elle que l'absence de conflit ? \\
La paix n'est-elle que l'absence de guerre ? \\
La paix perpétuelle \\
La paix sociale est-elle une fin en soi ? \\
La parenté \\
La parenté et la famille \\
La paresse \\
La parole \\
La parole publique \\
La participation \\
La participation des citoyens \\
La passion de la vérité \\
La passion de l'égalité \\
La passion du juste \\
La paternité \\
L'apathie \\
La patience \\
La patience est-elle une vertu ? \\
La patrie \\
La pauvreté \\
La peine capitale \\
La peinture est-elle une poésie muette ? \\
La peinture peut-elle être un art du temps ? \\
La pensée a-t-elle une histoire ? \\
La pensée collective \\
La pensée est-elle en lutte avec le langage ? \\
La pensée formelle peut-elle avoir un contenu ? \\
La pensée magique \\
La pensée peut-elle s'écrire ? \\
La perfectibilité \\
La perfection \\
La perfection en art \\
La perfection morale \\
La personnalité \\
La personne \\
La perspective \\
La pertinence \\
La perversion morale \\
La perversité \\
La peur de la mort \\
La peur de la nature \\
La peur de l'autre \\
La peur du châtiment \\
La peur du désordre \\
La philanthropie \\
La philosophie doit-elle se préoccuper du salut ? \\
La philosophie peut-elle disparaître ? \\
La philosophie peut-elle se passer de théologie ? \\
La philosophie première \\
La physique et la chimie \\
La pitié \\
La pitié est-elle morale ? \\
La pitié peut-elle fonder la morale ? \\
La place d'autrui \\
La place du hasard dans la science \\
La place du sujet dans la science \\
La plénitude \\
La pluralité \\
La pluralité des arts \\
La pluralité des cultures \\
La pluralité des langues \\
La pluralité des mondes \\
La pluralité des opinions \\
La pluralité des pouvoirs \\
La pluralité des sciences de la nature \\
La pluralité des sens de l'être \\
La poésie \\
La poésie et l'idée \\
La poésie pense-t-elle ? \\
La polémique \\
La politesse \\
La politique a-t-elle besoin de héros ? \\
La politique a-t-elle besoin de modèles ? \\
La politique a-t-elle besoin d'experts ? \\
La politique a-t-elle pour fin d'éliminer la violence ? \\
La politique consiste-t-elle à faire des compromis ? \\
La politique de la santé \\
La politique doit-elle être morale ? \\
La politique doit-elle être rationnelle ? \\
La politique doit-elle se mêler de l'art ? \\
La politique doit-elle viser la concorde ? \\
La politique doit-elle viser le consensus ? \\
La politique échappe-telle à l'exigence de vérité ? \\
La politique est-elle affaire de décision ? \\
La politique est-elle affaire de jugement ? \\
La politique est-elle architectonique ? \\
La politique est-elle la continuation de la guerre ? \\
La politique est-elle l'affaire de tous ? \\
La politique est-elle l'art des possibles ? \\
La politique est-elle l'art du possible ? \\
La politique est-elle naturelle ? \\
La politique est-elle par nature sujette à dispute ? \\
La politique est-elle plus importante que tout ? \\
La politique est-elle un art ? \\
La politique est-elle une science ? \\
La politique est-elle une technique ? \\
La politique est-elle un métier ? \\
La politique et la gloire \\
La politique et la ville \\
La politique et le mal \\
La politique et le politique \\
La politique et l'opinion \\
La politique exclut-elle le désordre ? \\
La politique implique-t-elle la hiérarchie ? \\
La politique peut-elle changer la société \\
La politique peut-elle changer le monde ? \\
La politique peut-elle être indépendante de la morale ? \\
La politique peut-elle être objet de science ? \\
La politique peut-elle être un objet de science ? \\
La politique peut-elle n'être qu'une pratique ? \\
La politique peut-elle se passer de croyances ? \\
La politique peut-elle unir les hommes ? \\
La politique repose-t-elle sur un contrat ? \\
La politique requière-t-elle le compromis \\
La politique scientifique \\
La politique suppose-t-elle la morale ? \\
La politique suppose-t-elle une idée de l'homme ? \\
L'apolitisme \\
La populace \\
La population \\
La possibilité \\
La possibilité logique \\
La possibilité métaphysique \\
La possibilité réelle \\
L'apparence \\
L'appartenance sociale \\
L'appel \\
L'appréciation de la nature \\
L'apprentissage de la langue \\
L'appropriation \\
L'approximation \\
La pratique de l'espace \\
La pratique des sciences met-elle à l'abri des préjugés ? \\
La précaution peut-elle être un principe ? \\
La précision \\
La première fois \\
La première vérité \\
La présence \\
La présence d'esprit \\
La présence du passé \\
La présomption \\
La preuve \\
La preuve de l'existence de Dieu \\
La prévision \\
L'\emph{a priori} \\
La prise de parti est-elle essentielle en politique ? \\
La prison \\
La prison est-elle utile ? \\
La privation \\
La probabilité \\
La probité \\
La productivité de l'art \\
La profondeur \\
La prohibition de l'inceste \\
La promenade \\
La promesse \\
La promesse et le contrat \\
La proposition \\
La propriété \\
La propriété est-elle une garantie de liberté ? \\
La protection \\
La protection sociale \\
La providence \\
La prudence \\
La psychologie est-elle une science de la nature ? \\
La psychologie est-elle une science ? \\
La publicité \\
La pudeur \\
La puissance \\
La puissance de la technique \\
La puissance de l'État \\
La puissance des contraires \\
La puissance des images \\
La puissance et l'acte \\
La pulsion \\
La punition \\
La pureté \\
La qualité \\
La question de l'œuvre d'art \\
La question sociale \\
La question : « qui ? » \\
La radicalité \\
La raison a-t-elle le droit d'expliquer ce que morale condamne ? \\
La raison a-t-elle une histoire ? \\
La raison d'état \\
La raison d'État \\
La raison du plus fort \\
La raison est-elle morale par elle-même ? \\
La raison est-elle suffisante ? \\
La raison peut-elle être immédiatement pratique ? \\
La raison peut-elle être pratique ? \\
La raison suffisante \\
La rationalité des choix politiques \\
La rationalité des comportements économiques \\
La rationalité du langage \\
La rationalité du marché \\
La rationalité en sciences sociales \\
La rationalité politique \\
L'arbitraire \\
L'arbitraire du signe \\
L'archéologie \\
L'architecte et l'ingénieur \\
L'architecture est-elle un art ? \\
La réaction en politique \\
La réalité \\
La réalité a-t-elle une forme logique ? \\
La réalité décrite par la science s'oppose-t-elle à la démonstration ? \\
La réalité de l'idéal \\
La réalité de l'idée \\
La réalité du beau \\
La réalité du futur \\
La réalité du possible \\
La réalité du sensible \\
La réalité du temps \\
La réalité peut-elle être virtuelle ? \\
La réception de l'œuvre d'art \\
La recherche de l'absolu \\
La recherche de la vérité \\
La recherche de la vérité dans les sciences humaines \\
La recherche des invariants \\
La recherche des origines \\
La recherche du bonheur \\
La recherche du bonheur suffit-elle à déterminer une morale ? \\
La recherche scientifique est-elle désintéressée ? \\
La réciprocité \\
La réciprocité est-elle indispensable à la communauté politique ? \\
La reconnaissance \\
La rectitude \\
La référence \\
La réflexion \\
La réflexion sur l'expérience participe-t-elle de l'expérience ? \\
La réforme \\
La réforme des institutions \\
La réfutation \\
La règle et l'exception \\
La relation \\
La relation de cause à effet \\
La relation de nécessité \\
La religion \\
La religion peut-elle faire lien social ? \\
La religion peut-elle suppléer la raison ? \\
La réminiscence \\
La renaissance \\
La rencontre \\
La réparation \\
La répétition \\
La représentation \\
La représentation en politique \\
La représentation politique \\
La reproduction \\
La reproduction sociale \\
La république \\
La réputation \\
La résignation \\
La résilience \\
La résistance à l'oppression \\
La résolution \\
La responsabilité \\
La responsabilité collective \\
La responsabilité de l'artiste \\
La responsabilité politique \\
La ressemblance \\
La restauration des œuvres d'art \\
La révélation \\
La rêverie \\
La révolte \\
La révolte peut-elle être un droit ? \\
La révolution \\
L'argent \\
L'argent et la valeur \\
L'argumentation \\
L'argumentation morale \\
L'argument d'autorité \\
La rhétorique \\
La rhétorique a-t-elle une valeur ? \\
La rhétorique est-elle un art ? \\
La richesse \\
La richesse du sensible \\
La richesse intérieure \\
La rigueur \\
La rigueur de la loi \\
La rigueur morale \\
L'aristocratie \\
L'art apprend-il à percevoir ? \\
L'art a-t-il des vertus thérapeutiques ? \\
L'art a-t-il plus de valeur que la vérité ? \\
L'art a-t-il une fin morale ? \\
L'art a-t-il une histoire ? \\
L'art a-t-il une valeur sociale ? \\
L'art contre la beauté ? \\
L'art d'écrire \\
L'art de gouverner \\
L'art de masse \\
L'art des images \\
L'art de vivre est-il un art ? \\
L'art doit-il être critique ? \\
L'art doit-il nous étonner ? \\
L'art dramatique \\
L'art du comédien \\
L'art échappe-t-il à la raison ? \\
L'art engagé \\
L'art est-il affaire de goût ? \\
L'art est-il affaire d'imagination ? \\
L'art est-il à lui-même son propre but ? \\
L'art est-il destiné à embellir ? \\
L'art est-il subversif ? \\
L'art est-il une critique de la culture ? \\
L'art est-il une expérience de la liberté ? \\
L'art est-il un langage ? \\
L'art est-il un mode de connaissance ? \\
L'art est-il un modèle pour la philosophie ? \\
L'art et la manière \\
L'art et la nature \\
L'art et la tradition \\
L'art et la vie \\
L'art et le divin \\
L'art et le mouvement \\
L'art et l'éphémère \\
L'art et le rêve \\
L'art et le sacré \\
L'art et les arts \\
L'art et l'immoralité \\
L'art et morale \\
L'art fait-il penser ? \\
L'artifice \\
L'artificiel \\
L'art imite-t-il la nature ? \\
L'artiste a-t-il besoin d'une idée de l'art ? \\
L'artiste a-t-il besoin d'un public ? \\
L'artiste a-t-il une méthode ? \\
L'artiste est-il le mieux placé pour comprendre son œuvre ? \\
L'artiste est-il maître de son œuvre ? \\
L'artiste et l'artisan \\
L'artiste exprime-t-il quelque chose ? \\
L'artiste peut-il se passer d'un maître ? \\
L'artiste sait-il ce qu'il fait ? \\
L'art modifie-t-il notre rapport au réel ? \\
L'art n'est-il pas toujours politique ? \\
L'art n'est-il pas toujours religieux ? \\
L'art n'est-il qu'une question de sentiment ? \\
L'art ou les arts \\
L'art peut-il changer le monde \\
L'art peut-il encore imiter la nature ? \\
L'art peut-il être utile ? \\
L'art peut-il n'être pas conceptuel ? \\
L'art peut-il nous rendre meilleurs ? \\
L'art peut-il prétendre à la vérité ? \\
L'art peut-il quelque chose contre la morale ? \\
L'art peut-il quelque chose pour la morale ? \\
L'art peut-il rendre le mouvement ? \\
L'art peut-il s'affranchir des lois ? \\
L'art peut-il s'enseigner ? \\
L'art peut-il se passer d'idéal ? \\
L'art politique \\
L'art pour l'art \\
L'art produit-il nécessairement des œuvres ? \\
L'art s'adresse-t-il à la sensibilité ? \\
L'art sait-il montrer ce que le langage ne peut pas dire ? \\
L'art s'apparente-t-il à la philosophie ? \\
L'art : expérience, exercice ou habitude ? \\
L'art : une arithmétique sensible ? \\
La ruine \\
La rumeur \\
La rupture \\
La ruse \\
La ruse technique \\
La sacralisation de l'œuvre \\
La sagesse \\
La sagesse et l'expérience \\
La sagesse rend-elle heureux ? \\
La sainteté \\
La sanction \\
La santé \\
La satisfaction des penchants \\
La scène théâtrale \\
L'ascèse \\
L'ascétisme \\
La science admet-elle des degrés de croyance ? \\
La science a-t-elle besoin du principe de causalité ? \\
La science a-t-elle des limites ? \\
La science a-t-elle le monopole de la vérité ? \\
La science a-t-elle une histoire ? \\
La science commence-t-elle avec la perception ? \\
La science découvre-t-elle ou construit-elle son objet ? \\
La science de l'être \\
La science de l'individuel \\
La science des mœurs \\
La science dévoile-t-elle le réel ? \\
La science doit-elle se fonder sur une idée de la nature ? \\
La science doit-elle se passer de l'idée de finalité ? \\
La science est-elle indépendante de toute métaphysique ? \\
La science est-elle une langue bien faite ? \\
La science et le mythe \\
La science et les sciences \\
La science et l'irrationnel \\
La science nous éloigne-t-elle des choses ? \\
La science nous indique-t-elle ce que nous devons faire ? \\
La science pense-t-elle ? \\
La science peut-elle guider notre conduite ? \\
La science peut-elle lutter contre les préjugés ? \\
La science peut-elle se passer de fondement ? \\
La science peut-elle se passer de métaphysique ? \\
La science peut-elle se passer d'hypothèses ? \\
La science peut-elle se passer d'institutions ? \\
La science peut-elle tout expliquer ? \\
La science politique \\
La science porte-elle au scepticisme ? \\
La science procède-t-elle par rectification ? \\
La sculpture \\
La sécularisation \\
La sécurité \\
La sécurité nationale \\
La sécurité publique \\
La séduction \\
La ségrégation \\
La sensibilité \\
La séparation \\
La séparation des pouvoirs \\
La sérénité \\
La servitude \\
La servitude volontaire \\
La sévérité \\
La sexualité \\
La signification \\
La signification dans l'œuvre \\
La signification en musique \\
La simplicité \\
La sincérité \\
La singularité \\
La situation \\
La sobriété \\
La socialisation des comportements \\
La société civile \\
La société civile et l'État \\
La société des nations \\
La société des savants \\
La société et l'État \\
La société existe-t-elle ? \\
La société peut-elle se passer de l'État ? \\
La sociologie de l'art nous permet-elle de comprendre l'art ? \\
La sociologie relativise-t-elle la valeur des œuvres d'art ? \\
La solidarité \\
La solitude \\
La solitude constitue-t-elle un obstacle à la citoyenneté ? \\
La sollicitude \\
La somme et le tout \\
La souffrance \\
La souffrance a-t-elle un sens moral ? \\
La souffrance a-t-elle un sens ? \\
La souffrance au travail \\
La souffrance d'autrui \\
La souffrance morale \\
La souveraineté \\
La souveraineté de l'État \\
La souveraineté du peuple \\
La souveraineté peut-elle se partager ? \\
La souveraineté populaire \\
La spécificité des sciences humaines \\
La spéculation \\
La sphère privée échappe-t-elle au politique ? \\
L'aspiration esthétique \\
La spontanéité \\
L'association \\
L'association des idées \\
La structure et le sujet \\
La superstition \\
La sûreté \\
La surveillance de la société \\
La survie \\
La sympathie \\
La sympathie peut-elle tenir lieu de moralité ? \\
La table rase \\
La technique est-elle moralement neutre ? \\
La technique fait-elle des miracles ? \\
La technique n'est-elle qu'une application de la science ? \\
La technique peut-elle améliorer l'homme ? \\
La technocratie \\
La technologie modifie-t-elle les rapports sociaux ? \\
La téléologie \\
La temporalité de l'œuvre d'art \\
La tendance \\
La tentation \\
La tentation réductionniste \\
La terre \\
La Terre et le Ciel \\
La terreur \\
La terreur morale \\
L'athéisme \\
La théogonie \\
La théologie rationnelle \\
La théorie et l'expérience \\
La tolérance \\
La tolérance a-t-elle des limites ? \\
La tolérance envers les intolérants \\
La tolérance est-elle un concept politique ? \\
La tolérance peut-elle constituer un problème pour la démocratie ? \\
L'atome \\
La totalitarisme \\
La totalité \\
La toute puissance \\
La toute-puissance \\
La toute puissance de la pensée \\
La trace \\
La trace et l'indice \\
La tradition \\
La traduction \\
La tranquillité \\
La transcendance \\
La transe \\
La transgression \\
La transmission \\
La transparence est-elle un idéal démocratique ? \\
La tristesse \\
L'attachement \\
L'attente \\
L'attention \\
L'attrait du beau \\
La tyrannie \\
La tyrannie de la majorité \\
La tyrannie du bonheur \\
L'audace \\
L'audace politique \\
L'au-delà \\
L'au-delà de l'être \\
L'autarcie \\
L'auteur et le créateur \\
L'authenticité \\
L'authenticité de l'œuvre d'art \\
L'autobiographie \\
L'autonomie \\
L'autonomie de l'art \\
L'autonomie de l'œuvre d'art \\
L'autonomie du théorique \\
L'autoportrait \\
L'autorité \\
L'autorité de la parole \\
L'autorité de la science \\
L'autorité de l'écrit \\
L'autorité de l'État \\
L'autorité morale \\
L'autorité politique \\
L'autre est-il le fondement de la conscience morale ? \\
L'autre monde \\
La valeur d'échange \\
La valeur de la science \\
La valeur de l'échange \\
La valeur de l'exemple \\
La valeur des choses \\
La valeur du consentement \\
La valeur d'une théorie scientifique se mesure-t-elle à son efficacité ? \\
La valeur du témoignage \\
La valeur du temps \\
La valeur du travail \\
La valeur morale \\
La validité \\
La vanité \\
La vanité est-elle toujours sans objet ? \\
L'avant-garde \\
L'avarice \\
La variété \\
La vénalité \\
La vengeance \\
L'avenir \\
L'avenir de l'humanité \\
L'avenir est-il imaginable ? \\
L'avenir existe-t-il ? \\
L'aventure \\
La véracité \\
La vérification \\
La vérité admet-elle des degrés ? \\
La vérité a-t-elle une histoire ? \\
La vérité de la fiction \\
La vérité de la religion \\
La vérité demande-t-elle du courage ? \\
La vérité des images \\
La vérité du déterminisme \\
La vérité d'une théorie dépend-elle de sa correspondance avec les faits ? \\
La vérité du roman \\
La vérité est-elle morale ? \\
La vérité scientifique est-elle relative ? \\
La vertu \\
La vertu de l'homme politique \\
La vertu du citoyen \\
La vertu, les vertus \\
La vertu peut-elle être excessive ? \\
La vertu peut-elle être purement morale ? \\
La vertu peut-elle s'enseigner ? \\
La vertu politique \\
L'aveu \\
L'aveu diminue-t-il la faute ? \\
L'aveuglement \\
La vie active \\
La vie collective est-elle nécessairement frustrante ? \\
La vie de la langue \\
La vie de l'esprit \\
La vie du droit \\
La vie est-elle la valeur suprême ? \\
La vie éternelle \\
La vie intérieure \\
La vie ordinaire \\
La vie peut-elle être éternelle ? \\
La vie politique \\
La vie politique est-elle aliénante ? \\
La vie privée \\
La vie psychique \\
La vie quotidienne \\
La vigilance \\
La ville \\
La ville et la campagne \\
La violence \\
La violence a-t-elle des degrés ? \\
La violence de l'État \\
La violence politique \\
La violence révolutionnaire \\
La violence sociale \\
La violence verbale \\
La virtualité \\
La virtuosité \\
La vocation \\
La voix \\
La voix de la conscience \\
La voix du peuple \\
La volonté constitue-t-elle le principe de la politique ? \\
La volonté du peuple \\
La volonté générale \\
La volonté peut-elle être collective ? \\
La volonté peut-elle être indéterminée ? \\
La volupté \\
La vraie morale se moque-t-elle de la morale ? \\
La vraisemblance \\
La vulgarité \\
La vulnérabilité \\
Le barbare \\
Le baroque \\
Le beau a-t-il une histoire ? \\
Le beau est-il aimable ? \\
Le beau est-il une valeur commune ? \\
Le beau et l'agréable \\
Le beau et le bien \\
Le beau et le sublime \\
Le beau naturel \\
Le besoin \\
Le besoin d'absolu \\
Le besoin de métaphysique est-il un besoin de connaissance ? \\
Le besoin de philosophie \\
Le besoin de vérité ? \\
Le bien commun \\
Le bien et le mal \\
Le bien et les biens \\
Le bien public \\
Le bien suppose-t-il la transcendance ? \\
Le bon et l'utile \\
Le bon goût \\
Le bonheur dans le mal \\
Le bonheur de la passion est-il sans lendemain ? \\
Le bonheur des citoyens est-il un idéal politique ? \\
Le bonheur des uns, le malheur des autres \\
Le bonheur est-il une fin morale ? \\
Le bonheur est-il une fin politique ? \\
Le bonheur est-il une valeur morale ? \\
Le bonheur est-il un principe politique ? \\
Le bonheur et la vertu \\
Le bon sens \\
Le bon usage des passions \\
Le bourgeois et le citoyen \\
Le bruit \\
Le cadavre \\
Le calcul \\
Le calcul des plaisirs \\
Le cannibalisme \\
Le capitalisme \\
Le capital social \\
Le caractère \\
L'écart \\
Le cas de conscience \\
Le catéchisme moral \\
Le cerveau et la pensée \\
L'échange des marchandises et les rapports humains \\
Le changement \\
L'échange symbolique \\
Le chant \\
Le chaos \\
Le charisme en politique \\
Le charme et la grâce \\
Le châtiment \\
Le chemin \\
Le choix \\
Le choix d'un destin \\
Le choix peut-il être éclairé ? \\
Le ciel et la terre \\
Le cinéma, art de la représentation ? \\
Le cinéma est-il un art comme les autres ? \\
Le cinéma est-il un art ou une industrie ? \\
Le cinéma est-il un art populaire ? \\
Le cinéma est-il un art ? \\
Le citoyen \\
Le citoyen peut-il être à la fois libre et soumis à l'État ? \\
Le classicisme \\
Le cœur \\
L'école des vertus \\
L'écologie est-elle un problème politique ? \\
L'écologie politique \\
L'écologie, une science humaine ? \\
Le combat contre l'injustice a-t-il une source morale ? \\
Le comique et le tragique \\
Le commencement \\
Le commerce des hommes \\
Le commerce est-il pacificateur ? \\
Le commun \\
Le comparatisme dans les sciences humaines \\
Le comportement \\
Le compromis \\
Le concept \\
Le concept de nature est-il un concept scientifique ? \\
Le concept de pulsion \\
Le concept de structure sociale \\
Le concept d'inconscient est-il nécessaire en sciences humaines ? \\
Le concret \\
Le conflit des devoirs \\
Le conflit est-il constitutif de la politique ? \\
Le conflit est-il la raison d'être de la politique ? \\
Le conformisme \\
Le conformisme moral \\
Le conformisme social \\
L'économie a-t-elle des lois ? \\
L'économie est-elle une science humaine ? \\
L'économie politique \\
L'économie psychique \\
L'économique et le politique \\
Le conseil \\
Le conseiller du prince \\
Le consensus \\
Le consentement des gouvernés \\
Le contingent \\
Le continu \\
Le contrat \\
Le contrôle social \\
Le convenable \\
Le corps dansant \\
Le corps dit-il quelque chose ? \\
Le corps est-il porteur de valeurs ? \\
Le corps est-il respectable ? \\
Le corps et la machine \\
Le corps et l'âme \\
Le corps et l'esprit \\
Le corps humain \\
Le corps humain est-il naturel ? \\
Le corps politique \\
Le corps propre \\
Le cosmopolitisme \\
Le cosmopolitisme peut-il devenir réalité ? \\
Le cosmopolitisme peut-il être réaliste ? \\
Le coup d'État \\
Le courage \\
Le courage politique \\
Le cours du temps \\
Le créé et l'incréé \\
Le cri \\
Le critère \\
L'écrit et l'oral \\
Le critique d'art \\
L'écriture des lois \\
L'écriture et la parole \\
Le culte des ancêtres \\
Le cynisme \\
Le danger \\
Le débat \\
Le débat politique \\
Le défaut \\
Le déguisement \\
Le dérèglement \\
Le dernier mot \\
Le désaccord \\
Le désespoir \\
Le désespoir est-il une faute morale ? \\
Le déshonneur \\
Le design \\
Le désintéressement \\
Le désintéressement esthétique \\
Le désir de gloire \\
Le désir de pouvoir \\
Le désir de savoir \\
Le désir d'éternité \\
Le désir de vérité \\
Le désir d'originalité \\
Le désir est-il sans limite ? \\
Le désir et la loi \\
Le désir et le manque \\
Le désir métaphysique \\
Le désir n'est-il qu'inquiétude ? \\
Le désœuvrement \\
Le désordre \\
Le désordre des choses \\
Le despote peut-il être éclairé ? \\
Le despotisme \\
Le dessin et la couleur \\
Le destin \\
Le désuet \\
Le détachement \\
Le détail \\
Le déterminisme \\
Le déterminisme social \\
Le deuil \\
Le devenir \\
Le devoir d'obéissance \\
Le devoir-être \\
Le dévouement \\
Le dialogue des philosophes \\
Le dialogue entre les cultures \\
Le dieu artiste \\
L'édification morale \\
Le dire et le faire \\
Le discernement \\
Le discontinu \\
Le discours politique \\
Le divers \\
Le divertissement \\
Le divin \\
Le dogmatisme \\
Le don \\
Le don de soi \\
Le don et l'échange \\
Le donné \\
Le double \\
Le doute dans les sciences \\
Le doute est-il une faiblesse de la pensée ? \\
Le drame \\
Le droit au Bonheur \\
Le droit au travail \\
Le droit de la guerre \\
Le droit de propriété \\
Le droit de punir \\
Le droit de révolte \\
Le droit des animaux \\
Le droit des gens \\
Le droit des peuples à disposer d'eux-mêmes \\
Le droit de veto \\
Le droit de vie et de mort \\
Le droit d'ingérence \\
Le droit d'intervention \\
Le droit doit-il être le seul régulateur de la vie sociale ? \\
Le droit du plus faible \\
Le droit du plus fort \\
Le droit du premier occupant \\
Le droit est-il une science humaine ? \\
Le droit humanitaire \\
Le droit international \\
Le droit peut-il être naturel ? \\
Le dualisme \\
L'éducation artistique \\
L'éducation civique \\
L'éducation des esprits \\
L'éducation du goût \\
L'éducation esthétique \\
L'éducation peut-elle être sentimentale ? \\
L'éducation physique \\
L'éducation politique \\
Le fait de vivre est-il un bien en soi ? \\
Le fait divers \\
Le fait et le droit \\
Le fait religieux \\
Le fait scientifique \\
Le fanatisme \\
Le fantastique \\
Le faux en art \\
Le faux et l'absurde \\
Le faux et le fictif \\
Le féminin et le masculin \\
Le féminisme \\
Le fétichisme \\
Le fétichisme de la marchandise \\
L'efficacité thérapeutique de la psychanalyse \\
L'efficience \\
Le finalisme \\
Le flegme \\
Le fond \\
Le fondement \\
Le fondement de l'induction \\
Le fond et la forme \\
Le formalisme \\
Le formalisme moral \\
Le fou \\
Le fragment \\
Le frivole \\
Le futur est-il contingent ? \\
L'égalité \\
L'égalité civile \\
L'égalité des chances \\
L'égalité des conditions \\
L'égalité des hommes et des femmes est-elle une question politique ? \\
L'égalité des sexes \\
Légalité et légitimité \\
Légalité et moralité \\
Le génie \\
Le génie du lieu \\
Le génie du mal \\
Le genre et l'espèce \\
Le genre humain \\
Le geste \\
Le geste créateur \\
Le geste et la parole \\
Légitimité et légalité \\
L'égoïsme \\
Le goût \\
Le goût de la polémique \\
Le goût des autres \\
Le goût du beau \\
Le goût du pouvoir \\
Le goût est-il une faculté ? \\
Le goût est-il une vertu sociale ? \\
Le goût se forme-t-il ? \\
Le goût : certitude ou conviction ? \\
Le gouvernement des experts \\
Le gouvernement des hommes et l'administration des choses \\
Le gouvernement des hommes libres \\
Le gouvernement des meilleurs \\
Le hasard \\
Le hasard existe-t-il ? \\
Le hasard fait-il bien les choses ? \\
Le hasard n'est il que la mesure de notre ignorance ? \\
Le haut et le bas \\
Le héros moral \\
Le je ne sais quoi \\
Le jeu \\
Le jeu social \\
Le joli, le beau \\
Le jugement \\
Le jugement de goût \\
Le jugement de goût est-il universel ? \\
Le jugement dernier \\
Le jugement de valeur est-il indifférent à la vérité ? \\
Le jugement moral \\
Le jugement politique \\
Le juste et le bien \\
Le juste milieu \\
Le langage de la pensée \\
Le langage de l'art \\
Le langage des sciences \\
Le langage est-il d'essence poétique ? \\
L'élégance \\
Le législateur \\
Le libre-arbitre \\
Le lien politique \\
Le lien social \\
Le lien social peut-il être compassionnel ? \\
Le lieu \\
Le lieu commun \\
Le lieu de la pensée \\
L'éloge de la démesure \\
Le loisir \\
Le luxe \\
Le lyrisme \\
Le mal \\
Le mal constitue-t-il une objection à l'existence de Dieu ? \\
Le malentendu \\
Le malheur \\
Le malin plaisir \\
Le mal métaphysique \\
L'émancipation \\
L'émancipation des femmes \\
Le maniérisme \\
Le manifeste politique \\
Le manque de culture \\
Le marché \\
Le marché de l'art \\
Le mariage \\
Le masque \\
Le mauvais goût \\
L'embarras du choix \\
Le mécanisme et la mécanique \\
Le méchant peut-il être heureux ? \\
Le meilleur \\
Le meilleur des mondes possible \\
Le meilleur régime \\
Le meilleur régime politique \\
Le même et l'autre \\
Le mensonge \\
Le mensonge de l'art ? \\
Le mensonge en politique \\
Le mensonge politique \\
Le mépris \\
Le mépris peut-il être justifié ? \\
Le mérite \\
Le mérite est-il le critère de la vertu ? \\
Le métaphysicien est-il un physicien à sa façon ? \\
Le métier \\
Le métier de politique \\
Le métier d'homme \\
Le mien et le tien \\
Le milieu \\
Le miracle \\
Le miroir \\
Le mode \\
Le mode d'existence de l'œuvre d'art \\
Le modèle en morale \\
Le modèle organiciste \\
Le moindre mal \\
Le monde à l'envers \\
Le monde de l'animal \\
Le monde de l'art \\
Le monde de la vie \\
Le monde de l'entreprise \\
Le monde des machines \\
Le monde des œuvres \\
Le monde des physiciens \\
Le monde des rêves \\
Le monde des sens \\
Le monde du rêve \\
Le monde est-il éternel ? \\
Le monde intérieur \\
Le monde politique \\
Le monde vrai \\
Le monopole de la violence légitime \\
Le monstre \\
Le monstrueux \\
Le moralisme \\
Le moraliste \\
Le mot et la chose \\
L'émotion \\
L'émotion esthétique peut-elle se communiquer ? \\
Le mot juste \\
Le mouvement \\
Le mouvement de la pensée \\
L'empathie \\
L'empathie est-elle nécessaire aux sciences sociales ? \\
L'empire \\
L'empire sur soi \\
Le multiculturalisme \\
Le musée \\
Le mystère \\
Le mysticisme \\
Le mythe est-il objet de science ? \\
Le naïf \\
Le narcissisme \\
Le naturalisme des sciences humaines et sociales \\
Le naturel \\
Le naturel et l'artificiel \\
L'encyclopédie \\
Le néant \\
Le nécessaire et le contingent \\
Le négatif \\
L'énergie \\
L'enfance \\
L'enfance de l'art \\
L'enfance est-elle ce qui doit être surmonté ? \\
L'engagement \\
L'engagement dans l'art \\
L'engagement politique \\
Le nihilisme \\
L'ennemi \\
L'ennemi intérieur \\
L'ennui \\
Le noble et le vil \\
Le nomade \\
Le nomadisme \\
Le nombre \\
Le nombre et la mesure \\
Le nom propre \\
Le non-sens \\
Le normal et le pathologique \\
L'enquête de terrain \\
L'enquête sociale \\
L'enthousiasme \\
L'enthousiasme est-il moral ? \\
L'entraide \\
Le nu \\
L'envie \\
L'environnement est-il un nouvel objet pour les sciences humaines ? \\
Le oui-dire \\
Le pacifisme \\
Le paradigme \\
Le paradoxe \\
Le pardon \\
Le pardon et l'oubli \\
Le partage \\
Le partage des biens \\
Le partage des connaissances \\
Le partage des savoirs \\
Le partage est-il une obligation morale ? \\
Le particulier \\
Le passage à l'acte \\
Le patriarcat \\
Le patrimoine \\
Le patrimoine artistique \\
Le patriotisme \\
Le paysage \\
Le pays natal \\
Le péché \\
Le pédagogue \\
Le pessimisme \\
Le peuple et les élites \\
Le phantasme \\
L'éphémère \\
Le phénomène \\
Le philanthrope \\
Le philosophe a-t-il des leçons à donner au politique ? \\
Le philosophe est-il le vrai politique ? \\
Le philosophe-roi \\
L'épistémologie est-elle une logique de la science ? \\
Le plaisir \\
Le plaisir a-t-il un rôle à jouer dans la morale ? \\
Le plaisir de l'art \\
Le plaisir d'imiter \\
Le plaisir esthétique \\
Le plaisir esthétique suppose-t-il une culture ? \\
Le plaisir est-il un bien ? \\
Le plaisir et le bien \\
Le pluralisme \\
Le pluralisme politique \\
Le poète réinvente-t-il la langue ? \\
Le poids du passé \\
Le poids du préjugé en politique \\
Le point de vue \\
Le point de vue de l'auteur \\
Le politique a-t-il à régler les passions humaines ? \\
Le politique doit-il être un technicien ? \\
Le politique doit-il se soucier des émotions ? \\
Le politique et le religieux \\
Le politique peut-il faire abstraction de la morale ? \\
Le populaire \\
Le populisme \\
Le portrait \\
Le possible et le probable \\
Le possible et le réel \\
Le pour et le contre \\
Le pouvoir absolu \\
Le pouvoir causal de l'inconscient \\
Le pouvoir corrompt-il nécessairement ? \\
Le pouvoir corrompt-il ? \\
Le pouvoir de la science \\
Le pouvoir de l'opinion \\
Le pouvoir des images \\
Le pouvoir des mots \\
Le pouvoir des sciences humaines et sociales \\
Le pouvoir du peuple \\
Le pouvoir législatif \\
Le pouvoir peut-il limiter le pouvoir ? \\
Le pouvoir peut-il se déléguer ? \\
Le pouvoir peut-il se passer de sa mise en scène ? \\
Le pouvoir politique est-il nécessairement coercitif ? \\
Le pouvoir politique repose-t-il sur un savoir ? \\
Le pouvoir souverain \\
Le pouvoir traditionnel \\
Le préférable \\
Le préjugé \\
Le premier devoir de l'État est-il de se défendre ? \\
Le premier et le primitif \\
Le premier principe \\
Le présent \\
Le primitivisme en art \\
Le prince \\
Le principe de causalité \\
Le principe de contradiction \\
Le principe d'égalité \\
Le principe de raison \\
Le principe de réalité \\
Le principe de réciprocité \\
Le principe d'identité \\
Le privilège de l'original \\
Le probable \\
Le problème de l'être \\
Le processus \\
Le processus de civilisation \\
Le prochain \\
Le proche et le lointain \\
Le profane \\
Le progrès \\
Le progrès des sciences \\
Le progrès des sciences infirme-t-il les résultats anciens ? \\
Le progrès en logique \\
Le progrès moral \\
Le progrès scientifique fait-il disparaître la superstition ? \\
Le progrès technique \\
Le projet \\
Le projet d'une paix perpétuelle est-il insensé ? \\
Le propre \\
Le propre de la musique \\
Le propriétaire \\
Le psychisme est-il objet de connaissance ? \\
Le public \\
Le public et le privé \\
Le pur et l'impur \\
Lequel, de l'art ou du réel, est-il une imitation de l'autre ? \\
L'équilibre des pouvoirs \\
L'équité \\
L'équivalence \\
L'équivoque \\
Le quotidien \\
Le raffinement \\
Le raisonnement par l'absurde \\
Le raisonnement scientifique \\
Le raisonnement suit-il des règles ? \\
Le rapport de l'homme à son milieu a-t-il une dimension morale ? \\
Le rationnel et le raisonnable \\
Le réalisme \\
Le réalisme de la science \\
Le récit \\
Le récit en histoire \\
Le réel est-il ce qui résiste ? \\
Le réel est-il rationnel ? \\
Le réel et le virtuel \\
Le réel peut-il être contradictoire ? \\
Le refoulement \\
Le refus \\
Le regard \\
Le regard du photographe \\
Le règlement politique des conflits ? \\
Le règne de l'homme \\
Le relativisme \\
Le relativisme culturel \\
Le relativisme moral \\
Le remords \\
Le renoncement \\
Le repentir \\
Le repos \\
Le respect \\
Le respect des convenances \\
Le respect des institutions \\
Le ressentiment \\
Le rêve \\
Le rêve et la veille \\
Le rien \\
Le rigorisme \\
Le risque \\
Le rôle de la théorie dans l'expérience scientifique \\
Le rôle des institutions \\
L'érotisme \\
Le royaume du possible \\
L'erreur et la faute \\
L'erreur et l'ignorance \\
L'erreur peut-elle jouer un rôle dans la connaissance scientifique ? \\
L'erreur politique, la faute politique \\
L'erreur scientifique \\
L'érudition \\
Le rythme \\
Le sacré \\
Le sacré et le profane \\
Le sacrifice \\
Les affaires publiques \\
Les affects sont-ils des objets sociologiques ? \\
Les agents sociaux poursuivent-ils l'utilité ? \\
Les agents sociaux sont-ils rationnels ? \\
Le salut \\
Les amis \\
Les analogies dans les sciences humaines \\
Les anciens et les modernes \\
Les animaux échappent-ils à la moralité ? \\
Les animaux ont-ils des droits ? \\
Les animaux pensent-ils ? \\
Les antagonismes sociaux \\
Les apparences font-elles partie du monde ? \\
Les archives \\
Les arts appliqués \\
Les arts communiquent-ils entre eux ? \\
Les arts de la mémoire \\
Les arts industriels \\
Les arts mineurs \\
Les arts nobles \\
Les arts ont-ils besoin de théorie ? \\
Les arts populaires \\
Les arts vivants \\
Le sauvage et le barbare \\
Le sauvage et le cultivé \\
Le savant et le politique \\
Le savoir a-t-il besoin d'être fondé ? \\
Le savoir du peintre \\
Le savoir émancipe-t-il ? \\
Le savoir est-il libérateur ? \\
Le savoir-faire \\
Le savoir se vulgarise-t-il ? \\
Le savoir utile au citoyen \\
Les beautés de la nature \\
Les beaux-arts sont-ils compatibles entre eux ? \\
Les bénéfices du doute \\
Les bénéfices moraux \\
Les biens communs \\
Les blessures de l'esprit \\
Les bonnes intentions \\
Les bonnes mœurs \\
Les bons sentiments \\
Le scandale \\
Les caractères moraux \\
Les catégories \\
Les causes et les effets \\
Les causes et les lois \\
Les causes finales \\
Les changements scientifiques et la réalité \\
Les chemins de traverse \\
Les choses \\
Les choses ont-elles une essence ? \\
Les cinq sens \\
Les circonstances \\
Les classes sociales \\
L'esclavage \\
L'esclave \\
L'esclave et son maître \\
Les conditions de la démocratie \\
Les conflits politiques \\
Les conflits politiques ne sont-ils que des conflits sociaux ? \\
Les conflits sociaux \\
Les conflits sociaux sont-ils des conflits de classe ? \\
Les conflits sociaux sont-ils des conflits politiques ? \\
Les connaissances scientifiques peuvent-elles être à la fois vraies et provisoires ? \\
Les connaissances scientifiques peuvent-elles être vulgarisées ? \\
Les conquêtes de la science \\
Les conséquences de l'action \\
Les coutumes \\
Les critères de vérité dans les sciences humaines \\
Les croyances politiques \\
Le scrupule \\
Les cultures sont-elles incommensurables ? \\
Les degrés de conscience \\
Les degrés de la beauté \\
Les devoirs à l'égard de la nature \\
Les devoirs de l'État \\
Les devoirs envers soi-même \\
Les dictionnaires \\
Les dilemmes moraux \\
Les dispositions sociales \\
Les distinctions sociales \\
Les divisions sociales \\
Les droits de l'enfant \\
Les droits de l'homme \\
Les droits de l'homme et ceux du citoyen \\
Les droits de l'homme ont-ils un fondement moral ? \\
Les droits de l'homme sont-ils une abstraction ? \\
Les droits et les devoirs \\
Les droits naturels imposent-ils une limite à la politique ? \\
Le secret \\
Le secret d'État \\
Les effets de l'esclavage \\
Les éléments \\
Le sens commun \\
Le sens de la mesure \\
Le sens de la situation \\
Le sens de l'État \\
Le sens de l'histoire \\
Le sens de l'Histoire \\
Le sens de l'humour \\
Le sens des mots \\
Les ensembles \\
Le sensible \\
Le sens interne \\
Le sens moral \\
Le sens musical \\
Le sentiment de l'existence \\
Le sentiment de l'injustice \\
Le sentiment esthétique \\
Le sentiment moral \\
Les entités mathématiques sont-elles des fictions ? \\
Les envieux \\
Le sérieux \\
Le serment \\
Les études comparatives \\
Les factions politiques \\
Les faits parlent-ils d'eux-mêmes ? \\
Les fausses sciences \\
Les fins de l'art \\
Les fins de l'éducation \\
Les fins dernières \\
Les fins naturelles et les fins morales \\
Les fonctions de l'image \\
Les fondements de l'État \\
Les forts et les faibles \\
Les foules \\
Les fous \\
Les frontières \\
Les frontières de l'art \\
Les fruits du travail \\
Les genres de Dieu \\
Les genres esthétiques \\
Les genres naturels \\
Les grands hommes \\
Les hasards de la vie \\
Les hommes de pouvoir \\
Les hommes et les dieux \\
Les hommes et les femmes \\
Les hommes n'agissent-ils que par intérêt ? \\
Les hommes sont-ils naturellement sociables ? \\
Les idées et les choses \\
Les idées politiques \\
Les idoles \\
Le silence \\
Le silence des lois \\
Les images empêchent-elles de penser ? \\
Le simple \\
Le simulacre \\
Les individus \\
Les industries culturelles \\
Les inégalités sociales \\
Les inégalités sociales sont-elles inévitables ? \\
Le singulier \\
Le singulier est-il objet de connaissance ? \\
Le singulier et le pluriel \\
Les institutions artistiques \\
Les instruments de la pensée \\
Les intentions de l'artiste \\
Les intentions et les conséquences \\
Les interdits \\
Les intérêts particuliers peuvent-ils tempérer l'autorité politique ? \\
Les invariants culturels \\
Les jeux du pouvoir \\
Les jugements analytiques \\
Les leçons de morale \\
Les libertés civiles \\
Les libertés fondamentales \\
Les liens sociaux \\
Les lieux du pouvoir \\
Les limites de la connaissance scientifique \\
Les limites de la démocratie \\
Les limites de la description \\
Les limites de la raison \\
Les limites de la tolérance \\
Les limites de la vertu \\
Les limites de l'État \\
Les limites de l'expérience \\
Les limites de l'humain \\
Les limites de l'imagination \\
Les limites du corps \\
Les limites du pouvoir \\
Les limites du pouvoir politique \\
Les lois causales \\
Les lois de la guerre \\
Les lois de la nature sont-elles contingentes ? \\
Les lois de la nature sont-elles de simples régularités ? \\
Les lois de la nature sont elles nécessaires ? \\
Les lois de l'art \\
Les lois de l'histoire \\
Les lois de l'hospitalité \\
Les lois du sang \\
Les lois scientifiques sont-elles des lois de la nature ? \\
Les lois sont-elles seulement utiles ? \\
Les machines \\
Les maladies de l'âme \\
Les marginaux \\
Les mathématiques du mouvement \\
Les mathématiques et la pensée de l'infini \\
Les mathématiques sont-elles réductibles à la logique ? \\
Les mathématiques sont-elles un langage ? \\
Les mathématiques sont-elles utiles au philosophe ? \\
Les mécanismes cérébraux \\
Les modalités \\
Les modèles \\
Les mœurs \\
Les mœurs et la morale \\
Les mondes possibles \\
Les mots et les choses \\
Les mots justes \\
Les moyens de l'autorité \\
Les moyens et la fin \\
Les moyens et les fins en art \\
Les nombres gouvernent-ils le monde ? \\
Les noms \\
Les noms propres \\
Les normes \\
Les normes du vivant \\
Les normes esthétiques \\
Les normes et les valeurs \\
Les nouvelles technologies transforment-elles l'idée de l'art ? \\
Les objets scientifiques \\
Le social et le politique \\
Les œuvres d'art ont-elles besoin d'un commentaire ? \\
Le sommeil et la veille \\
Les opérations de la pensée \\
Les opinions politiques \\
Le souci d'autrui résume-t-il la morale ? \\
Le souci de soi \\
Le souci du bien-être est-il politique ? \\
Le souverain bien \\
L'espace et le lieu \\
L'espace et le territoire \\
L'espace public \\
Les paroles et les actes \\
Les parties de l'âme \\
Les passions peuvent-elles être raisonnables ? \\
Les passions politiques \\
Les pauvres \\
L'espèce et l'individu \\
Le spectacle \\
Le spectacle de la nature \\
Le spectacle de la pensée \\
L'espérance est-elle une vertu ? \\
Les peuples ont-ils les gouvernements qu'ils méritent ? \\
Les plaisirs \\
Les plaisirs de l'amitié \\
Les poètes et la cité \\
Les pouvoirs de la religion \\
Les préjugés moraux \\
Les prêtres \\
Les principes de la démonstration \\
Les principes d'une science sont-ils des conventions ? \\
Les principes moraux \\
L'esprit critique \\
L'esprit de finesse \\
L'esprit de système \\
L'esprit est-il matériel ? \\
L'esprit est-il objet de science ? \\
L'esprit et la machine \\
L'esprit peut-il être malade ? \\
L'esprit peut-il être mesuré ? \\
L'esprit scientifique \\
Les problèmes politiques peuvent-ils se ramener à des problèmes techniques ? \\
Les problèmes politiques sont-ils des problèmes techniques ? \\
Les propositions métaphysiques sont-elles des illusions ? \\
Les proverbes \\
Les proverbes enseignent-ils quelque chose ? \\
Les proverbes nous instruisent-ils moralement ? \\
Les qualités esthétiques \\
Les questions métaphysiques ont-elles un sens ? \\
L'esquisse \\
Les raisons de vivre \\
Les règles de l'art \\
Les règles du jeu \\
Les règles d'un bon gouvernement \\
Les règles sociales \\
Les relations \\
Les représentants du peuple \\
Les reproductions \\
Les ressources humaines \\
Les révolutions scientifiques \\
Les révolutions techniques suscitent-elles des révolutions dans l'art ? \\
Les riches et les pauvres \\
Les rituels \\
Les rôles sociaux \\
Les ruines \\
Les sacrifices \\
Les sauvages \\
Les sciences décrivent-elles le réel ? \\
Les sciences de la vie et de la Terre \\
Les sciences de la vie visent-elles un objet irréductible à la matière ? \\
Les sciences de l'éducation \\
Les sciences de l'esprit \\
Les sciences de l'homme et l'évolution \\
Les sciences de l'homme ont-elles inventé leur objet ? \\
Les sciences de l'homme permettent-elles d'affiner la notion de responsabilité ? \\
Les sciences de l'homme peuvent-elles expliquer l'impuissance de la liberté ? \\
Les sciences de l'homme rendent-elles l'homme prévisible ? \\
Les sciences doivent-elle prétendre à l'unification ? \\
Les sciences du comportement \\
Les sciences et le vivant \\
Les sciences exactes \\
Les sciences forment-elle un système ? \\
Les sciences historiques \\
Les sciences humaines doivent-elles être transdisciplinaires ? \\
Les sciences humaines éliminent-elles la contingence du futur ? \\
Les sciences humaines et le droit \\
Les sciences humaines nous protègent-elles de l'essentialisme ? \\
Les sciences humaines ont-elles un objet commun ? \\
Les sciences humaines permettent-elles de comprendre la vie d'un homme ? \\
Les sciences humaines peuvent-elles adopter les méthodes des sciences de la nature ? \\
Les sciences humaines peuvent-elles se passer de la notion d'inconscient ? \\
Les sciences humaines présupposent-elles une définition de l'homme ? \\
Les sciences humaines sont-elles des sciences de la nature humaine ? \\
Les sciences humaines sont-elles des sciences de la vie humaine ? \\
Les sciences humaines sont-elles des sciences d'interprétation ? \\
Les sciences humaines sont-elles des sciences ? \\
Les sciences humaines sont-elles explicatives ou compréhensives ? \\
Les sciences humaines sont-elles normatives ? \\
Les sciences humaines sont-elles relativistes ? \\
Les sciences humaines sont-elles subversives ? \\
Les sciences humaines traitent-elles de l'individu ? \\
Les sciences humaines transforment-elles la notion de causalité ? \\
Les sciences naturelles \\
Les sciences ont-elles besoin d'une fondation métaphysique ? \\
Les sciences peuvent-elles penser l'individu ? \\
Les sciences sociales \\
Les sciences sociales peuvent-elles être expérimentales ? \\
Les sciences sociales sont-elles nécessairement inexactes ? \\
L'essence \\
Les sens peuvent-ils nous tromper ? \\
Les sentiments \\
Les sentiments peuvent-ils s'apprendre ? \\
Les services publics \\
Les signes de l'intelligence \\
Les sociétés évoluent-elles ? \\
Les sociétés ont-elles un inconscient ? \\
Les sociétés sont-elles hiérarchisables ? \\
Les sociétés sont-elles imprévisibles ? \\
Les structures expliquent-elles tout ? \\
Les systèmes \\
Le statut de l'axiome \\
Le statut des hypothèses dans la démarche scientifique \\
Les techniques artistiques \\
Les théories scientifiques sont-elles vraies ? \\
L'esthète \\
L'esthétique est-elle une métaphysique de l'art ? \\
L'esthétisme \\
L'estime de soi \\
Les traditions \\
Le style \\
Le sublime \\
Le succès \\
Le sujet \\
Le sujet de l'action \\
Le sujet de la pensée \\
Le sujet et l'objet \\
Le sujet moral \\
Les universaux \\
Les usages de l'art \\
Les valeurs de la République \\
Les vérités éternelles \\
Les vérités scientifiques sont-elles relatives ? \\
Les vertus \\
Les vertus politiques \\
Les visages du mal \\
Les vivants et les morts \\
Le syllogisme \\
Le symbole \\
Le symbolisme \\
Le symbolisme mathématique \\
Le système des beaux-arts \\
Le système des besoins \\
Le tableau \\
Le talent \\
Le talent et le génie \\
L'État a-t-il pour finalité de maintenir l'ordre ? \\
L'État de droit \\
L'état de la nature \\
L'état de nature \\
L'état d'exception \\
L'État doit-il disparaître ? \\
L'État doit-il éduquer le citoyen ? \\
L'État doit-il éduquer les citoyens ? \\
L'État doit-il faire le bonheur des citoyens ? \\
L'État est-il fin ou moyen ? \\
L'État est-il le garant de la propriété privée ? \\
L'État et la culture \\
L'État et la Nation \\
L'État et le marché \\
L'État et les Églises \\
L'État libéral \\
L'État peut-il créer la liberté ? \\
L'État peut-il être indifférent à la religion ? \\
L'État-providence \\
L'État universel \\
Le témoignage \\
Le temps est-il une dimension de la nature ? \\
Le temps ne fait-il que passer ? \\
Le temps perdu \\
Le temps se laisse-t-il décrire logiquement ? \\
L'éternel présent \\
L'éternité \\
L'éternité n'est-elle qu'une illusion ? \\
Le terrain \\
Le territoire \\
Le théâtre du monde \\
L'éthique à l'épreuve du tragique \\
L'éthique des plaisirs \\
L'éthique est-elle affaire de choix ? \\
L'éthique suppose-t-elle la liberté ? \\
L'ethnocentrisme \\
Le tiers exclu \\
L'étonnement \\
Le totalitarisme \\
Le totémisme \\
Le toucher \\
Le tourment moral \\
Le tout est-il la somme de ses parties ? \\
Le tragique \\
Le trait d'esprit \\
L'étranger \\
L'étrangeté \\
Le travail \\
Le travail artistique \\
Le travail est-il une valeur morale ? \\
Le travail rapproche-t-il les hommes ? \\
Le travail sur le terrain \\
L'être de la conscience \\
L'être de la vérité \\
L'être de l'image \\
L'être du possible \\
L'être en tant qu'être \\
L'être en tant qu'être est-il connaissable ? \\
L'être et la volonté \\
L'être et le bien \\
L'être et le néant \\
L'être et les êtres \\
L'être et l'essence \\
L'être et l'étant \\
L'être et le temps \\
L'être se confond-il avec l'être perçu ? \\
Le tribunal de l'histoire \\
Le vainqueur a-t-il tous les droits ? \\
Le vécu \\
L'événement \\
L'événement et le fait divers \\
L'événement manque-t-il d'être ? \\
Le verbalisme \\
Le verbe \\
Le vide \\
L'évidence \\
Le village global \\
Le virtuel \\
Le visage \\
Le visible et l'invisible \\
Le vivant comme problème pour la philosophie des sciences \\
Le volontaire et l'involontaire \\
L'évolution \\
L'évolution des langues \\
Le voyage \\
Le vrai a-t-il une histoire ? \\
Le vrai est-il à lui-même sa propre marque ? \\
Le vrai et le vraisemblable \\
Le vrai peut-il rester invérifiable ? \\
Le vraisemblable \\
Le vrai se réduit-il à l'utile ? \\
Le vulgaire \\
L'exactitude \\
L'excellence \\
L'exception \\
L'excès et le défaut \\
L'exclusion \\
L'excuse \\
L'exécution d'une œuvre d'art est-elle toujours une œuvre d'art ? \\
L'exemplaire \\
L'exemplarité \\
L'exemple \\
L'exercice de la vertu \\
L'exercice du pouvoir \\
L'exercice solitaire du pouvoir \\
L'exigence de vérité a-t-elle un sens moral ? \\
L'exigence morale \\
L'exil \\
L'existence de l'État dépend-elle d'un contrat ? \\
L'existence du mal \\
L'existence se démontre-t-elle ? \\
L'expérience \\
L'expérience artistique \\
L'expérience cruciale \\
L'expérience directe est-elle une connaissance ? \\
L'expérience en sciences humaines \\
L'expérience enseigne-elle quelque chose ? \\
L'expérience et l'expérimentation \\
L'expérience métaphysique \\
L'expérience morale \\
L'expérience sensible est-elle la seule source légitime de connaissance ? \\
L'expérimentation \\
L'expérimentation en psychologie \\
L'expérimentation en sciences sociales \\
L'expérimentation sur le vivant \\
L'expert et l'amateur \\
L'expertise \\
L'expertise politique \\
L'explication scientifique \\
L'exploitation de l'homme par l'homme \\
L'exposition \\
L'exposition de l'œuvre d'art \\
L'expression \\
L'expression artistique \\
L'expression de l'inconscient \\
L'expressivité musicale \\
L'extériorité \\
L'habileté \\
L'habitation \\
L'habitude \\
L'harmonie \\
L'hégémonie politique \\
L'héritage \\
L'hésitation \\
L'hétérogénéité sociale \\
L'hétéronomie \\
L'hétéronomie de l'art \\
L'histoire a-t-elle un sens ? \\
L'histoire de l'art \\
L'histoire de l'art est-elle celle des styles ? \\
L'histoire de l'art est-elle finie ? \\
L'histoire des arts est-elle liée à l'histoire des techniques ? \\
L'histoire des civilisations \\
L'histoire des sciences \\
L'histoire des sciences est-elle une histoire ? \\
L'histoire est-elle déterministe ? \\
L'histoire est-elle un roman vrai ? \\
L'histoire est-elle utile à la politique ? \\
L'histoire et la géographie \\
L'histoire peut-elle se répéter ? \\
L'histoire universelle est-elle l'histoire des guerres ? \\
L'histoire : enquête ou science ? \\
L'histoire : science ou récit ? \\
L'homme a-t-il une nature ? \\
L'homme de la rue \\
L'homme des droits de l'homme n'est-il qu'une fiction ? \\
L'homme des foules \\
L'homme des sciences de l'homme ? \\
L'homme des sciences humaines \\
L'homme d'État \\
L'homme est-il la mesure de toutes choses ? \\
L'homme est-il objet de science ? \\
L'homme est-il prisonnier du temps ? \\
L'homme est-il un animal politique ? \\
L'homme et la bête \\
L'homme et la machine \\
L'homme et la nature sont-ils commensurables ? \\
L'homme et le citoyen \\
L'homme injuste peut-il être heureux ? \\
L'homme, le citoyen, le soldat \\
L'homme peut-il changer ? \\
L'honnêteté \\
L'honneur \\
L'horizon \\
L'horreur \\
L'horrible \\
L'hospitalité \\
L'hospitalité a-t-elle un sens politique ? \\
L'hospitalité est-elle un devoir ? \\
L'humiliation \\
L'humilité \\
L'humour \\
L'humour et l'ironie \\
L'hybridation des arts \\
L'hypocrisie \\
L'hypothèse \\
L'hypothèse de l'inconscient \\
Liberté, égalité, fraternité \\
Liberté et libération \\
Liberté et nécessité \\
Liberté humaine et liberté divine \\
Liberté réelle, liberté formelle \\
Libertés publiques et culture politique \\
L'idéal de l'art \\
L'idéal de vérité \\
L'idéalisme \\
L'idéaliste \\
L'idéalité \\
L'idéal moral est-il vain ? \\
L'idéal-type \\
L'idée d'anthropologie \\
L'idée de beaux arts \\
L'idée de communauté \\
L'idée de connaissance approchée \\
L'idée de conscience collective \\
L'idée de continuité \\
L'idée de contrat social \\
L'idée de création \\
L'idée de crise \\
L'idée de Dieu \\
L'idée de domination \\
L'idée de forme sociale \\
L'idée de langue universelle \\
L'idée de logique \\
L'idée de logique transcendantale \\
L'idée de logique universelle \\
L'idée de loi logique \\
L'idée de loi naturelle \\
L'idée de mathesis universalis \\
L'idée de morale appliquée \\
L'idée de nation \\
L'idée d'encyclopédie \\
L'idée de norme \\
L'idée de perfection \\
L'idée de république \\
L'idée de rétribution est-elle nécessaire à la morale ? \\
L'idée de révolution \\
L'idée de science expérimentale \\
L'idée de substance \\
L'idée d'exactitude \\
L'idée de « sciences exactes » \\
L'idée d'un commencement absolu \\
L'idée d'une langue universelle \\
L'idée d'une science bien faite \\
L'idée esthétique \\
L'identité \\
L'identité et la différence \\
L'identité personnelle \\
L'idéologie \\
L'idolâtrie \\
L'ignoble \\
L'ignorance nous excuse-t-elle ? \\
L'illimité \\
L'illusion \\
L'illustration \\
L'image \\
L'imaginaire et le réel \\
L'imagination dans l'art \\
L'imagination dans les sciences \\
L'imagination esthétique \\
L'imagination nous éloigne-t-elle du réel ? \\
L'imagination politique \\
L'imitation \\
L'imitation a-t-elle une fonction morale ? \\
L'immanence \\
L'immédiat \\
L'immensité \\
L'immortalité de l'âme \\
L'immortalité des œuvres d'art \\
L'immuable \\
L'immutabilité \\
L'impardonnable \\
L'impartialité \\
L'impensable \\
L'impératif \\
L'imperceptible \\
L'implicite \\
L'importance des détails \\
L'impossible \\
L'imposteur \\
L'imprescriptible \\
L'impression \\
L'imprévisible \\
L'improbable \\
L'improvisation \\
L'improvisation dans l'art \\
L'imprudence \\
L'impuissance \\
L'impuissance de la raison \\
L'impuissance de l'État \\
L'impunité \\
L'inachevé \\
L'inaction \\
L'inapparent \\
L'incarnation \\
L'incertitude est-elle dans les choses ou dans les idées ? \\
L'incommensurabilité \\
L'incompréhensible \\
L'inconcevable \\
L'inconnu \\
L'inconscience \\
L'inconscient \\
L'inconscient collectif \\
L'inconscient de l'art \\
L'inconséquence \\
L'incorporel \\
L'incrédulité \\
L'inculture \\
L'indécidable \\
L'indécision \\
L'indéfini \\
L'indépassable \\
L'indétermination \\
L'indéterminé \\
L'indicible \\
L'indifférence \\
L'indifférence à la politique \\
L'individu \\
L'individualisme \\
L'individualisme a-t-il sa place en politique ? \\
L'individualisme méthodologique \\
L'individuel et le collectif \\
L'individu et la multitude \\
L'individu et le groupe \\
L'indivisible \\
L'induction \\
L'induction et la déduction \\
L'indulgence \\
L'industrie culturelle \\
L'industrie du beau \\
L'inégalité des chances \\
L'inégalité entre les hommes \\
L'inégalité naturelle \\
L'inertie \\
L'inesthétique \\
L'inexactitude et le savoir scientifique \\
L'infâme \\
L'infamie \\
L'inférence \\
L'infini \\
L'infinité de l'espace \\
L'influence \\
L'information \\
L'informe \\
L'informe et le difforme \\
L'ingratitude \\
L'inhibition \\
L'inhumain \\
L'inimaginable \\
L'inimitié \\
L'inintelligible \\
L'initiation \\
L'injonction \\
L'injustice \\
L'injustifiable \\
L'innocence \\
L'innommable \\
L'inobservable \\
L'inquiétant \\
L'inquiétude \\
L'insensé \\
L'insignifiant \\
L'insociable sociabilité \\
L'insouciance \\
L'insoumission \\
L'insoutenable \\
L'inspiration \\
L'instant \\
L'instinct \\
L'institution \\
L'institutionnalisation des conduites \\
L'institution scientifique \\
L'institution scolaire \\
L'instruction est-elle facteur de moralité ? \\
L'instrument mathématique en sciences humaines \\
L'instrument scientifique \\
L'insulte \\
L'insurrection \\
L'intangible \\
L'intellectuel \\
L'intelligence \\
L'intelligence de la main \\
L'intelligence de la matière \\
L'intelligence des bêtes \\
L'intelligence des foules \\
L'intelligence du sensible \\
L'intelligence du vivant \\
L'intelligence politique \\
L'intelligible \\
L'intempérance \\
L'intemporel \\
L'intention \\
L'intention morale \\
L'intention morale suffit-elle à constituer la valeur morale de l'action ? \\
L'intentionnalité \\
L'interdit \\
L'intérêt \\
L'intérêt bien compris \\
L'intérêt commun \\
L'intérêt général est-il le bien commun ? \\
L'intérêt peut-il être une valeur morale ? \\
L'intérêt public est-il une illusion ? \\
L'intérieur et l'extérieur \\
L'intériorisation des normes \\
L'intériorité \\
L'intériorité de l'œuvre \\
L'interprétation de la loi \\
L'interprétation des œuvres \\
L'interprétation est-elle un art ? \\
L'interrogation humaine \\
L'intime conviction \\
L'intimité \\
L'intolérable \\
L'intolérance \\
L'intraduisible \\
L'intransigeance \\
L'intuition \\
L'intuition a-t-elle une place en logique ? \\
L'intuition en mathématiques \\
L'intuition morale \\
L'inutile \\
L'invention \\
L'invention de soi \\
L'invisibilité \\
L'invisible \\
L'involontaire \\
Lire et écrire \\
L'ironie \\
L'irrationnel \\
L'irrationnel et le politique \\
L'irréel \\
L'irréfutable \\
L'irrégularité \\
L'irréparable \\
L'irreprésentable \\
L'irrésolution \\
L'irresponsabilité \\
L'irréversible \\
L'irrévocable \\
Littérature et réalité \\
L'ivresse \\
L'obéissance \\
L'obéissance à l'autorité \\
L'objectivité \\
L'objectivité de l'art \\
L'objectivité historique \\
L'objet \\
L'objet d'amour \\
L'objet de culte \\
L'objet de la littérature \\
L'objet de l'amour \\
L'objet de la politique \\
L'objet de la réflexion \\
L'objet de l'art \\
L'obligation \\
L'obligation d'échanger \\
L'obligation morale \\
L'obligation morale peut-elle se réduire à une obligation sociale ? \\
L'obscène \\
L'obscénité \\
L'obscurité \\
L'observation \\
L'observation participante \\
L'obsession \\
L'obstacle \\
L'obstacle épistémologique \\
L'occasion \\
L'œil et l'oreille \\
L'œuvre anonyme \\
L'œuvre d'art est-elle l'expression d'une idée ? \\
L'œuvre d'art est-elle toujours destinée à un public ? \\
L'œuvre d'art est-elle une belle apparence ? \\
L'œuvre d'art et sa reproduction \\
L'œuvre d'art et son auteur \\
L'œuvre d'art nous apprend-elle quelque chose ? \\
L'œuvre d'art représente-t-elle quelque chose ? \\
L'œuvre d'art totale \\
L'œuvre de fiction \\
L'œuvre de l'historien \\
L'œuvre et le produit \\
L'œuvre inachevée \\
L'offense \\
Logique et dialectique \\
Logique et existence \\
Logique et logiques \\
Logique et mathématique \\
Logique et mathématiques \\
Logique et métaphysique \\
Logique et méthode \\
Logique et ontologie \\
Logique et psychologie \\
Logique et vérité \\
Logique générale et logique transcendantale \\
Loi morale et loi politique \\
Loi naturelle et loi politique \\
Lois et règles en logique \\
L'oisiveté \\
L'oligarchie \\
L'ombre et la lumière \\
L'omniscience \\
L'opinion droite \\
L'opinion du citoyen \\
L'opinion publique \\
L'opinion vraie \\
L'opportunisme \\
L'opposant \\
L'opposition \\
L'ordinaire est-il ennuyeux ? \\
L'ordre \\
L'ordre des choses \\
L'ordre du monde \\
L'ordre du temps \\
L'ordre établi \\
L'ordre et la mesure \\
L'ordre moral \\
L'ordre politique peut-il exclure la violence ? \\
L'ordre public \\
L'ordre social \\
L'organique et le mécanique \\
L'organisation \\
L'orgueil \\
L'orientation \\
L'original et la copie \\
L'originalité \\
L'originalité en art \\
L'origine \\
L'origine de la culpabilité \\
L'origine de la négation \\
L'origine de l'art \\
L'origine des croyances \\
L'origine des langues \\
L'origine des langues est-elle un faux problème ? \\
L'origine des vertus \\
L'origine et le fondement \\
L'ornement \\
L'oubli \\
L'oubli des fautes \\
L'oubli est-il un échec de la mémoire ? \\
L'outil \\
L'un \\
L'unanimité est-elle un critère de légitimité ? \\
L'un est le multiple \\
L'un et le multiple \\
L'un et l'être \\
L'unité \\
L'unité dans le beau \\
L'unité de l'art \\
L'unité de la science \\
L'unité de l'œuvre d'art \\
L'unité des contraires \\
L'unité des langues \\
L'unité des sciences \\
L'unité des sciences humaines \\
L'unité des sciences humaines ? \\
L'unité du corps politique \\
L'univers \\
L'universel \\
L'universel et le particulier \\
L'universel et le singulier \\
L'univocité de l'étant \\
L'urbanité \\
L'urgence \\
L'usage \\
L'usage des fictions \\
L'usage des généalogies \\
L'usage des mots \\
L'usage des passions \\
L'usage du monde \\
L'utile et l'agréable \\
L'utilité de la poésie \\
L'utilité de l'art \\
L'utilité des préjugés \\
L'utilité des sciences humaines \\
L'utilité est-elle étrangère à la morale ? \\
L'utilité publique \\
L'utopie \\
L'utopie a-t-elle une signification politique ? \\
L'utopie en politique \\
Machine et organisme \\
Machines et liberté \\
Machines et mémoire \\
Magie et religion \\
Maître et serviteur \\
Maîtriser l'absence \\
Manger \\
Manquer de jugement \\
Masculin, féminin \\
Mathématiques et réalité \\
Mathématiques pures et mathématiques appliquées \\
Matière et corps \\
Matière et matériaux \\
Ma vraie nature \\
Mécanisme et finalité \\
Mémoire et fiction \\
Mémoire et responsabilité \\
Ménager les apparences \\
Mensonge et politique \\
Mentir \\
Mesurer \\
Métaphysique et histoire \\
Métaphysique et ontologie \\
Métaphysique et religion \\
Métaphysique spéciale, métaphysique générale \\
Métier et vocation \\
Mettre en ordre \\
Microscope et télescope \\
Misère et pauvreté \\
Mœurs, coutumes, lois \\
Moi d'abord \\
Mon corps \\
Mon corps est-il ma propriété ? \\
Mon corps m'appartient-il ? \\
Monde et nature \\
Montrer et démontrer \\
Morale et convention \\
Morale et éducation \\
Morale et histoire \\
Morale et liberté \\
Morale et politique sont-elles indépendantes ? \\
Morale et pratique \\
Morale et prudence \\
Morale et religion \\
Morale et sexualité \\
Morale et société \\
Morale et violence \\
Mourir \\
Mourir dans la dignité \\
Mourir pour des principes \\
Mourir pour la patrie \\
Murs et frontières \\
Musique et bruit \\
Mythe et histoire \\
Mythe et philosophie \\
Mythe et symbole \\
Mythes et idéologies \\
Naître \\
Nation et richesse \\
Nature et fonction du sacrifice \\
Nature et histoire \\
Nature et institutions \\
Naviguer \\
N'échange-t-on que des symboles ? \\
Négation et privation \\
Ne lèse personne \\
Ne pas raconter d'histoires \\
Ne pas savoir ce que l'on fait \\
Ne penser à rien \\
Ne penser qu'à soi \\
Ne prêche-t-on que les convertis ? \\
N'est-on juste que par crainte du châtiment ? \\
Névroses et psychoses \\
N'exprime t-on que ce dont on a conscience ? \\
Ni Dieu ni maître \\
Ni Dieu, ni maître \\
Nier le monde \\
Nier l'évidence \\
Ni regrets, ni remords \\
Nomade et sédentaire \\
Nommer \\
Normes morales et normes vitales \\
Notre besoin de fictions \\
Notre connaissance du réel se limite-t-elle au savoir scientifique ? \\
Notre corps pense-t-il ? \\
Notre ignorance nous excuse-t-elle ? \\
Nul n'est censé ignorer la loi \\
N'y a t-il de bonheur que dans l'instant ? \\
N'y a-t-il de rationalité que scientifique ? \\
N'y a-t-il de science qu'autant qu'il s'y trouve de mathématique ? \\
N'y a-t-il de science que du général ? \\
N'y a-t-il de sens que par le langage ? \\
N'y a-t-il de vérité que scientifique ? \\
N'y a-t-il qu'une substance ? \\
N'y a-t-il qu'un seul monde ? \\
Obéir \\
Obéir, est-ce se soumettre ? \\
Observation et expérimentation \\
Observer \\
Œuvre et événement \\
Ordre et désordre \\
Ordre et liberté \\
Organisme et milieu \\
Origine et commencement \\
Où commence la servitude ? \\
Où est le passé ? \\
Où est le pouvoir ? \\
Où est-on quand on pense ? \\
Où s'arrête l'espace public ? \\
Où sont les relations ? \\
Où suis-je quand je pense ? \\
Où suis-je ? \\
Par-delà beauté et laideur \\
Pardonner et oublier \\
Parfaire \\
Parier \\
Parler, est-ce communiquer ? \\
Parler pour ne rien dire \\
Par où commencer ? \\
Par quoi un individu se distingue-t-il d'un autre ? \\
Partager les richesses \\
Passer du fait au droit \\
Pâtir \\
Peindre \\
Peindre d'après nature \\
Peinture et histoire \\
Peinture et réalité \\
Pensée et réalité \\
Penser est-ce calculer ? \\
Penser, est-ce calculer ? \\
Penser et calculer \\
Penser et parler \\
Penser la technique \\
Penser le réel \\
Penser les sociétés comme des organismes \\
Penser par soi-même \\
Penser requiert-il un corps ? \\
Penser sans corps \\
Perception et aperception \\
Perception et jugement \\
Perception et mouvement \\
Percevoir, est-ce connaître ? \\
Percevoir et imaginer \\
Percevoir et sentir \\
Perdre la mémoire \\
Perdre ses habitudes \\
Perdre ses illusions \\
Perdre son âme \\
Persévérer dans son être \\
Persuader \\
Persuader et convaincre \\
Peuple et culture \\
Peuple et masse \\
Peuple et société \\
Peuples et masses \\
Peut-il être moral de tuer ? \\
Peut-il y avoir de bons tyrans ? \\
Peut-il y avoir de la politique sans conflit ? \\
Peut-il y avoir science sans intuition du vrai ? \\
Peut-il y avoir un droit à désobéir ? \\
Peut-il y avoir une philosophie applicable ? \\
Peut-il y avoir une philosophie politique sans Dieu ? \\
Peut-il y avoir une science politique ? \\
Peut-il y avoir une société des nations ? \\
Peut-il y avoir une société sans État ? \\
Peut-il y avoir une vérité en politique ? \\
Peut-on admettre un droit à la révolte ? \\
Peut-on aimer les animaux ? \\
Peut-on aimer l'humanité ? \\
Peut-on apprendre à vivre ? \\
Peut-on avoir raison tout seul ? \\
Peut-on changer de culture ? \\
Peut-on changer de logique ? \\
Peut-on changer le passé ? \\
Peut-on comparer deux philosophies ? \\
Peut-on concevoir une morale sans sanction ni obligation ? \\
Peut-on concevoir une société qui n'aurait plus besoin du droit ? \\
Peut-on concevoir un État mondial ? \\
Peut-on conclure de l'être au devoir-être ? \\
Peut-on connaître autrui ? \\
Peut-on considérer l'art comme un langage ? \\
Peut-on critiquer la démocratie ? \\
Peut-on décider de croire ? \\
Peut-on définir la vérité ? \\
Peut-on définir la vie ? \\
Peut-on définir le bien ? \\
Peut-on dire ce qui n'est pas ? \\
Peut-on dire de la connaissance scientifique qu'elle procède par approximation ? \\
Peut-on dire de l'art qu'il donne un monde en partage ? \\
Peut-on dire d'une image qu'elle parle ? \\
Peut-on dire d'une théorie scientifique qu'elle n'est jamais plus vraie qu'une autre mais seulement plus commode ? \\
Peut-on dire que la science ne nous fait pas connaître les choses mais les rapports entre les choses ? \\
Peut-on dire qu'est vrai ce qui correspond aux faits ? \\
Peut-on dire qu'une théorie physique en contredit une autre ? \\
Peut-on dire toute la vérité ? \\
Peut-on disposer de son corps ? \\
Peut-on distinguer différents types de causes ? \\
Peut-on douter de sa propre existence ? \\
Peut-on éclairer la liberté ? \\
Peut-on en appeler à la conscience contre la loi ? \\
Peut-on en savoir trop ? \\
Peut-on entreprendre d'éliminer la métaphysique ? \\
Peut-on établir une hiérarchie des arts ? \\
Peut-on être amoral ? \\
Peut-on être apolitique ? \\
Peut-on être citoyen du monde ? \\
Peut-on être en conflit avec soi-même ? \\
Peut-on être heureux tout seul ? \\
Peut-on être hors de soi ? \\
Peut-on être insensible à l'art ? \\
Peut-on être plus ou moins libre ? \\
Peut-on être sans opinion ? \\
Peut-on être trop sage ? \\
Peut-on expliquer une œuvre d'art ? \\
Peut-on faire de l'art avec tout ? \\
Peut-on faire du dialogue un modèle de relation morale ? \\
Peut-on faire l'économie de la notion de forme ? \\
Peut-on faire le mal en vue du bien ? \\
Peut-on faire l'inventaire du monde ? \\
Peut-on fixer des limites à la science ? \\
Peut-on fonder les droits de l'homme ? \\
Peut-on fonder les mathématiques ? \\
Peut-on fonder une morale sur la nature ? \\
Peut-on gouverner sans lois ? \\
Peut-on hiérarchiser les œuvres d'art ? \\
Peut-on innover en politique ? \\
Peut-on jamais aimer son prochain ? \\
Peut-on juger des œuvres d'art sans recourir à l'idée de beauté ? \\
Peut-on justifier la discrimination ? \\
Peut-on justifier la guerre ? \\
Peut-on justifier la raison d'État ? \\
Peut-on justifier le mensonge ? \\
Peut-on mesurer les phénomènes sociaux ? \\
Peut-on ne pas être matérialiste ? \\
Peut-on ne pas savoir ce que l'on fait ? \\
Peut-on ne rien vouloir ? \\
Peut-on objectiver le psychisme ? \\
Peut-on opposer justice et liberté ? \\
Peut-on opposer morale et technique ? \\
Peut-on parler d'art primitif ? \\
Peut-on parler de corruption des mœurs ? \\
Peut-on parler de droits des animaux ? \\
Peut-on parler des œuvres d'art ? \\
Peut-on parler de vérités métaphysiques ? \\
Peut-on parler de vertu politique ? \\
Peut-on parler d'un droit de la guerre ? \\
Peut-on parler d'une science de l'art ? \\
Peut-on parler d'un savoir poétique ? \\
Peut-on parler d'un travail intellectuel ? \\
Peut-on penser illogiquement ? \\
Peut-on penser la douleur ? \\
Peut-on penser la mort ? \\
Peut-on penser l'extériorité ? \\
Peut-on penser l'irrationnel ? \\
Peut-on penser sans concepts ? \\
Peut-on penser sans concept ? \\
Peut-on penser sans règles ? \\
Peut-on penser un art sans œuvres ? \\
Peut-on penser une métaphysique sans Dieu ? \\
Peut-on penser une volonté diabolique ? \\
Peut-on percevoir sans s'en apercevoir ? \\
Peut-on perdre la raison ? \\
Peut-on perdre sa liberté ? \\
Peut-on perdre son identité ? \\
Peut-on préconiser, dans les sciences humaines et sociales, l'imitation des sciences de la nature ? \\
Peut-on recommencer sa vie ? \\
Peut-on réduire la pensée à une espèce de comportement ? \\
Peut-on réduire une métaphysique à une conception du monde ? \\
Peut-on refuser la loi ? \\
Peut-on régner innocemment ? \\
Peut-on représenter l'espace ? \\
Peut-on reprocher à la morale d'être abstraite ? \\
Peut-on rester insensible à la beauté ? \\
Peut-on restreindre la logique à la pensée formelle ? \\
Peut-on réunir des arts différents dans une même œuvre ? \\
Peut-on revendiquer la paix comme un droit ? \\
Peut-on rire de tout ? \\
Peut-on s'abstenir de penser politiquement ? \\
Peut-on s'accorder sur des vérités morales ? \\
Peut-on savoir ce qui est bien ? \\
Peut-on se désintéresser de la politique ? \\
Peut-on se faire une idée de tout ? \\
Peut-on séparer politique et économie ? \\
Peut-on se passer de chef ? \\
Peut-on se passer de Dieu ? \\
Peut-on se passer de l'État ? \\
Peut-on se passer de représentants ? \\
Peut-on se passer des relations ? \\
Peut-on se passer d'un maître ? \\
Peut-on se punir soi-même ? \\
Peut-on se régler sur des exemples en politique ? \\
Peut-on se retirer du monde ? \\
Peut-on souhaiter le gouvernement des meilleurs ? \\
Peut-on tout définir ? \\
Peut-on tout démontrer ? \\
Peut-on tout dire ? \\
Peut-on tout mesurer ? \\
Peut-on tout soumettre à la discussion ? \\
Peut-on transiger avec les principes ? \\
Peut-on trouver du plaisir à l'ennui ? \\
Peut-on vivre avec les autres ? \\
Peut-on vivre dans le doute ? \\
Peut-on vivre sans art ? \\
Peut-on vivre sans aucune certitude ? \\
Peut-on vivre sans ressentiment ? \\
Peut-on vouloir le mal ? \\
Philosopher, est-ce apprendre à vivre ? \\
Philosophie et mathématiques \\
Philosophie et métaphysique \\
Physique et mathématiques \\
Physique et métaphysique \\
Plaider \\
Poésie et philosophie \\
Poésie et vérité \\
Point de vue du créateur et point de vue du spectateur \\
Politique et esthétique \\
Politique et mémoire \\
Politique et parole \\
Politique et participation \\
Politique et passions \\
Politique et propagande \\
Politique et secret \\
Politique et technologie \\
Politique et territoire \\
Politique et trahison \\
Pourquoi a-t-on peur de la folie ? \\
Pourquoi avoir recours à la notion d'inconscient ? \\
Pourquoi châtier ? \\
Pourquoi chercher un sens à l'histoire ? \\
Pourquoi commémorer ? \\
Pourquoi conserver les œuvres d'art ? \\
Pourquoi croyons-nous ? \\
Pourquoi définir ? \\
Pourquoi des artifices ? \\
Pourquoi des artistes ? \\
Pourquoi des cérémonies ? \\
Pourquoi des classifications ? \\
Pourquoi des conflits ? \\
Pourquoi des élections ? \\
Pourquoi des exemples ? \\
Pourquoi des géométries ? \\
Pourquoi des historiens ? \\
Pourquoi des hypothèses ? \\
Pourquoi des institutions ? \\
Pourquoi des logiciens ? \\
Pourquoi des lois ? \\
Pourquoi des métaphores ? \\
Pourquoi des modèles ? \\
Pourquoi des musées ? \\
Pourquoi des œuvres d'art ? \\
Pourquoi des poètes ? \\
Pourquoi des rites ? \\
Pourquoi des utopies ? \\
Pourquoi Dieu se soucierait-il des affaires humaines ? \\
Pourquoi dire la vérité ? \\
Pourquoi donner ? \\
Pourquoi écrire ? \\
Pourquoi écrit-on des lois ? \\
Pourquoi est-il difficile de rectifier une erreur ? \\
Pourquoi être exigeant ? \\
Pourquoi être moral ? \\
Pourquoi exiger la cohérence \\
Pourquoi faire de la politique ? \\
Pourquoi faire de l'histoire ? \\
Pourquoi faire la guerre ? \\
Pourquoi fait-on le mal ? \\
Pourquoi faudrait-il être cohérent ? \\
Pourquoi faut-il être cohérent ? \\
Pourquoi formaliser des arguments ? \\
Pourquoi la musique intéresse-t-elle le philosophe ? \\
Pourquoi la raison recourt-elle à l'hypothèse ? \\
Pourquoi l'art intéresse-t-il les philosophes ? \\
Pourquoi le droit international est-il imparfait ? \\
Pourquoi les États se font-ils la guerre ? \\
Pourquoi les mathématiques s'appliquent-elles à la réalité ? \\
Pourquoi les œuvres d'art résistent-elles au temps ? \\
Pourquoi l'ethnologue s'intéresse-t-il à la vie urbaine ? \\
Pourquoi lire des romans ? \\
Pourquoi nous souvenons-nous ? \\
Pourquoi obéir aux lois ? \\
Pourquoi parlons-nous ? \\
Pourquoi pensons-nous ? \\
Pourquoi plusieurs sciences ? \\
Pourquoi préférer l'original à sa reproduction ? \\
Pourquoi punir ? \\
Pourquoi sauver les phénomènes ? \\
Pourquoi se mettre à la place d'autrui ? \\
Pourquoi séparer les pouvoirs ? \\
Pourquoi s'inspirer de l'art antique ? \\
Pourquoi sommes-nous déçus par les œuvres d'un faussaire ? \\
Pourquoi sommes-nous moraux ? \\
Pourquoi un droit du travail ? \\
Pourquoi une instruction publique ? \\
Pourquoi y a-t-il des conflits insolubles ? \\
Pourquoi y a-t-il quelque chose plutôt que rien ? \\
Pourquoi y a-t-il plusieurs philosophies ? \\
Pouvoir et autorité \\
Pouvoir et contre-pouvoir \\
Pouvoir et politique \\
Pouvoir et puissance \\
Pouvoir et savoir \\
Pouvoirs et libertés \\
Pouvoir temporel et pouvoir spirituel \\
Pouvons-nous devenir meilleurs ? \\
Prédicats et relations \\
Prémisses et conclusions \\
Prendre des risques \\
Prendre le pouvoir \\
Prendre les armes \\
Prendre soin \\
Prendre son temps \\
Prendre une décision \\
Prendre une décision politique \\
Présence et absence \\
Présence et représentation \\
Prévoir \\
Prévoir les comportements humains \\
Primitif ou premier ? \\
Principe et cause \\
Principe et commencement \\
Probabilité et explication scientifique \\
Production et création \\
Proposition et jugement \\
Propriétés artistiques, propriétés esthétiques \\
Prose et poésie \\
Prospérité et sécurité \\
Protester \\
Prouver \\
Prouver en métaphysique \\
Providence et destin \\
Psychologie et contrôle des comportements \\
Psychologie et métaphysique \\
Psychologie et neurosciences \\
Publier \\
Puis-je être sûr de bien agir ? \\
Puis-je être universel ? \\
Puis-je ne rien croire ? \\
Pulsion et instinct \\
Pulsions et passions \\
Qualités premières, qualités secondes \\
Quand agit-on ? \\
Quand la guerre finira-t-elle ? \\
Quand la technique devient-elle art ? \\
Quand pense-t-on ? \\
Quand suis-je en faute ? \\
Quand y a-t-il œuvre ? \\
Quand y a-t-il paysage ? \\
Quand y a-t-il peuple ? \\
Quantification et pensée scientifique \\
Quantité et qualité \\
Qu'a perdu le fou ? \\
Qu'appelle-t-on chef-d'œuvre ? \\
Qu'apprend-on dans les livres ? \\
Qu'apprenons-nous de nos affects ? \\
Qu'a-t-on le droit de pardonner ? \\
Qu'a-t-on le droit d'interpréter ? \\
Qu'avons-nous à apprendre des historiens ? \\
Que cherchons-nous dans le regard des autres ? \\
Que connaissons-nous du vivant ? \\
Que construit le politique ? \\
Que crée l'artiste ? \\
Que déduire d'une contradiction ? \\
Que désirons-nous ? \\
Que disent les tables de vérité ? \\
Que dit la loi ? \\
Que dois-je à l'État ? \\
Que doit-on aux morts ? \\
Que doit-on faire de ses rêves ? \\
Que fait la police ? \\
Que faut-il craindre ? \\
Que faut-il savoir pour gouverner ? \\
Quel est le but d'une théorie physique ? \\
Quel est le but du travail scientifique ? \\
Quel est le pouvoir de l'art ? \\
Quel est le rôle de la créativité dans les sciences ? \\
Quel est le rôle du médecin ? \\
Quel est le sujet de l'histoire ? \\
Quel est l'être de l'illusion ? \\
Quel est l'objet de la métaphysique ? \\
Quel est l'objet de la philosophie politique ? \\
Quel est l'objet de la science ? \\
Quel est l'objet de l'échange ? \\
Quel est l'objet de l'esthétique ? \\
Quel est l'objet des sciences politiques ? \\
Quelle est la matière de l'œuvre d'art ? \\
Quelle est la réalité de la matière ? \\
Quelle est la spécificité de la communauté politique ? \\
Quelle idée le fanatique se fait-il de la vérité ? \\
Quelle politique fait-on avec les sciences humaines ? \\
Quelle réalité la science décrit-elle ? \\
Quelles actions permettent d'être heureux ? \\
Quelles règles la technique dicte-t-elle à l'art ? \\
Quelles sont les caractéristiques d'une proposition morale ? \\
Quelle valeur donner à la notion de « corps social » ? \\
Quel réel pour l'art ? \\
Quel rôle attribuer à l'intuition \emph{a priori} dans une philosophie des mathématiques ? \\
Quel rôle la logique joue-t-elle en mathématiques ? \\
Quel rôle l'imagination joue-t-elle en mathématiques ? \\
Quel sens y a-t-il à se demander si les sciences humaines sont vraiment des sciences ? \\
Quels sont les moyens légitimes de la politique ? \\
Que montre l'image ? \\
Que montre un tableau ? \\
Que ne peut-on pas expliquer ? \\
Que nous apporte l'art ? \\
Que nous apprend la psychanalyse de l'homme ? \\
Que nous apprend la sociologie des sciences ? \\
Que nous apprend le plaisir ? \\
Que nous apprend le toucher ? \\
Que nous apprend l'histoire de l'art ? \\
Que nous apprend l'histoire des sciences ? \\
Que nous apprend, sur la politique, l'utopie ? \\
Que nous apprennent les algorithmes sur nos sociétés ? \\
Que nous apprennent les controverses scientifiques ? \\
Que nous apprennent les faits divers ? \\
Que nous apprennent les langues étrangères ? \\
Que nous montrent les natures mortes ? \\
Que peindre ? \\
Que peint le peintre ? \\
Que perd la pensée en perdant l'écriture ? \\
Que peut la force ? \\
Que peut la politique ? \\
Que peut l'art ? \\
Que peut le politique ? \\
Que peut-on attendre de l'État ? \\
Que peut-on attendre du droit international ? \\
Que peut-on calculer ? \\
Que peut-on comprendre qu'on ne puisse expliquer ? \\
Que peut-on démontrer ? \\
Que peut-on dire de l'être ? \\
Que peut-on partager ? \\
Que peut un corps ? \\
Que pouvons-nous aujourd'hui apprendre des sciences d'autrefois ? \\
Que sais-je de ma souffrance ? \\
Que serait le meilleur des mondes ? \\
Que serait un art total ? \\
Que serait une démocratie parfaite ? \\
Que signifie apprendre ? \\
Que sondent les sondages d'opinion ? \\
Qu'est-ce le mal radical ? \\
Qu'est-ce qu'avoir conscience de soi ? \\
Qu'est-ce qu'avoir de l'expérience ? \\
Qu'est-ce qu'avoir du goût ? \\
Qu'est-ce qu'avoir du style ? \\
Qu'est-ce que calculer ? \\
Qu'est-ce que démontrer ? \\
Qu'est-ce que déraisonner ? \\
Qu'est-ce que Dieu pour un athée ? \\
Qu'est-ce que discuter ? \\
Qu'est-ce que faire autorité ? \\
Qu'est-ce que faire preuve d'humanité ? \\
Qu'est-ce que gouverner ? \\
Qu'est-ce que guérir ? \\
Qu'est-ce que juger ? \\
Qu'est-ce que la culture générale \\
Qu'est-ce que la psychologie ? \\
Qu'est-ce que l'art contemporain ? \\
Qu'est-ce que le désordre ? \\
Qu'est-ce que le moi ? \\
Qu'est-ce que le naturalisme ? \\
Qu'est-ce que l'enfance ? \\
Qu'est-ce que l'harmonie ? \\
Qu'est-ce que lire ? \\
Qu'est-ce que méditer ? \\
Qu'est-ce qu'enquêter ? \\
Qu'est-ce que parler ? \\
Qu'est-ce que perdre son temps ? \\
Qu'est-ce que prendre conscience ? \\
Qu'est-ce que prendre le pouvoir ? \\
Qu'est-ce que raisonner ? \\
Qu'est-ce que résoudre une contradiction ? \\
Qu'est-ce que réussir sa vie ? \\
Qu'est-ce que traduire ? \\
Qu'est-ce qu'être chez soi ? \\
Qu'est-ce qu'être comportementaliste ? \\
Qu'est-ce qu'être libéral ? \\
Qu'est-ce qu'être malade ? \\
Qu'est-ce qu'être républicain ? \\
Qu'est-ce qu'être sceptique ? \\
Qu'est-ce qu'être seul ? \\
Qu'est-ce qu'être souverain ? \\
Qu'est-ce qu'être un esclave ? \\
Qu'est-ce qu'être vivant ? \\
Qu'est-ce que un individu \\
Qu'est-ce qu'habiter ? \\
Qu'est-ce qui agit ? \\
Qu'est-ce qui apparaît ? \\
Qu'est-ce qui dépend de nous ? \\
Qu'est-ce qui est beau ? \\
Qu'est ce qui est concret ? \\
Qu'est ce qui est contre nature ? \\
Qu'est-ce qui est contre nature ? \\
Qu'est-ce qui est donné ? \\
Qu'est-ce qui est impossible ? \\
Qu'est-ce qui est indiscutable ? \\
Qu'est-ce qui est invérifiable ? \\
Qu'est ce qui est irréfutable ? \\
Qu'est-ce qui est mien ? \\
Qu'est-ce qui est moderne ? \\
Qu'est-ce qui est noble ? \\
Qu'est-ce qui est politique ? \\
Qu'est-ce qui est réel ? \\
Qu'est-ce qui est respectable ? \\
Qu'est ce qui est sacré ? \\
Qu'est-ce qui est spectaculaire ? \\
Qu'est-ce qui est sublime ? \\
Qu'est-ce qui est vital pour le vivant ? \\
Qu'est ce qui existe ? \\
Qu'est-ce qui existe ? \\
Qu'est-ce qui fait la force des lois ? \\
Qu'est-ce qui fait la justice des lois ? \\
Qu'est-ce qui fait la légitimité d'une autorité politique ? \\
Qu'est-ce qui fait la valeur de l'œuvre d'art ? \\
Qu'est-ce qui fait la valeur d'une croyance ? \\
Qu'est-ce qui fait le propre d'un corps propre ? \\
Qu'est-ce qui fait l'humanité d'un corps ? \\
Qu'est-ce qui fait l'unité d'une science ? \\
Qu'est-ce qui fait l'unité d'un peuple ? \\
Qu'est-ce qui fait un peuple ? \\
Qu'est-ce qu'ignore la science ? \\
Qu'est-ce qui n'est pas politique ? \\
Qu'est-ce qu'interpréter une œuvre d'art ? \\
Qu'est-ce qu'interpréter ? \\
Qu'est-ce qui rend l'objectivité difficile dans les sciences humaines ? \\
Qu'est-ce qui rend vrai un énoncé ? \\
Qu'est-ce qu'obéir ? \\
Qu'est-ce qu'on attend ? \\
Qu'est-ce qu'un abus de pouvoir ? \\
Qu'est-ce qu'un acte moral ? \\
Qu'est-ce qu'un acte symbolique ? \\
Qu'est-ce qu'un acteur ? \\
Qu'est-ce qu'un adversaire en politique ? \\
Qu'est-ce qu'un alter ego \\
Qu'est-ce qu'un animal domestique ? \\
Qu'est-ce qu'un animal ? \\
Qu'est-ce qu'un argument ? \\
Qu'est-ce qu'un artiste ? \\
Qu'est-ce qu'un art moral ? \\
Qu'est-ce qu'un auteur ? \\
Qu'est-ce qu'un axiome ? \\
Qu'est-ce qu'un bon citoyen ? \\
Qu'est-ce qu'un bon conseil ? \\
Qu'est-ce qu'un capital culturel ? \\
Qu'est-ce qu'un cas de conscience ? \\
Qu'est-ce qu'un chef d'œuvre ? \\
Qu'est-ce qu'un chef-d'œuvre ? \\
Qu'est-ce qu'un chef ? \\
Qu'est-ce qu'un citoyen ? \\
Qu'est-ce qu'un civilisé ? \\
Qu'est-ce qu'un concept scientifique ? \\
Qu'est-ce qu'un concept ? \\
Qu'est-ce qu'un conflit politique ? \\
Qu'est-ce qu'un contenu de conscience ? \\
Qu'est-ce qu'un contrat ? \\
Qu'est-ce qu'un contre-pouvoir ? \\
Qu'est-ce qu'un corps social ? \\
Qu'est-ce qu'un coup d'État ? \\
Qu'est-ce qu'un crime contre l'humanité ? \\
Qu'est-ce qu'un crime politique ? \\
Qu'est-ce qu'un déni ? \\
Qu'est-ce qu'un dieu ? \\
Qu'est-ce qu'un Dieu ? \\
Qu'est-ce qu'un document ? \\
Qu'est-ce qu'un dogme ? \\
Qu'est-ce qu'une alternative ? \\
Qu'est-ce qu'une âme ? \\
Qu'est-ce qu'une aporie ? \\
Qu'est-ce qu'une belle démonstration ? \\
Qu'est-ce qu'une bonne loi ? \\
Qu'est-ce qu'une bonne méthode ? \\
Qu'est-ce qu'une catégorie de l'être ? \\
Qu'est-ce qu'une catégorie ? \\
Qu'est-ce qu'une cause ? \\
Qu'est-ce qu'une chose ? \\
Qu'est-ce qu'une collectivité ? \\
Qu'est-ce qu'une communauté politique ? \\
Qu'est-ce qu'une communauté ? \\
Qu'est-ce qu'une conception scientifique du monde ? \\
Qu'est-ce qu'une conduite irrationnelle ? \\
Qu'est-ce qu'une connaissance non scientifique ? \\
Qu'est-ce qu'une constitution ? \\
Qu'est-ce qu'une crise politique ? \\
Qu'est-ce qu'une crise ? \\
Qu'est-ce qu'une croyance vraie ? \\
Qu'est-ce qu'une culture ? \\
Qu'est-ce qu'une découverte scientifique ? \\
Qu'est-ce qu'une découverte ? \\
Qu'est-ce qu'une discipline savante ? \\
Qu'est-ce qu'une école philosophique ? \\
Qu'est-ce qu'une éducation scientifique ? \\
Qu'est-ce qu'une époque ? \\
Qu'est-ce qu'une expérience cruciale ? \\
Qu'est-ce qu'une expérience de pensée ? \\
Qu'est-ce qu'une exposition ? \\
Qu'est-ce qu'une famille ? \\
Qu'est-ce qu'une fonction ? \\
Qu'est-ce qu'une forme ? \\
Qu'est-ce qu'une guerre juste ? \\
Qu'est-ce qu'une hypothèse scientifique ? \\
Qu'est-ce qu'une idée esthétique ? \\
Qu'est-ce qu'une idée incertaine ? \\
Qu'est-ce qu'une idée morale ? \\
Qu'est-ce qu'une idée vraie ? \\
Qu'est-ce qu'une idée ? \\
Qu'est-ce qu'une idéologie ? \\
Qu'est-ce qu'une image ? \\
Qu'est-ce qu'une institution ? \\
Qu'est-ce qu'une langue morte ? \\
Qu'est-ce qu'un élément ? \\
Qu'est-ce qu'une limite ? \\
Qu'est-ce qu'une logique sociale ? \\
Qu'est ce qu'une loi scientifique ? \\
Qu'est-ce qu'une loi scientifique ? \\
Qu'est-ce qu'une loi ? \\
Qu'est-ce qu'une machine ? \\
Qu'est-ce qu'une marchandise ? \\
Qu'est-ce qu'une mauvaise interprétation ? \\
Qu'est-ce qu'une méditation métaphysique ? \\
Qu'est-ce qu'une méditation ? \\
Qu'est-ce qu'une mentalité collective ? \\
Qu'est-ce qu'une méthode ? \\
Qu'est-ce qu'un empire ? \\
Qu'est-ce qu'une nation ? \\
Qu'est-ce qu'un enfant ? \\
Qu'est-ce qu'une norme sociale ? \\
Qu'est-ce qu'une norme ? \\
Qu'est-ce qu'une œuvre d'art authentique ? \\
Qu'est-ce qu'une œuvre d'art ? \\
Qu'est-ce qu'une œuvre ratée ? \\
Qu'est-ce qu'une œuvre ? \\
Qu'est-ce qu'une œuvre « géniale » ? \\
Qu'est-ce qu'une période en histoire ? \\
Qu'est-ce qu‘une philosophie première ? \\
Qu'est-ce qu'une phrase ? \\
Qu'est-ce qu'une politique sociale ? \\
Qu'est-ce qu'une preuve ? \\
Qu'est-ce qu'une promesse ? \\
Qu'est-ce qu'une propriété essentielle ? \\
Qu'est-ce qu'une psychologie scientifique ? \\
Qu'est-ce qu'une question dénuée de sens ? \\
Qu'est-ce qu'une question métaphysique ? \\
Qu'est-ce qu'une réfutation ? \\
Qu'est ce qu'une religion ? \\
Qu'est-ce qu'une représentation réussie ? \\
Qu'est-ce qu'une révolution scientifique ? \\
Qu'est-ce qu'une révolution ? \\
Qu'est-ce qu'une science rigoureuse ? \\
Qu'est-ce qu'une situation tragique ? \\
Qu'est-ce qu'une société mondialisée ? \\
Qu'est-ce qu'un esprit faux ? \\
Qu'est-ce qu'une substance ? \\
Qu'est-ce qu'une tradition ? \\
Qu'est-ce qu'une tragédie historique ? \\
Qu'est-ce qu'un être cultivé ? \\
Qu'est-ce qu'une valeur ? \\
Qu'est-ce qu'un événement historique ? \\
Qu'est-ce qu'un événement ? \\
Qu'est-ce qu'une vérité scientifique ? \\
Qu'est-ce qu'une vie réussie ? \\
Qu'est-ce qu'une ville ? \\
Qu'est-ce qu'une violence symbolique ? \\
Qu'est-ce qu'une vision du monde ? \\
Qu'est-ce qu'une vision scientifique du monde ? \\
Qu'est-ce qu'une volonté libre ? \\
Qu'est-ce qu'un exemple ? \\
Qu'est-ce qu'une « performance » ? \\
Qu'est-ce qu'un fait de société ? \\
Qu'est-ce qu'un fait historique ? \\
Qu'est-ce qu'un fait moral ? \\
Qu'est ce qu'un fait scientifique ? \\
Qu'est-ce qu'un fait social ? \\
Qu'est-ce qu'un faux problème ? \\
Qu'est-ce qu'un faux sentiment ? \\
Qu'est-ce qu'un film ? \\
Qu'est-ce qu'un geste artistique ? \\
Qu'est-ce qu'un gouvernement ? \\
Qu'est-ce qu'un grand philosophe ? \\
Qu'est-ce qu'un héros ? \\
Qu'est-ce qu'un homme bon ? \\
Qu'est-ce qu'un homme juste ? \\
Qu'est-ce qu'un homme seul ? \\
Qu'est-ce qu'un idéal moral ? \\
Qu'est-ce qu'un individu ? \\
Qu'est-ce qu'un jeu ? \\
Qu'est-ce qu'un laboratoire ? \\
Qu'est-ce qu'un législateur ? \\
Qu'est-ce qu'un lieu commun ? \\
Qu'est-ce qu'un livre ? \\
Qu'est-ce qu'un maître ? \\
Qu'est-ce qu'un marginal ? \\
Qu'est-ce qu'un mécanisme social ? \\
Qu'est-ce qu'un métaphysicien ? \\
Qu'est-ce qu'un modèle ? \\
Qu'est-ce qu'un moderne ? \\
Qu'est-ce qu'un monde ? \\
Qu'est-ce qu'un monstre ? \\
Qu'est-ce qu'un monument ? \\
Qu'est-ce qu'un mouvement politique \\
Qu'est-ce qu'un mythe ? \\
Qu'est-ce qu'un nombre ? \\
Qu'est-ce qu'un nom propre ? \\
Qu'est-ce qu'un objet d'art ? \\
Qu'est-ce qu'un objet esthétique ? \\
Qu'est-ce qu'un objet métaphysique ? \\
Qu'est-ce qu'un organisme ? \\
Qu'est-ce qu'un original ? \\
Qu'est-ce qu'un outil ? \\
Qu'est ce qu'un paradoxe ? \\
Qu'est-ce qu'un paradoxe ? \\
Qu'est-ce qu'un patrimoine ? \\
Qu'est-ce qu'un pédant ? \\
Qu'est-ce qu'un peuple \\
Qu'est-ce qu'un peuple libre ? \\
Qu'est-ce qu'un peuple ? \\
Qu'est-ce qu'un phénomène ? \\
Qu'est-ce qu'un plaisir pur ? \\
Qu'est-ce qu'un point de vue ? \\
Qu'est-ce qu'un primitif ? \\
Qu'est-ce qu'un prince juste ? \\
Qu'est-ce qu'un principe ? \\
Qu'est-ce qu'un problème éthique ? \\
Qu'est-ce qu'un problème métaphysique ? \\
Qu'est-ce qu'un problème philosophique ? \\
Qu'est-ce qu'un problème politique ? \\
Qu'est-ce qu'un problème scientifique ? \\
Qu'est-ce qu'un problème ? \\
Qu'est-ce qu'un produit culturel ? \\
Qu'est-ce qu'un programme politique ? \\
Qu'est-ce qu'un rapport de force ? \\
Qu'est-ce qu'un sage ? \\
Qu'est-ce qu'un sentiment moral ? \\
Qu'est-ce qu'un signe ? \\
Qu'est-ce qu'un sophisme ? \\
Qu'est-ce qu'un sophiste ? \\
Qu'est-ce qu'un souvenir ? \\
Qu'est-ce qu'un spécialiste ? \\
Qu'est-ce qu'un spectateur ? \\
Qu'est-ce qu'un style ? \\
Qu'est-ce qu'un symptôme ? \\
Qu'est-ce qu'un système philosophique ? \\
Qu'est-ce qu'un système ? \\
Qu'est-ce qu'un tableau \\
Qu'est-ce qu'un témoin ? \\
Qu'est-ce qu'un tout ? \\
Qu'est-ce qu'un trouble social ? \\
Qu'est-ce qu'un « champ artistique » ? \\
Question et problème \\
Qu'est qu'un régime politique ? \\
Que suppose le mouvement ? \\
Que valent les excuses ? \\
Que valent les idées générales ? \\
Que vaut en morale la justification par l'utilité ? \\
Que vaut la distinction entre nature et culture ? \\
Que vaut l'excuse : « C'est plus fort que moi » ? \\
Que vaut l'incertain ? \\
Que vaut une preuve contre un préjugé ? \\
Que veut dire introduire à la métaphysique ? \\
Que veut dire l'expression « aller au fond des choses » ? \\
Que voit-on dans une image ? \\
Que voit-on dans un miroir ? \\
Qu'exprime une œuvre d'art ? \\
Qui agit ? \\
Qui a le droit de juger ? \\
Qui a une histoire ? \\
Qui a une parole politique ? \\
Qui connaît le mieux mon corps ? \\
Qui doit faire les lois ? \\
Qui est le maître ? \\
Qui est métaphysicien ? \\
Qui est mon prochain ? \\
Qui est souverain ? \\
Qui fait la loi ? \\
Qui gouverne ? \\
Qui mérite d'être aimé ? \\
Qui parle ? \\
Qui pense ? \\
Qui peut parler ? \\
Qu'y a-t-il à comprendre dans une œuvre d'art ? \\
Qu'y a-t-il à comprendre en histoire ? \\
Qu'y a-t-il à l'origine de toutes choses ? \\
Qu'y a-t-il au fondement de l'objectivité ? \\
Raconter son histoire \\
Raison et politique \\
Raison et révélation \\
Raisonner et calculer \\
Rapports de force, rapport de pouvoir \\
Rassembler les hommes, est-ce les unir ? \\
Réalisme et idéalisme \\
Réalité et idéal \\
Rebuts et objets quelconques : une matière pour l'art ? \\
Recevoir \\
Récit et mémoire \\
Reconnaissons-nous le bien comme nous reconnaissons le vrai ? \\
Réforme et révolution \\
Réfutation et confirmation \\
Réfuter \\
Regarder \\
Regarder un tableau \\
Règle et commandement \\
Religion naturelle et religion révélée \\
Rendre la justice \\
Rendre raison \\
Rendre visible l'invisible \\
Renoncer au passé \\
Rentrer en soi-même \\
Répondre \\
Représentation et illusion \\
Reproduire, copier, imiter \\
Réprouver \\
République et démocratie \\
Résistance et soumission \\
Résister \\
Résister à l'oppression \\
Résister peut-il être un droit ? \\
Rêver \\
Revient-il à l'État d'assurer votre bonheur ? \\
Rien de nouveau sous le soleil \\
Rire \\
Rire et pleurer \\
Rites et cérémonies \\
Rythmes sociaux, rythmes naturels \\
Sait-on toujours ce que l'on fait ? \\
Sait-on toujours ce qu'on veut ? \\
S'aliéner \\
Sans foi ni loi \\
S'approprier une œuvre d'art \\
Sauver les apparences \\
Sauver les phénomènes \\
Savoir ce qu'on dit \\
Savoir de quoi on parle \\
Savoir, est-ce pouvoir ? \\
Savoir et liberté \\
Savoir et objectivité dans les sciences \\
Savoir et pouvoir \\
Savoir et rectification \\
Savoir être heureux \\
Savoir et vérifier \\
Savoir pour prévoir \\
Savoir, pouvoir \\
Savoir s'arrêter \\
Savoir vivre \\
Savons-nous ce que nous disons ? \\
Science et complexité \\
Science et démocratie \\
Science et histoire \\
Science et idéologie \\
Science et imagination \\
Science et magie \\
Science et opinion \\
Science et persuasion \\
Science et philosophie \\
Science et réalité \\
Science et religion \\
Science et sagesse \\
Science et société \\
Science et technique \\
Science et technologie \\
Science pure et science appliquée \\
Sciences de la nature et sciences de l'esprit \\
Sciences de la nature et sciences humaines \\
Sciences et philosophie \\
Sciences humaines et déterminisme \\
Sciences humaines et herméneutique \\
Sciences humaines et idéologie \\
Sciences humaines et liberté sont-elles compatibles ? \\
Sciences humaines et littérature \\
Sciences humaines et naturalisme \\
Sciences humaines et nature humaine \\
Sciences humaines et objectivité \\
Sciences humaines et philosophie \\
Sciences humaines, sciences de l'homme \\
Sciences sociales et humanités \\
Se conserver \\
Se cultiver \\
Sécurité et liberté \\
Se défendre \\
Se détacher des sens \\
Se faire justice \\
Se mentir à soi-même \\
Se mettre à la place d'autrui \\
S'ennuyer \\
Sens et fait \\
Sens et limites de la notion de capital culturel \\
Sens et sensibilité \\
Sens et sensible \\
Sens et structure \\
Sensible et intelligible \\
Sentir \\
Se parler et s'entendre \\
Se passer de philosophie \\
Se prendre au sérieux \\
Se retirer dans la pensée ? \\
Se retirer du monde \\
Servir \\
Servir l'État \\
Se taire \\
Seul le présent existe-t-il ? \\
Se voiler la face \\
Sexe et genre \\
S'exercer \\
S'exprimer \\
Sexualité et nature \\
Si Dieu n'existe pas, tout est-il permis ? \\
Signes, traces et indices \\
Signification et expression \\
Signification et vérité \\
Si l'esprit n'est pas une table rase, qu'est-il ? \\
S'indigner \\
S'indigner, est-ce un devoir ? \\
S'intéresser à l'art \\
Si tu veux, tu peux \\
Société et organisme \\
Sommes-nous capables d'agir de manière désintéressée ? \\
Sommes-nous des êtres métaphysiques ? \\
Sommes-nous libres de nos croyances ? \\
Sommes-nous libres de nos pensées ? \\
Sommes-nous libres de nos préférences morales ? \\
Sommes-nous responsables de ce que nous sommes ? \\
Sommes-nous toujours dépendants d'autrui ? \\
Sommes-nous tous contemporains ? \\
Sophismes et paradoxes \\
S'orienter \\
Sortir de soi \\
Soutenir une thèse \\
Structure et événement \\
Subir \\
Substance et sujet \\
Suffit-il d'être juste ? \\
Suffit-il de vouloir pour bien faire ? \\
Suffit-il pour être juste d'obéir aux lois et aux coutumes de son pays ? \\
Suis-je aussi ce que j'aurais pu être ? \\
Suis-je ma mémoire ? \\
Suis-je mon corps ? \\
Sujet et citoyen \\
Sujet et prédicat \\
Sur quoi fonder la propriété ? \\
Sur quoi fonder l'autorité ? \\
Sur quoi reposent nos certitudes ? \\
Sur quoi se fonde la connaissance scientifique ? \\
Surveillance et discipline \\
Surveiller son comportement \\
Survivre \\
Suspendre son assentiment \\
Suspendre son jugement \\
Syllogisme et démonstration \\
Tantôt je pense, tantôt je suis \\
Tautologie et contradiction \\
Technique et pratiques scientifiques \\
Témoigner \\
Temps et éternité \\
Temps et réalité \\
Tenir parole \\
Thème et variations \\
Théorie et modélisation \\
Tous les droits sont-ils formels ? \\
Tous les hommes désirent-ils connaître ? \\
Tous les hommes désirent-ils être heureux ? \\
Tout art est-il poésie ? \\
Tout a-t-il une raison d'être ? \\
Tout a-t-il un sens ? \\
Tout définir, tout démontrer \\
Tout devoir est-il l'envers d'un droit ? \\
Toute action politique est-elle collective ? \\
Toute chose a-t-elle une essence ? \\
Toute communauté est-elle politique ? \\
Toute connaissance autre que scientifique doit-elle être considérée comme une illusion ? \\
Toute connaissance est-elle historique ? \\
Toute expression est-elle métaphorique ? \\
Toute hiérarchie est-elle inégalitaire ? \\
Toute métaphysique implique-t-elle une transcendance ? \\
Tout énoncé est-il nécessairement vrai ou faux ? \\
Toute origine est-elle mythique ? \\
Toute peur est-elle irrationnelle ? \\
Toute philosophie est-elle systématique ? \\
Toute philosophie implique-t-elle une politique ? \\
Toutes les choses sont-elles singulières ? \\
Toutes les vérités  scientifiques sont-elles révisables ? \\
Tout est corps \\
Tout est-il connaissable ? \\
Tout est-il mesurable ? \\
Tout est-il nécessaire ? \\
Tout est-il politique ? \\
Tout est-il relatif ? \\
Tout est permis \\
Tout être est-il dans l'espace ? \\
Toute violence est-elle contre nature ? \\
Tout ou rien \\
Tout peut-il être objet de jugement esthétique ? \\
Tout peut-il n'être qu'apparence ? \\
Tout pouvoir est-il oppresseur ? \\
Tout pouvoir est-il politique ? \\
Tout pouvoir n'est-il pas abusif ? \\
Tout savoir \\
Tout savoir est-il fondé sur un savoir premier ? \\
Tout savoir est-il transmissible ? \\
Tradition et innovation \\
Tradition et raison \\
Traduire \\
Trahir \\
Traiter autrui comme une chose \\
Transcendance et altérité \\
Travail et subjectivité \\
Travail manuel, travail intellectuel \\
Tricher \\
Tuer le temps \\
Un acte désintéressé est-il possible ? \\
Un art sans sublimation est-il possible ? \\
Un bien peut-il sortir d'un mal ? \\
Un Dieu unique ? \\
Une action vertueuse se reconnaît-elle à sa difficulté ? \\
Une cause peut-elle être libre ? \\
Une culture de masse est-elle une culture ? \\
Une décision politique peut-elle être juste ? \\
Une éducation esthétique est-elle possible ? \\
Une éthique sceptique est-elle possible ? \\
Une explication peut-elle être réductrice ? \\
Une fiction peut-elle être vraie ? \\
Une guerre peut-elle être juste ? \\
Une ligne de conduite peut-elle tenir lieu de morale ? \\
Une logique non-formelle est-elle possible ? \\
Une loi n'est-elle qu'une règle ? \\
Une machine peut-elle penser ? \\
Une machine pourrait-elle penser ? \\
Une métaphysique athée est-elle possible ? \\
Une métaphysique peut-elle être sceptique ? \\
Une morale du plaisir est-elle concevable ? \\
Une morale peut-elle être dépassée ? \\
Une morale peut-elle être provisoire ? \\
Une morale peut-elle prétendre à l'universalité ? \\
Une morale sans Dieu \\
Une morale sans obligation est-elle possible ? \\
Une œuvre d'art doit-elle avoir un sens ? \\
Une œuvre d'art est-elle une marchandise ? \\
Une œuvre d'art peut-elle être laide ? \\
Une œuvre d'art s'explique-t-elle à partir de ses influences ? \\
Une œuvre est-elle toujours de son temps ? \\
Une perception peut-elle être illusoire ? \\
Une philosophie de l'amour est-elle possible ? \\
Une politique peut-elle se réclamer de la vie ? \\
Une religion civile est-elle possible ? \\
Une religion peut-elle être rationnelle ? \\
Une science de la culture est-elle possible ? \\
Une science de la morale est-elle possible ? \\
Une science des symboles est-elle possible ? \\
Une société juste est-elle une société sans conflits ? \\
Une société sans conflit est-elle possible ? \\
Une société sans État est-elle une société sans politique ? \\
Un État peut-il être trop étendu ? \\
Une théorie scientifique peut-elle devenir fausse ? \\
Une théorie scientifique peut-elle être ramenée à des propositions empiriques élémentaires ? \\
Une volonté peut-elle être générale ? \\
Un homme n'est-il que la somme de ses actes ? \\
Universalité et nécessité dans les sciences \\
Univocité et équivocité \\
Un jugement de goût est-il culturel ? \\
Un moment d'éternité \\
Un monde sans beauté \\
Un monde sans nature est-il pensable ? \\
Un pouvoir a-t-il besoin d'être légitime ? \\
Un problème scientifique peut-il être insoluble ? \\
Un sceptique peut-il être logicien ? \\
Un tableau peut-il être une dénonciation ? \\
Un vice, est-ce un manque ? \\
Utopie et tradition \\
Vanité des vanités \\
Vérité et fiction \\
Vérité et histoire \\
Vérité et sensibilité \\
Vérité et signification \\
Vérités de fait et vérités de raison \\
Vérités mathématiques, vérités philosophiques \\
Vertu et habitude \\
Vices privés, vertus publiques \\
Vie et existence \\
Vie et volonté \\
Vieillir \\
Violence et discours \\
Violence et politique \\
Vitalisme et mécanique \\
Vit-on au présent ? \\
Vivons-nous tous dans le même monde ? \\
Vivre au présent \\
Vivre comme si nous ne devions pas mourir \\
Vivre sans morale \\
Vivre sa vie \\
Vivre sous la conduite de la raison \\
Vivre vertueusement \\
Voir \\
Voir et entendre \\
Voir et toucher \\
Voir le meilleur et faire le pire \\
Vouloir ce que l'on peut \\
Vouloir le bien \\
Vouloir l'égalité \\
Voyager \\
Vulgariser la science ? \\
Y a-t-il continuité entre l'expérience et la science ? \\
Y a-t-il continuité ou discontinuité entre la pensée mythique et la science ? \\
Y a-t-il de fausses religions ? \\
Y a-t-il de l'inconcevable ? \\
Y a-t-il de l'indémontrable ? \\
Y a-t-il de l'irréparable ? \\
Y a-t-il des actes moralement indifférents ? \\
Y a-t-il des actions désintéressées ? \\
Y a-t-il des arts mineurs ? \\
Y a-t-il des canons de la beauté ? \\
Y a-t-il des certitudes historiques ? \\
Y a-t-il des choses qui échappent au droit ? \\
Y a-t-il des compétences politiques ? \\
Y a-t-il des critères du beau ? \\
Y a-t-il des croyances démocratiques ? \\
Y a-t-il des degrés dans la certitude ? \\
Y a-t-il des degrés de réalité ? \\
Y a-t-il des démonstrations en philosophie ? \\
Y a-t-il des devoirs envers soi-même ? \\
Y a-t-il des erreurs en politique ? \\
Y a-t-il des êtres mathématiques ? \\
Y a-t-il des expériences cruciales ? \\
Y a-t-il des faits moraux ? \\
Y a-t-il des faits sans essence ? \\
Y a-t-il des fins de la nature ? \\
Y a-t-il des fins dernières ? \\
Y a-t-il des fondements naturels à l'ordre social ? \\
Y a-t-il des guerres justes ? \\
Y a-t-il des héritages philosophiques ? \\
Y a-t-il des leçons de l'histoire ? \\
Y a-t-il des limites à l'exprimable ? \\
Y a-t-il des limites au droit ? \\
Y a-t-il des limites proprement morales à la discussion ? \\
Y a-t-il des lois du hasard ? \\
Y a-t-il des lois en histoire ? \\
Y a-t-il des lois injustes ? \\
Y a-t-il des lois morales ? \\
Y a-t-il des lois non écrites ? \\
Y a-t-il des mentalités collectives ? \\
Y a-t-il des normes naturelles ? \\
Y a-t-il des passions collectives ? \\
Y a-t-il des pathologies sociales ? \\
Y a-t-il des pensées folles ? \\
Y a-t-il des pensées inconscientes ? \\
Y a-t-il des preuves d'amour ? \\
Y a-t-il des propriétés singulières ? \\
Y a-t-il des révolutions en art ? \\
Y a-t-il des révolutions scientifiques ? \\
Y a-t-il des secrets de la nature ? \\
Y a-t-il des sociétés sans État ? \\
Y a-t-il des sociétés sans histoire ? \\
Y a-t-il des substances incorporelles ? \\
Y a-t-il des vertus mineures ? \\
Y a-t-il des violences justifiées ? \\
Y a-t-il des violences légitimes ? \\
Y a-t-il différentes manières de connaître ? \\
Y a-t-il du sacré dans la nature ? \\
Y a-t-il du synthétique \emph{a priori} ? \\
Y a-t-il encore des mythologies ? \\
Y a-t-il encore une sphère privée ? \\
Y a-t-il place pour l'idée de vérité en morale ? \\
Y a-t-il plusieurs manières de définir ? \\
Y a-t-il plusieurs nécessités ? \\
Y a-t-il un art de gouverner ? \\
Y a-t-il un art d'inventer ? \\
Y a-t-il un autre monde ? \\
Y a-t-il un beau idéal ? \\
Y a-t-il un besoin métaphysique ? \\
Y a-t-il un bien commun ? \\
Y a-t-il un bien plus précieux que la paix ? \\
Y a-t-il un canon de la beauté ? \\
Y a-t-il un critère du vrai ? \\
Y a-t-il un devoir d'être heureux ? \\
Y a-t-il un droit de mourir ? \\
Y a-t-il un droit de résistance ? \\
Y a-t-il un droit international ? \\
Y a-t-il une argumentation métaphysique ? \\
Y a-t-il une beauté morale ? \\
Y a-t-il une beauté naturelle ? \\
Y a-t-il une causalité historique ? \\
Y a-t-il une compétence en politique ? \\
Y a-t-il une connaissance du singulier ? \\
Y a-t-il une connaissance métaphysique ? \\
Y a-t-il une connaissance sensible ? \\
Y a-t-il une correspondance des arts ? \\
Y a-t-il une éthique de l'authenticité ? \\
Y a-t-il une expérience de la liberté ? \\
Y a-t-il une expérience de l'éternité ? \\
Y a-t-il une expérience du néant ? \\
Y a-t-il une fin dernière ? \\
Y a-t-il une forme morale de fanatisme ? \\
Y a-t-il une hiérarchie des êtres ? \\
Y a-t-il une hiérarchie des sciences ? \\
Y a-t-il une histoire de la vérité ? \\
Y a-t-il une intentionnalité collective ? \\
Y a-t-il une justice sans morale ? \\
Y a-t-il une logique de la découverte scientifique ? \\
Y a-t-il une logique de la découverte ? \\
Y a-t-il une métaphysique de l'amour ? \\
Y a-t-il une opinion publique mondiale ? \\
Y a-t-il une ou plusieurs philosophies ? \\
Y a-t-il une philosophie de la philosophie ? \\
Y a-t-il une philosophie première ? \\
Y a-t-il une place pour la morale dans l'économie ? \\
Y a-t-il une science de la vie mentale ? \\
Y a-t-il une science de l'être ? \\
Y a-t-il une science de l'individuel ? \\
Y a-t-il une science des principes ? \\
Y a-t-il une science du qualitatif ? \\
Y a-t-il une sensibilité esthétique ? \\
Y a-t-il une spécificité de la délibération politique ? \\
Y a-t-il une spécificité des sciences humaines ? \\
Y a-t-il une unité de la science ? \\
Y a-t-il une unité en psychologie ? \\
Y a-t-il une universalité des mathématiques ? \\
Y a-t-il une vérité des symboles ? \\
Y a-t-il une vérité du sentiment ? \\
Y a-t-il une vérité philosophique ? \\
Y a-t-il une vie de l'esprit ? \\
Y a-t-il un inconscient collectif ? \\
Y a-t-il un langage commun ? \\
Y a-t-il un langage du corps ? \\
Y a-t-il un mal absolu ? \\
Y a-t-il un monde extérieur ? \\
Y a-t-il un ordre des choses ? \\
Y a-t-il un principe du mal ? \\
Y a-t-il un progrès en art ? \\
Y a-t-il un progrès moral ? \\
Y a-t-il un savoir du contingent ? \\
Y a-t-il un savoir du corps ? \\
Y a-t-il un savoir du politique ? \\
Y a-t-il un savoir immédiat ? \\
Y a-t-il un temps des choses ? \\
Y a-t-il un usage moral des passions ? \\
« Aime, et fais ce que tu veux » \\
« Aimez vos ennemis » \\
« Après moi, le déluge » \\
« À l'impossible, nul n'est tenu » \\
« Bienheureuse faute » \\
« C'est humain » \\
« C'est la vie » \\
« Comment peut-on être persan ? » \\
« De la musique avant toute chose » \\
« Deviens qui tu es » \\
« Dieu est mort » \\
« Être contre » \\
« Expliquer les faits sociaux par des faits sociaux » \\
« Il faudrait rester des années entières pour contempler une telle œuvre » \\
« Je n'ai pas voulu cela » \\
« Je ne voulais pas cela » : en quoi les sciences humaines permettent-elles de comprendre cette excuse ? \\
« La logique » ou bien « les logiques » ? \\
« La vie des formes » \\
« La vie est un songe » \\
« L'enfer est pavé de bonnes intentions » \\
« L'histoire jugera » \\
« L'homme est la mesure de toute chose » \\
« Malheur aux vaincus » \\
« Ne fais pas à autrui ce que tu ne voudrais pas qu'on te fasse » \\
« Œil pour œil, dent pour dent » \\
« Petites causes, grands effets » \\
« Prendre ses désirs pour des réalités » \\
« Que nul n'entre ici s'il n'est géomètre » \\
« Rien n'est sans raison » \\
« Rien n'est simple » \\
« Sans titre » \\
« Sauver les phénomènes » \\
« Toute peine mérite salaire » \\
« Trop beau pour être vrai » \\
« Tu ne tueras point » \\


\subsection{Agrégation interne}
\label{sec-2-3}

\noindent
À chacun sa morale \\
Action et production \\
Agir justement fait-il de moi un homme juste ? \\
Agir sans raison \\
Amitié et société \\
Analyse et synthèse \\
Appartenons-nous à une culture ? \\
À quoi tient la force des religions ? \\
A-t-on le droit de se révolter ? \\
Autrui, est-ce n'importe quel autre ? \\
Avoir bonne conscience \\
Avoir des principes \\
Avoir des valeurs \\
Avoir le sens du devoir \\
Avoir le temps \\
Avons-nous des devoirs envers les animaux ? \\
Avons-nous des devoirs envers les générations futures ? \\
Avons-nous le devoir d'être heureux ? \\
Avons-nous une intuition du temps ? \\
À qui dois-je la vérité ? \\
À qui profite le travail ? \\
À quoi bon avoir mauvaise conscience ? \\
À quoi bon critiquer les autres ? \\
À quoi l'art nous rend-il sensibles ? \\
À quoi reconnaît-on la rationalité ? \\
À quoi reconnaît-on une œuvre d'art ? \\
À quoi sert la technique ? \\
À t-on le droit de faire tout ce qui est permis par la loi ? \\
Bien commun et bien public \\
Bonheur et autarcie \\
Catégories de pensée, catégories de langue \\
Cause et loi \\
Ce que la technique rend possible, peut-on jamais en empêcher la réalisation ? \\
Ce qui est démontré est-il nécessairement vrai ? \\
Ce qui n'a pas de prix \\
Ce qui n'est pas démontré peut-il être vrai ? \\
Ce qui n'est pas matériel peut-il être réel ? \\
Changer \\
Changer le monde \\
Chercher ses mots \\
Comment expliquer les phénomènes mentaux ? \\
Comment percevons-nous l'espace ? \\
Comment reconnaît-on un vivant ? \\
Comprendre autrui \\
Comprendre, est-ce interpréter ? \\
Comprendre l'inconscient \\
Concept et image \\
Conduire sa vie \\
Conflit et démocratie \\
Connaissons-nous mieux le présent que le passé ? \\
Connaître, est-ce connaître par les causes ? \\
Conscience et attention \\
Conscience et mémoire \\
Contempler \\
Croire, est-ce obéir ? \\
Croire pour savoir \\
Croyance et certitude \\
Croyance et vérité \\
Culture et civilisation \\
Démocratie et religion \\
Démontrer, argumenter, expérimenter \\
Démontrer est-il le privilège du mathématicien ? \\
Démontrer et argumenter \\
Dénaturer \\
De quelle liberté témoigne l'œuvre d'art ? \\
De quelle réalité nos perceptions témoignent-elles ? \\
De quoi peut-il y avoir science ? \\
Déraisonner \\
Devant qui sommes-nous responsables ? \\
Devoir et conformisme \\
Dieu tout-puissant \\
Doit-on corriger les inégalités sociales ? \\
Doit-on croire en l'humanité ? \\
Doit-on distinguer devoir moral et obligation sociale ? \\
Doit-on justifier les inégalités ? \\
Doit-on se justifier d'exister ? \\
Donner à chacun son dû \\
Donner, à quoi bon ? \\
Droit et démocratie \\
Droit et devoir sont-ils liés ? \\
Droit et morale \\
Droit et protection \\
Échange et partage \\
Échange et valeur \\
Échanger, est-ce risquer ? \\
En histoire, tout est-il affaire d'interprétation ? \\
En morale, est-ce seulement l'intention qui compte ? \\
En politique, faut-il refuser l'utopie ? \\
En politique, nécessité fait loi \\
En quel sens peut-on parler d'expérience possible ? \\
Entendement et raison \\
Entendre raison \\
Esprit et intériorité \\
Essence et existence \\
Est-ce la certitude qui fait la science ? \\
Est-ce la démonstration qui fait la science ? \\
Est-ce le corps qui perçoit ? \\
Est-ce l'utilité qui définit un objet technique ? \\
Esthétique et éthique \\
Est-il légitime d'affirmer que seul le présent existe ? \\
Établir la vérité, est-ce nécessairement démontrer ? \\
Être aliéné \\
Être au monde \\
Être citoyen \\
Être conscient de soi, est-ce être maître de soi ? \\
Être dans le temps \\
Être dans son droit \\
Être déterminé \\
Être heureux \\
Être là \\
Être libre, est-ce n'obéir qu'à soi-même ? \\
Être libre, même dans les fers \\
Être malade \\
Être matérialiste \\
Être quelqu'un \\
Être soi-même \\
Expérience et habitude \\
Expérience et interprétation \\
Expérience et phénomène \\
Expérimentation et vérification \\
Expérimenter \\
Expliquer, est-ce interpréter ? \\
Faire confiance \\
Faire des choix \\
Faire justice \\
Faire son devoir \\
Faudrait-il ne rien oublier ? \\
Faut-il accorder l'esprit aux bêtes ? \\
Faut-il apprendre à voir ? \\
Faut-il défendre l'ordre à tout prix ? \\
Faut-il dire tout haut ce que les autres pensent tout bas ? \\
Faut-il être fidèle à soi-même ? \\
Faut-il interpréter la loi ? \\
Faut-il opposer la théorie et la pratique ? \\
Faut-il que le réel ait un sens ? \\
Faut-il que les meilleurs gouvernent ? \\
Faut-il respecter la nature ? \\
Faut-il sauver les apparences ? \\
Faut-il tout démontrer ? \\
Faut-il tout interpréter ? \\
Faut-il vivre comme si nous étions immortels ? \\
Foi et bonne foi \\
Forcer à être libre \\
Former les esprits \\
Histoire et écriture \\
Histoire et morale \\
Inconscient et identité \\
Inconscient et inconscience \\
Interprétation et création \\
Interpréter, est-ce renoncer à prouver ? \\
Interpréter, est-ce savoir ? \\
Interpréter ou expliquer \\
Invention et création \\
Juger et connaître \\
Justice et égalité \\
Justice et force \\
Justice et vengeance \\
Justifier le mensonge \\
La barbarie \\
La barbarie de la technique \\
La bienveillance \\
L'absence de générosité \\
L'abstraction \\
La causalité historique \\
La cause première \\
L'accomplissement de soi \\
La certitude \\
La chair \\
La chute \\
La civilité \\
La cohérence \\
La colère \\
La comparaison \\
La concorde \\
La connaissance de la vie se confond-elle avec celle du vivant ? \\
La connaissance de l'histoire est-elle utile à l'action ? \\
La connaissance scientifique est-elle désintéressée ? \\
La conscience entrave-t-elle l'action ? \\
La conscience est-elle ce qui fait le sujet ? \\
La conscience est-elle ou n'est-elle pas ? \\
La conscience peut-elle être collective ? \\
La contingence \\
La contradiction \\
La contrainte des lois est-elle une violence ? \\
La contrainte supprime-t-elle la responsabilité ? \\
La conversation \\
La création \\
La critique du pouvoir peut-elle conduire à la désobéissance ? \\
La croyance est-elle signe de faiblesse ? \\
La croyance est-elle une opinion comme les autres ? \\
La croyance est-elle une opinion ? \\
La croyance peut-elle être rationnelle ? \\
L'action du temps \\
L'action politique \\
L'actualité \\
La culture est-elle une seconde nature ? \\
La culture et les cultures \\
La culture libère-t-elle des préjugés ? \\
La culture nous rend-elle meilleurs ? \\
La culture peut-elle être instituée ? \\
La culture peut-elle être objet de science ? \\
La curiosité \\
La définition \\
La démocratie peut-elle se passer de représentation ? \\
La démonstration obéit-elle à des lois ? \\
La déraison \\
La discursivité \\
La distinction sociale \\
La division du travail \\
La domination \\
La durée \\
La faiblesse d'esprit \\
La fidélité \\
La figure de l'ennemi \\
La fin des désirs \\
La fin des guerres \\
La fin du travail \\
La finitude \\
La folie des grandeurs \\
La fonction du philosophe est-elle de diriger l'État ? \\
La force de la vérité \\
La force de l'expérience \\
La force de l'habitude \\
La force de l'inconscient \\
La formation de l'esprit \\
La fraternité \\
La fraternité peut-elle se passer d'un fondement religieux ? \\
La générosité \\
La grammaire contraint-elle la pensée ? \\
La grandeur d'une culture \\
La guerre civile \\
La haine de la raison \\
La haine des machines \\
La honte \\
La joie \\
La joie de vivre \\
La jouissance \\
La juste mesure \\
La justice consiste-t-elle dans l'application de la loi ? \\
La justice peut-elle se passer de la force ? \\
La justice sociale \\
La laideur \\
La langue de la raison \\
La lettre et l'esprit \\
La liberté artistique \\
La liberté civile \\
La liberté des autres \\
La liberté d'expression \\
La liberté du savant \\
La liberté est-elle un fait ? \\
La liberté implique-t-elle l'indifférence ? \\
La liberté individuelle \\
La liberté peut-elle faire peur ? \\
La liberté peut-elle se constater ? \\
La liberté peut-elle se prouver ? \\
La liberté peut-elle se refuser ? \\
La liberté se prouve-t-elle ? \\
La liberté se réduit-elle au libre-arbitre ? \\
La logique est-elle la norme du vrai ? \\
La logique est-elle l'art de penser ? \\
La loi du désir \\
La loi éduque-t-elle ? \\
L'altruisme \\
La main \\
La maladie \\
La matière de la pensée \\
La matière de l'œuvre \\
La matière, est-ce le mal ? \\
La matière est-elle amorphe ? \\
La matière est-elle une vue de l'esprit ? \\
La matière et la forme \\
La matière n'est-elle qu'un obstacle ? \\
La matière pense-t-elle ? \\
La matière peut-elle être objet de connaissance ? \\
La matière vivante \\
La mauvaise foi \\
La méconnaissance de soi \\
La méfiance \\
La mémoire et l'histoire \\
La mesure du temps \\
La métaphore \\
L'amitié \\
L'amitié peut-elle obliger ? \\
La morale a-t-elle besoin d'un fondement ? \\
La morale est-elle affaire de sentiments ? \\
La mort de Dieu \\
L'amour de soi \\
L'amour est-il désir ? \\
L'amour est-il une vertu ? \\
L'amour maternel \\
L'anarchie \\
La nature est-elle écrite en langage mathématique ? \\
La nature est-elle sans histoire ? \\
La négligence \\
La neutralité \\
L'angoisse \\
L'animal a-t-il des droits ? \\
L'anormal \\
La nostalgie \\
La paix \\
La paix de la conscience \\
La paix sociale est-elle la finalité de la politique ? \\
La panne et la maladie \\
La passion n'est-elle que souffrance ? \\
L'apathie \\
La pauvreté \\
La perception est-elle l'interprétation du réel ? \\
La perception peut-elle être désintéressée ? \\
La persuasion \\
La peur des mots \\
La philosophie peut-elle être expérimentale ? \\
La politesse \\
La politique doit-elle refuser l'utopie ? \\
La politique est-elle extérieure au droit ? \\
La politique est-elle l'affaire de tous ? \\
La politique est-elle une technique ? \\
L'apprentissage \\
La présence \\
L'\emph{a priori} \\
La promesse \\
La propriété \\
La protection sociale \\
La prudence \\
La pudeur \\
La puissance du peuple \\
La raison a-t-elle une histoire ? \\
La raison d'État \\
La raison doit-elle être cultivée ? \\
La raison est-elle le pouvoir de distinguer le vrai du faux ? \\
La raison peut-elle errer ? \\
La raison peut-elle nous commander de croire ? \\
La raison pratique \\
La réalité de la vie s'épuise-t-elle dans celle des vivants ? \\
La réalité du temps se réduit-elle à la conscience que nous en avons ? \\
La réalité est-elle une idée ? \\
La réalité sociale \\
La reconnaissance \\
La rectitude du droit \\
La référence aux faits suffit-elle à garantir l'objectivité de la connaissance ? \\
La religion a-t-elle besoin d'un dieu ? \\
La religion civile \\
La religion est-elle la sagesse des pauvres ? \\
La rencontre d'autrui \\
La répétition \\
La résistance de la matière \\
L'argent \\
La rigueur \\
La rivalité \\
L'arme rhétorique \\
L'art de faire croire \\
L'art doit-il refaire le monde ? \\
L'art éduque-t-il l'homme ? \\
L'art est-il le produit de l'inconscient ? \\
L'art est-il le règne des apparences ? \\
L'art est-il un jeu ? \\
L'art et le temps \\
L'art exprime-t-il ce que nous ne saurions dire ? \\
L'artiste et la société \\
L'artiste et le savant \\
L'art nous fait-il mieux percevoir le réel ? \\
L'art peut-il contribuer à éduquer les hommes ? \\
L'art peut-il être abstrait ? \\
La santé est-elle un devoir ? \\
La science commence-t-elle avec la perception ? \\
La science commence-telle avec la perception ? \\
La science est-elle indépendante de toute métaphysique ? \\
La science n'est-elle qu'une activité théorique ? \\
La science n'est-elle qu'une fiction ? \\
La sécurité \\
La sensibilité \\
La servitude peut-elle être volontaire ? \\
La servitude volontaire \\
La sexualité \\
La simplicité du bien \\
La sincérité \\
La société civile \\
La société contre l'État \\
La société du genre humain \\
La société est-elle concevable sans le travail ? \\
La société peut-elle se passer de l'État ? \\
La société sans l'État \\
La solidarité \\
La solitude \\
La souffrance \\
La souffrance a-t-elle un sens ? \\
La souveraineté \\
La subjectivité \\
La succession des théories scientifiques \\
La superstition \\
La technique a-t-elle une histoire ? \\
La technique fait-elle violence à la nature ? \\
La technique n'est-elle qu'une application de la science ? \\
La technique permet-elle de réaliser tous les désirs ? \\
La tempérance \\
La théorie et la pratique \\
La totalité \\
La tradition \\
L'attention \\
La tyrannie \\
L'autorité de l'État \\
L'autre et les autres \\
La valeur de l'échange \\
La valeur d'une action se mesure-t-elle à sa réussite ? \\
La valeur du plaisir \\
La valeur du travail \\
L'avenir a-t-il une réalité ? \\
La vérité a-t-elle une histoire ? \\
La vérité de l'apparence \\
La vérité des sciences \\
La vérité doit-elle toujours être démontrée ? \\
La vérité est-elle éternelle ? \\
La vérité est-elle une construction ? \\
La vérité historique \\
La vérité n'est-elle qu'une erreur rectifiée ? \\
La vérité nous contraint-elle ? \\
La vérité peut-elle être équivoque ? \\
La vérité philosophique \\
La vertu du plaisir \\
La vertu peut-elle s'enseigner ? \\
L'aveu \\
La vie de l'esprit \\
La vie des machines \\
La vie est-elle le bien le plus précieux ? \\
La vie intérieure \\
La vie peut-elle être éternelle ? \\
La violence d'État \\
La voix de la conscience \\
La volonté peut-elle nous manquer ? \\
L'axiome \\
Le bon goût \\
Le bonheur des autres \\
Le bonheur est-il un accident ? \\
Le bonheur peut-il être un droit ? \\
Le bricolage \\
L'échange est-il un facteur de paix ? \\
Le citoyen a-t-il perdu toute naturalité ?L'étranger \\
Le commencement \\
Le commerce adoucit-il les mœurs ? \\
Le commerce équitable \\
Le commerce peut-il être équitable ? \\
Le complexe \\
Le conflit de devoirs \\
Le conflit des devoirs \\
Le conflit des interprétations \\
Le conflit entre la science et la religion est-il inévitable ? \\
Le corps du travailleur \\
Le corps et le temps \\
Le corps et l'instrument \\
Le corps n'est-il que matière ? \\
Le corps pense-t-il ? \\
Le corps politique \\
Le cosmopolitisme \\
Le cours des choses \\
L'écriture et la parole \\
L'écriture et la pensée \\
L'écriture ne sert-elle qu'à consigner la pensée ? \\
Le désir de reconnaissance \\
Le désir d'éternité \\
Le désir de vérité \\
Le désir d'immortalité \\
Le désir est-il l'essence de l'homme ? \\
Le désir n'est-il que l'épreuve d'un manque ? \\
Le désir peut-il se satisfaire de la réalité ? \\
Le désœuvrement \\
Le destin \\
Le devenir \\
Le devoir d'aimer \\
Le devoir et la dette \\
Le devoir se présente-t-il avec la force de l'évidence ? \\
Le dilemme \\
Le droit à la citoyenneté \\
Le droit à l'erreur \\
Le droit au bonheur \\
Le droit au respect de la vie privée \\
Le droit au travail \\
Le droit de vivre \\
Le droit peut-il se fonder sur la force ? \\
Le fait d'exister \\
Le fait social est-il une chose ? \\
Le fanatisme \\
Le féminisme \\
Le futur est-il contingent ? \\
L'égalité devant la loi \\
L'égalité peut-elle être une menace pour la liberté ? \\
Le gouvernement de soi et des autres \\
Le hasard \\
Le hasard n'est-il que la mesure de notre ignorance ? \\
Le jugement critique peut-il s'exercer sans culture ? \\
Le laboratoire \\
Le langage animal \\
Le langage du corps \\
Le langage ne sert-il qu'à communiquer ? \\
Le libre arbitre \\
Le libre échange \\
Le lieu de l'esprit \\
Le littéral et le figuré \\
Le loisir \\
L'émancipation \\
Le matériel \\
Le mépris \\
Le mien et le tien \\
Le miracle \\
Le moi \\
Le moi est-il haïssable ? \\
Le monde de la technique \\
Le monde du travail \\
L'empathie \\
L'emploi du temps \\
L'enfant \\
L'engendrement \\
L'ennui \\
Le passé a-t-il plus de réalité que l'avenir ? \\
Le passé peut-il être un objet de connaissance ? \\
Le paternalisme \\
Le patrimoine \\
Le philosophe a-t-il besoin de l'histoire ?Prouver et justifier \\
Le plaisir est-il la fin du désir ? \\
Le point de vue \\
Le possible \\
Le possible et le réel \\
Le pouvoir politique peut-il échapper à l'arbitraire ? \\
Le présent \\
L'épreuve de la liberté \\
Le principe de raison \\
Le privé et le public \\
Le prix de la liberté \\
Le profit est-il la fin de l'échange ? \\
Le progrès \\
L'équité \\
L'équivocité \\
L'équivocité du langage \\
Le quotidien \\
Le réel et l'idéal \\
Le relativisme \\
Le remords \\
Le retour à l'expérience \\
Le risque \\
Le risque technique \\
L'erreur \\
L'erreur et la faute \\
Le sacrifice \\
Le sacrifice de soi \\
Le savoir-faire \\
Les cérémonies \\
Les cinq sens \\
Les commandements divins \\
Le scrupule \\
Les devoirs envers soi-même \\
Les échanges, facteurs de paix ? \\
Le secret d'État \\
Le sens de l'État \\
Le sens de l'existence \\
Le sens de l'histoire \\
Le sens du silence \\
Le sensible est-il communicable ? \\
Le sensible est-il irréductible à l'intelligible ? \\
Les faits parlent-ils d'eux-mêmes ? \\
Les hommes sont-ils faits pour s'entendre ? \\
Les idées ont-elles une histoire ? \\
Les idées ont-elles une réalité ? \\
Le silence \\
Les leçons de l'expérience \\
Les limites de la vérité \\
Les limites de l'interprétation \\
Les limites de l'obéissance \\
Les limites du réel \\
Les limites du vivant \\
Les lois nous rendent-elles meilleurs ? \\
Les maladies de l'esprit \\
Les matériaux \\
Les mots disent-ils les choses ? \\
Les mots et les concepts \\
Le sommeil de la raison \\
Le souci de l'avenir \\
Le souci de soi est-il une attitude morale ? \\
L'espace de la perception \\
Les passions sont-elles un obstacle à la vie sociale ? \\
Les phénomènes inconscients sont-ils réductibles à une mécanique cérébrale ? \\
Le spirituel et le temporel \\
Les principes et les éléments \\
Les principes sont-ils indémontrables ?Qu'est-ce qu'être ensemble ? \\
L'esprit d'invention \\
L'esprit est-il plus aisé à connaître que le corps ? \\
L'esprit est-il une machine ? \\
L'esprit est-il un ensemble de facultés ? \\
L'esprit n'a-t-il jamais affaire qu'à lui-même ? \\
L'esprit peut-il être objet de science ? \\
L'esprit s'explique-t-il par une activité cérébrale ? \\
L'esprit tranquille \\
Les religions sont-elles des illusions ? \\
Les riches et les pauvres \\
Les sciences ne sont-elles qu'une description du monde ? \\
Les théories scientifiques décrivent-elles la réalité ? \\
L'estime de soi \\
Le sujet de droit \\
Le sujet moral \\
Les vertus de l'amour \\
Les vivants \\
Le symbole \\
Le talent et le génie \\
L'État est-il appelé à disparaître ? \\
L'État est-il un moindre mal ? \\
L'État et la guerre \\
L'État mondial \\
L'État peut-il être libéral ? \\
L'État providence \\
Le témoignage \\
Le temps de la liberté \\
Le temps de la science \\
Le temps du désir \\
Le temps est-il essentiellement destructeur ? \\
L'éternité \\
L'ethnocentrisme \\
Le travail est-il une fin ? \\
Le travail et l'œuvre \\
Le travail nous rend-il solidaires ? \\
Le travail sur soi \\
L'être et le devoir-être \\
Le tribunal de l'histoire \\
L'eugénisme \\
L'événement \\
L'évidence a-t-elle une valeur absolue ? \\
Le virtuel \\
Le vivant a-t-il des droits ? \\
Le vivant est-il entièrement connaissable ? \\
Le vrai doit-il être démontré ? \\
Le vrai se perçoit-il ? \\
L'existence de Dieu \\
L'existence est-elle un jeu ? \\
L'expérience de la liberté \\
L'expérience, est-ce l'observation ? \\
L'expérimentation \\
L'expérimentation sur l'être humain \\
L'expression du désir \\
L'habileté et la prudence \\
L'histoire est-elle avant tout mémoire ? \\
L'histoire est-elle le règne du hasard ? \\
L'histoire est-elle tragique ? \\
L'histoire est-elle un genre littéraire ? \\
L'histoire est-elle utile ? \\
L'histoire peut-elle être universelle ? \\
L'historien peut-il se passer du concept de causalité ? \\
L'homme est-il un être de devoir ? \\
L'homme est-il un être social par nature ? \\
L'homme injuste peut-il être heureux ? \\
L'homme libre est-il un homme seul ? \\
L'hospitalité \\
Libéral et libertaire \\
Liberté et habitude \\
Libre arbitre et liberté \\
L'idéal démonstratif \\
L'idée de continuité \\
L'idée d'éternité \\
L'identité personnelle est-elle donnée ou construite ? \\
L'identité relève-telle du champ politique ? \\
L'image \\
L'imaginaire \\
L'imagination et la raison \\
L'imitation \\
L'immatériel \\
L'impartialité des historiens \\
L'impuissance de l'État \\
L'incommensurable \\
L'inconscient a-t-il une histoire ? \\
L'inconscient est-il l'animal en nous ? \\
L'inconscient est-il une dimension de la conscience ? \\
L'inconscient n'est-il qu'un défaut de conscience ? \\
L'inculture \\
L'indépendance \\
L'indésirable \\
L'indice \\
L'indicible \\
L'indifférence \\
L'indiscutable \\
L'individu \\
L'individualisme \\
L'individu face à L'État \\
L'innocence \\
L'inquiétude \\
L'insatisfaction \\
L'inspiration \\
L'instant \\
L'instrument \\
L'intemporel \\
L'intérêt peut-il être une valeur morale ? \\
L'intériorité est-elle un mythe ? \\
L'interprétation de la nature \\
L'interprétation est-elle sans fin ? \\
L'introspection \\
L'intuition \\
L'irrationnel \\
L'irréfutable \\
L'irresponsabilité \\
L'irréversible \\
L'objectivité \\
L'objet de la psychologie \\
L'objet du désir \\
L'objet du désir en est-il la cause ? \\
L'œuvre d'art et le plaisir \\
L'œuvre du temps \\
Logique et réalité \\
Lois naturelles et lois civiles \\
L'opinion publique \\
L'opinion vraie \\
L'oral et l'écrit \\
L'ordre établi \\
L'ordre politique exclut-il la violence ? \\
L'organisation \\
L'organisation du vivant \\
L'origine des valeurs \\
L'oubli \\
L'outil et la machine \\
L'unité du vivant \\
L'usure des mots \\
L'utopie \\
Maîtriser le vivant \\
Matière et corps \\
Morale et identité \\
Naturel et artificiel \\
N'avons-nous affaire qu'au réel ? \\
Nécessité fait loi \\
Ne sait-on rien que par expérience ? \\
Ne sommes-nous véritablement maîtres que de nos pensées ? \\
Nier la vérité \\
Notre rapport au monde peut-il n'être que technique ? \\
N'y a-t-il de beauté qu'artistique ? \\
Obéir, est-ce se soumettre ? \\
Obéissance et servitude \\
Où commence l'interprétation ? \\
Où est-on quand on pense ? \\
Parler, est-ce ne pas agir ? \\
Partager sa vie \\
Partager ses sentiments \\
Penser, est-ce dire non ? \\
Penser la matière \\
Penser par soi-même \\
Perception et connaissance \\
Perception et création artistique \\
Perception et passivité \\
Perception et souvenir \\
Perception et vérité \\
Percevoir est-ce connaître ? \\
Percevoir, est-ce interpréter ? \\
Percevoir, est-ce juger ? \\
Percevoir, est-ce reconnaître ? \\
Percevoir et juger \\
Percevoir s'apprend-il ? \\
Perçoit-on les choses comme elles sont ? \\
Perdre la raison \\
Peut-on apprendre à être heureux ? \\
Peut-on apprendre à être libre ? \\
Peut-on avoir le droit de se révolter ? \\
Peut-on concevoir une science qui ne soit pas démonstrative ? \\
Peut-on convaincre quelqu'un de la beauté d'une œuvre d'art ? \\
Peut-on croire ce qu'on veut ? \\
Peut-on croire sans être crédule ? \\
Peut-on croire sans savoir pourquoi ? \\
Peut-on définir la vie ? \\
Peut-on démontrer qu'on ne rêve pas ? \\
Peut-on désirer ce qu'on possède ? \\
Peut-on dire que rien n'échappe à la technique ? \\
Peut-on distinguer le réel de l'imaginaire ? \\
Peut-on distinguer les faits de leurs interprétations ? \\
Peut-on être citoyen du monde ? \\
Peut-on être en conflit avec soi-même ? \\
Peut-on être heureux tout seul ? \\
Peut-on être injuste et heureux ? \\
Peut-on être seul ? \\
Peut-on être soi-même en société ? \\
Peut-on expliquer le mal ? \\
Peut-on expliquer le monde par la matière ? \\
Peut-on expliquer une œuvre d'art ? \\
Peut-on fonder la morale sur la pitié ? \\
Peut-on interpréter la nature ? \\
Peut-on lutter contre le destin ? \\
Peut-on lutter contre soi-même ? \\
Peut-on maîtriser l'inconscient ? \\
Peut-on manquer de culture ? \\
Peut-on ne pas interpréter ? \\
Peut-on ne pas savoir ce que l'on veut ? \\
Peut-on ne rien devoir à personne ? \\
Peut-on nier la réalité ? \\
Peut-on nier l'évidence ? \\
Peut-on nier l'existence de la matière ? \\
Peut-on opposer nature et culture ? \\
Peut-on parler d'un droit de résistance ? \\
Peut-on penser contre l'expérience ? \\
Peut-on penser la matière ? \\
Peut-on penser l'art sans référence au beau ? \\
Peut-on penser le monde sans la technique ? \\
Peut-on penser le temps sans l'espace ? \\
Peut-on penser l'œuvre d'art sans référence à l'idée de beauté ? \\
Peut-on penser sans son corps ? \\
Peut-on penser une conscience sans objet ? \\
Peut-on perdre sa dignité ? \\
Peut-on préférer l'ordre à la justice ? \\
Peut-on prouver l'existence ? \\
Peut-on reconnaître un sens à l'histoire sans lui assigner une fin ? \\
Peut-on réduire l'esprit à la matière ? \\
Peut-on réduire un homme à la somme de ses actes ? \\
Peut-on renoncer à ses droits ? \\
Peut-on renoncer au bonheur ? \\
Peut-on rester sceptique ? \\
Peut-on se connaître soi-même ? \\
Peut-on se passer de croire ? \\
Peut-on se passer de spiritualité ? \\
Peut-on se passer d'un maître ? \\
Peut-on suspendre le temps ? \\
Peut-on toujours savoir entièrement ce que l'on dit ? \\
Peut-on tout échanger ? \\
Peut-on transformer le réel ? \\
Peut-on vivre en marge de la société ? \\
Peut-on vivre hors du temps ? \\
Peut-on vivre sans rien espérer ? \\
Peut-on vouloir le mal ? \\
Peut-on vouloir sans désirer ? \\
Photographier le réel \\
Pour être heureux, faut-il renoncer à la perfection ? \\
Pourquoi démontrer ce que l'on sait être vrai ? \\
Pourquoi démontrer ? \\
Pourquoi des rites ? \\
Pourquoi écrire ? \\
Pourquoi écrit-on les lois ? \\
Pourquoi la réalité peut-elle dépasser la fiction ? \\
Pourquoi l'économie est-elle politique ? \\
Pourquoi les hommes mentent-ils ? \\
Pourquoi obéir ? \\
Pourquoi obéit-on aux lois ? \\
Pourquoi parler de fautes de goût ? \\
Pourquoi penser à la mort ? \\
Pourquoi punir ? \\
Pourquoi rechercher le bonheur ? \\
Pourquoi s'exprimer ? \\
Pourquoi transformer le monde ? \\
Pourquoi travailler ? \\
Pourquoi y a-t-il des religions ? \\
Pouvons-nous communiquer ce que nous sentons ? \\
Prévoir \\
Prouvez-le ! \\
Puis-je comprendre autrui ? \\
Puis-je douter de ma propre existence ? \\
Quand faut-il désobéir ? \\
Quand faut-il mentir ? \\
Quand peut-on se passer de théories ? \\
Quand y a-t-il de l'art ? \\
Que devons-nous à l'État ? \\
Que dois-je à autrui ? \\
Quel est le sujet du devenir ? \\
Quel est l'objet de la perception ? \\
Quel est l'objet du désir ? \\
Quel être peut être un sujet de droits ? \\
Quelle confiance accorder au langage ? \\
Quelle est la fin de la science ? \\
Quelle vérité y-a-t-il dans la perception ? \\
Que nous apprend la poésie ? \\
Que nous apprend l'expérience ? \\
Que percevons-nous du monde extérieur ? \\
Que percevons-nous ? \\
Que perçoit-on ? \\
Que peut expliquer l'histoire ? \\
Que peut la raison ? \\
Que peut-on cultiver ? \\
Que peut-on sur autrui ? \\
Que produit l'inconscient ? \\
Que sait-on de soi ? \\
Que savons-nous de l'inconscient ? \\
Qu'est-ce que le cinéma a changé dans l'idée que l'on se fait du temps ? \\
Qu'est-ce que traduire ? \\
Qu'est-ce qu'être asocial ? \\
Qu'est-ce qu'être moderne ? \\
Qu'est-ce qu'être un sujet ? \\
Qu'est ce qui est culturel ? \\
Qu'est-ce qui est historique ? \\
Qu'est-ce qui est hors-la-loi ? \\
Qu'est-ce qui est public ? \\
Qu'est-ce qui fait la valeur d'une œuvre d'art ? \\
Qu'est-ce qui fait qu'une théorie est vraie ? \\
Qu'est-ce qui fonde la croyance ? \\
Qu'est-ce qui justifie l'hypothèse d'un inconscient ? \\
Qu'est-ce qui ne s'achète pas ? \\
Qu'est-ce qui ne s'échange pas ? \\
Qu'est-ce qui n'est pas démontrable ? \\
Qu'est-ce qui n'existe pas ? \\
Qu'est-ce qu'un acte libre ? \\
Qu'est-ce qu'un artiste ? \\
Qu'est-ce qu'un axiome ? \\
Qu'est-ce qu'un chef-d'œuvre ? \\
Qu'est-ce qu'un échange juste ? \\
Qu'est-ce qu'une constitution ? \\
Qu'est-ce qu'une croyance rationnelle ? \\
Qu'est-ce qu'une époque ? \\
Qu'est-ce qu'une erreur ? \\
Qu'est-ce qu'une existence historique ? \\
Qu'est-ce qu'une expérience de pensée ? \\
Qu'est-ce qu'une explication matérialiste ? \\
Qu'est-ce qu'une guerre juste ? \\
Qu'est-ce qu'une histoire vraie ? \\
Qu'est-ce qu'une hypothèse ? \\
Qu'est-ce qu'une idée vraie ? \\
Qu'est-ce qu'une langue bien faite ? \\
Qu'est-ce qu'une marchandise ? \\
Qu'est-ce qu'une patrie ? \\
Qu'est-ce qu'une personne morale ? \\
Qu'est-ce qu'une preuve ? \\
Qu'est-ce qu'une révolution politique ? \\
Qu'est-ce qu'une science exacte ? \\
Qu'est-ce qu'une société juste ? \\
Qu'est-ce qu'une société ouverte ? \\
Qu'est-ce qu'un esprit faux ? \\
Qu'est-ce qu'une théorie ? \\
Qu'est-ce qu'un événement ? \\
Qu'est-ce qu'un fait scientifique ? \\
Qu'est-ce qu'un geste technique ? \\
Qu'est-ce qu'un gouvernement juste ? \\
Qu'est-ce qu'un gouvernement républicain ? \\
Qu'est-ce qu'un langage technique ? \\
Qu'est-ce qu'un mineur ? \\
Qu'est-ce qu'un monstre ? \\
Qu'est-ce qu'un objet mathématique ? \\
Qu'est-ce qu'un peuple ? \\
Qu'est-ce qu'un rival ? \\
Qu'est-ce qu'un sentiment vrai ? \\
Qu'est-ce qu'un visage ? \\
Qu'est-ce qu'un vrai changement ? \\
Que veut dire : « être cultivé » ? \\
Qui croire ? \\
Qui doit faire les lois ? \\
Qui écrit l'histoire ? \\
Qui est autorisé à me dire « tu dois » ? \\
Qui est mon prochain ? \\
Qui pense ? \\
Qui suis-je ? \\
Qui travaille ? \\
Qu'oppose-t-on à la vérité ? \\
Qu'y a-t-il au-delà du réel ? \\
Qu'y a-t-il d'universel dans la culture ? \\
Raison et langage \\
Raisonner par l'absurde \\
Réfuter une théorie \\
Regarder un tableau \\
Religion et liberté \\
Rendre raison \\
Rester soi-même \\
Revenir à la nature \\
Rêver \\
Savoir démontrer \\
Savoir et pouvoir \\
Savons-nous ce que nous disons ? \\
Science et religion \\
Sciences de la nature et sciences de l'esprit \\
Se connaître soi-même \\
Se cultiver \\
Sensation et perception \\
Sens et vérité \\
Se révolter \\
Se savoir mortel \\
Signe et symbole \\
Soigner \\
Sommes-nous conscients de nos mobiles ? \\
Sommes-nous faits pour la vérité ? \\
Sommes-nous les jouets de l'histoire ? \\
Suffit-il de bien juger pour bien faire ? \\
Suffit-il de faire son devoir ? \\
Suffit-il, pour croire, de le vouloir ? \\
Suis-je maître de ma conscience ? \\
Sujet et substance \\
Sur quoi fonder la légitimité de la loi ? \\
Sur quoi fonder la société ? \\
Sur quoi fonder le droit de punir ? \\
Suspendre son jugement \\
Système et structure \\
Temps et conscience \\
Tous les hommes désirent-ils être heureux ? \\
Toute conscience n'est-elle pas implicitement morale ? \\
Toute existence est-elle indémontrable ? \\
Toute expérience appelle-t-elle une interprétation ? \\
Toute passion fait-elle souffrir ? \\
Toutes les croyances se valent-elles ? \\
Toute vérité est-elle vérifiable ? \\
Tout peut-il se vendre ? \\
Tout savoir est-il pouvoir ? \\
Tout savoir est-il transmissible ? \\
Tout travail mérite salaire \\
Tradition et raison \\
Traiter des faits humains comme des choses, est-ce considérer l'homme comme une chose ? \\
Traiter les faits humains comme des choses, est-ce réduire les hommes à des choses ? \\
Travailler par plaisir, est-ce encore travailler ? \\
Travail manuel et travail intellectuel \\
Travail manuel, travail intellectuel \\
Un acte libre est-il un acte imprévisible ? \\
Un contrat peut-il être social ? \\
Une croyance infondée est-elle illégitime ? \\
Une existence se démontre-t-elle ? \\
Une fiction peut-elle être vraie ? \\
Une foi rationnelle \\
Une idée peut-elle être fausse ? \\
Une interprétation est-elle nécessairement subjective ? \\
Une œuvre d'art peut-elle être immorale ? \\
Une religion peut-elle être fausse ? \\
Une société n'est-elle qu'un ensemble d'individus ? \\
Une société sans religion est-elle possible ? \\
Un État mondial ? \\
Valeur artistique, valeur esthétique \\
Vérité et certitude \\
Vérité et réalité \\
Vieillir \\
Violence et histoire \\
Vivre, est-ce interpréter ? \\
Vivre, est-ce lutter contre la mort ? \\
Vivre sans religion, est-ce vivre sans espoir ? \\
Voir le meilleur et faire le pire \\
Voir le meilleur, faire le pire \\
Vouloir le mal \\
Y a-t-il de la raison dans la perception ? \\
Y a-t-il de l'impensable ? \\
Y a-t-il de l'indémontrable ? \\
Y a-t-il de l'inexprimable ? \\
Y a-t-il des actes désintéressés ? \\
Y a-t-il des arts mineurs ? \\
Y a-t-il des devoirs envers soi-même ? \\
Y a-t-il des expériences absolument certaines ? \\
Y a-t-il des guerres justes ? \\
Y a-t-il des inégalités justes ? \\
Y a-t-il des lois du vivant ? \\
Y a-t-il des modèles en morale ? \\
Y a-t-il des peuples sans histoire ? \\
Y a-t-il des règles de l'art ? \\
Y a-t-il des vérités philosophiques ? \\
Y a-t-il des vérités sans preuve ? \\
Y a-t-il du non-être ? \\
Y a-t-il lieu d'opposer matière et esprit ? \\
Y a-t-il un au-delà de la vérité ? \\
Y a-t-il un auteur de l'histoire ? \\
Y a-t-il un droit à la différence ? \\
Y a-t-il un droit du plus fort ? \\
Y a-t-il une bonne imitation ? \\
Y a-t-il une connaissance du singulier ? \\
Y a-t-il une expérience de l'éternité ? \\
Y a-t-il une expérience du néant ? \\
Y a-t-il une force du droit ? \\
Y a-t-il une logique des événements historiques ? \\
Y a-t-il une réalité du hasard ? \\
Y a-t-il une science de l'homme ? \\
Y a-t-il une science politique ? \\
Y a-t-il un esprit scientifique ? \\
Y a-t-il une technique de la nature ? \\
Y a-t-il une vérité des sentiments ? \\
Y a-t-il une vérité en histoire ? \\
Y a-t-il un inconscient psychique ? \\
Y a-t-il un inconscient social ? \\
Y a-t-il un langage de l'inconscient ? \\
Y a-t-il un sens du beau ? \\
Y a-t-il un sens moral ? \\
Y a-t-il un temps pour tout ? \\
« Je mens » \\
« La vraie morale se moque de la morale » \\
« Rien n'est sans raison » \\


\subsection{CAPES}
\label{sec-2-4}

\noindent
Abstraire, est-ce se couper du réel ? \\
Action et contemplation \\
Activité et passivité \\
Agir et faire \\
Agir et réagir \\
Ai-je des devoirs envers moi-même ? \\
Ai-je un corps ou suis-je mon corps ? \\
Aimer peut-il être un devoir ? \\
Ami et ennemi \\
Amour et inconscient \\
Analyser \\
Apparence et réalité \\
Apprend-on à percevoir ? \\
Apprendre à voir \\
Apprendre et enseigner \\
À quelles conditions une expérience est-elle possible ? \\
À quoi bon imiter la nature ? \\
À quoi nos illusions tiennent-elles ? \\
À quoi reconnaît-on qu'une expérience est scientifique ? \\
À quoi reconnaît-on qu'un événement est historique ? \\
À quoi reconnaît-on une religion ? \\
À quoi reconnaît-on un être vivant ? \\
À quoi sert la technique ? \\
À quoi sert l'État ? \\
À quoi servent les preuves de l'existence de Dieu ? \\
À quoi servent les religions ? \\
À quoi tient la force de l'État ? \\
Argent et liberté \\
Argumenter et démontrer \\
Art et beauté \\
Art et création \\
Art et illusion \\
Art et imagination \\
Art et jeu \\
Art et matière \\
Art et pouvoir \\
Art et représentation \\
Art et société \\
Art et Société \\
Art et symbole \\
Art et technique \\
Art et vérité \\
A-t-on besoin de certitudes ? \\
A-t-on besoin d'experts ? \\
A-t-on besoin d'un chef ? \\
A-t-on des devoirs envers soi-même ? \\
A-t-on le droit de se révolter ? \\
Au nom de qui rend-on justice ? \\
Au nom de quoi rend-on justice ? \\
Autorité et souveraineté \\
Autrui est-il pour moi un mystère ? \\
Autrui est-il un autre moi-même ? \\
Autrui m'est-il étranger ? \\
Avoir du jugement \\
Avoir du métier \\
Avoir du pouvoir \\
Avoir mauvaise conscience \\
Avoir raison \\
Avoir un corps \\
Avoir un destin \\
Avons-nous besoin de cérémonies ? \\
Avons-nous besoin de héros ? \\
Avons-nous besoin de maîtres ? \\
Avons-nous des devoirs à l'égard de la vérité ? \\
Avons-nous des devoirs envers la nature ? \\
Avons-nous des devoirs envers les autres êtres vivants ? \\
Avons-nous des devoirs envers nous-mêmes ? \\
Avons-nous des droits sur la nature ? \\
Avons-nous intérêt à la liberté d'autrui ? \\
Avons-nous le devoir de vivre ? \\
Avons-nous peur de la liberté ? \\
Avons-nous un devoir de vérité ? \\
À chacun selon son mérite \\
À quelle condition un travail est-il humain ? \\
À quelles conditions le vivant peut-il être objet de science ? \\
À quelles conditions une démarche est-elle scientifique ? \\
À quelles conditions une hypothèse est-elle scientifique ? \\
À qui doit-on le respect ? \\
À qui doit-on obéir ? \\
À qui faut-il obéir ? \\
À qui la faute ? \\
À quoi bon démontrer ? \\
À quoi bon se parler ? \\
À quoi la perception donne-t-elle accès ? \\
À quoi peut-on reconnaître une œuvre d'art ? \\
À quoi reconnaît-on la vérité ? \\
À quoi reconnaît-on le réel ? \\
À quoi reconnaît-on qu'une activité est un travail ? \\
À quoi reconnaît-on qu'une expérience est scientifique ? \\
À quoi reconnaît-on qu'une pensée est vraie ? \\
À quoi reconnaît-on un acte libre ? \\
À quoi reconnaît-on une attitude religieuse ? \\
À quoi reconnaît-on une bonne interprétation ? \\
À quoi reconnaît-on une idéologie ? \\
À quoi sert l'histoire ? \\
À quoi servent les images ? \\
À quoi servent les lois ? \\
À quoi servent les machines ? \\
À quoi servent les preuves ? \\
À quoi servent les symboles ? \\
À quoi servent les théories ? \\
À quoi servent les voyages ? \\
À quoi tient la force des religions ? \\
À quoi tient la valeur d'une pensée ? \\
À quoi tient le pouvoir des mots ? \\
À quoi tient notre humanité ? \\
Beauté et moralité \\
Beauté et vérité \\
Bêtise et méchanceté \\
Bien agir, est-ce toujours être moral ? \\
Bien commun et intérêt particulier \\
Bonheur et satisfaction \\
Bonheur et société \\
Bonheur et technique \\
Bonheur et vertu \\
Calculer et penser \\
Cause et condition \\
Cause et effet \\
Cause et loi \\
Cause et raison \\
Ce que je pense est-il nécessairement vrai ? \\
Ce qui dépasse la raison est-il nécessairement irréel ? \\
Ce qui est ordinaire est-il normal ? \\
Ce qui est vrai est-il toujours vérifiable ? \\
Ce qui ne peut s'acheter est-il dépourvu de valeur ? \\
Certitude et conviction \\
Chance et bonheur \\
Changer, est-ce devenir un autre ? \\
Changer le monde \\
Change-t-on avec le temps ? \\
Châtier, est ce faire honneur au criminel ? \\
Choisir, est-ce renoncer ? \\
Choisissons-nous qui nous sommes ? \\
Choix et raison \\
Chose et objet \\
Chose et personne \\
Civilisation et barbarie \\
Classer \\
Colère et indignation \\
Commander \\
Comment autrui peut-il m'aider à rechercher le bonheur ? \\
Comment chercher ce qu'on ignore ? \\
Comment comprendre les faits sociaux ? \\
Comment connaître nos devoirs ? \\
Comment dire la vérité ? \\
Comment distinguer le rêvé du perçu ? \\
Comment juger de la justesse d'une interprétation ? \\
Comment le passé peut-il demeurer présent ? \\
Comment l'erreur est-elle possible ? \\
Comment l'homme peut-il se représenter le temps ? \\
Comment penser le hasard ? \\
Comment penser l'éternel ? \\
Comment peut-on définir un être vivant ? \\
Comment peut-on être heureux ? \\
Comment puis-je devenir ce que je suis ? \\
Communauté et société \\
Comprendre \\
Comprendre le réel est-ce le dominer ? \\
Comprendre une démonstration \\
Concept et métaphore \\
Concurrence et égalité \\
Connaissance de soi et conscience de soi \\
Connaissance et perception \\
Connaissons-nous la réalité des choses ? \\
Connaît-on la vie ou bien connaît-on le vivant ? \\
Connaît-on la vie ou le vivant ? \\
Connaît-on les choses telles qu'elles sont ? \\
Connaître est-ce découvrir le réel ? \\
Connaître, est-ce dépasser les apparences ? \\
Connaître la vie ou le vivant ? \\
Conquérir \\
Conscience de soi et amour de soi \\
Conscience de soi et connaissance de soi \\
Conscience et connaissance \\
Conscience et conscience de soi \\
Conscience et liberté \\
Conscience et responsabilité \\
Conscience et subjectivité \\
Conscience et volonté \\
Contempler \\
Contrainte et obligation \\
Convaincre et persuader \\
Convient-il d'opposer explication et interprétation ? \\
Corps et espace \\
Corps et matière \\
Corps et nature \\
Crainte et espoir \\
Création et production \\
Créer et produire \\
Critiquer \\
Croire, est-ce renoncer au savoir ? \\
Croire et savoir \\
Croire que Dieu existe, est-ce croire en lui ? \\
Culpabilité et responsabilité \\
Culture et civilisation \\
Culture et différence \\
Culture et éducation \\
Culture et langage \\
Culture et savoir \\
Culture et technique \\
Culture et violence \\
Dans l'action, est-ce l'intention qui compte ? \\
Dans quel but les hommes se donnent-ils des lois ? \\
Dans quelle mesure est-on l'auteur de sa propre vie ? \\
Dans quelle mesure le temps nous appartient-il ? \\
Dans quelle mesure toute philosophie est-elle critique du langage ? \\
Débattre et dialoguer \\
Déchiffrer \\
Décider \\
Découverte et justification \\
Dématérialiser \\
Démocratie et opinion \\
Démocratie et représentation \\
Démontrer et argumenter \\
Démontrer par l'absurde \\
Dépend-il de soi d'être heureux ? \\
De quel droit l'État exerce-t-il un pouvoir ? \\
De quel droit punit-on ? \\
De quelle vérité l'art est-il capable ? \\
De quoi avons-nous vraiment besoin ? \\
De quoi dépend le bonheur ? \\
De quoi dépend notre bonheur ? \\
De quoi est fait mon présent ? \\
De quoi est-fait notre présent ? \\
De quoi la philosophie est-elle le désir ? \\
De quoi la vérité libère-t-elle ? \\
De quoi le devoir libère-t-il ? \\
De quoi les logiciens parlent-ils ? \\
De quoi l'État ne doit-il pas se mêler ? \\
De quoi parlent les mathématiques ? \\
De quoi peut-on être inconscient ? \\
De quoi peut-on faire l'expérience ? \\
De quoi pouvons-nous être sûrs ? \\
De quoi puis-je répondre ? \\
De quoi sommes-nous responsables ? \\
De quoi suis-je inconscient ? \\
Déraisonner, est-ce perdre de vue le réel ? \\
Désirer, est-ce être aliéné ? \\
Désir et besoin \\
Désir et bonheur \\
Désir et interdit \\
Désir et langage \\
Désir et manque \\
Désir et ordre \\
Désir et pouvoir \\
Désir et raison \\
Désir et réalité \\
Désir et volonté \\
Des lois justes suffisent-elles à assurer la justice ? \\
Devant qui sommes-nous responsables ? \\
Devoir et bonheur \\
Devoir et contrainte \\
Devoir et intérêt \\
Devoir et liberté \\
Devoir et plaisir \\
Devoir et prudence \\
Devoir et vertu \\
Devoirs envers les autres et devoirs envers soi-même \\
Devoirs et passions \\
Devons-nous dire la vérité ? \\
Dire, est-ce faire ? \\
Dire et exprimer \\
Dire et faire \\
Dire je \\
Dogme et opinion \\
Doit-on apprendre à percevoir ? \\
Doit-on apprendre à vivre ? \\
Doit-on bien juger pour bien faire ? \\
Doit-on changer ses désirs, plutôt que l'ordre du monde ? \\
Doit-on identifier l'âme à la conscience ? \\
Doit-on interpréter les rêves ? \\
Doit-on le respect au vivant ? \\
Doit-on mûrir pour la liberté ? \\
Doit-on rechercher le bonheur ? \\
Doit-on refuser d'interpréter ? \\
Doit-on respecter les êtres vivants ? \\
Doit-on se passer des utopies ? \\
Doit-on tenir le plaisir pour une fin ? \\
Doit-on toujours dire la vérité ? \\
Doit-on tout accepter de l'État ? \\
Doit-on tout attendre de l'État ? \\
Don et échange \\
Donner, à quoi bon ? \\
Donner et recevoir \\
Donner sa parole \\
Doute et raison \\
D'où vient la certitude ? \\
D'où vient la servitude ? \\
Droit et coutume \\
Droit et devoir \\
Droit et morale \\
Droit et violence \\
Droits de l'homme ou droits du citoyen ? \\
Droits et devoirs \\
Durée et instant \\
Durer \\
Échange et don \\
Échanger des idées \\
Échanger, est-ce créer de la valeur ? \\
Échanger, est-ce partager ? \\
Écouter et entendre \\
Écrire et parler \\
Égalité et différence \\
En quel sens l'État est-il rationnel ? \\
En quel sens le vivant a-t-il une histoire ? \\
En quel sens parler de lois de la pensée ? \\
En quel sens parler d'identité culturelle ? \\
En quel sens peut-on dire que la vérité s'impose ? \\
En quel sens peut-on dire que l'homme est un animal politique ? \\
En quel sens peut-on dire qu' « on expérimente avec sa raison » ? \\
En quel sens peut-on parler de la mort de l'art ? \\
En quel sens peut-on parler d'une culture technique ? \\
En quel sens peut-on parler d'une interprétation de la nature ? \\
En quoi la connaissance du vivant contribue-t-elle à la connaissance de l'homme ? \\
En quoi la méthode est-elle un art de penser ? \\
En quoi l'art peut-il intéresser le philosophe ? \\
En quoi le bien d'autrui m'importe-t-il ? \\
En quoi le bonheur est-il l'affaire de l'État ? \\
En quoi les vivants témoignent-ils d'une histoire ? \\
Entre l'opinion et la science, n'y a-t-il qu'une différence de degré ? \\
Erreur et faute \\
Erreur et illusion \\
Essence et existence \\
Est-ce à la raison de déterminer ce qui est réel ? \\
Est-ce de la force que l'État tient son autorité ? \\
Est-ce la majorité qui doit décider ? \\
Est-ce l'autorité qui fait la loi ? \\
Est-ce le cerveau qui pense ? \\
Est-ce l'échange utilitaire qui fait le lien social ? \\
Est-ce l'ignorance qui rend les hommes croyants ? \\
Est-ce l'intérêt qui fonde le lien social ? \\
Est-ce un devoir d'aimer son prochain ? \\
Est-il immoral de se rendre heureux ? \\
Est-il juste d'interpréter la loi ? \\
Est-il légitime d'opposer liberté et nécessité ? \\
Est-il naturel à l'homme de parler ? \\
Est-il possible de tout avoir pour être heureux ? \\
Est-il possible d'être immoral sans le savoir ? \\
Est-il raisonnable d'aimer ? \\
Est-il raisonnable d'être rationnel ? \\
Est-il raisonnable de vouloir maîtriser la nature ? \\
Est-il vrai que les animaux ne pensent pas ? \\
Est-il vrai que l'ignorant n'est pas libre ? \\
Est-il vrai que ma liberté s'arrête là où commence celle des autres ? \\
Est-il vrai que plus on échange, moins on se bat ? \\
Estime et respect \\
Estimer \\
Est-on l'auteur de sa propre vie ? \\
Est-on libre face à la vérité ? \\
Est-on sociable par nature ? \\
État et institutions \\
État et nation \\
État et Société \\
État et société civile \\
Éthique et Morale \\
Être à l'écoute de son désir, est-ce nier le désir de l'autre ? \\
Être bon juge \\
Être conscient, est-ce être maître de soi ? \\
Être cultivé rend-il meilleur ? \\
Être de son temps \\
Être dogmatique \\
Être et apparaître \\
Être et avoir \\
Être et avoir été \\
Être et devenir \\
Être et devoir être \\
Être et exister \\
Être et paraître \\
Être exemplaire \\
Être heureux, est-ce devoir ? \\
Être libre, est-ce dire non ? \\
Être libre est-ce faire ce que l'on veut ? \\
Être libre, est-ce pouvoir choisir ? \\
Être libre, est-ce se suffire à soi-même ? \\
Être raisonnable, est-ce renoncer à ses désirs ? \\
Être réaliste \\
Être sceptique \\
Être soi-même \\
Être spectateur \\
Être un sujet, est-ce être maître de soi ? \\
Être vertueux \\
Évidence et raison \\
Évidence et vérité \\
Évidences et préjugés \\
Évolution biologique et culture \\
Évolution et progrès \\
Excuser et pardonner \\
Existence et contingence \\
Exister, est-ce simplement vivre ? \\
Existe-t-il des choses en soi ? \\
Existe-t-il des désirs coupables ? \\
Existe-t-il une méthode pour rechercher la vérité ? \\
Existe-t-il une méthode pour trouver la vérité ? \\
Expérience et expérimentation \\
Expérience immédiate et expérimentation scientifique \\
Expliquer et comprendre \\
Fabriquer et créer \\
Faire confiance \\
Faire de nécessité vertu \\
Faire la paix \\
Faire le mal \\
Faire l'histoire \\
Faire son devoir, est-ce là toute la morale ? \\
Faisons-nous l'histoire ? \\
Fait et fiction \\
Fait et preuve \\
Fait et valeur \\
Faits et preuves \\
Faudrait-il ne rien oublier ? \\
Faudrait-il vivre sans passion ? \\
Faut-avoir peur de la technique ? \\
Faut-il accepter sa condition ? \\
Faut-il accorder de l'importance aux mots ? \\
Faut-il affirmer son identité ? \\
Faut-il aimer autrui pour le respecter ? \\
Faut-il aimer son prochain comme soi-même ? \\
Faut-il apprendre à être libre ? \\
Faut-il apprendre à vivre en renonçant au bonheur ? \\
Faut-il avoir peur des machines ? \\
Faut-il avoir peur d'être libre ? \\
Faut-il avoir peur du désordre ? \\
Faut-il changer ses désirs plutôt que l'ordre du monde ? \\
Faut-il chercher à satisfaire tous nos désirs ? \\
Faut-il chercher à se connaître ? \\
Faut-il chercher la paix à tout prix ? \\
Faut-il chercher le bonheur à tout prix ? \\
Faut-il chercher un sens à l'histoire ? \\
Faut-il choisir entre être heureux et être libre ? \\
Faut-il connaître l'Histoire pour gouverner ? \\
Faut-il craindre le développement des techniques ? \\
Faut-il craindre les machines ? \\
Faut-il craindre l'État ? \\
Faut-il craindre l'ordre ? \\
Faut-il croire en la science ? \\
Faut-il croire les historiens ? \\
Faut-il croire que l'histoire a un sens ? \\
Faut-il dire de la justice qu'elle n'existe pas ? \\
Faut-il distinguer désir et besoin ? \\
Faut-il douter de ce qu'on ne peut pas démontrer ? \\
Faut-il du passé faire table rase ? \\
Faut-il espérer pour agir ? \\
Faut-il être cohérent ? \\
Faut-il être libre pour être heureux ? \\
Faut-il être modéré ? \\
Faut-il être pragmatique ? \\
Faut-il faire confiance au progrès technique ? \\
Faut-il faire de nécessité vertu ? \\
Faut-il faire table rase du passé ? \\
Faut-il hiérarchiser les désirs ? \\
Faut-il hiérarchiser les formes de vie ? \\
Faut-il libérer l'humanité du travail ? \\
Faut-il limiter la souveraineté de l'État ? \\
Faut-il limiter le pouvoir de l'État ? \\
Faut-il ne manquer de rien pour être heureux ? \\
Faut-il obéir à la voix de sa conscience ? \\
Faut-il opposer histoire et mémoire ? \\
Faut-il opposer la matière et l'esprit ? \\
Faut-il opposer le don et l'échange ? \\
Faut-il opposer le temps vécu et le temps des choses ? \\
Faut-il opposer raison et sensation ? \\
Faut-il oublier le passé pour se donner un avenir ? \\
Faut-il pour le connaître faire du vivant un objet ? \\
Faut-il préférer l'art à la nature ? \\
Faut-il rechercher le bonheur ? \\
Faut-il rejeter tous les préjugés ? \\
Faut-il rejeter toute norme ? \\
Faut-il renoncer à faire du travail une valeur ? \\
Faut-il renoncer à l'idée d'âme ? \\
Faut-il respecter le vivant ? \\
Faut-il rompre avec le passé ? \\
Faut-il s'adapter ? \\
Faut-il s'affranchir des désirs ? \\
Faut-il s'aimer soi-même ? \\
Faut-il savoir mentir ? \\
Faut-il savoir pour agir ? \\
Faut-il se cultiver ? \\
Faut-il se détacher du monde ? \\
Faut-il se fier à sa propre raison ? \\
Faut-il se fier aux apparences ? \\
Faut-il se libérer du travail ? \\
Faut-il se méfier de l'intuition ? \\
Faut-il se méfier de ses désirs ? \\
Faut-il s'en tenir aux faits ? \\
Faut-il se rendre à l'évidence ? \\
Faut-il se ressembler pour former une société ? \\
Faut-il toujours éviter de se contredire ? \\
Faut-il toujours faire son devoir ? \\
Faut-il tout critiquer ? \\
Faut-il un corps pour penser ? \\
Faut-il vivre avec son temps ? \\
Faut-il vivre comme si nous ne devions jamais mourir ? \\
Faut-il vivre comme si on ne devait jamais mourir ? \\
Faut-il vivre hors de la société pour être heureux ? \\
Faut-il vouloir être heureux ? \\
Foi et raison \\
Foi et savoir \\
Foi et superstition \\
Force et violence \\
Forme et matière \\
Former et éduquer \\
Forme-t-on son esprit en transformant la matière ? \\
Gouvernement des hommes et administration des choses \\
Gouverner \\
Gouverner, est-ce régner ? \\
Guerre et politique \\
Habiter \\
Hasard et destin \\
Hier a-t-il plus de réalité que demain ? \\
Histoire et mémoire \\
Histoire et progrès \\
Histoire et violence \\
Histoire individuelle et histoire collective \\
Hypothèse et vérité \\
Ici et maintenant \\
Idéal et utopie \\
Idée et réalité \\
Identité et différence \\
Image et concept \\
Image et idée \\
Imagination et culture \\
Imagination et pouvoir \\
Imitation et création \\
Imitation et représentation \\
Inconscient et déterminisme \\
Inconscient et inconscience \\
Inconscient et instinct \\
Inconscient et liberté \\
Inconscient et mythes \\
Indépendance et autonomie \\
Indépendance et liberté \\
Individu et citoyen \\
Individu et communauté \\
Individu et société \\
Innocence et ignorance \\
Instruction et éducation \\
Instruire et éduquer \\
Intérêt général et bien commun \\
Interprétation et création \\
Interpréter \\
Interpréter, est-ce connaître ? \\
Interpréter est-il subjectif ? \\
Interpréter et traduire \\
Interprète-t-on à défaut de connaître ? \\
Interroger \\
Intuition et déduction \\
Invention et découverte \\
Jugement et réflexion \\
Jugement et vérité \\
Juger et sentir \\
Jusqu'à quel point la nature est-elle objet de science ? \\
Justice et charité \\
Justice et égalité \\
Justice et équité \\
Justice et pardon \\
Justice et vengeance \\
Justice et violence \\
La barbarie \\
La beauté du monde \\
La beauté est-elle dans les choses ? \\
La beauté est-elle intemporelle ? \\
La beauté morale \\
La beauté nous rend-elle meilleurs ? \\
La beauté s'explique-t-elle ? \\
La bête \\
La bête et l'animal \\
La bêtise \\
La bonne conscience \\
La bonne intention \\
La bonne volonté \\
La bonté \\
L'absence \\
L'absolu \\
L'absolu et le relatif \\
L'abstraction \\
L'abstrait et le concret \\
L'absurde \\
L'abus de pouvoir \\
L'académisme \\
La causalité \\
La causalité en histoire \\
La cause efficiente \\
La cause et l'effet \\
La certitude de mourir \\
La chair \\
La chance \\
La cité \\
La coexistence des libertés \\
La cohérence est-elle la norme du vrai ? \\
La cohérence logique est-elle une condition suffisante de la démonstration ? \\
La colère \\
La communauté scientifique \\
La compétence \\
La compréhension \\
La confiance \\
La confiance est-elle une vertu ? \\
La connaissance commune fait-elle obstacle à la vérité ? \\
La connaissance des principes \\
La connaissance du vivant est-elle désintéressée ? \\
La connaissance du vivant peut-elle être désintéressée  ? \\
La connaissance est-elle une contemplation ? \\
La connaissance et la croyance \\
La connaissance historique est-elle une interprétation des faits ? \\
La connaissance historique est-elle utile à l'homme ? \\
La connaissance objective doit-elle s'interdire toute interprétation ? \\
La connaissance objective exclut-elle toute forme de subjectivité ? \\
La connaissance scientifique \\
La connaissance sensible \\
La conscience \\
La conscience a-t-elle des degrés ? \\
La conscience collective \\
La conscience d'autrui est-elle impénétrable ? \\
La conscience de la mort est-elle une condition de la sagesse ? \\
La conscience de soi \\
La conscience de soi est-elle une donnée immédiate ? \\
La conscience de soi et l'identité personnelle \\
La conscience de soi suppose-t-elle autrui ? \\
La conscience du temps rend-elle l'existence tragique ? \\
La conscience est-elle nécessairement malheureuse ? \\
La conscience est-elle source d'illusions ? \\
La conscience est-elle toujours morale ? \\
La conscience est-elle une activité ? \\
La conscience est-elle une illusion ? \\
La conscience et l'inconscient \\
La conscience morale \\
La conscience morale n'est-elle que le fruit de l'éducation ? \\
La conscience morale n'est-elle que le produit de l'éducation ? \\
La conscience peut-elle nous tromper ? \\
La contingence de l'existence \\
La contradiction \\
La contrainte peut-elle être légitime ? \\
La controverse scientifique \\
La convention et l'arbitraire \\
La conviction \\
La corruption \\
La courtoisie \\
La coutume \\
La création \\
La création artistique \\
La création de valeur \\
La crédulité \\
La croyance et la foi \\
La croyance et la raison \\
La croyance peut-elle tenir lieu de savoir ? \\
La croyance religieuse échappe-t-elle à toute logique ? \\
L'acte et la parole \\
L'action \\
L'action et le risque \\
L'action politique \\
L'actualité \\
La culture \\
La culture est-elle la négation de la nature ? \\
La culture est-elle un luxe ? \\
La culture garantit-elle l'excellence humaine ? \\
La culture générale \\
La culture nous rend-elle plus humains ? \\
La culture nous unit-elle ? \\
La culture rend-elle plus humain ? \\
La culture technique \\
La curiosité \\
La décision \\
La défense de l'intérêt général est-il la fin dernière de la politique ? \\
La démarche scientifique exclut-elle tout recours à l'imagination ? \\
La démocratie, est-ce le pouvoir du plus grand nombre ? \\
La démocratie est-elle la loi du plus fort ? \\
La démocratie est-elle le règne de l'opinion ? \\
La démocratie peut-elle échapper à la démagogie ? \\
La démocratie peut-elle être représentative ? \\
La démonstration \\
La démonstration nous garantit-elle l'accès à la vérité ? \\
La démonstration supprime-t-elle le doute ? \\
La déraison \\
La désobéissance \\
La détermination \\
La dialectique \\
La différence des sexes est-elle un problème philosophique ? \\
La dignité \\
La discipline \\
La discorde \\
La discrétion \\
La disharmonie \\
La distinction \\
La diversité \\
La diversité des opinions conduit-elle à douter de tout ? \\
La division du travail \\
L'admiration \\
La douleur nous apprend-elle quelque chose ? \\
La faiblesse \\
La familiarité \\
La famille \\
La famille est-elle un modèle de société ? \\
La fatigue \\
La fermeté \\
La fête \\
La fiction \\
La fidélité \\
La finalité \\
La finalité est-elle nécessaire pour penser le vivant ? \\
La fin de l'État \\
La fin de l'histoire \\
La fin du travail \\
La fin et les moyens \\
La finitude \\
La fin justifie-t-elle les moyens ? \\
La foi \\
La foi est-elle aveugle ? \\
La foi est-elle rationnelle ? \\
La folie \\
La fonction \\
La fonction et l'organe \\
La force de la vérité \\
La force de l'esprit \\
La force de l'État est-elle nécessaire à la liberté des citoyens ? \\
La force de l'habitude \\
La force des idées \\
La force du droit \\
La force et le droit \\
La franchise \\
La fraternité \\
La fuite du temps est-elle nécessairement un malheur ? \\
La générosité \\
La grandeur \\
La guerre \\
La guerre et la paix \\
La guerre peut-elle être juste ? \\
La haine \\
La haine de la raison \\
La hiérarchie \\
La honte \\
La joie \\
La jurisprudence \\
La juste mesure \\
La justice \\
La justice a-t-elle un fondement rationnel ? \\
La justice est-elle l'affaire de l'État ? \\
La justice est-elle une vertu ? \\
La justice et la force \\
La justice et la loi \\
La justice et la paix \\
La justice et le droit \\
La justice et l'égalité \\
La justice n'est-elle qu'une institution ? \\
La justice n'est-elle qu'un idéal ? \\
La justice peut-elle se passer d'institutions ? \\
La justice sociale \\
La justice suppose-t-elle l'égalité ? \\
La lâcheté \\
La laideur \\
La langue et la parole \\
La légèreté \\
La légitime défense \\
La lettre et l'esprit \\
La liberté \\
La liberté a-t-elle un prix ? \\
La liberté comporte-t-elle des degrés ? \\
La liberté connaît-elle des excès ? \\
La liberté de croire \\
La liberté de l'interprète \\
La liberté de penser \\
La liberté d'expression est-elle nécessaire à la liberté de pensée ? \\
La liberté d'indifférence \\
La liberté du choix \\
La liberté est-elle le pouvoir de refuser ? \\
La liberté et l'égalité sont-elles compatibles ? \\
La liberté et le hasard \\
La liberté et le temps \\
La liberté fait-elle de nous des êtres meilleurs ? \\
La liberté implique-t-elle l'indifférence ? \\
La liberté n'est-elle qu'une illusion ? \\
La liberté nous rend-elle inexcusables ? \\
La liberté peut-elle être prouvée ? \\
La liberté peut-elle faire peur ? \\
La liberté peut-elle se refuser ? \\
La liberté requiert-elle le libre échange ? \\
La liberté s'achète-t-elle ? \\
La liberté se mérite-t-elle ? \\
La libre interprétation \\
L'aliénation \\
La loi \\
La loi dit-elle ce qui est juste ? \\
La loi est-elle une garantie contre l'injustice ? \\
La loi et la coutume \\
La loi et l'ordre \\
La loi peut-elle changer les mœurs ? \\
La loyauté \\
L'altérité \\
L'altruisme \\
L'altruisme n'est-il qu'un égoïsme bien compris ? \\
La machine \\
La main \\
La main et l'esprit \\
La maîtrise de soi \\
La majorité doit-elle toujours l'emporter ? \\
La majorité, force ou droit ? \\
La majorité peut-elle être tyrannique ? \\
La maladie \\
La maladie est-elle à l'organisme vivant ce que la panne est à la machine ? \\
La maladie est-elle indispensable à la connaissance du vivant ? \\
La marginalité \\
La mathématisation du réel \\
La matière \\
La matière est-elle plus facile à connaître que l'esprit ? \\
La matière et la vie \\
La matière et l'esprit \\
La matière n'est-elle que ce que l'on perçoit ? \\
La matière n'est-elle qu'une idée ? \\
La maturité \\
La mauvaise conscience \\
La mauvaise foi \\
La mauvaise volonté \\
L'ambiguïté \\
L'ambiguïté des mots peut-elle être heureuse ? \\
La méchanceté \\
L'âme et le corps sont-ils une seule et même chose ? \\
La méfiance \\
L'âme jouit-elle d'une vie propre ? \\
La mélancolie \\
La mémoire \\
La mémoire collective \\
La mémoire et l'oubli \\
La mesure \\
La mesure du temps \\
La métamorphose \\
La méthode \\
La méthode expérimentale est-elle appropriée à l'étude du vivant ? \\
L'amitié \\
L'amitié est-elle une vertu ? \\
La modération \\
La modestie \\
La morale a-t-elle à décider de la sexualité ? \\
La morale a-t-elle besoin de la notion de sainteté ? \\
La morale a-t-elle sa place dans l'économie ? \\
La morale consiste-t-elle à respecter le droit ? \\
La morale dépend-elle de la culture ? \\
La morale doit-elle être rationnelle ? \\
La morale est-elle affaire de convention ? \\
La morale est-elle affaire de sentiment ? \\
La morale est-elle condamnée à n'être qu'un champ de bataille ? \\
La morale est-elle désintéressée ? \\
La morale est-elle en conflit avec le désir ? \\
La morale est-elle une affaire de raison ? \\
La morale est-elle une affaire solitaire ? \\
La morale est-elle un fait de culture ? \\
La morale et la politique \\
La morale et la religion visent-elles les mêmes fins ? \\
La morale et les mœurs \\
La morale n'est-elle qu'un ensemble de conventions ? \\
La morale peut-elle se fonder sur les sentiments ? \\
La morale peut-elle s'enseigner ? \\
La morale s'enseigne-t-elle ? \\
La morale s'oppose-t-elle à la politique ? \\
La mort \\
La mort d'autrui \\
La mort de Dieu \\
L'amour de l'art \\
L'amour de la vie \\
L'amour de soi est-il immoral ? \\
L'amour du travail \\
L'amour et l'amitié \\
L'amour et le devoir \\
L'amour et le respect \\
L'amour fou \\
L'amour peut-il être raisonnable ? \\
L'amour peut-il être un devoir ? \\
L'amour propre \\
La multitude \\
L'anachronisme \\
L'analogie \\
L'analyse du langage ordinaire peut-elle avoir un intérêt philosophique ? \\
L'anarchie \\
La nation \\
La nature \\
La nature a-t-elle des droits ? \\
La nature est-elle écrite en langage mathématique ? \\
La nature est-elle prévisible ? \\
La nature est-elle une ressource ? \\
La nature est-elle un modèle ? \\
La nature fait-elle bien les choses ? \\
La nature peut-elle avoir des droits ? \\
La nature peut-elle constituer une norme ? \\
La nature peut-elle être un modèle ? \\
La nature peut-elle nous indiquer ce que nous devons faire ? \\
La nécessité \\
Langage et communication \\
Langage et logique \\
Langage et passions \\
Langage et pensée \\
Langage et pouvoir \\
Langage et société \\
L'angoisse \\
L'animal \\
L'animal et l'homme \\
La non-violence \\
La norme \\
La nostalgie \\
La nouveauté \\
La paix \\
La paix est-elle l'absence de guerres ? \\
La paix est-elle le plus grand des biens ? \\
La paix sociale \\
La paix sociale est-elle le but de la politique ? \\
La parole \\
La parole donnée \\
La parole et l'écriture \\
La parole et le geste \\
La parole intérieure \\
La passion de la connaissance \\
La passion de la liberté \\
La passion de la vérité \\
La passion de l'égalité \\
La passion est-elle immorale ? \\
La passion est-elle l'ennemi de la raison ? \\
La passion exclut-elle la lucidité ? \\
La passivité \\
La patience \\
La pauvreté \\
La pauvreté est-elle une injustice ? \\
La peine \\
La pénibilité du travail \\
La pensée \\
La pensée doit-elle se soumettre aux règles de la logique ? \\
La pensée et la conscience sont-elles une seule et même chose ? \\
La pensée peut-elle se passer de mots ? \\
La perception construit-elle son objet ? \\
La perception de l'espace est-elle innée ou acquise ? \\
La perception est-elle le premier degré de la connaissance ? \\
La perception est-elle une interprétation ? \\
La perception me donne-t-elle le réel ? \\
La perception peut-elle s'éduquer ? \\
La perfection \\
La perfection est-elle désirable ? \\
La permanence \\
La personne et l'individu \\
La peur \\
La peur de la science \\
La philosophie et son histoire \\
La philosophie rend-elle inefficace la propagande ? \\
La pitié \\
La plaisanterie \\
La pluralité des arts \\
La pluralité des interprétations \\
La pluralité des langues \\
La pluralité des religions \\
La police \\
La politesse \\
La politesse est-elle une vertu ? \\
La politique \\
La politique consiste-t-elle à faire cause commune ? \\
La politique est-elle l'affaire des spécialistes ? \\
La politique est-elle l'affaire de tous ? \\
La politique est-elle un art ? \\
La politique est-elle une affaire d'experts ? \\
La politique est-elle une science ? \\
La politique et la guerre \\
La politique et le bonheur \\
La politique n'est-elle que l'art de conquérir et de conserver le pouvoir ? \\
La poursuite de mon intérêt m'oppose-t-elle aux autres ? \\
L'apparence \\
L'apparence est-elle toujours trompeuse ? \\
L'apprentissage \\
L'apprentissage de la liberté \\
La précarité \\
La présence d'esprit \\
La prière \\
L'\emph{a priori} \\
La privation \\
La promesse \\
L'à propos \\
La propriété \\
La propriété et le travail \\
La prudence \\
La pudeur \\
La puissance \\
La punition \\
La pureté \\
La quantité et la qualité \\
La question « qui suis-je » admet-elle une réponse exacte ? \\
La raison \\
La raison a-t-elle pour fin la connaissance ? \\
La raison a-t-elle une histoire ? \\
La raison d'État \\
La raison d'État peut-elle être justifiée ? \\
La raison doit-elle critiquer la croyance ? \\
La raison doit-elle être notre guide ? \\
La raison doit-elle se soumettre au réel ? \\
La raison engendre-t-elle des illusions ? \\
La raison épuise-t-elle le réel ? \\
La raison est-elle l'esclave du désir ? \\
La raison est-elle plus fiable que l'expérience ? \\
La raison est-elle seulement affaire de logique ? \\
La raison et le réel \\
La raison et l'expérience \\
La raison et l'irrationnel \\
La raison ne connaît-elle du réel que ce qu'elle y met d'elle-même ? \\
La raison ne veut-elle que connaître ? \\
La raison peut-elle entrer en conflit avec elle-même ? \\
La raison peut-elle errer ? \\
La raison peut-elle se contredire ? \\
La raison peut-elle servir le mal ? \\
La raison suffisante \\
La raison transforme-t-elle le réel ? \\
La rationalité \\
L'arbitraire \\
La réalité des idées \\
La réalité des phénomènes \\
La réalité du désordre \\
La réalité du temps \\
La réalité n'est-elle qu'une construction ? \\
La réalité nourrit-elle la fiction ? \\
La réalité sensible \\
La recherche de la vérité peut-elle être désintéressée ? \\
La recherche du bonheur \\
La recherche du bonheur est-elle un idéal égoïste ? \\
La reconnaissance \\
La réflexion \\
La réfutation \\
La règle et l'exception \\
La régression \\
La religion \\
La religion conduit-elle l'homme au-delà de lui-même ? \\
La religion divise-t-elle les hommes ? \\
La religion est-elle contraire à la raison ? \\
La religion est-elle fondée sur la peur de la mort ? \\
La religion est-elle l'asile de l'ignorance ? \\
La religion est-elle une affaire privée ? \\
La religion est-elle une consolation pour les hommes ? \\
La religion est-elle un instrument de pouvoir ? \\
La religion et la croyance \\
La religion implique-t-elle la croyance en un être divin ? \\
La religion naturelle \\
La religion n'est-elle que l'affaire des croyants ? \\
La religion n'est-elle qu'une affaire privée ? \\
La religion n'est-elle qu'un fait de culture ? \\
La religion peut-elle n'être qu'une affaire privée ? \\
La religion relie-t-elle les hommes ? \\
La religion rend-elle l'homme heureux ? \\
La religion rend-elle meilleur ? \\
La religion repose-t-elle sur une illusion ? \\
La religion se distingue-t-elle de la superstition ? \\
La représentation \\
La représentation politique \\
La reproduction \\
La responsabilité \\
La responsabilité politique \\
La responsabilité politique n'est-elle le fait que de ceux qui gouvernent ? \\
La réussite \\
La révolte \\
La révolution \\
L'argent \\
L'argent est-il la mesure de tout échange ? \\
La rigueur \\
La rigueur des lois ? \\
L'art \\
L'art a-t-il besoin de théorie ? \\
L'art a-t-il pour fin le plaisir ? \\
L'art a-t-il une histoire ? \\
L'art a-t-il un rôle à jouer dans l'éducation ? \\
L'art change-t-il la vie ? \\
L'art de gouverner \\
L'art de juger \\
L'art de persuader \\
L'art de vivre \\
L'art d'interpréter \\
L'art donne-t-il nécessairement lieu à la production d'une œuvre ? \\
L'art éduque-t-il la perception ? \\
L'art est-il affaire d'apparence ? \\
L'art est-il le règne des apparences ? \\
L'art est-il moins nécessaire que la science ? \\
L'art est-il subversif ? \\
L'art est-il une affaire sérieuse ? \\
L'art est-il une histoire ? \\
L'art est-il universel ? \\
L'art est-il un luxe ? \\
L'art est-il un moyen de connaître ? \\
L'art est-il un refuge ? \\
L'art et la manière \\
L'art et la morale \\
L'art et la technique \\
L'art et la vie \\
L'art et le beau \\
L'art et le jeu \\
L'art et le réel \\
L'art et le sacré \\
L'art et l'illusion \\
L'art et l'invisible \\
L'artifice \\
L'artificiel \\
L'artiste a-t-il besoin de modèle ? \\
L'artiste doit-il être de son temps ? \\
L'artiste doit-il être original ? \\
L'artiste doit-il se donner des modèles ? \\
L'artiste doit-il se soucier du goût du public ? \\
L'artiste est-il souverain ? \\
L'artiste est-il un travailleur ? \\
L'artiste et l'artisan \\
L'artiste sait-il ce qu'il fait ? \\
L'artiste travaille-t-il ? \\
L'art n'est-il qu'un mode d'expression subjectif ? \\
L'art n'est qu'une affaire de goût ? \\
L'art nous détourne-t-il de la réalité ? \\
L'art nous mène-t-il au vrai ? \\
L'art nous réconcilie-t-il avec le monde ? \\
L'art parachève-t-il la nature ? \\
L'art participe-t-il à la vie politique ? \\
L'art peut-il être conceptuel ? \\
L'art peut-il être réaliste \\
L'art peut-il être sans œuvre ? \\
L'art peut-il ne pas être sacré ? \\
L'art peut-il s'enseigner ? \\
L'art peut-il se passer de règles ? \\
L'art peut-il se passer d'œuvres ? \\
L'art pour l'art \\
L'art rend-il les hommes meilleurs ? \\
L'art s'adresse-t-il à tous ? \\
L'art s'apprend-il ? \\
La ruse \\
La sagesse et la passion \\
La santé \\
La satisfaction \\
La science a-t-elle besoin d'une méthode ? \\
La science a-t-elle le monopole de la raison ? \\
La science commence-t-elle avec la perception ? \\
La science du vivant peut-elle se passer de l'idée de finalité ? \\
La science est-elle le lieu de la vérité ? \\
La science est-elle une connaissance du réel ? \\
La science nous éloigne-t-elle de la religion ? \\
La science nous indique-t-elle ce que nous devons faire ? \\
La science permet-elle de comprendre le monde ? \\
La science peut-elle être une métaphysique ? \\
La science peut-elle produire des croyances ? \\
La science peut-elle se passer de l'idée de finalité ? \\
La science politique \\
La science se limite-t-elle à constater les faits ? \\
La sensibilité \\
La sérénité \\
La servitude \\
La servitude volontaire \\
La simplicité \\
La sincérité \\
La société \\
La société civile \\
La société doit-elle reconnaître les désirs individuels ? \\
La société est-elle un organisme ? \\
La société et les échanges \\
La société et l'État \\
La société et l'individu \\
La société fait-elle l'homme ? \\
La société peut-elle être l'objet d'une science ? \\
La société peut-elle se passer de l'État ? \\
La société repose-t-elle sur l'altruisme ? \\
La solidarité \\
La solidarité est-elle naturelle ? \\
La solitude \\
La sollicitude \\
La souffrance d'autrui \\
La souffrance d'autrui m'importe-t-elle ? \\
La souffrance peut-elle être un mode de connaissance ? \\
La soumission à l'autorité \\
La souveraineté \\
La souveraineté de l'État \\
La souveraineté peut-elle se partager ? \\
La spontanéité \\
L'association des idées \\
La superstition \\
La sympathie \\
La technique \\
La technique accroît-elle notre liberté ? \\
La technique a-t-elle sa place en politique ? \\
La technique est-elle civilisatrice ? \\
La technique est-elle contre-nature ? \\
La technique est-elle le propre de l'homme ? \\
La technique est-elle neutre ? \\
La technique est-elle un savoir ? \\
La technique et le corps \\
La technique et le travail \\
La technique libère-t-elle les hommes ? \\
La technique ne fait-elle qu'appliquer la science ? \\
La technique ne pose-t-elle que des problèmes techniques ? \\
La technique n'est-elle pour l'homme qu'un moyen ? \\
La technique n'est-elle qu'un outil au service de l'homme ? \\
La technique n'existe-elle que pour satisfaire des besoins ? \\
La technique nous éloigne-t-elle de la nature ? \\
La technique nous éloigne-t-elle de la réalité ? \\
La technique nous libère-t-elle ? \\
La technique nous oppose-t-elle à la nature ? \\
La technique nous permet-elle de comprendre la nature ? \\
La technique peut-elle se déduire de la science ? \\
La technique peut-elle se passer de la science ? \\
La technique sert-elle nos désirs ? \\
La tentation \\
La théorie et la pratique \\
La théorie nous éloigne-t-elle de la réalité ? \\
La théorie scientifique \\
La tolérance \\
La tolérance est-elle une vertu ? \\
La tradition \\
La traduction \\
La transgression \\
L'attente \\
L'attention \\
L'attention caractérise-t-elle la conscience ? \\
La tyrannie des désirs \\
L'audace \\
L'au-delà \\
L'authenticité \\
L'autobiographie \\
L'automatisation \\
L'autonomie \\
L'autoportrait \\
L'autorité \\
L'autorité du droit \\
La valeur \\
La valeur de la vérité \\
La valeur morale de l'amour \\
La valeur morale d'une action se juge-t-elle à ses conséquences ? \\
La vengeance \\
L'avenir a-t-il une réalité ? \\
L'avenir peut-il être objet de connaissance ? \\
La vérification \\
La vérification expérimentale \\
La vérité \\
La vérité a-t-elle une histoire ? \\
La vérité donne-t-elle le droit d'être injuste ? \\
La vérité échappe-t-elle au temps ? \\
La vérité est-elle affaire de cohérence ? \\
La vérité est-elle contraignante ? \\
La vérité est-elle libératrice ? \\
La vérité est-elle une valeur ? \\
La vérité est-elle une ? \\
La vérité historique \\
La vérité peut-elle laisser indifférent ? \\
La vérité peut-elle se définir par le consensus ? \\
La vérité rend-elle heureux ? \\
La vertu \\
La vertu peut-elle être excessive ? \\
La vertu peut-elle s'enseigner ? \\
La vie de l'esprit \\
La vie de plaisirs \\
La vie en société impose-t-elle de n'être pas soi-même ? \\
La vie est-elle sacrée ? \\
La vie heureuse \\
La vie intérieure \\
La vie morale \\
La vie peut-elle être objet de science ? \\
La vie psychique \\
La vie sauvage \\
La vie sociale \\
La vie sociale est-elle toujours conflictuelle ? \\
La violence \\
La violence est-elle toujours destructrice ? \\
La violence peut-elle avoir raison ? \\
La violence peut-elle être gratuite ? \\
La violence verbale \\
La virtuosité \\
La vision peut-elle être le modèle de toute connaissance ? \\
La vocation \\
La voix de la raison \\
La volonté et le désir \\
La volonté peut-elle être générale ? \\
La volonté peut-elle nous manquer ? \\
La vue et le toucher \\
Le bavardage \\
Le beau est-il toujours moral ? \\
Le beau et l'agréable \\
Le beau et le bien \\
Le beau et le sublime \\
Le beau et l'utile \\
Le beau geste \\
Le bénéfice du doute \\
Le besoin de métaphysique est-il un besoin de connaissance ? \\
Le besoin de reconnaissance \\
Le besoin de théorie \\
Le besoin et le désir \\
Le bien commun est-il une illusion ? \\
Le bien commun et l'intérêt de tous \\
Le bien est-ce l'utile ? \\
Le bien est-il relatif ? \\
Le bien et le beau \\
Le bien et les biens \\
Le bien public \\
Le bonheur collectif \\
Le bonheur des sens \\
Le bonheur est-il au nombre de nos devoirs ? \\
Le bonheur est-il dans l'inconscience ? \\
Le bonheur est-il l'absence de maux ? \\
Le bonheur est-il la fin de la vie ? \\
Le bonheur est-il le bien suprême ? \\
Le bonheur est-il le but de la politique ? \\
Le bonheur est-il le prix de la vertu ? \\
Le bonheur est-il un but politique ? \\
Le bonheur est-il un droit ? \\
Le bonheur est-il une affaire privée ? \\
Le bonheur est-il un idéal ? \\
Le bonheur et la raison \\
Le bonheur et la technique \\
Le bonheur n'est-il qu'une idée ? \\
Le bonheur n'est-il qu'un idéal ? \\
Le bonheur peut-il être collectif ? \\
Le bonheur peut-il être le but de la politique ? \\
Le bonheur se calcule-t-il ? \\
Le bonheur se mérite-t-il ? \\
Le bricolage \\
Le calendrier \\
Le caractère \\
Le caractère sacré de la vie \\
Le cas de conscience \\
Le cerveau et la pensée \\
Le cerveau pense-t-il ? \\
L'échange constitue-t-il un lien social ? \\
L'échange économique fonde-t-il la société humaine \\
L'échange et l'usage \\
L'échange n'a-t-il de fondement qu'économique ? \\
L'échange ne porte-t-il que sur les choses ? \\
L'échange peut-il être désintéressé ? \\
Le châtiment \\
Le chef \\
Le chef d'œuvre \\
Le chef-d'œuvre \\
Le choix \\
Le choix et la liberté \\
Le citoyen \\
Le commencement \\
Le commerce \\
Le commerce adoucit-il les mœurs ? \\
Le commerce des idées \\
Le commerce unit-il les hommes ? \\
Le commun et le propre \\
Le concept \\
Le concept et l'exemple \\
Le conflit est-il une maladie sociale ? \\
L'économie et la politique \\
Le consentement \\
Le contentement \\
Le contrat \\
Le contrat de travail \\
Le contrat est-il au fondement de la politique ? \\
Le corps est-il négociable ? \\
Le corps et l'âme \\
Le corps et l'esprit \\
Le corps impose-t-il des perspectives ? \\
Le corps n'est-il qu'un mécanisme ? \\
Le corps obéit-il à l'esprit ? \\
Le corps politique \\
Le cosmopolitisme \\
Le courage \\
Le crime \\
L'écriture \\
Le dedans et le dehors \\
Le défaut \\
Le désespoir \\
Le désintéressement \\
Le désir a-t-il un objet ? \\
Le désir d'absolu \\
Le désir de l'autre \\
Le désir de savoir \\
Le désir de savoir est-il naturel ? \\
Le désir d'éternité \\
Le désir de vérité \\
Le désir du bonheur est-il universel ? \\
Le désir est-il aveugle ? \\
Le désir est-il ce qui nous fait vivre ? \\
Le désir est-il désir de l'autre ? \\
Le désir est-il le signe d'un manque ? \\
Le désir est-il l'essence de l'homme ? \\
Le désir est-il nécessairement l'expression d'un manque ? \\
Le désir est-il par nature illimité ? \\
Le désir et la culpabilité \\
Le désir et la loi \\
Le désir et le besoin \\
Le désir et le mal \\
Le désir et le manque \\
Le désir et le rêve \\
Le désir et le temps \\
Le désir et le travail \\
Le désir et l'interdit \\
Le désir n'est-il que manque ? \\
Le désir peut-il être désintéressé ? \\
Le désir peut-il ne pas avoir d'objet ? \\
Le désir peut-il nous rendre libre ? \\
Le despotisme \\
Le destin \\
Le devoir \\
Le devoir est-il l'expression de la contrainte sociale ? \\
Le devoir et le bonheur \\
Le devoir rend-il libre ? \\
Le devoir supprime-t-il la liberté ? \\
Le diable \\
Le dialogue \\
Le dialogue suffit-il à rompre la solitude ? \\
Le discernement \\
Le divertissement \\
Le don \\
Le don de soi \\
Le don est-il une modalité de l'échange ? \\
Le don et la dette \\
Le don et l'échange \\
Le doute \\
Le droit \\
Le droit à la différence met-il en péril l'égalité des droits ? \\
Le droit à la paresse \\
Le droit au bonheur \\
Le droit au travail \\
Le droit de mentir \\
Le droit de propriété \\
Le droit de résistance \\
Le droit divin \\
Le droit doit-il être indépendant de la morale ? \\
Le droit du plus faible \\
Le droit est-il facteur de paix ? \\
Le droit et la convention \\
Le droit et la force \\
Le droit et la liberté \\
Le droit et la loi \\
Le droit et la morale \\
Le droit et le devoir \\
Le droit n'est-il qu'une justice par défaut ? \\
Le droit peut-il échapper à l'histoire ? \\
Le droit peut-il être naturel ? \\
Le droit peut-il se passer de la morale ? \\
Le droit positif \\
Le droit sert-il à établir l'ordre ou la justice ? \\
L'éducation artistique \\
L'éducation esthétique \\
Le fait \\
Le fait divers \\
Le fait et l'événement \\
Le fantasme \\
L'efficacité \\
L'effort \\
L'effort moral \\
Le fini et l'infini \\
Le for intérieur \\
L'égalité \\
L'égalité est-elle toujours juste ? \\
Légalité et légitimité \\
Légalité et moralité \\
Le génie \\
Le génie est-il la marque de l'excellence artistique ? \\
Le génie et la règle \\
Le génie et le savant \\
Le geste et la parole \\
Le goût \\
Le goût de la liberté \\
Le goût des autres \\
Le goût s'éduque-t-il ? \\
Le gouvernement des meilleurs \\
Le hasard \\
Le hasard et la nécessité \\
Le hors-la-loi \\
Le je et le tu \\
Le jeu \\
Le jeu et le divertissement \\
Le jeu et le hasard \\
Le juge \\
Le jugement \\
Le jugement dernier \\
Le jugement moral \\
Le juste et le légal \\
Le langage \\
Le langage du corps \\
Le langage est-il d'essence poétique ? \\
Le langage est-il le lieu de la vérité ? \\
Le langage est-il logique ? \\
Le langage est-il une prise de possession des choses ? \\
Le langage est-il un instrument de connaissance ? \\
Le langage est-il un obstacle pour la pensée ? \\
Le langage et la pensée \\
Le langage masque-t-il la pensée ? \\
Le langage n'est-il qu'un instrument de communication ? \\
Le langage peut-il être un obstacle à la recherche de la vérité ? \\
Le langage rend-il l'homme plus puissant ? \\
Le langage traduit-il la pensée ? \\
Le langage trahit-il la pensée ? \\
L'élection \\
Le législateur \\
Le libre-arbitre \\
Le libre échange \\
Le lien social \\
Le livre de la nature \\
Le loisir \\
Le luxe \\
Le maître \\
Le mal \\
Le malentendu \\
Le mal être \\
Le mal existe-t-il ? \\
Le malheur \\
Le malheur est-il injuste ? \\
L'émancipation \\
Le marché \\
Le marché du travail \\
Le mariage \\
Le matérialisme \\
Le médiat et l'immédiat \\
Le meilleur est-il l'ennemi du bien ? \\
Le mensonge \\
Le mensonge peut-il être au service de la vérité ? \\
Le mépris \\
Le mérite \\
Le métier \\
Le métier de politique \\
Le mien et le tien \\
Le moi \\
Le moi est-il haïssable ? \\
Le moi est-il une fiction ? \\
Le moi est-il une illusion ? \\
Le moi et la conscience \\
Le moindre mal \\
Le moi n'est-il qu'une fiction ? \\
Le monde du travail \\
Le monde se réduit-il à ce que nous en voyons ? \\
Le monstre \\
Le monstrueux \\
Le mot et le geste \\
L'émotion \\
Le mouvement \\
Le multiple et l'un \\
Le musée \\
Le mystère \\
Le naturel et le fabriqué \\
L'encyclopédie \\
L'enfance \\
L'enfance est-elle en nous ce qui doit être abandonné ? \\
L'enfant \\
L'engagement \\
L'ennemi \\
L'ennui \\
Le non-être \\
L'enquête empirique rend-elle la métaphysique inutile ? \\
L'entendement et la volonté \\
L'envie \\
Le pardon \\
Le pardon et l'oubli \\
Le passé a-t-il plus de réalité que l'avenir ? \\
Le passé détermine-t-il notre présent ? \\
Le passé est-il ce qui a disparu ? \\
Le passé et le présent \\
Le passé existe-t-il ? \\
Le paysage \\
Le personnage et la personne \\
Le pessimisme \\
Le peuple \\
Le peuple et la nation \\
Le peuple peut-il se tromper ? \\
L'éphémère \\
Le phénomène \\
Le plaisir \\
Le plaisir de parler \\
Le plaisir des sens \\
Le plaisir esthétique \\
Le plaisir esthétique peut-il se partager ? \\
Le plaisir est-il tout le bonheur ? \\
Le plaisir et la joie \\
Le plaisir et la peine \\
Le plaisir peut-il être partagé ? \\
Le plaisir suffit-il au bonheur ? \\
Le poids de la société \\
Le poids du passé \\
Le possible et le réel \\
Le pouvoir \\
Le pouvoir corrompt-il toujours ? \\
Le pouvoir de l'État est-il arbitraire ? \\
Le pouvoir des mots \\
Le pouvoir et l'autorité \\
Le présent \\
L'épreuve du réel \\
Le principe \\
Le principe de raison suffisante \\
Le privé et le public \\
Le prix du travail \\
Le probable \\
Le profit \\
Le progrès \\
Le progrès est-il un mythe ? \\
Le progrès moral \\
Le progrès technique peut-il être aliénant ? \\
Le projet \\
Le propre du vivant est-il de tomber malade ? \\
Le provisoire \\
Le public et le privé \\
L e pur et l'impur \\
L'équité \\
L'équivocité \\
L'équivoque \\
Le quotidien \\
Le raffinement \\
Le rationnel et le raisonnable \\
Le rationnel et l'irrationnel \\
Le réalisme \\
Le récit historique \\
Le reconnaissance \\
Le réel \\
Le réel est-il ce que l'on croit ? \\
Le réel est-il ce que nous expérimentons ? \\
Le réel est-il ce que nous percevons ? \\
Le réel est-il ce qui apparaît ? \\
Le réel est-il ce qui est perçu ? \\
Le réel est-il inaccessible ? \\
Le réel est-il l'objet de la science ? \\
Le réel est-il objet d'interprétation ? \\
Le réel est-il rationnel ? \\
Le réel et la fiction \\
Le réel et le matériel \\
Le réel et le possible \\
Le réel et le virtuel \\
Le réel et le vrai \\
Le réel et l'imaginaire \\
Le réel et l'irréel \\
Le réel n'est-il qu'un ensemble de contraintes ? \\
Le réel obéit-il à la raison ? \\
Le réel résiste-t-il à la connaissance ? \\
Le réel se limite-t-il à ce que font connaître les théories scientifiques ? \\
Le réel se limite-t-il à ce que nous percevons ? \\
Le réel se réduit-il à ce que l'on perçoit ? \\
Le réel se réduit-il à l'objectivité ? \\
Le regard \\
Le relativisme \\
Le renoncement \\
Le respect \\
Le ressentiment \\
Le rien \\
Le risque \\
Le risque de la liberté \\
Le rôle de l'État est-il de faire régner la justice ? \\
Le rôle de l'État est-il de préserver la liberté de l'individu ? \\
Le rôle de l'historien est-il de juger ? \\
Le rôle des théories est-il d'expliquer ou de décrire ? \\
L'erreur \\
L'erreur et la faute \\
L'erreur et l'illusion \\
Le rythme \\
Le sacré \\
Le sacré et le profane \\
Le sacrifice \\
Les acteurs de l'histoire en sont-ils les auteurs ? \\
Les affects sont-ils déraisonnables ? \\
Le sage a-t-il besoin d'autrui ? \\
Les âges de la vie \\
Le salaire \\
Le salut vient-il de la raison ? \\
Les animaux ont-ils des droits ? \\
Les apparences sont-elles toujours trompeuses ? \\
Le sauvage \\
Le savant et l'ignorant \\
Le savoir exclut-il toute forme de croyance ? \\
Le savoir-faire \\
Le savoir rend-il libre ? \\
Les bêtes travaillent-elles ? \\
Les catégories \\
Les causes et les signes \\
Les classes sociales \\
L'esclavage des passions \\
Les coïncidences ont-elles des causes ? \\
Les commencements \\
Les conditions d'existence \\
Les conflits menacent-ils la société ? \\
Les considérations morales ont-elles leur place en politique ? \\
Les désirs ont-ils nécessairement un objet \\
Les devoirs de l'homme varient-ils selon la culture ? \\
Les devoirs de l'homme varient-ils selon les cultures ? \\
Les devoirs du citoyen \\
Les droits de l'individu \\
Les échanges \\
Les échanges économiques sont-ils facteurs de paix ? \\
Les échanges favorisent-ils la paix ? \\
Les échanges sont-ils facteurs de paix ? \\
Le secret \\
Le sens caché \\
Le sens commun \\
Le sens du devoir \\
Le sensible et l'intelligible \\
Le sensible peut-il être connu ? \\
Le sentiment \\
Le sentiment de liberté \\
Le sentiment d'injustice \\
Le sentiment d'injustice est-il naturel ? \\
Le sentiment du juste et de l'injuste \\
Les êtres vivants sont-ils des machines ? \\
Les événements historiques sont-ils de nature imprévisible ? \\
Les faits et les valeurs \\
Les faits existent-ils indépendamment de leur établissement par l'esprit humain ? \\
Les faits parlent-ils d'eux-mêmes ? \\
Les faits peuvent-ils faire autorité ? \\
Les fins de la culture \\
Les formes du vivant \\
Les générations \\
Les habitudes nous forment-elles ? \\
Les hommes naissent-ils libres ? \\
Les hommes ont-ils besoin de maîtres ? \\
Les hommes savent-ils ce qu'ils désirent ? \\
Les hommes sont-ils seulement le produit de leur culture ? \\
Les idées et les choses \\
Les idées ont-elles une existence éternelle ? \\
Le signe \\
Le silence \\
Le silence a-t-il un sens ? \\
Le simple \\
Le simple et le complexe \\
Les inégalités menacent-elles la société ? \\
Les inégalités sociales sont-elles naturelles ? \\
Les intentions et les actes \\
Les leçons de l'histoire \\
Les limites de la connaissance \\
Les limites de la raison \\
Les limites de l'expérience \\
Les lois \\
Les machines nous rendent-elles libres ? \\
Les maladies de l'âme \\
Les mathématiques parlent-elles du réel ? \\
Les mathématiques sont-elles un instrument ? \\
Les mœurs \\
Les monstres \\
Les mots disent-ils les choses ? \\
Les mots et les concepts \\
Les mots expriment-ils les choses ? \\
Les mots parviennent-ils à tout exprimer ? \\
Les mots sont-ils trompeurs ? \\
Les moyens et les fins \\
Les œuvres d'art sont-elles des réalités comme les autres ? \\
Le soi et le je \\
Le solipsisme \\
Le sommeil \\
Le souci de soi \\
Les outils \\
Le souverain bien \\
L'espace nous sépare-t-il ? \\
Les paroles et les actes \\
L'espérance \\
Les personnages de fiction peuvent-ils avoir une réalité ? \\
Les peuples font-ils l'histoire ? \\
Les preuves de la liberté \\
Les principes \\
Les principes de la morale dépendent-ils de la culture ? \\
L'esprit \\
L'esprit critique \\
L'esprit dépend-il du corps ? \\
L'esprit de système \\
L'esprit domine-t-il la matière ? \\
L'esprit est-il mieux connu que le corps ? \\
L'esprit est-il objet de science ? \\
L'esprit est-il plus difficile à connaître que la matière ? \\
L'esprit est-il une partie du corps ? \\
L'esprit humain progresse-t-il ? \\
Les progrès de la technique sont-ils nécessairement des progrès de la raison ? \\
Les progrès techniques constituent-ils des progrès de la civilisation ? \\
Les raisons de croire \\
Les religions peuvent-elles être objets de science ? \\
Les scélérats peuvent-ils être heureux ? \\
Les sciences décrivent-elles le réel ? \\
Les sciences permettent-elles de connaître la réalité-même ? \\
L'essence et l'existence \\
Les sens jugent-ils ? \\
Les sens sont-ils source d'illusion ? \\
Les sens sont-ils trompeurs ? \\
L'estime de soi \\
Le style \\
Le sublime \\
Le sujet \\
Le sujet et l'individu \\
Le sujet n'est-il qu'une fiction ? \\
Le sujet peut-il s'aliéner par un libre choix ? \\
Les valeurs morales ont-elles leur origine dans la raison ? \\
Les valeurs universelles \\
Les vérités empiriques \\
Les vérités éternelles \\
Les vérités sont-elles intemporelles ? \\
Les vertus du commerce \\
Les vivants peuvent-ils se passer des morts ? \\
Le système \\
Le système des arts \\
Le tact \\
Le talent \\
L'État \\
L'État a-t-il des intérêts propres ? \\
L'État a-t-il pour but de maintenir l'ordre ? \\
L'État a-t-il tous les droits ? \\
L'État contribue-t-il à pacifier les relations entre les hommes ? \\
L'État de droit \\
L'état de nature \\
L'État doit-il éduquer le peuple ? \\
L'État doit-il être fort ? \\
L'État doit-il être le plus fort ? \\
L'État doit-il être sans pitié ? \\
L'État doit-il préférer l'injustice au désordre ? \\
L'État doit-il reconnaître des limites à sa puissance ? \\
L'État doit-il se mêler de religion ? \\
L'État doit-il se préoccuper des arts ? \\
L'État doit-il se préoccuper du bonheur des citoyens ? \\
L'État doit-il se soucier de la morale ? \\
L'État doit-il veiller au bonheur des individus  ? \\
L'État est-il au service de la société ? \\
L'État est-il l'ennemi de la liberté ? \\
L'État est-il l'ennemi de l'individu ? \\
L'État est-il libérateur ? \\
L'État est-il toujours juste ? \\
L'État est-il un mal nécessaire ? \\
L'État est-il un tiers impartial ? \\
L'État est-il un « monstre froid » ? \\
L'État et la justice \\
L'État et la nation \\
L'État et la société \\
L'État et le droit \\
L'État et le peuple \\
L'État et les communautés \\
L'État et l'individu \\
L'État n'est-il qu'un instrument de domination ? \\
L'État nous rend-il meilleurs ? \\
L'État peut-il être impartial ? \\
L'État peut-il poursuivre une autre fin que sa propre puissance ? \\
L'État peut-il renoncer à la violence ? \\
Le technicien n'est-il qu'un exécutant ? \\
Le témoignage \\
Le temps \\
Le temps détruit-il tout ? \\
Le temps du bonheur \\
Le temps est-il destructeur ? \\
Le temps est-il en nous ou hors de nous ? \\
Le temps est-il la marque de notre impuissance ? \\
Le temps est-il notre allié ? \\
Le temps est-il une contrainte ? \\
Le temps est-il une réalité ? \\
Le temps et l'espace \\
Le temps libre \\
Le temps n'est-il pour l'homme que ce qui le limite ? \\
Le temps nous appartient-il ? \\
Le temps nous est-il compté ? \\
Le temps passe-t-il ? \\
Le temps perdu \\
L'éternel retour \\
L'éternité \\
L'étonnement \\
Le toucher \\
Le travail \\
Le travail a-t-il une valeur morale ? \\
Le travail est-il le propre de l'homme ? \\
Le travail est-il libérateur ? \\
Le travail est-il nécessaire au bonheur ? \\
Le travail est-il toujours une activité productrice ? \\
Le travail est-il un besoin ? \\
Le travail est-il une marchandise ? \\
Le travail est-il une valeur ? \\
Le travail est-il un rapport naturel de l'homme à la nature ? \\
Le travail et la propriété \\
Le travail et la technique \\
Le travail et le temps \\
Le travail fait-il de l'homme un être moral ? \\
Le travail fonde-t-il la propriété ? \\
Le travaille libère-t-il ? \\
Le travail manuel \\
Le travail manuel est-il sans pensée ? \\
Le travail unit-il ou sépare-t-il les hommes ? \\
L'être et le néant \\
L'être humain est-il la mesure de toute chose ? \\
L'être humain est-il par nature un être religieux \\
L'être imaginaire et l'être de raison \\
Le tribunal de l'histoire \\
Le troc \\
L'étude de l'histoire conduit-elle à désespérer l'homme ? \\
Le tyran \\
L'événement \\
L'événement historique a-t-il un sens par lui-même ? \\
Le vertige de la liberté \\
Le vice et la vertu \\
Le vide et le plein \\
L'évidence \\
L'évidence et la démonstration \\
L'évidence se passe-t-elle de démonstration ? \\
Le visible et l'invisible \\
Le vivant \\
Le vivant est-il entièrement connaissable ? \\
Le vivant est-il entièrement explicable ? \\
Le vivant est-il réductible au physico-chimique ? \\
Le vivant est-il un objet de science comme un autre ? \\
Le vivant et la machine \\
Le vivant et la mort \\
Le vivant et la sensibilité \\
Le vivant et la technique \\
Le vivant et le vécu \\
Le vivant et l'expérimentation \\
Le vivant et l'inerte \\
Le vivant n'est-il que matière ? \\
Le vivant n'est-il qu'une machine ingénieuse ? \\
Le vivant obéit-il à des lois ? \\
Le vivant obéit-il à une nécessité ? \\
Le volontaire et l'involontaire \\
Le voyage \\
Le vrai a-t-il une histoire ? \\
Le vrai et le bien \\
Le vrai et le vraisemblable \\
Le vrai se réduit-il à ce qui est vérifiable ? \\
L'exactitude \\
L'excellence des sens \\
L'exception \\
L'excès \\
L'excuse \\
L'existence \\
L'existence a-t-elle un sens ? \\
L'existence du mal met-elle en échec la raison ? \\
L'existence du passé \\
L'existence est-elle vaine ? \\
L'existence et le temps \\
L'existence se laisse-t-elle penser ? \\
L'expérience a-t-elle le même sens dans toutes les sciences ? \\
L'expérience d'autrui nous est-elle utile ? \\
L'expérience de l'injustice \\
L'expérience démontre-t-elle quelque chose ? \\
L'expérience de pensée \\
L'expérience et la sensation \\
L'expérience imaginaire \\
L'expérience instruit-elle ? \\
L'expérience morale \\
L'expérience peut-elle avoir raison des principes ? \\
L'expérience peut-elle contredire la théorie ? \\
L'expérience rend-elle raisonnable ? \\
L'expérience rend-elle responsable ? \\
L'expérience suffit-elle pour établir une vérité ? \\
L'expression \\
L'extinction du désir \\
L'habitude \\
L'habitude est-elle notre guide dans la vie ? \\
L'harmonie \\
L'hérédité \\
L'héritage \\
L'histoire a-t-elle un commencement et une fin ? \\
L'histoire a-t-elle une fin ? \\
L'histoire a-t-elle un sens ? \\
L'histoire des sciences \\
L'histoire du droit est-elle celle du progrès de la justice ? \\
L'histoire est-elle la connaissance du passé humain ? \\
L'histoire est-elle la mémoire de l'humanité ? \\
L'histoire est-elle la science du passé ? \\
L'histoire est-elle le récit objectif des faits passés  ? \\
L'histoire est-elle le théâtre des passions ? \\
L'histoire est-elle rationnelle ? \\
L'histoire est-elle une explication ou une justification du passé ? \\
L'histoire est-elle une science comme les autres ? \\
L'histoire est-elle une science ? \\
L'histoire jugera-t-elle ? \\
L'histoire n'a-t-elle pour objet que le passé ? \\
L'histoire n'est-elle que la connaissance du passé ? \\
L'histoire nous appartient-elle ? \\
L'histoire obéit-elle à des lois ? \\
L'histoire peut-elle être contemporaine ? \\
L'histoire se répète-t-elle ? \\
L'historien \\
L'historien peut-il être impartial ? \\
L'homme aime-t-il la justice pour elle-même ? \\
L'homme a-t-il besoin de l'art ? \\
L'homme a-t-il une place dans la nature ? \\
L'homme des droits de l'homme \\
L'homme d'État \\
L'homme est-il chez lui dans l'univers ? \\
L'homme est-il l'artisan de sa dignité ? \\
L'homme est-il le seul être à avoir une histoire ? \\
L'homme est-il le sujet de son histoire ? \\
L'homme est-il un animal comme un autre ? \\
L'homme est-il un animal dénaturé ? \\
L'homme est-il un animal politique ? \\
L'homme est-il un animal rationnel ? \\
L'homme est-il un animal social ? \\
L'homme est-il un animal ? \\
L'homme est-il un corps pensant ? \\
L'homme est-il un loup pour l'homme ? \\
L'homme et la machine \\
L'homme et l'animal \\
L'homme et le citoyen \\
L'homme injuste peut-il être heureux ? \\
L'homme se réalise-t-il dans le travail ? \\
L'honneur \\
L'honneur ? \\
L'hospitalité \\
L'humanité \\
L'humanité est-elle aimable ? \\
L'humilité \\
L'hypothèse \\
Liberté d'agir, liberté de penser \\
Liberté et courage \\
Liberté et déterminisme \\
Liberté et éducation \\
Liberté et égalité \\
Liberté et engagement \\
Liberté et existence \\
Liberté et indépendance \\
Liberté et libération \\
Liberté et licence \\
Liberté et nécessité \\
Liberté et pouvoir \\
Liberté et responsabilité \\
Liberté et savoir \\
Liberté et sécurité \\
Liberté et solitude \\
Libre arbitre et déterminisme sont-ils compatibles ? \\
Libre et heureux \\
L'idéal \\
L'idéal et le réel \\
L'idée de bonheur collectif a-t-elle un sens ? \\
L'idée de progrès \\
L'idée d'organisme \\
L'identité \\
L'identité personnelle \\
L'idéologie \\
L'idiot \\
L'ignorance \\
L'ignorance est-elle préférable à l'erreur ? \\
L'ignorance peut-elle être une excuse ? \\
L'illusion \\
L'imagination dans les sciences \\
L'imagination enrichit-elle la connaissance ? \\
L'imitation \\
L'immatériel \\
L'immédiat \\
L'immortalité \\
L'immortalité de l'âme \\
L'impartialité \\
L'impensable \\
L'imperceptible \\
L'impossible \\
L'imprévisible \\
L'inaperçu \\
L'inattendu \\
L'incertitude \\
L'incertitude interdit-elle de raisonner ? \\
L'inconscience \\
L'inconscient \\
L'inconscient est-il dans l'âme ou dans le corps ? \\
L'inconscient est-il une excuse ? \\
L'inconscient et l'involontaire \\
L'inconscient et l'oubli \\
L'inconscient n'est-il qu'une hypothèse ? \\
L'inconscient peut-il se manifester ? \\
L'indécidable \\
L'indéfini \\
L'indémontrable \\
L'indescriptible \\
L'indésirable \\
L'indice et la preuve \\
L'indicible \\
L'indicible et l'impensable \\
L'indicible et l'ineffable \\
L'indifférence \\
L'indifférence peut-elle être une vertu ? \\
L'indignation \\
L'indignité \\
L'individu a-t-il des droits ? \\
L'individu et l'espèce \\
L'induction \\
L'indulgence \\
L'ineffable et l'innommable \\
L'inestimable \\
L'inexistant \\
L'inexpérience \\
L'infini et l'indéfini \\
L'infinité de l'univers a-t-elle de quoi nous effrayer ? \\
L'ingratitude \\
L'inhumain \\
L'inimaginable \\
L'injustifiable \\
L'innocence \\
L'innovation \\
L'inquiétude \\
L'inquiétude peut-elle définir l'existence humaine ? \\
L'inquiétude peut-elle devenir l'existence humaine ? \\
L'insatisfaction \\
L'insouciance \\
L'instant \\
L'instant et la durée \\
L'instruction est-elle facteur de moralité ? \\
L'instrument \\
L'instrument et la machine \\
L'intellect \\
L'intelligence \\
L'intelligence artificielle \\
L'intelligence de la technique \\
L'intemporel \\
L'interdit est-il au fondement de la culture ? \\
L'intérêt constitue-t-il l'unique lien social ? \\
L'intérêt de la société l'emporte-t-il sur celui des individus ? \\
L'intérêt de l'État \\
L'intérêt est-il le principe de tout échange ? \\
L'intérêt général est-il la somme des intérêts particuliers ? \\
L'intériorité \\
L'interprétation \\
L'interprétation est-elle un art ? \\
L'interprétation est-elle une activité sans fin ? \\
L'interprète et le créateur \\
L'interprète sait-il ce qu'il cherche ? \\
L'intersubjectivité \\
L'intimité \\
L'intolérable \\
L'introspection \\
L'intuition \\
L'intuition intellectuelle \\
L'inutile \\
L'inutile est-il sans valeur ? \\
L'invention \\
L'invention et la découverte \\
L'invention technique \\
L'invisible \\
L'involontaire \\
Lire et écrire \\
L'irrationnel \\
L'irrationnel est-il pensable ? \\
L'irrationnel est-il toujours absurde ? \\
L'irrationnel existe-t-il ? \\
L'irréfléchi \\
L'irréfutable \\
L'irréparable \\
L'irrésolution \\
L'irresponsabilité \\
L'irréversibilité \\
L'obéissance \\
L'obéissance est-elle compatible avec la liberté ? \\
L'objectivité \\
L'objectivité de l'historien \\
L'objectivité historique est-elle synonyme de neutralité ? \\
L'objet et la chose \\
L'obligation \\
L'obscur \\
L'observation \\
L'occasion \\
L'œuvre \\
L'œuvre d'art a-t-elle un sens ? \\
L'œuvre d'art doit-elle être belle ? \\
L'œuvre d'art donne-t-elle à penser ? \\
L'œuvre d'art échappe-t-elle au temps ? \\
L'œuvre d'art échappe-t-elle nécessairement au temps ? \\
L'œuvre d'art est-elle une marchandise ? \\
L'œuvre d'art est-elle un objet d'échange ? \\
L'œuvre d'art est-elle un symbole ? \\
L'œuvre d'art instruit-elle ? \\
L'œuvre d'art nous apprend-elle quelque chose ? \\
Loisir et oisiveté \\
L'oisiveté \\
L'omniscience \\
L'opinion a-t-elle nécessairement tort ? \\
L'opinion est-elle un savoir ? \\
L'ordre des choses \\
L'ordre du monde \\
L'ordre du vivant est-il façonné par le hasard ? \\
L'ordre et le désordre \\
L'ordre social \\
L'ordre social peut-il être juste ? \\
L'organique \\
L'organique et l'inorganique \\
L'organisme \\
L'originalité \\
L'origine des idées \\
L'oubli \\
L'oubli et le pardon \\
L'outil \\
L'outil et la machine \\
L'ouverture d'esprit \\
L'unanimité est-elle un critère de vérité ? \\
L'unité de l'État \\
L'universel \\
L'universel et le particulier \\
L'urbanité \\
L'urgence \\
L'utile et l'agréable \\
L'utile et le beau \\
L'utile et l'inutile \\
L'utilité \\
L'utopie et l'idéologie \\
Machine et organisme \\
Maître et disciple \\
Maîtrise et puissance \\
Mal et liberté \\
Ma liberté s'arrête-t-elle où commence celle des autres ? \\
Mémoire et souvenir \\
Mentir \\
Modèle et copie \\
Mon corps \\
Mon corps est-il naturel ? \\
Mon corps fait-il obstacle à ma liberté ? \\
Mon prochain est-il mon semblable ? \\
Montrer et démontrer \\
Morale et calcul \\
Morale et économie \\
Morale et liberté \\
Moralité et utilité \\
Naît-on sujet ou le devient-on ? \\
N'apprend-on que par l'expérience ? \\
Narration et identité \\
Nature et artifice \\
Nature et convention \\
Nature et histoire \\
Nature et loi \\
Nature et morale \\
N'échange-t-on que ce qui a de la valeur ? \\
N'échange-t-on que par intérêt ? \\
Ne désire-t-on que ce dont on manque ? \\
Ne désirons-nous que ce qui est bon pour nous ? \\
Ne désirons-nous que les choses que nous estimons bonnes ? \\
Ne faire que son devoir \\
Ne rien devoir à personne \\
Ne veut-on que ce qui est désirable ? \\
Ne vit-on bien qu'avec ses amis ? \\
N'interprète-t-on que ce qui est équivoque ? \\
Nommer \\
Normes et valeurs \\
Nos convictions morales sont-elles le simple reflet de notre temps ? \\
Nos désirs nous appartiennent-ils ? \\
Nos désirs nous opposent-ils ? \\
Nos pensées dépendent-elles de nous ? \\
Nos pensées sont-elles entièrement en notre pouvoir ? \\
Notre liberté de pensée a-t-elle des limites ? \\
Notre rapport au monde est-il essentiellement technique ? \\
Nouveauté et tradition \\
N'y a-t-il de bonheur qu'éphémère ? \\
N'y a-t-il de devoirs qu'envers autrui ? \\
N'y a-t-il de droit qu'écrit ? \\
N'y a-t-il de foi que religieuse ? \\
N'y a-t-il de rationalité que scientifique ? \\
N'y a-t-il de réalité que de l'individuel ? \\
N'y a-t-il de savoir que livresque ? \\
N'y a-t-il de science que de ce qui est mathématisable ? \\
N'y a-t-il de vérité que vérifiable ? \\
N'y a-t-il de vérités que scientifiques ? \\
N'y a-t-il de vrai que le vérifiable ? \\
Obéissance et liberté \\
Obéissance et soumission \\
Objectivé et subjectivité \\
Observation et expérience \\
Observer et comprendre \\
Observer et expérimenter \\
Observer et interpréter \\
Opinion et ignorance \\
Ordre et désordre \\
Ordre et justice \\
Ordre et liberté \\
Organisme et milieu \\
Origine et fondement \\
Où commence la violence ? \\
Où commence ma liberté ? \\
Où est l'esprit ? \\
Outil et machine \\
Outil et organe \\
Paraître \\
Parier \\
Par le langage, peut-on agir sur la réalité ? \\
Parler, est-ce agir ? \\
Parler, est-ce communiquer ? \\
Parler, est-ce donner sa parole ? \\
Parler et agir \\
Parler, n'est-ce que désigner ? \\
Parler pour ne rien dire \\
Parole et pouvoir \\
Passions et intérêts \\
Penser, est-ce calculer ? \\
Penser, est-ce désobéir ? \\
Penser, est-ce se parler à soi-même ? \\
Penser et imaginer \\
Penser et parler \\
Penser et savoir \\
Penser et sentir \\
Penser l'avenir \\
Penser le changement \\
Penser par soi-même \\
Penser par soi-même, est-ce être l'auteur de ses pensées ? \\
Penser peut-il nous rendre heureux ? \\
Pense-t-on jamais seul ? \\
Perception et connaissance \\
Perception et imagination \\
Perception et sensation \\
Percevoir \\
Percevoir, est-ce interpréter ? \\
Percevoir, est-ce savoir ? \\
Percevoir, est-ce s'ouvrir au monde ? \\
Percevoir et concevoir \\
Percevoir et imaginer \\
Perçoit-on le réel tel qu'il est ? \\
Perçoit-on le réel ? \\
Perçoit-on les choses comme elles sont ? \\
Permanence et identité \\
Personne et individu \\
Peuple et multitude \\
Peut-il y avoir conflit entre nos devoirs ? \\
Peut-il y avoir des échanges équitables ? \\
Peut-il y avoir des lois de l'histoire ? \\
Peut-il y avoir des lois injustes ? \\
Peut-il y avoir des modèles en morale ? \\
Peut-il y avoir des vérités partielles ? \\
Peut-il y avoir esprit sans corps ? \\
Peut-il y avoir savoir-faire sans savoir ? \\
Peut-il y avoir une société sans État ? \\
Peut-il y avoir un État mondial ? \\
Peut-il y avoir une vérité en art ? \\
Peut-il y avoir un langage universel ? \\
Peut-on aimer l'autre tel qu'il est ? \\
Peut-on aimer sans perdre sa liberté ? \\
Peut-on aimer son prochain comme soi-même ? \\
Peut-on aimer une œuvre d'art sans la comprendre ? \\
Peut-on apprendre à mourir ? \\
Peut-on assimiler le vivant à une machine ? \\
Peut-on atteindre une certitude ? \\
Peut-on attribuer à chacun son dû ? \\
Peut-on avoir de bonnes raisons de ne pas dire la vérité ? \\
Peut-on avoir raison contre les faits ? \\
Peut-on avoir raison contre tout le monde ? \\
Peut-on avoir raisons contre les faits ? \\
Peut-on avoir raison tout seul ? \\
Peut-on cesser de croire ? \\
Peut-on cesser de désirer ? \\
Peut-on changer le cours de l'histoire ? \\
Peut-on changer le monde ? \\
Peut-on changer ses désirs ? \\
Peut-on choisir ses désirs ? \\
Peut-on commander à la nature ? \\
Peut-on communiquer ses perceptions à autrui ? \\
Peut-on communiquer son expérience ? \\
Peut-on comparer les cultures ? \\
Peut-on comparer l'organisme à une machine ? \\
Peut-on comprendre le présent ? \\
Peut-on comprendre un acte que l'on désapprouve ? \\
Peut-on concevoir une humanité sans art ? \\
Peut-on concevoir une science sans expérience ? \\
Peut-on concevoir une société juste sans que les hommes ne le soient ? \\
Peut-on concevoir une société sans État ? \\
Peut-on concilier bonheur et liberté ? \\
Peut-on connaître les choses telles qu'elles sont ? \\
Peut-on connaître l'esprit ? \\
Peut-on connaître le vivant sans le dénaturer ? \\
Peut-on connaître le vivant sans recourir à la notion de finalité ? \\
Peut-on connaître l'individuel ? \\
Peut-on connaître par intuition ? \\
Peut-on contredire l'expérience ? \\
Peut-on craindre la liberté ? \\
Peut-on critiquer la démocratie ? \\
Peut-on croire en rien ? \\
Peut-on décider d'être heureux ? \\
Peut-on définir la morale comme l'art d'être heureux ? \\
Peut-on définir le bonheur ? \\
Peut-on délimiter le réel ? \\
Peut-on dépasser la subjectivité ? \\
Peut-on désirer ce qui est ? \\
Peut-on désirer l'absolu ? \\
Peut-on désirer l'impossible ? \\
Peut-on désirer sans souffrir ? \\
Peut-on désobéir à l'État ? \\
Peut-on désobéir aux lois ? \\
Peut-on désobéir par devoir ? \\
Peut-on dire ce que l'on pense ? \\
Peut-on dire d'un homme qu'il est supérieur à un autre homme ? \\
Peut-on dire la vérité ? \\
Peut-on dire le singulier ? \\
Peut-on dire que les hommes font l'histoire ? \\
Peut-on dire que les machines travaillent pour nous ? \\
Peut-on dire que les mots pensent pour nous ? \\
Peut-on dire que l'humanité progresse ? \\
Peut-on dire que toutes les croyances se valent ? \\
Peut-on discuter des goûts et des couleurs ? \\
Peut-on distinguer entre de bons et de mauvais désirs ? \\
Peut-on distinguer entre les bons et les mauvais désirs ? \\
Peut-on donner un sens à son existence ? \\
Peut-on douter de sa propre existence ? \\
Peut-on douter de soi ? \\
Peut-on douter de toute vérité ? \\
Peut-on douter de tout ? \\
Peut-on échanger des idées ? \\
Peut-on échapper à ses désirs ? \\
Peut-on échapper à son temps ? \\
Peut-on échapper au cours de l'histoire ? \\
Peut-on échapper au temps ? \\
Peut-on éduquer la conscience ? \\
Peut-on en appeler à la conscience contre l'État ? \\
Peut-on être apolitique ? \\
Peut-on être dans le présent ? \\
Peut-on être en conflit avec soi-même ? \\
Peut-on être esclave de soi-même ? \\
Peut-on être heureux dans la solitude ? \\
Peut-on être heureux sans être sage ? \\
Peut-on être heureux sans philosophie ? \\
Peut-on être heureux sans s'en rendre compte ? \\
Peut-on être ignorant ? \\
Peut-on être indifférent à l'histoire ? \\
Peut-on être juste sans être impartial ? \\
Peut-on être méchant volontairement ? \\
Peut-on être obligé d'aimer ? \\
Peut-on être plus ou moins libre ? \\
Peut-on être sûr d'avoir raison ? \\
Peut-on être sûr de bien agir ? \\
Peut-on être sûr de ne pas se tromper ? \\
Peut-on être trop sensible ? \\
Peut-on étudier le passé de façon objective ? \\
Peut-on expérimenter sur le vivant ? \\
Peut-on expliquer le vivant ? \\
Peut-on expliquer une œuvre d'art ? \\
Peut-on faire de la politique sans supposer les hommes méchants ? \\
Peut-on faire de l'esprit un objet de science ? \\
Peut-on faire la philosophie de l'histoire ? \\
Peut-on faire le bien d'autrui malgré lui ? \\
Peut-on faire le mal innocemment ? \\
Peut-on faire l'expérience de la nécessité ? \\
Peut-on faire table rase du passé ? \\
Peut-on fonder la morale ? \\
Peut-on fonder le droit sur la morale ? \\
Peut-on fonder un droit de désobéir ? \\
Peut-on fonder une éthique sur la biologie ? \\
Peut-on fonder une morale sur le plaisir ? \\
Peut-on fuir la société ? \\
Peut-on haïr la raison ? \\
Peut-on haïr la vie ? \\
Peut-on haïr les images ? \\
Peut-on hiérarchiser les arts ? \\
Peut-on identifier le désir au besoin ? \\
Peut-on ignorer sa propre liberté ? \\
Peut-on ignorer volontairement la vérité ? \\
Peut-on imaginer l'avenir ? \\
Peut-on justifier le mal ? \\
Peut-on maîtriser la nature ? \\
Peut-on maîtriser le temps ? \\
Peut-on maîtriser l'évolution de la technique ? \\
Peut-on maîtriser ses désirs ? \\
Peut-on manquer de volonté ? \\
Peut-on mentir par humanité ? \\
Peut-on mesurer le temps ? \\
Peut-on moraliser la guerre ? \\
Peut-on ne croire en rien ? \\
Peut-on ne pas connaître son bonheur ? \\
Peut-on ne pas croire au progrès ? \\
Peut-on ne pas croire ? \\
Peut-on ne pas être égoïste ? \\
Peut-on ne pas être soi-même ? \\
Peut-on ne pas savoir ce que l'on dit ? \\
Peut-on ne pas savoir ce que l'on fait ? \\
Peut-on ne penser à rien ? \\
Peut-on nier le réel ? \\
Peut-on nier l'évidence ? \\
Peut-on nier l'existence de la matière ? \\
Peut-on opposer le loisir au travail ? \\
Peut-on ôter à l'homme sa liberté ? \\
Peut-on parler de dialogue des cultures ? \\
Peut-on parler de mondes imaginaires ? \\
Peut-on parler de nourriture spirituelle ? \\
Peut-on parler de problèmes techniques ? \\
Peut-on parler des miracles de la technique ? \\
Peut-on parler de travail intellectuel ? \\
Peut-on parler de vérité subjective ? \\
Peut-on parler de violence d'État ? \\
Peut-on parler d'une morale collective ? \\
Peut-on parler d'une religion de l'humanité ? \\
Peut-on parler d'un progrès dans l'histoire ? \\
Peut-on parler d'un progrès de la liberté ? \\
Peut-on parler pour ne rien dire ? \\
Peut-on penser ce qu'on ne peut dire ? \\
Peut-on penser contre l'expérience ? \\
Peut-on penser la matière ? \\
Peut-on penser la vie sans penser la mort ? \\
Peut-on penser la vie ? \\
Peut-on penser l'infini ? \\
Peut-on penser sans image ? \\
Peut-on penser sans les mots ? \\
Peut-on penser sans méthode ? \\
Peut-on penser sans préjugés ? \\
Peut-on penser un État sans violence ? \\
Peut-on percevoir sans juger ? \\
Peut-on perdre la raison ? \\
Peut-on perdre sa liberté ? \\
Peut-on perdre son temps ? \\
Peut-on prédire les événements ? \\
Peut-on prédire l'histoire ? \\
Peut-on préférer le bonheur à la vérité ? \\
Peut-on préférer l'injustice au désordre ? \\
Peut-on promettre le bonheur ? \\
Peut-on protéger les libertés sans les réduire ? \\
Peut-on prouver l'existence ? \\
Peut-on prouver une existence ? \\
Peut-on raconter sa vie ? \\
Peut-on ralentir la course du temps ? \\
Peut-on recommencer sa vie ? \\
Peut-on réduire le raisonnement au calcul ? \\
Peut-on refuser la violence ? \\
Peut-on refuser le vrai ? \\
Peut-on rendre raison de tout ? \\
Peut-on rendre raison du réel ? \\
Peut-on renoncer au bonheur ? \\
Peut-on réparer le vivant ? \\
Peut-on répondre d'autrui ? \\
Peut-on reprocher au langage d'être parfait ? \\
Peut-on résister au vrai ? \\
Peut-on rompre avec la société ? \\
Peut-on rompre avec le passé ? \\
Peut-on s'affranchir des lois ? \\
Peut-on s'attendre à tout ? \\
Peut-on savoir sans croire ? \\
Peut-on se choisir un destin ? \\
Peut-on se connaître soi-même ? \\
Peut-on se gouverner soi-même ? \\
Peut-on se mentir à soi-même ? \\
Peut-on se mettre à la place d'autrui ? \\
Peut-on se mettre à la place de l'autre ? \\
Peut-on se passer de croyances ? \\
Peut-on se passer de la religion ? \\
Peut-on se passer de l'État ? \\
Peut-on se passer d'État ? \\
Peut-on se passer de technique ? \\
Peut-on se passer de toute religion ? \\
Peut-on se passer d'idéal ? \\
Peut-on se passer d'un maître ? \\
Peut-on se prescrire une loi ? \\
Peut-on se tromper en se croyant heureux ? \\
Peut-on sympathiser avec l'ennemi ? \\
Peut-on tirer des leçons de l'histoire ? \\
Peut-on toujours faire ce qu'on doit ? \\
Peut-on tout analyser ? \\
Peut-on tout attendre de l'État ? \\
Peut-on tout démontrer ? \\
Peut-on tout dire ? \\
Peut-on tout donner ? \\
Peut-on tout échanger ? \\
Peut-on tout interpréter ? \\
Peut-on tout ordonner ? \\
Peut-on traiter un être vivant comme une machine ? \\
Peut-on vivre en sceptique ? \\
Peut-on vivre hors du temps ? \\
Peut-on vivre pour la vérité ? \\
Peut-on vivre sans désir ? \\
Peut-on vivre sans échange ? \\
Peut-on vivre sans le plaisir de vivre ? \\
Peut-on vivre sans lois ? \\
Peut-on vivre sans peur ? \\
Peut-on vivre sans réfléchir ? \\
Peut-on vivre sans sacré ? \\
Peut-on vouloir ce qu'on ne désire pas ? \\
Peut-on vouloir le bonheur d'autrui ? \\
Peut-on vouloir le mal ? \\
Philosophe-t-on pour être heureux ? \\
Physique et mathématiques \\
Physique et métaphysique \\
Pitié et compassion \\
Pitié et cruauté \\
Pitié et mépris \\
Plaisir et bonheur \\
Plusieurs religions valent-elles mieux qu'une seule ? \\
Poésie et philosophie \\
Politique et vérité \\
Possession et propriété \\
Pour connaître, suffit-il de démontrer ? \\
Pour être homme, faut-il être citoyen ? \\
Pour être libre, faut-il renoncer à être heureux ? \\
Pour être un bon observateur faut-il être un bon théoricien ? \\
Pour juger, faut-il seulement apprendre à raisonner ? \\
Pourquoi aller contre son désir ? \\
Pourquoi chercher à connaître le passé ? \\
Pourquoi chercher à se distinguer ? \\
Pourquoi chercher la vérité ? \\
Pourquoi cherche-t-on à connaître ? \\
Pourquoi défendre le faible ? \\
Pourquoi délibérer ? \\
Pourquoi des artistes ? \\
Pourquoi des cérémonies ? \\
Pourquoi des devoirs ? \\
Pourquoi des idoles ? \\
Pourquoi désire-t-on ce dont on n'a nul besoin ? \\
Pourquoi désirons-nous ? \\
Pourquoi des lois ? \\
Pourquoi des maîtres ? \\
Pourquoi des poètes ? \\
Pourquoi des utopies ? \\
Pourquoi dialogue-t-on ? \\
Pourquoi donner des leçons de morale ? \\
Pourquoi donner ? \\
Pourquoi échanger des idées ? \\
Pourquoi écrit-on les lois ? \\
Pourquoi écrit-on l'Histoire ? \\
Pourquoi écrit-on ? \\
Pourquoi être moral ? \\
Pourquoi étudier le vivant ? \\
Pourquoi étudier l'Histoire ? \\
Pourquoi faire confiance ? \\
Pourquoi faire son devoir ? \\
Pourquoi faut-il diviser le travail ? \\
Pourquoi faut-il être juste ? \\
Pourquoi faut-il travailler ? \\
Pourquoi interprète-t-on ? \\
Pourquoi joue-t-on ? \\
Pourquoi la justice a-t-elle besoin d'institutions ? \\
Pourquoi les sciences ont-elles une histoire ? \\
Pourquoi les sociétés ont-elles besoin de lois ? \\
Pourquoi l'homme travaille-t-il ? \\
Pourquoi lire les poètes ? \\
Pourquoi nous trompons-nous ? \\
Pourquoi nous-trompons nous ? \\
Pourquoi parle-t-on ? \\
Pourquoi punir ? \\
Pourquoi rechercher la vérité ? \\
Pourquoi refuse-t-on la conscience à l'animal ? \\
Pourquoi respecter autrui ? \\
Pourquoi respecter le droit ? \\
Pourquoi s'intéresser à l'histoire ? \\
Pourquoi s'interroger sur l'origine du langage ? \\
Pourquoi sommes-nous des êtres moraux ? \\
Pourquoi théoriser ? \\
Pourquoi transmettre ? \\
Pourquoi travailler ? \\
Pourquoi un fait devrait-il être établi ? \\
Pourquoi vivre ensemble ? \\
Pourquoi vouloir se connaître ? \\
Pourquoi y a-t-il des institutions ? \\
Pourquoi y a-t-il plusieurs sciences ? \\
Pourrait-on se passer de l'argent ? \\
Pouvoir et autorité \\
Pouvoir et devoir \\
Pouvoir et puissance \\
Pouvoir et savoir \\
Pouvons-nous connaître sans interpréter ? \\
Pouvons-nous désirer ce qui nous nuit ? \\
Pouvons-nous dissocier le réel de nos interprétations ? \\
Pouvons-nous faire l'expérience de la liberté ? \\
Prédire et expliquer \\
Prendre conscience \\
Prendre la parole \\
Prendre ses responsabilités \\
Prendre soin \\
Prendre son temps, est-ce le perdre ? \\
Preuve et démonstration \\
Production et création \\
Produire et créer \\
Prose et poésie \\
Prouver \\
Prouver et démontrer \\
Prouver et éprouver \\
Prouver et réfuter \\
Prudence et liberté \\
Puis-je être dans le vrai sans le savoir ? \\
Puis-je être libre sans être responsable ? \\
Puis-je faire confiance à mes sens ? \\
Puis-je invoquer l'inconscient sans ruiner la morale ? \\
Puis-je me passer d'imiter autrui ? \\
Puis-je ne pas vouloir ce que je désire ? \\
Puis-je répondre des autres ? \\
Puis-je savoir ce qui m'est propre ? \\
Punir \\
Punition et vengeance \\
Qu'ai-je le droit d'exiger d'autrui ? \\
Qu'ai-je le droit d'exiger des autres ? \\
Qu'aime-t-on dans l'amour ? \\
Qualité et quantité \\
Quand une autorité est-elle légitime ? \\
Qu'apprend-on des romans ? \\
Qu'apprend-on en commettant une faute ? \\
Qu'apprend-on quand on apprend à parler ? \\
Qu'attendons-nous de la technique ? \\
Qu'attendons-nous pour être heureux ? \\
Que célèbre l'art ? \\
Que démontrent nos actions ? \\
Que désirons-nous quand nous désirons savoir ?Qu'est-ce qu'un événement historique ? \\
Que désirons-nous ? \\
Que devons-nous à autrui ? \\
Que devons-nous à l'État ? \\
Que dois-je respecter en autrui ? \\
Que doit la pensée à l'écriture ? \\
Que doit la science à la technique ? \\
Que doit-on croire ? \\
Que doit-on désirer pour ne pas être déçu ? \\
Que faire de nos passions ? \\
Que faire des adversaires ? \\
Que faut-il absolument savoir ? \\
Que faut-il respecter ? \\
Que gagne-t-on à travailler ? \\
Quel est le contraire du travail ? \\
Quel est le poids du passé ? \\
Quel est le sens du progrès technique ? \\
Quel est l'objet de la biologie ? \\
Quel est l'objet de la métaphysique ? \\
Quelle causalité pour le vivant ? \\
Quelle est la cause du désir ? \\
Quelle est la fonction première de l'État ? \\
Quelle est la force de la loi ? \\
Quelle est la place de l'imagination dans la vie de l'esprit ? \\
Quelle est la place du hasard dans l'histoire ? \\
Quelle est la réalité de l'avenir ? \\
Quelle est la réalité d'une idée ? \\
Quelle est la réalité du passé ? \\
Quelle est la valeur d'une expérimentation ? \\
Quelle est la valeur du rêve ? \\
Quelle est la valeur du vivant ? \\
Quelle est l'unité du « je » ? \\
Quelle réalité attribuer à la matière ? \\
Quelle réalité l'art nous fait-il connaître ? \\
Quelle sorte d'histoire ont les sciences ? \\
Quelles sont les caractéristiques d'un être vivant ? \\
Quel sens donner à l'expression « gagner sa vie » ? \\
Que manque-t-il à une machine pour être vivante ? \\
Que mesure-t-on du temps ? \\
Que montre une démonstration ? \\
Que nous append l'histoire ? \\
Que nous apporte la vérité ? \\
Que nous apprend la définition de la vérité ? \\
Que nous apprend la fiction sur la réalité ? \\
Que nous apprend la maladie sur la santé ? \\
Que nous apprend la musique ? \\
Que nous apprend la vie ? \\
Que nous apprend l'expérience ? \\
Que nous apprennent les animaux sur nous-mêmes ? \\
Que nous apprennent les animaux ? \\
Que nous apprennent les machines ? \\
Que nous apprennent les métaphores ? \\
Que nous enseigne l'expérience ? \\
Que nous enseignent les sens ? \\
Que nous réserve l'avenir ? \\
Que peint le peintre ? \\
Que penser de l'adage : « Que la justice s'accomplisse, le monde dût-il périr » (Fiat justitia pereat mundus) ? \\
Que percevons-nous d'autrui ? \\
Que perdrait la pensée en perdant l'écriture ? \\
Que peut la musique ? \\
Que peut la volonté ? \\
Que peut le corps ? \\
Que peut l'esprit sur la matière ? \\
Que peut l'État ? \\
Que peut-on contre un préjugé ? \\
Que peut-on savoir de l'inconscient ? \\
Que peut-on savoir de soi ? \\
Que peut-on savoir du réel ? \\
Que peut-on savoir par expérience ? \\
Que pouvons-nous espérer de la connaissance du vivant ? \\
Que pouvons-nous faire de notre passé ? \\
Que produit l'inconscient ? \\
Que reste-t-il d'une existence ? \\
Que sait la conscience ? \\
Que sait-on du réel ? \\
Que serions-nous sans l'État ? \\
Que signifie être en guerre ? \\
Que signifie l'idée de technoscience ? \\
Que signifier « juger en son âme et conscience » ? \\
Que sont les apparences ? \\
Qu'est-ce qu'apprendre ? \\
Qu'est-ce qu'argumenter ? \\
Qu'est-ce que commencer ? \\
Qu'est-ce que composer une œuvre ? \\
Qu'est-ce que comprendre une œuvre d'art ? \\
Qu'est-ce que créer ? \\
Qu'est-ce que définir ? \\
Qu'est-ce que faire une expérience ? \\
Qu'est-ce que gouverner ? \\
Qu'est-ce que juger ? \\
Qu'est-ce que la causalité ? \\
Qu'est-ce que la science saisit du vivant ? \\
Qu'est-ce que le langage ordinaire ? \\
Qu'est-ce que le malheur ? \\
Qu'est-ce que le moi ? \\
Qu'est-ce que le présent ? \\
Qu'est-ce que le réel ? \\
Qu'est-ce que le sacré ? \\
Qu'est-ce que l'inconscient ? \\
Qu'est-ce que l'intérêt général ? \\
Qu'est-ce que manquer de culture ? \\
Qu'est-ce que parler le même langage ? \\
Qu'est-ce que parler ? \\
Qu'est-ce que prouver ? \\
Qu'est-ce que traduire ? \\
Qu'est-ce qu'être artiste ? \\
Qu'est-ce qu'être en vie ? \\
Qu'est-ce qu'être esclave ? \\
Qu'est-ce qu'être inhumain ? \\
Qu'est-ce qu'être l'auteur de son acte ? \\
Qu'est-ce qu'être malade ? \\
Qu'est-ce qu'être normal ? \\
Qu'est-ce qu'être réaliste ? \\
Qu'est-ce qu'être spirituel ? \\
Qu'est-ce que vérifier une théorie ? \\
Qu'est-ce que vivre bien ? \\
Qu'est-ce qu'exister pour un individu ? \\
Qu'est-ce qu'exister ? \\
Qu'est-ce que « parler le même langage » ? \\
Qu'est-ce qui distingue un vivant d'une machine ? \\
Qu'est-ce qui est absurde ? \\
Qu'est-ce qui est irrationnel ? \\
Qu'est-ce qui est irréversible ? \\
Qu'est-ce qui est naturel ? \\
Qu'est-ce qui est possible ? \\
Qu'est-ce qui est réel ? \\
Qu'est-ce qui est respectable ? \\
Qu'est-ce qui est scientifique ? \\
Qu'est-ce qui est vital ? \\
Qu'est-ce qui fait changer les sociétés ? \\
Qu'est-ce qui fait d'une activité un travail ? \\
Qu'est-ce qui fait la valeur de la technique ? \\
Qu'est-ce qui fait la valeur d'une existence ? \\
Qu'est-ce qui fait le pouvoir des mots ? \\
Qu'est-ce qui fait l'unité d'une science ? \\
Qu'est-ce qui fait l'unité d'un organisme ? \\
Qu'est-ce qui fait l'unité du vivant ? \\
Qu'est-ce qui fait un peuple ? \\
Qu'est-ce qui fonde le respect d'autrui ? \\
Qu'est-ce qui importe ? \\
Qu'est-ce qui menace la liberté ? \\
Qu'est-ce qui mesure la valeur d'un travail ? \\
Qu'est-ce qui n'a pas d'histoire ? \\
Qu'est-ce qu'interpréter une œuvre d'art ? \\
Qu'est-ce qui peut se transformer ? \\
Qu'est ce qui rapproche le vivant de la machine ? \\
Qu'est-ce qu'on ne peut comprendre ? \\
Qu'est-ce qu'un acte libre ? \\
Qu'est-ce qu'un alter ego ? \\
Qu'est-ce qu'un animal ? \\
Qu'est-ce qu'un bon citoyen ? \\
Qu'est-ce qu'un cas de conscience ? \\
Qu'est-ce qu'un chef-d'œuvre ? \\
Qu'est-ce qu'un citoyen libre ? \\
Qu'est-ce qu'un classique ? \\
Qu'est-ce qu'un concept ? \\
Qu'est-ce qu'un consommateur ? \\
Qu'est-ce qu'un désir satisfait ? \\
Qu'est-ce qu'une action juste ? \\
Qu'est-ce qu'une action politique ? \\
Qu'est-ce qu'une autorité légitime ? \\
Qu'est-ce qu'une belle forme ? \\
Qu'est-ce qu'une bonne délibération ? \\
Qu'est-ce qu'un échange juste ? \\
Qu'est-ce qu'un échange réussi ? \\
Qu'est-ce qu'une chose matérielle ? \\
Qu'est-ce qu'une communauté ? \\
Qu'est ce qu'une connaissance fiable ? \\
Qu'est-ce qu'une constitution ? \\
Qu'est-ce qu'une crise ? \\
Qu'est-ce qu'une erreur ? \\
Qu'est-ce qu'une expérience scientifique ? \\
Qu'est-ce qu'une fausse science ? \\
Qu'est-ce qu'une faute de goût ? \\
Qu'est-ce qu'une fiction ? \\
Qu'est-ce qu'une hypothèse scientifique ? \\
Qu'est-ce qu'une image ? \\
Qu'est-ce qu'une injustice ? \\
Qu'est-ce qu'une langue artificielle ? \\
Qu'est-ce qu'une libre interprétation ? \\
Qu'est ce qu'une mauvaise idée ? \\
Qu'est-ce qu'une méthode ? \\
Qu'est-ce qu'une œuvre d'art réaliste ? \\
Qu'est-ce qu'une parole vraie ? \\
Qu'est-ce qu'une preuve ? \\
Qu'est-ce qu'une république ? \\
Qu'est-ce qu'une révolution scientifique ? \\
Qu'est-ce qu'une révolution ? \\
Qu'est-ce qu'une science expérimentale ? \\
Qu'est-ce qu'une solution ? \\
Qu'est-ce qu'un esprit juste ? \\
Qu'est ce qu'un esprit libre ? \\
Qu'est-ce qu'un esprit libre ? \\
Qu'est-ce qu'un état de droit ? \\
Qu'est-ce qu'un État de droit ? \\
Qu'est-ce qu'un État libre ? \\
Qu'est-ce qu'une théorie scientifique ? \\
Qu'est-ce qu'une tradition ? \\
Qu'est-ce qu'un événement historique ? \\
Qu'est-ce qu'un événement ? \\
Qu'est-ce qu'une vérité contingente ? \\
Qu'est-ce qu'une vérité historique ? \\
Qu'est-ce qu'une vérité subjective ? \\
Qu'est-ce qu'une vie heureuse ? \\
Qu'est-ce qu'une vie humaine ? \\
Qu'est-ce qu'un exemple ? \\
Qu'est-ce qu'un expérimentateur ? \\
Qu'est-ce qu'un fait de culture ? \\
Qu'est-ce qu'un fait historique ? \\
Qu'est-ce qu'un fait ? \\
Qu'est-ce qu'un faux problème ? \\
Qu'est-ce qu'un faux ? \\
Qu'est-ce qu'un gouvernement démocratique ? \\
Qu'est-ce qu'un homme d'action ? \\
Qu'est-ce qu'un homme d'État ? \\
Qu'est-ce qu'un homme d'expérience ? \\
Qu'est-ce qu'un homme juste ? \\
Qu'est-ce qu'un homme méchant ? \\
Qu'est-ce qu'un homme politique ? \\
Qu'est-ce qu'un justicier ? \\
Qu'est-ce qu'un maître ? \\
Qu'est-ce qu'un monstre ? \\
Qu'est-ce qu'un musée ? \\
Qu'est-ce qu'un mythe ? \\
Qu'est-ce qu'un objet technique ? \\
Qu'est-ce qu'un outil ? \\
Qu'est-ce qu'un paradoxe ? \\
Qu'est-ce qu'un pauvre ? \\
Qu'est-ce qu'un peuple ? \\
Qu'est-ce qu'un problème scientifique ? \\
Qu'est-ce qu'un problème technique ? \\
Qu'est-ce qu'un problème ? \\
Qu'est-ce qu'un progrès technique ? \\
Qu'est-ce qu'un public ? \\
Qu'est-ce qu'un récit véridique ? \\
Qu'est-ce qu'un tabou ? \\
Qu'est-ce qu'un technicien ? \\
Qu'est-ce qu'un témoin ? \\
Qu'est-ce qu'un tyran ? \\
Que valent les mots ? \\
Que valent les théories ? \\
Que vaut la définition de l'homme comme animal doué de raison ? \\
Que veut dire : « le temps passe » ? \\
Qui accroît son savoir accroît sa douleur \\
Qui commande ? \\
Qui croire ? \\
Qui est digne du bonheur ? \\
Qui est libre ? \\
Qui est mon prochain ? \\
Qui est mon semblable ? \\
Qui est riche ? \\
Qui est sage ? \\
Qui fait la loi ? \\
Qui gouverne ? \\
Qui nous dicte nos devoirs ? \\
Qui parle quand je dis « je » ? \\
Qui parle ? \\
Qui peut avoir des droits ? \\
Qui peut me dire « tu ne dois pas » ? \\
Qui travaille ? \\
Raison et dialogue \\
Raison et folie \\
Raison et fondement \\
Raison et langage \\
Raison et tradition \\
Raisonnable et rationnel \\
Raisonner \\
Réalité et apparence \\
Réalité et perception \\
Réalité et représentation \\
Récit et histoire \\
Recourir au langage, est-ce renoncer à la violence ? \\
Réfuter \\
Règles sociales et loi morale \\
Regrets et remords \\
Religion et démocratie \\
Religion et moralité \\
Religion et politique \\
Religions et démocratie \\
Rendre justice \\
Répondre de soi \\
Représenter \\
République et démocratie \\
Résistance et obéissance \\
Respect et tolérance \\
Réussir sa vie \\
Rêver \\
Revient-il à l'État d'assurer le bonheur des citoyens ? \\
Révolte et révolution \\
Rhétorique et vérité \\
Richesse et pauvreté \\
Rire \\
Sait-on ce que l'on veut ? \\
Sait-on ce qu'on fait ? \\
Sait-on nécessairement ce que l'on désire ? \\
Sait-on toujours ce qu'on veut ? \\
S'amuser \\
Savoir est-ce cesser de croire ? \\
Savoir et croire \\
Savoir et démontrer \\
Savoir et pouvoir \\
Savoir et savoir faire \\
Science du vivant et finalisme \\
Science du vivant, science de l'inerte \\
Science et croyance \\
Science et métaphysique \\
Science et méthode \\
Science et mythe \\
Science et religion \\
Se cultiver \\
Se cultiver, est-ce s'affranchir de son appartenance culturelle ? \\
Se décider \\
Se faire comprendre \\
Se mentir à soi-même : est-ce possible ? \\
Se nourrir \\
Sensation et perception \\
Sens et existence \\
Sens et signification \\
Sens propre et sens figuré \\
Sentir et juger \\
Sentir et penser \\
Serions-nous heureux dans un ordre politique parfait ? \\
Serions-nous plus libres sans État ? \\
Servir, est-ce nécessairement renoncer à sa liberté ? \\
Se suffire à soi-même \\
S'exprimer \\
Signe et symbole \\
Sincérité et vérité \\
Si nous étions moraux, le droit serait-il inutile ? \\
Si tout est historique, tout est-il relatif ? \\
Société et communauté \\
Société humaines, sociétés animales \\
Sommes-nous dans le temps comme dans l'espace ? \\
Sommes-nous des sujets ? \\
Sommes-nous déterminés par notre culture ? \\
Sommes-nous faits pour le bonheur ? \\
Sommes-nous jamais certains d'avoir choisi librement ? \\
Sommes-nous les jouets de l'histoire ? \\
Sommes-nous libres face à l'évidence ? \\
Sommes-nous maîtres de nos désirs ? \\
Sommes-nous maîtres de nos paroles ? \\
Sommes-nous portés au bien ? \\
Sommes-nous prisonniers de nos désirs ? \\
Sommes-nous prisonniers de notre histoire ? \\
Sommes-nous prisonniers du temps ? \\
Sommes-nous responsables de ce dont nous n'avons pas conscience ? \\
Sommes-nous responsables de nos désirs ? \\
Sommes-nous responsables de nos opinions ? \\
Sommes-nous sujets de nos désirs ? \\
Sommes-nous toujours conscients des causes de nos désirs ?` \\
Soumission et servitude \\
Substance et accident \\
Suffit-il d'avoir raison ? \\
Suffit-il de bien juger pour bien faire ? \\
Suffit-il de faire son devoir pour être vertueux ? \\
Suffit-il de faire son devoir ? \\
Suffit-il de n'avoir rien fait pour être innocent ? \\
Suffit-il d'être vertueux pour être heureux ? \\
Suffit-il que nos intentions soient bonnes pour que nos actions le soient aussi ? \\
Suis-ce que j'ai conscience d'être ? \\
Suis-je ce que j'ai conscience d'être ? \\
Suis-je ce que je fais ? \\
Suis-je dans le temps comme je suis dans l'espace ? \\
Suis-je étranger à moi-même ? \\
Suis-je l'auteur de ce que je dis ? \\
Suis-je le mieux placé pour me connaître ? \\
Suis-je libre ? \\
Suis-je mon corps ? \\
Suis-je mon passé ? \\
Suis-je propriétaire de mon corps ? \\
Suis-je responsable de ce dont je n'ai pas conscience ? \\
Suis-je responsable de ce que je suis ? \\
Suis-je toujours autre que moi-même ? \\
Suivre son intuition \\
Surface et profondeur \\
Sur quoi fonder la justice ? \\
Sur quoi fonder l'autorité politique ? \\
Sur quoi fonder l'autorité ? \\
Sur quoi fonder le devoir ? \\
Sur quoi fonder le droit de punir ? \\
Sur quoi le langage doit-il se régler ? \\
Sur quoi repose la croyance au progrès ? \\
Survivre \\
Suspendre son jugement \\
Sympathie et respect \\
Talent et génie \\
Technique et idéologie \\
Technique et nature \\
Technique et progrès \\
Technique et savoir-faire \\
Technique et violence \\
Temps et commencement \\
Temps et création \\
Temps et histoire \\
Temps et irréversibilité \\
Temps et liberté \\
Temps et mémoire \\
Temps et vérité \\
Théorie et expérience \\
Toucher, sentir, goûter \\
Tous les hommes désirent-ils naturellement être heureux ? \\
Tous les hommes désirent-ils naturellement savoir ? \\
Tous les paradis sont-ils perdus ? \\
Tous les rapports humains sont-ils des échanges ? \\
Tout a-t-il une raison d'être ? \\
Tout ce qui est naturel est-il normal ? \\
Tout ce qui est rationnel est-il raisonnable ? \\
Tout ce qui est vrai doit-il être prouvé ? \\
Tout démontrer \\
Tout désir est-il égoïste ? \\
Tout désir est-il manque ? \\
Tout désir est-il une souffrance ? \\
Tout dire \\
Tout droit est-il un pouvoir ? \\
Toute compréhension implique-t-elle une interprétation ? \\
Toute connaissance est-elle hypothétique ? \\
Toute connaissance s'enracine-t-elle dans la perception ? \\
Toute conscience est-elle subjective ? \\
Toute description est-elle une interprétation ? \\
Toute faute est-elle une erreur ? \\
Toute inégalité est-elle injuste ? \\
Toute interprétation est-elle contestable ? \\
Toute interprétation est-elle subjective ? \\
Toute morale s'oppose-t-elle aux désirs ? \\
Toute relation humaine est-elle un échange ? \\
Toutes les fautes se valent-elles ? \\
Toutes les inégalités sont-elles des injustices ? \\
Toutes les interprétations se valent-elles ? \\
Toute société a-t-elle besoin d'une religion ? \\
Tout est-il historique ? \\
Tout est-il matière ? \\
Toute vérité est-elle démontrable ? \\
Toute vie est-elle intrinsèquement respectable ? \\
Tout futur est-il contingent ? \\
Tout ordre est-il une violence déguisée ? \\
Tout peut-il être objet d'échange ? \\
Tout peut-il être objet de science ? \\
Tout peut-il s'acheter ? \\
Tout peut-il se démontrer ? \\
Tout s'en va-t-il avec le temps ? \\
Tout se prête-il à la mesure ? \\
Tout travail est-il forcé ? \\
Tout travail est-il social ? \\
Tout vouloir \\
Tradition et liberté \\
Tradition et nouveauté \\
Tradition et transmission \\
Traduire, est-ce trahir ? \\
Transcendance et immanence \\
Transmettre \\
Travail, besoin, désir \\
Travail et aliénation \\
Travail et besoin \\
Travail et bonheur \\
Travail et capital \\
Travail et liberté \\
Travail et loisir \\
Travail et nécessité \\
Travail et œuvre \\
Travailler, est-ce faire œuvre ? \\
Travailler et œuvrer \\
Travaille-t-on pour soi-même ? \\
Travail manuel et travail intellectuel \\
Travail manuel, travail intellectuel \\
Un acte gratuit est-il possible ? \\
Un acte peut-il être inhumain ? \\
Un artiste doit-il être original ? \\
Un bien peut-il être commun ? \\
Un choix peut-il être rationnel ? \\
Un désir peut-il être coupable ? \\
Un désir peut-il être inconscient ? \\
Une activité inutile est-elle sans valeur ? \\
Une communauté politique n'est-elle qu'une communauté d'intérêt ? \\
Une connaissance peut-elle ne pas être relative ? \\
Une connaissance scientifique du vivant est-elle possible ? \\
Une croyance peut-elle être libre ? \\
Une croyance peut-elle être rationnelle ? \\
Une culture peut-elle être porteuse de valeurs universelles ? \\
Une destruction peut-elle être créatrice ? \\
Une éducation morale est-elle possible ? \\
Une expérience peut-elle être fictive ? \\
Une idée peut-elle être générale ? \\
Une imitation peut-elle être parfaite ? \\
Une interprétation peut-elle échapper à l'arbitraire ? \\
Une interprétation peut-elle être définitive ? \\
Une interprétation peut-elle être objective ? \\
Une interprétation peut-elle prétendre à la vérité ? \\
Une langue n'est-elle faite que de mots ? \\
Une loi peut-elle être injuste ? \\
Une morale sans obligation est-elle possible ? \\
Une morale sceptique est-elle possible ? \\
Une œuvre d'art a-t-elle toujours un sens ? \\
Une œuvre d'art doit-elle nécessairement être belle ? \\
Une œuvre d'art doit-elle plaire ? \\
Une œuvre d'art peut-elle être immorale ? \\
Une pensée contradictoire est-elle dénuée de valeur ? \\
Une perception peut-elle être illusoire ? \\
Une psychologie peut-elle être matérialiste ? \\
Une religion peut-elle se passer de pratiques ? \\
Une science de l'esprit est-elle possible ? \\
Une sensation peut-elle être fausse ? \\
Une société juste est-ce une société sans conflit ? \\
Une société sans État est-elle possible ? \\
Une société sans religion est-elle possible ? \\
Une société sans travail est-elle souhaitable ? \\
Une théorie peut-elle être vérifiée ? \\
Un être vivant peut-il être comparé à une œuvre d'art ? \\
Un événement historique est-il toujours imprévisible ? \\
Une vérité peut-elle être indicible ? \\
Une vérité peut-elle être provisoire ? \\
Une vie heureuse est-elle une vie de plaisirs ? \\
Une vie libre exclut-elle le travail ? \\
Un fait existe-t-il sans interprétation ? \\
Un gouvernement de savants est-il souhaitable ? \\
Un mensonge peut-il avoir une valeur morale ? \\
Un monde meilleur \\
Un peuple est-il responsable de son histoire ? \\
Un peuple est-il un rassemblement d'individus ? \\
Un peuple se définit-il par son histoire ? \\
Un problème moral peut-il recevoir une solution certaine ? \\
Un savoir peut-il être inconscient ? \\
User de violence peut-il être moral ? \\
Utilité et beauté \\
Vaut-il mieux subir ou commettre l'injustice ? \\
Vérité et apparence \\
Vérité et certitude \\
Vérité et efficacité \\
Vérité et exactitude \\
Vérité et illusion \\
Vérité et liberté \\
Vérité et réalité \\
Vérité et religion \\
Vérité et sincérité \\
Vérité et vérification \\
Vérité et vraisemblance \\
Vérité théorique, vérité pratique \\
Vice et délice \\
Vie politique et vie contemplative \\
Vie privée et vie publique \\
Vie publique et vie privée \\
Violence et force \\
Violence et pouvoir \\
Vivrait-on mieux sans désirs ? \\
Vivre en société, est-ce seulement vivre ensemble ? \\
Vivre, est-ce lutter pour survivre ? \\
Vivre, est-ce résister à la mort ? \\
Vivre et exister \\
Vivre libre \\
Vivre sa vie \\
Voir et entendre \\
Voir et savoir \\
Voir et toucher \\
Voit-on ce qu'on croit ? \\
Vouloir dire \\
Vouloir et pouvoir \\
Vouloir la paix sociale peut-il aller jusqu'à accepter l'injustice ? \\
Vouloir la solitude \\
Y a-t-il d'autres moyens que la démonstration pour établir la vérité ? \\
Y a-t-il de bons et de mauvais désirs ? \\
Y a-t-il de bons préjugés ? \\
Y a-t-il de justes inégalités ? \\
Y a-t-il de la fatalité dans la vie de l'homme ? \\
Y a-t-il de l'inconnaissable ? \\
Y a-t-il de l'indémontrable ? \\
Y a-t-il de l'indésirable ? \\
Y a-t-il de mauvais désirs ? \\
Y a-t-il des arts mineurs ? \\
Y a-t-il des biens inestimables ? \\
Y a-t-il des choses qu'on n'échange pas ? \\
Y a-t-il des connaissances dangereuses ? \\
Y a-t-il des contraintes légitimes ? \\
Y a-t-il des convictions philosophiques ? \\
Y a-t-il des correspondances entre les arts ? \\
Y a-t-il des croyances nécessaires ? \\
Y a-t-il des degrés de conscience ? \\
Y a-t-il des degrés de vérité ? \\
Y a-t-il des démonstrations en philosophie ? \\
Y a-t-il des devoirs envers soi ? \\
Y a-t-il des erreurs de la nature ? \\
Y a-t-il des évidences morales ? \\
Y a-t-il des expériences sans théorie ? \\
Y a-t-il des faits scientifiques ? \\
Y a-t-il des fins de la nature ? \\
Y a-t-il des guerres justes ? \\
Y a-t-il des illusions de la conscience ? \\
Y a-t-il des inégalités justes ? \\
Y a-t-il des injustices naturelles ? \\
Y a-t-il des leçons de l'histoire ? \\
Y a-t-il des limites à la connaissance ? \\
Y a-t-il des limites à la tolérance ? \\
Y a-t-il des mondes imaginaires ? \\
Y a-t-il des normes naturelles ? \\
Y a-t-il des objets qui n'existent pas ? \\
Y a-t-il des obstacles à la connaissance du vivant ? \\
Y a-t-il des perceptions insensibles ? \\
Y a-t-il des peuples sans histoire ? \\
Y a-t-il des plaisirs meilleurs que d'autres ? \\
Y a-t-il des principes de justice universels ? \\
Y a-t-il des progrès en art ? \\
Y a-t-il des questions sans réponse ? \\
Y a-t-il des raisons de douter de la raison ? \\
Y a-t-il des solutions en politique ? \\
Y a-t-il des sots métiers ? \\
Y a-t-il des valeurs absolues ? \\
Y a-t-il des valeurs propres à la science ? \\
Y a-t-il des vérités de fait ? \\
Y a-t-il des vérités définitives ? \\
Y a-t-il des vérités en art ? \\
Y a-t-il des vérités éternelles ? \\
Y a-t-il des vérités indémontrables ? \\
Y a-t-il des vérités indiscutables ? \\
Y a-t-il des vérités morales ? \\
Y a-t-il des vérités qui échappent à la raison ? \\
Y a-t-il différentes façons d'exister ? \\
Y a-t-il du nouveau dans l'histoire ? \\
Y a-t-il nécessairement du religieux dans l'art ? \\
Y a-t-il plusieurs sortes de matières ? \\
Y a-t-il progrès en art ? \\
Y a-t-il quoi que ce soit de nouveau dans l'histoire ? \\
Y a-t-il un art d'être heureux ? \\
Y a-t-il un art de vivre ? \\
Y a-t-il un art d'interpréter ? \\
Y a-t-il un art du bonheur ? \\
Y a-t-il un bonheur sans illusion ? \\
Y a-t-il un bon usage du temps ? \\
Y a-t-il un devoir de mémoire ? \\
Y a-t-il un devoir d'être heureux ? \\
Y a-t-il un droit au bonheur ? \\
Y a-t-il un droit au travail ? \\
Y a-t-il un droit de désobéissance ? \\
Y a-t-il un droit de mentir ? \\
Y a-t-il un droit de révolte ? \\
Y a-t-il un droit naturel ? \\
Y a-t-il une beauté propre à l'objet technique ? \\
Y a-t-il une causalité en histoire ? \\
Y a-t-il une compétence politique ? \\
Y a-t-il une condition humaine ? \\
Y a-t-il une conscience collective ? \\
Y a-t-il une enfance de l'humanité ? \\
Y a-t-il une expérience du temps ? \\
Y a-t-il une fonction propre à l'œuvre d'art ? \\
Y a-t-il une hiérarchie du vivant ? \\
Y a-t-il une histoire de la raison ? \\
Y a-t-il une histoire de la vérité ? \\
Y a-t-il une histoire universelle ? \\
Y a-t-il une irréversibilité du temps ? \\
Y a-t-il une justice naturelle ? \\
Y a-t-il une limite à la connaissance du vivant ? \\
Y a-t-il une limite au désir ? \\
Y a-t-il une logique dans l'histoire ? \\
Y a-t-il une logique du désir ? \\
Y a-t-il une méthode de l'interprétation ? \\
Y a-t-il une morale universelle ? \\
Y a-t-il une nature humaine ? \\
Y a-t-il une nécessité de l'erreur ? \\
Y a-t-il une nécessité morale ? \\
Y a-t-il une œuvre du temps ? \\
Y a-t-il une ou des morales ? \\
Y a-t-il une pensée technique ? \\
Y a-t-il une primauté du devoir sur le droit ? \\
Y a-t-il une rationalité du hasard ? \\
Y a-t-il une responsabilité de l'artiste ? \\
Y a-t-il une science politique ? \\
Y a-t-il une servitude volontaire ? \\
Y a-t-il une spécificité du vivant ? \\
Y a-t-il un État idéal ? \\
Y a-t-il une technique pour tout ? \\
Y a-t-il une unité des devoirs ? \\
Y a-t-il une unité des sciences ? \\
Y a-t-il une valeur de l'inutile ? \\
Y a-t-il une vérité de l'œuvre d'art ? \\
Y a-t-il une vérité des apparences ? \\
Y a-t-il une vérité des représentations ? \\
Y a-t-il une vertu de l'imitation ? \\
Y a-t-il une violence du droit ? \\
Y a-t-il un fondement de la croyance ? \\
Y a-t-il un jugement de l'histoire ? \\
Y a-t-il un langage du corps ? \\
Y a-t-il un moteur de l'histoire ? \\
Y a-t-il un objet du désir ? \\
Y a-t-il un ordre dans la nature ? \\
Y a-t-il un ordre des choses ? \\
Y a-t-il un ordre du monde ? \\
Y a-t-il un primat de la nature sur la culture ? \\
Y a-t-il un progrès du droit ? \\
Y a-t-il un progrès en art ? \\
Y a-t-il un propre de l'homme ? \\
Y a-t-il un rapport moral à soi-même ? \\
Y a-t-il un savoir de la justice ? \\
Y a-t-il un savoir du juste ? \\
Y a-t-il un sens moral ? \\
Y aura-t-il toujours des religions ? \\
« Aimer » se dit-il en un seul sens ? \\
« Connais-toi toi-même » \\
« Dans un bois aussi courbe que celui dont l'homme est fait on ne peut rien tailler de tout à fait droit » \\
« Il ne lui manque que la parole » \\
« Je ne l'ai pas fait exprès » \\
« Les bons comptes font les bons amis » \\
« Les faits sont là » \\
« L'histoire jugera » : quel sens faut-il accorder à cette expression ? \\
« Liberté, égalité, fraternité » \\
« Rien de ce qui est humain ne m'est étranger » \\
« Tout est relatif » \\
« Tradition n'est pas raison » \\
« Un instant d'éternité » \\
« Vis caché » \\


\subsection{CAPES externe}
\label{sec-2-5}

\noindent
Abstraire, est-ce se couper du réel ? \\
Action et contemplation \\
Activité et passivité \\
Agir et faire \\
Agir et réagir \\
Ai-je des devoirs envers moi-même ? \\
Ai-je un corps ou suis-je mon corps ? \\
Aimer peut-il être un devoir ? \\
Ami et ennemi \\
Amour et inconscient \\
Analyser \\
Apparence et réalité \\
Apprend-on à percevoir ? \\
Apprendre à voir \\
Apprendre et enseigner \\
À quelles conditions une expérience est-elle possible ? \\
À quoi bon imiter la nature ? \\
À quoi nos illusions tiennent-elles ? \\
À quoi sert la technique ? \\
À quoi servent les preuves de l'existence de Dieu ? \\
Argent et liberté \\
Argumenter et démontrer \\
Art et beauté \\
Art et création \\
Art et illusion \\
Art et imagination \\
Art et jeu \\
Art et matière \\
Art et pouvoir \\
Art et représentation \\
Art et société \\
Art et Société \\
Art et symbole \\
Art et technique \\
Art et vérité \\
A-t-on besoin de certitudes ? \\
A-t-on besoin d'experts ? \\
A-t-on besoin d'un chef ? \\
A-t-on des devoirs envers soi-même ? \\
A-t-on le droit de se révolter ? \\
Au nom de qui rend-on justice ? \\
Au nom de quoi rend-on justice ? \\
Autorité et souveraineté \\
Autrui est-il pour moi un mystère ? \\
Autrui est-il un autre moi-même ? \\
Autrui m'est-il étranger ? \\
Avoir du jugement \\
Avoir du métier \\
Avoir du pouvoir \\
Avoir mauvaise conscience \\
Avoir raison \\
Avoir un corps \\
Avoir un destin \\
Avons-nous besoin de cérémonies ? \\
Avons-nous besoin de héros ? \\
Avons-nous besoin de maîtres ? \\
Avons-nous des devoirs à l'égard de la vérité ? \\
Avons-nous des devoirs envers la nature ? \\
Avons-nous des devoirs envers nous-mêmes ? \\
Avons-nous des droits sur la nature ? \\
Avons-nous intérêt à la liberté d'autrui ? \\
Avons-nous le devoir de vivre ? \\
Avons-nous peur de la liberté ? \\
Avons-nous un devoir de vérité ? \\
À chacun selon son mérite \\
À quelle condition un travail est-il humain ? \\
À quelles conditions le vivant peut-il être objet de science ? \\
À quelles conditions une démarche est-elle scientifique ? \\
À quelles conditions une hypothèse est-elle scientifique ? \\
À qui doit-on le respect ? \\
À qui doit-on obéir ? \\
À qui faut-il obéir ? \\
À qui la faute ? \\
À quoi bon démontrer ? \\
À quoi la perception donne-t-elle accès ? \\
À quoi peut-on reconnaître une œuvre d'art ? \\
À quoi reconnaît-on la vérité ? \\
À quoi reconnaît-on le réel ? \\
À quoi reconnaît-on qu'une activité est un travail ? \\
À quoi reconnaît-on qu'une expérience est scientifique ? \\
À quoi reconnaît-on qu'une pensée est vraie ? \\
À quoi reconnaît-on un acte libre ? \\
À quoi reconnaît-on une attitude religieuse ? \\
À quoi reconnaît-on une bonne interprétation ? \\
À quoi reconnaît-on une idéologie ? \\
À quoi servent les images ? \\
À quoi servent les lois ? \\
À quoi servent les machines ? \\
À quoi servent les preuves ? \\
À quoi servent les symboles ? \\
À quoi servent les théories ? \\
À quoi servent les voyages ? \\
À quoi tient la force des religions ? \\
À quoi tient la valeur d'une pensée ? \\
À quoi tient le pouvoir des mots ? \\
À quoi tient notre humanité ? \\
Bêtise et méchanceté \\
Bien agir, est-ce toujours être moral ? \\
Bien commun et intérêt particulier \\
Bonheur et satisfaction \\
Bonheur et technique \\
Bonheur et vertu \\
Calculer et penser \\
Cause et condition \\
Cause et effet \\
Cause et loi \\
Cause et raison \\
Ce que je pense est-il nécessairement vrai ? \\
Ce qui est ordinaire est-il normal ? \\
Ce qui est vrai est-il toujours vérifiable ? \\
Ce qui ne peut s'acheter est-il dépourvu de valeur ? \\
Certitude et conviction \\
Changer, est-ce devenir un autre ? \\
Changer le monde \\
Change-t-on avec le temps ? \\
Châtier, est ce faire honneur au criminel ? \\
Choisir, est-ce renoncer ? \\
Choisissons-nous qui nous sommes ? \\
Choix et raison \\
Chose et objet \\
Chose et personne \\
Civilisation et barbarie \\
Classer \\
Colère et indignation \\
Commander \\
Comment autrui peut-il m'aider à rechercher le bonheur ? \\
Comment chercher ce qu'on ignore ? \\
Comment comprendre les faits sociaux ? \\
Comment connaître nos devoirs ? \\
Comment dire la vérité ? \\
Comment distinguer le rêvé du perçu ? \\
Comment juger de la justesse d'une interprétation ? \\
Comment le passé peut-il demeurer présent ? \\
Comment l'erreur est-elle possible ? \\
Comment penser le hasard ? \\
Comment penser l'éternel ? \\
Comment puis-je devenir ce que je suis ? \\
Communauté et société \\
Comprendre \\
Comprendre le réel est-ce le dominer ? \\
Comprendre une démonstration \\
Concept et métaphore \\
Concurrence et égalité \\
Connaissance de soi et conscience de soi \\
Connaissance et perception \\
Connaissons-nous la réalité des choses ? \\
Connaît-on la vie ou le vivant ? \\
Connaît-on les choses telles qu'elles sont ? \\
Connaître est-ce découvrir le réel ? \\
Connaître, est-ce dépasser les apparences ? \\
Connaître la vie ou le vivant ? \\
Conquérir \\
Conscience de soi et amour de soi \\
Conscience et connaissance \\
Conscience et conscience de soi \\
Conscience et responsabilité \\
Conscience et subjectivité \\
Conscience et volonté \\
Contempler \\
Contrainte et obligation \\
Convaincre et persuader \\
Convient-il d'opposer explication et interprétation ? \\
Corps et espace \\
Corps et matière \\
Corps et nature \\
Crainte et espoir \\
Création et production \\
Créer et produire \\
Critiquer \\
Croire, est-ce renoncer au savoir ? \\
Croire et savoir \\
Croire que Dieu existe, est-ce croire en lui ? \\
Culpabilité et responsabilité \\
Culture et civilisation \\
Culture et différence \\
Culture et éducation \\
Culture et langage \\
Culture et savoir \\
Culture et technique \\
Culture et violence \\
Dans l'action, est-ce l'intention qui compte ? \\
Dans quel but les hommes se donnent-ils des lois ? \\
Dans quelle mesure est-on l'auteur de sa propre vie ? \\
Dans quelle mesure toute philosophie est-elle critique du langage ? \\
Débattre et dialoguer \\
Déchiffrer \\
Décider \\
Découverte et justification \\
Dématérialiser \\
Démocratie et opinion \\
Démocratie et représentation \\
Démontrer et argumenter \\
Démontrer par l'absurde \\
De quel droit l'État exerce-t-il un pouvoir ? \\
De quel droit punit-on ? \\
De quelle vérité l'art est-il capable ? \\
De quoi avons-nous vraiment besoin ? \\
De quoi dépend notre bonheur ? \\
De quoi est fait mon présent ? \\
De quoi est-fait notre présent ? \\
De quoi la philosophie est-elle le désir ? \\
De quoi la vérité libère-t-elle ? \\
De quoi le devoir libère-t-il ? \\
De quoi les logiciens parlent-ils ? \\
De quoi parlent les mathématiques ? \\
De quoi peut-on être inconscient ? \\
De quoi peut-on faire l'expérience ? \\
De quoi pouvons-nous être sûrs ? \\
De quoi puis-je répondre ? \\
De quoi sommes-nous responsables ? \\
De quoi suis-je inconscient ? \\
Désirer, est-ce être aliéné ? \\
Désir et besoin \\
Désir et bonheur \\
Désir et interdit \\
Désir et langage \\
Désir et manque \\
Désir et ordre \\
Désir et pouvoir \\
Désir et raison \\
Désir et réalité \\
Désir et volonté \\
Des lois justes suffisent-elles à assurer la justice ? \\
Devant qui sommes-nous responsables ? \\
Devoir et bonheur \\
Devoir et contrainte \\
Devoir et intérêt \\
Devoir et liberté \\
Devoir et prudence \\
Devoir et vertu \\
Devoirs et passions \\
Devons-nous dire la vérité ? \\
Dire, est-ce faire ? \\
Dire et exprimer \\
Dire et faire \\
Dire je \\
Dogme et opinion \\
Doit-on apprendre à percevoir ? \\
Doit-on apprendre à vivre ? \\
Doit-on bien juger pour bien faire ? \\
Doit-on changer ses désirs, plutôt que l'ordre du monde ? \\
Doit-on identifier l'âme à la conscience ? \\
Doit-on interpréter les rêves ? \\
Doit-on le respect au vivant ? \\
Doit-on mûrir pour la liberté ? \\
Doit-on rechercher le bonheur ? \\
Doit-on refuser d'interpréter ? \\
Doit-on se passer des utopies ? \\
Doit-on tenir le plaisir pour une fin ? \\
Doit-on toujours dire la vérité ? \\
Don et échange \\
Donner, à quoi bon ? \\
Donner et recevoir \\
Donner sa parole \\
Doute et raison \\
D'où vient la certitude ? \\
D'où vient la servitude ? \\
Droit et coutume \\
Droit et devoir \\
Droit et morale \\
Droit et violence \\
Droits de l'homme ou droits du citoyen ? \\
Droits et devoirs \\
Durée et instant \\
Durer \\
Échange et don \\
Échanger des idées \\
Échanger, est-ce créer de la valeur ? \\
Échanger, est-ce partager ? \\
Écouter et entendre \\
Écrire et parler \\
Égalité et différence \\
En quel sens l'État est-il rationnel ? \\
En quel sens parler de lois de la pensée ? \\
En quel sens parler d'identité culturelle ? \\
En quel sens peut-on dire que la vérité s'impose ? \\
En quel sens peut-on dire que l'homme est un animal politique ? \\
En quel sens peut-on dire qu' « on expérimente avec sa raison » ? \\
En quel sens peut-on parler de la mort de l'art ? \\
En quel sens peut-on parler d'une culture technique ? \\
En quel sens peut-on parler d'une interprétation de la nature ? \\
En quoi la méthode est-elle un art de penser ? \\
En quoi l'art peut-il intéresser le philosophe ? \\
En quoi le bien d'autrui m'importe-t-il ? \\
En quoi les vivants témoignent-ils d'une histoire ? \\
Entre l'opinion et la science, n'y a-t-il qu'une différence de degré ? \\
Erreur et faute \\
Erreur et illusion \\
Essence et existence \\
Est-ce à la raison de déterminer ce qui est réel ? \\
Est-ce la majorité qui doit décider ? \\
Est-ce l'autorité qui fait la loi ? \\
Est-ce le cerveau qui pense ? \\
Est-ce l'échange utilitaire qui fait le lien social ? \\
Est-ce l'ignorance qui rend les hommes croyants ? \\
Est-ce l'intérêt qui fonde le lien social ? \\
Est-ce un devoir d'aimer son prochain ? \\
Est-il immoral de se rendre heureux ? \\
Est-il juste d'interpréter la loi ? \\
Est-il légitime d'opposer liberté et nécessité ? \\
Est-il naturel à l'homme de parler ? \\
Est-il possible de tout avoir pour être heureux ? \\
Est-il possible d'être immoral sans le savoir ? \\
Est-il raisonnable d'aimer ? \\
Est-il raisonnable d'être rationnel ? \\
Est-il raisonnable de vouloir maîtriser la nature ? \\
Est-il vrai que les animaux ne pensent pas ? \\
Est-il vrai que l'ignorant n'est pas libre ? \\
Est-il vrai que ma liberté s'arrête là où commence celle des autres ? \\
Est-il vrai que plus on échange, moins on se bat ? \\
Estime et respect \\
Estimer \\
Est-on l'auteur de sa propre vie ? \\
Est-on libre face à la vérité ? \\
Est-on sociable par nature ? \\
État et institutions \\
État et nation \\
État et Société \\
État et société civile \\
Éthique et Morale \\
Être bon juge \\
Être cultivé rend-il meilleur ? \\
Être de son temps \\
Être dogmatique \\
Être et apparaître \\
Être et avoir \\
Être et avoir été \\
Être et devenir \\
Être et devoir être \\
Être et exister \\
Être et paraître \\
Être exemplaire \\
Être libre, est-ce dire non ? \\
Être libre est-ce faire ce que l'on veut ? \\
Être libre, est-ce pouvoir choisir ? \\
Être libre, est-ce se suffire à soi-même ? \\
Être réaliste \\
Être sceptique \\
Être soi-même \\
Être spectateur \\
Être un sujet, est-ce être maître de soi ? \\
Être vertueux \\
Évidence et raison \\
Évidence et vérité \\
Évidences et préjugés \\
Évolution biologique et culture \\
Évolution et progrès \\
Excuser et pardonner \\
Existence et contingence \\
Exister, est-ce simplement vivre ? \\
Existe-t-il des choses en soi ? \\
Existe-t-il des désirs coupables ? \\
Existe-t-il une méthode pour rechercher la vérité ? \\
Existe-t-il une méthode pour trouver la vérité ? \\
Expérience et expérimentation \\
Expliquer et comprendre \\
Faire confiance \\
Faire de nécessité vertu \\
Faire la paix \\
Faire le mal \\
Faire l'histoire \\
Fait et fiction \\
Fait et preuve \\
Fait et valeur \\
Faits et preuves \\
Faudrait-il ne rien oublier ? \\
Faudrait-il vivre sans passion ? \\
Faut-avoir peur de la technique ? \\
Faut-il accepter sa condition ? \\
Faut-il affirmer son identité ? \\
Faut-il aimer autrui pour le respecter ? \\
Faut-il aimer son prochain comme soi-même ? \\
Faut-il apprendre à être libre ? \\
Faut-il avoir peur des machines ? \\
Faut-il avoir peur d'être libre ? \\
Faut-il avoir peur du désordre ? \\
Faut-il chercher à se connaître ? \\
Faut-il chercher la paix à tout prix ? \\
Faut-il chercher le bonheur à tout prix ? \\
Faut-il chercher un sens à l'histoire ? \\
Faut-il choisir entre être heureux et être libre ? \\
Faut-il craindre le développement des techniques ? \\
Faut-il craindre les machines ? \\
Faut-il craindre l'ordre ? \\
Faut-il croire en la science ? \\
Faut-il dire de la justice qu'elle n'existe pas ? \\
Faut-il douter de ce qu'on ne peut pas démontrer ? \\
Faut-il du passé faire table rase ? \\
Faut-il espérer pour agir ? \\
Faut-il être cohérent ? \\
Faut-il être modéré ? \\
Faut-il être pragmatique ? \\
Faut-il faire confiance au progrès technique ? \\
Faut-il faire de nécessité vertu ? \\
Faut-il hiérarchiser les formes de vie ? \\
Faut-il libérer l'humanité du travail ? \\
Faut-il limiter la souveraineté de l'État ? \\
Faut-il ne manquer de rien pour être heureux ? \\
Faut-il opposer la matière et l'esprit ? \\
Faut-il opposer le don et l'échange ? \\
Faut-il opposer le temps vécu et le temps des choses ? \\
Faut-il opposer raison et sensation ? \\
Faut-il pour le connaître faire du vivant un objet ? \\
Faut-il préférer l'art à la nature ? \\
Faut-il rejeter tous les préjugés ? \\
Faut-il rejeter toute norme ? \\
Faut-il renoncer à faire du travail une valeur ? \\
Faut-il renoncer à l'idée d'âme ? \\
Faut-il rompre avec le passé ? \\
Faut-il s'adapter ? \\
Faut-il s'affranchir des désirs ? \\
Faut-il s'aimer soi-même ? \\
Faut-il savoir mentir ? \\
Faut-il savoir pour agir ? \\
Faut-il se cultiver ? \\
Faut-il se détacher du monde ? \\
Faut-il se fier aux apparences ? \\
Faut-il se méfier de l'intuition ? \\
Faut-il s'en tenir aux faits ? \\
Faut-il se rendre à l'évidence ? \\
Faut-il se ressembler pour former une société ? \\
Faut-il toujours éviter de se contredire ? \\
Faut-il toujours faire son devoir ? \\
Faut-il tout critiquer ? \\
Faut-il un corps pour penser ? \\
Faut-il vivre avec son temps ? \\
Faut-il vivre comme si nous ne devions jamais mourir ? \\
Faut-il vivre comme si on ne devait jamais mourir ? \\
Faut-il vivre hors de la société pour être heureux ? \\
Faut-il vouloir être heureux ? \\
Foi et raison \\
Foi et savoir \\
Foi et superstition \\
Force et violence \\
Forme et matière \\
Former et éduquer \\
Forme-t-on son esprit en transformant la matière ? \\
Gouvernement des hommes et administration des choses \\
Gouverner \\
Gouverner, est-ce régner ? \\
Guerre et politique \\
Habiter \\
Hasard et destin \\
Hier a-t-il plus de réalité que demain ? \\
Histoire et mémoire \\
Histoire et violence \\
Histoire individuelle et histoire collective \\
Hypothèse et vérité \\
Ici et maintenant \\
Idéal et utopie \\
Idée et réalité \\
Identité et différence \\
Image et concept \\
Image et idée \\
Imagination et culture \\
Imagination et pouvoir \\
Imitation et création \\
Imitation et représentation \\
Inconscient et déterminisme \\
Inconscient et inconscience \\
Inconscient et instinct \\
Inconscient et liberté \\
Inconscient et mythes \\
Indépendance et autonomie \\
Indépendance et liberté \\
Individu et citoyen \\
Individu et communauté \\
Individu et société \\
Innocence et ignorance \\
Instruction et éducation \\
Instruire et éduquer \\
Intérêt général et bien commun \\
Interprétation et création \\
Interpréter \\
Interpréter, est-ce connaître ? \\
Interpréter est-il subjectif ? \\
Interpréter et traduire \\
Interprète-t-on à défaut de connaître ? \\
Interroger \\
Intuition et déduction \\
Invention et découverte \\
Jugement et réflexion \\
Jugement et vérité \\
Juger et sentir \\
Jusqu'à quel point la nature est-elle objet de science ? \\
Justice et charité \\
Justice et égalité \\
Justice et équité \\
Justice et vengeance \\
Justice et violence \\
La barbarie \\
La beauté du monde \\
La beauté est-elle dans les choses ? \\
La beauté est-elle intemporelle ? \\
La beauté morale \\
La beauté nous rend-elle meilleurs ? \\
La beauté s'explique-t-elle ? \\
La bête \\
La bête et l'animal \\
La bêtise \\
La bonne conscience \\
La bonne intention \\
La bonne volonté \\
La bonté \\
L'absence \\
L'absolu \\
L'absolu et le relatif \\
L'abstraction \\
L'abstrait et le concret \\
L'absurde \\
L'abus de pouvoir \\
L'académisme \\
La causalité en histoire \\
La cause efficiente \\
La cause et l'effet \\
La certitude de mourir \\
La chair \\
La chance \\
La cité \\
La coexistence des libertés \\
La cohérence est-elle la norme du vrai ? \\
La cohérence logique est-elle une condition suffisante de la démonstration ? \\
La colère \\
La communauté scientifique \\
La compétence \\
La compréhension \\
La confiance \\
La confiance est-elle une vertu ? \\
La connaissance commune fait-elle obstacle à la vérité ? \\
La connaissance des principes \\
La connaissance est-elle une contemplation ? \\
La connaissance et la croyance \\
La connaissance objective doit-elle s'interdire toute interprétation ? \\
La connaissance objective exclut-elle toute forme de subjectivité ? \\
La connaissance sensible \\
La conscience \\
La conscience a-t-elle des degrés ? \\
La conscience d'autrui est-elle impénétrable ? \\
La conscience de la mort est-elle une condition de la sagesse ? \\
La conscience de soi est-elle une donnée immédiate ? \\
La conscience de soi suppose-t-elle autrui ? \\
La conscience du temps rend-elle l'existence tragique ? \\
La conscience est-elle nécessairement malheureuse ? \\
La conscience est-elle une illusion ? \\
La conscience et l'inconscient \\
La conscience morale \\
La conscience morale n'est-elle que le produit de l'éducation ? \\
La contingence de l'existence \\
La contradiction \\
La contrainte peut-elle être légitime ? \\
La controverse scientifique \\
La convention et l'arbitraire \\
La conviction \\
La corruption \\
La courtoisie \\
La coutume \\
La création \\
La création artistique \\
La création de valeur \\
La crédulité \\
La croyance et la foi \\
La croyance et la raison \\
La croyance peut-elle tenir lieu de savoir ? \\
La croyance religieuse échappe-t-elle à toute logique ? \\
L'action \\
L'action et le risque \\
L'action politique \\
L'actualité \\
La culture \\
La culture est-elle la négation de la nature ? \\
La culture est-elle un luxe ? \\
La culture garantit-elle l'excellence humaine ? \\
La culture générale \\
La culture nous rend-elle plus humains ? \\
La culture nous unit-elle ? \\
La culture rend-elle plus humain ? \\
La culture technique \\
La curiosité \\
La décision \\
La défense de l'intérêt général est-il la fin dernière de la politique ? \\
La démarche scientifique exclut-elle tout recours à l'imagination ? \\
La démocratie, est-ce le pouvoir du plus grand nombre ? \\
La démocratie est-elle la loi du plus fort ? \\
La démocratie est-elle le règne de l'opinion ? \\
La démocratie peut-elle échapper à la démagogie ? \\
La démocratie peut-elle être représentative ? \\
La démonstration \\
La démonstration nous garantit-elle l'accès à la vérité ? \\
La démonstration supprime-t-elle le doute ? \\
La déraison \\
La désobéissance \\
La détermination \\
La dialectique \\
La différence des sexes est-elle un problème philosophique ? \\
La dignité \\
La discipline \\
La discorde \\
La discrétion \\
La disharmonie \\
La distinction \\
La diversité \\
La diversité des opinions conduit-elle à douter de tout ? \\
La division du travail \\
L'admiration \\
La douleur nous apprend-elle quelque chose ? \\
La faiblesse \\
La familiarité \\
La famille \\
La famille est-elle un modèle de société ? \\
La fatigue \\
La fermeté \\
La fête \\
La fiction \\
La fidélité \\
La finalité \\
La fin de l'État \\
La fin de l'histoire \\
La fin du travail \\
La fin et les moyens \\
La finitude \\
La fin justifie-t-elle les moyens ? \\
La foi \\
La foi est-elle aveugle ? \\
La foi est-elle rationnelle ? \\
La folie \\
La fonction \\
La fonction et l'organe \\
La force de la vérité \\
La force de l'esprit \\
La force de l'État est-elle nécessaire à la liberté des citoyens ? \\
La force de l'habitude \\
La force des idées \\
La force du droit \\
La force et le droit \\
La franchise \\
La fraternité \\
La générosité \\
La grandeur \\
La guerre \\
La guerre et la paix \\
La guerre peut-elle être juste ? \\
La haine \\
La haine de la raison \\
La hiérarchie \\
La honte \\
La joie \\
La jurisprudence \\
La juste mesure \\
La justice \\
La justice a-t-elle un fondement rationnel ? \\
La justice est-elle l'affaire de l'État ? \\
La justice est-elle une vertu ? \\
La justice et la force \\
La justice et la loi \\
La justice et la paix \\
La justice et le droit \\
La justice et l'égalité \\
La justice n'est-elle qu'une institution ? \\
La justice n'est-elle qu'un idéal ? \\
La justice peut-elle se passer d'institutions ? \\
La justice sociale \\
La justice suppose-t-elle l'égalité ? \\
La lâcheté \\
La laideur \\
La langue et la parole \\
La légèreté \\
La légitime défense \\
La lettre et l'esprit \\
La liberté \\
La liberté a-t-elle un prix ? \\
La liberté comporte-t-elle des degrés ? \\
La liberté connaît-elle des excès ? \\
La liberté de croire \\
La liberté de l'interprète \\
La liberté de penser \\
La liberté d'expression est-elle nécessaire à la liberté de pensée ? \\
La liberté d'indifférence \\
La liberté du choix \\
La liberté est-elle le pouvoir de refuser ? \\
La liberté et l'égalité sont-elles compatibles ? \\
La liberté et le hasard \\
La liberté et le temps \\
La liberté fait-elle de nous des êtres meilleurs ? \\
La liberté implique-t-elle l'indifférence ? \\
La liberté n'est-elle qu'une illusion ? \\
La liberté nous rend-elle inexcusables ? \\
La liberté peut-elle être prouvée ? \\
La liberté peut-elle faire peur ? \\
La liberté peut-elle se refuser ? \\
La liberté requiert-elle le libre échange ? \\
La liberté s'achète-t-elle ? \\
La liberté se mérite-t-elle ? \\
La libre interprétation \\
L'aliénation \\
La loi \\
La loi dit-elle ce qui est juste ? \\
La loi est-elle une garantie contre l'injustice ? \\
La loi et la coutume \\
La loi et l'ordre \\
La loi peut-elle changer les mœurs ? \\
La loyauté \\
L'altérité \\
L'altruisme \\
L'altruisme n'est-il qu'un égoïsme bien compris ? \\
La machine \\
La main \\
La main et l'esprit \\
La maîtrise de soi \\
La majorité doit-elle toujours l'emporter ? \\
La majorité, force ou droit ? \\
La majorité peut-elle être tyrannique ? \\
La maladie \\
La marginalité \\
La mathématisation du réel \\
La matière \\
La matière est-elle plus facile à connaître que l'esprit ? \\
La matière et la vie \\
La matière et l'esprit \\
La matière n'est-elle que ce que l'on perçoit ? \\
La matière n'est-elle qu'une idée ? \\
La maturité \\
La mauvaise foi \\
La mauvaise volonté \\
L'ambiguïté \\
L'ambiguïté des mots peut-elle être heureuse ? \\
La méchanceté \\
L'âme et le corps sont-ils une seule et même chose ? \\
La méfiance \\
L'âme jouit-elle d'une vie propre ? \\
La mélancolie \\
La mémoire \\
La mémoire collective \\
La mémoire et l'oubli \\
La mesure \\
La mesure du temps \\
La métamorphose \\
L'amitié \\
L'amitié est-elle une vertu ? \\
La modération \\
La modestie \\
La morale a-t-elle à décider de la sexualité ? \\
La morale a-t-elle besoin de la notion de sainteté ? \\
La morale a-t-elle sa place dans l'économie ? \\
La morale consiste-t-elle à respecter le droit ? \\
La morale doit-elle être rationnelle ? \\
La morale est-elle affaire de convention ? \\
La morale est-elle affaire de sentiment ? \\
La morale est-elle condamnée à n'être qu'un champ de bataille ? \\
La morale est-elle désintéressée ? \\
La morale est-elle en conflit avec le désir ? \\
La morale est-elle une affaire de raison ? \\
La morale est-elle une affaire solitaire ? \\
La morale est-elle un fait de culture ? \\
La morale et la politique \\
La morale et la religion visent-elles les mêmes fins ? \\
La morale et les mœurs \\
La morale n'est-elle qu'un ensemble de conventions ? \\
La morale peut-elle se fonder sur les sentiments ? \\
La morale peut-elle s'enseigner ? \\
La morale s'enseigne-t-elle ? \\
La morale s'oppose-t-elle à la politique ? \\
La mort \\
La mort d'autrui \\
La mort de Dieu \\
L'amour de l'art \\
L'amour de la vie \\
L'amour de soi est-il immoral ? \\
L'amour du travail \\
L'amour et l'amitié \\
L'amour et le devoir \\
L'amour et le respect \\
L'amour fou \\
L'amour peut-il être raisonnable ? \\
L'amour peut-il être un devoir ? \\
L'amour propre \\
La multitude \\
L'anachronisme \\
L'analogie \\
L'analyse du langage ordinaire peut-elle avoir un intérêt philosophique ? \\
L'anarchie \\
La nation \\
La nature \\
La nature a-t-elle des droits ? \\
La nature est-elle écrite en langage mathématique ? \\
La nature est-elle prévisible ? \\
La nature est-elle une ressource ? \\
La nature est-elle un modèle ? \\
La nature fait-elle bien les choses ? \\
La nature peut-elle avoir des droits ? \\
La nature peut-elle constituer une norme ? \\
La nature peut-elle être un modèle ? \\
La nature peut-elle nous indiquer ce que nous devons faire ? \\
La nécessité \\
Langage et communication \\
Langage et logique \\
Langage et passions \\
Langage et pensée \\
Langage et pouvoir \\
Langage et société \\
L'angoisse \\
L'animal \\
L'animal et l'homme \\
La non-violence \\
La norme \\
La nostalgie \\
La nouveauté \\
La paix \\
La paix est-elle l'absence de guerres ? \\
La paix est-elle le plus grand des biens ? \\
La paix sociale \\
La paix sociale est-elle le but de la politique ? \\
La parole \\
La parole donnée \\
La parole et l'écriture \\
La parole et le geste \\
La parole intérieure \\
La passion de la connaissance \\
La passion de la liberté \\
La passion de la vérité \\
La passion de l'égalité \\
La passion est-elle immorale ? \\
La passion est-elle l'ennemi de la raison ? \\
La passion exclut-elle la lucidité ? \\
La passivité \\
La patience \\
La pauvreté \\
La pauvreté est-elle une injustice ? \\
La peine \\
La pensée \\
La pensée doit-elle se soumettre aux règles de la logique ? \\
La pensée et la conscience sont-elles une seule et même chose ? \\
La pensée peut-elle se passer de mots ? \\
La perception construit-elle son objet ? \\
La perception de l'espace est-elle innée ou acquise ? \\
La perception est-elle le premier degré de la connaissance ? \\
La perception est-elle une interprétation ? \\
La perception me donne-t-elle le réel ? \\
La perception peut-elle s'éduquer ? \\
La perfection \\
La perfection est-elle désirable ? \\
La permanence \\
La personne et l'individu \\
La peur \\
La peur de la science \\
La philosophie et son histoire \\
La philosophie rend-elle inefficace la propagande ? \\
La pitié \\
La plaisanterie \\
La pluralité des arts \\
La pluralité des interprétations \\
La pluralité des langues \\
La pluralité des religions \\
La police \\
La politesse \\
La politesse est-elle une vertu ? \\
La politique \\
La politique consiste-t-elle à faire cause commune ? \\
La politique est-elle l'affaire des spécialistes ? \\
La politique est-elle l'affaire de tous ? \\
La politique est-elle un art ? \\
La politique est-elle une affaire d'experts ? \\
La politique est-elle une science ? \\
La politique et la guerre \\
La politique et le bonheur \\
La politique n'est-elle que l'art de conquérir et de conserver le pouvoir ? \\
La poursuite de mon intérêt m'oppose-t-elle aux autres ? \\
L'apparence \\
L'apparence est-elle toujours trompeuse ? \\
L'apprentissage \\
L'apprentissage de la liberté \\
La précarité \\
La présence d'esprit \\
La prière \\
L'\emph{a priori} \\
La privation \\
La promesse \\
L'à propos \\
La propriété \\
La propriété et le travail \\
La prudence \\
La pudeur \\
La puissance \\
La punition \\
La pureté \\
La quantité et la qualité \\
La question « qui suis-je » admet-elle une réponse exacte ? \\
La raison \\
La raison a-t-elle pour fin la connaissance ? \\
La raison a-t-elle une histoire ? \\
La raison d'État \\
La raison d'État peut-elle être justifiée ? \\
La raison doit-elle critiquer la croyance ? \\
La raison doit-elle être notre guide ? \\
La raison doit-elle se soumettre au réel ? \\
La raison et le réel \\
La raison et l'expérience \\
La raison et l'irrationnel \\
La raison ne veut-elle que connaître ? \\
La raison peut-elle entrer en conflit avec elle-même ? \\
La raison peut-elle errer ? \\
La raison peut-elle se contredire ? \\
La raison peut-elle servir le mal ? \\
La raison suffisante \\
La rationalité \\
L'arbitraire \\
La réalité des idées \\
La réalité des phénomènes \\
La réalité du désordre \\
La réalité du temps \\
La réalité n'est-elle qu'une construction ? \\
La réalité nourrit-elle la fiction ? \\
La réalité sensible \\
La recherche de la vérité peut-elle être désintéressée ? \\
La recherche du bonheur \\
La recherche du bonheur est-elle un idéal égoïste ? \\
La reconnaissance \\
La réflexion \\
La réfutation \\
La règle et l'exception \\
La régression \\
La religion \\
La religion conduit-elle l'homme au-delà de lui-même ? \\
La religion est-elle fondée sur la peur de la mort ? \\
La religion est-elle l'asile de l'ignorance ? \\
La religion est-elle une affaire privée ? \\
La religion est-elle un instrument de pouvoir ? \\
La religion et la croyance \\
La religion naturelle \\
La religion n'est-elle que l'affaire des croyants ? \\
La religion peut-elle n'être qu'une affaire privée ? \\
La représentation \\
La représentation politique \\
La reproduction \\
La responsabilité \\
La responsabilité politique \\
La responsabilité politique n'est-elle le fait que de ceux qui gouvernent ? \\
La réussite \\
La révolte \\
La révolution \\
L'argent \\
L'argent est-il la mesure de tout échange ? \\
La rigueur \\
La rigueur des lois ? \\
L'art \\
L'art a-t-il besoin de théorie ? \\
L'art a-t-il une histoire ? \\
L'art a-t-il un rôle à jouer dans l'éducation ? \\
L'art change-t-il la vie ? \\
L'art de gouverner \\
L'art de juger \\
L'art de persuader \\
L'art de vivre \\
L'art d'interpréter \\
L'art donne-t-il nécessairement lieu à la production d'une œuvre ? \\
L'art éduque-t-il la perception ? \\
L'art est-il affaire d'apparence ? \\
L'art est-il le règne des apparences ? \\
L'art est-il moins nécessaire que la science ? \\
L'art est-il subversif ? \\
L'art est-il une histoire ? \\
L'art est-il universel ? \\
L'art est-il un luxe ? \\
L'art est-il un moyen de connaître ? \\
L'art est-il un refuge ? \\
L'art et la manière \\
L'art et la morale \\
L'art et la technique \\
L'art et la vie \\
L'art et le jeu \\
L'art et le sacré \\
L'art et l'illusion \\
L'art et l'invisible \\
L'artifice \\
L'artificiel \\
L'artiste a-t-il besoin de modèle ? \\
L'artiste doit-il être de son temps ? \\
L'artiste doit-il se soucier du goût du public ? \\
L'artiste est-il souverain ? \\
L'artiste est-il un travailleur ? \\
L'artiste et l'artisan \\
L'artiste sait-il ce qu'il fait ? \\
L'artiste travaille-t-il ? \\
L'art n'est-il qu'un mode d'expression subjectif ? \\
L'art nous détourne-t-il de la réalité ? \\
L'art nous mène-t-il au vrai ? \\
L'art parachève-t-il la nature ? \\
L'art participe-t-il à la vie politique ? \\
L'art peut-il être conceptuel ? \\
L'art peut-il être réaliste \\
L'art peut-il être sans œuvre ? \\
L'art peut-il ne pas être sacré ? \\
L'art peut-il se passer de règles ? \\
L'art peut-il se passer d'œuvres ? \\
L'art pour l'art \\
L'art s'adresse-t-il à tous ? \\
L'art s'apprend-il ? \\
La ruse \\
La sagesse et la passion \\
La santé \\
La satisfaction \\
La science a-t-elle besoin d'une méthode ? \\
La science a-t-elle le monopole de la raison ? \\
La science commence-t-elle avec la perception ? \\
La science du vivant peut-elle se passer de l'idée de finalité ? \\
La science est-elle le lieu de la vérité ? \\
La science est-elle une connaissance du réel ? \\
La science nous indique-t-elle ce que nous devons faire ? \\
La science permet-elle de comprendre le monde ? \\
La science peut-elle être une métaphysique ? \\
La science peut-elle produire des croyances ? \\
La science peut-elle se passer de l'idée de finalité ? \\
La science politique \\
La science se limite-t-elle à constater les faits ? \\
La sensibilité \\
La sérénité \\
La servitude \\
La servitude volontaire \\
La simplicité \\
La sincérité \\
La société \\
La société civile \\
La société doit-elle reconnaître les désirs individuels ? \\
La société est-elle un organisme ? \\
La société et les échanges \\
La société et l'État \\
La société et l'individu \\
La société fait-elle l'homme ? \\
La société peut-elle être l'objet d'une science ? \\
La société repose-t-elle sur l'altruisme ? \\
La solidarité \\
La solidarité est-elle naturelle ? \\
La solitude \\
La sollicitude \\
La souffrance d'autrui \\
La souffrance d'autrui m'importe-t-elle ? \\
La souffrance peut-elle être un mode de connaissance ? \\
La soumission à l'autorité \\
La souveraineté \\
La souveraineté de l'État \\
La souveraineté peut-elle se partager ? \\
La spontanéité \\
L'association des idées \\
La superstition \\
La sympathie \\
La technique \\
La technique accroît-elle notre liberté ? \\
La technique a-t-elle sa place en politique ? \\
La technique est-elle civilisatrice ? \\
La technique est-elle contre-nature ? \\
La technique est-elle le propre de l'homme ? \\
La technique est-elle neutre ? \\
La technique est-elle un savoir ? \\
La technique et le corps \\
La technique et le travail \\
La technique libère-t-elle les hommes ? \\
La technique ne fait-elle qu'appliquer la science ? \\
La technique ne pose-t-elle que des problèmes techniques ? \\
La technique n'est-elle pour l'homme qu'un moyen ? \\
La technique n'est-elle qu'un outil au service de l'homme ? \\
La technique n'existe-elle que pour satisfaire des besoins ? \\
La technique nous éloigne-t-elle de la nature ? \\
La technique nous libère-t-elle ? \\
La technique nous oppose-t-elle à la nature ? \\
La technique nous permet-elle de comprendre la nature ? \\
La technique peut-elle se déduire de la science ? \\
La technique peut-elle se passer de la science ? \\
La technique sert-elle nos désirs ? \\
La tentation \\
La théorie et la pratique \\
La théorie nous éloigne-t-elle de la réalité ? \\
La théorie scientifique \\
La tolérance \\
La tolérance est-elle une vertu ? \\
La tradition \\
La traduction \\
La transgression \\
L'attente \\
L'attention \\
L'attention caractérise-t-elle la conscience ? \\
La tyrannie des désirs \\
L'audace \\
L'au-delà \\
L'authenticité \\
L'autobiographie \\
L'automatisation \\
L'autonomie \\
L'autoportrait \\
L'autorité \\
L'autorité du droit \\
La valeur \\
La valeur de la vérité \\
La valeur morale de l'amour \\
La valeur morale d'une action se juge-t-elle à ses conséquences ? \\
La vengeance \\
L'avenir a-t-il une réalité ? \\
L'avenir peut-il être objet de connaissance ? \\
La vérification \\
La vérité \\
La vérité a-t-elle une histoire ? \\
La vérité donne-t-elle le droit d'être injuste ? \\
La vérité échappe-t-elle au temps ? \\
La vérité est-elle affaire de cohérence ? \\
La vérité est-elle contraignante ? \\
La vérité est-elle libératrice ? \\
La vérité est-elle une valeur ? \\
La vérité est-elle une ? \\
La vérité historique \\
La vérité peut-elle laisser indifférent ? \\
La vérité peut-elle se définir par le consensus ? \\
La vérité rend-elle heureux ? \\
La vertu \\
La vertu peut-elle être excessive ? \\
La vie de l'esprit \\
La vie de plaisirs \\
La vie en société impose-t-elle de n'être pas soi-même ? \\
La vie est-elle sacrée ? \\
La vie heureuse \\
La vie intérieure \\
La vie morale \\
La vie peut-elle être objet de science ? \\
La vie psychique \\
La vie sauvage \\
La vie sociale \\
La vie sociale est-elle toujours conflictuelle ? \\
La violence \\
La violence est-elle toujours destructrice ? \\
La violence peut-elle avoir raison ? \\
La violence peut-elle être gratuite ? \\
La violence verbale \\
La virtuosité \\
La vision peut-elle être le modèle de toute connaissance ? \\
La vocation \\
La voix de la raison \\
La volonté peut-elle être générale ? \\
La volonté peut-elle nous manquer ? \\
La vue et le toucher \\
Le bavardage \\
Le beau est-il toujours moral ? \\
Le beau et l'agréable \\
Le beau et le bien \\
Le beau et le sublime \\
Le beau et l'utile \\
Le beau geste \\
Le bénéfice du doute \\
Le besoin de métaphysique est-il un besoin de connaissance ? \\
Le besoin de reconnaissance \\
Le besoin de théorie \\
Le besoin et le désir \\
Le bien commun est-il une illusion ? \\
Le bien commun et l'intérêt de tous \\
Le bien est-ce l'utile ? \\
Le bien est-il relatif ? \\
Le bien et le beau \\
Le bien et les biens \\
Le bien public \\
Le bonheur collectif \\
Le bonheur des sens \\
Le bonheur est-il au nombre de nos devoirs ? \\
Le bonheur est-il dans l'inconscience ? \\
Le bonheur est-il le but de la politique ? \\
Le bonheur est-il un but politique ? \\
Le bonheur est-il une affaire privée ? \\
Le bonheur est-il un idéal ? \\
Le bonheur et la raison \\
Le bonheur et la technique \\
Le bonheur n'est-il qu'une idée ? \\
Le bonheur peut-il être collectif ? \\
Le bonheur se calcule-t-il ? \\
Le calendrier \\
Le caractère \\
Le caractère sacré de la vie \\
Le cas de conscience \\
Le cerveau et la pensée \\
Le cerveau pense-t-il ? \\
L'échange constitue-t-il un lien social ? \\
L'échange économique fonde-t-il la société humaine \\
L'échange et l'usage \\
L'échange n'a-t-il de fondement qu'économique ? \\
L'échange ne porte-t-il que sur les choses ? \\
L'échange peut-il être désintéressé ? \\
Le châtiment \\
Le chef \\
Le chef d'œuvre \\
Le chef-d'œuvre \\
Le choix \\
Le choix et la liberté \\
Le citoyen \\
Le commencement \\
Le commerce \\
Le commerce adoucit-il les mœurs ? \\
Le commerce des idées \\
Le commerce unit-il les hommes ? \\
Le commun et le propre \\
Le concept \\
Le concept et l'exemple \\
Le conflit est-il une maladie sociale ? \\
L'économie et la politique \\
Le consentement \\
Le contentement \\
Le contrat \\
Le contrat de travail \\
Le contrat est-il au fondement de la politique ? \\
Le corps est-il négociable ? \\
Le corps et l'âme \\
Le corps et l'esprit \\
Le corps impose-t-il des perspectives ? \\
Le corps n'est-il qu'un mécanisme ? \\
Le corps obéit-il à l'esprit ? \\
Le corps politique \\
Le cosmopolitisme \\
Le courage \\
Le crime \\
L'écriture \\
Le dedans et le dehors \\
Le défaut \\
Le désespoir \\
Le désintéressement \\
Le désir d'absolu \\
Le désir de l'autre \\
Le désir de savoir \\
Le désir de savoir est-il naturel ? \\
Le désir d'éternité \\
Le désir de vérité \\
Le désir du bonheur est-il universel ? \\
Le désir est-il aveugle ? \\
Le désir est-il le signe d'un manque ? \\
Le désir est-il nécessairement l'expression d'un manque ? \\
Le désir et la culpabilité \\
Le désir et la loi \\
Le désir et le besoin \\
Le désir et le mal \\
Le désir et le manque \\
Le désir et le rêve \\
Le désir et le temps \\
Le désir et le travail \\
Le désir et l'interdit \\
Le désir n'est-il que manque ? \\
Le désir peut-il ne pas avoir d'objet ? \\
Le désir peut-il nous rendre libre ? \\
Le despotisme \\
Le destin \\
Le devoir \\
Le devoir est-il l'expression de la contrainte sociale ? \\
Le devoir et le bonheur \\
Le devoir rend-il libre ? \\
Le devoir supprime-t-il la liberté ? \\
Le diable \\
Le dialogue \\
Le dialogue suffit-il à rompre la solitude ? \\
Le discernement \\
Le divertissement \\
Le don \\
Le don de soi \\
Le don est-il une modalité de l'échange ? \\
Le don et la dette \\
Le don et l'échange \\
Le doute \\
Le droit \\
Le droit à la différence met-il en péril l'égalité des droits ? \\
Le droit à la paresse \\
Le droit au bonheur \\
Le droit au travail \\
Le droit de mentir \\
Le droit de propriété \\
Le droit de résistance \\
Le droit divin \\
Le droit doit-il être indépendant de la morale ? \\
Le droit du plus faible \\
Le droit est-il facteur de paix ? \\
Le droit et la convention \\
Le droit et la force \\
Le droit et la liberté \\
Le droit et la loi \\
Le droit et la morale \\
Le droit n'est-il qu'une justice par défaut ? \\
Le droit peut-il échapper à l'histoire ? \\
Le droit peut-il être naturel ? \\
Le droit peut-il se passer de la morale ? \\
Le droit positif \\
Le droit sert-il à établir l'ordre ou la justice ? \\
L'éducation artistique \\
L'éducation esthétique \\
Le fait \\
Le fait divers \\
Le fait et l'événement \\
Le fantasme \\
L'efficacité \\
L'effort \\
L'effort moral \\
Le fini et l'infini \\
Le for intérieur \\
L'égalité \\
L'égalité est-elle toujours juste ? \\
Légalité et légitimité \\
Légalité et moralité \\
Le génie \\
Le génie est-il la marque de l'excellence artistique ? \\
Le génie et la règle \\
Le génie et le savant \\
Le goût \\
Le goût de la liberté \\
Le goût des autres \\
Le goût s'éduque-t-il ? \\
Le gouvernement des meilleurs \\
Le hasard \\
Le hasard et la nécessité \\
Le hors-la-loi \\
Le je et le tu \\
Le jeu \\
Le jeu et le divertissement \\
Le jeu et le hasard \\
Le juge \\
Le jugement \\
Le jugement dernier \\
Le juste et le légal \\
Le langage \\
Le langage du corps \\
Le langage est-il d'essence poétique ? \\
Le langage est-il le lieu de la vérité ? \\
Le langage est-il logique ? \\
Le langage est-il une prise de possession des choses ? \\
Le langage est-il un instrument de connaissance ? \\
Le langage est-il un obstacle pour la pensée ? \\
Le langage masque-t-il la pensée ? \\
Le langage rend-il l'homme plus puissant ? \\
Le langage traduit-il la pensée ? \\
Le langage trahit-il la pensée ? \\
L'élection \\
Le législateur \\
Le libre-arbitre \\
Le libre échange \\
Le lien social \\
Le livre de la nature \\
Le loisir \\
Le luxe \\
Le maître \\
Le mal \\
Le malentendu \\
Le mal être \\
Le mal existe-t-il ? \\
Le malheur \\
Le malheur est-il injuste ? \\
L'émancipation \\
Le marché \\
Le marché du travail \\
Le mariage \\
Le matérialisme \\
Le médiat et l'immédiat \\
Le meilleur est-il l'ennemi du bien ? \\
Le mensonge \\
Le mensonge peut-il être au service de la vérité ? \\
Le mépris \\
Le mérite \\
Le métier \\
Le mien et le tien \\
Le moi \\
Le moi est-il haïssable ? \\
Le moi est-il une fiction ? \\
Le moi est-il une illusion ? \\
Le moindre mal \\
Le moi n'est-il qu'une fiction ? \\
Le monde du travail \\
Le monde se réduit-il à ce que nous en voyons ? \\
Le monstre \\
Le monstrueux \\
Le mot et le geste \\
L'émotion \\
Le mouvement \\
Le multiple et l'un \\
Le musée \\
Le mystère \\
Le naturel et le fabriqué \\
L'encyclopédie \\
L'enfance \\
L'enfance est-elle en nous ce qui doit être abandonné ? \\
L'enfant \\
L'engagement \\
L'ennemi \\
L'ennui \\
Le non-être \\
L'enquête empirique rend-elle la métaphysique inutile ? \\
L'entendement et la volonté \\
L'envie \\
Le pardon \\
Le pardon et l'oubli \\
Le passé a-t-il plus de réalité que l'avenir ? \\
Le passé est-il ce qui a disparu ? \\
Le passé et le présent \\
Le passé existe-t-il ? \\
Le paysage \\
Le personnage et la personne \\
Le pessimisme \\
Le peuple \\
Le peuple et la nation \\
Le peuple peut-il se tromper ? \\
L'éphémère \\
Le phénomène \\
Le plaisir \\
Le plaisir de parler \\
Le plaisir des sens \\
Le plaisir esthétique \\
Le plaisir esthétique peut-il se partager ? \\
Le plaisir est-il tout le bonheur ? \\
Le plaisir et la peine \\
Le plaisir peut-il être partagé ? \\
Le plaisir suffit-il au bonheur ? \\
Le poids de la société \\
Le poids du passé \\
Le possible et le réel \\
Le pouvoir \\
Le pouvoir corrompt-il toujours ? \\
Le pouvoir des mots \\
Le pouvoir et l'autorité \\
Le présent \\
L'épreuve du réel \\
Le principe \\
Le principe de raison suffisante \\
Le privé et le public \\
Le prix du travail \\
Le probable \\
Le profit \\
Le progrès \\
Le progrès est-il un mythe ? \\
Le progrès moral \\
Le progrès technique peut-il être aliénant ? \\
Le projet \\
Le propre du vivant est-il de tomber malade ? \\
Le provisoire \\
Le public et le privé \\
L e pur et l'impur \\
L'équité \\
L'équivocité \\
L'équivoque \\
Le quotidien \\
Le raffinement \\
Le rationnel et le raisonnable \\
Le réalisme \\
Le reconnaissance \\
Le réel \\
Le réel est-il ce que nous expérimentons ? \\
Le réel est-il ce que nous percevons ? \\
Le réel est-il ce qui apparaît ? \\
Le réel est-il ce qui est perçu ? \\
Le réel est-il inaccessible ? \\
Le réel est-il l'objet de la science ? \\
Le réel est-il objet d'interprétation ? \\
Le réel est-il rationnel ? \\
Le réel et la fiction \\
Le réel et le matériel \\
Le réel et le possible \\
Le réel et le virtuel \\
Le réel et le vrai \\
Le réel et l'imaginaire \\
Le réel et l'irréel \\
Le réel n'est-il qu'un ensemble de contraintes ? \\
Le réel résiste-t-il à la connaissance ? \\
Le réel se limite-t-il à ce que font connaître les théories scientifiques ? \\
Le regard \\
Le relativisme \\
Le renoncement \\
Le respect \\
Le ressentiment \\
Le rien \\
Le risque \\
Le risque de la liberté \\
Le rôle des théories est-il d'expliquer ou de décrire ? \\
L'erreur \\
L'erreur et la faute \\
L'erreur et l'illusion \\
Le rythme \\
Le sacré \\
Le sacré et le profane \\
Le sacrifice \\
Les acteurs de l'histoire en sont-ils les auteurs ? \\
Les affects sont-ils déraisonnables ? \\
Le sage a-t-il besoin d'autrui ? \\
Les âges de la vie \\
Le salaire \\
Le salut vient-il de la raison ? \\
Les animaux ont-ils des droits ? \\
Les apparences sont-elles toujours trompeuses ? \\
Le sauvage \\
Le savant et l'ignorant \\
Le savoir exclut-il toute forme de croyance ? \\
Le savoir-faire \\
Le savoir rend-il libre ? \\
Les bêtes travaillent-elles ? \\
Les catégories \\
Les causes et les signes \\
Les classes sociales \\
L'esclavage des passions \\
Les coïncidences ont-elles des causes ? \\
Les commencements \\
Les conditions d'existence \\
Les conflits menacent-ils la société ? \\
Les considérations morales ont-elles leur place en politique ? \\
Les devoirs de l'homme varient-ils selon la culture ? \\
Les devoirs de l'homme varient-ils selon les cultures ? \\
Les devoirs du citoyen \\
Les droits de l'individu \\
Les échanges \\
Les échanges économiques sont-ils facteurs de paix ? \\
Les échanges favorisent-ils la paix ? \\
Les échanges sont-ils facteurs de paix ? \\
Le secret \\
Le sens caché \\
Le sens commun \\
Le sens du devoir \\
Le sensible et l'intelligible \\
Le sensible peut-il être connu ? \\
Le sentiment \\
Le sentiment de liberté \\
Le sentiment d'injustice \\
Le sentiment d'injustice est-il naturel ? \\
Le sentiment du juste et de l'injuste \\
Les êtres vivants sont-ils des machines ? \\
Les faits et les valeurs \\
Les faits existent-ils indépendamment de leur établissement par l'esprit humain ? \\
Les faits parlent-ils d'eux-mêmes ? \\
Les faits peuvent-ils faire autorité ? \\
Les fins de la culture \\
Les formes du vivant \\
Les générations \\
Les habitudes nous forment-elles ? \\
Les hommes naissent-ils libres ? \\
Les hommes ont-ils besoin de maîtres ? \\
Les hommes savent-ils ce qu'ils désirent ? \\
Les hommes sont-ils seulement le produit de leur culture ? \\
Les idées et les choses \\
Les idées ont-elles une existence éternelle ? \\
Le signe \\
Le silence \\
Le silence a-t-il un sens ? \\
Le simple \\
Le simple et le complexe \\
Les inégalités menacent-elles la société ? \\
Les inégalités sociales sont-elles naturelles ? \\
Les intentions et les actes \\
Les leçons de l'histoire \\
Les limites de la connaissance \\
Les limites de la raison \\
Les limites de l'expérience \\
Les lois \\
Les machines nous rendent-elles libres ? \\
Les mathématiques parlent-elles du réel ? \\
Les mathématiques sont-elles un instrument ? \\
Les mœurs \\
Les monstres \\
Les mots disent-ils les choses ? \\
Les mots et les concepts \\
Les mots expriment-ils les choses ? \\
Les mots parviennent-ils à tout exprimer ? \\
Les mots sont-ils trompeurs ? \\
Les moyens et les fins \\
Les œuvres d'art sont-elles des réalités comme les autres ? \\
Le soi et le je \\
Le solipsisme \\
Le sommeil \\
Le souci de soi \\
Les outils \\
Le souverain bien \\
L'espace nous sépare-t-il ? \\
Les paroles et les actes \\
L'espérance \\
Les personnages de fiction peuvent-ils avoir une réalité ? \\
Les peuples font-ils l'histoire ? \\
Les preuves de la liberté \\
Les principes \\
Les principes de la morale dépendent-ils de la culture ? \\
L'esprit \\
L'esprit critique \\
L'esprit dépend-il du corps ? \\
L'esprit de système \\
L'esprit domine-t-il la matière ? \\
L'esprit est-il mieux connu que le corps ? \\
L'esprit est-il objet de science ? \\
L'esprit est-il plus difficile à connaître que la matière ? \\
L'esprit est-il une partie du corps ? \\
L'esprit humain progresse-t-il ? \\
Les progrès techniques constituent-ils des progrès de la civilisation ? \\
Les raisons de croire \\
Les scélérats peuvent-ils être heureux ? \\
Les sciences décrivent-elles le réel ? \\
L'essence et l'existence \\
Les sens jugent-ils ? \\
Les sens sont-ils source d'illusion ? \\
Les sens sont-ils trompeurs ? \\
L'estime de soi \\
Le style \\
Le sujet \\
Le sujet et l'individu \\
Le sujet n'est-il qu'une fiction ? \\
Le sujet peut-il s'aliéner par un libre choix ? \\
Les valeurs morales ont-elles leur origine dans la raison ? \\
Les valeurs universelles \\
Les vérités empiriques \\
Les vérités éternelles \\
Les vérités sont-elles intemporelles ? \\
Les vertus du commerce \\
Les vivants peuvent-ils se passer des morts ? \\
Le système \\
Le système des arts \\
Le tact \\
Le talent \\
L'État \\
L'État contribue-t-il à pacifier les relations entre les hommes ? \\
L'État de droit \\
L'état de nature \\
L'État doit-il être fort ? \\
L'État doit-il être le plus fort ? \\
L'État doit-il reconnaître des limites à sa puissance ? \\
L'État doit-il se mêler de religion ? \\
L'État doit-il se préoccuper des arts ? \\
L'État doit-il se préoccuper du bonheur des citoyens ? \\
L'État est-il au service de la société ? \\
L'État est-il l'ennemi de la liberté ? \\
L'État est-il un tiers impartial ? \\
L'État est-il un « monstre froid » ? \\
L'État et la justice \\
L'État et la nation \\
L'État et la société \\
L'État et le droit \\
L'État et le peuple \\
L'État et les communautés \\
L'État et l'individu \\
L'État nous rend-il meilleurs ? \\
L'État peut-il poursuivre une autre fin que sa propre puissance ? \\
Le technicien n'est-il qu'un exécutant ? \\
Le témoignage \\
Le temps \\
Le temps du bonheur \\
Le temps est-il destructeur ? \\
Le temps est-il en nous ou hors de nous ? \\
Le temps est-il notre allié ? \\
Le temps est-il une contrainte ? \\
Le temps est-il une réalité ? \\
Le temps et l'espace \\
Le temps libre \\
Le temps n'est-il pour l'homme que ce qui le limite ? \\
Le temps nous appartient-il ? \\
Le temps nous est-il compté ? \\
Le temps perdu \\
L'éternel retour \\
L'éternité \\
L'étonnement \\
Le toucher \\
Le travail \\
Le travail a-t-il une valeur morale ? \\
Le travail est-il le propre de l'homme ? \\
Le travail est-il libérateur ? \\
Le travail est-il nécessaire au bonheur ? \\
Le travail est-il toujours une activité productrice ? \\
Le travail est-il un besoin ? \\
Le travail est-il une marchandise ? \\
Le travail est-il une valeur ? \\
Le travail est-il un rapport naturel de l'homme à la nature ? \\
Le travail et la propriété \\
Le travail et la technique \\
Le travail fonde-t-il la propriété ? \\
Le travaille libère-t-il ? \\
Le travail manuel \\
Le travail manuel est-il sans pensée ? \\
Le travail unit-il ou sépare-t-il les hommes ? \\
L'être et le néant \\
L'être humain est-il la mesure de toute chose ? \\
L'être imaginaire et l'être de raison \\
Le tribunal de l'histoire \\
Le troc \\
L'étude de l'histoire conduit-elle à désespérer l'homme ? \\
Le tyran \\
L'événement \\
Le vertige de la liberté \\
Le vice et la vertu \\
Le vide et le plein \\
L'évidence \\
L'évidence et la démonstration \\
L'évidence se passe-t-elle de démonstration ? \\
Le visible et l'invisible \\
Le vivant \\
Le vivant est-il réductible au physico-chimique ? \\
Le vivant est-il un objet de science comme un autre ? \\
Le vivant et la machine \\
Le vivant et la mort \\
Le vivant et la sensibilité \\
Le vivant et la technique \\
Le vivant et le vécu \\
Le vivant et l'expérimentation \\
Le vivant et l'inerte \\
Le volontaire et l'involontaire \\
Le voyage \\
Le vrai a-t-il une histoire ? \\
Le vrai et le bien \\
Le vrai et le vraisemblable \\
Le vrai se réduit-il à ce qui est vérifiable ? \\
L'exactitude \\
L'excellence des sens \\
L'exception \\
L'excès \\
L'excuse \\
L'existence \\
L'existence a-t-elle un sens ? \\
L'existence du mal met-elle en échec la raison ? \\
L'existence du passé \\
L'existence est-elle vaine ? \\
L'existence et le temps \\
L'existence se laisse-t-elle penser ? \\
L'expérience a-t-elle le même sens dans toutes les sciences ? \\
L'expérience d'autrui nous est-elle utile ? \\
L'expérience de l'injustice \\
L'expérience démontre-t-elle quelque chose ? \\
L'expérience de pensée \\
L'expérience et la sensation \\
L'expérience imaginaire \\
L'expérience instruit-elle ? \\
L'expérience morale \\
L'expérience peut-elle avoir raison des principes ? \\
L'expérience peut-elle contredire la théorie ? \\
L'expérience rend-elle raisonnable ? \\
L'expérience rend-elle responsable ? \\
L'expérience suffit-elle pour établir une vérité ? \\
L'expression \\
L'extinction du désir \\
L'habitude \\
L'habitude est-elle notre guide dans la vie ? \\
L'harmonie \\
L'hérédité \\
L'héritage \\
L'histoire a-t-elle une fin ? \\
L'histoire a-t-elle un sens ? \\
L'histoire des sciences \\
L'histoire du droit est-elle celle du progrès de la justice ? \\
L'histoire est-elle la science du passé ? \\
L'histoire est-elle rationnelle ? \\
L'histoire est-elle une explication ou une justification du passé ? \\
L'histoire est-elle une science ? \\
L'histoire n'est-elle que la connaissance du passé ? \\
L'histoire se répète-t-elle ? \\
L'homme aime-t-il la justice pour elle-même ? \\
L'homme a-t-il besoin de l'art ? \\
L'homme a-t-il une place dans la nature ? \\
L'homme des droits de l'homme \\
L'homme d'État \\
L'homme est-il chez lui dans l'univers ? \\
L'homme est-il l'artisan de sa dignité ? \\
L'homme est-il le sujet de son histoire ? \\
L'homme est-il un animal comme un autre ? \\
L'homme est-il un animal dénaturé ? \\
L'homme est-il un animal politique ? \\
L'homme est-il un animal rationnel ? \\
L'homme est-il un animal social ? \\
L'homme est-il un animal ? \\
L'homme est-il un corps pensant ? \\
L'homme est-il un loup pour l'homme ? \\
L'homme et la machine \\
L'homme et l'animal \\
L'homme et le citoyen \\
L'homme injuste peut-il être heureux ? \\
L'homme se réalise-t-il dans le travail ? \\
L'honneur \\
L'honneur ? \\
L'hospitalité \\
L'humanité \\
L'humanité est-elle aimable ? \\
L'humilité \\
L'hypothèse \\
Liberté d'agir, liberté de penser \\
Liberté et courage \\
Liberté et déterminisme \\
Liberté et éducation \\
Liberté et égalité \\
Liberté et engagement \\
Liberté et existence \\
Liberté et indépendance \\
Liberté et libération \\
Liberté et licence \\
Liberté et nécessité \\
Liberté et pouvoir \\
Liberté et responsabilité \\
Liberté et savoir \\
Liberté et sécurité \\
Liberté et solitude \\
Libre arbitre et déterminisme sont-ils compatibles ? \\
Libre et heureux \\
L'idéal \\
L'idéal et le réel \\
L'idée de progrès \\
L'idée d'organisme \\
L'identité \\
L'identité personnelle \\
L'idéologie \\
L'idiot \\
L'ignorance \\
L'ignorance est-elle préférable à l'erreur ? \\
L'ignorance peut-elle être une excuse ? \\
L'illusion \\
L'imagination dans les sciences \\
L'imagination enrichit-elle la connaissance ? \\
L'imitation \\
L'immatériel \\
L'immédiat \\
L'immortalité \\
L'immortalité de l'âme \\
L'impartialité \\
L'impensable \\
L'imperceptible \\
L'impossible \\
L'imprévisible \\
L'inaperçu \\
L'inattendu \\
L'incertitude \\
L'incertitude interdit-elle de raisonner ? \\
L'inconscience \\
L'inconscient \\
L'inconscient est-il dans l'âme ou dans le corps ? \\
L'inconscient est-il une excuse ? \\
L'inconscient et l'involontaire \\
L'inconscient n'est-il qu'une hypothèse ? \\
L'inconscient peut-il se manifester ? \\
L'indécidable \\
L'indéfini \\
L'indémontrable \\
L'indescriptible \\
L'indésirable \\
L'indice et la preuve \\
L'indicible et l'impensable \\
L'indicible et l'ineffable \\
L'indifférence \\
L'indifférence peut-elle être une vertu ? \\
L'indignation \\
L'indignité \\
L'individu a-t-il des droits ? \\
L'individu et l'espèce \\
L'induction \\
L'indulgence \\
L'ineffable et l'innommable \\
L'inestimable \\
L'inexistant \\
L'infini et l'indéfini \\
L'infinité de l'univers a-t-elle de quoi nous effrayer ? \\
L'ingratitude \\
L'inhumain \\
L'inimaginable \\
L'injustifiable \\
L'innocence \\
L'innovation \\
L'inquiétude \\
L'inquiétude peut-elle définir l'existence humaine ? \\
L'inquiétude peut-elle devenir l'existence humaine ? \\
L'insatisfaction \\
L'insouciance \\
L'instant \\
L'instant et la durée \\
L'instruction est-elle facteur de moralité ? \\
L'instrument \\
L'intellect \\
L'intelligence \\
L'intelligence artificielle \\
L'intelligence de la technique \\
L'intemporel \\
L'interdit est-il au fondement de la culture ? \\
L'intérêt constitue-t-il l'unique lien social ? \\
L'intérêt de la société l'emporte-t-il sur celui des individus ? \\
L'intérêt de l'État \\
L'intérêt est-il le principe de tout échange ? \\
L'intérêt général est-il la somme des intérêts particuliers ? \\
L'intériorité \\
L'interprétation \\
L'interprétation est-elle un art ? \\
L'interprétation est-elle une activité sans fin ? \\
L'interprète et le créateur \\
L'interprète sait-il ce qu'il cherche ? \\
L'intersubjectivité \\
L'intimité \\
L'intolérable \\
L'introspection \\
L'intuition \\
L'intuition intellectuelle \\
L'inutile \\
L'inutile est-il sans valeur ? \\
L'invention \\
L'invention et la découverte \\
L'invention technique \\
L'invisible \\
L'involontaire \\
Lire et écrire \\
L'irrationnel \\
L'irrationnel est-il pensable ? \\
L'irrationnel est-il toujours absurde ? \\
L'irrationnel existe-t-il ? \\
L'irréfléchi \\
L'irréfutable \\
L'irréparable \\
L'irrésolution \\
L'irresponsabilité \\
L'irréversibilité \\
L'obéissance \\
L'obéissance est-elle compatible avec la liberté ? \\
L'objectivité \\
L'objectivité de l'historien \\
L'objet et la chose \\
L'obligation \\
L'obscur \\
L'observation \\
L'occasion \\
L'œuvre \\
L'œuvre d'art a-t-elle un sens ? \\
L'œuvre d'art donne-t-elle à penser ? \\
L'œuvre d'art échappe-t-elle au temps ? \\
L'œuvre d'art échappe-t-elle nécessairement au temps ? \\
L'œuvre d'art est-elle une marchandise ? \\
L'œuvre d'art est-elle un objet d'échange ? \\
L'œuvre d'art est-elle un symbole ? \\
L'œuvre d'art instruit-elle ? \\
Loisir et oisiveté \\
L'oisiveté \\
L'omniscience \\
L'opinion a-t-elle nécessairement tort ? \\
L'opinion est-elle un savoir ? \\
L'ordre des choses \\
L'ordre du monde \\
L'ordre et le désordre \\
L'ordre social \\
L'ordre social peut-il être juste ? \\
L'organique \\
L'organique et l'inorganique \\
L'organisme \\
L'originalité \\
L'origine des idées \\
L'oubli \\
L'oubli et le pardon \\
L'outil \\
L'outil et la machine \\
L'ouverture d'esprit \\
L'unanimité est-elle un critère de vérité ? \\
L'unité de l'État \\
L'universel \\
L'universel et le particulier \\
L'urbanité \\
L'urgence \\
L'utile et l'agréable \\
L'utile et le beau \\
L'utile et l'inutile \\
L'utilité \\
L'utopie et l'idéologie \\
Machine et organisme \\
Maître et disciple \\
Maîtrise et puissance \\
Mal et liberté \\
Ma liberté s'arrête-t-elle où commence celle des autres ? \\
Mémoire et souvenir \\
Mentir \\
Modèle et copie \\
Mon corps \\
Mon corps est-il naturel ? \\
Mon corps fait-il obstacle à ma liberté ? \\
Mon prochain est-il mon semblable ? \\
Montrer et démontrer \\
Morale et calcul \\
Morale et économie \\
Morale et liberté \\
Moralité et utilité \\
Naît-on sujet ou le devient-on ? \\
N'apprend-on que par l'expérience ? \\
Narration et identité \\
Nature et artifice \\
Nature et convention \\
Nature et histoire \\
Nature et loi \\
Nature et morale \\
N'échange-t-on que ce qui a de la valeur ? \\
N'échange-t-on que par intérêt ? \\
Ne faire que son devoir \\
Ne rien devoir à personne \\
Ne veut-on que ce qui est désirable ? \\
Ne vit-on bien qu'avec ses amis ? \\
N'interprète-t-on que ce qui est équivoque ? \\
Nommer \\
Normes et valeurs \\
Nos convictions morales sont-elles le simple reflet de notre temps ? \\
Nos pensées dépendent-elles de nous ? \\
Nos pensées sont-elles entièrement en notre pouvoir ? \\
Notre liberté de pensée a-t-elle des limites ? \\
Notre rapport au monde est-il essentiellement technique ? \\
Nouveauté et tradition \\
N'y a-t-il de devoirs qu'envers autrui ? \\
N'y a-t-il de droit qu'écrit ? \\
N'y a-t-il de foi que religieuse ? \\
N'y a-t-il de réalité que de l'individuel ? \\
N'y a-t-il de savoir que livresque ? \\
N'y a-t-il de science que de ce qui est mathématisable ? \\
N'y a-t-il de vérité que vérifiable ? \\
N'y a-t-il de vérités que scientifiques ? \\
N'y a-t-il de vrai que le vérifiable ? \\
Obéissance et liberté \\
Obéissance et soumission \\
Objectivé et subjectivité \\
Observer et expérimenter \\
Observer et interpréter \\
Opinion et ignorance \\
Ordre et désordre \\
Ordre et justice \\
Ordre et liberté \\
Organisme et milieu \\
Origine et fondement \\
Où commence la violence ? \\
Où commence ma liberté ? \\
Où est l'esprit ? \\
Outil et machine \\
Outil et organe \\
Paraître \\
Parier \\
Par le langage, peut-on agir sur la réalité ? \\
Parler, est-ce agir ? \\
Parler, est-ce communiquer ? \\
Parler, est-ce donner sa parole ? \\
Parler et agir \\
Parler, n'est-ce que désigner ? \\
Parole et pouvoir \\
Passions et intérêts \\
Penser, est-ce calculer ? \\
Penser, est-ce désobéir ? \\
Penser, est-ce se parler à soi-même ? \\
Penser et imaginer \\
Penser et parler \\
Penser et savoir \\
Penser et sentir \\
Penser l'avenir \\
Penser le changement \\
Penser par soi-même \\
Penser par soi-même, est-ce être l'auteur de ses pensées ? \\
Penser peut-il nous rendre heureux ? \\
Pense-t-on jamais seul ? \\
Perception et connaissance \\
Perception et imagination \\
Perception et sensation \\
Percevoir \\
Percevoir, est-ce interpréter ? \\
Percevoir, est-ce savoir ? \\
Percevoir, est-ce s'ouvrir au monde ? \\
Percevoir et concevoir \\
Percevoir et imaginer \\
Perçoit-on le réel tel qu'il est ? \\
Perçoit-on les choses comme elles sont ? \\
Permanence et identité \\
Personne et individu \\
Peuple et multitude \\
Peut-il y avoir conflit entre nos devoirs ? \\
Peut-il y avoir des échanges équitables ? \\
Peut-il y avoir des lois de l'histoire ? \\
Peut-il y avoir des lois injustes ? \\
Peut-il y avoir des modèles en morale ? \\
Peut-il y avoir des vérités partielles ? \\
Peut-il y avoir esprit sans corps ? \\
Peut-il y avoir savoir-faire sans savoir ? \\
Peut-il y avoir une société sans État ? \\
Peut-il y avoir un État mondial ? \\
Peut-il y avoir une vérité en art ? \\
Peut-il y avoir un langage universel ? \\
Peut-on aimer l'autre tel qu'il est ? \\
Peut-on aimer sans perdre sa liberté ? \\
Peut-on aimer son prochain comme soi-même ? \\
Peut-on aimer une œuvre d'art sans la comprendre ? \\
Peut-on apprendre à mourir ? \\
Peut-on assimiler le vivant à une machine ? \\
Peut-on atteindre une certitude ? \\
Peut-on attribuer à chacun son dû ? \\
Peut-on avoir de bonnes raisons de ne pas dire la vérité ? \\
Peut-on avoir raison contre les faits ? \\
Peut-on avoir raison contre tout le monde ? \\
Peut-on cesser de croire ? \\
Peut-on cesser de désirer ? \\
Peut-on changer le monde ? \\
Peut-on changer ses désirs ? \\
Peut-on choisir ses désirs ? \\
Peut-on commander à la nature ? \\
Peut-on communiquer ses perceptions à autrui ? \\
Peut-on communiquer son expérience ? \\
Peut-on comparer les cultures ? \\
Peut-on comparer l'organisme à une machine ? \\
Peut-on comprendre le présent ? \\
Peut-on comprendre un acte que l'on désapprouve ? \\
Peut-on concevoir une humanité sans art ? \\
Peut-on concevoir une science sans expérience ? \\
Peut-on concevoir une société juste sans que les hommes ne le soient ? \\
Peut-on concevoir une société sans État ? \\
Peut-on concilier bonheur et liberté ? \\
Peut-on connaître les choses telles qu'elles sont ? \\
Peut-on connaître l'esprit ? \\
Peut-on connaître le vivant sans le dénaturer ? \\
Peut-on connaître le vivant sans recourir à la notion de finalité ? \\
Peut-on connaître l'individuel ? \\
Peut-on connaître par intuition ? \\
Peut-on contredire l'expérience ? \\
Peut-on craindre la liberté ? \\
Peut-on critiquer la démocratie ? \\
Peut-on croire en rien ? \\
Peut-on décider d'être heureux ? \\
Peut-on définir la morale comme l'art d'être heureux ? \\
Peut-on définir le bonheur ? \\
Peut-on délimiter le réel ? \\
Peut-on dépasser la subjectivité ? \\
Peut-on désirer ce qui est ? \\
Peut-on désirer l'impossible ? \\
Peut-on désobéir aux lois ? \\
Peut-on désobéir par devoir ? \\
Peut-on dire ce que l'on pense ? \\
Peut-on dire d'un homme qu'il est supérieur à un autre homme ? \\
Peut-on dire la vérité ? \\
Peut-on dire le singulier ? \\
Peut-on dire que les machines travaillent pour nous ? \\
Peut-on dire que les mots pensent pour nous ? \\
Peut-on dire que l'humanité progresse ? \\
Peut-on dire que toutes les croyances se valent ? \\
Peut-on discuter des goûts et des couleurs ? \\
Peut-on donner un sens à son existence ? \\
Peut-on douter de sa propre existence ? \\
Peut-on douter de soi ? \\
Peut-on douter de toute vérité ? \\
Peut-on douter de tout ? \\
Peut-on échanger des idées ? \\
Peut-on échapper à son temps ? \\
Peut-on éduquer la conscience ? \\
Peut-on en appeler à la conscience contre l'État ? \\
Peut-on être apolitique ? \\
Peut-on être dans le présent ? \\
Peut-on être en conflit avec soi-même ? \\
Peut-on être esclave de soi-même ? \\
Peut-on être heureux dans la solitude ? \\
Peut-on être heureux sans être sage ? \\
Peut-on être heureux sans s'en rendre compte ? \\
Peut-on être ignorant ? \\
Peut-on être juste sans être impartial ? \\
Peut-on être méchant volontairement ? \\
Peut-on être obligé d'aimer ? \\
Peut-on être plus ou moins libre ? \\
Peut-on être sûr d'avoir raison ? \\
Peut-on être sûr de bien agir ? \\
Peut-on être sûr de ne pas se tromper ? \\
Peut-on être trop sensible ? \\
Peut-on étudier le passé de façon objective ? \\
Peut-on expliquer une œuvre d'art ? \\
Peut-on faire de la politique sans supposer les hommes méchants ? \\
Peut-on faire de l'esprit un objet de science ? \\
Peut-on faire la philosophie de l'histoire ? \\
Peut-on faire le bien d'autrui malgré lui ? \\
Peut-on faire le mal innocemment ? \\
Peut-on faire l'expérience de la nécessité ? \\
Peut-on faire table rase du passé ? \\
Peut-on fonder la morale ? \\
Peut-on fonder le droit sur la morale ? \\
Peut-on fonder un droit de désobéir ? \\
Peut-on fonder une éthique sur la biologie ? \\
Peut-on fonder une morale sur le plaisir ? \\
Peut-on fuir la société ? \\
Peut-on haïr la raison ? \\
Peut-on haïr la vie ? \\
Peut-on haïr les images ? \\
Peut-on ignorer sa propre liberté ? \\
Peut-on ignorer volontairement la vérité ? \\
Peut-on imaginer l'avenir ? \\
Peut-on justifier le mal ? \\
Peut-on maîtriser la nature ? \\
Peut-on maîtriser l'évolution de la technique ? \\
Peut-on manquer de volonté ? \\
Peut-on mentir par humanité ? \\
Peut-on mesurer le temps ? \\
Peut-on moraliser la guerre ? \\
Peut-on ne croire en rien ? \\
Peut-on ne pas croire au progrès ? \\
Peut-on ne pas croire ? \\
Peut-on ne pas être égoïste ? \\
Peut-on ne pas être soi-même ? \\
Peut-on ne pas savoir ce que l'on dit ? \\
Peut-on ne pas savoir ce que l'on fait ? \\
Peut-on ne penser à rien ? \\
Peut-on nier le réel ? \\
Peut-on nier l'évidence ? \\
Peut-on nier l'existence de la matière ? \\
Peut-on opposer le loisir au travail ? \\
Peut-on ôter à l'homme sa liberté ? \\
Peut-on parler de dialogue des cultures ? \\
Peut-on parler de mondes imaginaires ? \\
Peut-on parler de nourriture spirituelle ? \\
Peut-on parler de problèmes techniques ? \\
Peut-on parler des miracles de la technique ? \\
Peut-on parler de travail intellectuel ? \\
Peut-on parler de vérité subjective ? \\
Peut-on parler de violence d'État ? \\
Peut-on parler d'une morale collective ? \\
Peut-on parler d'une religion de l'humanité ? \\
Peut-on parler d'un progrès de la liberté ? \\
Peut-on parler pour ne rien dire ? \\
Peut-on penser ce qu'on ne peut dire ? \\
Peut-on penser contre l'expérience ? \\
Peut-on penser la matière ? \\
Peut-on penser la vie sans penser la mort ? \\
Peut-on penser la vie ? \\
Peut-on penser l'infini ? \\
Peut-on penser sans image ? \\
Peut-on penser sans les mots ? \\
Peut-on penser sans méthode ? \\
Peut-on penser sans préjugés ? \\
Peut-on percevoir sans juger ? \\
Peut-on perdre la raison ? \\
Peut-on perdre sa liberté ? \\
Peut-on perdre son temps ? \\
Peut-on prédire les événements ? \\
Peut-on préférer le bonheur à la vérité ? \\
Peut-on préférer l'injustice au désordre ? \\
Peut-on protéger les libertés sans les réduire ? \\
Peut-on prouver l'existence ? \\
Peut-on prouver une existence ? \\
Peut-on raconter sa vie ? \\
Peut-on recommencer sa vie ? \\
Peut-on réduire le raisonnement au calcul ? \\
Peut-on refuser la violence ? \\
Peut-on refuser le vrai ? \\
Peut-on rendre raison de tout ? \\
Peut-on répondre d'autrui ? \\
Peut-on reprocher au langage d'être parfait ? \\
Peut-on résister au vrai ? \\
Peut-on rompre avec la société ? \\
Peut-on rompre avec le passé ? \\
Peut-on s'affranchir des lois ? \\
Peut-on s'attendre à tout ? \\
Peut-on savoir sans croire ? \\
Peut-on se choisir un destin ? \\
Peut-on se connaître soi-même ? \\
Peut-on se gouverner soi-même ? \\
Peut-on se mentir à soi-même ? \\
Peut-on se mettre à la place d'autrui ? \\
Peut-on se mettre à la place de l'autre ? \\
Peut-on se passer de croyances ? \\
Peut-on se passer de l'État ? \\
Peut-on se passer d'État ? \\
Peut-on se passer de technique ? \\
Peut-on se passer de toute religion ? \\
Peut-on se passer d'idéal ? \\
Peut-on se passer d'un maître ? \\
Peut-on se prescrire une loi ? \\
Peut-on sympathiser avec l'ennemi ? \\
Peut-on tirer des leçons de l'histoire ? \\
Peut-on toujours faire ce qu'on doit ? \\
Peut-on tout analyser ? \\
Peut-on tout attendre de l'État ? \\
Peut-on tout démontrer ? \\
Peut-on tout dire ? \\
Peut-on tout donner ? \\
Peut-on tout échanger ? \\
Peut-on tout interpréter ? \\
Peut-on tout ordonner ? \\
Peut-on vivre en sceptique ? \\
Peut-on vivre hors du temps ? \\
Peut-on vivre pour la vérité ? \\
Peut-on vivre sans échange ? \\
Peut-on vivre sans le plaisir de vivre ? \\
Peut-on vivre sans lois ? \\
Peut-on vivre sans peur ? \\
Peut-on vivre sans réfléchir ? \\
Peut-on vivre sans sacré ? \\
Peut-on vouloir ce qu'on ne désire pas ? \\
Peut-on vouloir le mal ? \\
Philosophe-t-on pour être heureux ? \\
Physique et mathématiques \\
Physique et métaphysique \\
Pitié et compassion \\
Pitié et cruauté \\
Pitié et mépris \\
Plaisir et bonheur \\
Poésie et philosophie \\
Politique et vérité \\
Possession et propriété \\
Pour connaître, suffit-il de démontrer ? \\
Pour être homme, faut-il être citoyen ? \\
Pour être libre, faut-il renoncer à être heureux ? \\
Pour être un bon observateur faut-il être un bon théoricien ? \\
Pour juger, faut-il seulement apprendre à raisonner ? \\
Pourquoi aller contre son désir ? \\
Pourquoi chercher à se distinguer ? \\
Pourquoi chercher la vérité ? \\
Pourquoi cherche-t-on à connaître ? \\
Pourquoi défendre le faible ? \\
Pourquoi délibérer ? \\
Pourquoi des artistes ? \\
Pourquoi des cérémonies ? \\
Pourquoi des devoirs ? \\
Pourquoi des idoles ? \\
Pourquoi désirons-nous ? \\
Pourquoi des lois ? \\
Pourquoi des maîtres ? \\
Pourquoi des poètes ? \\
Pourquoi des utopies ? \\
Pourquoi dialogue-t-on ? \\
Pourquoi donner des leçons de morale ? \\
Pourquoi donner ? \\
Pourquoi échanger des idées ? \\
Pourquoi écrit-on les lois ? \\
Pourquoi écrit-on ? \\
Pourquoi être moral ? \\
Pourquoi faire confiance ? \\
Pourquoi faire son devoir ? \\
Pourquoi faut-il diviser le travail ? \\
Pourquoi faut-il être juste ? \\
Pourquoi faut-il travailler ? \\
Pourquoi interprète-t-on ? \\
Pourquoi joue-t-on ? \\
Pourquoi la justice a-t-elle besoin d'institutions ? \\
Pourquoi les sciences ont-elles une histoire ? \\
Pourquoi les sociétés ont-elles besoin de lois ? \\
Pourquoi l'homme travaille-t-il ? \\
Pourquoi lire les poètes ? \\
Pourquoi nous trompons-nous ? \\
Pourquoi nous-trompons nous ? \\
Pourquoi punir ? \\
Pourquoi rechercher la vérité ? \\
Pourquoi respecter autrui ? \\
Pourquoi respecter le droit ? \\
Pourquoi s'interroger sur l'origine du langage ? \\
Pourquoi sommes-nous des êtres moraux ? \\
Pourquoi théoriser ? \\
Pourquoi transmettre ? \\
Pourquoi travailler ? \\
Pourquoi un fait devrait-il être établi ? \\
Pourquoi vivre ensemble ? \\
Pourquoi vouloir se connaître ? \\
Pourquoi y a-t-il des institutions ? \\
Pourquoi y a-t-il plusieurs sciences ? \\
Pourrait-on se passer de l'argent ? \\
Pouvoir et autorité \\
Pouvoir et devoir \\
Pouvoir et puissance \\
Pouvoir et savoir \\
Pouvons-nous connaître sans interpréter ? \\
Pouvons-nous dissocier le réel de nos interprétations ? \\
Pouvons-nous faire l'expérience de la liberté ? \\
Prédire et expliquer \\
Prendre conscience \\
Prendre la parole \\
Prendre ses responsabilités \\
Prendre soin \\
Preuve et démonstration \\
Production et création \\
Produire et créer \\
Prose et poésie \\
Prouver \\
Prouver et démontrer \\
Prouver et éprouver \\
Prouver et réfuter \\
Prudence et liberté \\
Puis-je être dans le vrai sans le savoir ? \\
Puis-je être libre sans être responsable ? \\
Puis-je faire confiance à mes sens ? \\
Puis-je invoquer l'inconscient sans ruiner la morale ? \\
Puis-je me passer d'imiter autrui ? \\
Puis-je ne pas vouloir ce que je désire ? \\
Puis-je répondre des autres ? \\
Puis-je savoir ce qui m'est propre ? \\
Punir \\
Punition et vengeance \\
Qu'ai-je le droit d'exiger d'autrui ? \\
Qu'ai-je le droit d'exiger des autres ? \\
Qu'aime-t-on dans l'amour ? \\
Qualité et quantité \\
Quand une autorité est-elle légitime ? \\
Qu'apprend-on des romans ? \\
Qu'apprend-on en commettant une faute ? \\
Qu'apprend-on quand on apprend à parler ? \\
Qu'attendons-nous de la technique ? \\
Qu'attendons-nous pour être heureux ? \\
Que célèbre l'art ? \\
Que démontrent nos actions ? \\
Que devons-nous à autrui ? \\
Que dois-je respecter en autrui ? \\
Que doit la pensée à l'écriture ? \\
Que doit-on croire ? \\
Que doit-on désirer pour ne pas être déçu ? \\
Que faire de nos passions ? \\
Que faire des adversaires ? \\
Que faut-il absolument savoir ? \\
Que faut-il respecter ? \\
Que gagne-t-on à travailler ? \\
Quel est le poids du passé ? \\
Quel est le sens du progrès technique ? \\
Quel est l'objet de la biologie ? \\
Quel est l'objet de la métaphysique ? \\
Quelle causalité pour le vivant ? \\
Quelle est la cause du désir ? \\
Quelle est la fonction première de l'État ? \\
Quelle est la force de la loi ? \\
Quelle est la place de l'imagination dans la vie de l'esprit ? \\
Quelle est la réalité de l'avenir ? \\
Quelle est la réalité d'une idée ? \\
Quelle est la réalité du passé ? \\
Quelle est la valeur d'une expérimentation ? \\
Quelle est la valeur du rêve ? \\
Quelle est l'unité du « je » ? \\
Quelle réalité attribuer à la matière ? \\
Quelle réalité l'art nous fait-il connaître ? \\
Quelle sorte d'histoire ont les sciences ? \\
Quelles sont les caractéristiques d'un être vivant ? \\
Quel sens donner à l'expression « gagner sa vie » ? \\
Que manque-t-il à une machine pour être vivante ? \\
Que mesure-t-on du temps ? \\
Que montre une démonstration ? \\
Que nous apprend la fiction sur la réalité ? \\
Que nous apprend la maladie sur la santé ? \\
Que nous apprend la musique ? \\
Que nous apprend la vie ? \\
Que nous apprend l'expérience ? \\
Que nous apprennent les animaux sur nous-mêmes ? \\
Que nous apprennent les animaux ? \\
Que nous apprennent les machines ? \\
Que nous apprennent les métaphores ? \\
Que nous enseigne l'expérience ? \\
Que nous enseignent les sens ? \\
Que nous réserve l'avenir ? \\
Que peint le peintre ? \\
Que penser de l'adage : « Que la justice s'accomplisse, le monde dût-il périr » (Fiat justitia pereat mundus) ? \\
Que percevons-nous d'autrui ? \\
Que perdrait la pensée en perdant l'écriture ? \\
Que peut la musique ? \\
Que peut la volonté ? \\
Que peut le corps ? \\
Que peut l'esprit sur la matière ? \\
Que peut l'État ? \\
Que peut-on contre un préjugé ? \\
Que peut-on savoir de l'inconscient ? \\
Que peut-on savoir de soi ? \\
Que peut-on savoir par expérience ? \\
Que pouvons-nous faire de notre passé ? \\
Que produit l'inconscient ? \\
Que reste-t-il d'une existence ? \\
Que sait-on du réel ? \\
Que signifie être en guerre ? \\
Que signifie l'idée de technoscience ? \\
Que signifier « juger en son âme et conscience » ? \\
Que sont les apparences ? \\
Qu'est-ce qu'apprendre ? \\
Qu'est-ce qu'argumenter ? \\
Qu'est-ce que commencer ? \\
Qu'est-ce que composer une œuvre ? \\
Qu'est-ce que comprendre une œuvre d'art ? \\
Qu'est-ce que créer ? \\
Qu'est-ce que définir ? \\
Qu'est-ce que faire une expérience ? \\
Qu'est-ce que gouverner ? \\
Qu'est-ce que juger ? \\
Qu'est-ce que la causalité ? \\
Qu'est-ce que la science saisit du vivant ? \\
Qu'est-ce que le langage ordinaire ? \\
Qu'est-ce que le malheur ? \\
Qu'est-ce que le moi ? \\
Qu'est-ce que le présent ? \\
Qu'est-ce que le réel ? \\
Qu'est-ce que le sacré ? \\
Qu'est-ce que l'inconscient ? \\
Qu'est-ce que l'intérêt général ? \\
Qu'est-ce que manquer de culture ? \\
Qu'est-ce que parler le même langage ? \\
Qu'est-ce que parler ? \\
Qu'est-ce que prouver ? \\
Qu'est-ce que traduire ? \\
Qu'est-ce qu'être artiste ? \\
Qu'est-ce qu'être en vie ? \\
Qu'est-ce qu'être esclave ? \\
Qu'est-ce qu'être inhumain ? \\
Qu'est-ce qu'être l'auteur de son acte ? \\
Qu'est-ce qu'être malade ? \\
Qu'est-ce qu'être normal ? \\
Qu'est-ce qu'être réaliste ? \\
Qu'est-ce qu'être spirituel ? \\
Qu'est-ce que vérifier une théorie ? \\
Qu'est-ce que vivre bien ? \\
Qu'est-ce qu'exister pour un individu ? \\
Qu'est-ce qu'exister ? \\
Qu'est-ce que « parler le même langage » ? \\
Qu'est-ce qui distingue un vivant d'une machine ? \\
Qu'est-ce qui est absurde ? \\
Qu'est-ce qui est irrationnel ? \\
Qu'est-ce qui est irréversible ? \\
Qu'est-ce qui est naturel ? \\
Qu'est-ce qui est possible ? \\
Qu'est-ce qui est réel ? \\
Qu'est-ce qui est respectable ? \\
Qu'est-ce qui est scientifique ? \\
Qu'est-ce qui est vital ? \\
Qu'est-ce qui fait changer les sociétés ? \\
Qu'est-ce qui fait d'une activité un travail ? \\
Qu'est-ce qui fait la valeur de la technique ? \\
Qu'est-ce qui fait la valeur d'une existence ? \\
Qu'est-ce qui fait le pouvoir des mots ? \\
Qu'est-ce qui fait l'unité d'une science ? \\
Qu'est-ce qui fait l'unité d'un organisme ? \\
Qu'est-ce qui fait un peuple ? \\
Qu'est-ce qui fonde le respect d'autrui ? \\
Qu'est-ce qui importe ? \\
Qu'est-ce qui menace la liberté ? \\
Qu'est-ce qui mesure la valeur d'un travail ? \\
Qu'est-ce qui n'a pas d'histoire ? \\
Qu'est-ce qu'interpréter une œuvre d'art ? \\
Qu'est-ce qui peut se transformer ? \\
Qu'est-ce qu'on ne peut comprendre ? \\
Qu'est-ce qu'un acte libre ? \\
Qu'est-ce qu'un alter ego ? \\
Qu'est-ce qu'un animal ? \\
Qu'est-ce qu'un bon citoyen ? \\
Qu'est-ce qu'un cas de conscience ? \\
Qu'est-ce qu'un chef-d'œuvre ? \\
Qu'est-ce qu'un citoyen libre ? \\
Qu'est-ce qu'un classique ? \\
Qu'est-ce qu'un concept ? \\
Qu'est-ce qu'un consommateur ? \\
Qu'est-ce qu'une action juste ? \\
Qu'est-ce qu'une action politique ? \\
Qu'est-ce qu'une autorité légitime ? \\
Qu'est-ce qu'une belle forme ? \\
Qu'est-ce qu'une bonne délibération ? \\
Qu'est-ce qu'un échange juste ? \\
Qu'est-ce qu'un échange réussi ? \\
Qu'est-ce qu'une chose matérielle ? \\
Qu'est-ce qu'une communauté ? \\
Qu'est ce qu'une connaissance fiable ? \\
Qu'est-ce qu'une constitution ? \\
Qu'est-ce qu'une crise ? \\
Qu'est-ce qu'une erreur ? \\
Qu'est-ce qu'une expérience scientifique ? \\
Qu'est-ce qu'une fausse science ? \\
Qu'est-ce qu'une faute de goût ? \\
Qu'est-ce qu'une fiction ? \\
Qu'est-ce qu'une hypothèse scientifique ? \\
Qu'est-ce qu'une image ? \\
Qu'est-ce qu'une injustice ? \\
Qu'est-ce qu'une langue artificielle ? \\
Qu'est-ce qu'une libre interprétation ? \\
Qu'est ce qu'une mauvaise idée ? \\
Qu'est-ce qu'une méthode ? \\
Qu'est-ce qu'une œuvre d'art réaliste ? \\
Qu'est-ce qu'une parole vraie ? \\
Qu'est-ce qu'une preuve ? \\
Qu'est-ce qu'une république ? \\
Qu'est-ce qu'une révolution scientifique ? \\
Qu'est-ce qu'une révolution ? \\
Qu'est-ce qu'une science expérimentale ? \\
Qu'est-ce qu'une solution ? \\
Qu'est-ce qu'un esprit juste ? \\
Qu'est ce qu'un esprit libre ? \\
Qu'est-ce qu'un esprit libre ? \\
Qu'est-ce qu'un état de droit ? \\
Qu'est-ce qu'un État de droit ? \\
Qu'est-ce qu'un État libre ? \\
Qu'est-ce qu'une théorie scientifique ? \\
Qu'est-ce qu'une tradition ? \\
Qu'est-ce qu'un événement historique ? \\
Qu'est-ce qu'un événement ? \\
Qu'est-ce qu'une vérité contingente ? \\
Qu'est-ce qu'une vérité historique ? \\
Qu'est-ce qu'une vérité subjective ? \\
Qu'est-ce qu'une vie heureuse ? \\
Qu'est-ce qu'une vie humaine ? \\
Qu'est-ce qu'un exemple ? \\
Qu'est-ce qu'un expérimentateur ? \\
Qu'est-ce qu'un fait de culture ? \\
Qu'est-ce qu'un fait ? \\
Qu'est-ce qu'un faux problème ? \\
Qu'est-ce qu'un faux ? \\
Qu'est-ce qu'un gouvernement démocratique ? \\
Qu'est-ce qu'un homme d'action ? \\
Qu'est-ce qu'un homme d'État ? \\
Qu'est-ce qu'un homme d'expérience ? \\
Qu'est-ce qu'un homme juste ? \\
Qu'est-ce qu'un homme méchant ? \\
Qu'est-ce qu'un homme politique ? \\
Qu'est-ce qu'un justicier ? \\
Qu'est-ce qu'un maître ? \\
Qu'est-ce qu'un monstre ? \\
Qu'est-ce qu'un musée ? \\
Qu'est-ce qu'un mythe ? \\
Qu'est-ce qu'un outil ? \\
Qu'est-ce qu'un paradoxe ? \\
Qu'est-ce qu'un pauvre ? \\
Qu'est-ce qu'un peuple ? \\
Qu'est-ce qu'un problème scientifique ? \\
Qu'est-ce qu'un problème technique ? \\
Qu'est-ce qu'un problème ? \\
Qu'est-ce qu'un progrès technique ? \\
Qu'est-ce qu'un public ? \\
Qu'est-ce qu'un récit véridique ? \\
Qu'est-ce qu'un tabou ? \\
Qu'est-ce qu'un technicien ? \\
Qu'est-ce qu'un témoin ? \\
Qu'est-ce qu'un tyran ? \\
Que valent les mots ? \\
Que valent les théories ? \\
Que vaut la définition de l'homme comme animal doué de raison ? \\
Que veut dire : « le temps passe » ? \\
Qui accroît son savoir accroît sa douleur \\
Qui commande ? \\
Qui croire ? \\
Qui est digne du bonheur ? \\
Qui est libre ? \\
Qui est mon prochain ? \\
Qui est mon semblable ? \\
Qui est riche ? \\
Qui est sage ? \\
Qui fait la loi ? \\
Qui gouverne ? \\
Qui nous dicte nos devoirs ? \\
Qui parle quand je dis « je » ? \\
Qui parle ? \\
Qui peut avoir des droits ? \\
Qui peut me dire « tu ne dois pas » ? \\
Qui travaille ? \\
Raison et dialogue \\
Raison et folie \\
Raison et fondement \\
Raison et langage \\
Raison et tradition \\
Raisonnable et rationnel \\
Raisonner \\
Réalité et apparence \\
Réalité et perception \\
Réalité et représentation \\
Récit et histoire \\
Réfuter \\
Règles sociales et loi morale \\
Regrets et remords \\
Religion et démocratie \\
Religion et moralité \\
Religion et politique \\
Religions et démocratie \\
Rendre justice \\
Répondre de soi \\
Représenter \\
République et démocratie \\
Résistance et obéissance \\
Respect et tolérance \\
Réussir sa vie \\
Rêver \\
Revient-il à l'État d'assurer le bonheur des citoyens ? \\
Révolte et révolution \\
Rhétorique et vérité \\
Richesse et pauvreté \\
Rire \\
Sait-on ce que l'on veut ? \\
Sait-on ce qu'on fait ? \\
Sait-on nécessairement ce que l'on désire ? \\
Sait-on toujours ce qu'on veut ? \\
S'amuser \\
Savoir est-ce cesser de croire ? \\
Savoir et croire \\
Savoir et démontrer \\
Savoir et pouvoir \\
Savoir et savoir faire \\
Science du vivant et finalisme \\
Science du vivant, science de l'inerte \\
Science et croyance \\
Science et métaphysique \\
Science et méthode \\
Science et mythe \\
Science et religion \\
Se cultiver \\
Se cultiver, est-ce s'affranchir de son appartenance culturelle ? \\
Se décider \\
Se faire comprendre \\
Se mentir à soi-même : est-ce possible ? \\
Se nourrir \\
Sensation et perception \\
Sens et existence \\
Sens et signification \\
Sens propre et sens figuré \\
Sentir et juger \\
Sentir et penser \\
Serions-nous heureux dans un ordre politique parfait ? \\
Servir, est-ce nécessairement renoncer à sa liberté ? \\
Se suffire à soi-même \\
S'exprimer \\
Signe et symbole \\
Sincérité et vérité \\
Si nous étions moraux, le droit serait-il inutile ? \\
Si tout est historique, tout est-il relatif ? \\
Société et communauté \\
Société humaines, sociétés animales \\
Sommes-nous dans le temps comme dans l'espace ? \\
Sommes-nous des sujets ? \\
Sommes-nous déterminés par notre culture ? \\
Sommes-nous jamais certains d'avoir choisi librement ? \\
Sommes-nous libres face à l'évidence ? \\
Sommes-nous maîtres de nos paroles ? \\
Sommes-nous portés au bien ? \\
Sommes-nous prisonniers du temps ? \\
Sommes-nous responsables de nos désirs ? \\
Sommes-nous responsables de nos opinions ? \\
Sommes-nous sujets de nos désirs ? \\
Soumission et servitude \\
Substance et accident \\
Suffit-il d'avoir raison ? \\
Suffit-il de bien juger pour bien faire ? \\
Suffit-il de faire son devoir pour être vertueux ? \\
Suffit-il de faire son devoir ? \\
Suffit-il de n'avoir rien fait pour être innocent ? \\
Suffit-il que nos intentions soient bonnes pour que nos actions le soient aussi ? \\
Suis-ce que j'ai conscience d'être ? \\
Suis-je dans le temps comme je suis dans l'espace ? \\
Suis-je étranger à moi-même ? \\
Suis-je l'auteur de ce que je dis ? \\
Suis-je le mieux placé pour me connaître ? \\
Suis-je libre ? \\
Suis-je mon corps ? \\
Suis-je mon passé ? \\
Suis-je propriétaire de mon corps ? \\
Suis-je responsable de ce dont je n'ai pas conscience ? \\
Suis-je responsable de ce que je suis ? \\
Suis-je toujours autre que moi-même ? \\
Suivre son intuition \\
Surface et profondeur \\
Sur quoi fonder la justice ? \\
Sur quoi fonder l'autorité politique ? \\
Sur quoi fonder l'autorité ? \\
Sur quoi fonder le devoir ? \\
Sur quoi fonder le droit de punir ? \\
Sur quoi le langage doit-il se régler ? \\
Sur quoi repose la croyance au progrès ? \\
Survivre \\
Suspendre son jugement \\
Sympathie et respect \\
Talent et génie \\
Technique et idéologie \\
Technique et nature \\
Technique et savoir-faire \\
Technique et violence \\
Temps et commencement \\
Temps et création \\
Temps et histoire \\
Temps et irréversibilité \\
Temps et liberté \\
Temps et mémoire \\
Temps et vérité \\
Théorie et expérience \\
Toucher, sentir, goûter \\
Tous les hommes désirent-ils naturellement être heureux ? \\
Tous les hommes désirent-ils naturellement savoir ? \\
Tous les paradis sont-ils perdus ? \\
Tous les rapports humains sont-ils des échanges ? \\
Tout a-t-il une raison d'être ? \\
Tout ce qui est naturel est-il normal ? \\
Tout ce qui est vrai doit-il être prouvé ? \\
Tout démontrer \\
Tout désir est-il égoïste ? \\
Tout désir est-il une souffrance ? \\
Tout dire \\
Tout droit est-il un pouvoir ? \\
Toute compréhension implique-t-elle une interprétation ? \\
Toute connaissance est-elle hypothétique ? \\
Toute connaissance s'enracine-t-elle dans la perception ? \\
Toute conscience est-elle subjective ? \\
Toute description est-elle une interprétation ? \\
Toute faute est-elle une erreur ? \\
Toute inégalité est-elle injuste ? \\
Toute interprétation est-elle contestable ? \\
Toute interprétation est-elle subjective ? \\
Toute morale s'oppose-t-elle aux désirs ? \\
Toute relation humaine est-elle un échange ? \\
Toutes les fautes se valent-elles ? \\
Toutes les inégalités sont-elles des injustices ? \\
Toutes les interprétations se valent-elles ? \\
Toute société a-t-elle besoin d'une religion ? \\
Tout est-il matière ? \\
Toute vérité est-elle démontrable ? \\
Toute vie est-elle intrinsèquement respectable ? \\
Tout futur est-il contingent ? \\
Tout ordre est-il une violence déguisée ? \\
Tout peut-il être objet d'échange ? \\
Tout peut-il être objet de science ? \\
Tout peut-il s'acheter ? \\
Tout peut-il se démontrer ? \\
Tout s'en va-t-il avec le temps ? \\
Tout se prête-il à la mesure ? \\
Tout travail est-il forcé ? \\
Tout travail est-il social ? \\
Tout vouloir \\
Tradition et liberté \\
Tradition et nouveauté \\
Tradition et transmission \\
Traduire, est-ce trahir ? \\
Transcendance et immanence \\
Transmettre \\
Travail, besoin, désir \\
Travail et aliénation \\
Travail et besoin \\
Travail et bonheur \\
Travail et capital \\
Travail et liberté \\
Travail et loisir \\
Travail et nécessité \\
Travail et œuvre \\
Travailler, est-ce faire œuvre ? \\
Travailler et œuvrer \\
Travaille-t-on pour soi-même ? \\
Travail manuel, travail intellectuel \\
Un acte gratuit est-il possible ? \\
Un acte peut-il être inhumain ? \\
Un artiste doit-il être original ? \\
Un bien peut-il être commun ? \\
Un choix peut-il être rationnel ? \\
Un désir peut-il être coupable ? \\
Un désir peut-il être inconscient ? \\
Une activité inutile est-elle sans valeur ? \\
Une communauté politique n'est-elle qu'une communauté d'intérêt ? \\
Une connaissance peut-elle ne pas être relative ? \\
Une connaissance scientifique du vivant est-elle possible ? \\
Une croyance peut-elle être libre ? \\
Une croyance peut-elle être rationnelle ? \\
Une culture peut-elle être porteuse de valeurs universelles ? \\
Une destruction peut-elle être créatrice ? \\
Une idée peut-elle être générale ? \\
Une imitation peut-elle être parfaite ? \\
Une interprétation peut-elle échapper à l'arbitraire ? \\
Une interprétation peut-elle être définitive ? \\
Une interprétation peut-elle être objective ? \\
Une interprétation peut-elle prétendre à la vérité ? \\
Une langue n'est-elle faite que de mots ? \\
Une loi peut-elle être injuste ? \\
Une morale sans obligation est-elle possible ? \\
Une morale sceptique est-elle possible ? \\
Une œuvre d'art a-t-elle toujours un sens ? \\
Une œuvre d'art doit-elle nécessairement être belle ? \\
Une œuvre d'art doit-elle plaire ? \\
Une œuvre d'art peut-elle être immorale ? \\
Une pensée contradictoire est-elle dénuée de valeur ? \\
Une perception peut-elle être illusoire ? \\
Une psychologie peut-elle être matérialiste ? \\
Une science de l'esprit est-elle possible ? \\
Une sensation peut-elle être fausse ? \\
Une société juste est-ce une société sans conflit ? \\
Une société sans État est-elle possible ? \\
Une société sans religion est-elle possible ? \\
Une société sans travail est-elle souhaitable ? \\
Une théorie peut-elle être vérifiée ? \\
Une vérité peut-elle être indicible ? \\
Une vérité peut-elle être provisoire ? \\
Une vie libre exclut-elle le travail ? \\
Un fait existe-t-il sans interprétation ? \\
Un gouvernement de savants est-il souhaitable ? \\
Un mensonge peut-il avoir une valeur morale ? \\
Un monde meilleur \\
Un peuple est-il responsable de son histoire ? \\
Un peuple est-il un rassemblement d'individus ? \\
Un problème moral peut-il recevoir une solution certaine ? \\
Un savoir peut-il être inconscient ? \\
User de violence peut-il être moral ? \\
Utilité et beauté \\
Vaut-il mieux subir ou commettre l'injustice ? \\
Vérité et apparence \\
Vérité et certitude \\
Vérité et efficacité \\
Vérité et exactitude \\
Vérité et illusion \\
Vérité et liberté \\
Vérité et réalité \\
Vérité et religion \\
Vérité et sincérité \\
Vérité et vérification \\
Vérité et vraisemblance \\
Vérité théorique, vérité pratique \\
Vice et délice \\
Vie politique et vie contemplative \\
Vie privée et vie publique \\
Vie publique et vie privée \\
Violence et force \\
Violence et pouvoir \\
Vivrait-on mieux sans désirs ? \\
Vivre en société, est-ce seulement vivre ensemble ? \\
Vivre et exister \\
Vivre libre \\
Vivre sa vie \\
Voir et entendre \\
Voir et toucher \\
Voit-on ce qu'on croit ? \\
Vouloir dire \\
Vouloir et pouvoir \\
Vouloir la paix sociale peut-il aller jusqu'à accepter l'injustice ? \\
Vouloir la solitude \\
Y a-t-il d'autres moyens que la démonstration pour établir la vérité ? \\
Y a-t-il de bons préjugés ? \\
Y a-t-il de justes inégalités ? \\
Y a-t-il de la fatalité dans la vie de l'homme ? \\
Y a-t-il de l'inconnaissable ? \\
Y a-t-il de l'indémontrable ? \\
Y a-t-il de mauvais désirs ? \\
Y a-t-il des arts mineurs ? \\
Y a-t-il des biens inestimables ? \\
Y a-t-il des choses qu'on n'échange pas ? \\
Y a-t-il des connaissances dangereuses ? \\
Y a-t-il des contraintes légitimes ? \\
Y a-t-il des convictions philosophiques ? \\
Y a-t-il des correspondances entre les arts ? \\
Y a-t-il des croyances nécessaires ? \\
Y a-t-il des degrés de conscience ? \\
Y a-t-il des degrés de vérité ? \\
Y a-t-il des démonstrations en philosophie ? \\
Y a-t-il des devoirs envers soi ? \\
Y a-t-il des erreurs de la nature ? \\
Y a-t-il des évidences morales ? \\
Y a-t-il des expériences sans théorie ? \\
Y a-t-il des faits scientifiques ? \\
Y a-t-il des fins de la nature ? \\
Y a-t-il des guerres justes ? \\
Y a-t-il des inégalités justes ? \\
Y a-t-il des injustices naturelles ? \\
Y a-t-il des limites à la connaissance ? \\
Y a-t-il des limites à la tolérance ? \\
Y a-t-il des mondes imaginaires ? \\
Y a-t-il des normes naturelles ? \\
Y a-t-il des objets qui n'existent pas ? \\
Y a-t-il des perceptions insensibles ? \\
Y a-t-il des peuples sans histoire ? \\
Y a-t-il des plaisirs meilleurs que d'autres ? \\
Y a-t-il des principes de justice universels ? \\
Y a-t-il des questions sans réponse ? \\
Y a-t-il des raisons de douter de la raison ? \\
Y a-t-il des solutions en politique ? \\
Y a-t-il des sots métiers ? \\
Y a-t-il des valeurs absolues ? \\
Y a-t-il des valeurs propres à la science ? \\
Y a-t-il des vérités de fait ? \\
Y a-t-il des vérités définitives ? \\
Y a-t-il des vérités en art ? \\
Y a-t-il des vérités éternelles ? \\
Y a-t-il des vérités indémontrables ? \\
Y a-t-il des vérités indiscutables ? \\
Y a-t-il des vérités morales ? \\
Y a-t-il des vérités qui échappent à la raison ? \\
Y a-t-il différentes façons d'exister ? \\
Y a-t-il du nouveau dans l'histoire ? \\
Y a-t-il nécessairement du religieux dans l'art ? \\
Y a-t-il plusieurs sortes de matières ? \\
Y a-t-il progrès en art ? \\
Y a-t-il un art d'être heureux ? \\
Y a-t-il un art de vivre ? \\
Y a-t-il un art d'interpréter ? \\
Y a-t-il un art du bonheur ? \\
Y a-t-il un bonheur sans illusion ? \\
Y a-t-il un devoir de mémoire ? \\
Y a-t-il un devoir d'être heureux ? \\
Y a-t-il un droit au bonheur ? \\
Y a-t-il un droit au travail ? \\
Y a-t-il un droit de désobéissance ? \\
Y a-t-il un droit de mentir ? \\
Y a-t-il un droit de révolte ? \\
Y a-t-il un droit naturel ? \\
Y a-t-il une beauté propre à l'objet technique ? \\
Y a-t-il une causalité en histoire ? \\
Y a-t-il une compétence politique ? \\
Y a-t-il une condition humaine ? \\
Y a-t-il une conscience collective ? \\
Y a-t-il une enfance de l'humanité ? \\
Y a-t-il une fonction propre à l'œuvre d'art ? \\
Y a-t-il une histoire de la raison ? \\
Y a-t-il une histoire de la vérité ? \\
Y a-t-il une histoire universelle ? \\
Y a-t-il une justice naturelle ? \\
Y a-t-il une logique du désir ? \\
Y a-t-il une méthode de l'interprétation ? \\
Y a-t-il une morale universelle ? \\
Y a-t-il une nature humaine ? \\
Y a-t-il une nécessité de l'erreur ? \\
Y a-t-il une nécessité morale ? \\
Y a-t-il une œuvre du temps ? \\
Y a-t-il une ou des morales ? \\
Y a-t-il une pensée technique ? \\
Y a-t-il une primauté du devoir sur le droit ? \\
Y a-t-il une rationalité du hasard ? \\
Y a-t-il une responsabilité de l'artiste ? \\
Y a-t-il une science politique ? \\
Y a-t-il une servitude volontaire ? \\
Y a-t-il une spécificité du vivant ? \\
Y a-t-il une technique pour tout ? \\
Y a-t-il une unité des devoirs ? \\
Y a-t-il une unité des sciences ? \\
Y a-t-il une valeur de l'inutile ? \\
Y a-t-il une vérité de l'œuvre d'art ? \\
Y a-t-il une vérité des apparences ? \\
Y a-t-il une vérité des représentations ? \\
Y a-t-il une vertu de l'imitation ? \\
Y a-t-il une violence du droit ? \\
Y a-t-il un fondement de la croyance ? \\
Y a-t-il un jugement de l'histoire ? \\
Y a-t-il un langage du corps ? \\
Y a-t-il un moteur de l'histoire ? \\
Y a-t-il un objet du désir ? \\
Y a-t-il un ordre dans la nature ? \\
Y a-t-il un ordre des choses ? \\
Y a-t-il un ordre du monde ? \\
Y a-t-il un primat de la nature sur la culture ? \\
Y a-t-il un progrès du droit ? \\
Y a-t-il un progrès en art ? \\
Y a-t-il un propre de l'homme ? \\
Y a-t-il un rapport moral à soi-même ? \\
Y a-t-il un savoir de la justice ? \\
Y a-t-il un savoir du juste ? \\
Y a-t-il un sens moral ? \\
« Aimer » se dit-il en un seul sens ? \\
« Connais-toi toi-même » \\
« Dans un bois aussi courbe que celui dont l'homme est fait on ne peut rien tailler de tout à fait droit » \\
« Il ne lui manque que la parole » \\
« Je ne l'ai pas fait exprès » \\
« Les bons comptes font les bons amis » \\
« Les faits sont là » \\
« L'histoire jugera » : quel sens faut-il accorder à cette expression ? \\
« Liberté, égalité, fraternité » \\
« Rien de ce qui est humain ne m'est étranger » \\
« Tout est relatif » \\
« Tradition n'est pas raison » \\
« Un instant d'éternité » \\
« Vis caché » \\


\subsection{CAPES interne}
\label{sec-2-6}

\noindent
À quoi reconnaît-on qu'une expérience est scientifique ? \\
À quoi reconnaît-on qu'un événement est historique ? \\
À quoi reconnaît-on une religion ? \\
À quoi reconnaît-on un être vivant ? \\
À quoi sert l'État ? \\
À quoi servent les religions ? \\
À quoi tient la force de l'État ? \\
Art et illusion \\
Art et vérité \\
Avoir raison \\
Avons-nous des devoirs envers les autres êtres vivants ? \\
À quoi bon se parler ? \\
À quoi sert l'histoire ? \\
Beauté et moralité \\
Beauté et vérité \\
Bonheur et société \\
Ce qui dépasse la raison est-il nécessairement irréel ? \\
Chance et bonheur \\
Comment l'homme peut-il se représenter le temps ? \\
Comment peut-on définir un être vivant ? \\
Comment peut-on être heureux ? \\
Connaît-on la vie ou bien connaît-on le vivant ? \\
Conscience de soi et connaissance de soi \\
Conscience et liberté \\
Contrainte et obligation \\
Dans quelle mesure le temps nous appartient-il ? \\
Dépend-il de soi d'être heureux ? \\
De quoi dépend le bonheur ? \\
De quoi l'État ne doit-il pas se mêler ? \\
Déraisonner, est-ce perdre de vue le réel ? \\
Devoir et plaisir \\
Devoirs envers les autres et devoirs envers soi-même \\
Doit-on respecter les êtres vivants ? \\
Doit-on tout accepter de l'État ? \\
Doit-on tout attendre de l'État ? \\
En quel sens le vivant a-t-il une histoire ? \\
En quoi la connaissance du vivant contribue-t-elle à la connaissance de l'homme ? \\
En quoi le bonheur est-il l'affaire de l'État ? \\
Est-ce de la force que l'État tient son autorité ? \\
Être à l'écoute de son désir, est-ce nier le désir de l'autre ? \\
Être conscient, est-ce être maître de soi ? \\
Être heureux, est-ce devoir ? \\
Être raisonnable, est-ce renoncer à ses désirs ? \\
Expérience immédiate et expérimentation scientifique \\
Fabriquer et créer \\
Faire son devoir, est-ce là toute la morale ? \\
Faisons-nous l'histoire ? \\
Faut-il accorder de l'importance aux mots ? \\
Faut-il apprendre à vivre en renonçant au bonheur ? \\
Faut-il changer ses désirs plutôt que l'ordre du monde ? \\
Faut-il chercher à satisfaire tous nos désirs ? \\
Faut-il connaître l'Histoire pour gouverner ? \\
Faut-il craindre l'État ? \\
Faut-il croire les historiens ? \\
Faut-il croire que l'histoire a un sens ? \\
Faut-il distinguer désir et besoin ? \\
Faut-il être libre pour être heureux ? \\
Faut-il faire table rase du passé ? \\
Faut-il hiérarchiser les désirs ? \\
Faut-il limiter le pouvoir de l'État ? \\
Faut-il obéir à la voix de sa conscience ? \\
Faut-il opposer histoire et mémoire ? \\
Faut-il oublier le passé pour se donner un avenir ? \\
Faut-il rechercher le bonheur ? \\
Faut-il respecter le vivant ? \\
Faut-il se fier à sa propre raison ? \\
Faut-il se libérer du travail ? \\
Faut-il se méfier de ses désirs ? \\
Histoire et mémoire \\
Histoire et progrès \\
Justice et pardon \\
La causalité \\
La connaissance du vivant est-elle désintéressée ? \\
La connaissance du vivant peut-elle être désintéressée  ? \\
La connaissance historique est-elle une interprétation des faits ? \\
La connaissance historique est-elle utile à l'homme ? \\
La connaissance scientifique \\
La connaissance sensible \\
La conscience collective \\
La conscience de soi \\
La conscience de soi et l'identité personnelle \\
La conscience est-elle source d'illusions ? \\
La conscience est-elle toujours morale ? \\
La conscience est-elle une activité ? \\
La conscience morale \\
La conscience morale n'est-elle que le fruit de l'éducation ? \\
La conscience peut-elle nous tromper ? \\
L'acte et la parole \\
La finalité est-elle nécessaire pour penser le vivant ? \\
La fin du travail \\
La fuite du temps est-elle nécessairement un malheur ? \\
La joie \\
L'aliénation \\
La maladie est-elle à l'organisme vivant ce que la panne est à la machine ? \\
La maladie est-elle indispensable à la connaissance du vivant ? \\
La mauvaise conscience \\
La méthode \\
La méthode expérimentale est-elle appropriée à l'étude du vivant ? \\
La morale dépend-elle de la culture ? \\
Langage et communication \\
L'animal \\
La parole \\
La parole et l'écriture \\
La parole et le geste \\
La pénibilité du travail \\
La raison doit-elle se soumettre au réel ? \\
La raison engendre-t-elle des illusions ? \\
La raison épuise-t-elle le réel ? \\
La raison est-elle l'esclave du désir ? \\
La raison est-elle plus fiable que l'expérience ? \\
La raison est-elle seulement affaire de logique ? \\
La raison ne connaît-elle du réel que ce qu'elle y met d'elle-même ? \\
La raison transforme-t-elle le réel ? \\
La religion divise-t-elle les hommes ? \\
La religion est-elle contraire à la raison ? \\
La religion est-elle une consolation pour les hommes ? \\
La religion implique-t-elle la croyance en un être divin ? \\
La religion n'est-elle qu'une affaire privée ? \\
La religion n'est-elle qu'un fait de culture ? \\
La religion relie-t-elle les hommes ? \\
La religion rend-elle l'homme heureux ? \\
La religion rend-elle meilleur ? \\
La religion repose-t-elle sur une illusion ? \\
La religion se distingue-t-elle de la superstition ? \\
La responsabilité \\
L'art a-t-il pour fin le plaisir ? \\
L'art est-il moins nécessaire que la science ? \\
L'art est-il une affaire sérieuse ? \\
L'art est-il un luxe ? \\
L'art et le beau \\
L'art et le réel \\
L'artiste doit-il être original ? \\
L'artiste doit-il se donner des modèles ? \\
L'artiste et l'artisan \\
L'art n'est qu'une affaire de goût ? \\
L'art nous réconcilie-t-il avec le monde ? \\
L'art peut-il s'enseigner ? \\
L'art peut-il se passer de règles ? \\
L'art rend-il les hommes meilleurs ? \\
La science nous éloigne-t-elle de la religion ? \\
La société peut-elle se passer de l'État ? \\
La technique nous éloigne-t-elle de la réalité ? \\
La traduction \\
La vérification expérimentale \\
La vertu peut-elle s'enseigner ? \\
La vie intérieure \\
La volonté et le désir \\
Le bavardage \\
Le beau et le bien \\
Le beau et l'utile \\
Le bonheur est-il l'absence de maux ? \\
Le bonheur est-il la fin de la vie ? \\
Le bonheur est-il le bien suprême ? \\
Le bonheur est-il le prix de la vertu ? \\
Le bonheur est-il un droit ? \\
Le bonheur n'est-il qu'un idéal ? \\
Le bonheur peut-il être le but de la politique ? \\
Le bonheur se mérite-t-il ? \\
Le bricolage \\
Le chef d'œuvre \\
Le désir a-t-il un objet ? \\
Le désir est-il aveugle ? \\
Le désir est-il ce qui nous fait vivre ? \\
Le désir est-il désir de l'autre ? \\
Le désir est-il l'essence de l'homme ? \\
Le désir est-il par nature illimité ? \\
Le désir peut-il être désintéressé ? \\
Le dialogue \\
Le droit et le devoir \\
Le geste et la parole \\
Le jugement moral \\
Le langage et la pensée \\
Le langage n'est-il qu'un instrument de communication ? \\
Le langage peut-il être un obstacle à la recherche de la vérité ? \\
Le mal \\
Le malheur est-il injuste ? \\
Le métier de politique \\
Le moi et la conscience \\
Le passé détermine-t-il notre présent ? \\
Le pessimisme \\
Le plaisir et la joie \\
Le pouvoir de l'État est-il arbitraire ? \\
Le pouvoir des mots \\
Le progrès \\
Le rationnel et l'irrationnel \\
Le récit historique \\
Le réel est-il ce que l'on croit ? \\
Le réel est-il inaccessible ? \\
Le réel est-il rationnel ? \\
Le réel obéit-il à la raison ? \\
Le réel se limite-t-il à ce que nous percevons ? \\
Le réel se réduit-il à ce que l'on perçoit ? \\
Le réel se réduit-il à l'objectivité ? \\
Le rôle de l'État est-il de faire régner la justice ? \\
Le rôle de l'État est-il de préserver la liberté de l'individu ? \\
Le rôle de l'historien est-il de juger ? \\
Le salaire \\
Les désirs ont-ils nécessairement un objet \\
Les événements historiques sont-ils de nature imprévisible ? \\
Les maladies de l'âme \\
Le souverain bien \\
Les progrès de la technique sont-ils nécessairement des progrès de la raison ? \\
Les religions peuvent-elles être objets de science ? \\
Les sciences permettent-elles de connaître la réalité-même ? \\
Le sublime \\
L'État a-t-il des intérêts propres ? \\
L'État a-t-il pour but de maintenir l'ordre ? \\
L'État a-t-il tous les droits ? \\
L'État doit-il éduquer le peuple ? \\
L'État doit-il être sans pitié ? \\
L'État doit-il préférer l'injustice au désordre ? \\
L'État doit-il se soucier de la morale ? \\
L'État doit-il veiller au bonheur des individus  ? \\
L'État est-il l'ennemi de la liberté ? \\
L'État est-il l'ennemi de l'individu ? \\
L'État est-il libérateur ? \\
L'État est-il toujours juste ? \\
L'État est-il un mal nécessaire ? \\
L'État n'est-il qu'un instrument de domination ? \\
L'État nous rend-il meilleurs ? \\
L'État peut-il être impartial ? \\
L'État peut-il renoncer à la violence ? \\
Le temps détruit-il tout ? \\
Le temps est-il la marque de notre impuissance ? \\
Le temps libre \\
Le temps nous appartient-il ? \\
Le temps passe-t-il ? \\
Le travail et le temps \\
Le travail fait-il de l'homme un être moral ? \\
L'être humain est-il par nature un être religieux \\
L'événement \\
L'événement historique a-t-il un sens par lui-même ? \\
Le vivant est-il entièrement connaissable ? \\
Le vivant est-il entièrement explicable ? \\
Le vivant n'est-il que matière ? \\
Le vivant n'est-il qu'une machine ingénieuse ? \\
Le vivant obéit-il à des lois ? \\
Le vivant obéit-il à une nécessité ? \\
L'expérience de pensée \\
L'histoire a-t-elle un commencement et une fin ? \\
L'histoire a-t-elle une fin ? \\
L'histoire a-t-elle un sens ? \\
L'histoire est-elle la connaissance du passé humain ? \\
L'histoire est-elle la mémoire de l'humanité ? \\
L'histoire est-elle le récit objectif des faits passés  ? \\
L'histoire est-elle le théâtre des passions ? \\
L'histoire est-elle une science comme les autres ? \\
L'histoire est-elle une science ? \\
L'histoire jugera-t-elle ? \\
L'histoire n'a-t-elle pour objet que le passé ? \\
L'histoire nous appartient-elle ? \\
L'histoire obéit-elle à des lois ? \\
L'histoire peut-elle être contemporaine ? \\
L'histoire se répète-t-elle ? \\
L'historien \\
L'historien peut-il être impartial ? \\
L'homme est-il le seul être à avoir une histoire ? \\
L'homme injuste peut-il être heureux ? \\
Liberté et responsabilité \\
L'idée de bonheur collectif a-t-elle un sens ? \\
L'illusion \\
L'imitation \\
L'inconscient et l'oubli \\
L'indicible \\
L'inexpérience \\
L'instrument et la machine \\
L'objectivité historique est-elle synonyme de neutralité ? \\
L'œuvre d'art doit-elle être belle ? \\
L'œuvre d'art nous apprend-elle quelque chose ? \\
L'ordre du vivant est-il façonné par le hasard ? \\
L'organisme \\
L'outil et la machine \\
Machine et organisme \\
Ne désire-t-on que ce dont on manque ? \\
Ne désirons-nous que ce qui est bon pour nous ? \\
Ne désirons-nous que les choses que nous estimons bonnes ? \\
Nos désirs nous appartiennent-ils ? \\
Nos désirs nous opposent-ils ? \\
N'y a-t-il de bonheur qu'éphémère ? \\
N'y a-t-il de rationalité que scientifique ? \\
Observation et expérience \\
Observer et comprendre \\
Observer et expérimenter \\
Par le langage, peut-on agir sur la réalité ? \\
Parler pour ne rien dire \\
Perçoit-on le réel ? \\
Peut-on avoir raisons contre les faits ? \\
Peut-on avoir raison tout seul ? \\
Peut-on cesser de désirer ? \\
Peut-on changer le cours de l'histoire ? \\
Peut-on connaître le vivant sans le dénaturer ? \\
Peut-on contredire l'expérience ? \\
Peut-on désirer l'absolu ? \\
Peut-on désirer l'impossible ? \\
Peut-on désirer sans souffrir ? \\
Peut-on désobéir à l'État ? \\
Peut-on dire que les hommes font l'histoire ? \\
Peut-on distinguer entre de bons et de mauvais désirs ? \\
Peut-on distinguer entre les bons et les mauvais désirs ? \\
Peut-on douter de tout ? \\
Peut-on échapper à ses désirs ? \\
Peut-on échapper au cours de l'histoire ? \\
Peut-on échapper au temps ? \\
Peut-on être heureux sans philosophie ? \\
Peut-on être indifférent à l'histoire ? \\
Peut-on expérimenter sur le vivant ? \\
Peut-on expliquer le vivant ? \\
Peut-on expliquer une œuvre d'art ? \\
Peut-on hiérarchiser les arts ? \\
Peut-on identifier le désir au besoin ? \\
Peut-on maîtriser le temps ? \\
Peut-on maîtriser ses désirs ? \\
Peut-on ne pas connaître son bonheur ? \\
Peut-on parler d'un progrès dans l'histoire ? \\
Peut-on penser un État sans violence ? \\
Peut-on prédire l'histoire ? \\
Peut-on promettre le bonheur ? \\
Peut-on ralentir la course du temps ? \\
Peut-on rendre raison de tout ? \\
Peut-on rendre raison du réel ? \\
Peut-on renoncer au bonheur ? \\
Peut-on réparer le vivant ? \\
Peut-on se passer de la religion ? \\
Peut-on se passer de l'État ? \\
Peut-on se tromper en se croyant heureux ? \\
Peut-on traiter un être vivant comme une machine ? \\
Peut-on vivre sans désir ? \\
Peut-on vouloir le bonheur d'autrui ? \\
Plaisir et bonheur \\
Plusieurs religions valent-elles mieux qu'une seule ? \\
Pourquoi chercher à connaître le passé ? \\
Pourquoi désire-t-on ce dont on n'a nul besoin ? \\
Pourquoi désirons-nous ? \\
Pourquoi écrit-on l'Histoire ? \\
Pourquoi étudier le vivant ? \\
Pourquoi étudier l'Histoire ? \\
Pourquoi faire son devoir ? \\
Pourquoi parle-t-on ? \\
Pourquoi refuse-t-on la conscience à l'animal ? \\
Pourquoi s'intéresser à l'histoire ? \\
Pouvons-nous désirer ce qui nous nuit ? \\
Prendre son temps, est-ce le perdre ? \\
Production et création \\
Produire et créer \\
Puis-je invoquer l'inconscient sans ruiner la morale ? \\
Puis-je ne pas vouloir ce que je désire ? \\
Qu'attendons-nous pour être heureux ? \\
Que désirons-nous quand nous désirons savoir ?Qu'est-ce qu'un événement historique ? \\
Que désirons-nous ? \\
Que devons-nous à l'État ? \\
Que doit la science à la technique ? \\
Quel est le contraire du travail ? \\
Quelle est la fonction première de l'État ? \\
Quelle est la place du hasard dans l'histoire ? \\
Quelle est la valeur du vivant ? \\
Que nous append l'histoire ? \\
Que nous apporte la vérité ? \\
Que nous apprend la définition de la vérité ? \\
Que nous apprend la fiction sur la réalité ? \\
Que peut l'État ? \\
Que peut-on savoir du réel ? \\
Que pouvons-nous espérer de la connaissance du vivant ? \\
Que sait la conscience ? \\
Que serions-nous sans l'État ? \\
Qu'est-ce qui fait l'unité du vivant ? \\
Qu'est ce qui rapproche le vivant de la machine ? \\
Qu'est-ce qu'un désir satisfait ? \\
Qu'est-ce qu'un État libre ? \\
Qu'est-ce qu'un fait historique ? \\
Qu'est-ce qu'un objet technique ? \\
Recourir au langage, est-ce renoncer à la violence ? \\
Serions-nous plus libres sans État ? \\
Sommes-nous faits pour le bonheur ? \\
Sommes-nous les jouets de l'histoire ? \\
Sommes-nous maîtres de nos désirs ? \\
Sommes-nous prisonniers de nos désirs ? \\
Sommes-nous prisonniers de notre histoire ? \\
Sommes-nous prisonniers du temps ? \\
Sommes-nous responsables de ce dont nous n'avons pas conscience ? \\
Sommes-nous responsables de nos désirs ? \\
Sommes-nous toujours conscients des causes de nos désirs ?` \\
Suffit-il d'être vertueux pour être heureux ? \\
Suis-je ce que j'ai conscience d'être ? \\
Suis-je ce que je fais ? \\
Technique et progrès \\
Tout ce qui est rationnel est-il raisonnable ? \\
Tout désir est-il manque ? \\
Tout est-il historique ? \\
Travail manuel et travail intellectuel \\
Un désir peut-il être coupable ? \\
Une éducation morale est-elle possible ? \\
Une expérience peut-elle être fictive ? \\
Une religion peut-elle se passer de pratiques ? \\
Une société sans religion est-elle possible ? \\
Un être vivant peut-il être comparé à une œuvre d'art ? \\
Un événement historique est-il toujours imprévisible ? \\
Une vie heureuse est-elle une vie de plaisirs ? \\
Un peuple se définit-il par son histoire ? \\
Vérité et réalité \\
Vivre, est-ce lutter pour survivre ? \\
Vivre, est-ce résister à la mort ? \\
Voir et savoir \\
Y a-t-il de bons et de mauvais désirs ? \\
Y a-t-il de l'indésirable ? \\
Y a-t-il des illusions de la conscience ? \\
Y a-t-il des leçons de l'histoire ? \\
Y a-t-il des obstacles à la connaissance du vivant ? \\
Y a-t-il des progrès en art ? \\
Y a-t-il du nouveau dans l'histoire ? \\
Y a-t-il quoi que ce soit de nouveau dans l'histoire ? \\
Y a-t-il un bon usage du temps ? \\
Y a-t-il une expérience du temps ? \\
Y a-t-il une hiérarchie du vivant ? \\
Y a-t-il une irréversibilité du temps ? \\
Y a-t-il une limite à la connaissance du vivant ? \\
Y a-t-il une limite au désir ? \\
Y a-t-il une logique dans l'histoire ? \\
Y a-t-il un État idéal ? \\
Y aura-t-il toujours des religions ? \\


\subsection{ENS A​/​L}
\label{sec-2-7}

\noindent
2+2 = 4 \\
2+2 pourrait-il ne pas être égal à 4 ? \\
Abuser du pouvoir \\
Aide-toi, le ciel t'aidera \\
Ai-je un corps ? \\
Aimer ce qui est beau \\
Aimer la vie \\
Aimer son prochain \\
Aimer son prochain comme soi-même \\
Aime ton prochain comme toi-même \\
Amitié et société \\
Amour et amitié \\
Analyse et synthèse \\
Apprend-on à aimer ? \\
Apprend-on à être artiste ? \\
Apprend-on à penser ? \\
Apprend-on à voir ? \\
Apprendre à vivre \\
Apprendre à voir \\
Après moi le déluge \\
Art et apparences \\
Art et connaissance \\
Art et morale \\
Art et politique \\
Art et vérité \\
Artiste et artisan \\
Art populaire et art savant \\
Arts de l'espace et arts du temps \\
A-t-on besoin de fonder la connaissance ? \\
A-t-on besoin de spécialistes en politique ? \\
A-t-on des devoirs envers soi-même ? \\
A-t-on intérêt à tout savoir ? \\
A-t-on le droit de mentir ? \\
A-t-on le droit de résister ? \\
A-t-on le droit de s'évader ? \\
A-t-on l'obligation de pardonner ? \\
Autrui \\
Autrui est-il aimable ? \\
Autrui est-il mon semblable ? \\
Autrui est-il un autre moi-même ? \\
Autrui est-il un autre moi ? \\
Autrui me connaît-il mieux que moi-même ? \\
Aux armes, citoyens ! \\
Avez-vous une âme ? \\
Avoir confiance \\
Avoir des principes \\
Avoir du goût \\
Avoir du jugement \\
Avoir le choix \\
Avoir le sens de la situation \\
Avoir peur des mots \\
Avoir un corps \\
Avons-nous besoin d'amis ? \\
Avons-nous besoin de rêver ? \\
Avons-nous des devoirs envers les générations futures ? \\
Avons-nous des devoirs envers nous-mêmes ? \\
Avons-nous le droit de juger autrui ? \\
Avons-nous le droit d'être heureux ? \\
Avons-nous raison d'exiger toujours des raisons ? \\
Avons-nous un corps ? \\
Avons-nous une âme ? \\
Avons-nous une obligation envers les générations à venir ? \\
Axiomatiser, est-ce fonder ? \\
À chacun son dû \\
À l'impossible nul n'est tenu \\
À quelles conditions peut-on dire qu'une action est historique ? \\
À quelles conditions un choix peut-il être rationnel ? \\
À quelles conditions une théorie est-elle scientifique ? \\
À quelque chose malheur est bon \\
À qui devons-nous obéir ? \\
À quoi bon les regrets ? \\
À quoi bon les romans ? \\
À quoi bon raconter des histoires ? \\
À quoi bon voyager ? \\
À quoi la valeur d'un homme se mesure-t-elle ? \\
À quoi reconnaît-on l'injustice ? \\
À quoi reconnaît-on une œuvre d'art ? \\
À quoi sert la connaissance du passé ? \\
À quoi sert la mémoire ? \\
À quoi sert le contrat social ? \\
À quoi servent les émotions ? \\
À quoi servent les expériences ? \\
À quoi servent les fictions ? \\
À quoi servent les mythes ? \\
À quoi servent les utopies ? \\
À quoi tenons-nous ? \\
À quoi tient l'autorité ? \\
Besoin et désir \\
Besoins et désirs \\
Bonheur de chacun bonheur de tous \\
Catégories de langue, catégories de pensée \\
Cause et loi \\
Cause et raison \\
Causes et raisons \\
Ce qui vaut en théorie vaut-il toujours en pratique ? \\
Certitude et probabilité \\
Certitude et vérité \\
Changer d'opinion \\
Choisir \\
Choisit-on son corps ? \\
Chose et personne \\
Choses et personnes \\
Citoyen du monde ? \\
Civilisé, barbare, sauvage \\
Classer \\
Classicisme et romantisme \\
Commémorer \\
Comment conduire ses pensées ? \\
Comment croire au progrès ? \\
Comment dire la vérité ? \\
Comment dire l'individuel ? \\
Comment distinguer désirs et besoins ? \\
Comment être naturel ? \\
Comment évaluer la qualité de la vie ? \\
Comment fonder la propriété ? \\
Comment mesurer une sensation ? \\
Comment mesurer ? \\
Comment penser le mouvement ? \\
Comment percevons-nous l'espace ? \\
Comment peut-on définir la politique ? \\
Comment retrouver la nature ? \\
Comment sait-on qu'une chose existe ? \\
Comment savoir que l'on est dans l'erreur ? \\
Comment se mettre à la place d'autrui ? \\
Comment s'entendre ? \\
Comment s'orienter dans la pensée ? \\
Comment voyager dans le temps ? \\
Comparer les cultures \\
Comprendre le sens d'un texte \\
Concept et image \\
Concept et intuition \\
Concept et métaphore \\
Conflit et démocratie \\
Conflit et liberté \\
Connaissance et croyance \\
Connaissons-nous la réalité telle qu'elle est ? \\
Connais-toi toi-même \\
Connaît-on la vie ou connaît-on le vivant ? \\
Connaître autrui \\
Connaître l'infini \\
Connaître ses origines \\
Conscience et existence \\
Construire l'espace \\
Contempler \\
Contingence et nécessité \\
Continuité et discontinuité \\
Corps et esprit \\
Création et fabrication \\
Créer \\
Critiquer \\
Croire et savoir \\
Croit-on comme on veut ? \\
Croyance et connaissance \\
Culpabilité et responsabilité \\
Dans quelle mesure l'art est-il un fait social ? \\
Décider \\
Définir \\
De quoi avons-nous besoin ? \\
De quoi rit-on ? \\
De quoi sommes-nous coupables ? \\
De quoi sommes-nous responsables ? \\
De quoi suis-je responsable ? \\
Description et explication \\
Des inégalités peuvent-elles être justes ? \\
Désir et volonté \\
Désobéir \\
Détruire et construire \\
Devons-nous tenir certaines connaissances pour acquises ? \\
Dieu est-il mortel ? \\
Dieu est-il une invention humaine ? \\
Dire et faire \\
Dire et montrer \\
Dire l'individuel \\
Dire « je » \\
Disposer de son corps \\
Distinguer \\
Dois-je obéir à la loi ? \\
Doit-on bien juger pour bien faire ? \\
Doit-on chasser les artistes de la cité ? \\
Doit-on respecter la nature ? \\
Doit-on se mettre à la place d'autrui ? \\
Doit-on toujours rechercher la vérité ? \\
Doit-on tout contrôler ? \\
Don Juan \\
Donner \\
Donner des preuves \\
Donner sa parole \\
Donner son assentiment \\
Douter \\
D'où viennent nos idées ? \\
Dressage et éducation \\
Droit et coutume \\
Du passé pouvons-nous faire table rase ? \\
Échanger \\
Économie et société \\
Éducation et instruction \\
Éduquer et instruire \\
Égalité et solidarité \\
En finir avec les préjugés \\
Enquêter \\
En quoi le bien d'autrui m'importe-t-il ? \\
En quoi une œuvre d'art est-elle moderne ? \\
Enseigner et éduquer \\
Entendre \\
Entendre raison \\
Entre l'art et la nature, qui imite l'autre ? \\
Essence et existence \\
Est-ce à la fin que le sens apparaît ? \\
Est-ce la mémoire qui constitue mon identité ? \\
Est-ce seulement l'intention qui compte ? \\
Est-il difficile de savoir ce que l'on veut ? \\
Est-il difficile de savoir ce qu'on veut ? \\
Est-il difficile de vivre en société ? \\
Est-il juste de payer l'impôt ? \\
Est-il naturel de s'aimer soi-même ? \\
Est-il nécessaire d'espérer pour entreprendre ? \\
Est-il toujours avantageux de promouvoir son propre intérêt ? \\
Est-il toujours meilleur d'avoir le choix ? \\
Est-il utile d'avoir mal ? \\
Est-il vrai que nous ne nous tenons jamais au temps présent ? \\
Est-on libre de ne pas vouloir ce que l'on veut ? \\
Est-on responsable de son passé ? \\
État et société \\
Éthique et esthétique \\
Être citoyen du monde \\
Être cultivé, est-ce tout connaître ? \\
Être cynique \\
Être de son temps \\
Être équitable \\
Être et devenir \\
Être exemplaire \\
Être hors de soi \\
Être impossible \\
Être logique avec soi-même \\
Être majeur \\
Être malade \\
Être moderne \\
Être ou ne pas être, est-ce la question ? \\
Être précurseur \\
Être raisonnable, est-ce accepter la réalité telle qu'elle est ? \\
Être relativiste \\
Être sceptique \\
Être soi \\
Être soi-même \\
Être systématique \\
Évolution et progrès \\
Évolution et révolution \\
Exister \\
Existe-t-il de faux besoins ? \\
Existe-t-il des choses sans prix ? \\
Existe-t-il des croyances collectives ? \\
Existe-t-il des devoirs envers soi-même ? \\
Existe-t-il des signes naturels ? \\
Existe-t-il un art de penser ? \\
Existe-t-il un droit de mentir ? \\
Expérience et expérimentation \\
Expérience et vérité \\
Expérience, expérimentation \\
Expliquer \\
Expliquer et comprendre \\
Expliquer et interpréter \\
Faire apprendre \\
Faire douter \\
Faire justice \\
Faire la loi \\
Faire la paix \\
Faire table rase \\
Faire voir \\
Fait et théorie \\
Familles, je vous hais \\
Faut-il aimer son prochain ? \\
Faut-il aller toujours plus vite ? \\
Faut-il apprendre à voir ? \\
Faut-il avoir peur de la nature ? \\
Faut-il changer ses désirs plutôt que l'ordre du monde ? \\
Faut-il chasser les poètes ? \\
Faut-il condamner la fiction ? \\
Faut-il condamner la rhétorique ? \\
Faut-il contrôler les mœurs ? \\
Faut-il défendre ses convictions \\
Faut-il dépasser les apparences ? \\
Faut-il désespérer de l'humanité ? \\
Faut-il des frontières ? \\
Faut-il des héros ? \\
Faut-il des outils pour penser ? \\
Faut-il être à l'écoute du corps ? \\
Faut-il être connaisseur pour apprécier une œuvre d'art ? \\
Faut-il être idéaliste ? \\
Faut-il être logique avec soi-même ? \\
Faut-il être original ? \\
Faut-il être positif ? \\
Faut-il faire de nécessité vertu ? \\
Faut-il forcer les gens à participer à la vie politique ? \\
Faut-il garder ses illusions ? \\
Faut-il imaginer que nous sommes heureux ? \\
Faut-il laisser parler la nature ? \\
Faut-il lire des romans ? \\
Faut-il opposer nature et culture ? \\
Faut-il partager la souveraineté ? \\
Faut-il perdre son temps ? \\
Faut-il protéger la dignité humaine ? \\
Faut-il protéger la nature ? \\
Faut-il rechercher la simplicité ? \\
Faut-il rechercher l'harmonie ? \\
Faut-il regretter l'équivocité du langage ? \\
Faut-il renoncer à la certitude ? \\
Faut-il renoncer à l'impossible ? \\
Faut-il respecter la nature ? \\
Faut-il respecter les convenances ? \\
Faut-il rester impartial ? \\
Faut-il rester naturel ? \\
Faut-il sauver des vies à tout prix ? \\
Faut-il se contenter de peu ? \\
Faut-il se délivrer de la peur ? \\
Faut-il s'efforcer d'être moins personnel ? \\
Faut-il se fier à ce que l'on ressent ? \\
Faut-il se fier aux apparences ? \\
Faut-il se poser des questions métaphysiques ? \\
Faut-il se réjouir d'exister ? \\
Faut-il suivre ses intuitions ? \\
Faut-il surmonter son enfance ? \\
Faut-il toujours être en accord avec soi-même ? \\
Faut-il toujours garder espoir ? \\
Faut-il un commencement à tout ? \\
Faut-il une guerre pour mettre fin à toutes les guerres ? \\
Faut-il une théorie de la connaissance ? \\
Faut-il vaincre ses désirs plutôt que l'ordre du monde ? \\
Faut-il vivre dangereusement ? \\
Faut-il voir pour croire ? \\
Faut-il vouloir la paix de l'âme ? \\
Faut-il vouloir la transparence ? \\
Foi et raison \\
Forme et contenu \\
Fuir la civilisation \\
Génie et technique \\
Genre et espèce \\
Gouvernement des hommes et administration des choses \\
Habiter \\
Habiter le monde \\
Hériter \\
Hésiter \\
Histoire et fiction \\
Histoire et géographie \\
Honte, pudeur, embarras \\
Humour et ironie \\
Ici et maintenant \\
Identité et changement \\
Identité et égalité \\
Ignorer \\
Image, signe, symbole \\
Imagination et conception \\
Imagination et raison \\
Imaginer \\
Imiter \\
Information et opinion \\
Instruire et éduquer \\
Interpréter \\
Interroger et répondre \\
Intuition et déduction \\
Intuition et intellection \\
Je \\
Je est un autre \\
Jouer \\
Jouer son rôle \\
Juger \\
Juger et sentir \\
Jusqu'à quel point sommes-nous responsables de nos passions ? \\
Jusqu'à quel point suis-je mon propre maître ? \\
Jusqu'où interpréter ? \\
Jusqu'où peut-on dialoguer ? \\
Justice et force \\
Justice et utilité \\
Justice et vengeance \\
Justifier \\
La banalité \\
La barbarie \\
La beauté \\
La beauté du geste \\
La beauté est-elle affaire de goût ? \\
La belle âme \\
La bêtise \\
La bêtise et la méchanceté sont-elles liées intrinsèquement ? \\
La bienséance \\
La bienveillance \\
La biographie \\
L'abondance \\
La bonne conscience \\
La bonne éducation \\
L'absence \\
L'absolu \\
L'abstraction \\
L'abstrait et le concret \\
L'absurde \\
L'abus de pouvoir \\
La causalité \\
La causalité historique \\
La cause \\
La cause première \\
L'accès à la vérité \\
L'accident \\
La censure \\
La certitude \\
La chance \\
La citation \\
La citoyenneté \\
La civilisation \\
La clarté \\
La classification \\
La classification des arts \\
La cohérence \\
La cohérence est-elle une vertu ? \\
La colère \\
La comédie \\
La communauté \\
La communauté des savants \\
La communauté scientifique \\
La communication \\
La comparaison \\
La compréhension \\
La confiance \\
La confusion \\
La connaissance de Dieu \\
La connaissance de soi \\
La connaissance des passions \\
La connaissance du futur \\
La connaissance du monde \\
La connaissance du passé \\
La connaissance et la morale \\
La connaissance peut-elle être pratique ? \\
La connaissance peut-elle se passer de l'imagination ? \\
La conquête de l'espace \\
La conscience a-t-elle des moments ? \\
La conscience de soi \\
La conscience morale \\
La conscience universelle \\
La considération de l'utilité doit-elle déterminer toutes nos actions ? \\
La contemplation \\
La contingence \\
La contingence du monde \\
La contradiction \\
La contrainte \\
La contrainte en art \\
La conversation \\
La conversion \\
La couleur \\
La coutume \\
La création \\
La crédibilité \\
La crise \\
La critique \\
La critique d'art \\
La critique de l'État \\
La croissance \\
La croyance \\
La croyance religieuse se distingue-t-elle des autres formes de croyance ? \\
La cruauté \\
L'acte et la puissance \\
L'acte et l'œuvre \\
L'acte gratuit \\
L'action collective \\
L'action et son contexte \\
L'action humaine nécessite-t-elle la contingence du monde ? \\
L'actualité \\
La culpabilité \\
La culture \\
La culture artistique \\
La culture est-elle une question politique ? \\
La culture est-elle une seconde nature ? \\
La culture et les cultures \\
La curiosité \\
La danse \\
La décadence \\
La déception \\
La décision \\
La décision a-t-elle besoin de raisons ? \\
La déduction \\
La défense de la liberté \\
La définition \\
La délibération \\
La démesure \\
La démocratie \\
La démocratie a-t-elle des limites ? \\
La démocratie a-t-elle une histoire ? \\
La démocratie peut-elle se passer de représentation ? \\
La démonstration \\
La dérision \\
La désobéissance \\
La désobéissance civile \\
La deuxième chance \\
La déviance \\
La dialectique \\
La différence \\
La différence des sexes \\
La différence des sexes est-elle une question philosophique ? \\
La différence homme-femme \\
La difformité \\
La dignité \\
La direction de l'esprit \\
La discipline \\
La disgrâce \\
La disposition \\
La dissimulation \\
La distinction \\
La diversité des langues \\
La diversité des langues est-elle une diversité des pensées ? \\
La diversité des religions \\
La diversité des sciences \\
La division du travail \\
L'admiration \\
La douleur \\
La douleur est-elle utile ? \\
La durée \\
La faiblesse \\
La faiblesse de croire \\
La faiblesse de la volonté \\
La famille \\
La famille est-elle une institution politique ? \\
La fatalité \\
La fatigue \\
La fausseté \\
La faute \\
La faute et le péché \\
La faute et l'erreur \\
La fête \\
L'affirmation \\
La fiction \\
La fidélité \\
La fierté \\
La fièvre \\
La finalité \\
La fin de la guerre \\
La fin de la politique \\
La fin de l'art \\
La fin de l'histoire \\
La fin de l'homme \\
La fin du monde \\
La finitude \\
La fin justifie-t-elle les moyens ? \\
La foi est-elle irrationnelle ? \\
La folie \\
La fonction des exemples \\
La force de conviction \\
La force de la croyance \\
La force de la loi \\
La force de l'habitude \\
La force des choses \\
La force des faibles \\
La force des lois \\
La force et le droit \\
La forme \\
La fortune \\
La foule \\
La fragilité \\
La franchise \\
La franchise est-elle une vertu ? \\
La frivolité \\
La gauche et la droite \\
L'âge d'or \\
La généalogie \\
La généralisation \\
La générosité \\
La gloire \\
La gloire est-elle un bien ? \\
La grâce \\
La grammaire \\
La grandeur \\
La gratuité \\
L'agressivité \\
La grossièreté \\
La guerre \\
La guerre est-elle la politique continuée par d'autres moyens ? \\
La guerre et la paix \\
La guerre met-elle fin au droit ? \\
La haine \\
La haine de la raison \\
La hiérarchie \\
La honte \\
Laisser mourir, est-ce tuer ? \\
La jalousie \\
La jeunesse \\
La jeunesse est mécontente \\
La joie \\
La joie de vivre \\
La justice et l'égalité \\
La justice internationale \\
La justice sociale \\
La justification \\
La laideur \\
La légitimité démocratique \\
La lettre et l'esprit \\
La liberté d'autrui \\
La liberté de la volonté \\
La liberté de penser \\
La liberté peut-elle être une illusion ? \\
La liberté peut-elle s'aliéner ? \\
L'aliénation \\
La limite \\
La logique est-elle un art de penser ? \\
La logique est-elle utile à la métaphysique ? \\
La loi \\
La loi du plus fort \\
La loi et la coutume \\
La loi et la règle \\
La loi et les mœurs \\
La loi et l'ordre \\
La loi peut-elle être injuste ? \\
L'alter ego \\
L'altruisme \\
La machine \\
La magie \\
La magie des mots \\
La magie peut-elle être efficace ? \\
La main \\
La maîtrise de soi \\
La majorité \\
La maladie \\
La malveillance \\
La marchandise \\
La marge \\
La marginalité \\
La mathématisation du réel \\
La matière \\
La mauvaise foi \\
La mauvaise volonté \\
L'ambition \\
L'âme \\
La méchanceté \\
La méditation \\
L'âme et le corps \\
La méfiance \\
La mélancolie \\
L'amélioration des hommes peut-elle être considérée comme un objectif politique ? \\
La mémoire \\
La mémoire collective \\
La mémoire et l'histoire \\
La mémoire et l'oubli \\
La mesure \\
La métaphore \\
La métaphysique est-elle une science ? \\
La méthode \\
L'ami \\
L'ami du prince \\
La minorité \\
La misère \\
L'amitié \\
L'amitié relève-t-elle d'une décision ? \\
La mode \\
La modernité \\
La morale est-elle objet de science ? \\
La morale peut-elle être naturelle ? \\
La morale s'apprend-elle ? \\
La moralité est-elle affaire de principes ou de conséquences ? \\
La mort \\
La mort a-t-elle un sens ? \\
La mort de Dieu \\
La mort de l'art \\
La mort de l'homme \\
L'amour de l'argent \\
L'amour de la vérité \\
L'amour de l'humanité \\
L'amour de soi \\
L'amour du destin \\
L'amour et la haine \\
L'amour et l'amitié \\
L'amour implique-t-il le respect ? \\
L'amour peut-il être un devoir ? \\
L'amour propre \\
L'amour-propre \\
La multitude \\
La musique est-elle un langage ? \\
L'anachronisme \\
La naissance \\
La naïveté \\
La naïveté est-elle une vertu ? \\
L'analogie \\
L'analyse \\
L'analyse et la synthèse \\
L'anarchie \\
La nature artiste \\
La nature a-t-elle des droits ? \\
La nature a-t-elle une histoire ? \\
La nature est-elle belle ? \\
La nature est-elle bien faite ? \\
La nature est-elle écrite en langage mathématique ? \\
La nature est-elle sacrée ? \\
La nature est-elle sans histoire ? \\
La nature est-elle un système ? \\
La nature existe-t-elle ? \\
La nature morte \\
La nature ne fait pas de saut \\
La nature peut-elle être un modèle ? \\
La nécessité \\
La nécessité de l'oubli \\
La nécessité fait-elle loi ? \\
La négation \\
La négligence \\
La négligence est-elle une faute ? \\
La négociation \\
La neige est-elle blanche ? \\
Langage et pensée \\
L'angoisse \\
Langue et parole \\
L'animal \\
L'animal et la bête \\
L'animalité \\
L'animalité de l'animal, l'animalité de l'homme \\
La norme \\
La nostalgie \\
La notion de comportement \\
La notion de monde \\
La notion de nature humaine \\
La notion de nature humaine a-t-elle un sens ? \\
La notion de système \\
L'anthropocentrisme \\
L'anticipation \\
La nudité \\
La nuit \\
La ou les vertus ? \\
La paix \\
La paix est-elle l'absence de guerre ? \\
La paresse \\
La parole \\
La parole et l'écriture \\
La parole peut-elle être une arme ? \\
La partie et le tout \\
La passion \\
La patience \\
La pauvreté \\
La peine de mort \\
La peine de mort est-elle juste, injuste, et pourquoi ? \\
La peinture apprend-elle à voir ? \\
La peinture des mœurs \\
La pensée échappe-t-elle à la grammaire ? \\
La pensée est-elle une activité assimilable à un travail ? \\
La pensée formelle \\
La pensée obéit-elle à des lois ? \\
La perception \\
La perception est-elle source de connaissance ? \\
La perfection \\
La performance \\
La personne \\
La perspective \\
La perversion \\
La peur \\
La peur des machines \\
La philosophie est-elle abstraite ? \\
La philosophie peut-elle être populaire ? \\
La photographie est-elle un art ? \\
La pitié \\
La place d'autrui \\
La pluralité \\
La pluralité des arts \\
La pluralité des interprétations \\
La pluralité des langues \\
La pluralité des mondes \\
La pluralité des opinions \\
La poésie pense-t-elle ? \\
La politesse \\
La politique est-elle l'affaire de tous ? \\
La politique est-elle une science ? \\
La politique peut-elle se passer de croyance ? \\
La pornographie \\
L'apparence \\
La précaution \\
La précaution peut-elle être un principe ? \\
La présence \\
La présence de l'œuvre d'art \\
La présomption \\
La pression du groupe \\
La preuve \\
La preuve expérimentale \\
La prévision \\
La prière \\
L'\emph{a priori} \\
La prise du pouvoir \\
La probabilité \\
La promesse \\
La propriété \\
La propriété, est-ce un vol ? \\
La propriété est-elle un droit ? \\
La prose du monde \\
La protection sociale \\
La providence \\
La prudence \\
La psychanalyse est-elle une science ? \\
La publicité \\
La pudeur \\
La puissance \\
La puissance de l'image \\
La puissance de l'imagination \\
La puissance du langage \\
La pulsion \\
La punition \\
La qualité \\
La quantité \\
La quantité et la qualité \\
La question de l'origine \\
La question : « qui ? » \\
La raison a-t-elle des limites ? \\
La raison a-t-elle une histoire ? \\
La raison des mythes \\
La raison d'État \\
La raison d'être \\
La raison est-elle l'esclave des passions ? \\
La raison s'oppose-t-elle aux passions ? \\
La rareté \\
La rationalité des émotions \\
L'arbitraire \\
La réalité du futur \\
La réalité du mouvement \\
La réalité du passé \\
La réalité du possible \\
La réalité du rêve \\
La réalité virtuelle \\
La recherche de l'authenticité \\
La recherche de la vérité \\
La recherche des causes \\
La réciprocité \\
La reconnaissance \\
La réflexion \\
La réforme \\
La règle \\
La régularité \\
La relation \\
La religion \\
La religion est-elle l'opium du peuple ? \\
La religion peut-elle être civile ? \\
La renaissance \\
La rencontre \\
La répétition \\
La représentation \\
La reproduction \\
La résistance \\
La responsabilité \\
La responsabilité peut-elle être collective ? \\
La ressemblance \\
La rêverie \\
La révolte \\
La révolte peut-elle être un droit ? \\
La révolution \\
L'argent \\
L'argument d'autorité \\
La richesse \\
L'art abstrait \\
L'art a-t-il une histoire ? \\
L'art de la discussion \\
L'art de persuader \\
L'art de vivre \\
L'art d'inventer \\
L'art doit-il divertir ? \\
L'art donne-t-il à penser ? \\
L'art du corps \\
L'art du mensonge \\
L'art est-il au service du beau ? \\
L'art est-il imitatif ? \\
L'art est-il mensonger ? \\
L'art est-il un mode de connaissance ? \\
L'art et la morale \\
L'art et la nature \\
L'art et le temps \\
L'artificiel \\
L'artiste de soi-même \\
L'artiste est-il un créateur ? \\
L'artiste et l'artisan \\
L'artiste et le savant \\
L'artiste recherche-t-il le beau ? \\
L'art permet-il un accès au divin ? \\
L'art peut-il être brut ? \\
L'art peut-il être populaire ? \\
L'art peut-il se passer de la beauté ? \\
L'art peut-il se passer d'œuvres ? \\
L'art pour l'art \\
L'art vise-t-il le beau ? \\
La ruse \\
La sagesse \\
La santé \\
La santé est-elle un droit ou un devoir ? \\
La satisfaction \\
L'ascétisme \\
La science est-elle austère ? \\
La science exclut-elle l'imagination ? \\
La science pense-t-elle ? \\
La science peut-elle se passer de métaphysique ? \\
La seconde nature \\
La sécurité \\
La séduction \\
La sensation \\
La sensibilité \\
La servitude \\
La sexualité \\
La signification \\
La signification des mots \\
L'asile de l'ignorance \\
La simplicité \\
La simulation \\
La sincérité \\
La singularité \\
La sociabilité \\
La société précède-t-elle l'individu ? \\
La solidarité \\
La solitude \\
La souffrance \\
La souffrance a-t-elle une valeur morale ? \\
La souffrance d'autrui \\
La souveraineté \\
La souveraineté peut-elle être déléguée \\
La souveraineté peut-elle être limitée ? \\
La souveraineté peut-elle se partager ? \\
La spéculation \\
La spontanéité \\
La structure \\
La subjectivité \\
La substance \\
La subtilité \\
La superstition \\
La surface et la profondeur \\
La sympathie \\
La tautologie \\
La technique \\
La technique est-elle dangereuse ? \\
La technique est-elle libératrice ? \\
La technique est-elle moralement neutre ? \\
La technique est-elle neutre ? \\
La technique peut-elle améliorer l'homme ? \\
La technique repose-t-elle sur le génie du technicien ? \\
La télévision \\
La tempérance \\
La tentation \\
La terreur \\
L'athéisme \\
La théorie peut-elle nous égarer ? \\
La tolérance \\
La totalité \\
La tradition \\
La traduction \\
La trahison \\
La tranquillité \\
La transcendance \\
La transmission de pensée \\
La transparence \\
La tristesse \\
L'attente \\
L'attention \\
L'au-delà \\
L'authenticité \\
L'autobiographie \\
L'autocritique \\
L'autonomie \\
L'autorité \\
L'autorité de la loi \\
L'autorité de la science \\
L'autorité des lois \\
La valeur \\
La valeur de la pitié \\
La valeur de l'art \\
La valeur de la vie \\
La valeur de l'exemple \\
La valeur de l'hypothèse \\
La valeur des images \\
La valeur d'une œuvre \\
La valeur du travail \\
La valeur et le prix \\
L'avant-garde \\
La veille et le sommeil \\
La vengeance \\
L'avenir \\
La vérification \\
La vérité \\
La vérité a-t-elle une histoire ? \\
La vérité en art \\
La vérité est-elle intemporelle ? \\
La vérité est-elle triste ? \\
La vérité mathématique \\
La vérité peut-elle être tolérante ? \\
La vertu \\
La vertu peut-elle s'enseigner ? \\
La vertu s'enseigne-t-elle ? \\
L'aveu \\
La vie \\
La vie a-t-elle un sens ? \\
La vie en société est-elle naturelle à l'homme ? \\
La vie est-elle l'objet des sciences de la vie ? \\
La vie est-elle une valeur ? \\
La vie est-elle un roman ? \\
La vie est-elle un songe ? \\
La vieillesse \\
La vie sexuelle est-elle volontaire ? \\
La vie sociale est-elle une comédie ? \\
La ville \\
La ville et la campagne \\
La violence \\
La vision et le toucher \\
L'avocat du diable \\
La voix de la conscience \\
La volonté de savoir \\
La volonté générale est-elle la volonté de tous ? \\
La volonté peut-elle être générale ? \\
La vulgarisation \\
La vulgarité \\
Le bavardage \\
Le beau a-t-il une histoire ? \\
Le beau est-il universel ? \\
Le beau et le bien \\
Le beau et le bien sont-ils, au fond, identiques ? \\
Le beau et le joli \\
Le beau peut-il être bizarre ? \\
Le besoin de philosophie \\
Le besoin de sens \\
Le bien commun \\
Le bien d'autrui \\
Le bien et le mal \\
Le bien et l'utile \\
Le bien-être \\
Le bon goût \\
Le bonheur \\
Le bonheur des méchants \\
Le bonheur du juste \\
Le bonheur est-il affaire de volonté ? \\
Le bonheur est-il affaire privée ? \\
Le bon sens \\
Le bouc émissaire \\
Le bruit et la musique \\
Le but de l'association politique \\
Le calcul \\
Le capitalisme \\
Le cas de conscience \\
L'échange \\
Le changement \\
Le chaos \\
Le charme \\
Le châtiment \\
Le choix \\
Le choix des moyens \\
Le citoyen \\
Le clair et l'obscur \\
Le classicisme \\
Le cliché \\
Le cœur et la raison \\
Le comédien \\
Le comique \\
Le commencement \\
Le comment et le pourquoi \\
Le commerce adoucit-il les mœurs ? \\
Le commerce des idées \\
Le compromis \\
Le concept \\
Le concret et l'abstrait \\
Le conditionnel \\
Le conflit \\
Le conflit ? \\
Le confort intellectuel \\
L'économie est-elle une science ? \\
Le conscient et l'inconscient \\
Le conseil \\
Le consensus \\
Le consentement \\
Le corps est-il le reflet de l'âme ? \\
Le corps et l'esprit \\
Le corps peut-il être objet d'art ? \\
Le cosmopolitisme \\
Le courage \\
Le courage de penser \\
Le crime contre l'humanité \\
L'écrit et l'oral \\
L'écriture \\
L'écriture et la pensée \\
Le cynisme \\
Le dandysme \\
Le dégoût \\
Le dérèglement \\
Le désenchantement \\
Le désespoir \\
Le désintéressement \\
Le désir \\
Le désir de gloire \\
Le désir de savoir est-il naturel ? \\
Le désir est-il sans limite ? \\
Le désir et la loi \\
Le désordre \\
Le despotisme \\
Le dessin et la couleur \\
Le destin \\
Le détachement \\
Le détail \\
Le déterminisme \\
Le devenir \\
Le dialogue \\
Le dieu des philosophes \\
Le Dieu des philosophes \\
Le divertissement \\
Le dogmatisme \\
Le don \\
Le don est-il toujours généreux ? \\
Le donné \\
Le double \\
Le doute \\
Le doute peut-il être méthodique ? \\
Le droit à la révolte \\
Le droit de propriété \\
Le droit de punir \\
Le droit de résistance \\
Le droit des peuples à disposer d'eux-mêmes \\
Le droit du plus fort \\
Le droit naturel \\
Le droit peut-il être flexible ? \\
Le dualisme \\
Le factice \\
Le fait \\
Le fait de vivre constitue-t-il un bien en soi ? \\
Le fanatisme \\
Le fantastique \\
Le fatalisme \\
Le faux \\
Le féminin \\
Le féminin et le masculin \\
Le féminisme \\
L'effet et la cause \\
Le fil conducteur \\
Le fini \\
Le fini et l'infini \\
Le fond et la forme \\
Le formalisme \\
Le futur est-il contingent ? \\
L'égalité \\
L'égalité des chances \\
L'égalité est-elle souhaitable ? \\
L'égalité est-elle une condition de la liberté ? \\
Légalité et causalité \\
Légalité et légitimité \\
Le général et le particulier \\
Le génie \\
Le genre \\
Le genre et l'espèce \\
Le geste \\
L'égoïsme \\
Le goût \\
Le goût du risque \\
Le goût est-il affaire d'éducation ? \\
Le handicap \\
Le hasard \\
Le hasard est-il injuste ? \\
Le hasard et la nécessité \\
Le hasard existe-t-il ? \\
Le hasard fait-il bien les choses ? \\
Le haut \\
Le haut et le bas \\
Le héros \\
Le jeu \\
Le jeu et le sérieux \\
Le jugement de goût est-il désintéressé ? \\
Le jugement de valeur \\
Le juste et le bien \\
Le langage de la peinture \\
Le langage est-il assimilable à un outil ? \\
Le langage est-il l'auxiliaire de la pensée ? \\
Le langage est-il un instrument ? \\
Le langage et le réel \\
L'élection \\
Le légal et le légitime \\
L'élégance \\
Le légitime et le légal \\
Le libre arbitre \\
Le lien social \\
Le lieu commun \\
Le lieu et l'espace \\
Le livre \\
Le logique \\
Le loisir \\
Le luxe \\
Le lyrisme \\
Le maître et l'esclave \\
Le malentendu \\
Le masculin \\
Le masculin et le féminin \\
Le matérialisme \\
Le mauvais goût \\
L'embarras du choix \\
Le mécanisme \\
Le méchant \\
Le méchant peut-il être heureux ? \\
Le meilleur \\
Le meilleur des mondes \\
Le meilleur régime \\
Le même et l'autre \\
Le mensonge \\
Le mépris \\
Le mérite \\
Le mérite et les talents \\
Le merveilleux \\
Le mien et le tien \\
Le milieu \\
Le miracle \\
Le miroir \\
Le misanthrope \\
Le modèle \\
Le moi \\
Le moi est-il haïssable ? \\
Le moindre mal \\
Le moi n'est-il qu'une idée ? \\
Le monde \\
Le monde des idées \\
Le monde est-il écrit en langage mathématique ? \\
Le monde est-il ma représentation ? \\
Le monde est-il un théâtre ? \\
Le monde extérieur existe-t-il ? \\
Le monstre \\
Le moralisme \\
Le mot d'esprit \\
Le mot et la chose \\
L'émotion \\
L'émotion esthétique \\
Le mot juste \\
Le mot vie a-t-il plusieurs sens ? \\
Le mouvement \\
L'empirisme \\
L'emploi du temps \\
Le multiple \\
Le musée \\
Le Musée \\
Le mythe \\
Le naturel et l'artificiel \\
L'encyclopédie \\
Le néant \\
Le néant est-il ? \\
Le nécessaire et le contingent \\
Le nécessaire et le superflu \\
Le néologisme \\
L'énergie du désespoir \\
L'enfance \\
L'enfant \\
L'enfant et l'adulte \\
L'enfer est pavé de bonnes intentions \\
L'engagement politique \\
L'énigme \\
Le nihilisme \\
L'ennemi \\
L'ennemi intérieur \\
L'ennui \\
Le nombre \\
Le nom et le verbe \\
Le nominalisme \\
Le nom propre \\
Le non-sens \\
L'enquête \\
L'enseignement peut-il se passer d'exemples ? \\
L'enthousiasme \\
Le nu et la nudité \\
L'envie \\
Le pacifisme \\
Le paradigme \\
Le paradoxe \\
Le pardon \\
Le pardon et l'oubli \\
Le pardon peut-il être une obligation ? \\
Le pari \\
Le passage à l'acte \\
Le passé \\
Le passé a-t-il un intérêt ? \\
Le passé est-il réel ? \\
Le patriotisme \\
Le paysage \\
Le pessimisme \\
Le peuple \\
Le peuple est-il bête ? \\
Le philosophe s'écarte-t-il du réel ? \\
Le plagiat \\
Le plaisir \\
Le plaisir d'avoir mal \\
Le plaisir d'être libre \\
Le plaisir esthétique \\
Le plaisir est-il immoral ? \\
Le plaisir et la douleur \\
Le plaisir peut-il être immoral ? \\
Le pluralisme \\
Le plus grand bonheur pour le plus grand nombre \\
Le poids de la culture \\
Le poids du passé \\
Le point de vue \\
Le portrait \\
Le possible \\
Le possible et le réel \\
Le possible et le virtuel \\
Le possible et l'impossible \\
Le pourquoi et le comment \\
Le pouvoir corrompt-il ? \\
Le pouvoir des images \\
Le pouvoir des mots \\
Le pouvoir peut-il être limité ? \\
Le pragmatisme \\
Le préjugé \\
Le premier \\
Le présent \\
L'épreuve \\
Le primitif \\
Le principe \\
Le privé et le public \\
Le prix des choses \\
Le probable \\
Le progrès \\
Le progrès des sciences \\
Le provisoire \\
Le public et le privé \\
L'équité \\
Le quotidien \\
Le raisonnable et le rationnel \\
Le rationalisme peut-il être une passion ? \\
Le rationnel \\
Le rationnel et le raisonnable \\
Le réalisme \\
Le récit \\
Le recours à l'Histoire \\
Le réel et le nécessaire \\
Le réel et l'idéal \\
Le réel et l'irréel \\
Le réel peut-il échapper à la logique ? \\
Le refus de la vérité \\
Le regard \\
Le règne des passions \\
Le regret \\
Le remords \\
Le repos \\
Le respect \\
Le respect de la nature \\
Le respect de soi \\
Le respect de soi-même \\
Le ressentiment \\
Le retour à la nature \\
Le rêve \\
Le rêve et la réalité \\
Le riche et le pauvre \\
Le ridicule \\
Le rire \\
Le risque \\
Le rite \\
Le rituel \\
Le roman \\
L'erreur \\
L'erreur est-elle humaine ? \\
L'erreur et la faute \\
L'erreur et l'illusion \\
L'érudition \\
Le rythme \\
Le sacré \\
Le sacrifice \\
Les âges de la vie \\
Les âges de l'humanité \\
Les animaux ont-ils des droits ? \\
Les animaux pensent-ils ? \\
Les animaux peuvent-ils avoir des droits ? \\
Les apparences sont-elles toujours trompeuses ? \\
Les arts admettent-ils une hiérarchie ? \\
Les arts populaires \\
Le sauvage \\
Les avant-gardes \\
Le savoir absolu \\
Le savoir a-t-il des degrés ? \\
Le savoir est-il libérateur ? \\
Les besoins et les désirs \\
Les bonnes manières \\
Les bonnes résolutions \\
Les bons sentiments \\
Le scandale \\
Les caractères \\
Les catégories \\
Les causes et les raisons \\
Le scepticisme \\
Les choses et les événements \\
Les cinq sens \\
L'esclavage \\
Les coïncidences ont-elles des causes ? \\
Les convictions d'autrui sont-elles un argument ? \\
Les désirs et les valeurs \\
Les droits de l'enfant \\
Les droits de l'homme \\
Les droits de l'homme sont-ils les droits de la femme ? \\
Les écrans \\
Le secret \\
Les élites \\
Le semblable \\
Le sensationnel \\
Le sens commun \\
Le sens de la justice \\
Le sens de la vie \\
Le sens de l'histoire \\
Le sens de l'humour \\
Le sens du destin \\
Le sens du devoir \\
Le sens du travail \\
Le sens moral \\
Le sens moral est-il naturel ? \\
Le sentiment d'injustice \\
Le sérieux \\
Le serment \\
Les excuses \\
Les faits et les valeurs \\
Les faits parlent-ils d'eux-mêmes ? \\
Les fins de la technique sont-elles techniques ? \\
Les fins et les moyens \\
Les foules \\
Les frontières \\
Les hommes et les femmes \\
Les hommes sont-ils des animaux ? \\
Les hors-la-loi \\
Les hypothèses scientifiques ont-elles pour nature d'être confirmées ou infirmées ? \\
Les idées ont-elles une histoire ? \\
Le signe et le symbole \\
Le silence \\
Le silence signifie-t-il toujours l'échec du langage ? \\
Les images ont-elles un sens ? \\
Le simple et le complexe \\
Le simulacre \\
Les leçons de l'expérience \\
Les leçons de l'histoire \\
Les limites de la discussion \\
Les limites de la raison \\
Les limites de la science \\
Les limites de l'imaginaire \\
Les limites du langage \\
Les livres \\
Les lois de la nature \\
Les lois de la nature sont-elles contingentes ? \\
Les lois de la pensée \\
Les lois et les mœurs \\
Les machines pensent-elles ? \\
Les mathématiques consistent-elles seulement en des opérations de l'esprit ? \\
Les mathématiques ont-elles besoin d'un fondement ? \\
Les mathématiques se réduisent-elles à une pensée cohérente ? \\
Les mathématiques sont-elles un langage ? \\
Les méchants peuvent-ils être amis ? \\
Les mondes possibles \\
Les monstres \\
Les mots et la signification \\
Les moyens et les fins \\
Les noms propres ont-ils une signification ? \\
Les objets de la pensée \\
Les objets sont-ils colorés ? \\
Les œuvres d'art sont-elles des choses ? \\
Les œuvres d'art sont-elles éternelles ? \\
Le soleil se lèvera-t-il demain ? \\
Le sommeil \\
L'ésotérisme \\
Le souci de soi \\
Le souvenir \\
Le souverain bien \\
L'espace \\
L'espace et le lieu \\
Les passions ont-elles une place en politique ? \\
Les passions peuvent-elles être raisonnables ? \\
Les passions sont-elles toujours mauvaises ? \\
Les passions s'opposent-elles à la raison ? \\
L'espèce humaine \\
Le spectacle \\
Le spectacle de la nature \\
Les philosophes doivent-ils être rois ? \\
Les philosophies se classent-elles ? \\
L'espoir \\
L'espoir et la crainte \\
Le sport \\
Les possibles \\
L'esprit critique \\
L'esprit des lois \\
L'esprit de système \\
L'esprit et la lettre \\
L'esprit et le cerveau \\
L'esprit peut-il être objet de science ? \\
Les qualités sensibles sont-elles dans les choses ou dans l'esprit ? \\
Les raisons de croire \\
Les raisons d'espérer \\
Les règles de l'art \\
Les ressources naturelles \\
Les sciences ont-elles besoin de principes fondamentaux ? \\
Les sciences peuvent-elles exclure toute notion de finalité ? \\
Les sciences sont-elles une description du monde ? \\
Les sens nous trompent-ils ? \\
L'essentiel \\
Les sentiments ont-ils une histoire ? \\
Les signes de l'intelligence \\
L'esthétique \\
L'estime de soi \\
Le style \\
Le sublime \\
Le sujet \\
Le sujet de l'histoire \\
Le surnaturel \\
Les vertus cardinales \\
Les vertus ne sont-elles que des vices déguisés ? \\
Les vices privés peuvent-ils faire le bien public ? \\
Les vivants et les morts \\
Le tableau \\
Le tableau ? \\
Le tacite \\
Le tas et le tout \\
L'État a-t-il le droit de contrôler notre habillement ? \\
L'état de droit \\
L'État de droit \\
L'État doit-il être neutre ? \\
L'État et la protection \\
Le témoignage \\
Le témoignage des sens \\
Le témoin \\
Le temps \\
Le temps de la réflexion \\
Le temps est-il notre ennemi ? \\
Le temps est-il une prison ? \\
Le temps existe-t-il ? \\
Le temps libre \\
Le temps s'écoule-t-il ? \\
L'éternité \\
Le terrorisme est-il un acte de guerre ? \\
Le théâtre et l'existence \\
L'étonnement \\
Le toucher \\
Le tout et les parties \\
Le tragique \\
Le tragique et le comique \\
Le trait d'esprit \\
L'étranger \\
Le travail \\
Le travail du négatif \\
Le travail et le labeur \\
Le travail et l'œuvre \\
Le travail fonde-t-il la propriété ? \\
L'être et la relation \\
L'être et l'essence \\
L'être humain désire-t-il naturellement connaître ? \\
Le tribunal de l'histoire \\
Le trompe-l'œil \\
L'euthanasie \\
L'évaluation \\
Le vécu \\
Le vécu et la vérité \\
L'événement \\
Le vertige \\
Le vêtement \\
Le vide \\
L'évidence \\
Le virtuel \\
Le visible et l'invisible \\
Le vivant \\
Le vivant et la technique \\
Le volontaire et l'involontaire \\
Le volontarisme \\
Le voyage \\
Le voyage dans le temps \\
Le vrai a-t-il une histoire ? \\
Le vrai et le bien sont-ils analogues ? \\
Le vrai et le faux \\
Le vrai et le vraisemblable \\
Le vraisemblable \\
L'exception \\
L'exemple \\
L'exercice \\
L'exil \\
L'existence d'autrui \\
L'existence de Dieu \\
L'existence des idées \\
L'existence est-elle une propriété ? \\
L'expérience \\
L'expérience du désir \\
L'expérience du temps \\
L'expérience instruit-elle ? \\
L'expérience métaphysique \\
L'expérience morale \\
L'expérimentation \\
L'explication \\
L'expression \\
L'extraordinaire \\
L'extrémisme \\
Le « je ne sais quoi » \\
L'habileté \\
L'habitude \\
L'habitude a-t-elle des vertus ? \\
L'harmonie \\
L'harmonie du monde \\
L'héroïsme \\
L'hésitation \\
L'histoire a-t-elle des lois ? \\
L'Histoire a-t-elle un commencement ? \\
L'histoire a-t-elle un sens ? \\
L'histoire des sciences \\
L'Histoire des sciences \\
L'histoire est-elle la science de ce qui ne se répète jamais ? \\
L'histoire est-elle rationnelle ? \\
L'histoire est-elle une science ? \\
L'histoire jugera \\
L'histoire se répète-t-elle ? \\
L'homme a-t-il besoin de religion ? \\
L'homme est-il fait pour le travail ? \\
L'homme est-il la mesure de toute chose ? \\
L'homme est-il la mesure de toutes choses ? \\
L'homme est-il raisonnable par nature ? \\
L'homme est-il un animal comme les autres ? \\
L'homme est-il un animal ? \\
L'homme est-il un loup pour l'homme ? \\
L'homme libre est-il un homme seul ? \\
L'homme-machine \\
L'homme n'est-il qu'un animal comme les autres ? \\
L'honneur \\
L'horrible \\
L'hospitalité \\
L'humanité \\
L'humour \\
L'humour et l'ironie \\
L'hypocrisie \\
L'hypothèse \\
Liberté et démocratie \\
Liberté et déterminisme \\
Liberté et égalité \\
Liberté et licence \\
Liberté et société \\
L'idéal \\
L'idée de bonheur \\
L'idée de civilisation \\
L'idée de création \\
L'idée de déterminisme \\
L'idée de langue universelle \\
L'idée de modernité \\
L'idée de monde \\
L'idée d'encyclopédie \\
L'idée de paix \\
L'idée de progrès \\
L'idée de révolution \\
L'idée de système \\
L'idée d'histoire universelle \\
L'idée d'ordre \\
L'idée d'origine \\
L'identité \\
L'identité collective \\
L'identité personnelle \\
L'idéologie \\
L'idiot \\
L'idolâtrie \\
L'idole \\
L'ignorance \\
L'illusion \\
L'illusion de la liberté \\
L'image \\
L'imaginaire \\
L'imagination \\
L'imagination a-t-elle des limites ? \\
L'imagination est-elle libre ? \\
L'imagination est-elle maîtresse d'erreur et de fausseté ? \\
L'imitation \\
L'immortalité \\
L'impardonnable \\
L'impartialité \\
L'impartialité est-elle toujours désirable ? \\
L'impassibilité \\
L'imperceptible \\
L'impersonnel \\
L'impossible \\
L'imprévisible \\
L'improvisation \\
L'imprudence \\
L'inaliénable \\
L'inattendu \\
L'incertitude \\
L'incertitude du passé \\
L'incompréhensible \\
L'inconcevable \\
L'inconnu \\
L'inconscience \\
L'inconscient \\
L'inconstance \\
L'indécence \\
L'indécision \\
L'indétermination \\
L'indéterminé \\
L'indice \\
L'indicible \\
L'indifférence \\
L'indignation \\
L'individu \\
L'individualisme \\
L'individualisme est-il un égoïsme ? \\
L'individualité \\
L'induction \\
L'indulgence \\
L'inférence \\
L'infidélité \\
L'infini \\
L'ingénieur \\
L'inhumain \\
L'inhumanité \\
L'injustice est-elle préférable au désordre ? \\
L'inné et l'acquis \\
L'innocence \\
L'innommable \\
L'inquiétude \\
L'insensibilité \\
L'inspiration \\
L'instant \\
L'instant de la décision est-il une folie ? \\
L'instinct \\
L'institution \\
L'instruction \\
L'insurrection est-elle un droit ? \\
L'intelligence \\
L'intelligence des bêtes \\
L'intention \\
L'interdit \\
L'intéressant \\
L'intérêt \\
L'intérêt bien compris \\
L'intérêt commun \\
L'intérêt gouverne-t-il le monde ? \\
L'intérieur et l'extérieur \\
L'intériorité \\
L'interprétation \\
L'intime conviction \\
L'intolérable \\
L'intolérance \\
L'intraduisible \\
L'introspection est-elle une connaissance ? \\
L'intuition \\
L'inutile a-t-il de la valeur ? \\
L'invention \\
L'invention technique \\
L'invérifiable \\
L'invisible \\
L'involontaire \\
L'ironie \\
L'irrationalité \\
L'irrationnel \\
L'irréductible \\
L'irréel \\
L'irrégularité \\
L'irrésolution \\
L'ivresse \\
L'objectivité \\
L'objectivité scientifique \\
L'objet de l'amour \\
L'objet de l'intention \\
L'objet du désir \\
L'obligation \\
L'obscurantisme \\
L'obscurité \\
L'observation \\
L'occasion \\
L'œuvre d'art doit-elle être belle ? \\
L'œuvre d'art doit-elle nous émouvoir ? \\
L'œuvre d'art est-elle une marchandise ? \\
Logique et grammaire \\
Lois et coutumes \\
L'oligarchie \\
L'opinion \\
L'opportunité \\
L'optimisme \\
L'oral et l'écrit \\
L'ordre \\
L'ordre et le désordre \\
L'ordre international \\
L'organisation \\
L'orgueil \\
L'originalité \\
L'origine des langues \\
L'oubli \\
L'outil \\
L'un et le multiple \\
L'uniformité \\
L'unité \\
L'univers \\
L'universel \\
L'urgence \\
L'usage des fictions \\
L'usure \\
L'utile et le bien \\
L'utile et l'honnête \\
L'utilité de croire \\
L'utilité est-elle une valeur morale ? \\
L'utilité peut-elle être le principe de la moralité ? \\
L'utilité peut-elle être un critère pour juger de la valeur de nos actions ? \\
L'utopie \\
Maîtriser la technique \\
Majorité et minorité \\
Malaise dans la civilisation \\
Malheur aux vaincus \\
Masculin, féminin \\
Mémoire et identité \\
Mémoire et souvenir \\
Mesure et démesure \\
Mesurer \\
Mieux vaut subir que commettre l'injustice \\
Mon corps \\
Mon corps m'appartient-il ? \\
Monologue et dialogue \\
Mon semblable \\
Montrer, est-ce démontrer ? \\
Mourir pour la patrie \\
Mythe et connaissance \\
Mythe et histoire \\
Mythe et vérité \\
Nature et culture \\
Nature et institution \\
N'avons-nous de devoir qu'envers autrui ? \\
Nécessité et contingence \\
Nécessité et liberté \\
N'existe-t-il que des individus ? \\
N'existe-t-il que le présent ? \\
Nom propre et nom commun \\
Nos sens nous trompent-ils ? \\
Notre existence a-t-elle un sens si l'histoire n'en a pas ? \\
Notre ignorance nous excuse-t-elle ? \\
Nous et les autres \\
N'y a-t-il d'amitié qu'entre égaux ? \\
N'y a-t-il de connaissance que de l'universel ? \\
N'y a-t-il de démocratie que représentative ? \\
N'y a-t-il de droits que de l'homme ? \\
N'y a-t-il de science que du général ? \\
N'y a-t-il de science que du mesurable ? \\
N'y a-t-il de science qu'exacte ? \\
Objet et œuvre \\
Observer \\
Optimisme et pessimisme \\
Ordre et désordre \\
Origine et commencement \\
Où est mon esprit ? \\
Où s'arrête l'espace public ? \\
Où suis-je ? \\
Parler et agir \\
Parler pour ne rien dire \\
Parler pour quelqu'un \\
Paroles et actes \\
Partager \\
Passer le temps \\
Passions, intérêt, raison \\
Peindre, est-ce nécessairement feindre ? \\
Penser à rien \\
Penser est-il assimilable à un travail ? \\
Penser le changement \\
Penser le rien, est-ce ne rien penser ? \\
Penser par soi-même \\
Pensez-vous que vous avez une âme ? \\
Perception et connaissance \\
Percevoir, est-ce savoir ? \\
Percevoir et concevoir \\
Percevons-nous les choses telles qu'elles sont ? \\
Perdre la mémoire \\
Perdre le contrôle \\
Perdre ses illusions \\
Perdre son temps \\
Persuader et convaincre \\
Peut-il être préférable de ne pas savoir ? \\
Peut-il exister une action désintéressée ? \\
Peut-il y avoir conflit entre nos devoirs ? \\
Peut-il y avoir des expériences métaphysiques ? \\
Peut-il y avoir un art conceptuel ? \\
Peut-il y avoir un droit de la guerre ? \\
Peut-il y avoir un intérêt collectif ? \\
Peut-on agir machinalement ? \\
Peut-on aimer la vie plus que tout ? \\
Peut-on argumenter en morale ? \\
Peut-on avoir peur de soi-même ? \\
Peut-on avoir raison contre tous ? \\
Peut-on avoir raison tout seul ? \\
Peut-on choisir le mal ? \\
Peut-on choisir sa vie ? \\
Peut-on classer les arts ? \\
Peut-on comprendre ce qui est illogique ? \\
Peut-on concevoir une société qui n'aurait plus besoin du droit ? \\
Peut-on connaître le singulier ? \\
Peut-on créer un homme nouveau ? \\
Peut-on critiquer la religion ? \\
Peut-on croire ce qu'on veut ? \\
Peut-on décider de croire ? \\
Peut-on décider de tout ? \\
Peut-on délimiter l'humain ? \\
Peut-on dialoguer avec un ordinateur ? \\
Peut-on dire ce qui n'est pas ? \\
Peut-on dire d'une œuvre d'art qu'elle est ratée ? \\
Peut-on dire que tout est relatif ? \\
Peut-on douter de soi ? \\
Peut-on douter de tout ? \\
Peut-on échapper au temps ? \\
Peut-on éduquer le goût ? \\
Peut-on éduquer quelqu'un à être libre ? \\
Peut-on être citoyen du monde ? \\
Peut-on être complètement athée ? \\
Peut-on être en conflit avec soi-même ? \\
Peut-on être heureux dans la solitude ? \\
Peut-on être homme sans être citoyen ? \\
Peut-on être impartial ? \\
Peut-on être juste dans une situation injuste ? \\
Peut-on être juste dans une société injuste ? \\
Peut-on être maître de soi ? \\
Peut-on être sceptique de bonne foi ? \\
Peut-on être sceptique ? \\
Peut-on être seul avec soi-même ? \\
Peut-on forcer quelqu'un à être libre ? \\
Peut-on fuir hors du monde ? \\
Peut-on haïr la raison ? \\
Peut-on imposer la liberté ? \\
Peut-on justifier la guerre ? \\
Peut-on justifier ses choix ? \\
Peut-on maîtriser la technique ? \\
Peut-on maîtriser le risque ? \\
Peut-on maîtriser le temps ? \\
Peut-on montrer en cachant ? \\
Peut-on ne croire en rien ? \\
Peut-on ne pas être de son temps ? \\
Peut-on ne pas être soi-même ? \\
Peut-on ne pas savoir ce que l'on fait ? \\
Peut-on ne pas savoir ce que l'on veut ? \\
Peut-on ne pas savoir ce qu'on veut ? \\
Peut-on ne pas vouloir être heureux ? \\
Peut-on ne penser à rien ? \\
Peut-on ne vivre qu'au présent ? \\
Peut-on parler de ce qui n'existe pas ? \\
Peut-on parler d'un droit de la guerre ? \\
Peut-on parler d'une santé de l'âme ? \\
Peut-on parler d'un règne de la technique ? \\
Peut-on parler pour en rien dire ? \\
Peut-on parler pour ne rien dire ? \\
Peut-on penser l'impossible ? \\
Peut-on penser sans concept ? \\
Peut-on penser sans ordre ? \\
Peut-on penser sans préjugé ? \\
Peut-on penser sans savoir que l'on pense ? \\
Peut-on penser sans signes ? \\
Peut-on penser un droit international ? \\
Peut-on perdre son identité ? \\
Peut-on prévoir l'avenir ? \\
Peut-on prévoir le futur ? \\
Peut-on prouver l'existence de Dieu ? \\
Peut-on prouver l'existence de l'inconscient ? \\
Peut-on prouver l'existence ? \\
Peut-on renoncer à comprendre ? \\
Peut-on renoncer à soi ? \\
Peut-on renoncer au bonheur ? \\
Peut-on reprocher au langage d'être équivoque ? \\
Peut-on rester sceptique ? \\
Peut-on rire de tout ? \\
Peut-on se duper soi-même ? \\
Peut-on se fier à l'expérience vécue ? \\
Peut-on se fier à son intuition ? \\
Peut-on se mentir à soi-même ? \\
Peut-on se mettre à la place d'autrui ? \\
Peut-on s'en tenir au présent ? \\
Peut-on séparer l'homme et l'œuvre ? \\
Peut-on se passer de croyances ? \\
Peut-on se passer de croyance ? \\
Peut-on se passer de frontières ? \\
Peut-on se passer de méthode ? \\
Peut-on se passer de principes ? \\
Peut-on se passer de techniques de raisonnement ? \\
Peut-on se promettre quelque chose à soi-même ? \\
Peut-on se retirer du monde ? \\
Peut-on sortir de la subjectivité ? \\
Peut-on sortir de sa conscience ? \\
Peut-on suivre une règle ? \\
Peut-on tout définir ? \\
Peut-on tout démontrer ? \\
Peut-on tout désirer ? \\
Peut-on tout dire ? \\
Peut-on tout exprimer ? \\
Peut-on tout interpréter ? \\
Peut-on tout mathématiser ? \\
Peut-on tout mesurer ? \\
Peut-on tout pardonner ? \\
Peut-on tout partager ? \\
Peut-on tout prévoir ? \\
Peut-on tout prouver ? \\
Peut-on tout soumettre à la discussion ? \\
Peut-on traiter autrui comme un moyen ? \\
Peut-on vivre sans aimer ? \\
Peut-on vivre sans croyances ? \\
Peut-on vivre sans l'art ? \\
Peut-on vivre sans principes ? \\
Peut-on voir sans croire ? \\
Peut-on vouloir le mal pour le mal ? \\
Peut-on vouloir le mal ? \\
Peut-on vouloir l'impossible ? \\
Philosophie et système \\
Plaisir et douleur \\
Plaisirs, honneurs, richesses \\
Pluralité et unité \\
Poésie et philosophie \\
Pour qui se prend-on ? \\
Pourquoi communiquer ? \\
Pourquoi critiquer le conformisme ? \\
Pourquoi délibérer ? \\
Pourquoi des cérémonies ? \\
Pourquoi des châtiments ? \\
Pourquoi des fictions ? \\
Pourquoi des interdits ? \\
Pourquoi désirer la sagesse ? \\
Pourquoi des métaphores ? \\
Pourquoi des musées ? \\
Pourquoi des poètes ? \\
Pourquoi des religions ? \\
Pourquoi des rites ? \\
Pourquoi des traditions ? \\
Pourquoi domestiquer ? \\
Pourquoi écrire ? \\
Pourquoi être raisonnable ? \\
Pourquoi faut-il être poli ? \\
Pourquoi la critique ? \\
Pourquoi la curiosité est-elle un vilain défaut ? \\
Pourquoi la guerre ? \\
Pourquoi la prison ? \\
Pourquoi la prohibition de l'inceste ? \\
Pourquoi les droits de l'homme sont-ils universels ? \\
Pourquoi le sport ? \\
Pourquoi l'homme a-t-il des droits ? \\
Pourquoi lit-on des romans ? \\
Pourquoi mentir ? \\
Pourquoi ne s'entend-on pas sur la nature de ce qui est réel ? \\
Pourquoi nous soucier du sort des générations futures ? \\
Pourquoi obéit-on ? \\
Pourquoi parler de fautes de goût ? \\
Pourquoi parle-t-on d'une « société civile » ? \\
Pourquoi pas plusieurs dieux ? \\
Pourquoi pas ? \\
Pourquoi philosopher ? \\
Pourquoi pleure-t-on au cinéma ? \\
Pourquoi pleure-t-on ? \\
Pourquoi préférer l'original ? \\
Pourquoi préserver l'environnement ? \\
Pourquoi prier ? \\
Pourquoi prouver l'existence de Dieu ? \\
Pourquoi punir ? \\
Pourquoi raconter des histoires ? \\
Pourquoi respecter les anciens ? \\
Pourquoi sauver les apparences ? \\
Pourquoi se confesser ? \\
Pourquoi se divertir ? \\
Pourquoi se fier à autrui ? \\
Pourquoi se révolter ? \\
Pourquoi se soucier du futur ? \\
Pourquoi s'étonner ? \\
Pourquoi s'intéresser à l'origine ? \\
Pourquoi soigner son apparence ? \\
Pourquoi suivre l'actualité ? \\
Pourquoi tenir ses promesses ? \\
Pourquoi travailler ? \\
Pourquoi travaille-t-on ? \\
Pourquoi veut-on changer le monde ? \\
Pourquoi veut-on la vérité ? \\
Pourquoi vivons-nous ? \\
Pourquoi vouloir avoir raison ? \\
Pourquoi voulons-nous savoir ? \\
Pourquoi voyager ? \\
Pourquoi y a-t-il des conflits insolubles ? \\
Pourquoi y a-t-il du mal dans le monde ? \\
Pourquoi y a-t-il plusieurs façons de démontrer ? \\
Pourquoi y a-t-il plusieurs langues ? \\
Pourquoi y a-t-il quelque chose plutôt que rien ? \\
Pourquoi y a-t-il des lois ? \\
Pourquoi ? \\
Pourrions-nous comprendre une pensée non humaine ? \\
Pouvoir et savoir \\
Pouvons-nous être objectifs ? \\
Pouvons-nous justifier nos croyances ? \\
Prendre la parole \\
Prendre ses désirs pour des réalités \\
Prendre son temps \\
Prévoir \\
Principe et fondement \\
Produire et créer \\
Promettre \\
Prouver \\
Prouver Dieu \\
Prouver la force d'âme \\
Prouver l'existence de Dieu \\
Puis-je décider de croire ? \\
Puis-je dire « ceci est mon corps » ? \\
Puis-je être heureux dans un monde chaotique ? \\
Puis-je faire ce que je veux de mon corps ? \\
Puis-je ne croire que ce que je vois ? \\
Punir ou soigner ? \\
Qu'aime-t-on ? \\
Qualité et quantité \\
Quand faut-il désobéir ? \\
Quand y a-t-il de l'art ? \\
Qu'anticipent les romans d'anticipation ? \\
Quantité et qualité \\
Qu'avons-nous en commun ? \\
Que choisir ? \\
Que faire de nos émotions ? \\
Que faire ? \\
Que faut-il craindre ? \\
Que la nature soit explicable, est-ce explicable ? \\
Quel contrôle a-t-on sur son corps ? \\
Quel est le bon nombre d'amis ? \\
Quel est le but de la politique ? \\
Quel est le fondement de l'autorité ? \\
Quel est le sujet de la pensée ? \\
Quel est l'objet de la métaphysique ? \\
Quel est l'objet de l'amour ? \\
Quel est l'objet de l'histoire ? \\
Quel est l'objet du désir ? \\
Quelle est la fin de l'État ? \\
Quelle est la portée d'un exemple ? \\
Quelle est la valeur de l'expérience ? \\
Quelle est la valeur des hypothèses ? \\
Quelle est la valeur d'une œuvre d'art ? \\
Quelle est la valeur du témoignage ? \\
Quelle réalité peut-on accorder au temps ? \\
Quelles sont les limites de la souveraineté ? \\
Quelle valeur accorder à l'expérience ? \\
Quelle valeur devons accorder à l'expérience ? \\
Quelle valeur devons-nous accorder à l'expérience ? \\
Quelle valeur devons-nous accorder à l'intuition ? \\
Quelle valeur peut-on accorder à l'expérience ? \\
Quels désirs dois-je m'interdire ? \\
Quels sont les fondements de l'autorité ? \\
Quel usage peut-on faire des fictions ? \\
Que manque-t-il aux machines pour être des organismes ? \\
Que nous apprend la diversité des langues ? \\
Que nous apprend le cinéma ? \\
Que nous apprend le faux ? \\
Que nous apprend l'expérience ? \\
Que nous apprennent les controverses scientifiques ? \\
Que nous apprennent les illusions d'optique ? \\
Que nous apprennent les mythes ? \\
Que nous enseignent les œuvres d'art ? \\
Que nul n'entre ici s'il n'est géomètre \\
Que partage-t-on avec les animaux ? \\
Que peint le peintre ? \\
Que perdrait la pensée en perdant l'écriture ? \\
Que peut la philosophie ? \\
Que peut la science ? \\
Que peut la théorie ? \\
Que peut l'esprit ? \\
Que peut-on comprendre immédiatement ? \\
Que peut-on échanger ? \\
Que peut-on interdire ? \\
Que peut-on sur autrui ? \\
Que peut-on voir ? \\
Que prouvent les preuves de l'existence de Dieu ? \\
Que recherche l'artiste ? \\
Que répondre au sceptique ? \\
Que sais-je d'autrui ? \\
Que serait la vie sans l'art ? \\
Que signifie connaître ? \\
Que signifie être mortel ? \\
Que signifie la mort ? \\
Que signifient les mots ? \\
Que signifie pour l'homme être mortel ? \\
Qu'est-ce qu'agir ensemble ? \\
Qu'est-ce qu'aimer une œuvre d'art ? \\
Qu'est-ce qu'apprendre ? \\
Qu'est-ce qu'argumenter ? \\
Qu'est-ce qu'avoir un droit ? \\
Qu'est-ce que catégoriser ? \\
Qu'est-ce que comprendre ? \\
Qu'est-ce que créer ? \\
Qu'est-ce que croire ? \\
Qu'est-ce que décider ? \\
Qu'est-ce que définir ? \\
Qu'est-ce que démontrer ? \\
Qu'est-ce que déraisonner ? \\
Qu'est-ce qu'éduquer ? \\
Qu'est-ce que faire autorité ? \\
Qu'est-ce que faire preuve d'humanité ? \\
Qu'est-ce que guérir ? \\
Qu'est-ce que jouer ? \\
Qu'est-ce que juger ? \\
Qu'est-ce que la barbarie ? \\
Qu'est-ce que la critique ? \\
Qu'est-ce que la démocratie ? \\
Qu'est-ce que la folie ? \\
Qu'est-ce que la normalité ? \\
Qu'est-ce que la raison d'État ? \\
Qu'est-ce que la souveraineté ? \\
Qu'est-ce que la tragédie ? \\
Qu'est-ce que la valeur marchande ? \\
Qu'est-ce que la vie bonne ? \\
Qu'est-ce que le bonheur ? \\
Qu'est-ce que le cinéma donne à voir ? \\
Qu'est-ce que le courage ? \\
Qu'est-ce que le hasard ? \\
Qu'est-ce que le moi ? \\
Qu'est-ce que l'enfance ? \\
Qu'est-ce que le sens pratique ? \\
Qu'est-ce que le sublime ? \\
Qu'est-ce que le travail ? \\
Qu'est-ce que l'indifférence ? \\
Qu'est-ce que l'intuition ? \\
Qu'est ce que lire ? \\
Qu'est-ce que lire ? \\
Qu'est-ce que l'ordinaire ? \\
Qu'est-ce que maîtriser une technique ? \\
Qu'est-ce que mourir ? \\
Qu'est-ce qu'enseigner ? \\
Qu'est-ce que percevoir ? \\
Qu'est-ce que perdre la raison ? \\
Qu'est-ce que perdre son temps ? \\
Qu'est-ce que promettre ? \\
Qu'est-ce que réfuter une philosophie ? \\
Qu'est-ce que s'orienter ? \\
Qu'est-ce que témoigner ? \\
Qu'est-ce que traduire ? \\
Qu'est-ce que travailler ? \\
Qu'est-ce qu'être adulte ? \\
Qu'est-ce qu'être barbare ? \\
Qu'est-ce qu'être cohérent ? \\
Qu'est-ce qu'être cultivé ? \\
Qu'est-ce qu'être de son temps ? \\
Qu'est-ce qu'être efficace en politique ? \\
Qu'est-ce qu'être fidèle à soi-même ? \\
Qu'est-ce qu'être généreux ? \\
Qu'est-ce qu'être idéaliste ? \\
Qu'est-ce qu'être libre ? \\
Qu'est-ce qu'être maître de soi-même ? \\
Qu'est-ce qu'être malade ? \\
Qu'est-ce qu'être moderne ? \\
Qu'est-ce qu'être nihiliste ? \\
Qu'est-ce qu'être rationnel ? \\
Qu'est-ce qu'être sceptique ? \\
Qu'est-ce qu'être soi-même ? \\
Qu'est-ce qu'être témoin ? \\
Qu'est-ce qu'être un bon citoyen ? \\
Qu'est-ce qu'être ? \\
Qu'est-ce qu'expliquer ? \\
Qu'est-ce que « se rendre maître et possesseur de la nature » ? \\
Qu'est-ce qui dépend de nous ? \\
Qu'est-ce qui est absurde ? \\
Qu'est-ce qui est actuel ? \\
Qu'est-ce qui est culturel ? \\
Qu'est-ce qui est donné ? \\
Qu'est-ce qui est extérieur à ma conscience, ? \\
Qu'est-ce qui est hors la loi ? \\
Qu'est-ce qui est immoral ? \\
Qu'est-ce qui est mauvais dans l'égoïsme ? \\
Qu'est-ce qui est réel ? \\
Qu'est-ce qui est sauvage ? \\
Qu'est-ce qui est scientifique ? \\
Qu'est-ce qui est tragique ? \\
Qu'est-ce qui fait la force de la loi ? \\
Qu'est-ce qui fait la valeur d'une croyance ? \\
Qu'est-ce qui fait mon identité ? \\
Qu'est-ce qui fait un peuple ? \\
Qu'est-ce qui innocente le bourreau ? \\
Qu'est-ce qui justifie une croyance ? \\
Qu'est-ce qu'imaginer ? \\
Qu'est-ce qui n'existe pas ? \\
Qu'est-ce qui nous fait danser ? \\
Qu'est-ce qu'interpréter ? \\
Qu'est-ce qui plaît dans la musique ? \\
Qu'est-ce qu'on attend ? \\
Qu'est-ce qu'un abus de langage ? \\
Qu'est-ce qu'un accident ? \\
Qu'est-ce qu'un acteur ? \\
Qu'est-ce qu'un acte ? \\
Qu'est-ce qu'un ami ? \\
Qu'est-ce qu'un animal ? \\
Qu'est-ce qu'un art de vivre ? \\
Qu'est-ce qu'un artiste ? \\
Qu'est-ce qu'un auteur ? \\
Qu'est-ce qu'un bon gouvernement ? \\
Qu'est-ce qu'un bon jugement ? \\
Qu'est-ce qu'un caractère ? \\
Qu'est-ce qu'un cas de conscience ? \\
Qu'est-ce qu'un châtiment ? \\
Qu'est-ce qu'un chef d'œuvre ? \\
Qu'est-ce qu'un chef-d'œuvre ? \\
Qu'est-ce qu'un chef ? \\
Qu'est-ce qu'un citoyen ? \\
Qu'est-ce qu'un classique ? \\
Qu'est-ce qu'un code ? \\
Qu'est-ce qu'un concept ? \\
Qu'est-ce qu'un conflit de générations ? \\
Qu'est-ce qu'un contrat ? \\
Qu'est-ce qu'un créateur ? \\
Qu'est-ce qu'un crime contre l'humanité ? \\
Qu'est-ce qu'un crime ? \\
Qu'est-ce qu'un critère de vérité ? \\
Qu'est-ce qu'un détail ? \\
Qu'est-ce qu'un dialogue ? \\
Qu'est-ce qu'un dilemme ? \\
Qu'est-ce qu'une action réussie ? \\
Qu'est-ce qu'une analyse ? \\
Qu'est-ce qu'une aporie ? \\
Qu'est-ce qu'une avant-garde ? \\
Qu'est-ce qu'une belle mort ? \\
Qu'est-ce qu'une bête ? \\
Qu'est-ce qu'une bonne définition ? \\
Qu'est-ce qu'une bonne délibération ? \\
Qu'est-ce qu'une bonne éducation ? \\
Qu'est-ce qu'une bonne traduction ? \\
Qu'est-ce qu'une catastrophe ? \\
Qu'est-ce qu'une cause ? \\
Qu'est-ce qu'une civilisation ? \\
Qu'est-ce qu'une comédie ? \\
Qu'est-ce qu'une condition suffisante ? \\
Qu'est-ce qu'une contrainte ? \\
Qu'est-ce qu'une convention ? \\
Qu'est-ce qu'une conviction ? \\
Qu'est-ce qu'une crise ? \\
Qu'est-ce qu'une croyance rationnelle ? \\
Qu'est-ce qu'une croyance ? \\
Qu'est-ce qu'une décision politique ? \\
Qu'est-ce qu'une décision rationnelle ? \\
Qu'est-ce qu'une définition ? \\
Qu'est-ce qu'une démocratie ? \\
Qu'est-ce qu'une démonstration ? \\
Qu'est-ce qu'une exception ? \\
Qu'est-ce qu'une expérience cruciale ? \\
Qu'est-ce qu'une expérience de pensée ? \\
Qu'est-ce qu'une expérience scientifique ? \\
Qu'est-ce qu'une expérience ? \\
Qu'est-ce qu'une famille ? \\
Qu'est-ce qu'une fausse science ? \\
Qu'est-ce qu'une faute de goût ? \\
Qu'est-ce qu'une fonction ? \\
Qu'est-ce qu'une hypothèse scientifique ? \\
Qu'est-ce qu'une hypothèse ? \\
Qu'est-ce qu'une idée ? \\
Qu'est-ce qu'une idéologie ? \\
Qu'est-ce qu'une illusion ? \\
Qu'est-ce qu'une image ? \\
Qu'est-ce qu'une inégalité ? \\
Qu'est-ce qu'une injustice ? \\
Qu'est-ce qu'une institution ? \\
Qu'est-ce qu'une interprétation ? \\
Qu'est-ce qu'une invention technique ? \\
Qu'est-ce qu'une langue ? \\
Qu'est-ce qu'une loi de la nature ? \\
Qu'est-ce qu'une loi scientifique ? \\
Qu'est-ce qu'une loi ? \\
Qu'est-ce qu'une machine ? \\
Qu'est-ce qu'une maladie ? \\
Qu'est-ce qu'une métaphore ? \\
Qu'est-ce qu'une morale de la communication ? \\
Qu'est-ce qu'une nation ? \\
Qu'est-ce qu'un enfant ? \\
Qu'est-ce qu'un ennemi ? \\
Qu'est-ce qu'une norme ? \\
Qu'est-ce qu'une nouveauté ? \\
Qu'est-ce qu'une œuvre d'art ? \\
Qu'est-ce qu'une œuvre ? \\
Qu'est-ce qu'une parole libre ? \\
Qu'est-ce qu'une passion ? \\
Qu'est-ce qu'une pensée libre ? \\
Qu'est-ce qu'une personne ? \\
Qu'est-ce qu'une preuve ? \\
Qu'est-ce qu'une promesse ? \\
Qu'est-ce qu'une propriété ? \\
Qu'est-ce qu'une question ? \\
Qu'est-ce qu'une raison d'agir ? \\
Qu'est-ce qu'une règle ? \\
Qu'est-ce qu'une relation ? \\
Qu'est-ce qu'une révolution scientifique ? \\
Qu'est-ce qu'une révolution ? \\
Qu'est-ce qu'une science humaine ? \\
Qu'est-ce qu'un esclave ? \\
Qu'est-ce qu'une société juste ? \\
Qu'est-ce qu'un esprit libre ? \\
Qu'est-ce qu'un esprit profond ? \\
Qu'est-ce qu'une structure ? \\
Qu'est-ce qu'une théorie scientifique ? \\
Qu'est-ce qu'une tradition ? \\
Qu'est-ce qu'une tragédie ? \\
Qu'est-ce qu'un être vivant ? \\
Qu'est-ce qu'un événement ? \\
Qu'est-ce qu'une vertu ? \\
Qu'est-ce qu'une ville ? \\
Qu'est-ce qu'une vision du monde ? \\
Qu'est-ce qu'un exemple ? \\
Qu'est-ce qu'une « expérience de pensée » ? \\
Qu'est-ce qu'un fait divers ? \\
Qu'est-ce qu'un fait historique ? \\
Qu'est-ce qu'un fait social ? \\
Qu'est-ce qu'un fait ? \\
Qu'est-ce qu'un faux problème ? \\
Qu'est-ce qu'un génie ? \\
Qu'est-ce qu'un grand homme ou une grande femme ? \\
Qu'est-ce qu'un grand homme ? \\
Qu'est-ce qu'un grand philosophe ? \\
Qu'est-ce qu'un héros ? \\
Qu'est-ce qu'un homme libre ? \\
Qu'est-ce qu'un homme normal ? \\
Qu'est-ce qu'un idéaliste ? \\
Qu'est-ce qu'un idéal ? \\
Qu'est-ce qu'un intellectuel ? \\
Qu'est-ce qu'un jugement analytique ? \\
Qu'est-ce qu'un jugement de goût ? \\
Qu'est-ce qu'un livre ? \\
Qu'est-ce qu'un maître ? \\
Qu'est-ce qu'un miracle ? \\
Qu'est-ce qu'un modèle ? \\
Qu'est-ce qu'un monde \\
Qu'est-ce qu'un monde ? \\
Qu'est-ce qu'un monstre ? \\
Qu'est-ce qu'un mythe ? \\
Qu'est-ce qu'un objet ? \\
Qu'est-ce qu'un œuvre d'art ? \\
Qu'est-ce qu'un outil ? \\
Qu'est-ce qu'un paradoxe ? \\
Qu'est-ce qu'un paysage ? \\
Qu'est-ce qu'un peuple libre ? \\
Qu'est-ce qu'un peuple ? \\
Qu'est-ce qu'un philosophe ? \\
Qu'est-ce qu'un plaisir pur ? \\
Qu'est-ce qu'un portrait ? \\
Qu'est-ce qu'un post-moderne ? \\
Qu'est-ce qu'un précurseur ? \\
Qu'est-ce qu'un préjugé ? \\
Qu'est-ce qu'un principe ? \\
Qu'est-ce qu'un problème ? \\
Qu'est-ce qu'un programme ? \\
Qu'est-ce qu'un progrès scientifique ? \\
Qu'est-ce qu'un prophète ? \\
Qu'est-ce qu'un récit ? \\
Qu'est-ce qu'un réfutation ? \\
Qu'est-ce qu'un régime politique ? \\
Qu'est-ce qu'un savoir-faire ? \\
Qu'est-ce qu'un sentiment moral ? \\
Qu'est-ce qu'un signe ? \\
Qu'est-ce qu'un sophiste ? \\
Qu'est-ce qu'un style ? \\
Qu'est-ce qu'un symbole ? \\
Qu'est-ce qu'un système ? \\
Qu'est-ce qu'un tableau ? \\
Qu'est-ce qu'un traître ? \\
Qu'est-ce qu'un travail bien fait ? \\
Qu'est-ce qu'un tyran ? \\
Qu'est-ce qu'un vice ? \\
Qu'est-ce qu'un « être dégénéré » ? \\
Qu'est qu'une image ? \\
Que suis-je ? \\
Que valent les préjugés ? \\
Que vaut la décision de la majorité ? \\
Que vaut l'excuse : « C'est plus fort que moi » ? \\
Que vaut une preuve contre un préjugé ? \\
Que veut dire « essentiel » ? \\
Que veut dire « je t'aime » ? \\
Que veut dire « réel » ? \\
Que veut dire « respecter la nature » ? \\
Que veut dire : « je t'aime » ? \\
Que veut dire : « respecter la nature » ? \\
Que voit-on dans un miroir ? \\
Que voit-on dans un tableau ? \\
Que voyons-nous ? \\
Qui croire ? \\
Qui est citoyen ? \\
Qui est immoral ? \\
Qui fait la loi ? \\
Qui fait l'histoire ? \\
Qui meurt ? \\
Qui suis-je ? \\
Qui veut la fin veut les moyens \\
Qu'y a-t-il de sérieux dans le jeu ? \\
Qu'y a-t-il ? \\
Raconter sa vie \\
Raisonner par l'absurde \\
Rationnel et raisonnable \\
Refaire sa vie \\
Regarder un tableau \\
Religion et politique \\
Rester soi-même \\
Rêver \\
Rêvons-nous ? \\
Rien \\
Rien n'est sans raison \\
Roman et vérité \\
Sait-on ce qu'on veut ? \\
Sait-on toujours ce que l'on veut ? \\
Sauver les apparences \\
Sauver les phénomènes \\
Savoir est-ce se libérer ? \\
Savoir et savoir-faire \\
Savoir faire \\
Savoir renoncer \\
Savoir se décider \\
Savoir tout \\
Science et expérience \\
Science et hypothèse \\
Science et idéologie \\
Science et méthode \\
Science et objectivité \\
Science et technique \\
Se connaître soi-même \\
Se convertir \\
Se cultiver \\
Sécurité et liberté \\
Se décider \\
Se faire justice \\
S'engager \\
Sensation et perception \\
Se raconter des histoires \\
Serait-il immoral d'autoriser le commerce des organes humains ? \\
Seul \\
Seul le présent existe-t-il ? \\
Sexe et genre \\
Sexualité et féminité \\
Si\ldots{} alors \\
Si Dieu n'existe pas, tout est-il possible ? \\
Signe et symbole \\
Société et communauté \\
Société et religion \\
Soi \\
Sommes-nous dominés par la technique ? \\
Sommes-nous gouvernés par nos passions ? \\
Sommes-nous les jouets de nos pulsions ? \\
Sommes-nous perfectibles ? \\
Sommes-nous responsables de nos erreurs ? \\
Sommes-nous responsables de nos opinions ? \\
Sommes-nous responsables de nos passions ? \\
Sommes-nous soumis au temps ? \\
S'orienter \\
Soyez naturel ! \\
Sport et politique \\
Suffit-il d'être informé pour comprendre ? \\
Suis-je le même en des temps différents ? \\
Suis-je maître de mes pensées ? \\
Suis-je mon corps ? \\
Suis-je propriétaire de mon corps ? \\
Suis-je seul au monde ? \\
Suivre une règle \\
Superstition et religion \\
Sur quoi l'historien travaille-t-il ? \\
Sur quoi repose l'accord des esprits ? \\
Technique et intérêt \\
Technique et responsabilité \\
Tenir sa parole \\
Toujours plus vite ? \\
Tous les conflits peuvent-ils être résolus par le dialogue ? \\
Tous les hommes sont-ils égaux ? \\
Tous les plaisirs se valent-ils ? \\
Tout a-t-il une cause ? \\
Tout ce qui est excessif est-il insignifiant ? \\
Tout ce qui existe a-t-il un prix ? \\
Tout désir est-il désir de posséder ? \\
Toute connaissance commence-t-elle avec l'expérience ? \\
Toute connaissance consiste-t-elle en un savoir-faire ? \\
Toute conscience est-elle conscience de soi ? \\
Toute notre connaissance dérive-t-elle de l'expérience ? \\
Toute pensée revêt-elle nécessairement une forme linguistique ? \\
Toute peur est-elle irrationnelle ? \\
Toute philosophie constitue-t-elle une doctrine ? \\
Toutes les choses sont-elles singulières ? \\
Toutes les interprétations se valent-elles ? \\
Toutes les opinions se valent-elles ? \\
Toutes les opinions sont-elles bonnes à dire ? \\
Tout est-il à vendre ? \\
Tout est-il faux dans la fiction ? \\
Tout est-il mesurable ? \\
Tout est-il politique ? \\
Tout est-il quantifiable ? \\
Tout est-il relatif ? \\
Tout est relatif \\
Toute vérité est-elle bonne à dire ? \\
Toute vérité est-elle démontrable ? \\
Tout ou rien \\
Tout peut-il s'expliquer ? \\
Tout pouvoir corrompt-il ? \\
Tout pouvoir est-il oppresseur ? \\
Tout principe est-il un fondement ? \\
Tout savoir a-t-il une justification ? \\
Tout savoir peut-il se transmettre ? \\
Tradition et vérité \\
Traduction, Trahison \\
Traduire \\
Tragédie et comédie \\
Tu aimeras ton prochain comme toi-même \\
Tuer et laisser mourir \\
Tuer le temps \\
Un contrat peut-il être injuste ? \\
Un contrat peut-il être social ? \\
Une action peut-elle être désintéressée ? \\
Une action peut-elle être machinale ? \\
Une idée peut-elle être fausse ? \\
Une intention peut-elle être coupable ? \\
Une machine peut-elle avoir une mémoire ? \\
Une machine peut-elle penser ? \\
Une œuvre doit-elle nécessairement être belle ? \\
Une œuvre est-elle nécessairement singulière ? \\
Une société d'athées est-elle possible ? \\
Une théorie scientifique peut-elle être vraie ? \\
Un jeu peut-il être sérieux ? \\
Un langage universel est-il concevable ? \\
Un objet technique peut-il être beau ? \\
Un seul peut-il avoir raison contre tous ? \\
Vaut-il mieux subir l'injustice que la commettre ? \\
Vendre son corps \\
Vérité et certitude \\
Vérité et cohérence \\
Vérité et fiction \\
Vérité et poésie \\
Vérité et subjectivité \\
Vertu et perfection \\
Vivre \\
Vivre au présent \\
Vivre caché \\
Vivre, est-ce un droit ? \\
Vivre sa vie \\
Vivre selon la nature \\
Voir \\
Voir et savoir \\
Voir le meilleur et faire le pire \\
Voir un tableau \\
Vouloir, est-ce encore désirer ? \\
Vouloir être immortel \\
Vouloir le mal \\
Vouloir l'impossible \\
Vouloir oublier \\
Y a-t-il de bons préjugés ? \\
Y a-t-il de faux problèmes ? \\
Y a-t-il de l'impensable ? \\
Y a-t-il de l'incommunicable ? \\
Y a-t-il de l'inconnaissable ? \\
Y a-t-il de l'indicible ? \\
Y a-t-il de l'irréductible ? \\
Y a-t-il de l'irréfutable ? \\
Y a-t-il des acquis définitifs en science ? \\
Y a-t-il des actes de pensée ? \\
Y a-t-il des arts mineurs ? \\
Y a-t-il des connaissances désintéressées ? \\
Y a-t-il des critères de l'humanité ? \\
Y a-t-il des critères du goût ? \\
Y a-t-il des croyances rationnelles ? \\
Y a-t-il des déterminismes sociaux ? \\
Y a-t-il des devoirs envers soi-même ? \\
Y a-t-il des expériences de la liberté ? \\
Y a-t-il des expériences métaphysiques ? \\
Y a-t-il des faux problèmes ? \\
Y a-t-il des fins dans la nature ? \\
Y a-t-il des guerres justes ? \\
Y a-t-il des illusions nécessaires ? \\
Y a-t-il des instincts propres à l'Homme ? \\
Y a-t-il des interprétations fausses ? \\
Y a-t-il des leçons de l'histoire ? \\
Y a-t-il des limites à la connaissance ? \\
Y a-t-il des limites à la conscience ? \\
Y a-t-il des limites à l'exprimable ? \\
Y a-t-il des lois de la pensée ? \\
Y a-t-il des lois de l'histoire ? \\
Y a-t-il des lois de l'Histoire ? \\
Y a-t-il des normes naturelles ? \\
Y a-t-il des passions intraitables ? \\
Y a-t-il des passions raisonnables ? \\
Y a-t-il des plaisirs purs ? \\
Y a-t-il des régressions historiques ? \\
Y a-t-il des révolutions scientifiques ? \\
Y a-t-il des sciences de l'homme ? \\
Y a-t-il des sciences exactes ? \\
Y a-t-il des sentiments moraux ? \\
Y a-t-il des signes naturels ? \\
Y a-t-il des sociétés sans histoire ? \\
Y a-t-il des techniques de pensée ? \\
Y a-t-il des techniques du corps ? \\
Y a-t-il des valeurs naturelles ? \\
Y a-t-il des valeurs universelles ? \\
Y a-t-il des vérités éternelles ? \\
Y a-t-il des violences légitimes ? \\
Y a-t-il plusieurs libertés ? \\
Y a-t-il plusieurs morales ? \\
Y a-t-il plusieurs sortes de vérité ? \\
Y a-t-il un art de gouverner ? \\
Y a-t-il un art de penser ? \\
Y a-t-il un art de vivre ? \\
Y a-t-il un au-delà du langage ? \\
Y a-t-il un beau naturel ? \\
Y a-t-il un bien commun ? \\
Y a-t-il un bon usage des passions ? \\
Y a-t-il un critère de vérité ? \\
Y a-t-il un critère du vrai ? \\
Y a-t-il un devoir d'être heureux ? \\
Y a-t-il un droit de la guerre ? \\
Y a-t-il un droit de mentir ? \\
Y a-t-il un droit de résistance ? \\
Y a-t-il un droit d'ingérence ? \\
Y a-t-il un droit du plus faible ? \\
Y a-t-il un droit du plus fort ? \\
Y a-t-il un droit international ? \\
Y a-t-il un droit naturel ? \\
Y a-t-il un droit universel au mariage ? \\
Y a-t-il une cause première ? \\
Y a-t-il une connaissance du probable ? \\
Y a-t-il une connaissance du singulier ? \\
Y a-t-il une connaissance historique ? \\
Y a-t-il une connaissance sensible ? \\
Y a-t-il une éducation du goût ? \\
Y a-t-il une esthétique de la laideur ? \\
Y a-t-il une éthique des moyens ? \\
Y a-t-il une expérience de la liberté ? \\
Y a-t-il une fin de l'histoire ? \\
Y a-t-il une histoire de la nature ? \\
Y a-t-il une histoire de la raison ? \\
Y a-t-il une histoire de la vérité ? \\
Y a-t-il une intelligence du corps ? \\
Y a-t-il une langue de la philosophie ? \\
Y a-t-il une limite au développement scientifique ? \\
Y a-t-il une nature humaine ? \\
Y a-t-il une pensée sans signes ? \\
Y a-t-il une positivité de l'erreur ? \\
Y a-t-il une présence du passé ? \\
Y a-t-il une rationalité des sentiments ? \\
Y a-t-il une sagesse populaire ? \\
Y a-t-il une servitude volontaire ? \\
Y a-t-il une spécificité des sciences humaines ? \\
Y a-t-il une universalité du beau ? \\
Y a-t-il une vérité du sensible ? \\
Y a-t-il un langage animal ? \\
Y a-t-il un langage de la musique ? \\
Y a-t-il un langage de l'art ? \\
Y a-t-il un monde de l'art ? \\
Y a-t-il un progrès moral ? \\
Y a-t-il un savoir du contingent ? \\
Y a-t-il un savoir immédiat ? \\
Y a-t-il un sens moral ? \\
Y a-t-il un souverain bien ? \\
Y a-t-il un travail de la pensée ? \\
« Aux armes citoyens ! » \\
« À quelque chose malheur est bon » \\
« Ceci » \\
« Ce ne sont que des mots » \\
« C'est la vie » \\
« La vie est une scène » \\
« Les faits, rien que les faits » \\
« L'histoire jugera » \\
« L'homme est la mesure de toutes choses » \\
« Nul n'est censé ignorer la loi » \\
« Œil pour œil, dent pour dent » \\
« Quelle vanité que la peinture » \\
« Que nul n'entre ici s'il n'est géomètre » \\
« Que va-t-il se passer ? » \\
« Tu ne tueras point » \\


\subsection{ENS B​/​L}
\label{sec-2-8}

\noindent
Aimer, est-ce vraiment connaître ? \\
Analyse et synthèse \\
Apprendre à parler \\
Apprendre à philosopher \\
Apprendre à voir \\
À quelle expérience l'art nous convie-t-il ? \\
À quoi bon penser la fin du monde ? \\
À quoi bon voyager ? \\
À quoi reconnaît-on une œuvre d'art ? \\
À quoi servent les mythes ? \\
À quoi servent les utopies ? \\
Au nom de quoi le plaisir serait-il condamnable ? \\
Autrui est-il inconnaissable ? \\
Avoir \\
Avoir de la suite dans les idées \\
Avoir des principes \\
Avoir une idée \\
Avons-nous besoin d'amis ? \\
Avons-nous besoin d'utopies ? \\
Avons-nous des devoirs envers les animaux ? \\
Avons-nous des raisons d'espérer ? \\
Avons-nous un droit au droit ? \\
Avons-nous une responsabilité envers le passé ? \\
Avons-nous un libre arbitre ? \\
À quelles conditions une théorie peut-elle être scientifique ? \\
À quelles conditions un jugement est-il moral ? \\
À quels signes reconnaît-on la vérité ? \\
À qui est mon corps ? \\
À qui profite le crime ? \\
À quoi bon voyager ? \\
À quoi est-il impossible de s'habituer ? \\
À quoi sert la notion d'état de nature ? \\
Calculer et raisonner \\
Cause et loi \\
Certitude et vérité \\
Changer \\
Changer la vie \\
Choisir \\
Comment deux personnes peuvent-elles partager la même pensée ? \\
Comment distinguer entre l'amour et l'amitié ? \\
Comment distinguer l'amour de l'amitié ? \\
Comment distinguer le vrai du faux ? \\
Comment établir des critères d'équité ? \\
Comment exprimer l'identité ? \\
Comment la science progresse-t-elle ? \\
Comment ne pas être humaniste ? \\
Comment ne pas être libéral ? \\
Comment peut-on se trahir soi-même ? \\
Comment sait-on qu'une chose existe ? \\
Communauté, collectivité, société \\
Communauté et société \\
Comparaison n'est pas raison \\
Conclure \\
Conscience et mémoire \\
Contrôle et vigilance \\
Conventions sociales et moralité \\
Conviction et certitude \\
Création et production \\
Crime et châtiment \\
Crise et progrès \\
Croire et savoir \\
Croit-on ce que l'on veut ? \\
Croyance et choix \\
Culture et civilisation \\
Culture et communauté \\
Décider \\
Décrire, est-ce déjà expliquer ? \\
De quoi a-t-on conscience lorsqu'on a conscience de soi ? \\
De quoi avons-nous besoin ? \\
De quoi la philosophie est-elle le désir ? \\
De quoi parlent les mathématiques ? \\
Désirer et vouloir \\
Désir et besoin \\
Désir et volonté \\
Déterminisme et responsabilité \\
Devant qui est-on responsable ? \\
Devons-nous nous libérer de nos désirs ? \\
Devons-nous quelque chose à la nature ? \\
Devons-nous vivre comme si nous ne devions jamais mourir ? \\
Dialogue et délibération en démocratie \\
Dieu aurait-il pu mieux faire ? \\
Dire, est-ce faire ? \\
Discrimination et revendication \\
Discussion et conversation \\
Dois-je mériter mon bonheur ? \\
Doit-on rechercher l'harmonie ? \\
Doit-on souffrir de n'être pas compris ? \\
Dominer la nature \\
Donner \\
Donner sa parole \\
D'où viennent les concepts ? \\
D'où viennent les idées générales ? \\
D'où viennent les préjugés ? \\
Droits, garanties, protection \\
Efficacité et justice \\
En politique, ne faut-il croire qu'aux rapports de force ? \\
En quel sens les sciences ont-elles une histoire ? \\
En quel sens peut-on dire que le mal n'existe pas ? \\
En quoi la justice met-elle fin à la violence ? \\
En quoi la nature constitue-t-elle un modèle ? \\
En quoi la patience est-elle une vertu ? \\
En quoi la physique a-t-elle besoin des mathématiques ? \\
En quoi la sociologie est-elle fondamentale ? \\
En quoi le langage est-il constitutif de l'homme ? \\
En quoi les hommes restent-ils des enfants ? \\
En quoi une œuvre d'art est-elle moderne ? \\
Enseigner \\
Enseigner, est-ce transmettre un savoir ? \\
Entendre raison \\
Espace et représentation \\
Est-ce la mémoire qui constitue mon identité ? \\
Est-ce par désir de la vérité que l'homme cherche à savoir ? \\
Est-il naturel de s'aimer soi-même ? \\
Est-il possible d'améliorer l'homme ? \\
Est-il possible de ne croire à rien ? \\
Est-il possible de préparer l'avenir ? \\
Est-on responsable de son passé ? \\
État et nation \\
Étonnement et sidération \\
Être chez soi \\
Être compris \\
Être conséquent avec soi-même \\
Être dans son droit \\
Être et avoir \\
Être et devoir-être \\
Être et penser, est-ce la même chose ? \\
Être hors de soi \\
Être ou avoir \\
Être ou ne pas être \\
Être soi-même \\
Évidence et certitude \\
Existe-t-il des croyances collectives ? \\
Existe-t-il un vocabulaire neutre des droits fondamentaux ? \\
Expliquer et comprendre \\
Faire ce que l'on dit \\
Faire comme si \\
Faire de nécessité vertu \\
Faire la part des choses \\
Faisons-nous l'histoire ? \\
Fait-on de la politique pour changer les choses ? \\
Faits et valeurs \\
Faut-il avoir peur de la technique ? \\
Faut-il changer le monde ? \\
Faut-il concilier les contraires ? \\
Faut-il condamner le luxe ? \\
Faut-il craindre la mort ? \\
Faut-il craindre le pire ? \\
Faut-il craindre le regard d'autrui ? \\
Faut-il défendre la démocratie ? \\
Faut-il désirer la vérité ? \\
Faut-il détruire pour créer ? \\
Faut-il donner un sens à la souffrance ? \\
Faut-il douter de l'évidence \\
Faut-il être bon ? \\
Faut-il être modéré ? \\
Faut-il être réaliste ? \\
Faut-il imposer la vérité ? \\
Faut-il opposer l'État et la société ? \\
Faut-il opposer l'histoire et la fiction ? \\
Faut-il opposer nature et culture ? \\
Faut-il poser des limites à l'activité rationnelle ? \\
Faut-il protéger la nature ? \\
Faut-il protéger les faibles contre les forts ? \\
Faut-il reconnaître pour connaître ? \\
Faut-il résister à la peur de mourir ? \\
Faut-il respecter la nature ? \\
Faut-il rire ou pleurer ? \\
Faut-il s'adapter ? \\
Faut-il savoir obéir pour gouverner ? \\
Faut-il savoir prendre des risques ? \\
Faut-il se méfier de l'écriture ? \\
Faut-il se méfier des apparences ? \\
Faut-il s'en remettre à l'État pour limiter le pouvoir de l'État ? \\
Faut-il séparer la science et la technique ? \\
Faut-il séparer morale et politique ? \\
Faut-il toujours avoir raison ? \\
Faut-il toujours dire la vérité ? \\
Faut-il un commencement à tout ? \\
Faut-il vivre comme si l'on ne devait jamais mourir ? \\
Faut-il vouloir changer le monde ? \\
Fiction et virtualité \\
Foi et raison \\
Folie et raison \\
Force et violence \\
Fuir la civilisation \\
Gagner \\
Gouvernement et société \\
Guerres justes et injustes \\
Habiter \\
Habiter sur la terre \\
Histoire et devenir \\
Histoire et politique \\
Histoire et structure \\
Illusion et apparence \\
Incertitude et action \\
Individualisme et égoïsme \\
Intentions, plans et stratégies \\
Invention et imitation \\
Jouer \\
Jouer son rôle \\
Jugement de goût et jugement esthétique \\
Jugement moral et jugement empirique \\
Juger et décider \\
Justice et vengeance \\
Justification et politique \\
La banalité \\
La barbarie \\
La béatitude \\
La beauté de la nature \\
La beauté du diable \\
La beauté est-elle une promesse de bonheur ? \\
La beauté morale \\
La beauté peut-elle délivrer une vérité ? \\
La bestialité \\
La bêtise et la méchanceté sont-elles liées nécessairement ? \\
L'absence \\
L'absolu \\
L'abstraction \\
L'abstraction est-elle toujours utile à la science empirique ? \\
L'abstrait et le concret \\
L'abus de pouvoir \\
L'académisme \\
L'académisme et les fins de l'art \\
La calomnie \\
La causalité \\
L'accident \\
La censure \\
La certitude \\
La charité \\
La charité est-elle une vertu ? \\
L'achèvement de l'œuvre \\
La cité sans dieux \\
La civilisation \\
La classe moyenne \\
La classification des sciences \\
La cohérence \\
La cohérence est-elle un critère de vérité ? \\
La collection \\
La comédie \\
La comédie humaine \\
La comparaison \\
La compassion \\
La compétence \\
La condition de mortel \\
La confiance \\
La confiance en la raison \\
La connaissance du bien \\
La connaissance du singulier \\
La connaissance et la croyance \\
La connaissance et le vivant \\
La connaissance historique \\
La connaissance intuitive \\
La connaissance s'interdit-elle tout recours à l'imagination ? \\
La conquête de l'espace \\
La conscience \\
La conscience peut-elle être collective ? \\
La conséquence \\
La conservation \\
La consolation \\
La constitution \\
La contemplation \\
La contingence \\
La contingence est-elle la condition de la liberté ? \\
La contradiction \\
La contrainte déontologique \\
La convalescence \\
La conviction \\
La copie \\
La corruption \\
La couleur \\
La coutume \\
La création \\
La crise \\
La critique \\
La critique des théories \\
La croissance du savoir \\
La croyance peut-elle être rationnelle ? \\
La cruauté \\
L'acteur \\
L'action du temps \\
L'action intentionnelle \\
L'activité \\
L'activité se laisse-t-elle programmer ? \\
L'actualité \\
La culture de masse \\
La culture savante et la culture populaire \\
La culture : pour quoi faire ? \\
La curiosité \\
La danse \\
La décadence \\
La décision \\
La découverte de la vérité peut-elle être le fait du hasard ? \\
La démesure \\
La démocratie \\
La démocratie est-ce la fin du despotisme ? \\
La démocratie et les institutions de la justice \\
La démocratie et le statut de la loi \\
La démonstration \\
La dépense \\
La description \\
La dialectique \\
La différence \\
La différence culturelle \\
La différence sexuelle \\
La dignité \\
La discipline \\
La discrimination \\
La disponibilité \\
La disposition \\
La dispute \\
La dissidence \\
La distinction \\
La diversité humaine \\
La division de la volonté \\
La division du travail \\
La domestication \\
La domination \\
La douleur \\
L'adversité \\
La faiblesse \\
La famille est-elle naturelle ? \\
La famille est-elle une communauté naturelle ? \\
La famille et la cité \\
La famille et le droit \\
La fatigue \\
La faute et l'erreur \\
La femme est-elle l'avenir de l'homme ? \\
La fête \\
La fiction \\
La fidélité \\
La figure humaine \\
La fin de la métaphysique \\
La fin du monde \\
La fin du mythe \\
La folie \\
La force \\
La force d'âme \\
La force de l'habitude \\
La force des choses \\
La force des faibles \\
La force des idées \\
La force des lois \\
La force du social \\
La force et le droit \\
La force publique \\
La formalisation \\
La fortune \\
La fragilité \\
La frontière \\
L'âge atomique \\
L'âge d'or \\
La grâce \\
L'agression \\
La guérison \\
La guerre \\
La guerre est-elle l'essentiel de toute politique ? \\
La guerre mondiale \\
La haine de la raison \\
La haine des images \\
La haine et le mépris \\
La honte \\
La jalousie \\
La jeunesse \\
La justice peut-elle se fonder sur le compromis ? \\
La justification \\
La lâcheté \\
La laïcité \\
La laideur \\
La langue de la raison \\
La langue maternelle \\
La lassitude \\
L'aléatoire \\
La liberté comporte-t-elle des degrés ? \\
La liberté peut-elle s'affirmer sans violence ? \\
L'aliénation \\
La limite \\
La logique a-t-elle une histoire ? \\
La logique du pire \\
La logique est-elle une science ? \\
La loyauté \\
L'altruisme \\
La lutte des classes \\
La main \\
La maîtrise de la nature \\
La maîtrise de soi \\
L'amateur \\
La matière \\
La matière et la forme \\
La maturité \\
La mauvaise volonté \\
L'ambiguïté \\
La méchanceté \\
La médiation \\
La médiocrité \\
La méditation \\
L'âme et l'esprit \\
La mélancolie \\
La mémoire collective \\
La menace \\
La mesure \\
La mesure du temps \\
La métaphore \\
La minorité \\
L'amitié \\
L'amitié est-elle une vertu ? \\
La modération \\
La modernité \\
La morale doit-elle en appeler à la nature ? \\
La morale et le droit \\
La morale et les mœurs \\
La moralité consiste-t-elle à se contraindre soi-même ? \\
La moralité et le traitement des animaux \\
La moralité se réduit-elle aux sentiments ? \\
La mort \\
La mort de l'art \\
La mort de l'homme \\
L'amour a-t-il des raisons ? \\
L'amour de l'art \\
L'amour de la vie \\
L'amour de soi \\
L'amour et la mort \\
La multitude \\
La naissance \\
La naïveté \\
L'analogie \\
L'analyse \\
L'anarchie \\
La nation est-elle dépassée ? \\
La nature \\
La nature a-t-elle des droits ? \\
La nature a-t-elle un langage ? \\
La nature du fait moral \\
La nature est-elle artiste ? \\
La nature est-elle politique ? \\
La nature est-elle une norme ? \\
La nature et le beau \\
La nature peut-elle être belle ? \\
La nature se donne-t-elle à penser ? \\
L'anéantissement \\
La négation \\
La neutralité \\
L'angoisse \\
L'animal et la bête \\
L'animalité \\
La noblesse \\
L'anomalie \\
L'anormal \\
La nostalgie \\
La notion de finalité a-t-elle de l'intérêt pour le savant ? \\
La notion de genre littéraire \\
La nouveauté \\
L'anticipation \\
L'antinomie \\
La nudité \\
La nuit et le jour \\
La paix \\
La paresse \\
La parole et l'écriture \\
La part de l'ombre \\
La participation \\
La passion amoureuse \\
La passion de la justice \\
La passion de la vérité peut-elle être source d'erreur ? \\
La passion de l'égalité \\
La passivité \\
La patrie \\
La pauvreté \\
La peine \\
La peine de mort \\
La pensée a-t-elle une histoire ? \\
La pensée de l'espace \\
La pensée formelle \\
La pensée formelle est-elle privée d'objet ? \\
La pensée formelle est-elle une pensée vide ? \\
La pensée obéit-elle à des lois ? \\
La perception \\
La perfection morale \\
La personnalité \\
La perspective \\
La peur de la vérité \\
La philosophie a-t-elle une histoire ? \\
La philosophie doit-elle être une science ? \\
La philosophie est-elle une science ? \\
La philosophie et le sens commun \\
La philosophie et les sciences \\
La philosophie peut-elle être une science ? \\
La pitié \\
La pitié a-t-elle une valeur ? \\
La place de la philosophie dans la culture \\
La place publique \\
La pluralité \\
La pluralité des langues \\
La politesse \\
La politique doit-elle se mêler du bonheur ? \\
La politique est-elle affaire de compétence ? \\
La politique est-elle une science ? \\
La politique et les passions \\
La possession \\
L'apparence \\
L'apparence du pouvoir \\
La précaution \\
La présence \\
La présence d'esprit \\
La preuve \\
La prévision \\
La prison \\
La privation \\
La privation de liberté \\
La promesse \\
La propriété \\
La protection \\
La prudence \\
La publicité \\
La pudeur \\
La puissance \\
La pureté \\
La question de l'essence \\
La question des origines \\
La question sociale \\
La radicalité \\
La radicalité est-elle une exigence philosophique ? \\
La raison \\
La raison a-t-elle une histoire ? \\
La raison d'État \\
La raison est-elle toujours raisonnable ? \\
La raison et le réel \\
L'archive \\
La réaction \\
La réalité \\
La réalité de l'espace \\
La réalité du monde extérieur \\
La recherche \\
La recherche de la perfection \\
La recherche d'identité \\
La réciprocité \\
La réconciliation \\
La reconnaissance \\
La réflexion \\
La réfutation \\
La règle du jeu \\
La régression à l'infini \\
La relation \\
La relativité \\
La religion peut-elle être naturelle ? \\
La Renaissance \\
La rencontre \\
La répétition \\
La représentation \\
La reproduction \\
La reproduction des œuvres d'art \\
La république \\
La résistance \\
La résolution \\
La ressemblance \\
La révélation \\
La révolution \\
L'argent \\
L'argent est-il un mal nécessaire ? \\
La rhétorique \\
La rigueur \\
La rivalité \\
L'art a-t-il une histoire ? \\
L'art cinématographique \\
L'art décoratif \\
L'art du portrait \\
L'art est-il méthodique ? \\
L'art est-il subversif ? \\
L'art est-il une promesse de bonheur ? \\
L'art et la manière \\
L'art et la nouveauté \\
L'art et l'espace \\
L'artificiel \\
L'artiste \\
L'artiste et la sensation \\
L'artiste travaille-t-il ? \\
L'art populaire \\
La sagesse \\
La sanction \\
La santé \\
La santé mentale \\
La scène du monde \\
La science a-t-elle besoin d'un critère de démarcation entre science et non science ? \\
La science a-t-elle une histoire ? \\
La science est-elle un jeu ? \\
La science et la foi \\
La science peut-elle tout expliquer ? \\
La séduction \\
La sensibilité \\
La séparation \\
La sérénité \\
La servitude \\
La servitude volontaire \\
La simplicité \\
La sociabilité \\
La société existe-t-elle ? \\
La solitude \\
La sollicitude \\
La somme et le tout \\
La souffrance a-t-elle un sens ? \\
La souffrance des animaux \\
La souveraineté \\
La spontanéité \\
L'assentiment \\
L'association des idées \\
La standardisation \\
La structure \\
La subjectivité \\
La superstition \\
La surface et la profondeur \\
La surprise \\
La survie \\
La sympathie \\
La tâche d'exister \\
La technique crée-t-elle son propre monde ? \\
La technique est-elle l'application de la science ? \\
La technocratie \\
La tentation \\
La terre \\
La terreur \\
L'athéisme \\
La théorie \\
La totalité \\
La trace \\
La tradition \\
La traduction \\
La tragédie \\
La transcendance \\
La transgression des règles \\
La transmission \\
La transparence \\
La transparence des consciences \\
La tristesse \\
L'attention \\
L'attraction \\
La tyrannie \\
L'au-delà \\
L'authenticité \\
L'automate \\
L'automatisation du raisonnement \\
L'autonomie \\
L'autorité \\
L'autorité de la science \\
La valeur d'échange \\
La valeur de l'argent \\
La valeur de la vie \\
La valeur du consensus \\
La valeur du témoignage \\
La vanité \\
La vengeance \\
L'aventure \\
La vérification \\
La vérité de la perception \\
La vérité des arts \\
La vérité est-elle affaire de croyance ou de savoir ? \\
La vérité est-elle objective ? \\
La vérité se communique-t-elle ? \\
La vertu du citoyen \\
L'aveu \\
La vie de l'esprit \\
La vie des rêves \\
La vie est-elle objet de science ? \\
La vie est-elle un songe ? \\
La vie moderne \\
La vie psychique \\
La ville \\
La violence peut-elle être gratuite ? \\
La vitesse \\
La volonté peut-elle être générale ? \\
La vraie vie \\
La vue et l'ouïe \\
La vulgarité \\
Le bavardage \\
Le beau naturel \\
Le besoin \\
Le besoin de philosophie \\
Le besoin de signes \\
Le bien commun \\
Le bien et le mal \\
Le bien-être \\
Le bien public \\
L'éblouissement \\
Le bon Dieu \\
Le bon gouvernement \\
Le bonheur est-il affaire de vertu ? \\
Le bon régime \\
Le bon sens \\
Le calcul des plaisirs \\
Le cannibalisme \\
Le cas de conscience \\
Le cas particulier \\
Le cerveau pense-t-il ? \\
L'échange \\
Le chaos \\
Le choc esthétique \\
Le choix de philosopher \\
Le choix d'un métier \\
L'éclat \\
Le cœur \\
L'école de la vie \\
Le combat \\
Le commencement \\
Le commerce \\
Le commun \\
Le compromis \\
Le concept \\
Le concept de matière \\
Le concept de structure \\
Le concret \\
L'économie \\
L'économie a-t-elle des lois ? \\
L'économie des moyens \\
L'économique et le politique \\
Le conservatisme \\
Le contrat \\
Le corps et l'esprit \\
Le cosmopolitisme \\
Le cours des choses \\
Le cours du temps \\
Le crime contre l'humanité \\
Le crime inexpiable \\
Le critère \\
L'écriture peut-elle porter secours à la pensée ? \\
Le cynisme \\
Le déchet \\
Le délire \\
Le démoniaque \\
Le dépaysement \\
Le désaccord \\
Le désespoir \\
Le désintéressement \\
Le désir de connaître \\
Le désir d'égalité \\
Le désir de savoir \\
Le désir d'être autre \\
Le désir de vivre \\
Le désordre \\
Le deuil \\
Le devoir de mémoire \\
Le dialogue entre nations \\
Le dictionnaire \\
Le Dieu des philosophes \\
Le divertissement \\
Le divin \\
Le don \\
Le donné \\
Le doute est-il le principe de la méthode scientifique ? \\
Le droit à l'erreur \\
Le droit d'auteur \\
Le droit est-il une science ? \\
Le Droit et l'État \\
Le droit peut-il échapper à l'histoire ? \\
L'éducation du goût \\
Le fait et le droit \\
Le familier \\
Le fantastique \\
Le fatalisme l'incarnation \\
Le faux \\
Le faux en art \\
L'effectivité \\
L'efficacité des discours \\
Le fin mot de l'histoire \\
Le fondement \\
Le fondement de l'autorité \\
Le formalisme \\
L'égalité \\
L'égalité des citoyens \\
Légalité et légitimité \\
L'égarement \\
Le genre humain \\
Le goût du risque \\
Le gouvernement des meilleurs \\
Le grand art est-il de plaire ? \\
Le hasard \\
Le jeu \\
Le juste et le bien \\
Le langage de la morale \\
Le langage de la science \\
Le langage du corps \\
Le langage mathématique \\
Le langage rapproche-t-il ou sépare-t-il les hommes ? \\
L'élégance \\
L'élémentaire \\
Le lien causal \\
Le lien social \\
Le luxe \\
Le mal a-t-il des raisons ? \\
Le malentendu \\
Le marché \\
Le mariage est-il un contrat ? \\
Le matérialisme \\
Le matériel et le virtuel \\
Le mauvais goût \\
Le méchant est-il malheureux ? \\
Le même et l'autre \\
Le mensonge est-il la plus grande transgression ? \\
Le mépris \\
Le mérite \\
Le métier de philosophe \\
Le métier de savant \\
Le mieux est-il l'ennemi du bien ? \\
Le milieu \\
Le moi est-il objet de connaissance ? \\
Le moindre mal \\
Le moi n'est-il qu'une fiction ? \\
Le monde a-t-il besoin de moi ? \\
Le monde des images \\
Le monde est-il une marchandise ? \\
Le monde extérieur \\
Le monde intelligible \\
Le monde sensible \\
L'empathie est-elle possible ? \\
L'empirisme \\
L'empirisme exclut-il l'abstraction ? \\
Le musée \\
Le mystère \\
Le mysticisme \\
Le naturel \\
L'Encyclopédie \\
Le néant \\
Le négatif \\
L'énergie du désespoir \\
L'enfance de l'art \\
L'enfance est-elle ce qui doit être surmonté ? \\
L'enfant \\
L'énigme \\
L'ennui \\
Le nombre \\
L'enquête \\
L'enthousiasme \\
L'environnement \\
L'environnement est-il un problème politique ? \\
Le paradoxe \\
Le passage à l'acte \\
Le passé \\
Le pathologique \\
Le patrimoine de l'humanité \\
Le patriotisme est-il une vertu ? \\
Le paysage \\
Le philosophe et le sophiste \\
Le plaisir \\
Le plaisir et la douleur \\
Le plaisir et la jouissance \\
Le pluralisme \\
Le poétique \\
Le poids du passé \\
Le point de vue \\
Le point de vue d'autrui \\
Le portrait \\
Le possible \\
Le possible et le réel \\
Le possible et le virtuel \\
Le possible existe-t-il ? \\
Le pouvoir de l'habitude \\
Le pouvoir de l'imagination \\
Le pouvoir des images \\
Le pouvoir des mots \\
Le pouvoir des paroles \\
Le pouvoir du concept \\
Le pouvoir du peuple \\
Le pouvoir et la violence \\
Le pouvoir magique \\
Le pragmatisme \\
Le présent \\
Le principe \\
Le principe de non-contradiction \\
Le principe de réalité \\
Le prix des choses \\
Le probable \\
Le procès d'intention \\
Le progrès \\
Le progrès est-il réversible ? \\
Le propre de l'homme \\
Le provisoire \\
Le public \\
Le public et le privé \\
Le quelconque \\
L'équité \\
Le quotidien \\
Le racisme \\
Le raffinement \\
Le rationalisme \\
Le réel est-il rationnel ? \\
Le regard \\
Le relativisme \\
Le religieux est-il inutile ? \\
Le respect \\
Le ridicule \\
Le rite \\
Le roman \\
Le roman peut-il être philosophique ? \\
Le romantisme \\
L'érotisme \\
Le rythme \\
Le sacré et le profane \\
Le sacrifice \\
Les âges de la vie \\
Les Anciens et les Modernes \\
Les animaux pensent-ils ? \\
Les beaux-arts \\
Les belles choses \\
Les bienfaits de la coopération \\
Les biotechnologies \\
Le scandale \\
Les catastrophes \\
Les catégories \\
Le scepticisme a-t-il des limites ? \\
Les choses \\
Les cinq sens \\
Les conditions du dialogue \\
Les conséquences \\
Les disciplines scientifiques et leurs interfaces \\
Les droits de l'homme \\
Les droits de l'homme sont-ils une abstraction ? \\
Les droits des animaux \\
Le secret \\
Les enfants \\
Le sens commun \\
Le sens de la situation \\
Le sensible et la science \\
Le sens interne \\
Le sens moral \\
Le sentiment de culpabilité \\
Le sentiment d'injustice \\
Le service de l'État \\
Les faits parlent-ils d'eux-mêmes ? \\
Les fausses sciences \\
Les fins de la science \\
Les frontières \\
Les fruits du travail \\
Les genres de vie \\
Les héros \\
Les hommes sont-ils frères ? \\
Les idées ont-elles une histoire ? \\
Le silence a-t-il un sens ? \\
Les images empêchent-elles de penser ? \\
Les inégalités de la nature doivent-elles être compensées ? \\
Les langues que nous parlons sont-elles imparfaites ? \\
Les lettres et les sciences \\
Les limites de la démocratie \\
Les limites de la tolérance \\
Les limites de l'obéissance \\
Les limites du langage \\
Les lois de la guerre \\
Les lois de la pensée \\
Les lois et les armes \\
Les machines \\
Les mathématiques et la quantité \\
Les mathématiques et l'expérience \\
Les mathématiques ont-elles affaire au réel ? \\
Les mathématiques sont-elles un jeu de l'esprit ? \\
Les mondes possibles \\
Les morts \\
Les mots nous éloignent-ils des choses ? \\
Le soin \\
Le soldat \\
Les opérations de l'esprit \\
Le sophiste et le philosophe \\
Le souci \\
Le souci de soi \\
Le soupçon \\
Le souvenir \\
L'espace public \\
Les passions sont-elles toutes bonnes ? \\
L'espèce et l'individu \\
L'espèce humaine \\
Le spectacle de la nature \\
Le spectacle du monde \\
Les phénomènes \\
Le spirituel et le temporel \\
L'espoir peut-il être raisonnable ? \\
Le sport : s'accomplir ou se dépasser ? \\
Les preuves de l'existence de Dieu \\
L'esprit critique \\
L'esprit de corps \\
L'esprit du christianisme \\
L'esprit est-il une chose ? \\
L'esprit et la lettre \\
L'esprit peut-il être divisé ? \\
Les raisons du choix \\
Les rapports entre les hommes sont-ils des rapports de force ? \\
Les robots \\
Les sciences appliquées \\
Les sciences humaines traitent-elles de l'homme ? \\
Les sciences nous donnent-elles des normes ? \\
Les sciences sociales ont-elles un objet ? \\
L'essence de la technique \\
Les sentiments \\
Les techniques du corps \\
Les témoignages et la preuve \\
L'estime de soi \\
Le style \\
Le style et le beau \\
Le sublime \\
Le substitut \\
Le suffrage universel \\
Le sujet \\
Les vivants et les morts \\
Le symbole \\
Le tableau \\
L'état de guerre \\
L'État doit-il nous rendre meilleurs ? \\
L'État est-il un arbitre ? \\
L'État mondial \\
Le témoignage des sens \\
Le temps de l'histoire \\
L'éternité \\
Le théâtral \\
Le théâtre de l'histoire \\
L'étonnement \\
Le toucher \\
Le tragique \\
Le trait d'esprit \\
L'étranger \\
L'étrangeté \\
Le travail de la raison \\
L'étude \\
L'Europe \\
L'évaluation \\
L'évasion \\
L'événement \\
Le vestige \\
L'évidence \\
L'évidence est-elle critère de vérité ? \\
Le virtuel \\
L'évolution \\
Le voyage \\
Le vraisemblable \\
Le vraisemblable et le romanesque \\
L'exactitude \\
L'exception \\
L'excès \\
L'exemple en morale \\
L'exercice de la vertu \\
L'exercice de la volonté \\
L'existence est-elle pensable ? \\
L'expérience de la beauté \\
L'expérience de la liberté \\
L'expérience de la maladie \\
L'expérience de la vie \\
L'expérience de pensée \\
L'expérience du danger \\
L'expérience du mal \\
L'expérience et l'expérimentation \\
L'expérience nous apprend-elle quelque chose ? \\
L'expertise \\
L'expression \\
L'habitude \\
L'harmonie \\
L'hérésie \\
L'héritage \\
L'héroïsme \\
L'histoire a-t-elle des lois ? \\
L'histoire de l'art peut-elle arriver à son terme ? \\
L'histoire est-elle écrite par les vainqueurs ? \\
L'histoire naturelle \\
L'histoire n'est-elle qu'un récit ? \\
L'histoire se répète-t-elle ? \\
L'historicité des sciences \\
L'homme de l'art \\
L'homme est-il un animal politique ? \\
L'homme et la machine \\
L'homme et le citoyen \\
L'homme intérieur \\
L'homme pense-t-il toujours ? \\
L'homme peut-il être libéré du besoin ? \\
L'homme peut-il se représenter un monde sans l'homme ? \\
L'homme se reconnaît-il mieux dans le travail ou dans le loisir ? \\
L'honneur \\
L'horizon \\
L'hospitalité est-elle un devoir ? \\
L'humain \\
L'humour \\
L'humour et l'ironie \\
Libéral et libertaire \\
Libéralité et libéralisme \\
Liberté et libération \\
L'idéal \\
L'idéal systématique \\
L'idée de communisme \\
L'idée de crise \\
L'idée de destin a-t-elle un sens ? \\
L'idée de justice \\
L'idée de loi de la nature \\
L'idée de mal nécessaire \\
L'idée de métier \\
L'idée de monde \\
L'idée de science \\
L'idée d'Europe \\
L'idée de « sciences exactes » \\
L'idée d'univers \\
L'idée d'université \\
L'identification \\
L'identité personnelle \\
L'idéologie \\
L'illimité \\
L'illusion \\
L'image et le réel \\
L'imagination a-t-elle des limites ? \\
L'imitation \\
L'immédiat \\
L'immensité \\
L'impardonnable \\
L'impassibilité \\
L'impatience \\
L'impératif d'impartialité \\
L'impératif hypothétique \\
L'impersonnel \\
L'implicite \\
L'imprévisible \\
L'impunité \\
L'inaccessible \\
L'incertitude \\
L'inconnaissable \\
L'inconnu \\
L'inconscient \\
L'inconscient collectif \\
L'inconscient est-il un concept scientifique ? \\
L'incroyable \\
L'indécence \\
L'indécision \\
L'indémontrable \\
L'indépendance \\
L'indéterminé \\
L'indicible \\
L'indifférence \\
L'individualisme \\
L'induction \\
L'inéluctable \\
L'infini \\
L'information \\
L'ingénuité \\
L'inhumain \\
L'inimaginable \\
L'injustifiable \\
L'innocence \\
L'innommable \\
L'inquiétude \\
L'insignifiant \\
L'insolite \\
L'instant \\
L'instinct \\
L'institution \\
L'instrument \\
L'intelligence \\
L'interaction \\
L'interdisciplinarité \\
L'interdit \\
L'intérêt \\
L'intériorité \\
L'interprétation \\
L'interprétation est-elle une science ? \\
L'intimité \\
L'intolérable \\
L'introspection \\
L'intuition \\
L'inutile \\
L'invisible \\
L'ironie \\
L'irrationnel \\
L'irréparable \\
L'irréversible \\
L'ivresse \\
L'objection de conscience \\
L'objectivation \\
L'objectivité \\
L'objet de l'amour \\
L'objet des mathématiques \\
L'obligation \\
L'observation \\
L'occasion \\
L'omniscience \\
L'opinion publique \\
L'opposition \\
L'optimisme \\
L'ordinaire \\
L'ordre \\
L'ordre du monde \\
L'ordre est-il dans les choses ? \\
L'ordre public \\
L'organisation du travail \\
L'orgueil \\
L'orientation \\
L'Orient et l'Occident \\
L'origine de la violence \\
L'origine du droit \\
L'oubli \\
L'outil \\
L'un \\
L'un et le multiple \\
L'uniformité \\
L'unité de la science \\
L'unité des arts \\
L'unité du genre humain \\
L'universalisme \\
L'universel \\
L'usage \\
L'utile \\
Maître et disciple \\
Maladie et convalescence \\
Maladies du corps, maladies de l'âme \\
Ma parole m'engage-t-elle ? \\
Masculin, féminin \\
Médecine et philosophie \\
Méditer \\
Métaphysique et politique \\
Mon corps \\
Mon corps m'appartient-il ? \\
Morale et calcul \\
Morale et intérêt \\
Morale et technique \\
Moralité et connaissance \\
Mythe et pensée \\
Naître \\
Nature, monde, univers \\
Nécessité et contingence \\
Ne pas rire, ne pas pleurer, mais comprendre \\
Nos pensées dépendent-elles de nous ? \\
Nul n'est méchant volontairement \\
N'y a-t-il d'amitié qu'entre égaux ? \\
N'y a-t-il de certitude que mathématique ? \\
N'y a-t-il de propriété que privée ? \\
N'y a-t-il de science que de ce qui est mathématisable ? \\
N'y a-t-il de science que du mesurable ? \\
N'y a-t-il que des individus ? \\
Objectiver \\
Œil pour œil, dent pour dent \\
Origine et fondement \\
Parler, est-ce agir ? \\
Par où commencer ? \\
Peinture et histoire \\
Penser, est-ce calculer ? \\
Penser et raisonner \\
Penser l'impossible \\
Penser sans sujet \\
Percevoir et juger \\
Perdre son âme \\
Perdre son temps \\
Personne et individu \\
Personne n'est innocent \\
Peut-être se mettre à la place de l'autre ? \\
Peut-il y avoir une histoire universelle ? \\
Peut-il y avoir une philosophie appliquée ? \\
Peut-il y avoir une science de l'éducation ? \\
Peut-il y avoir un État mondial ? \\
Peut-on aimer ce qu'on ne connaît pas ? \\
Peut-on aimer son travail ? \\
Peut-on apprendre à être juste ? \\
Peut-on apprendre à penser ? \\
Peut-on avoir conscience de soi sans avoir conscience d'autrui ? \\
Peut-on avoir raison tout.e seul.e ? \\
Peut-on contester les droits de l'homme ? \\
Peut-on critiquer la démocratie ? \\
Peut-on décider de croire ? \\
Peut-on définir la vie ? \\
Peut-on désirer ce qu'on ne veut pas ? \\
Peut-on désirer l'impossible ? \\
Peut-on douter de sa propre existence ? \\
Peut-on douter de tout ? \\
Peut-on écrire comme on parle ? \\
Peut-on être en avance sur son temps ? \\
Peut-on être indifférent à son bonheur ? \\
Peut-on exercer son esprit ? \\
Peut-on expliquer une œuvre d'art ? \\
Peut-on faire la paix ? \\
Peut-on forcer un homme à être libre ? \\
Peut-on gâcher son talent ? \\
Peut-on inventer en morale ? \\
Peut-on légitimer la violence ? \\
Peut-on limiter l'expression de la volonté du peuple ? \\
Peut-on manipuler les esprits ? \\
Peut-on ne pas croire ? \\
Peut-on ne pas savoir ce que l'on dit ? \\
Peut-on ne pas savoir ce que l'on fait ? \\
Peut-on ne penser à rien ? \\
Peut-on nier l'évidence ? \\
Peut-on oublier de vivre ? \\
Peut-on parler de corruption des mœurs ? \\
Peut-on parler de droits des animaux ? \\
Peut-on parler de « travail intellectuel » ? \\
Peut-on parler d'une expérience religieuse ? \\
Peut-on parler pour ne rien dire ? \\
Peut-on penser la nouveauté ? \\
Peut-on penser le changement ? \\
Peut-on penser l'impossible ? \\
Peut-on penser sans images ? \\
Peut-on penser sans les signes ? \\
Peut-on penser un art sans œuvres ? \\
Peut-on penser une société sans État ? \\
Peut-on perdre la raison ? \\
Peut-on perdre son identité ? \\
Peut-on prouver l'existence du monde ? \\
Peut-on raconter sa vie ? \\
Peut-on raisonner sans règles ? \\
Peut-on refuser de voir la vérité ? \\
Peut-on refuser la loi ? \\
Peut-on représenter le peuple ? \\
Peut-on rester dans le doute ? \\
Peut-on revenir sur ses erreurs ? \\
Peut-on s'attendre à tout ? \\
Peut-on savoir quelque chose de l'avenir ? \\
Peut-on se désintéresser de la politique ? \\
Peut-on se désintéresser de son bonheur ? \\
Peut-on se fier à l'intuition ? \\
Peut-on se fier à son intuition ? \\
Peut-on se méfier de soi-même ? \\
Peut-on se mentir à soi-même \\
Peut-on se mentir à soi-même ? \\
Peut-on séparer politique et économie ? \\
Peut-on se passer de mythes ? \\
Peut-on se passer de religion ? \\
Peut-on se vouloir parfait ? \\
Peut-on suspendre son jugement ? \\
Peut-on tout démontrer ? \\
Peut-on tout dire ? \\
Peut-on tout enseigner ? \\
Peut-on tout expliquer ? \\
Peut-on tout imaginer ? \\
Peut-on tout imiter ? \\
Peut-on tout mesurer ? \\
Peut-on tout partager ? \\
Peut-on tout tolérer ? \\
Peut-on vivre sans illusions ? \\
Peut-on vivre sans passion ? \\
Peut-on vouloir le mal ? \\
Philosophie et poésie \\
Physique et métaphysique \\
Pluralisme et politique \\
Poésie et vérité \\
Poétique et prosaïque \\
Police et politique \\
Politique et coopération \\
Politique et unité \\
Pourquoi aimons-nous la musique ? \\
Pourquoi conserver les œuvres d'art ? \\
Pourquoi construire des monuments ? \\
Pourquoi critiquer la raison ? \\
Pourquoi défendre le faible ? \\
Pourquoi des artifices ? \\
Pourquoi des guerres ? \\
Pourquoi des institutions ? \\
Pourquoi désirer l'immortalité ? \\
Pourquoi des maîtres ? \\
Pourquoi des musées ? \\
Pourquoi des philosophes ? \\
Pourquoi des poètes ? \\
Pourquoi des psychologues ? \\
Pourquoi des sociologues ? \\
Pourquoi écrit-on les lois ? \\
Pourquoi exposer les œuvres d'art ? \\
Pourquoi imiter ? \\
Pourquoi le théâtre ? \\
Pourquoi l'homme est-il l'objet de plusieurs sciences ? \\
Pourquoi mentir ? \\
Pourquoi parler du travail comme d'un droit ? \\
Pourquoi parle-t-on d'économie politique ? \\
Pourquoi préférer l'original à la reproduction ? \\
Pourquoi punir ? \\
Pourquoi punit-on ? \\
Pourquoi rit-on ? \\
Pourquoi s'étonner ? \\
Pourquoi travaille-t-on ? \\
Pourquoi voulons-nous savoir ? \\
Pourquoi voyager ? \\
Pourquoi y a-t-il quelque chose plutôt que rien ? \\
Pour vivre heureux, vivons cachés \\
Pouvoir, magie, secret \\
Pouvons-nous être certains que nous ne rêvons pas ? \\
Pouvons-nous savoir ce que nous ignorons ? \\
Prédiction et probabilité \\
Prendre soin \\
Principes et stratégie \\
Production et création \\
Produire et créer \\
Puis-je aimer tous les hommes ? \\
Puis-je être sûr de ne pas me tromper ? \\
Puis-je être sûr que je ne rêve pas ? \\
Quand faut-il désobéir aux lois ? \\
Quand faut-il désobéir ? \\
Quand le temps passe, que reste-t-il ? \\
Qu'appelle-t-on destin ? \\
Qu'appelle-t-on penser ? \\
Qu'apprend-on dans les livres ? \\
Qu'apprend-on quand on apprend à parler ? \\
Qu'a-t-on le droit d'exiger ? \\
Que coûte une victoire ? \\
Que désire-t-on ? \\
Que disent les légendes ? \\
Que dit la musique ? \\
Que doit-on savoir avant d'agir ? \\
Que faire de la diversité des arts ? \\
Que faire de notre cerveau ? \\
Que fait la police ? \\
Que faut-il pour faire un monde ? \\
Que faut-il savoir pour agir ? \\
Quel est le fondement de la propriété ? \\
Quel est le pouvoir des mots ? La prévoyance \\
Quel est le rôle du concept en art ? \\
Quel est l'homme des Droits de l'homme ? \\
Quel est l'objet des mathématiques ? \\
Quel est l'objet des sciences humaines ? \\
Quelle est la spécificité de la communauté politique ? \\
Quelle est la valeur de l'expérience ? \\
Quelle idée le fanatique se fait-il de la vérité ? \\
Quelles sont les limites de la démonstration ? \\
Quels sont les droits de la conscience ? \\
Quel usage faut-il faire des exemples ? \\
Que nous apprend la grammaire ? \\
Que nous apprend l'étude du cerveau ? \\
Que nous apprennent les expériences de pensée ? \\
Que nous apprennent les jeux ? \\
Que nous apprennent les langues étrangères ? \\
Que nous apprennent les mythes ? \\
Que nous montre le cinéma ? \\
Que nous montrent les natures mortes ? \\
Que penser de la formule : « il faut suivre la nature » ? \\
Que penser de l'opposition travail manuel, travail intellectuel ? \\
Que peut la philosophie ? \\
Que prouvent les faits ? \\
Que signifie « donner le change » ? \\
Qu'est-ce la technique ? \\
Qu'est-ce qu'avoir de l'expérience ? \\
Qu'est-ce que comprendre ? \\
Qu'est-ce que créer ? \\
Qu'est-ce que Dieu pour athée ? \\
Qu'est-ce que guérir ? \\
Qu'est-ce que juger ? \\
Qu'est-ce que la critique ? \\
Qu'est-ce que la politique ? \\
Qu'est-ce que la science, si elle inclut la psychanalyse ? \\
Qu'est-ce que la scientificité ? \\
Qu'est-ce que la vie ? \\
Qu'est-ce que le cinéma donne à voir ? \\
Qu'est-ce que le dogmatisme ? \\
Qu'est-ce que le moi ? \\
Qu'est-ce que le nihilisme ? \\
Qu'est-ce que le pathologique nous apprend sur le normal ? \\
Qu'est-ce que le réel ? \\
Qu'est-ce que parler ? \\
Qu'est-ce que penser ? \\
Qu'est-ce que perdre sa liberté ? \\
Qu'est-ce que réfuter ? \\
Qu'est-ce que résister ? \\
Qu'est-ce que rester soi-même ? \\
Qu'est-ce qu'être dans le vrai ? \\
Qu'est-ce qu'être normal ? \\
Qu'est-ce qu'être réaliste ? \\
Qu'est-ce qu'être souverain ? \\
Qu'est-ce qu'être un sujet ? \\
Qu'est-ce qu'être vivant ? \\
Qu'est-ce que vérifier ? \\
Qu'est-ce que vivre ? \\
Qu'est-ce qui est concret ? \\
Qu'est-ce qui est essentiel ? \\
Qu'est-ce qui est le plus à craindre, l'ordre ou le désordre ? \\
Qu'est-ce qui est tragique ? \\
Qu'est-ce qui ne disparaît jamais ?/ \\
Qu'est-ce qu'on attend ? \\
Qu'est-ce qu'un ami ? \\
Qu'est-ce qu'un animal ? \\
Qu'est-ce qu'un artiste ? \\
Qu'est-ce qu'un axiome ? \\
Qu'est-ce qu'un chef d'œuvre ? \\
Qu'est-ce qu'un chef-d'œuvre ? \\
Qu'est-ce qu'un chef ? \\
Qu'est-ce qu'un choix éclairé ? \\
Qu'est-ce qu'un classique ? \\
Qu'est-ce qu'un concept philosophique ? \\
Qu'est-ce qu'un concept ? \\
Qu'est-ce qu'un contrat ? \\
Qu'est-ce qu'un contre-pouvoir ? \\
Qu'est-ce qu'un coup d'État ? \\
Qu'est-ce qu'un dieu ? \\
Qu'est-ce qu'un dilemme ? \\
Qu'est-ce qu'un dogme ? \\
Qu'est-ce qu'une action intentionnelle ? \\
Qu'est-ce qu'une belle mort ? \\
Qu'est-ce qu'une bonne traduction ? \\
Qu'est-ce qu'une connaissance par les faits ? \\
Qu'est-ce qu'une crise ? \\
Qu'est-ce qu'une croyance ? \\
Qu'est-ce qu'une découverte ? \\
Qu'est-ce qu'une éducation réussie ? \\
Qu'est-ce qu'une époque ? \\
Qu'est-ce qu'une expérience religieuse ? \\
Qu'est-ce qu'une grande cause ? \\
Qu'est-ce qu'une hypothèse ? \\
Qu'est-ce qu'une idée ? \\
Qu'est-ce qu'une illusion ? \\
Qu'est-ce qu'une institution ? \\
Qu'est-ce qu'une libération ? \\
Qu'est ce qu'une loi de la pensée ? \\
Qu'est-ce qu'une loi ? \\
Qu'est-ce qu'une machine ? \\
Qu'est-ce qu'une marchandise ? \\
Qu'est-ce qu'un empire ? \\
Qu'est-ce qu'une nation ? \\
Qu'est-ce qu'une norme ? \\
Qu'est-ce qu'une œuvre ratée ? \\
Qu'est-ce qu'une œuvre ? \\
Qu'est-ce qu'une personne ? \\
Qu'est-ce qu'une philosophie ? \\
Qu'est-ce qu'une preuve ? \\
Qu'est-ce qu'une règle de vie ? \\
Qu'est-ce qu'une règle ? \\
Qu'est-ce qu'une rencontre ? \\
Qu'est-ce qu'une révélation ? \\
Qu'est-ce qu'une révolution scientifique ? \\
Qu'est-ce qu'une révolution ? \\
Qu'est-ce qu'une science exacte ? \\
Qu'est-ce qu'une société libre ? \\
Qu'est-ce qu'un état mental ? \\
Qu'est-ce qu'un événement fondateur ? \\
Qu'est-ce qu'une vie humaine ? \\
Qu'est-ce qu'une ville ? \\
Qu'est-ce qu'une volonté raisonnable ? \\
Qu'est-ce qu'un exemple ? \\
Qu'est-ce qu'un expert ? \\
Qu'est-ce qu'un fait ? \\
Qu'est-ce qu'un faux problème ? \\
Qu'est-ce qu'un génie ? \\
Qu'est-ce qu'un grand philosophe ? \\
Qu'est-ce qu'un héros ? \\
Qu'est-ce qu'un homme sans éducation ? \\
Qu'est-ce qu'un homme seul ? \\
Qu'est-ce qu'un jeu ? \\
Qu'est-ce qu'un lieu commun ? \\
Qu'est-ce qu'un livre ? \\
Qu'est-ce qu'un maître ? \\
Qu'est-ce qu'un monstre ? \\
Qu'est-ce qu'un monument ? \\
Qu'est-ce qu'un mythe ? \\
Qu'est-ce qu'un nombre ? \\
Qu'est-ce qu'un ordre ? \\
Qu'est-ce qu'un paradoxe ? \\
Qu'est-ce qu'un peuple ? \\
Qu'est-ce qu'un plaisir pur ? \\
Qu'est-ce qu'un principe ? \\
Qu'est-ce qu'un problème politique ? \\
Qu'est-ce qu'un programmer ? \\
Qu'est-ce qu'un réseau ? \\
Qu'est-ce qu'un rhéteur ? \\
Qu'est-ce qu'un rite ? \\
Qu'est-ce qu'un sceptique ? \\
Qu'est-ce qu'un signe ? \\
Qu'est-ce qu'un sophisme ? \\
Qu'est-ce qu'un sophiste ? \\
Qu'est-ce qu'un symbole ? \\
Qu'est-ce qu'un système ? \\
Qu'est-ce qu'un temple ? \\
Qu'est-ce qu'un texte ? \\
Qu'est-ce qu'un traître ? \\
Que valent les préjugés ? \\
Que vaut l'excuse : « Je ne l'ai pas fait exprès» ? \\
Que vaut une preuve contre un préjugé ? \\
Que veut dire avoir raison ? \\
Que voit-on dans un miroir \\
Qu'expriment les mythes ? \\
Qui est compétent en matière politique ? \\
Qui est l'autre ? \\
Qui est l'homme des sciences humaines ? \\
Qui pense ? \\
Qui suis-je et qui es-tu ? \\
Raisonnement et expérimentation \\
Raisonner et calculer \\
Reconnaissance et inégalité \\
Réforme et révolution \\
Refuser et réfuter \\
Règle morale et norme juridique \\
Religion et violence \\
Répondre de soi \\
Représenter \\
Résister peut-il être un droit ? \\
Respecter la nature, est-ce renoncer à l'exploiter ? \\
Révolte et révolution \\
Rien n'est sans raison \\
Rituels et cérémonies \\
Sait-on toujours ce que l'on fait ? \\
Sauver les apparences \\
Savoir, est-ce pouvoir ? \\
Savoir et pouvoir \\
Science et abstraction \\
Science et certitude \\
Science et domination sociale \\
Science et invention \\
Science et libération \\
Science et magie \\
Sciences de la nature et sciences de l'esprit \\
Sciences empiriques et critères du vrai \\
Si l'État n'existait pas, faudrait-il l'inventer ? \\
Société et biologie \\
Socrate \\
Solitude et isolement \\
Sommes-nous maîtres de nos paroles ? \\
Substance et sujet \\
Suffit-il de bien juger pour bien faire ? \\
Suffit-il de voir le meilleur pour le suivre ? \\
Suis-je ma mémoire ? \\
Suivre la coutume \\
Technique et apprentissage \\
Témoigner \\
Tenir pour vrai \\
Théorie et modèle \\
Théorie et pratique \\
Toucher \\
Tous les désirs sont-ils naturels ? \\
Tous les paradis sont-ils perdus ? \\
Tout a-t-il un prix ? \\
Tout comprendre, est-ce tout pardonner ? \\
Toute morale implique-t-elle l'effort ? \\
Toute origine est-elle mythique ? \\
Toute science est-elle naturelle ? \\
Toutes les convictions sont-elles respectables ? \\
Toutes les fautes se valent-elles ? \\
Toutes les inégalités ont-elles une importance politique ? \\
Tout est-il affaire de point de vue ? \\
Tout est-il politique ? \\
Tout est vanité \\
Toute vérité est-elle vérifiable ? \\
Tout pouvoir est-il politique ? \\
Tout savoir est-il un pouvoir ? \\
Traduire \\
Traduire et interpréter \\
Travail manuel et travail intellectuel \\
Travail manuel, travail intellectuel \\
Trouver sa voie \\
Un art peut-il être populaire ? \\
Une existence se démontre-t-elle ? \\
Une loi peut-elle être injuste ? \\
Une machine peut-elle penser ? \\
Une philosophie peut-elle être réactionnaire ? \\
Une religion rationnelle est-elle possible ? \\
Une science de la conscience est-elle possible ? \\
Une science de l'éducation est-elle possible ? \\
Une société sans religion est-elle possible ? \\
Un objet technique peut-il être beau ? \\
Un philosophe a-t-il des devoirs envers la société ? \\
Vaincre la mort \\
Vaut-il mieux oublier ou pardonner ? \\
Vices privés, vertus publiques \\
Vie active, vie contemplative \\
Vivre et bien vivre \\
Vivre sa vie \\
Vivre selon la nature \\
Voir, observer, penser \\
Y a-t-il de faux besoins ? \\
Y a-t-il de l'impensable ? \\
Y a-t-il de l'incommunicable ? \\
Y a-t-il de l'universel ? \\
Y a-t-il des actes gratuits ? \\
Y a-t-il des actions désintéressées ? \\
Y a-t-il des barbares ? \\
Y a-t-il des croyances rationnelles ? \\
Y a-t-il des despotes éclairés ? \\
Y a-t-il des dilemmes moraux ? \\
Y a-t-il des droits sans devoirs ? \\
Y a-t-il des erreurs de la nature ? \\
Y a-t-il des facultés dans l'esprit ? \\
Y a-t-il des faits moraux ? \\
Y a-t-il des fondements naturels à l'ordre social ? \\
Y a-t-il des genres de plaisir ? \\
Y a-t-il des genres du plaisir ? \\
Y a-t-il des intuitions morales ? \\
Y a-t-il des limites à l'exprimable ? \\
Y a-t-il des lois du social ? \\
Y a-t-il des pathologies sociales ? \\
Y a-t-il des pensées folles ? \\
Y a-t-il des peuples sans histoire ? \\
Y a-t-il des règles de la guerre ? \\
Y a-t-il des sociétés sans histoire ? \\
Y a-t-il des valeurs objectives ? \\
Y a-t-il des vérités indiscutables ? \\
Y a-t-il des violences légitimes ? \\
Y a-t-il trop d'images ? \\
Y a-t-il un art de penser ? \\
Y a-t-il un devoir d'indignation ? \\
Y a-t-il un droit à la différence ? \\
Y a-t-il un droit au travail ? \\
Y a-t-il un droit naturel ? \\
Y a-t-il une causalité empirique ? \\
Y a-t-il une compétence politique ? \\
Y a-t-il une expérience de l'éternité ? \\
Y a-t-il une expérience du néant ? \\
Y a-t-il une fin de l'histoire ? \\
Y a-t-il une hiérarchie des sciences ? \\
Y a-t-il une mécanique des passions ? \\
Y a-t-il une médecine de l'âme ? \\
Y a-t-il une méthode propre aux sciences humaines ? \\
Y a-t-il un empire de la technique ? \\
Y a-t-il une nature humaine ? \\
Y a-t-il une nécessité de l'Histoire ? \\
Y a-t-il une philosophie de la nature ? \\
Y a-t-il une science de l'esprit ? \\
Y a-t-il une science du moi ? \\
Y a-t-il une unité des langages humains ? \\
Y a-t-il une vérité des apparences ? \\
Y a-t-il une vérité des sentiments ? \\
Y a-t-il une vertu de l'oubli ? \\
Y a-t-il un langage unifié de la science ? \\
Y a-t-il un ordre des choses ? \\
Y a-t-il un ordre du monde ? \\
Y a-t-il un progrès en philosophie ? \\
Y a-t-il un progrès moral ? \\
Y a-t-il un rythme de l'histoire ? \\
Y a-t-il un savoir du corps ? \\
Y a-t-il un savoir politique ? \\
Y a-t-il un savoir pratique \\
Y a-t-il un usage purement instrumental de la raison ? \\
« À quelque chose malheur est bon » \\
« Dieu est mort » \\
« Je ne crois que ce que je vois » \\
« Je préfère une injustice à un désordre » \\
« La crainte est le commencement de la sagesse » \\
« La critique est aisée » \\
« La science ne pense pas » \\
« L'enfer est pavé de bonnes intentions » \\
« Le seul problème philosophique vraiment sérieux, c'est le suicide » \\
« Le travail rend libre » \\
« L'histoire jugera » \\
« Nul n'est censé ignorer la loi » \\
« Pas de liberté pour les ennemis de la liberté » ? \\
« Pauvre bête » \\
« Pourquoi » \\
« Sauver les apparences » \\
« Toute peine mérite salaire » \\


\section{Tri par thème des sujets d'agrégation externe}
\label{sec-3}
\subsection{Philosophie générale}
\label{sec-3-1}

\noindent
Abolir la propriété \\
Action et événement \\
Affirmer et nier \\
Agir \\
Ai-je une âme ? \\
Aimer la nature \\
Aimer la vie \\
Aimer les lois \\
Aimer une œuvre d'art \\
À l'impossible nul n'est tenu \\
Analyser les mœurs \\
Apprend-on à penser ? \\
Apprend-on à voir ? \\
Apprendre à penser \\
Apprendre à vivre \\
Apprendre à voir \\
Apprendre s'apprend-il ? \\
Après-coup \\
\emph{A priori} et \emph{a posteriori} \\
À quoi bon discuter ? \\
À quoi bon ? \\
À quoi faut-il renoncer ? \\
À quoi la conscience nous donne-t-elle accès ? \\
À quoi sert la dialectique ? \\
À quoi sert la négation ? \\
À quoi sert l'écriture ? \\
À quoi tient la fermeté du vouloir ? \\
Argumenter \\
Art et critique \\
Art et finitude \\
Art et politique \\
Art et vérité \\
A-t-on des raisons de croire ce qu'on croit ? \\
A-t-on des raisons de croire ? \\
Attente et espérance \\
Aussitôt dit, aussitôt fait \\
Avoir de l'autorité \\
Avoir de l'esprit \\
Avoir de l'expérience \\
Avoir des ennemis \\
Avoir des principes \\
Avoir du style \\
Avoir mauvaise conscience \\
Avoir peur \\
Avoir un corps \\
Avoir une bonne mémoire \\
Avoir une idée \\
Avons-nous à apprendre des images ? \\
Avons-nous besoin de métaphysique ? \\
Avons-nous besoin de spectacles ? \\
Avons-nous besoin d'un libre arbitre ? \\
Avons-nous des devoirs envers les morts ? \\
Avons-nous une responsabilité envers le passé ? \\
Avons-nous un monde commun ? \\
À chacun ses goûts \\
À quelles conditions est-il acceptable de travailler ? \\
À quelles conditions une pensée est-elle libre ? \\
À qui la faute ? \\
À quoi faut-il être fidèle ? \\
À quoi reconnaît-on une œuvre d'art ? \\
À quoi sert un exemple ? \\
À quoi servent les règles ? \\
À quoi servent les statistiques ? \\
À quoi servent les utopies ? \\
À science nouvelle, nouvelle philosophie ? \\
Bâtir un monde \\
Bien jouer son rôle \\
Bien juger \\
Bien parler \\
Calculer \\
Calculer et penser \\
Cartographier \\
Ce que sait le poète \\
Ce qui est subjectif est-il arbitraire ? \\
Ce qu'il y a \\
Ce qu'on ne peut pas vendre \\
Certitude et vérité \\
Cesser d'espérer \\
C'est trop beau pour être vrai ! \\
Ceux qui savent doivent-ils gouverner ? \\
Changer \\
Choisir \\
Choisir ses souvenirs ? \\
Choisit-on ses souvenirs ? \\
Choisit-on son corps ? \\
Chose et objet \\
Classer \\
Classer et ordonner \\
Collectionner \\
Commémorer \\
Commencer \\
Commencer en philosophie \\
Comment assumer les conséquences de ses actes ? \\
Comment bien vivre ? \\
Comment comprendre une croyance qu'on ne partage pas ? \\
Comment définir la raison ? \\
Comment définir la signification \\
Comment devient-on raisonnable ? \\
Comment penser la diversité des langues ? \\
Comment peut-on être sceptique ? \\
Comment sait-on qu'on se comprend ? \\
Comment trancher une controverse ? \\
Comment vivre ensemble ? \\
Comme on dit \\
Communiquer \\
Communiquer et enseigner \\
Compétence et autorité \\
Compter sur soi \\
Concept et image \\
Concept et métaphore \\
Conception et perception \\
Conduire sa vie \\
Conduire ses pensées \\
Connaissance et expérience \\
Connaître par les causes \\
Connaître ses limites \\
Conscience et mémoire \\
Consensus et conflit \\
Considère-t-on jamais le temps en lui-même ? \\
Contempler \\
Corps et identité \\
Correspondre \\
Création et production \\
Créer \\
Crise et progrès \\
Critiquer \\
Critiquer la démocratie \\
Croire aux fictions \\
Croire en Dieu \\
Croire, est-ce être faible ? \\
Croire et savoir \\
Croire savoir \\
Cultivons notre jardin \\
Culture et civilisation \\
Décider \\
Décomposer les choses \\
Décrire, est-ce déjà expliquer ? \\
Définir l'art : à quoi bon ? \\
Déjouer \\
Délibérer, est-ce être dans l'incertitude ? \\
De l'utilité des voyages \\
Démêler le vrai du faux \\
Dénaturer \\
Dépasser les apparences ? \\
Dépasser l'humain \\
De quel bonheur sommes-nous capables ? \\
De quel droit ? \\
De quelle réalité témoignent nos perceptions ? \\
De quelle vérité l'opinion est-elle capable ? \\
De quoi doute un sceptique ? \\
De quoi est-on conscient ? \\
De quoi est-on malheureux ? \\
De quoi la forme est-elle la forme ? \\
De quoi la logique est-elle la science ? \\
De quoi l'art nous délivre-t-il ? \\
De quoi les métaphysiciens parlent-ils ? \\
De quoi n'avons-nous pas conscience ? \\
De quoi ne peut-on pas répondre ? \\
De quoi parlent les mathématiques ? \\
De quoi parlent les théories physiques ? \\
De quoi pâtit-on ? \\
De quoi somme-nous prisonniers ? \\
De quoi sommes-nous responsables ? \\
De quoi y a-t-il expérience ? \\
De quoi y a-t-il histoire ? \\
Désacraliser \\
Désirer \\
Désire-t-on la reconnaissance ? \\
Désobéir \\
Détruire pour reconstruire \\
Devenir citoyen \\
Devenir et évolution \\
Devient-on raisonnable ? \\
Dialectique et Philosophie \\
Dialoguer \\
Dieu aurait-il pu mieux faire ? \\
Dieu est-il une limite de la pensée ? \\
Dieu est mort \\
Dieu et César \\
Dieu, prouvé ou éprouvé ? \\
Dire ce qui est \\
Dire, est-ce faire ? \\
Dire et faire \\
Dire et montrer \\
Dire oui \\
Dire « je » \\
Diriger son esprit \\
Discussion et dialogue \\
Distinguer \\
Doit-on bien juger pour bien faire ? \\
Doit-on respecter la nature ? \\
Doit-on se faire l'avocat du diable ? \\
Doit-on tout calculer ? \\
Donner \\
Donner des exemples \\
Donner des raisons \\
Donner du sens \\
Donner raison \\
Donner raison, rendre raison \\
Donner sa parole \\
Donner un exemple \\
D'où vient aux objets techniques leur beauté ? \\
D'où vient la signification des mots ? \\
D'où vient le mal ? \\
D'où vient le plaisir de lire ? \\
D'où vient que l'histoire soit autre chose qu'un chaos ? \\
Échange et don \\
Éclairer \\
Écrire \\
Écrire l'histoire \\
Éducation de l'homme, éducation du citoyen \\
Égoïsme et méchanceté \\
Empirique et expérimental \\
En quel sens peut-on parler de transcendance ? \\
Enquêter \\
En quoi la matière s'oppose-t-elle à l'esprit ? \\
En quoi la technique fait-elle question ? \\
En quoi une insulte est-elle blessante ? \\
Enseigner \\
Entendement et raison \\
Entendre raison \\
Entrer en scène \\
Énumérer \\
Essayer \\
Essence et existence \\
Est-il difficile de savoir ce que l'on veut ? \\
Est-il difficile d'être heureux ? \\
Est-il judicieux de revenir sur ses décisions ? \\
Est-il possible de croire en la vie éternelle ? \\
Est-il possible de douter de tout ? \\
Est-il vrai qu'on apprenne de ses erreurs ? \\
Estimer \\
Est-on fondé à distinguer la justice et le droit ? \\
Est-on le produit d'une culture ? \\
État et société \\
Éternité et immortalité \\
Ethnologie et sociologie \\
Être affairé \\
Être, c'est agir \\
Être chez soi \\
Être compris \\
Être dans son bon droit \\
Être de mauvaise humeur \\
Être de son temps \\
Être égal à soi-même \\
Être en bonne santé \\
Être en désaccord \\
Être ensemble \\
Être est-ce agir ? \\
Être et avoir \\
Être et devoir être \\
Être exemplaire \\
Être hors de soi \\
Être logique \\
Être maître de soi \\
Être malade \\
Être méchant volontairement \\
Être né \\
Être pauvre \\
Être réaliste \\
Être sans scrupule \\
Être sceptique \\
Être seul avec sa conscience \\
Être seul avec soi-même \\
Être soi-même \\
Être spirituel \\
Être systématique \\
Être un artiste \\
Être un corps \\
Étudier \\
Évolution et progrès \\
Existence et essence \\
Exister \\
Existe-t-il des dilemmes moraux ? \\
Existe-t-il des questions sans réponse ? \\
Existe-t-il des sciences de différentes natures ? \\
Existe-t-il une opinion publique ? \\
Expérimenter \\
Expliquer \\
Expression et création \\
Expression et signification \\
Faire ce qu'on dit \\
Faire corps \\
Faire de nécessité vertu \\
Faire des choix \\
Faire école \\
Faire et laisser faire \\
Faire justice \\
Faire la paix \\
Faire la révolution \\
Faire l'histoire \\
Faire une expérience \\
Faut-il aimer la vie ? \\
Faut-il aimer son prochain comme soi-même ? \\
Faut-il aller au-delà des apparences ? \\
Faut-il avoir des ennemis ? \\
Faut-il avoir des principes ? \\
Faut-il avoir peur de la liberté ? \\
Faut-il avoir peur des habitudes ? \\
Faut-il concilier les contraires ? \\
Faut-il condamner la rhétorique ? \\
Faut-il condamner les illusions ? \\
Faut-il craindre le pire ? \\
Faut-il craindre les masses ? \\
Faut-il croire au progrès ? \\
Faut-il croire en quelque chose ? \\
Faut-il distinguer ce qui est de ce qui doit être ? \\
Faut-il être cosmopolite ? \\
Faut-il être mesuré en toutes choses ? \\
Faut-il être objectif ? \\
Faut-il laisser parler la nature ? \\
Faut-il ménager les apparences ? \\
Faut-il mépriser le luxe ? \\
Faut-il n'être jamais méchant ? \\
Faut-il opposer l'histoire et la fiction ? \\
Faut-il opposer rhétorique et philosophie ? \\
Faut-il parler pour avoir des idées générales ? \\
Faut-il perdre ses illusions ? \\
Faut-il préférer le bonheur à la vérité ? \\
Faut-il rechercher la certitude ? \\
Faut-il renoncer à son désir ? \\
Faut-il respecter la nature ? \\
Faut-il respecter les convenances ? \\
Faut-il rompre avec le passé ? \\
Faut-il se fier à la majorité ? \\
Faut-il se méfier de l'imagination ? \\
Faut-il suivre ses intuitions ? \\
Faut-il vivre comme si l'on ne devait jamais mourir ? \\
Fonder \\
Fonder la justice \\
Fonder une cité \\
Force et violence \\
Gérer et gouverner \\
Gouvernement des hommes et administration des choses \\
Grammaire et philosophie \\
Grandeur et décadence \\
Habiter \\
Habiter le monde \\
Haïr la raison \\
Histoire et fiction \\
Histoire et mémoire \\
Humour et ironie \\
Ici et maintenant \\
Identité et communauté \\
Illégalité et injustice \\
Imaginer \\
Indépendance et autonomie \\
Indépendance et liberté \\
Individuation et identité \\
Innocenter le devenir \\
Instinct et morale \\
Instruire et éduquer \\
Interpréter \\
Intuition et déduction \\
J'ai un corps \\
Je mens \\
Je sens, donc je suis \\
Je, tu, il \\
Jouer \\
Jouer un rôle \\
Juger \\
Jusqu'à quel point pouvons-nous juger autrui ? \\
Justice et vengeance \\
Justice et violence \\
Justifier \\
La banalité \\
L'abandon \\
La bassesse \\
La béatitude \\
La beauté \\
La beauté a-t-elle une histoire ? \\
La beauté de la nature \\
La beauté des ruines \\
La beauté du diable \\
La beauté du geste \\
La beauté du monde \\
La beauté est-elle l'objet d'une connaissance ? \\
La beauté peut-elle délivrer une vérité ? \\
La belle nature \\
La bestialité \\
La bêtise \\
La bêtise n'est-elle pas proprement humaine ? \\
La bibliothèque \\
La bienveillance \\
La biologie peut-elle se passer de causes finales ? \\
La bonne conscience \\
L'absence \\
L'absolu \\
L'abstraction \\
L'abstrait et le concret \\
L'abus de pouvoir \\
L'académisme dans l'art \\
La casuistique \\
La causalité \\
La cause \\
La cause et la raison \\
La cause première \\
L'accident \\
L'accidentel \\
L'accomplissement \\
L'accord \\
La censure \\
La certitude \\
La chair \\
La chance \\
La charité \\
La charité est-elle une vertu ? \\
L'achèvement de l'œuvre \\
La chose \\
La chose publique \\
La chronologie \\
La circonspection \\
La citation \\
La civilisation \\
La civilité \\
La clarté \\
La classification \\
La clémence \\
La colère \\
La comédie humaine \\
La communauté scientifique \\
La communication \\
La compassion \\
La compétence \\
La compréhension \\
La concorde \\
La condition \\
La condition humaine \\
La confiance \\
La connaissance animale \\
La connaissance a-t-elle des limites ? \\
La connaissance du singulier \\
La connaissance mathématique \\
La conquête \\
La conscience de soi \\
La conscience historique \\
La conséquence \\
La constance \\
La constitution \\
La contemplation \\
La contingence \\
La contingence du futur \\
La continuité \\
La contradiction \\
La contradiction réside-t-elle dans les choses ? \\
La contrainte \\
La conversation \\
La conversion \\
La conviction \\
La coopération \\
La corruption \\
La couleur \\
La coutume \\
La crainte des Dieux \\
La crainte et l'ignorance \\
La création de l'humanité \\
La créativité \\
La critique \\
La croissance \\
La croyance est-elle l'asile de l'ignorance ? \\
La cruauté \\
L'acteur \\
L'actualité \\
L'actuel \\
La culpabilité \\
La culture générale \\
La curiosité \\
La curiosité est-elle à l'origine du savoir ? \\
La danse \\
La décadence \\
La décence \\
La déception \\
La déduction \\
La déficience \\
La définition \\
La délibération \\
La démagogie \\
La démence \\
La démesure \\
La démocratie et les experts \\
La dépendance \\
La déraison \\
La désillusion \\
La désinvolture \\
La désobéissance civile \\
La dialectique \\
La dialectique est-elle une science ? \\
La différence \\
La différence des arts \\
La différence des sexes \\
La différence sexuelle \\
La difformité \\
La dignité \\
La dignité humaine \\
La digression \\
La discipline \\
La discrétion \\
La discussion \\
La distance \\
La distinction \\
La distraction \\
La diversion \\
La diversité des cultures \\
La diversité des langues \\
La diversité des perceptions \\
La diversité des sciences \\
La division du travail \\
L'admiration \\
La domination du corps \\
La domination sociale \\
La douleur \\
La droiture \\
La faiblesse de la démocratie \\
La famille \\
La fatalité \\
La fatigue \\
La faute \\
La fête \\
La fiction \\
La fidélité \\
La finalité \\
La fin de l'histoire \\
La fin du monde \\
La finitude \\
La folie \\
La fonction de penser peut-elle se déléguer ? \\
La force \\
La force de la loi \\
La force de l'habitude \\
La force de l'idée \\
La force des idées \\
La formation du goût \\
La formation d'une conscience \\
La fortune \\
La foule \\
La fragilité \\
La fraude \\
La frivolité \\
La frontière \\
La futilité \\
L'âge d'or \\
La générosité \\
La genèse \\
La géométrie \\
La grâce \\
La grammaire \\
La grammaire contraint-elle notre pensée ? \\
La grandeur \\
La grandeur d'âme \\
La gratuité \\
L'agressivité \\
La guérison \\
La guerre et la paix \\
La haine de la pensée \\
La haine de la raison \\
La haine de soi \\
La hiérarchie \\
La honte \\
La jalousie \\
La jeunesse \\
La joie \\
La jouissance \\
La jurisprudence \\
La juste mesure \\
La juste peine \\
La justice a-t-elle besoin des institutions ? \\
La justice de l'État \\
La justice divine \\
La justice peut-elle se passer d'institutions ? \\
La justification \\
La laideur \\
La langue maternelle \\
La leçon des choses \\
La lecture \\
La légende \\
La légitimation \\
La lettre et l'esprit \\
La liberté de parole \\
La liberté se prouve-t-elle ? \\
L'aliénation \\
La limite \\
La logique est-elle un art de raisonner ? \\
La logique pourrait-elle nous surprendre ? \\
La loi du genre \\
La loi du marché \\
La loi peut-elle changer les mœurs ? \\
La lumière de la vérité \\
La lumière naturelle \\
La machine \\
La magie \\
La main \\
La main et l'outil \\
La maîtrise \\
La maîtrise de la langue \\
La maîtrise de soi \\
La maîtrise du temps \\
La majesté \\
La majorité \\
La maladie \\
La malchance \\
La manière \\
L'amateurisme \\
La matière, est-ce l'informe ? \\
La matière n'est-elle qu'une idée ? \\
La matière sensible \\
La maturité \\
La mauvaise foi \\
La mauvaise volonté \\
L'ambiguïté \\
La méchanceté \\
La médecine est-elle une science ? \\
L'âme des bêtes \\
La médiation \\
L'âme est-elle immortelle ? \\
La mélancolie \\
La mémoire \\
La mémoire sélective \\
La mesure \\
La mesure des choses \\
La métaphore \\
La méthode \\
La minorité \\
La misère \\
La misologie \\
L'amitié \\
L'amitié est-elle une vertu ? \\
La mode \\
La modernité \\
La monnaie \\
La morale de l'athée \\
La morale doit-elle en appeler à la nature ? \\
La morale est-elle ennemie du bonheur ? \\
La morale est-elle fondée sur la liberté ? \\
La moralité des lois \\
La mort dans l'âme \\
La mort fait-elle partie de la vie ? \\
L'amour de la liberté \\
L'amour de l'art \\
L'amour de soi \\
L'amour et la haine \\
L'amour et l'amitié \\
L'amour et la mort \\
L'amour peut-il être absolu ? \\
L'amour-propre \\
L'amour vrai \\
La multitude \\
La musique est-elle un langage ? \\
La naissance \\
La naïveté \\
L'analogie \\
L'analyse \\
L'analyse du vécu \\
L'anarchie \\
La nation est-elle dépassée ? \\
La nature a-t-elle une histoire ? \\
La nature des choses \\
La nature est-elle muette ? \\
La nature est-elle sacrée ? \\
La nature est-elle sauvage ? \\
La nature morte \\
L'anecdotique \\
La nécessité \\
La nécessité des contradictions \\
La nécessité des signes \\
La négation \\
La négligence est-elle une faute ? \\
La neutralité \\
Langage et communication \\
Langage et réalité \\
L'angoisse \\
Langue et parole \\
L'animalité \\
L'animal nous apprend-il quelque chose sur l'homme ? \\
La noblesse \\
L'anomalie \\
L'anonymat \\
L'anormal \\
La normalité \\
La norme et le fait \\
La nostalgie \\
La notion de barbarie a-t-elle un sens ? \\
La notion de loi a-t-elle une unité ? \\
La notion de point de vue \\
La nouveauté \\
L'anthropocentrisme \\
L'anticipation \\
La nuance \\
La nudité \\
La paix \\
La paix est-elle moins naturelle que la guerre ? \\
La paix n'est-elle que l'absence de conflit ? \\
La parenté \\
La paresse \\
La parole \\
La participation \\
La passion de la vérité \\
La passion de l'égalité \\
La passion du juste \\
La paternité \\
L'apathie \\
La pauvreté \\
La peine capitale \\
La pensée a-t-elle une histoire ? \\
La pensée est-elle en lutte avec le langage ? \\
La pensée peut-elle s'écrire ? \\
La perfectibilité \\
La perfection \\
La perfection morale \\
La personnalité \\
La perspective \\
La perversité \\
La peur de la mort \\
La peur de la nature \\
La peur de l'autre \\
La philanthropie \\
La philosophie doit-elle se préoccuper du salut ? \\
La philosophie peut-elle disparaître ? \\
La philosophie peut-elle se passer de théologie ? \\
La philosophie première \\
La physique et la chimie \\
La pitié \\
La place d'autrui \\
La plénitude \\
La pluralité \\
La pluralité des arts \\
La pluralité des langues \\
La poésie et l'idée \\
La poésie pense-t-elle ? \\
La polémique \\
La politesse \\
La possibilité \\
L'apparence \\
L'appel \\
L'appropriation \\
L'approximation \\
La pratique de l'espace \\
La précaution peut-elle être un principe ? \\
La précision \\
La première fois \\
La première vérité \\
La présence \\
La présence d'esprit \\
La présence du passé \\
La présomption \\
La preuve \\
La prévision \\
L'\emph{a priori} \\
La prison \\
La prison est-elle utile ? \\
La privation \\
La profondeur \\
La promenade \\
La promesse \\
La promesse et le contrat \\
La proposition \\
La propriété \\
La protection \\
La providence \\
La prudence \\
La publicité \\
La pudeur \\
La puissance \\
La puissance de la technique \\
La puissance de l'État \\
La pulsion \\
La punition \\
La pureté \\
La qualité \\
La question : « qui ? » \\
La radicalité \\
La raison a-t-elle le droit d'expliquer ce que morale condamne ? \\
La raison a-t-elle une histoire ? \\
La raison d'état \\
La raison d'État \\
La raison du plus fort \\
La raison est-elle suffisante ? \\
La rationalité du langage \\
L'arbitraire \\
La réalité de l'idéal \\
La réalité de l'idée \\
La réalité du futur \\
La réalité du possible \\
La réalité peut-elle être virtuelle ? \\
La recherche de l'absolu \\
La recherche de la vérité \\
La recherche des origines \\
La recherche du bonheur \\
La réciprocité \\
La reconnaissance \\
La rectitude \\
La référence \\
La réflexion \\
La réflexion sur l'expérience participe-t-elle de l'expérience ? \\
La réforme \\
La réfutation \\
La règle et l'exception \\
La relation \\
La religion peut-elle suppléer la raison ? \\
La réminiscence \\
La renaissance \\
La rencontre \\
La réparation \\
La répétition \\
La représentation \\
La reproduction \\
La réputation \\
La résilience \\
La résolution \\
La responsabilité \\
La responsabilité collective \\
La ressemblance \\
La révélation \\
La rêverie \\
La révolte peut-elle être un droit ? \\
La révolution \\
L'argent \\
L'argumentation \\
L'argument d'autorité \\
La rhétorique \\
La rhétorique a-t-elle une valeur ? \\
La richesse \\
La richesse du sensible \\
La richesse intérieure \\
La rigueur \\
La rigueur de la loi \\
L'art contre la beauté ? \\
L'art est-il affaire de goût ? \\
L'art est-il une critique de la culture ? \\
L'art et la manière \\
L'art et le divin \\
L'artifice \\
L'artificiel \\
L'art n'est-il qu'une question de sentiment ? \\
L'art peut-il changer le monde \\
L'art peut-il n'être pas conceptuel ? \\
L'art pour l'art \\
L'art sait-il montrer ce que le langage ne peut pas dire ? \\
La ruine \\
La rumeur \\
La rupture \\
La ruse \\
La ruse technique \\
La sagesse et l'expérience \\
La sagesse rend-elle heureux ? \\
La santé \\
L'ascèse \\
La science a-t-elle des limites ? \\
La science commence-t-elle avec la perception ? \\
La science et le mythe \\
La science nous éloigne-t-elle des choses ? \\
La science pense-t-elle ? \\
La science peut-elle se passer de fondement ? \\
La science peut-elle tout expliquer ? \\
La sculpture \\
La sécurité \\
La séduction \\
La sensibilité \\
La séparation \\
La séparation des pouvoirs \\
La sérénité \\
La sévérité \\
La sexualité \\
La signification \\
La simplicité \\
La sincérité \\
La singularité \\
La situation \\
La sobriété \\
La société des nations \\
La solitude \\
La souffrance a-t-elle un sens ? \\
La souffrance d'autrui \\
La souffrance morale \\
La souveraineté \\
La souveraineté du peuple \\
La spéculation \\
La spontanéité \\
L'association \\
L'association des idées \\
La superstition \\
La survie \\
La sympathie \\
La sympathie peut-elle tenir lieu de moralité ? \\
La table rase \\
La technique fait-elle des miracles ? \\
La technique peut-elle améliorer l'homme ? \\
La technocratie \\
La téléologie \\
La tendance \\
La tentation \\
La terre \\
La Terre et le Ciel \\
La terreur \\
L'athéisme \\
La théologie rationnelle \\
La tolérance a-t-elle des limites ? \\
L'atome \\
La totalité \\
La toute-puissance \\
La toute puissance de la pensée \\
La trace \\
La trace et l'indice \\
La tradition \\
La traduction \\
La tranquillité \\
La transcendance \\
La tristesse \\
L'attachement \\
L'attente \\
L'attention \\
La tyrannie \\
La tyrannie du bonheur \\
L'audace \\
L'au-delà \\
L'auteur et le créateur \\
L'authenticité \\
L'autobiographie \\
L'autonomie \\
L'autoportrait \\
L'autorité \\
L'autorité de la parole \\
L'autorité de l'écrit \\
L'autorité morale \\
L'autre monde \\
La valeur d'échange \\
La valeur des choses \\
La valeur du temps \\
La valeur du travail \\
La vanité \\
La vanité est-elle toujours sans objet ? \\
L'avant-garde \\
L'avarice \\
La variété \\
La vengeance \\
L'avenir \\
L'avenir de l'humanité \\
L'avenir est-il imaginable ? \\
L'avenir existe-t-il ? \\
L'aventure \\
La vérification \\
La vérité a-t-elle une histoire ? \\
La vérité de la religion \\
La vérité demande-t-elle du courage ? \\
La vérité des images \\
La vérité scientifique est-elle relative ? \\
La vertu \\
La vertu du citoyen \\
La vertu peut-elle être excessive ? \\
L'aveu \\
L'aveuglement \\
La vie active \\
La vie de la langue \\
La vie de l'esprit \\
La vie du droit \\
La vie intérieure \\
La vie ordinaire \\
La vie peut-elle être éternelle ? \\
La vie psychique \\
La vie quotidienne \\
La ville \\
La ville et la campagne \\
La violence \\
La violence a-t-elle des degrés ? \\
La violence verbale \\
La virtualité \\
La vocation \\
La voix \\
La voix de la conscience \\
La voix du peuple \\
La volonté du peuple \\
La volonté générale \\
La volonté peut-elle être indéterminée ? \\
La volupté \\
La vraisemblance \\
La vulgarité \\
La vulnérabilité \\
Le barbare \\
Le beau a-t-il une histoire ? \\
Le beau est-il aimable ? \\
Le beau et l'agréable \\
Le beau naturel \\
Le besoin d'absolu \\
Le besoin de philosophie \\
Le besoin de vérité ? \\
Le bien commun \\
Le bon goût \\
Le bonheur dans le mal \\
Le bonheur de la passion est-il sans lendemain ? \\
Le bonheur est-il une valeur morale ? \\
Le bon sens \\
Le bon usage des passions \\
Le bruit \\
Le cadavre \\
Le caractère \\
L'écart \\
Le cas de conscience \\
Le changement \\
Le chant \\
Le chaos \\
Le charme et la grâce \\
Le chemin \\
Le choix d'un destin \\
Le choix peut-il être éclairé ? \\
Le ciel et la terre \\
Le cinéma est-il un art comme les autres ? \\
Le classicisme \\
Le cœur \\
Le comique et le tragique \\
Le commencement \\
Le commerce des hommes \\
Le commun \\
Le compromis \\
Le concept \\
Le conformisme \\
Le conseil \\
Le consensus \\
Le contingent \\
Le contrat \\
Le corps dit-il quelque chose ? \\
Le corps est-il porteur de valeurs ? \\
Le corps et la machine \\
Le corps et l'âme \\
Le corps humain \\
Le corps politique \\
Le corps propre \\
Le cosmopolitisme \\
Le cosmopolitisme peut-il être réaliste ? \\
Le courage \\
Le cours du temps \\
Le cri \\
Le critère \\
L'écrit et l'oral \\
L'écriture des lois \\
L'écriture et la parole \\
Le cynisme \\
Le danger \\
Le débat \\
Le défaut \\
Le déguisement \\
Le dernier mot \\
Le désaccord \\
Le désespoir \\
Le déshonneur \\
Le désir de gloire \\
Le désir de pouvoir \\
Le désir de savoir \\
Le désir d'éternité \\
Le désir de vérité \\
Le désir est-il sans limite ? \\
Le désir et la loi \\
Le désir et le manque \\
Le désir n'est-il qu'inquiétude ? \\
Le désœuvrement \\
Le désordre \\
Le désordre des choses \\
Le dessin et la couleur \\
Le destin \\
Le désuet \\
Le détachement \\
Le détail \\
Le deuil \\
Le devenir \\
Le dialogue des philosophes \\
Le dire et le faire \\
Le discernement \\
Le discontinu \\
Le divers \\
Le divertissement \\
Le divin \\
Le dogmatisme \\
Le don de soi \\
Le don et l'échange \\
Le donné \\
Le double \\
Le doute est-il une faiblesse de la pensée ? \\
Le drame \\
Le droit au travail \\
Le droit de la guerre \\
Le droit de punir \\
Le droit des animaux \\
Le droit de veto \\
Le droit de vie et de mort \\
Le droit d'intervention \\
Le droit du plus faible \\
Le droit international \\
Le droit peut-il être naturel ? \\
Le dualisme \\
L'éducation des esprits \\
L'éducation peut-elle être sentimentale ? \\
L'éducation physique \\
Le fait de vivre est-il un bien en soi ? \\
Le fait divers \\
Le fait et le droit \\
Le fanatisme \\
Le faux et le fictif \\
Le féminin et le masculin \\
Le finalisme \\
Le flegme \\
Le fond et la forme \\
Le formalisme \\
Le fragment \\
Le futur est-il contingent ? \\
L'égalité \\
L'égalité des chances \\
Légalité et légitimité \\
Le génie \\
Le génie du lieu \\
Le génie du mal \\
Le genre humain \\
Le geste \\
Le geste et la parole \\
Le goût \\
Le goût de la polémique \\
Le goût des autres \\
Le goût du pouvoir \\
Le hasard \\
Le hasard fait-il bien les choses ? \\
Le haut et le bas \\
Le je ne sais quoi \\
Le jeu \\
Le joli, le beau \\
Le jugement dernier \\
Le jugement de valeur est-il indifférent à la vérité ? \\
Le juste et le bien \\
Le langage de la pensée \\
Le langage est-il d'essence poétique ? \\
L'élégance \\
Le libre-arbitre \\
Le lien social \\
Le lieu \\
Le lieu commun \\
L'éloge de la démesure \\
Le loisir \\
Le luxe \\
Le malentendu \\
Le malheur \\
Le malin plaisir \\
L'émancipation \\
Le manque de culture \\
Le marché \\
Le masque \\
Le mauvais goût \\
L'embarras du choix \\
Le meilleur \\
Le meilleur régime politique \\
Le même et l'autre \\
Le mensonge de l'art ? \\
Le mensonge en politique \\
Le mépris peut-il être justifié ? \\
Le mérite \\
Le métier \\
Le métier d'homme \\
Le mien et le tien \\
Le milieu \\
Le miracle \\
Le miroir \\
Le moindre mal \\
Le monde à l'envers \\
Le monde de l'animal \\
Le monde de la vie \\
Le monde des machines \\
Le monde des œuvres \\
Le monde des physiciens \\
Le monde des rêves \\
Le monde des sens \\
Le monde du rêve \\
Le monde est-il éternel ? \\
Le monstre \\
Le monstrueux \\
Le moralisme \\
Le mot et la chose \\
L'émotion \\
Le mot juste \\
Le mouvement \\
Le mouvement de la pensée \\
Le musée \\
Le mystère \\
Le mysticisme \\
Le naïf \\
Le narcissisme \\
Le naturel \\
Le naturel et l'artificiel \\
L'encyclopédie \\
Le néant \\
Le négatif \\
L'énergie \\
L'enfance \\
L'enfance de l'art \\
L'enfance est-elle ce qui doit être surmonté ? \\
L'engagement \\
Le nihilisme \\
L'ennemi \\
L'ennui \\
Le noble et le vil \\
Le nomade \\
Le nom propre \\
L'enthousiasme \\
Le nu \\
Le oui-dire \\
Le pacifisme \\
Le paradoxe \\
Le pardon \\
Le pardon et l'oubli \\
Le partage \\
Le partage des biens \\
Le partage des savoirs \\
Le passage à l'acte \\
Le paysage \\
Le pays natal \\
Le pédagogue \\
Le pessimisme \\
Le phantasme \\
L'éphémère \\
Le phénomène \\
Le philanthrope \\
Le plaisir \\
Le plaisir de l'art \\
Le plaisir et le bien \\
Le pluralisme \\
Le poids du passé \\
Le point de vue \\
Le populaire \\
Le portrait \\
Le possible et le réel \\
Le pour et le contre \\
Le pouvoir des images \\
Le pouvoir des mots \\
Le pouvoir politique est-il nécessairement coercitif ? \\
Le préjugé \\
Le présent \\
Le principe de raison \\
Le principe de réalité \\
Le privilège de l'original \\
Le probable \\
Le processus \\
Le prochain \\
Le proche et le lointain \\
Le profane \\
Le progrès \\
Le progrès technique \\
Le projet \\
Le projet d'une paix perpétuelle est-il insensé ? \\
Le propre \\
Le public et le privé \\
Le pur et l'impur \\
L'équilibre des pouvoirs \\
L'équité \\
L'équivalence \\
L'équivoque \\
Le quotidien \\
Le raffinement \\
Le rationnel et le raisonnable \\
Le réalisme \\
Le récit \\
Le réel est-il ce qui résiste ? \\
Le refus \\
Le regard \\
Le regard du photographe \\
Le règne de l'homme \\
Le relativisme \\
Le repos \\
Le respect \\
Le ressentiment \\
Le rêve \\
Le rêve et la veille \\
Le rien \\
Le risque \\
Le rôle des institutions \\
L'érotisme \\
L'erreur et la faute \\
L'erreur et l'ignorance \\
L'érudition \\
Le rythme \\
Le sacré \\
Le salut \\
Les amis \\
Les anciens et les modernes \\
Les animaux ont-ils des droits ? \\
Les animaux pensent-ils ? \\
Les apparences font-elles partie du monde ? \\
Les archives \\
Les arts de la mémoire \\
Le sauvage et le barbare \\
Le sauvage et le cultivé \\
Le savoir a-t-il besoin d'être fondé ? \\
Le savoir du peintre \\
Le savoir émancipe-t-il ? \\
Le savoir est-il libérateur ? \\
Le savoir-faire \\
Le savoir se vulgarise-t-il ? \\
Les beautés de la nature \\
Les bénéfices du doute \\
Les blessures de l'esprit \\
Les bons sentiments \\
Les catégories \\
Les causes et les effets \\
Les chemins de traverse \\
Les choses ont-elles une essence ? \\
Les cinq sens \\
Les circonstances \\
L'esclavage \\
L'esclave \\
L'esclave et son maître \\
Les conséquences de l'action \\
Les degrés de conscience \\
Les dictionnaires \\
Les droits de l'enfant \\
Les droits de l'homme \\
Le secret \\
Les effets de l'esclavage \\
Les éléments \\
Le sens commun \\
Le sens de la mesure \\
Le sens de la situation \\
Le sens de l'histoire \\
Le sens de l'Histoire \\
Le sens de l'humour \\
Le sens des mots \\
Le sensible \\
Le sens interne \\
Le sens musical \\
Le sentiment de l'existence \\
Le sentiment de l'injustice \\
Le sentiment esthétique \\
Le sentiment moral \\
Les entités mathématiques sont-elles des fictions ? \\
Les envieux \\
Le sérieux \\
Le serment \\
Les faits parlent-ils d'eux-mêmes ? \\
Les fins de l'éducation \\
Les fins dernières \\
Les forts et les faibles \\
Les frontières \\
Les fruits du travail \\
Les grands hommes \\
Les hasards de la vie \\
Les hommes et les dieux \\
Les idoles \\
Le silence \\
Les images empêchent-elles de penser ? \\
Le simulacre \\
Les inégalités sociales sont-elles inévitables ? \\
Le singulier \\
Le singulier est-il objet de connaissance ? \\
Le singulier et le pluriel \\
Les instruments de la pensée \\
Les intentions et les conséquences \\
Les limites de la description \\
Les limites de la raison \\
Les limites de la tolérance \\
Les limites de la vertu \\
Les limites de l'humain \\
Les limites de l'imagination \\
Les limites du corps \\
Les lois causales \\
Les lois de la guerre \\
Les lois de l'hospitalité \\
Les machines \\
Les maladies de l'âme \\
Les mathématiques sont-elles utiles au philosophe ? \\
Les mœurs \\
Les mœurs et la morale \\
Les mots et les choses \\
Les mots justes \\
Les moyens et la fin \\
Les nombres gouvernent-ils le monde ? \\
Les noms \\
Les noms propres \\
Les normes \\
Les normes du vivant \\
Les normes et les valeurs \\
Le sommeil et la veille \\
Les opérations de la pensée \\
Le souverain bien \\
L'espace et le lieu \\
L'espace et le territoire \\
Les paroles et les actes \\
Les parties de l'âme \\
L'espèce et l'individu \\
Le spectacle \\
Le spectacle de la nature \\
Le spectacle de la pensée \\
Les plaisirs \\
Les pouvoirs de la religion \\
L'esprit critique \\
L'esprit de finesse \\
L'esprit de système \\
L'esprit est-il matériel ? \\
L'esprit et la machine \\
L'esprit peut-il être malade ? \\
L'esprit peut-il être mesuré ? \\
L'esprit scientifique \\
Les proverbes \\
Les proverbes enseignent-ils quelque chose ? \\
L'esquisse \\
Les raisons de vivre \\
Les règles de l'art \\
Les règles du jeu \\
Les représentants du peuple \\
Les ruines \\
Les sciences humaines sont-elles des sciences ? \\
L'essence \\
Les sens peuvent-ils nous tromper ? \\
Les sentiments \\
Les sentiments peuvent-ils s'apprendre ? \\
Les signes de l'intelligence \\
Les systèmes \\
L'estime de soi \\
Le style \\
Le sublime \\
Le succès \\
Le sujet \\
Le sujet de l'action \\
Le sujet et l'objet \\
Les vertus \\
Les visages du mal \\
Les vivants et les morts \\
Le symbole \\
Le talent \\
L'État de droit \\
L'état de la nature \\
L'état d'exception \\
L'État doit-il éduquer les citoyens ? \\
L'État et les Églises \\
Le témoignage \\
Le temps est-il une dimension de la nature ? \\
Le temps ne fait-il que passer ? \\
Le temps perdu \\
L'éternité n'est-elle qu'une illusion ? \\
Le territoire \\
Le théâtre du monde \\
L'éthique à l'épreuve du tragique \\
L'ethnocentrisme \\
L'étonnement \\
Le toucher \\
Le tout est-il la somme de ses parties ? \\
Le trait d'esprit \\
L'étranger \\
L'étrangeté \\
Le travail rapproche-t-il les hommes ? \\
L'être de la conscience \\
L'être du possible \\
L'être et le bien \\
L'être se confond-il avec l'être perçu ? \\
Le tribunal de l'histoire \\
Le vécu \\
L'événement \\
Le verbalisme \\
Le verbe \\
Le vide \\
L'évidence \\
Le virtuel \\
Le visage \\
Le visible et l'invisible \\
L'évolution \\
Le vrai a-t-il une histoire ? \\
Le vrai et le vraisemblable \\
Le vrai peut-il rester invérifiable ? \\
Le vraisemblable \\
Le vulgaire \\
L'exactitude \\
L'excellence \\
L'exception \\
L'excès et le défaut \\
L'exclusion \\
L'excuse \\
L'exemplaire \\
L'exemplarité \\
L'exemple \\
L'exercice de la vertu \\
L'exigence de vérité a-t-elle un sens moral ? \\
L'exil \\
L'expérience \\
L'expérience directe est-elle une connaissance ? \\
L'expérience enseigne-elle quelque chose ? \\
L'expérience et l'expérimentation \\
L'expérimentation \\
L'expérimentation sur le vivant \\
L'exposition \\
L'expression \\
L'extériorité \\
L'habileté \\
L'habitation \\
L'habitude \\
L'harmonie \\
L'héritage \\
L'hésitation \\
L'histoire a-t-elle un sens ? \\
L'histoire de l'art \\
L'histoire des sciences \\
L'histoire peut-elle se répéter ? \\
L'histoire universelle est-elle l'histoire des guerres ? \\
L'homme a-t-il une nature ? \\
L'homme des sciences humaines \\
L'homme d'État \\
L'homme est-il la mesure de toutes choses ? \\
L'homme est-il prisonnier du temps ? \\
L'homme et la bête \\
L'homme et la machine \\
L'homme et la nature sont-ils commensurables ? \\
L'homme peut-il changer ? \\
L'honneur \\
L'horizon \\
L'horreur \\
L'horrible \\
L'hospitalité \\
L'hospitalité est-elle un devoir ? \\
L'humiliation \\
L'humilité \\
L'humour \\
L'humour et l'ironie \\
L'hypocrisie \\
L'hypothèse \\
Liberté et libération \\
Liberté et nécessité \\
L'idéalisme \\
L'idéaliste \\
L'idéalité \\
L'idée d'anthropologie \\
L'idée de beaux arts \\
L'idée de création \\
L'idée de crise \\
L'idée de Dieu \\
L'idée de langue universelle \\
L'idée de logique \\
L'idée d'encyclopédie \\
L'idée de perfection \\
L'idée de substance \\
L'idée d'exactitude \\
L'idée d'une langue universelle \\
L'idée d'une science bien faite \\
L'identité \\
L'identité et la différence \\
L'identité personnelle \\
L'idéologie \\
L'idolâtrie \\
L'ignoble \\
L'ignorance nous excuse-t-elle ? \\
L'illimité \\
L'illusion \\
L'image \\
L'imagination nous éloigne-t-elle du réel ? \\
L'imitation \\
L'immédiat \\
L'immensité \\
L'immutabilité \\
L'impardonnable \\
L'impartialité \\
L'impensable \\
L'impératif \\
L'imperceptible \\
L'implicite \\
L'importance des détails \\
L'impossible \\
L'imposteur \\
L'imprescriptible \\
L'impression \\
L'imprévisible \\
L'improbable \\
L'improvisation \\
L'imprudence \\
L'impuissance \\
L'impunité \\
L'inachevé \\
L'inaction \\
L'inapparent \\
L'incarnation \\
L'incertitude est-elle dans les choses ou dans les idées ? \\
L'incommensurabilité \\
L'inconcevable \\
L'inconnu \\
L'inconscience \\
L'inconscient \\
L'inconséquence \\
L'incrédulité \\
L'inculture \\
L'indécidable \\
L'indécision \\
L'indéfini \\
L'indépassable \\
L'indétermination \\
L'indéterminé \\
L'indicible \\
L'indifférence \\
L'indifférence à la politique \\
L'individu \\
L'individualisme \\
L'induction \\
L'inégalité des chances \\
L'inégalité entre les hommes \\
L'inégalité naturelle \\
L'inertie \\
L'infâme \\
L'infamie \\
L'infini \\
L'influence \\
L'information \\
L'informe \\
L'ingratitude \\
L'inhumain \\
L'inimaginable \\
L'inimitié \\
L'inintelligible \\
L'initiation \\
L'injustice \\
L'innocence \\
L'innommable \\
L'inobservable \\
L'inquiétant \\
L'inquiétude \\
L'insensé \\
L'insignifiant \\
L'insouciance \\
L'insoumission \\
L'insoutenable \\
L'inspiration \\
L'instant \\
L'instinct \\
L'institution \\
L'insulte \\
L'intellectuel \\
L'intelligence \\
L'intelligence de la main \\
L'intelligence de la matière \\
L'intelligence des bêtes \\
L'intelligence du sensible \\
L'intelligence du vivant \\
L'intelligible \\
L'intemporel \\
L'intention \\
L'intentionnalité \\
L'interdit \\
L'intérêt \\
L'intérêt bien compris \\
L'intérieur et l'extérieur \\
L'intériorité \\
L'interprétation est-elle un art ? \\
L'intime conviction \\
L'intimité \\
L'intolérable \\
L'intolérance \\
L'intraduisible \\
L'intransigeance \\
L'intuition \\
L'inutile \\
L'invention \\
L'invention de soi \\
L'invisibilité \\
Lire et écrire \\
L'ironie \\
L'irrationnel \\
L'irréfutable \\
L'irrégularité \\
L'irréparable \\
L'irrésolution \\
L'irresponsabilité \\
L'irréversible \\
L'irrévocable \\
L'ivresse \\
L'obéissance \\
L'objectivité \\
L'objet \\
L'objet d'amour \\
L'objet de l'amour \\
L'objet de la réflexion \\
L'objet de l'art \\
L'obligation \\
L'obscénité \\
L'obscurité \\
L'observation \\
L'obsession \\
L'obstacle \\
L'occasion \\
L'œil et l'oreille \\
Logique et vérité \\
L'oisiveté \\
L'ombre et la lumière \\
L'opinion droite \\
L'opinion publique \\
L'opinion vraie \\
L'opposition \\
L'ordinaire est-il ennuyeux ? \\
L'ordre \\
L'ordre des choses \\
L'ordre du monde \\
L'ordre du temps \\
L'ordre établi \\
L'ordre social \\
L'organique et le mécanique \\
L'organisation \\
L'orgueil \\
L'orientation \\
L'originalité \\
L'origine \\
L'origine de la culpabilité \\
L'origine de la négation \\
L'origine des langues \\
L'origine des langues est-elle un faux problème ? \\
L'origine et le fondement \\
L'ornement \\
L'oubli \\
L'oubli est-il un échec de la mémoire ? \\
L'outil \\
L'un est le multiple \\
L'un et le multiple \\
L'unité \\
L'unité dans le beau \\
L'unité de l'art \\
L'unité des sciences \\
L'unité des sciences humaines \\
L'universel \\
L'universel et le particulier \\
L'urbanité \\
L'urgence \\
L'usage \\
L'usage des fictions \\
L'usage des généalogies \\
L'usage des mots \\
L'usage des passions \\
L'usage du monde \\
L'utile et l'agréable \\
L'utilité de la poésie \\
L'utilité des préjugés \\
L'utopie \\
Maître et serviteur \\
Maîtriser l'absence \\
Manger \\
Manquer de jugement \\
Masculin, féminin \\
Matière et corps \\
Matière et matériaux \\
Ma vraie nature \\
Mémoire et fiction \\
Ménager les apparences \\
Mentir \\
Métier et vocation \\
Mettre en ordre \\
Microscope et télescope \\
Misère et pauvreté \\
Moi d'abord \\
Mon corps \\
Mon corps est-il ma propriété ? \\
Mon corps m'appartient-il ? \\
Morale et prudence \\
Mourir \\
Mythe et histoire \\
Mythe et philosophie \\
Naître \\
Nature et histoire \\
Nature et institutions \\
Naviguer \\
Ne pas raconter d'histoires \\
Ne pas savoir ce que l'on fait \\
Ne penser à rien \\
Ne prêche-t-on que les convertis ? \\
N'exprime t-on que ce dont on a conscience ? \\
Nier le monde \\
Nier l'évidence \\
Ni regrets, ni remords \\
Nomade et sédentaire \\
Nommer \\
Notre besoin de fictions \\
N'y a t-il de bonheur que dans l'instant ? \\
N'y a-t-il de science que du général ? \\
N'y a-t-il de sens que par le langage ? \\
N'y a-t-il de vérité que scientifique ? \\
N'y a-t-il qu'une substance ? \\
Obéir, est-ce se soumettre ? \\
Observer \\
Ordre et désordre \\
Ordre et liberté \\
Où commence la servitude ? \\
Où est le passé ? \\
Où est le pouvoir ? \\
Où est-on quand on pense ? \\
Où s'arrête l'espace public ? \\
Où suis-je quand je pense ? \\
Où suis-je ? \\
Pardonner et oublier \\
Parfaire \\
Parier \\
Parler, est-ce communiquer ? \\
Parler pour ne rien dire \\
Par où commencer ? \\
Par quoi un individu se distingue-t-il d'un autre ? \\
Partager les richesses \\
Passer du fait au droit \\
Peindre d'après nature \\
Peinture et histoire \\
Pensée et réalité \\
Penser, est-ce calculer ? \\
Penser et calculer \\
Penser et parler \\
Penser la technique \\
Penser le réel \\
Penser par soi-même \\
Penser requiert-il un corps ? \\
Perception et aperception \\
Perception et jugement \\
Perception et mouvement \\
Percevoir, est-ce connaître ? \\
Percevoir et imaginer \\
Percevoir et sentir \\
Perdre la mémoire \\
Perdre ses habitudes \\
Perdre ses illusions \\
Perdre son âme \\
Persuader \\
Persuader et convaincre \\
Peut-il y avoir de bons tyrans ? \\
Peut-il y avoir une philosophie applicable ? \\
Peut-on aimer les animaux ? \\
Peut-on aimer l'humanité ? \\
Peut-on apprendre à vivre ? \\
Peut-on avoir raison tout seul ? \\
Peut-on changer de logique ? \\
Peut-on changer le passé ? \\
Peut-on comparer deux philosophies ? \\
Peut-on connaître autrui ? \\
Peut-on considérer l'art comme un langage ? \\
Peut-on décider de croire ? \\
Peut-on dire d'une image qu'elle parle ? \\
Peut-on dire toute la vérité ? \\
Peut-on éclairer la liberté ? \\
Peut-on en savoir trop ? \\
Peut-on être en conflit avec soi-même ? \\
Peut-on être heureux tout seul ? \\
Peut-on être hors de soi ? \\
Peut-on être plus ou moins libre ? \\
Peut-on être sans opinion ? \\
Peut-on être trop sage ? \\
Peut-on faire l'économie de la notion de forme ? \\
Peut-on faire l'inventaire du monde ? \\
Peut-on fixer des limites à la science ? \\
Peut-on fonder les mathématiques ? \\
Peut-on fonder une morale sur la nature ? \\
Peut-on jamais aimer son prochain ? \\
Peut-on justifier le mensonge ? \\
Peut-on ne pas être matérialiste ? \\
Peut-on ne pas savoir ce que l'on fait ? \\
Peut-on ne rien vouloir ? \\
Peut-on parler de corruption des mœurs ? \\
Peut-on parler de droits des animaux ? \\
Peut-on parler d'un droit de la guerre ? \\
Peut-on parler d'un travail intellectuel ? \\
Peut-on penser illogiquement ? \\
Peut-on penser la douleur ? \\
Peut-on penser la mort ? \\
Peut-on penser l'irrationnel ? \\
Peut-on penser sans concepts ? \\
Peut-on penser sans concept ? \\
Peut-on penser sans règles ? \\
Peut-on percevoir sans s'en apercevoir ? \\
Peut-on perdre la raison ? \\
Peut-on perdre sa liberté ? \\
Peut-on perdre son identité ? \\
Peut-on recommencer sa vie ? \\
Peut-on représenter l'espace ? \\
Peut-on rester insensible à la beauté ? \\
Peut-on rire de tout ? \\
Peut-on savoir ce qui est bien ? \\
Peut-on se faire une idée de tout ? \\
Peut-on se passer d'un maître ? \\
Peut-on se retirer du monde ? \\
Peut-on tout démontrer ? \\
Peut-on tout dire ? \\
Peut-on tout mesurer ? \\
Peut-on tout soumettre à la discussion ? \\
Peut-on trouver du plaisir à l'ennui ? \\
Peut-on vivre avec les autres ? \\
Peut-on vivre dans le doute ? \\
Peut-on vivre sans aucune certitude ? \\
Peut-on vivre sans ressentiment ? \\
Peut-on vouloir le mal ? \\
Philosopher, est-ce apprendre à vivre ? \\
Philosophie et mathématiques \\
Philosophie et métaphysique \\
Plaider \\
Poésie et philosophie \\
Pourquoi a-t-on peur de la folie ? \\
Pourquoi avoir recours à la notion d'inconscient ? \\
Pourquoi chercher un sens à l'histoire ? \\
Pourquoi commémorer ? \\
Pourquoi croyons-nous ? \\
Pourquoi des artifices ? \\
Pourquoi des artistes ? \\
Pourquoi des cérémonies ? \\
Pourquoi des classifications ? \\
Pourquoi des conflits ? \\
Pourquoi des exemples ? \\
Pourquoi des historiens ? \\
Pourquoi des hypothèses ? \\
Pourquoi des métaphores ? \\
Pourquoi des modèles ? \\
Pourquoi des musées ? \\
Pourquoi des poètes ? \\
Pourquoi des rites ? \\
Pourquoi dire la vérité ? \\
Pourquoi donner ? \\
Pourquoi écrire ? \\
Pourquoi est-il difficile de rectifier une erreur ? \\
Pourquoi être exigeant ? \\
Pourquoi être moral ? \\
Pourquoi exiger la cohérence \\
Pourquoi faire de l'histoire ? \\
Pourquoi fait-on le mal ? \\
Pourquoi faudrait-il être cohérent ? \\
Pourquoi lire des romans ? \\
Pourquoi nous souvenons-nous ? \\
Pourquoi parlons-nous ? \\
Pourquoi pensons-nous ? \\
Pourquoi préférer l'original à sa reproduction ? \\
Pourquoi punir ? \\
Pourquoi sauver les phénomènes ? \\
Pourquoi se mettre à la place d'autrui ? \\
Pourquoi sommes-nous déçus par les œuvres d'un faussaire ? \\
Pourquoi sommes-nous moraux ? \\
Pourquoi un droit du travail ? \\
Pourquoi y a-t-il des conflits insolubles ? \\
Pourquoi y a-t-il plusieurs philosophies ? \\
Pouvoir et puissance \\
Pouvoirs et libertés \\
Prendre des risques \\
Prendre son temps \\
Prendre une décision \\
Présence et représentation \\
Principe et commencement \\
Prose et poésie \\
Prouver \\
Providence et destin \\
Publier \\
Puis-je être sûr de bien agir ? \\
Puis-je être universel ? \\
Puis-je ne rien croire ? \\
Pulsion et instinct \\
Pulsions et passions \\
Qualités premières, qualités secondes \\
Quand agit-on ? \\
Quand la guerre finira-t-elle ? \\
Quand pense-t-on ? \\
Quand suis-je en faute ? \\
Quand y a-t-il œuvre ? \\
Qu'a perdu le fou ? \\
Qu'apprend-on dans les livres ? \\
Qu'apprenons-nous de nos affects ? \\
Qu'a-t-on le droit d'interpréter ? \\
Qu'avons-nous à apprendre des historiens ? \\
Que cherchons-nous dans le regard des autres ? \\
Que connaissons-nous du vivant ? \\
Que désirons-nous ? \\
Que dit la loi ? \\
Que doit-on faire de ses rêves ? \\
Que faut-il craindre ? \\
Quel est le rôle de la créativité dans les sciences ? \\
Quel est le rôle du médecin ? \\
Quel est l'être de l'illusion ? \\
Quel est l'objet de la science ? \\
Quel est l'objet de l'échange ? \\
Quelle est la réalité de la matière ? \\
Quelle idée le fanatique se fait-il de la vérité ? \\
Quelles actions permettent d'être heureux ? \\
Quel rôle l'imagination joue-t-elle en mathématiques ? \\
Que montre l'image ? \\
Que ne peut-on pas expliquer ? \\
Que nous apprend le plaisir ? \\
Que nous apprend le toucher ? \\
Que nous apprennent les controverses scientifiques ? \\
Que nous apprennent les langues étrangères ? \\
Que nous montrent les natures mortes ? \\
Que peindre ? \\
Que peint le peintre ? \\
Que perd la pensée en perdant l'écriture ? \\
Que peut la force ? \\
Que peut l'art ? \\
Que peut-on calculer ? \\
Que peut-on comprendre qu'on ne puisse expliquer ? \\
Que peut-on démontrer ? \\
Que peut-on partager ? \\
Que peut un corps ? \\
Que pouvons-nous aujourd'hui apprendre des sciences d'autrefois ? \\
Que sais-je de ma souffrance ? \\
Que serait le meilleur des mondes ? \\
Que serait un art total ? \\
Que signifie apprendre ? \\
Qu'est-ce qu'avoir conscience de soi ? \\
Qu'est-ce qu'avoir de l'expérience ? \\
Qu'est-ce qu'avoir du style ? \\
Qu'est-ce que démontrer ? \\
Qu'est-ce que déraisonner ? \\
Qu'est-ce que Dieu pour un athée ? \\
Qu'est-ce que discuter ? \\
Qu'est-ce que faire autorité ? \\
Qu'est-ce que faire preuve d'humanité ? \\
Qu'est-ce que gouverner ? \\
Qu'est-ce que guérir ? \\
Qu'est-ce que juger ? \\
Qu'est-ce que la culture générale \\
Qu'est-ce que le désordre ? \\
Qu'est-ce que le moi ? \\
Qu'est-ce que le naturalisme ? \\
Qu'est-ce que l'enfance ? \\
Qu'est-ce que l'harmonie ? \\
Qu'est-ce que lire ? \\
Qu'est-ce que méditer ? \\
Qu'est-ce qu'enquêter ? \\
Qu'est-ce que parler ? \\
Qu'est-ce que perdre son temps ? \\
Qu'est-ce que prendre conscience ? \\
Qu'est-ce que raisonner ? \\
Qu'est-ce que résoudre une contradiction ? \\
Qu'est-ce que réussir sa vie ? \\
Qu'est-ce que traduire ? \\
Qu'est-ce qu'être chez soi ? \\
Qu'est-ce qu'être malade ? \\
Qu'est-ce qu'être sceptique ? \\
Qu'est-ce qu'être seul ? \\
Qu'est-ce qu'être vivant ? \\
Qu'est-ce que un individu \\
Qu'est-ce qu'habiter ? \\
Qu'est-ce qui agit ? \\
Qu'est-ce qui apparaît ? \\
Qu'est-ce qui dépend de nous ? \\
Qu'est ce qui est concret ? \\
Qu'est ce qui est contre nature ? \\
Qu'est-ce qui est contre nature ? \\
Qu'est-ce qui est donné ? \\
Qu'est-ce qui est impossible ? \\
Qu'est ce qui est irréfutable ? \\
Qu'est-ce qui est mien ? \\
Qu'est-ce qui est moderne ? \\
Qu'est-ce qui est noble ? \\
Qu'est-ce qui est réel ? \\
Qu'est-ce qui est respectable ? \\
Qu'est ce qui est sacré ? \\
Qu'est-ce qui est sublime ? \\
Qu'est-ce qui est vital pour le vivant ? \\
Qu'est ce qui existe ? \\
Qu'est-ce qui existe ? \\
Qu'est-ce qui fait la valeur d'une croyance ? \\
Qu'est-ce qui fait le propre d'un corps propre ? \\
Qu'est-ce qui fait l'humanité d'un corps ? \\
Qu'est-ce qui fait l'unité d'une science ? \\
Qu'est-ce qui fait l'unité d'un peuple ? \\
Qu'est-ce qui fait un peuple ? \\
Qu'est-ce qu'interpréter une œuvre d'art ? \\
Qu'est-ce qu'interpréter ? \\
Qu'est-ce qu'obéir ? \\
Qu'est-ce qu'on attend ? \\
Qu'est-ce qu'un abus de pouvoir ? \\
Qu'est-ce qu'un acteur ? \\
Qu'est-ce qu'un alter ego \\
Qu'est-ce qu'un animal domestique ? \\
Qu'est-ce qu'un animal ? \\
Qu'est-ce qu'un auteur ? \\
Qu'est-ce qu'un axiome ? \\
Qu'est-ce qu'un bon conseil ? \\
Qu'est-ce qu'un cas de conscience ? \\
Qu'est-ce qu'un chef d'œuvre ? \\
Qu'est-ce qu'un chef-d'œuvre ? \\
Qu'est-ce qu'un citoyen ? \\
Qu'est-ce qu'un concept ? \\
Qu'est-ce qu'un contenu de conscience ? \\
Qu'est-ce qu'un contrat ? \\
Qu'est-ce qu'un coup d'État ? \\
Qu'est-ce qu'un crime contre l'humanité ? \\
Qu'est-ce qu'un déni ? \\
Qu'est-ce qu'un dieu ? \\
Qu'est-ce qu'un Dieu ? \\
Qu'est-ce qu'un dogme ? \\
Qu'est-ce qu'une alternative ? \\
Qu'est-ce qu'une aporie ? \\
Qu'est-ce qu'une bonne loi ? \\
Qu'est-ce qu'une bonne méthode ? \\
Qu'est-ce qu'une catégorie ? \\
Qu'est-ce qu'une cause ? \\
Qu'est-ce qu'une chose ? \\
Qu'est-ce qu'une collectivité ? \\
Qu'est-ce qu'une communauté ? \\
Qu'est-ce qu'une conduite irrationnelle ? \\
Qu'est-ce qu'une crise ? \\
Qu'est-ce qu'une découverte ? \\
Qu'est-ce qu'une école philosophique ? \\
Qu'est-ce qu'une expérience cruciale ? \\
Qu'est-ce qu'une expérience de pensée ? \\
Qu'est-ce qu'une famille ? \\
Qu'est-ce qu'une forme ? \\
Qu'est-ce qu'une hypothèse scientifique ? \\
Qu'est-ce qu'une idée incertaine ? \\
Qu'est-ce qu'une idée ? \\
Qu'est-ce qu'une image ? \\
Qu'est-ce qu'une institution ? \\
Qu'est-ce qu'une langue morte ? \\
Qu'est-ce qu'une limite ? \\
Qu'est-ce qu'une loi ? \\
Qu'est-ce qu'une machine ? \\
Qu'est-ce qu'une marchandise ? \\
Qu'est-ce qu'une mauvaise interprétation ? \\
Qu'est-ce qu'une méditation ? \\
Qu'est-ce qu'une méthode ? \\
Qu'est-ce qu'un empire ? \\
Qu'est-ce qu'une nation ? \\
Qu'est-ce qu'un enfant ? \\
Qu'est-ce qu'une norme ? \\
Qu'est-ce qu'une œuvre d'art ? \\
Qu'est-ce qu'une œuvre ? \\
Qu'est-ce qu'une phrase ? \\
Qu'est-ce qu'une promesse ? \\
Qu'est-ce qu'une question dénuée de sens ? \\
Qu'est-ce qu'une question métaphysique ? \\
Qu'est-ce qu'une réfutation ? \\
Qu'est ce qu'une religion ? \\
Qu'est-ce qu'une représentation réussie ? \\
Qu'est-ce qu'une révolution ? \\
Qu'est-ce qu'une situation tragique ? \\
Qu'est-ce qu'un esprit faux ? \\
Qu'est-ce qu'une substance ? \\
Qu'est-ce qu'une tradition ? \\
Qu'est-ce qu'une tragédie historique ? \\
Qu'est-ce qu'un être cultivé ? \\
Qu'est-ce qu'une valeur ? \\
Qu'est-ce qu'un événement historique ? \\
Qu'est-ce qu'un événement ? \\
Qu'est-ce qu'une ville ? \\
Qu'est-ce qu'une vision du monde ? \\
Qu'est-ce qu'un exemple ? \\
Qu'est-ce qu'un fait historique ? \\
Qu'est-ce qu'un faux problème ? \\
Qu'est-ce qu'un faux sentiment ? \\
Qu'est-ce qu'un grand philosophe ? \\
Qu'est-ce qu'un héros ? \\
Qu'est-ce qu'un homme bon ? \\
Qu'est-ce qu'un homme seul ? \\
Qu'est-ce qu'un individu ? \\
Qu'est-ce qu'un jeu ? \\
Qu'est-ce qu'un laboratoire ? \\
Qu'est-ce qu'un législateur ? \\
Qu'est-ce qu'un lieu commun ? \\
Qu'est-ce qu'un livre ? \\
Qu'est-ce qu'un maître ? \\
Qu'est-ce qu'un modèle ? \\
Qu'est-ce qu'un moderne ? \\
Qu'est-ce qu'un monde ? \\
Qu'est-ce qu'un monstre ? \\
Qu'est-ce qu'un monument ? \\
Qu'est-ce qu'un mythe ? \\
Qu'est-ce qu'un nombre ? \\
Qu'est-ce qu'un nom propre ? \\
Qu'est-ce qu'un objet esthétique ? \\
Qu'est-ce qu'un organisme ? \\
Qu'est-ce qu'un original ? \\
Qu'est-ce qu'un outil ? \\
Qu'est-ce qu'un pédant ? \\
Qu'est-ce qu'un peuple \\
Qu'est-ce qu'un peuple libre ? \\
Qu'est-ce qu'un peuple ? \\
Qu'est-ce qu'un phénomène ? \\
Qu'est-ce qu'un plaisir pur ? \\
Qu'est-ce qu'un point de vue ? \\
Qu'est-ce qu'un principe ? \\
Qu'est-ce qu'un problème éthique ? \\
Qu'est-ce qu'un problème philosophique ? \\
Qu'est-ce qu'un problème ? \\
Qu'est-ce qu'un rapport de force ? \\
Qu'est-ce qu'un sage ? \\
Qu'est-ce qu'un signe ? \\
Qu'est-ce qu'un sophisme ? \\
Qu'est-ce qu'un sophiste ? \\
Qu'est-ce qu'un souvenir ? \\
Qu'est-ce qu'un spécialiste ? \\
Qu'est-ce qu'un style ? \\
Qu'est-ce qu'un symptôme ? \\
Qu'est-ce qu'un système philosophique ? \\
Qu'est-ce qu'un système ? \\
Qu'est-ce qu'un tableau \\
Qu'est-ce qu'un témoin ? \\
Qu'est-ce qu'un tout ? \\
Que suppose le mouvement ? \\
Que valent les excuses ? \\
Que valent les idées générales ? \\
Que vaut la distinction entre nature et culture ? \\
Que vaut l'excuse : « C'est plus fort que moi » ? \\
Que vaut l'incertain ? \\
Que vaut une preuve contre un préjugé ? \\
Que veut dire l'expression « aller au fond des choses » ? \\
Que voit-on dans une image ? \\
Que voit-on dans un miroir ? \\
Qui agit ? \\
Qui a le droit de juger ? \\
Qui a une histoire ? \\
Qui connaît le mieux mon corps ? \\
Qui doit faire les lois ? \\
Qui est le maître ? \\
Qui est métaphysicien ? \\
Qui fait la loi ? \\
Qui mérite d'être aimé ? \\
Qui parle ? \\
Qui pense ? \\
Qui peut parler ? \\
Qu'y a-t-il à comprendre en histoire ? \\
Qu'y a-t-il à l'origine de toutes choses ? \\
Qu'y a-t-il au fondement de l'objectivité ? \\
Raconter son histoire \\
Raison et révélation \\
Réalisme et idéalisme \\
Réalité et idéal \\
Recevoir \\
Récit et mémoire \\
Reconnaissons-nous le bien comme nous reconnaissons le vrai ? \\
Réforme et révolution \\
Réfuter \\
Regarder \\
Regarder un tableau \\
Religion naturelle et religion révélée \\
Rendre la justice \\
Renoncer au passé \\
Rentrer en soi-même \\
Répondre \\
Représentation et illusion \\
République et démocratie \\
Résistance et soumission \\
Résister \\
Rêver \\
Rien de nouveau sous le soleil \\
Rire \\
Rire et pleurer \\
Sait-on toujours ce que l'on fait ? \\
Sait-on toujours ce qu'on veut ? \\
S'aliéner \\
Sans foi ni loi \\
S'approprier une œuvre d'art \\
Savoir ce qu'on dit \\
Savoir de quoi on parle \\
Savoir, est-ce pouvoir ? \\
Savoir et liberté \\
Savoir être heureux \\
Savoir, pouvoir \\
Savoir s'arrêter \\
Savoir vivre \\
Savons-nous ce que nous disons ? \\
Science et imagination \\
Science et philosophie \\
Science et société \\
Science et technique \\
Sciences et philosophie \\
Se conserver \\
Se cultiver \\
Se défendre \\
Se faire justice \\
S'ennuyer \\
Sens et fait \\
Sens et sensible \\
Sentir \\
Se parler et s'entendre \\
Se passer de philosophie \\
Se prendre au sérieux \\
Se retirer dans la pensée ? \\
Se retirer du monde \\
Servir \\
Se taire \\
Se voiler la face \\
S'exercer \\
S'exprimer \\
Si Dieu n'existe pas, tout est-il permis ? \\
Signification et expression \\
S'indigner, est-ce un devoir ? \\
S'intéresser à l'art \\
Sommes-nous capables d'agir de manière désintéressée ? \\
Sommes-nous libres de nos croyances ? \\
Sommes-nous libres de nos pensées ? \\
Sommes-nous responsables de ce que nous sommes ? \\
Sophismes et paradoxes \\
S'orienter \\
Sortir de soi \\
Soutenir une thèse \\
Subir \\
Substance et sujet \\
Suffit-il d'être juste ? \\
Suis-je aussi ce que j'aurais pu être ? \\
Suis-je ma mémoire ? \\
Suis-je mon corps ? \\
Sujet et citoyen \\
Sur quoi fonder la propriété ? \\
Sur quoi reposent nos certitudes ? \\
Surveiller son comportement \\
Survivre \\
Suspendre son assentiment \\
Suspendre son jugement \\
Tantôt je pense, tantôt je suis \\
Témoigner \\
Temps et réalité \\
Tenir parole \\
Tous les droits sont-ils formels ? \\
Tous les hommes désirent-ils connaître ? \\
Tous les hommes désirent-ils être heureux ? \\
Tout a-t-il un sens ? \\
Tout définir, tout démontrer \\
Toute chose a-t-elle une essence ? \\
Toute expression est-elle métaphorique ? \\
Toute origine est-elle mythique ? \\
Toute peur est-elle irrationnelle ? \\
Toute philosophie est-elle systématique ? \\
Tout est corps \\
Tout est-il connaissable ? \\
Tout est-il mesurable ? \\
Tout est-il nécessaire ? \\
Toute violence est-elle contre nature ? \\
Tout ou rien \\
Tout peut-il n'être qu'apparence ? \\
Tout savoir \\
Tout savoir est-il transmissible ? \\
Tradition et innovation \\
Tradition et raison \\
Traduire \\
Trahir \\
Transcendance et altérité \\
Travail et subjectivité \\
Travail manuel, travail intellectuel \\
Tuer le temps \\
Un acte désintéressé est-il possible ? \\
Une éducation esthétique est-elle possible ? \\
Une explication peut-elle être réductrice ? \\
Une fiction peut-elle être vraie ? \\
Une loi n'est-elle qu'une règle ? \\
Une machine peut-elle penser ? \\
Une machine pourrait-elle penser ? \\
Une morale du plaisir est-elle concevable ? \\
Une perception peut-elle être illusoire ? \\
Une philosophie de l'amour est-elle possible ? \\
Une religion civile est-elle possible ? \\
Une religion peut-elle être rationnelle ? \\
Une science des symboles est-elle possible ? \\
Une théorie scientifique peut-elle devenir fausse ? \\
Une volonté peut-elle être générale ? \\
Un homme n'est-il que la somme de ses actes ? \\
Un moment d'éternité \\
Un monde sans beauté \\
Un monde sans nature est-il pensable ? \\
Un pouvoir a-t-il besoin d'être légitime ? \\
Vanité des vanités \\
Vérité et fiction \\
Vérité et sensibilité \\
Vérité et signification \\
Vérités mathématiques, vérités philosophiques \\
Vie et existence \\
Vieillir \\
Violence et discours \\
Violence et politique \\
Vit-on au présent ? \\
Vivons-nous tous dans le même monde ? \\
Vivre au présent \\
Vivre comme si nous ne devions pas mourir \\
Vivre sa vie \\
Vivre sous la conduite de la raison \\
Voir \\
Voir et entendre \\
Voir et toucher \\
Vouloir ce que l'on peut \\
Voyager \\
Vulgariser la science ? \\
Y a-t-il continuité entre l'expérience et la science ? \\
Y a-t-il de fausses religions ? \\
Y a-t-il de l'inconcevable ? \\
Y a-t-il des actions désintéressées ? \\
Y a-t-il des arts mineurs ? \\
Y a-t-il des canons de la beauté ? \\
Y a-t-il des certitudes historiques ? \\
Y a-t-il des choses qui échappent au droit ? \\
Y a-t-il des croyances démocratiques ? \\
Y a-t-il des degrés dans la certitude ? \\
Y a-t-il des degrés de réalité ? \\
Y a-t-il des démonstrations en philosophie ? \\
Y a-t-il des devoirs envers soi-même ? \\
Y a-t-il des faits sans essence ? \\
Y a-t-il des fins de la nature ? \\
Y a-t-il des guerres justes ? \\
Y a-t-il des héritages philosophiques ? \\
Y a-t-il des leçons de l'histoire ? \\
Y a-t-il des limites à l'exprimable ? \\
Y a-t-il des limites au droit ? \\
Y a-t-il des lois non écrites ? \\
Y a-t-il des pensées folles ? \\
Y a-t-il des pensées inconscientes ? \\
Y a-t-il des preuves d'amour ? \\
Y a-t-il des secrets de la nature ? \\
Y a-t-il des substances incorporelles ? \\
Y a-t-il des violences justifiées ? \\
Y a-t-il des violences légitimes ? \\
Y a-t-il du sacré dans la nature ? \\
Y a-t-il plusieurs manières de définir ? \\
Y a-t-il un autre monde ? \\
Y a-t-il un besoin métaphysique ? \\
Y a-t-il un bien commun ? \\
Y a-t-il un canon de la beauté ? \\
Y a-t-il un droit de résistance ? \\
Y a-t-il un droit international ? \\
Y a-t-il une connaissance du singulier ? \\
Y a-t-il une connaissance sensible ? \\
Y a-t-il une éthique de l'authenticité ? \\
Y a-t-il une expérience de la liberté ? \\
Y a-t-il une expérience de l'éternité ? \\
Y a-t-il une expérience du néant ? \\
Y a-t-il une fin dernière ? \\
Y a-t-il une ou plusieurs philosophies ? \\
Y a-t-il une philosophie de la philosophie ? \\
Y a-t-il une philosophie première ? \\
Y a-t-il une science de l'être ? \\
Y a-t-il une vérité des symboles ? \\
Y a-t-il une vérité du sentiment ? \\
Y a-t-il une vérité philosophique ? \\
Y a-t-il une vie de l'esprit ? \\
Y a-t-il un langage commun ? \\
Y a-t-il un langage du corps ? \\
Y a-t-il un mal absolu ? \\
Y a-t-il un monde extérieur ? \\
Y a-t-il un ordre des choses ? \\
Y a-t-il un progrès moral ? \\
Y a-t-il un savoir du corps ? \\
Y a-t-il un savoir immédiat ? \\
Y a-t-il un temps des choses ? \\
« Aimez vos ennemis » \\
« Après moi, le déluge » \\
« À l'impossible, nul n'est tenu » \\
« C'est humain » \\
« C'est la vie » \\
« De la musique avant toute chose » \\
« Deviens qui tu es » \\
« Dieu est mort » \\
« Être contre » \\
« Je n'ai pas voulu cela » \\
« La vie est un songe » \\
« L'histoire jugera » \\
« L'homme est la mesure de toute chose » \\
« Malheur aux vaincus » \\
« Petites causes, grands effets » \\
« Prendre ses désirs pour des réalités » \\
« Que nul n'entre ici s'il n'est géomètre » \\
« Rien n'est sans raison » \\
« Rien n'est simple » \\
« Sauver les phénomènes » \\
« Toute peine mérite salaire » \\
« Trop beau pour être vrai » \\
« Tu ne tueras point » \\


\subsection{Esthétique}
\label{sec-3-2}

\noindent
Art et authenticité \\
Art et divertissement \\
Art et émotion \\
Art et folie \\
Art et forme \\
Art et illusion \\
Art et image \\
Art et interdit \\
Art et jeu \\
Art et marchandise \\
Art et mélancolie \\
Art et mémoire \\
Art et métaphysique \\
Art et politique \\
Art et propagande \\
Art et religieux \\
Art et religion \\
Art et représentation \\
Art et technique \\
Art et transgression \\
Art et vérité \\
Arts de l'espace et arts du temps \\
Avoir du goût \\
Avons-nous besoin d'experts en matière d'art ? \\
Beauté et vérité \\
Beauté naturelle et beauté artistique \\
Beauté réelle, beauté idéale \\
Certaines œuvres d'art ont-elles plus de valeur que d'autres ? \\
Cinéma et réalité \\
Comment devient-on artiste ? \\
Comment juger d'une œuvre d'art ? \\
Comment reconnaît-on une œuvre d'art ? \\
Composition et construction \\
Contemplation et distraction \\
Contempler \\
Créativité et contrainte \\
De quoi l'expérience esthétique est-elle l'expérience ? \\
Donner une représentation \\
Écouter \\
En quel sens une œuvre d'art est-elle un document ? \\
En quoi l'œuvre d'art donne-t-elle à penser ? \\
Enseigner l'art \\
Esthétique et poétique \\
Être acteur \\
Expérience esthétique et sens commun \\
Expression et création \\
Faut-il distinguer esthétique et philosophie de l'art ? \\
Faut-il opposer l'art à la connaissance ? \\
Forme et rythme \\
Imitation et création \\
Imiter, est-ce copier ? \\
Interpréter une œuvre d'art \\
Invention et création \\
Jugement esthétique et jugement de valeur \\
La beauté a-t-elle une histoire ? \\
La beauté des corps \\
La beauté est-elle dans le regard ou dans la chose vue ? \\
La beauté est-elle partout ? \\
La beauté est-elle sensible ? \\
La beauté est-elle une promesse de bonheur ? \\
La beauté et la grâce \\
La beauté idéale \\
La beauté naturelle \\
L'absence d'œuvre \\
L'abstraction \\
L'abstraction en art \\
L'académisme \\
La catharsis \\
La censure \\
L'achèvement de l'œuvre \\
La classification des arts \\
La composition \\
La conscience de soi de l'art \\
La couleur \\
La création artistique \\
La création dans l'art \\
La critique d'art \\
L'acteur et son rôle \\
La culture artistique \\
La culture est-elle nécessaire à l'appréciation d'une œuvre d'art ? \\
La danse est-elle l'œuvre du corps ? \\
La fiction \\
La figuration \\
La fin de l'art \\
La fonction de l'art \\
La force de l'art \\
La genèse de l'œuvre \\
La hiérarchie des arts \\
La laideur \\
La liberté créatrice \\
La liberté de l'artiste \\
L'allégorie \\
L'amateur \\
La métaphore \\
La mode \\
La modernité dans les arts \\
La monumentalité \\
La mort de l'art \\
L'amour de l'art \\
La musique a-t-elle une essence ? \\
La musique de film \\
La musique et le bruit \\
La nature est-elle artiste ? \\
La nature morte \\
La nature peut-elle être belle ? \\
La nouveauté en art \\
La peinture est-elle une poésie muette ? \\
La peinture peut-elle être un art du temps ? \\
La perfection en art \\
La pluralité des arts \\
La poésie \\
L'appréciation de la nature \\
La productivité de l'art \\
La profondeur \\
La puissance des images \\
La question de l'œuvre d'art \\
L'architecte et l'ingénieur \\
L'architecture est-elle un art ? \\
La réalité du beau \\
La réception de l'œuvre d'art \\
La répétition \\
La responsabilité de l'artiste \\
La restauration des œuvres d'art \\
La rhétorique est-elle un art ? \\
L'art apprend-il à percevoir ? \\
L'art a-t-il des vertus thérapeutiques ? \\
L'art a-t-il plus de valeur que la vérité ? \\
L'art a-t-il une fin morale ? \\
L'art a-t-il une histoire ? \\
L'art a-t-il une valeur sociale ? \\
L'art d'écrire \\
L'art de masse \\
L'art des images \\
L'art de vivre est-il un art ? \\
L'art doit-il être critique ? \\
L'art doit-il nous étonner ? \\
L'art dramatique \\
L'art du comédien \\
L'art échappe-t-il à la raison ? \\
L'art engagé \\
L'art est-il affaire d'imagination ? \\
L'art est-il à lui-même son propre but ? \\
L'art est-il destiné à embellir ? \\
L'art est-il subversif ? \\
L'art est-il une expérience de la liberté ? \\
L'art est-il un langage ? \\
L'art est-il un mode de connaissance ? \\
L'art est-il un modèle pour la philosophie ? \\
L'art et la nature \\
L'art et la tradition \\
L'art et la vie \\
L'art et le mouvement \\
L'art et l'éphémère \\
L'art et le rêve \\
L'art et le sacré \\
L'art et les arts \\
L'art et l'immoralité \\
L'art et morale \\
L'art fait-il penser ? \\
L'art imite-t-il la nature ? \\
L'artiste a-t-il besoin d'une idée de l'art ? \\
L'artiste a-t-il besoin d'un public ? \\
L'artiste a-t-il une méthode ? \\
L'artiste est-il le mieux placé pour comprendre son œuvre ? \\
L'artiste est-il maître de son œuvre ? \\
L'artiste et l'artisan \\
L'artiste exprime-t-il quelque chose ? \\
L'artiste peut-il se passer d'un maître ? \\
L'artiste sait-il ce qu'il fait ? \\
L'art modifie-t-il notre rapport au réel ? \\
L'art n'est-il pas toujours politique ? \\
L'art n'est-il pas toujours religieux ? \\
L'art ou les arts \\
L'art peut-il encore imiter la nature ? \\
L'art peut-il être utile ? \\
L'art peut-il nous rendre meilleurs ? \\
L'art peut-il prétendre à la vérité ? \\
L'art peut-il quelque chose contre la morale ? \\
L'art peut-il quelque chose pour la morale ? \\
L'art peut-il rendre le mouvement ? \\
L'art peut-il s'affranchir des lois ? \\
L'art peut-il s'enseigner ? \\
L'art peut-il se passer d'idéal ? \\
L'art pour l'art \\
L'art produit-il nécessairement des œuvres ? \\
L'art s'adresse-t-il à la sensibilité ? \\
L'art s'apparente-t-il à la philosophie ? \\
L'art : expérience, exercice ou habitude ? \\
L'art : une arithmétique sensible ? \\
La sacralisation de l'œuvre \\
La scène théâtrale \\
La signification dans l'œuvre \\
La signification en musique \\
L'aspiration esthétique \\
La temporalité de l'œuvre d'art \\
L'attrait du beau \\
L'authenticité de l'œuvre d'art \\
L'autonomie de l'art \\
L'autonomie de l'œuvre d'art \\
L'autoportrait \\
L'avant-garde \\
La vérité de la fiction \\
La vérité du roman \\
La virtuosité \\
Le baroque \\
Le beau est-il une valeur commune ? \\
Le beau et le bien \\
Le beau et le sublime \\
Le bon goût \\
Le cinéma, art de la représentation ? \\
Le cinéma est-il un art ou une industrie ? \\
Le cinéma est-il un art populaire ? \\
Le cinéma est-il un art ? \\
Le corps dansant \\
Le critique d'art \\
Le design \\
Le désintéressement esthétique \\
Le désir d'originalité \\
Le dieu artiste \\
L'éducation artistique \\
L'éducation du goût \\
L'éducation esthétique \\
Le fantastique \\
Le faux en art \\
Le formalisme \\
Le frivole \\
Le geste \\
Le geste créateur \\
Le goût du beau \\
Le goût est-il une faculté ? \\
Le goût est-il une vertu sociale ? \\
Le goût se forme-t-il ? \\
Le goût : certitude ou conviction ? \\
Le jugement de goût \\
Le jugement de goût est-il universel ? \\
Le langage de l'art \\
Le lyrisme \\
Le maniérisme \\
Le marché de l'art \\
Le mauvais goût \\
Le mode d'existence de l'œuvre d'art \\
Le monde de l'art \\
L'émotion esthétique peut-elle se communiquer ? \\
Le musée \\
L'enfance de l'art \\
L'engagement dans l'art \\
Le patrimoine artistique \\
Le paysage \\
Le plaisir d'imiter \\
Le plaisir esthétique \\
Le plaisir esthétique suppose-t-il une culture ? \\
Le poète réinvente-t-il la langue ? \\
Le point de vue de l'auteur \\
Le portrait \\
Le pouvoir des images \\
Le primitivisme en art \\
Le propre de la musique \\
Le public \\
Lequel, de l'art ou du réel, est-il une imitation de l'autre ? \\
Le réalisme \\
Le rythme \\
Les arts appliqués \\
Les arts communiquent-ils entre eux ? \\
Les arts industriels \\
Les arts mineurs \\
Les arts nobles \\
Les arts ont-ils besoin de théorie ? \\
Les arts populaires \\
Les arts vivants \\
Les beaux-arts sont-ils compatibles entre eux ? \\
Les degrés de la beauté \\
Le sentiment esthétique \\
Les fins de l'art \\
Les fonctions de l'image \\
Les frontières de l'art \\
Les genres esthétiques \\
Les institutions artistiques \\
Les intentions de l'artiste \\
Les lois de l'art \\
Les moyens et les fins en art \\
Les normes esthétiques \\
Les nouvelles technologies transforment-elles l'idée de l'art ? \\
Les œuvres d'art ont-elles besoin d'un commentaire ? \\
Les poètes et la cité \\
Les qualités esthétiques \\
L'esquisse \\
Les règles de l'art \\
Les reproductions \\
Les révolutions techniques suscitent-elles des révolutions dans l'art ? \\
Les techniques artistiques \\
L'esthète \\
L'esthétique est-elle une métaphysique de l'art ? \\
L'esthétisme \\
Le style \\
Le sublime \\
Les usages de l'art \\
Le symbolisme \\
Le système des beaux-arts \\
Le tableau \\
Le talent et le génie \\
Le tragique \\
Le travail artistique \\
L'exécution d'une œuvre d'art est-elle toujours une œuvre d'art ? \\
L'expérience artistique \\
L'expert et l'amateur \\
L'exposition de l'œuvre d'art \\
L'expression \\
L'expression artistique \\
L'expressivité musicale \\
L'harmonie \\
L'hétéronomie de l'art \\
L'histoire de l'art \\
L'histoire de l'art est-elle celle des styles ? \\
L'histoire de l'art est-elle finie ? \\
L'histoire des arts est-elle liée à l'histoire des techniques ? \\
L'hybridation des arts \\
L'idéal de l'art \\
L'idée esthétique \\
L'illustration \\
L'imaginaire et le réel \\
L'imagination dans l'art \\
L'imagination esthétique \\
L'imitation \\
L'immortalité des œuvres d'art \\
L'improvisation \\
L'improvisation dans l'art \\
L'inconscient de l'art \\
L'industrie culturelle \\
L'industrie du beau \\
L'inesthétique \\
L'informe et le difforme \\
L'inspiration \\
L'intériorité de l'œuvre \\
L'interprétation des œuvres \\
L'irreprésentable \\
Littérature et réalité \\
L'objectivité de l'art \\
L'objet de la littérature \\
L'objet de l'art \\
L'œuvre anonyme \\
L'œuvre d'art est-elle l'expression d'une idée ? \\
L'œuvre d'art est-elle toujours destinée à un public ? \\
L'œuvre d'art est-elle une belle apparence ? \\
L'œuvre d'art et sa reproduction \\
L'œuvre d'art et son auteur \\
L'œuvre d'art nous apprend-elle quelque chose ? \\
L'œuvre d'art représente-t-elle quelque chose ? \\
L'œuvre d'art totale \\
L'œuvre de fiction \\
L'œuvre et le produit \\
L'œuvre inachevée \\
L'original et la copie \\
L'originalité en art \\
L'origine de l'art \\
L'ornement \\
L'unité de l'œuvre d'art \\
L'utilité de l'art \\
Musique et bruit \\
Œuvre et événement \\
Par-delà beauté et laideur \\
Peindre \\
Peinture et réalité \\
Peut-on dire de l'art qu'il donne un monde en partage ? \\
Peut-on établir une hiérarchie des arts ? \\
Peut-on être insensible à l'art ? \\
Peut-on expliquer une œuvre d'art ? \\
Peut-on faire de l'art avec tout ? \\
Peut-on hiérarchiser les œuvres d'art ? \\
Peut-on juger des œuvres d'art sans recourir à l'idée de beauté ? \\
Peut-on parler d'art primitif ? \\
Peut-on parler des œuvres d'art ? \\
Peut-on parler d'une science de l'art ? \\
Peut-on parler d'un savoir poétique ? \\
Peut-on penser un art sans œuvres ? \\
Peut-on réunir des arts différents dans une même œuvre ? \\
Peut-on vivre sans art ? \\
Poésie et vérité \\
Point de vue du créateur et point de vue du spectateur \\
Pourquoi conserver les œuvres d'art ? \\
Pourquoi des musées ? \\
Pourquoi des œuvres d'art ? \\
Pourquoi la musique intéresse-t-elle le philosophe ? \\
Pourquoi l'art intéresse-t-il les philosophes ? \\
Pourquoi les œuvres d'art résistent-elles au temps ? \\
Pourquoi s'inspirer de l'art antique ? \\
Production et création \\
Propriétés artistiques, propriétés esthétiques \\
Quand la technique devient-elle art ? \\
Quand y a-t-il paysage ? \\
Qu'appelle-t-on chef-d'œuvre ? \\
Que crée l'artiste ? \\
Quel est le pouvoir de l'art ? \\
Quel est l'objet de l'esthétique ? \\
Quelle est la matière de l'œuvre d'art ? \\
Quelles règles la technique dicte-t-elle à l'art ? \\
Quel réel pour l'art ? \\
Que montre un tableau ? \\
Que nous apporte l'art ? \\
Que nous apprend l'histoire de l'art ? \\
Que peint le peintre ? \\
Qu'est-ce qu'avoir du goût ? \\
Qu'est-ce que l'art contemporain ? \\
Qu'est-ce qui est beau ? \\
Qu'est-ce qui est spectaculaire ? \\
Qu'est-ce qui fait la valeur de l'œuvre d'art ? \\
Qu'est-ce qu'interpréter une œuvre d'art ? \\
Qu'est-ce qu'un artiste ? \\
Qu'est-ce qu'un art moral ? \\
Qu'est-ce qu'une exposition ? \\
Qu'est-ce qu'une idée esthétique ? \\
Qu'est-ce qu'une œuvre d'art authentique ? \\
Qu'est-ce qu'une œuvre ratée ? \\
Qu'est-ce qu'une œuvre « géniale » ? \\
Qu'est-ce qu'une « performance » ? \\
Qu'est-ce qu'un film ? \\
Qu'est-ce qu'un geste artistique ? \\
Qu'est-ce qu'un objet d'art ? \\
Qu'est-ce qu'un produit culturel ? \\
Qu'est-ce qu'un spectateur ? \\
Qu'est-ce qu'un style ? \\
Qu'est-ce qu'un « champ artistique » ? \\
Qu'exprime une œuvre d'art ? \\
Qu'y a-t-il à comprendre dans une œuvre d'art ? \\
Rebuts et objets quelconques : une matière pour l'art ? \\
Regarder \\
Rendre visible l'invisible \\
Reproduire, copier, imiter \\
Sens et sensibilité \\
Thème et variations \\
Tout art est-il poésie ? \\
Tout peut-il être objet de jugement esthétique ? \\
Un art sans sublimation est-il possible ? \\
Une œuvre d'art doit-elle avoir un sens ? \\
Une œuvre d'art est-elle une marchandise ? \\
Une œuvre d'art peut-elle être laide ? \\
Une œuvre d'art s'explique-t-elle à partir de ses influences ? \\
Une œuvre est-elle toujours de son temps ? \\
Un jugement de goût est-il culturel ? \\
Un tableau peut-il être une dénonciation ? \\
Y a-t-il des arts mineurs ? \\
Y a-t-il des critères du beau ? \\
Y a-t-il des révolutions en art ? \\
Y a-t-il un beau idéal ? \\
Y a-t-il une beauté naturelle ? \\
Y a-t-il une correspondance des arts ? \\
Y a-t-il une sensibilité esthétique ? \\
Y a-t-il un progrès en art ? \\
« De la musique avant toute chose » \\
« Il faudrait rester des années entières pour contempler une telle œuvre » \\
« La vie des formes » \\
« Sans titre » \\


\subsection{Logique et épistémologie}
\label{sec-3-3}

\noindent
Analyse et synthèse \\
À quelles conditions une démarche est-elle scientifique ? \\
À quelles conditions une hypothèse est-elle scientifique ? \\
À quelles conditions un énoncé est-il doué de sens ? \\
À quoi la logique peut-elle servir dans les sciences ? \\
À quoi reconnaît-on la vérité ? \\
À quoi reconnaît-on qu'une théorie est scientifique ? \\
À quoi servent les sciences ? \\
À quoi tient la vérité d'une interprétation ? \\
À quelles conditions une explication est-elle scientifique ? \\
À quelles conditions une hypothèse est-elle scientifique ? \\
À quoi sert la logique ? \\
Calculer et penser \\
Catégories logiques et catégories linguistiques \\
Ce qui est faux est-il dénué de sens ? \\
Classer \\
Comment justifier l'autonomie des sciences de la vie ? \\
Comment peut-on choisir entre différentes hypothèses ? \\
Concevoir et juger \\
Connaissance commune et connaissance scientifique \\
Connaissance, croyance, conjecture \\
Connaissance du futur et connaissance du passé \\
Connaissance et croyance \\
Connaître, est-ce connaître par les causes ? \\
Connaître et comprendre \\
Contradiction et opposition \\
Convention et observation \\
Croire et savoir \\
Croyance et probabilité \\
Découverte et invention \\
Découverte et invention dans les sciences \\
Découvrir \\
Décrire \\
Déduction et expérience \\
Définir la vérité, est-ce la connaître ? \\
Définition et démonstration \\
Définition nominale et définition réelle \\
Définitions, axiomes, postulats \\
Démonstration et argumentation \\
Démonstration et déduction \\
De quelle certitude la science est-elle capable ? \\
De quoi la logique est-elle la science ? \\
Des événements aléatoires peuvent-ils obéir à des lois ? \\
D'où vient la certitude dans les sciences ? \\
Épistémologie générale et épistémologie des sciences particulières \\
Erreur et illusion \\
Espace mathématique et espace physique \\
Est-ce par son objet ou par ses méthodes qu'une science peut se définir ? \\
Est-il vrai qu'en science, « rien n'est donné, tout est construit » ? \\
Évidence et certitude \\
Expérience et approximation \\
Expérience et expérimentation \\
Explication et prévision \\
Expliquer \\
Expliquer et comprendre \\
Expliquer et interpréter \\
Extension et compréhension \\
Fonction et prédicat \\
Forger des hypothèses \\
Formaliser et axiomatiser \\
Identité et indiscernabilité \\
Interpréter et expliquer \\
Intuition et concept \\
Intuition et déduction \\
Jugement analytique et jugement synthétique \\
Juger et raisonner \\
Justifier et prouver \\
L'abstraction \\
L'abstrait est-il en dehors de l'espace et du temps ? \\
L'abstrait et le concret \\
La causalité \\
La causalité en histoire \\
La causalité suppose-t-elle des lois ? \\
La certitude \\
La classification des sciences \\
La cohérence est-elle un critère de la vérité ? \\
La communauté scientifique \\
La connaissance adéquate \\
La connaissance commune est-elle le point de départ de la science ? \\
La connaissance de la vie \\
La connaissance des causes \\
La connaissance des principes \\
La connaissance du futur \\
La connaissance du singulier \\
La connaissance du vivant \\
La connaissance est-elle une croyance justifiée ? \\
La connaissance objective \\
La connaissance scientifique abolit-elle toute croyance ? \\
La connaissance scientifique n'est-elle qu'une croyance argumentée ? \\
La contingence des lois de la nature \\
La contradiction \\
La culture scientifique \\
La déduction \\
La définition \\
La démonstration \\
La dialectique \\
La grammaire et la logique \\
La hiérarchie des énoncés scientifiques \\
L'aléatoire \\
La liberté de la science \\
La limite \\
La logique a-t-elle une histoire ? \\
La logique a-t-elle un intérêt philosophique ? \\
La logique décrit-elle le monde ? \\
La logique est-elle indépendante de la psychologie ? \\
La logique est-elle un art de penser ? \\
La logique est-elle une discipline normative ? \\
La logique est-elle une forme de calcul ? \\
La logique est-elle une science de la vérité ? \\
La logique est-elle utile à la métaphysique ? \\
La logique et le réel \\
La logique nous apprend-elle quelque chose sur le langage ordinaire ? \\
La logique peut-elle se passer de la métaphysique ? \\
La logique : découverte ou invention ? \\
La maîtrise de la nature \\
La mathématique est-elle une ontologie ? \\
La matière \\
La mesure \\
La mesure du temps \\
La méthode \\
La méthode de la science \\
La modalité \\
La naissance de la science \\
L'analogie \\
La nature est-elle écrite en langage mathématique ? \\
La nature et le monde \\
La nature parle-t-elle le langage des mathématiques ? \\
La nécessité historique \\
La négation \\
Langage ordinaire et langage de la science \\
La notion de possible \\
La notion d'évolution \\
La notion physique de force \\
La pensée formelle peut-elle avoir un contenu ? \\
La pertinence \\
La place du hasard dans la science \\
La place du sujet dans la science \\
La pluralité des sciences de la nature \\
La politique scientifique \\
La possibilité logique \\
L'approximation \\
La pratique des sciences met-elle à l'abri des préjugés ? \\
La preuve \\
L'\emph{a priori} \\
La probabilité \\
La proposition \\
La psychologie est-elle une science ? \\
La réalité a-t-elle une forme logique ? \\
La réalité décrite par la science s'oppose-t-elle à la démonstration ? \\
La recherche de la vérité \\
La recherche scientifique est-elle désintéressée ? \\
La réfutation \\
La relation de cause à effet \\
La relation de nécessité \\
La science admet-elle des degrés de croyance ? \\
La science a-t-elle besoin du principe de causalité ? \\
La science a-t-elle le monopole de la vérité ? \\
La science a-t-elle une histoire ? \\
La science commence-t-elle avec la perception ? \\
La science découvre-t-elle ou construit-elle son objet ? \\
La science de l'individuel \\
La science dévoile-t-elle le réel ? \\
La science doit-elle se fonder sur une idée de la nature ? \\
La science doit-elle se passer de l'idée de finalité ? \\
La science est-elle indépendante de toute métaphysique ? \\
La science est-elle une langue bien faite ? \\
La science et les sciences \\
La science et l'irrationnel \\
La science nous indique-t-elle ce que nous devons faire ? \\
La science peut-elle lutter contre les préjugés ? \\
La science peut-elle se passer de métaphysique ? \\
La science peut-elle se passer d'hypothèses ? \\
La science peut-elle se passer d'institutions ? \\
La science porte-elle au scepticisme ? \\
La science procède-t-elle par rectification ? \\
La somme et le tout \\
La technique n'est-elle qu'une application de la science ? \\
La théorie et l'expérience \\
L'autonomie du théorique \\
L'autorité de la science \\
La valeur de la science \\
La valeur d'une théorie scientifique se mesure-t-elle à son efficacité ? \\
La validité \\
La vérité admet-elle des degrés ? \\
La vérité du déterminisme \\
La vérité d'une théorie dépend-elle de sa correspondance avec les faits ? \\
Le calcul \\
Le cerveau et la pensée \\
Le concept \\
Le concept de nature est-il un concept scientifique ? \\
Le contingent \\
Le continu \\
Le déterminisme \\
Le doute dans les sciences \\
Le fait scientifique \\
Le faux et l'absurde \\
Le fondement de l'induction \\
Le formalisme \\
Le genre et l'espèce \\
Le hasard existe-t-il ? \\
Le hasard n'est il que la mesure de notre ignorance ? \\
Le jugement \\
Le langage des sciences \\
Le mécanisme et la mécanique \\
Le mouvement \\
Le nécessaire et le contingent \\
Le nombre et la mesure \\
Le non-sens \\
Le normal et le pathologique \\
Le paradigme \\
Le partage des connaissances \\
L'épistémologie est-elle une logique de la science ? \\
Le possible et le probable \\
Le pouvoir de la science \\
Le principe de contradiction \\
Le principe d'identité \\
Le progrès des sciences \\
Le progrès des sciences infirme-t-il les résultats anciens ? \\
Le progrès en logique \\
Le progrès scientifique fait-il disparaître la superstition ? \\
Le raisonnement par l'absurde \\
Le raisonnement scientifique \\
Le raisonnement suit-il des règles ? \\
Le réalisme de la science \\
Le rôle de la théorie dans l'expérience scientifique \\
L'erreur peut-elle jouer un rôle dans la connaissance scientifique ? \\
L'erreur scientifique \\
Les causes et les lois \\
Les changements scientifiques et la réalité \\
Les connaissances scientifiques peuvent-elles être à la fois vraies et provisoires ? \\
Les connaissances scientifiques peuvent-elles être vulgarisées ? \\
Les conquêtes de la science \\
Les ensembles \\
Les fausses sciences \\
Les genres naturels \\
Les jugements analytiques \\
Les limites de la connaissance scientifique \\
Les lois de la nature sont-elles de simples régularités ? \\
Les lois de la nature sont elles nécessaires ? \\
Les lois de l'histoire \\
Les lois scientifiques sont-elles des lois de la nature ? \\
Les mathématiques du mouvement \\
Les mathématiques et la pensée de l'infini \\
Les mathématiques sont-elles réductibles à la logique ? \\
Les mathématiques sont-elles un langage ? \\
Les modalités \\
Les modèles \\
Les mondes possibles \\
Les objets scientifiques \\
Les principes de la démonstration \\
Les principes d'une science sont-ils des conventions ? \\
L'esprit est-il objet de science ? \\
L'esprit scientifique \\
Les relations \\
Les révolutions scientifiques \\
Les sciences décrivent-elles le réel ? \\
Les sciences de la vie et de la Terre \\
Les sciences de la vie visent-elles un objet irréductible à la matière ? \\
Les sciences de l'esprit \\
Les sciences doivent-elle prétendre à l'unification ? \\
Les sciences et le vivant \\
Les sciences exactes \\
Les sciences forment-elle un système ? \\
Les sciences historiques \\
Les sciences humaines peuvent-elles adopter les méthodes des sciences de la nature ? \\
Les sciences naturelles \\
Les sciences sociales \\
Les sciences sociales sont-elles nécessairement inexactes ? \\
Le statut de l'axiome \\
Le statut des hypothèses dans la démarche scientifique \\
Les théories scientifiques sont-elles vraies ? \\
Les vérités scientifiques sont-elles relatives ? \\
Le syllogisme \\
Le symbolisme mathématique \\
Le temps se laisse-t-il décrire logiquement ? \\
Le tiers exclu \\
L'étonnement \\
L'évidence \\
Le vivant comme problème pour la philosophie des sciences \\
Le vrai est-il à lui-même sa propre marque ? \\
Le vrai se réduit-il à l'utile ? \\
L'expérience \\
L'expérience cruciale \\
L'expérience sensible est-elle la seule source légitime de connaissance ? \\
L'expérimentation \\
L'expérimentation en psychologie \\
L'explication scientifique \\
L'histoire des sciences est-elle une histoire ? \\
L'homme est-il objet de science ? \\
L'hypothèse \\
L'idéal de vérité \\
L'idée de connaissance approchée \\
L'idée de continuité \\
L'idée de logique transcendantale \\
L'idée de logique universelle \\
L'idée de loi logique \\
L'idée de loi naturelle \\
L'idée de mathesis universalis \\
L'idée de norme \\
L'idée de science expérimentale \\
L'idée de « sciences exactes » \\
L'identité \\
L'imagination dans les sciences \\
L'indifférence \\
L'induction \\
L'induction et la déduction \\
L'inexactitude et le savoir scientifique \\
L'inférence \\
L'institution scientifique \\
L'instrument scientifique \\
L'intuition \\
L'intuition a-t-elle une place en logique ? \\
L'intuition en mathématiques \\
L'objectivité \\
L'objectivité historique \\
L'obstacle épistémologique \\
Logique et dialectique \\
Logique et existence \\
Logique et logiques \\
Logique et mathématique \\
Logique et mathématiques \\
Logique et métaphysique \\
Logique et méthode \\
Logique et ontologie \\
Logique et psychologie \\
Logique générale et logique transcendantale \\
Lois et règles en logique \\
L'ordre du monde \\
L'ordre et la mesure \\
L'unité de la science \\
L'universel et le particulier \\
L'universel et le singulier \\
Machine et organisme \\
Mathématiques et réalité \\
Mathématiques pures et mathématiques appliquées \\
Mécanisme et finalité \\
Mesurer \\
Montrer et démontrer \\
Notre connaissance du réel se limite-t-elle au savoir scientifique ? \\
N'y a-t-il de rationalité que scientifique ? \\
N'y a-t-il de science qu'autant qu'il s'y trouve de mathématique ? \\
Observation et expérimentation \\
Observer \\
Organisme et milieu \\
Où sont les relations ? \\
Penser est-ce calculer ? \\
Peut-il y avoir science sans intuition du vrai ? \\
Peut-on changer de logique ? \\
Peut-on définir la vérité ? \\
Peut-on définir la vie ? \\
Peut-on dire de la connaissance scientifique qu'elle procède par approximation ? \\
Peut-on dire d'une théorie scientifique qu'elle n'est jamais plus vraie qu'une autre mais seulement plus commode ? \\
Peut-on dire que la science ne nous fait pas connaître les choses mais les rapports entre les choses ? \\
Peut-on dire qu'est vrai ce qui correspond aux faits ? \\
Peut-on dire qu'une théorie physique en contredit une autre ? \\
Peut-on distinguer différents types de causes ? \\
Peut-on penser illogiquement ? \\
Peut-on préconiser, dans les sciences humaines et sociales, l'imitation des sciences de la nature ? \\
Peut-on réduire la pensée à une espèce de comportement ? \\
Peut-on restreindre la logique à la pensée formelle ? \\
Peut-on se passer des relations ? \\
Peut-on tout définir ? \\
Peut-on tout démontrer ? \\
Physique et mathématiques \\
Pourquoi définir ? \\
Pourquoi des géométries ? \\
Pourquoi des logiciens ? \\
Pourquoi est-il difficile de rectifier une erreur ? \\
Pourquoi faut-il être cohérent ? \\
Pourquoi formaliser des arguments ? \\
Pourquoi la raison recourt-elle à l'hypothèse ? \\
Pourquoi les mathématiques s'appliquent-elles à la réalité ? \\
Pourquoi plusieurs sciences ? \\
Prédicats et relations \\
Prémisses et conclusions \\
Prévoir \\
Probabilité et explication scientifique \\
Proposition et jugement \\
Quantification et pensée scientifique \\
Quantité et qualité \\
Que déduire d'une contradiction ? \\
Que disent les tables de vérité ? \\
Quel est le but d'une théorie physique ? \\
Quel est le but du travail scientifique ? \\
Quelle réalité la science décrit-elle ? \\
Quel rôle attribuer à l'intuition \emph{a priori} dans une philosophie des mathématiques ? \\
Quel rôle la logique joue-t-elle en mathématiques ? \\
Quel sens y a-t-il à se demander si les sciences humaines sont vraiment des sciences ? \\
Que nous apprend l'histoire des sciences ? \\
Que peut-on calculer ? \\
Qu'est-ce que calculer ? \\
Qu'est-ce que la psychologie ? \\
Qu'est-ce qui est indiscutable ? \\
Qu'est-ce qui est invérifiable ? \\
Qu'est-ce qu'ignore la science ? \\
Qu'est-ce qui rend vrai un énoncé ? \\
Qu'est-ce qu'un argument ? \\
Qu'est-ce qu'un concept scientifique ? \\
Qu'est-ce qu'une belle démonstration ? \\
Qu'est-ce qu'une catégorie ? \\
Qu'est-ce qu'une conception scientifique du monde ? \\
Qu'est-ce qu'une connaissance non scientifique ? \\
Qu'est-ce qu'une croyance vraie ? \\
Qu'est-ce qu'une découverte scientifique ? \\
Qu'est-ce qu'une discipline savante ? \\
Qu'est-ce qu'une éducation scientifique ? \\
Qu'est-ce qu'une fonction ? \\
Qu'est-ce qu'une hypothèse scientifique ? \\
Qu'est-ce qu'une idée vraie ? \\
Qu'est ce qu'une loi scientifique ? \\
Qu'est-ce qu'une loi scientifique ? \\
Qu'est-ce qu'une preuve ? \\
Qu'est-ce qu'une psychologie scientifique ? \\
Qu'est-ce qu'une révolution scientifique ? \\
Qu'est-ce qu'une science rigoureuse ? \\
Qu'est-ce qu'une vérité scientifique ? \\
Qu'est-ce qu'une vision scientifique du monde ? \\
Qu'est ce qu'un fait scientifique ? \\
Qu'est-ce qu'un modèle ? \\
Qu'est-ce qu'un nombre ? \\
Qu'est ce qu'un paradoxe ? \\
Qu'est-ce qu'un paradoxe ? \\
Qu'est-ce qu'un principe ? \\
Qu'est-ce qu'un problème scientifique ? \\
Question et problème \\
Que vaut une preuve contre un préjugé ? \\
Raisonner et calculer \\
Réfutation et confirmation \\
Sauver les phénomènes \\
Savoir et objectivité dans les sciences \\
Savoir et pouvoir \\
Savoir et rectification \\
Savoir et vérifier \\
Savoir pour prévoir \\
Science et complexité \\
Science et histoire \\
Science et idéologie \\
Science et magie \\
Science et opinion \\
Science et persuasion \\
Science et réalité \\
Science et religion \\
Science et sagesse \\
Science et technologie \\
Science pure et science appliquée \\
Sciences de la nature et sciences de l'esprit \\
Sciences de la nature et sciences humaines \\
Signification et vérité \\
Sujet et prédicat \\
Sur quoi se fonde la connaissance scientifique ? \\
Syllogisme et démonstration \\
Tautologie et contradiction \\
Technique et pratiques scientifiques \\
Théorie et modélisation \\
Toute connaissance autre que scientifique doit-elle être considérée comme une illusion ? \\
Toute connaissance est-elle historique ? \\
Tout énoncé est-il nécessairement vrai ou faux ? \\
Toutes les vérités  scientifiques sont-elles révisables ? \\
Tout savoir est-il fondé sur un savoir premier ? \\
Une logique non-formelle est-elle possible ? \\
Une théorie scientifique peut-elle être ramenée à des propositions empiriques élémentaires ? \\
Universalité et nécessité dans les sciences \\
Un problème scientifique peut-il être insoluble ? \\
Un sceptique peut-il être logicien ? \\
Vérité et histoire \\
Vitalisme et mécanique \\
Y a-t-il de l'indémontrable ? \\
Y a-t-il des expériences cruciales ? \\
Y a-t-il des lois du hasard ? \\
Y a-t-il des propriétés singulières ? \\
Y a-t-il des révolutions scientifiques ? \\
Y a-t-il différentes manières de connaître ? \\
Y a-t-il du synthétique \emph{a priori} ? \\
Y a-t-il plusieurs nécessités ? \\
Y a-t-il un art d'inventer ? \\
Y a-t-il un critère du vrai ? \\
Y a-t-il une hiérarchie des sciences ? \\
Y a-t-il une histoire de la vérité ? \\
Y a-t-il une logique de la découverte scientifique ? \\
Y a-t-il une logique de la découverte ? \\
Y a-t-il une science de l'individuel ? \\
Y a-t-il une science du qualitatif ? \\
Y a-t-il une unité de la science ? \\
Y a-t-il une universalité des mathématiques ? \\
« La logique » ou bien « les logiques » ? \\


\subsection{Métaphysique}
\label{sec-3-4}

\noindent
Apparaître \\
Apparence et réalité \\
Au-delà \\
Au-delà de la nature ? \\
Catégories de l'être, catégories de langue \\
Causes premières et causes secondes \\
Ce qui fut et ce qui sera \\
Ce qui n'est pas réel est-il impossible ? \\
Ce qui passe et ce qui demeure \\
Certitude et vérité \\
Chaque science porte-t-elle une métaphysique qui lui est propre ? \\
Chose et objet \\
Connaître et penser \\
Consistance et précarité \\
Contingence et nécessité \\
Création et production \\
Définir, est-ce déterminer l'essence ? \\
Devenir autre \\
Dieu pense-t-il ? \\
Dieu peut-il tout faire ? \\
Dire le monde \\
En quel sens la métaphysique a-t-elle une histoire ? \\
En quel sens la métaphysique est-elle une science ? \\
En quel sens parler de structure métaphysique ? \\
En quoi la connaissance de la matière peut-elle relever de la métaphysique ? \\
Être cause de soi \\
Être dans l'esprit \\
Être dans le temps \\
Être et devenir \\
Être et devoir être \\
Être et être pensé \\
Être et ne plus être \\
Être et représentation \\
Être et sens \\
Être par soi \\
Être sans cause \\
Être une chose qui pense \\
Être, vie et pensée \\
Faire de la métaphysique, est-ce se détourner du monde ? \\
Fait et essence \\
Grammaire et métaphysique \\
Ici et maintenant \\
Identité et différence \\
Il y a \\
Infini et indéfini \\
La béatitude \\
L'absence \\
L'absence de fondement \\
L'accident \\
La chose en soi \\
La contradiction \\
La création \\
L'acte \\
La destruction \\
La division \\
La dualité \\
La fin \\
La fin de la métaphysique \\
La limite \\
L'altérité \\
La manifestation \\
La matière \\
La matière première \\
L'âme, le monde et Dieu \\
La métaphysique a-t-elle ses fictions ? \\
La métaphysique est-elle le fondement de la morale ? \\
La métaphysique est-elle nécessairement une réflexion sur Dieu ? \\
La métaphysique peut-elle être autre chose qu'une science recherchée ? \\
La métaphysique peut-elle faire appel à l'expérience ? \\
La métaphysique se définit-elle par son objet ou sa démarche ? \\
La multiplicité \\
La naissance \\
La nature et la grâce \\
La négation \\
La notion d'ordre \\
L'antériorité \\
La participation \\
La perfection \\
La pluralité des mondes \\
La pluralité des sens de l'être \\
La possibilité métaphysique \\
La possibilité réelle \\
La preuve de l'existence de Dieu \\
L'\emph{a priori} \\
La puissance des contraires \\
La puissance et l'acte \\
La raison suffisante \\
La réalité \\
La réalité du sensible \\
La réalité du temps \\
La recherche de l'absolu \\
La relation \\
La science de l'être \\
La singularité \\
La spontanéité \\
La toute puissance \\
La transcendance \\
L'au-delà de l'être \\
La vie de l'esprit \\
La vie éternelle \\
Le besoin de métaphysique est-il un besoin de connaissance ? \\
Le changement \\
Le concret \\
Le corps et l'esprit \\
Le créé et l'incréé \\
Le désir métaphysique \\
Le divin \\
L'efficience \\
Le fond \\
Le fondement \\
Le hasard \\
Le lieu de la pensée \\
Le mal constitue-t-il une objection à l'existence de Dieu ? \\
Le mal métaphysique \\
Le meilleur des mondes possible \\
Le même et l'autre \\
Le métaphysicien est-il un physicien à sa façon ? \\
Le miracle \\
Le mode \\
Le monde intérieur \\
Le monde vrai \\
Le néant \\
L'ennui \\
Le nombre \\
Le particulier \\
L'éphémère \\
Le phénomène \\
Le premier principe \\
Le principe de causalité \\
Le principe de contradiction \\
Le principe d'identité \\
Le problème de l'être \\
Le réel est-il rationnel ? \\
Le réel et le virtuel \\
Le réel peut-il être contradictoire ? \\
Le royaume du possible \\
Les causes finales \\
Les choses \\
Les genres de Dieu \\
Les idées et les choses \\
Le simple \\
Les individus \\
Les limites de la raison \\
Les limites de l'expérience \\
Les lois de la nature sont-elles contingentes ? \\
Les nombres gouvernent-ils le monde ? \\
Le souverain bien \\
Les propositions métaphysiques sont-elles des illusions ? \\
Les questions métaphysiques ont-elles un sens ? \\
Les sciences ont-elles besoin d'une fondation métaphysique ? \\
Le sujet de la pensée \\
Les universaux \\
Les vérités éternelles \\
L'éternel présent \\
L'éternité \\
L'être de la vérité \\
L'être de l'image \\
L'être en tant qu'être \\
L'être en tant qu'être est-il connaissable ? \\
L'être et la volonté \\
L'être et le bien \\
L'être et le néant \\
L'être et les êtres \\
L'être et l'essence \\
L'être et l'étant \\
L'être et le temps \\
L'événement manque-t-il d'être ? \\
Le vide \\
L'existence du mal \\
L'existence se démontre-t-elle ? \\
L'expérience métaphysique \\
Liberté humaine et liberté divine \\
L'idée de Dieu \\
L'idée d'un commencement absolu \\
L'immanence \\
L'immédiat \\
L'immortalité de l'âme \\
L'immuable \\
L'impossible \\
L'impuissance de la raison \\
L'inapparent \\
L'incompréhensible \\
L'incorporel \\
L'indéterminé \\
L'individu \\
L'indivisible \\
L'infinité de l'espace \\
L'inquiétude \\
L'intangible \\
L'interrogation humaine \\
L'invisible \\
L'irréel \\
Logique et métaphysique \\
L'omniscience \\
L'ordre des choses \\
L'un \\
L'un et le multiple \\
L'un et l'être \\
L'unité des contraires \\
L'univers \\
L'universel \\
L'univocité de l'étant \\
Métaphysique et histoire \\
Métaphysique et ontologie \\
Métaphysique et religion \\
Métaphysique spéciale, métaphysique générale \\
Monde et nature \\
Négation et privation \\
Notre corps pense-t-il ? \\
N'y a-t-il qu'un seul monde ? \\
Origine et commencement \\
Pâtir \\
Penser sans corps \\
Persévérer dans son être \\
Peut-on dire ce qui n'est pas ? \\
Peut-on douter de sa propre existence ? \\
Peut-on entreprendre d'éliminer la métaphysique ? \\
Peut-on parler de vérités métaphysiques ? \\
Peut-on penser l'extériorité ? \\
Peut-on penser une métaphysique sans Dieu ? \\
Peut-on réduire une métaphysique à une conception du monde ? \\
Peut-on se passer de Dieu ? \\
Peut-on tout définir ? \\
Physique et métaphysique \\
Pourquoi Dieu se soucierait-il des affaires humaines ? \\
Pourquoi y a-t-il quelque chose plutôt que rien ? \\
Présence et absence \\
Principe et cause \\
Prouver en métaphysique \\
Psychologie et métaphysique \\
Quantité et qualité \\
Quel est l'objet de la métaphysique ? \\
Que peut-on dire de l'être ? \\
Qu'est-ce qui est réel ? \\
Qu'est-ce qu'une âme ? \\
Qu'est-ce qu'une catégorie de l'être ? \\
Qu'est-ce qu'un élément ? \\
Qu'est-ce qu'une méditation métaphysique ? \\
Qu'est-ce qu‘une philosophie première ? \\
Qu'est-ce qu'une propriété essentielle ? \\
Qu'est-ce qu'une question métaphysique ? \\
Qu'est-ce qu'un métaphysicien ? \\
Qu'est-ce qu'un objet métaphysique ? \\
Qu'est-ce qu'un principe ? \\
Qu'est-ce qu'un problème métaphysique ? \\
Que veut dire introduire à la métaphysique ? \\
Rendre raison \\
Sauver les apparences \\
Se détacher des sens \\
Sensible et intelligible \\
Seul le présent existe-t-il ? \\
Si l'esprit n'est pas une table rase, qu'est-il ? \\
Sommes-nous des êtres métaphysiques ? \\
Temps et éternité \\
Tout a-t-il une raison d'être ? \\
Toute métaphysique implique-t-elle une transcendance ? \\
Toutes les choses sont-elles singulières ? \\
Tout est-il relatif ? \\
Tout être est-il dans l'espace ? \\
Un Dieu unique ? \\
Une cause peut-elle être libre ? \\
Une métaphysique athée est-elle possible ? \\
Une métaphysique peut-elle être sceptique ? \\
Univocité et équivocité \\
Vérités de fait et vérités de raison \\
Vie et volonté \\
Y a-t-il des degrés de réalité ? \\
Y a-t-il des êtres mathématiques ? \\
Y a-t-il une argumentation métaphysique ? \\
Y a-t-il une connaissance métaphysique ? \\
Y a-t-il une expérience de l'éternité ? \\
Y a-t-il une hiérarchie des êtres ? \\
Y a-t-il une métaphysique de l'amour ? \\
Y a-t-il une science des principes ? \\
Y a-t-il un principe du mal ? \\
Y a-t-il un savoir du contingent ? \\


\subsection{Morale}
\label{sec-3-5}

\noindent
À chacun son dû \\
Agir moralement, est-ce lutter contre ses idées ? \\
Aimer ses proches \\
Aimer son prochain comme soi-même \\
Avons-nous des devoirs envers les animaux ? \\
Avons-nous des devoirs envers le vivant ? \\
À quoi servent les doctrines morales ? \\
Ce que la morale autorise, l'État peut-il légitimement l'interdire ? \\
Ce qui dépend de moi \\
C'est pour ton bien \\
Changer ses désirs plutôt que l'ordre du monde \\
Chercher son intérêt, est-ce être immoral ? \\
Comment traiter les animaux ? \\
Compatir \\
Composer avec les circonstances \\
Conduire sa vie \\
Conviction et responsabilité \\
Défendre son honneur \\
Démériter \\
Désintérêt et désintéressement \\
Désobéir \\
Devant qui sommes-nous responsables ? \\
Doit-on bien juger pour bien faire ? \\
Doit-on répondre de ce qu'on est devenu ? \\
Doit-on toujours dire la vérité ? \\
Donner sa parole \\
Droits et devoirs \\
Égoïsme et altruisme \\
Égoïsme et individualisme \\
Enfance et moralité \\
Éprouver sa valeur \\
Est-ce pour des raisons morales qu'il faut protéger l'environnement ? \\
Esthétisme et moralité \\
Est-il mauvais de suivre son désir ? \\
Est-il parfois bon de mentir ? \\
Est-on responsable de ce qu'on n'a pas voulu ? \\
Est-on responsable de l'avenir de l'humanité \\
Éthique et authenticité \\
Être content de soi \\
Être en règle avec soi-même \\
Être juge et partie \\
Être méchant \\
Être sans cœur \\
Expliquer et justifier \\
Faire de sa vie une œuvre d'art \\
Faire la morale \\
Fait et valeur \\
Faut-il être bon ? \\
Faut-il expliquer la morale par son utilité ? \\
Faut-il mieux vivre comme si nous ne devions jamais mourir ? \\
Faut-il prendre soin de soi ? \\
Faut-il se délivrer des passions ? \\
Garder la mesure \\
Haïr \\
Je ne l'ai pas fait exprès \\
Jouir sans entraves \\
Juger en conscience \\
Jusqu'où peut-on soigner ? \\
La beauté morale \\
La belle âme \\
La bienfaisance \\
La bienveillance \\
La bonne conscience \\
La bonne volonté \\
La casuistique \\
La censure \\
La clémence \\
La communauté morale \\
La confiance \\
La connaissance suppose-t-elle une éthique ? \\
La conscience est-elle intrinsèquement morale ? \\
La conscience morale \\
La conscience morale est-elle innée ? \\
La contingence \\
La corruption \\
La cruauté \\
La culpabilité \\
La culture morale \\
La décision morale \\
La délibération \\
La délibération en morale \\
La déontologie \\
La détermination \\
La dette \\
La dignité \\
La discipline \\
La disposition morale \\
La docilité est-elle un vice ou une vertu ? \\
L'adoucissement des mœurs \\
La droiture \\
La duplicité \\
La faiblesse de la volonté \\
La famille est-elle le lieu de la formation morale ? \\
La faute \\
La fidélité \\
La fidélité à soi \\
La force d'âme \\
La force est-elle une vertu ? \\
La franchise \\
La franchise est-elle une vertu ? \\
La fraternité est-elle un idéal moral ? \\
La générosité \\
La gentillesse \\
La gratitude \\
L'agressivité \\
La honte \\
La juste colère \\
La justice \\
La justice est-elle une notion morale ? \\
La libération des mœurs \\
La liberté, est-ce l'indépendance à l'égard des passions ? \\
La liberté morale \\
La loi \\
La louange et le blâme \\
La loyauté \\
L'altruisme \\
La magnanimité \\
La maîtrise de soi \\
La mauvaise conscience \\
La mauvaise foi \\
La mauvaise volonté \\
La misanthropie \\
L'amitié \\
La modération \\
La modération est-elle l'essence de la vertu ? \\
La morale a-t-elle besoin d'être fondée ? \\
La morale a-t-elle besoin d'un au-delà ? \\
La morale commune \\
La morale consiste-t-elle à suivre la nature ? \\
La morale de l'intérêt \\
La morale des fables \\
La morale doit-elle fournir des préceptes ? \\
La morale du citoyen \\
La morale du plus fort \\
La morale est-elle affaire de jugement ? \\
La morale est-elle affaire de sentiment ? \\
La morale est-elle ennemie du bonheur ? \\
La morale est-elle incompatible avec le déterminisme ? \\
La morale est-elle l'ennemie de la vie ? \\
La morale est-elle nécessairement répressive ? \\
La morale est-elle un art de vivre ? \\
La morale est-elle une affaire d'habitude ? \\
La morale est-elle un fait social ? \\
La morale et le droit \\
La morale peut-elle être fondée sur la science ? \\
La morale peut-elle être naturelle ? \\
La morale peut-elle être un calcul ? \\
La morale peut-elle être une science ? \\
La morale peut-elle se passer d'un fondement religieux ? \\
La morale suppose-t-elle le libre arbitre ? \\
La moralité n'est-elle que dressage ? \\
La moralité réside-t-elle dans l'intention ? \\
L'amour de l'humanité \\
L'amour de soi \\
L'amour et la justice \\
L'amour-propre \\
La nature du bien \\
La nature est-elle digne de respect ? \\
La neutralité \\
L'angélisme \\
L'animal peut-il être un sujet moral ? \\
L'apathie \\
La patience \\
La patience est-elle une vertu ? \\
La personne \\
La perversion morale \\
La perversité \\
La peur du châtiment \\
La pitié \\
La pitié est-elle morale ? \\
La pitié peut-elle fonder la morale ? \\
La politesse \\
La politique doit-elle être morale ? \\
La probité \\
La promesse \\
La prudence \\
La pudeur \\
La raison est-elle morale par elle-même ? \\
La raison peut-elle être immédiatement pratique ? \\
La raison peut-elle être pratique ? \\
La recherche du bonheur suffit-elle à déterminer une morale ? \\
La reconnaissance \\
La règle et l'exception \\
La réparation \\
La résignation \\
La responsabilité \\
L'argumentation morale \\
La rigueur morale \\
La ruse \\
La sagesse \\
La sainteté \\
La sanction \\
La satisfaction des penchants \\
L'ascèse \\
L'ascétisme \\
La science peut-elle guider notre conduite ? \\
La servitude \\
La sincérité \\
La sollicitude \\
La souffrance \\
La souffrance a-t-elle un sens moral ? \\
La sympathie \\
La technique est-elle moralement neutre ? \\
La tentation \\
La terreur morale \\
La tolérance \\
La transgression \\
L'autarcie \\
L'authenticité \\
L'autonomie \\
L'autorité morale \\
L'autre est-il le fondement de la conscience morale ? \\
La valeur de l'exemple \\
La valeur morale \\
La vanité \\
La vénalité \\
La vengeance \\
La véracité \\
La vérité est-elle morale ? \\
La vertu, les vertus \\
La vertu peut-elle être purement morale ? \\
La vertu peut-elle s'enseigner ? \\
L'aveu diminue-t-il la faute ? \\
La vie est-elle la valeur suprême ? \\
La vigilance \\
La voix de la conscience \\
La vraie morale se moque-t-elle de la morale ? \\
Le bien et le mal \\
Le bien et les biens \\
Le bien suppose-t-il la transcendance ? \\
Le bon et l'utile \\
Le bonheur des uns, le malheur des autres \\
Le bonheur est-il une fin morale ? \\
Le bonheur et la vertu \\
Le calcul des plaisirs \\
Le cas de conscience \\
Le catéchisme moral \\
Le châtiment \\
Le choix \\
L'école des vertus \\
Le combat contre l'injustice a-t-il une source morale ? \\
Le conflit des devoirs \\
Le conformisme \\
Le conformisme moral \\
Le convenable \\
Le corps est-il respectable ? \\
Le courage \\
Le cynisme \\
Le dérèglement \\
Le désespoir est-il une faute morale ? \\
Le désintéressement \\
Le devoir-être \\
Le dévouement \\
L'édification morale \\
Le droit de punir \\
Le formalisme moral \\
Légalité et moralité \\
L'égoïsme \\
Le héros moral \\
Le jugement moral \\
Le juste milieu \\
Le mal \\
Le méchant peut-il être heureux ? \\
Le mensonge \\
Le mépris \\
Le mérite est-il le critère de la vertu ? \\
Le modèle en morale \\
Le moindre mal \\
Le moraliste \\
L'empire sur soi \\
L'enthousiasme est-il moral ? \\
L'entraide \\
L'envie \\
Le pardon \\
Le partage est-il une obligation morale ? \\
Le péché \\
Le plaisir a-t-il un rôle à jouer dans la morale ? \\
Le plaisir est-il un bien ? \\
Le préférable \\
Le principe de réciprocité \\
Le progrès moral \\
L'équité \\
Le rapport de l'homme à son milieu a-t-il une dimension morale ? \\
Le relativisme moral \\
Le remords \\
Le renoncement \\
Le repentir \\
Le respect \\
Le ressentiment \\
Le rigorisme \\
L'erreur et la faute \\
Le sacrifice \\
Les animaux échappent-ils à la moralité ? \\
Les bénéfices moraux \\
Les bonnes intentions \\
Les bonnes mœurs \\
Le scandale \\
Les caractères moraux \\
Le scrupule \\
Les devoirs à l'égard de la nature \\
Les devoirs envers soi-même \\
Les dilemmes moraux \\
Les droits de l'homme ont-ils un fondement moral ? \\
Le sens moral \\
Les fins naturelles et les fins morales \\
Les hommes n'agissent-ils que par intérêt ? \\
Les leçons de morale \\
Les mœurs \\
Le souci d'autrui résume-t-il la morale ? \\
Le souci de soi \\
Le souverain bien \\
Les paroles et les actes \\
Les passions peuvent-elles être raisonnables ? \\
L'espérance est-elle une vertu ? \\
Les plaisirs de l'amitié \\
Les préjugés moraux \\
Les principes moraux \\
Les proverbes nous instruisent-ils moralement ? \\
Le sujet moral \\
L'éthique des plaisirs \\
L'éthique est-elle affaire de choix ? \\
L'éthique suppose-t-elle la liberté ? \\
Le tourment moral \\
Le travail est-il une valeur morale ? \\
Le volontaire et l'involontaire \\
L'excuse \\
L'exigence morale \\
L'expérience morale \\
L'habileté \\
L'hétéronomie \\
L'homme injuste peut-il être heureux ? \\
L'honnêteté \\
L'honneur \\
L'hospitalité \\
L'humiliation \\
L'humilité \\
L'hypocrisie \\
L'hypothèse de l'inconscient \\
L'idéal moral est-il vain ? \\
L'idée de morale appliquée \\
L'idée de rétribution est-elle nécessaire à la morale ? \\
L'ignoble \\
L'imitation a-t-elle une fonction morale ? \\
L'impardonnable \\
L'impartialité \\
L'impératif \\
L'indifférence \\
L'indulgence \\
L'inhibition \\
L'injonction \\
L'injustifiable \\
L'innocence \\
L'instruction est-elle facteur de moralité ? \\
L'insulte \\
L'intempérance \\
L'intention morale \\
L'intention morale suffit-elle à constituer la valeur morale de l'action ? \\
L'interdit \\
L'intérêt \\
L'intérêt peut-il être une valeur morale ? \\
L'intolérable \\
L'intuition morale \\
L'involontaire \\
L'irrésolution \\
L'obligation \\
L'obligation morale \\
L'obligation morale peut-elle se réduire à une obligation sociale ? \\
L'obscène \\
L'offense \\
Loi morale et loi politique \\
L'opportunisme \\
L'ordre moral \\
L'orgueil \\
L'origine des vertus \\
L'oubli des fautes \\
L'utilité est-elle étrangère à la morale ? \\
Mémoire et responsabilité \\
Mentir \\
Morale et convention \\
Morale et éducation \\
Morale et histoire \\
Morale et liberté \\
Morale et pratique \\
Morale et religion \\
Morale et sexualité \\
Morale et société \\
Morale et violence \\
Mourir dans la dignité \\
Mourir pour des principes \\
Ne lèse personne \\
Ne penser qu'à soi \\
N'est-on juste que par crainte du châtiment ? \\
Normes morales et normes vitales \\
Notre ignorance nous excuse-t-elle ? \\
Obéir \\
Peut-il être moral de tuer ? \\
Peut-on concevoir une morale sans sanction ni obligation ? \\
Peut-on conclure de l'être au devoir-être ? \\
Peut-on définir le bien ? \\
Peut-on disposer de son corps ? \\
Peut-on être amoral ? \\
Peut-on faire du dialogue un modèle de relation morale ? \\
Peut-on faire le mal en vue du bien ? \\
Peut-on ne pas savoir ce que l'on fait ? \\
Peut-on opposer morale et technique ? \\
Peut-on penser une volonté diabolique ? \\
Peut-on reprocher à la morale d'être abstraite ? \\
Peut-on s'accorder sur des vérités morales ? \\
Peut-on se punir soi-même ? \\
Peut-on transiger avec les principes ? \\
Peut-on vouloir le mal ? \\
Pourquoi châtier ? \\
Pourquoi être moral ? \\
Pouvons-nous devenir meilleurs ? \\
Prendre soin \\
Protester \\
Qu'a-t-on le droit de pardonner ? \\
Que doit-on aux morts ? \\
Quelles sont les caractéristiques d'une proposition morale ? \\
Qu'est-ce le mal radical ? \\
Qu'est-ce qui est respectable ? \\
Qu'est-ce qu'un acte moral ? \\
Qu'est-ce qu'une idée morale ? \\
Qu'est-ce qu'une vie réussie ? \\
Qu'est-ce qu'une volonté libre ? \\
Qu'est-ce qu'un fait moral ? \\
Qu'est-ce qu'un homme juste ? \\
Qu'est-ce qu'un idéal moral ? \\
Qu'est-ce qu'un sentiment moral ? \\
Que vaut en morale la justification par l'utilité ? \\
Qui est mon prochain ? \\
Règle et commandement \\
Réprouver \\
Sans foi ni loi \\
Se mentir à soi-même \\
Se mettre à la place d'autrui \\
S'indigner \\
Si tu veux, tu peux \\
Sommes-nous libres de nos préférences morales ? \\
Sommes-nous toujours dépendants d'autrui ? \\
Suffit-il de vouloir pour bien faire ? \\
Tout devoir est-il l'envers d'un droit ? \\
Tout est permis \\
Trahir \\
Traiter autrui comme une chose \\
Tricher \\
Un bien peut-il sortir d'un mal ? \\
Une action vertueuse se reconnaît-elle à sa difficulté ? \\
Une éthique sceptique est-elle possible ? \\
Une ligne de conduite peut-elle tenir lieu de morale ? \\
Une morale du plaisir est-elle concevable ? \\
Une morale peut-elle être dépassée ? \\
Une morale peut-elle être provisoire ? \\
Une morale peut-elle prétendre à l'universalité ? \\
Une morale sans Dieu \\
Une morale sans obligation est-elle possible ? \\
Une science de la morale est-elle possible ? \\
Un vice, est-ce un manque ? \\
Vertu et habitude \\
Vices privés, vertus publiques \\
Vivre sans morale \\
Vivre vertueusement \\
Voir le meilleur et faire le pire \\
Vouloir le bien \\
Y a-t-il de l'irréparable ? \\
Y a-t-il des actes moralement indifférents ? \\
Y a-t-il des devoirs envers soi-même ? \\
Y a-t-il des faits moraux ? \\
Y a-t-il des fins dernières ? \\
Y a-t-il des limites proprement morales à la discussion ? \\
Y a-t-il des lois morales ? \\
Y a-t-il des normes naturelles ? \\
Y a-t-il des vertus mineures ? \\
Y a-t-il place pour l'idée de vérité en morale ? \\
Y a-t-il un devoir d'être heureux ? \\
Y a-t-il un droit de mourir ? \\
Y a-t-il une beauté morale ? \\
Y a-t-il une forme morale de fanatisme ? \\
Y a-t-il une justice sans morale ? \\
Y a-t-il une place pour la morale dans l'économie ? \\
Y a-t-il un usage moral des passions ? \\
« Aime, et fais ce que tu veux » \\
« À l'impossible, nul n'est tenu » \\
« Bienheureuse faute » \\
« L'enfer est pavé de bonnes intentions » \\
« Ne fais pas à autrui ce que tu ne voudrais pas qu'on te fasse » \\
« Œil pour œil, dent pour dent » \\
« Tu ne tueras point » \\


\subsection{Politique}
\label{sec-3-6}

\noindent
Amitié et société \\
Apprendre à gouverner \\
À quoi juger l'action d'un gouvernement ? \\
Art et politique \\
A-t-on des droits contre l'État ? \\
Autorité et pouvoir \\
Avons-nous besoin de partis politiques ? \\
Avons-nous besoin de traditions ? \\
À quoi reconnaît-on qu'une politique est juste ? \\
À quoi reconnaît-on un bon gouvernement ? \\
À quoi sert la notion de contrat social ? \\
À quoi sert la notion d'état de nature ? \\
À quoi sert l'État ? \\
À quoi servent les élections ? \\
Ceux qui savent doivent-ils gouverner ? \\
Cité juste ou citoyen juste ? \\
Citoyen et soldat \\
Commémorer \\
Comment décider, sinon à la majorité ? \\
Comment juger de la politique ? \\
Comment penser un pouvoir qui ne corrompe pas ? \\
Communauté et société \\
Connaissance historique et action politique \\
Conseiller le prince \\
Conservatisme et tradition \\
Constitution et lois \\
Consumérisme et démocratie \\
Contrainte et désobéissance \\
Crime et châtiment \\
Démocrates et démagogues \\
Démocratie ancienne et démocratie moderne \\
Démocratie et anarchie \\
Démocratie et démagogie \\
Démocratie et impérialisme \\
Démocratie et représentation \\
Démocratie et république \\
Démocratie et transparence \\
De quoi l'État doit-il être propriétaire ? \\
Désir et politique \\
Des nations peuvent-elles former une société ? \\
Désobéir aux lois \\
Désobéissance et résistance \\
Des sociétés sans État sont-elles des sociétés politiques ? \\
Division du travail et cohésion sociale \\
D'où la politique tire-t-elle sa légitimité ? \\
Droit naturel et loi naturelle \\
Droits de l'homme et droits du citoyen \\
Droits et devoirs sont-ils réciproques ? \\
Économie et politique \\
Éduquer le citoyen \\
Égalité des droits, égalité des conditions \\
En politique n'y a-t-il que des rapports de force ? \\
En politique, peut-on faire table rase du passé ? \\
En politique, y a-t-il des modèles ? \\
En quoi une discussion est-elle politique ? \\
Espace public et vie privée \\
Est-il bon qu'un seul commande ? \\
Est-il possible d'être neutre politiquement ? \\
État et nation \\
Être citoyen du monde \\
Existe-t-il un bien commun qui soit la norme de la vie politique ? \\
Faire de la politique \\
Faire la loi \\
Faire la paix \\
Faut-il considérer le droit pénal comme instituant une violence légitime ? \\
Faut-il craindre la révolution ? \\
Faut-il craindre les foules ? \\
Faut-il détruire l'État ? \\
Faut-il diriger l'économie ? \\
Faut-il être réaliste en politique ? \\
Faut-il fuir la politique ? \\
Faut-il limiter la souveraineté ? \\
Faut-il limiter l'exercice de la puissance publique ? \\
Faut-il opposer à la politique la souveraineté du droit ? \\
Faut-il penser l'État comme un corps ? \\
Faut-il préférer une injustice au désordre ? \\
Faut-il se méfier du volontarisme politique ? \\
Faut-il tolérer les intolérants ? \\
Faut-il vouloir changer le monde ? \\
Faut-il vouloir la paix ? \\
Fonder une cite \\
Gouverner \\
Gouverner, administrer, gérer \\
Gouverner, est-ce prévoir ? \\
Gouverner et se gouverner \\
Groupe, classe, société \\
Guerre et politique \\
Imaginaire et politique \\
L'abus de pouvoir \\
La chose publique \\
La cité idéale \\
La citoyenneté \\
La civilité \\
La clause de conscience \\
La comédie du pouvoir \\
La communauté internationale \\
La communication est-elle nécessaire à la démocratie ? \\
La compassion risque-t-elle d'abolir l'exigence politique ? \\
La compétence technique peut-elle fonder l'autorité publique ? \\
La conscience politique \\
La constitution \\
La contestation \\
La contrôle social \\
La corruption politique \\
L'action politique \\
L'action politique a-t-elle un fondement rationnel ? \\
L'action politique peut-elle se passer de mots ? \\
La culture démocratique \\
La culture est-elle affaire de politique ? \\
La décision politique \\
La défense nationale \\
La délibération politique \\
La démagogie \\
La démocratie conduit-elle au règne de l'opinion ? \\
La démocratie est-elle le pire des régimes politiques ? \\
La démocratie est-elle moyen ou fin ? \\
La démocratie est-elle nécessairement libérale ? \\
La démocratie est-elle possible ? \\
La démocratie n'est-elle que la force des faibles ? \\
La démocratie participative \\
La désobéissance civile \\
La dictature \\
La discrimination \\
La division des pouvoirs \\
La domination \\
La droit de conquête \\
La fin de la politique \\
La fin de la politique est-elle l'établissement de la justice ? \\
La fin de l'État \\
La fin justifie-t-elle les moyens ? \\
La fonction première de l'État est-elle de durer ? \\
La force de la loi \\
La force des lois \\
La force du pouvoir \\
La force fait-elle le droit ? \\
La formation des citoyens \\
La fraternité a-t-elle un sens politique ? \\
La guerre civile \\
La guerre est-elle la continuation de la politique par d'autres moyens ? \\
La guerre est-elle la continuation de la politique ? \\
La guerre et la paix \\
La guerre juste \\
La guerre totale \\
La justice consiste-t-elle à traiter tout le monde de la même manière ? \\
La justice entre les générations \\
La justice sociale \\
La justice : moyen ou fin de la politique ? \\
La laïcité \\
La légitimité \\
La liberté civile \\
La liberté de culte \\
La liberté des citoyens \\
La liberté d'opinion \\
La liberté politique \\
La loi et le règlement \\
La loyauté \\
La lutte des classes \\
La majorité peut-elle être tyrannique ? \\
L'ambition politique \\
La meilleure constitution \\
L'amitié est-elle un principe politique ? \\
La modération est-elle une vertu politique ? \\
La morale politique \\
L'amour des lois \\
L'anarchie \\
La nation \\
La nation et l'État \\
La neutralité de l'État \\
L'animal politique \\
La notion de progrès a-t-elle un sens en politique ? \\
La notion de sujet en politique \\
La paix \\
La paix civile \\
La paix est-elle possible ? \\
La paix n'est-elle que l'absence de guerre ? \\
La paix perpétuelle \\
La paix sociale est-elle une fin en soi ? \\
La parole publique \\
La participation des citoyens \\
La patrie \\
La pauvreté \\
La peur du désordre \\
La pluralité des opinions \\
La pluralité des pouvoirs \\
La politique a-t-elle besoin de héros ? \\
La politique a-t-elle besoin de modèles ? \\
La politique a-t-elle besoin d'experts ? \\
La politique a-t-elle pour fin d'éliminer la violence ? \\
La politique consiste-t-elle à faire des compromis ? \\
La politique de la santé \\
La politique doit-elle être rationnelle ? \\
La politique doit-elle se mêler de l'art ? \\
La politique doit-elle viser la concorde ? \\
La politique doit-elle viser le consensus ? \\
La politique échappe-telle à l'exigence de vérité ? \\
La politique est-elle affaire de décision ? \\
La politique est-elle affaire de jugement ? \\
La politique est-elle architectonique ? \\
La politique est-elle la continuation de la guerre ? \\
La politique est-elle l'affaire de tous ? \\
La politique est-elle l'art des possibles ? \\
La politique est-elle l'art du possible ? \\
La politique est-elle naturelle ? \\
La politique est-elle par nature sujette à dispute ? \\
La politique est-elle plus importante que tout ? \\
La politique est-elle un art ? \\
La politique est-elle une science ? \\
La politique est-elle une technique ? \\
La politique est-elle un métier ? \\
La politique et la gloire \\
La politique et la ville \\
La politique et le mal \\
La politique et le politique \\
La politique et l'opinion \\
La politique exclut-elle le désordre ? \\
La politique implique-t-elle la hiérarchie ? \\
La politique peut-elle changer la société \\
La politique peut-elle changer le monde ? \\
La politique peut-elle être indépendante de la morale ? \\
La politique peut-elle être objet de science ? \\
La politique peut-elle être un objet de science ? \\
La politique peut-elle n'être qu'une pratique ? \\
La politique peut-elle se passer de croyances ? \\
La politique peut-elle unir les hommes ? \\
La politique repose-t-elle sur un contrat ? \\
La politique requière-t-elle le compromis \\
La politique suppose-t-elle la morale ? \\
La politique suppose-t-elle une idée de l'homme ? \\
L'apolitisme \\
La populace \\
La prise de parti est-elle essentielle en politique ? \\
La propriété \\
La propriété est-elle une garantie de liberté ? \\
La protection sociale \\
La prudence \\
La raison d'État \\
La rationalité des choix politiques \\
La rationalité politique \\
La réaction en politique \\
La réciprocité est-elle indispensable à la communauté politique ? \\
La réforme des institutions \\
La religion \\
La religion peut-elle faire lien social ? \\
La représentation en politique \\
La représentation politique \\
La république \\
La résistance à l'oppression \\
La responsabilité politique \\
La révolte \\
L'aristocratie \\
L'art de gouverner \\
L'art politique \\
La science politique \\
La sécurité nationale \\
La sécurité publique \\
La ségrégation \\
La séparation des pouvoirs \\
La servitude volontaire \\
La société civile \\
La société civile et l'État \\
La société et l'État \\
La société peut-elle se passer de l'État ? \\
La solidarité \\
La solitude constitue-t-elle un obstacle à la citoyenneté ? \\
La souveraineté \\
La souveraineté de l'État \\
La souveraineté du peuple \\
La souveraineté peut-elle se partager ? \\
La souveraineté populaire \\
La sphère privée échappe-t-elle au politique ? \\
La sûreté \\
La surveillance de la société \\
La technocratie \\
La terreur \\
La tolérance \\
La tolérance envers les intolérants \\
La tolérance est-elle un concept politique ? \\
La tolérance peut-elle constituer un problème pour la démocratie ? \\
La totalitarisme \\
La transparence est-elle un idéal démocratique ? \\
La tyrannie \\
La tyrannie de la majorité \\
L'audace politique \\
L'autorité de l'État \\
L'autorité politique \\
La valeur du consentement \\
La vertu de l'homme politique \\
La vertu politique \\
La vie collective est-elle nécessairement frustrante ? \\
La vie politique \\
La vie politique est-elle aliénante ? \\
La vie privée \\
La violence de l'État \\
La violence politique \\
La violence révolutionnaire \\
La volonté constitue-t-elle le principe de la politique ? \\
La volonté générale \\
La volonté peut-elle être collective ? \\
Le bien public \\
Le bonheur des citoyens est-il un idéal politique ? \\
Le bonheur est-il une fin politique ? \\
Le bonheur est-il un principe politique ? \\
Le bourgeois et le citoyen \\
Le charisme en politique \\
Le citoyen \\
Le citoyen peut-il être à la fois libre et soumis à l'État ? \\
L'écologie est-elle un problème politique ? \\
L'écologie politique \\
Le commerce est-il pacificateur ? \\
Le conflit est-il constitutif de la politique ? \\
Le conflit est-il la raison d'être de la politique ? \\
L'économie politique \\
Le conseiller du prince \\
Le consensus \\
Le consentement des gouvernés \\
Le cosmopolitisme \\
Le cosmopolitisme peut-il devenir réalité ? \\
Le coup d'État \\
Le courage politique \\
Le débat politique \\
Le despote peut-il être éclairé ? \\
Le despotisme \\
Le devoir d'obéissance \\
Le discours politique \\
Le droit au Bonheur \\
Le droit de propriété \\
Le droit de punir \\
Le droit de révolte \\
Le droit des gens \\
Le droit des peuples à disposer d'eux-mêmes \\
Le droit de vie et de mort \\
Le droit d'ingérence \\
Le droit doit-il être le seul régulateur de la vie sociale ? \\
Le droit du plus fort \\
Le droit du premier occupant \\
Le droit humanitaire \\
L'éducation civique \\
L'éducation politique \\
Le fanatisme \\
Le fétichisme \\
L'égalité civile \\
L'égalité des chances \\
L'égalité des conditions \\
L'égalité des hommes et des femmes est-elle une question politique ? \\
Légitimité et légalité \\
Le gouvernement des experts \\
Le gouvernement des hommes et l'administration des choses \\
Le gouvernement des hommes libres \\
Le gouvernement des meilleurs \\
Le jugement politique \\
Le législateur \\
Le lien politique \\
Le lien social \\
Le lien social peut-il être compassionnel ? \\
L'émancipation des femmes \\
Le manifeste politique \\
Le meilleur régime \\
Le mensonge politique \\
Le métier de politique \\
Le monde politique \\
Le monopole de la violence légitime \\
L'empire \\
Le multiculturalisme \\
L'engagement politique \\
L'ennemi intérieur \\
Le pacifisme \\
Le patriotisme \\
Le peuple et les élites \\
Le philosophe a-t-il des leçons à donner au politique ? \\
Le philosophe est-il le vrai politique ? \\
Le philosophe-roi \\
Le pluralisme politique \\
Le poids du préjugé en politique \\
Le politique a-t-il à régler les passions humaines ? \\
Le politique doit-il être un technicien ? \\
Le politique doit-il se soucier des émotions ? \\
Le politique et le religieux \\
Le politique peut-il faire abstraction de la morale ? \\
Le populisme \\
Le pouvoir absolu \\
Le pouvoir corrompt-il nécessairement ? \\
Le pouvoir corrompt-il ? \\
Le pouvoir de l'opinion \\
Le pouvoir du peuple \\
Le pouvoir législatif \\
Le pouvoir peut-il limiter le pouvoir ? \\
Le pouvoir peut-il se déléguer ? \\
Le pouvoir peut-il se passer de sa mise en scène ? \\
Le pouvoir politique est-il nécessairement coercitif ? \\
Le pouvoir politique repose-t-il sur un savoir ? \\
Le pouvoir souverain \\
Le premier devoir de l'État est-il de se défendre ? \\
Le prince \\
Le principe d'égalité \\
Le public et le privé \\
Le règlement politique des conflits ? \\
Le respect des institutions \\
L'erreur politique, la faute politique \\
Les affaires publiques \\
Le savant et le politique \\
Le savoir utile au citoyen \\
Les biens communs \\
L'esclavage \\
Les conditions de la démocratie \\
Les conflits politiques \\
Les conflits politiques ne sont-ils que des conflits sociaux ? \\
Les conflits sociaux sont-ils des conflits politiques ? \\
Les croyances politiques \\
Les devoirs de l'État \\
Les droits de l'homme \\
Les droits de l'homme et ceux du citoyen \\
Les droits de l'homme sont-ils une abstraction ? \\
Les droits et les devoirs \\
Les droits naturels imposent-ils une limite à la politique ? \\
Le secret d'État \\
Le sens de l'État \\
Les factions politiques \\
Les fondements de l'État \\
Les frontières \\
Les grands hommes \\
Les hommes de pouvoir \\
Les hommes sont-ils naturellement sociables ? \\
Les idées politiques \\
Le silence des lois \\
Les inégalités sociales \\
Les intérêts particuliers peuvent-ils tempérer l'autorité politique ? \\
Les jeux du pouvoir \\
Les libertés civiles \\
Les libertés fondamentales \\
Les lieux du pouvoir \\
Les limites de la démocratie \\
Les limites de l'État \\
Les limites du pouvoir \\
Les limites du pouvoir politique \\
Les lois sont-elles seulement utiles ? \\
Les moyens de l'autorité \\
Le social et le politique \\
Les opinions politiques \\
Le souci du bien-être est-il politique ? \\
L'espace public \\
Les passions politiques \\
Les pauvres \\
Les peuples ont-ils les gouvernements qu'ils méritent ? \\
Les problèmes politiques peuvent-ils se ramener à des problèmes techniques ? \\
Les problèmes politiques sont-ils des problèmes techniques ? \\
Les règles d'un bon gouvernement \\
Les services publics \\
Les valeurs de la République \\
Les vertus politiques \\
L'État a-t-il pour finalité de maintenir l'ordre ? \\
L'État de droit \\
L'état de nature \\
L'état d'exception \\
L'État doit-il disparaître ? \\
L'État doit-il éduquer le citoyen ? \\
L'État doit-il éduquer les citoyens ? \\
L'État doit-il faire le bonheur des citoyens ? \\
L'État est-il fin ou moyen ? \\
L'État est-il le garant de la propriété privée ? \\
L'État et la culture \\
L'État et la Nation \\
L'État et le marché \\
L'État libéral \\
L'État peut-il créer la liberté ? \\
L'État peut-il être indifférent à la religion ? \\
L'État-providence \\
L'État universel \\
Le territoire \\
Le totalitarisme \\
Le travail \\
Le vainqueur a-t-il tous les droits ? \\
L'exclusion \\
L'exercice du pouvoir \\
L'exercice solitaire du pouvoir \\
L'existence de l'État dépend-elle d'un contrat ? \\
L'expertise politique \\
L'exploitation de l'homme par l'homme \\
L'hégémonie politique \\
L'histoire est-elle utile à la politique ? \\
L'homme des droits de l'homme n'est-il qu'une fiction ? \\
L'homme d'État \\
L'homme est-il un animal politique ? \\
L'homme et le citoyen \\
L'homme, le citoyen, le soldat \\
L'hospitalité a-t-elle un sens politique ? \\
Liberté, égalité, fraternité \\
Liberté réelle, liberté formelle \\
Libertés publiques et culture politique \\
L'idée de communauté \\
L'idée de contrat social \\
L'idée de domination \\
L'idée de nation \\
L'idée de république \\
L'idée de révolution \\
L'imagination politique \\
L'impartialité \\
L'impuissance de l'État \\
L'individualisme a-t-il sa place en politique ? \\
L'insociable sociabilité \\
L'insoumission \\
L'insurrection \\
L'intelligence politique \\
L'intérêt commun \\
L'intérêt général est-il le bien commun ? \\
L'intérêt public est-il une illusion ? \\
L'interprétation de la loi \\
L'irrationnel et le politique \\
L'objet de la politique \\
Loi naturelle et loi politique \\
L'oligarchie \\
L'opinion du citoyen \\
L'opinion publique \\
L'opposant \\
L'ordre politique peut-il exclure la violence ? \\
L'ordre public \\
L'unanimité est-elle un critère de légitimité ? \\
L'unité du corps politique \\
L'utilité publique \\
L'utopie a-t-elle une signification politique ? \\
L'utopie en politique \\
Mensonge et politique \\
Mœurs, coutumes, lois \\
Morale et politique sont-elles indépendantes ? \\
Mourir pour la patrie \\
Murs et frontières \\
Nation et richesse \\
Ni Dieu ni maître \\
Ni Dieu, ni maître \\
Nul n'est censé ignorer la loi \\
Peuple et société \\
Peuples et masses \\
Peut-il y avoir de la politique sans conflit ? \\
Peut-il y avoir un droit à désobéir ? \\
Peut-il y avoir une philosophie politique sans Dieu ? \\
Peut-il y avoir une science politique ? \\
Peut-il y avoir une société des nations ? \\
Peut-il y avoir une société sans État ? \\
Peut-il y avoir une vérité en politique ? \\
Peut-on admettre un droit à la révolte ? \\
Peut-on concevoir une société qui n'aurait plus besoin du droit ? \\
Peut-on concevoir un État mondial ? \\
Peut-on critiquer la démocratie ? \\
Peut-on en appeler à la conscience contre la loi ? \\
Peut-on être apolitique ? \\
Peut-on être citoyen du monde ? \\
Peut-on fonder les droits de l'homme ? \\
Peut-on gouverner sans lois ? \\
Peut-on innover en politique ? \\
Peut-on justifier la discrimination ? \\
Peut-on justifier la guerre ? \\
Peut-on justifier la raison d'État ? \\
Peut-on opposer justice et liberté ? \\
Peut-on parler de vertu politique ? \\
Peut-on refuser la loi ? \\
Peut-on régner innocemment ? \\
Peut-on revendiquer la paix comme un droit ? \\
Peut-on s'abstenir de penser politiquement ? \\
Peut-on se désintéresser de la politique ? \\
Peut-on séparer politique et économie ? \\
Peut-on se passer de chef ? \\
Peut-on se passer de l'État ? \\
Peut-on se passer de représentants ? \\
Peut-on se passer d'un maître ? \\
Peut-on se régler sur des exemples en politique ? \\
Peut-on souhaiter le gouvernement des meilleurs ? \\
Politique et esthétique \\
Politique et mémoire \\
Politique et parole \\
Politique et participation \\
Politique et passions \\
Politique et propagande \\
Politique et secret \\
Politique et technologie \\
Politique et territoire \\
Politique et trahison \\
Pourquoi des élections ? \\
Pourquoi des institutions ? \\
Pourquoi des lois ? \\
Pourquoi des utopies ? \\
Pourquoi écrit-on des lois ? \\
Pourquoi faire de la politique ? \\
Pourquoi faire la guerre ? \\
Pourquoi le droit international est-il imparfait ? \\
Pourquoi les États se font-ils la guerre ? \\
Pourquoi obéir aux lois ? \\
Pourquoi punir ? \\
Pourquoi séparer les pouvoirs ? \\
Pourquoi une instruction publique ? \\
Pouvoir et autorité \\
Pouvoir et contre-pouvoir \\
Pouvoir et politique \\
Pouvoir et savoir \\
Pouvoir temporel et pouvoir spirituel \\
Prendre le pouvoir \\
Prendre les armes \\
Prendre une décision politique \\
Prospérité et sécurité \\
Quand y a-t-il peuple ? \\
Que construit le politique ? \\
Que dois-je à l'État ? \\
Que fait la police ? \\
Que faut-il savoir pour gouverner ? \\
Quel est l'objet de la philosophie politique ? \\
Quel est l'objet des sciences politiques ? \\
Quelle est la spécificité de la communauté politique ? \\
Quelle valeur donner à la notion de « corps social » ? \\
Quels sont les moyens légitimes de la politique ? \\
Que nous apprend, sur la politique, l'utopie ? \\
Que peut la politique ? \\
Que peut le politique ? \\
Que peut-on attendre de l'État ? \\
Que peut-on attendre du droit international ? \\
Que serait une démocratie parfaite ? \\
Qu'est-ce que gouverner ? \\
Qu'est-ce que prendre le pouvoir ? \\
Qu'est-ce qu'être libéral ? \\
Qu'est-ce qu'être républicain ? \\
Qu'est-ce qu'être souverain ? \\
Qu'est-ce qu'être un esclave ? \\
Qu'est-ce qui est politique ? \\
Qu'est-ce qui fait la force des lois ? \\
Qu'est-ce qui fait la justice des lois ? \\
Qu'est-ce qui fait la légitimité d'une autorité politique ? \\
Qu'est-ce qui fait un peuple ? \\
Qu'est-ce qui n'est pas politique ? \\
Qu'est-ce qu'un adversaire en politique ? \\
Qu'est-ce qu'un bon citoyen ? \\
Qu'est-ce qu'un chef ? \\
Qu'est-ce qu'un conflit politique ? \\
Qu'est-ce qu'un contre-pouvoir ? \\
Qu'est-ce qu'un crime politique ? \\
Qu'est-ce qu'une communauté politique ? \\
Qu'est-ce qu'une constitution ? \\
Qu'est-ce qu'une crise politique ? \\
Qu'est-ce qu'une guerre juste ? \\
Qu'est-ce qu'une idéologie ? \\
Qu'est-ce qu'une politique sociale ? \\
Qu'est-ce qu'une violence symbolique ? \\
Qu'est-ce qu'un gouvernement ? \\
Qu'est-ce qu'un mouvement politique \\
Qu'est-ce qu'un peuple ? \\
Qu'est-ce qu'un prince juste ? \\
Qu'est-ce qu'un problème politique ? \\
Qu'est-ce qu'un programme politique ? \\
Qu'est qu'un régime politique ? \\
Qui a une parole politique ? \\
Qui est souverain ? \\
Qui gouverne ? \\
Raison et politique \\
Rapports de force, rapport de pouvoir \\
Rassembler les hommes, est-ce les unir ? \\
Réforme et révolution \\
République et démocratie \\
Résister à l'oppression \\
Résister peut-il être un droit ? \\
Revient-il à l'État d'assurer votre bonheur ? \\
Science et démocratie \\
Sécurité et liberté \\
Servir l'État \\
Société et organisme \\
Suffit-il pour être juste d'obéir aux lois et aux coutumes de son pays ? \\
Sur quoi fonder l'autorité ? \\
Surveillance et discipline \\
Toute action politique est-elle collective ? \\
Toute communauté est-elle politique ? \\
Toute hiérarchie est-elle inégalitaire ? \\
Toute philosophie implique-t-elle une politique ? \\
Tout est-il politique ? \\
Tout pouvoir est-il oppresseur ? \\
Tout pouvoir est-il politique ? \\
Tout pouvoir n'est-il pas abusif ? \\
Une décision politique peut-elle être juste ? \\
Une guerre peut-elle être juste ? \\
Une politique peut-elle se réclamer de la vie ? \\
Une société juste est-elle une société sans conflits ? \\
Une société sans conflit est-elle possible ? \\
Une société sans État est-elle une société sans politique ? \\
Un État peut-il être trop étendu ? \\
Utopie et tradition \\
Vices privés, vertus publiques \\
Vouloir l'égalité \\
Y a-t-il des compétences politiques ? \\
Y a-t-il des erreurs en politique ? \\
Y a-t-il des fondements naturels à l'ordre social ? \\
Y a-t-il des guerres justes ? \\
Y a-t-il des lois injustes ? \\
Y a-t-il un art de gouverner ? \\
Y a-t-il un bien plus précieux que la paix ? \\
Y a-t-il une compétence en politique ? \\
Y a-t-il une opinion publique mondiale ? \\
Y a-t-il une spécificité de la délibération politique ? \\
Y a-t-il un savoir du politique ? \\


\subsection{Sciences humaines}
\label{sec-3-7}

\noindent
Acteurs sociaux et usages sociaux \\
Animal politique ou social ? \\
Anomalie et anomie \\
Anthropologie et ontologie \\
Anthropologie et politique \\
Apprentissage et conditionnement \\
À quoi bon les sciences humaines et sociales ? \\
Avons-nous une identité ? \\
Castes et classes \\
Causes et motivations \\
Classes et histoire \\
Comment les sociétés changent-elles ? \\
Cultes et rituels \\
Culture et civilisation \\
Culture et conscience \\
De quelle science humaine la folie peut-elle être l'objet ? \\
De quoi les sciences humaines nous instruisent-elles ? \\
Des comportements économiques peuvent-ils être rationnels ? \\
Des motivations peuvent-elles être sociales ? \\
Des peuples sans histoire \\
Déterminisme psychique et déterminisme physique \\
Documents et monuments \\
Économie politique et politique économique \\
En quel sens l'anthropologie peut-elle être historique ? \\
En quel sens peut-on parler de la vie sociale comme d'un jeu ? \\
En quoi les sciences humaines nous éclairent-elles sur la barbarie ? \\
En quoi les sciences humaines sont-elles normatives ? \\
Enseigner, instruire, éduquer \\
Espace et structure sociale \\
Ethnologie et cinéma \\
Ethnologie et ethnocentrisme \\
Être l'entrepreneur de soi-même \\
Être mère \\
Être père \\
Famille et tribu \\
Faut-il considérer les faits sociaux comme des choses ? \\
Faut-il enfermer ? \\
Folie et société \\
Guérir \\
Histoire et anthropologie \\
Histoire et ethnologie \\
Histoire et géographie \\
Histoire et mémoire \\
Homo religiosus \\
Imitation et identification \\
Inconscient et langage \\
Individu et société \\
Information et communication \\
Interdire et prohiber \\
Interpréter et formaliser dans les sciences humaines \\
La causalité historique \\
La chasse et la guerre \\
La comédie sociale \\
La concurrence \\
La condition sociale \\
La cosmogonie \\
La criminalité \\
La crise sociale \\
L'action collective \\
La cuisine \\
La culture de masse \\
La culture d'entreprise \\
La descendance \\
La distinction de genre \\
La distinction de la nature et de la culture est-elle un fait de culture ? \\
La diversité des religions \\
La diversité humaine \\
La division des tâches \\
La division du travail \\
La domination \\
La famille \\
La fête \\
La finalité des sciences humaines \\
La force du social \\
La foule \\
La géographie \\
L'agriculture \\
La hiérarchie \\
La liberté intéresse-t-elle les sciences humaines ? \\
La littérature peut-elle suppléer les sciences de l'homme ? \\
La magie \\
La maîtrise du feu \\
L'âme concerne-t-elle les sciences humaines ? \\
La mémoire collective \\
La mémoire et l'individu \\
La mesure de l'intelligence \\
La mode \\
La modélisation en sciences sociales \\
La modernité \\
La mondialisation \\
La naissance \\
La naissance de l'homme \\
Langage, langue et parole \\
L'animisme \\
La notion d'administration \\
La notion de civilisation \\
La notion de classe dominante \\
La notion de classe sociale \\
La notion de corps social \\
La notion de loi dans les sciences de la nature et dans les sciences de l'homme \\
La notion de peuple \\
La notion d'intérêt \\
L'anthropologie est-elle une ontologie ? \\
La parenté \\
La parenté et la famille \\
La pauvreté \\
La pensée collective \\
La pensée magique \\
La pluralité des cultures \\
La pluralité des langues \\
La politesse \\
La population \\
L'appartenance sociale \\
L'apprentissage de la langue \\
La prohibition de l'inceste \\
La psychologie est-elle une science de la nature ? \\
La question sociale \\
La rationalité des comportements économiques \\
La rationalité du marché \\
La rationalité en sciences sociales \\
L'arbitraire du signe \\
L'archéologie \\
La recherche de la vérité dans les sciences humaines \\
La recherche des invariants \\
La reproduction sociale \\
La réputation \\
L'argent \\
L'argent et la valeur \\
La rumeur \\
La science des mœurs \\
La sécularisation \\
La socialisation des comportements \\
La société des savants \\
La société existe-t-elle ? \\
La sociologie de l'art nous permet-elle de comprendre l'art ? \\
La sociologie relativise-t-elle la valeur des œuvres d'art ? \\
La solitude \\
La souffrance au travail \\
La spécificité des sciences humaines \\
La structure et le sujet \\
La technologie modifie-t-elle les rapports sociaux ? \\
La tentation réductionniste \\
La théogonie \\
La tradition \\
La traduction \\
La transe \\
La transmission \\
La valeur de l'échange \\
La valeur du témoignage \\
La ville \\
La violence sociale \\
Le besoin \\
Le cannibalisme \\
Le capitalisme \\
Le capital social \\
L'échange des marchandises et les rapports humains \\
L'échange symbolique \\
L'écologie, une science humaine ? \\
Le comparatisme dans les sciences humaines \\
Le comportement \\
Le concept de pulsion \\
Le concept de structure sociale \\
Le concept d'inconscient est-il nécessaire en sciences humaines ? \\
Le conformisme social \\
L'économie a-t-elle des lois ? \\
L'économie est-elle une science humaine ? \\
L'économie politique \\
L'économie psychique \\
L'économique et le politique \\
Le contrôle social \\
Le corps humain est-il naturel ? \\
Le culte des ancêtres \\
Le déterminisme social \\
Le dialogue entre les cultures \\
Le don \\
Le droit est-il une science humaine ? \\
L'éducation physique \\
Le fait religieux \\
Le féminisme \\
Le fétichisme de la marchandise \\
L'efficacité thérapeutique de la psychanalyse \\
Le fou \\
L'égalité des sexes \\
Le jeu social \\
Le mariage \\
Le modèle organiciste \\
Le monde de l'entreprise \\
L'empathie \\
L'empathie est-elle nécessaire aux sciences sociales ? \\
Le mythe est-il objet de science ? \\
Le naturalisme des sciences humaines et sociales \\
Le nomadisme \\
L'enquête de terrain \\
L'enquête sociale \\
L'environnement est-il un nouvel objet pour les sciences humaines ? \\
Le partage des savoirs \\
Le patriarcat \\
Le patrimoine \\
Le pouvoir causal de l'inconscient \\
Le pouvoir des mots \\
Le pouvoir des sciences humaines et sociales \\
Le pouvoir traditionnel \\
Le premier et le primitif \\
Le processus de civilisation \\
Le propriétaire \\
Le psychisme est-il objet de connaissance ? \\
Le public et le privé \\
Le récit en histoire \\
Le refoulement \\
Le relativisme culturel \\
Le respect des convenances \\
Le rêve \\
Le sacré et le profane \\
Le sacrifice \\
Les affects sont-ils des objets sociologiques ? \\
Les agents sociaux poursuivent-ils l'utilité ? \\
Les agents sociaux sont-ils rationnels ? \\
Les analogies dans les sciences humaines \\
Les antagonismes sociaux \\
Les archives \\
Les classes sociales \\
Les conflits sociaux \\
Les conflits sociaux sont-ils des conflits de classe ? \\
Les coutumes \\
Les critères de vérité dans les sciences humaines \\
Les cultures sont-elles incommensurables ? \\
Les dispositions sociales \\
Les distinctions sociales \\
Les divisions sociales \\
Les études comparatives \\
Les foules \\
Les fous \\
Les frontières \\
Les hommes et les femmes \\
Les industries culturelles \\
Les inégalités sociales \\
Les interdits \\
Les invariants culturels \\
Les liens sociaux \\
Les lois du sang \\
Les marginaux \\
Les mécanismes cérébraux \\
Les prêtres \\
Les règles sociales \\
Les ressources humaines \\
Les riches et les pauvres \\
Les rituels \\
Les rôles sociaux \\
Les sacrifices \\
Les sauvages \\
Les sciences de l'éducation \\
Les sciences de l'homme et l'évolution \\
Les sciences de l'homme ont-elles inventé leur objet ? \\
Les sciences de l'homme permettent-elles d'affiner la notion de responsabilité ? \\
Les sciences de l'homme peuvent-elles expliquer l'impuissance de la liberté ? \\
Les sciences de l'homme rendent-elles l'homme prévisible ? \\
Les sciences du comportement \\
Les sciences humaines doivent-elles être transdisciplinaires ? \\
Les sciences humaines éliminent-elles la contingence du futur ? \\
Les sciences humaines et le droit \\
Les sciences humaines nous protègent-elles de l'essentialisme ? \\
Les sciences humaines ont-elles un objet commun ? \\
Les sciences humaines permettent-elles de comprendre la vie d'un homme ? \\
Les sciences humaines peuvent-elles se passer de la notion d'inconscient ? \\
Les sciences humaines présupposent-elles une définition de l'homme ? \\
Les sciences humaines sont-elles des sciences de la nature humaine ? \\
Les sciences humaines sont-elles des sciences de la vie humaine ? \\
Les sciences humaines sont-elles des sciences d'interprétation ? \\
Les sciences humaines sont-elles des sciences ? \\
Les sciences humaines sont-elles explicatives ou compréhensives ? \\
Les sciences humaines sont-elles normatives ? \\
Les sciences humaines sont-elles relativistes ? \\
Les sciences humaines sont-elles subversives ? \\
Les sciences humaines traitent-elles de l'individu ? \\
Les sciences humaines transforment-elles la notion de causalité ? \\
Les sciences peuvent-elles penser l'individu ? \\
Les sciences sociales peuvent-elles être expérimentales ? \\
Les sociétés évoluent-elles ? \\
Les sociétés ont-elles un inconscient ? \\
Les sociétés sont-elles hiérarchisables ? \\
Les sociétés sont-elles imprévisibles ? \\
Les structures expliquent-elles tout ? \\
Les traditions \\
Le système des besoins \\
Le terrain \\
Le totémisme \\
L'étranger \\
Le travail sur le terrain \\
L'événement et le fait divers \\
Le village global \\
L'évolution des langues \\
Le voyage \\
L'expérience en sciences humaines \\
L'expérimentation en sciences sociales \\
L'expertise \\
L'expression de l'inconscient \\
L'héritage \\
L'hétérogénéité sociale \\
L'histoire des civilisations \\
L'histoire est-elle déterministe ? \\
L'histoire est-elle un roman vrai ? \\
L'histoire et la géographie \\
L'histoire : enquête ou science ? \\
L'histoire : science ou récit ? \\
L'homme de la rue \\
L'homme des foules \\
L'homme des sciences de l'homme ? \\
L'homme est-il objet de science ? \\
L'idéal-type \\
L'idée de conscience collective \\
L'idée de forme sociale \\
L'inconscient \\
L'inconscient collectif \\
L'individualisme méthodologique \\
L'individuel et le collectif \\
L'individu et la multitude \\
L'individu et le groupe \\
L'initiation \\
L'institutionnalisation des conduites \\
L'institution scolaire \\
L'instrument mathématique en sciences humaines \\
L'intelligence des foules \\
L'interdit \\
L'intériorisation des normes \\
L'obéissance à l'autorité \\
L'objet de culte \\
L'obligation d'échanger \\
L'observation participante \\
L'œuvre de l'historien \\
L'opinion publique \\
L'ordre social \\
L'origine des croyances \\
L'unité des langues \\
L'unité des sciences humaines \\
L'unité des sciences humaines ? \\
L'utilité des sciences humaines \\
Machines et liberté \\
Machines et mémoire \\
Magie et religion \\
Masculin, féminin \\
Mythe et symbole \\
Mythes et idéologies \\
Nature et fonction du sacrifice \\
N'échange-t-on que des symboles ? \\
Névroses et psychoses \\
Penser les sociétés comme des organismes \\
Peuple et culture \\
Peuple et masse \\
Peut-on changer de culture ? \\
Peut-on mesurer les phénomènes sociaux ? \\
Peut-on objectiver le psychisme ? \\
Pourquoi des cérémonies ? \\
Pourquoi l'ethnologue s'intéresse-t-il à la vie urbaine ? \\
Prévoir les comportements humains \\
Primitif ou premier ? \\
Psychologie et contrôle des comportements \\
Psychologie et neurosciences \\
Quel est le sujet de l'histoire ? \\
Quelle politique fait-on avec les sciences humaines ? \\
Que nous apprend la psychanalyse de l'homme ? \\
Que nous apprend la sociologie des sciences ? \\
Que nous apprennent les algorithmes sur nos sociétés ? \\
Que nous apprennent les faits divers ? \\
Que sondent les sondages d'opinion ? \\
Qu'est-ce que lire ? \\
Qu'est-ce qu'être comportementaliste ? \\
Qu'est-ce qui rend l'objectivité difficile dans les sciences humaines ? \\
Qu'est-ce qu'un acte symbolique ? \\
Qu'est-ce qu'un capital culturel ? \\
Qu'est-ce qu'un civilisé ? \\
Qu'est-ce qu'un corps social ? \\
Qu'est-ce qu'un document ? \\
Qu'est-ce qu'une culture ? \\
Qu'est-ce qu'une époque ? \\
Qu'est-ce qu'une institution ? \\
Qu'est-ce qu'une logique sociale ? \\
Qu'est-ce qu'une mentalité collective ? \\
Qu'est-ce qu'une norme sociale ? \\
Qu'est-ce qu'une période en histoire ? \\
Qu'est-ce qu'une société mondialisée ? \\
Qu'est-ce qu'un fait de société ? \\
Qu'est-ce qu'un fait social ? \\
Qu'est-ce qu'un individu ? \\
Qu'est-ce qu'un marginal ? \\
Qu'est-ce qu'un mécanisme social ? \\
Qu'est-ce qu'un monument ? \\
Qu'est-ce qu'un mythe ? \\
Qu'est-ce qu'un patrimoine ? \\
Qu'est-ce qu'un primitif ? \\
Qu'est-ce qu'un symptôme ? \\
Qu'est-ce qu'un trouble social ? \\
Qui a une histoire ? \\
Rites et cérémonies \\
Rythmes sociaux, rythmes naturels \\
Sciences humaines et déterminisme \\
Sciences humaines et herméneutique \\
Sciences humaines et idéologie \\
Sciences humaines et liberté sont-elles compatibles ? \\
Sciences humaines et littérature \\
Sciences humaines et naturalisme \\
Sciences humaines et nature humaine \\
Sciences humaines et objectivité \\
Sciences humaines et philosophie \\
Sciences humaines, sciences de l'homme \\
Sciences sociales et humanités \\
Sens et limites de la notion de capital culturel \\
Sens et structure \\
Sexe et genre \\
Sexualité et nature \\
Signes, traces et indices \\
Sommes-nous tous contemporains ? \\
Structure et événement \\
Une culture de masse est-elle une culture ? \\
Une science de la culture est-elle possible ? \\
Voyager \\
Y a-t-il continuité ou discontinuité entre la pensée mythique et la science ? \\
Y a-t-il des lois en histoire ? \\
Y a-t-il des mentalités collectives ? \\
Y a-t-il des passions collectives ? \\
Y a-t-il des pathologies sociales ? \\
Y a-t-il des sociétés sans État ? \\
Y a-t-il des sociétés sans histoire ? \\
Y a-t-il encore des mythologies ? \\
Y a-t-il encore une sphère privée ? \\
Y a-t-il une causalité historique ? \\
Y a-t-il une intentionnalité collective ? \\
Y a-t-il une science de la vie mentale ? \\
Y a-t-il une spécificité des sciences humaines ? \\
Y a-t-il une unité en psychologie ? \\
Y a-t-il un inconscient collectif ? \\
« Comment peut-on être persan ? » \\
« Expliquer les faits sociaux par des faits sociaux » \\
« Je ne voulais pas cela » : en quoi les sciences humaines permettent-elles de comprendre cette excuse ? \\


\section{Tri par type}
\label{sec-4}

\subsection{Question}
\label{sec-4-1}

\noindent
2+2 pourrait-il ne pas être égal à 4 ? \\
Abstraire, est-ce se couper du réel ? \\
Agir justement fait-il de moi un homme juste ? \\
Agir moralement, est-ce lutter contre ses idées ? \\
Ai-je des devoirs envers moi-même ? \\
Ai-je un corps ou suis-je mon corps ? \\
Ai-je un corps ? \\
Ai-je une âme ? \\
Aimer, est-ce vraiment connaître ? \\
Aimer peut-il être un devoir ? \\
Animal politique ou social ? \\
Appartenons-nous à une culture ? \\
Apprend-on à aimer ? \\
Apprend-on à être artiste ? \\
Apprend-on à penser ? \\
Apprend-on à percevoir ? \\
Apprend-on à voir ? \\
Apprendre s'apprend-il ? \\
À quelle expérience l'art nous convie-t-il ? \\
À quelles conditions une démarche est-elle scientifique ? \\
À quelles conditions une expérience est-elle possible ? \\
À quelles conditions une hypothèse est-elle scientifique ? \\
À quelles conditions un énoncé est-il doué de sens ? \\
À quoi bon discuter ? \\
À quoi bon imiter la nature ? \\
À quoi bon les sciences humaines et sociales ? \\
À quoi bon penser la fin du monde ? \\
À quoi bon voyager ? \\
À quoi bon ? \\
À quoi faut-il renoncer ? \\
À quoi juger l'action d'un gouvernement ? \\
À quoi la conscience nous donne-t-elle accès ? \\
À quoi la logique peut-elle servir dans les sciences ? \\
À quoi nos illusions tiennent-elles ? \\
À quoi reconnaît-on la vérité ? \\
À quoi reconnaît-on qu'une expérience est scientifique ? \\
À quoi reconnaît-on qu'une théorie est scientifique ? \\
À quoi reconnaît-on qu'un événement est historique ? \\
À quoi reconnaît-on une œuvre d'art ? \\
À quoi reconnaît-on une religion ? \\
À quoi reconnaît-on un être vivant ? \\
À quoi sert la dialectique ? \\
À quoi sert la négation ? \\
À quoi sert la technique ? \\
À quoi sert l'écriture ? \\
À quoi sert l'État ? \\
À quoi servent les mythes ? \\
À quoi servent les preuves de l'existence de Dieu ? \\
À quoi servent les religions ? \\
À quoi servent les sciences ? \\
À quoi servent les utopies ? \\
À quoi tient la fermeté du vouloir ? \\
À quoi tient la force de l'État ? \\
À quoi tient la force des religions ? \\
À quoi tient la vérité d'une interprétation ? \\
A-t-on besoin de certitudes ? \\
A-t-on besoin de fonder la connaissance ? \\
A-t-on besoin de spécialistes en politique ? \\
A-t-on besoin d'experts ? \\
A-t-on besoin d'un chef ? \\
A-t-on des devoirs envers soi-même ? \\
A-t-on des droits contre l'État ? \\
A-t-on des raisons de croire ce qu'on croit ? \\
A-t-on des raisons de croire ? \\
A-t-on intérêt à tout savoir ? \\
A-t-on le droit de mentir ? \\
A-t-on le droit de résister ? \\
A-t-on le droit de se révolter ? \\
A-t-on le droit de s'évader ? \\
A-t-on l'obligation de pardonner ? \\
Au-delà de la nature ? \\
Au nom de qui rend-on justice ? \\
Au nom de quoi le plaisir serait-il condamnable ? \\
Au nom de quoi rend-on justice ? \\
Autrui, est-ce n'importe quel autre ? \\
Autrui est-il aimable ? \\
Autrui est-il inconnaissable ? \\
Autrui est-il mon semblable ? \\
Autrui est-il pour moi un mystère ? \\
Autrui est-il un autre moi-même ? \\
Autrui est-il un autre moi ? \\
Autrui me connaît-il mieux que moi-même ? \\
Autrui m'est-il étranger ? \\
Avez-vous une âme ? \\
Avons-nous à apprendre des images ? \\
Avons-nous besoin d'amis ? \\
Avons-nous besoin de cérémonies ? \\
Avons-nous besoin de héros ? \\
Avons-nous besoin de maîtres ? \\
Avons-nous besoin de métaphysique ? \\
Avons-nous besoin de partis politiques ? \\
Avons-nous besoin de rêver ? \\
Avons-nous besoin de spectacles ? \\
Avons-nous besoin de traditions ? \\
Avons-nous besoin d'experts en matière d'art ? \\
Avons-nous besoin d'un libre arbitre ? \\
Avons-nous besoin d'utopies ? \\
Avons-nous des devoirs à l'égard de la vérité ? \\
Avons-nous des devoirs envers la nature ? \\
Avons-nous des devoirs envers les animaux ? \\
Avons-nous des devoirs envers les autres êtres vivants ? \\
Avons-nous des devoirs envers les générations futures ? \\
Avons-nous des devoirs envers les morts ? \\
Avons-nous des devoirs envers le vivant ? \\
Avons-nous des devoirs envers nous-mêmes ? \\
Avons-nous des droits sur la nature ? \\
Avons-nous des raisons d'espérer ? \\
Avons-nous intérêt à la liberté d'autrui ? \\
Avons-nous le devoir d'être heureux ? \\
Avons-nous le devoir de vivre ? \\
Avons-nous le droit de juger autrui ? \\
Avons-nous le droit d'être heureux ? \\
Avons-nous peur de la liberté ? \\
Avons-nous raison d'exiger toujours des raisons ? \\
Avons-nous un corps ? \\
Avons-nous un devoir de vérité ? \\
Avons-nous un droit au droit ? \\
Avons-nous une âme ? \\
Avons-nous une identité ? \\
Avons-nous une intuition du temps ? \\
Avons-nous une obligation envers les générations à venir ? \\
Avons-nous une responsabilité envers le passé ? \\
Avons-nous un libre arbitre ? \\
Avons-nous un monde commun ? \\
Axiomatiser, est-ce fonder ? \\
À quelle condition un travail est-il humain ? \\
À quelles conditions est-il acceptable de travailler ? \\
À quelles conditions le vivant peut-il être objet de science ? \\
À quelles conditions peut-on dire qu'une action est historique ? \\
À quelles conditions un choix peut-il être rationnel ? \\
À quelles conditions une démarche est-elle scientifique ? \\
À quelles conditions une explication est-elle scientifique ? \\
À quelles conditions une hypothèse est-elle scientifique ? \\
À quelles conditions une pensée est-elle libre ? \\
À quelles conditions une théorie est-elle scientifique ? \\
À quelles conditions une théorie peut-elle être scientifique ? \\
À quelles conditions un jugement est-il moral ? \\
À quels signes reconnaît-on la vérité ? \\
À qui devons-nous obéir ? \\
À qui dois-je la vérité ? \\
À qui doit-on le respect ? \\
À qui doit-on obéir ? \\
À qui est mon corps ? \\
À qui faut-il obéir ? \\
À qui la faute ? \\
À qui profite le crime ? \\
À qui profite le travail ? \\
À quoi bon avoir mauvaise conscience ? \\
À quoi bon critiquer les autres ? \\
À quoi bon démontrer ? \\
À quoi bon les regrets ? \\
À quoi bon les romans ? \\
À quoi bon raconter des histoires ? \\
À quoi bon se parler ? \\
À quoi bon voyager ? \\
À quoi est-il impossible de s'habituer ? \\
À quoi faut-il être fidèle ? \\
À quoi la perception donne-t-elle accès ? \\
À quoi l'art nous rend-il sensibles ? \\
À quoi la valeur d'un homme se mesure-t-elle ? \\
À quoi peut-on reconnaître une œuvre d'art ? \\
À quoi reconnaît-on la rationalité ? \\
À quoi reconnaît-on la vérité ? \\
À quoi reconnaît-on le réel ? \\
À quoi reconnaît-on l'injustice ? \\
À quoi reconnaît-on qu'une activité est un travail ? \\
À quoi reconnaît-on qu'une expérience est scientifique ? \\
À quoi reconnaît-on qu'une pensée est vraie ? \\
À quoi reconnaît-on qu'une politique est juste ? \\
À quoi reconnaît-on un acte libre ? \\
À quoi reconnaît-on un bon gouvernement ? \\
À quoi reconnaît-on une attitude religieuse ? \\
À quoi reconnaît-on une bonne interprétation ? \\
À quoi reconnaît-on une idéologie ? \\
À quoi reconnaît-on une œuvre d'art ? \\
À quoi sert la connaissance du passé ? \\
À quoi sert la logique ? \\
À quoi sert la mémoire ? \\
À quoi sert la notion de contrat social ? \\
À quoi sert la notion d'état de nature ? \\
À quoi sert la technique ? \\
À quoi sert le contrat social ? \\
À quoi sert l'État ? \\
À quoi sert l'histoire ? \\
À quoi sert un exemple ? \\
À quoi servent les doctrines morales ? \\
À quoi servent les élections ? \\
À quoi servent les émotions ? \\
À quoi servent les expériences ? \\
À quoi servent les fictions ? \\
À quoi servent les images ? \\
À quoi servent les lois ? \\
À quoi servent les machines ? \\
À quoi servent les mythes ? \\
À quoi servent les preuves ? \\
À quoi servent les règles ? \\
À quoi servent les statistiques ? \\
À quoi servent les symboles ? \\
À quoi servent les théories ? \\
À quoi servent les utopies ? \\
À quoi servent les voyages ? \\
À quoi tenons-nous ? \\
À quoi tient la force des religions ? \\
À quoi tient l'autorité ? \\
À quoi tient la valeur d'une pensée ? \\
À quoi tient le pouvoir des mots ? \\
À quoi tient notre humanité ? \\
À science nouvelle, nouvelle philosophie ? \\
À t-on le droit de faire tout ce qui est permis par la loi ? \\
Bien agir, est-ce toujours être moral ? \\
Ce que je pense est-il nécessairement vrai ? \\
Ce que la morale autorise, l'État peut-il légitimement l'interdire ? \\
Ce que la technique rend possible, peut-on jamais en empêcher la réalisation ? \\
Ce qui dépasse la raison est-il nécessairement irréel ? \\
Ce qui est démontré est-il nécessairement vrai ? \\
Ce qui est faux est-il dénué de sens ? \\
Ce qui est ordinaire est-il normal ? \\
Ce qui est subjectif est-il arbitraire ? \\
Ce qui est vrai est-il toujours vérifiable ? \\
Ce qui ne peut s'acheter est-il dépourvu de valeur ? \\
Ce qui n'est pas démontré peut-il être vrai ? \\
Ce qui n'est pas matériel peut-il être réel ? \\
Ce qui n'est pas réel est-il impossible ? \\
Ce qui vaut en théorie vaut-il toujours en pratique ? \\
Certaines œuvres d'art ont-elles plus de valeur que d'autres ? \\
Ceux qui savent doivent-ils gouverner ? \\
Changer, est-ce devenir un autre ? \\
Change-t-on avec le temps ? \\
Chaque science porte-t-elle une métaphysique qui lui est propre ? \\
Châtier, est ce faire honneur au criminel ? \\
Chercher son intérêt, est-ce être immoral ? \\
Choisir, est-ce renoncer ? \\
Choisir ses souvenirs ? \\
Choisissons-nous qui nous sommes ? \\
Choisit-on ses souvenirs ? \\
Choisit-on son corps ? \\
Cité juste ou citoyen juste ? \\
Citoyen du monde ? \\
Comment assumer les conséquences de ses actes ? \\
Comment autrui peut-il m'aider à rechercher le bonheur ? \\
Comment bien vivre ? \\
Comment chercher ce qu'on ignore ? \\
Comment comprendre les faits sociaux ? \\
Comment comprendre une croyance qu'on ne partage pas ? \\
Comment conduire ses pensées ? \\
Comment connaître nos devoirs ? \\
Comment croire au progrès ? \\
Comment décider, sinon à la majorité ? \\
Comment définir la raison ? \\
Comment deux personnes peuvent-elles partager la même pensée ? \\
Comment devient-on artiste ? \\
Comment devient-on raisonnable ? \\
Comment dire la vérité ? \\
Comment dire l'individuel ? \\
Comment distinguer désirs et besoins ? \\
Comment distinguer entre l'amour et l'amitié ? \\
Comment distinguer l'amour de l'amitié ? \\
Comment distinguer le rêvé du perçu ? \\
Comment distinguer le vrai du faux ? \\
Comment établir des critères d'équité ? \\
Comment être naturel ? \\
Comment évaluer la qualité de la vie ? \\
Comment expliquer les phénomènes mentaux ? \\
Comment exprimer l'identité ? \\
Comment fonder la propriété ? \\
Comment juger de la justesse d'une interprétation ? \\
Comment juger de la politique ? \\
Comment juger d'une œuvre d'art ? \\
Comment justifier l'autonomie des sciences de la vie ? \\
Comment la science progresse-t-elle ? \\
Comment le passé peut-il demeurer présent ? \\
Comment l'erreur est-elle possible ? \\
Comment les sociétés changent-elles ? \\
Comment l'homme peut-il se représenter le temps ? \\
Comment mesurer une sensation ? \\
Comment mesurer ? \\
Comment ne pas être humaniste ? \\
Comment ne pas être libéral ? \\
Comment penser la diversité des langues ? \\
Comment penser le hasard ? \\
Comment penser le mouvement ? \\
Comment penser l'éternel ? \\
Comment penser un pouvoir qui ne corrompe pas ? \\
Comment percevons-nous l'espace ? \\
Comment peut-on choisir entre différentes hypothèses ? \\
Comment peut-on définir la politique ? \\
Comment peut-on définir un être vivant ? \\
Comment peut-on être heureux ? \\
Comment peut-on être sceptique ? \\
Comment peut-on se trahir soi-même ? \\
Comment puis-je devenir ce que je suis ? \\
Comment reconnaît-on une œuvre d'art ? \\
Comment reconnaît-on un vivant ? \\
Comment retrouver la nature ? \\
Comment sait-on qu'on se comprend ? \\
Comment sait-on qu'une chose existe ? \\
Comment savoir que l'on est dans l'erreur ? \\
Comment se mettre à la place d'autrui ? \\
Comment s'entendre ? \\
Comment s'orienter dans la pensée ? \\
Comment traiter les animaux ? \\
Comment trancher une controverse ? \\
Comment vivre ensemble ? \\
Comment voyager dans le temps ? \\
Comprendre, est-ce interpréter ? \\
Comprendre le réel est-ce le dominer ? \\
Connaissons-nous la réalité des choses ? \\
Connaissons-nous la réalité telle qu'elle est ? \\
Connaissons-nous mieux le présent que le passé ? \\
Connaît-on la vie ou bien connaît-on le vivant ? \\
Connaît-on la vie ou connaît-on le vivant ? \\
Connaît-on la vie ou le vivant ? \\
Connaît-on les choses telles qu'elles sont ? \\
Connaître, est-ce connaître par les causes ? \\
Connaître est-ce découvrir le réel ? \\
Connaître, est-ce dépasser les apparences ? \\
Connaître la vie ou le vivant ? \\
Considère-t-on jamais le temps en lui-même ? \\
Convient-il d'opposer explication et interprétation ? \\
Croire, est-ce être faible ? \\
Croire, est-ce obéir ? \\
Croire, est-ce renoncer au savoir ? \\
Croire que Dieu existe, est-ce croire en lui ? \\
Croit-on ce que l'on veut ? \\
Croit-on comme on veut ? \\
Dans l'action, est-ce l'intention qui compte ? \\
Dans quel but les hommes se donnent-ils des lois ? \\
Dans quelle mesure est-on l'auteur de sa propre vie ? \\
Dans quelle mesure l'art est-il un fait social ? \\
Dans quelle mesure le temps nous appartient-il ? \\
Dans quelle mesure toute philosophie est-elle critique du langage ? \\
Décrire, est-ce déjà expliquer ? \\
Définir, est-ce déterminer l'essence ? \\
Définir l'art : à quoi bon ? \\
Définir la vérité, est-ce la connaître ? \\
Délibérer, est-ce être dans l'incertitude ? \\
Démontrer est-il le privilège du mathématicien ? \\
Dépasser les apparences ? \\
Dépend-il de soi d'être heureux ? \\
De quel bonheur sommes-nous capables ? \\
De quel droit l'État exerce-t-il un pouvoir ? \\
De quel droit punit-on ? \\
De quel droit ? \\
De quelle certitude la science est-elle capable ? \\
De quelle liberté témoigne l'œuvre d'art ? \\
De quelle réalité nos perceptions témoignent-elles ? \\
De quelle réalité témoignent nos perceptions ? \\
De quelle science humaine la folie peut-elle être l'objet ? \\
De quelle vérité l'art est-il capable ? \\
De quelle vérité l'opinion est-elle capable ? \\
De quoi a-t-on conscience lorsqu'on a conscience de soi ? \\
De quoi avons-nous besoin ? \\
De quoi avons-nous vraiment besoin ? \\
De quoi dépend le bonheur ? \\
De quoi dépend notre bonheur ? \\
De quoi doute un sceptique ? \\
De quoi est fait mon présent ? \\
De quoi est-fait notre présent ? \\
De quoi est-on conscient ? \\
De quoi est-on malheureux ? \\
De quoi la forme est-elle la forme ? \\
De quoi la logique est-elle la science ? \\
De quoi la philosophie est-elle le désir ? \\
De quoi l'art nous délivre-t-il ? \\
De quoi la vérité libère-t-elle ? \\
De quoi le devoir libère-t-il ? \\
De quoi les logiciens parlent-ils ? \\
De quoi les métaphysiciens parlent-ils ? \\
De quoi les sciences humaines nous instruisent-elles ? \\
De quoi l'État doit-il être propriétaire ? \\
De quoi l'État ne doit-il pas se mêler ? \\
De quoi l'expérience esthétique est-elle l'expérience ? \\
De quoi n'avons-nous pas conscience ? \\
De quoi ne peut-on pas répondre ? \\
De quoi parlent les mathématiques ? \\
De quoi parlent les théories physiques ? \\
De quoi pâtit-on ? \\
De quoi peut-il y avoir science ? \\
De quoi peut-on être inconscient ? \\
De quoi peut-on faire l'expérience ? \\
De quoi pouvons-nous être sûrs ? \\
De quoi puis-je répondre ? \\
De quoi rit-on ? \\
De quoi somme-nous prisonniers ? \\
De quoi sommes-nous coupables ? \\
De quoi sommes-nous responsables ? \\
De quoi suis-je inconscient ? \\
De quoi suis-je responsable ? \\
De quoi y a-t-il expérience ? \\
De quoi y a-t-il histoire ? \\
Déraisonner, est-ce perdre de vue le réel ? \\
Des comportements économiques peuvent-ils être rationnels ? \\
Des événements aléatoires peuvent-ils obéir à des lois ? \\
Des inégalités peuvent-elles être justes ? \\
Désirer, est-ce être aliéné ? \\
Désire-t-on la reconnaissance ? \\
Des lois justes suffisent-elles à assurer la justice ? \\
Des motivations peuvent-elles être sociales ? \\
Des nations peuvent-elles former une société ? \\
Des sociétés sans État sont-elles des sociétés politiques ? \\
Devant qui est-on responsable ? \\
Devant qui sommes-nous responsables ? \\
Devient-on raisonnable ? \\
Devons-nous dire la vérité ? \\
Devons-nous nous libérer de nos désirs ? \\
Devons-nous quelque chose à la nature ? \\
Devons-nous tenir certaines connaissances pour acquises ? \\
Devons-nous vivre comme si nous ne devions jamais mourir ? \\
Dieu aurait-il pu mieux faire ? \\
Dieu est-il mortel ? \\
Dieu est-il une invention humaine ? \\
Dieu est-il une limite de la pensée ? \\
Dieu pense-t-il ? \\
Dieu peut-il tout faire ? \\
Dieu, prouvé ou éprouvé ? \\
Dire, est-ce faire ? \\
Dois-je mériter mon bonheur ? \\
Dois-je obéir à la loi ? \\
Doit-on apprendre à percevoir ? \\
Doit-on apprendre à vivre ? \\
Doit-on bien juger pour bien faire ? \\
Doit-on changer ses désirs, plutôt que l'ordre du monde ? \\
Doit-on chasser les artistes de la cité ? \\
Doit-on corriger les inégalités sociales ? \\
Doit-on croire en l'humanité ? \\
Doit-on distinguer devoir moral et obligation sociale ? \\
Doit-on identifier l'âme à la conscience ? \\
Doit-on interpréter les rêves ? \\
Doit-on justifier les inégalités ? \\
Doit-on le respect au vivant ? \\
Doit-on mûrir pour la liberté ? \\
Doit-on rechercher le bonheur ? \\
Doit-on rechercher l'harmonie ? \\
Doit-on refuser d'interpréter ? \\
Doit-on répondre de ce qu'on est devenu ? \\
Doit-on respecter la nature ? \\
Doit-on respecter les êtres vivants ? \\
Doit-on se faire l'avocat du diable ? \\
Doit-on se justifier d'exister ? \\
Doit-on se mettre à la place d'autrui ? \\
Doit-on se passer des utopies ? \\
Doit-on souffrir de n'être pas compris ? \\
Doit-on tenir le plaisir pour une fin ? \\
Doit-on toujours dire la vérité ? \\
Doit-on toujours rechercher la vérité ? \\
Doit-on tout accepter de l'État ? \\
Doit-on tout attendre de l'État ? \\
Doit-on tout calculer ? \\
Doit-on tout contrôler ? \\
Donner, à quoi bon ? \\
D'où la politique tire-t-elle sa légitimité ? \\
D'où viennent les concepts ? \\
D'où viennent les idées générales ? \\
D'où viennent les préjugés ? \\
D'où viennent nos idées ? \\
D'où vient aux objets techniques leur beauté ? \\
D'où vient la certitude dans les sciences ? \\
D'où vient la certitude ? \\
D'où vient la servitude ? \\
D'où vient la signification des mots ? \\
D'où vient le mal ? \\
D'où vient le plaisir de lire ? \\
D'où vient que l'histoire soit autre chose qu'un chaos ? \\
Droit et devoir sont-ils liés ? \\
Droits de l'homme ou droits du citoyen ? \\
Droits et devoirs sont-ils réciproques ? \\
Du passé pouvons-nous faire table rase ? \\
Échanger, est-ce créer de la valeur ? \\
Échanger, est-ce partager ? \\
Échanger, est-ce risquer ? \\
En histoire, tout est-il affaire d'interprétation ? \\
En morale, est-ce seulement l'intention qui compte ? \\
En politique, faut-il refuser l'utopie ? \\
En politique, ne faut-il croire qu'aux rapports de force ? \\
En politique n'y a-t-il que des rapports de force ? \\
En politique, peut-on faire table rase du passé ? \\
En politique, y a-t-il des modèles ? \\
En quel sens la métaphysique a-t-elle une histoire ? \\
En quel sens la métaphysique est-elle une science ? \\
En quel sens l'anthropologie peut-elle être historique ? \\
En quel sens les sciences ont-elles une histoire ? \\
En quel sens l'État est-il rationnel ? \\
En quel sens le vivant a-t-il une histoire ? \\
En quel sens parler de lois de la pensée ? \\
En quel sens parler de structure métaphysique ? \\
En quel sens parler d'identité culturelle ? \\
En quel sens peut-on dire que la vérité s'impose ? \\
En quel sens peut-on dire que le mal n'existe pas ? \\
En quel sens peut-on dire que l'homme est un animal politique ? \\
En quel sens peut-on dire qu' « on expérimente avec sa raison » ? \\
En quel sens peut-on parler de la mort de l'art ? \\
En quel sens peut-on parler de la vie sociale comme d'un jeu ? \\
En quel sens peut-on parler de transcendance ? \\
En quel sens peut-on parler d'expérience possible ? \\
En quel sens peut-on parler d'une culture technique ? \\
En quel sens peut-on parler d'une interprétation de la nature ? \\
En quel sens une œuvre d'art est-elle un document ? \\
En quoi la connaissance de la matière peut-elle relever de la métaphysique ? \\
En quoi la connaissance du vivant contribue-t-elle à la connaissance de l'homme ? \\
En quoi la justice met-elle fin à la violence ? \\
En quoi la matière s'oppose-t-elle à l'esprit ? \\
En quoi la méthode est-elle un art de penser ? \\
En quoi la nature constitue-t-elle un modèle ? \\
En quoi la patience est-elle une vertu ? \\
En quoi la physique a-t-elle besoin des mathématiques ? \\
En quoi l'art peut-il intéresser le philosophe ? \\
En quoi la sociologie est-elle fondamentale ? \\
En quoi la technique fait-elle question ? \\
En quoi le bien d'autrui m'importe-t-il ? \\
En quoi le bonheur est-il l'affaire de l'État ? \\
En quoi le langage est-il constitutif de l'homme ? \\
En quoi les hommes restent-ils des enfants ? \\
En quoi les sciences humaines nous éclairent-elles sur la barbarie ? \\
En quoi les sciences humaines sont-elles normatives ? \\
En quoi les vivants témoignent-ils d'une histoire ? \\
En quoi l'œuvre d'art donne-t-elle à penser ? \\
En quoi une discussion est-elle politique ? \\
En quoi une insulte est-elle blessante ? \\
En quoi une œuvre d'art est-elle moderne ? \\
Enseigner, est-ce transmettre un savoir ? \\
Entre l'art et la nature, qui imite l'autre ? \\
Entre l'opinion et la science, n'y a-t-il qu'une différence de degré ? \\
Est-ce à la fin que le sens apparaît ? \\
Est-ce à la raison de déterminer ce qui est réel ? \\
Est-ce de la force que l'État tient son autorité ? \\
Est-ce la certitude qui fait la science ? \\
Est-ce la démonstration qui fait la science ? \\
Est-ce la majorité qui doit décider ? \\
Est-ce la mémoire qui constitue mon identité ? \\
Est-ce l'autorité qui fait la loi ? \\
Est-ce le cerveau qui pense ? \\
Est-ce l'échange utilitaire qui fait le lien social ? \\
Est-ce le corps qui perçoit ? \\
Est-ce l'ignorance qui rend les hommes croyants ? \\
Est-ce l'intérêt qui fonde le lien social ? \\
Est-ce l'utilité qui définit un objet technique ? \\
Est-ce par désir de la vérité que l'homme cherche à savoir ? \\
Est-ce par son objet ou par ses méthodes qu'une science peut se définir ? \\
Est-ce pour des raisons morales qu'il faut protéger l'environnement ? \\
Est-ce seulement l'intention qui compte ? \\
Est-ce un devoir d'aimer son prochain ? \\
Est-il bon qu'un seul commande ? \\
Est-il difficile de savoir ce que l'on veut ? \\
Est-il difficile de savoir ce qu'on veut ? \\
Est-il difficile d'être heureux ? \\
Est-il difficile de vivre en société ? \\
Est-il immoral de se rendre heureux ? \\
Est-il judicieux de revenir sur ses décisions ? \\
Est-il juste de payer l'impôt ? \\
Est-il juste d'interpréter la loi ? \\
Est-il légitime d'affirmer que seul le présent existe ? \\
Est-il légitime d'opposer liberté et nécessité ? \\
Est-il mauvais de suivre son désir ? \\
Est-il naturel à l'homme de parler ? \\
Est-il naturel de s'aimer soi-même ? \\
Est-il nécessaire d'espérer pour entreprendre ? \\
Est-il parfois bon de mentir ? \\
Est-il possible d'améliorer l'homme ? \\
Est-il possible de croire en la vie éternelle ? \\
Est-il possible de douter de tout ? \\
Est-il possible de ne croire à rien ? \\
Est-il possible de préparer l'avenir ? \\
Est-il possible de tout avoir pour être heureux ? \\
Est-il possible d'être immoral sans le savoir ? \\
Est-il possible d'être neutre politiquement ? \\
Est-il raisonnable d'aimer ? \\
Est-il raisonnable d'être rationnel ? \\
Est-il raisonnable de vouloir maîtriser la nature ? \\
Est-il toujours avantageux de promouvoir son propre intérêt ? \\
Est-il toujours meilleur d'avoir le choix ? \\
Est-il utile d'avoir mal ? \\
Est-il vrai que les animaux ne pensent pas ? \\
Est-il vrai que l'ignorant n'est pas libre ? \\
Est-il vrai que ma liberté s'arrête là où commence celle des autres ? \\
Est-il vrai que nous ne nous tenons jamais au temps présent ? \\
Est-il vrai qu'en science, « rien n'est donné, tout est construit » ? \\
Est-il vrai que plus on échange, moins on se bat ? \\
Est-il vrai qu'on apprenne de ses erreurs ? \\
Est-on fondé à distinguer la justice et le droit ? \\
Est-on l'auteur de sa propre vie ? \\
Est-on le produit d'une culture ? \\
Est-on libre de ne pas vouloir ce que l'on veut ? \\
Est-on libre face à la vérité ? \\
Est-on responsable de ce qu'on n'a pas voulu ? \\
Est-on responsable de son passé ? \\
Est-on sociable par nature ? \\
Établir la vérité, est-ce nécessairement démontrer ? \\
Être à l'écoute de son désir, est-ce nier le désir de l'autre ? \\
Être conscient de soi, est-ce être maître de soi ? \\
Être conscient, est-ce être maître de soi ? \\
Être cultivé, est-ce tout connaître ? \\
Être cultivé rend-il meilleur ? \\
Être est-ce agir ? \\
Être et penser, est-ce la même chose ? \\
Être heureux, est-ce devoir ? \\
Être libre, est-ce dire non ? \\
Être libre est-ce faire ce que l'on veut ? \\
Être libre, est-ce n'obéir qu'à soi-même ? \\
Être libre, est-ce pouvoir choisir ? \\
Être libre, est-ce se suffire à soi-même ? \\
Être ou ne pas être, est-ce la question ? \\
Être raisonnable, est-ce accepter la réalité telle qu'elle est ? \\
Être raisonnable, est-ce renoncer à ses désirs ? \\
Être un sujet, est-ce être maître de soi ? \\
Exister, est-ce simplement vivre ? \\
Existe-t-il de faux besoins ? \\
Existe-t-il des choses en soi ? \\
Existe-t-il des choses sans prix ? \\
Existe-t-il des croyances collectives ? \\
Existe-t-il des désirs coupables ? \\
Existe-t-il des devoirs envers soi-même ? \\
Existe-t-il des dilemmes moraux ? \\
Existe-t-il des questions sans réponse ? \\
Existe-t-il des sciences de différentes natures ? \\
Existe-t-il des signes naturels ? \\
Existe-t-il un art de penser ? \\
Existe-t-il un bien commun qui soit la norme de la vie politique ? \\
Existe-t-il un droit de mentir ? \\
Existe-t-il une méthode pour rechercher la vérité ? \\
Existe-t-il une méthode pour trouver la vérité ? \\
Existe-t-il une opinion publique ? \\
Existe-t-il un vocabulaire neutre des droits fondamentaux ? \\
Expliquer, est-ce interpréter ? \\
Faire de la métaphysique, est-ce se détourner du monde ? \\
Faire son devoir, est-ce là toute la morale ? \\
Faisons-nous l'histoire ? \\
Fait-on de la politique pour changer les choses ? \\
Faudrait-il ne rien oublier ? \\
Faudrait-il vivre sans passion ? \\
Faut-avoir peur de la technique ? \\
Faut-il accepter sa condition ? \\
Faut-il accorder de l'importance aux mots ? \\
Faut-il accorder l'esprit aux bêtes ? \\
Faut-il affirmer son identité ? \\
Faut-il aimer autrui pour le respecter ? \\
Faut-il aimer la vie ? \\
Faut-il aimer son prochain comme soi-même ? \\
Faut-il aimer son prochain ? \\
Faut-il aller au-delà des apparences ? \\
Faut-il aller toujours plus vite ? \\
Faut-il apprendre à être libre ? \\
Faut-il apprendre à vivre en renonçant au bonheur ? \\
Faut-il apprendre à voir ? \\
Faut-il avoir des ennemis ? \\
Faut-il avoir des principes ? \\
Faut-il avoir peur de la liberté ? \\
Faut-il avoir peur de la nature ? \\
Faut-il avoir peur de la technique ? \\
Faut-il avoir peur des habitudes ? \\
Faut-il avoir peur des machines ? \\
Faut-il avoir peur d'être libre ? \\
Faut-il avoir peur du désordre ? \\
Faut-il changer le monde ? \\
Faut-il changer ses désirs plutôt que l'ordre du monde ? \\
Faut-il chasser les poètes ? \\
Faut-il chercher à satisfaire tous nos désirs ? \\
Faut-il chercher à se connaître ? \\
Faut-il chercher la paix à tout prix ? \\
Faut-il chercher le bonheur à tout prix ? \\
Faut-il chercher un sens à l'histoire ? \\
Faut-il choisir entre être heureux et être libre ? \\
Faut-il concilier les contraires ? \\
Faut-il condamner la fiction ? \\
Faut-il condamner la rhétorique ? \\
Faut-il condamner le luxe ? \\
Faut-il condamner les illusions ? \\
Faut-il connaître l'Histoire pour gouverner ? \\
Faut-il considérer le droit pénal comme instituant une violence légitime ? \\
Faut-il considérer les faits sociaux comme des choses ? \\
Faut-il contrôler les mœurs ? \\
Faut-il craindre la mort ? \\
Faut-il craindre la révolution ? \\
Faut-il craindre le développement des techniques ? \\
Faut-il craindre le pire ? \\
Faut-il craindre le regard d'autrui ? \\
Faut-il craindre les foules ? \\
Faut-il craindre les machines ? \\
Faut-il craindre les masses ? \\
Faut-il craindre l'État ? \\
Faut-il craindre l'ordre ? \\
Faut-il croire au progrès ? \\
Faut-il croire en la science ? \\
Faut-il croire en quelque chose ? \\
Faut-il croire les historiens ? \\
Faut-il croire que l'histoire a un sens ? \\
Faut-il défendre la démocratie ? \\
Faut-il défendre l'ordre à tout prix ? \\
Faut-il dépasser les apparences ? \\
Faut-il désespérer de l'humanité ? \\
Faut-il des frontières ? \\
Faut-il des héros ? \\
Faut-il désirer la vérité ? \\
Faut-il des outils pour penser ? \\
Faut-il détruire l'État ? \\
Faut-il détruire pour créer ? \\
Faut-il dire de la justice qu'elle n'existe pas ? \\
Faut-il dire tout haut ce que les autres pensent tout bas ? \\
Faut-il diriger l'économie ? \\
Faut-il distinguer ce qui est de ce qui doit être ? \\
Faut-il distinguer désir et besoin ? \\
Faut-il distinguer esthétique et philosophie de l'art ? \\
Faut-il donner un sens à la souffrance ? \\
Faut-il douter de ce qu'on ne peut pas démontrer ? \\
Faut-il du passé faire table rase ? \\
Faut-il enfermer ? \\
Faut-il espérer pour agir ? \\
Faut-il être à l'écoute du corps ? \\
Faut-il être bon ? \\
Faut-il être cohérent ? \\
Faut-il être connaisseur pour apprécier une œuvre d'art ? \\
Faut-il être cosmopolite ? \\
Faut-il être fidèle à soi-même ? \\
Faut-il être idéaliste ? \\
Faut-il être libre pour être heureux ? \\
Faut-il être logique avec soi-même ? \\
Faut-il être mesuré en toutes choses ? \\
Faut-il être modéré ? \\
Faut-il être objectif ? \\
Faut-il être original ? \\
Faut-il être positif ? \\
Faut-il être pragmatique ? \\
Faut-il être réaliste en politique ? \\
Faut-il être réaliste ? \\
Faut-il expliquer la morale par son utilité ? \\
Faut-il faire confiance au progrès technique ? \\
Faut-il faire de nécessité vertu ? \\
Faut-il faire table rase du passé ? \\
Faut-il forcer les gens à participer à la vie politique ? \\
Faut-il fuir la politique ? \\
Faut-il garder ses illusions ? \\
Faut-il hiérarchiser les désirs ? \\
Faut-il hiérarchiser les formes de vie ? \\
Faut-il imaginer que nous sommes heureux ? \\
Faut-il imposer la vérité ? \\
Faut-il interpréter la loi ? \\
Faut-il laisser parler la nature ? \\
Faut-il libérer l'humanité du travail ? \\
Faut-il limiter la souveraineté de l'État ? \\
Faut-il limiter la souveraineté ? \\
Faut-il limiter le pouvoir de l'État ? \\
Faut-il limiter l'exercice de la puissance publique ? \\
Faut-il lire des romans ? \\
Faut-il ménager les apparences ? \\
Faut-il mépriser le luxe ? \\
Faut-il mieux vivre comme si nous ne devions jamais mourir ? \\
Faut-il ne manquer de rien pour être heureux ? \\
Faut-il n'être jamais méchant ? \\
Faut-il obéir à la voix de sa conscience ? \\
Faut-il opposer à la politique la souveraineté du droit ? \\
Faut-il opposer histoire et mémoire ? \\
Faut-il opposer la matière et l'esprit ? \\
Faut-il opposer l'art à la connaissance ? \\
Faut-il opposer la théorie et la pratique ? \\
Faut-il opposer le don et l'échange ? \\
Faut-il opposer l'État et la société ? \\
Faut-il opposer le temps vécu et le temps des choses ? \\
Faut-il opposer l'histoire et la fiction ? \\
Faut-il opposer nature et culture ? \\
Faut-il opposer raison et sensation ? \\
Faut-il opposer rhétorique et philosophie ? \\
Faut-il oublier le passé pour se donner un avenir ? \\
Faut-il parler pour avoir des idées générales ? \\
Faut-il partager la souveraineté ? \\
Faut-il penser l'État comme un corps ? \\
Faut-il perdre ses illusions ? \\
Faut-il perdre son temps ? \\
Faut-il poser des limites à l'activité rationnelle ? \\
Faut-il pour le connaître faire du vivant un objet ? \\
Faut-il préférer l'art à la nature ? \\
Faut-il préférer le bonheur à la vérité ? \\
Faut-il préférer une injustice au désordre ? \\
Faut-il prendre soin de soi ? \\
Faut-il protéger la dignité humaine ? \\
Faut-il protéger la nature ? \\
Faut-il protéger les faibles contre les forts ? \\
Faut-il que le réel ait un sens ? \\
Faut-il que les meilleurs gouvernent ? \\
Faut-il rechercher la certitude ? \\
Faut-il rechercher la simplicité ? \\
Faut-il rechercher le bonheur ? \\
Faut-il rechercher l'harmonie ? \\
Faut-il reconnaître pour connaître ? \\
Faut-il regretter l'équivocité du langage ? \\
Faut-il rejeter tous les préjugés ? \\
Faut-il rejeter toute norme ? \\
Faut-il renoncer à faire du travail une valeur ? \\
Faut-il renoncer à la certitude ? \\
Faut-il renoncer à l'idée d'âme ? \\
Faut-il renoncer à l'impossible ? \\
Faut-il renoncer à son désir ? \\
Faut-il résister à la peur de mourir ? \\
Faut-il respecter la nature ? \\
Faut-il respecter les convenances ? \\
Faut-il respecter le vivant ? \\
Faut-il rester impartial ? \\
Faut-il rester naturel ? \\
Faut-il rire ou pleurer ? \\
Faut-il rompre avec le passé ? \\
Faut-il s'adapter ? \\
Faut-il s'affranchir des désirs ? \\
Faut-il s'aimer soi-même ? \\
Faut-il sauver des vies à tout prix ? \\
Faut-il sauver les apparences ? \\
Faut-il savoir mentir ? \\
Faut-il savoir obéir pour gouverner ? \\
Faut-il savoir pour agir ? \\
Faut-il savoir prendre des risques ? \\
Faut-il se contenter de peu ? \\
Faut-il se cultiver ? \\
Faut-il se délivrer de la peur ? \\
Faut-il se délivrer des passions ? \\
Faut-il se détacher du monde ? \\
Faut-il s'efforcer d'être moins personnel ? \\
Faut-il se fier à ce que l'on ressent ? \\
Faut-il se fier à la majorité ? \\
Faut-il se fier à sa propre raison ? \\
Faut-il se fier aux apparences ? \\
Faut-il se libérer du travail ? \\
Faut-il se méfier de l'écriture ? \\
Faut-il se méfier de l'imagination ? \\
Faut-il se méfier de l'intuition ? \\
Faut-il se méfier des apparences ? \\
Faut-il se méfier de ses désirs ? \\
Faut-il se méfier du volontarisme politique ? \\
Faut-il s'en remettre à l'État pour limiter le pouvoir de l'État ? \\
Faut-il s'en tenir aux faits ? \\
Faut-il séparer la science et la technique ? \\
Faut-il séparer morale et politique ? \\
Faut-il se poser des questions métaphysiques ? \\
Faut-il se réjouir d'exister ? \\
Faut-il se rendre à l'évidence ? \\
Faut-il se ressembler pour former une société ? \\
Faut-il suivre ses intuitions ? \\
Faut-il surmonter son enfance ? \\
Faut-il tolérer les intolérants ? \\
Faut-il toujours avoir raison ? \\
Faut-il toujours dire la vérité ? \\
Faut-il toujours être en accord avec soi-même ? \\
Faut-il toujours éviter de se contredire ? \\
Faut-il toujours faire son devoir ? \\
Faut-il toujours garder espoir ? \\
Faut-il tout critiquer ? \\
Faut-il tout démontrer ? \\
Faut-il tout interpréter ? \\
Faut-il un commencement à tout ? \\
Faut-il un corps pour penser ? \\
Faut-il une guerre pour mettre fin à toutes les guerres ? \\
Faut-il une théorie de la connaissance ? \\
Faut-il vaincre ses désirs plutôt que l'ordre du monde ? \\
Faut-il vivre avec son temps ? \\
Faut-il vivre comme si l'on ne devait jamais mourir ? \\
Faut-il vivre comme si nous étions immortels ? \\
Faut-il vivre comme si nous ne devions jamais mourir ? \\
Faut-il vivre comme si on ne devait jamais mourir ? \\
Faut-il vivre dangereusement ? \\
Faut-il vivre hors de la société pour être heureux ? \\
Faut-il voir pour croire ? \\
Faut-il vouloir changer le monde ? \\
Faut-il vouloir être heureux ? \\
Faut-il vouloir la paix de l'âme ? \\
Faut-il vouloir la paix ? \\
Faut-il vouloir la transparence ? \\
Forme-t-on son esprit en transformant la matière ? \\
Gouverner, est-ce prévoir ? \\
Gouverner, est-ce régner ? \\
Hier a-t-il plus de réalité que demain ? \\
Imiter, est-ce copier ? \\
Interpréter, est-ce connaître ? \\
Interpréter, est-ce renoncer à prouver ? \\
Interpréter, est-ce savoir ? \\
Interpréter est-il subjectif ? \\
Interprète-t-on à défaut de connaître ? \\
Jusqu'à quel point la nature est-elle objet de science ? \\
Jusqu'à quel point pouvons-nous juger autrui ? \\
Jusqu'à quel point sommes-nous responsables de nos passions ? \\
Jusqu'à quel point suis-je mon propre maître ? \\
Jusqu'où interpréter ? \\
Jusqu'où peut-on dialoguer ? \\
Jusqu'où peut-on soigner ? \\
La beauté a-t-elle une histoire ? \\
La beauté est-elle affaire de goût ? \\
La beauté est-elle dans le regard ou dans la chose vue ? \\
La beauté est-elle dans les choses ? \\
La beauté est-elle intemporelle ? \\
La beauté est-elle l'objet d'une connaissance ? \\
La beauté est-elle partout ? \\
La beauté est-elle sensible ? \\
La beauté est-elle une promesse de bonheur ? \\
La beauté nous rend-elle meilleurs ? \\
La beauté peut-elle délivrer une vérité ? \\
La beauté s'explique-t-elle ? \\
La bêtise et la méchanceté sont-elles liées intrinsèquement ? \\
La bêtise et la méchanceté sont-elles liées nécessairement ? \\
La bêtise n'est-elle pas proprement humaine ? \\
La biologie peut-elle se passer de causes finales ? \\
L'abstraction est-elle toujours utile à la science empirique ? \\
L'abstrait est-il en dehors de l'espace et du temps ? \\
La causalité suppose-t-elle des lois ? \\
La charité est-elle une vertu ? \\
La cohérence est-elle la norme du vrai ? \\
La cohérence est-elle un critère de la vérité ? \\
La cohérence est-elle un critère de vérité ? \\
La cohérence est-elle une vertu ? \\
La cohérence logique est-elle une condition suffisante de la démonstration ? \\
La communication est-elle nécessaire à la démocratie ? \\
La compassion risque-t-elle d'abolir l'exigence politique ? \\
La compétence technique peut-elle fonder l'autorité publique ? \\
La confiance est-elle une vertu ? \\
La connaissance a-t-elle des limites ? \\
La connaissance commune est-elle le point de départ de la science ? \\
La connaissance commune fait-elle obstacle à la vérité ? \\
La connaissance de la vie se confond-elle avec celle du vivant ? \\
La connaissance de l'histoire est-elle utile à l'action ? \\
La connaissance du vivant est-elle désintéressée ? \\
La connaissance du vivant peut-elle être désintéressée  ? \\
La connaissance est-elle une contemplation ? \\
La connaissance est-elle une croyance justifiée ? \\
La connaissance historique est-elle une interprétation des faits ? \\
La connaissance historique est-elle utile à l'homme ? \\
La connaissance objective doit-elle s'interdire toute interprétation ? \\
La connaissance objective exclut-elle toute forme de subjectivité ? \\
La connaissance peut-elle être pratique ? \\
La connaissance peut-elle se passer de l'imagination ? \\
La connaissance scientifique abolit-elle toute croyance ? \\
La connaissance scientifique est-elle désintéressée ? \\
La connaissance scientifique n'est-elle qu'une croyance argumentée ? \\
La connaissance s'interdit-elle tout recours à l'imagination ? \\
La connaissance suppose-t-elle une éthique ? \\
La conscience a-t-elle des degrés ? \\
La conscience a-t-elle des moments ? \\
La conscience d'autrui est-elle impénétrable ? \\
La conscience de la mort est-elle une condition de la sagesse ? \\
La conscience de soi est-elle une donnée immédiate ? \\
La conscience de soi suppose-t-elle autrui ? \\
La conscience du temps rend-elle l'existence tragique ? \\
La conscience entrave-t-elle l'action ? \\
La conscience est-elle ce qui fait le sujet ? \\
La conscience est-elle intrinsèquement morale ? \\
La conscience est-elle nécessairement malheureuse ? \\
La conscience est-elle ou n'est-elle pas ? \\
La conscience est-elle source d'illusions ? \\
La conscience est-elle toujours morale ? \\
La conscience est-elle une activité ? \\
La conscience est-elle une illusion ? \\
La conscience morale est-elle innée ? \\
La conscience morale n'est-elle que le fruit de l'éducation ? \\
La conscience morale n'est-elle que le produit de l'éducation ? \\
La conscience peut-elle être collective ? \\
La conscience peut-elle nous tromper ? \\
La considération de l'utilité doit-elle déterminer toutes nos actions ? \\
La contingence est-elle la condition de la liberté ? \\
La contradiction réside-t-elle dans les choses ? \\
La contrainte des lois est-elle une violence ? \\
La contrainte peut-elle être légitime ? \\
La contrainte supprime-t-elle la responsabilité ? \\
La critique du pouvoir peut-elle conduire à la désobéissance ? \\
La croyance est-elle l'asile de l'ignorance ? \\
La croyance est-elle signe de faiblesse ? \\
La croyance est-elle une opinion comme les autres ? \\
La croyance est-elle une opinion ? \\
La croyance peut-elle être rationnelle ? \\
La croyance peut-elle tenir lieu de savoir ? \\
La croyance religieuse échappe-t-elle à toute logique ? \\
La croyance religieuse se distingue-t-elle des autres formes de croyance ? \\
L'action humaine nécessite-t-elle la contingence du monde ? \\
L'action politique a-t-elle un fondement rationnel ? \\
L'action politique peut-elle se passer de mots ? \\
L'activité se laisse-t-elle programmer ? \\
La culture est-elle affaire de politique ? \\
La culture est-elle la négation de la nature ? \\
La culture est-elle nécessaire à l'appréciation d'une œuvre d'art ? \\
La culture est-elle une question politique ? \\
La culture est-elle une seconde nature ? \\
La culture est-elle un luxe ? \\
La culture garantit-elle l'excellence humaine ? \\
La culture libère-t-elle des préjugés ? \\
La culture nous rend-elle meilleurs ? \\
La culture nous rend-elle plus humains ? \\
La culture nous unit-elle ? \\
La culture peut-elle être instituée ? \\
La culture peut-elle être objet de science ? \\
La culture rend-elle plus humain ? \\
La culture : pour quoi faire ? \\
La curiosité est-elle à l'origine du savoir ? \\
La danse est-elle l'œuvre du corps ? \\
La décision a-t-elle besoin de raisons ? \\
La découverte de la vérité peut-elle être le fait du hasard ? \\
La défense de l'intérêt général est-il la fin dernière de la politique ? \\
La démarche scientifique exclut-elle tout recours à l'imagination ? \\
La démocratie a-t-elle des limites ? \\
La démocratie a-t-elle une histoire ? \\
La démocratie conduit-elle au règne de l'opinion ? \\
La démocratie est-ce la fin du despotisme ? \\
La démocratie, est-ce le pouvoir du plus grand nombre ? \\
La démocratie est-elle la loi du plus fort ? \\
La démocratie est-elle le pire des régimes politiques ? \\
La démocratie est-elle le règne de l'opinion ? \\
La démocratie est-elle moyen ou fin ? \\
La démocratie est-elle nécessairement libérale ? \\
La démocratie est-elle possible ? \\
La démocratie n'est-elle que la force des faibles ? \\
La démocratie peut-elle échapper à la démagogie ? \\
La démocratie peut-elle être représentative ? \\
La démocratie peut-elle se passer de représentation ? \\
La démonstration nous garantit-elle l'accès à la vérité ? \\
La démonstration obéit-elle à des lois ? \\
La démonstration supprime-t-elle le doute ? \\
La dialectique est-elle une science ? \\
La différence des sexes est-elle une question philosophique ? \\
La différence des sexes est-elle un problème philosophique ? \\
La distinction de la nature et de la culture est-elle un fait de culture ? \\
La diversité des langues est-elle une diversité des pensées ? \\
La diversité des opinions conduit-elle à douter de tout ? \\
La docilité est-elle un vice ou une vertu ? \\
La douleur est-elle utile ? \\
La douleur nous apprend-elle quelque chose ? \\
La famille est-elle le lieu de la formation morale ? \\
La famille est-elle naturelle ? \\
La famille est-elle une communauté naturelle ? \\
La famille est-elle une institution politique ? \\
La famille est-elle un modèle de société ? \\
La femme est-elle l'avenir de l'homme ? \\
La finalité est-elle nécessaire pour penser le vivant ? \\
La fin de la politique est-elle l'établissement de la justice ? \\
La fin justifie-t-elle les moyens ? \\
La foi est-elle aveugle ? \\
La foi est-elle irrationnelle ? \\
La foi est-elle rationnelle ? \\
La fonction de penser peut-elle se déléguer ? \\
La fonction du philosophe est-elle de diriger l'État ? \\
La fonction première de l'État est-elle de durer ? \\
La force de l'État est-elle nécessaire à la liberté des citoyens ? \\
La force est-elle une vertu ? \\
La force fait-elle le droit ? \\
La franchise est-elle une vertu ? \\
La fraternité a-t-elle un sens politique ? \\
La fraternité est-elle un idéal moral ? \\
La fraternité peut-elle se passer d'un fondement religieux ? \\
La fuite du temps est-elle nécessairement un malheur ? \\
La gloire est-elle un bien ? \\
La grammaire contraint-elle la pensée ? \\
La grammaire contraint-elle notre pensée ? \\
La guerre est-elle la continuation de la politique par d'autres moyens ? \\
La guerre est-elle la continuation de la politique ? \\
La guerre est-elle la politique continuée par d'autres moyens ? \\
La guerre est-elle l'essentiel de toute politique ? \\
La guerre met-elle fin au droit ? \\
La guerre peut-elle être juste ? \\
Laisser mourir, est-ce tuer ? \\
La justice a-t-elle besoin des institutions ? \\
La justice a-t-elle un fondement rationnel ? \\
La justice consiste-t-elle à traiter tout le monde de la même manière ? \\
La justice consiste-t-elle dans l'application de la loi ? \\
La justice est-elle l'affaire de l'État ? \\
La justice est-elle une notion morale ? \\
La justice est-elle une vertu ? \\
La justice n'est-elle qu'une institution ? \\
La justice n'est-elle qu'un idéal ? \\
La justice peut-elle se fonder sur le compromis ? \\
La justice peut-elle se passer de la force ? \\
La justice peut-elle se passer d'institutions ? \\
La justice suppose-t-elle l'égalité ? \\
La justice : moyen ou fin de la politique ? \\
La liberté a-t-elle un prix ? \\
La liberté comporte-t-elle des degrés ? \\
La liberté connaît-elle des excès ? \\
La liberté d'expression est-elle nécessaire à la liberté de pensée ? \\
La liberté, est-ce l'indépendance à l'égard des passions ? \\
La liberté est-elle le pouvoir de refuser ? \\
La liberté est-elle un fait ? \\
La liberté et l'égalité sont-elles compatibles ? \\
La liberté fait-elle de nous des êtres meilleurs ? \\
La liberté implique-t-elle l'indifférence ? \\
La liberté intéresse-t-elle les sciences humaines ? \\
La liberté n'est-elle qu'une illusion ? \\
La liberté nous rend-elle inexcusables ? \\
La liberté peut-elle être prouvée ? \\
La liberté peut-elle être une illusion ? \\
La liberté peut-elle faire peur ? \\
La liberté peut-elle s'affirmer sans violence ? \\
La liberté peut-elle s'aliéner ? \\
La liberté peut-elle se constater ? \\
La liberté peut-elle se prouver ? \\
La liberté peut-elle se refuser ? \\
La liberté requiert-elle le libre échange ? \\
La liberté s'achète-t-elle ? \\
La liberté se mérite-t-elle ? \\
La liberté se prouve-t-elle ? \\
La liberté se réduit-elle au libre-arbitre ? \\
La littérature peut-elle suppléer les sciences de l'homme ? \\
La logique a-t-elle une histoire ? \\
La logique a-t-elle un intérêt philosophique ? \\
La logique décrit-elle le monde ? \\
La logique est-elle indépendante de la psychologie ? \\
La logique est-elle la norme du vrai ? \\
La logique est-elle l'art de penser ? \\
La logique est-elle un art de penser ? \\
La logique est-elle un art de raisonner ? \\
La logique est-elle une discipline normative ? \\
La logique est-elle une forme de calcul ? \\
La logique est-elle une science de la vérité ? \\
La logique est-elle une science ? \\
La logique est-elle utile à la métaphysique ? \\
La logique nous apprend-elle quelque chose sur le langage ordinaire ? \\
La logique peut-elle se passer de la métaphysique ? \\
La logique pourrait-elle nous surprendre ? \\
La logique : découverte ou invention ? \\
La loi dit-elle ce qui est juste ? \\
La loi éduque-t-elle ? \\
La loi est-elle une garantie contre l'injustice ? \\
La loi peut-elle changer les mœurs ? \\
La loi peut-elle être injuste ? \\
L'altruisme n'est-il qu'un égoïsme bien compris ? \\
La magie peut-elle être efficace ? \\
La majorité doit-elle toujours l'emporter ? \\
La majorité, force ou droit ? \\
La majorité peut-elle être tyrannique ? \\
La maladie est-elle à l'organisme vivant ce que la panne est à la machine ? \\
La maladie est-elle indispensable à la connaissance du vivant ? \\
La mathématique est-elle une ontologie ? \\
La matière, est-ce le mal ? \\
La matière, est-ce l'informe ? \\
La matière est-elle amorphe ? \\
La matière est-elle plus facile à connaître que l'esprit ? \\
La matière est-elle une vue de l'esprit ? \\
La matière n'est-elle que ce que l'on perçoit ? \\
La matière n'est-elle qu'une idée ? \\
La matière n'est-elle qu'un obstacle ? \\
La matière pense-t-elle ? \\
La matière peut-elle être objet de connaissance ? \\
L'ambiguïté des mots peut-elle être heureuse ? \\
L'âme concerne-t-elle les sciences humaines ? \\
La médecine est-elle une science ? \\
L'âme est-elle immortelle ? \\
L'âme et le corps sont-ils une seule et même chose ? \\
L'âme jouit-elle d'une vie propre ? \\
L'amélioration des hommes peut-elle être considérée comme un objectif politique ? \\
La métaphysique a-t-elle ses fictions ? \\
La métaphysique est-elle le fondement de la morale ? \\
La métaphysique est-elle nécessairement une réflexion sur Dieu ? \\
La métaphysique est-elle une science ? \\
La métaphysique peut-elle être autre chose qu'une science recherchée ? \\
La métaphysique peut-elle faire appel à l'expérience ? \\
La métaphysique se définit-elle par son objet ou sa démarche ? \\
La méthode expérimentale est-elle appropriée à l'étude du vivant ? \\
L'amitié est-elle une vertu ? \\
L'amitié est-elle un principe politique ? \\
L'amitié peut-elle obliger ? \\
L'amitié relève-t-elle d'une décision ? \\
La modération est-elle l'essence de la vertu ? \\
La modération est-elle une vertu politique ? \\
La morale a-t-elle à décider de la sexualité ? \\
La morale a-t-elle besoin de la notion de sainteté ? \\
La morale a-t-elle besoin d'être fondée ? \\
La morale a-t-elle besoin d'un au-delà ? \\
La morale a-t-elle besoin d'un fondement ? \\
La morale a-t-elle sa place dans l'économie ? \\
La morale consiste-t-elle à respecter le droit ? \\
La morale consiste-t-elle à suivre la nature ? \\
La morale dépend-elle de la culture ? \\
La morale doit-elle en appeler à la nature ? \\
La morale doit-elle être rationnelle ? \\
La morale doit-elle fournir des préceptes ? \\
La morale est-elle affaire de convention ? \\
La morale est-elle affaire de jugement ? \\
La morale est-elle affaire de sentiments ? \\
La morale est-elle affaire de sentiment ? \\
La morale est-elle condamnée à n'être qu'un champ de bataille ? \\
La morale est-elle désintéressée ? \\
La morale est-elle en conflit avec le désir ? \\
La morale est-elle ennemie du bonheur ? \\
La morale est-elle fondée sur la liberté ? \\
La morale est-elle incompatible avec le déterminisme ? \\
La morale est-elle l'ennemie de la vie ? \\
La morale est-elle nécessairement répressive ? \\
La morale est-elle objet de science ? \\
La morale est-elle un art de vivre ? \\
La morale est-elle une affaire de raison ? \\
La morale est-elle une affaire d'habitude ? \\
La morale est-elle une affaire solitaire ? \\
La morale est-elle un fait de culture ? \\
La morale est-elle un fait social ? \\
La morale et la religion visent-elles les mêmes fins ? \\
La morale n'est-elle qu'un ensemble de conventions ? \\
La morale peut-elle être fondée sur la science ? \\
La morale peut-elle être naturelle ? \\
La morale peut-elle être un calcul ? \\
La morale peut-elle être une science ? \\
La morale peut-elle se fonder sur les sentiments ? \\
La morale peut-elle s'enseigner ? \\
La morale peut-elle se passer d'un fondement religieux ? \\
La morale s'apprend-elle ? \\
La morale s'enseigne-t-elle ? \\
La morale s'oppose-t-elle à la politique ? \\
La morale suppose-t-elle le libre arbitre ? \\
La moralité consiste-t-elle à se contraindre soi-même ? \\
La moralité est-elle affaire de principes ou de conséquences ? \\
La moralité n'est-elle que dressage ? \\
La moralité réside-t-elle dans l'intention ? \\
La moralité se réduit-elle aux sentiments ? \\
La mort a-t-elle un sens ? \\
La mort fait-elle partie de la vie ? \\
L'amour a-t-il des raisons ? \\
L'amour de soi est-il immoral ? \\
L'amour est-il désir ? \\
L'amour est-il une vertu ? \\
L'amour implique-t-il le respect ? \\
L'amour peut-il être absolu ? \\
L'amour peut-il être raisonnable ? \\
L'amour peut-il être un devoir ? \\
La musique a-t-elle une essence ? \\
La musique est-elle un langage ? \\
La naïveté est-elle une vertu ? \\
L'analyse du langage ordinaire peut-elle avoir un intérêt philosophique ? \\
La nation est-elle dépassée ? \\
La nature a-t-elle des droits ? \\
La nature a-t-elle une histoire ? \\
La nature a-t-elle un langage ? \\
La nature est-elle artiste ? \\
La nature est-elle belle ? \\
La nature est-elle bien faite ? \\
La nature est-elle digne de respect ? \\
La nature est-elle écrite en langage mathématique ? \\
La nature est-elle muette ? \\
La nature est-elle politique ? \\
La nature est-elle prévisible ? \\
La nature est-elle sacrée ? \\
La nature est-elle sans histoire ? \\
La nature est-elle sauvage ? \\
La nature est-elle une norme ? \\
La nature est-elle une ressource ? \\
La nature est-elle un modèle ? \\
La nature est-elle un système ? \\
La nature existe-t-elle ? \\
La nature fait-elle bien les choses ? \\
La nature parle-t-elle le langage des mathématiques ? \\
La nature peut-elle avoir des droits ? \\
La nature peut-elle constituer une norme ? \\
La nature peut-elle être belle ? \\
La nature peut-elle être un modèle ? \\
La nature peut-elle nous indiquer ce que nous devons faire ? \\
La nature se donne-t-elle à penser ? \\
La nécessité fait-elle loi ? \\
La négligence est-elle une faute ? \\
La neige est-elle blanche ? \\
L'animal a-t-il des droits ? \\
L'animal nous apprend-il quelque chose sur l'homme ? \\
L'animal peut-il être un sujet moral ? \\
La notion de barbarie a-t-elle un sens ? \\
La notion de finalité a-t-elle de l'intérêt pour le savant ? \\
La notion de loi a-t-elle une unité ? \\
La notion de nature humaine a-t-elle un sens ? \\
La notion de progrès a-t-elle un sens en politique ? \\
L'anthropologie est-elle une ontologie ? \\
La ou les vertus ? \\
La paix est-elle l'absence de guerres ? \\
La paix est-elle l'absence de guerre ? \\
La paix est-elle le plus grand des biens ? \\
La paix est-elle moins naturelle que la guerre ? \\
La paix est-elle possible ? \\
La paix n'est-elle que l'absence de conflit ? \\
La paix n'est-elle que l'absence de guerre ? \\
La paix sociale est-elle la finalité de la politique ? \\
La paix sociale est-elle le but de la politique ? \\
La paix sociale est-elle une fin en soi ? \\
La parole peut-elle être une arme ? \\
La passion de la vérité peut-elle être source d'erreur ? \\
La passion est-elle immorale ? \\
La passion est-elle l'ennemi de la raison ? \\
La passion exclut-elle la lucidité ? \\
La passion n'est-elle que souffrance ? \\
La patience est-elle une vertu ? \\
La pauvreté est-elle une injustice ? \\
La peine de mort est-elle juste, injuste, et pourquoi ? \\
La peinture apprend-elle à voir ? \\
La peinture est-elle une poésie muette ? \\
La peinture peut-elle être un art du temps ? \\
La pensée a-t-elle une histoire ? \\
La pensée doit-elle se soumettre aux règles de la logique ? \\
La pensée échappe-t-elle à la grammaire ? \\
La pensée est-elle en lutte avec le langage ? \\
La pensée est-elle une activité assimilable à un travail ? \\
La pensée et la conscience sont-elles une seule et même chose ? \\
La pensée formelle est-elle privée d'objet ? \\
La pensée formelle est-elle une pensée vide ? \\
La pensée formelle peut-elle avoir un contenu ? \\
La pensée obéit-elle à des lois ? \\
La pensée peut-elle s'écrire ? \\
La pensée peut-elle se passer de mots ? \\
La perception construit-elle son objet ? \\
La perception de l'espace est-elle innée ou acquise ? \\
La perception est-elle le premier degré de la connaissance ? \\
La perception est-elle l'interprétation du réel ? \\
La perception est-elle source de connaissance ? \\
La perception est-elle une interprétation ? \\
La perception me donne-t-elle le réel ? \\
La perception peut-elle être désintéressée ? \\
La perception peut-elle s'éduquer ? \\
La perfection est-elle désirable ? \\
La philosophie a-t-elle une histoire ? \\
La philosophie doit-elle être une science ? \\
La philosophie doit-elle se préoccuper du salut ? \\
La philosophie est-elle abstraite ? \\
La philosophie est-elle une science ? \\
La philosophie peut-elle disparaître ? \\
La philosophie peut-elle être expérimentale ? \\
La philosophie peut-elle être populaire ? \\
La philosophie peut-elle être une science ? \\
La philosophie peut-elle se passer de théologie ? \\
La philosophie rend-elle inefficace la propagande ? \\
La photographie est-elle un art ? \\
La pitié a-t-elle une valeur ? \\
La pitié est-elle morale ? \\
La pitié peut-elle fonder la morale ? \\
La poésie pense-t-elle ? \\
La politesse est-elle une vertu ? \\
La politique a-t-elle besoin de héros ? \\
La politique a-t-elle besoin de modèles ? \\
La politique a-t-elle besoin d'experts ? \\
La politique a-t-elle pour fin d'éliminer la violence ? \\
La politique consiste-t-elle à faire cause commune ? \\
La politique consiste-t-elle à faire des compromis ? \\
La politique doit-elle être morale ? \\
La politique doit-elle être rationnelle ? \\
La politique doit-elle refuser l'utopie ? \\
La politique doit-elle se mêler de l'art ? \\
La politique doit-elle se mêler du bonheur ? \\
La politique doit-elle viser la concorde ? \\
La politique doit-elle viser le consensus ? \\
La politique échappe-telle à l'exigence de vérité ? \\
La politique est-elle affaire de compétence ? \\
La politique est-elle affaire de décision ? \\
La politique est-elle affaire de jugement ? \\
La politique est-elle architectonique ? \\
La politique est-elle extérieure au droit ? \\
La politique est-elle la continuation de la guerre ? \\
La politique est-elle l'affaire des spécialistes ? \\
La politique est-elle l'affaire de tous ? \\
La politique est-elle l'art des possibles ? \\
La politique est-elle l'art du possible ? \\
La politique est-elle naturelle ? \\
La politique est-elle par nature sujette à dispute ? \\
La politique est-elle plus importante que tout ? \\
La politique est-elle un art ? \\
La politique est-elle une affaire d'experts ? \\
La politique est-elle une science ? \\
La politique est-elle une technique ? \\
La politique est-elle un métier ? \\
La politique exclut-elle le désordre ? \\
La politique implique-t-elle la hiérarchie ? \\
La politique n'est-elle que l'art de conquérir et de conserver le pouvoir ? \\
La politique peut-elle changer le monde ? \\
La politique peut-elle être indépendante de la morale ? \\
La politique peut-elle être objet de science ? \\
La politique peut-elle être un objet de science ? \\
La politique peut-elle n'être qu'une pratique ? \\
La politique peut-elle se passer de croyances ? \\
La politique peut-elle se passer de croyance ? \\
La politique peut-elle unir les hommes ? \\
La politique repose-t-elle sur un contrat ? \\
La politique suppose-t-elle la morale ? \\
La politique suppose-t-elle une idée de l'homme ? \\
La poursuite de mon intérêt m'oppose-t-elle aux autres ? \\
L'apparence est-elle toujours trompeuse ? \\
La pratique des sciences met-elle à l'abri des préjugés ? \\
La précaution peut-elle être un principe ? \\
La prise de parti est-elle essentielle en politique ? \\
La prison est-elle utile ? \\
La propriété, est-ce un vol ? \\
La propriété est-elle un droit ? \\
La propriété est-elle une garantie de liberté ? \\
La psychanalyse est-elle une science ? \\
La psychologie est-elle une science de la nature ? \\
La psychologie est-elle une science ? \\
La question « qui suis-je » admet-elle une réponse exacte ? \\
La question : « qui ? » \\
La radicalité est-elle une exigence philosophique ? \\
La raison a-t-elle des limites ? \\
La raison a-t-elle le droit d'expliquer ce que morale condamne ? \\
La raison a-t-elle pour fin la connaissance ? \\
La raison a-t-elle une histoire ? \\
La raison d'État peut-elle être justifiée ? \\
La raison doit-elle critiquer la croyance ? \\
La raison doit-elle être cultivée ? \\
La raison doit-elle être notre guide ? \\
La raison doit-elle se soumettre au réel ? \\
La raison engendre-t-elle des illusions ? \\
La raison épuise-t-elle le réel ? \\
La raison est-elle le pouvoir de distinguer le vrai du faux ? \\
La raison est-elle l'esclave des passions ? \\
La raison est-elle l'esclave du désir ? \\
La raison est-elle morale par elle-même ? \\
La raison est-elle plus fiable que l'expérience ? \\
La raison est-elle seulement affaire de logique ? \\
La raison est-elle suffisante ? \\
La raison est-elle toujours raisonnable ? \\
La raison ne connaît-elle du réel que ce qu'elle y met d'elle-même ? \\
La raison ne veut-elle que connaître ? \\
La raison peut-elle entrer en conflit avec elle-même ? \\
La raison peut-elle errer ? \\
La raison peut-elle être immédiatement pratique ? \\
La raison peut-elle être pratique ? \\
La raison peut-elle nous commander de croire ? \\
La raison peut-elle se contredire ? \\
La raison peut-elle servir le mal ? \\
La raison s'oppose-t-elle aux passions ? \\
La raison transforme-t-elle le réel ? \\
L'architecture est-elle un art ? \\
La réalité a-t-elle une forme logique ? \\
La réalité décrite par la science s'oppose-t-elle à la démonstration ? \\
La réalité de la vie s'épuise-t-elle dans celle des vivants ? \\
La réalité du temps se réduit-elle à la conscience que nous en avons ? \\
La réalité est-elle une idée ? \\
La réalité n'est-elle qu'une construction ? \\
La réalité nourrit-elle la fiction ? \\
La réalité peut-elle être virtuelle ? \\
La recherche de la vérité peut-elle être désintéressée ? \\
La recherche du bonheur est-elle un idéal égoïste ? \\
La recherche du bonheur suffit-elle à déterminer une morale ? \\
La recherche scientifique est-elle désintéressée ? \\
La réciprocité est-elle indispensable à la communauté politique ? \\
La référence aux faits suffit-elle à garantir l'objectivité de la connaissance ? \\
La réflexion sur l'expérience participe-t-elle de l'expérience ? \\
La religion a-t-elle besoin d'un dieu ? \\
La religion conduit-elle l'homme au-delà de lui-même ? \\
La religion divise-t-elle les hommes ? \\
La religion est-elle contraire à la raison ? \\
La religion est-elle fondée sur la peur de la mort ? \\
La religion est-elle la sagesse des pauvres ? \\
La religion est-elle l'asile de l'ignorance ? \\
La religion est-elle l'opium du peuple ? \\
La religion est-elle une affaire privée ? \\
La religion est-elle une consolation pour les hommes ? \\
La religion est-elle un instrument de pouvoir ? \\
La religion implique-t-elle la croyance en un être divin ? \\
La religion n'est-elle que l'affaire des croyants ? \\
La religion n'est-elle qu'une affaire privée ? \\
La religion n'est-elle qu'un fait de culture ? \\
La religion peut-elle être civile ? \\
La religion peut-elle être naturelle ? \\
La religion peut-elle faire lien social ? \\
La religion peut-elle n'être qu'une affaire privée ? \\
La religion peut-elle suppléer la raison ? \\
La religion relie-t-elle les hommes ? \\
La religion rend-elle l'homme heureux ? \\
La religion rend-elle meilleur ? \\
La religion repose-t-elle sur une illusion ? \\
La religion se distingue-t-elle de la superstition ? \\
La responsabilité peut-elle être collective ? \\
La responsabilité politique n'est-elle le fait que de ceux qui gouvernent ? \\
La révolte peut-elle être un droit ? \\
L'argent est-il la mesure de tout échange ? \\
L'argent est-il un mal nécessaire ? \\
La rhétorique a-t-elle une valeur ? \\
La rhétorique est-elle un art ? \\
La rigueur des lois ? \\
L'art apprend-il à percevoir ? \\
L'art a-t-il besoin de théorie ? \\
L'art a-t-il des vertus thérapeutiques ? \\
L'art a-t-il plus de valeur que la vérité ? \\
L'art a-t-il pour fin le plaisir ? \\
L'art a-t-il une fin morale ? \\
L'art a-t-il une histoire ? \\
L'art a-t-il une valeur sociale ? \\
L'art a-t-il un rôle à jouer dans l'éducation ? \\
L'art change-t-il la vie ? \\
L'art contre la beauté ? \\
L'art de vivre est-il un art ? \\
L'art doit-il divertir ? \\
L'art doit-il être critique ? \\
L'art doit-il nous étonner ? \\
L'art doit-il refaire le monde ? \\
L'art donne-t-il à penser ? \\
L'art donne-t-il nécessairement lieu à la production d'une œuvre ? \\
L'art échappe-t-il à la raison ? \\
L'art éduque-t-il la perception ? \\
L'art éduque-t-il l'homme ? \\
L'art est-il affaire d'apparence ? \\
L'art est-il affaire de goût ? \\
L'art est-il affaire d'imagination ? \\
L'art est-il à lui-même son propre but ? \\
L'art est-il au service du beau ? \\
L'art est-il destiné à embellir ? \\
L'art est-il imitatif ? \\
L'art est-il le produit de l'inconscient ? \\
L'art est-il le règne des apparences ? \\
L'art est-il mensonger ? \\
L'art est-il méthodique ? \\
L'art est-il moins nécessaire que la science ? \\
L'art est-il subversif ? \\
L'art est-il une affaire sérieuse ? \\
L'art est-il une critique de la culture ? \\
L'art est-il une expérience de la liberté ? \\
L'art est-il une histoire ? \\
L'art est-il une promesse de bonheur ? \\
L'art est-il universel ? \\
L'art est-il un jeu ? \\
L'art est-il un langage ? \\
L'art est-il un luxe ? \\
L'art est-il un mode de connaissance ? \\
L'art est-il un modèle pour la philosophie ? \\
L'art est-il un moyen de connaître ? \\
L'art est-il un refuge ? \\
L'art exprime-t-il ce que nous ne saurions dire ? \\
L'art fait-il penser ? \\
L'art imite-t-il la nature ? \\
L'artiste a-t-il besoin de modèle ? \\
L'artiste a-t-il besoin d'une idée de l'art ? \\
L'artiste a-t-il besoin d'un public ? \\
L'artiste a-t-il une méthode ? \\
L'artiste doit-il être de son temps ? \\
L'artiste doit-il être original ? \\
L'artiste doit-il se donner des modèles ? \\
L'artiste doit-il se soucier du goût du public ? \\
L'artiste est-il le mieux placé pour comprendre son œuvre ? \\
L'artiste est-il maître de son œuvre ? \\
L'artiste est-il souverain ? \\
L'artiste est-il un créateur ? \\
L'artiste est-il un travailleur ? \\
L'artiste exprime-t-il quelque chose ? \\
L'artiste peut-il se passer d'un maître ? \\
L'artiste recherche-t-il le beau ? \\
L'artiste sait-il ce qu'il fait ? \\
L'artiste travaille-t-il ? \\
L'art modifie-t-il notre rapport au réel ? \\
L'art n'est-il pas toujours politique ? \\
L'art n'est-il pas toujours religieux ? \\
L'art n'est-il qu'une question de sentiment ? \\
L'art n'est-il qu'un mode d'expression subjectif ? \\
L'art n'est qu'une affaire de goût ? \\
L'art nous détourne-t-il de la réalité ? \\
L'art nous fait-il mieux percevoir le réel ? \\
L'art nous mène-t-il au vrai ? \\
L'art nous réconcilie-t-il avec le monde ? \\
L'art parachève-t-il la nature ? \\
L'art participe-t-il à la vie politique ? \\
L'art permet-il un accès au divin ? \\
L'art peut-il contribuer à éduquer les hommes ? \\
L'art peut-il encore imiter la nature ? \\
L'art peut-il être abstrait ? \\
L'art peut-il être brut ? \\
L'art peut-il être conceptuel ? \\
L'art peut-il être populaire ? \\
L'art peut-il être sans œuvre ? \\
L'art peut-il être utile ? \\
L'art peut-il ne pas être sacré ? \\
L'art peut-il n'être pas conceptuel ? \\
L'art peut-il nous rendre meilleurs ? \\
L'art peut-il prétendre à la vérité ? \\
L'art peut-il quelque chose contre la morale ? \\
L'art peut-il quelque chose pour la morale ? \\
L'art peut-il rendre le mouvement ? \\
L'art peut-il s'affranchir des lois ? \\
L'art peut-il s'enseigner ? \\
L'art peut-il se passer de la beauté ? \\
L'art peut-il se passer de règles ? \\
L'art peut-il se passer d'idéal ? \\
L'art peut-il se passer d'œuvres ? \\
L'art produit-il nécessairement des œuvres ? \\
L'art rend-il les hommes meilleurs ? \\
L'art s'adresse-t-il à la sensibilité ? \\
L'art s'adresse-t-il à tous ? \\
L'art sait-il montrer ce que le langage ne peut pas dire ? \\
L'art s'apparente-t-il à la philosophie ? \\
L'art s'apprend-il ? \\
L'art vise-t-il le beau ? \\
L'art : expérience, exercice ou habitude ? \\
L'art : une arithmétique sensible ? \\
La sagesse rend-elle heureux ? \\
La santé est-elle un devoir ? \\
La santé est-elle un droit ou un devoir ? \\
La science admet-elle des degrés de croyance ? \\
La science a-t-elle besoin d'un critère de démarcation entre science et non science ? \\
La science a-t-elle besoin d'une méthode ? \\
La science a-t-elle besoin du principe de causalité ? \\
La science a-t-elle des limites ? \\
La science a-t-elle le monopole de la raison ? \\
La science a-t-elle le monopole de la vérité ? \\
La science a-t-elle une histoire ? \\
La science commence-t-elle avec la perception ? \\
La science commence-telle avec la perception ? \\
La science découvre-t-elle ou construit-elle son objet ? \\
La science dévoile-t-elle le réel ? \\
La science doit-elle se fonder sur une idée de la nature ? \\
La science doit-elle se passer de l'idée de finalité ? \\
La science du vivant peut-elle se passer de l'idée de finalité ? \\
La science est-elle austère ? \\
La science est-elle indépendante de toute métaphysique ? \\
La science est-elle le lieu de la vérité ? \\
La science est-elle une connaissance du réel ? \\
La science est-elle une langue bien faite ? \\
La science est-elle un jeu ? \\
La science exclut-elle l'imagination ? \\
La science n'est-elle qu'une activité théorique ? \\
La science n'est-elle qu'une fiction ? \\
La science nous éloigne-t-elle de la religion ? \\
La science nous éloigne-t-elle des choses ? \\
La science nous indique-t-elle ce que nous devons faire ? \\
La science pense-t-elle ? \\
La science permet-elle de comprendre le monde ? \\
La science peut-elle être une métaphysique ? \\
La science peut-elle guider notre conduite ? \\
La science peut-elle lutter contre les préjugés ? \\
La science peut-elle produire des croyances ? \\
La science peut-elle se passer de fondement ? \\
La science peut-elle se passer de l'idée de finalité ? \\
La science peut-elle se passer de métaphysique ? \\
La science peut-elle se passer d'hypothèses ? \\
La science peut-elle se passer d'institutions ? \\
La science peut-elle tout expliquer ? \\
La science porte-elle au scepticisme ? \\
La science procède-t-elle par rectification ? \\
La science se limite-t-elle à constater les faits ? \\
La servitude peut-elle être volontaire ? \\
La société doit-elle reconnaître les désirs individuels ? \\
La société est-elle concevable sans le travail ? \\
La société est-elle un organisme ? \\
La société existe-t-elle ? \\
La société fait-elle l'homme ? \\
La société peut-elle être l'objet d'une science ? \\
La société peut-elle se passer de l'État ? \\
La société précède-t-elle l'individu ? \\
La société repose-t-elle sur l'altruisme ? \\
La sociologie de l'art nous permet-elle de comprendre l'art ? \\
La sociologie relativise-t-elle la valeur des œuvres d'art ? \\
La solidarité est-elle naturelle ? \\
La solitude constitue-t-elle un obstacle à la citoyenneté ? \\
La souffrance a-t-elle une valeur morale ? \\
La souffrance a-t-elle un sens moral ? \\
La souffrance a-t-elle un sens ? \\
La souffrance d'autrui m'importe-t-elle ? \\
La souffrance peut-elle être un mode de connaissance ? \\
La souveraineté peut-elle être limitée ? \\
La souveraineté peut-elle se partager ? \\
La sphère privée échappe-t-elle au politique ? \\
La sympathie peut-elle tenir lieu de moralité ? \\
La technique accroît-elle notre liberté ? \\
La technique a-t-elle sa place en politique ? \\
La technique a-t-elle une histoire ? \\
La technique crée-t-elle son propre monde ? \\
La technique est-elle civilisatrice ? \\
La technique est-elle contre-nature ? \\
La technique est-elle dangereuse ? \\
La technique est-elle l'application de la science ? \\
La technique est-elle le propre de l'homme ? \\
La technique est-elle libératrice ? \\
La technique est-elle moralement neutre ? \\
La technique est-elle neutre ? \\
La technique est-elle un savoir ? \\
La technique fait-elle des miracles ? \\
La technique fait-elle violence à la nature ? \\
La technique libère-t-elle les hommes ? \\
La technique ne fait-elle qu'appliquer la science ? \\
La technique ne pose-t-elle que des problèmes techniques ? \\
La technique n'est-elle pour l'homme qu'un moyen ? \\
La technique n'est-elle qu'une application de la science ? \\
La technique n'est-elle qu'un outil au service de l'homme ? \\
La technique n'existe-elle que pour satisfaire des besoins ? \\
La technique nous éloigne-t-elle de la nature ? \\
La technique nous éloigne-t-elle de la réalité ? \\
La technique nous libère-t-elle ? \\
La technique nous oppose-t-elle à la nature ? \\
La technique nous permet-elle de comprendre la nature ? \\
La technique permet-elle de réaliser tous les désirs ? \\
La technique peut-elle améliorer l'homme ? \\
La technique peut-elle se déduire de la science ? \\
La technique peut-elle se passer de la science ? \\
La technique repose-t-elle sur le génie du technicien ? \\
La technique sert-elle nos désirs ? \\
La technologie modifie-t-elle les rapports sociaux ? \\
La théorie nous éloigne-t-elle de la réalité ? \\
La théorie peut-elle nous égarer ? \\
La tolérance a-t-elle des limites ? \\
La tolérance est-elle un concept politique ? \\
La tolérance est-elle une vertu ? \\
La tolérance peut-elle constituer un problème pour la démocratie ? \\
La transparence est-elle un idéal démocratique ? \\
L'attention caractérise-t-elle la conscience ? \\
L'autre est-il le fondement de la conscience morale ? \\
La valeur d'une action se mesure-t-elle à sa réussite ? \\
La valeur d'une théorie scientifique se mesure-t-elle à son efficacité ? \\
La valeur morale d'une action se juge-t-elle à ses conséquences ? \\
La vanité est-elle toujours sans objet ? \\
L'avenir a-t-il une réalité ? \\
L'avenir est-il imaginable ? \\
L'avenir existe-t-il ? \\
L'avenir peut-il être objet de connaissance ? \\
La vérité admet-elle des degrés ? \\
La vérité a-t-elle une histoire ? \\
La vérité demande-t-elle du courage ? \\
La vérité doit-elle toujours être démontrée ? \\
La vérité donne-t-elle le droit d'être injuste ? \\
La vérité d'une théorie dépend-elle de sa correspondance avec les faits ? \\
La vérité échappe-t-elle au temps ? \\
La vérité est-elle affaire de cohérence ? \\
La vérité est-elle affaire de croyance ou de savoir ? \\
La vérité est-elle contraignante ? \\
La vérité est-elle éternelle ? \\
La vérité est-elle intemporelle ? \\
La vérité est-elle libératrice ? \\
La vérité est-elle morale ? \\
La vérité est-elle objective ? \\
La vérité est-elle triste ? \\
La vérité est-elle une construction ? \\
La vérité est-elle une valeur ? \\
La vérité est-elle une ? \\
La vérité n'est-elle qu'une erreur rectifiée ? \\
La vérité nous contraint-elle ? \\
La vérité peut-elle être équivoque ? \\
La vérité peut-elle être tolérante ? \\
La vérité peut-elle laisser indifférent ? \\
La vérité peut-elle se définir par le consensus ? \\
La vérité rend-elle heureux ? \\
La vérité scientifique est-elle relative ? \\
La vérité se communique-t-elle ? \\
La vertu peut-elle être excessive ? \\
La vertu peut-elle être purement morale ? \\
La vertu peut-elle s'enseigner ? \\
La vertu s'enseigne-t-elle ? \\
L'aveu diminue-t-il la faute ? \\
La vie a-t-elle un sens ? \\
La vie collective est-elle nécessairement frustrante ? \\
La vie en société est-elle naturelle à l'homme ? \\
La vie en société impose-t-elle de n'être pas soi-même ? \\
La vie est-elle la valeur suprême ? \\
La vie est-elle le bien le plus précieux ? \\
La vie est-elle l'objet des sciences de la vie ? \\
La vie est-elle objet de science ? \\
La vie est-elle sacrée ? \\
La vie est-elle une valeur ? \\
La vie est-elle un roman ? \\
La vie est-elle un songe ? \\
La vie peut-elle être éternelle ? \\
La vie peut-elle être objet de science ? \\
La vie politique est-elle aliénante ? \\
La vie sexuelle est-elle volontaire ? \\
La vie sociale est-elle toujours conflictuelle ? \\
La vie sociale est-elle une comédie ? \\
La violence a-t-elle des degrés ? \\
La violence est-elle toujours destructrice ? \\
La violence peut-elle avoir raison ? \\
La violence peut-elle être gratuite ? \\
La vision peut-elle être le modèle de toute connaissance ? \\
La volonté constitue-t-elle le principe de la politique ? \\
La volonté générale est-elle la volonté de tous ? \\
La volonté peut-elle être collective ? \\
La volonté peut-elle être générale ? \\
La volonté peut-elle être indéterminée ? \\
La volonté peut-elle nous manquer ? \\
La vraie morale se moque-t-elle de la morale ? \\
Le beau a-t-il une histoire ? \\
Le beau est-il aimable ? \\
Le beau est-il toujours moral ? \\
Le beau est-il une valeur commune ? \\
Le beau est-il universel ? \\
Le beau et le bien sont-ils, au fond, identiques ? \\
Le beau peut-il être bizarre ? \\
Le besoin de métaphysique est-il un besoin de connaissance ? \\
Le besoin de vérité ? \\
Le bien commun est-il une illusion ? \\
Le bien est-ce l'utile ? \\
Le bien est-il relatif ? \\
Le bien suppose-t-il la transcendance ? \\
Le bonheur de la passion est-il sans lendemain ? \\
Le bonheur des citoyens est-il un idéal politique ? \\
Le bonheur est-il affaire de vertu ? \\
Le bonheur est-il affaire de volonté ? \\
Le bonheur est-il affaire privée ? \\
Le bonheur est-il au nombre de nos devoirs ? \\
Le bonheur est-il dans l'inconscience ? \\
Le bonheur est-il l'absence de maux ? \\
Le bonheur est-il la fin de la vie ? \\
Le bonheur est-il le bien suprême ? \\
Le bonheur est-il le but de la politique ? \\
Le bonheur est-il le prix de la vertu ? \\
Le bonheur est-il un accident ? \\
Le bonheur est-il un but politique ? \\
Le bonheur est-il un droit ? \\
Le bonheur est-il une affaire privée ? \\
Le bonheur est-il une fin morale ? \\
Le bonheur est-il une fin politique ? \\
Le bonheur est-il une valeur morale ? \\
Le bonheur est-il un idéal ? \\
Le bonheur est-il un principe politique ? \\
Le bonheur n'est-il qu'une idée ? \\
Le bonheur n'est-il qu'un idéal ? \\
Le bonheur peut-il être collectif ? \\
Le bonheur peut-il être le but de la politique ? \\
Le bonheur peut-il être un droit ? \\
Le bonheur se calcule-t-il ? \\
Le bonheur se mérite-t-il ? \\
Le cerveau pense-t-il ? \\
L'échange constitue-t-il un lien social ? \\
L'échange est-il un facteur de paix ? \\
L'échange n'a-t-il de fondement qu'économique ? \\
L'échange ne porte-t-il que sur les choses ? \\
L'échange peut-il être désintéressé ? \\
Le choix peut-il être éclairé ? \\
Le cinéma, art de la représentation ? \\
Le cinéma est-il un art comme les autres ? \\
Le cinéma est-il un art ou une industrie ? \\
Le cinéma est-il un art populaire ? \\
Le cinéma est-il un art ? \\
Le citoyen a-t-il perdu toute naturalité ?L'étranger \\
Le citoyen peut-il être à la fois libre et soumis à l'État ? \\
L'écologie est-elle un problème politique ? \\
L'écologie, une science humaine ? \\
Le combat contre l'injustice a-t-il une source morale ? \\
Le commerce adoucit-il les mœurs ? \\
Le commerce est-il pacificateur ? \\
Le commerce peut-il être équitable ? \\
Le commerce unit-il les hommes ? \\
Le concept de nature est-il un concept scientifique ? \\
Le concept d'inconscient est-il nécessaire en sciences humaines ? \\
Le conflit entre la science et la religion est-il inévitable ? \\
Le conflit est-il constitutif de la politique ? \\
Le conflit est-il la raison d'être de la politique ? \\
Le conflit est-il une maladie sociale ? \\
Le conflit ? \\
L'économie a-t-elle des lois ? \\
L'économie est-elle une science humaine ? \\
L'économie est-elle une science ? \\
Le contrat est-il au fondement de la politique ? \\
Le corps dit-il quelque chose ? \\
Le corps est-il le reflet de l'âme ? \\
Le corps est-il négociable ? \\
Le corps est-il porteur de valeurs ? \\
Le corps est-il respectable ? \\
Le corps humain est-il naturel ? \\
Le corps impose-t-il des perspectives ? \\
Le corps n'est-il que matière ? \\
Le corps n'est-il qu'un mécanisme ? \\
Le corps obéit-il à l'esprit ? \\
Le corps pense-t-il ? \\
Le corps peut-il être objet d'art ? \\
Le cosmopolitisme peut-il devenir réalité ? \\
Le cosmopolitisme peut-il être réaliste ? \\
L'écriture ne sert-elle qu'à consigner la pensée ? \\
L'écriture peut-elle porter secours à la pensée ? \\
Le désespoir est-il une faute morale ? \\
Le désir a-t-il un objet ? \\
Le désir de savoir est-il naturel ? \\
Le désir du bonheur est-il universel ? \\
Le désir est-il aveugle ? \\
Le désir est-il ce qui nous fait vivre ? \\
Le désir est-il désir de l'autre ? \\
Le désir est-il le signe d'un manque ? \\
Le désir est-il l'essence de l'homme ? \\
Le désir est-il nécessairement l'expression d'un manque ? \\
Le désir est-il par nature illimité ? \\
Le désir est-il sans limite ? \\
Le désir n'est-il que l'épreuve d'un manque ? \\
Le désir n'est-il que manque ? \\
Le désir n'est-il qu'inquiétude ? \\
Le désir peut-il être désintéressé ? \\
Le désir peut-il ne pas avoir d'objet ? \\
Le désir peut-il nous rendre libre ? \\
Le désir peut-il se satisfaire de la réalité ? \\
Le despote peut-il être éclairé ? \\
Le devoir est-il l'expression de la contrainte sociale ? \\
Le devoir rend-il libre ? \\
Le devoir se présente-t-il avec la force de l'évidence ? \\
Le devoir supprime-t-il la liberté ? \\
Le dialogue suffit-il à rompre la solitude ? \\
Le don est-il toujours généreux ? \\
Le don est-il une modalité de l'échange ? \\
Le doute est-il le principe de la méthode scientifique ? \\
Le doute est-il une faiblesse de la pensée ? \\
Le doute peut-il être méthodique ? \\
Le droit à la différence met-il en péril l'égalité des droits ? \\
Le droit doit-il être indépendant de la morale ? \\
Le droit doit-il être le seul régulateur de la vie sociale ? \\
Le droit est-il facteur de paix ? \\
Le droit est-il une science humaine ? \\
Le droit est-il une science ? \\
Le droit n'est-il qu'une justice par défaut ? \\
Le droit peut-il échapper à l'histoire ? \\
Le droit peut-il être flexible ? \\
Le droit peut-il être naturel ? \\
Le droit peut-il se fonder sur la force ? \\
Le droit peut-il se passer de la morale ? \\
Le droit sert-il à établir l'ordre ou la justice ? \\
L'éducation peut-elle être sentimentale ? \\
Le fait de vivre constitue-t-il un bien en soi ? \\
Le fait de vivre est-il un bien en soi ? \\
Le fait social est-il une chose ? \\
Le futur est-il contingent ? \\
L'égalité des hommes et des femmes est-elle une question politique ? \\
L'égalité est-elle souhaitable ? \\
L'égalité est-elle toujours juste ? \\
L'égalité est-elle une condition de la liberté ? \\
L'égalité peut-elle être une menace pour la liberté ? \\
Le génie est-il la marque de l'excellence artistique ? \\
Le goût est-il affaire d'éducation ? \\
Le goût est-il une faculté ? \\
Le goût est-il une vertu sociale ? \\
Le goût s'éduque-t-il ? \\
Le goût se forme-t-il ? \\
Le goût : certitude ou conviction ? \\
Le grand art est-il de plaire ? \\
Le hasard est-il injuste ? \\
Le hasard existe-t-il ? \\
Le hasard fait-il bien les choses ? \\
Le hasard n'est il que la mesure de notre ignorance ? \\
Le hasard n'est-il que la mesure de notre ignorance ? \\
Le jugement critique peut-il s'exercer sans culture ? \\
Le jugement de goût est-il désintéressé ? \\
Le jugement de goût est-il universel ? \\
Le jugement de valeur est-il indifférent à la vérité ? \\
Le langage est-il assimilable à un outil ? \\
Le langage est-il d'essence poétique ? \\
Le langage est-il l'auxiliaire de la pensée ? \\
Le langage est-il le lieu de la vérité ? \\
Le langage est-il logique ? \\
Le langage est-il une prise de possession des choses ? \\
Le langage est-il un instrument de connaissance ? \\
Le langage est-il un instrument ? \\
Le langage est-il un obstacle pour la pensée ? \\
Le langage masque-t-il la pensée ? \\
Le langage ne sert-il qu'à communiquer ? \\
Le langage n'est-il qu'un instrument de communication ? \\
Le langage peut-il être un obstacle à la recherche de la vérité ? \\
Le langage rapproche-t-il ou sépare-t-il les hommes ? \\
Le langage rend-il l'homme plus puissant ? \\
Le langage traduit-il la pensée ? \\
Le langage trahit-il la pensée ? \\
Le lien social peut-il être compassionnel ? \\
Le mal a-t-il des raisons ? \\
Le mal constitue-t-il une objection à l'existence de Dieu ? \\
Le mal existe-t-il ? \\
Le malheur est-il injuste ? \\
Le mariage est-il un contrat ? \\
Le méchant est-il malheureux ? \\
Le méchant peut-il être heureux ? \\
Le meilleur est-il l'ennemi du bien ? \\
Le mensonge de l'art ? \\
Le mensonge est-il la plus grande transgression ? \\
Le mensonge peut-il être au service de la vérité ? \\
Le mépris peut-il être justifié ? \\
Le mérite est-il le critère de la vertu ? \\
Le métaphysicien est-il un physicien à sa façon ? \\
Le mieux est-il l'ennemi du bien ? \\
Le moi est-il haïssable ? \\
Le moi est-il objet de connaissance ? \\
Le moi est-il une fiction ? \\
Le moi est-il une illusion ? \\
Le moi n'est-il qu'une fiction ? \\
Le moi n'est-il qu'une idée ? \\
Le monde a-t-il besoin de moi ? \\
Le monde est-il écrit en langage mathématique ? \\
Le monde est-il éternel ? \\
Le monde est-il ma représentation ? \\
Le monde est-il une marchandise ? \\
Le monde est-il un théâtre ? \\
Le monde extérieur existe-t-il ? \\
Le monde se réduit-il à ce que nous en voyons ? \\
L'émotion esthétique peut-elle se communiquer ? \\
Le mot vie a-t-il plusieurs sens ? \\
L'empathie est-elle nécessaire aux sciences sociales ? \\
L'empathie est-elle possible ? \\
L'empirisme exclut-il l'abstraction ? \\
Le mythe est-il objet de science ? \\
Le néant est-il ? \\
L'enfance est-elle ce qui doit être surmonté ? \\
L'enfance est-elle en nous ce qui doit être abandonné ? \\
L'enquête empirique rend-elle la métaphysique inutile ? \\
L'enseignement peut-il se passer d'exemples ? \\
L'enthousiasme est-il moral ? \\
L'environnement est-il un nouvel objet pour les sciences humaines ? \\
L'environnement est-il un problème politique ? \\
Le pardon peut-il être une obligation ? \\
Le partage est-il une obligation morale ? \\
Le passé a-t-il plus de réalité que l'avenir ? \\
Le passé a-t-il un intérêt ? \\
Le passé détermine-t-il notre présent ? \\
Le passé est-il ce qui a disparu ? \\
Le passé est-il réel ? \\
Le passé existe-t-il ? \\
Le passé peut-il être un objet de connaissance ? \\
Le patriotisme est-il une vertu ? \\
Le peuple est-il bête ? \\
Le peuple peut-il se tromper ? \\
Le philosophe a-t-il besoin de l'histoire ?Prouver et justifier \\
Le philosophe a-t-il des leçons à donner au politique ? \\
Le philosophe est-il le vrai politique ? \\
Le philosophe s'écarte-t-il du réel ? \\
L'épistémologie est-elle une logique de la science ? \\
Le plaisir a-t-il un rôle à jouer dans la morale ? \\
Le plaisir esthétique peut-il se partager ? \\
Le plaisir esthétique suppose-t-il une culture ? \\
Le plaisir est-il immoral ? \\
Le plaisir est-il la fin du désir ? \\
Le plaisir est-il tout le bonheur ? \\
Le plaisir est-il un bien ? \\
Le plaisir peut-il être immoral ? \\
Le plaisir peut-il être partagé ? \\
Le plaisir suffit-il au bonheur ? \\
Le poète réinvente-t-il la langue ? \\
Le politique a-t-il à régler les passions humaines ? \\
Le politique doit-il être un technicien ? \\
Le politique doit-il se soucier des émotions ? \\
Le politique peut-il faire abstraction de la morale ? \\
Le possible existe-t-il ? \\
Le pouvoir corrompt-il nécessairement ? \\
Le pouvoir corrompt-il toujours ? \\
Le pouvoir corrompt-il ? \\
Le pouvoir de l'État est-il arbitraire ? \\
Le pouvoir peut-il être limité ? \\
Le pouvoir peut-il limiter le pouvoir ? \\
Le pouvoir peut-il se déléguer ? \\
Le pouvoir peut-il se passer de sa mise en scène ? \\
Le pouvoir politique est-il nécessairement coercitif ? \\
Le pouvoir politique peut-il échapper à l'arbitraire ? \\
Le pouvoir politique repose-t-il sur un savoir ? \\
Le premier devoir de l'État est-il de se défendre ? \\
Le profit est-il la fin de l'échange ? \\
Le progrès des sciences infirme-t-il les résultats anciens ? \\
Le progrès est-il réversible ? \\
Le progrès est-il un mythe ? \\
Le progrès scientifique fait-il disparaître la superstition ? \\
Le progrès technique peut-il être aliénant ? \\
Le projet d'une paix perpétuelle est-il insensé ? \\
Le propre du vivant est-il de tomber malade ? \\
Le psychisme est-il objet de connaissance ? \\
Lequel, de l'art ou du réel, est-il une imitation de l'autre ? \\
Le raisonnement suit-il des règles ? \\
Le rapport de l'homme à son milieu a-t-il une dimension morale ? \\
Le rationalisme peut-il être une passion ? \\
Le réel est-il ce que l'on croit ? \\
Le réel est-il ce que nous expérimentons ? \\
Le réel est-il ce que nous percevons ? \\
Le réel est-il ce qui apparaît ? \\
Le réel est-il ce qui est perçu ? \\
Le réel est-il ce qui résiste ? \\
Le réel est-il inaccessible ? \\
Le réel est-il l'objet de la science ? \\
Le réel est-il objet d'interprétation ? \\
Le réel est-il rationnel ? \\
Le réel n'est-il qu'un ensemble de contraintes ? \\
Le réel obéit-il à la raison ? \\
Le réel peut-il échapper à la logique ? \\
Le réel peut-il être contradictoire ? \\
Le réel résiste-t-il à la connaissance ? \\
Le réel se limite-t-il à ce que font connaître les théories scientifiques ? \\
Le réel se limite-t-il à ce que nous percevons ? \\
Le réel se réduit-il à ce que l'on perçoit ? \\
Le réel se réduit-il à l'objectivité ? \\
Le règlement politique des conflits ? \\
Le religieux est-il inutile ? \\
Le rôle de l'État est-il de faire régner la justice ? \\
Le rôle de l'État est-il de préserver la liberté de l'individu ? \\
Le rôle de l'historien est-il de juger ? \\
Le rôle des théories est-il d'expliquer ou de décrire ? \\
Le roman peut-il être philosophique ? \\
L'erreur est-elle humaine ? \\
L'erreur peut-elle jouer un rôle dans la connaissance scientifique ? \\
Les acteurs de l'histoire en sont-ils les auteurs ? \\
Les affects sont-ils déraisonnables ? \\
Les affects sont-ils des objets sociologiques ? \\
Le sage a-t-il besoin d'autrui ? \\
Les agents sociaux poursuivent-ils l'utilité ? \\
Les agents sociaux sont-ils rationnels ? \\
Le salut vient-il de la raison ? \\
Les animaux échappent-ils à la moralité ? \\
Les animaux ont-ils des droits ? \\
Les animaux pensent-ils ? \\
Les animaux peuvent-ils avoir des droits ? \\
Les apparences font-elles partie du monde ? \\
Les apparences sont-elles toujours trompeuses ? \\
Les arts admettent-ils une hiérarchie ? \\
Les arts communiquent-ils entre eux ? \\
Les arts ont-ils besoin de théorie ? \\
Le savoir a-t-il besoin d'être fondé ? \\
Le savoir a-t-il des degrés ? \\
Le savoir émancipe-t-il ? \\
Le savoir est-il libérateur ? \\
Le savoir exclut-il toute forme de croyance ? \\
Le savoir rend-il libre ? \\
Le savoir se vulgarise-t-il ? \\
Les beaux-arts sont-ils compatibles entre eux ? \\
Les bêtes travaillent-elles ? \\
Le scepticisme a-t-il des limites ? \\
Les choses ont-elles une essence ? \\
Les coïncidences ont-elles des causes ? \\
Les conflits menacent-ils la société ? \\
Les conflits politiques ne sont-ils que des conflits sociaux ? \\
Les conflits sociaux sont-ils des conflits de classe ? \\
Les conflits sociaux sont-ils des conflits politiques ? \\
Les connaissances scientifiques peuvent-elles être à la fois vraies et provisoires ? \\
Les connaissances scientifiques peuvent-elles être vulgarisées ? \\
Les considérations morales ont-elles leur place en politique ? \\
Les convictions d'autrui sont-elles un argument ? \\
Les cultures sont-elles incommensurables ? \\
Les devoirs de l'homme varient-ils selon la culture ? \\
Les devoirs de l'homme varient-ils selon les cultures ? \\
Les droits de l'homme ont-ils un fondement moral ? \\
Les droits de l'homme sont-ils les droits de la femme ? \\
Les droits de l'homme sont-ils une abstraction ? \\
Les droits naturels imposent-ils une limite à la politique ? \\
Les échanges économiques sont-ils facteurs de paix ? \\
Les échanges, facteurs de paix ? \\
Les échanges favorisent-ils la paix ? \\
Les échanges sont-ils facteurs de paix ? \\
Le sensible est-il communicable ? \\
Le sensible est-il irréductible à l'intelligible ? \\
Le sensible peut-il être connu ? \\
Le sens moral est-il naturel ? \\
Le sentiment d'injustice est-il naturel ? \\
Les entités mathématiques sont-elles des fictions ? \\
Les êtres vivants sont-ils des machines ? \\
Les événements historiques sont-ils de nature imprévisible ? \\
Les faits existent-ils indépendamment de leur établissement par l'esprit humain ? \\
Les faits parlent-ils d'eux-mêmes ? \\
Les faits peuvent-ils faire autorité ? \\
Les fins de la technique sont-elles techniques ? \\
Les habitudes nous forment-elles ? \\
Les hommes n'agissent-ils que par intérêt ? \\
Les hommes naissent-ils libres ? \\
Les hommes ont-ils besoin de maîtres ? \\
Les hommes savent-ils ce qu'ils désirent ? \\
Les hommes sont-ils des animaux ? \\
Les hommes sont-ils faits pour s'entendre ? \\
Les hommes sont-ils frères ? \\
Les hommes sont-ils naturellement sociables ? \\
Les hommes sont-ils seulement le produit de leur culture ? \\
Les hypothèses scientifiques ont-elles pour nature d'être confirmées ou infirmées ? \\
Les idées ont-elles une existence éternelle ? \\
Les idées ont-elles une histoire ? \\
Les idées ont-elles une réalité ? \\
Le silence a-t-il un sens ? \\
Le silence signifie-t-il toujours l'échec du langage ? \\
Les images empêchent-elles de penser ? \\
Les images ont-elles un sens ? \\
Les inégalités de la nature doivent-elles être compensées ? \\
Les inégalités menacent-elles la société ? \\
Les inégalités sociales sont-elles inévitables ? \\
Les inégalités sociales sont-elles naturelles ? \\
Le singulier est-il objet de connaissance ? \\
Les intérêts particuliers peuvent-ils tempérer l'autorité politique ? \\
Les langues que nous parlons sont-elles imparfaites ? \\
Les lois de la nature sont-elles contingentes ? \\
Les lois de la nature sont-elles de simples régularités ? \\
Les lois de la nature sont elles nécessaires ? \\
Les lois nous rendent-elles meilleurs ? \\
Les lois scientifiques sont-elles des lois de la nature ? \\
Les lois sont-elles seulement utiles ? \\
Les machines nous rendent-elles libres ? \\
Les machines pensent-elles ? \\
Les mathématiques consistent-elles seulement en des opérations de l'esprit ? \\
Les mathématiques ont-elles affaire au réel ? \\
Les mathématiques ont-elles besoin d'un fondement ? \\
Les mathématiques parlent-elles du réel ? \\
Les mathématiques se réduisent-elles à une pensée cohérente ? \\
Les mathématiques sont-elles réductibles à la logique ? \\
Les mathématiques sont-elles un instrument ? \\
Les mathématiques sont-elles un jeu de l'esprit ? \\
Les mathématiques sont-elles un langage ? \\
Les mathématiques sont-elles utiles au philosophe ? \\
Les méchants peuvent-ils être amis ? \\
Les mots disent-ils les choses ? \\
Les mots expriment-ils les choses ? \\
Les mots nous éloignent-ils des choses ? \\
Les mots parviennent-ils à tout exprimer ? \\
Les mots sont-ils trompeurs ? \\
Les nombres gouvernent-ils le monde ? \\
Les noms propres ont-ils une signification ? \\
Les nouvelles technologies transforment-elles l'idée de l'art ? \\
Les objets sont-ils colorés ? \\
Les œuvres d'art ont-elles besoin d'un commentaire ? \\
Les œuvres d'art sont-elles des choses ? \\
Les œuvres d'art sont-elles des réalités comme les autres ? \\
Les œuvres d'art sont-elles éternelles ? \\
Le soleil se lèvera-t-il demain ? \\
Le souci d'autrui résume-t-il la morale ? \\
Le souci de soi est-il une attitude morale ? \\
Le souci du bien-être est-il politique ? \\
L'espace nous sépare-t-il ? \\
Les passions ont-elles une place en politique ? \\
Les passions peuvent-elles être raisonnables ? \\
Les passions sont-elles toujours mauvaises ? \\
Les passions sont-elles toutes bonnes ? \\
Les passions sont-elles un obstacle à la vie sociale ? \\
Les passions s'opposent-elles à la raison ? \\
L'espérance est-elle une vertu ? \\
Les personnages de fiction peuvent-ils avoir une réalité ? \\
Les peuples font-ils l'histoire ? \\
Les peuples ont-ils les gouvernements qu'ils méritent ? \\
Les phénomènes inconscients sont-ils réductibles à une mécanique cérébrale ? \\
Les philosophes doivent-ils être rois ? \\
Les philosophies se classent-elles ? \\
L'espoir peut-il être raisonnable ? \\
Le sport : s'accomplir ou se dépasser ? \\
Les principes de la morale dépendent-ils de la culture ? \\
Les principes d'une science sont-ils des conventions ? \\
Les principes sont-ils indémontrables ?Qu'est-ce qu'être ensemble ? \\
L'esprit dépend-il du corps ? \\
L'esprit domine-t-il la matière ? \\
L'esprit est-il matériel ? \\
L'esprit est-il mieux connu que le corps ? \\
L'esprit est-il objet de science ? \\
L'esprit est-il plus aisé à connaître que le corps ? \\
L'esprit est-il plus difficile à connaître que la matière ? \\
L'esprit est-il une chose ? \\
L'esprit est-il une machine ? \\
L'esprit est-il un ensemble de facultés ? \\
L'esprit est-il une partie du corps ? \\
L'esprit humain progresse-t-il ? \\
L'esprit n'a-t-il jamais affaire qu'à lui-même ? \\
L'esprit peut-il être divisé ? \\
L'esprit peut-il être malade ? \\
L'esprit peut-il être mesuré ? \\
L'esprit peut-il être objet de science ? \\
L'esprit s'explique-t-il par une activité cérébrale ? \\
Les problèmes politiques peuvent-ils se ramener à des problèmes techniques ? \\
Les problèmes politiques sont-ils des problèmes techniques ? \\
Les progrès de la technique sont-ils nécessairement des progrès de la raison ? \\
Les progrès techniques constituent-ils des progrès de la civilisation ? \\
Les propositions métaphysiques sont-elles des illusions ? \\
Les proverbes enseignent-ils quelque chose ? \\
Les proverbes nous instruisent-ils moralement ? \\
Les qualités sensibles sont-elles dans les choses ou dans l'esprit ? \\
Les questions métaphysiques ont-elles un sens ? \\
Les rapports entre les hommes sont-ils des rapports de force ? \\
Les religions peuvent-elles être objets de science ? \\
Les religions sont-elles des illusions ? \\
Les révolutions techniques suscitent-elles des révolutions dans l'art ? \\
Les scélérats peuvent-ils être heureux ? \\
Les sciences décrivent-elles le réel ? \\
Les sciences de la vie visent-elles un objet irréductible à la matière ? \\
Les sciences de l'homme ont-elles inventé leur objet ? \\
Les sciences de l'homme permettent-elles d'affiner la notion de responsabilité ? \\
Les sciences de l'homme peuvent-elles expliquer l'impuissance de la liberté ? \\
Les sciences de l'homme rendent-elles l'homme prévisible ? \\
Les sciences doivent-elle prétendre à l'unification ? \\
Les sciences forment-elle un système ? \\
Les sciences humaines doivent-elles être transdisciplinaires ? \\
Les sciences humaines éliminent-elles la contingence du futur ? \\
Les sciences humaines nous protègent-elles de l'essentialisme ? \\
Les sciences humaines ont-elles un objet commun ? \\
Les sciences humaines permettent-elles de comprendre la vie d'un homme ? \\
Les sciences humaines peuvent-elles adopter les méthodes des sciences de la nature ? \\
Les sciences humaines peuvent-elles se passer de la notion d'inconscient ? \\
Les sciences humaines présupposent-elles une définition de l'homme ? \\
Les sciences humaines sont-elles des sciences de la nature humaine ? \\
Les sciences humaines sont-elles des sciences de la vie humaine ? \\
Les sciences humaines sont-elles des sciences d'interprétation ? \\
Les sciences humaines sont-elles des sciences ? \\
Les sciences humaines sont-elles explicatives ou compréhensives ? \\
Les sciences humaines sont-elles normatives ? \\
Les sciences humaines sont-elles relativistes ? \\
Les sciences humaines sont-elles subversives ? \\
Les sciences humaines traitent-elles de l'homme ? \\
Les sciences humaines traitent-elles de l'individu ? \\
Les sciences humaines transforment-elles la notion de causalité ? \\
Les sciences ne sont-elles qu'une description du monde ? \\
Les sciences nous donnent-elles des normes ? \\
Les sciences ont-elles besoin de principes fondamentaux ? \\
Les sciences ont-elles besoin d'une fondation métaphysique ? \\
Les sciences permettent-elles de connaître la réalité-même ? \\
Les sciences peuvent-elles exclure toute notion de finalité ? \\
Les sciences peuvent-elles penser l'individu ? \\
Les sciences sociales ont-elles un objet ? \\
Les sciences sociales peuvent-elles être expérimentales ? \\
Les sciences sociales sont-elles nécessairement inexactes ? \\
Les sciences sont-elles une description du monde ? \\
Les sens jugent-ils ? \\
Les sens nous trompent-ils ? \\
Les sens peuvent-ils nous tromper ? \\
Les sens sont-ils source d'illusion ? \\
Les sens sont-ils trompeurs ? \\
Les sentiments ont-ils une histoire ? \\
Les sentiments peuvent-ils s'apprendre ? \\
Les sociétés évoluent-elles ? \\
Les sociétés ont-elles un inconscient ? \\
Les sociétés sont-elles hiérarchisables ? \\
Les sociétés sont-elles imprévisibles ? \\
Les structures expliquent-elles tout ? \\
Les théories scientifiques décrivent-elles la réalité ? \\
Les théories scientifiques sont-elles vraies ? \\
L'esthétique est-elle une métaphysique de l'art ? \\
Le sujet n'est-il qu'une fiction ? \\
Le sujet peut-il s'aliéner par un libre choix ? \\
Les valeurs morales ont-elles leur origine dans la raison ? \\
Les vérités scientifiques sont-elles relatives ? \\
Les vérités sont-elles intemporelles ? \\
Les vertus ne sont-elles que des vices déguisés ? \\
Les vices privés peuvent-ils faire le bien public ? \\
Les vivants peuvent-ils se passer des morts ? \\
Le tableau ? \\
L'État a-t-il des intérêts propres ? \\
L'État a-t-il le droit de contrôler notre habillement ? \\
L'État a-t-il pour but de maintenir l'ordre ? \\
L'État a-t-il pour finalité de maintenir l'ordre ? \\
L'État a-t-il tous les droits ? \\
L'État contribue-t-il à pacifier les relations entre les hommes ? \\
L'État doit-il disparaître ? \\
L'État doit-il éduquer le citoyen ? \\
L'État doit-il éduquer le peuple ? \\
L'État doit-il éduquer les citoyens ? \\
L'État doit-il être fort ? \\
L'État doit-il être le plus fort ? \\
L'État doit-il être neutre ? \\
L'État doit-il être sans pitié ? \\
L'État doit-il faire le bonheur des citoyens ? \\
L'État doit-il nous rendre meilleurs ? \\
L'État doit-il préférer l'injustice au désordre ? \\
L'État doit-il reconnaître des limites à sa puissance ? \\
L'État doit-il se mêler de religion ? \\
L'État doit-il se préoccuper des arts ? \\
L'État doit-il se préoccuper du bonheur des citoyens ? \\
L'État doit-il se soucier de la morale ? \\
L'État doit-il veiller au bonheur des individus  ? \\
L'État est-il appelé à disparaître ? \\
L'État est-il au service de la société ? \\
L'État est-il fin ou moyen ? \\
L'État est-il le garant de la propriété privée ? \\
L'État est-il l'ennemi de la liberté ? \\
L'État est-il l'ennemi de l'individu ? \\
L'État est-il libérateur ? \\
L'État est-il toujours juste ? \\
L'État est-il un arbitre ? \\
L'État est-il un mal nécessaire ? \\
L'État est-il un moindre mal ? \\
L'État est-il un tiers impartial ? \\
L'État est-il un « monstre froid » ? \\
L'État n'est-il qu'un instrument de domination ? \\
L'État nous rend-il meilleurs ? \\
L'État peut-il créer la liberté ? \\
L'État peut-il être impartial ? \\
L'État peut-il être indifférent à la religion ? \\
L'État peut-il être libéral ? \\
L'État peut-il poursuivre une autre fin que sa propre puissance ? \\
L'État peut-il renoncer à la violence ? \\
Le technicien n'est-il qu'un exécutant ? \\
Le temps détruit-il tout ? \\
Le temps est-il destructeur ? \\
Le temps est-il en nous ou hors de nous ? \\
Le temps est-il essentiellement destructeur ? \\
Le temps est-il la marque de notre impuissance ? \\
Le temps est-il notre allié ? \\
Le temps est-il notre ennemi ? \\
Le temps est-il une contrainte ? \\
Le temps est-il une dimension de la nature ? \\
Le temps est-il une prison ? \\
Le temps est-il une réalité ? \\
Le temps existe-t-il ? \\
Le temps ne fait-il que passer ? \\
Le temps n'est-il pour l'homme que ce qui le limite ? \\
Le temps nous appartient-il ? \\
Le temps nous est-il compté ? \\
Le temps passe-t-il ? \\
Le temps s'écoule-t-il ? \\
Le temps se laisse-t-il décrire logiquement ? \\
L'éternité n'est-elle qu'une illusion ? \\
Le terrorisme est-il un acte de guerre ? \\
L'éthique est-elle affaire de choix ? \\
L'éthique suppose-t-elle la liberté ? \\
Le tout est-il la somme de ses parties ? \\
Le travail a-t-il une valeur morale ? \\
Le travail est-il le propre de l'homme ? \\
Le travail est-il libérateur ? \\
Le travail est-il nécessaire au bonheur ? \\
Le travail est-il toujours une activité productrice ? \\
Le travail est-il un besoin ? \\
Le travail est-il une fin ? \\
Le travail est-il une marchandise ? \\
Le travail est-il une valeur morale ? \\
Le travail est-il une valeur ? \\
Le travail est-il un rapport naturel de l'homme à la nature ? \\
Le travail fait-il de l'homme un être moral ? \\
Le travail fonde-t-il la propriété ? \\
Le travaille libère-t-il ? \\
Le travail manuel est-il sans pensée ? \\
Le travail nous rend-il solidaires ? \\
Le travail rapproche-t-il les hommes ? \\
Le travail unit-il ou sépare-t-il les hommes ? \\
L'être en tant qu'être est-il connaissable ? \\
L'être humain désire-t-il naturellement connaître ? \\
L'être humain est-il la mesure de toute chose ? \\
L'être se confond-il avec l'être perçu ? \\
L'étude de l'histoire conduit-elle à désespérer l'homme ? \\
Le vainqueur a-t-il tous les droits ? \\
L'événement historique a-t-il un sens par lui-même ? \\
L'événement manque-t-il d'être ? \\
L'évidence a-t-elle une valeur absolue ? \\
L'évidence est-elle critère de vérité ? \\
L'évidence se passe-t-elle de démonstration ? \\
Le vivant a-t-il des droits ? \\
Le vivant est-il entièrement connaissable ? \\
Le vivant est-il entièrement explicable ? \\
Le vivant est-il réductible au physico-chimique ? \\
Le vivant est-il un objet de science comme un autre ? \\
Le vivant n'est-il que matière ? \\
Le vivant n'est-il qu'une machine ingénieuse ? \\
Le vivant obéit-il à des lois ? \\
Le vivant obéit-il à une nécessité ? \\
Le vrai a-t-il une histoire ? \\
Le vrai doit-il être démontré ? \\
Le vrai est-il à lui-même sa propre marque ? \\
Le vrai et le bien sont-ils analogues ? \\
Le vrai peut-il rester invérifiable ? \\
Le vrai se perçoit-il ? \\
Le vrai se réduit-il à ce qui est vérifiable ? \\
Le vrai se réduit-il à l'utile ? \\
L'exécution d'une œuvre d'art est-elle toujours une œuvre d'art ? \\
L'exigence de vérité a-t-elle un sens moral ? \\
L'existence a-t-elle un sens ? \\
L'existence de l'État dépend-elle d'un contrat ? \\
L'existence du mal met-elle en échec la raison ? \\
L'existence est-elle pensable ? \\
L'existence est-elle une propriété ? \\
L'existence est-elle un jeu ? \\
L'existence est-elle vaine ? \\
L'existence se démontre-t-elle ? \\
L'existence se laisse-t-elle penser ? \\
L'expérience a-t-elle le même sens dans toutes les sciences ? \\
L'expérience d'autrui nous est-elle utile ? \\
L'expérience démontre-t-elle quelque chose ? \\
L'expérience directe est-elle une connaissance ? \\
L'expérience enseigne-elle quelque chose ? \\
L'expérience, est-ce l'observation ? \\
L'expérience instruit-elle ? \\
L'expérience nous apprend-elle quelque chose ? \\
L'expérience peut-elle avoir raison des principes ? \\
L'expérience peut-elle contredire la théorie ? \\
L'expérience rend-elle raisonnable ? \\
L'expérience rend-elle responsable ? \\
L'expérience sensible est-elle la seule source légitime de connaissance ? \\
L'expérience suffit-elle pour établir une vérité ? \\
L'habitude a-t-elle des vertus ? \\
L'habitude est-elle notre guide dans la vie ? \\
L'histoire a-t-elle des lois ? \\
L'histoire a-t-elle un commencement et une fin ? \\
L'Histoire a-t-elle un commencement ? \\
L'histoire a-t-elle une fin ? \\
L'histoire a-t-elle un sens ? \\
L'histoire de l'art est-elle celle des styles ? \\
L'histoire de l'art est-elle finie ? \\
L'histoire de l'art peut-elle arriver à son terme ? \\
L'histoire des arts est-elle liée à l'histoire des techniques ? \\
L'histoire des sciences est-elle une histoire ? \\
L'histoire du droit est-elle celle du progrès de la justice ? \\
L'histoire est-elle avant tout mémoire ? \\
L'histoire est-elle déterministe ? \\
L'histoire est-elle écrite par les vainqueurs ? \\
L'histoire est-elle la connaissance du passé humain ? \\
L'histoire est-elle la mémoire de l'humanité ? \\
L'histoire est-elle la science de ce qui ne se répète jamais ? \\
L'histoire est-elle la science du passé ? \\
L'histoire est-elle le récit objectif des faits passés  ? \\
L'histoire est-elle le règne du hasard ? \\
L'histoire est-elle le théâtre des passions ? \\
L'histoire est-elle rationnelle ? \\
L'histoire est-elle tragique ? \\
L'histoire est-elle une explication ou une justification du passé ? \\
L'histoire est-elle une science comme les autres ? \\
L'histoire est-elle une science ? \\
L'histoire est-elle un genre littéraire ? \\
L'histoire est-elle un roman vrai ? \\
L'histoire est-elle utile à la politique ? \\
L'histoire est-elle utile ? \\
L'histoire jugera-t-elle ? \\
L'histoire n'a-t-elle pour objet que le passé ? \\
L'histoire n'est-elle que la connaissance du passé ? \\
L'histoire n'est-elle qu'un récit ? \\
L'histoire nous appartient-elle ? \\
L'histoire obéit-elle à des lois ? \\
L'histoire peut-elle être contemporaine ? \\
L'histoire peut-elle être universelle ? \\
L'histoire peut-elle se répéter ? \\
L'histoire se répète-t-elle ? \\
L'histoire universelle est-elle l'histoire des guerres ? \\
L'histoire : enquête ou science ? \\
L'histoire : science ou récit ? \\
L'historien peut-il être impartial ? \\
L'historien peut-il se passer du concept de causalité ? \\
L'homme aime-t-il la justice pour elle-même ? \\
L'homme a-t-il besoin de l'art ? \\
L'homme a-t-il besoin de religion ? \\
L'homme a-t-il une nature ? \\
L'homme a-t-il une place dans la nature ? \\
L'homme des droits de l'homme n'est-il qu'une fiction ? \\
L'homme des sciences de l'homme ? \\
L'homme est-il chez lui dans l'univers ? \\
L'homme est-il fait pour le travail ? \\
L'homme est-il la mesure de toute chose ? \\
L'homme est-il la mesure de toutes choses ? \\
L'homme est-il l'artisan de sa dignité ? \\
L'homme est-il le seul être à avoir une histoire ? \\
L'homme est-il le sujet de son histoire ? \\
L'homme est-il objet de science ? \\
L'homme est-il prisonnier du temps ? \\
L'homme est-il raisonnable par nature ? \\
L'homme est-il un animal comme les autres ? \\
L'homme est-il un animal comme un autre ? \\
L'homme est-il un animal dénaturé ? \\
L'homme est-il un animal politique ? \\
L'homme est-il un animal rationnel ? \\
L'homme est-il un animal social ? \\
L'homme est-il un animal ? \\
L'homme est-il un corps pensant ? \\
L'homme est-il un être de devoir ? \\
L'homme est-il un être social par nature ? \\
L'homme est-il un loup pour l'homme ? \\
L'homme et la nature sont-ils commensurables ? \\
L'homme injuste peut-il être heureux ? \\
L'homme libre est-il un homme seul ? \\
L'homme n'est-il qu'un animal comme les autres ? \\
L'homme pense-t-il toujours ? \\
L'homme peut-il changer ? \\
L'homme peut-il être libéré du besoin ? \\
L'homme peut-il se représenter un monde sans l'homme ? \\
L'homme se réalise-t-il dans le travail ? \\
L'homme se reconnaît-il mieux dans le travail ou dans le loisir ? \\
L'honneur ? \\
L'hospitalité a-t-elle un sens politique ? \\
L'hospitalité est-elle un devoir ? \\
L'humanité est-elle aimable ? \\
Libre arbitre et déterminisme sont-ils compatibles ? \\
L'idéal moral est-il vain ? \\
L'idée de bonheur collectif a-t-elle un sens ? \\
L'idée de destin a-t-elle un sens ? \\
L'idée de rétribution est-elle nécessaire à la morale ? \\
L'identité personnelle est-elle donnée ou construite ? \\
L'identité relève-telle du champ politique ? \\
L'ignorance est-elle préférable à l'erreur ? \\
L'ignorance nous excuse-t-elle ? \\
L'ignorance peut-elle être une excuse ? \\
L'imagination a-t-elle des limites ? \\
L'imagination enrichit-elle la connaissance ? \\
L'imagination est-elle libre ? \\
L'imagination est-elle maîtresse d'erreur et de fausseté ? \\
L'imagination nous éloigne-t-elle du réel ? \\
L'imitation a-t-elle une fonction morale ? \\
L'impartialité est-elle toujours désirable ? \\
L'incertitude est-elle dans les choses ou dans les idées ? \\
L'incertitude interdit-elle de raisonner ? \\
L'inconscient a-t-il une histoire ? \\
L'inconscient est-il dans l'âme ou dans le corps ? \\
L'inconscient est-il l'animal en nous ? \\
L'inconscient est-il un concept scientifique ? \\
L'inconscient est-il une dimension de la conscience ? \\
L'inconscient est-il une excuse ? \\
L'inconscient n'est-il qu'un défaut de conscience ? \\
L'inconscient n'est-il qu'une hypothèse ? \\
L'inconscient peut-il se manifester ? \\
L'indifférence peut-elle être une vertu ? \\
L'individualisme a-t-il sa place en politique ? \\
L'individualisme est-il un égoïsme ? \\
L'individu a-t-il des droits ? \\
L'infinité de l'univers a-t-elle de quoi nous effrayer ? \\
L'injustice est-elle préférable au désordre ? \\
L'inquiétude peut-elle définir l'existence humaine ? \\
L'inquiétude peut-elle devenir l'existence humaine ? \\
L'instant de la décision est-il une folie ? \\
L'instruction est-elle facteur de moralité ? \\
L'insurrection est-elle un droit ? \\
L'intention morale suffit-elle à constituer la valeur morale de l'action ? \\
L'interdit est-il au fondement de la culture ? \\
L'intérêt constitue-t-il l'unique lien social ? \\
L'intérêt de la société l'emporte-t-il sur celui des individus ? \\
L'intérêt est-il le principe de tout échange ? \\
L'intérêt général est-il la somme des intérêts particuliers ? \\
L'intérêt général est-il le bien commun ? \\
L'intérêt gouverne-t-il le monde ? \\
L'intérêt peut-il être une valeur morale ? \\
L'intérêt public est-il une illusion ? \\
L'intériorité est-elle un mythe ? \\
L'interprétation est-elle sans fin ? \\
L'interprétation est-elle un art ? \\
L'interprétation est-elle une activité sans fin ? \\
L'interprétation est-elle une science ? \\
L'interprète sait-il ce qu'il cherche ? \\
L'introspection est-elle une connaissance ? \\
L'intuition a-t-elle une place en logique ? \\
L'inutile a-t-il de la valeur ? \\
L'inutile est-il sans valeur ? \\
L'irrationnel est-il pensable ? \\
L'irrationnel est-il toujours absurde ? \\
L'irrationnel existe-t-il ? \\
L'obéissance est-elle compatible avec la liberté ? \\
L'objectivité historique est-elle synonyme de neutralité ? \\
L'objet du désir en est-il la cause ? \\
L'obligation morale peut-elle se réduire à une obligation sociale ? \\
L'œuvre d'art a-t-elle un sens ? \\
L'œuvre d'art doit-elle être belle ? \\
L'œuvre d'art doit-elle nous émouvoir ? \\
L'œuvre d'art donne-t-elle à penser ? \\
L'œuvre d'art échappe-t-elle au temps ? \\
L'œuvre d'art échappe-t-elle nécessairement au temps ? \\
L'œuvre d'art est-elle l'expression d'une idée ? \\
L'œuvre d'art est-elle toujours destinée à un public ? \\
L'œuvre d'art est-elle une belle apparence ? \\
L'œuvre d'art est-elle une marchandise ? \\
L'œuvre d'art est-elle un objet d'échange ? \\
L'œuvre d'art est-elle un symbole ? \\
L'œuvre d'art instruit-elle ? \\
L'œuvre d'art nous apprend-elle quelque chose ? \\
L'œuvre d'art représente-t-elle quelque chose ? \\
L'opinion a-t-elle nécessairement tort ? \\
L'opinion est-elle un savoir ? \\
L'ordinaire est-il ennuyeux ? \\
L'ordre du vivant est-il façonné par le hasard ? \\
L'ordre est-il dans les choses ? \\
L'ordre politique exclut-il la violence ? \\
L'ordre politique peut-il exclure la violence ? \\
L'ordre social peut-il être juste ? \\
L'origine des langues est-elle un faux problème ? \\
L'oubli est-il un échec de la mémoire ? \\
L'unanimité est-elle un critère de légitimité ? \\
L'unanimité est-elle un critère de vérité ? \\
L'unité des sciences humaines ? \\
L'utilité est-elle étrangère à la morale ? \\
L'utilité est-elle une valeur morale ? \\
L'utilité peut-elle être le principe de la moralité ? \\
L'utilité peut-elle être un critère pour juger de la valeur de nos actions ? \\
L'utopie a-t-elle une signification politique ? \\
Ma liberté s'arrête-t-elle où commence celle des autres ? \\
Ma parole m'engage-t-elle ? \\
Mon corps est-il ma propriété ? \\
Mon corps est-il naturel ? \\
Mon corps fait-il obstacle à ma liberté ? \\
Mon corps m'appartient-il ? \\
Mon prochain est-il mon semblable ? \\
Montrer, est-ce démontrer ? \\
Morale et politique sont-elles indépendantes ? \\
Naît-on sujet ou le devient-on ? \\
N'apprend-on que par l'expérience ? \\
N'avons-nous affaire qu'au réel ? \\
N'avons-nous de devoir qu'envers autrui ? \\
N'échange-t-on que ce qui a de la valeur ? \\
N'échange-t-on que des symboles ? \\
N'échange-t-on que par intérêt ? \\
Ne désire-t-on que ce dont on manque ? \\
Ne désirons-nous que ce qui est bon pour nous ? \\
Ne désirons-nous que les choses que nous estimons bonnes ? \\
Ne prêche-t-on que les convertis ? \\
Ne sait-on rien que par expérience ? \\
Ne sommes-nous véritablement maîtres que de nos pensées ? \\
N'est-on juste que par crainte du châtiment ? \\
Ne veut-on que ce qui est désirable ? \\
Ne vit-on bien qu'avec ses amis ? \\
N'existe-t-il que des individus ? \\
N'existe-t-il que le présent ? \\
N'exprime t-on que ce dont on a conscience ? \\
N'interprète-t-on que ce qui est équivoque ? \\
Nos convictions morales sont-elles le simple reflet de notre temps ? \\
Nos désirs nous appartiennent-ils ? \\
Nos désirs nous opposent-ils ? \\
Nos pensées dépendent-elles de nous ? \\
Nos pensées sont-elles entièrement en notre pouvoir ? \\
Nos sens nous trompent-ils ? \\
Notre connaissance du réel se limite-t-elle au savoir scientifique ? \\
Notre corps pense-t-il ? \\
Notre existence a-t-elle un sens si l'histoire n'en a pas ? \\
Notre ignorance nous excuse-t-elle ? \\
Notre liberté de pensée a-t-elle des limites ? \\
Notre rapport au monde est-il essentiellement technique ? \\
Notre rapport au monde peut-il n'être que technique ? \\
N'y a-t-il d'amitié qu'entre égaux ? \\
N'y a-t-il de beauté qu'artistique ? \\
N'y a t-il de bonheur que dans l'instant ? \\
N'y a-t-il de bonheur qu'éphémère ? \\
N'y a-t-il de certitude que mathématique ? \\
N'y a-t-il de connaissance que de l'universel ? \\
N'y a-t-il de démocratie que représentative ? \\
N'y a-t-il de devoirs qu'envers autrui ? \\
N'y a-t-il de droit qu'écrit ? \\
N'y a-t-il de droits que de l'homme ? \\
N'y a-t-il de foi que religieuse ? \\
N'y a-t-il de propriété que privée ? \\
N'y a-t-il de rationalité que scientifique ? \\
N'y a-t-il de réalité que de l'individuel ? \\
N'y a-t-il de savoir que livresque ? \\
N'y a-t-il de science qu'autant qu'il s'y trouve de mathématique ? \\
N'y a-t-il de science que de ce qui est mathématisable ? \\
N'y a-t-il de science que du général ? \\
N'y a-t-il de science que du mesurable ? \\
N'y a-t-il de science qu'exacte ? \\
N'y a-t-il de sens que par le langage ? \\
N'y a-t-il de vérité que scientifique ? \\
N'y a-t-il de vérité que vérifiable ? \\
N'y a-t-il de vérités que scientifiques ? \\
N'y a-t-il de vrai que le vérifiable ? \\
N'y a-t-il que des individus ? \\
N'y a-t-il qu'une substance ? \\
N'y a-t-il qu'un seul monde ? \\
Obéir, est-ce se soumettre ? \\
Où commence la servitude ? \\
Où commence la violence ? \\
Où commence l'interprétation ? \\
Où commence ma liberté ? \\
Où est le passé ? \\
Où est le pouvoir ? \\
Où est l'esprit ? \\
Où est mon esprit ? \\
Où est-on quand on pense ? \\
Où s'arrête l'espace public ? \\
Où sont les relations ? \\
Où suis-je quand je pense ? \\
Où suis-je ? \\
Par le langage, peut-on agir sur la réalité ? \\
Parler, est-ce agir ? \\
Parler, est-ce communiquer ? \\
Parler, est-ce donner sa parole ? \\
Parler, est-ce ne pas agir ? \\
Parler, n'est-ce que désigner ? \\
Par où commencer ? \\
Par quoi un individu se distingue-t-il d'un autre ? \\
Peindre, est-ce nécessairement feindre ? \\
Penser est-ce calculer ? \\
Penser, est-ce calculer ? \\
Penser, est-ce désobéir ? \\
Penser, est-ce dire non ? \\
Penser, est-ce se parler à soi-même ? \\
Penser est-il assimilable à un travail ? \\
Penser le rien, est-ce ne rien penser ? \\
Penser par soi-même, est-ce être l'auteur de ses pensées ? \\
Penser peut-il nous rendre heureux ? \\
Penser requiert-il un corps ? \\
Pense-t-on jamais seul ? \\
Pensez-vous que vous avez une âme ? \\
Percevoir est-ce connaître ? \\
Percevoir, est-ce connaître ? \\
Percevoir, est-ce interpréter ? \\
Percevoir, est-ce juger ? \\
Percevoir, est-ce reconnaître ? \\
Percevoir, est-ce savoir ? \\
Percevoir, est-ce s'ouvrir au monde ? \\
Percevoir s'apprend-il ? \\
Percevons-nous les choses telles qu'elles sont ? \\
Perçoit-on le réel tel qu'il est ? \\
Perçoit-on le réel ? \\
Perçoit-on les choses comme elles sont ? \\
Peut-être se mettre à la place de l'autre ? \\
Peut-il être moral de tuer ? \\
Peut-il être préférable de ne pas savoir ? \\
Peut-il exister une action désintéressée ? \\
Peut-il y avoir conflit entre nos devoirs ? \\
Peut-il y avoir de bons tyrans ? \\
Peut-il y avoir de la politique sans conflit ? \\
Peut-il y avoir des échanges équitables ? \\
Peut-il y avoir des expériences métaphysiques ? \\
Peut-il y avoir des lois de l'histoire ? \\
Peut-il y avoir des lois injustes ? \\
Peut-il y avoir des modèles en morale ? \\
Peut-il y avoir des vérités partielles ? \\
Peut-il y avoir esprit sans corps ? \\
Peut-il y avoir savoir-faire sans savoir ? \\
Peut-il y avoir science sans intuition du vrai ? \\
Peut-il y avoir un art conceptuel ? \\
Peut-il y avoir un droit à désobéir ? \\
Peut-il y avoir un droit de la guerre ? \\
Peut-il y avoir une histoire universelle ? \\
Peut-il y avoir une philosophie applicable ? \\
Peut-il y avoir une philosophie appliquée ? \\
Peut-il y avoir une philosophie politique sans Dieu ? \\
Peut-il y avoir une science de l'éducation ? \\
Peut-il y avoir une science politique ? \\
Peut-il y avoir une société des nations ? \\
Peut-il y avoir une société sans État ? \\
Peut-il y avoir un État mondial ? \\
Peut-il y avoir une vérité en art ? \\
Peut-il y avoir une vérité en politique ? \\
Peut-il y avoir un intérêt collectif ? \\
Peut-il y avoir un langage universel ? \\
Peut-on admettre un droit à la révolte ? \\
Peut-on agir machinalement ? \\
Peut-on aimer ce qu'on ne connaît pas ? \\
Peut-on aimer l'autre tel qu'il est ? \\
Peut-on aimer la vie plus que tout ? \\
Peut-on aimer les animaux ? \\
Peut-on aimer l'humanité ? \\
Peut-on aimer sans perdre sa liberté ? \\
Peut-on aimer son prochain comme soi-même ? \\
Peut-on aimer son travail ? \\
Peut-on aimer une œuvre d'art sans la comprendre ? \\
Peut-on apprendre à être heureux ? \\
Peut-on apprendre à être juste ? \\
Peut-on apprendre à être libre ? \\
Peut-on apprendre à mourir ? \\
Peut-on apprendre à penser ? \\
Peut-on apprendre à vivre ? \\
Peut-on argumenter en morale ? \\
Peut-on assimiler le vivant à une machine ? \\
Peut-on atteindre une certitude ? \\
Peut-on attribuer à chacun son dû ? \\
Peut-on avoir conscience de soi sans avoir conscience d'autrui ? \\
Peut-on avoir de bonnes raisons de ne pas dire la vérité ? \\
Peut-on avoir le droit de se révolter ? \\
Peut-on avoir peur de soi-même ? \\
Peut-on avoir raison contre les faits ? \\
Peut-on avoir raison contre tous ? \\
Peut-on avoir raison contre tout le monde ? \\
Peut-on avoir raisons contre les faits ? \\
Peut-on avoir raison tout.e seul.e ? \\
Peut-on avoir raison tout seul ? \\
Peut-on cesser de croire ? \\
Peut-on cesser de désirer ? \\
Peut-on changer de culture ? \\
Peut-on changer de logique ? \\
Peut-on changer le cours de l'histoire ? \\
Peut-on changer le monde ? \\
Peut-on changer le passé ? \\
Peut-on changer ses désirs ? \\
Peut-on choisir le mal ? \\
Peut-on choisir sa vie ? \\
Peut-on choisir ses désirs ? \\
Peut-on classer les arts ? \\
Peut-on commander à la nature ? \\
Peut-on communiquer ses perceptions à autrui ? \\
Peut-on communiquer son expérience ? \\
Peut-on comparer deux philosophies ? \\
Peut-on comparer les cultures ? \\
Peut-on comparer l'organisme à une machine ? \\
Peut-on comprendre ce qui est illogique ? \\
Peut-on comprendre le présent ? \\
Peut-on comprendre un acte que l'on désapprouve ? \\
Peut-on concevoir une humanité sans art ? \\
Peut-on concevoir une morale sans sanction ni obligation ? \\
Peut-on concevoir une science qui ne soit pas démonstrative ? \\
Peut-on concevoir une science sans expérience ? \\
Peut-on concevoir une société juste sans que les hommes ne le soient ? \\
Peut-on concevoir une société qui n'aurait plus besoin du droit ? \\
Peut-on concevoir une société sans État ? \\
Peut-on concevoir un État mondial ? \\
Peut-on concilier bonheur et liberté ? \\
Peut-on conclure de l'être au devoir-être ? \\
Peut-on connaître autrui ? \\
Peut-on connaître les choses telles qu'elles sont ? \\
Peut-on connaître le singulier ? \\
Peut-on connaître l'esprit ? \\
Peut-on connaître le vivant sans le dénaturer ? \\
Peut-on connaître le vivant sans recourir à la notion de finalité ? \\
Peut-on connaître l'individuel ? \\
Peut-on connaître par intuition ? \\
Peut-on considérer l'art comme un langage ? \\
Peut-on contester les droits de l'homme ? \\
Peut-on contredire l'expérience ? \\
Peut-on convaincre quelqu'un de la beauté d'une œuvre d'art ? \\
Peut-on craindre la liberté ? \\
Peut-on créer un homme nouveau ? \\
Peut-on critiquer la démocratie ? \\
Peut-on critiquer la religion ? \\
Peut-on croire ce qu'on veut ? \\
Peut-on croire en rien ? \\
Peut-on croire sans être crédule ? \\
Peut-on croire sans savoir pourquoi ? \\
Peut-on décider de croire ? \\
Peut-on décider de tout ? \\
Peut-on décider d'être heureux ? \\
Peut-on définir la morale comme l'art d'être heureux ? \\
Peut-on définir la vérité ? \\
Peut-on définir la vie ? \\
Peut-on définir le bien ? \\
Peut-on définir le bonheur ? \\
Peut-on délimiter le réel ? \\
Peut-on délimiter l'humain ? \\
Peut-on démontrer qu'on ne rêve pas ? \\
Peut-on dépasser la subjectivité ? \\
Peut-on désirer ce qui est ? \\
Peut-on désirer ce qu'on ne veut pas ? \\
Peut-on désirer ce qu'on possède ? \\
Peut-on désirer l'absolu ? \\
Peut-on désirer l'impossible ? \\
Peut-on désirer sans souffrir ? \\
Peut-on désobéir à l'État ? \\
Peut-on désobéir aux lois ? \\
Peut-on désobéir par devoir ? \\
Peut-on dialoguer avec un ordinateur ? \\
Peut-on dire ce que l'on pense ? \\
Peut-on dire ce qui n'est pas ? \\
Peut-on dire de la connaissance scientifique qu'elle procède par approximation ? \\
Peut-on dire de l'art qu'il donne un monde en partage ? \\
Peut-on dire d'une image qu'elle parle ? \\
Peut-on dire d'une œuvre d'art qu'elle est ratée ? \\
Peut-on dire d'une théorie scientifique qu'elle n'est jamais plus vraie qu'une autre mais seulement plus commode ? \\
Peut-on dire d'un homme qu'il est supérieur à un autre homme ? \\
Peut-on dire la vérité ? \\
Peut-on dire le singulier ? \\
Peut-on dire que la science ne nous fait pas connaître les choses mais les rapports entre les choses ? \\
Peut-on dire que les hommes font l'histoire ? \\
Peut-on dire que les machines travaillent pour nous ? \\
Peut-on dire que les mots pensent pour nous ? \\
Peut-on dire que l'humanité progresse ? \\
Peut-on dire que rien n'échappe à la technique ? \\
Peut-on dire qu'est vrai ce qui correspond aux faits ? \\
Peut-on dire que toutes les croyances se valent ? \\
Peut-on dire que tout est relatif ? \\
Peut-on dire qu'une théorie physique en contredit une autre ? \\
Peut-on dire toute la vérité ? \\
Peut-on discuter des goûts et des couleurs ? \\
Peut-on disposer de son corps ? \\
Peut-on distinguer différents types de causes ? \\
Peut-on distinguer entre de bons et de mauvais désirs ? \\
Peut-on distinguer entre les bons et les mauvais désirs ? \\
Peut-on distinguer le réel de l'imaginaire ? \\
Peut-on distinguer les faits de leurs interprétations ? \\
Peut-on donner un sens à son existence ? \\
Peut-on douter de sa propre existence ? \\
Peut-on douter de soi ? \\
Peut-on douter de toute vérité ? \\
Peut-on douter de tout ? \\
Peut-on échanger des idées ? \\
Peut-on échapper à ses désirs ? \\
Peut-on échapper à son temps ? \\
Peut-on échapper au cours de l'histoire ? \\
Peut-on échapper au temps ? \\
Peut-on éclairer la liberté ? \\
Peut-on écrire comme on parle ? \\
Peut-on éduquer la conscience ? \\
Peut-on éduquer le goût ? \\
Peut-on éduquer quelqu'un à être libre ? \\
Peut-on en appeler à la conscience contre la loi ? \\
Peut-on en appeler à la conscience contre l'État ? \\
Peut-on en savoir trop ? \\
Peut-on entreprendre d'éliminer la métaphysique ? \\
Peut-on établir une hiérarchie des arts ? \\
Peut-on être amoral ? \\
Peut-on être apolitique ? \\
Peut-on être citoyen du monde ? \\
Peut-on être complètement athée ? \\
Peut-on être dans le présent ? \\
Peut-on être en avance sur son temps ? \\
Peut-on être en conflit avec soi-même ? \\
Peut-on être esclave de soi-même ? \\
Peut-on être heureux dans la solitude ? \\
Peut-on être heureux sans être sage ? \\
Peut-on être heureux sans philosophie ? \\
Peut-on être heureux sans s'en rendre compte ? \\
Peut-on être heureux tout seul ? \\
Peut-on être homme sans être citoyen ? \\
Peut-on être hors de soi ? \\
Peut-on être ignorant ? \\
Peut-on être impartial ? \\
Peut-on être indifférent à l'histoire ? \\
Peut-on être indifférent à son bonheur ? \\
Peut-on être injuste et heureux ? \\
Peut-on être insensible à l'art ? \\
Peut-on être juste dans une situation injuste ? \\
Peut-on être juste dans une société injuste ? \\
Peut-on être juste sans être impartial ? \\
Peut-on être maître de soi ? \\
Peut-on être méchant volontairement ? \\
Peut-on être obligé d'aimer ? \\
Peut-on être plus ou moins libre ? \\
Peut-on être sans opinion ? \\
Peut-on être sceptique de bonne foi ? \\
Peut-on être sceptique ? \\
Peut-on être seul avec soi-même ? \\
Peut-on être seul ? \\
Peut-on être soi-même en société ? \\
Peut-on être sûr d'avoir raison ? \\
Peut-on être sûr de bien agir ? \\
Peut-on être sûr de ne pas se tromper ? \\
Peut-on être trop sage ? \\
Peut-on être trop sensible ? \\
Peut-on étudier le passé de façon objective ? \\
Peut-on exercer son esprit ? \\
Peut-on expérimenter sur le vivant ? \\
Peut-on expliquer le mal ? \\
Peut-on expliquer le monde par la matière ? \\
Peut-on expliquer le vivant ? \\
Peut-on expliquer une œuvre d'art ? \\
Peut-on faire de la politique sans supposer les hommes méchants ? \\
Peut-on faire de l'art avec tout ? \\
Peut-on faire de l'esprit un objet de science ? \\
Peut-on faire du dialogue un modèle de relation morale ? \\
Peut-on faire la paix ? \\
Peut-on faire la philosophie de l'histoire ? \\
Peut-on faire le bien d'autrui malgré lui ? \\
Peut-on faire l'économie de la notion de forme ? \\
Peut-on faire le mal en vue du bien ? \\
Peut-on faire le mal innocemment ? \\
Peut-on faire l'expérience de la nécessité ? \\
Peut-on faire l'inventaire du monde ? \\
Peut-on faire table rase du passé ? \\
Peut-on fixer des limites à la science ? \\
Peut-on fonder la morale sur la pitié ? \\
Peut-on fonder la morale ? \\
Peut-on fonder le droit sur la morale ? \\
Peut-on fonder les droits de l'homme ? \\
Peut-on fonder les mathématiques ? \\
Peut-on fonder un droit de désobéir ? \\
Peut-on fonder une éthique sur la biologie ? \\
Peut-on fonder une morale sur la nature ? \\
Peut-on fonder une morale sur le plaisir ? \\
Peut-on forcer quelqu'un à être libre ? \\
Peut-on forcer un homme à être libre ? \\
Peut-on fuir hors du monde ? \\
Peut-on fuir la société ? \\
Peut-on gâcher son talent ? \\
Peut-on gouverner sans lois ? \\
Peut-on haïr la raison ? \\
Peut-on haïr la vie ? \\
Peut-on haïr les images ? \\
Peut-on hiérarchiser les arts ? \\
Peut-on hiérarchiser les œuvres d'art ? \\
Peut-on identifier le désir au besoin ? \\
Peut-on ignorer sa propre liberté ? \\
Peut-on ignorer volontairement la vérité ? \\
Peut-on imaginer l'avenir ? \\
Peut-on imposer la liberté ? \\
Peut-on innover en politique ? \\
Peut-on interpréter la nature ? \\
Peut-on inventer en morale ? \\
Peut-on jamais aimer son prochain ? \\
Peut-on juger des œuvres d'art sans recourir à l'idée de beauté ? \\
Peut-on justifier la discrimination ? \\
Peut-on justifier la guerre ? \\
Peut-on justifier la raison d'État ? \\
Peut-on justifier le mal ? \\
Peut-on justifier le mensonge ? \\
Peut-on justifier ses choix ? \\
Peut-on légitimer la violence ? \\
Peut-on limiter l'expression de la volonté du peuple ? \\
Peut-on lutter contre le destin ? \\
Peut-on lutter contre soi-même ? \\
Peut-on maîtriser la nature ? \\
Peut-on maîtriser la technique ? \\
Peut-on maîtriser le risque ? \\
Peut-on maîtriser le temps ? \\
Peut-on maîtriser l'évolution de la technique ? \\
Peut-on maîtriser l'inconscient ? \\
Peut-on maîtriser ses désirs ? \\
Peut-on manipuler les esprits ? \\
Peut-on manquer de culture ? \\
Peut-on manquer de volonté ? \\
Peut-on mentir par humanité ? \\
Peut-on mesurer les phénomènes sociaux ? \\
Peut-on mesurer le temps ? \\
Peut-on montrer en cachant ? \\
Peut-on moraliser la guerre ? \\
Peut-on ne croire en rien ? \\
Peut-on ne pas connaître son bonheur ? \\
Peut-on ne pas croire au progrès ? \\
Peut-on ne pas croire ? \\
Peut-on ne pas être de son temps ? \\
Peut-on ne pas être égoïste ? \\
Peut-on ne pas être matérialiste ? \\
Peut-on ne pas être soi-même ? \\
Peut-on ne pas interpréter ? \\
Peut-on ne pas savoir ce que l'on dit ? \\
Peut-on ne pas savoir ce que l'on fait ? \\
Peut-on ne pas savoir ce que l'on veut ? \\
Peut-on ne pas savoir ce qu'on veut ? \\
Peut-on ne pas vouloir être heureux ? \\
Peut-on ne penser à rien ? \\
Peut-on ne rien devoir à personne ? \\
Peut-on ne rien vouloir ? \\
Peut-on ne vivre qu'au présent ? \\
Peut-on nier la réalité ? \\
Peut-on nier le réel ? \\
Peut-on nier l'évidence ? \\
Peut-on nier l'existence de la matière ? \\
Peut-on objectiver le psychisme ? \\
Peut-on opposer justice et liberté ? \\
Peut-on opposer le loisir au travail ? \\
Peut-on opposer morale et technique ? \\
Peut-on opposer nature et culture ? \\
Peut-on ôter à l'homme sa liberté ? \\
Peut-on oublier de vivre ? \\
Peut-on parler d'art primitif ? \\
Peut-on parler de ce qui n'existe pas ? \\
Peut-on parler de corruption des mœurs ? \\
Peut-on parler de dialogue des cultures ? \\
Peut-on parler de droits des animaux ? \\
Peut-on parler de mondes imaginaires ? \\
Peut-on parler de nourriture spirituelle ? \\
Peut-on parler de problèmes techniques ? \\
Peut-on parler des miracles de la technique ? \\
Peut-on parler des œuvres d'art ? \\
Peut-on parler de travail intellectuel ? \\
Peut-on parler de vérités métaphysiques ? \\
Peut-on parler de vérité subjective ? \\
Peut-on parler de vertu politique ? \\
Peut-on parler de violence d'État ? \\
Peut-on parler de « travail intellectuel » ? \\
Peut-on parler d'un droit de la guerre ? \\
Peut-on parler d'un droit de résistance ? \\
Peut-on parler d'une expérience religieuse ? \\
Peut-on parler d'une morale collective ? \\
Peut-on parler d'une religion de l'humanité ? \\
Peut-on parler d'une santé de l'âme ? \\
Peut-on parler d'une science de l'art ? \\
Peut-on parler d'un progrès dans l'histoire ? \\
Peut-on parler d'un progrès de la liberté ? \\
Peut-on parler d'un règne de la technique ? \\
Peut-on parler d'un savoir poétique ? \\
Peut-on parler d'un travail intellectuel ? \\
Peut-on parler pour en rien dire ? \\
Peut-on parler pour ne rien dire ? \\
Peut-on penser ce qu'on ne peut dire ? \\
Peut-on penser contre l'expérience ? \\
Peut-on penser illogiquement ? \\
Peut-on penser la douleur ? \\
Peut-on penser la matière ? \\
Peut-on penser la mort ? \\
Peut-on penser la nouveauté ? \\
Peut-on penser l'art sans référence au beau ? \\
Peut-on penser la vie sans penser la mort ? \\
Peut-on penser la vie ? \\
Peut-on penser le changement ? \\
Peut-on penser le monde sans la technique ? \\
Peut-on penser le temps sans l'espace ? \\
Peut-on penser l'extériorité ? \\
Peut-on penser l'impossible ? \\
Peut-on penser l'infini ? \\
Peut-on penser l'irrationnel ? \\
Peut-on penser l'œuvre d'art sans référence à l'idée de beauté ? \\
Peut-on penser sans concepts ? \\
Peut-on penser sans concept ? \\
Peut-on penser sans images ? \\
Peut-on penser sans image ? \\
Peut-on penser sans les mots ? \\
Peut-on penser sans les signes ? \\
Peut-on penser sans méthode ? \\
Peut-on penser sans ordre ? \\
Peut-on penser sans préjugés ? \\
Peut-on penser sans préjugé ? \\
Peut-on penser sans règles ? \\
Peut-on penser sans savoir que l'on pense ? \\
Peut-on penser sans signes ? \\
Peut-on penser sans son corps ? \\
Peut-on penser un art sans œuvres ? \\
Peut-on penser un droit international ? \\
Peut-on penser une conscience sans objet ? \\
Peut-on penser une métaphysique sans Dieu ? \\
Peut-on penser une société sans État ? \\
Peut-on penser un État sans violence ? \\
Peut-on penser une volonté diabolique ? \\
Peut-on percevoir sans juger ? \\
Peut-on percevoir sans s'en apercevoir ? \\
Peut-on perdre la raison ? \\
Peut-on perdre sa dignité ? \\
Peut-on perdre sa liberté ? \\
Peut-on perdre son identité ? \\
Peut-on perdre son temps ? \\
Peut-on préconiser, dans les sciences humaines et sociales, l'imitation des sciences de la nature ? \\
Peut-on prédire les événements ? \\
Peut-on prédire l'histoire ? \\
Peut-on préférer le bonheur à la vérité ? \\
Peut-on préférer l'injustice au désordre ? \\
Peut-on préférer l'ordre à la justice ? \\
Peut-on prévoir l'avenir ? \\
Peut-on prévoir le futur ? \\
Peut-on promettre le bonheur ? \\
Peut-on protéger les libertés sans les réduire ? \\
Peut-on prouver l'existence de Dieu ? \\
Peut-on prouver l'existence de l'inconscient ? \\
Peut-on prouver l'existence du monde ? \\
Peut-on prouver l'existence ? \\
Peut-on prouver une existence ? \\
Peut-on raconter sa vie ? \\
Peut-on raisonner sans règles ? \\
Peut-on ralentir la course du temps ? \\
Peut-on recommencer sa vie ? \\
Peut-on reconnaître un sens à l'histoire sans lui assigner une fin ? \\
Peut-on réduire la pensée à une espèce de comportement ? \\
Peut-on réduire le raisonnement au calcul ? \\
Peut-on réduire l'esprit à la matière ? \\
Peut-on réduire une métaphysique à une conception du monde ? \\
Peut-on réduire un homme à la somme de ses actes ? \\
Peut-on refuser de voir la vérité ? \\
Peut-on refuser la loi ? \\
Peut-on refuser la violence ? \\
Peut-on refuser le vrai ? \\
Peut-on régner innocemment ? \\
Peut-on rendre raison de tout ? \\
Peut-on rendre raison du réel ? \\
Peut-on renoncer à comprendre ? \\
Peut-on renoncer à ses droits ? \\
Peut-on renoncer à soi ? \\
Peut-on renoncer au bonheur ? \\
Peut-on réparer le vivant ? \\
Peut-on répondre d'autrui ? \\
Peut-on représenter le peuple ? \\
Peut-on représenter l'espace ? \\
Peut-on reprocher à la morale d'être abstraite ? \\
Peut-on reprocher au langage d'être équivoque ? \\
Peut-on reprocher au langage d'être parfait ? \\
Peut-on résister au vrai ? \\
Peut-on rester dans le doute ? \\
Peut-on rester insensible à la beauté ? \\
Peut-on rester sceptique ? \\
Peut-on restreindre la logique à la pensée formelle ? \\
Peut-on réunir des arts différents dans une même œuvre ? \\
Peut-on revendiquer la paix comme un droit ? \\
Peut-on revenir sur ses erreurs ? \\
Peut-on rire de tout ? \\
Peut-on rompre avec la société ? \\
Peut-on rompre avec le passé ? \\
Peut-on s'abstenir de penser politiquement ? \\
Peut-on s'accorder sur des vérités morales ? \\
Peut-on s'affranchir des lois ? \\
Peut-on s'attendre à tout ? \\
Peut-on savoir ce qui est bien ? \\
Peut-on savoir quelque chose de l'avenir ? \\
Peut-on savoir sans croire ? \\
Peut-on se choisir un destin ? \\
Peut-on se connaître soi-même ? \\
Peut-on se désintéresser de la politique ? \\
Peut-on se désintéresser de son bonheur ? \\
Peut-on se duper soi-même ? \\
Peut-on se faire une idée de tout ? \\
Peut-on se fier à l'expérience vécue ? \\
Peut-on se fier à l'intuition ? \\
Peut-on se fier à son intuition ? \\
Peut-on se gouverner soi-même ? \\
Peut-on se méfier de soi-même ? \\
Peut-on se mentir à soi-même ? \\
Peut-on se mettre à la place d'autrui ? \\
Peut-on se mettre à la place de l'autre ? \\
Peut-on s'en tenir au présent ? \\
Peut-on séparer l'homme et l'œuvre ? \\
Peut-on séparer politique et économie ? \\
Peut-on se passer de chef ? \\
Peut-on se passer de croire ? \\
Peut-on se passer de croyances ? \\
Peut-on se passer de croyance ? \\
Peut-on se passer de Dieu ? \\
Peut-on se passer de frontières ? \\
Peut-on se passer de la religion ? \\
Peut-on se passer de l'État ? \\
Peut-on se passer de méthode ? \\
Peut-on se passer de mythes ? \\
Peut-on se passer de principes ? \\
Peut-on se passer de religion ? \\
Peut-on se passer de représentants ? \\
Peut-on se passer de spiritualité ? \\
Peut-on se passer des relations ? \\
Peut-on se passer d'État ? \\
Peut-on se passer de techniques de raisonnement ? \\
Peut-on se passer de technique ? \\
Peut-on se passer de toute religion ? \\
Peut-on se passer d'idéal ? \\
Peut-on se passer d'un maître ? \\
Peut-on se prescrire une loi ? \\
Peut-on se promettre quelque chose à soi-même ? \\
Peut-on se punir soi-même ? \\
Peut-on se régler sur des exemples en politique ? \\
Peut-on se retirer du monde ? \\
Peut-on se tromper en se croyant heureux ? \\
Peut-on se vouloir parfait ? \\
Peut-on sortir de la subjectivité ? \\
Peut-on sortir de sa conscience ? \\
Peut-on souhaiter le gouvernement des meilleurs ? \\
Peut-on suivre une règle ? \\
Peut-on suspendre le temps ? \\
Peut-on suspendre son jugement ? \\
Peut-on sympathiser avec l'ennemi ? \\
Peut-on tirer des leçons de l'histoire ? \\
Peut-on toujours faire ce qu'on doit ? \\
Peut-on toujours savoir entièrement ce que l'on dit ? \\
Peut-on tout analyser ? \\
Peut-on tout attendre de l'État ? \\
Peut-on tout définir ? \\
Peut-on tout démontrer ? \\
Peut-on tout désirer ? \\
Peut-on tout dire ? \\
Peut-on tout donner ? \\
Peut-on tout échanger ? \\
Peut-on tout enseigner ? \\
Peut-on tout expliquer ? \\
Peut-on tout exprimer ? \\
Peut-on tout imaginer ? \\
Peut-on tout imiter ? \\
Peut-on tout interpréter ? \\
Peut-on tout mathématiser ? \\
Peut-on tout mesurer ? \\
Peut-on tout ordonner ? \\
Peut-on tout pardonner ? \\
Peut-on tout partager ? \\
Peut-on tout prévoir ? \\
Peut-on tout prouver ? \\
Peut-on tout soumettre à la discussion ? \\
Peut-on tout tolérer ? \\
Peut-on traiter autrui comme un moyen ? \\
Peut-on traiter un être vivant comme une machine ? \\
Peut-on transformer le réel ? \\
Peut-on transiger avec les principes ? \\
Peut-on trouver du plaisir à l'ennui ? \\
Peut-on vivre avec les autres ? \\
Peut-on vivre dans le doute ? \\
Peut-on vivre en marge de la société ? \\
Peut-on vivre en sceptique ? \\
Peut-on vivre hors du temps ? \\
Peut-on vivre pour la vérité ? \\
Peut-on vivre sans aimer ? \\
Peut-on vivre sans art ? \\
Peut-on vivre sans aucune certitude ? \\
Peut-on vivre sans croyances ? \\
Peut-on vivre sans désir ? \\
Peut-on vivre sans échange ? \\
Peut-on vivre sans illusions ? \\
Peut-on vivre sans l'art ? \\
Peut-on vivre sans le plaisir de vivre ? \\
Peut-on vivre sans lois ? \\
Peut-on vivre sans passion ? \\
Peut-on vivre sans peur ? \\
Peut-on vivre sans principes ? \\
Peut-on vivre sans réfléchir ? \\
Peut-on vivre sans ressentiment ? \\
Peut-on vivre sans rien espérer ? \\
Peut-on vivre sans sacré ? \\
Peut-on voir sans croire ? \\
Peut-on vouloir ce qu'on ne désire pas ? \\
Peut-on vouloir le bonheur d'autrui ? \\
Peut-on vouloir le mal pour le mal ? \\
Peut-on vouloir le mal ? \\
Peut-on vouloir l'impossible ? \\
Peut-on vouloir sans désirer ? \\
Philosopher, est-ce apprendre à vivre ? \\
Philosophe-t-on pour être heureux ? \\
Plusieurs religions valent-elles mieux qu'une seule ? \\
Pour connaître, suffit-il de démontrer ? \\
Pour être heureux, faut-il renoncer à la perfection ? \\
Pour être homme, faut-il être citoyen ? \\
Pour être libre, faut-il renoncer à être heureux ? \\
Pour être un bon observateur faut-il être un bon théoricien ? \\
Pour juger, faut-il seulement apprendre à raisonner ? \\
Pour qui se prend-on ? \\
Pourquoi aimons-nous la musique ? \\
Pourquoi aller contre son désir ? \\
Pourquoi a-t-on peur de la folie ? \\
Pourquoi avoir recours à la notion d'inconscient ? \\
Pourquoi châtier ? \\
Pourquoi chercher à connaître le passé ? \\
Pourquoi chercher à se distinguer ? \\
Pourquoi chercher la vérité ? \\
Pourquoi chercher un sens à l'histoire ? \\
Pourquoi cherche-t-on à connaître ? \\
Pourquoi commémorer ? \\
Pourquoi communiquer ? \\
Pourquoi conserver les œuvres d'art ? \\
Pourquoi construire des monuments ? \\
Pourquoi critiquer la raison ? \\
Pourquoi critiquer le conformisme ? \\
Pourquoi croyons-nous ? \\
Pourquoi défendre le faible ? \\
Pourquoi définir ? \\
Pourquoi délibérer ? \\
Pourquoi démontrer ce que l'on sait être vrai ? \\
Pourquoi démontrer ? \\
Pourquoi des artifices ? \\
Pourquoi des artistes ? \\
Pourquoi des cérémonies ? \\
Pourquoi des châtiments ? \\
Pourquoi des classifications ? \\
Pourquoi des conflits ? \\
Pourquoi des devoirs ? \\
Pourquoi des élections ? \\
Pourquoi des exemples ? \\
Pourquoi des fictions ? \\
Pourquoi des géométries ? \\
Pourquoi des guerres ? \\
Pourquoi des historiens ? \\
Pourquoi des hypothèses ? \\
Pourquoi des idoles ? \\
Pourquoi des institutions ? \\
Pourquoi des interdits ? \\
Pourquoi désirer la sagesse ? \\
Pourquoi désirer l'immortalité ? \\
Pourquoi désire-t-on ce dont on n'a nul besoin ? \\
Pourquoi désirons-nous ? \\
Pourquoi des logiciens ? \\
Pourquoi des lois ? \\
Pourquoi des maîtres ? \\
Pourquoi des métaphores ? \\
Pourquoi des modèles ? \\
Pourquoi des musées ? \\
Pourquoi des œuvres d'art ? \\
Pourquoi des philosophes ? \\
Pourquoi des poètes ? \\
Pourquoi des psychologues ? \\
Pourquoi des religions ? \\
Pourquoi des rites ? \\
Pourquoi des sociologues ? \\
Pourquoi des traditions ? \\
Pourquoi des utopies ? \\
Pourquoi dialogue-t-on ? \\
Pourquoi Dieu se soucierait-il des affaires humaines ? \\
Pourquoi dire la vérité ? \\
Pourquoi domestiquer ? \\
Pourquoi donner des leçons de morale ? \\
Pourquoi donner ? \\
Pourquoi échanger des idées ? \\
Pourquoi écrire ? \\
Pourquoi écrit-on des lois ? \\
Pourquoi écrit-on les lois ? \\
Pourquoi écrit-on l'Histoire ? \\
Pourquoi écrit-on ? \\
Pourquoi est-il difficile de rectifier une erreur ? \\
Pourquoi être exigeant ? \\
Pourquoi être moral ? \\
Pourquoi être raisonnable ? \\
Pourquoi étudier le vivant ? \\
Pourquoi étudier l'Histoire ? \\
Pourquoi exposer les œuvres d'art ? \\
Pourquoi faire confiance ? \\
Pourquoi faire de la politique ? \\
Pourquoi faire de l'histoire ? \\
Pourquoi faire la guerre ? \\
Pourquoi faire son devoir ? \\
Pourquoi fait-on le mal ? \\
Pourquoi faudrait-il être cohérent ? \\
Pourquoi faut-il diviser le travail ? \\
Pourquoi faut-il être cohérent ? \\
Pourquoi faut-il être juste ? \\
Pourquoi faut-il être poli ? \\
Pourquoi faut-il travailler ? \\
Pourquoi formaliser des arguments ? \\
Pourquoi imiter ? \\
Pourquoi interprète-t-on ? \\
Pourquoi joue-t-on ? \\
Pourquoi la critique ? \\
Pourquoi la curiosité est-elle un vilain défaut ? \\
Pourquoi la guerre ? \\
Pourquoi la justice a-t-elle besoin d'institutions ? \\
Pourquoi la musique intéresse-t-elle le philosophe ? \\
Pourquoi la prison ? \\
Pourquoi la prohibition de l'inceste ? \\
Pourquoi la raison recourt-elle à l'hypothèse ? \\
Pourquoi la réalité peut-elle dépasser la fiction ? \\
Pourquoi l'art intéresse-t-il les philosophes ? \\
Pourquoi l'économie est-elle politique ? \\
Pourquoi le droit international est-il imparfait ? \\
Pourquoi les droits de l'homme sont-ils universels ? \\
Pourquoi les États se font-ils la guerre ? \\
Pourquoi les hommes mentent-ils ? \\
Pourquoi les mathématiques s'appliquent-elles à la réalité ? \\
Pourquoi les œuvres d'art résistent-elles au temps ? \\
Pourquoi le sport ? \\
Pourquoi les sciences ont-elles une histoire ? \\
Pourquoi les sociétés ont-elles besoin de lois ? \\
Pourquoi le théâtre ? \\
Pourquoi l'ethnologue s'intéresse-t-il à la vie urbaine ? \\
Pourquoi l'homme a-t-il des droits ? \\
Pourquoi l'homme est-il l'objet de plusieurs sciences ? \\
Pourquoi l'homme travaille-t-il ? \\
Pourquoi lire des romans ? \\
Pourquoi lire les poètes ? \\
Pourquoi lit-on des romans ? \\
Pourquoi mentir ? \\
Pourquoi ne s'entend-on pas sur la nature de ce qui est réel ? \\
Pourquoi nous soucier du sort des générations futures ? \\
Pourquoi nous souvenons-nous ? \\
Pourquoi nous trompons-nous ? \\
Pourquoi nous-trompons nous ? \\
Pourquoi obéir aux lois ? \\
Pourquoi obéir ? \\
Pourquoi obéit-on aux lois ? \\
Pourquoi obéit-on ? \\
Pourquoi parler de fautes de goût ? \\
Pourquoi parler du travail comme d'un droit ? \\
Pourquoi parle-t-on d'économie politique ? \\
Pourquoi parle-t-on d'une « société civile » ? \\
Pourquoi parle-t-on ? \\
Pourquoi parlons-nous ? \\
Pourquoi pas plusieurs dieux ? \\
Pourquoi pas ? \\
Pourquoi penser à la mort ? \\
Pourquoi pensons-nous ? \\
Pourquoi philosopher ? \\
Pourquoi pleure-t-on au cinéma ? \\
Pourquoi pleure-t-on ? \\
Pourquoi plusieurs sciences ? \\
Pourquoi préférer l'original à la reproduction ? \\
Pourquoi préférer l'original à sa reproduction ? \\
Pourquoi préférer l'original ? \\
Pourquoi préserver l'environnement ? \\
Pourquoi prier ? \\
Pourquoi prouver l'existence de Dieu ? \\
Pourquoi punir ? \\
Pourquoi punit-on ? \\
Pourquoi raconter des histoires ? \\
Pourquoi rechercher la vérité ? \\
Pourquoi rechercher le bonheur ? \\
Pourquoi refuse-t-on la conscience à l'animal ? \\
Pourquoi respecter autrui ? \\
Pourquoi respecter le droit ? \\
Pourquoi respecter les anciens ? \\
Pourquoi rit-on ? \\
Pourquoi sauver les apparences ? \\
Pourquoi sauver les phénomènes ? \\
Pourquoi se confesser ? \\
Pourquoi se divertir ? \\
Pourquoi se fier à autrui ? \\
Pourquoi se mettre à la place d'autrui ? \\
Pourquoi séparer les pouvoirs ? \\
Pourquoi se révolter ? \\
Pourquoi se soucier du futur ? \\
Pourquoi s'étonner ? \\
Pourquoi s'exprimer ? \\
Pourquoi s'inspirer de l'art antique ? \\
Pourquoi s'intéresser à l'histoire ? \\
Pourquoi s'intéresser à l'origine ? \\
Pourquoi s'interroger sur l'origine du langage ? \\
Pourquoi soigner son apparence ? \\
Pourquoi sommes-nous déçus par les œuvres d'un faussaire ? \\
Pourquoi sommes-nous des êtres moraux ? \\
Pourquoi sommes-nous moraux ? \\
Pourquoi suivre l'actualité ? \\
Pourquoi tenir ses promesses ? \\
Pourquoi théoriser ? \\
Pourquoi transformer le monde ? \\
Pourquoi transmettre ? \\
Pourquoi travailler ? \\
Pourquoi travaille-t-on ? \\
Pourquoi un droit du travail ? \\
Pourquoi une instruction publique ? \\
Pourquoi un fait devrait-il être établi ? \\
Pourquoi veut-on changer le monde ? \\
Pourquoi veut-on la vérité ? \\
Pourquoi vivons-nous ? \\
Pourquoi vivre ensemble ? \\
Pourquoi vouloir avoir raison ? \\
Pourquoi vouloir se connaître ? \\
Pourquoi voulons-nous savoir ? \\
Pourquoi voyager ? \\
Pourquoi y a-t-il des conflits insolubles ? \\
Pourquoi y a-t-il des institutions ? \\
Pourquoi y a-t-il des religions ? \\
Pourquoi y a-t-il du mal dans le monde ? \\
Pourquoi y a-t-il plusieurs façons de démontrer ? \\
Pourquoi y a-t-il plusieurs langues ? \\
Pourquoi y a-t-il plusieurs sciences ? \\
Pourquoi y a-t-il quelque chose plutôt que rien ? \\
Pourquoi y a-t-il des lois ? \\
Pourquoi y a-t-il plusieurs philosophies ? \\
Pourquoi ? \\
Pourrait-on se passer de l'argent ? \\
Pourrions-nous comprendre une pensée non humaine ? \\
Pouvons-nous communiquer ce que nous sentons ? \\
Pouvons-nous connaître sans interpréter ? \\
Pouvons-nous désirer ce qui nous nuit ? \\
Pouvons-nous devenir meilleurs ? \\
Pouvons-nous dissocier le réel de nos interprétations ? \\
Pouvons-nous être certains que nous ne rêvons pas ? \\
Pouvons-nous être objectifs ? \\
Pouvons-nous faire l'expérience de la liberté ? \\
Pouvons-nous justifier nos croyances ? \\
Pouvons-nous savoir ce que nous ignorons ? \\
Prendre son temps, est-ce le perdre ? \\
Primitif ou premier ? \\
Puis-je aimer tous les hommes ? \\
Puis-je comprendre autrui ? \\
Puis-je décider de croire ? \\
Puis-je dire « ceci est mon corps » ? \\
Puis-je douter de ma propre existence ? \\
Puis-je être dans le vrai sans le savoir ? \\
Puis-je être heureux dans un monde chaotique ? \\
Puis-je être libre sans être responsable ? \\
Puis-je être sûr de bien agir ? \\
Puis-je être sûr de ne pas me tromper ? \\
Puis-je être sûr que je ne rêve pas ? \\
Puis-je être universel ? \\
Puis-je faire ce que je veux de mon corps ? \\
Puis-je faire confiance à mes sens ? \\
Puis-je invoquer l'inconscient sans ruiner la morale ? \\
Puis-je me passer d'imiter autrui ? \\
Puis-je ne croire que ce que je vois ? \\
Puis-je ne pas vouloir ce que je désire ? \\
Puis-je ne rien croire ? \\
Puis-je répondre des autres ? \\
Puis-je savoir ce qui m'est propre ? \\
Punir ou soigner ? \\
Qu'ai-je le droit d'exiger d'autrui ? \\
Qu'ai-je le droit d'exiger des autres ? \\
Qu'aime-t-on dans l'amour ? \\
Qu'aime-t-on ? \\
Quand agit-on ? \\
Quand faut-il désobéir aux lois ? \\
Quand faut-il désobéir ? \\
Quand faut-il mentir ? \\
Quand la guerre finira-t-elle ? \\
Quand la technique devient-elle art ? \\
Quand le temps passe, que reste-t-il ? \\
Quand pense-t-on ? \\
Quand peut-on se passer de théories ? \\
Quand suis-je en faute ? \\
Quand une autorité est-elle légitime ? \\
Quand y a-t-il de l'art ? \\
Quand y a-t-il œuvre ? \\
Quand y a-t-il paysage ? \\
Quand y a-t-il peuple ? \\
Qu'anticipent les romans d'anticipation ? \\
Qu'a perdu le fou ? \\
Qu'appelle-t-on chef-d'œuvre ? \\
Qu'appelle-t-on destin ? \\
Qu'appelle-t-on penser ? \\
Qu'apprend-on dans les livres ? \\
Qu'apprend-on des romans ? \\
Qu'apprend-on en commettant une faute ? \\
Qu'apprend-on quand on apprend à parler ? \\
Qu'apprenons-nous de nos affects ? \\
Qu'a-t-on le droit de pardonner ? \\
Qu'a-t-on le droit d'exiger ? \\
Qu'a-t-on le droit d'interpréter ? \\
Qu'attendons-nous de la technique ? \\
Qu'attendons-nous pour être heureux ? \\
Qu'avons-nous à apprendre des historiens ? \\
Qu'avons-nous en commun ? \\
Que célèbre l'art ? \\
Que cherchons-nous dans le regard des autres ? \\
Que choisir ? \\
Que connaissons-nous du vivant ? \\
Que construit le politique ? \\
Que coûte une victoire ? \\
Que crée l'artiste ? \\
Que déduire d'une contradiction ? \\
Que démontrent nos actions ? \\
Que désire-t-on ? \\
Que désirons-nous quand nous désirons savoir ?Qu'est-ce qu'un événement historique ? \\
Que désirons-nous ? \\
Que devons-nous à autrui ? \\
Que devons-nous à l'État ? \\
Que disent les légendes ? \\
Que disent les tables de vérité ? \\
Que dit la loi ? \\
Que dit la musique ? \\
Que dois-je à autrui ? \\
Que dois-je à l'État ? \\
Que dois-je respecter en autrui ? \\
Que doit la pensée à l'écriture ? \\
Que doit la science à la technique ? \\
Que doit-on aux morts ? \\
Que doit-on croire ? \\
Que doit-on désirer pour ne pas être déçu ? \\
Que doit-on faire de ses rêves ? \\
Que doit-on savoir avant d'agir ? \\
Que faire de la diversité des arts ? \\
Que faire de nos émotions ? \\
Que faire de nos passions ? \\
Que faire de notre cerveau ? \\
Que faire des adversaires ? \\
Que faire ? \\
Que fait la police ? \\
Que faut-il absolument savoir ? \\
Que faut-il craindre ? \\
Que faut-il pour faire un monde ? \\
Que faut-il respecter ? \\
Que faut-il savoir pour agir ? \\
Que faut-il savoir pour gouverner ? \\
Que gagne-t-on à travailler ? \\
Que la nature soit explicable, est-ce explicable ? \\
Quel contrôle a-t-on sur son corps ? \\
Quel est le bon nombre d'amis ? \\
Quel est le but de la politique ? \\
Quel est le but d'une théorie physique ? \\
Quel est le but du travail scientifique ? \\
Quel est le contraire du travail ? \\
Quel est le fondement de la propriété ? \\
Quel est le fondement de l'autorité ? \\
Quel est le poids du passé ? \\
Quel est le pouvoir de l'art ? \\
Quel est le pouvoir des mots ? La prévoyance \\
Quel est le rôle de la créativité dans les sciences ? \\
Quel est le rôle du concept en art ? \\
Quel est le rôle du médecin ? \\
Quel est le sens du progrès technique ? \\
Quel est le sujet de la pensée ? \\
Quel est le sujet de l'histoire ? \\
Quel est le sujet du devenir ? \\
Quel est l'être de l'illusion ? \\
Quel est l'homme des Droits de l'homme ? \\
Quel est l'objet de la biologie ? \\
Quel est l'objet de la métaphysique ? \\
Quel est l'objet de l'amour ? \\
Quel est l'objet de la perception ? \\
Quel est l'objet de la philosophie politique ? \\
Quel est l'objet de la science ? \\
Quel est l'objet de l'échange ? \\
Quel est l'objet de l'esthétique ? \\
Quel est l'objet de l'histoire ? \\
Quel est l'objet des mathématiques ? \\
Quel est l'objet des sciences humaines ? \\
Quel est l'objet des sciences politiques ? \\
Quel est l'objet du désir ? \\
Quel être peut être un sujet de droits ? \\
Quelle causalité pour le vivant ? \\
Quelle confiance accorder au langage ? \\
Quelle est la cause du désir ? \\
Quelle est la fin de la science ? \\
Quelle est la fin de l'État ? \\
Quelle est la fonction première de l'État ? \\
Quelle est la force de la loi ? \\
Quelle est la matière de l'œuvre d'art ? \\
Quelle est la place de l'imagination dans la vie de l'esprit ? \\
Quelle est la place du hasard dans l'histoire ? \\
Quelle est la portée d'un exemple ? \\
Quelle est la réalité de la matière ? \\
Quelle est la réalité de l'avenir ? \\
Quelle est la réalité d'une idée ? \\
Quelle est la réalité du passé ? \\
Quelle est la spécificité de la communauté politique ? \\
Quelle est la valeur de l'expérience ? \\
Quelle est la valeur des hypothèses ? \\
Quelle est la valeur d'une expérimentation ? \\
Quelle est la valeur d'une œuvre d'art ? \\
Quelle est la valeur du rêve ? \\
Quelle est la valeur du témoignage ? \\
Quelle est la valeur du vivant ? \\
Quelle est l'unité du « je » ? \\
Quelle idée le fanatique se fait-il de la vérité ? \\
Quelle politique fait-on avec les sciences humaines ? \\
Quelle réalité attribuer à la matière ? \\
Quelle réalité l'art nous fait-il connaître ? \\
Quelle réalité la science décrit-elle ? \\
Quelle réalité peut-on accorder au temps ? \\
Quelles actions permettent d'être heureux ? \\
Quelle sorte d'histoire ont les sciences ? \\
Quelles règles la technique dicte-t-elle à l'art ? \\
Quelles sont les caractéristiques d'une proposition morale ? \\
Quelles sont les caractéristiques d'un être vivant ? \\
Quelles sont les limites de la démonstration ? \\
Quelles sont les limites de la souveraineté ? \\
Quelle valeur accorder à l'expérience ? \\
Quelle valeur devons accorder à l'expérience ? \\
Quelle valeur devons-nous accorder à l'expérience ? \\
Quelle valeur devons-nous accorder à l'intuition ? \\
Quelle valeur donner à la notion de « corps social » ? \\
Quelle valeur peut-on accorder à l'expérience ? \\
Quelle vérité y-a-t-il dans la perception ? \\
Quel réel pour l'art ? \\
Quel rôle attribuer à l'intuition \emph{a priori} dans une philosophie des mathématiques ? \\
Quel rôle la logique joue-t-elle en mathématiques ? \\
Quel rôle l'imagination joue-t-elle en mathématiques ? \\
Quels désirs dois-je m'interdire ? \\
Quel sens donner à l'expression « gagner sa vie » ? \\
Quel sens y a-t-il à se demander si les sciences humaines sont vraiment des sciences ? \\
Quels sont les droits de la conscience ? \\
Quels sont les fondements de l'autorité ? \\
Quels sont les moyens légitimes de la politique ? \\
Quel usage faut-il faire des exemples ? \\
Quel usage peut-on faire des fictions ? \\
Que manque-t-il à une machine pour être vivante ? \\
Que manque-t-il aux machines pour être des organismes ? \\
Que mesure-t-on du temps ? \\
Que montre l'image ? \\
Que montre une démonstration ? \\
Que montre un tableau ? \\
Que ne peut-on pas expliquer ? \\
Que nous append l'histoire ? \\
Que nous apporte l'art ? \\
Que nous apporte la vérité ? \\
Que nous apprend la définition de la vérité ? \\
Que nous apprend la diversité des langues ? \\
Que nous apprend la fiction sur la réalité ? \\
Que nous apprend la grammaire ? \\
Que nous apprend la maladie sur la santé ? \\
Que nous apprend la musique ? \\
Que nous apprend la poésie ? \\
Que nous apprend la psychanalyse de l'homme ? \\
Que nous apprend la sociologie des sciences ? \\
Que nous apprend la vie ? \\
Que nous apprend le cinéma ? \\
Que nous apprend le faux ? \\
Que nous apprend le plaisir ? \\
Que nous apprend le toucher ? \\
Que nous apprend l'étude du cerveau ? \\
Que nous apprend l'expérience ? \\
Que nous apprend l'histoire de l'art ? \\
Que nous apprend l'histoire des sciences ? \\
Que nous apprend, sur la politique, l'utopie ? \\
Que nous apprennent les algorithmes sur nos sociétés ? \\
Que nous apprennent les animaux sur nous-mêmes ? \\
Que nous apprennent les animaux ? \\
Que nous apprennent les controverses scientifiques ? \\
Que nous apprennent les expériences de pensée ? \\
Que nous apprennent les faits divers ? \\
Que nous apprennent les illusions d'optique ? \\
Que nous apprennent les jeux ? \\
Que nous apprennent les langues étrangères ? \\
Que nous apprennent les machines ? \\
Que nous apprennent les métaphores ? \\
Que nous apprennent les mythes ? \\
Que nous enseigne l'expérience ? \\
Que nous enseignent les œuvres d'art ? \\
Que nous enseignent les sens ? \\
Que nous montre le cinéma ? \\
Que nous montrent les natures mortes ? \\
Que nous réserve l'avenir ? \\
Que partage-t-on avec les animaux ? \\
Que peindre ? \\
Que peint le peintre ? \\
Que penser de l'adage : « Que la justice s'accomplisse, le monde dût-il périr » (Fiat justitia pereat mundus) ? \\
Que penser de la formule : « il faut suivre la nature » ? \\
Que penser de l'opposition travail manuel, travail intellectuel ? \\
Que percevons-nous d'autrui ? \\
Que percevons-nous du monde extérieur ? \\
Que percevons-nous ? \\
Que perçoit-on ? \\
Que perd la pensée en perdant l'écriture ? \\
Que perdrait la pensée en perdant l'écriture ? \\
Que peut expliquer l'histoire ? \\
Que peut la force ? \\
Que peut la musique ? \\
Que peut la philosophie ? \\
Que peut la politique ? \\
Que peut la raison ? \\
Que peut l'art ? \\
Que peut la science ? \\
Que peut la théorie ? \\
Que peut la volonté ? \\
Que peut le corps ? \\
Que peut le politique ? \\
Que peut l'esprit sur la matière ? \\
Que peut l'esprit ? \\
Que peut l'État ? \\
Que peut-on attendre de l'État ? \\
Que peut-on attendre du droit international ? \\
Que peut-on calculer ? \\
Que peut-on comprendre immédiatement ? \\
Que peut-on comprendre qu'on ne puisse expliquer ? \\
Que peut-on contre un préjugé ? \\
Que peut-on cultiver ? \\
Que peut-on démontrer ? \\
Que peut-on dire de l'être ? \\
Que peut-on échanger ? \\
Que peut-on interdire ? \\
Que peut-on partager ? \\
Que peut-on savoir de l'inconscient ? \\
Que peut-on savoir de soi ? \\
Que peut-on savoir du réel ? \\
Que peut-on savoir par expérience ? \\
Que peut-on sur autrui ? \\
Que peut-on voir ? \\
Que peut un corps ? \\
Que pouvons-nous aujourd'hui apprendre des sciences d'autrefois ? \\
Que pouvons-nous espérer de la connaissance du vivant ? \\
Que pouvons-nous faire de notre passé ? \\
Que produit l'inconscient ? \\
Que prouvent les faits ? \\
Que prouvent les preuves de l'existence de Dieu ? \\
Que recherche l'artiste ? \\
Que répondre au sceptique ? \\
Que reste-t-il d'une existence ? \\
Que sais-je d'autrui ? \\
Que sais-je de ma souffrance ? \\
Que sait la conscience ? \\
Que sait-on de soi ? \\
Que sait-on du réel ? \\
Que savons-nous de l'inconscient ? \\
Que serait la vie sans l'art ? \\
Que serait le meilleur des mondes ? \\
Que serait un art total ? \\
Que serait une démocratie parfaite ? \\
Que serions-nous sans l'État ? \\
Que signifie apprendre ? \\
Que signifie connaître ? \\
Que signifie être en guerre ? \\
Que signifie être mortel ? \\
Que signifie la mort ? \\
Que signifie l'idée de technoscience ? \\
Que signifient les mots ? \\
Que signifie pour l'homme être mortel ? \\
Que signifier « juger en son âme et conscience » ? \\
Que signifie « donner le change » ? \\
Que sondent les sondages d'opinion ? \\
Que sont les apparences ? \\
Qu'est-ce la technique ? \\
Qu'est-ce le mal radical ? \\
Qu'est-ce qu'agir ensemble ? \\
Qu'est-ce qu'aimer une œuvre d'art ? \\
Qu'est-ce qu'apprendre ? \\
Qu'est-ce qu'argumenter ? \\
Qu'est-ce qu'avoir conscience de soi ? \\
Qu'est-ce qu'avoir de l'expérience ? \\
Qu'est-ce qu'avoir du goût ? \\
Qu'est-ce qu'avoir du style ? \\
Qu'est-ce qu'avoir un droit ? \\
Qu'est-ce que calculer ? \\
Qu'est-ce que catégoriser ? \\
Qu'est-ce que commencer ? \\
Qu'est-ce que composer une œuvre ? \\
Qu'est-ce que comprendre une œuvre d'art ? \\
Qu'est-ce que comprendre ? \\
Qu'est-ce que créer ? \\
Qu'est-ce que croire ? \\
Qu'est-ce que décider ? \\
Qu'est-ce que définir ? \\
Qu'est-ce que démontrer ? \\
Qu'est-ce que déraisonner ? \\
Qu'est-ce que Dieu pour athée ? \\
Qu'est-ce que Dieu pour un athée ? \\
Qu'est-ce que discuter ? \\
Qu'est-ce qu'éduquer ? \\
Qu'est-ce que faire autorité ? \\
Qu'est-ce que faire preuve d'humanité ? \\
Qu'est-ce que faire une expérience ? \\
Qu'est-ce que gouverner ? \\
Qu'est-ce que guérir ? \\
Qu'est-ce que jouer ? \\
Qu'est-ce que juger ? \\
Qu'est-ce que la barbarie ? \\
Qu'est-ce que la causalité ? \\
Qu'est-ce que la critique ? \\
Qu'est-ce que la démocratie ? \\
Qu'est-ce que la folie ? \\
Qu'est-ce que la normalité ? \\
Qu'est-ce que la politique ? \\
Qu'est-ce que la psychologie ? \\
Qu'est-ce que la raison d'État ? \\
Qu'est-ce que l'art contemporain ? \\
Qu'est-ce que la science saisit du vivant ? \\
Qu'est-ce que la science, si elle inclut la psychanalyse ? \\
Qu'est-ce que la scientificité ? \\
Qu'est-ce que la souveraineté ? \\
Qu'est-ce que la tragédie ? \\
Qu'est-ce que la valeur marchande ? \\
Qu'est-ce que la vie bonne ? \\
Qu'est-ce que la vie ? \\
Qu'est-ce que le bonheur ? \\
Qu'est-ce que le cinéma a changé dans l'idée que l'on se fait du temps ? \\
Qu'est-ce que le cinéma donne à voir ? \\
Qu'est-ce que le courage ? \\
Qu'est-ce que le désordre ? \\
Qu'est-ce que le dogmatisme ? \\
Qu'est-ce que le hasard ? \\
Qu'est-ce que le langage ordinaire ? \\
Qu'est-ce que le malheur ? \\
Qu'est-ce que le moi ? \\
Qu'est-ce que le naturalisme ? \\
Qu'est-ce que l'enfance ? \\
Qu'est-ce que le nihilisme ? \\
Qu'est-ce que le pathologique nous apprend sur le normal ? \\
Qu'est-ce que le présent ? \\
Qu'est-ce que le réel ? \\
Qu'est-ce que le sacré ? \\
Qu'est-ce que le sens pratique ? \\
Qu'est-ce que le sublime ? \\
Qu'est-ce que le travail ? \\
Qu'est-ce que l'harmonie ? \\
Qu'est-ce que l'inconscient ? \\
Qu'est-ce que l'indifférence ? \\
Qu'est-ce que l'intérêt général ? \\
Qu'est-ce que l'intuition ? \\
Qu'est ce que lire ? \\
Qu'est-ce que lire ? \\
Qu'est-ce que l'ordinaire ? \\
Qu'est-ce que maîtriser une technique ? \\
Qu'est-ce que manquer de culture ? \\
Qu'est-ce que méditer ? \\
Qu'est-ce que mourir ? \\
Qu'est-ce qu'enquêter ? \\
Qu'est-ce qu'enseigner ? \\
Qu'est-ce que parler le même langage ? \\
Qu'est-ce que parler ? \\
Qu'est-ce que penser ? \\
Qu'est-ce que percevoir ? \\
Qu'est-ce que perdre la raison ? \\
Qu'est-ce que perdre sa liberté ? \\
Qu'est-ce que perdre son temps ? \\
Qu'est-ce que prendre conscience ? \\
Qu'est-ce que prendre le pouvoir ? \\
Qu'est-ce que promettre ? \\
Qu'est-ce que prouver ? \\
Qu'est-ce que raisonner ? \\
Qu'est-ce que réfuter une philosophie ? \\
Qu'est-ce que réfuter ? \\
Qu'est-ce que résister ? \\
Qu'est-ce que résoudre une contradiction ? \\
Qu'est-ce que rester soi-même ? \\
Qu'est-ce que réussir sa vie ? \\
Qu'est-ce que s'orienter ? \\
Qu'est-ce que témoigner ? \\
Qu'est-ce que traduire ? \\
Qu'est-ce que travailler ? \\
Qu'est-ce qu'être adulte ? \\
Qu'est-ce qu'être artiste ? \\
Qu'est-ce qu'être asocial ? \\
Qu'est-ce qu'être barbare ? \\
Qu'est-ce qu'être chez soi ? \\
Qu'est-ce qu'être cohérent ? \\
Qu'est-ce qu'être comportementaliste ? \\
Qu'est-ce qu'être cultivé ? \\
Qu'est-ce qu'être dans le vrai ? \\
Qu'est-ce qu'être de son temps ? \\
Qu'est-ce qu'être efficace en politique ? \\
Qu'est-ce qu'être en vie ? \\
Qu'est-ce qu'être esclave ? \\
Qu'est-ce qu'être fidèle à soi-même ? \\
Qu'est-ce qu'être généreux ? \\
Qu'est-ce qu'être idéaliste ? \\
Qu'est-ce qu'être inhumain ? \\
Qu'est-ce qu'être l'auteur de son acte ? \\
Qu'est-ce qu'être libéral ? \\
Qu'est-ce qu'être libre ? \\
Qu'est-ce qu'être maître de soi-même ? \\
Qu'est-ce qu'être malade ? \\
Qu'est-ce qu'être moderne ? \\
Qu'est-ce qu'être nihiliste ? \\
Qu'est-ce qu'être normal ? \\
Qu'est-ce qu'être rationnel ? \\
Qu'est-ce qu'être réaliste ? \\
Qu'est-ce qu'être républicain ? \\
Qu'est-ce qu'être sceptique ? \\
Qu'est-ce qu'être seul ? \\
Qu'est-ce qu'être soi-même ? \\
Qu'est-ce qu'être souverain ? \\
Qu'est-ce qu'être spirituel ? \\
Qu'est-ce qu'être témoin ? \\
Qu'est-ce qu'être un bon citoyen ? \\
Qu'est-ce qu'être un esclave ? \\
Qu'est-ce qu'être un sujet ? \\
Qu'est-ce qu'être vivant ? \\
Qu'est-ce qu'être ? \\
Qu'est-ce que vérifier une théorie ? \\
Qu'est-ce que vérifier ? \\
Qu'est-ce que vivre bien ? \\
Qu'est-ce que vivre ? \\
Qu'est-ce qu'exister pour un individu ? \\
Qu'est-ce qu'exister ? \\
Qu'est-ce qu'expliquer ? \\
Qu'est-ce que « parler le même langage » ? \\
Qu'est-ce que « se rendre maître et possesseur de la nature » ? \\
Qu'est-ce qu'habiter ? \\
Qu'est-ce qui agit ? \\
Qu'est-ce qui apparaît ? \\
Qu'est-ce qui dépend de nous ? \\
Qu'est-ce qui distingue un vivant d'une machine ? \\
Qu'est-ce qui est absurde ? \\
Qu'est-ce qui est actuel ? \\
Qu'est-ce qui est beau ? \\
Qu'est ce qui est concret ? \\
Qu'est-ce qui est concret ? \\
Qu'est ce qui est contre nature ? \\
Qu'est-ce qui est contre nature ? \\
Qu'est ce qui est culturel ? \\
Qu'est-ce qui est culturel ? \\
Qu'est-ce qui est donné ? \\
Qu'est-ce qui est essentiel ? \\
Qu'est-ce qui est extérieur à ma conscience, ? \\
Qu'est-ce qui est historique ? \\
Qu'est-ce qui est hors la loi ? \\
Qu'est-ce qui est hors-la-loi ? \\
Qu'est-ce qui est immoral ? \\
Qu'est-ce qui est impossible ? \\
Qu'est-ce qui est indiscutable ? \\
Qu'est-ce qui est invérifiable ? \\
Qu'est-ce qui est irrationnel ? \\
Qu'est ce qui est irréfutable ? \\
Qu'est-ce qui est irréversible ? \\
Qu'est-ce qui est le plus à craindre, l'ordre ou le désordre ? \\
Qu'est-ce qui est mauvais dans l'égoïsme ? \\
Qu'est-ce qui est mien ? \\
Qu'est-ce qui est moderne ? \\
Qu'est-ce qui est naturel ? \\
Qu'est-ce qui est noble ? \\
Qu'est-ce qui est politique ? \\
Qu'est-ce qui est possible ? \\
Qu'est-ce qui est public ? \\
Qu'est-ce qui est réel ? \\
Qu'est-ce qui est respectable ? \\
Qu'est ce qui est sacré ? \\
Qu'est-ce qui est sauvage ? \\
Qu'est-ce qui est scientifique ? \\
Qu'est-ce qui est spectaculaire ? \\
Qu'est-ce qui est sublime ? \\
Qu'est-ce qui est tragique ? \\
Qu'est-ce qui est vital pour le vivant ? \\
Qu'est-ce qui est vital ? \\
Qu'est ce qui existe ? \\
Qu'est-ce qui existe ? \\
Qu'est-ce qui fait changer les sociétés ? \\
Qu'est-ce qui fait d'une activité un travail ? \\
Qu'est-ce qui fait la force de la loi ? \\
Qu'est-ce qui fait la force des lois ? \\
Qu'est-ce qui fait la justice des lois ? \\
Qu'est-ce qui fait la légitimité d'une autorité politique ? \\
Qu'est-ce qui fait la valeur de la technique ? \\
Qu'est-ce qui fait la valeur de l'œuvre d'art ? \\
Qu'est-ce qui fait la valeur d'une croyance ? \\
Qu'est-ce qui fait la valeur d'une existence ? \\
Qu'est-ce qui fait la valeur d'une œuvre d'art ? \\
Qu'est-ce qui fait le pouvoir des mots ? \\
Qu'est-ce qui fait le propre d'un corps propre ? \\
Qu'est-ce qui fait l'humanité d'un corps ? \\
Qu'est-ce qui fait l'unité d'une science ? \\
Qu'est-ce qui fait l'unité d'un organisme ? \\
Qu'est-ce qui fait l'unité d'un peuple ? \\
Qu'est-ce qui fait l'unité du vivant ? \\
Qu'est-ce qui fait mon identité ? \\
Qu'est-ce qui fait qu'une théorie est vraie ? \\
Qu'est-ce qui fait un peuple ? \\
Qu'est-ce qui fonde la croyance ? \\
Qu'est-ce qui fonde le respect d'autrui ? \\
Qu'est-ce qu'ignore la science ? \\
Qu'est-ce qui importe ? \\
Qu'est-ce qui innocente le bourreau ? \\
Qu'est-ce qui justifie l'hypothèse d'un inconscient ? \\
Qu'est-ce qui justifie une croyance ? \\
Qu'est-ce qu'imaginer ? \\
Qu'est-ce qui menace la liberté ? \\
Qu'est-ce qui mesure la valeur d'un travail ? \\
Qu'est-ce qui n'a pas d'histoire ? \\
Qu'est-ce qui ne disparaît jamais ?/ \\
Qu'est-ce qui ne s'achète pas ? \\
Qu'est-ce qui ne s'échange pas ? \\
Qu'est-ce qui n'est pas démontrable ? \\
Qu'est-ce qui n'est pas politique ? \\
Qu'est-ce qui n'existe pas ? \\
Qu'est-ce qui nous fait danser ? \\
Qu'est-ce qu'interpréter une œuvre d'art ? \\
Qu'est-ce qu'interpréter ? \\
Qu'est-ce qui peut se transformer ? \\
Qu'est-ce qui plaît dans la musique ? \\
Qu'est ce qui rapproche le vivant de la machine ? \\
Qu'est-ce qui rend l'objectivité difficile dans les sciences humaines ? \\
Qu'est-ce qui rend vrai un énoncé ? \\
Qu'est-ce qu'obéir ? \\
Qu'est-ce qu'on attend ? \\
Qu'est-ce qu'on ne peut comprendre ? \\
Qu'est-ce qu'un abus de langage ? \\
Qu'est-ce qu'un abus de pouvoir ? \\
Qu'est-ce qu'un accident ? \\
Qu'est-ce qu'un acte libre ? \\
Qu'est-ce qu'un acte moral ? \\
Qu'est-ce qu'un acte symbolique ? \\
Qu'est-ce qu'un acteur ? \\
Qu'est-ce qu'un acte ? \\
Qu'est-ce qu'un adversaire en politique ? \\
Qu'est-ce qu'un alter ego ? \\
Qu'est-ce qu'un ami ? \\
Qu'est-ce qu'un animal domestique ? \\
Qu'est-ce qu'un animal ? \\
Qu'est-ce qu'un argument ? \\
Qu'est-ce qu'un art de vivre ? \\
Qu'est-ce qu'un artiste ? \\
Qu'est-ce qu'un art moral ? \\
Qu'est-ce qu'un auteur ? \\
Qu'est-ce qu'un axiome ? \\
Qu'est-ce qu'un bon citoyen ? \\
Qu'est-ce qu'un bon conseil ? \\
Qu'est-ce qu'un bon gouvernement ? \\
Qu'est-ce qu'un bon jugement ? \\
Qu'est-ce qu'un capital culturel ? \\
Qu'est-ce qu'un caractère ? \\
Qu'est-ce qu'un cas de conscience ? \\
Qu'est-ce qu'un châtiment ? \\
Qu'est-ce qu'un chef d'œuvre ? \\
Qu'est-ce qu'un chef-d'œuvre ? \\
Qu'est-ce qu'un chef ? \\
Qu'est-ce qu'un choix éclairé ? \\
Qu'est-ce qu'un citoyen libre ? \\
Qu'est-ce qu'un citoyen ? \\
Qu'est-ce qu'un civilisé ? \\
Qu'est-ce qu'un classique ? \\
Qu'est-ce qu'un code ? \\
Qu'est-ce qu'un concept philosophique ? \\
Qu'est-ce qu'un concept scientifique ? \\
Qu'est-ce qu'un concept ? \\
Qu'est-ce qu'un conflit de générations ? \\
Qu'est-ce qu'un conflit politique ? \\
Qu'est-ce qu'un consommateur ? \\
Qu'est-ce qu'un contenu de conscience ? \\
Qu'est-ce qu'un contrat ? \\
Qu'est-ce qu'un contre-pouvoir ? \\
Qu'est-ce qu'un corps social ? \\
Qu'est-ce qu'un coup d'État ? \\
Qu'est-ce qu'un créateur ? \\
Qu'est-ce qu'un crime contre l'humanité ? \\
Qu'est-ce qu'un crime politique ? \\
Qu'est-ce qu'un crime ? \\
Qu'est-ce qu'un critère de vérité ? \\
Qu'est-ce qu'un déni ? \\
Qu'est-ce qu'un désir satisfait ? \\
Qu'est-ce qu'un détail ? \\
Qu'est-ce qu'un dialogue ? \\
Qu'est-ce qu'un dieu ? \\
Qu'est-ce qu'un Dieu ? \\
Qu'est-ce qu'un dilemme ? \\
Qu'est-ce qu'un document ? \\
Qu'est-ce qu'un dogme ? \\
Qu'est-ce qu'une action intentionnelle ? \\
Qu'est-ce qu'une action juste ? \\
Qu'est-ce qu'une action politique ? \\
Qu'est-ce qu'une action réussie ? \\
Qu'est-ce qu'une alternative ? \\
Qu'est-ce qu'une âme ? \\
Qu'est-ce qu'une analyse ? \\
Qu'est-ce qu'une aporie ? \\
Qu'est-ce qu'une autorité légitime ? \\
Qu'est-ce qu'une avant-garde ? \\
Qu'est-ce qu'une belle démonstration ? \\
Qu'est-ce qu'une belle forme ? \\
Qu'est-ce qu'une belle mort ? \\
Qu'est-ce qu'une bête ? \\
Qu'est-ce qu'une bonne définition ? \\
Qu'est-ce qu'une bonne délibération ? \\
Qu'est-ce qu'une bonne éducation ? \\
Qu'est-ce qu'une bonne loi ? \\
Qu'est-ce qu'une bonne méthode ? \\
Qu'est-ce qu'une bonne traduction ? \\
Qu'est-ce qu'une catastrophe ? \\
Qu'est-ce qu'une catégorie de l'être ? \\
Qu'est-ce qu'une catégorie ? \\
Qu'est-ce qu'une cause ? \\
Qu'est-ce qu'un échange juste ? \\
Qu'est-ce qu'un échange réussi ? \\
Qu'est-ce qu'une chose matérielle ? \\
Qu'est-ce qu'une chose ? \\
Qu'est-ce qu'une civilisation ? \\
Qu'est-ce qu'une collectivité ? \\
Qu'est-ce qu'une comédie ? \\
Qu'est-ce qu'une communauté politique ? \\
Qu'est-ce qu'une communauté ? \\
Qu'est-ce qu'une conception scientifique du monde ? \\
Qu'est-ce qu'une condition suffisante ? \\
Qu'est-ce qu'une conduite irrationnelle ? \\
Qu'est ce qu'une connaissance fiable ? \\
Qu'est-ce qu'une connaissance non scientifique ? \\
Qu'est-ce qu'une connaissance par les faits ? \\
Qu'est-ce qu'une constitution ? \\
Qu'est-ce qu'une contrainte ? \\
Qu'est-ce qu'une convention ? \\
Qu'est-ce qu'une conviction ? \\
Qu'est-ce qu'une crise politique ? \\
Qu'est-ce qu'une crise ? \\
Qu'est-ce qu'une croyance rationnelle ? \\
Qu'est-ce qu'une croyance vraie ? \\
Qu'est-ce qu'une croyance ? \\
Qu'est-ce qu'une culture ? \\
Qu'est-ce qu'une décision politique ? \\
Qu'est-ce qu'une décision rationnelle ? \\
Qu'est-ce qu'une découverte scientifique ? \\
Qu'est-ce qu'une découverte ? \\
Qu'est-ce qu'une définition ? \\
Qu'est-ce qu'une démocratie ? \\
Qu'est-ce qu'une démonstration ? \\
Qu'est-ce qu'une discipline savante ? \\
Qu'est-ce qu'une école philosophique ? \\
Qu'est-ce qu'une éducation réussie ? \\
Qu'est-ce qu'une éducation scientifique ? \\
Qu'est-ce qu'une époque ? \\
Qu'est-ce qu'une erreur ? \\
Qu'est-ce qu'une exception ? \\
Qu'est-ce qu'une existence historique ? \\
Qu'est-ce qu'une expérience cruciale ? \\
Qu'est-ce qu'une expérience de pensée ? \\
Qu'est-ce qu'une expérience religieuse ? \\
Qu'est-ce qu'une expérience scientifique ? \\
Qu'est-ce qu'une expérience ? \\
Qu'est-ce qu'une explication matérialiste ? \\
Qu'est-ce qu'une exposition ? \\
Qu'est-ce qu'une famille ? \\
Qu'est-ce qu'une fausse science ? \\
Qu'est-ce qu'une faute de goût ? \\
Qu'est-ce qu'une fiction ? \\
Qu'est-ce qu'une fonction ? \\
Qu'est-ce qu'une forme ? \\
Qu'est-ce qu'une grande cause ? \\
Qu'est-ce qu'une guerre juste ? \\
Qu'est-ce qu'une histoire vraie ? \\
Qu'est-ce qu'une hypothèse scientifique ? \\
Qu'est-ce qu'une hypothèse ? \\
Qu'est-ce qu'une idée esthétique ? \\
Qu'est-ce qu'une idée incertaine ? \\
Qu'est-ce qu'une idée morale ? \\
Qu'est-ce qu'une idée vraie ? \\
Qu'est-ce qu'une idée ? \\
Qu'est-ce qu'une idéologie ? \\
Qu'est-ce qu'une illusion ? \\
Qu'est-ce qu'une image ? \\
Qu'est-ce qu'une inégalité ? \\
Qu'est-ce qu'une injustice ? \\
Qu'est-ce qu'une institution ? \\
Qu'est-ce qu'une interprétation ? \\
Qu'est-ce qu'une invention technique ? \\
Qu'est-ce qu'une langue artificielle ? \\
Qu'est-ce qu'une langue bien faite ? \\
Qu'est-ce qu'une langue morte ? \\
Qu'est-ce qu'une langue ? \\
Qu'est-ce qu'un élément ? \\
Qu'est-ce qu'une libération ? \\
Qu'est-ce qu'une libre interprétation ? \\
Qu'est-ce qu'une limite ? \\
Qu'est-ce qu'une logique sociale ? \\
Qu'est-ce qu'une loi de la nature ? \\
Qu'est ce qu'une loi de la pensée ? \\
Qu'est ce qu'une loi scientifique ? \\
Qu'est-ce qu'une loi scientifique ? \\
Qu'est-ce qu'une loi ? \\
Qu'est-ce qu'une machine ? \\
Qu'est-ce qu'une maladie ? \\
Qu'est-ce qu'une marchandise ? \\
Qu'est ce qu'une mauvaise idée ? \\
Qu'est-ce qu'une mauvaise interprétation ? \\
Qu'est-ce qu'une méditation métaphysique ? \\
Qu'est-ce qu'une méditation ? \\
Qu'est-ce qu'une mentalité collective ? \\
Qu'est-ce qu'une métaphore ? \\
Qu'est-ce qu'une méthode ? \\
Qu'est-ce qu'une morale de la communication ? \\
Qu'est-ce qu'un empire ? \\
Qu'est-ce qu'une nation ? \\
Qu'est-ce qu'un enfant ? \\
Qu'est-ce qu'un ennemi ? \\
Qu'est-ce qu'une norme sociale ? \\
Qu'est-ce qu'une norme ? \\
Qu'est-ce qu'une nouveauté ? \\
Qu'est-ce qu'une œuvre d'art authentique ? \\
Qu'est-ce qu'une œuvre d'art réaliste ? \\
Qu'est-ce qu'une œuvre d'art ? \\
Qu'est-ce qu'une œuvre ratée ? \\
Qu'est-ce qu'une œuvre ? \\
Qu'est-ce qu'une œuvre « géniale » ? \\
Qu'est-ce qu'une parole libre ? \\
Qu'est-ce qu'une parole vraie ? \\
Qu'est-ce qu'une passion ? \\
Qu'est-ce qu'une patrie ? \\
Qu'est-ce qu'une pensée libre ? \\
Qu'est-ce qu'une période en histoire ? \\
Qu'est-ce qu'une personne morale ? \\
Qu'est-ce qu'une personne ? \\
Qu'est-ce qu‘une philosophie première ? \\
Qu'est-ce qu'une philosophie ? \\
Qu'est-ce qu'une phrase ? \\
Qu'est-ce qu'une politique sociale ? \\
Qu'est-ce qu'une preuve ? \\
Qu'est-ce qu'une promesse ? \\
Qu'est-ce qu'une propriété essentielle ? \\
Qu'est-ce qu'une propriété ? \\
Qu'est-ce qu'une psychologie scientifique ? \\
Qu'est-ce qu'une question dénuée de sens ? \\
Qu'est-ce qu'une question métaphysique ? \\
Qu'est-ce qu'une question ? \\
Qu'est-ce qu'une raison d'agir ? \\
Qu'est-ce qu'une réfutation ? \\
Qu'est-ce qu'une règle de vie ? \\
Qu'est-ce qu'une règle ? \\
Qu'est-ce qu'une relation ? \\
Qu'est ce qu'une religion ? \\
Qu'est-ce qu'une rencontre ? \\
Qu'est-ce qu'une représentation réussie ? \\
Qu'est-ce qu'une république ? \\
Qu'est-ce qu'une révélation ? \\
Qu'est-ce qu'une révolution politique ? \\
Qu'est-ce qu'une révolution scientifique ? \\
Qu'est-ce qu'une révolution ? \\
Qu'est-ce qu'une science exacte ? \\
Qu'est-ce qu'une science expérimentale ? \\
Qu'est-ce qu'une science humaine ? \\
Qu'est-ce qu'une science rigoureuse ? \\
Qu'est-ce qu'un esclave ? \\
Qu'est-ce qu'une situation tragique ? \\
Qu'est-ce qu'une société juste ? \\
Qu'est-ce qu'une société libre ? \\
Qu'est-ce qu'une société mondialisée ? \\
Qu'est-ce qu'une société ouverte ? \\
Qu'est-ce qu'une solution ? \\
Qu'est-ce qu'un esprit faux ? \\
Qu'est-ce qu'un esprit juste ? \\
Qu'est ce qu'un esprit libre ? \\
Qu'est-ce qu'un esprit libre ? \\
Qu'est-ce qu'un esprit profond ? \\
Qu'est-ce qu'une structure ? \\
Qu'est-ce qu'une substance ? \\
Qu'est-ce qu'un état de droit ? \\
Qu'est-ce qu'un État de droit ? \\
Qu'est-ce qu'un État libre ? \\
Qu'est-ce qu'un état mental ? \\
Qu'est-ce qu'une théorie scientifique ? \\
Qu'est-ce qu'une théorie ? \\
Qu'est-ce qu'une tradition ? \\
Qu'est-ce qu'une tragédie historique ? \\
Qu'est-ce qu'une tragédie ? \\
Qu'est-ce qu'un être cultivé ? \\
Qu'est-ce qu'un être vivant ? \\
Qu'est-ce qu'une valeur ? \\
Qu'est-ce qu'un événement fondateur ? \\
Qu'est-ce qu'un événement historique ? \\
Qu'est-ce qu'un événement ? \\
Qu'est-ce qu'une vérité contingente ? \\
Qu'est-ce qu'une vérité historique ? \\
Qu'est-ce qu'une vérité scientifique ? \\
Qu'est-ce qu'une vérité subjective ? \\
Qu'est-ce qu'une vertu ? \\
Qu'est-ce qu'une vie heureuse ? \\
Qu'est-ce qu'une vie humaine ? \\
Qu'est-ce qu'une vie réussie ? \\
Qu'est-ce qu'une ville ? \\
Qu'est-ce qu'une violence symbolique ? \\
Qu'est-ce qu'une vision du monde ? \\
Qu'est-ce qu'une vision scientifique du monde ? \\
Qu'est-ce qu'une volonté libre ? \\
Qu'est-ce qu'une volonté raisonnable ? \\
Qu'est-ce qu'un exemple ? \\
Qu'est-ce qu'un expérimentateur ? \\
Qu'est-ce qu'un expert ? \\
Qu'est-ce qu'une « expérience de pensée » ? \\
Qu'est-ce qu'une « performance » ? \\
Qu'est-ce qu'un fait de culture ? \\
Qu'est-ce qu'un fait de société ? \\
Qu'est-ce qu'un fait divers ? \\
Qu'est-ce qu'un fait historique ? \\
Qu'est-ce qu'un fait moral ? \\
Qu'est ce qu'un fait scientifique ? \\
Qu'est-ce qu'un fait scientifique ? \\
Qu'est-ce qu'un fait social ? \\
Qu'est-ce qu'un fait ? \\
Qu'est-ce qu'un faux problème ? \\
Qu'est-ce qu'un faux sentiment ? \\
Qu'est-ce qu'un faux ? \\
Qu'est-ce qu'un film ? \\
Qu'est-ce qu'un génie ? \\
Qu'est-ce qu'un geste artistique ? \\
Qu'est-ce qu'un geste technique ? \\
Qu'est-ce qu'un gouvernement démocratique ? \\
Qu'est-ce qu'un gouvernement juste ? \\
Qu'est-ce qu'un gouvernement républicain ? \\
Qu'est-ce qu'un gouvernement ? \\
Qu'est-ce qu'un grand homme ou une grande femme ? \\
Qu'est-ce qu'un grand homme ? \\
Qu'est-ce qu'un grand philosophe ? \\
Qu'est-ce qu'un héros ? \\
Qu'est-ce qu'un homme bon ? \\
Qu'est-ce qu'un homme d'action ? \\
Qu'est-ce qu'un homme d'État ? \\
Qu'est-ce qu'un homme d'expérience ? \\
Qu'est-ce qu'un homme juste ? \\
Qu'est-ce qu'un homme libre ? \\
Qu'est-ce qu'un homme méchant ? \\
Qu'est-ce qu'un homme normal ? \\
Qu'est-ce qu'un homme politique ? \\
Qu'est-ce qu'un homme sans éducation ? \\
Qu'est-ce qu'un homme seul ? \\
Qu'est-ce qu'un idéaliste ? \\
Qu'est-ce qu'un idéal moral ? \\
Qu'est-ce qu'un idéal ? \\
Qu'est-ce qu'un individu ? \\
Qu'est-ce qu'un intellectuel ? \\
Qu'est-ce qu'un jeu ? \\
Qu'est-ce qu'un jugement analytique ? \\
Qu'est-ce qu'un jugement de goût ? \\
Qu'est-ce qu'un justicier ? \\
Qu'est-ce qu'un laboratoire ? \\
Qu'est-ce qu'un langage technique ? \\
Qu'est-ce qu'un législateur ? \\
Qu'est-ce qu'un lieu commun ? \\
Qu'est-ce qu'un livre ? \\
Qu'est-ce qu'un maître ? \\
Qu'est-ce qu'un marginal ? \\
Qu'est-ce qu'un mécanisme social ? \\
Qu'est-ce qu'un métaphysicien ? \\
Qu'est-ce qu'un mineur ? \\
Qu'est-ce qu'un miracle ? \\
Qu'est-ce qu'un modèle ? \\
Qu'est-ce qu'un moderne ? \\
Qu'est-ce qu'un monde ? \\
Qu'est-ce qu'un monstre ? \\
Qu'est-ce qu'un monument ? \\
Qu'est-ce qu'un musée ? \\
Qu'est-ce qu'un mythe ? \\
Qu'est-ce qu'un nombre ? \\
Qu'est-ce qu'un nom propre ? \\
Qu'est-ce qu'un objet d'art ? \\
Qu'est-ce qu'un objet esthétique ? \\
Qu'est-ce qu'un objet mathématique ? \\
Qu'est-ce qu'un objet métaphysique ? \\
Qu'est-ce qu'un objet technique ? \\
Qu'est-ce qu'un objet ? \\
Qu'est-ce qu'un œuvre d'art ? \\
Qu'est-ce qu'un ordre ? \\
Qu'est-ce qu'un organisme ? \\
Qu'est-ce qu'un original ? \\
Qu'est-ce qu'un outil ? \\
Qu'est ce qu'un paradoxe ? \\
Qu'est-ce qu'un paradoxe ? \\
Qu'est-ce qu'un patrimoine ? \\
Qu'est-ce qu'un pauvre ? \\
Qu'est-ce qu'un paysage ? \\
Qu'est-ce qu'un pédant ? \\
Qu'est-ce qu'un peuple libre ? \\
Qu'est-ce qu'un peuple ? \\
Qu'est-ce qu'un phénomène ? \\
Qu'est-ce qu'un philosophe ? \\
Qu'est-ce qu'un plaisir pur ? \\
Qu'est-ce qu'un point de vue ? \\
Qu'est-ce qu'un portrait ? \\
Qu'est-ce qu'un post-moderne ? \\
Qu'est-ce qu'un précurseur ? \\
Qu'est-ce qu'un préjugé ? \\
Qu'est-ce qu'un primitif ? \\
Qu'est-ce qu'un prince juste ? \\
Qu'est-ce qu'un principe ? \\
Qu'est-ce qu'un problème éthique ? \\
Qu'est-ce qu'un problème métaphysique ? \\
Qu'est-ce qu'un problème philosophique ? \\
Qu'est-ce qu'un problème politique ? \\
Qu'est-ce qu'un problème scientifique ? \\
Qu'est-ce qu'un problème technique ? \\
Qu'est-ce qu'un problème ? \\
Qu'est-ce qu'un produit culturel ? \\
Qu'est-ce qu'un programme politique ? \\
Qu'est-ce qu'un programmer ? \\
Qu'est-ce qu'un programme ? \\
Qu'est-ce qu'un progrès scientifique ? \\
Qu'est-ce qu'un progrès technique ? \\
Qu'est-ce qu'un prophète ? \\
Qu'est-ce qu'un public ? \\
Qu'est-ce qu'un rapport de force ? \\
Qu'est-ce qu'un récit véridique ? \\
Qu'est-ce qu'un récit ? \\
Qu'est-ce qu'un réfutation ? \\
Qu'est-ce qu'un régime politique ? \\
Qu'est-ce qu'un réseau ? \\
Qu'est-ce qu'un rhéteur ? \\
Qu'est-ce qu'un rite ? \\
Qu'est-ce qu'un rival ? \\
Qu'est-ce qu'un sage ? \\
Qu'est-ce qu'un savoir-faire ? \\
Qu'est-ce qu'un sceptique ? \\
Qu'est-ce qu'un sentiment moral ? \\
Qu'est-ce qu'un sentiment vrai ? \\
Qu'est-ce qu'un signe ? \\
Qu'est-ce qu'un sophisme ? \\
Qu'est-ce qu'un sophiste ? \\
Qu'est-ce qu'un souvenir ? \\
Qu'est-ce qu'un spécialiste ? \\
Qu'est-ce qu'un spectateur ? \\
Qu'est-ce qu'un style ? \\
Qu'est-ce qu'un symbole ? \\
Qu'est-ce qu'un symptôme ? \\
Qu'est-ce qu'un système philosophique ? \\
Qu'est-ce qu'un système ? \\
Qu'est-ce qu'un tableau ? \\
Qu'est-ce qu'un tabou ? \\
Qu'est-ce qu'un technicien ? \\
Qu'est-ce qu'un témoin ? \\
Qu'est-ce qu'un temple ? \\
Qu'est-ce qu'un texte ? \\
Qu'est-ce qu'un tout ? \\
Qu'est-ce qu'un traître ? \\
Qu'est-ce qu'un travail bien fait ? \\
Qu'est-ce qu'un trouble social ? \\
Qu'est-ce qu'un tyran ? \\
Qu'est-ce qu'un vice ? \\
Qu'est-ce qu'un visage ? \\
Qu'est-ce qu'un vrai changement ? \\
Qu'est-ce qu'un « champ artistique » ? \\
Qu'est-ce qu'un « être dégénéré » ? \\
Qu'est qu'une image ? \\
Qu'est qu'un régime politique ? \\
Que suis-je ? \\
Que suppose le mouvement ? \\
Que valent les excuses ? \\
Que valent les idées générales ? \\
Que valent les mots ? \\
Que valent les préjugés ? \\
Que valent les théories ? \\
Que vaut en morale la justification par l'utilité ? \\
Que vaut la décision de la majorité ? \\
Que vaut la définition de l'homme comme animal doué de raison ? \\
Que vaut la distinction entre nature et culture ? \\
Que vaut l'excuse : « C'est plus fort que moi » ? \\
Que vaut l'excuse : « Je ne l'ai pas fait exprès» ? \\
Que vaut l'incertain ? \\
Que vaut une preuve contre un préjugé ? \\
Que veut dire avoir raison ? \\
Que veut dire introduire à la métaphysique ? \\
Que veut dire l'expression « aller au fond des choses » ? \\
Que veut dire « essentiel » ? \\
Que veut dire « je t'aime » ? \\
Que veut dire « réel » ? \\
Que veut dire « respecter la nature » ? \\
Que veut dire : « être cultivé » ? \\
Que veut dire : « je t'aime » ? \\
Que veut dire : « le temps passe » ? \\
Que veut dire : « respecter la nature » ? \\
Que voit-on dans une image ? \\
Que voit-on dans un miroir ? \\
Que voit-on dans un tableau ? \\
Que voyons-nous ? \\
Qu'expriment les mythes ? \\
Qu'exprime une œuvre d'art ? \\
Qui agit ? \\
Qui a le droit de juger ? \\
Qui a une histoire ? \\
Qui a une parole politique ? \\
Qui commande ? \\
Qui connaît le mieux mon corps ? \\
Qui croire ? \\
Qui doit faire les lois ? \\
Qui écrit l'histoire ? \\
Qui est autorisé à me dire « tu dois » ? \\
Qui est citoyen ? \\
Qui est compétent en matière politique ? \\
Qui est digne du bonheur ? \\
Qui est immoral ? \\
Qui est l'autre ? \\
Qui est le maître ? \\
Qui est l'homme des sciences humaines ? \\
Qui est libre ? \\
Qui est métaphysicien ? \\
Qui est mon prochain ? \\
Qui est mon semblable ? \\
Qui est riche ? \\
Qui est sage ? \\
Qui est souverain ? \\
Qui fait la loi ? \\
Qui fait l'histoire ? \\
Qui gouverne ? \\
Qui mérite d'être aimé ? \\
Qui meurt ? \\
Qui nous dicte nos devoirs ? \\
Qui parle quand je dis « je » ? \\
Qui parle ? \\
Qui pense ? \\
Qui peut avoir des droits ? \\
Qui peut me dire « tu ne dois pas » ? \\
Qui peut parler ? \\
Qui suis-je et qui es-tu ? \\
Qui suis-je ? \\
Qui travaille ? \\
Qu'oppose-t-on à la vérité ? \\
Qu'y a-t-il à comprendre dans une œuvre d'art ? \\
Qu'y a-t-il à comprendre en histoire ? \\
Qu'y a-t-il à l'origine de toutes choses ? \\
Qu'y a-t-il au-delà du réel ? \\
Qu'y a-t-il au fondement de l'objectivité ? \\
Qu'y a-t-il de sérieux dans le jeu ? \\
Qu'y a-t-il d'universel dans la culture ? \\
Qu'y a-t-il ? \\
Rassembler les hommes, est-ce les unir ? \\
Rebuts et objets quelconques : une matière pour l'art ? \\
Reconnaissons-nous le bien comme nous reconnaissons le vrai ? \\
Recourir au langage, est-ce renoncer à la violence ? \\
Résister peut-il être un droit ? \\
Respecter la nature, est-ce renoncer à l'exploiter ? \\
Revient-il à l'État d'assurer le bonheur des citoyens ? \\
Revient-il à l'État d'assurer votre bonheur ? \\
Rêvons-nous ? \\
Sait-on ce que l'on veut ? \\
Sait-on ce qu'on fait ? \\
Sait-on ce qu'on veut ? \\
Sait-on nécessairement ce que l'on désire ? \\
Sait-on toujours ce que l'on fait ? \\
Sait-on toujours ce que l'on veut ? \\
Sait-on toujours ce qu'on veut ? \\
Savoir est-ce cesser de croire ? \\
Savoir, est-ce pouvoir ? \\
Savoir est-ce se libérer ? \\
Savons-nous ce que nous disons ? \\
Sciences humaines et liberté sont-elles compatibles ? \\
Se cultiver, est-ce s'affranchir de son appartenance culturelle ? \\
Se mentir à soi-même : est-ce possible ? \\
Serait-il immoral d'autoriser le commerce des organes humains ? \\
Se retirer dans la pensée ? \\
Serions-nous heureux dans un ordre politique parfait ? \\
Serions-nous plus libres sans État ? \\
Servir, est-ce nécessairement renoncer à sa liberté ? \\
Seul le présent existe-t-il ? \\
Si Dieu n'existe pas, tout est-il permis ? \\
Si Dieu n'existe pas, tout est-il possible ? \\
Si l'esprit n'est pas une table rase, qu'est-il ? \\
Si l'État n'existait pas, faudrait-il l'inventer ? \\
S'indigner, est-ce un devoir ? \\
Si nous étions moraux, le droit serait-il inutile ? \\
Si tout est historique, tout est-il relatif ? \\
Sommes-nous capables d'agir de manière désintéressée ? \\
Sommes-nous conscients de nos mobiles ? \\
Sommes-nous dans le temps comme dans l'espace ? \\
Sommes-nous des êtres métaphysiques ? \\
Sommes-nous des sujets ? \\
Sommes-nous déterminés par notre culture ? \\
Sommes-nous dominés par la technique ? \\
Sommes-nous faits pour la vérité ? \\
Sommes-nous faits pour le bonheur ? \\
Sommes-nous gouvernés par nos passions ? \\
Sommes-nous jamais certains d'avoir choisi librement ? \\
Sommes-nous les jouets de l'histoire ? \\
Sommes-nous les jouets de nos pulsions ? \\
Sommes-nous libres de nos croyances ? \\
Sommes-nous libres de nos pensées ? \\
Sommes-nous libres de nos préférences morales ? \\
Sommes-nous libres face à l'évidence ? \\
Sommes-nous maîtres de nos désirs ? \\
Sommes-nous maîtres de nos paroles ? \\
Sommes-nous perfectibles ? \\
Sommes-nous portés au bien ? \\
Sommes-nous prisonniers de nos désirs ? \\
Sommes-nous prisonniers de notre histoire ? \\
Sommes-nous prisonniers du temps ? \\
Sommes-nous responsables de ce dont nous n'avons pas conscience ? \\
Sommes-nous responsables de ce que nous sommes ? \\
Sommes-nous responsables de nos désirs ? \\
Sommes-nous responsables de nos erreurs ? \\
Sommes-nous responsables de nos opinions ? \\
Sommes-nous responsables de nos passions ? \\
Sommes-nous soumis au temps ? \\
Sommes-nous sujets de nos désirs ? \\
Sommes-nous toujours conscients des causes de nos désirs ?` \\
Sommes-nous toujours dépendants d'autrui ? \\
Sommes-nous tous contemporains ? \\
Suffit-il d'avoir raison ? \\
Suffit-il de bien juger pour bien faire ? \\
Suffit-il de faire son devoir pour être vertueux ? \\
Suffit-il de faire son devoir ? \\
Suffit-il de n'avoir rien fait pour être innocent ? \\
Suffit-il d'être informé pour comprendre ? \\
Suffit-il d'être juste ? \\
Suffit-il d'être vertueux pour être heureux ? \\
Suffit-il de voir le meilleur pour le suivre ? \\
Suffit-il de vouloir pour bien faire ? \\
Suffit-il, pour croire, de le vouloir ? \\
Suffit-il pour être juste d'obéir aux lois et aux coutumes de son pays ? \\
Suffit-il que nos intentions soient bonnes pour que nos actions le soient aussi ? \\
Suis-ce que j'ai conscience d'être ? \\
Suis-je aussi ce que j'aurais pu être ? \\
Suis-je ce que j'ai conscience d'être ? \\
Suis-je ce que je fais ? \\
Suis-je dans le temps comme je suis dans l'espace ? \\
Suis-je étranger à moi-même ? \\
Suis-je l'auteur de ce que je dis ? \\
Suis-je le même en des temps différents ? \\
Suis-je le mieux placé pour me connaître ? \\
Suis-je libre ? \\
Suis-je maître de ma conscience ? \\
Suis-je maître de mes pensées ? \\
Suis-je ma mémoire ? \\
Suis-je mon corps ? \\
Suis-je mon passé ? \\
Suis-je propriétaire de mon corps ? \\
Suis-je responsable de ce dont je n'ai pas conscience ? \\
Suis-je responsable de ce que je suis ? \\
Suis-je seul au monde ? \\
Suis-je toujours autre que moi-même ? \\
Sur quoi fonder la justice ? \\
Sur quoi fonder la légitimité de la loi ? \\
Sur quoi fonder la propriété ? \\
Sur quoi fonder la société ? \\
Sur quoi fonder l'autorité politique ? \\
Sur quoi fonder l'autorité ? \\
Sur quoi fonder le devoir ? \\
Sur quoi fonder le droit de punir ? \\
Sur quoi le langage doit-il se régler ? \\
Sur quoi l'historien travaille-t-il ? \\
Sur quoi repose l'accord des esprits ? \\
Sur quoi repose la croyance au progrès ? \\
Sur quoi reposent nos certitudes ? \\
Sur quoi se fonde la connaissance scientifique ? \\
Toujours plus vite ? \\
Tous les conflits peuvent-ils être résolus par le dialogue ? \\
Tous les désirs sont-ils naturels ? \\
Tous les droits sont-ils formels ? \\
Tous les hommes désirent-ils connaître ? \\
Tous les hommes désirent-ils être heureux ? \\
Tous les hommes désirent-ils naturellement être heureux ? \\
Tous les hommes désirent-ils naturellement savoir ? \\
Tous les hommes sont-ils égaux ? \\
Tous les paradis sont-ils perdus ? \\
Tous les plaisirs se valent-ils ? \\
Tous les rapports humains sont-ils des échanges ? \\
Tout art est-il poésie ? \\
Tout a-t-il une cause ? \\
Tout a-t-il une raison d'être ? \\
Tout a-t-il un prix ? \\
Tout a-t-il un sens ? \\
Tout ce qui est excessif est-il insignifiant ? \\
Tout ce qui est naturel est-il normal ? \\
Tout ce qui est rationnel est-il raisonnable ? \\
Tout ce qui est vrai doit-il être prouvé ? \\
Tout ce qui existe a-t-il un prix ? \\
Tout comprendre, est-ce tout pardonner ? \\
Tout désir est-il désir de posséder ? \\
Tout désir est-il égoïste ? \\
Tout désir est-il manque ? \\
Tout désir est-il une souffrance ? \\
Tout devoir est-il l'envers d'un droit ? \\
Tout droit est-il un pouvoir ? \\
Toute action politique est-elle collective ? \\
Toute chose a-t-elle une essence ? \\
Toute communauté est-elle politique ? \\
Toute compréhension implique-t-elle une interprétation ? \\
Toute connaissance autre que scientifique doit-elle être considérée comme une illusion ? \\
Toute connaissance commence-t-elle avec l'expérience ? \\
Toute connaissance consiste-t-elle en un savoir-faire ? \\
Toute connaissance est-elle historique ? \\
Toute connaissance est-elle hypothétique ? \\
Toute connaissance s'enracine-t-elle dans la perception ? \\
Toute conscience est-elle conscience de soi ? \\
Toute conscience est-elle subjective ? \\
Toute conscience n'est-elle pas implicitement morale ? \\
Toute description est-elle une interprétation ? \\
Toute existence est-elle indémontrable ? \\
Toute expérience appelle-t-elle une interprétation ? \\
Toute expression est-elle métaphorique ? \\
Toute faute est-elle une erreur ? \\
Toute hiérarchie est-elle inégalitaire ? \\
Toute inégalité est-elle injuste ? \\
Toute interprétation est-elle contestable ? \\
Toute interprétation est-elle subjective ? \\
Toute métaphysique implique-t-elle une transcendance ? \\
Toute morale implique-t-elle l'effort ? \\
Toute morale s'oppose-t-elle aux désirs ? \\
Tout énoncé est-il nécessairement vrai ou faux ? \\
Toute notre connaissance dérive-t-elle de l'expérience ? \\
Toute origine est-elle mythique ? \\
Toute passion fait-elle souffrir ? \\
Toute pensée revêt-elle nécessairement une forme linguistique ? \\
Toute peur est-elle irrationnelle ? \\
Toute philosophie constitue-t-elle une doctrine ? \\
Toute philosophie est-elle systématique ? \\
Toute philosophie implique-t-elle une politique ? \\
Toute relation humaine est-elle un échange ? \\
Toute science est-elle naturelle ? \\
Toutes les choses sont-elles singulières ? \\
Toutes les convictions sont-elles respectables ? \\
Toutes les croyances se valent-elles ? \\
Toutes les fautes se valent-elles ? \\
Toutes les inégalités ont-elles une importance politique ? \\
Toutes les inégalités sont-elles des injustices ? \\
Toutes les interprétations se valent-elles ? \\
Toutes les opinions se valent-elles ? \\
Toutes les opinions sont-elles bonnes à dire ? \\
Toutes les vérités  scientifiques sont-elles révisables ? \\
Toute société a-t-elle besoin d'une religion ? \\
Tout est-il affaire de point de vue ? \\
Tout est-il à vendre ? \\
Tout est-il connaissable ? \\
Tout est-il faux dans la fiction ? \\
Tout est-il historique ? \\
Tout est-il matière ? \\
Tout est-il mesurable ? \\
Tout est-il nécessaire ? \\
Tout est-il politique ? \\
Tout est-il quantifiable ? \\
Tout est-il relatif ? \\
Tout être est-il dans l'espace ? \\
Toute vérité est-elle bonne à dire ? \\
Toute vérité est-elle démontrable ? \\
Toute vérité est-elle vérifiable ? \\
Toute vie est-elle intrinsèquement respectable ? \\
Toute violence est-elle contre nature ? \\
Tout futur est-il contingent ? \\
Tout ordre est-il une violence déguisée ? \\
Tout peut-il être objet d'échange ? \\
Tout peut-il être objet de jugement esthétique ? \\
Tout peut-il être objet de science ? \\
Tout peut-il n'être qu'apparence ? \\
Tout peut-il s'acheter ? \\
Tout peut-il se démontrer ? \\
Tout peut-il se vendre ? \\
Tout peut-il s'expliquer ? \\
Tout pouvoir corrompt-il ? \\
Tout pouvoir est-il oppresseur ? \\
Tout pouvoir est-il politique ? \\
Tout pouvoir n'est-il pas abusif ? \\
Tout principe est-il un fondement ? \\
Tout savoir a-t-il une justification ? \\
Tout savoir est-il fondé sur un savoir premier ? \\
Tout savoir est-il pouvoir ? \\
Tout savoir est-il transmissible ? \\
Tout savoir est-il un pouvoir ? \\
Tout savoir peut-il se transmettre ? \\
Tout s'en va-t-il avec le temps ? \\
Tout se prête-il à la mesure ? \\
Tout travail est-il forcé ? \\
Tout travail est-il social ? \\
Traduire, est-ce trahir ? \\
Traiter des faits humains comme des choses, est-ce considérer l'homme comme une chose ? \\
Traiter les faits humains comme des choses, est-ce réduire les hommes à des choses ? \\
Travailler, est-ce faire œuvre ? \\
Travailler par plaisir, est-ce encore travailler ? \\
Travaille-t-on pour soi-même ? \\
Un acte désintéressé est-il possible ? \\
Un acte gratuit est-il possible ? \\
Un acte libre est-il un acte imprévisible ? \\
Un acte peut-il être inhumain ? \\
Un artiste doit-il être original ? \\
Un art peut-il être populaire ? \\
Un art sans sublimation est-il possible ? \\
Un bien peut-il être commun ? \\
Un bien peut-il sortir d'un mal ? \\
Un choix peut-il être rationnel ? \\
Un contrat peut-il être injuste ? \\
Un contrat peut-il être social ? \\
Un désir peut-il être coupable ? \\
Un désir peut-il être inconscient ? \\
Un Dieu unique ? \\
Une action peut-elle être désintéressée ? \\
Une action peut-elle être machinale ? \\
Une action vertueuse se reconnaît-elle à sa difficulté ? \\
Une activité inutile est-elle sans valeur ? \\
Une cause peut-elle être libre ? \\
Une communauté politique n'est-elle qu'une communauté d'intérêt ? \\
Une connaissance peut-elle ne pas être relative ? \\
Une connaissance scientifique du vivant est-elle possible ? \\
Une croyance infondée est-elle illégitime ? \\
Une croyance peut-elle être libre ? \\
Une croyance peut-elle être rationnelle ? \\
Une culture de masse est-elle une culture ? \\
Une culture peut-elle être porteuse de valeurs universelles ? \\
Une décision politique peut-elle être juste ? \\
Une destruction peut-elle être créatrice ? \\
Une éducation esthétique est-elle possible ? \\
Une éducation morale est-elle possible ? \\
Une éthique sceptique est-elle possible ? \\
Une existence se démontre-t-elle ? \\
Une expérience peut-elle être fictive ? \\
Une explication peut-elle être réductrice ? \\
Une fiction peut-elle être vraie ? \\
Une guerre peut-elle être juste ? \\
Une idée peut-elle être fausse ? \\
Une idée peut-elle être générale ? \\
Une imitation peut-elle être parfaite ? \\
Une intention peut-elle être coupable ? \\
Une interprétation est-elle nécessairement subjective ? \\
Une interprétation peut-elle échapper à l'arbitraire ? \\
Une interprétation peut-elle être définitive ? \\
Une interprétation peut-elle être objective ? \\
Une interprétation peut-elle prétendre à la vérité ? \\
Une langue n'est-elle faite que de mots ? \\
Une ligne de conduite peut-elle tenir lieu de morale ? \\
Une logique non-formelle est-elle possible ? \\
Une loi n'est-elle qu'une règle ? \\
Une loi peut-elle être injuste ? \\
Une machine peut-elle avoir une mémoire ? \\
Une machine peut-elle penser ? \\
Une machine pourrait-elle penser ? \\
Une métaphysique athée est-elle possible ? \\
Une métaphysique peut-elle être sceptique ? \\
Une morale du plaisir est-elle concevable ? \\
Une morale peut-elle être dépassée ? \\
Une morale peut-elle être provisoire ? \\
Une morale peut-elle prétendre à l'universalité ? \\
Une morale sans obligation est-elle possible ? \\
Une morale sceptique est-elle possible ? \\
Une œuvre d'art a-t-elle toujours un sens ? \\
Une œuvre d'art doit-elle avoir un sens ? \\
Une œuvre d'art doit-elle nécessairement être belle ? \\
Une œuvre d'art doit-elle plaire ? \\
Une œuvre d'art est-elle une marchandise ? \\
Une œuvre d'art peut-elle être immorale ? \\
Une œuvre d'art peut-elle être laide ? \\
Une œuvre d'art s'explique-t-elle à partir de ses influences ? \\
Une œuvre doit-elle nécessairement être belle ? \\
Une œuvre est-elle nécessairement singulière ? \\
Une œuvre est-elle toujours de son temps ? \\
Une pensée contradictoire est-elle dénuée de valeur ? \\
Une perception peut-elle être illusoire ? \\
Une philosophie de l'amour est-elle possible ? \\
Une philosophie peut-elle être réactionnaire ? \\
Une politique peut-elle se réclamer de la vie ? \\
Une psychologie peut-elle être matérialiste ? \\
Une religion civile est-elle possible ? \\
Une religion peut-elle être fausse ? \\
Une religion peut-elle être rationnelle ? \\
Une religion peut-elle se passer de pratiques ? \\
Une religion rationnelle est-elle possible ? \\
Une science de la conscience est-elle possible ? \\
Une science de la culture est-elle possible ? \\
Une science de la morale est-elle possible ? \\
Une science de l'éducation est-elle possible ? \\
Une science de l'esprit est-elle possible ? \\
Une science des symboles est-elle possible ? \\
Une sensation peut-elle être fausse ? \\
Une société d'athées est-elle possible ? \\
Une société juste est-ce une société sans conflit ? \\
Une société juste est-elle une société sans conflits ? \\
Une société n'est-elle qu'un ensemble d'individus ? \\
Une société sans conflit est-elle possible ? \\
Une société sans État est-elle possible ? \\
Une société sans État est-elle une société sans politique ? \\
Une société sans religion est-elle possible ? \\
Une société sans travail est-elle souhaitable ? \\
Un État mondial ? \\
Un État peut-il être trop étendu ? \\
Une théorie peut-elle être vérifiée ? \\
Une théorie scientifique peut-elle devenir fausse ? \\
Une théorie scientifique peut-elle être ramenée à des propositions empiriques élémentaires ? \\
Une théorie scientifique peut-elle être vraie ? \\
Un être vivant peut-il être comparé à une œuvre d'art ? \\
Un événement historique est-il toujours imprévisible ? \\
Une vérité peut-elle être indicible ? \\
Une vérité peut-elle être provisoire ? \\
Une vie heureuse est-elle une vie de plaisirs ? \\
Une vie libre exclut-elle le travail ? \\
Une volonté peut-elle être générale ? \\
Un fait existe-t-il sans interprétation ? \\
Un gouvernement de savants est-il souhaitable ? \\
Un homme n'est-il que la somme de ses actes ? \\
Un jeu peut-il être sérieux ? \\
Un jugement de goût est-il culturel ? \\
Un langage universel est-il concevable ? \\
Un mensonge peut-il avoir une valeur morale ? \\
Un monde sans nature est-il pensable ? \\
Un objet technique peut-il être beau ? \\
Un peuple est-il responsable de son histoire ? \\
Un peuple est-il un rassemblement d'individus ? \\
Un peuple se définit-il par son histoire ? \\
Un philosophe a-t-il des devoirs envers la société ? \\
Un pouvoir a-t-il besoin d'être légitime ? \\
Un problème moral peut-il recevoir une solution certaine ? \\
Un problème scientifique peut-il être insoluble ? \\
Un savoir peut-il être inconscient ? \\
Un sceptique peut-il être logicien ? \\
Un seul peut-il avoir raison contre tous ? \\
Un tableau peut-il être une dénonciation ? \\
Un vice, est-ce un manque ? \\
User de violence peut-il être moral ? \\
Vaut-il mieux oublier ou pardonner ? \\
Vaut-il mieux subir l'injustice que la commettre ? \\
Vaut-il mieux subir ou commettre l'injustice ? \\
Vit-on au présent ? \\
Vivons-nous tous dans le même monde ? \\
Vivrait-on mieux sans désirs ? \\
Vivre en société, est-ce seulement vivre ensemble ? \\
Vivre, est-ce interpréter ? \\
Vivre, est-ce lutter contre la mort ? \\
Vivre, est-ce lutter pour survivre ? \\
Vivre, est-ce résister à la mort ? \\
Vivre, est-ce un droit ? \\
Vivre sans religion, est-ce vivre sans espoir ? \\
Voit-on ce qu'on croit ? \\
Vouloir, est-ce encore désirer ? \\
Vouloir la paix sociale peut-il aller jusqu'à accepter l'injustice ? \\
Vulgariser la science ? \\
Y a-t-il continuité entre l'expérience et la science ? \\
Y a-t-il continuité ou discontinuité entre la pensée mythique et la science ? \\
Y a-t-il d'autres moyens que la démonstration pour établir la vérité ? \\
Y a-t-il de bons et de mauvais désirs ? \\
Y a-t-il de bons préjugés ? \\
Y a-t-il de fausses religions ? \\
Y a-t-il de faux besoins ? \\
Y a-t-il de faux problèmes ? \\
Y a-t-il de justes inégalités ? \\
Y a-t-il de la fatalité dans la vie de l'homme ? \\
Y a-t-il de la raison dans la perception ? \\
Y a-t-il de l'impensable ? \\
Y a-t-il de l'incommunicable ? \\
Y a-t-il de l'inconcevable ? \\
Y a-t-il de l'inconnaissable ? \\
Y a-t-il de l'indémontrable ? \\
Y a-t-il de l'indésirable ? \\
Y a-t-il de l'indicible ? \\
Y a-t-il de l'inexprimable ? \\
Y a-t-il de l'irréductible ? \\
Y a-t-il de l'irréfutable ? \\
Y a-t-il de l'irréparable ? \\
Y a-t-il de l'universel ? \\
Y a-t-il de mauvais désirs ? \\
Y a-t-il des acquis définitifs en science ? \\
Y a-t-il des actes de pensée ? \\
Y a-t-il des actes désintéressés ? \\
Y a-t-il des actes gratuits ? \\
Y a-t-il des actes moralement indifférents ? \\
Y a-t-il des actions désintéressées ? \\
Y a-t-il des arts mineurs ? \\
Y a-t-il des barbares ? \\
Y a-t-il des biens inestimables ? \\
Y a-t-il des canons de la beauté ? \\
Y a-t-il des certitudes historiques ? \\
Y a-t-il des choses qui échappent au droit ? \\
Y a-t-il des choses qu'on n'échange pas ? \\
Y a-t-il des compétences politiques ? \\
Y a-t-il des connaissances dangereuses ? \\
Y a-t-il des connaissances désintéressées ? \\
Y a-t-il des contraintes légitimes ? \\
Y a-t-il des convictions philosophiques ? \\
Y a-t-il des correspondances entre les arts ? \\
Y a-t-il des critères de l'humanité ? \\
Y a-t-il des critères du beau ? \\
Y a-t-il des critères du goût ? \\
Y a-t-il des croyances démocratiques ? \\
Y a-t-il des croyances nécessaires ? \\
Y a-t-il des croyances rationnelles ? \\
Y a-t-il des degrés dans la certitude ? \\
Y a-t-il des degrés de conscience ? \\
Y a-t-il des degrés de réalité ? \\
Y a-t-il des degrés de vérité ? \\
Y a-t-il des démonstrations en philosophie ? \\
Y a-t-il des despotes éclairés ? \\
Y a-t-il des déterminismes sociaux ? \\
Y a-t-il des devoirs envers soi-même ? \\
Y a-t-il des devoirs envers soi ? \\
Y a-t-il des dilemmes moraux ? \\
Y a-t-il des droits sans devoirs ? \\
Y a-t-il des erreurs de la nature ? \\
Y a-t-il des erreurs en politique ? \\
Y a-t-il des êtres mathématiques ? \\
Y a-t-il des évidences morales ? \\
Y a-t-il des expériences absolument certaines ? \\
Y a-t-il des expériences cruciales ? \\
Y a-t-il des expériences de la liberté ? \\
Y a-t-il des expériences métaphysiques ? \\
Y a-t-il des expériences sans théorie ? \\
Y a-t-il des facultés dans l'esprit ? \\
Y a-t-il des faits moraux ? \\
Y a-t-il des faits sans essence ? \\
Y a-t-il des faits scientifiques ? \\
Y a-t-il des faux problèmes ? \\
Y a-t-il des fins dans la nature ? \\
Y a-t-il des fins de la nature ? \\
Y a-t-il des fins dernières ? \\
Y a-t-il des fondements naturels à l'ordre social ? \\
Y a-t-il des genres de plaisir ? \\
Y a-t-il des genres du plaisir ? \\
Y a-t-il des guerres justes ? \\
Y a-t-il des héritages philosophiques ? \\
Y a-t-il des illusions de la conscience ? \\
Y a-t-il des illusions nécessaires ? \\
Y a-t-il des inégalités justes ? \\
Y a-t-il des injustices naturelles ? \\
Y a-t-il des instincts propres à l'Homme ? \\
Y a-t-il des interprétations fausses ? \\
Y a-t-il des intuitions morales ? \\
Y a-t-il des leçons de l'histoire ? \\
Y a-t-il des limites à la connaissance ? \\
Y a-t-il des limites à la conscience ? \\
Y a-t-il des limites à la tolérance ? \\
Y a-t-il des limites à l'exprimable ? \\
Y a-t-il des limites au droit ? \\
Y a-t-il des limites proprement morales à la discussion ? \\
Y a-t-il des lois de la pensée ? \\
Y a-t-il des lois de l'histoire ? \\
Y a-t-il des lois de l'Histoire ? \\
Y a-t-il des lois du hasard ? \\
Y a-t-il des lois du social ? \\
Y a-t-il des lois du vivant ? \\
Y a-t-il des lois en histoire ? \\
Y a-t-il des lois injustes ? \\
Y a-t-il des lois morales ? \\
Y a-t-il des lois non écrites ? \\
Y a-t-il des mentalités collectives ? \\
Y a-t-il des modèles en morale ? \\
Y a-t-il des mondes imaginaires ? \\
Y a-t-il des normes naturelles ? \\
Y a-t-il des objets qui n'existent pas ? \\
Y a-t-il des obstacles à la connaissance du vivant ? \\
Y a-t-il des passions collectives ? \\
Y a-t-il des passions intraitables ? \\
Y a-t-il des passions raisonnables ? \\
Y a-t-il des pathologies sociales ? \\
Y a-t-il des pensées folles ? \\
Y a-t-il des pensées inconscientes ? \\
Y a-t-il des perceptions insensibles ? \\
Y a-t-il des peuples sans histoire ? \\
Y a-t-il des plaisirs meilleurs que d'autres ? \\
Y a-t-il des plaisirs purs ? \\
Y a-t-il des preuves d'amour ? \\
Y a-t-il des principes de justice universels ? \\
Y a-t-il des progrès en art ? \\
Y a-t-il des propriétés singulières ? \\
Y a-t-il des questions sans réponse ? \\
Y a-t-il des raisons de douter de la raison ? \\
Y a-t-il des règles de la guerre ? \\
Y a-t-il des règles de l'art ? \\
Y a-t-il des régressions historiques ? \\
Y a-t-il des révolutions en art ? \\
Y a-t-il des révolutions scientifiques ? \\
Y a-t-il des sciences de l'homme ? \\
Y a-t-il des sciences exactes ? \\
Y a-t-il des secrets de la nature ? \\
Y a-t-il des sentiments moraux ? \\
Y a-t-il des signes naturels ? \\
Y a-t-il des sociétés sans État ? \\
Y a-t-il des sociétés sans histoire ? \\
Y a-t-il des solutions en politique ? \\
Y a-t-il des sots métiers ? \\
Y a-t-il des substances incorporelles ? \\
Y a-t-il des techniques de pensée ? \\
Y a-t-il des techniques du corps ? \\
Y a-t-il des valeurs absolues ? \\
Y a-t-il des valeurs naturelles ? \\
Y a-t-il des valeurs objectives ? \\
Y a-t-il des valeurs propres à la science ? \\
Y a-t-il des valeurs universelles ? \\
Y a-t-il des vérités de fait ? \\
Y a-t-il des vérités définitives ? \\
Y a-t-il des vérités en art ? \\
Y a-t-il des vérités éternelles ? \\
Y a-t-il des vérités indémontrables ? \\
Y a-t-il des vérités indiscutables ? \\
Y a-t-il des vérités morales ? \\
Y a-t-il des vérités philosophiques ? \\
Y a-t-il des vérités qui échappent à la raison ? \\
Y a-t-il des vérités sans preuve ? \\
Y a-t-il des vertus mineures ? \\
Y a-t-il des violences justifiées ? \\
Y a-t-il des violences légitimes ? \\
Y a-t-il différentes façons d'exister ? \\
Y a-t-il différentes manières de connaître ? \\
Y a-t-il du non-être ? \\
Y a-t-il du nouveau dans l'histoire ? \\
Y a-t-il du sacré dans la nature ? \\
Y a-t-il du synthétique \emph{a priori} ? \\
Y a-t-il encore des mythologies ? \\
Y a-t-il encore une sphère privée ? \\
Y a-t-il lieu d'opposer matière et esprit ? \\
Y a-t-il nécessairement du religieux dans l'art ? \\
Y a-t-il place pour l'idée de vérité en morale ? \\
Y a-t-il plusieurs libertés ? \\
Y a-t-il plusieurs manières de définir ? \\
Y a-t-il plusieurs morales ? \\
Y a-t-il plusieurs nécessités ? \\
Y a-t-il plusieurs sortes de matières ? \\
Y a-t-il plusieurs sortes de vérité ? \\
Y a-t-il progrès en art ? \\
Y a-t-il quoi que ce soit de nouveau dans l'histoire ? \\
Y a-t-il trop d'images ? \\
Y a-t-il un art de gouverner ? \\
Y a-t-il un art de penser ? \\
Y a-t-il un art d'être heureux ? \\
Y a-t-il un art de vivre ? \\
Y a-t-il un art d'interpréter ? \\
Y a-t-il un art d'inventer ? \\
Y a-t-il un art du bonheur ? \\
Y a-t-il un au-delà de la vérité ? \\
Y a-t-il un au-delà du langage ? \\
Y a-t-il un auteur de l'histoire ? \\
Y a-t-il un autre monde ? \\
Y a-t-il un beau idéal ? \\
Y a-t-il un beau naturel ? \\
Y a-t-il un besoin métaphysique ? \\
Y a-t-il un bien commun ? \\
Y a-t-il un bien plus précieux que la paix ? \\
Y a-t-il un bonheur sans illusion ? \\
Y a-t-il un bon usage des passions ? \\
Y a-t-il un bon usage du temps ? \\
Y a-t-il un canon de la beauté ? \\
Y a-t-il un critère de vérité ? \\
Y a-t-il un critère du vrai ? \\
Y a-t-il un devoir de mémoire ? \\
Y a-t-il un devoir d'être heureux ? \\
Y a-t-il un devoir d'indignation ? \\
Y a-t-il un droit à la différence ? \\
Y a-t-il un droit au bonheur ? \\
Y a-t-il un droit au travail ? \\
Y a-t-il un droit de désobéissance ? \\
Y a-t-il un droit de la guerre ? \\
Y a-t-il un droit de mentir ? \\
Y a-t-il un droit de mourir ? \\
Y a-t-il un droit de résistance ? \\
Y a-t-il un droit de révolte ? \\
Y a-t-il un droit d'ingérence ? \\
Y a-t-il un droit du plus faible ? \\
Y a-t-il un droit du plus fort ? \\
Y a-t-il un droit international ? \\
Y a-t-il un droit naturel ? \\
Y a-t-il un droit universel au mariage ? \\
Y a-t-il une argumentation métaphysique ? \\
Y a-t-il une beauté morale ? \\
Y a-t-il une beauté naturelle ? \\
Y a-t-il une beauté propre à l'objet technique ? \\
Y a-t-il une bonne imitation ? \\
Y a-t-il une causalité empirique ? \\
Y a-t-il une causalité en histoire ? \\
Y a-t-il une causalité historique ? \\
Y a-t-il une cause première ? \\
Y a-t-il une compétence en politique ? \\
Y a-t-il une compétence politique ? \\
Y a-t-il une condition humaine ? \\
Y a-t-il une connaissance du probable ? \\
Y a-t-il une connaissance du singulier ? \\
Y a-t-il une connaissance historique ? \\
Y a-t-il une connaissance métaphysique ? \\
Y a-t-il une connaissance sensible ? \\
Y a-t-il une conscience collective ? \\
Y a-t-il une correspondance des arts ? \\
Y a-t-il une éducation du goût ? \\
Y a-t-il une enfance de l'humanité ? \\
Y a-t-il une esthétique de la laideur ? \\
Y a-t-il une éthique de l'authenticité ? \\
Y a-t-il une éthique des moyens ? \\
Y a-t-il une expérience de la liberté ? \\
Y a-t-il une expérience de l'éternité ? \\
Y a-t-il une expérience du néant ? \\
Y a-t-il une expérience du temps ? \\
Y a-t-il une fin de l'histoire ? \\
Y a-t-il une fin dernière ? \\
Y a-t-il une fonction propre à l'œuvre d'art ? \\
Y a-t-il une force du droit ? \\
Y a-t-il une forme morale de fanatisme ? \\
Y a-t-il une hiérarchie des êtres ? \\
Y a-t-il une hiérarchie des sciences ? \\
Y a-t-il une hiérarchie du vivant ? \\
Y a-t-il une histoire de la nature ? \\
Y a-t-il une histoire de la raison ? \\
Y a-t-il une histoire de la vérité ? \\
Y a-t-il une histoire universelle ? \\
Y a-t-il une intelligence du corps ? \\
Y a-t-il une intentionnalité collective ? \\
Y a-t-il une irréversibilité du temps ? \\
Y a-t-il une justice naturelle ? \\
Y a-t-il une justice sans morale ? \\
Y a-t-il une langue de la philosophie ? \\
Y a-t-il une limite à la connaissance du vivant ? \\
Y a-t-il une limite au désir ? \\
Y a-t-il une limite au développement scientifique ? \\
Y a-t-il une logique dans l'histoire ? \\
Y a-t-il une logique de la découverte scientifique ? \\
Y a-t-il une logique de la découverte ? \\
Y a-t-il une logique des événements historiques ? \\
Y a-t-il une logique du désir ? \\
Y a-t-il une mécanique des passions ? \\
Y a-t-il une médecine de l'âme ? \\
Y a-t-il une métaphysique de l'amour ? \\
Y a-t-il une méthode de l'interprétation ? \\
Y a-t-il une méthode propre aux sciences humaines ? \\
Y a-t-il une morale universelle ? \\
Y a-t-il un empire de la technique ? \\
Y a-t-il une nature humaine ? \\
Y a-t-il une nécessité de l'erreur ? \\
Y a-t-il une nécessité de l'Histoire ? \\
Y a-t-il une nécessité morale ? \\
Y a-t-il une œuvre du temps ? \\
Y a-t-il une opinion publique mondiale ? \\
Y a-t-il une ou des morales ? \\
Y a-t-il une ou plusieurs philosophies ? \\
Y a-t-il une pensée sans signes ? \\
Y a-t-il une pensée technique ? \\
Y a-t-il une philosophie de la nature ? \\
Y a-t-il une philosophie de la philosophie ? \\
Y a-t-il une philosophie première ? \\
Y a-t-il une place pour la morale dans l'économie ? \\
Y a-t-il une positivité de l'erreur ? \\
Y a-t-il une présence du passé ? \\
Y a-t-il une primauté du devoir sur le droit ? \\
Y a-t-il une rationalité des sentiments ? \\
Y a-t-il une rationalité du hasard ? \\
Y a-t-il une réalité du hasard ? \\
Y a-t-il une responsabilité de l'artiste ? \\
Y a-t-il une sagesse populaire ? \\
Y a-t-il une science de la vie mentale ? \\
Y a-t-il une science de l'esprit ? \\
Y a-t-il une science de l'être ? \\
Y a-t-il une science de l'homme ? \\
Y a-t-il une science de l'individuel ? \\
Y a-t-il une science des principes ? \\
Y a-t-il une science du moi ? \\
Y a-t-il une science du qualitatif ? \\
Y a-t-il une science politique ? \\
Y a-t-il une sensibilité esthétique ? \\
Y a-t-il une servitude volontaire ? \\
Y a-t-il une spécificité de la délibération politique ? \\
Y a-t-il une spécificité des sciences humaines ? \\
Y a-t-il une spécificité du vivant ? \\
Y a-t-il un esprit scientifique ? \\
Y a-t-il un État idéal ? \\
Y a-t-il une technique de la nature ? \\
Y a-t-il une technique pour tout ? \\
Y a-t-il une unité de la science ? \\
Y a-t-il une unité des devoirs ? \\
Y a-t-il une unité des langages humains ? \\
Y a-t-il une unité des sciences ? \\
Y a-t-il une unité en psychologie ? \\
Y a-t-il une universalité des mathématiques ? \\
Y a-t-il une universalité du beau ? \\
Y a-t-il une valeur de l'inutile ? \\
Y a-t-il une vérité de l'œuvre d'art ? \\
Y a-t-il une vérité des apparences ? \\
Y a-t-il une vérité des représentations ? \\
Y a-t-il une vérité des sentiments ? \\
Y a-t-il une vérité des symboles ? \\
Y a-t-il une vérité du sensible ? \\
Y a-t-il une vérité du sentiment ? \\
Y a-t-il une vérité en histoire ? \\
Y a-t-il une vérité philosophique ? \\
Y a-t-il une vertu de l'imitation ? \\
Y a-t-il une vertu de l'oubli ? \\
Y a-t-il une vie de l'esprit ? \\
Y a-t-il une violence du droit ? \\
Y a-t-il un fondement de la croyance ? \\
Y a-t-il un inconscient collectif ? \\
Y a-t-il un inconscient psychique ? \\
Y a-t-il un inconscient social ? \\
Y a-t-il un jugement de l'histoire ? \\
Y a-t-il un langage animal ? \\
Y a-t-il un langage commun ? \\
Y a-t-il un langage de la musique ? \\
Y a-t-il un langage de l'art ? \\
Y a-t-il un langage de l'inconscient ? \\
Y a-t-il un langage du corps ? \\
Y a-t-il un langage unifié de la science ? \\
Y a-t-il un mal absolu ? \\
Y a-t-il un monde de l'art ? \\
Y a-t-il un monde extérieur ? \\
Y a-t-il un moteur de l'histoire ? \\
Y a-t-il un objet du désir ? \\
Y a-t-il un ordre dans la nature ? \\
Y a-t-il un ordre des choses ? \\
Y a-t-il un ordre du monde ? \\
Y a-t-il un primat de la nature sur la culture ? \\
Y a-t-il un principe du mal ? \\
Y a-t-il un progrès du droit ? \\
Y a-t-il un progrès en art ? \\
Y a-t-il un progrès en philosophie ? \\
Y a-t-il un progrès moral ? \\
Y a-t-il un propre de l'homme ? \\
Y a-t-il un rapport moral à soi-même ? \\
Y a-t-il un rythme de l'histoire ? \\
Y a-t-il un savoir de la justice ? \\
Y a-t-il un savoir du contingent ? \\
Y a-t-il un savoir du corps ? \\
Y a-t-il un savoir du juste ? \\
Y a-t-il un savoir du politique ? \\
Y a-t-il un savoir immédiat ? \\
Y a-t-il un savoir politique ? \\
Y a-t-il un sens du beau ? \\
Y a-t-il un sens moral ? \\
Y a-t-il un souverain bien ? \\
Y a-t-il un temps des choses ? \\
Y a-t-il un temps pour tout ? \\
Y a-t-il un travail de la pensée ? \\
Y a-t-il un usage moral des passions ? \\
Y a-t-il un usage purement instrumental de la raison ? \\
Y aura-t-il toujours des religions ? \\
« Aimer » se dit-il en un seul sens ? \\
« Comment peut-on être persan ? » \\
« Je ne voulais pas cela » : en quoi les sciences humaines permettent-elles de comprendre cette excuse ? \\
« La logique » ou bien « les logiques » ? \\
« L'histoire jugera » : quel sens faut-il accorder à cette expression ? \\
« Pas de liberté pour les ennemis de la liberté » ? \\
« Que va-t-il se passer ? » \\


\subsection{Question en « peut »}
\label{sec-4-2}

\noindent
Aimer peut-il être un devoir ? \\
À quoi la logique peut-elle servir dans les sciences ? \\
À quelles conditions le vivant peut-il être objet de science ? \\
À quelles conditions peut-on dire qu'une action est historique ? \\
À quelles conditions un choix peut-il être rationnel ? \\
À quelles conditions une théorie peut-elle être scientifique ? \\
À quoi peut-on reconnaître une œuvre d'art ? \\
Ce que la morale autorise, l'État peut-il légitimement l'interdire ? \\
Ce que la technique rend possible, peut-on jamais en empêcher la réalisation ? \\
Ce qui ne peut s'acheter est-il dépourvu de valeur ? \\
Ce qui n'est pas démontré peut-il être vrai ? \\
Ce qui n'est pas matériel peut-il être réel ? \\
Ce qu'on ne peut pas vendre \\
Comment autrui peut-il m'aider à rechercher le bonheur ? \\
Comment le passé peut-il demeurer présent ? \\
Comment l'homme peut-il se représenter le temps ? \\
Comment peut-on choisir entre différentes hypothèses ? \\
Comment peut-on définir la politique ? \\
Comment peut-on définir un être vivant ? \\
Comment peut-on être heureux ? \\
Comment peut-on être sceptique ? \\
Comment peut-on se trahir soi-même ? \\
De quelle science humaine la folie peut-elle être l'objet ? \\
De quoi ne peut-on pas répondre ? \\
De quoi peut-il y avoir science ? \\
De quoi peut-on être inconscient ? \\
De quoi peut-on faire l'expérience ? \\
Dieu peut-il tout faire ? \\
En politique, peut-on faire table rase du passé ? \\
En quel sens l'anthropologie peut-elle être historique ? \\
En quel sens peut-on dire que la vérité s'impose ? \\
En quel sens peut-on dire que le mal n'existe pas ? \\
En quel sens peut-on dire que l'homme est un animal politique ? \\
En quel sens peut-on dire qu' « on expérimente avec sa raison » ? \\
En quel sens peut-on parler de la mort de l'art ? \\
En quel sens peut-on parler de la vie sociale comme d'un jeu ? \\
En quel sens peut-on parler de transcendance ? \\
En quel sens peut-on parler d'expérience possible ? \\
En quel sens peut-on parler d'une culture technique ? \\
En quel sens peut-on parler d'une interprétation de la nature ? \\
En quoi la connaissance de la matière peut-elle relever de la métaphysique ? \\
En quoi l'art peut-il intéresser le philosophe ? \\
Est-ce par son objet ou par ses méthodes qu'une science peut se définir ? \\
Faut-il douter de ce qu'on ne peut pas démontrer ? \\
Jusqu'où peut-on dialoguer ? \\
Jusqu'où peut-on soigner ? \\
La beauté peut-elle délivrer une vérité ? \\
La biologie peut-elle se passer de causes finales ? \\
La compétence technique peut-elle fonder l'autorité publique ? \\
La connaissance du vivant peut-elle être désintéressée  ? \\
La connaissance peut-elle être pratique ? \\
La connaissance peut-elle se passer de l'imagination ? \\
La conscience peut-elle être collective ? \\
La conscience peut-elle nous tromper ? \\
La contrainte peut-elle être légitime ? \\
La critique du pouvoir peut-elle conduire à la désobéissance ? \\
La croyance peut-elle être rationnelle ? \\
La croyance peut-elle tenir lieu de savoir ? \\
L'action politique peut-elle se passer de mots ? \\
La culture peut-elle être instituée ? \\
La culture peut-elle être objet de science ? \\
La découverte de la vérité peut-elle être le fait du hasard ? \\
La démocratie peut-elle échapper à la démagogie ? \\
La démocratie peut-elle être représentative ? \\
La démocratie peut-elle se passer de représentation ? \\
La fonction de penser peut-elle se déléguer ? \\
La fraternité peut-elle se passer d'un fondement religieux ? \\
La guerre peut-elle être juste ? \\
La justice peut-elle se fonder sur le compromis ? \\
La justice peut-elle se passer de la force ? \\
La justice peut-elle se passer d'institutions ? \\
La liberté peut-elle être prouvée ? \\
La liberté peut-elle être une illusion ? \\
La liberté peut-elle faire peur ? \\
La liberté peut-elle s'affirmer sans violence ? \\
La liberté peut-elle s'aliéner ? \\
La liberté peut-elle se constater ? \\
La liberté peut-elle se prouver ? \\
La liberté peut-elle se refuser ? \\
La littérature peut-elle suppléer les sciences de l'homme ? \\
La logique peut-elle se passer de la métaphysique ? \\
La loi peut-elle changer les mœurs ? \\
La loi peut-elle être injuste ? \\
La magie peut-elle être efficace ? \\
La majorité peut-elle être tyrannique ? \\
La matière peut-elle être objet de connaissance ? \\
L'ambiguïté des mots peut-elle être heureuse ? \\
L'amélioration des hommes peut-elle être considérée comme un objectif politique ? \\
La métaphysique peut-elle être autre chose qu'une science recherchée ? \\
La métaphysique peut-elle faire appel à l'expérience ? \\
L'amitié peut-elle obliger ? \\
La morale peut-elle être fondée sur la science ? \\
La morale peut-elle être naturelle ? \\
La morale peut-elle être un calcul ? \\
La morale peut-elle être une science ? \\
La morale peut-elle se fonder sur les sentiments ? \\
La morale peut-elle s'enseigner ? \\
La morale peut-elle se passer d'un fondement religieux ? \\
L'amour peut-il être absolu ? \\
L'amour peut-il être raisonnable ? \\
L'amour peut-il être un devoir ? \\
L'analyse du langage ordinaire peut-elle avoir un intérêt philosophique ? \\
La nature peut-elle avoir des droits ? \\
La nature peut-elle constituer une norme ? \\
La nature peut-elle être belle ? \\
La nature peut-elle être un modèle ? \\
La nature peut-elle nous indiquer ce que nous devons faire ? \\
L'animal peut-il être un sujet moral ? \\
La parole peut-elle être une arme ? \\
La passion de la vérité peut-elle être source d'erreur ? \\
La peinture peut-elle être un art du temps ? \\
La pensée formelle peut-elle avoir un contenu ? \\
La pensée peut-elle s'écrire ? \\
La pensée peut-elle se passer de mots ? \\
La perception peut-elle être désintéressée ? \\
La perception peut-elle s'éduquer ? \\
La philosophie peut-elle disparaître ? \\
La philosophie peut-elle être expérimentale ? \\
La philosophie peut-elle être populaire ? \\
La philosophie peut-elle être une science ? \\
La philosophie peut-elle se passer de théologie ? \\
La pitié peut-elle fonder la morale ? \\
La politique peut-elle changer la société \\
La politique peut-elle changer le monde ? \\
La politique peut-elle être indépendante de la morale ? \\
La politique peut-elle être objet de science ? \\
La politique peut-elle être un objet de science ? \\
La politique peut-elle n'être qu'une pratique ? \\
La politique peut-elle se passer de croyances ? \\
La politique peut-elle se passer de croyance ? \\
La politique peut-elle unir les hommes ? \\
La précaution peut-elle être un principe ? \\
La raison d'État peut-elle être justifiée ? \\
La raison peut-elle entrer en conflit avec elle-même ? \\
La raison peut-elle errer ? \\
La raison peut-elle être immédiatement pratique ? \\
La raison peut-elle être pratique ? \\
La raison peut-elle nous commander de croire ? \\
La raison peut-elle se contredire ? \\
La raison peut-elle servir le mal ? \\
La réalité peut-elle être virtuelle ? \\
La recherche de la vérité peut-elle être désintéressée ? \\
La religion peut-elle être civile ? \\
La religion peut-elle être naturelle ? \\
La religion peut-elle faire lien social ? \\
La religion peut-elle n'être qu'une affaire privée ? \\
La religion peut-elle suppléer la raison ? \\
La responsabilité peut-elle être collective ? \\
La révolte peut-elle être un droit ? \\
L'art a-t-il des vertus thérapeutiques ? \\
L'artiste peut-il se passer d'un maître ? \\
L'art peut-il changer le monde \\
L'art peut-il contribuer à éduquer les hommes ? \\
L'art peut-il encore imiter la nature ? \\
L'art peut-il être abstrait ? \\
L'art peut-il être brut ? \\
L'art peut-il être conceptuel ? \\
L'art peut-il être populaire ? \\
L'art peut-il être réaliste \\
L'art peut-il être sans œuvre ? \\
L'art peut-il être utile ? \\
L'art peut-il ne pas être sacré ? \\
L'art peut-il n'être pas conceptuel ? \\
L'art peut-il nous rendre meilleurs ? \\
L'art peut-il prétendre à la vérité ? \\
L'art peut-il quelque chose contre la morale ? \\
L'art peut-il quelque chose pour la morale ? \\
L'art peut-il rendre le mouvement ? \\
L'art peut-il s'affranchir des lois ? \\
L'art peut-il s'enseigner ? \\
L'art peut-il se passer de la beauté ? \\
L'art peut-il se passer de règles ? \\
L'art peut-il se passer d'idéal ? \\
L'art peut-il se passer d'œuvres ? \\
L'art sait-il montrer ce que le langage ne peut pas dire ? \\
La science du vivant peut-elle se passer de l'idée de finalité ? \\
La science peut-elle être une métaphysique ? \\
La science peut-elle guider notre conduite ? \\
La science peut-elle lutter contre les préjugés ? \\
La science peut-elle produire des croyances ? \\
La science peut-elle se passer de fondement ? \\
La science peut-elle se passer de l'idée de finalité ? \\
La science peut-elle se passer de métaphysique ? \\
La science peut-elle se passer d'hypothèses ? \\
La science peut-elle se passer d'institutions ? \\
La science peut-elle tout expliquer ? \\
La servitude peut-elle être volontaire ? \\
La société peut-elle être l'objet d'une science ? \\
La société peut-elle se passer de l'État ? \\
La souffrance peut-elle être un mode de connaissance ? \\
La souveraineté peut-elle être déléguée \\
La souveraineté peut-elle être limitée ? \\
La souveraineté peut-elle se partager ? \\
La sympathie peut-elle tenir lieu de moralité ? \\
La technique peut-elle améliorer l'homme ? \\
La technique peut-elle se déduire de la science ? \\
La technique peut-elle se passer de la science ? \\
La théorie peut-elle nous égarer ? \\
La tolérance peut-elle constituer un problème pour la démocratie ? \\
L'avenir peut-il être objet de connaissance ? \\
La vérité peut-elle être équivoque ? \\
La vérité peut-elle être tolérante ? \\
La vérité peut-elle laisser indifférent ? \\
La vérité peut-elle se définir par le consensus ? \\
La vertu peut-elle être excessive ? \\
La vertu peut-elle être purement morale ? \\
La vertu peut-elle s'enseigner ? \\
La vie peut-elle être éternelle ? \\
La vie peut-elle être objet de science ? \\
La violence peut-elle avoir raison ? \\
La violence peut-elle être gratuite ? \\
La vision peut-elle être le modèle de toute connaissance ? \\
La volonté peut-elle être collective ? \\
La volonté peut-elle être générale ? \\
La volonté peut-elle être indéterminée ? \\
La volonté peut-elle nous manquer ? \\
Le beau peut-il être bizarre ? \\
Le bonheur peut-il être collectif ? \\
Le bonheur peut-il être le but de la politique ? \\
Le bonheur peut-il être un droit ? \\
L'échange peut-il être désintéressé ? \\
Le choix peut-il être éclairé ? \\
Le citoyen peut-il être à la fois libre et soumis à l'État ? \\
Le commerce peut-il être équitable ? \\
Le corps peut-il être objet d'art ? \\
Le cosmopolitisme peut-il devenir réalité ? \\
Le cosmopolitisme peut-il être réaliste ? \\
L'écriture peut-elle porter secours à la pensée ? \\
Le désir peut-il être désintéressé ? \\
Le désir peut-il ne pas avoir d'objet ? \\
Le désir peut-il nous rendre libre ? \\
Le désir peut-il se satisfaire de la réalité ? \\
Le despote peut-il être éclairé ? \\
Le doute peut-il être méthodique ? \\
Le droit peut-il échapper à l'histoire ? \\
Le droit peut-il être flexible ? \\
Le droit peut-il être naturel ? \\
Le droit peut-il se fonder sur la force ? \\
Le droit peut-il se passer de la morale ? \\
L'éducation peut-elle être sentimentale ? \\
L'efficacité thérapeutique de la psychanalyse \\
L'égalité peut-elle être une menace pour la liberté ? \\
Le jugement critique peut-il s'exercer sans culture ? \\
Le langage peut-il être un obstacle à la recherche de la vérité ? \\
Le lien social peut-il être compassionnel ? \\
Le méchant peut-il être heureux ? \\
Le mensonge peut-il être au service de la vérité ? \\
Le mépris peut-il être justifié ? \\
L'émotion esthétique peut-elle se communiquer ? \\
L'enseignement peut-il se passer d'exemples ? \\
Le pardon peut-il être une obligation ? \\
Le passé peut-il être un objet de connaissance ? \\
Le peuple peut-il se tromper ? \\
Le plaisir esthétique peut-il se partager ? \\
Le plaisir peut-il être immoral ? \\
Le plaisir peut-il être partagé ? \\
Le politique peut-il faire abstraction de la morale ? \\
Le pouvoir peut-il être limité ? \\
Le pouvoir peut-il limiter le pouvoir ? \\
Le pouvoir peut-il se déléguer ? \\
Le pouvoir peut-il se passer de sa mise en scène ? \\
Le pouvoir politique peut-il échapper à l'arbitraire ? \\
Le progrès technique peut-il être aliénant ? \\
Le rationalisme peut-il être une passion ? \\
Le réel peut-il échapper à la logique ? \\
Le réel peut-il être contradictoire ? \\
Le roman peut-il être philosophique ? \\
L'erreur peut-elle jouer un rôle dans la connaissance scientifique ? \\
Le sensible peut-il être connu ? \\
L'espoir peut-il être raisonnable ? \\
L'esprit peut-il être divisé ? \\
L'esprit peut-il être malade ? \\
L'esprit peut-il être mesuré ? \\
L'esprit peut-il être objet de science ? \\
Le sujet peut-il s'aliéner par un libre choix ? \\
L'État peut-il créer la liberté ? \\
L'État peut-il être impartial ? \\
L'État peut-il être indifférent à la religion ? \\
L'État peut-il être libéral ? \\
L'État peut-il poursuivre une autre fin que sa propre puissance ? \\
L'État peut-il renoncer à la violence ? \\
Le vrai peut-il rester invérifiable ? \\
L'expérience peut-elle avoir raison des principes ? \\
L'expérience peut-elle contredire la théorie ? \\
L'histoire de l'art peut-elle arriver à son terme ? \\
L'histoire peut-elle être contemporaine ? \\
L'histoire peut-elle être universelle ? \\
L'histoire peut-elle se répéter ? \\
L'historien peut-il être impartial ? \\
L'historien peut-il se passer du concept de causalité ? \\
L'homme injuste peut-il être heureux ? \\
L'homme peut-il changer ? \\
L'homme peut-il être libéré du besoin ? \\
L'homme peut-il se représenter un monde sans l'homme ? \\
L'ignorance peut-elle être une excuse ? \\
L'inconscient peut-il se manifester ? \\
L'indifférence peut-elle être une vertu ? \\
L'inquiétude peut-elle définir l'existence humaine ? \\
L'inquiétude peut-elle devenir l'existence humaine ? \\
L'intérêt peut-il être une valeur morale ? \\
L'obligation morale peut-elle se réduire à une obligation sociale ? \\
L'ordre politique peut-il exclure la violence ? \\
L'ordre social peut-il être juste ? \\
L'utilité peut-elle être le principe de la moralité ? \\
L'utilité peut-elle être un critère pour juger de la valeur de nos actions ? \\
Notre rapport au monde peut-il n'être que technique ? \\
Par le langage, peut-on agir sur la réalité ? \\
Penser peut-il nous rendre heureux ? \\
Peut-être se mettre à la place de l'autre ? \\
Peut-il être moral de tuer ? \\
Peut-il être préférable de ne pas savoir ? \\
Peut-il exister une action désintéressée ? \\
Peut-il y avoir conflit entre nos devoirs ? \\
Peut-il y avoir de bons tyrans ? \\
Peut-il y avoir de la politique sans conflit ? \\
Peut-il y avoir des échanges équitables ? \\
Peut-il y avoir des expériences métaphysiques ? \\
Peut-il y avoir des lois de l'histoire ? \\
Peut-il y avoir des lois injustes ? \\
Peut-il y avoir des modèles en morale ? \\
Peut-il y avoir des vérités partielles ? \\
Peut-il y avoir esprit sans corps ? \\
Peut-il y avoir savoir-faire sans savoir ? \\
Peut-il y avoir science sans intuition du vrai ? \\
Peut-il y avoir un art conceptuel ? \\
Peut-il y avoir un droit à désobéir ? \\
Peut-il y avoir un droit de la guerre ? \\
Peut-il y avoir une histoire universelle ? \\
Peut-il y avoir une philosophie applicable ? \\
Peut-il y avoir une philosophie appliquée ? \\
Peut-il y avoir une philosophie politique sans Dieu ? \\
Peut-il y avoir une science de l'éducation ? \\
Peut-il y avoir une science politique ? \\
Peut-il y avoir une société des nations ? \\
Peut-il y avoir une société sans État ? \\
Peut-il y avoir un État mondial ? \\
Peut-il y avoir une vérité en art ? \\
Peut-il y avoir une vérité en politique ? \\
Peut-il y avoir un intérêt collectif ? \\
Peut-il y avoir un langage universel ? \\
Peut-on admettre un droit à la révolte ? \\
Peut-on agir machinalement ? \\
Peut-on aimer ce qu'on ne connaît pas ? \\
Peut-on aimer l'autre tel qu'il est ? \\
Peut-on aimer la vie plus que tout ? \\
Peut-on aimer les animaux ? \\
Peut-on aimer l'humanité ? \\
Peut-on aimer sans perdre sa liberté ? \\
Peut-on aimer son prochain comme soi-même ? \\
Peut-on aimer son travail ? \\
Peut-on aimer une œuvre d'art sans la comprendre ? \\
Peut-on apprendre à être heureux ? \\
Peut-on apprendre à être juste ? \\
Peut-on apprendre à être libre ? \\
Peut-on apprendre à mourir ? \\
Peut-on apprendre à penser ? \\
Peut-on apprendre à vivre ? \\
Peut-on argumenter en morale ? \\
Peut-on assimiler le vivant à une machine ? \\
Peut-on atteindre une certitude ? \\
Peut-on attribuer à chacun son dû ? \\
Peut-on avoir conscience de soi sans avoir conscience d'autrui ? \\
Peut-on avoir de bonnes raisons de ne pas dire la vérité ? \\
Peut-on avoir le droit de se révolter ? \\
Peut-on avoir peur de soi-même ? \\
Peut-on avoir raison contre les faits ? \\
Peut-on avoir raison contre tous ? \\
Peut-on avoir raison contre tout le monde ? \\
Peut-on avoir raisons contre les faits ? \\
Peut-on avoir raison tout.e seul.e ? \\
Peut-on avoir raison tout seul ? \\
Peut-on cesser de croire ? \\
Peut-on cesser de désirer ? \\
Peut-on changer de culture ? \\
Peut-on changer de logique ? \\
Peut-on changer le cours de l'histoire ? \\
Peut-on changer le monde ? \\
Peut-on changer le passé ? \\
Peut-on changer ses désirs ? \\
Peut-on choisir le mal ? \\
Peut-on choisir sa vie ? \\
Peut-on choisir ses désirs ? \\
Peut-on classer les arts ? \\
Peut-on commander à la nature ? \\
Peut-on communiquer ses perceptions à autrui ? \\
Peut-on communiquer son expérience ? \\
Peut-on comparer deux philosophies ? \\
Peut-on comparer les cultures ? \\
Peut-on comparer l'organisme à une machine ? \\
Peut-on comprendre ce qui est illogique ? \\
Peut-on comprendre le présent ? \\
Peut-on comprendre un acte que l'on désapprouve ? \\
Peut-on concevoir une humanité sans art ? \\
Peut-on concevoir une morale sans sanction ni obligation ? \\
Peut-on concevoir une science qui ne soit pas démonstrative ? \\
Peut-on concevoir une science sans expérience ? \\
Peut-on concevoir une société juste sans que les hommes ne le soient ? \\
Peut-on concevoir une société qui n'aurait plus besoin du droit ? \\
Peut-on concevoir une société sans État ? \\
Peut-on concevoir un État mondial ? \\
Peut-on concilier bonheur et liberté ? \\
Peut-on conclure de l'être au devoir-être ? \\
Peut-on connaître autrui ? \\
Peut-on connaître les choses telles qu'elles sont ? \\
Peut-on connaître le singulier ? \\
Peut-on connaître l'esprit ? \\
Peut-on connaître le vivant sans le dénaturer ? \\
Peut-on connaître le vivant sans recourir à la notion de finalité ? \\
Peut-on connaître l'individuel ? \\
Peut-on connaître par intuition ? \\
Peut-on considérer l'art comme un langage ? \\
Peut-on contester les droits de l'homme ? \\
Peut-on contredire l'expérience ? \\
Peut-on convaincre quelqu'un de la beauté d'une œuvre d'art ? \\
Peut-on craindre la liberté ? \\
Peut-on créer un homme nouveau ? \\
Peut-on critiquer la démocratie ? \\
Peut-on critiquer la religion ? \\
Peut-on croire ce qu'on veut ? \\
Peut-on croire en rien ? \\
Peut-on croire sans être crédule ? \\
Peut-on croire sans savoir pourquoi ? \\
Peut-on décider de croire ? \\
Peut-on décider de tout ? \\
Peut-on décider d'être heureux ? \\
Peut-on définir la morale comme l'art d'être heureux ? \\
Peut-on définir la vérité ? \\
Peut-on définir la vie ? \\
Peut-on définir le bien ? \\
Peut-on définir le bonheur ? \\
Peut-on délimiter le réel ? \\
Peut-on délimiter l'humain ? \\
Peut-on démontrer qu'on ne rêve pas ? \\
Peut-on dépasser la subjectivité ? \\
Peut-on désirer ce qui est ? \\
Peut-on désirer ce qu'on ne veut pas ? \\
Peut-on désirer ce qu'on possède ? \\
Peut-on désirer l'absolu ? \\
Peut-on désirer l'impossible ? \\
Peut-on désirer sans souffrir ? \\
Peut-on désobéir à l'État ? \\
Peut-on désobéir aux lois ? \\
Peut-on désobéir par devoir ? \\
Peut-on dialoguer avec un ordinateur ? \\
Peut-on dire ce que l'on pense ? \\
Peut-on dire ce qui n'est pas ? \\
Peut-on dire de la connaissance scientifique qu'elle procède par approximation ? \\
Peut-on dire de l'art qu'il donne un monde en partage ? \\
Peut-on dire d'une image qu'elle parle ? \\
Peut-on dire d'une œuvre d'art qu'elle est ratée ? \\
Peut-on dire d'une théorie scientifique qu'elle n'est jamais plus vraie qu'une autre mais seulement plus commode ? \\
Peut-on dire d'un homme qu'il est supérieur à un autre homme ? \\
Peut-on dire la vérité ? \\
Peut-on dire le singulier ? \\
Peut-on dire que la science ne nous fait pas connaître les choses mais les rapports entre les choses ? \\
Peut-on dire que les hommes font l'histoire ? \\
Peut-on dire que les machines travaillent pour nous ? \\
Peut-on dire que les mots pensent pour nous ? \\
Peut-on dire que l'humanité progresse ? \\
Peut-on dire que rien n'échappe à la technique ? \\
Peut-on dire qu'est vrai ce qui correspond aux faits ? \\
Peut-on dire que toutes les croyances se valent ? \\
Peut-on dire que tout est relatif ? \\
Peut-on dire qu'une théorie physique en contredit une autre ? \\
Peut-on dire toute la vérité ? \\
Peut-on discuter des goûts et des couleurs ? \\
Peut-on disposer de son corps ? \\
Peut-on distinguer différents types de causes ? \\
Peut-on distinguer entre de bons et de mauvais désirs ? \\
Peut-on distinguer entre les bons et les mauvais désirs ? \\
Peut-on distinguer le réel de l'imaginaire ? \\
Peut-on distinguer les faits de leurs interprétations ? \\
Peut-on donner un sens à son existence ? \\
Peut-on douter de sa propre existence ? \\
Peut-on douter de soi ? \\
Peut-on douter de toute vérité ? \\
Peut-on douter de tout ? \\
Peut-on échanger des idées ? \\
Peut-on échapper à ses désirs ? \\
Peut-on échapper à son temps ? \\
Peut-on échapper au cours de l'histoire ? \\
Peut-on échapper au temps ? \\
Peut-on éclairer la liberté ? \\
Peut-on écrire comme on parle ? \\
Peut-on éduquer la conscience ? \\
Peut-on éduquer le goût ? \\
Peut-on éduquer quelqu'un à être libre ? \\
Peut-on en appeler à la conscience contre la loi ? \\
Peut-on en appeler à la conscience contre l'État ? \\
Peut-on en savoir trop ? \\
Peut-on entreprendre d'éliminer la métaphysique ? \\
Peut-on établir une hiérarchie des arts ? \\
Peut-on être amoral ? \\
Peut-on être apolitique ? \\
Peut-on être citoyen du monde ? \\
Peut-on être complètement athée ? \\
Peut-on être dans le présent ? \\
Peut-on être en avance sur son temps ? \\
Peut-on être en conflit avec soi-même ? \\
Peut-on être esclave de soi-même ? \\
Peut-on être heureux dans la solitude ? \\
Peut-on être heureux sans être sage ? \\
Peut-on être heureux sans philosophie ? \\
Peut-on être heureux sans s'en rendre compte ? \\
Peut-on être heureux tout seul ? \\
Peut-on être homme sans être citoyen ? \\
Peut-on être hors de soi ? \\
Peut-on être ignorant ? \\
Peut-on être impartial ? \\
Peut-on être indifférent à l'histoire ? \\
Peut-on être indifférent à son bonheur ? \\
Peut-on être injuste et heureux ? \\
Peut-on être insensible à l'art ? \\
Peut-on être juste dans une situation injuste ? \\
Peut-on être juste dans une société injuste ? \\
Peut-on être juste sans être impartial ? \\
Peut-on être maître de soi ? \\
Peut-on être méchant volontairement ? \\
Peut-on être obligé d'aimer ? \\
Peut-on être plus ou moins libre ? \\
Peut-on être sans opinion ? \\
Peut-on être sceptique de bonne foi ? \\
Peut-on être sceptique ? \\
Peut-on être seul avec soi-même ? \\
Peut-on être seul ? \\
Peut-on être soi-même en société ? \\
Peut-on être sûr d'avoir raison ? \\
Peut-on être sûr de bien agir ? \\
Peut-on être sûr de ne pas se tromper ? \\
Peut-on être trop sage ? \\
Peut-on être trop sensible ? \\
Peut-on étudier le passé de façon objective ? \\
Peut-on exercer son esprit ? \\
Peut-on expérimenter sur le vivant ? \\
Peut-on expliquer le mal ? \\
Peut-on expliquer le monde par la matière ? \\
Peut-on expliquer le vivant ? \\
Peut-on expliquer une œuvre d'art ? \\
Peut-on faire de la politique sans supposer les hommes méchants ? \\
Peut-on faire de l'art avec tout ? \\
Peut-on faire de l'esprit un objet de science ? \\
Peut-on faire du dialogue un modèle de relation morale ? \\
Peut-on faire la paix ? \\
Peut-on faire la philosophie de l'histoire ? \\
Peut-on faire le bien d'autrui malgré lui ? \\
Peut-on faire l'économie de la notion de forme ? \\
Peut-on faire le mal en vue du bien ? \\
Peut-on faire le mal innocemment ? \\
Peut-on faire l'expérience de la nécessité ? \\
Peut-on faire l'inventaire du monde ? \\
Peut-on faire table rase du passé ? \\
Peut-on fixer des limites à la science ? \\
Peut-on fonder la morale sur la pitié ? \\
Peut-on fonder la morale ? \\
Peut-on fonder le droit sur la morale ? \\
Peut-on fonder les droits de l'homme ? \\
Peut-on fonder les mathématiques ? \\
Peut-on fonder un droit de désobéir ? \\
Peut-on fonder une éthique sur la biologie ? \\
Peut-on fonder une morale sur la nature ? \\
Peut-on fonder une morale sur le plaisir ? \\
Peut-on forcer quelqu'un à être libre ? \\
Peut-on forcer un homme à être libre ? \\
Peut-on fuir hors du monde ? \\
Peut-on fuir la société ? \\
Peut-on gâcher son talent ? \\
Peut-on gouverner sans lois ? \\
Peut-on haïr la raison ? \\
Peut-on haïr la vie ? \\
Peut-on haïr les images ? \\
Peut-on hiérarchiser les arts ? \\
Peut-on hiérarchiser les œuvres d'art ? \\
Peut-on identifier le désir au besoin ? \\
Peut-on ignorer sa propre liberté ? \\
Peut-on ignorer volontairement la vérité ? \\
Peut-on imaginer l'avenir ? \\
Peut-on imposer la liberté ? \\
Peut-on innover en politique ? \\
Peut-on interpréter la nature ? \\
Peut-on inventer en morale ? \\
Peut-on jamais aimer son prochain ? \\
Peut-on juger des œuvres d'art sans recourir à l'idée de beauté ? \\
Peut-on justifier la discrimination ? \\
Peut-on justifier la guerre ? \\
Peut-on justifier la raison d'État ? \\
Peut-on justifier le mal ? \\
Peut-on justifier le mensonge ? \\
Peut-on justifier ses choix ? \\
Peut-on légitimer la violence ? \\
Peut-on limiter l'expression de la volonté du peuple ? \\
Peut-on lutter contre le destin ? \\
Peut-on lutter contre soi-même ? \\
Peut-on maîtriser la nature ? \\
Peut-on maîtriser la technique ? \\
Peut-on maîtriser le risque ? \\
Peut-on maîtriser le temps ? \\
Peut-on maîtriser l'évolution de la technique ? \\
Peut-on maîtriser l'inconscient ? \\
Peut-on maîtriser ses désirs ? \\
Peut-on manipuler les esprits ? \\
Peut-on manquer de culture ? \\
Peut-on manquer de volonté ? \\
Peut-on mentir par humanité ? \\
Peut-on mesurer les phénomènes sociaux ? \\
Peut-on mesurer le temps ? \\
Peut-on montrer en cachant ? \\
Peut-on moraliser la guerre ? \\
Peut-on ne croire en rien ? \\
Peut-on ne pas connaître son bonheur ? \\
Peut-on ne pas croire au progrès ? \\
Peut-on ne pas croire ? \\
Peut-on ne pas être de son temps ? \\
Peut-on ne pas être égoïste ? \\
Peut-on ne pas être matérialiste ? \\
Peut-on ne pas être soi-même ? \\
Peut-on ne pas interpréter ? \\
Peut-on ne pas savoir ce que l'on dit ? \\
Peut-on ne pas savoir ce que l'on fait ? \\
Peut-on ne pas savoir ce que l'on veut ? \\
Peut-on ne pas savoir ce qu'on veut ? \\
Peut-on ne pas vouloir être heureux ? \\
Peut-on ne penser à rien ? \\
Peut-on ne rien devoir à personne ? \\
Peut-on ne rien vouloir ? \\
Peut-on ne vivre qu'au présent ? \\
Peut-on nier la réalité ? \\
Peut-on nier le réel ? \\
Peut-on nier l'évidence ? \\
Peut-on nier l'existence de la matière ? \\
Peut-on objectiver le psychisme ? \\
Peut-on opposer justice et liberté ? \\
Peut-on opposer le loisir au travail ? \\
Peut-on opposer morale et technique ? \\
Peut-on opposer nature et culture ? \\
Peut-on ôter à l'homme sa liberté ? \\
Peut-on oublier de vivre ? \\
Peut-on parler d'art primitif ? \\
Peut-on parler de ce qui n'existe pas ? \\
Peut-on parler de corruption des mœurs ? \\
Peut-on parler de dialogue des cultures ? \\
Peut-on parler de droits des animaux ? \\
Peut-on parler de mondes imaginaires ? \\
Peut-on parler de nourriture spirituelle ? \\
Peut-on parler de problèmes techniques ? \\
Peut-on parler des miracles de la technique ? \\
Peut-on parler des œuvres d'art ? \\
Peut-on parler de travail intellectuel ? \\
Peut-on parler de vérités métaphysiques ? \\
Peut-on parler de vérité subjective ? \\
Peut-on parler de vertu politique ? \\
Peut-on parler de violence d'État ? \\
Peut-on parler de « travail intellectuel » ? \\
Peut-on parler d'un droit de la guerre ? \\
Peut-on parler d'un droit de résistance ? \\
Peut-on parler d'une expérience religieuse ? \\
Peut-on parler d'une morale collective ? \\
Peut-on parler d'une religion de l'humanité ? \\
Peut-on parler d'une santé de l'âme ? \\
Peut-on parler d'une science de l'art ? \\
Peut-on parler d'un progrès dans l'histoire ? \\
Peut-on parler d'un progrès de la liberté ? \\
Peut-on parler d'un règne de la technique ? \\
Peut-on parler d'un savoir poétique ? \\
Peut-on parler d'un travail intellectuel ? \\
Peut-on parler pour en rien dire ? \\
Peut-on parler pour ne rien dire ? \\
Peut-on penser ce qu'on ne peut dire ? \\
Peut-on penser contre l'expérience ? \\
Peut-on penser illogiquement ? \\
Peut-on penser la douleur ? \\
Peut-on penser la matière ? \\
Peut-on penser la mort ? \\
Peut-on penser la nouveauté ? \\
Peut-on penser l'art sans référence au beau ? \\
Peut-on penser la vie sans penser la mort ? \\
Peut-on penser la vie ? \\
Peut-on penser le changement ? \\
Peut-on penser le monde sans la technique ? \\
Peut-on penser le temps sans l'espace ? \\
Peut-on penser l'extériorité ? \\
Peut-on penser l'impossible ? \\
Peut-on penser l'infini ? \\
Peut-on penser l'irrationnel ? \\
Peut-on penser l'œuvre d'art sans référence à l'idée de beauté ? \\
Peut-on penser sans concepts ? \\
Peut-on penser sans concept ? \\
Peut-on penser sans images ? \\
Peut-on penser sans image ? \\
Peut-on penser sans les mots ? \\
Peut-on penser sans les signes ? \\
Peut-on penser sans méthode ? \\
Peut-on penser sans ordre ? \\
Peut-on penser sans préjugés ? \\
Peut-on penser sans préjugé ? \\
Peut-on penser sans règles ? \\
Peut-on penser sans savoir que l'on pense ? \\
Peut-on penser sans signes ? \\
Peut-on penser sans son corps ? \\
Peut-on penser un art sans œuvres ? \\
Peut-on penser un droit international ? \\
Peut-on penser une conscience sans objet ? \\
Peut-on penser une métaphysique sans Dieu ? \\
Peut-on penser une société sans État ? \\
Peut-on penser un État sans violence ? \\
Peut-on penser une volonté diabolique ? \\
Peut-on percevoir sans juger ? \\
Peut-on percevoir sans s'en apercevoir ? \\
Peut-on perdre la raison ? \\
Peut-on perdre sa dignité ? \\
Peut-on perdre sa liberté ? \\
Peut-on perdre son identité ? \\
Peut-on perdre son temps ? \\
Peut-on préconiser, dans les sciences humaines et sociales, l'imitation des sciences de la nature ? \\
Peut-on prédire les événements ? \\
Peut-on prédire l'histoire ? \\
Peut-on préférer le bonheur à la vérité ? \\
Peut-on préférer l'injustice au désordre ? \\
Peut-on préférer l'ordre à la justice ? \\
Peut-on prévoir l'avenir ? \\
Peut-on prévoir le futur ? \\
Peut-on promettre le bonheur ? \\
Peut-on protéger les libertés sans les réduire ? \\
Peut-on prouver l'existence de Dieu ? \\
Peut-on prouver l'existence de l'inconscient ? \\
Peut-on prouver l'existence du monde ? \\
Peut-on prouver l'existence ? \\
Peut-on prouver une existence ? \\
Peut-on raconter sa vie ? \\
Peut-on raisonner sans règles ? \\
Peut-on ralentir la course du temps ? \\
Peut-on recommencer sa vie ? \\
Peut-on reconnaître un sens à l'histoire sans lui assigner une fin ? \\
Peut-on réduire la pensée à une espèce de comportement ? \\
Peut-on réduire le raisonnement au calcul ? \\
Peut-on réduire l'esprit à la matière ? \\
Peut-on réduire une métaphysique à une conception du monde ? \\
Peut-on réduire un homme à la somme de ses actes ? \\
Peut-on refuser de voir la vérité ? \\
Peut-on refuser la loi ? \\
Peut-on refuser la violence ? \\
Peut-on refuser le vrai ? \\
Peut-on régner innocemment ? \\
Peut-on rendre raison de tout ? \\
Peut-on rendre raison du réel ? \\
Peut-on renoncer à comprendre ? \\
Peut-on renoncer à ses droits ? \\
Peut-on renoncer à soi ? \\
Peut-on renoncer au bonheur ? \\
Peut-on réparer le vivant ? \\
Peut-on répondre d'autrui ? \\
Peut-on représenter le peuple ? \\
Peut-on représenter l'espace ? \\
Peut-on reprocher à la morale d'être abstraite ? \\
Peut-on reprocher au langage d'être équivoque ? \\
Peut-on reprocher au langage d'être parfait ? \\
Peut-on résister au vrai ? \\
Peut-on rester dans le doute ? \\
Peut-on rester insensible à la beauté ? \\
Peut-on rester sceptique ? \\
Peut-on restreindre la logique à la pensée formelle ? \\
Peut-on réunir des arts différents dans une même œuvre ? \\
Peut-on revendiquer la paix comme un droit ? \\
Peut-on revenir sur ses erreurs ? \\
Peut-on rire de tout ? \\
Peut-on rompre avec la société ? \\
Peut-on rompre avec le passé ? \\
Peut-on s'abstenir de penser politiquement ? \\
Peut-on s'accorder sur des vérités morales ? \\
Peut-on s'affranchir des lois ? \\
Peut-on s'attendre à tout ? \\
Peut-on savoir ce qui est bien ? \\
Peut-on savoir quelque chose de l'avenir ? \\
Peut-on savoir sans croire ? \\
Peut-on se choisir un destin ? \\
Peut-on se connaître soi-même ? \\
Peut-on se désintéresser de la politique ? \\
Peut-on se désintéresser de son bonheur ? \\
Peut-on se duper soi-même ? \\
Peut-on se faire une idée de tout ? \\
Peut-on se fier à l'expérience vécue ? \\
Peut-on se fier à l'intuition ? \\
Peut-on se fier à son intuition ? \\
Peut-on se gouverner soi-même ? \\
Peut-on se méfier de soi-même ? \\
Peut-on se mentir à soi-même \\
Peut-on se mentir à soi-même ? \\
Peut-on se mettre à la place d'autrui ? \\
Peut-on se mettre à la place de l'autre ? \\
Peut-on s'en tenir au présent ? \\
Peut-on séparer l'homme et l'œuvre ? \\
Peut-on séparer politique et économie ? \\
Peut-on se passer de chef ? \\
Peut-on se passer de croire ? \\
Peut-on se passer de croyances ? \\
Peut-on se passer de croyance ? \\
Peut-on se passer de Dieu ? \\
Peut-on se passer de frontières ? \\
Peut-on se passer de la religion ? \\
Peut-on se passer de l'État ? \\
Peut-on se passer de méthode ? \\
Peut-on se passer de mythes ? \\
Peut-on se passer de principes ? \\
Peut-on se passer de religion ? \\
Peut-on se passer de représentants ? \\
Peut-on se passer de spiritualité ? \\
Peut-on se passer des relations ? \\
Peut-on se passer d'État ? \\
Peut-on se passer de techniques de raisonnement ? \\
Peut-on se passer de technique ? \\
Peut-on se passer de toute religion ? \\
Peut-on se passer d'idéal ? \\
Peut-on se passer d'un maître ? \\
Peut-on se prescrire une loi ? \\
Peut-on se promettre quelque chose à soi-même ? \\
Peut-on se punir soi-même ? \\
Peut-on se régler sur des exemples en politique ? \\
Peut-on se retirer du monde ? \\
Peut-on se tromper en se croyant heureux ? \\
Peut-on se vouloir parfait ? \\
Peut-on sortir de la subjectivité ? \\
Peut-on sortir de sa conscience ? \\
Peut-on souhaiter le gouvernement des meilleurs ? \\
Peut-on suivre une règle ? \\
Peut-on suspendre le temps ? \\
Peut-on suspendre son jugement ? \\
Peut-on sympathiser avec l'ennemi ? \\
Peut-on tirer des leçons de l'histoire ? \\
Peut-on toujours faire ce qu'on doit ? \\
Peut-on toujours savoir entièrement ce que l'on dit ? \\
Peut-on tout analyser ? \\
Peut-on tout attendre de l'État ? \\
Peut-on tout définir ? \\
Peut-on tout démontrer ? \\
Peut-on tout désirer ? \\
Peut-on tout dire ? \\
Peut-on tout donner ? \\
Peut-on tout échanger ? \\
Peut-on tout enseigner ? \\
Peut-on tout expliquer ? \\
Peut-on tout exprimer ? \\
Peut-on tout imaginer ? \\
Peut-on tout imiter ? \\
Peut-on tout interpréter ? \\
Peut-on tout mathématiser ? \\
Peut-on tout mesurer ? \\
Peut-on tout ordonner ? \\
Peut-on tout pardonner ? \\
Peut-on tout partager ? \\
Peut-on tout prévoir ? \\
Peut-on tout prouver ? \\
Peut-on tout soumettre à la discussion ? \\
Peut-on tout tolérer ? \\
Peut-on traiter autrui comme un moyen ? \\
Peut-on traiter un être vivant comme une machine ? \\
Peut-on transformer le réel ? \\
Peut-on transiger avec les principes ? \\
Peut-on trouver du plaisir à l'ennui ? \\
Peut-on vivre avec les autres ? \\
Peut-on vivre dans le doute ? \\
Peut-on vivre en marge de la société ? \\
Peut-on vivre en sceptique ? \\
Peut-on vivre hors du temps ? \\
Peut-on vivre pour la vérité ? \\
Peut-on vivre sans aimer ? \\
Peut-on vivre sans art ? \\
Peut-on vivre sans aucune certitude ? \\
Peut-on vivre sans croyances ? \\
Peut-on vivre sans désir ? \\
Peut-on vivre sans échange ? \\
Peut-on vivre sans illusions ? \\
Peut-on vivre sans l'art ? \\
Peut-on vivre sans le plaisir de vivre ? \\
Peut-on vivre sans lois ? \\
Peut-on vivre sans passion ? \\
Peut-on vivre sans peur ? \\
Peut-on vivre sans principes ? \\
Peut-on vivre sans réfléchir ? \\
Peut-on vivre sans ressentiment ? \\
Peut-on vivre sans rien espérer ? \\
Peut-on vivre sans sacré ? \\
Peut-on voir sans croire ? \\
Peut-on vouloir ce qu'on ne désire pas ? \\
Peut-on vouloir le bonheur d'autrui ? \\
Peut-on vouloir le mal pour le mal ? \\
Peut-on vouloir le mal ? \\
Peut-on vouloir l'impossible ? \\
Peut-on vouloir sans désirer ? \\
Pourquoi la réalité peut-elle dépasser la fiction ? \\
Quand peut-on se passer de théories ? \\
Quel être peut être un sujet de droits ? \\
Quelle réalité peut-on accorder au temps ? \\
Quelle valeur peut-on accorder à l'expérience ? \\
Quel usage peut-on faire des fictions ? \\
Que ne peut-on pas expliquer ? \\
Que peut expliquer l'histoire ? \\
Que peut la force ? \\
Que peut la musique ? \\
Que peut la philosophie ? \\
Que peut la politique ? \\
Que peut la raison ? \\
Que peut l'art ? \\
Que peut la science ? \\
Que peut la théorie ? \\
Que peut la volonté ? \\
Que peut le corps ? \\
Que peut le politique ? \\
Que peut l'esprit sur la matière ? \\
Que peut l'esprit ? \\
Que peut l'État ? \\
Que peut-on attendre de l'État ? \\
Que peut-on attendre du droit international ? \\
Que peut-on calculer ? \\
Que peut-on comprendre immédiatement ? \\
Que peut-on comprendre qu'on ne puisse expliquer ? \\
Que peut-on contre un préjugé ? \\
Que peut-on cultiver ? \\
Que peut-on démontrer ? \\
Que peut-on dire de l'être ? \\
Que peut-on échanger ? \\
Que peut-on interdire ? \\
Que peut-on partager ? \\
Que peut-on savoir de l'inconscient ? \\
Que peut-on savoir de soi ? \\
Que peut-on savoir du réel ? \\
Que peut-on savoir par expérience ? \\
Que peut-on sur autrui ? \\
Que peut-on voir ? \\
Que peut un corps ? \\
Qu'est-ce qui peut se transformer ? \\
Qu'est-ce qu'on ne peut comprendre ? \\
Qui peut avoir des droits ? \\
Qui peut me dire « tu ne dois pas » ? \\
Qui peut parler ? \\
Résister peut-il être un droit ? \\
Tout peut-il être objet d'échange ? \\
Tout peut-il être objet de jugement esthétique ? \\
Tout peut-il être objet de science ? \\
Tout peut-il n'être qu'apparence ? \\
Tout peut-il s'acheter ? \\
Tout peut-il se démontrer ? \\
Tout peut-il se vendre ? \\
Tout peut-il s'expliquer ? \\
Tout savoir peut-il se transmettre ? \\
Un acte peut-il être inhumain ? \\
Un art peut-il être populaire ? \\
Un bien peut-il être commun ? \\
Un bien peut-il sortir d'un mal ? \\
Un choix peut-il être rationnel ? \\
Un contrat peut-il être injuste ? \\
Un contrat peut-il être social ? \\
Un désir peut-il être coupable ? \\
Un désir peut-il être inconscient ? \\
Une action peut-elle être désintéressée ? \\
Une action peut-elle être machinale ? \\
Une cause peut-elle être libre ? \\
Une connaissance peut-elle ne pas être relative ? \\
Une croyance peut-elle être libre ? \\
Une croyance peut-elle être rationnelle ? \\
Une culture peut-elle être porteuse de valeurs universelles ? \\
Une décision politique peut-elle être juste ? \\
Une destruction peut-elle être créatrice ? \\
Une expérience peut-elle être fictive ? \\
Une explication peut-elle être réductrice ? \\
Une fiction peut-elle être vraie ? \\
Une guerre peut-elle être juste ? \\
Une idée peut-elle être fausse ? \\
Une idée peut-elle être générale ? \\
Une imitation peut-elle être parfaite ? \\
Une intention peut-elle être coupable ? \\
Une interprétation peut-elle échapper à l'arbitraire ? \\
Une interprétation peut-elle être définitive ? \\
Une interprétation peut-elle être objective ? \\
Une interprétation peut-elle prétendre à la vérité ? \\
Une ligne de conduite peut-elle tenir lieu de morale ? \\
Une loi peut-elle être injuste ? \\
Une machine peut-elle avoir une mémoire ? \\
Une machine peut-elle penser ? \\
Une métaphysique peut-elle être sceptique ? \\
Une morale peut-elle être dépassée ? \\
Une morale peut-elle être provisoire ? \\
Une morale peut-elle prétendre à l'universalité ? \\
Une œuvre d'art peut-elle être immorale ? \\
Une œuvre d'art peut-elle être laide ? \\
Une perception peut-elle être illusoire ? \\
Une philosophie peut-elle être réactionnaire ? \\
Une politique peut-elle se réclamer de la vie ? \\
Une psychologie peut-elle être matérialiste ? \\
Une religion peut-elle être fausse ? \\
Une religion peut-elle être rationnelle ? \\
Une religion peut-elle se passer de pratiques ? \\
Une sensation peut-elle être fausse ? \\
Un État peut-il être trop étendu ? \\
Une théorie peut-elle être vérifiée ? \\
Une théorie scientifique peut-elle devenir fausse ? \\
Une théorie scientifique peut-elle être ramenée à des propositions empiriques élémentaires ? \\
Une théorie scientifique peut-elle être vraie ? \\
Un être vivant peut-il être comparé à une œuvre d'art ? \\
Une vérité peut-elle être indicible ? \\
Une vérité peut-elle être provisoire ? \\
Une volonté peut-elle être générale ? \\
Un jeu peut-il être sérieux ? \\
Un mensonge peut-il avoir une valeur morale ? \\
Un objet technique peut-il être beau ? \\
Un problème moral peut-il recevoir une solution certaine ? \\
Un problème scientifique peut-il être insoluble ? \\
Un savoir peut-il être inconscient ? \\
Un sceptique peut-il être logicien ? \\
Un seul peut-il avoir raison contre tous ? \\
Un tableau peut-il être une dénonciation ? \\
User de violence peut-il être moral ? \\
Vouloir ce que l'on peut \\
Vouloir la paix sociale peut-il aller jusqu'à accepter l'injustice ? \\
« Comment peut-on être persan ? » \\
« Dans un bois aussi courbe que celui dont l'homme est fait on ne peut rien tailler de tout à fait droit » \\


\subsection{Mot unique}
\label{sec-4-3}
\noindent
Agir \\
Analyser \\
Apparaître \\
Après-coup \\
Argumenter \\
Au-delà \\
Autrui \\
Avoir \\
Calculer \\
Cartographier \\
Changer \\
Choisir \\
Classer \\
Collectionner \\
Commander \\
Commémorer \\
Commencer \\
Communiquer \\
Compatir \\
Comprendre \\
Conclure \\
Conquérir \\
Contempler \\
Correspondre \\
Créer \\
Critiquer \\
Déchiffrer \\
Décider \\
Découvrir \\
Décrire \\
Définir \\
Déjouer \\
Dématérialiser \\
Démériter \\
Dénaturer \\
Déraisonner \\
Désacraliser \\
Désirer \\
Désobéir \\
Dialoguer \\
Distinguer \\
Donner \\
Douter \\
Durer \\
Échanger \\
Éclairer \\
Écouter \\
Écrire \\
Enquêter \\
Enseigner \\
Entendre \\
Énumérer \\
Essayer \\
Estimer \\
Étudier \\
Exister \\
Expérimenter \\
Expliquer \\
Fonder \\
Gagner \\
Gouverner \\
Guérir \\
Habiter \\
Haïr \\
Hériter \\
Hésiter \\
Ignorer \\
Imaginer \\
Imiter \\
Interpréter \\
Interroger \\
Je \\
Jouer \\
Juger \\
Justifier \\
Manger \\
Méditer \\
Mentir \\
Mesurer \\
Mourir \\
Naître \\
Naviguer \\
Nommer \\
Obéir \\
Objectiver \\
Observer \\
Paraître \\
Parfaire \\
Parier \\
Partager \\
Pâtir \\
Peindre \\
Percevoir \\
Persuader \\
Plaider \\
Pourquoi ? \\
Prévoir \\
Promettre \\
Protester \\
Prouver \\
Prouvez-le ! \\
Publier \\
Punir \\
Qu'aime-t-on ? \\
Raisonner \\
Recevoir \\
Réfuter \\
Regarder \\
Répondre \\
Représenter \\
Réprouver \\
Résister \\
Rêver \\
Rêvons-nous ? \\
Rien \\
Rire \\
S'aliéner \\
S'amuser \\
S'engager \\
S'ennuyer \\
Sentir \\
Servir \\
Seul \\
S'exercer \\
S'exprimer \\
S'indigner \\
Socrate \\
Soi \\
Soigner \\
S'orienter \\
Subir \\
Survivre \\
Témoigner \\
Toucher \\
Traduire \\
Trahir \\
Transmettre \\
Tricher \\
Vieillir \\
Vivre \\
Voir \\
Voyager \\
« Ceci » \\
« Pourquoi » \\


\subsection{Le X}
\label{sec-4-4}
\noindent
La banalité \\
L'abandon \\
La barbarie \\
La bassesse \\
La béatitude \\
La beauté \\
La bestialité \\
La bête \\
La bêtise \\
La bibliothèque \\
La bienfaisance \\
La bienséance \\
La bienveillance \\
La biographie \\
L'abondance \\
La bonté \\
L'absence \\
L'absolu \\
L'abstraction \\
L'absurde \\
L'académisme \\
La calomnie \\
La casuistique \\
La catharsis \\
La causalité \\
La cause \\
L'accident \\
L'accidentel \\
L'accomplissement \\
L'accord \\
La censure \\
La certitude \\
La chair \\
La chance \\
La charité \\
La chose \\
La chronologie \\
La chute \\
La circonspection \\
La citation \\
La cité \\
La citoyenneté \\
La civilisation \\
La civilité \\
La clarté \\
La classification \\
La clémence \\
La cohérence \\
La colère \\
La collection \\
La comédie \\
La communauté \\
La communication \\
La comparaison \\
La compassion \\
La compétence \\
La composition \\
La compréhension \\
La concorde \\
La concurrence \\
La condition \\
La confiance \\
La confusion \\
La conquête \\
La conscience \\
La conséquence \\
La conservation \\
La consolation \\
La constance \\
La constitution \\
La contemplation \\
La contestation \\
La contingence \\
La continuité \\
La contradiction \\
La contrainte \\
La convalescence \\
La conversation \\
La conversion \\
La conviction \\
La coopération \\
La copie \\
La corruption \\
La cosmogonie \\
La couleur \\
La courtoisie \\
La coutume \\
La création \\
La créativité \\
La crédibilité \\
La crédulité \\
La criminalité \\
La crise \\
La critique \\
La croissance \\
La croyance \\
La cruauté \\
L'acte \\
L'acteur \\
L'action \\
L'activité \\
L'actualité \\
L'actuel \\
La cuisine \\
La culpabilité \\
La culture \\
La curiosité \\
La danse \\
La décadence \\
La décence \\
La déception \\
La décision \\
La déduction \\
La déficience \\
La définition \\
La délibération \\
La démagogie \\
La démence \\
La démesure \\
La démocratie \\
La démonstration \\
La déontologie \\
La dépendance \\
La dépense \\
La déraison \\
La dérision \\
La descendance \\
La description \\
La désillusion \\
La désinvolture \\
La désobéissance \\
La destruction \\
La détermination \\
La dette \\
La déviance \\
La dialectique \\
La dictature \\
La différence \\
La difformité \\
La dignité \\
La digression \\
La discipline \\
La discorde \\
La discrétion \\
La discrimination \\
La discursivité \\
La discussion \\
La disgrâce \\
La disharmonie \\
La disponibilité \\
La disposition \\
La dispute \\
La dissidence \\
La dissimulation \\
La distance \\
La distinction \\
La distraction \\
La diversion \\
La diversité \\
La division \\
L'admiration \\
La domestication \\
La domination \\
La douleur \\
La droiture \\
La dualité \\
La duplicité \\
La durée \\
L'adversité \\
La faiblesse \\
La familiarité \\
La famille \\
La fatalité \\
La fatigue \\
La fausseté \\
La faute \\
La fermeté \\
La fête \\
L'affirmation \\
La fiction \\
La fidélité \\
La fierté \\
La fièvre \\
La figuration \\
La fin \\
La finalité \\
La finitude \\
La foi \\
La folie \\
La fonction \\
La force \\
La formalisation \\
La forme \\
La fortune \\
La foule \\
La fragilité \\
La franchise \\
La fraternité \\
La fraude \\
La frivolité \\
La frontière \\
La futilité \\
La généalogie \\
La généralisation \\
La générosité \\
La genèse \\
La gentillesse \\
La géographie \\
La géométrie \\
La gloire \\
La grâce \\
La grammaire \\
La grandeur \\
La gratitude \\
La gratuité \\
L'agression \\
L'agressivité \\
L'agriculture \\
La grossièreté \\
La guérison \\
La guerre \\
La haine \\
La hiérarchie \\
La honte \\
La jalousie \\
La jeunesse \\
La joie \\
La jouissance \\
La jurisprudence \\
La justice \\
La justification \\
La lâcheté \\
La laïcité \\
La laideur \\
La lassitude \\
L'aléatoire \\
La lecture \\
La légende \\
La légèreté \\
La légitimation \\
La légitimité \\
La liberté \\
L'aliénation \\
La limite \\
L'allégorie \\
La loi \\
La loyauté \\
L'altérité \\
L'altruisme \\
La machine \\
La magie \\
La magnanimité \\
La main \\
La maîtrise \\
La majesté \\
La majorité \\
La maladie \\
La malchance \\
La malveillance \\
La manière \\
La manifestation \\
La marchandise \\
La marge \\
La marginalité \\
L'amateur \\
L'amateurisme \\
La matière \\
La maturité \\
L'ambiguïté \\
L'ambition \\
L'âme \\
La méchanceté \\
La médiation \\
La médiocrité \\
La méditation \\
La méfiance \\
La mélancolie \\
La mémoire \\
La menace \\
La mesure \\
La métamorphose \\
La métaphore \\
La méthode \\
L'ami \\
La minorité \\
La misanthropie \\
La misère \\
La misologie \\
L'amitié \\
La modalité \\
La mode \\
La modération \\
La modernité \\
La modestie \\
La mondialisation \\
La monnaie \\
La monumentalité \\
La mort \\
La multiplicité \\
La multitude \\
L'anachronisme \\
La naissance \\
La naïveté \\
L'analogie \\
L'analyse \\
L'anarchie \\
La nation \\
La nature \\
L'anéantissement \\
L'anecdotique \\
La nécessité \\
La négation \\
La négligence \\
La négociation \\
La neutralité \\
L'angélisme \\
L'angoisse \\
L'animal \\
L'animalité \\
L'animisme \\
La noblesse \\
L'anomalie \\
L'anonymat \\
L'anormal \\
La normalité \\
La norme \\
La nostalgie \\
La nouveauté \\
L'antériorité \\
L'anthropocentrisme \\
L'anticipation \\
L'antinomie \\
La nuance \\
La nudité \\
La nuit \\
La paix \\
La parenté \\
La paresse \\
La parole \\
La participation \\
La passion \\
La passivité \\
La paternité \\
L'apathie \\
La patience \\
La patrie \\
La pauvreté \\
La peine \\
La pensée \\
La perception \\
La perfectibilité \\
La perfection \\
La performance \\
La permanence \\
La personnalité \\
La personne \\
La perspective \\
La persuasion \\
La pertinence \\
La perversion \\
La perversité \\
La peur \\
La philanthropie \\
La pitié \\
La plaisanterie \\
La plénitude \\
La pluralité \\
La poésie \\
La polémique \\
La police \\
La politesse \\
La politique \\
L'apolitisme \\
La populace \\
La population \\
La pornographie \\
La possession \\
La possibilité \\
L'apparence \\
L'appel \\
L'apprentissage \\
L'appropriation \\
L'approximation \\
La précarité \\
La précaution \\
La précision \\
La présence \\
La présomption \\
La preuve \\
La prévision \\
La prière \\
La prison \\
La privation \\
La probabilité \\
La probité \\
La profondeur \\
La promenade \\
La promesse \\
La proposition \\
La propriété \\
La protection \\
La providence \\
La prudence \\
La publicité \\
La pudeur \\
La puissance \\
La pulsion \\
La punition \\
La pureté \\
La qualité \\
La quantité \\
La radicalité \\
La raison \\
La rareté \\
La rationalité \\
L'arbitraire \\
L'archéologie \\
L'archive \\
La réaction \\
La réalité \\
La recherche \\
La réciprocité \\
La réconciliation \\
La reconnaissance \\
La rectitude \\
La référence \\
La réflexion \\
La réforme \\
La réfutation \\
La règle \\
La régression \\
La régularité \\
La relation \\
La relativité \\
La religion \\
La réminiscence \\
La renaissance \\
La Renaissance \\
La rencontre \\
La réparation \\
La répétition \\
La représentation \\
La reproduction \\
La république \\
La réputation \\
La résignation \\
La résilience \\
La résistance \\
La résolution \\
La responsabilité \\
La ressemblance \\
La réussite \\
La révélation \\
La rêverie \\
La révolte \\
La révolution \\
L'argent \\
L'argumentation \\
La rhétorique \\
La richesse \\
La rigueur \\
L'aristocratie \\
La rivalité \\
L'art \\
L'artifice \\
L'artificiel \\
L'artiste \\
La ruine \\
La rumeur \\
La rupture \\
La ruse \\
La sagesse \\
La sainteté \\
La sanction \\
La santé \\
La satisfaction \\
L'ascèse \\
L'ascétisme \\
La sculpture \\
La sécularisation \\
La sécurité \\
La séduction \\
La ségrégation \\
La sensation \\
La sensibilité \\
La séparation \\
La sérénité \\
La servitude \\
La sévérité \\
La sexualité \\
La signification \\
La simplicité \\
La simulation \\
La sincérité \\
La singularité \\
La situation \\
La sobriété \\
La sociabilité \\
La société \\
La solidarité \\
La solitude \\
La sollicitude \\
La souffrance \\
La souveraineté \\
La spéculation \\
La spontanéité \\
L'assentiment \\
L'association \\
La standardisation \\
La structure \\
La subjectivité \\
La substance \\
La subtilité \\
La superstition \\
La sûreté \\
La surprise \\
La survie \\
La sympathie \\
La tautologie \\
La technique \\
La technocratie \\
La téléologie \\
La télévision \\
La tempérance \\
La tendance \\
La tentation \\
La terre \\
La terreur \\
L'athéisme \\
La théogonie \\
La théorie \\
La tolérance \\
L'atome \\
La totalitarisme \\
La totalité \\
La trace \\
La tradition \\
La traduction \\
La tragédie \\
La trahison \\
La tranquillité \\
La transcendance \\
La transe \\
La transgression \\
La transmission \\
La transparence \\
La tristesse \\
L'attachement \\
L'attente \\
L'attention \\
L'attraction \\
La tyrannie \\
L'audace \\
L'autarcie \\
L'authenticité \\
L'autobiographie \\
L'autocritique \\
L'automate \\
L'automatisation \\
L'autonomie \\
L'autoportrait \\
L'autorité \\
La valeur \\
La validité \\
La vanité \\
L'avarice \\
La variété \\
La vénalité \\
La vengeance \\
L'avenir \\
L'aventure \\
La véracité \\
La vérification \\
La vérité \\
La vertu \\
L'aveu \\
L'aveuglement \\
La vie \\
La vieillesse \\
La vigilance \\
La ville \\
La violence \\
La virtualité \\
La virtuosité \\
La vitesse \\
La vocation \\
La voix \\
La volupté \\
La vraisemblance \\
La vulgarisation \\
La vulgarité \\
La vulnérabilité \\
L'axiome \\
Le barbare \\
Le baroque \\
Le bavardage \\
Le besoin \\
L'éblouissement \\
Le bonheur \\
Le bricolage \\
Le bruit \\
Le cadavre \\
Le calcul \\
Le calendrier \\
Le cannibalisme \\
Le capitalisme \\
Le caractère \\
L'écart \\
L'échange \\
Le changement \\
Le chant \\
Le chaos \\
Le charme \\
Le châtiment \\
Le chef \\
Le chemin \\
Le choix \\
Le citoyen \\
Le classicisme \\
L'éclat \\
Le cliché \\
Le cœur \\
Le combat \\
Le comédien \\
Le comique \\
Le commencement \\
Le commerce \\
Le commun \\
Le complexe \\
Le comportement \\
Le compromis \\
Le concept \\
Le concret \\
Le conditionnel \\
Le conflit \\
Le conformisme \\
L'économie \\
Le conseil \\
Le consensus \\
Le consentement \\
Le conservatisme \\
Le contentement \\
Le contingent \\
Le continu \\
Le contrat \\
Le convenable \\
Le cosmopolitisme \\
Le courage \\
Le cri \\
Le crime \\
Le critère \\
L'écriture \\
Le cynisme \\
Le dandysme \\
Le danger \\
Le débat \\
Le déchet \\
Le défaut \\
Le dégoût \\
Le déguisement \\
Le délire \\
Le démoniaque \\
Le dépaysement \\
Le dérèglement \\
Le désaccord \\
Le désenchantement \\
Le désespoir \\
Le déshonneur \\
Le design \\
Le désintéressement \\
Le désir \\
Le désœuvrement \\
Le désordre \\
Le despotisme \\
Le destin \\
Le désuet \\
Le détachement \\
Le détail \\
Le déterminisme \\
Le deuil \\
Le devenir \\
Le devoir \\
Le dévouement \\
Le diable \\
Le dialogue \\
Le dictionnaire \\
Le dilemme \\
Le discernement \\
Le discontinu \\
Le divers \\
Le divertissement \\
Le divin \\
Le dogmatisme \\
Le don \\
Le donné \\
Le double \\
Le doute \\
Le drame \\
Le droit \\
Le dualisme \\
Le factice \\
Le fait \\
Le familier \\
Le fanatisme \\
Le fantasme \\
Le fantastique \\
Le fatalisme \\
Le faux \\
Le féminin \\
Le féminisme \\
Le fétichisme \\
L'effectivité \\
L'efficacité \\
L'efficience \\
L'effort \\
Le finalisme \\
Le fini \\
Le flegme \\
Le fond \\
Le fondement \\
Le formalisme \\
Le fou \\
Le fragment \\
Le frivole \\
L'égalité \\
L'égarement \\
Le génie \\
Le genre \\
Le geste \\
L'égoïsme \\
Le goût \\
Le handicap \\
Le hasard \\
Le haut \\
Le héros \\
Le jeu \\
Le juge \\
Le jugement \\
Le laboratoire \\
Le langage \\
L'élection \\
L'élégance \\
Le législateur \\
L'élémentaire \\
Le lieu \\
Le livre \\
Le logique \\
Le loisir \\
Le luxe \\
Le lyrisme \\
Le maître \\
Le mal \\
Le malentendu \\
Le malheur \\
L'émancipation \\
Le maniérisme \\
Le marché \\
Le mariage \\
Le masculin \\
Le masque \\
Le matérialisme \\
Le matériel \\
Le mécanisme \\
Le méchant \\
Le meilleur \\
Le mensonge \\
Le mépris \\
Le mérite \\
Le merveilleux \\
Le métier \\
Le milieu \\
Le miracle \\
Le miroir \\
Le misanthrope \\
Le mode \\
Le modèle \\
Le moi \\
Le monde \\
Le monstre \\
Le monstrueux \\
Le moralisme \\
Le moraliste \\
L'émotion \\
Le mouvement \\
L'empathie \\
L'empire \\
L'empirisme \\
Le multiculturalisme \\
Le multiple \\
Le musée \\
Le Musée \\
Le mystère \\
Le mysticisme \\
Le mythe \\
Le naïf \\
Le narcissisme \\
Le naturel \\
L'encyclopédie \\
L'Encyclopédie \\
Le néant \\
Le négatif \\
Le néologisme \\
L'énergie \\
L'enfance \\
L'enfant \\
L'engagement \\
L'engendrement \\
L'énigme \\
Le nihilisme \\
L'ennemi \\
L'ennui \\
Le nomade \\
Le nomadisme \\
Le nombre \\
Le nominalisme \\
L'enquête \\
L'enthousiasme \\
L'entraide \\
Le nu \\
L'envie \\
L'environnement \\
Le pacifisme \\
Le paradigme \\
Le paradoxe \\
Le pardon \\
Le pari \\
Le partage \\
Le particulier \\
Le passé \\
Le paternalisme \\
Le pathologique \\
Le patriarcat \\
Le patrimoine \\
Le patriotisme \\
Le paysage \\
Le péché \\
Le pédagogue \\
Le pessimisme \\
Le peuple \\
Le phantasme \\
L'éphémère \\
Le phénomène \\
Le philanthrope \\
Le plagiat \\
Le plaisir \\
Le pluralisme \\
Le poétique \\
Le populaire \\
Le populisme \\
Le portrait \\
Le possible \\
Le pouvoir \\
Le pragmatisme \\
Le préférable \\
Le préjugé \\
Le premier \\
Le présent \\
L'épreuve \\
Le primitif \\
Le prince \\
Le principe \\
Le probable \\
Le processus \\
Le prochain \\
Le profane \\
Le profit \\
Le progrès \\
Le projet \\
Le propre \\
Le propriétaire \\
Le provisoire \\
Le public \\
Le quelconque \\
L'équité \\
L'équivalence \\
L'équivocité \\
L'équivoque \\
Le quotidien \\
Le racisme \\
Le raffinement \\
Le rationalisme \\
Le rationnel \\
Le réalisme \\
Le récit \\
Le reconnaissance \\
Le réel \\
Le refoulement \\
Le refus \\
Le regard \\
Le regret \\
Le relativisme \\
Le remords \\
Le renoncement \\
Le repentir \\
Le repos \\
Le respect \\
Le ressentiment \\
Le rêve \\
Le ridicule \\
Le rien \\
Le rigorisme \\
Le rire \\
Le risque \\
Le rite \\
Le rituel \\
Le roman \\
Le romantisme \\
L'érotisme \\
L'erreur \\
L'érudition \\
Le rythme \\
Le sacré \\
Le sacrifice \\
Le salaire \\
Le salut \\
Les amis \\
Les archives \\
Le sauvage \\
Les biotechnologies \\
Le scandale \\
Les caractères \\
Les catastrophes \\
Les catégories \\
Le scepticisme \\
Les cérémonies \\
Les choses \\
Les circonstances \\
L'esclavage \\
L'esclave \\
Les commencements \\
Les conséquences \\
Les coutumes \\
Le scrupule \\
Les dictionnaires \\
Les échanges \\
Les écrans \\
Le secret \\
Les éléments \\
Les élites \\
Le semblable \\
Les enfants \\
Le sensationnel \\
Les ensembles \\
Le sensible \\
Le sentiment \\
Les envieux \\
Le sérieux \\
Le serment \\
Les excuses \\
Les foules \\
Les fous \\
Les frontières \\
Les générations \\
Les héros \\
Les idoles \\
Le signe \\
Le silence \\
Le simple \\
Le simulacre \\
Les individus \\
Le singulier \\
Les interdits \\
Les livres \\
Les lois \\
Les machines \\
Les marginaux \\
Les matériaux \\
Les modalités \\
Les modèles \\
Les mœurs \\
Les monstres \\
Les morts \\
Les noms \\
Les normes \\
Le soin \\
Le soldat \\
Le solipsisme \\
Le sommeil \\
L'ésotérisme \\
Le souci \\
Le soupçon \\
Les outils \\
Le souvenir \\
L'espace \\
Les pauvres \\
Le spectacle \\
L'espérance \\
Les phénomènes \\
Les plaisirs \\
L'espoir \\
Le sport \\
Les possibles \\
Les prêtres \\
Les principes \\
L'esprit \\
Les proverbes \\
L'esquisse \\
Les relations \\
Les reproductions \\
Les rituels \\
Les robots \\
Les ruines \\
Les sacrifices \\
Les sauvages \\
L'essence \\
L'essentiel \\
Les sentiments \\
Les systèmes \\
L'esthète \\
L'esthétique \\
L'esthétisme \\
Les traditions \\
Le style \\
Le sublime \\
Le substitut \\
Le succès \\
Le sujet \\
Les universaux \\
Le surnaturel \\
Les vertus \\
Les vivants \\
Le syllogisme \\
Le symbole \\
Le symbolisme \\
Le système \\
Le tableau \\
Le tacite \\
Le tact \\
Le talent \\
L'État \\
Le témoignage \\
Le témoin \\
Le temps \\
L'éternité \\
Le terrain \\
Le territoire \\
Le théâtral \\
L'ethnocentrisme \\
L'étonnement \\
Le totalitarisme \\
Le totémisme \\
Le toucher \\
Le tragique \\
L'étranger \\
L'étrangeté \\
Le travail \\
Le troc \\
L'étude \\
Le tyran \\
L'eugénisme \\
L'Europe \\
L'euthanasie \\
L'évaluation \\
L'évasion \\
Le vécu \\
L'événement \\
Le verbalisme \\
Le verbe \\
Le vertige \\
Le vestige \\
Le vêtement \\
Le vide \\
L'évidence \\
Le virtuel \\
Le visage \\
Le vivant \\
Le volontarisme \\
L'évolution \\
Le voyage \\
Le vraisemblable \\
Le vulgaire \\
L'exactitude \\
L'excellence \\
L'exception \\
L'excès \\
L'exclusion \\
L'excuse \\
L'exemplaire \\
L'exemplarité \\
L'exemple \\
L'exercice \\
L'exil \\
L'existence \\
L'expérience \\
L'expérimentation \\
L'expertise \\
L'explication \\
L'exposition \\
L'expression \\
L'extériorité \\
L'extraordinaire \\
L'extrémisme \\
L'habileté \\
L'habitation \\
L'habitude \\
L'harmonie \\
L'hérédité \\
L'hérésie \\
L'héritage \\
L'héroïsme \\
L'hésitation \\
L'hétéronomie \\
L'historien \\
L'honnêteté \\
L'honneur \\
L'horizon \\
L'horreur \\
L'horrible \\
L'hospitalité \\
L'humain \\
L'humanité \\
L'humiliation \\
L'humilité \\
L'humour \\
L'hypocrisie \\
L'hypothèse \\
L'idéal \\
L'idéalisme \\
L'idéaliste \\
L'idéalité \\
L'identification \\
L'identité \\
L'idéologie \\
L'idiot \\
L'idolâtrie \\
L'idole \\
L'ignoble \\
L'ignorance \\
L'illimité \\
L'illusion \\
L'illustration \\
L'image \\
L'imaginaire \\
L'imagination \\
L'imitation \\
L'immanence \\
L'immatériel \\
L'immédiat \\
L'immensité \\
L'immortalité \\
L'immuable \\
L'immutabilité \\
L'impardonnable \\
L'impartialité \\
L'impassibilité \\
L'impatience \\
L'impensable \\
L'impératif \\
L'imperceptible \\
L'impersonnel \\
L'implicite \\
L'impossible \\
L'imposteur \\
L'imprescriptible \\
L'impression \\
L'imprévisible \\
L'improbable \\
L'improvisation \\
L'imprudence \\
L'impuissance \\
L'impunité \\
L'inaccessible \\
L'inachevé \\
L'inaction \\
L'inaliénable \\
L'inaperçu \\
L'inapparent \\
L'inattendu \\
L'incarnation \\
L'incertitude \\
L'incommensurabilité \\
L'incommensurable \\
L'incompréhensible \\
L'inconcevable \\
L'inconnaissable \\
L'inconnu \\
L'inconscience \\
L'inconscient \\
L'inconséquence \\
L'inconstance \\
L'incorporel \\
L'incrédulité \\
L'incroyable \\
L'inculture \\
L'indécence \\
L'indécidable \\
L'indécision \\
L'indéfini \\
L'indémontrable \\
L'indépassable \\
L'indépendance \\
L'indescriptible \\
L'indésirable \\
L'indétermination \\
L'indéterminé \\
L'indice \\
L'indicible \\
L'indifférence \\
L'indignation \\
L'indignité \\
L'indiscutable \\
L'individu \\
L'individualisme \\
L'individualité \\
L'indivisible \\
L'induction \\
L'indulgence \\
L'inéluctable \\
L'inertie \\
L'inesthétique \\
L'inestimable \\
L'inexistant \\
L'inexpérience \\
L'infâme \\
L'infamie \\
L'inférence \\
L'infidélité \\
L'infini \\
L'influence \\
L'information \\
L'informe \\
L'ingénieur \\
L'ingénuité \\
L'ingratitude \\
L'inhibition \\
L'inhumain \\
L'inhumanité \\
L'inimaginable \\
L'inimitié \\
L'inintelligible \\
L'initiation \\
L'injonction \\
L'injustice \\
L'injustifiable \\
L'innocence \\
L'innommable \\
L'innovation \\
L'inobservable \\
L'inquiétant \\
L'inquiétude \\
L'insatisfaction \\
L'insensé \\
L'insensibilité \\
L'insignifiant \\
L'insolite \\
L'insouciance \\
L'insoumission \\
L'insoutenable \\
L'inspiration \\
L'instant \\
L'instinct \\
L'institution \\
L'instruction \\
L'instrument \\
L'insulte \\
L'insurrection \\
L'intangible \\
L'intellect \\
L'intellectuel \\
L'intelligence \\
L'intelligible \\
L'intempérance \\
L'intemporel \\
L'intention \\
L'intentionnalité \\
L'interaction \\
L'interdisciplinarité \\
L'interdit \\
L'intéressant \\
L'intérêt \\
L'intériorité \\
L'interprétation \\
L'intersubjectivité \\
L'intimité \\
L'intolérable \\
L'intolérance \\
L'intraduisible \\
L'intransigeance \\
L'introspection \\
L'intuition \\
L'inutile \\
L'invention \\
L'invérifiable \\
L'invisibilité \\
L'invisible \\
L'involontaire \\
L'ironie \\
L'irrationalité \\
L'irrationnel \\
L'irréductible \\
L'irréel \\
L'irréfléchi \\
L'irréfutable \\
L'irrégularité \\
L'irréparable \\
L'irreprésentable \\
L'irrésolution \\
L'irresponsabilité \\
L'irréversibilité \\
L'irréversible \\
L'irrévocable \\
L'ivresse \\
L'obéissance \\
L'objectivation \\
L'objectivité \\
L'objet \\
L'obligation \\
L'obscène \\
L'obscénité \\
L'obscur \\
L'obscurantisme \\
L'obscurité \\
L'observation \\
L'obsession \\
L'obstacle \\
L'occasion \\
L'œuvre \\
L'offense \\
L'oisiveté \\
L'oligarchie \\
L'omniscience \\
L'opinion \\
L'opportunisme \\
L'opportunité \\
L'opposant \\
L'opposition \\
L'optimisme \\
L'ordinaire \\
L'ordre \\
L'organique \\
L'organisation \\
L'organisme \\
L'orgueil \\
L'orientation \\
L'originalité \\
L'origine \\
L'ornement \\
L'oubli \\
L'outil \\
L'un \\
L'uniformité \\
L'unité \\
L'univers \\
L'universalisme \\
L'universel \\
L'urbanité \\
L'urgence \\
L'usage \\
L'usure \\
L'utile \\
L'utilité \\
L'utopie \\


\subsection{X et Y}
\label{sec-4-5}

\noindent
Acteurs sociaux et usages sociaux \\
Action et contemplation \\
Action et événement \\
Action et production \\
Activité et passivité \\
Affirmer et nier \\
Agir et faire \\
Agir et réagir \\
Ami et ennemi \\
Amitié et société \\
Amour et amitié \\
Amour et inconscient \\
Analyse et synthèse \\
Anomalie et anomie \\
Anthropologie et ontologie \\
Anthropologie et politique \\
Apparence et réalité \\
Apprendre et enseigner \\
Apprentissage et conditionnement \\
\emph{A priori} et \emph{a posteriori} \\
À quoi bon les sciences humaines et sociales ? \\
Argent et liberté \\
Argumenter et démontrer \\
Art et apparences \\
Art et authenticité \\
Art et beauté \\
Art et connaissance \\
Art et création \\
Art et critique \\
Art et divertissement \\
Art et émotion \\
Art et finitude \\
Art et folie \\
Art et forme \\
Art et illusion \\
Art et image \\
Art et imagination \\
Art et interdit \\
Art et jeu \\
Art et marchandise \\
Art et matière \\
Art et mélancolie \\
Art et mémoire \\
Art et métaphysique \\
Art et morale \\
Art et politique \\
Art et pouvoir \\
Art et propagande \\
Art et religieux \\
Art et religion \\
Art et représentation \\
Art et société \\
Art et Société \\
Art et symbole \\
Art et technique \\
Art et transgression \\
Art et vérité \\
Artiste et artisan \\
Art populaire et art savant \\
Arts de l'espace et arts du temps \\
Attente et espérance \\
Autorité et pouvoir \\
Autorité et souveraineté \\
Beauté et moralité \\
Beauté et vérité \\
Beauté naturelle et beauté artistique \\
Besoin et désir \\
Besoins et désirs \\
Bêtise et méchanceté \\
Bien commun et bien public \\
Bien commun et intérêt particulier \\
Bonheur et autarcie \\
Bonheur et satisfaction \\
Bonheur et société \\
Bonheur et technique \\
Bonheur et vertu \\
Calculer et penser \\
Calculer et raisonner \\
Castes et classes \\
Catégories logiques et catégories linguistiques \\
Cause et condition \\
Cause et effet \\
Cause et loi \\
Cause et raison \\
Causes et motivations \\
Causes et raisons \\
Causes premières et causes secondes \\
Ce qui fut et ce qui sera \\
Ce qui passe et ce qui demeure \\
Certitude et conviction \\
Certitude et probabilité \\
Certitude et vérité \\
Chance et bonheur \\
Choix et raison \\
Chose et objet \\
Chose et personne \\
Choses et personnes \\
Cinéma et réalité \\
Citoyen et soldat \\
Civilisation et barbarie \\
Classer et ordonner \\
Classes et histoire \\
Classicisme et romantisme \\
Colère et indignation \\
Comment distinguer désirs et besoins ? \\
Comment distinguer entre l'amour et l'amitié ? \\
Communauté et société \\
Communiquer et enseigner \\
Compétence et autorité \\
Composition et construction \\
Concept et image \\
Concept et intuition \\
Concept et métaphore \\
Conception et perception \\
Concevoir et juger \\
Concurrence et égalité \\
Conflit et démocratie \\
Conflit et liberté \\
Connaissance commune et connaissance scientifique \\
Connaissance de soi et conscience de soi \\
Connaissance du futur et connaissance du passé \\
Connaissance et croyance \\
Connaissance et expérience \\
Connaissance et perception \\
Connaissance historique et action politique \\
Connaître et comprendre \\
Connaître et penser \\
Conscience de soi et amour de soi \\
Conscience de soi et connaissance de soi \\
Conscience et attention \\
Conscience et connaissance \\
Conscience et conscience de soi \\
Conscience et existence \\
Conscience et liberté \\
Conscience et mémoire \\
Conscience et responsabilité \\
Conscience et subjectivité \\
Conscience et volonté \\
Consensus et conflit \\
Conservatisme et tradition \\
Consistance et précarité \\
Constitution et lois \\
Consumérisme et démocratie \\
Contemplation et distraction \\
Contingence et nécessité \\
Continuité et discontinuité \\
Contradiction et opposition \\
Contrainte et désobéissance \\
Contrainte et obligation \\
Contrôle et vigilance \\
Convaincre et persuader \\
Convention et observation \\
Conventions sociales et moralité \\
Conviction et certitude \\
Conviction et responsabilité \\
Convient-il d'opposer explication et interprétation ? \\
Corps et espace \\
Corps et esprit \\
Corps et identité \\
Corps et matière \\
Corps et nature \\
Crainte et espoir \\
Création et fabrication \\
Création et production \\
Créativité et contrainte \\
Créer et produire \\
Crime et châtiment \\
Crise et progrès \\
Croire et savoir \\
Croyance et certitude \\
Croyance et choix \\
Croyance et connaissance \\
Croyance et probabilité \\
Croyance et vérité \\
Culpabilité et responsabilité \\
Cultes et rituels \\
Culture et civilisation \\
Culture et communauté \\
Culture et conscience \\
Culture et différence \\
Culture et éducation \\
Culture et langage \\
Culture et savoir \\
Culture et technique \\
Culture et violence \\
Débattre et dialoguer \\
Découverte et invention \\
Découverte et invention dans les sciences \\
Découverte et justification \\
Déduction et expérience \\
Définition et démonstration \\
Définition nominale et définition réelle \\
Démocrates et démagogues \\
Démocratie ancienne et démocratie moderne \\
Démocratie et anarchie \\
Démocratie et démagogie \\
Démocratie et impérialisme \\
Démocratie et opinion \\
Démocratie et religion \\
Démocratie et représentation \\
Démocratie et république \\
Démocratie et transparence \\
Démonstration et argumentation \\
Démonstration et déduction \\
Démontrer et argumenter \\
Description et explication \\
Désintérêt et désintéressement \\
Désirer et vouloir \\
Désir et besoin \\
Désir et bonheur \\
Désir et interdit \\
Désir et langage \\
Désir et manque \\
Désir et ordre \\
Désir et politique \\
Désir et pouvoir \\
Désir et raison \\
Désir et réalité \\
Désir et volonté \\
Désobéissance et résistance \\
Déterminisme et responsabilité \\
Déterminisme psychique et déterminisme physique \\
Détruire et construire \\
Devenir et évolution \\
Devoir et bonheur \\
Devoir et conformisme \\
Devoir et contrainte \\
Devoir et intérêt \\
Devoir et liberté \\
Devoir et plaisir \\
Devoir et prudence \\
Devoir et vertu \\
Devoirs envers les autres et devoirs envers soi-même \\
Devoirs et passions \\
Dialectique et Philosophie \\
Dialogue et délibération en démocratie \\
Dieu et César \\
Dire et exprimer \\
Dire et faire \\
Dire et montrer \\
Discrimination et revendication \\
Discussion et conversation \\
Discussion et dialogue \\
Division du travail et cohésion sociale \\
Documents et monuments \\
Dogme et opinion \\
Doit-on distinguer devoir moral et obligation sociale ? \\
Don et échange \\
Donner et recevoir \\
Doute et raison \\
Dressage et éducation \\
Droit et coutume \\
Droit et démocratie \\
Droit et devoir \\
Droit et devoir sont-ils liés ? \\
Droit et morale \\
Droit et protection \\
Droit et violence \\
Droit naturel et loi naturelle \\
Droits de l'homme et droits du citoyen \\
Droits et devoirs \\
Droits et devoirs sont-ils réciproques ? \\
Durée et instant \\
Échange et don \\
Échange et partage \\
Échange et valeur \\
Économie et politique \\
Économie et société \\
Économie politique et politique économique \\
Écouter et entendre \\
Écrire et parler \\
Éducation et instruction \\
Éduquer et instruire \\
Efficacité et justice \\
Égalité et différence \\
Égalité et solidarité \\
Égoïsme et altruisme \\
Égoïsme et individualisme \\
Égoïsme et méchanceté \\
Empirique et expérimental \\
Enfance et moralité \\
Enseigner et éduquer \\
Entendement et raison \\
Entre l'art et la nature, qui imite l'autre ? \\
Entre l'opinion et la science, n'y a-t-il qu'une différence de degré ? \\
Épistémologie générale et épistémologie des sciences particulières \\
Erreur et faute \\
Erreur et illusion \\
Espace et représentation \\
Espace et structure sociale \\
Espace mathématique et espace physique \\
Espace public et vie privée \\
Esprit et intériorité \\
Essence et existence \\
Esthétique et éthique \\
Esthétique et poétique \\
Esthétisme et moralité \\
Est-il légitime d'opposer liberté et nécessité ? \\
Estime et respect \\
Est-on fondé à distinguer la justice et le droit ? \\
État et institutions \\
État et nation \\
État et société \\
État et Société \\
État et société civile \\
Éternité et immortalité \\
Éthique et authenticité \\
Éthique et esthétique \\
Éthique et Morale \\
Ethnologie et cinéma \\
Ethnologie et ethnocentrisme \\
Ethnologie et sociologie \\
Étonnement et sidération \\
Être et apparaître \\
Être et avoir \\
Être et avoir été \\
Être et devenir \\
Être et devoir être \\
Être et devoir-être \\
Être et être pensé \\
Être et exister \\
Être et ne plus être \\
Être et paraître \\
Être et penser, est-ce la même chose ? \\
Être et représentation \\
Être et sens \\
Être juge et partie \\
Être, vie et pensée \\
Évidence et certitude \\
Évidence et raison \\
Évidence et vérité \\
Évidences et préjugés \\
Évolution biologique et culture \\
Évolution et progrès \\
Évolution et révolution \\
Excuser et pardonner \\
Existence et contingence \\
Existence et essence \\
Expérience esthétique et sens commun \\
Expérience et approximation \\
Expérience et expérimentation \\
Expérience et habitude \\
Expérience et interprétation \\
Expérience et phénomène \\
Expérience et vérité \\
Expérience immédiate et expérimentation scientifique \\
Expérimentation et vérification \\
Explication et prévision \\
Expliquer et comprendre \\
Expliquer et interpréter \\
Expliquer et justifier \\
Expression et création \\
Expression et signification \\
Extension et compréhension \\
Fabriquer et créer \\
Faire et laisser faire \\
Fait et essence \\
Fait et fiction \\
Fait et preuve \\
Fait et théorie \\
Fait et valeur \\
Faits et preuves \\
Faits et valeurs \\
Famille et tribu \\
Faut-il choisir entre être heureux et être libre ? \\
Faut-il distinguer désir et besoin ? \\
Faut-il distinguer esthétique et philosophie de l'art ? \\
Faut-il opposer histoire et mémoire ? \\
Faut-il opposer la matière et l'esprit ? \\
Faut-il opposer la théorie et la pratique ? \\
Faut-il opposer le don et l'échange ? \\
Faut-il opposer l'État et la société ? \\
Faut-il opposer le temps vécu et le temps des choses ? \\
Faut-il opposer l'histoire et la fiction ? \\
Faut-il opposer nature et culture ? \\
Faut-il opposer raison et sensation ? \\
Faut-il opposer rhétorique et philosophie ? \\
Faut-il séparer la science et la technique ? \\
Faut-il séparer morale et politique ? \\
Fiction et virtualité \\
Foi et bonne foi \\
Foi et raison \\
Foi et savoir \\
Foi et superstition \\
Folie et raison \\
Folie et société \\
Fonction et prédicat \\
Force et violence \\
Formaliser et axiomatiser \\
Forme et contenu \\
Forme et matière \\
Forme et rythme \\
Former et éduquer \\
Génie et technique \\
Genre et espèce \\
Gérer et gouverner \\
Gouvernement des hommes et administration des choses \\
Gouvernement et société \\
Gouverner et se gouverner \\
Grammaire et métaphysique \\
Grammaire et philosophie \\
Grandeur et décadence \\
Guerre et politique \\
Guerres justes et injustes \\
Hasard et destin \\
Histoire et anthropologie \\
Histoire et devenir \\
Histoire et écriture \\
Histoire et ethnologie \\
Histoire et fiction \\
Histoire et géographie \\
Histoire et mémoire \\
Histoire et morale \\
Histoire et politique \\
Histoire et progrès \\
Histoire et structure \\
Histoire et violence \\
Histoire individuelle et histoire collective \\
Humour et ironie \\
Hypothèse et vérité \\
Ici et maintenant \\
Idéal et utopie \\
Idée et réalité \\
Identité et changement \\
Identité et communauté \\
Identité et différence \\
Identité et égalité \\
Identité et indiscernabilité \\
Illégalité et injustice \\
Illusion et apparence \\
Image et concept \\
Image et idée \\
Imaginaire et politique \\
Imagination et conception \\
Imagination et culture \\
Imagination et pouvoir \\
Imagination et raison \\
Imitation et création \\
Imitation et identification \\
Imitation et représentation \\
Incertitude et action \\
Inconscient et déterminisme \\
Inconscient et identité \\
Inconscient et inconscience \\
Inconscient et instinct \\
Inconscient et langage \\
Inconscient et liberté \\
Inconscient et mythes \\
Indépendance et autonomie \\
Indépendance et liberté \\
Individualisme et égoïsme \\
Individuation et identité \\
Individu et citoyen \\
Individu et communauté \\
Individu et société \\
Infini et indéfini \\
Information et communication \\
Information et opinion \\
Innocence et ignorance \\
Instinct et morale \\
Instruction et éducation \\
Instruire et éduquer \\
Intentions, plans et stratégies \\
Interdire et prohiber \\
Intérêt général et bien commun \\
Interprétation et création \\
Interpréter et expliquer \\
Interpréter et formaliser dans les sciences humaines \\
Interpréter et traduire \\
Interroger et répondre \\
Intuition et concept \\
Intuition et déduction \\
Intuition et intellection \\
Invention et création \\
Invention et découverte \\
Invention et imitation \\
Jugement analytique et jugement synthétique \\
Jugement de goût et jugement esthétique \\
Jugement esthétique et jugement de valeur \\
Jugement et réflexion \\
Jugement et vérité \\
Jugement moral et jugement empirique \\
Juger et connaître \\
Juger et décider \\
Juger et raisonner \\
Juger et sentir \\
Justice et charité \\
Justice et égalité \\
Justice et équité \\
Justice et force \\
Justice et pardon \\
Justice et utilité \\
Justice et vengeance \\
Justice et violence \\
Justification et politique \\
Justifier et prouver \\
La beauté et la grâce \\
La bête et l'animal \\
La bêtise et la méchanceté sont-elles liées intrinsèquement ? \\
La bêtise et la méchanceté sont-elles liées nécessairement ? \\
L'absolu et le relatif \\
L'abstrait est-il en dehors de l'espace et du temps ? \\
L'abstrait et le concret \\
L'académisme et les fins de l'art \\
La cause et la raison \\
La cause et l'effet \\
La chasse et la guerre \\
La connaissance et la croyance \\
La connaissance et la morale \\
La connaissance et le vivant \\
La conscience de soi et l'identité personnelle \\
La conscience et l'inconscient \\
La convention et l'arbitraire \\
La crainte et l'ignorance \\
La croyance et la foi \\
La croyance et la raison \\
L'acte et la parole \\
L'acte et la puissance \\
L'acte et l'œuvre \\
L'acteur et son rôle \\
L'action et le risque \\
L'action et son contexte \\
La culture et les cultures \\
La culture savante et la culture populaire \\
La démocratie et les experts \\
La démocratie et les institutions de la justice \\
La démocratie et le statut de la loi \\
La distinction de la nature et de la culture est-elle un fait de culture ? \\
La famille et la cité \\
La famille et le droit \\
La faute et le péché \\
La faute et l'erreur \\
La fin et les moyens \\
La fonction et l'organe \\
La force et le droit \\
La gauche et la droite \\
La grammaire et la logique \\
La guerre et la paix \\
La haine et le mépris \\
La justice et la force \\
La justice et la loi \\
La justice et la paix \\
La justice et le droit \\
La justice et l'égalité \\
La langue et la parole \\
La lettre et l'esprit \\
La liberté et l'égalité sont-elles compatibles ? \\
La liberté et le hasard \\
La liberté et le temps \\
La logique et le réel \\
La loi et la coutume \\
La loi et la règle \\
La loi et le règlement \\
La loi et les mœurs \\
La loi et l'ordre \\
La louange et le blâme \\
La main et l'esprit \\
La main et l'outil \\
La matière et la forme \\
La matière et la vie \\
La matière et l'esprit \\
L'âme et le corps \\
L'âme et le corps sont-ils une seule et même chose ? \\
L'âme et l'esprit \\
L'âme, le monde et Dieu \\
La mémoire et l'histoire \\
La mémoire et l'individu \\
La mémoire et l'oubli \\
La morale et la politique \\
La morale et la religion visent-elles les mêmes fins ? \\
La morale et le droit \\
La morale et les mœurs \\
La moralité et le traitement des animaux \\
L'amour et la haine \\
L'amour et la justice \\
L'amour et l'amitié \\
L'amour et la mort \\
L'amour et le devoir \\
L'amour et le respect \\
La musique et le bruit \\
L'analyse et la synthèse \\
La nation et l'État \\
La nature et la grâce \\
La nature et le beau \\
La nature et le monde \\
Langage et communication \\
Langage et logique \\
Langage et passions \\
Langage et pensée \\
Langage et pouvoir \\
Langage et réalité \\
Langage et société \\
Langage, langue et parole \\
Langage ordinaire et langage de la science \\
Langue et parole \\
L'animal et la bête \\
L'animal et l'homme \\
La norme et le fait \\
La notion de loi dans les sciences de la nature et dans les sciences de l'homme \\
La nuit et le jour \\
La panne et la maladie \\
La parenté et la famille \\
La parole et l'écriture \\
La parole et le geste \\
La partie et le tout \\
La peine de mort est-elle juste, injuste, et pourquoi ? \\
La pensée et la conscience sont-elles une seule et même chose ? \\
La personne et l'individu \\
La philosophie et le sens commun \\
La philosophie et les sciences \\
La philosophie et son histoire \\
La physique et la chimie \\
La poésie et l'idée \\
La politique et la gloire \\
La politique et la guerre \\
La politique et la ville \\
La politique et le bonheur \\
La politique et le mal \\
La politique et le politique \\
La politique et les passions \\
La politique et l'opinion \\
La politique n'est-elle que l'art de conquérir et de conserver le pouvoir ? \\
La promesse et le contrat \\
La propriété et le travail \\
La puissance et l'acte \\
La quantité et la qualité \\
La raison et le réel \\
La raison et l'expérience \\
La raison et l'irrationnel \\
L'architecte et l'ingénieur \\
La règle et l'exception \\
La religion et la croyance \\
L'argent et la valeur \\
L'art et la manière \\
L'art et la morale \\
L'art et la nature \\
L'art et la nouveauté \\
L'art et la technique \\
L'art et la tradition \\
L'art et la vie \\
L'art et le beau \\
L'art et le divin \\
L'art et le jeu \\
L'art et le mouvement \\
L'art et l'éphémère \\
L'art et le réel \\
L'art et le rêve \\
L'art et le sacré \\
L'art et les arts \\
L'art et l'espace \\
L'art et le temps \\
L'art et l'illusion \\
L'art et l'immoralité \\
L'art et l'invisible \\
L'art et morale \\
L'artiste et l'artisan \\
L'artiste et la sensation \\
L'artiste et la société \\
L'artiste et le savant \\
La sagesse et la passion \\
La sagesse et l'expérience \\
La science a-t-elle besoin d'un critère de démarcation entre science et non science ? \\
La science et la foi \\
La science et le mythe \\
La science et les sciences \\
La science et l'irrationnel \\
La société civile et l'État \\
La société et les échanges \\
La société et l'État \\
La société et l'individu \\
La somme et le tout \\
La structure et le sujet \\
La surface et la profondeur \\
La technique et le corps \\
La technique et le travail \\
La Terre et le Ciel \\
La théorie et la pratique \\
La théorie et l'expérience \\
La trace et l'indice \\
L'auteur et le créateur \\
L'autre et les autres \\
La valeur et le prix \\
La veille et le sommeil \\
La ville et la campagne \\
La vision et le toucher \\
La volonté et le désir \\
La vue et le toucher \\
La vue et l'ouïe \\
Le beau et l'agréable \\
Le beau et le bien \\
Le beau et le bien sont-ils, au fond, identiques ? \\
Le beau et le joli \\
Le beau et le sublime \\
Le beau et l'utile \\
Le besoin et le désir \\
Le bien commun et l'intérêt de tous \\
Le bien et le beau \\
Le bien et le mal \\
Le bien et les biens \\
Le bien et l'utile \\
Le bon et l'utile \\
Le bonheur et la raison \\
Le bonheur et la technique \\
Le bonheur et la vertu \\
Le bourgeois et le citoyen \\
Le bruit et la musique \\
Le cerveau et la pensée \\
L'échange des marchandises et les rapports humains \\
L'échange et l'usage \\
Le charme et la grâce \\
Le choix et la liberté \\
Le ciel et la terre \\
Le citoyen peut-il être à la fois libre et soumis à l'État ? \\
Le clair et l'obscur \\
Le cœur et la raison \\
Le comique et le tragique \\
Le comment et le pourquoi \\
Le commun et le propre \\
Le concept et l'exemple \\
Le concret et l'abstrait \\
Le conflit entre la science et la religion est-il inévitable ? \\
L'économie et la politique \\
L'économique et le politique \\
Le conscient et l'inconscient \\
Le corps et la machine \\
Le corps et l'âme \\
Le corps et l'esprit \\
Le corps et le temps \\
Le corps et l'instrument \\
Le créé et l'incréé \\
L'écrit et l'oral \\
L'écriture et la parole \\
L'écriture et la pensée \\
Le dedans et le dehors \\
Le désir et la culpabilité \\
Le désir et la loi \\
Le désir et le besoin \\
Le désir et le mal \\
Le désir et le manque \\
Le désir et le rêve \\
Le désir et le temps \\
Le désir et le travail \\
Le désir et l'interdit \\
Le dessin et la couleur \\
Le devoir et la dette \\
Le devoir et le bonheur \\
Le dire et le faire \\
Le don et la dette \\
Le don et l'échange \\
Le droit de vie et de mort \\
Le droit et la convention \\
Le droit et la force \\
Le droit et la liberté \\
Le droit et la loi \\
Le droit et la morale \\
Le droit et le devoir \\
Le Droit et l'État \\
Le fait et le droit \\
Le fait et l'événement \\
Le faux et l'absurde \\
Le faux et le fictif \\
Le féminin et le masculin \\
L'effet et la cause \\
Le fini et l'infini \\
Le fond et la forme \\
L'égalité des hommes et des femmes est-elle une question politique ? \\
Légalité et causalité \\
Légalité et légitimité \\
Légalité et moralité \\
Le général et le particulier \\
Le génie et la règle \\
Le génie et le savant \\
Le genre et l'espèce \\
Le geste et la parole \\
Légitimité et légalité \\
Le gouvernement des hommes et l'administration des choses \\
Le gouvernement de soi et des autres \\
Le hasard et la nécessité \\
Le haut et le bas \\
Le je et le tu \\
Le jeu et le divertissement \\
Le jeu et le hasard \\
Le jeu et le sérieux \\
Le juste et le bien \\
Le juste et le légal \\
Le langage et la pensée \\
Le langage et le réel \\
Le légal et le légitime \\
Le légitime et le légal \\
Le lieu et l'espace \\
Le littéral et le figuré \\
Le maître et l'esclave \\
Le masculin et le féminin \\
Le matériel et le virtuel \\
Le mécanisme et la mécanique \\
Le médiat et l'immédiat \\
Le même et l'autre \\
Le mérite et les talents \\
Le mien et le tien \\
Le moi et la conscience \\
Le mot et la chose \\
Le mot et le geste \\
Le multiple et l'un \\
Le naturalisme des sciences humaines et sociales \\
Le naturel et l'artificiel \\
Le naturel et le fabriqué \\
Le nécessaire et le contingent \\
Le nécessaire et le superflu \\
L'enfant et l'adulte \\
Le noble et le vil \\
Le nombre et la mesure \\
Le nom et le verbe \\
Le normal et le pathologique \\
L'entendement et la volonté \\
Le nu et la nudité \\
Le pardon et l'oubli \\
Le passé et le présent \\
Le personnage et la personne \\
Le peuple et la nation \\
Le peuple et les élites \\
Le philosophe a-t-il besoin de l'histoire ?Prouver et justifier \\
Le philosophe et le sophiste \\
Le plaisir et la douleur \\
Le plaisir et la joie \\
Le plaisir et la jouissance \\
Le plaisir et la peine \\
Le plaisir et le bien \\
Le politique et le religieux \\
Le possible et le probable \\
Le possible et le réel \\
Le possible et le virtuel \\
Le possible et l'impossible \\
Le pour et le contre \\
Le pourquoi et le comment \\
Le pouvoir des sciences humaines et sociales \\
Le pouvoir et l'autorité \\
Le pouvoir et la violence \\
Le premier et le primitif \\
Le privé et le public \\
Le proche et le lointain \\
Le public et le privé \\
L e pur et l'impur \\
Le pur et l'impur \\
Le raisonnable et le rationnel \\
Le rationnel et le raisonnable \\
Le rationnel et l'irrationnel \\
Le réel et la fiction \\
Le réel et le matériel \\
Le réel et le nécessaire \\
Le réel et le possible \\
Le réel et le virtuel \\
Le réel et le vrai \\
Le réel et l'idéal \\
Le réel et l'imaginaire \\
Le réel et l'irréel \\
Le rêve et la réalité \\
Le rêve et la veille \\
Le riche et le pauvre \\
L'erreur et la faute \\
L'erreur et l'ignorance \\
L'erreur et l'illusion \\
Le sacré et le profane \\
Les anciens et les modernes \\
Les Anciens et les Modernes \\
Le sauvage et le barbare \\
Le sauvage et le cultivé \\
Le savant et le politique \\
Le savant et l'ignorant \\
Les besoins et les désirs \\
Les causes et les effets \\
Les causes et les lois \\
Les causes et les raisons \\
Les causes et les signes \\
Les changements scientifiques et la réalité \\
Les choses et les événements \\
L'esclave et son maître \\
Les connaissances scientifiques peuvent-elles être à la fois vraies et provisoires ? \\
Les désirs et les valeurs \\
Les disciplines scientifiques et leurs interfaces \\
Les droits de l'homme et ceux du citoyen \\
Les droits et les devoirs \\
Le sensible et la science \\
Le sensible et l'intelligible \\
Le sentiment du juste et de l'injuste \\
Les faits et les valeurs \\
Les fins et les moyens \\
Les fins naturelles et les fins morales \\
Les forts et les faibles \\
Les hommes et les dieux \\
Les hommes et les femmes \\
Les idées et les choses \\
Le signe et le symbole \\
Le simple et le complexe \\
Le singulier et le pluriel \\
Les intentions et les actes \\
Les intentions et les conséquences \\
Les lettres et les sciences \\
Les lois et les armes \\
Les lois et les mœurs \\
Les mathématiques et la pensée de l'infini \\
Les mathématiques et la quantité \\
Les mathématiques et l'expérience \\
Les mœurs et la morale \\
Les mots et la signification \\
Les mots et les choses \\
Les mots et les concepts \\
Les moyens et la fin \\
Les moyens et les fins \\
Les moyens et les fins en art \\
Les normes et les valeurs \\
Le social et le politique \\
Le soi et le je \\
Le sommeil et la veille \\
Le sophiste et le philosophe \\
L'espace et le lieu \\
L'espace et le territoire \\
Les paroles et les actes \\
L'espèce et l'individu \\
Le spirituel et le temporel \\
Les poètes et la cité \\
L'espoir et la crainte \\
Les principes et les éléments \\
L'esprit et la lettre \\
L'esprit et la machine \\
L'esprit et le cerveau \\
Les riches et les pauvres \\
Les sciences de la vie et de la Terre \\
Les sciences de l'homme et l'évolution \\
Les sciences et le vivant \\
Les sciences humaines et le droit \\
L'essence et l'existence \\
Les témoignages et la preuve \\
Le style et le beau \\
Le sujet et l'individu \\
Le sujet et l'objet \\
Les vivants et les morts \\
Le talent et le génie \\
Le tas et le tout \\
L'État et la culture \\
L'État et la guerre \\
L'État et la justice \\
L'État et la nation \\
L'État et la Nation \\
L'État et la protection \\
L'État et la société \\
L'État et le droit \\
L'État et le marché \\
L'État et le peuple \\
L'État et les communautés \\
L'État et les Églises \\
L'État et l'individu \\
Le temps et l'espace \\
Le théâtre et l'existence \\
Le tout et les parties \\
Le tragique et le comique \\
Le travail et la propriété \\
Le travail et la technique \\
Le travail et le labeur \\
Le travail et le temps \\
Le travail et l'œuvre \\
L'être et la relation \\
L'être et la volonté \\
L'être et le bien \\
L'être et le devoir-être \\
L'être et le néant \\
L'être et les êtres \\
L'être et l'essence \\
L'être et l'étant \\
L'être et le temps \\
L'être imaginaire et l'être de raison \\
Le vécu et la vérité \\
L'événement et le fait divers \\
Le vice et la vertu \\
Le vide et le plein \\
L'évidence et la démonstration \\
Le visible et l'invisible \\
Le vivant et la machine \\
Le vivant et la mort \\
Le vivant et la sensibilité \\
Le vivant et la technique \\
Le vivant et le vécu \\
Le vivant et l'expérimentation \\
Le vivant et l'inerte \\
Le volontaire et l'involontaire \\
Le vrai et le bien \\
Le vrai et le bien sont-ils analogues ? \\
Le vrai et le faux \\
Le vrai et le vraisemblable \\
Le vraisemblable et le romanesque \\
L'excès et le défaut \\
L'existence et le temps \\
L'expérience et la sensation \\
L'expérience et l'expérimentation \\
L'expert et l'amateur \\
L'habileté et la prudence \\
L'histoire a-t-elle un commencement et une fin ? \\
L'histoire et la géographie \\
L'homme et la bête \\
L'homme et la machine \\
L'homme et la nature sont-ils commensurables ? \\
L'homme et l'animal \\
L'homme et le citoyen \\
L'humour et l'ironie \\
Libéral et libertaire \\
Libéralité et libéralisme \\
Liberté et courage \\
Liberté et démocratie \\
Liberté et déterminisme \\
Liberté et éducation \\
Liberté et égalité \\
Liberté et engagement \\
Liberté et existence \\
Liberté et habitude \\
Liberté et indépendance \\
Liberté et libération \\
Liberté et licence \\
Liberté et nécessité \\
Liberté et pouvoir \\
Liberté et responsabilité \\
Liberté et savoir \\
Liberté et sécurité \\
Liberté et société \\
Liberté et solitude \\
Liberté humaine et liberté divine \\
Libertés publiques et culture politique \\
Libre arbitre et déterminisme sont-ils compatibles ? \\
Libre arbitre et liberté \\
Libre et heureux \\
L'idéal et le réel \\
L'identité et la différence \\
L'image et le réel \\
L'imaginaire et le réel \\
L'imagination est-elle maîtresse d'erreur et de fausseté ? \\
L'imagination et la raison \\
L'inconscient et l'involontaire \\
L'inconscient et l'oubli \\
L'indice et la preuve \\
L'indicible et l'impensable \\
L'indicible et l'ineffable \\
L'individuel et le collectif \\
L'individu et la multitude \\
L'individu et le groupe \\
L'individu et l'espèce \\
L'induction et la déduction \\
L'ineffable et l'innommable \\
L'inexactitude et le savoir scientifique \\
L'infini et l'indéfini \\
L'informe et le difforme \\
L'inné et l'acquis \\
L'instant et la durée \\
L'instrument et la machine \\
L'intérieur et l'extérieur \\
L'interprète et le créateur \\
L'invention et la découverte \\
Lire et écrire \\
L'irrationnel et le politique \\
Littérature et réalité \\
L'objet et la chose \\
L'œil et l'oreille \\
L'œuvre d'art et le plaisir \\
L'œuvre d'art et sa reproduction \\
L'œuvre d'art et son auteur \\
L'œuvre et le produit \\
Logique et dialectique \\
Logique et existence \\
Logique et grammaire \\
Logique et logiques \\
Logique et mathématique \\
Logique et mathématiques \\
Logique et métaphysique \\
Logique et méthode \\
Logique et ontologie \\
Logique et psychologie \\
Logique et réalité \\
Logique et vérité \\
Logique générale et logique transcendantale \\
Loi morale et loi politique \\
Loi naturelle et loi politique \\
Lois et coutumes \\
Lois et règles en logique \\
Loisir et oisiveté \\
Lois naturelles et lois civiles \\
L'ombre et la lumière \\
L'oral et l'écrit \\
L'ordre et la mesure \\
L'ordre et le désordre \\
L'organique et le mécanique \\
L'organique et l'inorganique \\
L'Orient et l'Occident \\
L'original et la copie \\
L'origine et le fondement \\
L'oubli et le pardon \\
L'outil et la machine \\
L'un et le multiple \\
L'un et l'être \\
L'universel et le particulier \\
L'universel et le singulier \\
L'utile et l'agréable \\
L'utile et le beau \\
L'utile et le bien \\
L'utile et l'honnête \\
L'utile et l'inutile \\
L'utopie et l'idéologie \\
Machine et organisme \\
Machines et liberté \\
Machines et mémoire \\
Magie et religion \\
Maître et disciple \\
Maître et serviteur \\
Maîtrise et puissance \\
Majorité et minorité \\
Maladie et convalescence \\
Mal et liberté \\
Mathématiques et réalité \\
Mathématiques pures et mathématiques appliquées \\
Matière et corps \\
Matière et matériaux \\
Mécanisme et finalité \\
Médecine et philosophie \\
Mémoire et fiction \\
Mémoire et identité \\
Mémoire et responsabilité \\
Mémoire et souvenir \\
Mensonge et politique \\
Mesure et démesure \\
Métaphysique et histoire \\
Métaphysique et ontologie \\
Métaphysique et politique \\
Métaphysique et religion \\
Métier et vocation \\
Microscope et télescope \\
Misère et pauvreté \\
Modèle et copie \\
Monde et nature \\
Monologue et dialogue \\
Montrer et démontrer \\
Morale et calcul \\
Morale et convention \\
Morale et économie \\
Morale et éducation \\
Morale et histoire \\
Morale et identité \\
Morale et intérêt \\
Morale et liberté \\
Morale et politique sont-elles indépendantes ? \\
Morale et pratique \\
Morale et prudence \\
Morale et religion \\
Morale et sexualité \\
Morale et société \\
Morale et technique \\
Morale et violence \\
Moralité et connaissance \\
Moralité et utilité \\
Murs et frontières \\
Musique et bruit \\
Mythe et connaissance \\
Mythe et histoire \\
Mythe et pensée \\
Mythe et philosophie \\
Mythe et symbole \\
Mythe et vérité \\
Mythes et idéologies \\
Narration et identité \\
Nation et richesse \\
Nature et artifice \\
Nature et convention \\
Nature et culture \\
Nature et fonction du sacrifice \\
Nature et histoire \\
Nature et institution \\
Nature et institutions \\
Nature et loi \\
Nature et morale \\
Naturel et artificiel \\
Nécessité et contingence \\
Nécessité et liberté \\
Négation et privation \\
Névroses et psychoses \\
Nomade et sédentaire \\
Nom propre et nom commun \\
Normes et valeurs \\
Normes morales et normes vitales \\
Nous et les autres \\
Nouveauté et tradition \\
Obéissance et liberté \\
Obéissance et servitude \\
Obéissance et soumission \\
Objectivé et subjectivité \\
Objet et œuvre \\
Observation et expérience \\
Observation et expérimentation \\
Observer et comprendre \\
Observer et expérimenter \\
Observer et interpréter \\
Œuvre et événement \\
Opinion et ignorance \\
Optimisme et pessimisme \\
Ordre et désordre \\
Ordre et justice \\
Ordre et liberté \\
Organisme et milieu \\
Origine et commencement \\
Origine et fondement \\
Outil et machine \\
Outil et organe \\
Par-delà beauté et laideur \\
Pardonner et oublier \\
Parler et agir \\
Parole et pouvoir \\
Paroles et actes \\
Passions et intérêts \\
Peinture et histoire \\
Peinture et réalité \\
Pensée et réalité \\
Penser et calculer \\
Penser et imaginer \\
Penser et parler \\
Penser et raisonner \\
Penser et savoir \\
Penser et sentir \\
Perception et aperception \\
Perception et connaissance \\
Perception et création artistique \\
Perception et imagination \\
Perception et jugement \\
Perception et mouvement \\
Perception et passivité \\
Perception et sensation \\
Perception et souvenir \\
Perception et vérité \\
Percevoir et concevoir \\
Percevoir et imaginer \\
Percevoir et juger \\
Percevoir et sentir \\
Permanence et identité \\
Personne et individu \\
Persuader et convaincre \\
Peuple et culture \\
Peuple et masse \\
Peuple et multitude \\
Peuple et société \\
Peuples et masses \\
Peut-on concilier bonheur et liberté ? \\
Peut-on discuter des goûts et des couleurs ? \\
Peut-on distinguer entre de bons et de mauvais désirs ? \\
Peut-on distinguer entre les bons et les mauvais désirs ? \\
Peut-on être injuste et heureux ? \\
Peut-on opposer justice et liberté ? \\
Peut-on opposer morale et technique ? \\
Peut-on opposer nature et culture ? \\
Peut-on préconiser, dans les sciences humaines et sociales, l'imitation des sciences de la nature ? \\
Peut-on séparer l'homme et l'œuvre ? \\
Peut-on séparer politique et économie ? \\
Philosophie et mathématiques \\
Philosophie et métaphysique \\
Philosophie et poésie \\
Philosophie et système \\
Physique et mathématiques \\
Physique et métaphysique \\
Pitié et compassion \\
Pitié et cruauté \\
Pitié et mépris \\
Plaisir et bonheur \\
Plaisir et douleur \\
Pluralisme et politique \\
Pluralité et unité \\
Poésie et philosophie \\
Poésie et vérité \\
Poétique et prosaïque \\
Point de vue du créateur et point de vue du spectateur \\
Police et politique \\
Politique et coopération \\
Politique et esthétique \\
Politique et mémoire \\
Politique et parole \\
Politique et participation \\
Politique et passions \\
Politique et propagande \\
Politique et secret \\
Politique et technologie \\
Politique et territoire \\
Politique et trahison \\
Politique et unité \\
Politique et vérité \\
Possession et propriété \\
Pouvoir et autorité \\
Pouvoir et contre-pouvoir \\
Pouvoir et devoir \\
Pouvoir et politique \\
Pouvoir et puissance \\
Pouvoir et savoir \\
Pouvoirs et libertés \\
Pouvoir temporel et pouvoir spirituel \\
Prédicats et relations \\
Prédiction et probabilité \\
Prédire et expliquer \\
Prémisses et conclusions \\
Présence et absence \\
Présence et représentation \\
Preuve et démonstration \\
Principe et cause \\
Principe et commencement \\
Principe et fondement \\
Principes et stratégie \\
Probabilité et explication scientifique \\
Production et création \\
Produire et créer \\
Proposition et jugement \\
Prose et poésie \\
Prospérité et sécurité \\
Prouver et démontrer \\
Prouver et éprouver \\
Prouver et réfuter \\
Providence et destin \\
Prudence et liberté \\
Psychologie et contrôle des comportements \\
Psychologie et métaphysique \\
Psychologie et neurosciences \\
Pulsion et instinct \\
Pulsions et passions \\
Punition et vengeance \\
Qualité et quantité \\
Quantification et pensée scientifique \\
Quantité et qualité \\
Que signifier « juger en son âme et conscience » ? \\
Qu'est-ce que « se rendre maître et possesseur de la nature » ? \\
Question et problème \\
Que vaut la distinction entre nature et culture ? \\
Qui suis-je et qui es-tu ? \\
Raison et dialogue \\
Raison et folie \\
Raison et fondement \\
Raison et langage \\
Raison et politique \\
Raison et révélation \\
Raison et tradition \\
Raisonnable et rationnel \\
Raisonnement et expérimentation \\
Raisonner et calculer \\
Rationnel et raisonnable \\
Réalisme et idéalisme \\
Réalité et apparence \\
Réalité et idéal \\
Réalité et perception \\
Réalité et représentation \\
Rebuts et objets quelconques : une matière pour l'art ? \\
Récit et histoire \\
Récit et mémoire \\
Reconnaissance et inégalité \\
Réforme et révolution \\
Refuser et réfuter \\
Réfutation et confirmation \\
Règle et commandement \\
Règle morale et norme juridique \\
Règles sociales et loi morale \\
Regrets et remords \\
Religion et démocratie \\
Religion et liberté \\
Religion et moralité \\
Religion et politique \\
Religion et violence \\
Religion naturelle et religion révélée \\
Religions et démocratie \\
Représentation et illusion \\
République et démocratie \\
Résistance et obéissance \\
Résistance et soumission \\
Respect et tolérance \\
Révolte et révolution \\
Rhétorique et vérité \\
Richesse et pauvreté \\
Rire et pleurer \\
Rites et cérémonies \\
Rituels et cérémonies \\
Roman et vérité \\
Savoir et croire \\
Savoir et démontrer \\
Savoir et liberté \\
Savoir et objectivité dans les sciences \\
Savoir et pouvoir \\
Savoir et rectification \\
Savoir et savoir faire \\
Savoir et savoir-faire \\
Savoir et vérifier \\
Science du vivant et finalisme \\
Science et abstraction \\
Science et certitude \\
Science et complexité \\
Science et croyance \\
Science et démocratie \\
Science et domination sociale \\
Science et expérience \\
Science et histoire \\
Science et hypothèse \\
Science et idéologie \\
Science et imagination \\
Science et invention \\
Science et libération \\
Science et magie \\
Science et métaphysique \\
Science et méthode \\
Science et mythe \\
Science et objectivité \\
Science et opinion \\
Science et persuasion \\
Science et philosophie \\
Science et réalité \\
Science et religion \\
Science et sagesse \\
Science et société \\
Science et technique \\
Science et technologie \\
Science pure et science appliquée \\
Sciences de la nature et sciences de l'esprit \\
Sciences de la nature et sciences humaines \\
Sciences empiriques et critères du vrai \\
Sciences et philosophie \\
Sciences humaines et déterminisme \\
Sciences humaines et herméneutique \\
Sciences humaines et idéologie \\
Sciences humaines et liberté sont-elles compatibles ? \\
Sciences humaines et littérature \\
Sciences humaines et naturalisme \\
Sciences humaines et nature humaine \\
Sciences humaines et objectivité \\
Sciences humaines et philosophie \\
Sciences sociales et humanités \\
Sécurité et liberté \\
Sensation et perception \\
Sens et existence \\
Sens et fait \\
Sens et limites de la notion de capital culturel \\
Sens et sensibilité \\
Sens et sensible \\
Sens et signification \\
Sens et structure \\
Sens et vérité \\
Sensible et intelligible \\
Sens propre et sens figuré \\
Sentir et juger \\
Sentir et penser \\
Se parler et s'entendre \\
Sexe et genre \\
Sexualité et féminité \\
Sexualité et nature \\
Signe et symbole \\
Signes, traces et indices \\
Signification et expression \\
Signification et vérité \\
Sincérité et vérité \\
Société et biologie \\
Société et communauté \\
Société et organisme \\
Société et religion \\
Solitude et isolement \\
Sophismes et paradoxes \\
Soumission et servitude \\
Sport et politique \\
Structure et événement \\
Substance et accident \\
Substance et sujet \\
Suffit-il pour être juste d'obéir aux lois et aux coutumes de son pays ? \\
Sujet et citoyen \\
Sujet et prédicat \\
Sujet et substance \\
Superstition et religion \\
Surface et profondeur \\
Surveillance et discipline \\
Syllogisme et démonstration \\
Sympathie et respect \\
Système et structure \\
Talent et génie \\
Tautologie et contradiction \\
Technique et apprentissage \\
Technique et idéologie \\
Technique et intérêt \\
Technique et nature \\
Technique et pratiques scientifiques \\
Technique et progrès \\
Technique et responsabilité \\
Technique et savoir-faire \\
Technique et violence \\
Temps et commencement \\
Temps et conscience \\
Temps et création \\
Temps et éternité \\
Temps et histoire \\
Temps et irréversibilité \\
Temps et liberté \\
Temps et mémoire \\
Temps et réalité \\
Temps et vérité \\
Thème et variations \\
Théorie et expérience \\
Théorie et modèle \\
Théorie et modélisation \\
Théorie et pratique \\
Tradition et innovation \\
Tradition et liberté \\
Tradition et nouveauté \\
Tradition et raison \\
Tradition et transmission \\
Tradition et vérité \\
Traduire et interpréter \\
Tragédie et comédie \\
Transcendance et altérité \\
Transcendance et immanence \\
Travail et aliénation \\
Travail et besoin \\
Travail et bonheur \\
Travail et capital \\
Travail et liberté \\
Travail et loisir \\
Travail et nécessité \\
Travail et œuvre \\
Travail et subjectivité \\
Travailler et œuvrer \\
Travail manuel et travail intellectuel \\
Tuer et laisser mourir \\
Universalité et nécessité dans les sciences \\
Univocité et équivocité \\
Utilité et beauté \\
Utopie et tradition \\
Vérité et apparence \\
Vérité et certitude \\
Vérité et cohérence \\
Vérité et efficacité \\
Vérité et exactitude \\
Vérité et fiction \\
Vérité et histoire \\
Vérité et illusion \\
Vérité et liberté \\
Vérité et poésie \\
Vérité et réalité \\
Vérité et religion \\
Vérité et sensibilité \\
Vérité et signification \\
Vérité et sincérité \\
Vérité et subjectivité \\
Vérité et vérification \\
Vérité et vraisemblance \\
Vérités de fait et vérités de raison \\
Vertu et habitude \\
Vertu et perfection \\
Vice et délice \\
Vie et existence \\
Vie et volonté \\
Vie politique et vie contemplative \\
Vie privée et vie publique \\
Vie publique et vie privée \\
Violence et discours \\
Violence et force \\
Violence et histoire \\
Violence et politique \\
Violence et pouvoir \\
Vitalisme et mécanique \\
Vivre et bien vivre \\
Vivre et exister \\
Voir et entendre \\
Voir et savoir \\
Voir et toucher \\
Voir le meilleur et faire le pire \\
Vouloir et pouvoir \\
Y a-t-il continuité entre l'expérience et la science ? \\
Y a-t-il continuité ou discontinuité entre la pensée mythique et la science ? \\
Y a-t-il de bons et de mauvais désirs ? \\
Y a-t-il lieu d'opposer matière et esprit ? \\
« Aime, et fais ce que tu veux » \\


\subsection{X ou Y}
\label{sec-4-6}

\noindent
Ai-je un corps ou suis-je mon corps ? \\
Animal politique ou social ? \\
Cité juste ou citoyen juste ? \\
Connaît-on la vie ou bien connaît-on le vivant ? \\
Connaît-on la vie ou connaît-on le vivant ? \\
Connaît-on la vie ou le vivant ? \\
Connaître la vie ou le vivant ? \\
Dieu, prouvé ou éprouvé ? \\
Droits de l'homme ou droits du citoyen ? \\
Est-ce par son objet ou par ses méthodes qu'une science peut se définir ? \\
Être ou avoir \\
Être ou ne pas être \\
Être ou ne pas être, est-ce la question ? \\
Faut-il rire ou pleurer ? \\
Interpréter ou expliquer \\
La beauté est-elle dans le regard ou dans la chose vue ? \\
La conscience est-elle ou n'est-elle pas ? \\
La démocratie est-elle moyen ou fin ? \\
La docilité est-elle un vice ou une vertu ? \\
La justice : moyen ou fin de la politique ? \\
La logique : découverte ou invention ? \\
La majorité, force ou droit ? \\
La métaphysique se définit-elle par son objet ou sa démarche ? \\
La moralité est-elle affaire de principes ou de conséquences ? \\
La ou les vertus ? \\
La perception de l'espace est-elle innée ou acquise ? \\
L'art ou les arts \\
L'art : expérience, exercice ou habitude ? \\
La santé est-elle un droit ou un devoir ? \\
La science découvre-t-elle ou construit-elle son objet ? \\
La vérité est-elle affaire de croyance ou de savoir ? \\
Le cinéma est-il un art ou une industrie ? \\
Le droit sert-il à établir l'ordre ou la justice ? \\
Le goût : certitude ou conviction ? \\
Le langage rapproche-t-il ou sépare-t-il les hommes ? \\
Lequel, de l'art ou du réel, est-il une imitation de l'autre ? \\
Le rôle des théories est-il d'expliquer ou de décrire ? \\
Les hypothèses scientifiques ont-elles pour nature d'être confirmées ou infirmées ? \\
Le sport : s'accomplir ou se dépasser ? \\
Les qualités sensibles sont-elles dans les choses ou dans l'esprit ? \\
Les sciences humaines sont-elles explicatives ou compréhensives ? \\
L'État est-il fin ou moyen ? \\
Le temps est-il en nous ou hors de nous ? \\
Le travail unit-il ou sépare-t-il les hommes ? \\
L'histoire est-elle une explication ou une justification du passé ? \\
L'histoire : enquête ou science ? \\
L'histoire : science ou récit ? \\
L'homme se reconnaît-il mieux dans le travail ou dans le loisir ? \\
L'identité personnelle est-elle donnée ou construite ? \\
L'incertitude est-elle dans les choses ou dans les idées ? \\
L'inconscient est-il dans l'âme ou dans le corps ? \\
Naît-on sujet ou le devient-on ? \\
Peut-on être plus ou moins libre ? \\
Primitif ou premier ? \\
Punir ou soigner ? \\
Qu'est-ce qui est le plus à craindre, l'ordre ou le désordre ? \\
Qu'est-ce qu'un grand homme ou une grande femme ? \\
Tout énoncé est-il nécessairement vrai ou faux ? \\
Tout ou rien \\
Vaut-il mieux oublier ou pardonner ? \\
Vaut-il mieux subir ou commettre l'injustice ? \\
Y a-t-il continuité ou discontinuité entre la pensée mythique et la science ? \\
Y a-t-il une ou des morales ? \\
Y a-t-il une ou plusieurs philosophies ? \\
« La logique » ou bien « les logiques » ? \\


\subsection{Citation}
\label{sec-4-7}

\noindent
Dire « je » \\
En quel sens peut-on dire qu' « on expérimente avec sa raison » ? \\
Est-il vrai qu'en science, « rien n'est donné, tout est construit » ? \\
La question « qui suis-je » admet-elle une réponse exacte ? \\
La question : « qui ? » \\
L'État est-il un « monstre froid » ? \\
Le « je ne sais quoi » \\
L'idée de « sciences exactes » \\
Peut-on parler de « travail intellectuel » ? \\
Pourquoi parle-t-on d'une « société civile » ? \\
Puis-je dire « ceci est mon corps » ? \\
Quelle est l'unité du « je » ? \\
Quelle valeur donner à la notion de « corps social » ? \\
Quel sens donner à l'expression « gagner sa vie » ? \\
Que penser de l'adage : « Que la justice s'accomplisse, le monde dût-il périr » (Fiat justitia pereat mundus) ? \\
Que penser de la formule : « il faut suivre la nature » ? \\
Que signifier « juger en son âme et conscience » ? \\
Que signifie « donner le change » ? \\
Qu'est-ce que « parler le même langage » ? \\
Qu'est-ce que « se rendre maître et possesseur de la nature » ? \\
Qu'est-ce qu'une œuvre « géniale » ? \\
Qu'est-ce qu'une « expérience de pensée » ? \\
Qu'est-ce qu'une « performance » ? \\
Qu'est-ce qu'un « champ artistique » ? \\
Qu'est-ce qu'un « être dégénéré » ? \\
Que vaut l'excuse : « C'est plus fort que moi » ? \\
Que vaut l'excuse : « Je ne l'ai pas fait exprès» ? \\
Que veut dire l'expression « aller au fond des choses » ? \\
Que veut dire « essentiel » ? \\
Que veut dire « je t'aime » ? \\
Que veut dire « réel » ? \\
Que veut dire « respecter la nature » ? \\
Que veut dire : « être cultivé » ? \\
Que veut dire : « je t'aime » ? \\
Que veut dire : « le temps passe » ? \\
Que veut dire : « respecter la nature » ? \\
Qui est autorisé à me dire « tu dois » ? \\
Qui parle quand je dis « je » ? \\
Qui peut me dire « tu ne dois pas » ? \\
« Aime, et fais ce que tu veux » \\
« Aimer » se dit-il en un seul sens ? \\
« Aimez vos ennemis » \\
« Après moi, le déluge » \\
« Aux armes citoyens ! » \\
« À l'impossible, nul n'est tenu » \\
« À quelque chose malheur est bon » \\
« Bienheureuse faute » \\
« Ceci » \\
« Ce ne sont que des mots » \\
« C'est humain » \\
« C'est la vie » \\
« Comment peut-on être persan ? » \\
« Connais-toi toi-même » \\
« Dans un bois aussi courbe que celui dont l'homme est fait on ne peut rien tailler de tout à fait droit » \\
« De la musique avant toute chose » \\
« Deviens qui tu es » \\
« Dieu est mort » \\
« Être contre » \\
« Expliquer les faits sociaux par des faits sociaux » \\
« Il faudrait rester des années entières pour contempler une telle œuvre » \\
« Il ne lui manque que la parole » \\
« Je mens » \\
« Je n'ai pas voulu cela » \\
« Je ne crois que ce que je vois » \\
« Je ne l'ai pas fait exprès » \\
« Je ne voulais pas cela » : en quoi les sciences humaines permettent-elles de comprendre cette excuse ? \\
« Je préfère une injustice à un désordre » \\
« La crainte est le commencement de la sagesse » \\
« La critique est aisée » \\
« La logique » ou bien « les logiques » ? \\
« La science ne pense pas » \\
« La vie des formes » \\
« La vie est une scène » \\
« La vie est un songe » \\
« La vraie morale se moque de la morale » \\
« L'enfer est pavé de bonnes intentions » \\
« Les bons comptes font les bons amis » \\
« Le seul problème philosophique vraiment sérieux, c'est le suicide » \\
« Les faits, rien que les faits » \\
« Les faits sont là » \\
« Le travail rend libre » \\
« L'histoire jugera » \\
« L'histoire jugera » : quel sens faut-il accorder à cette expression ? \\
« L'homme est la mesure de toute chose » \\
« L'homme est la mesure de toutes choses » \\
« Liberté, égalité, fraternité » \\
« Malheur aux vaincus » \\
« Ne fais pas à autrui ce que tu ne voudrais pas qu'on te fasse » \\
« Nul n'est censé ignorer la loi » \\
« Œil pour œil, dent pour dent » \\
« Pas de liberté pour les ennemis de la liberté » ? \\
« Pauvre bête » \\
« Petites causes, grands effets » \\
« Pourquoi » \\
« Prendre ses désirs pour des réalités » \\
« Quelle vanité que la peinture » \\
« Que nul n'entre ici s'il n'est géomètre » \\
« Que va-t-il se passer ? » \\
« Rien de ce qui est humain ne m'est étranger » \\
« Rien n'est sans raison » \\
« Rien n'est simple » \\
« Sans titre » \\
« Sauver les apparences » \\
« Sauver les phénomènes » \\
« Toute peine mérite salaire » \\
« Tout est relatif » \\
« Tradition n'est pas raison » \\
« Trop beau pour être vrai » \\
« Tu ne tueras point » \\
« Un instant d'éternité » \\
« Vis caché » \\
% Emacs 24.5.1 (Org mode 8.2.10)
\end{document}
