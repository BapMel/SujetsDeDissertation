% Created 2023-01-17 mar. 15:12
% Intended LaTeX compiler: pdflatex
\documentclass[a4paper,12pt]{article}
\usepackage[utf8]{inputenc}
\usepackage[T1]{fontenc}
\usepackage{graphicx}
\usepackage{longtable}
\usepackage{wrapfig}
\usepackage{rotating}
\usepackage[normalem]{ulem}
\usepackage{amsmath}
\usepackage{amssymb}
\usepackage{capt-of}
\usepackage{hyperref}
\usepackage[french]{babel}
\usepackage[frenchb]{babel}
\usepackage{lmodern}
\DeclareUnicodeCharacter{00A0}{~}
\DeclareUnicodeCharacter{200B}{}
\author{Baptiste Mélès}
\date{27 septembre 2022}
\title{11 000 sujets de dissertation de philosophie}
\hypersetup{
 pdfauthor={Baptiste Mélès},
 pdftitle={11 000 sujets de dissertation de philosophie},
 pdfkeywords={},
 pdfsubject={},
 pdfcreator={Emacs 24.5.1 (Org mode 8.2.10)}, 
 pdflang={French}}
\begin{document}

\maketitle
On trouvera ici, classés par ordre alphabétique puis par concours et
par type, quelque 11 000 sujets de dissertation de philosophie.

Cette liste est la compilation de tous les sujets donnés aux épreuves
écrites et orales de près d'une centaine de concours :
\begin{enumerate}
\item agrégation externe (14 années : 2008-2021) ;
\item agrégation interne (13 années : 2009-2021) ;
\item CAPES externe (12 années : 2010-2021) ;
\item CAPES interne (11 années : 2011-2021) ;
\item École normale supérieure de Paris, concours A​/​L (20 années :
2002-2021) ;
\item École normale supérieure de Paris, concours B​/​L (20 années :
2002-2021).
\end{enumerate}

L'auteur vous sera reconnaissant de lui signaler toute coquille
ou erreur\footnote{Documents pédagogiques de Baptiste Mélès (CNRS, AHP-PReST,
Université de Lorraine) :
\begin{enumerate}
\item « Méthode de la dissertation philosophique »
(\url{http://baptiste.meles.free.fr/site/B.Meles-Methode\_dissertation.pdf},
2010-2022) ;
\item « Méthode du commentaire de texte philosophique »
(\url{http://baptiste.meles.free.fr/site/B.Meles-Methode\_commentaire\_texte.pdf},
2007-2016)
\item « Méthodologie du mémoire de Master »
(\url{http://baptiste.meles.free.fr/site/B.Meles-Memoire\_Master.pdf},
2014-2019) ;
\item « Le travail personnel en philosophie, de la licence à l'agrégation »
(\url{http://baptiste.meles.free.fr/site/B.Meles-Travail\_perso.pdf},
2008-2016) ;
\item « Bibliographie de philosophie » (avec Paul Clavier,
\url{http://baptiste.meles.free.fr/site/P\_Clavier-B\_Meles-Bibliographie\_philosophie.pdf}, 2022)
\item « Les tables de vérité en braille »
(\url{http://baptiste.meles.free.fr/site/B.Meles-Table\_verite\_braille.pdf},
2011).
\end{enumerate}}.


\newpage

\tableofcontents

\newpage

\section{Liste complète}
\label{sec:org39e5b6e}

\noindent

\section{Tri par concours}
\label{sec:orgede0c79}

\subsection{Agrégation}
\label{sec:orgf0fa48b}

\noindent


\subsection{Agrégation externe}
\label{sec:orgde99a47}

\noindent


\subsection{Agrégation interne}
\label{sec:orga28676e}

\noindent


\subsection{CAPES}
\label{sec:org6914e32}

\noindent


\subsection{CAPES externe}
\label{sec:org3b54b6b}

\noindent


\subsection{CAPES interne}
\label{sec:orgfc50157}

\noindent


\subsection{ENS A​/​L}
\label{sec:orgb26ad69}

\noindent


\subsection{ENS B​/​L}
\label{sec:org184f523}

\noindent


\section{Tri par thème des sujets d'agrégation externe}
\label{sec:org13d5862}
\subsection{Philosophie générale}
\label{sec:org8a29126}

\noindent


\subsection{Esthétique}
\label{sec:org9adbe09}

\noindent


\subsection{Logique et épistémologie}
\label{sec:org32fada4}

\noindent


\subsection{Métaphysique}
\label{sec:org292c149}

\noindent


\subsection{Morale}
\label{sec:orgb0b18d8}

\noindent


\subsection{Politique}
\label{sec:org6a33845}

\noindent


\subsection{Sciences humaines}
\label{sec:orgf1a7274}

\noindent


\section{Tri par type}
\label{sec:org59d05b4}

\subsection{Question}
\label{sec:org2eac89b}

\noindent


\subsection{Question en « peut »}
\label{sec:org5ebc87c}

\noindent


\subsection{Mot unique}
\label{sec:org8d73d18}
\noindent


\subsection{Le X}
\label{sec:orgfffd433}
\noindent


\subsection{X et Y}
\label{sec:org096e44b}

\noindent


\subsection{X ou Y}
\label{sec:org6e75ce9}

\noindent


\subsection{Citation}
\label{sec:orga0a7545}

\noindent
Emacs 24.5.1 (Org mode 8.2.10)
\end{document}